\documentclass[10pt]{amsart} %Default font size is 10pt
%\documentclass[11pt, oneside]{book}

\IfFileExists{srcltx.sty}{\usepackage{srcltx}}

\usepackage{comment}
\usepackage{graphicx}
\usepackage[utf8]{inputenc}
\usepackage{xspace,amssymb,amsfonts,euscript}
\usepackage{amsthm,amsmath}
\usepackage{palatino}
\usepackage{euscript}
\usepackage{tikz-cd}
\usepackage{shuffle}
%\usepackage{hyperref}
\usepackage{url}
\input xy \xyoption {all}

%Margin size:
%\usepackage[margin=40mm]{geometry}
%\usepackage[a4paper, total={135mm, 200mm}]{geometry}
%If planning on printing read https://texfaq.org/FAQ-marginmanual



%Line spacing
%\linespread{1.33}

%%\numberwithin{equation}{section}
%\renewcommand{\thesection}{\arabic{section}} %this gets rid of the chapter numbering, so this is numbered like an article.
%%Number equations by sections
\numberwithin{equation}{section}



\RequirePackage{color}
\definecolor{myred}{rgb}{0.75,0,0}
\definecolor{mygreen}{rgb}{0,0.5,0}
\definecolor{myblue}{rgb}{0,0,0.65}



\RequirePackage{ifpdf}
\ifpdf
  \IfFileExists{pdfsync.sty}{\RequirePackage{pdfsync}}{}
  \RequirePackage[pdftex,
   colorlinks = true,
   urlcolor = myblue, % \href{...}{...} external (URL)
   citecolor = mygreen, % \cite{...}
   linkcolor = myred, % \ref{...} and \pageref{...}
   pagebackref,
%   pdfpagemode=None,
   bookmarksopen=true]{hyperref}
\else
  \RequirePackage[hypertex]{hyperref}
\fi

\RequirePackage{ae, aecompl, aeguill} % to have pretty pdf

% gothic, blackboard

\def\AG{{\mathfrak A}}  \def\ag{{\mathfrak a}}  \def\AM{{\mathbb{A}}}
\def\BG{{\mathfrak B}}  \def\bg{{\mathfrak b}}  \def\BM{{\mathbb{B}}}
\def\CG{{\mathfrak C}}  \def\cg{{\mathfrak c}}  \def\CM{{\mathbb{C}}}
\def\DG{{\mathfrak D}}  \def\dg{{\mathfrak d}}  \def\DM{{\mathbb{D}}}
\def\EG{{\mathfrak E}}  \def\eg{{\mathfrak e}}  \def\EM{{\mathbb{E}}}
\def\FG{{\mathfrak F}}  \def\fg{{\mathfrak f}}  \def\FM{{\mathbb{F}}}
\def\GG{{\mathfrak G}}  \def\gg{{\mathfrak g}}  \def\GM{{\mathbb{G}}}
\def\HG{{\mathfrak H}}  \def\hg{{\mathfrak h}}  \def\HM{{\mathbb{H}}}
\def\IG{{\mathfrak I}}  \def\ig{{\mathfrak i}}  \def\IM{{\mathbb{I}}}
\def\JG{{\mathfrak J}}  \def\jg{{\mathfrak j}}  \def\JM{{\mathbb{J}}}
\def\KG{{\mathfrak K}}  \def\kg{{\mathfrak k}}  \def\KM{{\mathbb{K}}}
\def\LG{{\mathfrak L}}  \def\lg{{\mathfrak l}}  \def\LM{{\mathbb{L}}}
\def\MG{{\mathfrak M}}  \def\mg{{\mathfrak m}}  \def\MM{{\mathbb{M}}}
\def\NG{{\mathfrak N}}  \def\ng{{\mathfrak n}}  \def\NM{{\mathbb{N}}}
\def\OG{{\mathfrak O}}  \def\og{{\mathfrak o}}  \def\OM{{\mathbb{O}}}
\def\PG{{\mathfrak P}}  \def\pg{{\mathfrak p}}  \def\PM{{\mathbb{P}}}
\def\QG{{\mathfrak Q}}  \def\qg{{\mathfrak q}}  \def\QM{{\mathbb{Q}}}
\def\RG{{\mathfrak R}}  \def\rg{{\mathfrak r}}  \def\RM{{\mathbb{R}}}
\def\SG{{\mathfrak S}}  \def\sg{{\mathfrak s}}  \def\SM{{\mathbb{S}}}
\def\TG{{\mathfrak T}}  \def\tg{{\mathfrak t}}  \def\TM{{\mathbb{T}}}
\def\UG{{\mathfrak U}}  \def\ug{{\mathfrak u}}  \def\UM{{\mathbb{U}}}
\def\VG{{\mathfrak V}}  \def\vg{{\mathfrak v}}  \def\VM{{\mathbb{V}}}
\def\WG{{\mathfrak W}}  \def\wg{{\mathfrak w}}  \def\WM{{\mathbb{W}}}
\def\XG{{\mathfrak X}}  \def\xg{{\mathfrak x}}  \def\XM{{\mathbb{X}}}
\def\YG{{\mathfrak Y}}  \def\yg{{\mathfrak y}}  \def\YM{{\mathbb{Y}}}
\def\ZG{{\mathfrak Z}}  \def\zg{{\mathfrak z}}  \def\ZM{{\mathbb{Z}}}

\newcommand{\km}{\Bbbk}

% bold, mathcal

\def\AB{{\mathbf A}}  \def\ab{{\mathbf a}}  \def\AC{{\mathcal{A}}}
\def\BB{{\mathbf B}}  \def\bb{{\mathbf b}}  \def\BC{{\mathcal{B}}}
\def\CB{{\mathbf C}}  \def\cb{{\mathbf c}}  \def\CC{{\mathcal{C}}}
\def\DB{{\mathbf D}}  \def\db{{\mathbf d}}  \def\DC{{\mathcal{D}}}
\def\EB{{\mathbf E}}  \def\eb{{\mathbf e}}  \def\EC{{\mathcal{E}}}
\def\FB{{\mathbf F}}  \def\fb{{\mathbf f}}  \def\FC{{\mathcal{F}}}
\def\GB{{\mathbf G}}  \def\gb{{\mathbf g}}  \def\GC{{\mathcal{G}}}
\def\HB{{\mathbf H}}  \def\hb{{\mathbf h}}  \def\HC{{\mathcal{H}}}
\def\IB{{\mathbf I}}  \def\ib{{\mathbf i}}  \def\IC{{\mathcal{I}}}
\def\JB{{\mathbf J}}  \def\jb{{\mathbf j}}  \def\JC{{\mathcal{J}}}
\def\KB{{\mathbf K}}  \def\kb{{\mathbf k}}  \def\KC{{\mathcal{K}}}
\def\LB{{\mathbf L}}  \def\lb{{\mathbf l}}  \def\LC{{\mathcal{L}}}
\def\MB{{\mathbf M}}  \def\mb{{\mathbf m}}  \def\MC{{\mathcal{M}}}
\def\NB{{\mathbf N}}  \def\nb{{\mathbf n}}  \def\NC{{\mathcal{N}}}
\def\OB{{\mathbf O}}  \def\ob{{\mathbf o}}  \def\OC{{\mathcal{O}}}
\def\PB{{\mathbf P}}  \def\pb{{\mathbf p}}  \def\PC{{\mathcal{P}}}
\def\QB{{\mathbf Q}}  \def\qb{{\mathbf q}}  \def\QC{{\mathcal{Q}}}
\def\RB{{\mathbf R}}  \def\rb{{\mathbf r}}  \def\RC{{\mathcal{R}}}
\def\SB{{\mathbf S}}  \def\sb{{\mathbf s}}  \def\SC{{\mathcal{S}}}
\def\TB{{\mathbf T}}  \def\tb{{\mathbf t}}  \def\TC{{\mathcal{T}}}
\def\UB{{\mathbf U}}  \def\ub{{\mathbf u}}  \def\UC{{\mathcal{U}}}
\def\VB{{\mathbf V}}  \def\vb{{\mathbf v}}  \def\VC{{\mathcal{V}}}
\def\WB{{\mathbf W}}  \def\wb{{\mathbf w}}  \def\WC{{\mathcal{W}}}
\def\XB{{\mathbf X}}  \def\xb{{\mathbf x}}  \def\XC{{\mathcal{X}}}
\def\YB{{\mathbf Y}}  \def\yb{{\mathbf y}}  \def\YC{{\mathcal{Y}}}
\def\ZB{{\mathbf Z}}  \def\zb{{\mathbf z}}  \def\ZC{{\mathcal{Z}}}

% EuScript

\def\AS{{\EuScript A}}
\def\BS{{\EuScript B}}
\def\CS{{\EuScript C}}
\def\DS{{\EuScript D}}
\def\ES{{\EuScript E}}
\def\FS{{\EuScript F}}
\def\GS{{\EuScript G}}
\def\HS{{\EuScript H}}
\def\IS{{\EuScript I}}
\def\JS{{\EuScript J}}
\def\KS{{\EuScript K}}
\def\LS{{\EuScript L}}
\def\MS{{\EuScript M}}
\def\NS{{\EuScript N}}
\def\OS{{\EuScript O}}
\def\PS{{\EuScript P}}
\def\QS{{\EuScript Q}}
\def\RS{{\EuScript R}}
\def\SS{{\EuScript S}}
\def\TS{{\EuScript T}}
\def\US{{\EuScript U}}
\def\VS{{\EuScript V}}
\def\WS{{\EuScript W}}
\def\XS{{\EuScript X}}
\def\YS{{\EuScript Y}}
\def\ZS{{\EuScript Z}}

%make \l=\lambda and \strokel=\l 

\let\strokel\l
\renewcommand\l{\lambda}

% greek

\def\a{\alpha}
\def\b{\beta}
\def\g{\gamma}
\def\G{\Gamma}
\def\d{\delta}
\def\D{\Delta}
\def\e{\varepsilon}
\def\ph{\varphi}
%\def\l{\lambda}
\def\L{\Lambda}
\def\O{\Omega}
\def\r{\rho}
\def\s{\sigma}
\def\S{\Sigma}
\def\th{\theta}
\def\Th{\Theta}
\def\t{\tau}
\def\z{\zeta}
\def\k{\kappa}




\newcommand{\nc}{\newcommand} \newcommand{\renc}{\renewcommand}



\def\reg{{\mathrm{reg}}}
\def\rs{{\mathrm{rs}}}

\DeclareMathOperator{\Obj}{Obj}
\DeclareMathOperator{\Mor}{Mor}

\DeclareMathOperator{\Ker}{Ker}
\DeclareMathOperator{\Cok}{Cok}
\DeclareMathOperator{\Res}{Res}

%\DeclareMathOperator{\H}{H}

\def\wt{\widetilde}
\def\ov{\overline}
\def\un{\underline}

\def\p{{}^p}

\def\to{\rightarrow}

\nc{\ic}{\mathbf{IC}}

\nc{\gl}{{\mathfrak{gl}}}

\DeclareMathOperator{\Hom}{Hom}
\DeclareMathOperator{\RHom}{RHom}
\DeclareMathOperator{\Nat}{Nat}
\DeclareMathOperator{\supp}{supp} 
\DeclareMathOperator{\End}{End} 
\DeclareMathOperator{\Rep}{Rep}
\DeclareMathOperator{\id}{Id}
\DeclareMathOperator{\For}{For}
\DeclareMathOperator{\Rg}{R \G}
\DeclareMathOperator{\colim}{colim}
\DeclareMathOperator{\im}{im}

\DeclareMathOperator{\Rad}{rad}

\DeclareMathOperator{\chara}{char}
\DeclareMathOperator{\rank}{rank}

\DeclareMathOperator{\stab}{stab}

\DeclareMathOperator{\Proj}{Proj}
\DeclareMathOperator{\eval}{eval}

\DeclareMathOperator{\Pol}{Pol}

\DeclareMathOperator{\real}{real}

\newtheorem{thm}{Theorem}[section] %change section to subsection to label as in an article
\newtheorem{lemma}[thm]{Lemma}
\newtheorem{definition}[thm]{Definition/Theorem}
\newtheorem{prop}[thm]{Proposition}
\newtheorem{cor}[thm]{Corollary}
\newtheorem{claim}[thm]{Claim}  
\newtheorem{conj}[thm]{Conjecture}
\newtheorem{desi}[thm]{Desideratum}
\newtheorem{porism}[thm]{Porism}


%manualtheorem: to manually add numbering to theorem
\newtheorem{manualtheoreminner}{Theorem}
\newenvironment{manualtheorem}[1]{%
  \renewcommand\themanualtheoreminner{#1}%
  \manualtheoreminner
}{\endmanualtheoreminner}

%manualcorollary
\newtheorem{manualcorollaryinner}{Corollary}
\newenvironment{manualcorollary}[1]{%
  \renewcommand\themanualcorollaryinner{#1}%
  \manualcorollaryinner
}{\endmanualcorollaryinner}

%manualproposition
\newtheorem{manualpropositioninner}{Proposition}
\newenvironment{manualproposition}[1]{%
  \renewcommand\themanualpropositioninner{#1}%
  \manualpropositioninner
}{\endmanualpropositioninner}

%manualporism
\newtheorem{manualporisminner}{Porism}
\newenvironment{manualporism}[1]{%
  \renewcommand\themanualporisminner{#1}%
  \manualporisminner
}{\endmanualporisminner}

\theoremstyle{definition}
\newtheorem{defn}[thm]{Definition}
\newtheorem{defi}[thm]{Definition}
\newtheorem{ex}[thm]{Example}
\newtheorem{remark}[thm]{Remark}
\newtheorem{question}[thm]{Question}
\newtheorem{openquestion}[thm]{Open Question}

%manualdefinition
\newtheorem{manualdefinitioninner}{Definition}
\newenvironment{manualdefinition}[1]{%
  \renewcommand\themanualdefinitioninner{#1}%
  \manualdefinitioninner
}{\endmanualdefinitioninner}

%\theoremstyle{remark}





\DeclareMathOperator{\Ext}{Ext}
\DeclareMathOperator{\Spec}{Spec}
\DeclareMathOperator{\Ind}{Ind}
\DeclareMathOperator{\ev}{ev}
\DeclareMathOperator{\codim}{codim}

\DeclareMathOperator{\cok}{cok}

\DeclareMathOperator{\chr}{char}

\DeclareMathOperator{\GL}{GL}

\newcommand{\into}{\hookrightarrow}

\newcommand{\weq}{\xrightarrow{\sim}}
\newcommand{\cofib}{\rightarrowtail}
\newcommand{\fib}{\twoheadrightarrow}


\def\uk{\underline{\km}}
\def\utimes{\underline{\otimes}}
\def\uHom{\underline{\mathcal{H} om}}

\def\oO{\overline{O}}

\def\Gr{{\EuScript Gr}}
\def\Gras{\Gr}

\def\Mod{\text{-}\mathrm{Mod}}
\def\rMod{\mathrm{Mod}\text{-}}

\def\mod{\text{-}\mathrm{mod}}
\def\rmod{\mathrm{mod}\text{-}}

\DeclareMathOperator{\Loc}{Loc}
\DeclareMathOperator{\Perv}{Perv}

\DeclareMathOperator{\ad}{ad}

\DeclareMathOperator{\U}{U}

\def\d{\mathbf d}
\def\gl{\mathfrak{gl}}
\def\ic{\mathbf{IC}}
\def\Grdbar{\overline{\Gr^\d}}
\def\Grcon{(\Gr^{\varpi_1})^{*d}}
\def\Olambar{\overline{\OC_\lambda}}
\def\Olamtil{\widetilde{\OC_\lambda}}

\newcommand{\TMp}{\:{}^\backprime\mathbb{T}}

\DeclareMathOperator{\res}{res}
\newcommand{\resp}{\mathop{{}^\backprime\mathrm{res}}\nolimits}
\DeclareMathOperator{\ind}{ind}

\newcommand{\indLG}{\ind_L^G}
\newcommand{\resLG}{\res_L^G}
\newcommand{\respLG}{\resp_L^G}
\def\sgn{\mathrm{sgn}}

\def\multiset#1#2{\ensuremath{\left(\kern-.3em\left(\genfrac{}{}{0pt}{}{#1}{#2}\right)\kern-.3em\right)}}

% *** quiver ***
% A package for drawing commutative diagrams exported from https://q.uiver.app.
%
% This package is currently a wrapper around the `tikz-cd` package, importing necessary TikZ
% libraries, and defining a new TikZ style for curves of a fixed height.
%
% Version: 1.1.0
% Authors:
% - varkor (https://github.com/varkor)
% - AndréC (https://tex.stackexchange.com/users/138900/andr%C3%A9c)

\NeedsTeXFormat{LaTeX2e}
\ProvidesPackage{quiver}[2021/01/11 quiver]

% `tikz-cd` is necessary to draw commutative diagrams.
\RequirePackage{tikz-cd}
% `amssymb` is necessary for `\lrcorner` and `\ulcorner`.
\RequirePackage{amssymb}
% `calc` is necessary to draw curved arrows.
\usetikzlibrary{calc}
% `pathmorphing` is necessary to draw squiggly arrows.
\usetikzlibrary{decorations.pathmorphing}

% A TikZ style for curved arrows of a fixed height, due to AndréC.
\tikzset{curve/.style={settings={#1},to path={(\tikztostart)
    .. controls ($(\tikztostart)!\pv{pos}!(\tikztotarget)!\pv{height}!270:(\tikztotarget)$)
    and ($(\tikztostart)!1-\pv{pos}!(\tikztotarget)!\pv{height}!270:(\tikztotarget)$)
    .. (\tikztotarget)\tikztonodes}},
    settings/.code={\tikzset{quiver/.cd,#1}
        \def\pv##1{\pgfkeysvalueof{/tikz/quiver/##1}}},
    quiver/.cd,pos/.initial=0.35,height/.initial=0}

% TikZ arrowhead/tail styles.
\tikzset{tail reversed/.code={\pgfsetarrowsstart{tikzcd to}}}
\tikzset{2tail/.code={\pgfsetarrowsstart{Implies[reversed]}}}
\tikzset{2tail reversed/.code={\pgfsetarrowsstart{Implies}}}
% TikZ arrow styles.
\tikzset{no body/.style={/tikz/dash pattern=on 0 off 1mm}}

% Hyperlinks for bibliography
%\newcommand{\arxiv}[1]{\href{http://arxiv.org/abs/#1}{\tt arXiv:\nolinkurl{#1}}}
%\newcommand{\doi}[1]{\href{http://dx.doi.org/#1}{{\tt DOI:#1}}}
%\newcommand{\euclid}[1]{\href{http://projecteuclid.org/euclid.cmp/#1}{{\tt #1}}}
%\newcommand{\mathscinet}[1]{\href{http://www.ams.org/mathscinet-getitem?mr=#1}{\tt #1}}
%\newcommand{\googlebooks}[1]{(preview at \href{http://books.google.com/books?id=#1}{google books})}



%%%%%%%%%%%%%%%%%%%%%%%%%%%%%%%%%%%%%%%%%%%%%%%%%%%%%%%%%%%%%%%%%%%%%%%%%%%%%%%%%%%%%%%%%%%%%%%%%%%%%%%%%%%%%%%%%%%%%%%%%%%%%%%%%%%%%%%%%%%%%%%%%%%%%%%%%%%%%%
\title[Stratifications of abelian categories]{Stratifications of abelian categories}



\author{Giulian Wiggins}
%\address{School of Mathematical and Physical Sciences\\ Macquarie University \\ Australia}
%\email{g.wiggins@maths.usyd.edu.au}
%\thanks{
%This article revises and extends part of the author's PhD thesis [Wig23]. This work was supported by Australian Research Council grant DP19010243.
%}






\begin{document}

\maketitle


\begin{abstract} 
This paper studies abelian categories that can be decomposed into smaller abelian categories via iterated recollements - such a decomposition we call a {\em stratification}. Examples include the categories of (equivariant) perverse sheaves and $\e$-stratified categories (in particular highest weight categories) in the sense of Brundan-Stroppel \cite{BS18}.

We give necessary and sufficient conditions for an abelian category with a stratification to be equivalent to a category of finite dimensional modules of a finite dimensional algebra - this generalizes the main result of Cipriani-Woolf \cite{CW21}. Furthermore, we give necessary and sufficient conditions for such a category to be $\e$-stratified - this generalizes the characterisation of highest weight categories given by Krause \cite{Kra17a}.
\end{abstract}



%\newpage
%
%\tableofcontents
%
%\newpage



\section{Introduction}\label{intro}



A {\em recollement} of abelian categories is a short exact sequence of abelian categories
\[
\begin{tikzcd}
	{0} & {\AC_Z} & \AC & {\AC_U} & {0}
	\arrow[from=1-1, to=1-2]
	\arrow["{i_*}", from=1-2, to=1-3]
	\arrow["{j^*}", from=1-3, to=1-4]
	\arrow[from=1-4, to=1-5]
\end{tikzcd}
\]
in which $\AC_Z$ is a Serre subcategory of $\AC$ (with Serre quotient $\AC_U$), $i_*$ has both a left and right adjoint, and $j^*$ has fully-faithful left and right adjoints. 
% In this case we say that $\AC$ is a {\em gluing} of $\AC_{Z}$ and $\AC_U$.
We usually denote such a recollement of abelian categories by the  diagram
\[
\begin{tikzcd}
	{\AC_Z} && \AC && {\AC_U}
	\arrow["{i_*}", from=1-1, to=1-3]
	\arrow["{i^{*}}"', shift right=5, from=1-3, to=1-1]
	\arrow["{i^{!}}", shift left=5, from=1-3, to=1-1]
	\arrow["{j^*}", from=1-3, to=1-5]
	\arrow["{j_{!}}"', shift right=5, from=1-5, to=1-3]
	\arrow["{j_{*}}", shift left=5, from=1-5, to=1-3]
\end{tikzcd}
\]
where $(i^*, i_*, i^!)$ and $(j_!, j^*, j_*)$ are adjoint triples.

This definition is motivated by recollements of triangulated categories as defined by Beilinson, Bernstein and Deligne \cite{BBD82} to generalize  Grothendieck's six functors relating the constructible derived category, $\DC (X, \km)$, of sheaves on a variety $X$ (with coefficients in a field $\km$) and the constructible derived categories, $\DC (Z, \km)$ and $\DC(U, \km)$, of sheaves on a closed subvariety $Z \subset X$ and open complement $U := X \backslash Z$ i.e. the situation:
\[
\begin{tikzcd}
	{\DC (Z, \km)} && \DC (X, \km) && {\DC (U, \km)}
	\arrow["{i_{*}}", from=1-1, to=1-3]
	\arrow["{i^{*}}"', shift right=5, from=1-3, to=1-1]
	\arrow["{i^{!}}", shift left=5, from=1-3, to=1-1]
	\arrow["{j^{*}}", from=1-3, to=1-5]
	\arrow["{j_{!}}"', shift right=5, from=1-5, to=1-3]
	\arrow["{j_{*}}", shift left=5, from=1-5, to=1-3]
\end{tikzcd}
\]
Here, $i \colon Z \into X$ is the closed embedding and $j \colon U \into X$ is the complimentary open embedding.

Note that if $\SC h (X, \km)$ is the category of sheaves on $X$, and ${\HC}^{i} \colon \DC (X, \km) \to \SC h (X, \km)$ is the $i$-th cohomology functor, then the following is an example of a recollement of abelian categories.
\[
\begin{tikzcd}
	{\SC h (Z, \km)} && {\SC h (X, \km)} && {\SC h (U, \km)}
	\arrow["{i_{*}}", from=1-1, to=1-3]
	\arrow["{{\HC}^{0} i^{*}}"', shift right=5, from=1-3, to=1-1]
	\arrow["{{\HC}^{0} i^{!}}", shift left=5, from=1-3, to=1-1]
	\arrow["{j^{*}}", from=1-3, to=1-5]
	\arrow["{{\HC}^{0} j_{!}}"', shift right=5, from=1-5, to=1-3]
	\arrow["{{\HC}^{0} j_{*}}", shift left=5, from=1-5, to=1-3]
\end{tikzcd}
\]
More generally, given a recollement of triangulated categories with compatible $t$-structures\footnote{Given a recollement of triangulated categories $\DC_{Z}$, $\DC$, $\DC_{U}$, we consider $t$-structures on these categories to be {\em compatible} if 
\begin{align*}
\DC^{\leq 0} &= \{
X \in \DC ~|~ i^{*}X \in \DC^{\leq 0}_{Z}, \quad j^* X \in \DC_{U}^{\leq 0}
\}, \\
\DC^{\geq 0} &= \{
X \in \DC ~|~ i^{!}X \in \DC^{\geq 0}_{Z}, \quad j^! X \in \DC_{U}^{\geq 0}
\} .
\end{align*}
 }, there is a canonical recollement of abelian categories on the hearts of the $t$-structure \cite[Proposition 1.4.16]{BBD82} defined using the zero-th cohomology functor arising from the $t$-structure (as in the above example).

In representation theory, recollements of abelian categories arise from modules of algebras $A$ with an idempotent $e$. In this situation there is a recollement
\[
\begin{tikzcd}
	{0} & {A / AeA \Mod} & A \Mod & {eAe \Mod} & {0}
	\arrow[from=1-1, to=1-2]
	\arrow["{i_*}", from=1-2, to=1-3]
	\arrow["{j^*}", from=1-3, to=1-4]
	\arrow[from=1-4, to=1-5]
\end{tikzcd}
\]
where $i_* \colon A / AeA \Mod \to A \Mod $ is the restriction functor, and $j^* \colon A \Mod \to eAe \Mod$ is the functor sending an $A$-module $M$ to the $eAe$-module $eM$.

Recollements often arise as part of an iterated collection of recollements, giving a kind of filtration of a larger category by smaller categories. Consider, for example, the category, $P_{\L} (X, \km)$, of perverse sheaves that are constructible with respect to a stratification, $X := \bigcup_{\l \in \L}X_{\l}$, of a variety $X$. 
%Regard the set $\L$ as a poset with the closure order: $\l \leq \mu$ if $X_{\l} \subset \overline{X_{\mu}}$. 
For each strata $X_{\l}$, there is a recollement of abelian categories
\[
\begin{tikzcd}
	{0} & {P_{\L} (\overline{X_{\l}} \setminus X_{\l}, \km)} & {P_{\L} (\overline{X_{\l}}, \km)} & {\Loc^{ft} (X_{\l}, \km)} & {0}
	\arrow[from=1-1, to=1-2]
	\arrow[from=1-2, to=1-3]
	\arrow[from=1-3, to=1-4]
	\arrow[from=1-4, to=1-5]
\end{tikzcd}
\]
where $\Loc^{ft} (X_{\l}, \km)$ is the category of local systems on $X_{\l}$ of finite type. 
%More generally, if $\L' \subset \L$ is a lower\footnote{A subposet $\L' \subset \L$ is {\em lower} if for any $\l, \mu \in \L$, $\mu \leq \l$ and $\l \in \L'$ implies $\mu \in \L'$.} subposet in which $\l \in \L'$ is maximal, and $X_{\L'} \subset X$ is the closed subvariety $X_{\L'} := \bigcup_{\l \in \L'} X_{\l}$, then the following is a recollement of abelian categories.
More generally, if $\L' \subset \L$ is such that $X_{\L'} := \bigcup_{\l \in \L'} X_{\l}$ is a closed subvariety of $X$, and $X_{\l}$ is open in $X_{\L'}$, then the following is a recollement of abelian categories.
\[
\begin{tikzcd}
	{0} & {P_{\L} (X_{\L' \setminus \{\l\} }, \km)} & {P_{\L} (X_{\L'}, \km)} & {\Loc^{ft} (X_{\l}, \km)} & {0}
	\arrow[from=1-1, to=1-2]
	\arrow[from=1-2, to=1-3]
	\arrow[from=1-3, to=1-4]
	\arrow[from=1-4, to=1-5]
\end{tikzcd}
\]

Similar iterations of recollement occur in highest weight categories (as defined in \cite{CPS88b}). Indeed, let $\AC$ be a highest weight category with respect to a poset $\L$, and let $\{ \D_{\l} ~|~ \l \in \L \}$ be the collection of standard objects in $\AC$. For each lower\footnote{A subposet $\L' \subset \L$ is {\em lower} if for any $\l, \mu \in \L$, $\mu \leq \l$ and $\l \in \L'$ implies $\mu \in \L'$.} subposet $\L' \subset \L$, let $\AC_{\L'}$ be the Serre subcategory generated by objects $\{ \D_{\l} ~|~ \l \in \L' \}$. For each lower subposet $\L' \subset \L$ and  maximal $\l \in \L'$ there is a recollement of abelian categories
\[
\begin{tikzcd}
	{0} & {\AC_{\L' \setminus \{\l\}}} & {\AC_{\L'}} & {\End_{\AC} (\D_{\l})^{op} \Mod} & {0}
	\arrow[from=1-1, to=1-2]
	\arrow[from=1-2, to=1-3]
	\arrow["{j^*}", from=1-3, to=1-4]
	\arrow[,from=1-4, to=1-5]
\end{tikzcd}
\]
where $j^* := \Hom(\D_{\l}, -): \AC_{\L'} \to \End_{\AC} (\D_{\l})^{op} \Mod$.

We unify these examples by the following notion.

\begin{manualdefinition}{\ref{stratification}}
A  {\em stratification} of an abelian category $\AC$ by a non-empty poset $\L$ consists of the following data
\begin{enumerate}
\item[(i)]
For each lower subposet $\L' \subset \L$, define a Serre subcategory, $\AC_{\L'}$, of $\AC$. 
\item[(ii)]
For each  $\l \in \L$, define an abelian category $\AC_{\l}$. These we call {\em strata categories}.
\end{enumerate}
This data must satisfy the conditions
\begin{enumerate}
\item
$\AC_{\emptyset} = 0$ and $\AC_{\L} = \AC$.
\item
For each pair of lower subposets $\L_{1}' \subset \L_{2}' \subset \L$, there are inclusions of Serre categories 
$ \AC_{\L_{1}'} \into \AC_{\L_{2}'}$.
\item
For each lower subposet $\L' \subset \L$, and maximal $\l \in \L'$ there is a recollement
\[
\begin{tikzcd}
	{0} & {\AC_{\L' \setminus \{\l\}}} & \AC_{\L'} & {\AC_{\l}} & {0}
	\arrow[from=1-1, to=1-2]
	\arrow[from=1-2, to=1-3]
	\arrow[from=1-3, to=1-4]
	\arrow[from=1-4, to=1-5]
\end{tikzcd}
\]
\end{enumerate}
\end{manualdefinition}

For example, the category $P_{\L} (X, \km)$ has a stratification by the poset $\L$ with closure order: $\l \leq \mu$ if $X_{\l} \subset \overline{X_{\mu}}$. Further examples include equivariant perverse sheaves, highest weight categories, and  $\e$-stratified categories in the sense of Brundan and Stroppel \cite{BS18}.

Recall that, for a field $\km$, a $\km$-linear abelian category $\AC$ is {\em finite over $\km$} if $\AC$ is equivalent to a category of finite dimensional modules of a finite dimensional $\km$-algebra.
This paper is inspired by a result of Cipriani and Woolf \cite[Corollary 5.2]{CW21} that says that a category of perverse sheaves (with coefficients in a field) on a space stratified by finitely many strata is finite if and only if the same is true for each category of finite type local systems on each stratum. We extend this result by showing that an abelian category with a stratification by a finite poset is finite if and only if the same if true of all strata categories (Corollary \ref{recolenoughprojcor1}). Moreover, we give necessary and sufficient conditions for a finite abelian category to be $\e$-stratified (Theorem \ref{eStratTheorem}). This result specializes to give necessary and sufficient conditions for a finite abelian category to be standardly stratified in the sense of Cline, Parshall, Scott \cite{CPS96}, and further specialises to recover a characterisation of highest weight categories due to Krause \cite{Kra17a}.



%%%%%%%%%%%%%%%%%%%%%%%%%%%%%%%%%%%%%%%%%%%%%%%%%%%%%%%%%%%%%%%%%%%%%%%%%%
 %%%%%%%%%%%%%%%%%%%%%%%%%%%%%%%%%%%%%%%%%%%%%%%%%%%%%%%%%%%%%%%%%%%%%%%
 \subsection{Outline and explanation of main results}
Let $\AC$ be an abelian category with a stratification by a poset $\L$. For each $\l \in \L$, define the Serre quotient functor $j^{\l} \colon \AC_{\{\mu \in \L ~|~ \mu \leq \l \}} \to \AC_{\l}$, and let $j_{!}^{\l} \colon \AC_{\l} \to \AC_{\{\mu \in \L ~|~ \mu \leq \l \}}$ and $j_{*}^{\l} : \AC_{\l} \to \AC_{\{\mu \in \L ~|~ \mu \leq \l \}}$ be the left and right adjoints of $j^{\l}$. By a slight abuse of notation, write $j_{!}^{\l} \colon \AC_{\l} \to \AC$ and $j_{*}^{\l} \colon \AC_{\l} \to \AC$ for the functors obtained by postcomposing with the inclusion functor $\AC_{\{\mu \in \L ~|~ \mu \leq \l \}} \into \AC$.

In Section \ref{recprel} we recall some basic features of recollements and stratifications, and describe examples of stratifications of abelian categories.

In Section \ref{recintext} we define, for each $\l \in \L$, the {\em intermediate-extension functor} $j^{\l}_{!*}: \AC_{\l} \to \AC$:
\[
j_{!*}^{\l} X := \im(\overline{1_X}:  j_{!}^{\l} X \to j_{*}^{\l} X ),
\]
where $\overline{1_X}$ is the morphism corresponding to the identity $1_X$ under the natural isomorphism
\[
\Hom_{\AC} (j_{!}^{\l} X, j_{*}^{\l} X) \simeq \Hom_{\AC_{\l}} ( X, j^{\l} j_{*}^{\l} X) \simeq \Hom_{\AC_{\l}} ( X,  X) .
\]
A result of  Kuhn \cite[Proposition 4.6]{Kuh94} (see also Proposition \ref{recol4})  is that every simple object $L \in \AC$ is of the form $j^{\l}_{!*} L_{\l}$, for a unique (up to isomorphism) simple object $L_{\l} \in \AC_{\l}$ and unique $\l \in \L$. Proposition \ref{recol41} says that if $\AC$ is an abelian category with a stratification by a finite poset, then every object in $\AC$ has a finite filtration by simple objects if and only if the same is true of all the strata categories. 

%The proofs here are almost identical to the standard proofs of these results in the theory of perverse sheaves (see e.g. \cite[Chapter 3]{Ach21}).

In Section \ref{rechom} we prove the main result of the paper (Theorem \ref{recolenoughproj}) that describes when the middle category in a recollement has enough projectives.

\begin{manualtheorem}{\ref{recolenoughproj}}
Consider a recollement:
\[
\begin{tikzcd}
	{0} & {\AC_Z} & \AC & {\AC_U} & {0}
	\arrow[from=1-1, to=1-2]
	\arrow["{i_*}", from=1-2, to=1-3]
	\arrow["{j^*}", from=1-3, to=1-4]
	\arrow[from=1-4, to=1-5]
\end{tikzcd}
\]
Suppose $\AC_U$ and $\AC_Z$ have finitely many simple objects and every object has a finite filtration by simple objects. Suppose moreover that for any simple objects 
 $A, B$ in $\AC$,
\[
\dim_{\End_{\AC} (B)} \Ext^{1}_{\AC} (A, B) < \infty .
\]
Then $\AC$ has enough  projectives if and only if both $\AC_U$ and $\AC_Z$ have enough projectives.\footnote{Since $\End_{\AC} (B)$ is a division ring, any $\End_{\AC} (B)$-module is free. For an $\End_{\AC} (B)$-module $M$, $\dim_{\End_{\AC} (B)} M$ is the rank of $M$ as a free $\End_{\AC} (B)$-module.}
\end{manualtheorem}

%Recall that, for a field $\km$, a $\km$-linear abelian category $\AC$ is {\em finite over $\km$} if $\AC$ is equivalent to a category of finite dimensional modules of a finite dimensional $\km$-algebra.
A category $\AC$ is finite over $\km$ if and only if $\AC$ has finite dimensional $\Hom$-spaces, finitely many simple objects, enough projectives, and every object has finite filtration by simple objects. Such categories also have finite dimensional $\Ext$-spaces (see e.g. Proposition \ref{finExt}). In particular, the following corollary follows from 
Theorem \ref{recolenoughproj}.


\begin{manualcorollary}{\ref{recolenoughprojcor}}
For any field $\km$, a $\km$-linear abelian category with a stratification by a finite poset is finite if and only if the same is true for all strata categories.
\end{manualcorollary}


%Corollary \ref{recolenoughprojcor} generalises a result of Cipriani-Woolf \cite[Theorem 4.6]{CW21} that says that a category of perverse sheaves (with coefficients in a field) on a space stratified by finitely many strata is finite if and only if the same is true for each category of finite type local systems on each stratum. 


To state the results in Sections \ref{standardsec} and \ref{BrunStropSection}, let $\AC$ have enough projectives and injectives, finitely many simple objects, and suppose every object in $\AC$ has finite length. Suppose $\AC$ has a stratification by a finite poset $\L$. Let $B$ be a set indexing the simple objects in $\AC$ (up to isomorphism) and write $L(b)$ (respectively $P(b)$, $I(b)$) for the simple (respectively projective indecomposable, injective indecomposable) object corresponding to $b \in B$.

Define the {\em stratification function }
\[
\rho: B \to \L
\]
 that maps each $b \in B$ to the corresponding $\l \in \L$ in which $L(b) = j^{\l}_{!*} L_{\l} (b)$ for some simple object $L_{\l} (b) \in \AC_{\l}$. 
 Write $P_{\l} (b)$ and $I_{\l} (b)$ for the projective cover and injective envelope of $L_{\l} (b)$ in $\AC_{\l}$. 
 
For $b \in B$ and $ \l = \r (b)$, define the {\em standard} and {\em costandard} objects
\[
\D (b) := j^{\l}_! P_{\l} (b), \qquad \nabla (b) := j^{\l}_* I_{\l} (b).
\]
%The following result follows from the proof of Theorem \ref{recolenoughproj}.
%
%\begin{manualporism}{\ref{recolenoughprojcor}}
%Every projective indecomposable, $P(b)$, in $\AC$ has a filtration by quotients of standard objects, $\D (b')$, in which $\rho (b') \geq \rho (b)$. Dually, every injective indecomposable, $I(b)$, of $\AC$ has a filtration by subobjects of costandard objects, $\nabla (b')$, in which $\rho (b') \geq \rho (b)$.
%\end{manualporism}

Porism \ref{standards} says that every projective indecomposable, $P(b)$, in $\AC$ has a filtration by quotients of standard objects, $\D (b')$, in which $\rho (b') \geq \rho (b)$. Dually, every injective indecomposable, $I(b)$, of $\AC$ has a filtration by subobjects of costandard objects, $\nabla (b')$, in which $\rho (b') \geq \rho (b)$. This result follows from the construction of projective indecomposable objects given in the proof of Theorem \ref{recolenoughproj}.
%This result follows from the  proof of Theorem \ref{recolenoughproj}. More precisely, 

Standard and costandard objects play a crucial role in representation theoretic applications of stratifications of abelian categories, where one often requires that projective and/or injective indecomposable objects have filtrations by standard and/or costandard objects. For example, both of these conditions are required in Cline-Parshall-Scott's definition of highest weight category \cite{CPS88b}. 

Categories whose projective indecomposables have filtrations by standard objects have been widely studied - beginning with the work of Dlab \cite{Dla96} and Cline, Parshall and Scott \cite{CPS96}. 
Categories in which both projective objects have a filtration by standard objects and injective objects have a filtration by costandard objects have been studied by various authors (see e.g. \cite{Fri07}, \cite{CZ19}, \cite{LW15}). Brundan and Stroppel \cite{BS18} (based on the work of \cite{ADL98}) define a general framework called an {\em $\e$-stratified category} that includes these situations as special cases. 
We recall this definition now. 

For a {\em sign function} $\e \colon \Lambda
\to \{+,-\}$, define
the {\em $\e$-standard} and {\em $\e$-costandard objects}
\begin{align*}
\D_\e(b) &:= \left\{
\begin{array}{ll}
j^{\l}_! P_{\l} (b)&\text{if $\e(\l) = +$}\\
j^{\l}_! L_{\l}(b)&\text{if $\e(\l) = -$}
\end{array}
\right.,&
\nabla_\e(b) &:= \left\{
\begin{array}{ll}
 j^{\l}_* L_{\l}(b)&\text{if $\e(\l) = +$}\\
 j^{\l}_* I_{\l} (b)&\text{if $\e(\l) = -$}
\end{array}
\right.,
\end{align*}
where $\l = \rho (b)$.



%Categories in which projective and/or injective indecomposable objects have filtrations by standard and/or costandard objects have been widely studied, begining with Cline-Parshall-Scott's definition of highest weight category in \cite{CPS88b}. Categories whose projective indecomposables have filtrations by standard objects where originally studied by Dlab \cite{Dla96} and Cline, Parshall and Scott \cite{CPS96}, where these are called {\em standardly stratified categories}. Categories in which both projective objects have a filtration by standard objects and injective objects have a filtration by costandard objects have been studied by various authors (see e.g. \cite{Fri07}, \cite{CZ19}, \cite{LW15}). These situations fit into the  general  framework of $\e$-stratified categories (defined by Brundan and Stroppel \cite{BS18}). 
Brundan and Stroppel \cite{BS18} say that $\AC$  is {\em $\e$-stratified} if the following equivalent conditions hold:
\begin{enumerate}
\item
Every projective indecomposable, $P(b)$, has a filtration by $\e$-standard objects, $\D_{\e} (b')$, in which $\rho (b') \geq \rho (b)$. 
\item
Every injective indecomposable, $I(b)$, has a filtration by $\e$-costandard objects, $\nabla_{\e} (b')$, in which $\rho (b') \geq \rho (b)$.
\end{enumerate}

Note that $\AC$ is a {\em highest weight category} if and only if each $\AC_{\l}$ is semisimple and $\AC$ is $\e$-stratified for any (and all) $\e \colon \L \to \{+,-\}$.


The following result gives a criterion for a finite abelian category to be $\e$-stratified.

\begin{manualtheorem}{\ref{eStratTheorem}}
A finite abelian category $\AC$ with a stratification by a finite poset $\L$ is $\e$-stratified (for a function $\e \colon \L \to \{+,-\}$) if and only if the following conditions hold:
\begin{enumerate}
\item
For each inclusion of Serre subcategories $i_* \colon \AC_{\L'} \to \AC_{\L}$, and objects $X,Y \in \AC_{\L'}$, there is a natural isomorphism 
$
\Ext^{2}_{\AC_{\L'}} (X,Y) \simeq  \Ext^{2}_{\AC} (i_*X,i_*Y) .
$
\item
For each $\l \in \L$,
\begin{enumerate}
\item
If $\e (\l) = +$ then $j^{\l}_* : \AC_{\l} \to \AC$ is exact.
\item
If $\e (\l) = -$ then $j^{\l}_! : \AC_{\l} \to \AC$ is exact.
\end{enumerate}
\end{enumerate}
\end{manualtheorem}

Recollements satisfying Condition (1) are known as {\em 2-homological recollements}. These are studied by Psaroudakis in \cite{Psa13}.

One application of Theorem \ref{eStratTheorem} is a criterion for a category of (equivariant) perverse sheaves to be $\e$-stratified - although we remark that checking these conditions (particularly the first condition) is difficult in general. 
%Examples in which an abelian category with a stratification is shown to be a highest weight category include  \cite[Theorem 3.3.1]{BGS96}, \cite[Theorem 6.8]{BM19}, \cite[Theorem 7.2]{BR21}, and \cite[Theorem 5.2]{Gou22}. 

\subsection{Acknowledgement}
This article revises and extends part of the author's PhD thesis \cite{Wig23}. This paper has benefited from discussions with Oded Yacobi, Kevin Coulembier, and feedback from an examiner of \cite{Wig23}. This work was supported by Australian Research Council grant DP190102432.
%The author declares no completing interests.
 
 
 %%%%%%%%%%%%%%%%%%%%%%%%%%%%%%%%%%%%%%%%%%%%%%%%%%%%%%%%%%%%%%%%%%%%%%%%%%%%%%%
%%%%%%%%%%%%%%%%%%%%%%%%%%%%%%%%%%%%%%%%%%%%%%%%%%%%%%%%%%%%


\section{Preliminaries}\label{recprel} 


We begin with a definition of recollement. The notation used in this definition will be used throughout the paper.

\begin{defn}\label{recdef}

A {\em recollement of abelian categories} consists of three abelian categories $\AC_Z$, $\AC$ and $\AC_U$ and functors:
\begin{equation}\label{recollement0}
\begin{tikzcd}
	{\AC_Z} && \AC && {\AC_U}
	\arrow["{i_*}", from=1-1, to=1-3]
	\arrow["{i^{*}}"', shift right=5, from=1-3, to=1-1]
	\arrow["{i^{!}}", shift left=5, from=1-3, to=1-1]
	\arrow["{j^*}", from=1-3, to=1-5]
	\arrow["{j_{!}}"', shift right=5, from=1-5, to=1-3]
	\arrow["{j_{*}}", shift left=5, from=1-5, to=1-3]
\end{tikzcd}
\end{equation}
satisfying the conditions:
\begin{enumerate}
\item[(R1)]
$(i^*, i_*, i^!)$ and $(j_!, j^*, j_*)$ are adjoint triples.
\item[(R2)]
The functors $i_*$, $j_!$, $j_*$ are fully-faithful. Equivalently the adjunction maps $i^* i_* \to \id \to i^! i_*$ and $j^* j_* \to \id \to j^* j_!$ are isomorphisms.
\item[(R3)]
The functors satisfy  $j^* i_* = 0$ (and so by adjunction $i^* j_! = 0 = i^! j_*$).
%\[
%\Hom(j_! A, i_* B) = 0 = \Hom(i_* A, j_* B)
%\]
\item[(R4)]
The adjunction maps produce exact sequences for each object $X \in \AC$:
\begin{align}
j_! j^* X \to &X \to i_* i^* X \to 0 \label{adex1}\\
0 \to i_* i^! X \to &X \to j_* j^* X \label{adex2}
\end{align}
\end{enumerate}
Alternatively, condition (R4) can be replaced by the condition
\begin{enumerate}
\item[(R4')]
For any object $X \in \AC$, if $j^* X = 0$ then X is in the essential image of $i_*$.
\end{enumerate}


A {\em recollement of triangulated categories} is defined in the same way as a recollement of abelian categories except that condition (R4) is replaced by the existence of the triangles:
\begin{align}
j_! j^* X \to &X \to i_* i^* X \to  \\
 i_* i^! X \to &X \to j_* j^* X \to
\end{align}
for each object $X$.
%
%Given a recollement of abelian/triangulated categories with objects and morphisms as in (\ref{recollement}) we say that the category $\AC$ is a {\em gluing} of the categories $\AC_Z$ and $\AC_U$.
\end{defn}

\begin{remark}
The interchangibility of (R4) and (R4') follows from the following argument. If  $j^* X = 0$ then (R4) implies that $i_* i^! X \simeq X \simeq i_* i^* X$ and so $X$ is in the essential image of $i_*$. Conversely let $\mu: j_! j^* \to \id$ and $\eta: \id \to i_* i^*$ be the adjunction natural transformations. Then there is a commutative diagram
\[
\begin{tikzcd}
	{j_! j^* X} & X & {\cok \mu_X} & 0 \\
	0 & {i_* i^*X} & {i_* i^*(\cok \mu_X)} & 0
	\arrow["{\mu_X}", from=1-1, to=1-2]
	\arrow[from=1-2, to=1-3]
	\arrow["{\eta_{j_! j^* X}}"', from=1-1, to=2-1]
	\arrow["{\eta_X}"', from=1-2, to=2-2]
	\arrow["{\eta_{\cok \mu_X}}", from=1-3, to=2-3]
	\arrow[from=2-1, to=2-2]
	\arrow[from=2-2, to=2-3]
	\arrow[from=1-3, to=1-4]
	\arrow[from=2-3, to=2-4]
\end{tikzcd}
\]
in which the rows are exact. By applying $j^*$ to the top row we see that $j^* \cok \mu_X = 0$ and so (R4') implies that $\cok \mu_X \simeq i_* i^*(\cok \mu_X) \simeq i_* i^*X$. Equation (\ref{adex2}) holds by a similar argument.
\end{remark}

Write $\AC^Z$ for the essential image of $i_*$. To reconcile Definition \ref{recdef} with the definition of recollement in Section \ref{intro}, note that by (R2), $\AC^Z \simeq \AC_Z$, and by (R4'), $\AC^Z$ is the kernel of the exact functor $j^*$ and is hence a Serre subcategory of $\AC$.


It will be useful to note that if we extend the sequences (\ref{adex1}) and (\ref{adex2}) to exact sequences
\begin{align}
0 \to K \to j_! j^* X \to &X \to i_* i^* X \to 0 \label{adex3}\\
0 \to i_* i^! X \to &X \to j_* j^* X \to K'  \to 0\label{adex4}
\end{align}
then $K$ and $K'$ are in $\AC^Z$. Indeed, by applying the exact functor $j^*$ to (\ref{adex3}) we get that $j^* K = 0$ and so $i_* i^! K \simeq K \simeq i_* i^* K$.  Likewise by applying $j^*$ to (\ref{adex4}) we get that $K' \in \AC^Z$.

Given a recollement of abelian or triangulated categories with objects and morphisms as in (\ref{recollement0}), the opposite categories form the following recollement
\begin{equation*}
\begin{tikzcd}
	{\AC_{Z}^{op}} && \AC^{op} && {\AC_{U}^{op}}
	\arrow["{i_{*}}", from=1-1, to=1-3]
	\arrow["{i^{!}}"', shift right=5, from=1-3, to=1-1]
	\arrow["{i^{*}}", shift left=5, from=1-3, to=1-1]
	\arrow["{j^{*}}", from=1-3, to=1-5]
	\arrow["{j_{*}}"', shift right=5, from=1-5, to=1-3]
	\arrow["{j_{!}}", shift left=5, from=1-5, to=1-3]
\end{tikzcd}
\end{equation*}
which we call the {\em opposite recollement}. 

The following proposition describes a useful way to characterise the functors $i^*$ and $i^!$ in any recollement.

\begin{prop}\label{recol1}
Let $\AC$ be an abelian category with a recollement as in (\ref{recollement0}). Then for any object $X \in \AC$:
\begin{enumerate}
\item[(i)]
$i_* i^{*} X$ is the largest quotient object of $X$ in $\AC^Z$.
\item[(ii)]
$i_* i^{!} X$ is the largest subobject of $X$ in $\AC^Z$.
\end{enumerate}
\end{prop}

\begin{proof}
By the adjunction $(i_*, i^*)$ and since $i_*$ is fully-faithful we have natural isomorphisms for $X \in \AC$, $Y \in \AC_Z$:
\[
\Hom_{\AC}(i_* i^* X, i_* Y)
\simeq
\Hom_{\AC_Z}(i^* X, Y)
\simeq
\Hom_{\AC} (X, i_* Y)
\]
sending $f$ to $f \circ \eta$ where $\eta : X \to i_* i^* X$ is the adjunction unit. In particular any morphism $X \to i_* Y$ factors through $ i_* i^* X$. Statement (i) follows. 
Statement (ii) follows by taking the opposite recollement.
\end{proof}


%Recall that a poset $\L$ is called {\em well-founded} if every subset $\L' \subset \L$ has at least one minimal element i.e. an element $m$ in which there are no $m' \in \L'$ satisfying $m' \leq m$. This is equivalent to requiring that $\L$ has no infinite descending chains. Two important examples are finite posets and the inclusion poset for closed sets in an irreducible topological space with Zariski topology. 
%%More generally, topological spaces whose closed sets form a well-founded poset are called {\em Noetherian topological spaces}.
%
%A useful property of well-founded posets is the principle of {\em Noetherian induction}. That is, to check that a property $P(\l)$ is true for any element $\l$ in a well-founded poset $\L$ it is enough to show the following properties:
%\begin{itemize}
%\item[(i)]
%$P(\l)$ is true for all minimal elements of $\l$.
%\item[(ii)]
%For every non-minimal $\l \in \L$, if $P(\mu)$ is true for any $\mu < \l$ then $P(\l)$ is true.
%\end{itemize}

Say that a subset $\L'$ of a poset $\L$ is {\em lower} if for any $\l \in \L'$, if $\mu \leq \l$ then $\mu \in \L'$.

%Any poset $\L$ induces a posetal category $\L_{cat}$ whose objects are elements of $\L$, and with one morphism $f_{\l \leq \mu}: \l \to \mu$ wherever $\l \leq \mu$ (composition of morphisms is defined in the unique possible way).

\begin{defn}\label{stratification}
A {\em stratification} of an abelian/triangulated category $\AC$ by a non-empty poset $\L$ consists of the following data:
\begin{enumerate}
\item[(i)]
An assignment of an abelian/triangulated category $\AC_{\L'}$ for every lower $\L' \subset \L$, and for lower subsets $\L'' \subset \L' \subset \L$ an embedding $i_{\L'', \L'*}: \AC_{\L''} \into \AC_{\L'}$.
\item[(ii)]
For each $\l \in \L$ an abelian/triangulated category $\AC_{\l}$. We call these {\em strata categories}.
\end{enumerate}
This data must satisfy the following conditions
\begin{enumerate}
\item[(S1)]
$\AC_{\emptyset} = 0$ and $\AC_{\L} =\AC$.
\item[(S2)]
For each $\l \in \L$ and lower subset $\L' \subset \L$ in which $\l \in \L'$ is maximal, the functor $i_* = i_{\L' \backslash \{ \l \},\L' *}$ fits into a recollement 
\[
\begin{tikzcd}
	{\AC_{\L' \backslash \{\l\}}} && \AC_{\L'} && {\AC_{\l}}
	\arrow["{i_{*}}", from=1-1, to=1-3]
	\arrow["{i^{*}}"', shift right=5, from=1-3, to=1-1]
	\arrow["{i^{!}}", shift left=5, from=1-3, to=1-1]
	\arrow["{j^{*}}", from=1-3, to=1-5]
	\arrow["{j_{!}}"', shift right=5, from=1-5, to=1-3]
	\arrow["{j_{*}}", shift left=5, from=1-5, to=1-3]
\end{tikzcd}
\]
\item[(S3)]
If $\L'' \subset \L'$ are lower subsets of $\L$, and $\l \in \L$ is a maximal element of both $\L''$ and $\L'$, then the following diagram of functors commutes
\[
% https://q.uiver.app/?q=WzAsMyxbMCwwLCJcXEFDX3tcXEwnJ30iXSxbMSwwLCJcXEFDX3tcXEwnfSJdLFsxLDEsIlxcQUNfe1xcbH0iXSxbMCwxLCJpX3tcXEwnLCBcXEwnJyAqfSIsMCx7InN0eWxlIjp7InRhaWwiOnsibmFtZSI6Imhvb2siLCJzaWRlIjoidG9wIn19fV0sWzAsMiwiIiwyLHsic3R5bGUiOnsiaGVhZCI6eyJuYW1lIjoiZXBpIn19fV0sWzEsMiwiIiwwLHsic3R5bGUiOnsiaGVhZCI6eyJuYW1lIjoiZXBpIn19fV1d
\begin{tikzcd}
	{\AC_{\L''}} & {\AC_{\L'}} \\
	& {\AC_{\l}}
	\arrow["{i_{\L', \L'' *}}", hook, from=1-1, to=1-2]
	\arrow[from=1-1, to=2-2]
	\arrow[from=1-2, to=2-2]
\end{tikzcd}
\]
\end{enumerate}
\end{defn}

%Recall that a poset $\L$ is {\em well-founded} if $\L$ has no infinite descending chains.
%%i.e. each subset $\L' \subset \L$ has an element $m$ in which there are no $m' \in \L'$ satisfying $m' \leq m$. 
%Two important examples are finite posets and the inclusion poset for closed sets in an irreducible topological space with Zariski topology. 
%In this paper we only study stratifications of abelian categories by well-founded posets.


\begin{remark}
For an abelian category $\AC$, consider a poset $\L$ whose cardinality, $|\L|$, is maximal among all posets $\L'$ in which $\AC$ has a stratification by $\L '$. The value $| \L|$ is equal to the {\em stratification dimension} of $\AC$, as defined by Psaroudakis \cite[Definition 4.19]{Psa13}.
\end{remark}



We proceed with some important examples of recollements and stratifications.
%We will often use without mentioning a result of Beilinson-Bernstein-Deligne \cite[Proposition 1.4.16]{BBD82} that says that given a recollement of triangulated categories with a $t$-structure one obtains a recollement on the hearts of the $t$-structure by applying zero-th cohomology.
%A proof of this result is given in Section \ref{rectstruct}.


%\begin{ex}[Direct sum category]\label{recex00}
%The simplest example of a recollement is the direct sum of abelian categories:
%\[
%\begin{tikzcd}
%	{\AC} && \AC \oplus \BC && {\BC}
%	\arrow["{i_{*}}", from=1-1, to=1-3]
%	\arrow["{i^{*}}"', shift right=5, from=1-3, to=1-1]
%	\arrow["{i^{!}}", shift left=5, from=1-3, to=1-1]
%	\arrow["{j^{*}}", from=1-3, to=1-5]
%	\arrow["{j_{!}}"', shift right=5, from=1-5, to=1-3]
%	\arrow["{j_{*}}", shift left=5, from=1-5, to=1-3]
%\end{tikzcd}
%\]
%where $i_*$ and $j_! = j_*$ are the inclusion functors, and $i^* = i^!$ and $j^*$ are projection functors.
%\end{ex}



%\begin{ex}[Grothendieck's derived functors]\label{recex0} Let $X$ be a variety. Let $i \colon Z \into X$ be the inclusion of a closed subvariety and let $j \colon U \into X$ be the inclusion of the open complement. Write $\DC^b (X, \km)$ for the bounded derived category of sheaves on $X$ with coefficients in a Noetherian ring $\km$ of finite global dimension.
%%field can be replaced with Noethian ring with finite global dimension and the following category will still be triangulated
%The direct image $i_* \colon \DC^b (Z , \km) \into \DC^b (X, \km)$ fits into a recollement of triangulated categories
%\[
%\begin{tikzcd}
%	{\DC^b (Z, \km)} && \DC^b (X, \km) && {\DC^b (U, \km)}
%	\arrow["{i_{*}}", from=1-1, to=1-3]
%	\arrow["{i^{*}}"', shift right=5, from=1-3, to=1-1]
%	\arrow["{i^{!}}", shift left=5, from=1-3, to=1-1]
%	\arrow["{j^{*}}", from=1-3, to=1-5]
%	\arrow["{j_{!}}"', shift right=5, from=1-5, to=1-3]
%	\arrow["{j_{*}}", shift left=5, from=1-5, to=1-3]
%\end{tikzcd}
%\]
%Indeed this is the original and motivating example of a recollement of triangulated categories.
%After taking zero-th cohomology we get a recollement of abelian categories of sheaves:
%\[
%\begin{tikzcd}
%	{\SC h (Z, \km)} && \SC h (X, \km) && {\SC h (U, \km)}
%	\arrow["{i_{*}}", from=1-1, to=1-3]
%	\arrow["{{\HC}^{0} i^{*}}"', shift right=5, from=1-3, to=1-1]
%	\arrow["{{\HC}^{0} i^{!}}", shift left=5, from=1-3, to=1-1]
%	\arrow["{j^{*}}", from=1-3, to=1-5]
%	\arrow["{ {\HC}^{0} j_{!}}"', shift right=5, from=1-5, to=1-3]
%	\arrow["{{\HC}^{0} j_{*}}", shift left=5, from=1-5, to=1-3]
%\end{tikzcd}
%\]
%Likewise if we take zero-th perverse cohomology for any perversity function, we get a recollement of the abelian categories of perverse sheaves.
%% (this is a consequence of Theorem \ref{BBDtheorem}).
%\end{ex}

\begin{ex}[Constructible sheaves with respect to a stratification]\label{recex1}
A {\em stratification} of a quasiprojective complex variety $X$ is a finite collection, $\{X_{\l} \}_{\l \in \L}$, of disjoint, smooth, connected, locally closed subvarieties, called {\em strata}, in which
$X = \bigcup_{\l \in \L} X_{\l}$ and 
for each $\l \in \L$, $\overline{X_{\l}}$ is a union of strata. In this case we equip $\L$ with the partial order
\[
\mu \leq \l  \text{ if $X_{\mu} \subset \overline{X_{\l}}$.}
\]
We use $\L$ to refer to the stratification of $X$.

%Important examples include the decomposition of a Grassmannian manifold (or more generally a flag variety) into Schubert varieties.!!FIX!!

For a variety $X$, let $\Loc^{ft} (X, \km)$ be the category of local systems on $X$ of finite type with coefficients in a field $\km$. Recall that, by taking monodromy, $\Loc^{ft} (X, \km)$ is equivalent to the category, $\km[\pi_1 (X_{\l})] \mod_{fg}$, of finitely generated $\km[\pi_1 (X_{\l})]$-modules (see e.g. \cite[Theorem 1.7.9]{Ach21}).

Say that a sheaf $\FC$ on $X$ is {\em constructible} with respect to a stratification, $\L$, of $X$ if $\FC |_{X_{\l}}$ is a local system of finite type for each $\l \in \L$. Write $\DC^{b}_{\L} (X, \km)$ for the full triangulated subcategory of $\DC^{b} (X, \km)$ consisting of objects $\FC$ in which $H^k (\FC)$ is constructible with respect to $\L$.

Say that a stratification, $\L$, of $X$ is {\em good} if for any $\l \in \L$ and any object $\LC \in \Loc^{ft} (X_{\l}, \km)$, we have $j_{\l *} \LC \in \DC^{b}_{\L} (X, \km)$, where $j_{\l}: X_{\l} \into X$ is the embedding, and $j_{\l *}$ is the derived pushforward. It is difficult to tell in general whether a stratification is good (see \cite[Remark 2.3.21]{Ach21} for a discussion of these dificulties). A stratification satisfying the {\em Whitney regularity conditions} \cite{Wit65} is good. In particular, if an algebraic group $G$ acts on $X$ with finitely many orbits (each connected), then the stratification of $X$ by $G$-orbits is good (see e.g. \cite[Exercise 6.5.2]{Ach21}).
%Geordie's Illustrated Guide to Perverse Sheaves 

%Let $cl(\L)$ be the set of lower subsets of a poset $\L$. 
Given a good stratification $\L$ on $X$, the triangulated category $\DC^{b}_{\L} (X, \km)$ has a stratification by $\L$ with strata categories 
\[
\DC_{\l} := \DC^b (\Loc^{ft}(X_{\l}, \km)) \simeq \DC^{b} (\km[\pi_1 (X_{\l})] \mod_{fg})
\]
 and Serre subcategories $\DC_{\L'} := \DC^{b}_{\L'} (\bigcup_{\l \in \L'} X_{\l})$ for each lower $\L' \subset \L$.

For a perversity function $p: \L \to \ZM$, the category ${}^p P_{\L} (X, \km)$ of perverse sheaves on $X$ with respect to the stratification $\L$ (and perversity function $p$) is the full subcategory of $\DC^{b}_{\L} (X , \km)$ consisting of complexes $\FC$ in which for any strata $h_{\l}: X_{\l} \into X$:
\begin{itemize}
\item[(i)]
 $\HC^d (h_{\l}^* \FC) = 0$ if $d > p(\l)$,
\item[(ii)]
 $\HC^d (h_{\l}^! \FC) = 0$ if $d < p(\l)$,
\end{itemize}
where $\HC^d (\FC)$ refers to the $d$-th cohomology sheaf of $\FC$. The category $\AC = {}^p P_{\L} (X, \km)$ is abelian and has a stratification by $\L$, with strata categories 
\[
\AC_{\l} = \Loc^{ft}(X_{\l}, \km) [p(\l)] \simeq \km[\pi_1 (X_{\l})] \mod_{fg}.
\]
%(see e.g. Theorem \ref{BBDtheorem}).
\end{ex}

\begin{ex}[$G$-equivariant perverse sheaves]
%Another example of a stratification arises in the theory of equivariant perverse sheaves as defined in \cite{BL94}. 
%We will briefly review this theory. We recommend the reader consult \cite[Chapter 6]{Ach21} for more details.

%!!!Let $G$ be a connected algebraic group acting on a variety $X$ and let $\s, pr_2: G \times X \to X$ be the action map $\s (g,x) = g \cdot x$ and projection map $pr_2 (g,x) = x$. The category $P_G (X , \km)$, of $G$-equivariant perverse sheaves on $X$ is equivalent to the full subcategory of $P (X , \km)$ consisting of objects $\FC$ in which 
%$pr_{2}^* \FC [\dim G] \simeq \s^* \FC [\dim G]$ in $P (G \times X , \km)$.!!!

For a complex algebraic group $G$ and quasiprojective complex $G$-variety $X$, a {\em $G$-equivariant perverse sheaf} on $X$ is, roughly speaking, a perverse sheaf on $X$ with a $G$-action compatible with the $G$-action on $X$ (see e.g. \cite[Definition 6.2.3]{Ach21} for a precise definition). The category, $P_G (X, \km)$, of $G$-equivariant perverse sheaves is the heart of a $t$-structure on the {\em $G$-equivariant derived category}, $\DC_G (X, \km)$ defined by Bernstein-Lunts \cite{BL94}. For a $G$-equivariant map of $G$-varieties $h: X \to Y$, there are equivariant versions of the (proper) pushforward and (proper) pullback functors: $h_*, h_!, h^!, h^*$. If $i \colon Z \into X$ is the inclusion of a $G$-invariant closed subvariety with open complement $j \colon U \into X$, then there is a recollement of triangulated categories
\[
\begin{tikzcd}
	{\DC_{G}^b (Z, \km)} && \DC_{G}^b (X, \km) && {\DC_{G}^b (U, \km)}
	\arrow["{i_{*}}", from=1-1, to=1-3]
	\arrow["{i^{*}}"', shift right=5, from=1-3, to=1-1]
	\arrow["{i^{!}}", shift left=5, from=1-3, to=1-1]
	\arrow["{j^{*}}", from=1-3, to=1-5]
	\arrow["{j_{!}}"', shift right=5, from=1-5, to=1-3]
	\arrow["{j_{*}}", shift left=5, from=1-5, to=1-3]
\end{tikzcd}
\]


 If $X$ is a homogeneous $G$-variety, then every $G$-equivariant perverse sheaf is a finite type local system (shifted by $\dim_{\CM} X$). Moreover, in this case,
\begin{equation}
P_G (X , \km) \simeq \km [G^{x} / (G^{x})^{\circ}] \mod_{fg},
\end{equation}
where $G^x \subset G$ is the stabilizer of a point $x \in X$, and $(G^{x})^{\circ}$ is the connected component of $G^{x}$ containing the identity element (see e.g. \cite[Proposition 6.2.13]{Ach21} for a proof of this statement).  

Suppose $G$ acts on $X$ with finitely many orbits (each connected). Let $\L$ be a set indexing the set of $G$-orbits in $X$, and write $\OC_{\l}$ for the orbit corresponding to $\l \in \L$. Consider $\L$ as a poset with the closure order: $\l \leq \mu$ if $\OC_{\l} \subset \overline{\OC_{\mu}}$. Then the category $\AC = P_{G} (X, \km)$ has a stratification with strata categories 
\[
\AC_{\l} = P_G (\OC_{\l} , \km) \simeq \km [G^{x_{\l}} / (G^{x_{\l}})^{\circ}] \mod_{fg},
\]
where $x_{\l} \in \OC_{\l}$.
% Achar exercise 6.5.2

\end{ex}

\begin{ex}[Modules with idempotents]\label{recex2}
For a ring $A$, let $\rMod A$ be the category of all right $A$-modules, and $\rmod A$ be the category of finitely presented right $A$-modules. Let  $e$ be an idempotent in $A$, and define the inclusion functor $i_* \colon  \rMod A / AeA \to \rMod A$. Note that $\rMod A / AeA$ is equivalent to the Serre subcategory of $\rMod A$ consisting of modules annihilated by $e$. There is a corresponding Serre quotient $j^*: \rMod A \to \rMod eAe $ defined
\[
j^* := \Hom_{A} (eA, -) \simeq - \otimes_A Ae .
\]
i.e. $j^* M = Me$ for any object $M \in \rMod A$.  These functors fit into a recollement of abelian categories
\[
\begin{tikzcd}
	{\rMod A / AeA} &&  \rMod A && {\rMod eAe}
	\arrow["{i_{*}}", from=1-1, to=1-3]
	\arrow["{i^{*}}"', shift right=5, from=1-3, to=1-1]
	\arrow["{i^{!}}", shift left=5, from=1-3, to=1-1]
	\arrow["{j^{*}}", from=1-3, to=1-5]
	\arrow["{j_{!}}"', shift right=5, from=1-5, to=1-3]
	\arrow["{j_{*}}", shift left=5, from=1-5, to=1-3]
\end{tikzcd}
\]
where for any right $A$-module $M$:
\begin{enumerate}
\item[(i)]
$i^* M$ is the largest quotient, $N$,  of $M$ in which $Ne = 0$.
\item[(ii)]
$i^! M$ is the largest subobject, $N$, of $M$ in which $Ne = 0$.
\end{enumerate}
Moreover $j_! := - \otimes_{eAe} eA$ and $j_* := \Hom_{eAe} (Ae, -)$.

If $A$ is right artinian and has enough injectives then the inclusion $i_* \colon \rmod A / AeA \to \rmod A$ fits into a recollement 
\[
\begin{tikzcd}
	{\rmod A / AeA} &&  \rmod A && {\rmod eAe}
	\arrow["{i_{*}}", from=1-1, to=1-3]
	\arrow["{i^{*}}"', shift right=5, from=1-3, to=1-1]
	\arrow["{i^{!}}", shift left=5, from=1-3, to=1-1]
	\arrow["{j^{*}}", from=1-3, to=1-5]
	\arrow["{j_{!}}"', shift right=5, from=1-5, to=1-3]
	\arrow["{j_{*}}", shift left=5, from=1-5, to=1-3]
\end{tikzcd}
\]
in which $j^*$ has left adjoint $j_! = - \otimes_{eAe} eA$ (see e.g \cite[Lemma 2.5]{Kra17a}). 
%To construct the left adjoint functor $j_*$ of $j^*$, first consider an injective cogenerator $E$ of $\rmod A$. Then $\rmod A \simeq ( \rmod \End_A (E))^{op}$ and $\rmod eAe \simeq ( \rmod \End_A (eEe))^{op}$. Under these equivalences the !!!!

\end{ex}

%Many algebras !!!
%Consider a poset $\L$ as a quiver with path algebra $A := \km[\L]$ and let $\{e_{\l }\}_{\l \in \L}$ be the set of idempotents in $A$ indexed by vertices. The category $\rmod A$ has a stratification with Serre subcategories 
%\[
%A_{\L'} := A / (\sum_{\l \notin \L'} A e_{\l} A)
%\]
%and strata categories $A_{\l} := e_{\l} A e_{\l} \simeq \rmod \km$.
%
%More generally, consider an algebra $A$ has a generating set of elements $\{e_{\l}\}_{\l \in \L}$ indexed by a poset $\L$ and satisfying the properties:
%\begin{itemize}
%\item[(A1)]
%$e_{\l}^2 - e_{\l} \in \sum_{\mu > \l} A e_{\mu} A$.
%\item[(A2)]
%$e_{\mu} A e_{\l} = 0$ if $\mu \ngeq \l$.
%\item[(A2)]
%If $\l \nleq \mu$ and $\mu \nleq \l$ then $e_{\l} A e_{\mu} = 0$.
%\end{itemize}
%Then $\rMod A$ has a stratification by $\L$ with Serre subcategories $A_{\L'} := A / (\sum_{\l \notin \L'} A e_{\l} A)$.
%
%Reference CPS-stratified algebras https://agostoni.web.elte.hu//papers/centralizer.pdf
%\end{ex}
%
%
%
%
%\begin{ex}
%
%For a finite poset $\L$, say that an algebra $A$ is {\em $\L$-stratifiable} is $A$ has a generating set of elements $\{e_{\l}\}_{\l \in \L}$ in which
%\begin{itemize}
%\item[(1)]
%$e_{\l}^2 - e_{\l} \in \sum_{\mu > \l} A e_{\mu} A$
%\item[(2)]
%$e_{\l} A e_{\mu} = 0$ if $\mu \nleq \l$.
%\end{itemize}
%For example, if we consider the poset $\L$ as a quiver, it's path algebra $\km[\L]$ is $\L$-stratifiable when $\{\e_{\l}\}_{\l \in \L}$ is the set of idempotents indexed by the vertices. 
%%Categories of modules for quasi-hereditary algebras can be defined by iterations of such recollements.
%\begin{itemize}
%\item[(1)]
%$e_{\l}^2 - e_{\l} \in $
%\item[(2)]
%$e_{\mu} A e_{\l} = 0$ if $\mu \ngeq \l$.
%\end{itemize}
%\end{ex}
%
%
%\begin{ex}[Quasi-hereditary algebras]\label{recex21}
%Let $\km$ be a field and $A$ a finite dimensional $\km$-algebra. Recall that $A$ is {\em quasi-hereditary} if there is an idempotent $e \in A$ in which $AeA$ is projective, $eAe$ is semisimple, and $A/AeA$ is either $0$ or quasihereditary.
%\end{ex}
%
%\begin{ex}[Highest weight categories]\label{recex3}
%
%An important class of stratifications are the highest weight categories. We recall their definition now.
%
%Let $\km$ a field.
%Let $\SC$ be a finite dimenisional Schurian $\km$-algebra (i.e. $\End_{\SC} (L) = \km$ for every simple $\SC$-module $L$) in which $\SC \Mod$ has enough injectives. 
%
%Let $\{ L_{\l} ~|~ \l \in \L^{+}\}$ be a complete set of nonisomorphic simple $\SC$-modules. Write $P_{\l}$ and $I_{\l}$ for the projective cover and injective envelope of $L_{\l}$. Given a poset $\leq$ on the set $\L^{+}$, define the {\em standard module}, $\D_{\l}$, to be the largest quotient of $P_{\l}$ with all composition factors in $\{L_{\l} ~|~  \mu \leq \l\}$. The {\em costandard module}, $\nabla_{\l}$, is the largest submodule of $I_{\l}$ whose composition factors are in $\{L_{\l} ~|~  \mu \leq \l\}$.
%
%Say that $\SC \Mod$ is a {\em highest weight category} if the following equivalent conditions are satisfied for all $\l \in \L^{+}$:
%\begin{itemize}
%\item[(1)]
%The kernel of the quotient $P_{\l} \to \D_{\l}$ has a filtration by standard modules $\D_{\mu}$ satisfying $\mu > \l$.
%\item[(2)]
%The quotient $I_{\l}/ \nabla_{\l}$ has a filtration by costandard modules $\nabla_{\mu}$ satisfying $\mu > \l$.
%\end{itemize}
%In this case we call the pair $(\SC, \L^{+})$ a quasi-hereditary algebra. 
%
%\begin{remark}
%Why conditions (1) and (2) are equivalent: Use injective cogenerator to give equivalence of categories with opposite of module category: SHOULD WE BE LOOKING AT FINITELY GENERATED MODULES?
%\end{remark}
%
%
%
%We now show that every highest weight category is a stratfication.
%Conversely, suppose that $\AC$ is a highest weight category. For a closed subset $Z \subset \L$ define $\AC_Z$ to be the Serre subcategory of $\AC$ generated by standard objects $\D_{\l}$, for $\l \in Z$. Equivalently, $\AC_Z$ is the full subcategory of $\AC$ with objects !!!! 
%Define the categories $\SC_{\l} := \rmod \End_{\AC} (\D_{\l})$. The recollement !! is defined !!
%
%\end{ex}

\begin{ex}[Macpherson-Vilonen construction]\label{recex01} 
Macpherson and Vilonen \cite{MV86} define the category of perverse sheaves on a stratified variety via iterations of a formal method for constructing recollements of an abelian categories We recall this formal construction here.


Let $\AC_{Z}$, $\AC_{U}$ be abelian categories, $F: \AC_{U} \to \AC_{Z}$ be a right exact functor, $G: \AC_U \to \AC_Z$ be a left exact functor, and let $\e \colon F \to G$ be a natural transformation. 

%Macpherson and Vilonen \cite{MV86} give a method for constructing recollements of abelian categories. Namely, given two abelian categories $\AC_{Z}$, $\AC_{U}$, define {\em MV-data} for $\AC_{Z}, \AC_U$ to be a collection $(F,G, \e)$, where $F: \AC_{U} \to \AC_{Z}$ is a right exact functor, $G: \AC_U \to \AC_Z$ is a left exact functor, and $\e \colon F \to G$ is a natural transformation.
%%Consider the following situation:
%%% https://q.uiver.app/?q=WzAsMixbMCwwLCJcXEFDX1oiXSxbMiwwLCJcXEFDX1UiXSxbMSwwLCJGIiwyLHsiY3VydmUiOjJ9XSxbMSwwLCJHIiwwLHsiY3VydmUiOi0yfV0sWzIsMywiXFxlIiwwLHsic2hvcnRlbiI6eyJzb3VyY2UiOjIwLCJ0YXJnZXQiOjIwfX1dXQ==
%%\[\begin{tikzcd}
%%	{\AC_Z} && {\AC_U}
%%	\arrow[""{name=0, anchor=center, inner sep=0}, "F"', curve={height=12pt}, from=1-3, to=1-1]
%%	\arrow[""{name=1, anchor=center, inner sep=0}, "G", curve={height=-12pt}, from=1-3, to=1-1]
%%	\arrow["\e", shorten <=3pt, shorten >=3pt, Rightarrow, from=0, to=1]
%%\end{tikzcd}\]
%%in which $F$ is right exact and $G$ is left exact.


Macpherson and Vilonen \cite{MV86} define a category, $\AC (\e)$, whose objects are tuples $(X_U, X_Z, \a, \b)$, where $(X_U,X_Z) \in \Obj \AC_U \times \Obj \AC_Z$ and $\a : F(X_U) \to X_Z$, $\b: X_Z \to G(X_U)$ are morphisms in $\AC_Z$ in which the following diagram commutes
\[\begin{tikzcd}
	{F(X_U)} && {G(X_U)} \\
	& {X_Z}
	\arrow["\e_{X_U}", from=1-1, to=1-3]
	\arrow["\a"', from=1-1, to=2-2]
	\arrow["\b"', from=2-2, to=1-3]
\end{tikzcd}\]
A morphism $(X_U, X_Z, \a, \b) \to (X_U', X_Z', \a', \b')$ is a pair
\[
f = (f_U : X_U \to X_U', f_Z: X_Z \to X_Z') \in \Mor \AC_U \times \Mor \AC_Z
\]
in which the following prism commutes:
% https://q.uiver.app/?q=WzAsNyxbMCwwLCJGKFhfVSkiXSxbMiwwLCJHKFhfVSkiXSxbMSwxLCJYX1oiXSxbMCwyLCJGKFhfe1V9JykiXSxbMiwyLCJHKFhfe1V9JykiXSxbMSwzLCJYX3tafSciXSxbMiw0XSxbMCwyLCJcXGEiLDJdLFsyLDEsIlxcYiIsMl0sWzAsMSwiXFxlX3tYX1V9Il0sWzAsMywiRihmX1UpIiwyXSxbMSw0LCJHKGZfVSkiXSxbMyw0LCJcXGVfe1hfe1V9J30iLDAseyJsYWJlbF9wb3NpdGlvbiI6MjB9XSxbMiw1LCJmX1oiLDAseyJsYWJlbF9wb3NpdGlvbiI6MjB9XSxbMyw1LCJcXGEnIiwyXSxbNSw0LCJcXGInIiwyXV0=
\[\begin{tikzcd}
	{F(X_U)} && {G(X_U)} \\
	& {X_Z} \\
	{F(X_{U}')} && {G(X_{U}')} \\
	& {X_{Z}'} \\
	&& {}
	\arrow["\a"', from=1-1, to=2-2]
	\arrow["\b"', from=2-2, to=1-3]
	\arrow["{\e_{X_U}}", from=1-1, to=1-3]
	\arrow["{F(f_U)}"', from=1-1, to=3-1]
	\arrow["{G(f_U)}", from=1-3, to=3-3]
	\arrow["{\e_{X_{U}'}}"{pos=0.2}, from=3-1, to=3-3]
	\arrow["{f_Z}"{pos=0.2}, crossing over, from=2-2, to=4-2]
	\arrow["{\a'}"', from=3-1, to=4-2]
	\arrow["{\b'}"', from=4-2, to=3-3]
\end{tikzcd}\]
%NB. The back face always commutes, so we don't need the whole prism in the definition.
Macpherson and Vilonen show \cite[Proposition 1.1]{MV86} that the category $\AC(\e)$ is abelian.
%{\em Construction of kernels}: Let $f: (X_U, X_Z, \a, \b) \to (X_U', X_Z', \a', \b')$ be a morphism in $\AC(\e)$. Consider the commutative diagram
%\[\begin{tikzcd}
%	{F (\ker f_U)} && {G (\ker f_U)} \\
%	{\ker F (f_U)} & {\ker f_Z} & {\ker G(f_U)}
%	\arrow[from=1-1, to=2-1]
%	\arrow["\a", from=2-1, to=2-2]
%	\arrow["\b", from=2-2, to=2-3]
%	\arrow["\simeq", from=1-3, to=2-3]
%	\arrow["{\e_{\ker f_U}}", from=1-1, to=1-3]
%\end{tikzcd}\]
%where the the vertical maps arise from the definition of the kernel, and the right vertical map is an isomorphism due to the left exactness of $G$. The kernel of $f$ is the corresponding diagram
%% https://q.uiver.app/?q=WzAsMyxbMCwwLCJGIChcXGtlciBmX1UpIl0sWzIsMCwiRyhcXGtlciBmX1UpIl0sWzEsMSwiXFxrZXIgZl9aIl0sWzAsMV0sWzAsMiwiXFxhXzAiLDJdLFsyLDEsIlxcYl8wIiwyXV0=
%\[\begin{tikzcd}
%	{F (\ker f_U)} && {G(\ker f_U)} \\
%	& {\ker f_Z}
%	\arrow["{\e_{\ker f_U}}", from=1-1, to=1-3]
%	\arrow["{\a_0}"', from=1-1, to=2-2]
%	\arrow["{\b_0}"', from=2-2, to=1-3]
%\end{tikzcd}\]
%
%% https://q.uiver.app/?q=WzAsNSxbMCwwLCJGIChcXGtlciBmX1UpIl0sWzAsMSwiXFxrZXIgRiAoZl9VKSJdLFsxLDEsIlxca2VyIGZfWiJdLFsyLDEsIlxca2VyIEcoZl9VKSJdLFsyLDAsIkcgKFxca2VyIGZfVSkiXSxbMCwxXSxbMSwyLCJcXGEiLDJdLFsyLDMsIlxcYiIsMl0sWzQsMywiXFxzaW1lcSJdLFswLDQsIlxcZV97XFxrZXIgZl9VfSJdLFswLDIsIlxcYV8wIl0sWzIsNCwiXFxiXzAiXV0=
%% \[\begin{tikzcd}
%% 	{F (\ker f_U)} && {G (\ker f_U)} \\
%% 	{\ker F (f_U)} & {\ker f_Z} & {\ker G(f_U)}
%% 	\arrow[from=1-1, to=2-1]
%% 	\arrow["\a"', from=2-1, to=2-2]
%% 	\arrow["\b"', from=2-2, to=2-3]
%% 	\arrow["\simeq", from=1-3, to=2-3]
%% 	\arrow["{\e_{\ker f_U}}", from=1-1, to=1-3]
%% 	\arrow["{\a_0}", from=1-1, to=2-2]
%% 	\arrow["{\b_0}", from=2-2, to=1-3]
%% \end{tikzcd}\]
Moreover they show that the category $\AC (\e)$ fits into a recollement
\[
\begin{tikzcd}
	{\AC_Z} && \AC(\e) && {\AC_U}
	\arrow["{i_*}", from=1-1, to=1-3]
	\arrow["{i^{*}}"', shift right=5, from=1-3, to=1-1]
	\arrow["{i^{!}}", shift left=5, from=1-3, to=1-1]
	\arrow["{j^*}", from=1-3, to=1-5]
	\arrow["{j_{!}}"', shift right=5, from=1-5, to=1-3]
	\arrow["{j_{*}}", shift left=5, from=1-5, to=1-3]
\end{tikzcd}
\]
in which
\begin{align*}
i_* (X_Z) = (0,X_Z,0,0), \qquad & j^* (X_U, X_Z, \a, \b) = X_U, \\
i^* (X_U, X_Z, \a, \b) = \cok \a, \qquad & j_! (X_U) = (X_U, F(X_U), 1_{F(X_U)}, \e_{X_U}), \\
i^! (X_U, X_Z, \a, \b) = \ker \b, \qquad & j_* (X_U) = (X_U, G(X_U), \e_{X_U}, 1_{G(X_U)}).
\end{align*}

%Macpherson and Vilonen \cite{MV86} use iterations of this formal construction to construct the category of perverse sheaves on a stratified variety. 

The functor $j_{!*} : \AC_U \to \AC (\e)$ defined $j_{!*} (X_U) = (X_U, \im \e_{X_U}, \e_{X_U}, 1)$ is a special case of an intermediate-extension functor defined in Definition \ref{intextfunctor} below.
Note that every simple object in $\AC (\e)$ is either of the form $i_* L$ for a simple object $L$ in $\AC_Z$, or of the form $j_{!*} L$ for a simple object in $\AC_U$. This is a special case of Proposition \ref{recol4} below.

Note that the functor $i_* \colon \AC_U \to \AC (\e)$ has an exact retraction $i^{!*} : \AC (\e) \to \AC_Z$ defined $i^{!*} (X_U, X_Z, \a, \b) = X_Z$.
This is a special feature of recollements built using Macpherson-Vilonen's construction - the functor $i_*$ does not usually have an exact retraction in a general recollement.

Franjou and Pirashvili \cite[Theorem 8.7]{FP04} give necessary and sufficient conditions for a recollement of abelian categories to be equivalent to a recollement built using the Macpherson-Vilonen construction.

%Macpherson and Vilonen \cite{MV86} use this construction to give a construction of perverse sheaves that 



\end{ex}



%%%%%%%%%%%%%%%%%%%%%%%%%%%%%%%%%%%%%%%%%%%%%%%%%%%%%%%%%

\section{The intermediate-extension functor}\label{recintext}

Consider again a recollement:
\begin{equation}\label{recollement}
\begin{tikzcd}
	{\AC_Z} && \AC && {\AC_U}
	\arrow["{i_{*}}", from=1-1, to=1-3]
	\arrow["{i^{*}}"', shift right=5, from=1-3, to=1-1]
	\arrow["{i^{!}}", shift left=5, from=1-3, to=1-1]
	\arrow["{j^{*}}", from=1-3, to=1-5]
	\arrow["{j_{!}}"', shift right=5, from=1-5, to=1-3]
	\arrow["{j_{*}}", shift left=5, from=1-5, to=1-3]
\end{tikzcd}
\end{equation}

In this section we study the full subcategory, $\AC^U \into \AC$, whose objects have no subobjects or quotients in $\AC^Z := \im i_*$. A result of Kuhn \cite[Proposition 4.6(3)]{Kuh94}
(see also Proposition \ref{recol3}) is that the restricted functor $j^* \colon \AC^U \to \AC_U$ is an equivalence of categories. The quasi-inverse $j_{!*}: \AC_U \to \AC^U$ is defined as follows.

\begin{defn}[$j_{!*}: \AC_U \to \AC^U$]\label{intextfunctor}
For an object $X \in \AC_U$, let $\overline{1_X}:  j_! X \to j_* X$ be the morphism corresponding to the identity on $X$ under the isomorphism
\[
\Hom_{\AC} (j_! X, j_* X) \simeq \Hom_{\AC_U} ( X, j^* j_* X) \simeq \Hom_{\AC_U} ( X,  X) .
\]
Define
\[
j_{!*} X := \im(\overline{1_X}:  j_! X \to j_* X ) \in \AC .
\]
\end{defn}
It is easy to check that if $X \in \AC_U$ then $j_{!*} X \in \AC^U$. Indeed as $i^! j_* X = 0$, $j_* X$ has no subobjects in $\AC^Z$. In particular, as $j_{!*} X$ is a subobject of $j_* X$ it cannot have any subobjects in $\AC^Z$. Likewise as $j_{!*} X$ is a quotient of $j_{!} X$ it cannot have any quotients in $\AC^Z$. 

We call the functor 
$
j_{!*}: \AC_U \to \AC
$
an {\em intermediate-extension functor}.
 

\begin{remark}
Not every subquotient of an object in $\AC^U$ need be in $\AC^U$. In particular, an object in $\AC^U$ may still have simple composition factors in $\AC^Z$.
\end{remark}

The following proposition is due to Kuhn \cite[Proposition 4.6(3)]{Kuh94}. We include a proof for completeness.

\begin{prop}\label{recol3}
Let $\AC$ be an abelian category with a recollement of abelian categories as in (\ref{recollement}).
Then 
$j^* \colon \AC^U \to \AC_U$ is an equivalence of categories with quasi-inverse
$j_{!*}: \AC_U \to \AC^U$.
\end{prop}

\begin{proof}
Since $\AC_U$ is a Serre quotient of $\AC$ by $\AC_Z$, if $X \in \AC$ has no nonzero quotient objects in $\AC^Z$, and $Y \in \AC$ has no nonzero subobjects in $\AC^Z$, then
\[
\Hom_{\AC}(X, Y) \simeq \Hom_{\AC_U} (j^* X, j^* Y) .
\]
In particular, $j^* \colon \AC^U \to \AC_U$ is fully-faithful. To show that $j^*$ is essentially surjective it suffices to show that for any object $X \in \AC_U$, $j^* j_{!*} X \simeq X$. Now, as $j^*$ is exact:
\[
j^* j_{!*} X = j^* \im (j_! X \to j_* X) \simeq \im (j^* j_! X \to j^* j_* X) \simeq \im (\id: X \to X) = X.
\]
The result follows.
\end{proof}


%\begin{prop}\label{recol3}
%Let $\AC$ be an abelian category with a recollement of abelian categories as in (\ref{recollement}).
%Then 
%\begin{enumerate}
%\item[(i)]
%If $X \in \AC$ has no nonzero quotient objects in $\AC^Z$, and $Y \in \AC$ has no nonzero subobjects in $\AC^Z$ (i.e. $i^* X = 0$ and $i^! Y = 0$), then
%\[
%\Hom_{\AC}(X, Y) \simeq \Hom_{\AC_U} (j^* X, j^* Y) .
%\]
%\item[(ii)]
%$j^* \colon \AC^U \to \AC_U$ is an equivalence of categories with quasi-inverse
%$j_{!*}: \AC_U \to \AC^U$.
%\end{enumerate}
%\end{prop}
%
%\begin{proof}
%If $i^*X = 0$ then (\ref{adex3}) gives an exact sequence 
%\[
%0 \to K \to j_! j^* X \to X \to 0
%\]
%in which $K \simeq i_* i^! K$.
%So applying $\Hom(-, Y)$ we get the exact sequence
%\[
%0 \to \Hom (X, Y) \to \Hom (j_! j^* X, Y) \to \Hom (i_* i^! K, Y) .
%\]
%Applying adjunctions gives the exact sequence
%\[
%0 \to \Hom (X, Y) \to \Hom ( j^* X, j^* Y) \to \Hom ( i^! K, i^!Y) .
%\]
%Statement (i) follows as $i^! Y = 0$.
%
%A corollary of statement (i) is that $j^* \colon \AC^U \to \AC_U$ is fully-faithful. 
%\end{proof}

If $\AC$ has a stratification by a finite poset $\L$, then for each $\l \in \L$, there is a functor $j^{\l}_{!*}: \AC_{\l} \to \AC$ defined by the composition 
\[\begin{tikzcd}
	{\AC_{\l}} & {\AC_{\{ \mu \in \L ~|~ \mu \leq \l \}}} & \AC
	\arrow["{j_{!*}}", from=1-1, to=1-2]
	\arrow[hook, from=1-2, to=1-3]
\end{tikzcd}\]


%The following result follows immediately from the previous proposition.

The following is an immediate consequence of \cite[Proposition 4.7]{Kuh94}. 
%We include a proof for completeness.


\begin{prop}\label{recol4}
Let $\AC$ be an abelian category with a stratification by a finite poset $\L$.
Every simple object $L \in \AC$ is of the form $j^{\l}_{!*} L_{\l}$, for a unique (up to isomorphism) simple object $L_{\l} \in \AC_{\l}$ and unique $\l \in \L$.
\end{prop}

\begin{proof}
Suppose $\AC$ fits into a recollement as in (\ref{recollement}).
By Proposition \ref{recol3}, if $L \in \AC_U$ is a simple object, then $j_{!*} L$ is a simple object in $\AC$.
Moreover all the simple objects of $\AC$ are either of the form $i_{*} L$ for a simple object $L \in \AC_Z$, or of the form $j_{ !*} L$ for a simple object $L \in \AC_U$.
The statement follows via an induction argument on $|\L|$.
\end{proof}

%\begin{prop}\label{recol4}
%Suppose we have a recollement of abelian categories as in (\ref{recollement}).
%If $L \in \AC_U$ is a simple object, then $j_{!*} L$ is a simple object in $\AC$.
%Moreover all the simple objects of $\AC$ are either of the forms:
%\begin{enumerate}
%\item[(i)]
%$i_{*} L$ for a simple object $L \in \AC_Z$.
%\item[(ii)]
%$j_{ !*} L$ for a simple object $L \in \AC_U$.
%\end{enumerate}
%%In particular, for any stratification of an abelian category $\AC$, the simple objects of $\AC$ are all and only the intermediate-extensions of the simple objects on the strata categories.
%\end{prop}

The following properties of the intermediate-extension functor will be useful. Statement (i) of Proposition \ref{recol31} is due to Kuhn \cite[Proposition 4.8]{Kuh94}.

\begin{prop}\label{recol31}
Let $\AC$ be an abelian category with a recollement of abelian categories as in (\ref{recollement}).
Then 
\begin{enumerate}
\item[(i)]
The functor $j_{!*}: \AC_U \to \AC$ maps injective morphisms to injective morphisms and surjective morphisms to surjective morphisms.
\item[(ii)]
If $X \in \AC$ has no nonzero quotient objects in $\AC^Z$ then there is a canonical short exact sequence
\[
0 \to i_* i^! X \to X \to j_{!*} j^{*} X \to 0
\]
\item[(iii)]
If $X \in \AC$ has no nonzero subobjects in $\AC^Z$ then there is a canonical short exact sequence
\[
0 \to j_{!*} j^{*} X \to X \to i_* i^* X \to 0
\]
\item[(iv)]
Let
\[
0 \to X' \to X \to X'' \to 0
\]
be an exact sequence in $\AC_U$. The object $j_{!*} X$ has a filtration 
\[
0 = X_0 \subset X_1 \subset X_2 \subset X_3 = j_{!*} X
\]
 in which $X_1 \simeq j_{!*} X'$, $X_3/X_2 \simeq j_{!*} X''$, and $X_2 / X_1 \in \AC^Z$.
\end{enumerate}
\end{prop}

\begin{proof}
Let $f: X \to Y$ be a map in $\AC_U$ and define objects $K_1$, $K_2$ in $\AC$ by the exact sequence
\[
0 \to K_1 \to j_{!*} X \to j_{!*} Y \to K_2 \to 0 
\]
To prove statement (i) it suffices to show that if $j^* K_i = 0$ then $K_i = 0$. If $j^* K_i = 0$ then by (R4'), $K_i \in \AC^Z$. It follows in either case that $K_i = 0$ since $j_{!*}X$ and $j_{!*} Y$ are in $\AC^U$.

To prove statement (ii), let $X \in \AC$ have no nonzero quotients in $\AC^Z$ and consider the short exact sequence 
\[
0 \to i_* i^! X \to X \to K \to 0 
\]
Applying $i^!$ to the sequence we see that $i^! K = 0$ and so $K \in \AC^U$. So $ K \simeq j_{!*} j^* K$ and (by applying $j^*$ to this sequence) $j^* X \simeq j^* K$. Statement (ii) follows immediately. The proof of statement (iii) is similar.

To show Statement (iv), let $0 \to X' \to X \to X'' \to 0$ be an exact sequence in $\AC_U$. Let $X_1$ be the image of the injection $j_{!*} X' \to j_{!*} X$ and let $X_2$ be the kernel of the surjection $j_{!*} X \to j_{!*} X''$. Then $X_1 \simeq j_{!*} X'$ and $X_3/X_2 \simeq j_{!*} X''$. Moreover $X_2 / X_1 \in \AC^Z$ since
\[
j^*(X_2 / X_1) \simeq j^* X_2 / j^* X_1 \simeq \ker (X \to X'') / \im (X' \to X) = 0 .
\]
\end{proof}


Say that an abelian category is a {\em length category} if every object has a finite filtration by simple objects.

\begin{prop}\label{recol41}
If $\AC$ is an abelian category with a stratification by a finite poset, then $\AC$ is a length category if and only if all the strata categories are length categories.
\end{prop}


\begin{proof}
Let $\AC$ be an abelian category fitting into a recollement of abelian categories as in (\ref{recollement}). 
It suffices to show that  $\AC$ is a length category if and only if both $\AC_Z$ and $\AC_U$ are length categories. The result follows from this statement by  induction.

Let $X$ be an object in $\AC$ and let $K$ be defined by the short exact sequence
\[
0 \to i_* i^! X \to X \to K \to 0
\]
Then $i^! K = 0$ and so applying Proposition \ref{recol31}(iii) we get the short exact sequence
\[
0 \to j_{!*} j^* K \to K \to i_* i^* K \to 0
\]
By Proposition \ref{recol31}(iv) and the exactness of $i_* : \AC_Z \to \AC$,  if every object in $\AC_Z$ and every object in $\AC_U$ has a finite filtration then so do both $j_{!*} j^* K$ and $i_* i^* K$. It follows that $K$ has a finite filtration and hence so does $X$.
The converse statement is obvious.
\end{proof}


% To end this section we state another result of Kuhn \cite[Proposition 4.8]{Kuh94} that will be needed in the next section.
%\begin{prop}\label{recol311}
%The functor $j_{!*}: \AC_U \to \AC$ maps injective morphisms to injective morphisms and surjective morphisms to surjective morphisms.
%\end{prop}
%%%%%%%%%%%%%%%%%%%%%%%%%%%%%%%%%%%%%%%%%%%%%%%%%%%%%%%%%%

\section{Recollements with enough projectives/injectives}\label{rechom}

In this section we study the relationship between projective covers of objects in the different categories of a recollement. More precisely, let $\AC$ be a category fitting into a recollement as in (\ref{recollement}). Proposition \ref{recol42} says that if $\AC$ has enough projectives/injectives then so does $\AC_U$. Proposition \ref{recol43} says that if $\AC$ is an abelian length category then if $\AC$ has enough projectives/injectives then so does $\AC_Z$. Proposition \ref{recol5} says that if $X \in \AC_U$ has a projective cover $P$ in $\AC_U$ then $j_! P$ is a projective cover in $\AC$ of $j_{!*} X$. 

Unfortunately it is not easy to find a projective cover in $\AC$ of an object $i_* X \in \AC^Z$, even if a projective cover of $X$ exists in $\AC_Z$. Theorem \ref{recolenoughproj} gives sufficient conditions for such a projective cover to exist. A consequence of Theorem \ref{recolenoughproj} (Corollary \ref{recolenoughprojcor}) is that a category $\AC$ with a stratification by a finite poset is equivalent to a category of finite dimensional modules of a finite dimensional algebra if and only if the same is true of the strata categories.


\subsection{Projective covers}

Recall that a surjection $\phi \colon X \to Y$ is {\em essential} if for any morphism $\a: X' \to X$, if $\phi \circ \a$ is surjective then $\a$ is surjective. Equivalently $\phi \colon X \to Y$ is essential if it's kernel, $\ker \phi$, is a superfluous submodule of $K$ i.e. for any subobject $U \subset X$, if $U + \ker \phi = X$ then $U = X$.

If $P \to X$ is an essential surjection and $P$ is projective then we call $P$ (or more accurately the morphism $P \to X$) a {\em projective cover} of $X$.  
%The projective cover of $X$ (if it exists) factors through every other essential cover of $X$, and is unique up to isomorphism \cite[Corollary 3.5]{Kra15}.
The projective cover of $X$ (if it exists)  is unique up to isomorphism (see e.g. \cite[Corollary 3.5]{Kra15}).

The dual concept of an essential surjection is called an {\em essential extension}. If $X \to I$ is an essential extension and $I$ is injective then this extension is called the {\em injective envelope} of $X$. 

If $L \in \AC$ is a simple object and $P$ is projective then $\phi \colon P \to L$ is a projective cover if and only if the following equivalent conditions hold:
\begin{itemize}
\item[(i)]
$\ker \phi$ is the unique maximal subobject of $P$.
\item[(ii)]
The endomorphism ring of $P$ is local.
%local means every element is either nilpotent or an isomorphism. If P is irreducible and finite dimensional, then by Fitting's Lemma the endomorphism ring is local. If endomorphism ring is local and $P = A \oplus B $, then map $P \to B \to P$ must be isomorphism or zero.
\end{itemize}
See e.g. \cite[Lemma 3.6]{Kra15}. The following additional characterisation of projective covers of simple objects will be useful.

\begin{prop}\label{KSproj}
Let $\AC$ be an abelian category. Let $P \in \AC$ be a projective object and $L \in \AC$ be a simple object. A map $P \to L$ is a projective cover if and only if for any simple object $L'$
\begin{equation}\label{dimensioncondition}
\dim_{\End_{\AC} (L')} \Hom_{\AC} (P, L') = 
\begin{cases}
1 &\text{ if $L = L'$,} \\
0 &\text{ otherwise.}
\end{cases}
\end{equation}
\end{prop}

\begin{remark}
For an object $B \in \AC$, if $\End_{\AC} (B)$ is a division ring then any $\End_{\AC} (B)$-module is free. In this case we write $\dim_{\End_{\AC} (B)} M$ for the rank of an $\End_{\AC} (B)$-module $M$.
\end{remark}

\begin{proof}[Proof of Proposition \ref{KSproj}]
Let $\phi \colon P \to L$ be a nonzero morphism from a projective object $P$ to simple object $L$. Then $\ker \phi$ is the unique maximal subobject of $P$ if and only if the following two statements hold:
\begin{enumerate}
\item
$\Hom_{\AC} (P, L') = 0$ whenever $L \neq L'$.  
\item
For any nonzero morphism $\phi': P \to L$, there is an isomorphism $f: \ker \phi \to \ker \phi'$ making the following diagram commute:
\[
\begin{tikzcd}
	{\ker \phi} & P \\
	{\ker \phi'} & P
	\arrow[from=1-1, to=1-2]
	\arrow["f"', from=1-1, to=2-1]
	\arrow["{\id_P}", from=1-2, to=2-2]
	\arrow[from=2-1, to=2-2]
\end{tikzcd}
\]
\end{enumerate}
Note that this diagram can be extended to a commutative diagram with exact rows
\begin{equation}\label{commutative}
\begin{tikzcd}
	0 & {\ker \phi} & P & L & 0 \\
	\\
	0 & {\ker \phi'} & P & L & 0
	\arrow[from=1-1, to=1-2]
	\arrow[from=1-2, to=1-3]
	\arrow["f"', from=1-2, to=3-2]
	\arrow["\phi", from=1-3, to=1-4]
	\arrow["{\id_P}"', from=1-3, to=3-3]
	\arrow[from=1-4, to=1-5]
	\arrow["{f'}", from=1-4, to=3-4]
	\arrow[from=3-1, to=3-2]
	\arrow[from=3-2, to=3-3]
	\arrow["{\phi'}"', from=3-3, to=3-4]
	\arrow[from=3-4, to=3-5]
\end{tikzcd}
\end{equation}
Moreover the existence of an an isomorphism $f: \ker \phi \to \ker \phi'$ making the left square commute in (\ref{commutative}) is equivalent to the existence of an isomorphism $f': L \to L$ making the right square of (\ref{commutative}) commute.
In particular, $(2)$ is equivalent to the statement
\begin{enumerate}
\item[$(2')$]
For any nonzero morphism $\phi': P \to L$, there is an isomorphism  $f': L \to L$ such that $\phi' = f' \circ \phi$.
\end{enumerate}
Statement $(2')$ is equivalent to the equation $\dim_{\End_{\AC} (L)} \Hom_{\AC} (P, L) =1$. The result follows.
\end{proof}

%
%\begin{proof}[Proof of Proposition \ref{KSproj}]
%Let $\phi \colon P \to L$ be a nonzero morphism from a projective object $P$ to simple object $L$. Then $\ker \phi$ is the unique maximal subobject of $P$ if and only if the following two statements hold:
%\begin{enumerate}
%\item
%$\Hom_{\AC} (P, L') = 0$ whenever $L \neq L'$.  
%\item
%For any nonzero morphism $\phi': P \to L$, there is are isomorphisms $f: \ker \phi \to \ker \phi'$ and $f': L \to L$ making the following diagram commute:
%\[
%\begin{tikzcd}
%	{\ker \phi} & P & L \\
%	{\ker \phi'} & P & L
%	\arrow[from=1-1, to=1-2]
%	\arrow["f"', from=1-1, to=2-1]
%	\arrow["\phi", from=1-2, to=1-3]
%	\arrow["{\id_P}"', from=1-2, to=2-2]
%	\arrow["{f'}", from=1-3, to=2-3]
%	\arrow[from=2-1, to=2-2]
%	\arrow["{\phi'}"', from=2-2, to=2-3]
%\end{tikzcd}
%\]
%\end{enumerate}
%Indeed the existence of an an isomorphism $f: \ker \phi \to \ker \phi'$ making the left square commute is equivalent to the existence of an isomorphism $f': L \to L$ making the right square commute.
%In particular, $(2)$ is equivalent to the saying that for any nonzero morphism $\phi': P \to L$, there is an isomorphism  $f': L \to L$ such that $\phi' = f' \circ \phi$. In other words, $\dim_{\End_{\AC} (L)} \Hom_{\AC} (P, L) =1$. The result follows.
%\end{proof}



An abelian category has {\em enough projectives} (resp. {\em enough injectives}) if every object has a projective cover (resp. injective envelope).

%An abelian category $\AC$ is a {\em Krull-Schmidt category} if every object in $\AC$ is a finite direct sum of objects with local endomorphism rings. For example, any abelian length category is a Krull-Schmidt category. 
%
%In a Krull-Schmidt category, the projective covers of simple objects are exactly the projective indecomposable objects.
%We will need the following well-known characterisation of projective covers of simple objects in Krull-Schmidt categories.

%\begin{prop}\label{KSproj}
%Let $\AC$ be a Krull-Schmidt category. Let $P \in \AC$ be a projective object and $L \in \AC$ be a simple object. A map $P \to L$ is a projective cover if and only if for any simple object $L'$
%\begin{equation}\label{dimensioncondition}
%\dim_{\End_{\AC} (L')} \Hom_{\AC} (P, L') = 
%\begin{cases}
%1 &\text{ if $L = L'$,} \\
%0 &\text{ otherwise.}
%\end{cases}
%\end{equation}
%\end{prop}
%
%\begin{remark}
%If $\End_{\AC} (B)$ is a division ring then any $\End_{\AC} (B)$-module is free. In this case we write $\dim_{\End_{\AC} (B)} M$ for the rank of an $\End_{\AC} (B)$-module $M$.
%\end{remark}
%
%
%\begin{proof}[Proof of Proposition \ref{KSproj}]
%Let $\phi \colon P \to L$ be a projective cover of a simple object. Since $\ker \phi$ is the unique maximal subobject of $P$, $\Hom_{\AC} (P, L') = 0$ whenever $L \neq L'$. To show equation (\ref{dimensioncondition}), it remains to show that 
%\[
%\dim_{\End_{\AC} (L)} \Hom_{\AC} (P, L)
%=
% \dim_{\End_{\AC} (L)}  \End_{\AC} (L)
%=1.
%\]
%For this we show that the $\End_{\AC} (L)$-equivariant map 
%\[
%- \circ \phi : \End_{\AC} (L) \to \Hom_{\AC} (P, L)
%\]
%is an isomorphism. Since $\phi$ is a surjection this map is injective. To show surjectivity, let $f \in \Hom_{\AC} (P, L)$ be nonzero. Then as $\ker f$ is a maximal subobject of $P$, $\ker \phi \subset \ker f$, and so $f$ factors through $\phi$. 
%
%
%Conversely, if (\ref{dimensioncondition}) holds, then if $P = P_1 \oplus P_2$, only one $P_i$ can have a simple quotient and the other must be zero. In particular, $P$ is indecomposable.
%\end{proof}

%\begin{proof}[Proof of Proposition \ref{KSproj}]
%Let $\phi \colon P \to L$ be a projective cover of a simple object. Since $\ker \phi$ is the unique maximal subobject of $\phi$, $\Hom_{\AC} (P, L') = 0$ whenever $L \neq L'$. To show that 
%\[
%\dim_{\End_{\AC} (L)} \Hom_{\AC} (P, L)
%=
% \dim_{\End_{\AC} (L)}  \End_{\AC} (L)
%=1
%\]
%it suffices to show that the $\End_{\AC} (L)$-equivariant map 
%\[
%- \circ \phi : \End_{\AC} (L) \to \Hom_{\AC} (P, L)
%\]
%is an isomorphism. Since $\phi$ is a surjection this map is injective. To show surjectivity, let $f \in \Hom_{\AC} (P, L)$ be nonzero. Then as $\ker f$ is a maximal subobject of $P$, $\ker \phi \subset \ker f$, and so $f$ factors through $\phi$. 
%
%
%Conversely, if (\ref{dimensioncondition}) holds, then if $P = P_1 \oplus P_2$, only one $P_i$ can have a simple quotient and the other must be zero. In particular, $P$ is indecomposable.
%\end{proof}

%\begin{proof}[Proof of Proposition \ref{KSproj}]
%Let $\phi \colon P \to L$ be a projective cover of a simple object. Since $\ker \phi$ is the unique maximal subobject of $\phi$, $\Hom_{\AC} (P, L') = 0$ whenever $L \neq L'$. To show equation (\ref{dimensioncondition}), it remains to show that the $\End_{\AC} (L)$-equivariant map 
%\[
%- \circ \phi : \End_{\AC} (L) \to \Hom_{\AC} (P, L)
%\]
%is an isomorphism. Since $\phi$ is a surjection this map is injective. To show surjectivity, let $f \in \Hom_{\AC} (P, L)$ be nonzero. Then as $\ker f$ is a maximal subobject of $P$, $\ker \phi \subset \ker f$, and so $f$ factors through $\phi$. Since $- \circ \phi : \End_{\AC} (L) \to \Hom_{\AC} (P, L)$ is an $\End_{\AC} (L)$-equivariant isomorphism, 
%\[
%\dim_{\End_{\AC} (L)} \Hom_{\AC} (P, L)
%=
% \dim_{\End_{\AC} (L)}  \End_{\AC} (L)
%=1.
%\]
%
%Conversely, if (\ref{dimensioncondition}) holds, then if $P = P_1 \oplus P_2$, only one $P_i$ can have a simple quotient and the other must be zero. In particular, $P$ is indecomposable.
%\end{proof}

\begin{prop}\label{FinProj}
Let $\AC$ be an abelian length category. Then $\AC$ has enough projectives if and only if every simple object in $\AC$ has a projective cover.
\end{prop}

\begin{proof}
Let $\AC$ be an abelian length category in which every simple object has a projective cover.
Let $X$ be an object in $\AC$ and write $\Rad X$ for the intersection of all maximal subobjects of $X$. 
%It is straightforward to check that $\Rad X$ is superfluous
% and so the quotient $X \to X / \Rad X$ is essential. 

The quotient object $X / \Rad X$ is semisimple. Indeed, since $\AC$ is a length category, $\Rad X$ is the intersection of finitely many maximal subobjects $M_i \subset X$.  In particular there is an injection
\[
X / \Rad X \into \bigoplus_i X/ M_i
\]
%defined $x + \Rad X \mapsto (x +M_1, x + M_2, \ldots)$ . 
given by the diagonal map.
Since $X / \Rad X$ embeds into a semisimple object, it is itself semisimple.

%We remark also that $\Rad X $ is a superfluous submodule of $X$ - this is straightforward to check.

Since $X / \Rad X$ is semisimple, it has a projective cover $P$.  
Since $P$ is projective the essential surjection $P \to X / \Rad X$ factors through $X$, defining a morphism $P \to X$ that fits into the commutative diagram
\begin{equation*}
\begin{tikzcd}
	X & {X/\Rad X} \\
	P
	\arrow[two heads, from=1-1, to=1-2]
	\arrow[from=2-1, to=1-1]
	\arrow[two heads, from=2-1, to=1-2]
\end{tikzcd}
\end{equation*}
We show that this map $P \to X$ defines a projective cover of $X$.

It is straightforward to check that $\Rad X$ is a superfluous subobject of $X$. Hence the map $X \to X / \Rad X$ is an essential surjection. It follows that the map $P \to X$ is surjective. One can check directly that $P \to X$ is essential using the fact that the maps  $P \to X /\Rad X$ and $X \to X / \Rad X$ are essential (see e.g. \cite[Lemma 3.1]{Kra15}). The result follows.



\end{proof}

\subsection{Ext-finiteness}

For $k \in \NM$, say that an abelian category $\AC$ is {\em $\Ext^k$-finite} if for any simple objects
 $A, B$ in $\AC$, 
\[
\dim_{\End_{\AC} (B)} \Ext^{k}_{\AC} (A, B) < \infty .
\]
%\End_{\AC} (Y) is a division ring

We remark that if $\AC$ is a $\km$-linear category, for some field $\km$, then\footnote{If $\{e_i\}_{i \in I}$ is a $\km$-basis of $\End_{\AC} (B) $ and $\{f_j\}_{j \in J}$ is a $\End_{\AC} (B)$-basis of  $\Ext^{k}_{\AC} (A,B)$, then $\{ e_i\circ   f_j\}_{(i,j) \in I \times J}$ is a $\km$-basis of  $\Ext^{k}_{\AC} (A,B) $.}

%\[
%\dim_{\End_{\AC} (B)} \Ext^{k}_{\AC} (A,B) = \frac{ \dim_{\km} \Ext^{k}_{\AC} (A,B) }{ \dim_{\km} \End_{\AC} (B)}.
%\]
\[
\dim_{\km} \Ext^{k}_{\AC} (A,B) =  \dim_{\km} \End_{\AC} (B) \times \dim_{\End_{\AC} (B)} \Ext^{k}_{\AC} (A,B).
\]
%Indeed if $\{e_i\}_{i \in I}$ is a $\km$-basis of $\End_{\AC} (B) $ and $\{f_j\}_{j \in J}$ is a $\End_{\AC} (B)$-basis of  $\Ext^{k}_{\AC} (A,B)$, then $\{ e_i\circ   f_j\}_{(i,j) \in I \times J}$ is a $\km$-basis of  $\Ext^{k}_{\AC} (A,B) $.

In particular, $\AC$ is $\Ext^k$-finite whenever $\dim_{\km} \Ext^{k}_{\AC} (A,B) < \infty$ for every simple object $A, B$. The converse is true if the endomorphism ring of every simple object has finite $\km$-dimension (e.g. if $\km$ is algebraically closed).


The $\Ext^1$-finiteness property allows us to construct new objects using the {\em universal extension construction} as defined in \cite{Rin91} - we recall this construction in Section \ref{MainResults}. This construction is particularly useful for constructing projective generators (see e.g. \cite[Theorem 3.2.1]{BGS96}, \cite[Theorem 4.6]{CW21}). 
%\cite[8.2]{Gab73}
We require $\Ext^1$-finiteness for this purpose in Theorem \ref{recolenoughproj}.

In this section we give two results about $\Ext$-finiteness (Propositions \ref{finExt} and \ref{enoughprojandinj}) that will be needed in the discussion following Theorem \ref{recolenoughproj}.

\begin{prop}\label{finExt} 
Any abelian length category with enough projectives is $\Ext^k$-finite for every $k \in \NM$.
\end{prop}

\begin{proof}
Let $\AC$ be an abelian length category.
Let $X$ be an object in $\AC$, and let $Y$ be a simple object in $\AC$. 
%We use induction on $k$ to show that  $\dim_{\End_{\AC} (Y)} \Ext^{k}_{\AC} (X, Y) < \infty$ for each $k \geq 0$.
Consider a projective presentation of X:
\[
0 \to K \to P \to X \to 0
\] 
Since $\Ext^{k}_{\AC} (P, Y) = 0$ there is a $\End_{\AC} (Y)$-equivariant surjection of $\Ext^{k-1}_{\AC} (K, Y)$ onto $\Ext^{k}_{\AC} (X, Y)$ for each $k > 0$. Since $\AC$ is a length category, 
\[
\dim_{\End_{\AC} (Y)} \Hom_{\AC} (X, Y) < \infty.
\]
The result follows by induction.
\end{proof}



%The following two propositions give useful criteria for a category to be $\Ext^k$-finite.



Say that a $\km$-linear abelian category is {\em finite over $\km$} if $\AC$ is a length category with enough projectives, finitely many simple objects, and finite dimensional $\Hom$-spaces. It is well-known that $\AC$ is finite over $\km$ if and only if there is a finite-dimensional $\km$-algebra $A$ in which $\AC$ is equivalent to  the category, $A \mod$, of modules that are finite dimensional as $\km$-vector spaces. Indeed, if $\{ P_{\l} \}_{\l \in \L}$ are the projective indecomposables in $\AC$ (up to isomorphism), then $A = \End_{\AC} (\bigoplus_{\l \in \L} P_{\l})^{op}$ and $\Hom_{\AC} (\bigoplus_{\l \in \L} P_{\l}, -): \AC \to A \mod$ is an equivalence of categories. Note that there is a contravariant equivalence $\Hom_{\km} (- , \km): A \mod \to A^{op} \mod$. In particular, any finite abelian category has enough injectives.

\begin{prop}\label{enoughprojandinj}
Let $\km$ be a field and let $\AC$ be a $\km$-linear abelian category with a stratification by a finite poset in which every strata category is a finite abelian category. Then $\AC$ is $\Ext^1$-finite.
\end{prop}

\begin{proof}
By the assumptions on the strata categories, $\AC$ has finite dimensional $\Hom$-spaces.
Suppose $\AC$ has a recollement with objects and morphisms as in (\ref{recollement}).
Suppose $\AC_U$ and $\AC_Z$ have enough projectives, and $\AC_U$ has enough injectives.
By Proposition \ref{finExt}, both $\AC_Z$ and $\AC_U$ are $\Ext^1$-finite. We show that $\AC$ is $\Ext^1$-finite. It suffices to show that $\dim_{\km} \Ext_{\AC}^1 (X, Y) < \infty$ for all simple objects $X, Y$.

Since $\AC^Z$ is a Serre subcategory of $\AC$, this is true whenever $X$ and $Y$ are both in $\AC^Z$. 
Let $L \in \AC_U$ be simple and let $j_{!*} L$ have projective and injective presentations:
\begin{align*}
& 0 \to K \to j_! P \to j_{!*} L \to 0 \\
& 0 \to  j_{!*} L \to j_* I \to K' \to 0
\end{align*}
The projective presentation implies that $\Hom_{\AC} (K, Y)$ surjects onto $\Ext_{\AC}^1 (j_{!*} L, Y)$. The injective presentation implies that  $\Hom_{\AC} (Y, K')$ surjects onto $\Ext_{\AC}^1 (Y, j_{!*} L)$. It follows that $\AC$ is $\Ext^1$-finite. The result follows by an induction argument.
\end{proof}


\subsection{Main results}\label{MainResults}

This section includes our results about recollements and projective covers.

\begin{prop}\label{recol42}
Let $\AC$ be an abelian category with a recollement of abelian categories as in (\ref{recollement}).
Then
\begin{enumerate}
\item[(i)]
If $\phi: X \to Y$ is an essential surjection in $\AC$ and $i^* Y = 0$ then $i^*X = 0$ and $j^* (\phi): j^* X \to j^* Y$ is an essential surjection.
\item[(ii)]
If $P \in \AC$ is projective and $i^* P = 0$ then $j^* P \in \AC_U$ is projective. In particular if $\phi: P \to X$ is a projective cover in $\AC$ and $i^* X = 0$ then $j^* (\phi): j^* P \to j^* X$ is a projective cover in $\AC_U$.
\end{enumerate}
In particular, if $\AC$ has enough projectives then so does $\AC_U$.
\end{prop}

\begin{proof}
Let $\phi \colon X \to Y$ be an essential surjection in $\AC$ and suppose $i^* Y = 0$. To show that $i^* X = 0$ it suffices to show that the canonical map $\epsilon_X : j_! j^* X \to X$ is surjective. This follows from the following commutative diagram since $\phi$ is essential.
\[
\begin{tikzcd}
	X && Y \\
	{j_! j^* X} && {j_! j^* Y}
	\arrow["{j_! j^* \phi}"', two heads, from=2-1, to=2-3]
	\arrow["{\epsilon_X}", from=2-1, to=1-1]
	\arrow["{\epsilon_Y}"', two heads, from=2-3, to=1-3]
	\arrow["\phi", two heads, from=1-1, to=1-3]
\end{tikzcd}
\]
Let $\a : X' \to j^* X$ be a morphism in $\AC_U$, in which $j^*( \phi) \circ \a: X' \to j^* Y$ is surjective. Then $\epsilon_Y \circ j_! j^*( \phi) \circ j_! \a: j_! X' \to Y$ is surjective and so (since $\phi$ is essential) $\epsilon_X \circ j_! \a : j_! X' \to X$ is surjective. Hence $j^* (\epsilon_X \circ j_! \a) \simeq \a :X' \to j^* X$ is surjective. This proves (i).

%If $P \in \AC$ is projective and $i^*P = 0$ then the functor
%\[
%\Hom_{\AC_U} (j^* P, -) \simeq \Hom_{\AC} (j_! j^* P, j_! (-)) \simeq \Hom_{\AC}(P, j_! (-)) : \AC_U \to \ZM \mod
%\]
%is exact. Here the last isomorphism follows from the sequence (\ref{adex3}). It follows that $j^* P$ is projective. Statement (ii) follows.

For Statement (ii) it remains to show that if $P \in \AC$ is projective and $i^*P = 0$ then the functor $\Hom_{A_U} (j^* P, -) \simeq \Hom_{\AC} (P, j_*(-))$ is exact. We proceed as in the proof of \cite[Lemma 4.3]{CW21}. Given a short exact sequence $0 \to X \to Y \to Z \to 0$ in $\AC_U$, there is a short exact sequence
\[
0 \to j_* X \to j_* Y \to j_* Z \to i_* C \to 0
\]
for some $C \in \AC_Z$. Applying $\Hom_{\AC}(P,-)$ to this sequence gives the exact sequence
\[
0 \to \Hom_{\AC} (P, j_* X) \to \Hom_{\AC}(P, j_* Y) \to \Hom_{\AC} (P, j_* Z ) \to 0
\]
since $\Hom_{\AC} (P, i_* C ) \simeq \Hom_{\AC_Z} (i^* P, C) =0$.
\end{proof}

\begin{prop}\label{recol43}
Suppose $\AC$ is an abelian length category with a recollement of abelian categories as in (\ref{recollement}).
If $P \to L$ is a projective cover in $\AC$ of a simple object $L \in \AC^Z$, then $i^* P \to i^* L$ is a projective cover in $\AC_Z$. In particular, if $\AC$ has enough projectives then so does $\AC_Z$.
\end{prop}

\begin{proof}
Since $i^*$ is the left adjoint of an exact functor it preserves projective objects. For any projective object $P \in \AC$ and simple object $L' \in \AC^Z$, $\Hom_{\AC_Z} (i^* P,  i^* L') \simeq
\Hom_{\AC} (P, i_* i^* L') \simeq  \Hom_{\AC} (P,  L')$. The first statement in the proposition follows from Proposition \ref{KSproj}. The second statement follows from Proposition \ref{FinProj}.
\end{proof}

%The following result is shown in \cite[where?]{CW21} for the case of perverse sheaves, although their proof holds for all Krull-Schmidt categories.

\begin{prop}\label{recol5}
Let $\AC$ be an abelian category with a recollement of abelian categories as in (\ref{recollement}).
If $X \to Y$ is an essential surjection in $\AC_U$ then the composition $j_! X \to j_{!*} X \to j_{!*} Y$ is an essential surjection in $\AC$. In particular:
\begin{enumerate}
\item[(i)]
The canonical surjection $j_! X \to j_{!*}X$ is essential.
\item[(ii)]
If $P \to X$ is a projective cover of $X$ in $\AC_U$ then $j_! P \to j_! X \to j_{!*} X$ is a projective cover of  $j_{!*} X$ in $\AC$.
\end{enumerate}

%
%If $X \to Y$ is an essential extension in $\AC_U$ then the composition $ j_{!*} X \to j_{!*} Y \to j_* Y$ is an essential extension in $\AC$. In particular:
%\begin{enumerate}
%\item[(a)]
%The canonical injection $ j_{!*}X \to j_* X$ is essential.
%\item[(b)]
%If $X \to I$ is an injective envelope of $X$ in $\AC_U$ then  $j_{!*} X \to j_{*} X \to j_* I$ is an injective envelope of $j_{!*} X$ in $\AC$.
%\end{enumerate}

\end{prop}

\begin{proof}
Let $\phi \colon X \to Y$ be an essential surjection in $\AC_U$. The map $\phi': j_! X \to j_{!*} X \to j_{!*} Y$ is surjective by Proposition \ref{recol31}(i). Let $\a: X' \to j_! X$ be a morphism in which $\phi' \circ \a$ is surjective. Now, $j^* (\phi') = \phi \colon X \to Y$ and since $j^*$ is exact, $j^*(\phi' \circ \a) = \phi \circ j^* (\a) : j^* X' \to Y$ is surjective. Since $\phi$ is essential it follows that $j^* (\a): j^* X' \to X$ is surjective in $\AC_U$ and so $j_! j^* (\a): j_! j^* X' \to j_! X $ is surjective in $\AC$. The surjectivity of $\a$ follows from the commutative triangle
\[
\begin{tikzcd}
	{j_! j^* X'} && {j_! X} \\
	{X'}
	\arrow["{j_! j^* (\a)}", two heads, from=1-1, to=1-3]
	\arrow[from=1-1, to=2-1]
	\arrow["\a"', from=2-1, to=1-3]
\end{tikzcd}
\]
in which the downward arrow is the adjunction counit.
\end{proof}

The following result holds by an almost identical argument.

\begin{prop}
The intermediate-extension functor preserves essential surjections and essential extensions. 
\end{prop}



The following is the main result of this section. 

\begin{thm}\label{recolenoughproj}
Let $\AC$ be an abelian length category with finitely many simple objects, and a recollement of abelian categories as in (\ref{recollement}). If $\AC$ is $\Ext^1$-finite then $\AC$ has enough  projectives if and only if both $\AC_U$ and $\AC_Z$ have enough projectives. Dually if $\AC^{op}$ is $\Ext^1$-finite then $\AC$ has enough  injectives if and only if both $\AC_U$ and $\AC_Z$ have enough injectives.
\end{thm}

Before giving the proof of this theorem we will explain one important ingredient: the {\em universal extension} as described by Ringel \cite{Rin91}.

Let $A, B$ be objects in an abelian category $\AC$ in which $\End_{\AC} (B)$ is a division ring and $d := \dim_{\End_{\AC} (B)} \Ext_{\AC}^1(A,B) < \infty$. We form the {\em universal extension} $\EC \in \Ext^{1}_{\AC} (A, B^{\oplus d})$ by the following process. First let $E_1, \ldots, E_d \in \Ext_{\AC}^1 (A, B)$ be an $\End_{\AC} (B)$-basis. The diagonal map $\D: A \to A^{\oplus d}$ induces a map $\Ext_{\AC}^1 (A^{\oplus d}, B^{\oplus d}) \to \Ext^{1}_{\AC} (A, B^{\oplus d})$.
Let $\EC$ be the image of $E_1 \oplus \cdots \oplus E_d$ under this map. 
%That is, $\EC$ fits into the following commutative diagram with exact rows.
%\[\begin{tikzcd}
%	0 & {B^{\oplus d}} & {\bigoplus_i E_i} & {A^{\oplus d}} & 0 \\
%	0 & {B^{\oplus d}} & \EC & A & 0
%	\arrow[from=1-1, to=1-2]
%	\arrow[from=1-2, to=1-3]
%	\arrow[from=1-3, to=1-4]
%	\arrow[from=1-4, to=1-5]
%	\arrow["\id", from=2-2, to=1-2]
%	\arrow[from=2-3, to=1-3]
%	\arrow["\D", from=2-4, to=1-4]
%	\arrow[from=2-3, to=2-4]
%	\arrow[from=2-2, to=2-3]
%	\arrow[from=2-1, to=2-2]
%	\arrow[from=2-4, to=2-5]
%	\arrow["\ulcorner"{anchor=center, pos=0.125, rotate=90}, draw=none, from=2-3, to=1-4]
%\end{tikzcd}\]
Note that the $\End_{\AC} (B)$-equivariant map $\Hom_{\AC} (B^{\oplus d}, B) \to \Ext_{\AC}^1 (A,B)$ induced by the short exact sequence
\[
0 \to B^{\oplus d} \to \EC \to A \to 0
\]
is surjective (this is easy to check on the basis of $\Ext_{\AC}^1 (A,B)$).
%In particular if $B$ is projective or $\Hom_{\AC} (A, B) \simeq \Hom(E, B)$ then $\Ext_{\AC}^1 (E, B) = 0$. 

When $B_1, \ldots, B_n$ are objects in $\AC$ in which each ring $\End_{\AC} (B_i)$ is a division ring and $d_i := \dim_{\End_{\AC} (B_i)} \Ext_{\AC}^1(A,B_i) < \infty$, we also talk about a {\em universal extension} 
$\EC \in \Ext_{\AC}^1 (A, \bigoplus_i B_{i}^{\oplus d_i})$ constructed in the following way. Let $\EC_i \in \Ext^{1}_{\AC} (A, B_{i}^{\oplus d_i})$ be a universal extension (as defined in the previous paragraph) and define $\EC$ to be the image of $\EC_1 \oplus \cdots \oplus  \EC_n$ under the map  $\Ext_{\AC}^1 (\bigoplus_i A, \bigoplus_i B^{\oplus d_i}) \to \Ext^{1}_{\AC} (A, \bigoplus_i B^{\oplus d_i})$ induced by the diagonal map $\D: A \to A^{\oplus d}$. Then $\EC$ has the property that the short exact sequence 
\[
0 \to \bigoplus_i B^{\oplus d_i} \to \EC \to A \to 0
\]
induces a surjection $\Hom_{\AC} (\bigoplus_i B_{i}^{\oplus d_i}, B_j) \to \Ext_{\AC}^{1} (A, B_j)$ for each $j = 1, \ldots, n$.

Dually, if $\End_{\AC} (A)^{op}$ is a division ring and $\dim_{\End_{\AC} (A)^{op}} \Ext^1 (A,B) < \infty$ then one can form a universal extension $\EC' \in \Ext_{\AC}^1 (A^{\otimes d},B)$ using the codiagonal map $\delta: B^{\oplus d} \to B$ instead of the diagonal map.


\begin{proof}[Proof of Theorem \ref{recolenoughproj}]
Suppose that $\AC_U$ and $\AC_Z$ have enough projectives.

Suppose $\AC^Z$ has simple objects $L_1, \ldots, L_m$ with projective covers $\bar{P}_1, \ldots, \bar{P}_m$ in $\AC^Z$. Suppose $\AC^U$ has simple objects $L_{m+1}, \ldots, L_{m+n}$. By Proposition \ref{recol5} every simple object in $\AC^U$ has a projective cover in $\AC$.
It suffices to construct a projective cover in $\AC$ of each simple object in $\AC^Z$. This amounts to finding, for each $1 \leq t \leq m$, a projective object, $P_t$, whose unique simple quotient is $L_t$.

Fix $1 \leq t \leq m$.

{\em Step 1. Define $P_t$.}
For simple object $L_{m+k} \in \AC^U$, let $P_{m+k}$ denote its projective cover in $\AC$.
Define $Q$ to be a maximal length quotient of 
\[
P := \bigoplus_{k = 1}^{n} P_{m + k}^{\oplus \dim_{\End_{\AC} (L_{m+k})} \Ext^{1}_{\AC} (\bar{P}_t, L_{m + k})}
\]
in which there is an extension 
\begin{equation}\label{Qdef}
0 \to Q \to \EC  \to \bar{P}_t \to 0
\end{equation}
inducing an isomorphism $\Hom_{\AC} (Q, L) \simeq \Ext^{1}_{\AC} (\bar{P}_t , L)$ for each $L \in \AC^U$. 
That is $0 = \Hom_{\AC} (\bar{P}_t, L) \simeq \Hom_{\AC} (\EC, L)$ and $\Ext_{\AC}^1 (\EC, L)$ injects into $\Ext_{\AC}^1 (Q, L)$.

Let $P_t$ be any choice of such $\EC$.

{\em Step 2. $P_t$ is well-defined.}
To show that the maximal quotient $Q$ exists, we just need to find one quotient of $P$ with the required property. Then since $\AC$ is a length category  there exists a maximal length quotient with the required property.
Since $\AC$ has finite $\Ext^1$-spaces, we can let
\[
R = \bigoplus_{k =1}^{n} L_{m + k}^{\oplus \dim_{\End_{\AC} (L_{m+k})} \Ext^{1}_{\AC} (\bar{P}_t, L_{m + k})}
\]
and form the universal extension
\[
0 \to R \to \EC  \to \bar{P}_t \to 0
\]

Since this is a universal extension it induces a surjection $\Hom_{\AC} (R, L_{m+k}) \to \Ext^{1}_{\AC} (\bar{P}_t , L_{m+k})$ for each $L_{m+k} \in \AC^U$. This map is an isomorphism since 
\[
\dim_{\End_{\AC} (L_{m+k})} \Hom_{\AC} (R, L_{m+k}) =\dim_{\End_{\AC} (L_{m+k})} \Ext^{1}_{\AC} (\bar{P}_t, L_{m+k}).
\]

{\em Step 3. $P_t $ has unique simple quotient $L_t$.}
By definition of $P_t$ and by the $(i^*, i_*)$-adjunction, for any simple module $L \in \AC$:
\[
\Hom_{\AC} (P_t, L) \simeq 
\Hom_{\AC} (\overline{P}_t, L)
\simeq
\Hom_{\AC} (\overline{P}_t, i_* i^* L)
\]
% last equality due to adjunction
and so the only simple quotient of $P_t$ is $L_t$ with multiplicity one.

{\em Step 4. $P_t$ is projective.} We show that $\Ext^{1}_{\AC} (P_{t}, L) = 0$ for each simple $L \in \AC$. 
For any simple $L \in \AC$ there is an exact sequence
\begin{equation}\label{step4}
0 \to \Ext^{1}_{\AC} (P_{t}, L) \to \Ext^{1}_{\AC} (Q, L) \to \Ext^{2}_{\AC} (\bar{P}_t, L)
\end{equation}
Indeed if $L \in \AC^U$ then this holds because $\Hom_{\AC} (Q, L) \simeq \Ext^{1}_{\AC} (\bar{P}_t , L)$. If $L \in \AC^Z$ then (\ref{step4}) holds since $\Ext_{\AC}^{1} (\bar{P}_t, L) = 0$.

To show that $P_t$ is projective it suffices to show that the third map in (\ref{step4}) is injective for any $L \in \AC$. Suppose for contradiction that there is a nontrivial extension
\begin{equation}\label{Q'}
0 \to L \to Q' \to Q \to 0
\end{equation}
in the kernel of this map. Then there is an object $\EC \in \AC$ fitting into the following diagram
\begin{equation}\label{exactgrid}
\begin{tikzcd}
	& 0 & 0 & 0 \\
	0 & L & {Q'} & {Q} & 0 \\
	0 & L & \EC & {P_{t}} & 0 \\
	0 & 0 & {\bar{P}_t} & {\bar{P}_t} & 0 \\
	& 0 & 0 & 0
	\arrow[from=2-1, to=2-2]
	\arrow[from=2-2, to=2-3]
	\arrow[from=2-3, to=2-4]
	\arrow[from=2-4, to=2-5]
	\arrow[from=1-4, to=2-4]
	\arrow[from=2-4, to=3-4]
	\arrow[from=3-4, to=4-4]
	\arrow[from=4-4, to=5-4]
	\arrow[from=4-3, to=5-3]
	\arrow[from=3-3, to=4-3]
	\arrow[from=2-3, to=3-3]
	\arrow[from=1-3, to=2-3]
	\arrow[from=3-1, to=3-2]
	\arrow[from=3-2, to=3-3]
	\arrow[from=3-3, to=3-4]
	\arrow[from=3-4, to=3-5]
	\arrow[from=1-2, to=2-2]
	\arrow[from=2-2, to=3-2]
	\arrow[from=3-2, to=4-2]
	\arrow[from=4-2, to=4-3]
	\arrow[from=4-3, to=4-4]
	\arrow[from=4-4, to=4-5]
	\arrow[from=4-1, to=4-2]
	\arrow[from=4-2, to=5-2]
\end{tikzcd}
\end{equation}
in which each row and column is exact.
For each $L' \in \AC^U$ the sequence (\ref{Q'}) induces an exact sequence
\[
0 \to \Hom_{\AC}(Q, L') \to \Hom_{\AC}(Q', L') \to \Hom_{\AC}(L, L') 
\]
Of course, $\Hom_{\AC}(L, L') = 0$ if $L \neq L'$. If $L = L'$ the third map must be zero. Indeed if $f: L \to L$ factors through the inclusion $\iota: L \to Q'$ via a map $g: Q' \to L$, then $f^{-1} \circ g : Q' \to L$ is a retraction of $\iota$. This contradicts the assumption that (\ref{Q'}) does not split.
%Since (\ref{Q'}) does not split there is an isomorphism $ \Hom_{\AC}(Q, L') \simeq \Hom_{\AC}(Q', L')$ for any $L' \in \AC^U$.
Hence, for any $L' \in \AC^U$, there is an isomorphism
\begin{equation}\label{maximality}
\Hom_{\AC}(Q', L') \simeq  \Hom_{\AC}(Q, L') \simeq \Ext^{1}_{\AC} (\bar{P}_t, L') .
\end{equation}
Since $P$ is projective the quotient $P \to Q$ fits into the diagram
\[
\begin{tikzcd}
	& P \\
	{Q'} & {Q}
	\arrow[two heads, from=1-2, to=2-2]
	\arrow["\varphi"', from=1-2, to=2-1]
	\arrow[two heads, from=2-1, to=2-2]
\end{tikzcd}
\]
Now $\varphi$ cannot be surjective, as, by (\ref{maximality}), this would contradict the maximality of $Q$. Thus the image of $\varphi$ is isomorphic to $Q$ and so the sequence (\ref{Q'}) splits. This is a contradiction. Hence $P_t$ is projective.
\end{proof}

\begin{cor}\label{recolenoughprojcor1}
Let $\AC$ be an abelian category with a stratification in which every strata category is a length category, and has finitely many simple objects. 

If $\AC$ is $\Ext^1$-finite (respectively $\AC^{op}$ is $\Ext^1$-finite) then $\AC$ has enough projectives (respectively injectives) if and only if every strata category has enough projectives (respectively injectives).
\end{cor}

\begin{proof}

By Proposition \ref{recol41}, every category $\AC_{\L'}$ (for lower $\L' \subset \L$) satisfies the conditions of Theorem \ref{recolenoughproj}. So we can obtain a projective cover in $\AC$ of any simple object $j^{\l}_{!*} L$ by repeatedly applying the construction in the proof of Theorem \ref{recolenoughproj} to larger and larger Serre subcategories of $\AC$.
\end{proof}

The following result follows immediately from Proposition \ref{enoughprojandinj} and Corollary \ref{recolenoughprojcor1}.

\begin{cor}\label{recolenoughprojcor}
Let $\AC$ be a $\km$-linear abelian category with a stratification by a finite poset. Then $\AC$ is finite over $\km$ if and only if the same is true of every strata category.
\end{cor}

From this result we recover the following result of Cipriani-Woolf.

\begin{cor}[Corollary 5.2 of \cite{CW21}]\label{CipWoo}
Let $P_{\L} (X, \km)$ be the category of perverse sheaves (with coefficients in a field $\km$) on a quasiprojective complex variety $X$ that is constructible with respect to a stratification, $X = \bigcup_{\l \in \L} X_{\l}$, with finitely many strata.
Then $P_{\L}(X, \km)$ is finite over $\km$ if and only if each category $\km [\pi_1 (X_{\l})] \mod_{fg}$ is finite over $\km$.
\end{cor}


For example, if $X$ has a finite stratification, $X = \bigcup_{\l \in \L} X_{\l}$, in which each $X_{\l}$ has finite fundamental group, then the category $P_{\L} (X, \km)$ is finite over $\km$.

\begin{cor}\label{enoughprojEquiv}
Let $G$ be an algebraic group and let $X$ be a $G$-variety with finitely many orbits, each connected. Let $\km$ be a field. The category, $P_{G} (X, \km)$, of $G$-equivariant perverse sheaves is finite over $\km$ if and only if for each $G$-orbit $\OC_{\l}$ and $x \in \OC_{\l}$, the category $\km [G^{x} / (G^{x})^{\circ}] \mod_{fg}$ is finite over $\km$.
\end{cor}


%%%%%%%%%%%%%%%%%%%%%%%%%%%%%%%%%%%%%%%%%%%%%%%%%%%%%%%%%%%%%%%%%%%%%%%%%%%%%%%%%


\section{Standard and costandard objects}\label{standardsec}


In this section we focus on abelian length categories $\AC$ with finitely many simples, enough projectives and injectives, and admitting a stratification by a finite non-empty poset $\L$. Let $B$ be a set indexing the simple objects of $\AC$ up to isomorphism. Write $L(b)$ for the simple object corresponding to $b \in B$, and write $P(b)$ and $I(b)$ for the projective cover and injective envelope of $L(b)$. 

For each $\l \in \L$, write $\AC_{\leq \l} := \AC_{\{ \mu \in \L ~|~ \mu \leq \l \}}$ and let $j^{\l}_{!}: \AC_{\l} \to \AC$ be the composition 
\[
\begin{tikzcd}
	{\AC_{\l}} & {\AC_{\leq \l}} & \AC
	\arrow["{j_{!}}", from=1-1, to=1-2]
	\arrow[hook, from=1-2, to=1-3]
\end{tikzcd}
\]
Define  $j^{\l}_{*}: \AC_{\l} \to \AC$ and  $j^{\l}_{!*}: \AC_{\l} \to \AC$ likewise. Let $j^{\l} \colon \AC_{\leq \l} \to \AC_{\l}$ denote the Serre quotient functor.

Define the {\em stratification function} $\r: B \to \L$ that assigns to each $b \in B$ the $\l \in \L$ in which $L(b) = j_{!*}^{\l} L_{\l} (b)$. Let $P_{\l} (b)$ and $I_{\l} (b)$ be the projective cover and injective envelope of the simple object $L_{\l}(b)$ in $\AC_{\l}$.

For $b \in B$, define the {\em standard} and {\em costandard} objects
\[
\D (b) := j^{\l}_! P_{\l} (b), \qquad \nabla (b) := j^{\l}_* I_{\l} (b),
\]
where $ \l = \r (b)$.

The following result follows from the proofs of Theorem \ref{recolenoughproj} and Corollary \ref{recolenoughprojcor1}.

\begin{porism}\label{standards}
Let $\AC$ be an abelian category with a stratification by a finite non-empty poset $\L$, in which every strata category is a length category with finitely many simple objects. Let $\r: B \to \L$ be the stratification function for $\AC$. 

If $\AC$ has enough projectives then for each $\l \in \L$ and $b \in \r^{-1} (\l)$, the projective indecomposable object $P(b) \in \AC$ fits into a short exact sequence
\[
0 \to Q (b) \to P (b) \to \D (b) \to 0
\]
in which
$Q(b)$ has a filtration by quotients of $\D( b')$ satisfying $\r(b') > \r (b)$. 

If $\AC$ has enough injectives then for each $\l \in \L$ and $b \in \r^{-1} (\l)$, the injective indecomposable object $I(b) \in \AC$ fits into a short exact sequence
\[
0 \to \nabla (b) \to I (b) \to Q' (b) \to 0
\]
in which
$Q'(b)$ has a filtration by subobjects of $\nabla( b')$ satisfying $\r(b') > \r (b)$. 

\end{porism}

\begin{proof}
We just prove the first statement by induction on $|\L|$. The base case $|\L| = 1$ is trivial. 

Consider the projective cover, $P(b)$, of simple object $L(b)$ in $\AC$. If $\r (b)$ is maximal then $P(b) \simeq \D (b)$ and the result holds. Suppose $\r (b)$ is not maximal, and let $\mu \in \L$ be a maximal element. Consider the recollement
\[
\begin{tikzcd}
	{\AC_{< \mu}} && \AC && {\AC_{\mu}}
	\arrow["{i_{*}}", from=1-1, to=1-3]
	\arrow["{i^{*}}"', shift right=5, from=1-3, to=1-1]
	\arrow["{i^{!}}", shift left=5, from=1-3, to=1-1]
	\arrow["{j^{*}}", from=1-3, to=1-5]
	\arrow["{j_{!}}"', shift right=5, from=1-5, to=1-3]
	\arrow["{j_{*}}", shift left=5, from=1-5, to=1-3]
\end{tikzcd}
\]
Let $P_{< \mu} (b)$ and $\D_{< \mu} (b)$ be the projective indecomposable and standard object in $\AC_{< \mu}$ corresponding to the simple object $i^* L(b) \in \AC_{< \mu}$. By induction there is a short exact sequence
\[
0 \to Q_{< \mu} (b) \to P_{< \mu} (b) \to \D_{< \mu} (b) \to 0
\]
in which $Q_{< \mu} (b)$ has a filtration by quotients of standard objects $\D_{< \mu} ( b')$ satisfying $ \r(b') > \r (b)$. Since $i_*$ is exact we get the following short exact sequence in $\AC$:
\begin{equation}\label{porproof1}
0 \to i_* Q_{< \mu} (b) \to i_* P_{< \mu} (b) \to \D (b) \to 0
\end{equation}
and  $i_* Q_{< \mu} (b)$ has a filtration by quotients of standard objects $\D ( b')$ satisfying $\mu > \r(b') > \r (b)$.
By applying the construction in Step 1 of the proof of Theorem \ref{recolenoughproj}, $P(b)$ fits into the short exact sequence in $\AC$:
\begin{equation}\label{porproof2}
0 \to Q_{\mu} (b) \to P (b) \to i_* P_{< \mu} (b) \to 0
\end{equation}
%and $Q_{\mu} (b)$ is a quotient of a direct sum of standard objects of the form $\D (b')$ in which $\r (b') = \mu$. 
and $Q_{\mu} (b)$ is the direct sum of quotients of standard objects of the form $\D (b')$ in which $\r (b') = \mu$. 
%quotient of direct sum is the direct sum of quotientrs
Combining (\ref{porproof1}) and (\ref{porproof2}) gives the following diagram with exact rows and columns:
\[
\begin{tikzcd}
	& 0 & 0 & 0 \\
	0 & Q_{\mu} (b) & Q_{\mu} (b) & 0 & 0 \\
	0 & Q (b) & P (b) & \D (b) & 0 \\
	0 & i_* Q_{< \mu} (b)  & i_* P_{< \mu} (b) & \D (b) & 0 \\
	& 0 & 0 & 0
	\arrow[from=2-1, to=2-2]
	\arrow[from=2-2, to=2-3]
	\arrow[from=2-3, to=2-4]
	\arrow[from=2-4, to=2-5]
	\arrow[from=1-4, to=2-4]
	\arrow[from=2-4, to=3-4]
	\arrow[from=3-4, to=4-4]
	\arrow[from=4-4, to=5-4]
	\arrow[from=4-3, to=5-3]
	\arrow[from=3-3, to=4-3]
	\arrow[from=2-3, to=3-3]
	\arrow[from=1-3, to=2-3]
	\arrow[from=3-1, to=3-2]
	\arrow[from=3-2, to=3-3]
	\arrow[from=3-3, to=3-4]
	\arrow[from=3-4, to=3-5]
	\arrow[from=1-2, to=2-2]
	\arrow[from=2-2, to=3-2]
	\arrow[from=3-2, to=4-2]
	\arrow[from=4-2, to=4-3]
	\arrow[from=4-3, to=4-4]
	\arrow[from=4-4, to=4-5]
	\arrow[from=4-1, to=4-2]
	\arrow[from=4-2, to=5-2]
\end{tikzcd}
\]
The result follows.

\end{proof}

\begin{ex}[Projective Perverse Sheaves]\label{perverseStandards}
Let $X = \bigcup_{\l \in \L} X_{\l}$ be a stratification of a quasiprojective complex variety. 
%We demonstrate Porism \ref{standards} in the perverse sheaves category, $P_{\L}(X, \km)$ !!. 
Let $\rho: B \to \L$ be a stratification function for the perverse sheaves category $P_{\L}(X, \km)$.
For $\l \in \L$, write $j^{\l}: X_{\l} \into X$ for the embedding of the corresponding stratum. 
Let ${}^{p} {\HC^{0}}: \DC_{\L} (X, \km) \to P_{\L} (X, \km)$ be the zero-th perverse cohomology functor. 

Consider a simple perverse sheaf, $\LC_{\l} (b)$, in $P(X_{\l},\km)$ with projective cover $\PC_{\l} (b)$. Let $\LC (b) := j_{!*}^{\l} \LC_{\l} (b)$ be the corresponding simple perverse sheaf in $P_{\L}(X, \km)$, and let $\PC(b)$ be the projective cover of $\LC (b)$ in $P_{\L}(X, \km)$.

If $\l \in \L$ is maximal then $\PC (b)$ is isomorphic to the standard object $\D (b) := {}^{p} {\HC^{0}} (j_{!}^{\l} \PC_{\l}(b))$. 

Suppose otherwise that $\l \in \L$ is not maximal. Consider the closed and open inclusions
\[
\begin{tikzcd}
	{\overline{X_{\l}}} & X & {X \setminus \overline{X_{\l}}}
	\arrow["i", hook, from=1-1, to=1-2]
	\arrow["j"', hook', from=1-3, to=1-2]
\end{tikzcd}
\]

Then $\PC (b)$ fits into an exact sequence 
\[
\begin{tikzcd}
	{{}^p \HC^0 (j_! j^* \PC (b)) } & {\PC (b)} & {{}^p \HC^0 (i_* i^* \PC (b))} & 0
	\arrow[from=1-1, to=1-2]
	\arrow[from=1-2, to=1-3]
	\arrow[from=1-3, to=1-4]
\end{tikzcd}
\]
\end{ex}
Moreover $ {{}^p \HC^0 (i_* i^* \PC (b))} \simeq \D (b)$ and the image of the map ${{}^p \HC^0 (j_! j^* \PC (b)) } \to {\PC (b)}$ has a filtration by quotients of $\D( b')$ satisfying $\r(b') > \r (b)$. Indeed the later statement can be checked by induction of the size of the stratification $|\L|$, using the facts that $j^* \PC (b)$ is projective in $P_{\L \setminus \overline{\{\l\}}}( X \setminus \overline{X_{\l}}, \km)$ and the functor ${{}^p \HC^0 (j_!(-))}: P_{\L \setminus \overline{\{\l\}}}( X \setminus \overline{X_{\l}}) \to P_{\L}(X, \km)$ is right exact.

%Two questions arise naturally from Porism \ref{standards}:
%\begin{itemize}
%\item
%Under what conditions do projective indecomposables in $\AC$ have a filtration by standard objects?
%\item
%Under what conditions do injective indecomposables in $\AC$ have a filtration by costandard objects? 
%\end{itemize}
%
%These questions are answered by Theorem \ref{eStratTheorem} below.

\section{Brundan and Stroppel's $\e$-stratified categories}\label{BrunStropSection}



%Categories in which projective and/or injective indecomposable objects have filtrations by standard and/or costandard objects have been widely studied, begining with Cline-Parshall-Scott's definition of highest weight category in \cite{CPS88b}. Categories whose projective indecomposables have filtrations by standard objects where originally studied by Dlab \cite{Dla96} and Cline, Parshall and Scott \cite{CPS96}, where these are called {\em standardly stratified categories}. Categories in which both projective objects have a filtration by standard objects and injective objects have a filtration by costandard objects have been studied by various authors (see e.g. \cite{Fri07}, \cite{CZ19}, \cite{LW15}). These situations fit into the  general  framework of $\e$-stratified categories (defined by Brundan and Stroppel \cite{BS18}). 

In this section we give necessary and sufficient conditions for a finite abelian category with a stratification by a finite poset to be $\e$-stratified in the sense of Brundan and Stroppel \cite{BS18} (Theorem \ref{eStratTheorem}). This result specializes to a characterization of highest weight categories originally given by Krause \cite[Theorem 3.4]{Kra17a}. We recall this characterisation in Corollary \ref{hwcresult}.


Let $\AC$ be an abelian length category with finitely many simples, enough projectives and injectives, and with a stratification by a finite non-empty poset $\L$. Let $\rho: B \to \L$ be the stratification function corresponding to this stratification.

For $b \in B$ and $\l = \rho (b) \in \L$, define {\em proper standard} and {\em proper costandard} objects 
\[
\overline{\D} (b) := j^{\l}_! L_{\l}(b),  \qquad \overline{\nabla} (b) := j^{\l}_* L_{\l}(b) .
\]

For a {\em sign function} $\e \colon \Lambda
\to \{+, - \}$, define
the {\em $\e$-standard} and {\em $\e$-costandard objects}
\begin{align*}
\D_\e(b) &:= \left\{
\begin{array}{ll}
\Delta(b)&\text{if $\e(\rho(b)) = +$}\\
\overline\Delta(b)&\text{if $\e(\rho(b)) = -$}
\end{array}
\right.,&
\nabla_\e(b) &:= \left\{
\begin{array}{ll}
\overline\nabla(b)&\text{if $\e(\rho(b)) = +$}\\
\nabla(b)&\text{if $\e(\rho(b)) = -$}
\end{array}
\right. .
\end{align*}

Note that since $j_{!}^{\l}$ and $j_{*}^{\l}$ are fully-faithful, these objects have local endomorphism rings and are hence irreducible.
Note also that if $\r (b) \nleq \r (b')$ then for any $\e : \L \to \{+,-\}$,
\begin{equation}\label{exceptional}
\Hom_{\AC} (\D_{\e} (b), \D_{\e}(b')) = 0 = \Hom_{\AC} (\nabla_{\e} (b'), \nabla_{\e} (b)) .
\end{equation}
Indeed the only simple quotient of $\D_{\e} (b)$ is $L(b)$, and all simple subobjects, $L(b'')$, of $\D_{\e}(b')$ satisfy $\r(b'') \leq \r (b')$. Likewise the only simple subobject of $\nabla_{\e} (b)$ is $L(b)$, and all simple quotients, $L(b'')$, of $\nabla_{\e} (b')$ satisfy $\r(b'') \leq \r (b')$. 


The following definition is due to Brundan and Stroppel \cite{BS18}.\footnote{In Brundan and Stoppel's definition, $\e$-stratified categories need not be finite (they satisfy slightly weaker finiteness conditions).}

\begin{defn}[$\e$-stratified category]\label{BSdefn}
Let $\AC$ be a finite abelian category with a stratification by a poset $\L$ and stratification function $\rho: B \to \L$. Let $\e \colon \L \to \{+, -\}$ be a function. 
Say that $\AC$ is an {\em $\e$-stratified category} if the following equivalent conditions are satisfied:
\begin{enumerate}
\item[($\e$-S1)]
For every $b \in B$, the projective indecomposable $P(b)$ has a filtration by objects of the form $\D_{\e} (b')$, where $\r (b') \geq \r (b)$.
\item[($\e$-S2)]
For every $b \in B$, the injective indecomposable $I(b)$ has a filtration by objects of the form $\nabla_{\e} (b')$, where $\r (b') \geq \r (b)$.
\end{enumerate}
\end{defn}
The equivalence of statements ($\e$-S1) and ($\e$-S2) is shown in \cite[Theorem 2.2]{ADL98}. A proof of this fact can also be found in \cite[Theorem 3.5]{BS18}.


\begin{remark}
A finite abelian category $A \mod$ is $\e$-stratified if and only if $A$ is a {\em stratified algebra} in the sense of \cite{ADL98}.
\end{remark}

We introduce the next concept.

\begin{defn}[$\e$-exact stratification]\label{BSdefn}
For a function $\e \colon \L \to \{+,-\}$, say that a stratification of an abelian category $\AC$ by a  finite non-empty poset $\L$ is {\em $\e$-exact} if for all $\l \in \L$ the following hold:
\begin{enumerate}
\item[(i)]
If $\e (\l) = +$ then the functor $j_{*}^{\l}\colon \AC_{\l} \to \AC$ is exact.
\item[(ii)]
If $\e(\l) = -$ then the functor $j_{!}^{\l} \colon \AC_{\l} \to \AC$ is exact.
\end{enumerate}
\end{defn}

For $k \in \NM$, say that a recollement of abelian categories (as in (\ref{recollement})) is {\em $k$-homological} if for all $n \leq k$ and $X,Y \in \AC_{Z}$,
\[
\Ext^{n}_{\AC_{Z}}(X, Y) \simeq \Ext^{n}_{\AC} (i_{*} X, i_{*} Y).
\]
Say that a recollement of abelian categories is {\em homological} if it is $k$-homological for all $k \in \NM$.
A study of homological recollements is given in \cite{Psa13}.

Say that a stratification of an abelian category is {\em $k$-homological} (respectively {\em homological}) if each of the recollements in the data of the stratification is $k$-homological (respectively  homological).

The following theorem is the main result of this section.

\begin{thm}\label{eStratTheorem}
Let $\AC$ be a finite abelian category. Then, for any finite non-empty poset $\L$ and function $\e \colon \L \to \{+, -\}$, the following statements are equivalent:
\begin{enumerate}
\item
$\AC$ has an $\e$-exact homological stratification by $\L$.
\item
$\AC$ has an $\e$-exact 2-homological stratification by $\L$.
\item
$\AC$ is $\e$-stratified.
\end{enumerate}
\end{thm}

To prove this theorem we need the following lemma.

\begin{lemma}\label{2homologicalcorollary}
Let $\AC$ be an abelian length category with finitely many simple objects, and fitting into a 2-homological recollement of abelian categories as in (\ref{recollement}).

If $\AC$ has enough projectives then for any projective object $P \in \AC$, there is a short exact sequence
\[
0 \to j_! j^* P \to P \to i_* i^* P \to 0
\]

If $\AC$ has enough injectives then for any injective object $I \in \AC$, there is a short exact sequence
\[
0 \to i_* i^! I \to I \to j_* j^* I \to 0
\]
\end{lemma}

\begin{proof}
Let $P \in \AC$ be a projective indecomposable object in $\AC$. 
If $i^* P = 0$ then $j_! j^* P \simeq P$ and the result holds.

Suppose that $i^* P \neq 0$.
Consider the exact sequence
\[
0 \to K \to j_! j^* P \to P \to i_* i^* P \to 0
\]
where $K \in \AC^Z$. Since $i^* P$ is projective in $\AC_Z$ and the recollement is 2-homological, $\Ext^{2}_{\AC} (i_* i^* P, K) = 0$. In particular, there is an object $\EC \in \AC$ and surjection $q: \EC \to P$ that fits into the following diagram in which all rows and columns are exact.
% see https://math.stackexchange.com/questions/1109787/ext-groups-due-to-yoneda-why-is-this-class-zero for explanation
% https://q.uiver.app/?q=WzAsMjEsWzEsMCwiMCJdLFsyLDAsIjAiXSxbMywwLCIwIl0sWzEsMSwiSyJdLFsxLDIsIksiXSxbMSwzLCIwIl0sWzEsNCwiMCJdLFsyLDMsImlfKiBpXiogUCJdLFszLDMsImlfKiBpXiogUCJdLFs0LDMsIjAiXSxbMCwzLCIwIl0sWzAsMiwiMCJdLFsyLDIsIlxcRUMiXSxbMywyLCJQIl0sWzQsMiwiMCJdLFswLDEsIjAiXSxbMiwxLCJqXyEgal4hIFAiXSxbMywxLCJqX3shKn0gal4hIFAiXSxbNCwxLCIwIl0sWzMsNCwiMCJdLFsyLDQsIjAiXSxbMCwzXSxbMyw0XSxbNCw1XSxbNSw2XSxbNSw3XSxbNyw4XSxbOCw5XSxbMTAsNV0sWzExLDRdLFs0LDEyXSxbMTIsMTMsInEiXSxbMTMsMTRdLFsxNSwzXSxbMywxNl0sWzE2LDE3XSxbMTcsMThdLFsyLDE3XSxbMTcsMTNdLFsxMyw4XSxbOCwxOV0sWzEsMTZdLFsxNiwxMl0sWzEyLDddLFs3LDIwXV0=
\begin{equation}\label{Ext2}
\begin{tikzcd}
	& 0 & 0 & 0 \\
	0 & K & {j_! j^! P} & {Q} & 0 \\
	0 & K & \EC & P & 0 \\
	0 & 0 & {i_* i^* P} & {i_* i^* P} & 0 \\
	& 0 & 0 & 0
	\arrow[from=1-2, to=2-2]
	\arrow[from=2-2, to=3-2]
	\arrow[from=3-2, to=4-2]
	\arrow[from=4-2, to=5-2]
	\arrow[from=4-2, to=4-3]
	\arrow[from=4-3, to=4-4]
	\arrow[from=4-4, to=4-5]
	\arrow[from=4-1, to=4-2]
	\arrow[from=3-1, to=3-2]
	\arrow[from=3-2, to=3-3]
	\arrow["q", from=3-3, to=3-4]
	\arrow[from=3-4, to=3-5]
	\arrow[from=2-1, to=2-2]
	\arrow[from=2-2, to=2-3]
	\arrow[from=2-3, to=2-4]
	\arrow[from=2-4, to=2-5]
	\arrow[from=1-4, to=2-4]
	\arrow[from=2-4, to=3-4]
	\arrow[from=3-4, to=4-4]
	\arrow[from=4-4, to=5-4]
	\arrow[from=1-3, to=2-3]
	\arrow[from=2-3, to=3-3]
	\arrow[from=3-3, to=4-3]
	\arrow[from=4-3, to=5-3]
\end{tikzcd}
\end{equation}
Since $P$ is projective, there is a map $\a: P \to \EC$  making the following diagram commute.
% https://q.uiver.app/?q=WzAsNCxbMiwwLCJQIl0sWzEsMCwiXFxFQyJdLFswLDAsIlAiXSxbMSwxLCJpXyppXipQIl0sWzIsMSwiXFxhIl0sWzEsMCwicSIsMCx7InN0eWxlIjp7ImhlYWQiOnsibmFtZSI6ImVwaSJ9fX1dLFsyLDMsIiIsMix7InN0eWxlIjp7ImhlYWQiOnsibmFtZSI6ImVwaSJ9fX1dLFsxLDMsIiIsMix7InN0eWxlIjp7ImhlYWQiOnsibmFtZSI6ImVwaSJ9fX1dLFswLDMsIiIsMCx7InN0eWxlIjp7ImhlYWQiOnsibmFtZSI6ImVwaSJ9fX1dXQ==
\[
\begin{tikzcd}
	P & \EC & P \\
	& {i_*i^*P}
	\arrow["\a", from=1-1, to=1-2]
	\arrow["q", two heads, from=1-2, to=1-3]
	\arrow[two heads, from=1-1, to=2-2]
	\arrow[two heads, from=1-2, to=2-2]
	\arrow[two heads, from=1-3, to=2-2]
\end{tikzcd}
\]
Since $P$ is indecomposable, the endomorphism $q \circ \a : P \to P$ is either nilpotent or an automorphism. Since $i_* i^* P$ is nonzero, this map is not nilpotent, and hence is an automorphism of $P$. In particular, the first two rows of Diagram (\ref{Ext2}) split. Hence $K$ is a quotient of $j_! j^! P$. Since $j_! j^! P$ has no nonzero quotient in $\AC^Z$, $K =0$. The first result follows.

The second result holds by the dual argument.
\end{proof}

\begin{proof}[Proof of Theorem \ref{eStratTheorem}]
The implication $(1) \implies (2)$ is obvious. 

For the remainder of the proof, fix a maximal element $\l \in \L$ and recollement
\begin{equation}\label{eStratProofRec}
\begin{tikzcd}
	{\AC_{< \l}} && \AC && {\AC_{\l}}
	\arrow["{i_{*}}", from=1-1, to=1-3]
	\arrow["{i^{*}}"', shift right=5, from=1-3, to=1-1]
	\arrow["{i^{!}}", shift left=5, from=1-3, to=1-1]
	\arrow["{j^{*}}", from=1-3, to=1-5]
	\arrow["{j_{!}}"', shift right=5, from=1-5, to=1-3]
	\arrow["{j_{*}}", shift left=5, from=1-5, to=1-3]
\end{tikzcd}
\end{equation}


$(2) \implies (3)$. Let $P (b) \in \AC$ be a projective indecomposable object. Suppose that every projective indecomposable, $P(b')$, in $\AC_{< \l}$ has a filtration by $\e$-standard objects, $\D_{\e} (b'')$, in which $\rho (b'') \geq \rho (b')$.
In particular $i_* i^* P(b)$ has a filtration by $\e$-standard objects, $\D_{\e} (b')$, in which $\rho (b') \geq \rho (b)$.

By Lemma \ref{2homologicalcorollary}, if the recollement (\ref{eStratProofRec}) is 2-homological then $P(b)$ fits into a short exact sequence 
\[
0 \to j_! j^* P(b) \to P(b) \to i_* i^* P(b) \to 0
\]

If $j_*$ is exact then $j^* P (b)$ is projective (since $j^*$ is the left adjoint of an exact functor) and so $j_! j^* P (b)$ is a direct sum of standard objects.
If $j_!$ is exact then $j_! j^* P (b)$ has a filtration by proper standard objects.

In particular, in either case $\e (\l) = \pm$, $P(b)$ has a  filtration by $\e$-standard objects, $\D_{\e} (b')$, in which $\rho (b') \geq \rho (b)$.
The result follows by induction on $|\L|$.


$(3) \implies (1)$. Brundan-Stroppel show that if $\AC$ is $\e$-stratified then $\AC$ has an $\e$-exact stratification \cite[Theorem 3.5]{BS18} and
\begin{equation}\label{FamousExtCondition}
\Ext^{n}_{\AC} (\D_{\e} (b), \nabla_{\e} (b')) = 0
\end{equation}
 for all $b,b' \in B$ and $n \in \NM$ \cite[Theorem 3.14]{BS18}. 

Let $\AC$ be $\e$-stratified abelian length category.
Let $P$ be a projective object in $\AC_{< \l}$, and $I$ be an injective object in $\AC_{< \l}$. Then $P$ and $i_*P$ have a filtration by $\e$-standard objects, and $I$ and $i_* I$ have a filtration by $\e$-costandard objects. Since $\AC$ is a length category, then it follows from (\ref{FamousExtCondition}) that $\Ext^{n}_{\AC}  (i_* P, i_* I) = 0$. Since this is true for any projective object $P$ and injective object $I$ in $\AC_{< \l}$, the recollement (\ref{eStratProofRec}) is homological (see e.g. \cite[Theorem 3.9]{Psa13}). The result follows by induction on $|\L|$.

\end{proof}

%%%%%%%%%%%%%%%%%%%%%%%%%%%%%%%%%%%%%%%%%%%%%%%%%%%%%%%%%%%%%%%%%%%%%%%%%%%%%%%%
\subsection{Highest weight categories}

Let $\km$ be a field. Say that a $\km$-linear abelian category $\AC$ is a {\em highest weight category} with respect to a finite poset $\L$ if $\AC$ is finite over $\km$, and for every $\l \in \L$ there is a projective indecomposable, $P_{\l}$, that fits into a short exact sequence in $\AC$:
\[
0 \to U_{\l} \to P_{\l} \to \D_{\l} \to 0
\]
in which the following hold:
\begin{enumerate}
\item[(HW1)]
$\End_{\AC} (\D_{\l})$ is a division ring for all $\l \in \L$.
\item[(HW2)]
$\Hom_{\AC} (\D_{\l}, \D_{\mu}) = 0$ whenever $\l \nleq \mu$.% $\l > \mu$.
\item[(HW3)]
$U_{\l}$ has a filtration by objects $\D_{\mu}$ in which $\l < \mu$.
\item[(HW4)]
$\bigoplus_{\l \in \L} P_{\l}$ is a projective generator of $\AC$.
\end{enumerate}

The following characterisation of highest weight categories is shown by Krause \cite[Theorem 3.4]{Kra17a}. We give a new proof using Theorem \ref{eStratTheorem}.

\begin{cor}\label{hwcresult}
Let $\km$ be a field, and let $\AC$ be a $\km$-linear abelian category. The following statements are equivalent.
\begin{enumerate}
\item[(1)]
$\AC$ is a highest weight category.
\item[(2)]
$\AC$ has a homological stratification with respect to $\L$  in which every strata category is equivalent to $\rmod \G_{\l}$ for some finite dimensional division algebra $\G_{\l}$.
\item[(3)]
$\AC$ has a 2-homological stratification with respect to $\L$  in which every strata category is equivalent to $\rmod \G_{\l}$ for some finite dimensional division algebra $\G_{\l}$.
\end{enumerate}
\end{cor}

\begin{proof}
%{\em Step 1. (i) implies (ii).}
$(1) \implies (2)$.
If $\AC$ is a highest weight category with respect to $\L$, then a homological stratification of $\AC$ by $\L$ is constructed as follows: For a lower subposet $\L' \subset \L$ define $\AC_{\L'}$ to be the Serre subcategory of $\AC$ generated by the standard objects $\D_{\l}$ in which $\l \in \L'$. Then for any maximal $\mu \in \L'$ there is a homological recollement of abelian categories
\[
\begin{tikzcd}
	{\AC_{\L' \setminus \{\mu\}}} && \AC_{\L'} && {\rmod \End_{\AC} (\D_{\l})}
	\arrow["{i_{*}}", from=1-1, to=1-3]
	\arrow["{i^{*}}"', shift right=5, from=1-3, to=1-1]
	\arrow["{i^{!}}", shift left=5, from=1-3, to=1-1]
	\arrow["{ j^{*}}", from=1-3, to=1-5]
	\arrow["{j_{!}}"', shift right=5, from=1-5, to=1-3]
	\arrow["{j_{*}}", shift left=5, from=1-5, to=1-3]
\end{tikzcd}
\]
in which $j^* = \Hom_{\AC_{\L'}} (\D_{\mu}, -)$ (see the proof of \cite[Theorem 3.4]{Kra17a}).

%{\em Step 2. (ii) implies (iii).} 
$(2) \implies (3).$ This is obvious.

%{\em Step 3. (iii) implies (i).}
$(3) \implies (1).$
 Suppose $\AC$ has a 2-homological stratification with strata categories of the form $\AC_{\l} = \rmod \G_{\l}$ for a finite dimensional division ring $\G_{\l}$. Then $\AC$ is finite (by Corollary \ref{recolenoughprojcor}).
Let $L_{\l}$ denote the unique simple object in $\AC_{\l}$. Define $\D_{\l} := j_{!}^{\l} L_{\l}$ and let $P_{\l}$ be the projective cover of $j_{!*}^{\l} L_{\l}$ in $\AC$. Statement (HW1) holds since $j_!$ is fully-faithful, Statement (HW2) is exactly equation (\ref{exceptional}), Statement (HW3) follows from Theorem \ref{eStratTheorem} (since each $j_{!}^{\l}$ is exact), and Statement (HW4) is obvious. 
\end{proof}



%
%Let $\AC$ be an abelian category with a stratification by a finite poset $\L$ in which every strata category is semisimple. Then all standard objects in $\AC$ are proper standard objects (and all costandard objects are proper costandard objects). Moreover the functors $j^{\l}_{*}, j^{\l}_{!} : \AC_{\l} \to \AC$ are exact for all $\l \in \L$. 
%If an abelian category $\AC$ has a stratification in which each strata category is semisimple, then all standard objects in $\AC$ are proper standard objects (and all costandard objects are proper costandard objects). In particular, if such an abelian category $\AC$ is $\e$-stratified for 
%
%An important special case of $\e$-stratified categories are highest weight categories (originally defined in \cite{CPS88b}). A finite abelian category $\AC$ is a {\em highest weight category} with respect to a poset $\L$ if $\AC$ has a stratification by the poset $\L$ in which each strata category $\AC_{\l}$ has a unique simple object.
%
%A finite abelian category $\AC$ is a {\em highest weight category} with respect to a poset $\L$ (in the sense of \cite{CPS88b}) if $\AC$ is $\e$-stratified 

%%%%%%%%%%%%%%%%%%%%%%%%%%%%%%%%%%%%%%%%%%%%%%%%%%%%%%%%%%%%%%%%%%%%%%%%%%%%%
%\subsection{Geometric application}
%
%\begin{cor}\label{enoughprojEquiv}
%Let $G$ be an algebraic group and let $X$ be a $G$-variety with finitely many orbits, each connected. Let $\L$ be !! and $\e$ !!
%
% Let $\km$ be a field. The category, $P_{G} (X, \km)$, of $G$-equivariant perverse sheaves is an $\e$-stratified finite category  if and only if for each $G$-orbit $\OC_{\l}$ and $x \in \OC_{\l}$, the category $\km [G^{x} / (G^{x})^{\circ}] \mod_{fg}$ is finite over $\km$, and !!
%\end{cor}

%%%%%%%%%%%%%%%%%%%%%%%%%%%%%%%%%%%%%%%%%%%%%%%%%%%%%%%%%%%%%%%%%%%%%%%%%%%%%%%








% ----------------------------------------------------------------
%\newcommand{\urlprefix}{}
\bibliographystyle{alpha}
\bibliography{bibliography/bibliography}
% ----------------------------------------------------------------
\vspace{1cm}
\begin{thebibliography}{MOY98}

%
%
%\bibitem[AB88]{AB88}
%Kaan Akin, David A. Buchsbaum.
%\newblock Characteristic-free representation theory of the general linear group, II. Homological considerations.
%\newblock {\em Adv. Math.} 72:171-210, 1988

\bibitem[Ach21]{Ach21}
Pramod N. Achar.
\newblock {\em Perverse sheaves and applications to representation theory}.
\newblock Mathematical Surveys and Monographs, no. 258, {\em American Mathematical Society}, Providence, RI, 2021

\bibitem[ADL98]{ADL98}
Istvan  \'Agoston, Vlastimil Dlab, Erzsebet Luk\'acs.
\newblock Stratified algebras.
\newblock {\em Math. Rep. Acad. Sci. Canada} 20:22-28, 1998

%\bibitem[ADLU98]{ADLU98}
%Istvan \'Agoston, Vlastimil Dlab, Erzsebet Luk\'acs, Luise Unger.
%\newblock Standardly stratified algebras and tilting.
%\newblock {\em J. Algebra} 144-160, 2000

%\bibitem[AH08]{AH08}
%Pramod N. Achar, Anthony Henderson.
%\newblock Orbit closures in the enhanced nilpotent cone.
%\newblock {\em Adv. Math.} 219(1):27-62, 2008
%
%\bibitem[AHJR14]{AHJR14}
%Pramod N. Achar, Anthony Henderson, Daniel Juteau, Simon Riche.
%\newblock Weyl group actions on the Springer sheaf.
%\newblock {\em Proc. Lond. Math. Soc.} 108(6):1501-1528, 2014
%
%\bibitem[AHJR16]{AHJR16}
%Pramod N. Achar, Anthony Henderson, Daniel Juteau, Simon Riche.
%\newblock Modular generalized Springer correspondence I: the general linear group.
%\newblock {\em J. Eur. Math. Soc.} 18:1405-1436, 2016
%
%\bibitem[AHR15]{AHR15}
%Pramod N. Achar, Anthony Henderson, Simon Riche.
%\newblock Geometric Satake, Springer correspondence, and small representations II.
%\newblock {\em Represent. Theory} 19:94-166, 2015
%
%\bibitem[Aki89]{Aki89}
%Kaan Akin.
%\newblock Extensions of symmetric tensors by alternating tensors.
%\newblock {\em J. Algebra} 121(2):358-363, 1989
%
%\bibitem[AM15]{AM12}
%Pramod N. Achar, Carl Mautner.
%\newblock Sheaves on nilpotent cones, Fourier transform, and a geometric Ringel duality.
%\newblock {\em Mosc. Math. J.} 15(3):407-423, 2015
%
%\bibitem[AR17]{AR17}
%Cosima Aquilino, Rebecca Reischuk.
%\newblock The monoidal structure on strict polynomial functors.
%\newblock  {\em J. Algebra} 485:213-229, 2017

\bibitem[BBD82]{BBD82}
Alexander Beilinson, Joseph Bernstein, Pierre Deligne.
\newblock Faisceaux pervers.
\newblock {\em Ast\'{e}risque} 100, 1982



%\bibitem[BD09]{BD09}
%David Benson, Stephen Doty.
%\newblock Schur-Weyl duality over finite fields.
%\newblock {\em Arch. Math. (Basel)} 93(5): 425-435, 2009

\bibitem[BGS96]{BGS96}
Alexander Beilinson, Victor Ginzburg, Wolfgang Soergel.
\newblock Koszul duality patterns in representation theory.
\newblock {\em J. Amer. Math. Soc.} volume 9, 2:473-527, 1996

%\bibitem[BK89]{BK89}
%Alexei I. Bondal, Mikhail M. Kapranov.
%\newblock Representable functors, Serre functors, and mutations.
%\newblock {\em Izv. Akad. Nauk SSSR Ser. Mat.} 53(6):1183-1205, 1989

\bibitem[BL94]{BL94}
Joseph Bernstein, Valery Lunts.
\newblock {\em Equivariant sheaves and functors}.
\newblock Lecture Notes in Mathematics, volume 1578, {\em Springer-Verlag}, 1994 
%%%%%

%\bibitem[BLM90]{BLM90}
%Alexander Beilinson, George Lusztig, Robert MacPherson.
%\newblock A geometric setting for the quantum deformation of $\GL_n$.
%\newblock {\em Duke Math. J.} 61:655-677, 1990


%\bibitem[BM19]{BM19}
%Tom Braden, Carl Mautner.
%\newblock Ringel duality for perverse sheaves on hypertoric varieties.
%\newblock {\em Adv. Math.} 344:35-98, 2019

%\bibitem[BO01]{BO01}
%Alexei Bondal, Dmitri Orlov.
%\newblock Reconstruction of a variety from the derived category and groups of autoequivalences.
%\newblock {\em Compositio Mathematica} 125:327-344, 2001

%\bibitem[BR22]{BR21}
%Adam Brown, Anna Romanov.
%\newblock Contravariant pairings between standard Whittaker modules and Verma modules.
%\newblock {\em J. Algebra} 609:145-179, 2022

\bibitem[BS24]{BS18}
Jonathan Brundan, Catharina Stroppel.
\newblock Semi-infinite highest weight categories.
\newblock {\em Mem. Amer. Math. Soc.} 293:1-152, 2024

%\bibitem[Bry09]{Bry09}
%R. M. Bryant.
%\newblock Lie powers of infinite-dimensional modules.
%\newblock {\em Beitr\"age Algebra Geom.} 50:179-193, 2009

%\bibitem[Cha08]{Cha08}
%Marcin Cha\strokel upnik. 
%\newblock Koszul duality and extensions of exponential functors.
%\newblock {\em Adv. Math.} 208(3): 969-982, 2008


%\bibitem[CG97]{CG97}
%Neil Chriss, Victor Ginzburg.
%\newblock Representation Theory and complex geometry.
%\newblock {\em Birkh\"{a}user Boston Inc.}, Boston MA, 1997 

%\bibitem[Ch08]{Ch08}
%Marcin Cha?upnik.
%\newblock Koszul duality and extensions of exponential functors.
%\newblock  {\em Adv. Math} 218(3):969-982, 2008
%possibly the original use of ringel duality for strict polynomial functors

%\bibitem[CKM14]{CKM14}
%Sabin Cautis, Joel Kamnitzer, Scott Morrison.
%\newblock Webs and quantum skew Howe duality.
%\newblock {\em Mathematische Annalen} 360:351-390, 2014
%
%\bibitem[CM93]{CM93}
%David H. Collingwood, William M. McGovern.
%\newblock {\em Nilpotent orbits in semisimple Lie algebras},
%\newblock Van Nostrand Reinhold Co., New York, 1993
%
%\bibitem[CL74]{CL74}
%Roger W. Carter, George Lusztig.
%\newblock On the modular representations of the general linear and symmetric groups.
%\newblock  {\em Math Z.} 136:193-242, 1974
%Modular Schur-Weyl see Mautner phd

%\bibitem[CPS88a]{CPS88a}
%E. Cline, B.J. Parshall, L.L Scott.
%\newblock Algebraic stratification in representation categories.
%\newblock {\em J. Algebra} 117:504-521, 1988

\bibitem[CPS88]{CPS88b}
Edward Cline, Brian J. Parshall, Leonard L. Scott.
\newblock Finite-dimensional algebras and highest weight categories.
\newblock {\em J. Reine Angew Math} 391:85-99, 1988

\bibitem[CPS96]{CPS96}
Edward Cline, Brian J. Parshall, Leonard L. Scott.
\newblock Stratifying endomorphism algebras.
\newblock {\em Mem. Amer. Math. Soc.} 124:1-119, 1996

\bibitem[CW22]{CW21}
Alessio Cipriani, Jon Woolf.
\newblock When are there enough projective perverse sheaves?
\newblock {\em Glasgow Mathematical Journal} 64(1):185-196, 2022

\bibitem[CZ19]{CZ19}
Kevin Coulembier, Ruibin Zhang.
\newblock Borelic pairs for stratified algebras.
\newblock {\em Adv. Math.} 345:53-115, 2019

%\bibitem[Day70]{Day70}
%Brian Day.
%\newblock Construction of biclosed categories.
%\newblock Ph.D. thesis, University of New South Wales, 1970
%
%
%\bibitem[DG02]{DG02}
%Stephen Doty, Anthony Giaquinto.
%\newblock Presenting Schur Algebras.
%\newblock {\em Int. Matg. Res. Not.} 36:1907-1944, 2002
%
%
%\bibitem[DiG95]{DiG95}
%Persi Diaconis, Anil Gangolli.
%\newblock Rectangular arrays with fixed margins,
%\newblock in: {\em Discrete Probability and Algorithms},
%\newblock Springer-Verlag, Berlin/New York, 15-41, 1995 
%
%\bibitem[DGS09]{DGS09}
%Stephen Doty, Anthony Giaquinto, John Sullivan.
%\newblock On the defining relations for generalized $q$-Schur algebras.
%\newblock {\em Adv. in Math.} 221:955-982, 2009

\bibitem[Dla96]{Dla96}
Vlastimil Dlab.
\newblock Quasi-hereditary algebras revisited.
\newblock {\em An. Stiin. Univ. Ovidius Constantza} 4:43-54, 1996 

%\bibitem[Don86]{Don86}
%Stephen Donkin.
%\newblock On Schur algebras and related algebras I.
%\newblock {\em J. Algebra} 104:310-328, 1986

%\bibitem[Don87]{Don87}
%Stephen Donkin.
%\newblock On Schur algebras and related algebras II.
%\newblock {\em J. Algebra} 111(2):354-364, 1987



%\bibitem[Don93]{Don93}
%Stephen Donkin.
%\newblock On tilting modules for algebraic groups.
%\newblock {\em Mathematische Zeitschrift} 212:39-60, 1993
%Schur algebra is Ringel self dual

%\bibitem[Don94a]{Don94b}
%Stephen Donkin.
%\newblock On Schur algebras and related algebras III: integral representations.
%\newblock {\em Math. Proc. of Cambridge Philosophical Society} 116(1):37-55, 1994

%\bibitem[Don94b]{Don94b}
%Stephen Donkin.
%\newblock On Schur algebras and related algebras IV. The blocks of the Schur algebras.
%\newblock {\em J. Algebra} 168(2):400-429, 1994

%\bibitem[Dot03]{Dot03}
%Stephen Doty.
%\newblock Presenting generalized $q$-Schur algebras.
%\newblock {\em Representation Theory} 7:196-213 (electronic), 2003.

%\bibitem[Dot09]{Dot09}
%Stephen Doty.
%\newblock Schur-Weyl duality in positive characteristic.
%\newblock {\em Contemporary Math.} 478:15-28, 2009.

%\bibitem[DR04]{DR04}
%James M. Douglass, Gerhand R\"{o}hrle.
%\newblock The geometry of generalized Steinberg varieties.
%\newblock {\em Advances in Mathematics} 187(2):396-416, 2004.
%%They also have another paper called Homology of generalized steinberg varieties that might be relevant

%\bibitem[DR09]{DR09}
%James M. Douglass, Gerhand R\"{o}hrle.
%\newblock The Steinberg variety and representations of reductive groups.
%\newblock {\em J. Algebra} 321(11):3158-3196, 2009.


\bibitem[Fri07]{Fri07}
Anders Frisk.
\newblock Dlab's theorem and tilting modules for stratified algebras.
\newblock {\em J. Algebra} 314:507-537, 2007.

\bibitem[FP04]{FP04}
Vincent Franjou, Teimuraz Pirashvili.
\newblock Comparison of abelian categories recollements.
\newblock {\em Documenta Mathematica} 9:41-56, 2004.

%\bibitem[FS97]{FS97}
%Eric M. Friedlander, Andrei Suslin.
%\newblock Cohomology of finite group schemes over a field.
%\newblock {\em Inventiones Mathematicae} 127(2):209-270, 1997

\bibitem[Gab73]{Gab73}
Pierre Gabriel
\newblock Indecomposable representations II.
\newblock in: {\em Symposia Mathematica, vol. 11.}
\newblock {\em Academic Press}: pp 81-104, 1973

\bibitem[Wit65]{Wit65}
Hassler Whitney.
\newblock Local properties of analytic varieties.
\newblock in: {\em Differential and Combinatorial Topology (A Symposium in Honor of Marston Morse).}
\newblock {\em Princeton University Press}: pp 205-244, Princeton N.J., 1965

%\bibitem[Ger59]{Ger59}
%Murray Gerstenhaber.
%\newblock On nilalgebras and linear varieties of nilpotent matrices, III.
%\newblock {\em Annals of Mathematics} 70(1):167-205, 1959

%\bibitem[Gou22]{Gou22}
%Valentin Gouttard.
%\newblock Perverse monodromic sheaves.
%\newblock {\em J. London Math. Soc.} 106(1):388-424, 2022

%\bibitem[Gre80]{Gr80}
%James A. Green.
%\newblock {\em Polynomial representations of $GL_n$}, 
%\newblock volume 830 of {\em Lecture notes in mathematics}.
%\newblock Springer, Berlin, 1980

%\bibitem[Hap87]{Hap87}
%Dieter Happel.
%\newblock On the derived category of a finite-dimensional algebra.
%\newblock {\em Commentarii mathematici Helvetici} 62:339-389, 1987

%\bibitem[Hen15]{Hen15}
%Anthony Henderson.
%\newblock Singularities of nilpotent orbit closures.
%\newblock {\em  Revue Roumaine de Mathematiques Pures et Appliquees} 60(4):441-469, 2015
%
%\bibitem[HR82]{HR82}
%Dieter Happel, Claus M. Ringel.
%\newblock Tilted Algebras.
%\newblock {\em Trans. Amer. Math. Soc.} 274:339-443, 1982

%\bibitem[HTT08]{HTT08}
%Ryoshi Hotta, Kiyoshi Takeuchi, Toshiyuki Tanisaki.
%\newblock{\em D-Modules, Perverse Sheaves, and Representation Theory},
%\newblock English Edition, Birkhauser, Boston, 2008


%
%\bibitem[JK81]{JK81}
%Gordon Douglas James, Adalbert Kerber.
%\newblock {\em The Representation Theory of the Symmetric Group},
%\newblock volume 16 of {\em Encyclopedia of Mathematics and its Applications}.
%\newblock Addison-Wesley, Reading, MA, 1981
%
%\bibitem[JMW12]{JMW12}
%Daniel Juteau, Carl Mautner, Geordie Williamson.
%\newblock Perverse sheaves and modular representation theory,
%\newblock in: {\em Geometric methods in representation theory II},
%\newblock {\em S\'eminares et Congr\`es} 24:313-350, 2012
%add that this is a good source of extra information

%\bibitem[Jut07]{Jut07}
%Daniel Juteau.
%\newblock Modular Springer correspondence and decomposition matrices.
%\newblock Ph.D. thesis, Universit\'{e} Paris, 2007

%\bibitem[KK99]{KK99}
%Michael Klucznik, Steffen K\"{o}nig.
%\newblock Characteristic tilting modules over quasi-hereditary algebras


%\bibitem[Ko66]{Ko66}
%Bertram Kostant.
%\newblock Groups over Z,
%\newblock in: {\em Algebraic Groups and Discontinuous Subgroups},


%\bibitem[Kra13]{Kra13}
%Henning Krause.
%\newblock Koszul, Ringel and Serre duality for strict polynomial functors.
%\newblock {\em Compositio Mathematica} 149(6):996-1018, 2013

\bibitem[Kra15]{Kra15}
Henning Krause.
\newblock Krull-Schmidt categories and projective covers.
\newblock {\em Expositiones Mathematicae} 33(4):535-549, 2015


\bibitem[Kra17]{Kra17a}
Henning Krause.
\newblock Highest weight categories and recollements.
\newblock {\em Annales de l'Institut Fourier } 67(6):2679-2701, 2017

%\bibitem[Kra17b]{Kra17b}
%Henning Krause.
%\newblock Highest weight categories and strict polynomial functors.
%\newblock {\em Representation theory - Current trends and perspectives},
%\newblock European Mathematical Society,
%pp. 331-373, 2017

%\bibitem[KS90]{KS90}
%Masaki Kashiwara, Pierre Schapira.
%\newblock {\em Sheaves on manifolds},
%\newblock volume 292 of {\em Grundlehren der mathematischen Wissenschaften}.
%\newblock Springer-Verlag, 1990

\bibitem[Kuh94]{Kuh94}
Nicholas Kuhn.
\newblock Generic representations of the finite general linear groups and the Steenrod algebra: II.
\newblock {\em K-Theory } 8:395-428, 1994



%\bibitem[Lus81]{Lu81}
%George Lusztig.
%\newblock Green polynomials and singularities of unipotent classes.
%\newblock {\em Adv. Math.} 42:169-178, 1981
%
%\bibitem[Lus84]{Lu84}
%George Lusztig.
%\newblock Intersection cohomology complexes on a reductive group.
%\newblock {\em Invent. Math.} 75:205-272, 1984
%
%\bibitem[Lus93]{Lu93}
%George Lusztig.
%\newblock {\em Introduction to quantum groups},
%\newblock volume 110 of {\em Progress in Mathematics}.
%\newblock Birkh\"{a}user Boston Inc., Boston, MA, 1993

\bibitem[LW15]{LW15}
Ivan Losev, Ben Webster.
\newblock On uniqueness of tensor products of irreducible categorifications.
\newblock {\em Selecta Math.} 21:345-377, 2015

%\bibitem[Mau10]{Mau10}
%Carl Mautner.
%\newblock Sheaf theoretic methods in modular representation theory.
%\newblock Ph.D. thesis, University of Texas at Austin, 2010

%\bibitem[Mau14]{Mau14}
%Carl Mautner.
%\newblock A geometric Schur functor.
%\newblock {\em Selecta Math. (N.S.)} 20(4):961-977, 2014

%\bibitem[M\'en01]{Men01}
%Miguel A. M\'endez.
%\newblock Directed graphs and the combinatorics of the polynomial representations of $GL_n (\CM)$.
%\newblock {\em Annals of Combinatorics} 5:459-478, 2001
%
%\bibitem[Mir04]{Mir04}
%Ivan Mircovi\'{c}.
%\newblock Character sheaves on reductive Lie algebras.
%\newblock {\em Mosc. Math. J.} 4(4):897-910, 2004
%
%\bibitem[Miy86]{Miy86}
%Yoichi Miyashita.
%\newblock Tilting modules of finite projective dimension.
%\newblock {\em Math. Zeit.} 193:113-146, 1986 
%
%\bibitem[MS08]{MS08}
%Volodymyr Mazorchuk, Catharina Stroppel.
%\newblock Projective-injective modules, Serre functors and symmetric algebras.
%\newblock {\em J. Reine Angew. Math.} 616:131-165, 2008
%%use as reference to see more about ringel and serre functors
%
\bibitem[MV86]{MV86}
Robert MacPherson, Kari Vilonen.
\newblock Elementary construction of perverse sheaves.
\newblock {\em Invent. Math.} 84:403-435, 1986
%
%
%\bibitem[MV07]{MV07}
%Ivan Mircovi\'{c}, Kari Vilonen.
%\newblock Geometric Langlands duality and representations of algebraic groups over commutative rings.
%\newblock {\em Annals of Mathematics} 166(1):95-143, 2007

\bibitem[Psa14]{Psa13}
Chrysostomos Psaroudakis.
\newblock Homological theory of recollements of abelian categories.
\newblock {\em J. Algebra} 398:63-110, 2014

%\bibitem[Rei16]{Rei16}
%Rebecca Reischuk.
%\newblock The monoidal structure on strict polynomial functors.
%\newblock Ph.D. thesis, Universit\"{a}t Bielefeld, 2016

%\bibitem[Rei16]{Rei16}
%Rebecca Reischuk.
%\newblock The adjoints of the Schur Functor.
%\newblock Preprint. https://arxiv.org/abs/1601.03513. 2016

%\bibitem[Ric89]{Ric89}
%Jeremy Rickard.
%\newblock Morita theory for derived categories.
%\newblock {\em J. London Math. Soc.} s2-39(3):436-456, 1989
%
\bibitem[Rin91]{Rin91}
Claus M. Ringel.
\newblock The category of modules with good filtrations over a quasi-hereditary algebra has almost split sequences.
\newblock {\em Math. Zeit.} 208:209-223, 1991

%\bibitem[Rou08]{Rou08}
%Raphael Rouquier.
%\newblock q-Scur algebras and complex reflection groups.
%\newblock {\em Mosc. Math. J.} 8(1):119-158, 2008

%\bibitem[Sch1901]{Sch01}
%Issai Schur.
%\newblock \"{U}ber eine Klasse von Matrizen, die sich einer gegebenen Matrix zuordnen lasse.
%\newblock Dissertation, Berlin, 1901
%
%
%\bibitem[Ser87]{Ser87}
%Jean-Pierre Serre.
%\newblock Complex Semisimple Lie Algebras.
%\newblock {\em Springer-Verlag}, New York, 1987
%
%\bibitem[Tot97]{Tot97}
%B. Totaro.
%\newblock Projective resolutions of representations of $\GL (n)$
%\newblock {\em J. Reine Angew. Math}, 482, 1997
%
%%\bibitem[Tou13]{Tou13}
%%Antoine Touz\'e.
%%\newblock Ringel duality and derivatives of non-additive functors.
%%\newblock {\em J. Pure Appl. Algebra} 217(9):1642-1673, 2013
%%%A use of Ringel duality functor for strict polynomial functors, together with Chalupnik. The introduction has a 
%
%\bibitem[Tou14]{Tou14}
%Antoine Touz\'e.
%\newblock Applications of functor (co)homology.
%\newblock in: {\em An alpine expedition through algebraic topology},
%\newblock volume 617 of {\em Contemporary Mathematics}: pp. 259-277, 
%American Mathematical Society, Providence, RI, 2014
%2014
%
%
%\bibitem[W19]{W19}
%Giulian Wiggins.
%\newblock Presentations of categories of modules using the Cautis-Kamnitzer-Morrison principle.
%\newblock {\em Journal of Combinatorial Algebra} 3(1):71-112, 2019
% 
%\bibitem[W21]{W21}
%Giulian Wiggins.
%\newblock Rectangular arrays with fixed margins.
%\newblock GitHub repository,
%\newblock https://github.com/giulianwiggins/Rectangular-arrays-with-fixed-margins, 2021

\bibitem[Wig23]{Wig23}
Giulian Wiggins.
\newblock Stratified categories and a geometric approach to representations of the Schur algebra.
\newblock PhD Thesis. The University of Sydney, 2023

\bibitem[Wit65]{Wit65}
Hassler Whitney.
\newblock Local properties of analytic varieties.
\newblock in: {\em Differential and Combinatorial Topology (A Symposium in Honor of Marston Morse).}
\newblock {\em Princeton University Press}: pp 205-244, Princeton N.J., 1965

\end{thebibliography}
\end{document}

