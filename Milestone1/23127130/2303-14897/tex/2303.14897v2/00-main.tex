\documentclass[10pt,twocolumn,letterpaper]{article}
\usepackage{iccv}             
\usepackage{times}
\usepackage{epsfig}
\usepackage{graphicx}
\usepackage{amsmath,bm}
\usepackage{amssymb}
\usepackage{multirow}
\usepackage{tabstackengine}
\usepackage{ctable}
\usepackage{threeparttable, tablefootnote}
\usepackage{array, caption, floatrow, makecell, booktabs}
\floatsetup[table]{capposition=top}
\TABstackMath
\TABstackMathstyle{\displaystyle}
\setstackgap{S}{12pt}
\setstacktabbedgap{7pt}
\TABbinary
% Include other packages here, before hyperref.
\newcommand{\tablestyle}[2]{\setlength{\tabcolsep}{#1}\renewcommand{\arraystretch}{#2}\centering\footnotesize}
% If you comment hyperref and then uncomment it, you should delete
% egpaper.aux before re-running latex.  (Or just hit 'q' on the first latex
% run, let it finish, and you should be clear).
\usepackage[pagebackref=true,breaklinks=true,letterpaper=true,colorlinks,bookmarks=false]{hyperref}

\iccvfinalcopy % *** Uncomment this line for the final submission

\def\iccvPaperID{***} % *** Enter the ICCV Paper ID here
\def\httilde{\mbox{\tt\raisebox{-.5ex}{\symbol{126}}}}

% Pages are numbered in submission mode, and unnumbered in camera-ready
%\ificcvfinal\pagestyle{empty}\fi


\newcommand{\authnote}[2]{$\ll$\textsf{\footnotesize #1 notes: #2}$\gg$}

\newcommand{\gu}[1]{{\color{orange}\authnote{Gu}{#1}}}
\newcommand{\yang}[1]{{\color{blue}\authnote{Yang}{#1}}}
\newcommand{\chuan}[1]{{\color{red}\authnote{Chuan}{#1}}}
\newcommand{\js}[1]{{\color{teal}[Jiaming: #1]}}



\begin{document}

%%%%%%%%% TITLE
\title{Seer: Language Instructed Video Prediction with Latent Diffusion Models}

\author{
Xianfan~Gu\textsuperscript{1}
\quad Chuan~Wen\textsuperscript{1,2,3}
\quad Jiaming Song\textsuperscript{4}
\quad Yang~Gao\textsuperscript{1,2,3}\\
\textsuperscript{1}Shanghai Qi Zhi Institute \quad \textsuperscript{2}IIIS, Tsinghua University \\
\textsuperscript{3}Shanghai Artificial Intelligence Laboratory \quad
\textsuperscript{4}NVIDIA\\
}
\maketitle
% Remove page # from the first page of camera-ready.
%\ificcvfinal\thispagestyle{empty}\fi
%%%%%%%%% ABSTRACT
\begin{abstract}
Imagining the future trajectory is the key for robots to make sound planning and successfully reach their goals. Therefore, text-conditioned video prediction (TVP) is an essential task to facilitate general robot policy learning, i.e., predicting future video frames with a given language instruction and reference frames. It is a highly challenging task to ground task-level goals specified by instructions and high-fidelity frames together, requiring large-scale data and computation. To tackle this task and empower robots with the ability to foresee the future, we propose a sample and computation-efficient model, named \textbf{Seer}, by inflating the pretrained text-to-image (T2I) stable diffusion models along the temporal axis. We inflate the denoising U-Net and language conditioning model with two novel techniques, Autoregressive Spatial-Temporal Attention and Frame Sequential Text Decomposer, to propagate the rich prior knowledge in the pretrained T2I models across the frames. With the well-designed architecture, Seer makes it possible to generate high-fidelity, coherent, and instruction-aligned video frames by fine-tuning a few layers on a small amount of data. The experimental results on Something Something V2 (SSv2) and Bridgedata datasets demonstrate our superior video prediction performance with around 210-hour training on 4 RTX 3090 GPUs: decreasing the FVD of the current SOTA model from 290 to 200 on SSv2 and achieving at least 70\% preference in the human evaluation. \url{https://seervideodiffusion.github.io/}
\end{abstract}

%%%%%%%%% BODY TEXT
Humans are multimodal learners. We communicate with each other about things that we have experienced and knowledge we have gained using our senses---most commonly including sight as well as hearing, touch, smell, and taste. Our communication channel is limited to a single modality---spoken language, signed language, or text---but a reader or listener is expected to use his or her imagination to visualize and reason about the content being described. In general, language is used to describe scenes, events, and images; the words used to describe these are used to conjure up a visual impression in the listener. Therefore, it is natural to consider the types of visual reasoning used in understanding language, and to ask how well we can currently model them with computational methods.

Consider, for instance, the questions in Figure \ref{fig:teaser}. Concreteness is typically correlated with how well a concept can be visually imagined. For example, a concrete word such as \emph{present} often has a unique visual representation. In addition, common associations such as \emph{ocean}$\rightarrow$\emph{blue} (color) and \emph{corn chip}$\rightarrow$\emph{triangle} (shape) reflect properties of an imagined visual representation of the item in question. These properties may be difficult to infer from text alone without prior knowledge gained from visual input; for instance, a number of studies have investigated the partial ability of blind English speakers to predict color associations and how it differs from the intuition of sighted speakers\footnote{This phenomenon is illustrated in \href{https://www.youtube.com/watch?v=59YN8_lg6-U}{this interview} with Tommy Edison, a congenitally blind man, in which he describes his understanding and frequent confusion regarding color associations.}~\cite{van2021blind, saysani2021seeing, saysani2018colour, shepard1992representation, marmor1978age}. 

There has been a wealth of recent research vision-and-language (V\&L) tasks involving both text and image data, and the use of vision-language pretraining (VLP) to create models that are able to reason jointly about both of these modalities together~\cite{chen2020uniter,kim2021vilt,li2021align,chen2022vlp}. Notable in this regard is CLIP~\cite{radford2021learning}, consisting of paired text and image encoders jointly trained on a contrastive objective, that learns to align text and image embeddings in a shared semantic space. On the other hand, text encoder models such as BERT~\cite{devlin2018bert} learn to reason about text in a unimodal vacuum, with knowledge derived from pretraining tasks that only involve textual data.




Prior work has investigated the performance of multimodally trained text encoders on various natural language understanding (NLU) tasks with mixed results, sometimes finding that they are outperformed by unimodal models~\cite{iki2021effect} and at other times suggesting improved performance~\cite{wang2021simvlm}. However, these works fine-tune the models under consideration on NLU tasks before evaluation, making it difficult to disentangle the effects of multimodal pretraining and fine-tuning configuration on the observed performance. Additionally, these works do not address the distinction between NLU tasks requiring implicit visual reasoning and ones that are purely non-visual. We refer to natural language inference involving implicit visual reasoning as \emph{visual language understanding} (VLU) and propose a suite of VLU tasks that may be used to evaluate visual reasoning capabilities of pretrained text encoders, focusing primarily on zero-shot methods.

We compare multimodally trained text encoders such as that of CLIP to BERT and other unimodally trained text encoders, evaluating their performance on our suite of VLU tasks. We evaluate these models in without modifying their internal weights in order to probe their knowledge obtained during pretraining. A key design aspect of these tests is the probing method used to evaluate knowledge. Previous work has probed the knowledge of BERT and similar models using a masked language modelling (MLM) paradigm~\cite{petroni2019language, rogers2020primer}, but this cannot be directly applied to CLIP since it was not pretrained with MLM. We therefore propose a new zero-shot probing method that we term \emph{Stroop probing}. This is based on the psychological Stroop effect~\cite{macleod1991half} (described in Section \ref{sec:probing}), which suggests that salient items should have a stronger interference effect on the representation of their context.

Strikingly, we find that the multimodally trained text encoders under consideration outperform unimodally trained text encoders on VLU tasks, both when comparing to much larger encoders as well as ones of comparable size. We also compare these models on baseline NLU tasks that do not involve visual reasoning and find that models such as CLIP underperform on these tasks, demonstrating that they do not have a global advantage on NLU tasks. We conclude that exposure to images during pretraining improves performance on text-only tasks that require visual reasoning. Furthermore, our findings isolate the effect of the text component of multimodal models for tasks such as text to image generation, providing principled guidelines for understanding the knowledge that such models inject into downstream vision tasks.




\section{Related Work}

\subsection{Text-to-Image Generation}
Since Scott Reed et al.~\citep{reed2016generative} firstly set up the T2I generation task and proposed a GAN-based method, this multi-modal generation task has attracted the attention of the computer vision community. 
DALL-E~\citep{ramesh2021dalle} makes a breakthrough by modeling the T2I generation task as a sequence-to-sequence translation task with a VQ-VAE~\citep{van2017vqvae} and Transformer~\citep{vaswani2017attention}. Since then, many variants have been proposed with an improved image tokenizer~\citep{yu2022parti}, hierarchical Transformers~\citep{ding2022cogview2} or domain-specific knowledge~\citep{gafni2022make}.
With the recent progress of Denoising Diffusion Probabilistic Models (DDPM)~\citep{ddpm}, the diffusion models have been widely used for T2I generation tasks~\citep{glide,imagen,ramesh2022dalle2}. Specifically, GLIDE~\citep{glide} proposes classifier-free guidance for T2I diffusion models to improve image quality. For a better alignment between text and image, DALL-E 2~\citep{ramesh2022dalle2} proposed to denoise CLIP~\citep{radford2021clip} image embedding conditioned on CLIP text embedding, which integrated high-level semantic information. To reduce the computation cost of the denoising process in pixel space, Latent Diffusion Model (LDM) employs VAE~\citep{Kingma2014vae} to operate in the latent space. 
Seer takes advantage of the prior language-vision knowledge of pretrained LDM and inflates it along the time axis.

\subsection{Text-to-Video Generation}
In contrast to the huge success of Text-to-Image (T2I) generation, Text-to-Video (T2V) generation is still underexplored due to the limitation of the large text-video data annotation and computing resources.  Inspired by the various variants of T2I generation, recent T2V studies have attempted to explore compatible variants for video generation modeling. GODIVA~\citep{godiva} first proposes a VQ-VAE based auto-regressive model with three-dimensional sparse attention for T2V generation. N\"UWA~\citep{nuwa} further improves it by designing a 3D encoder with 3D nearby attention and achieves competitive performance on multi-task generation. Unlike the single frame-rate T2V approaches trained from scratch on large-scale text-video datasets, CogVideo~\citep{hong2023cogvideo} proposes a multi-frame-rate hierarchical model for T2V generation. This approach leverages the pre-trained module of T2I CogView-2~\citep{ding2022cogview2}.

Motivated by the remarkable progress of T2I diffusion models~\citep{ramesh2022dalle2,imagen,ldm}, Make-A-Video~\citep{makeavideo}, MagicVideo~\citep{magicvideo}, Tune-A-Video~\citep{tuneavideo} and Imagen Video~\citep{imagenvideo} transfer the 2D diffusion models to 3D models by incorporating temporal modules in T2V generation. In contrast to Imagen Video, all other three methods utilize the prior knowledge of T2I pre-trained model. Similarly, we use the pre-trained weight of the 2D T2I diffusion model in our 3D T2V model. Varying from the aforementioned methods, our method Seer utilizes autoregressive attention on both spatial and temporal spaces to generate high-fidelity and coherent video frames. And Seer is able to handle the task-level video prediction by decomposing the language condition into fine-grained sub-instruction.


\section{Preliminaries}
\noindent\textbf{Denoising Diffusion Probabilistic Models with classifier-free guidance:}
Diffusion models are probabilistic models that approximate the data distribution by iteratively adding noise and denoising through a forward/reverse Gaussian Diffusion Process~\citep{ddpm,song2020score}. The forward process applies noise at each time step $t\in{0,...,T}$ to the data distribution $\mathbf{x}_{0}$, creating a noisy sample $\mathbf{x}_t$ where $\mathbf{x}_t = \sqrt{\bar{\alpha}_t}\mathbf{x}_0+\sqrt{1-\bar{\alpha}_t}\bm{\epsilon}$ ($\bm{\epsilon}\sim\mathcal{N}(\boldsymbol{0},\boldsymbol{I})$), and $\bar{\alpha}_t$ is the accumulation of the noise schedule $\alpha_{0:T}$ defined by $\bar{\alpha}_t=\prod^t_{s=1}\alpha_s$. To denoise images, the diffusion process uses a reparameterized variant of Gaussian noise prediction $\bm{\epsilon}_\theta(\mathbf{x}_t,t)$ targeting Gaussian noise $\bm{\epsilon}$. The reverse process $p(\mathbf{x}_{t-1}|\mathbf{x}_{t})$ of the Markov Chain generates new samples from Gaussian noise, which is approximated by Bayes' theorem as $q(\mathbf{x}_{t-1}|\mathbf{x}_t,\mathbf{x}_0)$, where $\mathbf{x}_0$ is derived from the forward process as $\mathbf{x}_0 = \frac{1}{\sqrt{\bar{\alpha}_t}}(\mathbf{x}_t-\sqrt{1-\bar{\alpha}_t\bm{\epsilon}_\theta(\mathbf{x}_t,t)})$.

Classifier-free guidance~\citep{clsfree} is introduced for conditional diffusion models to generate images without requiring an extra image classifier. A conditional model with a parameterized reverse process $p(\mathbf{x}_{t-1}|\mathbf{x}_t,\mathbf{c})$ uses a conditional identifier $\mathbf{c}$ through $\bm{\epsilon}_{\theta}(\mathbf{x}_t,t,\mathbf{c})$. To predict an unconditional score, the conditional identifier is replaced with a null token $\O$ and denoted as $\bm{\epsilon}_{\theta}(\mathbf{x}_t,t,\mathbf{c}=\O)$. Classifier-free guidance can then be approximated as a linear combination of conditional and unconditional predictions:
\vspace{-3pt}
\begin{equation}
   \bm{\tilde{\epsilon}}_{\theta}(\mathbf{x}_t,t,\mathbf{c}) = (1+w)\bm{\epsilon}_{\theta}(\mathbf{x}_t,t,\mathbf{c})-w\bm{\epsilon}_{\theta}(\mathbf{x}_t,t,\mathbf{c}=\O),
   \vspace{-3pt}
\end{equation}
where $w$ is the guidance scale. Text-video and text-image-based diffusion models~\citep{ldm,imagen,glide,vdm,makeavideo} use DDPM with classifier-free guidance. This diffusion method can be adapted to various tasks with flexibility.

\noindent\textbf{Latent Diffusion Models:} 
Compared with image diffusion, video diffusion has significantly higher computation costs because it needs to process multiple frames.
Recent works have explored the computation-efficient version of diffusion modeling, such as latent diffusion model (LDM)~\citep{ldm}. LDM proposes the VAE-based latent diffusion, including a KL-regularized autoencoder for encoding/decoding latent representation $\bm{\varepsilon}(\mathbf{x})$, and a diffusion model to operate on the latent space $\mathbf{z}_t$.
For the conditional generation, LDM introduces a domain-specific encoder $\bm{\tau}_\theta$ to the projection of condition $\mathbf{y}$ for various modality generations. Thus, the objective of LDM is: 
\vspace{-5pt}
\begin{equation}
    \vspace{-10pt}
    L_{\mathrm{LDM}} = \mathbb{E}_{t,\bm{\varepsilon}(\mathbf{x}),\mathbf{y},\bm{\epsilon}\sim\mathcal{N}(\boldsymbol{0},\boldsymbol{I})}\Bigr[\|\bm{\epsilon} - \bm{\epsilon}_\theta(\bm{z}_t,t,\bm{\tau}_\theta(\mathbf{y}))\|^2\Bigr]
\end{equation}

\section{Methodology}\label{sec:method}
In this paper, we aim to explore an efficient diffusion method to predict coherent video frames guided by language instructions, which requires learning to parse natural language, understand the scene, and ground the language and scene together. However, it is challenging to directly apply conventional video diffusion models for TVP due to the following problems: (1) The limited labeled text-video data and computational resources. (2) Low fidelity of frame generation. (3) Lack of fine-grained instruction for each frame in the task-level videos.
 
%\gu{Specifically, inheriting from I3D~\cite{i3d} technique, we build an inflated 3D U-Net to extend the prior knowledge contained in Stable Diffusion across the frames to generate high-quality and coherent frames by inserting computation-efficient spatial-temporal attention layers (Sec.~\ref{sec:efficientnet}).} As for the language conditioning model, we propose a novel Frame Sequential Text (FSText) Decomposer to adaptively decompose the text instruction into sub-conditions for each frame (Sec.~\ref{sec:fstext}).

\subsection{Overview of Seer}
\label{sec:inflate}
Motivated by the robust generative capabilities of text-to-image (T2I) diffusion models, we leverage the prior knowledge implied in pretrained T2I models by inflating the 2D U-Net~\citep{ldm} and incorporating temporally consistent layers. However, the inflated video diffusion model guided solely by coarse global language instruction tends to generate irrelevant T2I outcomes and fails to maintain temporal coherency between video frames. To address this limitation and provide precise and controllable guidance for our inflated model, we introduce a novel temporal decomposition component for language instruction, this component decomposes global instruction as temporally aligned sub-instruction for delicate task-level guidance, which significantly enhances the fidelity and coherency of predicted video.

Our Seer method comprises two main components: the video diffusion and the language conditioning modules. We propose to enhance these two components to facilitate high-fidelity frame synthesis and the temporal alignment of text instructions, respectively. Specifically, as shown in Figure~\ref{fig:pipeline} (a), we utilize two pathways to implement the conditional diffusion process guided by reference frames and language: \textbf{1)} We incorporate the spatial-temporal module discussed in Section~\ref{sec:efficientnet} into the Inflated 3D U-Net. This integration enables the propagation of contextual information from reference frames to future frames within the spatial-temporal space, allowing for coherent motion prediction based on the reference frames. \textbf{2)} To plan continuous motion with fine-grained language guidance, we introduce a Frame Sequential Text (FSText) Decomposer in Section~\ref{sec:fstext}. This module transforms global language instructions into multi-timestep sub-instructions that are synchronized with video. Subsequently, we inject these frame-wise subinstruction tokens into the intermediate latent space of the video frames at each time step.
With this design, we merely train the spatial-temporal layers and FSText module from scratch while freezing the remaining pretrained modules within our 3D inflated U-Net. These two modules are jointly trained by the diffusion objective, where $f_\theta$ is our FSText decomposer, $\bm{\tau}$ is the frozen CLIP text encoder, and $\mathbf{y}$ is the input text:
\begin{equation}
    L_{\mathrm{diffusion}} = \mathbb{E}_{t,\bm{\varepsilon}(\mathbf{x}),\mathbf{y},\bm{\epsilon}\sim\mathcal{N}(\boldsymbol{0},\boldsymbol{I})}\Bigr[\|\bm{\epsilon} - \bm{\epsilon}_\theta(\mathbf{z}_t,t,f_\theta(\bm{\tau}(\mathbf{y})))\|^2\Bigr],
\end{equation}


%\chuan{I find this paragraph is similar to the last paragraph of page 4. Just keep one to reduce redundancy.}
%Since the T2I latent diffusion models consist of two main components: the image diffusion module and the language conditioning module~\cite{ldm}. We propose to inflate these two parts to perform the synthesis of high-fidelity frames and the temporal decomposition of text instructions, respectively. \gu{Specifically, as shown in Figure~\ref{fig:pipeline} (a), we incorporate the computation-efficient spatial-temporal module into the Inflated 3D U-Net for optimizing temporal consistency in Section~\ref{sec:efficientnet}, and we propose a Frame Sequential Text (FSText) Decomposer for text conditioning module in Section~\ref{sec:fstext}.} Overall, we adopt two pathways to implement the conditional diffusion process of language guidance and reference frames. During training, we stack the latent space of the reference frames with the noisy latent space of the remaining frames along the temporal dimension. During inference, we predict future frames by propagating the prior reference frames and Gaussian noise through the Inflated 3D U-Net. For text conditioning, we employ FSText Decomposer to incorporate the text condition into the diffusion model.





\subsection{Data \& Computation-efficient 3D Network}\label{sec:efficientnet}
To design a computation-efficient visual backbone for our video diffusion model,  we refer to some relevant works on lifting 2D to 3D video modeling~\citep{i3d} and efficient attention computation~\citep{swin, videoswin}. In general, we leverage the latent diffusion model (LDMs) pretrained on T2I tasks to build a text-video model. Our inflated 3D U-Net consists of two principal components as illustrated in Figure~\ref{fig:pipeline} (b): \textbf{1)} The 3D spatial layers, where we draw inspiration from I3D~\citep{i3d} and enhance the 2D convolution kernel from ($3 \times 3$) to a 3D counterpart ($1 \times 3 \times 3$)  with an added video frames axis from the pre-trained 2D modules, consisting of a series of 2D ResNet blocks and Spatial Attention Blocks.
\begin{wrapfigure}[17]{r}{0.42\textwidth}
\centering
\vspace{-10pt}
\includegraphics[width=0.8\linewidth]{fig/wintempatten.pdf}
\vspace{-10pt}
\caption{Variants of temporal attention, only the blue tokens attend to the current token in the red box. Red dashed arrows indicate the direction of attention. And the orange boxes indicate the local window region ($2\times 2$ window in this case).}
\label{fig:tempattn}
\end{wrapfigure}
\textbf{2)} The temporal layers, play a crucial role in our visual backbone for propagating contextual information from the reference frame's image prior across the temporal sequence. We investigated various temporal attention and incorporated them into our 3D U-Net architecture. Our empirical observations indicate that bi-directional temporal attention tends to disregard guidance from reference frames, and both bi-directional and directed temporal attention struggle to capture dependencies among spatial regions, as discussed in Section~\ref{sec:ablate:temp}. To address these limitations while reducing complexity, we employ an efficient approach that builds upon the concept of window attention~\citep{swin} in 3D space: the implementation of local window attention in an autoregressive manner across spatial-temporal dimensions. As illustrated in Figure~\ref{fig:tempattn}, we establish fixed local windows for each spatial region with a window size of $m \times m$ relative to the global frame sequence $n$. Within this framework, we compute self-attention using a causal mask, considering both local spatial and global temporal dimensions within the 3D space. This effectively constrains pixel propagation from the future temporal-spatial sequence.

Finally, We maintain the acquired knowledge from the 2D modules by freezing all pretrained weights and exclusively training the spatiotemporal attention layers during fine-tuning. Overall, through a combination of frozen pre-trained spatial layers and lightweight spatiotemporal layers, our inflated 3D U-Net not only retains crucial knowledge but also enhances fine-tuning efficiency.

%Our empirical findings indicate that bi-directional temporal attention tends to disregard visual guidance from reference frames in time sequence (as discussed in Section~\ref{sec:ablate:temp}), and both bi-directional and directed temporal attention also miss out on capturing dependencies among nearby spatial regions, resulting in suboptimal frame quality. To address these limitations and enhance generation while reducing complexity, we employ an efficient approach that builds upon the concept of window attention~\cite{swin} in 3D space: the implementation of local window attention in an autoregressive manner across spatial-temporal dimensions. As illustrated in Figure~\ref{fig:tempattn}, we establish fixed local windows for each spatial region with a window size of $m \times m$ relative to the global frame sequence $n$. Within this framework, we compute self-attention with a causal mask across both local spatial and global temporal space within the 3D space, effectively constraining the pixel propagation from the future temporal-spatial sequence.
%The utilization of text-to-image (T2I) priors enhances the generative and imaginative capabilities of video generative models~\citep{makeavideo}. In this spirit, we leverage the latent diffusion model (LDMs) pretrained on T2I tasks, inflating it along the temporal axis. Our proposed latent diffusion model consists of two principal components as illustrated in Figure~\ref{fig:pipeline} (b): \textbf{1)} The 3D spatial layers inflated from the pre-trained image diffusion module, consisting of a series of 2D ResNet blocks and Spatial Attention Blocks. To adapt these 2D modules for 3D processing, we draw inspiration from I3D~\cite{i3d} to enhance the 2D convolution kernel from ($3 \times 3$) to a 3D counterpart ($1 \times 3 \times 3$)  with an added video frames axis. \textbf{2)} The temporal layers propagate contextual information from the image prior across the temporal sequence. We maintain the learned text-to-image knowledge from 2D modules by freezing all pre-trained weights during fine-tuning. This strategy not only retains crucial knowledge but also enhances fine-tuning efficiency.
%In comparison to plain spatial-temporal attention, the application of local windows to spatial regions significantly reduces computational overhead while delivering high-fidelity generation results. Notably, we observe that adopting SAWT-Atten has only a marginal $2.13\%$ computation speed lag compared to directed and bidirectional temporal attention, as shown in Appendix Table~\ref{table:speed_temp}.
%To address these limitations and enhance generation while reducing complexity, we employ an efficient approach: implementing local window attention in an autoregressive manner on both temporal and spatial spaces. Specifically, in this paper, extending from window attention~\cite{swin} in 3D space, we adopt a fixed window strategy with window sizes of $8\times8$, $4\times4$, $4\times4$, and $4\times4$ at stages 1, 2, 3, and 4 of U-Net encoder, respectively, within the U-Net network which utilizes multi-scale features. This strategy replaces the shifted window technique in Swin-Attention~\cite{swin}. Within the Scaled Autoregressive Window Temporal Attention (SAWT-Attn) layer (illustrated in Figure~\ref{fig:tempattn}), we extend vanilla temporal attention into spatial space through window attention for each spatial area with the local window of size $m \times m$, alongside the global video frame sequence $n$ across this window. The SAWT-Attn layer conducts self-attention on this extended sequence with a causal mask, integrating both spatial and temporal dimensions and restricting the model from learning future temporal-spatial tokens.
%As it operates in both spatial and temporal spaces, frame generation attends not only to prior frames but also to adjacent spatial regions, resulting in high-fidelity generation performance.
%In this context, for a video clip with $n$ frames, each is projected into $n\times H/K \times W/K \times 4$ latent vectors. Here $n$ signifies the frame count, $K$ indicates downsample ratio of VAE encoder, $(H/K, W/K)$ represents spatial dimensions, and $4$ corresponds to the number of latent channels.

\subsection{Frame Sequential Text Decomposer}\label{sec:fstext}
For the language conditioning module, since our 3D inflated U-Net is built upon a pretrained text-to-image model, we noticed that using a text-to-image prior alongside a global instruction tends to provide strong semantic guidance, which can override the scene in reference frames, deviating from the intended guidance for prediction based on the existing scenes.
To address the above limitation and better capture long-term dependencies from both text and reference frames, we introduce the Frame Sequential Text (FSText) Decomposer. This novel approach decomposes the global instruction into fine-grained sub-instructions, aligning with each frame. We further explore the interpretability of sub-instruction embeddings in Section~\ref{sec:results:subins}.
%the existing methods~\citep{magicvideo,makeavideo,tuneavideo} simply encode a single text embedding for the whole video with a CLIP text encoder~\cite{radford2021clip}.However, since text instructions often pertain to the overall task, understanding progress at each time step becomes challenging with a global instruction embedding
\begin{figure*}
\centering
\vspace{-30pt}
\includegraphics[width=0.9\linewidth]{fig/seq_text_transformer.pdf}
\vspace{-10pt}
\caption{Frame Sequential Text Decomposer is shown in (a). We start by initializing the weight of the network to project identity vectors from CLIP text tokens. We then optimize the generated text tokens via the diffusion process (b), where frame-individual cross-attention is denoted by ``fic-attn."}
\label{fig:fseq}
\vspace{-10pt}
\end{figure*}
To derive a sequence of temporally aligned sub-instruction embeddings from the global instruction generated by the CLIP text encoder~\cite{radford2021clip}, we employ a transformer-based temporal network designed to fulfill three essential properties for meaningful sub-instructions: \textbf{1)} Contextual aggregation, which ensures that the inner tokens of each sub-instruction aggregate contextual information within the sentence. \textbf{2)} Semantic inheritance, the semantic information of these sub-instructions is inherited directly from the global instruction \textbf{3)} Temporal consistency ensures alignment between the sub-instructions and the time sequence, thereby facilitating the generation of temporally consistent video. Based on these properties,  our network consists of  three key components: \textbf{a)} To achieve the property of contextual aggregation, we employ Text-Sequential Attention, akin to BERT, a bidirectional self-attention layer~\citep{bert} to capture global dependencies among different positions within text sentences. \textbf{b)} To ensure semantic inheritance, we use Cross-Attention, responsible for projecting the global instruction's textual sequence onto the inner tokens of each sub-instruction, this component ensures that all sub-instructions contain essential global instruction signals for guiding video frame generation. \textbf{c)} To maintain temporal consistency, we adopt temporal Attention, a directed attention layer to capture temporal dependencies along the frame axis, which enhances temporal consistency among the generated sub-instructions throughout the video.
\begin{wrapfigure}[10]{r}{0.4\textwidth}
\centering
\vspace{-10pt}
\includegraphics[width=0.9\linewidth]{fig/fstext_module.pdf}
\vspace{-10pt}
\caption{The FSText attention of sub-instruction tokens.}
\label{fig:fstextpipline}
\end{wrapfigure}
Specifically, as shown in Figure~\ref{fig:fstextpipline}, we start with a global CLIP text embedding, denoted as $(l, C)$, where $l$ signifies the text sentence length and $C$ is the channel size, we initialize learnable tokens with shape $(n, l, C)$ where $n$ denotes the number of frames. The tokens are fed into the text sequential attention layer to perform self-attention along the $l$ axis. Subsequently, the cross-attention layer employs these learnable tokens as queries and the global text embedding as keys and values, resulting in a one-to-multiple projection from the global text into $n$ time steps. This yields $(n, l, C)$ tokens for $n$ frame containing task instruction information. Finally, the temporal attention layer conducts directed attention along the $n$ axis for each token in the textual sequence, transforming the macro-instruction progress into frame-specific guidance.

After getting $n$ sub-instruction embeddings corresponding to each frame, the next step is to inject this guidance into the diffusion process, which is commonly completed by a cross-attention layer. As shown in Figure~\ref{fig:fseq} (b), different from the existing works~\citep{magicvideo,tuneavideo} that calculate the cross-attention between the global instruction embedding and $n$ frames. In our cross-attention layer, where cross-attention is calculated separately between visual latent vectors and sub-instruction embeddings for each frame, and the results from all frames are then concatenated, an attention mechanism we refer to as frame-individual cross-attention (\textit{fic-attn}).

\paragraph{Initialization~~} We find initialization is critical to FSText decomposer. Especially, the random initialization fails to approximate the distribution of text embeddings in the pretrained T2I model and results in poor performance. To guarantee the sub-instruction embeddings become a close approximation of the CLIP text embedding, we employ an initialization strategy by enforcing the FSText decomposer to be an identity function (Note that this initialization step is completed before the diffusion process. We ablate this design in Section~\ref{sec:ablate:fstext}). It can be achieved by this objective:
\begin{equation}
    L_{\mathrm{identity}} = \|f_\theta(\bm{\tau}(\mathbf{y})) - \bm{\tau}(\mathbf{y})\|^2
\end{equation}

%\subsection{Inflated 3D U-Net with Autoregressive Spatial-Temporal Attention}\label{sec:tempoal}
%We inflate the Text-to-Image (T2I) pre-trained 2D U-Net to our Inflated 3D U-Net as illustrated in Figure~\ref{fig:pipeline} (b). A standard 2D U-Net block of LDMs consists of a series of 2D ResNet blocks and Spatial Attention Blocks including spatial self-attention and cross-attention modules. Similar to~\cite{vdm}, we replace the $3 \times 3$ 2D convolution kernel with a $1 \times 3 \times 3$ 3D convolution kernel with an additional axis of video frames. Additionally, to further boost the performance of capturing the inter-frame dependency, we incorporate temporal attention after every spatial cross-attention layer. In Figure~\ref{fig:tempattn}, we explore various types of temporal attention, including: (1) bi-directional temporal attention~\cite{vdm,makeavideo,imagenvideo}, which employs a full self-attention across all tokens along the temporal dimension; (2) directed temporal attention~\cite{magicvideo}, which uses a masked attention mechanism that follows the direction of the video sequence along the temporal dimension; and (3) autoregressive spatial-temporal attention: a novel technique proposed by us, which uses causal attention to autoregressively generates the frames on both spatial and temporal dimensions by flattening the tokens into a long sequence.

%We empirically observe that the two existing temporal attention layers cannot achieve promising performance on the TVP task. Bi-directional temporal attention tends to neglect the visual content guidance of the reference frames during the generation process (see Section~\ref{sec:ablate:temp}). 
%And the directed temporal attention fails to capture the dependency of nearby spatial regions and thus generates low-quality frames, while it adheres to the temporal sequence constraint.

%To handle the limitations of bi-directional and directed temporal attention, we introduce the Autoregressive Spatial-Temporal Attention (AST-Attn) mechanism shown in Figure~\ref{fig:tempattn}.
%Given $n$ frames video, a video clip is projected into $n\times s$ latent vectors (where $s$ is the length of a latent vector in each frame) by the pre-trained VAE encoder. We flatten the latent vectors of both temporal and spatial dimensions ($n\times s$) into one dimension. 
%Then, AST-Attn performs self-attention on this long sequence with a causal mask that prevents the model from learning from future temporal-spatial tokens. Because it performs in both spatial and temporal spaces, the frame generation will attend to not only the previous frames but also the nearby spatial regions, which results in high-fidelity generation performance.
%While the calculation of Autoregressive Spatial-Temporal Attention (AST-Attn) involves both temporal and spatial dimensions, its computational complexity remains manageable due to the design of the Inflated 3D U-Net, which maintains the complexity of spatial compression rates and channel depths within a controllable range. Specifically, in the AST-Attn layers of Inflated 3D U-Net, higher-resolution features have more spatial tokens but are computed in lower embedding dimensions, while lower-resolution features have fewer spatial tokens but are computed in higher embedding dimensions. In practice, we observe that adopting AST-Attn has only a $0.4\%$ computation speed lag compared to directed and bidirectional temporal attention.

%By incorporating Autoregressive Spatial-Temporal Attention in the Inflated 3D U-Net, we can generate high-fidelity and coherent video frames with minimal fine-tuning. Specifically, we merely fine-tune the proposed autoregressive spatial-temporal attention layers and freeze the rest of the pre-trained layers in our Inflated 3D U-Net. 
\iffalse
\begin{figure}
\centering
\includegraphics[width=1.0\linewidth]{fig/inflate_unet.pdf}
\caption{The overview of inflated 3D U-Net, we inflate the pre-trained T2I latent diffusion model (LDM) by expanding the 2D Conv kernel to 3D kernels and connecting the cross-attention layer with the trainable causal temporal attention layer.}
\label{fig:inflate3d}
\end{figure}
\fi




\section{Experiments}
In this section, we evaluate the proposed method Seer on the text-conditioned video prediction task. 
We compare against various recent methods and conduct ablation studies on the techniques presented in Section~\ref{sec:method}.

\begin{figure}
\centering
\includegraphics[width=1.0\linewidth]{fig/prediction.pdf}
\caption{Visualization results of text-conditioned video prediction (conditioned on first 2 frames) on Something-Something V2. TAV refers to Tune-A-Video.}
\vspace{-10pt}
\label{fig:tvp:sthv2}
\end{figure}
\begin{table*}
\centering\small
\tablestyle{2pt}{1.1}
\setlength{\tabcolsep}{6pt}
{\caption{\textbf{ Text-conditioned video prediction (TVP) results on Something-Something V2 (SSv2) and Bridgedata (Bridge).} We report the FVD and KVD metrics of each method. We can see that our method Seer achieves the lowest FVD and KVD values in both SSv2 and Bridgedata, illustrating our superior performance on the challenging TVP task.
}
\label{table:tvp}}
\vspace*{-3mm}
\begin{tabular}{cccc|cc|cc}
\specialrule{.1em}{.05em}{.05em} 
 \multirow{2}{*}{Method} & \multirow{2}{*}{Pre.-weight} & \multirow{2}{*}{Text} & \multirow{2}{*}{Resolution} & \multicolumn{2}{c|}{\textbf{SSv2} (ref. = 2)} & \multicolumn{2}{c}{\textbf{Bridge} (ref. = 1)}\\
   &  &  &  & FVD$\downarrow$ & KVD$\downarrow$  & FVD$\downarrow$ & KVD$\downarrow$\\
 \hline
TATS~\cite{tats} & video & No  & $128\times 128$ & 428.1 & 2177 & 1253 & 6213\\
 MCVD~\cite{mcvd} & No & No  & $256\times 256$ & 1407 & 3.80 & 1427 & 2.50\\
Tune-A-Video~\cite{tuneavideo} & txt-img & Yes & $256\times 256$ & 291.4 & 0.91 & 515.7 & 2.01\\
Seer (Ours) & txt-img & Yes & $256\times 256$ & $\bf 200.1$ & $\bf 0.30$ & $\bf 507.3$ & $\bf 1.37$\\
\specialrule{.1em}{.05em}{.05em} 
\end{tabular}
\vspace{-10pt}
\end{table*}
\begin{figure}
\centering
\includegraphics[width=1.0\linewidth]{fig/tvp_bridgedata.pdf}
\caption{Visualization of Text-conditioned Video Prediction on Bridgedata. ``Ref." refers to reference frames and TAV refers to Tune-A-Video.}
\vspace{-10pt}
\label{fig:tvp:bridge}
\end{figure}


\subsection{Datasets}
We conduct experiments on two text-video datasets: Something Something-V2 (SSv2)~\cite{sthv2}, which contains videos of human behaviors involving everyday objects accompanied by language instructions, and BridgeData~\cite{bridge} that is rendered by a photo-realistic kitchen simulator with text prompts.
Because the SSv2 validation set is too large (with over 200k samples), we follow \cite{ucf101} to evaluate the first 2048 samples during evaluation to save testing time.
For BridgeData, we split the dataset into an $80\%$ training set and $20\%$ validation set, and evaluate all validation samples. To reduce complexity, we downsample each video clip to 12 frames for SSv2 and 16 frames for BridgeData during both training and evaluation. Moreover, to provide a fair comparison with recent unreleased video generative model baselines~\cite{vdm,lvdm,magicvideo,makeavideo}, we also included results on the UCF-101 dataset~\cite{ucf101} in Appendix~\ref{appendix:sec:ucf101}.

\subsection{Implementation Details}
We use the pre-trained weights of the Tex-to-Image Latent Diffusion Model (LDM), Stable Diffusion-v1.5~\cite{ldm}
, to initialize the VAE, ResNet Blocks and Spatial-Cross Attention layers of the 3D U-Net. We freeze both the pre-trained VAE and the pre-trained modules of the 3D U-Net, and only fine-tune the Autoregressive Spatial-Temporal Attention Layers. To fine-tune the FSText Decomposer, we initialized it as the identity function of the CLIP text embedding, as described in Section~\ref{sec:fstext}. We train the models on Something Something-V2 and BridgeData with an image resolution of $256 \times 256$ for 20k training steps. In the evaluation stage, we speed up the sampling process with the fast sampler DDIM~\cite{ddim} and denoise the prediction with conditional guidance of 7.5 for 30 timesteps. Please refer to Appendix~\ref{appendix:sec:impl} for more details on hyperparameters.

\subsection{Evaluation Settings}
\noindent\textbf{Baselines.~~}\label{sec:baseline} We compare Seer with three publicly released baselines for video generation (1) conditional video diffusion methods: \textit{Tune-A-Video}~\cite{tuneavideo} and \textit{Masked Conditional Video Diffusion} (MCVD)\cite{mcvd}; and (2) autoregressive-based transformer method: \textit{Time-Agnostic VQGAN and Time-Sensitive Transformer} (TATS)\cite{tats}. Since Tune-A-Video is also the Text-to-Image inflated video diffusion model, and both MCVD and TATS are long video generative models for video prediction, they conform to our benchmark that requires predicting task-level movements. We further fine-tune Tune-A-Video, TATS, and train MCVD on the training sets of SSv2 and Bridgedata for 300k training steps.

\noindent\textbf{Machine Evaluation.~~}\label{sec:exp:tvp} We evaluate the text-conditioned video prediction of several baseline methods on Something Something-V2 (SSv2) (with 2 reference frames) and Bridgedata (with 1 reference frame). Additionally, we conduct several ablation studies of our proposed modules on SSv2. We report the Fréchet Video Distance (FVD) and Kernel Video Distance (KVD) metrics in our evaluation. FVD and KVD are calculated with the Kinetics-400 pre-trained I3D model~\cite{i3d}. We evaluate FVD and KVD on 2,048 SSv2 samples and 5,558 Bridgedata samples in the validation sets. For FVD metrics, we follow the evaluation code of VideoGPT~\cite{videogpt}. We further evaluate the class-conditioned video prediction of our method on the UCF-101 dataset~\cite{ucf101} and present the comparison results in Appendix~\ref{appendix:sec:ucf101}.

\begin{figure}
\centering
\includegraphics[width=0.9\linewidth]{fig/human_eval.pdf}
\caption{Human evaluation results. Preference percentage for text-conditioned video manipulation on SSv2.}
\vspace{-8pt}
\label{fig:humaneval}
\end{figure}

\noindent\textbf{Human Evaluation.~~}\label{sec:exp:humaneval} Besides evaluating the models on the standard validation sets, we also manually modify the text prompts to provide richer testing results, called text-conditioned video manipulation. Because of the absence of ground-truth frames, we conducted a human evaluation of text-conditioned video manipulation using 99 video clips from the validation set of SSv2. We manually modified partial text prompts and generated 99 predicted videos for each method. Then, we invited 54 anonymous evaluators to rate the quality of the prediction, with a higher priority placed on the semantic contents in the videos and an intermediate priority placed on the fidelity of the video frames. We report the percentage of overall preference choices among the 99 video clips. More details are introduced in Appendix~\ref{appendix:sec:humaneval}.

\begin{figure}
\centering
\includegraphics[width=1.0\linewidth]{fig/manipulation.pdf}
\caption{Visualization of Text-conditioned Video Prediction with the original (a) and manually modified (b) text prompts on Something-Something V2. ``Ref." is reference frames and TAV refers to Tune-A-Video.}
\vspace{-10pt}
\label{fig:tvm:sthv2}
\end{figure}

\subsection{Main Results}\label{sec:main-results}
\noindent\textbf{Quantitative Results.~~} Table~\ref{table:tvp} presents the text-condtioned video prediction results on Something Something-V2 (SSv2) and BridgeData. Seer achieves the best performance among all baselines, with the lowest Fréchet Video Distance (FVD) of 200.1 and Kinematic Distance (KVD) of 0.3 in SSv2, and the lowest FVD of 507 and KVD of 1.37 in BridgeData. Notably, Seer and Tune-A-Video both incorporate text conditioning, and the results highlight Seer's superior text-video alignment performance, especially on SSv2.

The results of the human evaluation in the text-conditioned video manipulation experiment are shown in Figure~\ref{fig:humaneval}. Our proposed Seer outperforms the other baselines in terms of both semantic content and fidelity of video, with a preference rate of at least $72.2\%$ in comparison. This indicates that Seer is effective in generating high-quality video clips that are faithful to the input text prompts.

\begin{table}
\centering\small
\tablestyle{2pt}{1.0}
\setlength{\tabcolsep}{4pt}
\floatsetup{floatrowsep=qquad, captionskip=1.5pt}
\begin{floatrow}[2]
\ttabbox
{\begin{tabular}{c|cc}
init. weight & FVD$\downarrow$ & KVD$\downarrow$\\
 \hline
 \textit{random} & 367.9 & 0.75\\
\textit{identity}(Ours) & 200.1 & 0.30\\
\end{tabular}
{\caption[ftext]{\textbf{Init. weight ablation results of FSText}}
\label{table:ablation:weight}
}
}
\hfill
\ttabbox
{\begin{tabular}{c|cc}
temp. attn. & FVD$\downarrow$ & KVD$\downarrow$\\
 \hline
 \textit{bi-direct.} & 258.2 & 0.56\\
 \textit{directed.} & 222.3 & 0.40\\
 \textit{autoreg.}(Ours) &200.1 & 0.30\\
\end{tabular}}
{\caption[temp]{\textbf{Ablation study of temporal attention}}
\label{table:ablation:tempattn}
}
\end{floatrow}

\begin{floatrow}[1]
\ttabbox[1.0\linewidth]
{\begin{tabular}{cc|cc}
\specialrule{.1em}{.05em}{.05em} 
fine-tune& FSText. & FVD$\downarrow$ & KVD$\downarrow$\\
 \hline
 \textit{temp-attn.} & & 328.2 & 1.26\\
 \textit{cross}+\textit{temp-attn.} & & 249.9 & 0.73\\
 \textit{temp-attn.}(Ours) & \checkmark &200.1 & 0.30\\
 \textit{cross}+\textit{temp-attn.} & \checkmark & 1807 & 5.12\\
\specialrule{.1em}{.05em}{.05em} 
\end{tabular}}
{\caption[fine]{\textbf{Ablation study of Fine-tune settings}}
\label{table:ablation:finetune}
}
\end{floatrow}
\end{table}


\iffalse
\begin{table}
\centering\small
\tablestyle{2pt}{1.0}
\setlength{\tabcolsep}{5pt}
\caption{\textbf{ Fine-tune settings and component design }.}
\label{table:ablation:finetune}
\vspace*{-3mm}
\begin{tabular}{cc|cc}
\specialrule{.1em}{.05em}{.05em} 
fine-tune& FSText. & FVD$\downarrow$ & KVD$\downarrow$\\
 \hline
 \textit{temp-attn.} & & 328.2 & 1.26\\
 \textit{cross}+\textit{temp-attn.} & & 249.9 & 0.73\\
 \textit{temp-attn.}(Ours) & \checkmark &200.1 & 0.30\\
 \textit{cross}+\textit{temp-attn.} & \checkmark & 1807 & 5.12\\
\specialrule{.1em}{.05em}{.05em} 
\end{tabular}
\end{table}
\fi


\noindent\textbf{Qualitative Results.~~} Figure~\ref{fig:tvp:sthv2} compares the text-conditioned video prediction (TVP) performance of Seer and Tune-A-Video on Something Something-V2 (SSv2). While Tune-A-Video can align simple text prompts with video in some cases, it struggles to consistently track the spatial appearance of reference frames in later predictions. For instance, in the ``taking glass from desk" samples, Tune-A-Video fails to generate a coherent motion trajectory and corrupts the pixels in the background, generating a new video instead of predicting from the reference frames. In contrast, Seer generates relatively coherent motion and better aligns the predictions with text prompts. Additionally, Seer can generate hidden objects by leveraging the imaging capability of the pretrained text-to-image diffusion model, which flexibly tackles occlusion issues in video prediction. In the ``tearing a piece of paper into two pieces" sample, Seer accurately predicts that a man is hidden behind the paper and generates coherent frames including the man's face. Figure~\ref{fig:tvp:bridge} compares Seer and Tune-A-Video's TVP performance on Bridgedata, illustrating that Seer achieves better text-video alignment, including the alignment of instructed behavior and target objects in future frames, and predicts a more coherent video with higher fidelity.

Figure~\ref{fig:tvm:sthv2} shows a comparison of Seer and Tune-A-Video for text-conditioned video prediction and manipulation on Something Something-V2 (SSv2). Tune-A-Video tends to mainly focus on Text-to-Video alignment, usually ignoring the directional temporal movement of the video. For example, in the case of ``turning the camera left while filming wall mounted fan", Tune-A-Video generates a semantic movement when the word "left" is replaced with "right" in the sentence, but fails to maintain temporal consistency in the video background during the transition from the middle to the last frame. In contrast, Seer performs better in handling the temporal dynamics of the video and achieving more precise text-video alignment in video manipulation.


\subsection{Ablation study}
In this section, we evaluate the effect of different components of our method in the TVP task on the SSv2 dataset.

\noindent\textbf{FSText Decomposer.~~}\label{sec:ablate:fstext}
Table~\ref{table:ablation:weight} compares different weight initialization strategies of FSText decomposer. The results demonstrate that using identity initialization described in Section~\ref{sec:fstext} yields higher prediction quality compared with random initialization. This finding demonstrates that identity initialization is necessary for the temporal-text projection of FSText decomposer. We also provide additional ablation results of FSText decomposer in Appendix~\ref{appendix:sec:fstext}.

\noindent\textbf{Temporal Attention.~~}\label{sec:ablate:temp}
As shown in Table~\ref{table:ablation:tempattn} studies the effectiveness of different types of temporal attention. Our autoregressive spatial-temporal attention (autoreg.) outperforms both bi-directional temporal attention (bi-direct.) and directed temporal attention (directed.), resulting in the lowest FVD and KVD scores. We also find that directed temporal attention further improves video prediction performance compared to bi-directional temporal attention because it utilizes the inductive bias of sequential generation.

\noindent\textbf{Fine-tune Setting.~~}
We compare various fine-tuning settings of 3D Inflated U-Net, and the results are presented in Table~\ref{table:ablation:finetune}. Our default setting involves fine-tuning both FSText decomposer (FSText.) and autoregressive spatial-temporal attention (AST-Attn.) layers (\textit{temp-attn.}), while freezing the remaining modules in 3D U-Net. For the ``\textit{temp-attn.}" setting, we only finetune the AST-Attn. layers and freeze all other components. In the "\textit{cross+temp-attn.}" setting, we jointly update the parameters of Spatial-Cross Attention layers and AST-Attn. layers. We further fine-tune the "\textit{cross+temp-attn.}" together with FSText decomposer. We observe that our default setting achieves the highest quality of video prediction among all these settings, indicating that fine-tuning FSText decomposer is critical. Based on our default setting, further fine-tuning "\textit{cross+temp-attn.}" causes the performance of Seer to drop a lot, even the worst among all fine-tuning settings. These results suggest that the optimization of the FSText decomposer is strongly guided by the frozen conditional diffusion prior, and additional fine-tuning of cross-attention leads to uncontrollable guidance to the FSText decomposer.

\iffalse
\begin{table}
\centering\small
\tablestyle{2pt}{1.0}
\setlength{\tabcolsep}{5pt}
\caption{\textbf{ Type of temporal attention }}
\label{table:ablation:tempattn}
\vspace*{-3mm}
\begin{tabular}{c|cc}
temp. attn. & FVD$\downarrow$ & KVD$\downarrow$\\
 \hline
 \textit{bi-direct.} & 258.2 & 0.56\\
 \textit{directed.} & 222.3 & 0.40\\
 \textit{autoreg.}(Ours) &200.1 & 0.30\\
\specialrule{.1em}{.05em}{.05em} 
\end{tabular}
\end{table}
\fi
\section{Conclusion}\label{conclusion}
% Repeat what has been done
In this paper, we have introduced a hybrid coupled fluid-particle implementation with geometrically resolved particles. We use \gls{gpu}s for the fluid dynamics, whereas the particle simulation runs on \gls{cpu}s.
% Restate problem
We have addressed the issue that in multiphysics simulations, different methodologies can have distinctly different computational properties, implying that the best-suited hardware architecture may differ between the simulation components.
% Summarize arguments and findings
The paper has examined the performance of this approach for two cases of a fluidized bed simulation that differ significantly in terms of the number of particles per volume.
The particle methodology scales poorly, reaching a parallel efficiency of about 45\% when using only six \gls{cpu} cores.
This scaling result supports our initial assumption that the particle simulation part of such a coupled simulation would not benefit from a \gls{gpu} parallelization.
The penalty introduced by the hybrid implementation (i.e., \gls{cpu}-\gls{gpu} communication) is negligible because we are transferring only a small amount of data per particle but no fluid cells.
The performance of the fluid simulation is close to utilizing the whole memory bandwidth of the A100, implying that the \gls{gpu} is the best choice for the fluid simulation.
In both cases, the \gls{gpu} routines take most of the run time.
In a weak scaling benchmark, the hybrid fluid-particle implementation reaches a parallel efficiency of 71\% in the dilute case and 53\% in the dense case when using 1024 \gls{cpu}-\gls{gpu} pairs.
The \gls{pd} methodology requires 32 \gls{cpu}-\gls{cpu} communications per time step which is the driving force for the decrease of the overall parallel efficiency. Our results are limited insofar as different numbers of particle sub-cycles, fluid cells per diameter, etc., will result in different performance results.
% Key takeaways from the paper
We have formulated four criteria that a hybrid implementation must meet to be suitable for the responsible use of heterogeneous supercomputers.
The performance results have shown that our hybrid implementation fulfills all criteria making it suitable for large-scale simulations on heterogeneous supercomputers.
% Future work / Outlook
In the future, we plan to investigate the particle communications in more detail regarding the bottleneck and optimization possibilities. We have shown the acceleration potential of hybrid implementations. Therefore, we plan to run coupled fluid-particle simulations of unprecedented sizes to better understand complex physical phenomena occurring in sea and river beds.


{\small
\bibliographystyle{ieee_fullname}
\bibliography{egbib}
}

\clearpage
\appendix
\section{Additional Experimental Results}

\subsection{Implementation Details of Baselines}
We compare three baselines in our paper. For Tune-A-Video, to ensure a fair comparison, we use the pre-trained weight of Stable Diffusion-v1.5\footnote{https://github.com/CompVis/stable-diffusion} (same as our model) to initialize the UNet and we fine-tune the model with an image resolution of $256 \times 256$ on the training sets of Something Something-V2 (SSv2) and Bridgedata for 200k training steps. For MCVD, we train the model with an image resolution of $256 \times 256$ for 300k training steps. For TATS, we fine-tune the pre-trained UCF-101 model with an image resolution of $128 \times 128$ on the training sets of SSv2 and Bridgedata for 300k training steps.



\subsection{Evaluation Details and Results of UCF-101}~\label{appendix:sec:ucf101}
Most prior text-conditioned video generation methods~\cite{hong2023cogvideo,vdm,makeavideo,magicvideo} evaluate their performance on the UCF-101~\cite{ucf101} benchmark. However, since our proposed method, Seer, is designed for text-conditioned video prediction (TVP) on task-level video datasets, the UCF-101 benchmark, which evaluates class-conditioned video prediction on random short-horizon video clips, is not an ideal evaluation benchmark for TVP. Nonetheless, in order to fairly compare these baselines, we still evaluate the class-conditioned video prediction performance of Seer on UCF-101. 

\paragraph{Settings}Specifically, we fine-tune our model with a video resolution of $16\times256\times256$ on UCF-101. Following the evaluation protocols of ~\cite{hong2023cogvideo}, Seer predicts the videos conditioned on 5 reference frames during fine-tuning and inference stage. We report FVD and Inception score (IS) metrics on the UCF-101 dataset~\cite{ucf101}. The IS is calculated by a C3D model\cite{c3d} that is pre-trained on the Sports-1M dataset~\cite{sports} and fine-tuned on UCF101. We follow the evaluation code of TGAN-v2~\cite{tganv2} to calculate IS metric. Following ~\cite{hong2023cogvideo,vdm,makeavideo}, we evaluate the FVD metric with 2,048 samples and IS metric with 100k samples in the validation set of UCF-101.

\paragraph{Results} Table~\ref{table:tvp:ucf} presents the class-conditioned video prediction results on UCF-101, demonstrating that Seer outperforms CogVideo~\cite{hong2023cogvideo} and MagicVideo~\cite{magicvideo}, but falls short of Make-A-Video~\cite{makeavideo}. Make-A-Video employs unlabelled video pre-training on temporal layers and achieves the best performance among all other methods. While Make-A-Video shows superior performance on FVD and IS, Seer has the potential to further improve its generation performance by addressing the following two limitations. First, Seer has not been pre-trained on video data. Second, Seer obtains latent vectors via a pre-trained 2D VAE, which has not been fine-tuned on UCF-101 and limits the video generation quality of Seer (with 259.4 FVD and 68.16 IS reconstruction quality). However, as we focus on the text-conditioned video prediction task, addressing the above limitations on UCF-101 is out of the scope of this paper.

\begin{table*}
\begin{threeparttable}
\centering\small
\tablestyle{2pt}{1.1}
\setlength{\tabcolsep}{5pt}
\caption{\textbf{ Class-conditioned video prediction performance on UCF-101} we evaluate the Seer on the UCF-101 with 16-frames-long videos. Ex.data indicates that the model has been pre-trained or fine-tuned on extra datasets.
}
\label{table:tvp:ucf}
\begin{tabular}{cccc|cc}
\specialrule{.1em}{.05em}{.05em} 
 Method & Ex.data & Cond. & Resolution & FVD$\downarrow$ & IS$\uparrow$\\
 \hline
MoCoGAN-HD~\cite{mocogan} & No & Class.  & $256\times 256$ & 700\tiny{$\pm$24} & 33.95\tiny{$\pm$0.25}\\
VideoGPT~\cite{videogpt} & No & No  & $128\times 128$ & - & 24.69\tiny{$\pm$0.30}\\
RaMViD~\cite{ramvid} & No & No  & $128\times 128$ & - & 21.71\tiny{$\pm$0.21}\\
StyleGAN-V~\cite{styleganv} & No & No  & $128\times 128$ & - & 23.94\tiny{$\pm$0.73}\\
DIGAN~\cite{digan} & No & No  & & 577\tiny{$\pm$22} & 32.70\tiny{$\pm$0.35}\\
TGANv2~\cite{tganv2} & No & Class.  & $128\times 128$ & 1431.0 & 26.60\tiny{$\pm$0.47}\\
VDM~\cite{vdm} & No & No  & $64\times 64$ & - & 57.80\tiny{$\pm$1.3}\\
TATS-base~\cite{tats} & No & Class.  & $128\times 128$ & 278\tiny{$\pm$11} & 79.28\tiny{$\pm$0.38}\\
MCVD~\cite{mcvd} & No & No  & $64\times 64$ & 1143.0 & -\\
LVDM~\cite{lvdm} & No & No  & $256\times 256$ & 372\tiny{$\pm$11} & 27\tiny{$\pm$1}\\
MAGVIT-B~\cite{magvit} & No & Class.  & $128\times 128$ & 159\tiny{$\pm$2} & 83.55\tiny{$\pm$0.14}\\
 \hline
CogVideo~\cite{hong2023cogvideo} & txt-img \& txt-video & Class.  & $160\times 160$ & 626 & 50.46\\
Make-A-Video~\cite{makeavideo} & txt-img \& video & Class.  & $256\times 256$ & 81.25 & 82.55\\
MagicVideo~\cite{magicvideo} & txt-img \& txt-video & Class.  & & 699 & -\\
\textbf{Seer(Ours)} & txt-img & Class. & $256\times 256$ & 287.8 & 57.74\\
\specialrule{.1em}{.05em}{.05em} 
\textbf{pre-trained VAE}\tnote{*} & - & - & $256\times 256$ & 259.4 & 68.16\\
\specialrule{.1em}{.05em}{.05em} 
\end{tabular}
\begin{tablenotes}\footnotesize
    \item [*] we evaluate the reconstruction quality of pre-trained 2D VAE in this table, the pre-trained 2D VAE is initialized with the pre-trained weight from Stable Diffusion-v1.5 without extra fine-tuning.
\end{tablenotes}
\end{threeparttable}
\end{table*}



\subsection{Evaluation Results of Sampling Steps}
To further investigate the generation effects of sampling steps during evaluation, we conduct a comparison between Seer and Tune-A-Video. We apply a series of DDIM sampling steps (10, 20, 30, 40, 50 DDIM steps), as shown in Figure~\ref{fig:ddimstep}. Seer consistently outperformed Tune-A-Video in terms of both FVD and KVD, with improvements observed from 20 DDIM steps to 50 DDIM steps. Particularly noteworthy is Seer's advantage in video quality (280.7 FVD and 0.73 KVD) compared to Tune-A-Video (419.3 FVD and 1.5 KVD) when using only 10 DDIM steps, demonstrating Seer's ability to sample high-fidelity videos efficiently with minimal denoising steps.
\begin{figure}
\centering
\includegraphics[width=1.0\linewidth]{fig_appendix/ddim_curve.pdf}
\caption{Evaluation results of Seer and Tune-A-Video with DDIM sampling steps ranging from 10 to 50 on the Something-Something V2 dataset.}
\vspace{-8pt}
\label{fig:ddimstep}
\end{figure}



\subsection{Additional Ablation Results}~\label{appendix:sec:fstext}
\paragraph{FSText layer depth}In this section, we additionally investigate the impact of FSText Decomposer's layer depth in Table~\ref{table:ablation:layer}. Our default setting (8-layer FSText Decomposer) outperforms shallower models (2-layer and 4-layer) in terms of FVD. Though the 4-layer model shows a marginal advantage over the 8-layer model in terms of KVD, our experiments indicate that the 8-layer FSText Decomposer shows a remarkable advantage on FVD metrics and exhibits robustness in text-video alignment. Therefore, we adopt the 8-layer FSText Decomposer as the default setting for Seer.

\begin{table}
\centering\small
\tablestyle{2pt}{1.0}
\setlength{\tabcolsep}{5pt}
\caption{\textbf{Layer depth in FSText Decomposer}.}
\label{table:ablation:layer}
\vspace*{-3mm}
\begin{tabular}{c|cc}
num. layers.& FVD$\downarrow$ & KVD$\downarrow$\\
 \hline
 2 & 238.6 & 0.51\\
 4 & 229.7 & 0.23\\
8(Ours) &200.1 & 0.30\\
\end{tabular}
\end{table}

\paragraph{Qualitative results of fine-tuning ablation} We conduct a qualitative analysis of various fine-tune settings. We provide additional visualizations of Fine-tune Setting ablation in Section 5.5 of the main paper. Figure~\ref{fig:ablate:finetune} shows the results of different settings. Among these settings, our default setting \textit{``temp+FSText"} stands out as it preserves a higher-level temporal consistency in video prediction starting from reference frames and also delivers superior text-based video motion compared to the other fine-tune settings. 
\begin{figure}
\centering
\includegraphics[width=1.0\linewidth]{fig_appendix/ablation_visualization.pdf}
\caption{Additional qualitative results of fine-tuning ablation. \textit{“temp+FSText.”} is our default setting.}
\vspace{-8pt}
\label{fig:ablate:finetune}
\end{figure}


\begin{table}
\centering\small
\tablestyle{2pt}{1.1}
\caption{Hyperparameters and details of Fine Tuning/Inference}
\label{table:hyperparam:finetune}
\begin{tabular}{c|cc}
\hline
param. & value\\
\hline
optim. & AdamW\\
Adam-$\beta_1$ &  0.9\\
Adam-$\beta_2$ &  0.99\\
Adam-$\epsilon$ &  $1e^{-8}$\\
weight decay &  $1e^{-2}$\\
lr &  $1.28e^{-5}$\\
end lr & 0.0\\
lr sche. & cosine\\
noise sche. & cosine\\
train batch size& 1/GPU\\
grad. acc.& 2\\
warmup steps& 10k\\
resolution& $256 \times 256$\\
train. steps & 200k\\
train. hardware & 4 RTX 3090\\
val. batch size& 2/GPU\\
sampler& DDIM\\
sampling steps & 30\\
guidance scale & 7.5\\
\hline
\end{tabular}
\end{table}

\begin{table}
\centering\small
\tablestyle{2pt}{1.1}
\caption{Hyperparameters of 3D U-Net}
\label{table:3dunet}
\begin{tabular}{c|cc}
\hline
hyperparam. & value\\
\hline
input/output channels &  4\\
Base channels & 320\\
Channel multipliers&  1,2,4,4\\
3D Downsample blocks &  4\\
3D Upsample blocks &  4\\
Number of layers (per block) &  2\\
\hline
Modules of layer & 3D ResnetBlock\\
 & Spatial-cross Atten.\\
 & ATS Atten.\\
 & Down./Up. 3D ResnetBlock\\
\hline
Dimension of atten. heads &  8\\
activation function &  SiLU\\
Dimension of cross-atten. &  768\\
\hline
\end{tabular}
\end{table}

\begin{table}
\centering\small
\tablestyle{2pt}{1.1}
\caption{Hyperparameters of FSText Decomposer}
\label{table:hyperparam:fstext}
\begin{tabular}{c|cc}
\hline
hyperparam. & value\\
\hline
learnable tokens channels &  768\\
output channels &  768\\
Base channels & 768\\
Number of layers &  8\\
\hline
Modules of layer & Seq-cross Atten.\\
 & Feedforward\\
 & Directed temporal Atten.\\
 & Feedforward\\
\hline
Number of atten. heads &  8\\
Dimension of cross-atten. &  768\\
\hline
\end{tabular}
\end{table}




\section{Implementation Details}~\label{appendix:sec:impl}

\subsection{Fine-tuning and Sampling}\label{sec:finetuneparam}
 In this section, we list the hyperparameters, fine-tuning details, sampling details, and hardware information of our model in Table~\ref{table:hyperparam:finetune}.
 
\subsection{Architecture information}\label{sec:arch}
In this section, we list the hyperparameters of 3D U-Net in Table~\ref{table:3dunet} and hyperparameters of FSText Decomposer in Table~\ref{table:hyperparam:fstext}.



\section{Visualization} 

\subsection{Additional qualitative results} 
We provide additional visualization on Something-Something v2 (SSv2) of our text-conditioned video prediction in Figure~\ref{fig:ssv2pred}, and text-conditioned video prediction/manipulation results in Figure~\ref{fig:ssv2mani}. Additionally, we provide the visualization on BridgeData of text-conditioned video prediction in Figure~\ref{fig:bridgepred} and text-conditioned video prediction/manipulation in Figure~\ref{fig:bridgemani}.
\section{Human Evaluation Details}~\label{appendix:sec:humaneval} 
To evaluate the quality of video predictions according to human preferences, we conducted a human evaluation with 99 video clips on the validation set of the Something-Something V2 dataset (SSv2), the evaluation process involved 54 anonymous evaluators. To eliminate biases towards specific baselines, we randomly selected 20 questions for each evaluator. Each single-choice question consisted of a ground-truth video as a reference, a manually modified text instruction, and two video prediction results generated by Seer and another baseline method. The evaluators were required to choose the video clip that is more consistent with the text instruction and has higher fidelity from the two options.
To ensure the clarity of the questions, we provided an example to explain the options in each questionnaire. Moreover, we recommended that evaluators prioritize video predictions with strong text-based motions as their first preference and the fidelity of the generated video as their second preference. For reference, Figure~\ref{fig:humanevalexp} provides a screenshot of an example questionnaire.

In total, we collected 342 responses for the Seer vs. TATS comparison, 363 responses for the Seer vs. Tune-A-Video comparison, and 357 responses for the Seer vs. MCVD comparison. And the results in the main paper Figure 7 are calculated based on the collected questionnaires.
\clearpage
\begin{figure}
\centering
\includegraphics[width=1.0\linewidth]{fig_appendix/sth_predict.pdf}
\caption{Text-conditioned video prediction of Seer on SSv2.}
\vspace{-8pt}
\label{fig:ssv2pred}
\end{figure}
\begin{figure}
\centering
\includegraphics[width=1.0\linewidth]{fig_appendix/sth_manipulate.pdf}
\caption{Text-conditioned video prediction/manipulation of Seer on SSv2, where ``pred." refers to prediction, ``mani." refers to manipulation.}
\vspace{-8pt}
\label{fig:ssv2mani}
\end{figure}
\begin{figure*}
\centering
\includegraphics[width=0.9\linewidth]{fig_appendix/bridge_pred.pdf}
\caption{Text-conditioned video prediction of Seer on BridgeData.}
\vspace{-8pt}
\label{fig:bridgepred}
\end{figure*}
\begin{figure*}
\centering
\includegraphics[width=1.0\linewidth]{fig_appendix/bridge_manipulate.pdf}
\caption{Text-conditioned video prediction/manipulation of Seer on BridgeData, where ``pred." refers to prediction, ``mani." refers to manipulation.}
\vspace{-8pt}
\label{fig:bridgemani}
\end{figure*}


\begin{figure}
\centering
\includegraphics[width=1.0\linewidth]{fig_appendix/screenshot.PNG}
\caption{Screenshot of a questionnaire example shown to human evaluators.}
\vspace{-8pt}
\label{fig:humanevalexp}
\end{figure}
\end{document}