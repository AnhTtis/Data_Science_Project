% Leveraging simulation data is important to adapt machine learning models to new domains and deal with change in sensors.
% However, there is a domain gap between synthetic data generated from the simulator and data collected on the roads. 
% This domain gap can hinder the performance of detection models when trained on simulation data. 
% Sim-to-real domain transfer aims to post-process simulation images to reduce this domain gap and to enable the effective use of simulation data in model training.

% Current literature focuses on either paired or unpaired image-to-image translation. 
% However, simulators are capable of generating scenes that loosely mimic a real world scene’s lighting, world, and composition. 
% We can leverage this loose pairing and other features provided by the simulator (eg: semantic segmentation and class labels) to create a targeted architecture specifically for our use cases.


% Advancements in graphics technology have led to increased utilization of simulated data for training machine learning models.
% However, the data generated in simulation often differs from real-world data, creating a distribution gap that can reduce the effectiveness of models trained on simulation data when applied to real-world scenarios.
% To address this gap, sim-to-real domain transfer aims to modify simulation images to better align them with real-world data, thereby enabling the efficient use of simulation data in model training.

Advancements in graphics technology has increased the use of simulated data for training machine learning models. 
However, the simulated data often differs from real-world data, creating a distribution gap that can decrease the efficacy of models trained on simulation data in real-world applications.
To mitigate this gap, sim-to-real domain transfer modifies simulated images to better match real-world data, enabling the effective use of simulation data in model training.

Sim-to-real transfer utilizes image translation methods, which are divided into two main categories: paired and unpaired image-to-image translation. Paired image translation requires a perfect pixel match, making it difficult to apply in practice due to the lack of pixel-wise correspondence between simulation and real-world data.
Unpaired image translation, while more suitable for sim-to-real transfer, is still challenging to learn for complex natural scenes. 
To address these challenges, we propose a third category: approximately-paired sim-to-real translation, where the source and target images do not need to be exactly paired.
Our approximately-paired method, AptSim2Real, exploits the fact that simulators can generate scenes loosely resembling real-world scenes in terms of lighting, environment, and composition. Our novel training strategy results in significant qualitative and quantitative improvements, with up to a $24\%$ improvement in FID score compared to the state-of-the-art unpaired image-translation methods.
% Must Capture (clearly): Why should someone read the paper
% - New category of problem. Why important. Results. Little bit of method
% - very important 
% Categories:
% - Title
% - Title & Abstract
% - Title, Abstract, and Figures 
% - Titles, Abstract, Figures, Tables
% - Few people read the entire paper 
