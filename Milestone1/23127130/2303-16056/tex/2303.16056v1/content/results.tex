\section{Results}\label{sec:results}

In order to test the capabilities of the \gls{snpe} algorithm, we consider a multi-compartmental model which consists of a chain of passive compartments, see \cref{fig:evaluation}.
Such multi-compartmental models have been used to model dendrites and axons \citep{rall1962electrophysiology, fatt1951analysis}.
Each compartment $i$ is connected to a leak potential $V_\text{leak}$ via a leak conductance $g_\text{leak}^i$ and to the neighboring compartment via an axial conductance $g_\text{axial}^{i \leftrightarrow i + 1}$, compare \cref{eq:mc_current}.
These conductances serve as our parameters $\myvec{\theta}$, all other parameters are fixed.

\begin{figure}
	\tikzsetnextfilename{evaluation}
	\begin{tikzpicture}
		\newlength{\heightmodel}
		\setlength{\heightmodel}{0.35in}
		\coordinate (a) at (0,0);
		\node[panel] at (a) {\input{../fig/mc_model.tex}};
		\node[panel] at ($ (a) - (0, \heightmodel) $) {\importpgf{psp_evaluation}};
	\end{tikzpicture}
	\caption{Model of a passive compartment chain.
		The parameters of the model are given by the leak conductance in each compartment $g_\text{leak}^i$ and the axial conductance between compartments $g_\text{axial}^{i \leftrightarrow i + 1}$.
		In our experiment we observe the propagation of \glsfmtfirstpl{psp}.
		Here we show membrane traces of neurons which were emulated on the \glsfmtlong{bss2} system.
		We inject a synaptic input (vertical lines) in one compartment after another and record the membrane potential in each compartment (different rows).
		From these traces we extract the height of the \glsfmttext{psp} $h_{ij}$.
		We use the matrix of all heights $\mymat{H}$, the heights resulting from an input to the first compartment $\myvec{F} = [h_{00}, h_{10}, h_{20}, h_{30}]$ or the decay constant $\tau$ from an exponential fit to $\myvec{F}$ as observables.
		The scale bar in the lower right corner indicates the voltage and time in the hardware domain.}
	\label{fig:evaluation}
\end{figure}

We inject synaptic inputs in the different compartments and observe how the \glspl{psp} propagate along the chain, \cref{fig:evaluation}.
Since we are only interested in the passive propagation, we disable the spiking threshold, this is equivalent to $V_\text{thres} \to \infty$.
Due to the low-pass properties of the passive chain, the response in the first compartment broadens and its height decreases as the synaptic input is injected more distal from the first compartment, compare first row in \cref{fig:evaluation}.
A similar behavior is visible when we look at the voltage traces in the second compartment: the \glspl{psp} broaden and flatten for more distal inputs.
Since we consider a finite chain, we can also see that an input at the end of the chain affects the membrane potential more strongly, for example $h_{10} > h_{12}$.

The height of the \glspl{psp} depends on the leak and axial conductance \citep{fatt1951analysis}.
If we look at the response at the injection site, a higher leak or axial conductance will result in lower heights as less charge can accumulate in the compartment.
Therefore, the \gls{psp} heights $\mymat{H}$ or quantities derived form them are suitable observations $\myvec{x}$ that can be used to infer parameters $\myvec{\theta}$.

\subsection{Two-dimensional Parameter Space}\label{sec:2d}
Before looking at a higher-dimensional parameter space in the next section, we reduce the dimensions of the parameter space to two by setting the leak and axial conductance for all compartments and connections to the same digital value\footnote{Due to the production induced mismatch between analog circuits, the same digital values lead to different conductances on the \gls{bss2} system.}; $g_\text{leak}^i = g_\text{leak}\; \forall i \in \{0, 1, 2, 3\}$ and $g_\text{axial}^{i \leftrightarrow i + 1} = g_\text{axial}\; \forall i \in \{0, 1, 2\}$.
This allows us to easily visualize and analyze the approximated posteriors.
Furthermore, the low-dimensional parameter space allows us to perform a grid search in a reasonable amount of time.
The grid search result can give an intuition about the behavior of the chain and can be used as a comparison to the approximated posterior obtained with the \gls{snpe} algorithm.

\begin{figure*}
	\tikzsetnextfilename{2d}
	\newlength{\width}
	\setlength{\width}{0.333333\doublecolumn}
	\newlength{\heighttop}
	\setlength{\heighttop}{2in}

	\begin{tikzpicture}
		\coordinate (a) at (0,0);
		\node[panel] at (a) {\importpgf{grid_search}};
		\node[fig label] at (a) {A};
		\coordinate (b) at (0, -\heighttop);
		\node[panel] at (b) {\importpgf{traces_scaled}};
		\node[fig label] at (b) {B};

		\coordinate (c) at (\width, 0);
		\node[panel] at (c) {\importpgf{posterior}};
		\node[fig label] at (c) {C};

		\coordinate (d) at (2\width, 0);
		\node[panel] at (d) {\importpgf{posterior_samples}};
		\node[fig label] at (d) {D};
		\coordinate (e) at (2\width, -\heighttop);
		\node[panel] at (e) {\importpgf{traces}};
		\node[fig label] at (e) {E};
	\end{tikzpicture}
	\caption{
		Propagation of \glsfmtfirstpl{psp} in a passive chain of four compartments emulated on the \glsfmtlong{bss2} system.
		Leak and axial conductance are set to the same value for all compartments and connections between compartments.
		\subcaption{A} Grid search of the decay constant $\tau$; the decay constant is given in units of \enquote{compartments} and calculated by fitting an exponential to the \glsfmtlongpl{psp} which result from an input to the first compartment, compare \cref{fig:evaluation}.
			We divided the parameter space in an evenly spaced grid with \num{40} values in each dimension and recorded the resulting \glsfmttext{psp} heights in each compartment, compare \cref{fig:evaluation}.
			Next, we fitted an exponential decay to the \glsfmttext{psp} which resulted from an input to the first compartment and extracted the decay constant $\tau$, \cref{fig:exponential} shows the exponential fits for some exemplary measurements.
			The decay constant $\tau$ decreases as the leak conductance $g_\text{leak}$ is increased or the axial conductance $g_\text{axial}$ is reduced.
			The white contour lines mark regions with equal decay constant and show a correlation between leak and axial conductance.
			Traces recorded at the numbered points are displayed in panel B and E.
		\subcaption{B} Example traces recorded at different locations in the parameter space, compare panel A.
			The traces are scaled relative to the height in the first compartment $h_{00}$.
			Due to the faster emulation of the neural dynamics on \gls{bss2}, the time scales are in the microsecond rather than in the millisecond range.
		\subcaption{C} Posterior obtained with the \glsfmtlong{snpe} algorithm.
			The posterior shows a high density in the parameter region where the target decay constant $\tau^*$ was recorded, \Circled{2}.
			As expected from the grid search result in panel A, we also see a correlation between the leak and axial conductance.
			Points where the decay constant is significantly lower/higher than the target observation show a low probability density, \Circled{0} and \Circled{3}.
		\subcaption{D} \num{500} random samples drawn from the approximated posteriors for two different types of observations.
			The green points represent samples drawn from the posterior which is shown in panel C.
			The samples show a correlation between both parameters.
			If the absolute heights of the \glsfmttext{psp} which result from an input to the first compartment $\myvec{F} = [h_{00}, h_{10}, h_{20}, h_{30}]$ are chosen as observations (blue), the samples scatter around point \Circled{2} where the original target $\myvec{F}^*$ was recorded.
			The histograms at the top and right of the scatter plot show histograms of the parameter distribution in one dimension.
		\subcaption{E} Same traces as in panel B but shown on an absolute scale.
			While traces \Circled{1} and \Circled{2} share a similar decay constant $\tau$, compare panel A and B, their absolute height differs.
	}
	\label{fig:2d}
\end{figure*}

\subsubsection{Grid Search}\label{sec:grid}
In order to obtain an overview of the model behavior, we perform a grid search over the two-dimensional parameter space.
We create a grid of parameters by choosing equally spaced values of the leak conductance and the axial conductance which span the whole parameter range.
The model is then emulated with these parameters on the \gls{bss2} system and the membrane traces in the different compartments are recorded.
In order to easily visualize the results, we choose a one-dimensional observable.
Exponential fits to the maximal height of propagating \glspl{psp} were used in other publications to classify the attenuation of \glspl{psp} in apical dendrites \citep{berger2001high}.
Similarly, we fit an exponential to the \gls{psp} heights which result from an input to the first compartment $\myvec{F} = [h_{00}, h_{10}, h_{20}, h_{30}]$ and analyze the exponential decay constant $\tau$, \figGridsearch.
The decay constant increases with increasing axial conductance $g_\text{axial}$ and decreasing leak conductance $g_\text{leak}$.
Even though the exponential is just an approximation for the attenuation of transient inputs in multi-compartmental models, a correlation between leak and axial conductance is expected \citep{fatt1951analysis, rall1962electrophysiology}.

The responses of the membrane potentials to a synaptic input in the first compartment are displayed in \figTracesRelative.
For a low leak and a large axial conductance \Circled{0}, the attenuation is the weakest and the \gls{psp} is still clearly visible in the last compartment.
Parameters on the same contour line show as expected similar attenuation, \Circled{1} and \Circled{2}, even though the exact shape of the \glspl{psp} differ.
For a large leak and a low axial conductance, \Circled{3}, the \gls{psp} decays quickly and is already near zero in the third compartment.


\subsubsection{Simulation Based Inference}\label{sec:2d-sbi}
We will now use the \gls{snpe} algorithm to infer possible parameters $\myvec{\theta} = [g_\text{leak}, g_\text{axial}]$ which reproduce a target observation $\myvec{x}^* = [\tau^*]$.
Furthermore, we will investigate how the posterior distribution changes when we use a more informative observation $\myvec{x}^* = \myvec{F}^* = [h_{00}^*, h_{10}^*, h_{20}^*, h_{30}^*]$, compare \cref{fig:evaluation}.

In the case where a target observation $\myvec{x}^*$ is given by an experiment, the true posterior and the optimal parameters which replicate the observation are typically unknown.
This makes it hard to assess the quality of the approximated posterior found with the \gls{snpe} algorithm.
Therefore, we will explicitly choose target parameters $\myvec{\theta}^*$, emulate our model with these parameters on \gls{bss2} and measure an \enquote{artificial} target observation $\myvec{x}^* = \tau^*$.
This allows us to perform a closure test and check whether the \gls{snpe} algorithm is able to estimate a posterior which agrees with the initial observation.

We pick target parameters $\myvec{\theta}^*$ at the center of the parameter space and execute the model with these parameters \num{100} times to account for trial-to-trial variations due to temporal noise.
The mean of the observed decay constants is our target observation $\myvec{x}^* = [\overline{\tau}^*] = 1.17 \pm 0.04$; the decay constant is in units of \enquote{compartments}.

We use a uniform distribution over all possible parameters as a prior distribution $p(\myvec{\theta})$ and execute the \gls{snpe} algorithm to obtain an approximation of the posterior distribution $p\left( \myvec{\theta} \mid \myvec{x}^* \right)$.

The correlation between the leak $g_\text{leak}$ and the axial conductance $g_\text{axial}$ is clearly visible in the approximated posterior, \figPosterior.
The posterior distribution shows high densities for parameters $\myvec{\theta}$ which reproduced observations near the target observation during the grid search, compare contour lines in \figGridsearch.
For larger leak conductances $g_\text{leak}$ the posterior probability decreases.
This also agrees with the spread of the contour lines fitted to the grid search result.

In order to perform a \gls{ppc}, we draw samples $\{\myvec{\theta}_i\}$ from the posterior distribution, \figPosterior, configure our model with them and compare the observations $\{\myvec{x}_i\}$ with the target observation $\myvec{x}^*$.
We measure a mean decay constant of $\overline{\myvec{\tau}} = 1.20 \pm 0.09$ which agrees within uncertainty with the target $\myvec{\tau}^* = 1.17 \pm 0.04$.
Therefore, we conclude that the approximated posterior is in agreement with the target observation $\myvec{\tau}^*$.

If we want to rediscover parameters which are closer to our original parameters $\myvec{\theta}^*$, we need more informative observations.
While the \gls{psp} heights show a similar decay for different sets of leak and axial conductance, \figTracesRelative, the absolute heights of the \glspl{psp} differ, \figTracesAbsolute.
We can therefore use the \glspl{psp} heights which result from an input to the first compartment $\myvec{F}$ as a target observation $\myvec{x}^* = \myvec{F}^*$ to further constrain possible parameters.
Samples $\{\myvec{\theta}_i\}$ drawn from the posterior are now scattered around the original parameter $\myvec{\theta}^*$ in the parameter space and both parameters seem uncorrelated, \figSamples; the Pearson correlation coefficient decreases from \num{0.93} to \num{0.06}.
The marginal distribution of the leak and axial conductance are bell-shaped and show a high density near the target parameter $\myvec{\theta}^*$.

In order to validate the results we obtained in this section, we implemented a computer simulation in the simulation library Arbor \citep{abi2019arbor}.
The topology of the parameter space as well as the shape of the approximated posteriors agree well between emulation and simulation, see \cref{fig:arbor}.

\subsection{Multidimensional Parameter Space}\label{sec:multi}

\begin{figure*}
	\tikzsetnextfilename{md}
	\begin{tikzpicture}
		\coordinate (a) at (0,0);
		\node[panel] at (a) {\importpgf{1d_marginals}};
		\node[fig label] at (a) {A};

		\coordinate (b) at (0.5\doublecolumn, 0);
		\node[panel] at (b) {\importpgf{observations_md}};
		\node[fig label] at (b) {B};
	\end{tikzpicture}
	\caption{
		Results of the \glsfmtlong{snpe} for a compartment chain of \num{4} compartments and setting parameters individually for each compartment and connection between them.
		Emulations were performed on the neuromorphic \glsfmtlong{bss2} system.
		\subcaption{A} Histograms of \num{1000} parameters drawn from the approximated posterior.
			If we chose the heights $\myvec{F}$ of the \glsfmtfirstpl{psp} which result from an input to the first compartment as a target observation (blue), the distribution of the leak conductance in the first compartments is bell-shaped and peaks near the target parameter (dotted line).
			The leak conductance is roughly uniformly distributed in later compartments.
			The distributions of the axial conductance are bell-shaped and broaden for later compartments.
			Choosing all heights $\mymat{H}$ as a target (orange) leads to narrower distributions.
			All histograms are now bell-shaped with a peak near the target (dotted line).
		\subcaption{B} \Acrlong{ppc}.
			The passive chain is configured with the parameters $\{\myvec{\theta}_i\}$ drawn in panel A and the \glsfmttext{psp} heights in all compartments $\{\mymat{H}_i\}$ are measured on the \glsfmtlong{bss2} system.
			These \glsfmttext{psp} heights are compared to the observation $\mymat{H}^*$ which represents the measurement with the target parameters $\myvec{\theta}^*$.
			The vertical lines show the mean deviation of the observations $\{\mymat{H}_i\}$ from this target $\mymat{H}^*$ while the horizontal bars illustrate the standard deviation of this deviation.
			As mentioned in the introduction, analog hardware is subject to temporal noise.
			Therefore, the target hardware was configured to the target parameters $\myvec{\theta}^*$ \num{100} times and the mean \glsfmttext{psp} heights were chosen as a target $\mymat{H}^*$; the deviation in this panel are scaled by the standard deviation $\sigma^*$ of these \num{100} measurements (each height deviation $h_{ij}$ is divided by the standard deviation of this height $\sigma^*_{ij}$).
			For all \glsfmttext{psp} heights the mean observation is within \numrange{1}{2} standard deviations of the initial target.
			When a more informative observation $\mymat{H}$ is chosen, the standard deviations of the heights which do not include inputs or measurements in the first compartment more than halves.
	}
	\label{fig:md}
\end{figure*}

In order to increase the problem complexity, we set the leak and axial conductance for each compartment and connection individually.
For four compartments this results in a total of seven parameters.

As in the previous section we use a uniform prior and the \gls{psp} heights caused by an input to the first compartment as a target ($\myvec{x}^*=\myvec{F}^*$).
We then execute the \gls{snpe} algorithm and draw samples from the approximated posterior $p(\myvec{\theta} \mid \myvec{x}^*)$.

The marginal distribution of the sampled leak conductance in the first compartment $g_\text{leak}^0$ is bell-shaped and peaks near the target parameter, \figMarginals.
The almost uniform distributions of the leak conductances in the other compartments indicate that they are not relevant for the chosen observation.
In contrast, the marginal distribution of all axial conductances are bell-shaped with a high density around the original parameters.
The distributions of the axial conductance become broader as we advance along the chain, suggesting that the influence on the observable weakens for axial resistances later in the chain.

We once again use a \gls{ppc} to check if samples drawn from the approximated posterior $\{\myvec{\theta}_i\}$ reproduce the target observation.
The mean difference between observations $\{\mymat{H}_i\}$ obtained with these parameters and a target observation $\mymat{H}^*$ are displayed in \figPPC.
$\mymat{H}$ describes the observation of all \gls{psp} heights for inputs to different compartments and the target observation $\myvec{F}^*$ was extracted from $\mymat{H}^*$, see \cref{fig:evaluation}.

The mean of the \gls{psp} heights for an input to the first compartment (first column) is near the initial target values; the standard deviation is in the range of \numrange{1}{2}~$\sigma^*$ where $\sigma^*$ is the standard deviation of the measurements which were used to extract the target observation $\mymat{H}^*$.
A similar behavior can be seen if we only look at the response in the first compartment (first row), indicating that for these observations the leak conductance in the first compartment and the axial conductances are most relevant.
For the other \gls{psp} height the mean is still in the two-sigma range of the initial target observation, but the standard deviation of the observations is significantly higher.
This is expected since we did not use these \gls{psp} heights as an observation and the higher standard deviation can be attributed to the broad posterior distribution of the leak and axial conductance in later compartments.

Similar to the two-dimensional case, we can consider a higher-dimensional observation as a target to retrieve narrower posterior distributions.
When we choose all \gls{psp} heights as a target ($\myvec{x}^* = \mymat{H}^*$) the posterior distribution becomes narrower, \figMarginals.
Now the one-dimensional marginals of all parameters are bell-shaped.
The marginals of the axial conductance show a narrower distribution than these of the leak conductance, indicating that the given observation is more sensitive to the axial conductance.
The posterior distribution of the leak conductance in the first and last compartment show a similar confidence, while the distributions are broader for the two compartments within the chain.

The sharpening of the posterior distribution is also visible in the results of the \gls{ppc}, \figPPC.
Here the standard deviation of the observations is now in the range of \numrange{1}{2}~$\sigma^*$ for all \gls{psp} heights.
