\section{Discussion}
\glsresetall

We have shown that the \gls{snpe} algorithm can be used to parameterize the analog neuromorphic \gls{bss2} system.
To be able to validate the approximated posteriors found by the \gls{snpe} algorithm, we selected a multi-compartmental model which takes the form of a chain of passive compartments.
We chose the leak conductance as well as the axial conductance between compartments as parameters and observe how \glspl{psp} propagate along the chain.
This model allows us to easily change the dimensionality of the parameter space as well as the choice of observable and evaluate how this influences the approximated posteriors.

In all our experiments, we pick a set of target parameters, extract an observation with these parameters and then use the \gls{snpe} algorithm to approximate the posterior distribution of the parameters which reproduce this given observation.

As a first step, we considered a two-dimensional parameter space where we set all leak conductances and axial conductances to the same value.
The low dimensionality of the parameter space allowed us to perform a grid search in a reasonable amount of time.
The posterior approximation found with the \gls{snpe} algorithm agrees with the results from this grid search.
In both cases we find a correlation between the leak and axial conductance if we look at the attenuation of \glspl{psp}; this argees with theoretical expectations \citep{fatt1951analysis, rall1962electrophysiology}.
To be able to find such correlation is one of the advantages of a posterior approximation over traditional parameter search results which usually only yield a set of parameters which reproduce the given observation but do not illustrate the relation between different parameters.

When we chose a more informative observation, specifically the height of the \glspl{psp} which result from an input to the first compartment, the posterior distribution of the parameters narrows and the correlation between leak and axial conductance vanishes.
We further show that the algorithm is capable of finding appropriate posterior approximations for several, random values of the target parameters.
The approximations are even in agreement with the target parameters if they lie at the edges of the parameter space.
This indicates that the algorithm is able to deal with the hard parameter limits which are dictated by the neuromorphic hardware.

Next, we increased the dimensionality of the parameter space by adjusting each leak and axial conductance individually; resulting in a seven-dimensional parameter space.
We show that the marginal distributions of samples drawn from the posterior approximation have a high density around the target parameters.
Furthermore, we perform \glspl{ppc} to confirm, that the parameters drawn from the approximated posterior produce emulation results which are in agreement with the target observation.
As in the two-dimensional case, increasing the dimensionality of the observable leads to narrower posterior distributions.

Simulation which we performed in the simulation library Arbor show results which are comparable to the emulation results on \gls{bss2} \citep{abi2019arbor}.
The topology of the two-dimensional parameter spaces agree well between simulation and emulation.
Even though the approximated posteriors are narrower for the simulations, the overall shape resembles the shape of the posterior distributions found for emulations on \gls{bss2}.

Our current results are limited to observations which were created by the model itself.
Therefore, we can be certain, that model parameters exist which reproduce the given observation.
In subsequent studies, we will use the \gls{snpe} algorithm to replicate observations which are provided by another model such as a numerical simulation or by physiological experiments.
As seen in the grid search results, we dealt with a rather smooth parameter space, where the observations change gradually with the model parameters.
In future, we will challenge the \gls{snpe} algorithm with somatic and dendritic spikes.

In summary, we show that the \gls{snpe} algorithm is suitable to find posterior approximations for parameters of the analog neuromorphic \gls{bss2} system.
The algorithm is able to deal with the inherent trial-to-trial variations and the fixed parameter ranges of the \gls{bss2} system.
