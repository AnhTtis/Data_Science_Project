\documentclass[11pt]{article}




\usepackage[utf8]{inputenc} % allow utf-8 input
\usepackage[T1]{fontenc}    % use 8-bit T1 fonts
\usepackage{hyperref}       % hyperlinks
\usepackage{url}            % simple URL typesetting
\usepackage{booktabs}       % professional-quality tables
\usepackage{amsfonts}       % blackboard math symbols
\usepackage{nicefrac}       % compact symbols for 1/2, etc.
\usepackage{microtype}      % microtypography
\usepackage{lipsum}
\usepackage{graphicx}
\usepackage{comment}
\usepackage{outlines}
\usepackage{enumitem}
\graphicspath{{media/}}
\usepackage{tabularx}
\usepackage{todo}
\usepackage{colortbl}
\usepackage{booktabs}
\usepackage{array}
\usepackage{float}
\usepackage[table]{xcolor}
\usepackage{caption} 
\captionsetup[table]{skip=6pt}
\usepackage{newtxtext,newtxmath}
\usepackage{pdflscape}
\usepackage{threeparttable}
\usepackage{authblk, tikz}
\usepackage{dcolumn, ulem}
\usepackage{subfigure}
% \usepackage{lscape}
\usepackage[round]{natbib}
% \usepackage[sectionbib,square]{natbib}
% \usepackage{dcolumn}
% \usepackage[singlespacing]{setspace}
\usepackage[margin=1in]{geometry}
% \usepackage{numprint}
% \usepackage{adjustbox}
% \usepackage{comment}
% \usepackage{titling}

% organize your images and other figures under media/ folder

  
%% Title (provisional)
\title{GPTs are GPTs: An Early Look at the Labor Market Impact Potential of Large Language Models} 

\author[1]{Tyna Eloundou}
\author[1,2]{Sam Manning}
\author[1]{Pamela Mishkin\thanks{Corresponding author (pamela@openai.com). Authors contributed equally and are listed alphabetically.}}
\author[3]{Daniel Rock}
\affil[1]{OpenAI}
\affil[2]{OpenResearch}
\affil[3]{University of Pennsylvania}


\begin{document}
\maketitle

\begin{abstract}

We investigate the potential implications of Generative Pre-trained Transformer (GPT) models and related technologies on the U.S. labor market. Using a new rubric, we assess occupations based on their correspondence with GPT capabilities, incorporating both human expertise and classifications from GPT-4. Our findings indicate that approximately 80\% of the U.S. workforce could have at least 10\% of their work tasks affected by the introduction of GPTs, while around 19\% of workers may see at least 50\% of their tasks impacted. The influence spans all wage levels, with higher-income jobs potentially facing greater exposure. Notably, the impact is not limited to industries with higher recent productivity growth. We conclude that Generative Pre-trained Transformers exhibit characteristics of general-purpose technologies (GPTs), suggesting that as these models could have notable economic, social, and policy implications.

%(alternatively, how much time a human could save using an LLM-powered system for help in the occupation)
\end{abstract}


% \keywords{Large language models \and Labor markets \and Generative Pre-trained Transformers \and General-purpose technologies}

% \begin{comment}
% - We need 3-4 Figures to really make the paper appealing to gensci journals.

% Candidate Figure 1: Flow showing how the measures are produced (1-3 panels... one for each approach)
% Panel with the automation/augmentation correlation scatterplot at the task level
% Panel with the agreement / disagreement areas

% Candidate Figure 2: Binscatters on wages and employment; Regional map + industry info (employment + productivity growth stuff). Demo stats w/variance breakdown into occupational variation, maybe age, gender, etc.

% Candidate Figure 3: All about validity. Not sure how to plot / visualize. Earlier effort plot can be bar charts?
% External validity component can be a table with use cases and maybe growth in API queries. Job postings... need to scope what we'd do there... same table idea for companies with products?

% Candidate Figure 4: Growth of language models (papers, api usage etc.). Diff-in-diff plot with ChatGPT launch on github stats for huggingface relative to other libraries. Others?



% \section{Overview}
% \end{comment}


% \begin{figure}
%     \centering
%     \includegraphics[width=0.8\textwidth]{figures/image_capabilities.png}
%     \caption{On the left is the trajectory of GAN progress on face generation as compiled by Ian Goodfellow from \citep{Goodfellow2014-sy, Radford2015-dd, Liu2016-uc, Karras2017-th, Karras2018-gh}, on the right is an image generated from DALL-E 2.0 \citep{dalle2} using outpainting. Today's image generators can create images of arbitrary size and, accordingly, resolution of any subject and in any framing.}
%     \label{fig:goodfellow}
% \end{figure}



\section{Introduction}

As shown in Figure \ref{fig:exames}, recent years, months, and weeks have seen remarkable progress in the field of generative AI and large language models (LLMs). While the public often associates LLMs with various iterations of the Generative Pre-trained Transformer (GPT), LLMs can be trained using a range of architectures, and are not limited to transformer-based models \citep{47751}. %These advances in performance may be attributed to breakthroughs in the Transformer architecture \citep{vaswani2017attention} and research advances which demonstrated significant returns to investment in more data, parameters and computing power \citep{kaplan2020scaling, hoffmann_training_2022}. 
LLMs can process and produce various forms of sequential data, including assembly language, protein sequences and chess games, extending beyond natural language applications alone. In this paper, we use LLMs and GPTs somewhat interchangeably, and specify in our rubric that these should be considered similar to the GPT-family of models available via ChatGPT or the OpenAI Playground (which at the time of labeling included models in the GPT-3.5 family but not in the GPT-4 family). We examine GPTs with text- and code-generating abilities and employ the term "generative AI" to additionally include modalities such as images or audio. %The swift development of generative AI capabilities across diverse modalities poses challenges in understanding their progression and predicting their potential impacts on human labor (see Figure \ref{fig:exames}).

\begin{figure}
    \centering
    \includegraphics[width=.9\textwidth]{gpt_4_exams_new.png}
    \caption{To get a sense of how quickly model capabilities are progressing -- consider the jump in exam performance between GPT-3.5 and GPT-4 \citep{gpt4}.}
    \label{fig:exames}
\end{figure}

Our study is motivated less by the progress of these models alone though, and more by the breadth, scale, and capabilities we've seen in the complementary technologies developed around them. The role of complementary technologies remains to be seen, but maximizing the impact of LLMs appears contingent on integrating them with larger systems \citep{bresnahan2019artificial, agrawal2021ai}. While we focus much of this discussion on the generative capabilities of LLMs, there may be new types of software and machine communication made possible by use of LLMs for other tasks including things like embeddings which make it possible to build custom search applications or tasks like summarization and classification where it can be unclear where to draw the distinction over what is or is not generative.

%Our study is motivated by several distinct contextual factors: 1) the rapid evolution of GPT capabilities across text, code, and other modalities, 2) the emergence and improvement of unexpected abilities over time, and 3) generative models' capacity to act as discriminators and select outputs that align with designer objectives. While the measurement of automation and general ML capabilities is ongoing, progress in LLMs in particular merits separate investigation.



To contextualize this progression and complement labor impact forecasts of technology, we propose a new rubric for understanding LLM capabilities and their potential effects on jobs. This rubric (\ref{exposure_tax}) measures the overall exposure of tasks to GPTs, following the spirit of prior work on quantifying exposure to machine learning \citep{Brynjolfsson2018, SeamansRajFelten2018, Webb2020}. We define exposure as a proxy for potential economic impact without distinguishing between labor-augmenting or labor-displacing effects. We employ human annotators and GPT-4 itself as a classifier to apply this rubric to occupational data in the U.S. economy, primarily sourced from the O*NET database.\footnote{This is distinct from recent social science research that makes use of advanced language models to simulate human behavior \citep{horton2023large, Sorensen_2022}}\footnote{While our exposure rubric does not necessarily tie the concept of language models to any particular model, we were strongly motivated by our observed capabilities of GPT-4 and the suite of capabilities we saw in development with OpenAI's launch partners \citep{gpt4}.}

To construct our primary exposure dataset, we collected both human annotations and GPT-4 classifications, using a prompt tuned for agreement with a sample of labels from the authors. We observe similar agreement levels in GPT-4 responses and between human and machine evaluations, when aggregated to the task level. This measure reflects an estimate of the technical capacity to make human labor more efficient; however, social, economic, regulatory, or other determinants imply that technical feasibility does not guarantee labor productivity or automation outcomes. Our analysis indicates that approximately 19\% of jobs have at least 50\% of their tasks exposed when considering both current model capabilities and anticipated tools built upon them. Human assessments suggest that only 3\% of U.S. workers have over half of their tasks exposed to GPT when considering existing language and code capabilities without additional software or modalities. Accounting for other generative models and complementary technologies, our human estimates indicate that up to 49\% of workers could have half or more of their tasks exposed to LLMs. 

Our findings consistently show across both human and GPT-4 annotations that most occupations exhibit some degree of exposure to LLMs, with varying exposure levels across different types of work. Occupations with higher wages generally present with high exposure, a result contrary to similar evaluations of overall machine learning exposure \citep{brynjolfssonQuantifyingDistributionMachine2023}. When regressing exposure measures on skillsets using O*NET's skill rubric, we discover that roles heavily reliant on science and critical thinking skills show a negative correlation with exposure, while programming and writing skills are positively associated with LLM exposure. Following \citet{autor2022new}, we examine barriers to entry by "Job Zones" and find that occupational exposure to LLMs weakly increases with the difficulty of job preparation. In other words, workers facing higher (lower) barriers to entry in their jobs tend to experience more (less) exposure to LLMs.

We further compare our measurements to previous efforts documenting the distribution of automation exposure in the economy and find broadly consistent results. Most other technology exposure measures we examine are statistically significantly correlated with our preferred exposure measure, while measures of manual routineness and robotics exposure show negative correlations. The variance explained by these earlier efforts \citep{acemoglu2011skills, FreyOsborne2017, Brynjolfsson2018, SeamansRajFelten2018, Webb2020, brynjolfssonQuantifyingDistributionMachine2023}, along with wage controls, ranges from 60 to 72\%, indicating that 28 to 40\% of the variation in our AI exposure measure remains unaccounted for by previous technology exposure measurements. 

We analyze exposure by industry and discover that information processing industries (4-digit NAICS) exhibit high exposure, while manufacturing, agriculture, and mining demonstrate lower exposure. The connection between productivity growth in the past decade and overall GPT exposure appears weak, suggesting a potential optimistic case that future productivity gains from LLMs may not exacerbate possible cost disease effects \citep{baumol2012cost}. \footnote{Baumol's cost disease is a theory that explains why the cost of labor-intensive services, such as healthcare and education, increases over time. This happens because wages for skilled workers in other industries increase, but there is no corresponding increase in productivity or efficiency in these service industries. Therefore, the cost of labor in these industries becomes relatively more expensive compared to other goods and services in the economy.} 
% Based on our study, it appears that the influence of LLMs is likely to be widespread, considering their consistent improvement in capabilities over time. It's important to note that this potential impact could be even greater when considering the integration of LLMs with other technologies. However, it's worth clarifying that the paper doesn't necessarily require an assumption of continued improvement in capabilities.

Our analysis indicates that the impacts of LLMs like GPT-4, are likely to be pervasive. While LLMs have consistently improved in capabilities over time, their growing economic effect is expected to persist and increase even if we halt the development of new capabilities today. We also find that the potential impact of LLMs expands significantly when we take into account the development of complementary technologies. Collectively, these characteristics imply that Generative Pre-trained Transformers (GPTs) are general-purpose technologies (GPTs).\footnote{For the remainder of the paper, we use GPT to refer to large language models generally as exemplified by those available via OpenAI, and we spell out general-purpose technologies when it is used outside of stating "GPTs are GPTs."} \citep{bresnahan1995general, lipsey2005economic}. \citep{goldfarb2023could} argue that machine learning as a broad category is likely a general-purpose technology. Our evidence supports a wider impact, as even subsets of machine learning software meet the criteria for general-purpose technology status independently. This paper's primary contributions are to provide a set of measurements of LLM impact potential and to demonstrate the use case of applying LLMs to develop such measurements efficiently and at scale. Additionally, we showcase the general-purpose potential of LLMs. If "GPTs are GPTs," the eventual trajectory of LLM development and application may be challenging for policymakers to predict and regulate. As with other general-purpose technologies, much of these algorithms' potential will emerge across a broad range of economically valuable use cases, including the creation of new types of work \citep{acemoglu2018race, autor2022new} Our research serves to measure what is technically feasible now, but necessarily will miss the evolving impact potential of the LLMs over time.

The paper is structured as follows: Section \ref{sec:litreview} reviews relevant prior work, Section \ref{sec:data} discusses methods and data collection, Section \ref{sec:econimpact} presents summary statistics and results, Section \ref{sec:validation} relates our measurements to earlier efforts, Section \ref{sec:discussion} explores results, and Section \ref{sec:conclusion} offers concluding remarks.

\section{Literature Review}
\label{sec:litreview}

\subsection{The Advancement of Large Language Models}
\label{sec:llmlit}
In recent years, large language models (LLMs) have risen to prominence in the field of artificial intelligence (AI) research, showcasing their ability to tackle a wide array of complex language-based tasks. This progress has been fueled by multiple factors, including increased model parameter count, greater training data volume, and enhanced training configurations \citep{brown2020language, radford2019language, hernandez2021scaling, kaplan2020scaling}. Broad, state-of-the-art LLMs, such as LaMDA \citep{thoppilan2022lamda} and GPT-4 \citep{gpt4}, excel in diverse applications like translation, classification, creative writing, and code generation—capabilities that previously demanded specialized, task-specific models developed by expert engineers using domain-specific data.

% Although not all LLMs are GPTs, our focus here lies on GPTs. We also emphasize generative capabilities, acknowledging the substantial value AI has brought in recent years through the development of cutting-edge discriminators and embeddings.

Concurrently, researchers have improved the steerability, reliability, and utility of these models using methods like fine-tuning and reinforcement learning with human feedback \citep{ouyang2022training, bai_training_2022}. These advancements enhance the models' ability to discern user intent, rendering them more user-friendly and practical. Moreover, recent studies reveal the potential of LLMs to program and control other digital tools, such as APIs, search engines, and even other generative AI systems \citep{schick2023toolformer, mialon2023augmented, langchain}. This enables seamless integration of individual components for better utility, performance, and generalization. In the long run, these trends suggest that LLMs may be capable of executing any task typically performed at a computer.

For the most part, generative AI models have predominantly been deployed as modular specialists, carrying out specific tasks, like generating images from captions or transcribing text from speech. However, we argue that it is essential to adopt a broader perspective, recognizing LLMs as crucial building blocks for additional tools. While constructing these tools and integrating them into comprehensive systems will take time and necessitate significant reconfiguration of existing processes across the economy, we already observe emerging adoption trends. Despite their limitations, LLMs are becoming increasingly integrated into specialized applications in areas such as writing assistance, coding, and legal research, paving the way for businesses and individuals to adopt GPTs more widely.

We emphasize the significance of these complementary technologies, partly because out-of-the-box general-purpose GPTs may continue to be unreliable for various tasks due to issues such as factual inaccuracies, inherent biases, privacy concerns, and disinformation risks \citep{abid_persistent_2021, schramowski_large_2022, goldstein_generative_2023, 4systemcard}. However, specialized workflows—including tooling, software, or human-in-the-loop systems—can help address these shortcomings by incorporating domain-specific expertise. For example, \href{https://casetext.com/}{Casetext} offers LLM-based legal research tools that provide lawyers with quicker and more accurate legal research results, utilizing embeddings and summarization to counter the risk that GPT-4 provides innacurate details about a legal case or set of documents. \href{https://github.com/features/copilot}{GitHub Copilot} is a coding assistant that employs LLMs to generate code snippets and autocomplete code, which users can then accept or reject based on their expertise. In other words, while it's true that on its own GPT-4 does not "know what time it is," it's easy enough to give it a watch.

Furthermore, a positive feedback loop may emerge as LLMs surpass a specific performance threshold, allowing them to assist in building the very tooling that enhances their usefulness and usability across various contexts. This could lower the cost and engineering expertise required to create such tools, potentially accelerating LLM adoption and integration even further. \citep{chen2021evaluating, peng2023impact} LLMs can also become valuable assets in machine learning model development—serving as coding assistants for researchers, data labeling services, or synthetic data generators. There is potential for such models to contribute to economic decision-making at the task level, for instance, by refining methods for task and sub-task allocation between humans and machines \citep{Singla2015LearningTH,Shahaf2010GeneralizedTM}. As LLMs improve over time and better align with user preferences, we can anticipate a continuous enhancement in performance. However, it is essential to recognize that these trends also bring a variety of serious risks. \citep{hazard_analysis, Weidinger2022, solaiman_release}


\subsection{The Economic Impacts of Automation Technologies}

A large and growing body of literature addresses the labor market impacts of artificial intelligence and automation technologies broadly defined. The concept of skill-biased technological change and the task model of automation—often considered the standard framework for understanding technology's influence on labor—originated from research demonstrating that technological progress raises the demand for skilled workers over unskilled workers \citep{katz1992changes}. Numerous studies have built upon this concept, exploring the effects of technological change and automation on workers within a task-based framework \citep{autor2003skill, acemoglu_autor_2011, acemoglu2018race}. This strand of research has shown that workers involved in routine and repetitive tasks are at a higher risk of technology-driven displacement, a phenomenon known as routine-biased technological change. More recent studies have distinguished between technology's task-displacement and task-reinstatement effects (where new technology increases the need for a wider array of labor-intensive tasks) \citep{acemoglu2018race, acemoglu2019automation}. Several studies have shown that automation technologies have resulted in wage inequality in the US, driven by relative wage declines for workers specializing in routine tasks \citep{autor2006polarization, van2011wage, acemoglu2022tasks}. 

Prior research has employed various approaches to estimate the overlap between AI capabilities and the tasks and activities workers undertake in different occupations. These methods include mapping patent descriptions to worker task descriptions \citep{Webb2020, meindl2021exposure}, linking AI capabilities to occupational abilities documented in the O*NET database \citep{SeamansRajFelten2018, felten2023will}, aligning AI task benchmark evaluations with worker tasks via cognitive abilities \citep{Tolan2021}, labeling automation potential for a subset of US occupations and using machine learning classifiers to estimate this potential for all other US occupations \citep{FreyOsborne2017}, modeling task-level automation and aggregating the results to occupation-level insights \citep{arntz2017revisiting}, expert forecasts \citep{grace2018will}, and most relevantly to this paper, devising a new rubric to assess worker activities for their suitability for machine learning \citep{Brynjolfsson2018, brynjolfssonQuantifyingDistributionMachine2023}. Some of these approaches have found exposure to AI technologies at the task-level tends to be diversified within occupation. Considering each job as a bundle of tasks, it would be rare to find any occupation for which AI tools could do nearly all of the work. \citep{autor2022new} finds as well that automation and augmentation exposures tend to be positively correlated. There is also a growing set of studies examining specific economic impacts and opportunities for LLMs \citep{bommasani2021opportunities, felten2023will, korinek2023language, mollick2022new, noy2023experimental, peng2023impact}. Alongside this work, our measurements help characterize the broader potential relevance of language models to the labor market. 


General-purpose technologies (e.g. printing, the steam engine) (GPTs) are characterized by widespread proliferation, continuous improvement, and the generation of complementary innovations \citep{bresnahan1995general, lipsey2005economic}. Their far-reaching consequences, which unfold over decades, are difficult to anticipate, particularly in relation to labor demand \citep{bessen2018artificial, korinek2018artificial, acemoglu2020ai, benzell2021}. The realization of general purpose technologies' full potential requires extensive co-invention \citep{bresnahan1995general, bresnahan1996technical, bresnahan2002information,lipsey2005economic, dixon2021robot}, a costly and time-consuming process involving the discovery of new business procedures \citep{david1990dynamo,bresnahan1999computerisation, frey2019technology,brynjolfsson2021productivity,feigenbaum2021organizational}. Consequently, many studies of machine learning technologies focus on systems-level adoption, arguing that organizational systems may require redesign to effectively take advantage of novel machine learning advancements \citep{bresnahan2019artificial, agrawal2021ai, goldfarb2023could}. Appropriately designed systems can yield considerable business value and improve firm performance \citep{rock2019engineering, babina2021artificial, zolas2021advanced}, with AI tools facilitating the discovery process \citep{cockburn2018impact, cheng2022innovae}. By employing task-level information to assess whether LLMs fulfill GPT criteria, we seek to merge the two perspectives for understanding the technology-labor relationship.

We attempt to build on these diverse literature streams in several ways. Echoing \citep{felten2023will}, we focus our analysis on the impact of LLMs, rather than addressing machine learning or automation technologies more broadly. Additionally, we propose a novel method that employs LLMs, specifically GPT-4, to assess tasks for exposure and automation potential, thereby bolstering human scoring efforts. Subsequently, we aggregate our findings to occupations and industries, capturing the overall potential exposure in the contemporary U.S. labor market.

\begin{comment}
    \1 uniqueness here: 
    \2 focus on generative models as an especially important domain for software
    \2 identification of potential for heterogeneous impact
    \2 dataset for researchers across a few fronts
    \3 expert-validated scoring  
    \3 task-level 
    \3 automation/augmentation
    \3 characterization of differences between human and machine answers
    \4 where humans think machines can do it but machines don't
    \4 where machines think machines can do it but humans don't
    \3 early evidence of fast pace of adoption (partly due to UI...)
    \3 early evidence on the emergence (or lackthereof) of new skills and tasks
\end{comment}  



\section{Methods and Data Collection}
\label{sec:data}

\subsection{Data on Activities and Tasks Performed by Occupation in the US}
\label{subsec:task_data}

We use the O*NET 27.2 database \citep{onet272}, which contains information on 1,016 occupations, including their respective Detailed Work Activities (DWAs) and tasks. A DWA is a comprehensive action that is part of completing task, such as "Study scripts to determine project requirements." A task, on the other hand, is an occupation-specific unit of work that may be associated with none, one, or multiple DWAs. We offer a sample of tasks and DWAs in Table \ref{tab:onet}. The two datasets we use consist of:
\begin{itemize}
    \item 19,265 tasks, where each task featuring a "task description" and a corresponding occupation, and with most tasks associated with one or more DWAs
    \item 2,087 DWAs, where most DWAs are connected to one or more tasks, and tasks may be associated with one or more DWAs, though some tasks lack any associated DWAs
\end{itemize}

\input{tables/onet_tables}


% \input{onet_tables}

\subsection{Data on Wages, Employment, and Demographics}


% We use O*NET and the Bureau of Labor Statistics.

We obtain employment and wage data from the 2020 and 2021 Occupational Employment series provided by the Bureau of Labor Statistics. This dataset encompasses occupational titles, the number of workers in each occupation, and occupation-level employment projections for 2031, typical education required for entry in an occupation and on-the-job training required to attain competency in an occupation \citep{bls_employment_occupation}. We use the BLS-recommended crosswalk to O*NET \citep{bls_crosswalk} to link the O*NET task and DWA dataset and the BLS Labor Force Demographics \citep{bls_demographics}, which is derived from the Current Population Survey (CPS). Both of these data sources are collected by the U.S. government and primarily capture workers who are not self-employed, are documented, and are working in the so-called formal economy.

%CUT:The dataset represents 158 million workers. Since occupation names do not exactly match the ones in the O*NET dataset, we join the datasets using the {BLS-recommended crosswalk to O*NET}. We also derive demographic information about workers from the {BLS Labor Force Demographics} database, which draws from the Current Population Survey (CPS).

\subsection{Exposure}


We present our results based on an exposure rubric, in which we define \textbf{exposure} as a measure of whether access to a GPT or GPT-powered system would reduce the time required for a human to perform a specific DWA or complete a task by at least 50 percent. We provide a summary of our rubric below, while the complete rubric can be found in \ref{exposure_tax}. When we have labels for DWAs, we first aggregate at the task level before aggregating at the occupation level.



\vspace{20pt}

\tikzstyle{mybox} = [draw=black, fill=black!2, very thick,
    rectangle, rounded corners, inner sep=20pt, inner ysep=20pt]
\tikzstyle{fancytitle} =[fill=black, text=white, rounded corners]

\begin{tikzpicture}\small
\node [mybox] (box){%
    \begin{minipage}{0.8\textwidth}
  No exposure (E0) if:
    \begin{itemize}[itemsep=1pt,parsep=1pt,topsep=1pt] % Adjust spacing here
        \item there is no or minimal reduction in the time required to complete the activity or task while maintaining equivalent quality or
        \item using any combination of the capabilities described in accordance with the below criteria would decrease the quality of the activity/task output.
    \end{itemize}
     Direct exposure (E1) if:
    \begin{itemize}[itemsep=1pt,parsep=1pt,topsep=1pt] % Adjust spacing here
        \item using solely the theoretical LLM or GPT-4 described via ChatGPT or the OpenAI playground can decrease the time required to complete the DWA or task by at least half (50\%).
    \end{itemize}
     LLM+ Exposed (E2) if:
    \begin{itemize}[itemsep=1pt,parsep=1pt,topsep=1pt] % Adjust spacing here
        \item access to the LLM alone would not reduce the time required to complete the activity/task by at least half, but
        \item additional software could be developed on to the LLM that could reduce the time it takes to complete the specific activity/task with quality by at least half. Among these systems, we count access to image generation systems. \footnote{In practice, as can be seen in the full rubric in \ref{exposure_tax}, we categorize access to image capabilities separately (E3) to facilitate annotation, though we combine E2 and E3 for all analyses.}
\end{itemize}
    \end{minipage}
};
\node[fancytitle, right=10pt] at (box.north west) {Exposure overview};
\end{tikzpicture}

\vspace{20pt}

We set the exposure threshold at a potential 50 \% reduction in time required to complete a specific DWA or task while maintaining consistent quality. We anticipate that adoption will be highest and most immediate for applications that realize a considerable increase in productivity. Although this threshold is somewhat arbitrary, it was selected for ease of interpretation by annotators. \footnote{Moreover, regardless of the chosen threshold, we guessed that the real-world reduction in task time would likely be slightly or significantly lower than our estimates, leading us to opt for a relatively high threshold. In our own validation labeling, we found that this corresponded closely to whether GPT or GPT-powered applications could perform the core part of a task or nearly the entire task.}

\begin{table}[h!]
\centering
\begin{tabular}{@{}lllrr@{}}
\toprule
\textbf{Comparison} & \multicolumn{1}{c}{$\mathbf{\gamma}$}& \multicolumn{1}{c}{\textbf{Weighting}} & \multicolumn{1}{c}{\textbf{Agreement}} & \multicolumn{1}{c}{\textbf{Pearson's}} \\ \midrule

GPT-4, Rubric 1; Human & $\alpha$ & E1 & 80.8\% & 0.223 \\
                                & $\beta$ & E1 + .5*E2 & 65.6\% & 0.591 \\
                                & $\zeta$ & E1 + E2 & 82.1\% & 0.654 \\ \midrule
GPT-4, Rubric 2; Human & $\alpha$ & E1 & 81.8\% & 0.221 \\
                                & $\beta$ & E1 + .5*E2 & 65.6\% & 0.538 \\
                                & $\zeta$ & E1 + E2 & 79.5\% & 0.589 \\ \midrule
GPT-4, Rubric 1; GPT-4, Rubric 2 & $\alpha$ & E1 & 91.1\% & 0.611 \\
                                  & $\beta$ & E1 + .5*E2 & 76.0\% & 0.705 \\
                                  & $\zeta$ & E1 + E2 & 82.4\% & 0.680 \\ \bottomrule
\end{tabular}
\caption{Model and human comparison of agreement and Pearson's correlation scores. The agreement score is determined by looking at how often the two groups agree on the annotation (e.g. E0, E1 or E2). In the paper we use GPT-4, Rubric 1.}
\label{tab:comparison}
\end{table}


We then collected both human and GPT-4-generated annotations using the exposure rubric, which underlie the bulk of the analyses in this paper.
\begin{itemize}
    \item \textit{Human Ratings:} We obtained human annotations by applying the rubric to each O*NET Detailed Worker Activity (DWA) and a subset of all O*NET tasks and then aggregated those DWA and task scores\footnote{The authors annotated DWAs that clearly required a high degree of physicality or manual dexterity, and the contracted annotators labeled the remaining activities, along with a subset of tasks including those without associated DWAs and those for which there was no clear task-level annotation after aggregating the DWA annotations.} at the task and occupation levels. To ensure the quality of these annotations, the authors personally labeled a large sample of tasks and DWAs and enlisted experienced human annotators who have extensively reviewed GPT outputs as part of OpenAI's alignment work \citep{ouyang2022training}.
    \item \textit{GPT-4 Ratings:} We administered a similar rubric to an early version of GPT-4 \citep{gpt4} but on all task/occupation pairs rather than DWAs. We made slight modifications to the rubric (which was used as a "prompt" to the model in this case) to enhance agreement with a set of human labels. Full agreement rates are given in Table \ref{tab:comparison}.
\end{itemize}


We construct three primary measures for our dependent variable of interest: (i) \textbf{$\alpha$}, corresponding to E1 in the exposure rubric above, anticipated to represent the lower bound of the proportion of exposed tasks within an occupation,\footnote{Despite recent advances of multimodal GPT models \citep{gpt4}, vision capabilities were not included in the assessment of $\alpha$ exposure.} (ii) \textbf{$\beta$}, which is the sum of E1 and 0.5*E2, where the 0.5 weight on E2 is intended to account for exposure when deploying the technology via complementary tools and applications necessitates additional investment, and (iii) \textbf{$\zeta$}, the sum of E1 and E2, an upper bound of exposure that provides an assessment of maximal exposure to GPT and GPT-powered software. We summarize agreement between annotation groups and measures in Table \ref{tab:comparison}. For the remainder of the analysis, if not specified, the reader may assume that we refer to $\beta$ exposure -- meaning all tasks directly exposed via tools like ChatGPT or the OpenAI Playground are considered twice as exposed as tasks requiring some complementary innovation.


\subsection{Limitations of our methodology}
\subsubsection{Subjective human judgments}
A fundamental limitation of our approach lies in the subjectivity of the labeling. In our study, we employ annotators who are familiar with the GPT models' capabilities. However, this group is not occupationally diverse, potentially leading to biased judgments regarding GPTs' reliability and effectiveness in performing tasks within unfamiliar occupations. We acknowledge that obtaining high-quality labels for each task in an occupation requires workers engaged in those occupations or, at a minimum, possessing in-depth knowledge of the diverse tasks within those occupations. This represents an important area for future work in validating these results.


\subsubsection{Measuring GPTs with GPT-4}

Recent research indicates that GPT-4 serves as an effective discriminator, capable of applying intricate taxonomies and responding to changes in wording and emphasis. \citep{gpt4} The outcomes of GPT-4 task classification are sensitive to alterations in the rubric's wording, the prompt's order and composition, the presence or absence of specific examples in the rubric, the level of detail provided, and key term definitions. Iterating on the prompt, based on observed outcomes in a small validation set, can enhance the agreement between model outputs and the rubric's intent. Consequently, there are slight differences between the rubric presented to humans and the one used for GPT-4. This decision was made deliberately to guide the model towards reasonable labels without excessively influencing human annotators. As a result, we use multiple annotation sources, but none should be considered the definitive ground truth relative to the others. In the analysis, we will present results from human annotators as our primary results. Further improvement and innovation in crafting effective rubrics for LLM classification remain possible. Still, we observe a high degree of agreement between human ratings and GPT-4 ratings at the occupation level concerning overall exposure to GPT systems (see Table \ref{tab:comparison}, Figure \ref{two_agreement_figs}).
\begin{figure}
    \centering
    \begin{minipage}{0.45\textwidth}
        \centering
        \includegraphics[width=\textwidth]{figures/agreement_binscatter.png}
    \end{minipage}\hfill
    \begin{minipage}{0.5\textwidth}
        \centering
        \includegraphics[width=\textwidth]{figures/DVE_humanmean.png}
    \end{minipage}
        \caption{Human raters (x-axis) and GPT-4 ratings (y-axis) show a high degree of agreement about GPT exposure by occupation. Near the highest levels of exposure following the $\beta$ method of aggregating exposure scores to occupations, GPT-4 ratings tend to be lower than Human ratings. We present the raw scatter and the binscatter. Near the top end of exposure ratings, humans are on average more likely to rate an occupation as exposed.}
        \label{fig:humangpt_exp_regscatter}
    \label{two_agreement_figs}    
\end{figure}
% \label{sec:caveats}
% % \begin{outline}[enumerate]
% In addition to those highlighted above, we note:

% \begin{enumerate}
\subsubsection{Additional Weaknesses}

\begin{itemize}

%\textbf{Brittleness of human judgment.} Making accurate predictions about the future applications of language models remains a formidable challenge, even for domain experts \citep{gpt4}. This is due to various factors, including emergent capabilities, human perception biases, and shifts in technological development, which can impact the accuracy and reliability of predictions about LLMs' potential for impacting worker tasks.

\item \textbf{Validity of task-based framework.} It is unclear to what extent occupations can be entirely broken down into tasks, and whether this approach systemtically omits certain categories of skills or tasks that are tacitly required for competent performance of a job. Additionally, tasks can be composed of sub-tasks, some of which are more automatable than others. Some tasks may function as pre-cursor to other tasks, such that the completion of downstream tasks is dependent on precursor tasks. If indeed, the task-based breakdown is not a valid representation of how most work in an occupation is performed, our exposure analysis would largely be invalidated.


\item \textbf{Relative vs. absolute measures.} It is likely best to interpret these measures as relative measures, e.g. an occupation with an estimated 0.6 exposure should likely be interpreted as just much more exposed than one with 0.1 exposure.

\item \textbf{Lack of expertise and task interpretation.} Human annotators were mostly unaware of the specific occupations mapped to each DWA during the labeling process. This led to unclear logic for aggregating tasks and occupations, as well as some evident discrepancies in labels, demonstrated in Table \ref{tab:onet}. We experimented with various aggregation methods and discovered that even with a maximum-matching approach (taking the matching human<>model label if one existed), the agreement remained relatively consistent. Ultimately, we collected additional labels for task/occupation pairs where there was significant disagreement.

\item \textbf{Forward-looking and subject to change, with some early evidence.} Accurately predicting future LLM applications remains a significant challenge, even for experts \citep{gpt4}. Emergent capabilities, human perception biases, and technological development shifts can all affect the accuracy and reliability of predictions regarding LLMs' potential impact on worker tasks. Our projections are inherently forward-looking and based on current trends, evidence, and perceptions of technological possibilities. As a result, they may change as new advancements arise in the field. For example, some tasks that seem unlikely for LLMs to impact today might change with the introduction of new model capabilities. Conversely, tasks that appear exposed might face unforeseen challenges limiting language model applications. %Assessments of tasks that are today perceived as requiring human oversight or in-person communication would potentially be different in the future as model capabilities improve and user trust increases. 

\item \textbf{Sources of disagreement.} While we did not rigorously examine sources of disagreement, we found a few places where humans and the model tended to get "stuck" in their assessments:
\begin{itemize}
    \item Tasks or activities where while an LLM could theoretically help or accomplish the task, adopting it to do so would require multiple people to change their habits or expectations (e.g. meetings, negotiations)
    \item Tasks or activities where there is currently some regulation that requires human oversight or norm that suggests human judgment or empathy (e.g. making decisions, counseling), and
    \item Tasks or activities where there already exists a technology that can reasonably automate the task (e.g. making reservations). 
\end{itemize}
\end{itemize}


% \noindent \textbf{Arbitrary exposure threshold}. The decision to use a threshold of 50\% for determining task automation exposure was made for two reasons: (i) it is easy to interpret for most people and (ii) use in production systems might reduce actual time savings, so we wanted to set the threshold high enough.


\section{Results}
\label{sec:econimpact}

 
General-purpose technologies are relatively rare and characterized by their pervasiveness, improvement over time, and the development of significant co-invention and spillovers \citep{lipsey2005economic}. Our assessment of GPTs' (Generative Pre-trained Transformers) impact on the labor market is limited since it does not consider total factor productivity or capital input potential. In addition to their influence on labor, GPTs may also influence these dimensions.

At this stage, certain GPT criteria are easier to evaluate than others. For example, assessing the long-term impact of these models' capabilities and the growth of complementary applications and systems is more feasible in the long run. Our primary focus at this early stage is to test the hypothesis that GPT language models have a pervasive influence on the economy, similar to \citep{goldfarb2023could}'s analysis of machine learning diffusion through job postings to assess machine learning's GPT potential as an algorithmic category. Rather than using job postings or studying machine learning in general, examining the task evaluation approach with both human and GPT annotations may reveal whether GPT impacts are limited to a small set of similar tasks or occupations.

Our findings suggest that, based on their task-level capabilities, GPTs have the potential to significantly affect a diverse range of occupations within the U.S. economy, demonstrating a key attribute of general-purpose technologies. In the following sections, we discuss results across various roles and wage structures. Additional results on the relative exposure of industries within the U.S. economy can be found in Appendix \ref{subsec:aggregates}.

\subsection{Summary Statistics}

Summary statistics for these measures can be found in Table \ref{tab:summarystats}. Both human and GPT-4 annotations indicate that average occupation-level $\alpha$ values fall between 0.14 and 0.15, suggesting that, for the median occupation, approximately 15\% of tasks are directly exposed to GPTs. This figure increases to over 30\% for $\beta$ and surpasses 50\% for $\zeta$. Coincidentally, human and GPT-4 annotations also tag between 15\% and 14\% of total tasks in the dataset as being exposed to GPTs.

Based on the $\beta$ values, we estimate that 80\% of workers belong to an occupation with at least one task exposed to GPTs, while 19\% of workers are in an occupation where over half the tasks are labeled as exposed.

Although the potential for tasks to be affected is extensive, GPTs must be incorporated into broader systems to realize this potential fully. As is common with general-purpose technologies, such co-invention barriers may impede the rapid diffusion of GPTs into economic applications. Additionally, predicting the need for human oversight is challenging, especially for tasks where model capabilities equal or surpass human levels. While the requirement for human supervision may initially slow down the adoption and diffusion rate, users of GPTs and GPT-powered systems are likely to become increasingly acquainted with the technology over time, particularly in terms of understanding when and how to trust its outputs.



\input{tables/summary_stats}
\subsection{Wages and Employment}
\label{subsec:labor}

\begin{figure}
    \centering
    % \includegraphics[width=\textwidth]{threshold_workers.png}
     \includegraphics[width=\textwidth]{exposure_employees_newest.png}
    \caption{Exposure intensity across the economy, displayed on the left in terms of percent of affected occupations and on the right as percent of affected workers. The distribution of exposure is similar across occupations and across workers, suggesting that worker concentration in occupations is not highly correlated with occupational exposure to GPTs or GPT-powered software. We do however expect that it could be more highly correlated with investment in developing GPT-powered software for particular domains.}
\label{fig:exposure_final}
\end{figure}

In Figure \ref{fig:exposure_final}, we present the exposure intensity across the economy. The first plot displays exposure in terms of total workers, while the second plot shows exposure in terms of total occupations. Each point on the graph represents the estimated percentage of workers (and occupations) on the y-axis with an exposure level ($\alpha$, $\beta$, and $\zeta$) indicated on the x-axis. For example, human annotators determined that 2.4\% of workers are $\alpha_{50}$-exposed, 18.6\% are $\beta_{50}$-exposed, and 49.6\% are $\zeta_{50}$-exposed, where the threshold of 50\% comes from the x-axis and the percentage of workers comes from the y axis in the right plot of Figure 2. At any given point on the x-axis, the vertical distance between the $\alpha$ and the $\zeta$ represents the exposure potential attributable to tools and applications beyond direct exposure to GPTs. The distribution of exposure is similar for both workers and occupations, suggesting that worker concentration in occupations does not have a strong correlation with occupational exposure to GPT or GPT-powered software.

\begin{comment}
Binscatter results showing the average occupational binary value weighted by task importance against the log annual wage of the occupation are shown in Figure \ref{fig:codex_binscatter}. To a point, higher wage occupations tend to be more exposed to GPT. The red line denotes a third-order polynomial approximation.
\end{comment}
% \includegraphics[width=0.5\textwidth]{figures/codex_logsal_binscatter.png}
% \label{fig:codex_binscatter} \\

% In Figure \ref{fig:human_binscatter}, we show the human $\beta$ labels for LLM exposure averaged over all tasks in the occupation. Higher wage occupations tend to be more exposed, though the very highest wage occupations have slightly less exposure than the peak. 
%In this case, we code a completely unexposed task as a 0, a fully exposed task as a 1, and a task with impact potential subject to additional conditions or partial exposure as 0.5.  The overall trend is highly similar to the naive binary encodings for GPT or Codex-style model relevance. 

\begin{comment}
\begin{figure}
    \centering
\includegraphics[width=0.42\textwidth]{figures/meanhuman_logsal_binscatter.png}
    \caption{Human labels ($\beta$) for GPT-exposure averaged over all tasks in the occupation. Higher wage occupations tend to be more exposed, though the very highest wage occupations have slightly less exposure than the peak. }
\label{fig:human_binscatter}
\end{figure}

\begin{figure}
    \centering
\includegraphics[width=0.52\textwidth]{figures/dv_logsal_binscatter.png}
  \caption{GPT-4 evaluation of LLM exposure by occupation (and therefore, the model's estimate of its own potential). GPT-4 annotations estimate a slightly higher exposure for high wage occupations than human counterparts.}\label{fig:dve_binscatter}  
\end{figure}
\end{comment}
 
% Figure \ref{fig:dve_binscatter} shows the GPT-4 evaluation of LLM exposure by occupation (and therefore, the model's estimate of its own potential). 
Aggregated at the occupation level, human and GPT-4 annotations exhibit qualitative similarities and tend to correlate, as demonstrated in Figure \ref{binscatters}. Human annotations estimate marginally lower exposure for high-wage occupations compared to GPT-4 annotations. While there are numerous low-wage occupations with high exposure and high-wage occupations with low exposure, the overall trend in the binscatter plot reveals that higher wages are associated with increased exposure to GPT.


\input{tables/top_jobs_sample}

%\todo{create regression table and partial out automation from exposure}. 

\begin{comment}
\begin{figure}
    \centering
\includegraphics[width=0.52\textwidth]{figures/meanhuman_totemp_binscatter.png}
    \caption{Human ratings of overall exposure to LLMs aggregated to the occupation-level (y-axis) against log total employment (x-axis). Current employment levels appear to be orthogonal to potential LLM exposure.}
\label{fig:meanhuman_totemp}
\end{figure}
\begin{figure}
    \centering
\includegraphics[width=0.52\textwidth]{figures/dvE_totemp_binscatter.png}
    \caption{Human ratings of overall exposure to GPT-4 aggregated to the occupation-level (y-axis) against log total employment (x-axis).}
\label{fig:dve_totemp} 
\end{figure}
\end{comment}

The potential exposure to GPTs seems to have little correlation with current employment levels. In Figure \ref{binscatters}, both human and GPT-4 ratings of overall exposure are aggregated to the occupation-level (y-axis) and compared with the log of total employment (x-axis). Neither plot reveals significant differences in GPT exposure across varying employment levels.



\begin{figure}[htbp]\small
\centering
\begin{minipage}[t]{0.48\textwidth}
\centering
\includegraphics[width=\linewidth]{figures/meanhuman_totemp_binscatter.png}
\label{fig:meanhuman_totemp}
\end{minipage}
\hfill
\begin{minipage}[t]{0.48\textwidth}
\centering
\includegraphics[width=\linewidth]{figures/dvE_totemp_binscatter.png}
\label{fig:dve_totemp}
\end{minipage}
%     \caption{Caption}
%     \label{fig:my_label}
% \end{figure}
% \begin{figure}
\begin{minipage}[t]{0.48\textwidth}
\centering
\includegraphics[width=\linewidth]{figures/meanhuman_logsal_binscatter.png}
\label{fig:human_binscatter}
\end{minipage}
\hfill
\begin{minipage}[t]{0.48\textwidth}
\centering
\includegraphics[width=\linewidth]{figures/dv_logsal_binscatter.png}
\label{fig:dve_binscatter}
\end{minipage}
  \caption{The binscatter plots depict the exposure to language models (LLMs) in various occupations, as assessed by both human evaluators and GPT-4. These plots compare the exposure to GPT ($\beta$) at the occupation level against the log of total employment within an occupation and log of the median annual wage for occupations. While some discrepancies exist, both human and GPT-4 assessments indicate that higher wage occupations tend to be more exposed to LLMs. Additionally, numerous lower wage occupations demonstrate high exposure based on our rubrics. Core tasks receive twice the weight of supplemental tasks within occupations when calculating average exposure scores. Employment and wage data are sourced from the BLS-OES survey conducted in May 2021.}
  \label{binscatters}
\end{figure}


% In \ref{fig:exposure_thresh_e1e2}, we see that 
% \includegraphics[width=0.5\textwidth]{figures/exposure_2023_expected_human_emp.png}
% \1 distribution of impact by wage / skill level (people push back against high wage = high skill)

\subsection{Skill Importance} 

In this section, we investigate the relationship between the importance of a skill for an occupation (as annotated in the O*NET dataset) and our exposure measures. We first take the Basic Skills provided by O*NET (skill definitions can be found in Appendix \ref{sec:skills_definitions}) and normalize the measure of skill importance for each occupation to enhance interpretability. Then we perform a regression analysis on our exposure measures ($\alpha$, $\beta$, $\zeta$) to examine the strength of associations between skill importance and exposure.

Our findings indicate that the importance of \textbf{science} and \textbf{critical thinking} skills are strongly negatively associated with exposure, suggesting that occupations requiring these skills are less likely to be impacted by current language models. Conversely, \textbf{programming} and \textbf{writing} skills show a strong positive association with exposure, implying that occupations involving these skills are more susceptible to being influenced by language models (see Table \ref{tab:skills} for detailed results). 


% Of course, the skills interact. For instance, professions rated as having high Programming exposure will likely be more exposed, but there is likely a differential between those also rated as having high critical thinking and those having low critical thinking.

\begin{table}\tiny\centering
\def\sym#1{\ifmmode^{#1}\else\(^{#1}\)\fi}
\caption{OLS Regression Results of Exposure Measures on O*NET Skills}
\begin{tabular}{l | c | c | c }      
& $\alpha$ & $\beta$ & $\zeta$ \\
& (std err) &  (std err) & (std err)  \\
\midrule

% Constant & 0.082  & & -0.112 & & 0.300 \\
Constant & 0.082*** & -0.112*** & 0.300*** \\
 & (0.011) & (0.011) & (0.057)   \\
Active Listening & 0.128** & 0.214*** & 0.449*** \\
 & (0.047) & (0.043) & (0.027) \\
Mathematics & -0.127*** & 0.161*** & 0.787*** \\
 & (0.026)   & (0.021) & (0.049) \\
Reading Comprehension & 0.153*** & 0.470*** & -0.346*** \\
& (0.041) & (0.037) & (0.017) \\
Science & -0.114*** & -0.230*** & -0.346*** \\
& (0.014) & (0.012) & (0.017) \\
Speaking & -0.028 & 0.133*** & 0.294*** \\
& (0.039) & (0.033) &    (0.042)    \\
Writing & 0.368*** & 0.467*** & 0.566*** \\
& (0.042) &    (0.037) & (0.047) \\
Active Learning & -0.157*** & -0.065** & 0.028 \\
& (0.027) & (0.024) & (0.032) \\
Critical Thinking & -0.264*** & -0.196*** & -0.129** \\
& (0.036) & (0.033) & (0.042) \\
Learning Strategies & -0.072* & -0.209*** & -0.346*** \\
& (0.028) & (0.025) & (0.034) \\
Monitoring & -0.067** & -0.149*** & -0.232*** \\
& (0.023) & 0.020) & (0.026) \\
Programming & 0.637*** & 0.623*** & 0.609*** \\
& (0.030) & (0.022) & (0.024) \\
\end{tabular}
\label{tab:skills}
\end{table}



\subsection{Barriers to Entry}


Next, we examine barriers to entry to better understand if there is differentiation in exposure due to types of jobs. One such proxy is an O*NET occupation-level descriptor called the "Job Zone." \href{https://www.onetonline.org/help/online/zones}{A Job Zone} groups occupations that are similar in (a) the level of education needed to get a job in the occupation, (b) the amount of related experience required to do the work, and (c) the extent of on-the-job training needed to do the work. In the ONET database, there are 5 Job Zones, with Job Zone 1 requiring the least amount of preparation (3 months) and Job Zone 5 requiring the most extensive amount of preparation, 4 or more years. We observe that median income increases monotonically across Job Zones as the level of preparation needed also increases, with the median worker in Job Zone 1 earning $\$30,230$ and the median worker in Job Zone 5 earning $\$80,980$.

All of our measures ($\alpha$, $\beta$, and $\zeta$) show an identical pattern, that is, exposure increases from Job Zone 1 to Job Zone 4, and either remains similar or decreases at Job Zone 5. Similar to Figure \ref{fig:exposure_final} in \ref{fig:job_zone_plots}, we plot the percentage of workers at every threshold of exposure. We find that, on average, the percentage of workers in occupations with greater than 50\% $\beta$ exposure in Job Zones 1 through 5 have $\beta$ at 0.00\% (Job Zone 1), 6.11\% (Job Zone 2), 10.57\% (Job Zone 3), 34.5\% (Job Zone 4), and 26.45\% (Job Zone 5), respectively. 




% From the O*Net Website: \url{https://www.onetonline.org/help/online/zones#zone1}

\begin{table}[h!]
\centering
\scriptsize
% \begin{tabular}{>{\raggedright\arraybackslash}p{.5cm} p{2cm} >{\raggedright\arraybackslash}p{2.5cm} >{\raggedright\arraybackslash}p{3.2cm} l | l l | l l | l l}
\begin{tabular}{>{\raggedright\arraybackslash}p{.4cm} >{\raggedright\arraybackslash}p{1.9cm} >{\raggedright\arraybackslash}p{2.4cm} >{\raggedright\arraybackslash}p{3cm} | >{\raggedright\arraybackslash}p{.9cm} | >{\raggedright\arraybackslash}p{.9cm} | >{\raggedleft\arraybackslash}p{.39cm} >{\raggedleft\arraybackslash}p{.39cm} | >{\raggedleft\arraybackslash}p{.39cm}>{\raggedleft\arraybackslash}p{.39cm} | >{\raggedleft\arraybackslash}p{.39cm}>{\raggedleft\arraybackslash}p{.39cm}}
\toprule
\rowcolor{gray!20}
\textbf{Job Zone} & \textbf{Preparation Required} & \textbf{Education Required} & \textbf{Example Occupations} & \textbf{Median Income} & Tot Emp (000) & \textbf{H} $\pmb{\alpha}$ & \textbf{M} $\pmb{\alpha}$ & \textbf{H} $\pmb{\beta}$ & \textbf{M} $\pmb{\beta}$ & \textbf{H} \ $\pmb{\zeta}$\  & \textbf{M} $\pmb{\zeta}$ \\
\midrule
1 & None or little (0-3 months) & High school diploma or GED (optional) & Food preparation workers, dishwashers, floor sanders & \$30,230 & 13,100 & 3.71 & 3.84 & 6.45 & 5.97 & 9.19 & 8.11 \\
% \midrule 
\rowcolor{gray!10}
2 & Some (3-12 months) & High school diploma & Orderlies, customer service representatives, tellers & \$38,215 & 73,962 & 7.03 & 11.88 & 15.74 & 19.54 & 24.45 & 27.19 \\

3 & Medium (1-2 years) & Vocational school, on-the-job training, or associate's degree & Electricians, barbers, medical assistants & 54,815 & 37,881 & 11.28 & 13.72 & 26.08 & 32.17 & 40.88 & 50.62 \\
% \midrule
\rowcolor{gray!10}
4 & Considerable (2-4 years) & Bachelor's degree & Database administrators, graphic designers, cost estimators & \$77,345 &   56,833 & 22.68 & 17.82 & 46.78 & 51.30 & 70.87 & 84.78 \\
% \rowcolor{gray!10}
5 & Extensive (4+ years) & Master's degree or higher & Pharmacists, lawyers, astronomers & \$81,980 & 21,221 & 22.81 & 13.36 & 43.11 & 44.64 & 63.41 & 75.92 \\
\bottomrule
\end{tabular}
\caption{Exposure to GPTs by Job Zone}
\end{table}



\label{table:job_zone_table}

\begin{figure}
    \centering
    % \includegraphics[width=.95\textwidth]{job_zones2.png}
    \includegraphics[width=.95\textwidth]{exposure_2023_expected_human_jobzone_newest.png}
    \caption{$\beta$ exposure ratings of occupations in the five Job Zones, which are groups of similar occupations that are classified according to the level of education, experience, and on-the-job training needed to perform them.}
    \label{fig:job_zone_plots}
\end{figure}


\subsubsection{Typical Education Needed for Entry}

Since inclusion in a Job Zone accounts for both the education required—which itself is a proxy for skill acquisition—and the preparation required, we seek data to disentangle these variables. We use two variables from the Bureau of Labor Statistics' Occupational data: "Typical Education Needed for Entry" and "On-the-job Training Required to Attain Competency" in an occupation. By examining these factors, we aim to uncover trends with potential implications for the workforce. There are 3,504,000 workers for whom we lack data on education and on-the-job training requirements, and they are therefore excluded from the summary tables.

Our analysis suggests that individuals holding Bachelor's, Master's, and professional degrees are more exposed to GPTs and GPT-powered software than those without formal educational credentials (see Table \ref{tab:on_the_job_training}). Interestingly, we also find that individuals with some college education but no degree exhibit a high level of exposure to GPTs and GPT-powered software. Upon examining the table displaying barriers to entry, we observe that the jobs with the least exposure require the longest training, potentially offering a lower payoff (in terms of median income) once competency is achieved. Conversely, jobs with no on-the-job training required or only internship/residency required appear to yield higher income but are more exposed to GPT.



\include{tables/on_the_job_training}



\section{Validation of Measures}
\label{sec:validation}

\subsection{Comparison to Earlier Efforts}
\label{subsec:othermeasures}
This paper aims to build on a number of previous empirical studies examining the occupational exposure to advances in AI and/or automation. Previous studies have used a variety of methods, including:
\begin{itemize}
    \item Using occupational taxonomies like O*NET to characterize which occupations have routine vs. non-routine and manual vs. cognitive task content \citep{autor2003skill, acemoglu2011skills}.
    \item Mapping text descriptions of tasks to descriptions of technological advances in patents. \citep{NBERw29552, Webb2020}
    \item Linking capabilities of AI systems to occupational abilities and aggregating exposure estimates to the occupations where those abilities are required. \citep{SeamansRajFelten2018, felten2023will}
    \item Mapping the results of AI task benchmark evaluations (ImageNet, Robocup, etc.) to 59 worker tasks through a set of 14 cognitive abilities drawn from the cognitive science literature. \citep{Tolan2021}
    \item Expert labeling of automation potential for a set of O*NET occupations where experts had high confidence, combined with a probabilistic classifier to estimate automation potential for the remainder of O*NET occupations. \citep{FreyOsborne2017}
    \item Developing a rubric for evaluating the "suitability for machine learning" (SML) of activities that workers are completing in the economy \citep{brynjolfsson2017can,Brynjolfsson2018, brynjolfssonQuantifyingDistributionMachine2023}. 
\end{itemize}

We provide a set of summary statistics on many of these prior efforts in Table \ref{tab:autoscores_sumstats}. 




This paper's methodology primarily builds upon the SML approach by developing a rubric to evaluate the overlap between LLM capabilities and worker tasks as reported in the O*NET database. Table \ref{tab:autoscores_regression} presents the results of OLS regressions of our new LLM exposure measurements on occupation-level exposure measures from \citep{SeamansRajFelten2018} ("AI Occupational Exposure Score" in the table), \citep{FreyOsborne2017} (Frey \& Osborne Automation), scores from all three technologies in \citep{Webb2020}, normalized routine manual and cognitive scores from \citep{acemoglu2011skills}, and \citep{Brynjolfsson2018,brynjolfssonQuantifyingDistributionMachine2023} (SML). We also use annualized occupational salaries from the most recent BLS Occupational Employment Survey as a control. There are four separate output variables representing new scores in this paper that are predicted by earlier efforts.

GPT-4 Exposure Rating 1 corresponds to our overall exposure rubric as evaluated by GPT-4, where full exposure potential is coded as 1, no exposure potential is coded as 0, and partial exposure (E2 in our labeling scheme) is coded as 0.5. GPT-4 Exposure Rating 2 is scored similarly for overall exposure, but with a slightly different prompt. The results are very similar across the two prompts. GPT-4 Automation Rating applies our "T" rubric, coding no automation exposure from LLMs as 0, full automation exposure as 1, and levels 2, 3, and 4 as 0.25, 0.5, and 0.75, respectively. Finally, Human Exposure Rating represents the same rubric as in GPT-4 Exposure Rating 1 but is scored by humans, as discussed in an earlier section of the paper. These results correspond to the $\beta$ set of statistics presented above.


The results across each type of measurement are consistent. We find generally positive and statistically significant correlations between our LLM exposure measures and previous measurements targeting software and AI. Encouragingly, the SML exposure scores by occupation show significant and positive associations with the exposure scores we develop in this paper, demonstrating a level of cohesion between the two studies with similar approaches. The Webb software and AI patent-based measures, SML, and normalized (demeaned and divided by standard deviation) routine cognitive scores all exhibit positive associations with some of our measures.

Software, SML, and routine cognitive scores all show positive and statistically significant associations with LLM exposure scores at a 1\% level. Coefficients on AI scores from \citep{Webb2020} are also positive and statistically significant at a 5\% level, but our secondary prompt on overall exposure to LLMs in columns 3 and 4 does not exhibit a statistically significant relationship. For the most part, the AI Occupational Exposure Score is not correlated with our exposure measures. Webb's Robot exposure scores, routine manual task content, and the overall Automation metric from \citep{FreyOsborne2017} are all negatively correlated with our primary GPT-4 and human-assessed overall exposure ratings, conditional on the other measurements. This negative correlation reflects the limited exposure of physical tasks to LLMs. Manual work is not exposed to LLMs or even LLMs with additional systems integration for the time being. Our automation rubric results are also uncorrelated with \citep{FreyOsborne2017} measures.

Low correlations with \citep{SeamansRajFelten2018} and \citep{FreyOsborne2017} could potentially be explained by differences in approaches. Linking AI capabilities to worker abilities or scoring exposure directly based on the occupation's characteristics, rather than aggregating up to the occupation from DWA or task-level scoring (as in the SML paper and our own), offer a slightly different perspective on the content of occupations.

In all regressions, the $R^2$ ranges between 60.7\% (column 3) and 72.8\% (column 5). This suggests that our measure, which explicitly focuses on LLM capabilities, has between 28 and 40\% unexplained variance compared to other measurements. Particularly in the case of AI-related exposure scores, we anticipate that a combination of other measurements would have a strong correlation with our scores. However, earlier efforts had limited information about the future progress of LLM technologies. We expect that our understanding of future machine learning technologies is similarly imperfectly captured by our rubric today.

\begin{table}[]\small
\begin{threeparttable}
    \centering
    \resizebox{\textwidth}{!}{
    \input{tables/autoscores_summary}
    }
    \caption{Summary statistics for a suite of prior efforts to measure occupational exposure to AI and automation. We have also included summary statistics for measurements newly presented in this work. We include all measures from \citep{Webb2020}, normalized routine cognitive and manual scores from \citep{acemoglu2011skills} (means may deviate slightly from 0 due to imperfect matching of occupational groups), Suitability for Machine Learning from \citep{brynjolfsson2017can, Brynjolfsson2018, brynjolfssonQuantifyingDistributionMachine2023}, AI Occupational Exposure from \citep{SeamansRajFelten2018}, and Automation exposure from \citep{FreyOsborne2017}. We include as many occupations as we can match, but since O*NET taxonomies have changed as these measures have been developed, some of the roles may be missing from the most recent version of O*NET 6-digit occupations.}
    \label{tab:autoscores_sumstats}
\end{threeparttable}
\end{table}



\begin{table}[]\small
\begin{threeparttable}
    \centering
    \resizebox{\textwidth}{!}{
    \input{tables/autoscores_regression}
    }
    \caption{Regression of GPT-exposure cores on prior efforts. Regression coefficients from exposure measures from our rubrics on earlier efforts to quantify occupational exposure to AI and automation. We also include annualized wages from the BLS-OES survey in May 2021. Each measure is kept in its original scale, with the exception of routine cognitive and routine manual scores from \citep{acemoglu2011skills}. Those two scores are standardized to mean zero and variance 1. Generally we find strong positive associations with previous efforts, though large residual variance to still be explained by our new measures. Columns 1 and 2 are based on our main $\beta$ exposure measure from GPT-4 ratings. Columns 3 and 4 are based on a similar slightly different exposure rubric also rated by GPT-4 for robustness. Columns 5 and 6 reflect human ratings on the same rubric as columns 1 and 2.}
    \label{tab:autoscores_regression}
\end{threeparttable}
\end{table}



%\input{tables/reg_results_other_measures}


\begin{comment}
\begin{outline}[enumerate]
\1 SML
\1 Felten, Raj, and Seamans
\1 Webb
\1 Arntz et al
\1 Current slate of models vs. other efforts and why it's not automation
\1 SML validity comparison (new)
\end{outline}

\subsection{External Validity}
\label{subsec:externalval}
\begin{outline}[enumerate]
\1 @drock: job postings analysis?
\1 @drock: companies with products matched to tasks -- publicly available products
\end{outline}
\end{comment}

%\includegraphics{}




%\section{An Early Look at LLM Adoption}
%\label{sec:adoption}

% Cite adoption paper by Acemoglu et al 2022 \citep{acemoglu_2022_automation_survey} showing current low levels of AI adoption by firms but high worker exposure due to larger firms being more likely to adopt. Contrast with recent uptick in both individual use and interest in ChatGPT and explosion of new apps being built on top of models deployed through APIs. Quick synthesis of twitter and Hackernews scraping for ChatGPT use. 

% Recent studies show that language models like ChatGPT and GitHub Copilot can significantly enhance work efficiency and quality. On a narrow task performed by Upwork workers, ChatGPT reduced task completion time by 0.8 SDs and increased output quality by 0.4 SDs, particularly benefiting low-ability workers and reducing productivity inequality. The model also led to task restructuring, with workers focusing on idea generation and editing, and relying on the model for drafting. This resulted in higher job satisfaction and self-efficacy. \citep{Noy2023} Similarly, GitHub Copilot users reported increased job fulfillment and reduced frustration. The tool allowed developers to focus on satisfying tasks while offloading repetitive work, helping maintain flow state and conserving mental energy. \citep{sida_copilot}

% These results align with forthcoming findings on the impact of integrating machine learning into the workflow of customer service agents, wherein the productivity gap between experienced an inexperienced workers shrunk, and the work became more enjoyable. (cite Lindsey Raymond paper)

% Taken together we see early signs...

\begin{comment}
\begin{outline}[enumerate]
\1 summary stats on growth of language model stuff
\1 difference-in-difference on huggingface github stats following chatGPT launch @drock
\1 news/tweet 

Anecdotally, we see reports that people are using the models in their jobs and finding them useful. 
\2 https://www.cnn.com/2023/01/28/tech/chatgpt-real-estate/index.html -- 41-9022.00
\2 

Additionally, early studies show that these models are useful.

Recent studies show that language models like ChatGPT and GitHub Copilot can significantly enhance work efficiency and quality. On a narrow task performed by Upwork workers, ChatGPT reduced task completion time by 0.8 SDs and increased output quality by 0.4 SDs, particularly benefiting low-ability workers and reducing productivity inequality. The model also led to task restructuring, with workers focusing on idea generation and editing, and relying on the model for drafting. This resulted in higher job satisfaction and self-efficacy.

These results align with earlier findings on the impact of integrating GPT-3 into the workflow of customer service agents, wherein the productivity gap between experienced an inexperienced workers shrunk, and the work became more enjoyable. (cite Lindsey Raymond paper)

Similarly, GitHub Copilot users reported increased job fulfillment and reduced frustration. The tool allowed developers to focus on satisfying tasks while offloading repetitive work, helping maintain flow state and conserving mental energy.



\1 funding
\1 publicly reported stuff on number of users of xyz
\1 other ideas in this area?
\end{outline}
\end{comment}
%%%% TYNA STOP POINT.
\section{Discussion}
\label{sec:discussion}


\subsection{GPTs as a General-Purpose Technology}

Earlier in this paper we discuss the possibility that GPTs could be classified as a general-purpose technology. This classification requires GPTs to meet three core criteria: improvement over time, pervasiveness throughout the economy, and the ability to spawn complementary innovations \citep{lipsey2005economic}. Evidence from the AI and machine learning literature thoroughly demonstrates that GPTs meet the first criteria -- they are improving in capabilities over time with the ability to complete or be helpful for an increasingly complex set of tasks and use-cases (see \ref{sec:llmlit}). This paper presents evidence to support the latter two criteria, finding that GPTs on their own can have pervasive impacts across the economy, and that complementary innovations enabled by GPTs -- particularly via software and digital tools -- can have widespread application to economic activity.

%Of course, there are some surprising results when we look at the most exposed occupations (see Table \ref{table:top_jobs}). Both human annotators and GPT-4 index Mathematicians as exposed to GPTs, with all of the gain coming from access to GPT-powered software on top of the model.

Figure \ref{fig:exposure_final} offers one illustration of the potential economic impact of complementary software built on top of LLMs. Taking the difference in the y-axis (the share of all occupations) between $\alpha$ and $\zeta$ at a given point along the x-axis (the share of tasks within an occupation that are exposed) gives the aggregate within-occupation exposure potential attributable to tools and software over and above direct exposure from LLMs on their own. The difference in means across all tasks between $\alpha$ and $\zeta$ of 0.42 using the GPT-4 annotations and 0.32 using the human annotations (see Figure \ref{tab:summarystats}), suggests that the average impact of GPT-powered software on task-exposure may be more than twice as large as the mean exposure from LLMs on their own (mean $\zeta$ of 0.14 based on both human annotations and GPT-4 annotations). While our findings suggest that out-of-the-box these models are relevant to a meaningful share of workers and tasks, they also suggest that the software innovations they spawn could drive a much broader impact.

One component of the pervasiveness of a technology is its level of adoption by businesses and users. This paper does not systematically analyze adoption of these models, however, there is early qualitative evidence that adoption and use of LLMs is becoming increasingly widespread. The power of relatively simple UI improvements on top of LLMs was evident in the rollout of ChatGPT -- wherein versions of the underlying model had been previously available via API, but usage skyrocketed after the release of the ChatGPT interface. \citep{chow_chatgpt_2023, chatgptblog} Following this release, a number of commercial surveys indicate that firm and worker adoption of LLMs has increased over the past several months. \citep{constantz_bloomberg, resumebuildersuvey}

Widespread adoption of these models, however, necessitates the identification of existing bottlenecks. A key determinant of their utility is the level of confidence humans place in them, as well as habits. For instance, in the legal profession, the models' usefulness hinges upon whether legal professionals can trust their output without resorting to verifying original documents or conducting independent research. The cost and flexibility of the technology, worker and firm preferences, and incentives also play a significant role in the adoption of tools built on top of LLMs. In this way, adoption may be driven by progress on some of the ethical and safety risks associated with LLMs: bias, making up facts, and misalignment to name a few \cite{4systemcard}.

%Cite adoption paper by Acemoglu et al 2022 \citep{acemoglu_2022_automation_survey} showing current low levels of AI adoption by firms but high worker exposure due to larger firms being more likely to adopt. Contrast with recent uptick in both individual use and interest in ChatGPT and explosion of new apps being built on top of models deployed through APIs. Quick synthesis of twitter and Hackernews scraping for ChatGPT use. 

% Recent studies show that language models like ChatGPT and GitHub Copilot can significantly enhance work efficiency and quality. On a narrow task performed by Upwork workers, ChatGPT reduced task completion time by 0.8 SDs and increased output quality by 0.4 SDs, particularly benefiting low-ability workers and reducing productivity inequality. The model also led to task restructuring, with workers focusing on idea generation and editing, and relying on the model for drafting. This resulted in higher job satisfaction and self-efficacy. \citep{Noy2023} Similarly, GitHub Copilot users reported increased job fulfillment and reduced frustration. The tool allowed developers to focus on satisfying tasks while offloading repetitive work, helping maintain flow state and conserving mental energy. \citep{sida_copilot}

% These results align with forthcoming findings on the impact of integrating machine learning into the workflow of customer service agents, wherein the productivity gap between experienced an inexperienced workers shrunk, and the work became more enjoyable. 

Moreover, the adoption of LLMs will vary across different economic sectors due to factors such as data availability, regulatory quality, innovation culture, and the distribution of power and interests. Consequently, a comprehensive understanding of the adoption and of large language models by workers and firms requires a more in-depth exploration of these intricacies.

One possibility is that time savings and seamless application will hold greater importance than quality improvement for the majority of tasks. Another is that the initial focus will  be on augmentation, followed by automation \citep{huang2018artificial}. One way this might take shape is that an augmentation phase where jobs first become more precarious (writers become freelancers) could play out prior to full automation.


% One component impacting different generation technologies diffusing at different rates -- one factor affecting this is ...

%Future work to understand the differential impacts of various software integrations as well as the adoption process for both software developers and their users would be invaluable. 



\subsection{Implications for US Public Policy}

The introduction of automation technologies, including LLMs, has previously been linked to heightened economic disparity and labor disruption, which may give rise to adverse downstream effects.\citep{acemoglu2022demographics, Acemoglu2002, Moll2021, Klinova2021, Weidinger2021, Weidinger2022} Our results examining worker exposure in the United States underscore the need for societal and policy preparedness to the potential economic disruption posed by LLMs and the complementary technologies that they spawn. While it is outside the scope of this paper to recommend specific policy prescriptions to smooth the transition to an economy with increasingly widespread LLM adoption, prior work such as \citep{Autor2022} has articulated several important directions for US policy related to education, worker training, reforms to safety net programs, and more. 

\begin{comment}
    
\textbf{Discussion of relationship between our results here and other outcomes like job satisfaction, overreliance/dependency, impacts on skill development.}

In light of our findings that subsequent generations of LLMs possess the capacity to significantly transform and modify the employment landscape in the US, it becomes imperative to contextualize these results within the broader milieu of workforce dynamics and human development. In this discussion, we elucidate three key areas that warrant attention in understanding the implications of LLMs in these domains:

1. Job Satisfaction: The proliferation of LLMs in various occupational sectors may precipitate shifts in job roles and responsibilities, altering the nature of work and potentially impacting employee job satisfaction. As tasks traditionally performed by humans are increasingly automated, workers may experience a reconfiguration of their roles that could engender a sense of diminished agency and fulfillment. Conversely, the reduction of repetitive and menial tasks may also present opportunities for employees to engage in more meaningful work. We recommend further empirical inquiry into the nuanced relationships between LLM integration and job satisfaction..

2. Overreliance and skill development: The increased integration of LLMs into various occupational sectors raises concerns about overreliance and dependency on these technologies.\citep{passi2022overreliance} Studies have found that overreliance on AI systems, characterized by acceptance of incorrect recommendations and modification of responses to align with AI suggestions, can contribute to diminished human autonomy and critical thinking. Furthermore, dependency on LLMs for decision-making and task completion may result in the underdevelopment of critical human skills, such as problem-solving, analysis, and creativity. This can lead to increased vulnerability in the face of technological failures or security breaches. As the demand for certain skills diminishes in light of LLM-driven automation, there may be a concomitant need to reorient skill development initiatives to equip the workforce with competencies that are complementary to LLMs. This includes skills such as AI literacy and the ability to check sources. Our research underscores that reskilling to jobs that are available today may not be sufficient to make workers robust to changes in the labor market given the rapid development of AI technologies.

\textbf{Why you should care about economic impacts}

\textbf{Case for this being different, research on how much automation jobs/industries can absorb or withstand}
\end{comment}

\subsection{Limitations and Future Work}

This study possesses several limitations that warrant further investigation. Primarily, our focus on the United States restricts the generalizability of our findings to other nations where the adoption and impact of generative models may differ due to factors such as industrial organization, technological infrastructure, regulatory frameworks, linguistic diversity, and cultural contexts. We hope to address this limitation by extending the study's scope and by sharing our methods so other researchers can build on them.

Subsequent research efforts should consider two additional studies: one exploring GPT adoption patterns across various sectors and occupations, and another scrutinizing the actual capabilities and limitations of state-of-the-art models in relation to worker activities beyond the scope of our exposure scores. For example, despite recent advances in multimodal capabilities with GPT-4, we did not consider vision capabilities in the $\alpha$ ratings on direct GPT-exposure. \citep{gpt4} Future work should consider the impact of such capability advances as they unfold. We acknowledge that there may be discrepancies between theoretical and practical performance, particularly in complex, open-ended, and domain-specific tasks. %Furthermore, our scores do not consider ease of use of LLM-enabled tools, which may be critical to the adoption of generative models.

%The emergence of experimental studies examining the potential benefits and challenges of incorporating generative models into work processes is a noteworthy development. However, these studies also highlight the limitations of task-centric approaches to job design, suggesting the need to consider broader work process integration when assessing generative models' potential.

%In conclusion, although the promise of LLMs is undeniable, considerable work remains in discerning how they can be effectively integrated into work processes and identifying the factors influencing their adoption. By delving deeper into these issues, we can develop a more sophisticated understanding of the potential advantages and challenges posed by language models for individual workers and society at large.

%\# FEEL FREE TO MOVE AROUND
%This study examines various measures that we define as "exposure", "automation", and "augmentation." We also set specific definitions that are ours and may not be moored in how development and adoption will play out in the real world. For instance, one task being fully automated might simply mean the task gets converted into a tool that other workers use as input to augment their productivity. Conversely, sufficient augmentation might make the impacted jobs more precarious, as a greater share of their tasks become automated and may set the impacted workers on the path of automation. 



% At the moment the evidence suggests that language models will soon be pervasive throughout the economy, improving over time, and spawning complementary innovations. These criteria justify classification of language models as general-purpose technologies. Many occupations have some exposure to LLMs and that exposure is greater when considering the potential complementarities between LLMs and other types of software and technology. Predicting the course of performance improvement within language model domains as well as how they might augment other forms of economic innovation will be challenging. We recommend ongoing efforts to track these technologies and their impacts.



\section{Conclusion}
\label{sec:conclusion}

In conclusion, this study offers an examination of the potential impact of LLMs, specifically GPTs, on various occupations and industries within the U.S. economy. By applying a new rubric for understanding LLM capabilities and their potential effects on jobs, we have observed that most occupations exhibit some degree of exposure to GPTs, with higher-wage occupations generally presenting more tasks with high exposure. Our analysis indicates that approximately 19 \% of jobs have at least 50\% of their tasks exposed to GPTs when considering both current model capabilities and anticipated GPT-powered software. 

Our research aims to highlight the general-purpose potential of GPTs and their possible implications for US workers. Previous literature demonstrates the impressive improvements of GPTs to date (see \ref{sec:llmlit}). Our findings confirm the hypothesis that these technologies can have pervasive impacts across a wide swath of occupations in the US, and that additional advancements supported by GPTs, mainly through software and digital tools, can have significant effects on a range of economic activities. However, while the technical capacity for GPTs to make human labor more efficient appears evident, it is important to recognize that social, economic, regulatory, and other factors may influence actual labor productivity outcomes. As capabilities continue to evolve, the impact of GPTs on the economy will likely persist and increase, posing challenges for policymakers in predicting and regulating their trajectory.

Further research is necessary to explore the broader implications of GPT advancements, including their potential to augment or displace human labor, their impact on job quality, impacts on inequality, skill development, and numerous other outcomes. By seeking to understand the capabilities and potential effects of GPTs on the workforce, policymakers and stakeholders can make more informed decisions to navigate the complex landscape of AI and its role in shaping the future of work.

\subsection{GPT Conclusion (GPT-4's Version)}

Generative Pre-trained Transformers (GPTs) generate profound transformations, garnering potential technological growth, permeating tasks, greatly impacting professions. This study probes GPTs’ potential trajectories, presenting a groundbreaking rubric to gauge tasks’ GPT exposure, particularly in the U.S. labor market.

\subsection{GPT Conclusion (Author-Augmented Version)}

Generative Pre-trained Transformers (GPTs) generate profound transformations, garnering potential technological growth, permeating tasks, gutting professional management. Gauging possible trajectories? Generate pioneering taxonomies, gather policymakers together, generalize past today.

\begin{comment}
    
Conclusion: We want to emphasize diversified job design. No job fully lines up to the canonical setup.  Industrial organization. Job creators should think about what jobs should look like. Sometimes we design jobs to act as machines. Let's think about what constitutes a job or a function that is suited to a person.

The "right" kind of AI? General-purpose AI or generative AI models aren't the level where this choice is made: applications are.

While we believe our findings can help guide impact assessments in the short to medium run, and should be enough to motivate more attention to the pace of development and the need for attention from policymakers, we also believe there are likely many additional findings in the datasets we collected and methods we have considered for future work.
\end{comment}

\begin{comment}
The documentation for \verb+natbib+ may be found at
\begin{center}
  \url{http://mirrors.ctan.org/macros/latex/contrib/natbib/natnotes.pdf}
\end{center}
Of note is the command \verb+\citet+, which produces citations
appropriate for use in inline text.  For example,
\begin{verbatim}
   \citet{hasselmo} investigated\dots
\end{verbatim}
produces
\begin{quote}
  Hasselmo, et al.\ (1995) investigated\dots
\end{quote}

\begin{center}
  \url{https://www.ctan.org/pkg/booktabs}
\end{center}






\begin{figure}
  \centering
  \fbox{\rule[-.5cm]{4cm}{4cm} \rule[-.5cm]{4cm}{0cm}}
  \caption{Sample figure caption.}
  \label{fig:fig1}
\end{figure}

\subsection{Tables}
\lipsum[12]
See awesome Table~\ref{tab:table}.

\begin{table}
 \caption{Sample table title}
  \centering
  \begin{tabular}{lll}
    \toprule
    \multicolumn{2}{c}{Part}                   \\
    \cmidrule(r){1-2}
    Name     & Description     & Size ($\mu$m) \\
    \midrule
    Dendrite & Input terminal  & $\sim$100     \\
    Axon     & Output terminal & $\sim$10      \\
    Soma     & Cell body       & up to $10^6$  \\
    \bottomrule
  \end{tabular}
  \label{tab:table}
\end{table}
\end{comment}

\section*{Acknowledgments}
Thank you to the group of annotators who helped us annotate task exposure, including Muhammad Ahmed Saeed, Bongane Zitha, Merve Özen Şenen, J.J., and Peter Hoeschele. We also thank Lauryn Fuld, Ashley Glat, Michael Lampe, and Julia Susser for excellent research assistance. We thank Miles Brundage for significant feedback on this paper.

We thank Todor Markov and Vik Goel for setting up the infrastructure to run our taxonomies against GPT-4. We thank Lama Ahmad, Donald Bakong, Seth Benzell, Erik Brynjolfsson, Parfait Eloundou-Enyegue, Carl Frey, Sarah Giroux, Gillian Hadfield, Johannes Heidecke, Alan Hickey,  Eric Horvitz, Shengli Hu, Ashyana Kachra,  Christina Kim, Katya Klinova, Daniel Kokotajlo, Gretchen Krueger, Michael Lampe, Aalok Mehta, Larissa Schiavo, Daniel Selsam, Sarah Shoker, Prasanna Tambe, and Jeff Wu for feedback and edits at various stages of the project.

\section*{LLM assistance statement}
GPT-4 and ChatGPT were used for writing, coding, and formatting assistance in this project.

\appendix

\section{Taxonomies}\label{taxonomies}

\subsection{Exposure} \label{exposure_tax} 
\input{exposure_taxonomy}

% \subsection{Automation}\label{automation_tax}
% \# T Automation Rubric

1. Determine if the occupation/task pair meets the definition of T0 No-Automation Exposure. If it does, label it as T0 and stop.
2. If the occupation/task pair does not meet the definition of T0 No-Automation Exposure, determine if the occupation/task pair meets one of the other definitions above and select the label that applies:

- T4: Full automation exposure
- T3: High automation exposure
- T2: Moderate automation exposure
- T1: No automation exposure

\#\# Rubric

LLM++ is a software tool that can perform a wide range of digital tasks but cannot program physical machines or objects.

Consider the most powerful OpenAI large language model (LLM). This model can complete many tasks that can be formulated as having text, image, or audio input and text, image, or audio output. The model also cannot draw up-to-date facts (those from <1 year ago) unless they are captured in the input. This system cannot accurately retrieve very detailed information from image inputs, such as measurements of dimensions within an image.

On its own, the LLM can perform tasks that can be reduced to:
- Writing and transforming text and code according to complex instructions,
- Providing edits to existing text or code following specifications,
- Writing code that can help perform simple, repetitive digital tasks,
- Debugging code or software
- Programming in computer languages like Python, C++, Java, Stata
- Assisting with a data analysis
- Translating text between languages,
- Summarizing short documents,
- Providing feedback on documents,
- Answering questions about a document,
- Generating questions a user might want to ask about a document,
- Writing questions for an interview or assessment,
- Writing and responding to emails, including ones that involve refuting information or engaging in a negotiation (but only if the negotiation is via written correspondence),
- Maintain records of written data,
- Prepare training materials based on general knowledge,
- Searching for a relevant document over a small number of documents,
- Reading text from PDFs,
- Scanning images,
- Creating or editing digital images according to instructions,
- Captioning an image,
- Transcribing text, or
- Informing anyone of any information via any written or spoken medium.

Assume you are a worker with an average level of expertise in your role trying to complete the given task. You have access to the LLM as well as any other existing software or computer hardware tools mentioned in the task. You also have access to any commonly available technical tools accessible via a laptop (e.g. a microphone, speakers, etc.). You do not have access to any other physical tools or materials.

You also have access to LLM++ which is additional software developed on top of the LLM. LLM++ includes capabilities such as:
- Summarizing documents longer than 2000 words and answering questions about those documents,
- Retrieving up-to-date facts from the Internet and using those facts in combination with the LLM capabilities,
- Searching over an organization’s existing knowledge, data, or documents and retrieving information,
- Retrieving highly specialized domain knowledge,
- Making recommendations given data or written input,
- Analyzing written information to inform decisions,
- Preparing training materials based on highly specialized knowledge,
- Performing legal research,
- Providing counsel on issues,
- Writing lesson plans, and
- Maintaining databases.

We refer to this suite of software tools as LLM++. LLM++ can NOT program physical machines or objects.

Please label the given task according to the rubric below.

**T0 No-Automation Exposure** A class of tasks for which LLM++ can not conceivably perform any aspect of the task in any manner. 

**T4 Full Automation Exposure** A class of tasks where, in most contexts in which this task is currently performed by a human, LLM++ can complete all aspects of this task with high quality when prompted by a human. The output does not normally require oversight by a human. Oversight is not normally required for tasks labeled T4 because the consequences for failure or inaccuracy are small for this task, human judgment is not necessary to complete this task, and generative models can consistently perform this task with very high quality.

**T3 High Automation Exposure** A class of tasks where, in most contexts in which this task is currently performed by a human, LLM++ could complete 90-100% of the components of the task when prompted, but the output requires oversight from a human. Oversight is normally required because the consequences for failure or inaccuracy are significant for this task, human judgment is necessary to complete this task, and/or generative models cannot perform all aspects of this task with high quality consistently. These tasks rely almost exclusively on the processing of digital information, but human judgment is needed to ensure that any digital outputs from LLM++ are high enough quality to be acceptable for the particular context.

** T2 Moderate Automation Exposure** A class of tasks where, in most contexts in which this task is performed by a human, LLM++ could complete between 50%-90% of the components of the task at high quality. These tasks normally rely heavily on the processing of digital information, but a significant portion of the task also involves actions that LLM++ cannot perform with high quality. These tasks require at least some human action beyond just double-checking generative model outputs (such as interpretation, judgment, human-to-human communication, or physical actions).

** T1 Low Automation Exposure** A class of tasks where, in most contexts in which this task is performed by a human, LLM++ could complete between 0%-50% of the components of the task at high quality. These tasks normally rely only partially on the processing of digital information, while the majority of the task involves actions that LLM++ cannot perform with high quality. A majority of the actions that need to be taken to complete this task require a human to perform the action.

\#\# Definitions

**High quality** means someone receiving or reviewing the output would not be able to tell the difference between whether it came from LLM++ or a human. For tasks that require a lot of interaction during the completion of the task (e.g. meetings, negotiations), high quality means the people you were interacting with either would not know or would not care that they were interacting with with LLM++.

**Digital information** or **information that can easily be expressed digitally** includes but is not limited to text, audio, images, PDFs, books, and code.

\#\# Annotation Examples

Occupation: Special Education Teachers, Preschool
Task: Develop individual educational plans (IEPs) designed to promote students' educational, physical, or social development.
Automation score (T0/T1/T2/T3/T4): T2
Explanation: While LLM++ can assist in the creation of educational plans, this task requires significant interpretation and judgment by a human in order to be effective. Special Education Teachers must consider the individual needs and abilities of each student, as well as communicate with parents and other educators to ensure the plan is appropriate and effective. While LLM++ can provide suggestions and potential plans, it cannot make the final decision or communicate with stakeholders.

Occupation: Construction and Building Inspectors
Task: Inspect and monitor construction sites to ensure adherence to safety standards, building codes, or specifications.
Automation score (T0/T1/T2/T3/T4): T1
Explanation: While LLM++ can provide some support in the inspection and monitoring of construction sites, this task involves a significant amount of physical inspection and communication with workers and supervisors on-site. Additionally, the inspector may need to make on-the-spot judgments based on what they observe and may need to refer to local building codes and regulations which may not be available in digital format. As such, this task requires a significant amount of human input and oversight.

Occupation: Food and Tobacco Roasting, Baking, and Drying Machine Operators and Tenders
Task: Dump sugar dust from collectors into melting tanks and add water to reclaim sugar lost during processing.
Automation score (T0/T1/T2/T3/T4): T0
Explanation: This task requires physical actions that cannot be completed by LLM++.

Occupation: Historians
Task: Translate or request translation of reference materials.
Automation score (T0/T1/T2/T3/T4): T4
Explanation: LLM++ can easily complete this task by translating text from one language to another. Human intervention is not usually necessary.

Occupation: Electronics Engineer
Task: Prepare documentation containing information such as confidential descriptions or specifications of proprietary hardware or software, product development or introduction schedules, product costs, or information about product performance weaknesses.
Automation score (T0/T1/T2/T3/T4): T2
Explanation: LLM++ can assist with documentation but human judgment is needed for understanding specifications, costs, and performance weaknesses. Thus, this task falls under T2 Moderate Automation Exposure.

Occupation: Advanced Practice Psychiatric Nurse
Task: Write prescriptions for psychotropic medications as allowed by state regulations and collaborative practice agreements.
Automation score (T0/T1/T2/T3/T4): T1
Explanation: LLM++ could be used to help create prescriptions given digital input about the patient’s context, but a human would need to double-check the information to ensure that it meets regulations and is appropriate for the patient.

\#\# Now, apply the above rubric to the example below




\section{ONET Basic Skills Definitions}
\label{sec:skills_definitions}

\subsection*{Basic Skills}

Developed capacities that facilitate learning or the more rapid acquisition of knowledge.

\subsection*{Content}

Background structures needed to work with and acquire more specific skills in a variety of different domains.

\begin{itemize}
    \item \textbf{Reading Comprehension} — Understanding written sentences and paragraphs in work-related documents.
    \item \textbf{Active Listening} — Giving full attention to what other people are saying, taking time to understand the points being made, asking questions as appropriate, and not interrupting at inappropriate times.
    \item \textbf{Writing} — Communicating effectively in writing as appropriate for the needs of the audience.
    \item \textbf{Speaking} — Talking to others to convey information effectively.
    \item \textbf{Mathematics} — Using mathematics to solve problems.
    \item \textbf{Science} — Using scientific rules and methods to solve problems.
\end{itemize}

\subsection*{Process}

Procedures that contribute to the more rapid acquisition of knowledge and skill across a variety of domains

\begin{itemize}
    \item \textbf{Critical Thinking} — Using logic and reasoning to identify the strengths and weaknesses of alternative solutions, conclusions or approaches to problems.
    \item \textbf{Active Learning} — Understanding the implications of new information for both current and future problem-solving and decision-making.
    \item \textbf{Learning Strategies} — Selecting and using training/instructional methods and procedures appropriate for the situation when learning or teaching new things.
    \item \textbf{Monitoring} — Monitoring/Assessing performance of yourself, other individuals, or organizations to make improvements or take corrective action.
\end{itemize}

\subsection*{Cross-Functional Skills}
Note: We selected only Programming from the list of cross-functional skills because of our prior knowledge about the models' ability to code.
\begin{itemize}
    \item \textbf{Programming} - Writing computer programs for various purposes.
\end{itemize}


\section{Education}
\label{subsec:educ_appendix}

\begin{table}[h]
    \centering
    \input{tables/education_table}
    \caption{Mean exposure scores for occupations, grouped by typical education needed for entry into the occupation. Alongside exposure scores, we display the median of median annual income for each occupation, as well as the total number of workers in each group, in thousands.}
    \label{tab:education}
\end{table}

\section{Industrial and Productivity Exposure}
\label{subsec:aggregates}

Which regions are most exposed (map) to automation and augmentation

\begin{figure}[p]
\centering
\includegraphics[width=\textheight, angle=270]{figures/industryexp.png}
\caption{}
\label{fig:meanhuman_indexp}
\end{figure}

\begin{figure}[p]
\centering
\includegraphics[width=\textheight, angle=270]{figures/industryexp_dve.png}
\caption{}
\label{fig:dve_indexp}
\end{figure}

Figures \ref{fig:meanhuman_indexp} and \ref{fig:dve_indexp} show the overall employment-weighted relative exposure of 3-digit NAICS industries according to human raters and our algorithmic exposure rubric respectively. The impact potential is present across nearly all industries, with wide heterogeneity. Both methods agree generally on relative exposures: data processing, information processing, and hospitals all have high exposure.

%Table XX (PUT A TABLE SHOWING RELATIVE EXPOSURES) describes the relative exposures according to different evaluation regimes. 



\includegraphics[width=0.5\textwidth]{figures/tfpemp_dv_facetscatter.png}
\label{fig:tfp_dv}
\includegraphics[width=0.5\textwidth]{figures/laborprodemp_dv_facetscatter.png}
\label{fig:lp_dv}
Recent productivity growth (both total factor and labor) appears uncorrelated with exposure as well. Figures \ref{fig:tfp_dv} and \ref{fig:lp_dv} show little relationship between productivity growth since 2012 and current exposure to LLMs as rated by the model. A high correlation between already fast-growing productive industries and exposure might mean an exacerbation of Baumol's cost disease. In other words, if LLMs are likely to increase productivity differentially across industries, one concern is that the most productive would become even more productive. With inelastic demand for the production of those industries, the most productive sectors would shrink as a proportion of inputs in the economy. We see little to suggest this will be the case. Productivity growth since 2012 and exposure to LLM technologies appear unrelated. 


% \section{Demographic Variation in Exposure}
% \label{subsec:demographics}

% \begin{table}[h]
%     \centering
%     \input{tables/demographics_table}
%     \caption{Demographic Differences in Exposure}
%     \label{tab:demographics}
% \end{table}

% From the table above, we see that the proportion of women employed in an occupation is positively and significantly associated with an occupation's exposure to GPTs. Across all measures, we see the proportion of Asian people in an occupation to be positively correlated with GPT-exposure and that of Latino people to be negatively correlated. Demographic groups are unevenly distributed across occupations.
% % part of the effects are simply that people are unevenly distributed in occupations.

\section{Occupations Without Any Exposed Tasks}
\label{subsec:noexposure}
\input{tables/bottom_jobs}


%Bibliography
% \bibliographystyle{unsrt}  

\bibliographystyle{apalike} 

\bibliography{references}  


\end{document}


