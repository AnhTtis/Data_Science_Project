\textbf{Implications for public policy}

Our findings suggest that LLMs and the software built on top of them will have diffuse and transformative impacts on work in the United States. The pace of progress in model capabilities also suggests that impacts could begin to come about rather quickly as more applications are developed that leverage increasingly powerful models. Exposure risk is not uniform across all workers, and policymakers should therefore expect reductions and expansions in the demand for various skills to affect some workers more than others, leading to differential impacts on the frequency of AI-induced job loss and the need to transition between jobs and occupations. Furthermore, the realized impacts on the labor market will not be solely determined by model capabilities or the task-composition of work, but are instead interdependent on a range of complex factors including how these systems are adopted, how they are positioned relative to humans and other technology within various production processes, macroeconomic trends, regulation around AI deployment and adoption, and other determinants. 

Job loss and economic disruption resulting from technological advances can cause significant long-lasting harms to individuals and families and public policy can play a role in reducing these harms while maximizing the positive economic impact of these systems. See 

We highlight three areas of potential public policy opportunities in light of our findings. \footnote{Several of these ideas are drawn from the \href{https://workofthefuture.mit.edu/wp-content/uploads/2021/01/2020-Final-Report4.pdf}{final report of the MIT Task Force on the Work of the Future}, which has a much more detailed analysis of policy options to consider in response to the labor market impacts of advances in artificial intelligence} 

\begin{enumerate}
\item \textbf{Increased economic security through enhanced worker safety nets}

Expanded wage insurance options for workers who are displaced from a job and transition to lower-wage employment can help smooth job transitions for workers and provide added economic security, relieving constraints that may allow for investments in skill development and increasing earnings and employment. \cite{hyman2021wage} One example of a similar program is the Reemployment Trade Adjustment Assistance (RTAA) program, in which eligible workers who are displaced from their job due to a trade-related event can receive up to fifty percent of the gap between their pre-and post-displacement wages for up to two years. 

Similarly, reforms to unemployment insurance in the U.S. can help safeguard against the harms associated with shocks to labor demand. Several proposals to modernize UI benefits such as determining eligibility based on hours worked rather than earnings, and expanding eligibility to workers seeking part-time work and those working as independent contractors warrant consideration by policymakers.


\item \textbf{Investment in the development and evaluation of innovative worker training programs and AI literacy initiatives}

Investment in worker training and AI literacy initiatives can help prepare workers for the changing labor market and ensure they have the skills needed to succeed in the future of work. One approach is to invest in apprenticeship programs that combine on-the-job training with classroom instruction. These programs can help workers gain the technical skills and practical experience needed to succeed in high-demand occupations. \cite{Reed2012} Policymakers should consider investing in initiatives that promote AI literacy, such as training programs that teach workers how to use and understand AI technologies. \cite{AIliteracy2020, milesblog} Establishing public-private partnerships and providing incentives for businesses to invest in workforce development alongside AI adoption is another promising approach to help workers adapt to the changing nature of many jobs.


\item \textbf{Investments in technology and information systems to make job transitions easier for workers}

Improving the quality and accessibility of labor market information for workers and employers through, for example, expanded public-private data sharing partnerships between employers and American Job Centers can help provide up-to-date information on job openings, reducing search costs and mitigating the effects of job transitions for workers.  

\end{enumerate}

In the longer term, larger-scale policy innovation may be needed if model capabilities advance to the point where aggregate labor demand diminishes. Given the difficulty of predicting long-run labor market impacts, it is worthwhile to experiment and prototype large-scale safety net expansions such as Universal Basic Income and consider updated tax policy proposals in a future with significantly less demand for human labor.