\documentclass[twocolumn,notitlepage,prl,superscriptaddress,longbibliography]{revtex4-2}
\usepackage{amsfonts}
\usepackage{textcomp}
\usepackage{times}
\usepackage{graphicx}
\usepackage{float}
\usepackage{latexsym,amsmath,amssymb,bm,euscript}
\usepackage{color}
\usepackage{subfigure}
\usepackage{epstopdf}
\usepackage[colorlinks=true,linkcolor=blue,citecolor=blue,urlcolor=blue]{hyperref}
\usepackage{hyperref}
\usepackage{soul}
\usepackage[normalem]{ulem}
\usepackage{mathrsfs}
\usepackage{amsmath}
\usepackage{amstext}
\usepackage[bottom]{footmisc}
\usepackage{amsbsy}
\usepackage{ulem}

\newcommand{\gss}[1]{{\color{blue} #1}}

%%%%%%%%%
\begin{document}

%%%%%%
\title{Sign structure in the square-lattice $t$-$t^\prime$-$J$ model and numerical consequences} 
\author{Xin Lu}\thanks{Both authors contributed equally to this work.}
\affiliation{School of Physics, Beihang University, Beijing 100191, China}

\author{Jia-Xin Zhang}\thanks{Both authors contributed equally to this work.}
\affiliation{Institute for Advanced Study, Tsinghua University, Beijing 100084, China}

\author{Shou-Shu Gong}
\email{shoushu.gong$@$buaa.edu.cn}
\affiliation{School of Physics, Beihang University, Beijing 100191, China}
\affiliation{School of Physical Sciences, Great Bay University, Dongguan 523000, China}

\author{D. N. Sheng}  
\email{donna.sheng1$@$csun.edu}
\affiliation{Department of Physics and Astronomy, California State University, Northridge, California 91330, USA}

\author{Zheng-Yu Weng}
\email{weng@mail.tsinghua.edu.cn}
\affiliation{Institute for Advanced Study, Tsinghua University, Beijing 100084, China}

%%%
\begin{abstract}
Understanding the doped Mott insulator is a central challenge in condensed matter physics. 
This study identifies an intrinsic Berry-phase-like sign structure for the square-lattice $t$-$t'$-$J$ model with the nearest-neighbor ($t$) and next-nearest-neighbor hopping ($t'$), which could help explain the origin of the quasi-long-range superconducting and stripe phases observed through density matrix renormalization group (DMRG) calculation. 
We first demonstrate that the hole binding underlies both the superconducting and stripe orders, and then show that the hole pairing generically disappears once the phase-string or mutual statistics component of the sign structure is switched off in DMRG calculation. In the latter case, the superexchange interaction no longer plays a crucial role in shaping the charge dynamics, where a Fermi-liquid-like phase with small hole Fermi pockets is found. It is in sharp contrast to the large Fermi surfaces in either the stripe phase found at $t'/t<0$ or the superconducting phase at $t'/t>0$ in the original $t$-$t'$-$J$ model on the 6-leg ladder.
%reduce to a simple Fermi-liquid-like phase with small hole Fermi pockets that are distinctly different from the large Fermi surfaces in the original model.  % without the phase-string or associated mutual statistics. % the sign of $t'/t$ , and neither does . %with only a smooth band-structure effect on the Fermi pockets 
\end{abstract}

\maketitle

%%%%%%Introduction
{\it Introduction.---} 
Understanding the emergence of unconventional superconductivity (SC) and its pairing mechanism is a major task in condensed matter physics~\cite{Keimer_2015,Proust_2019}. 
Since the unconventional SC is usually realized by doping the parent Mott insulators, the Hubbard and effective $t$-$J$ models are commonly taken as the minimal models to study SC in doped Mott insulators~\cite{Keimer_2015,Proust_2019,Anderson_1987,Anderson_2004,Wen_2006,IJMPB2007,Fukuyama2008,Arovas_2022}.
While analytical solutions for two-dimensional correlated systems may be still not well controlled, numerical calculations have found different quantum phases in the doped square-lattice Hubbard and $t$-$J$ models~\cite{White_PRL_1998,White_1999,White2003,Hager2005,Corboz_PRL_2014,Simons_PRX_2015,Ehlers2017,Garnet_Science_2017,EWHuang2017,Ido2018,Jiang_2018,Corboz_2019,QinMingPu_PRX_2020,Mingpu_2022,Xu2022,Martins_2001,Sorella2002,Shih2004,White_2009,Eberlein2014,Kivelson_PRB_2017,HCJ_Science_2019,Chung_PRB_2020,Jiang_PRR_2020,Lu_2022,White_PRB_2022,White_PNAS_2021,Jiang_PRL_2021,Gong_PRL_2021,Jiang_Kivelson_Lee_2023,Wietek_PRL_2022,Wietek2021,Qu2022,Jiang2023,Xu2023}. In particular, some phases have been established for quasi-1D systems, including the spin and charge intertwined charge density wave (CDW) order with relatively weak SC correlation~\cite{White_PRL_1998,White_1999,White2003,Hager2005,Corboz_PRL_2014,Simons_PRX_2015,Ehlers2017,Garnet_Science_2017,EWHuang2017,Ido2018,Jiang_2018,Corboz_2019,QinMingPu_PRX_2020,Mingpu_2022,Xu2022}, as well as the Luther-Emery liquid with quasi-long-range SC and CDW orders~\cite{Luther_PRL_1974,Balents_PRB_1996} by turning on the next-nearest-neighbor (NNN) hopping $t'$ with both positive and negative $t'/t$ ($t$ is the nearest-neighbor (NN) hopping)~\cite{Kivelson_PRB_2017,HCJ_Science_2019,Chung_PRB_2020,Jiang_PRR_2020,Lu_2022}.

Recently, density matrix renormalization group (DMRG) studies on $t$-$t'$-$J$ model~\cite{White_PRB_2022,White_PNAS_2021,Jiang_PRL_2021,Gong_PRL_2021,Jiang_Kivelson_Lee_2023,Wietek_PRL_2022} find that while the CDW phase persists at $t'/t < 0$~\cite{White_PRB_2022,White_PNAS_2021}, it gives way to a robust $d$-wave SC phase with tuning $t'/t>0$ ~\cite{White_PNAS_2021,Jiang_PRL_2021,Gong_PRL_2021,Jiang_Kivelson_Lee_2023} on 6- and 8-leg ladders.   
The Fermi surface topology~\cite{Gong_PRL_2021,White_PNAS_2021} indicates that the SC phase provides a natural understanding on the cuprates with electron doping, and the $t'/t<0$ side should correspond to the hole doping~\cite{Kim1998,Pavarini2001,Tanaka2004}, leaving the emergent SC at hole doping still puzzling.
Interestingly, DMRG results indicate hole binding may also exist in the CDW phase~\cite{White_PNAS_2021}.
These  numerical results raise important questions such as what is the origin of the hole binding? Whether the holes are indeed paired in the CDW phase, and if they are, what is the nature of this state?

In this work, combined with the state-of-the-art DMRG calculations, we demonstrate that in the $t$-$t'$-$J$ model near the optimal doping, the hole binding underlies both the SC and CDW phases as a common feature either on the $t'/t < 0$ or $t'/t \geq 0$ side. 
By switching off the phase-string, which is analytically identified in the present model as the mutual-statistical sign structure~\cite{Weng.Sheng.1996,Ting.Weng.1997,Zaanen.Wu.2008}, we show that the hole binding is diminished, and both the stripe and SC phases are transited to a Fermi-liquid-like (FL-like) state~\cite{Jiang_Chen_Weng_2020}. Correspondingly, the original large Fermi surfaces are reduced to the small Fermi pockets in the absence of the phase-string that characterizes the many-body quantum entanglement between the spin and charge degrees of freedom.%~\cite{Weng.Sheng.1996,Ting.Weng.1997,Zaanen.Wu.2008,Weng_PRB_1999,IJMPB2007}. %Our DMRG calculations indicate that the phase-string is a key ingredient responsible for the stripe and SC orders with emergent large Fermi surfaces.



%%%%%%
{\it Exact sign structure of the $t$-$t'$-$J$ model.---}
We first discuss the sign structure in the $t$-$t'$-$J$ model, which is defined as $H_{t\text{-}t^\prime \text{-}J}=H_{t\text{-}t^\prime} +H_J$, with both the NN $\langle i j\rangle$ and NNN $\langle\langle i j\rangle\rangle$ hopping terms 
\begin{equation}\label{Ht}
  H_{t-t^{\prime}}\equiv - t \sum_{\langle i j\rangle \sigma}  c_{i, \sigma}^{\dagger} c_{j, \sigma}-t^{\prime} \sum_{\langle\langle i j\rangle\rangle\sigma} c_{i, \sigma}^{\dagger} c_{j, \sigma}+\text {h.c.},
\end{equation}
as well as the NN superexchange term
%\begin{equation*}\label{HJ}
$H_J = J \sum_{\langle ij \rangle}({\bf S}_i \cdot {\bf S}_j - \frac{1}{4} {n}_i {n}_j)$,
%\end{equation*}
where ${c}^{\dagger}_{i,\sigma}$ and ${c}_{i,\sigma}$ are the creation and annihilation operators for the electron with spin $\sigma/2$ ($\sigma = \pm 1$) at the site $i$, ${\bf S}_{i}$ is the spin-$1/2$ operator, and ${n}_i \equiv \sum_{\sigma} {c}^{\dagger}_{i,\sigma} {c}_{i,\sigma}$ is the electron number operator.

The central theme to be established in this work is that the physics of the $t$-$t'$-$J$ model is essentially dictated by the sign structure of the model just like that a FL state is determined by the Fermi sign structure (statistics) in a conventional weakly interacting system. Here the sign structure refers to the sign factor $\tau_C$ in the following partition function for the system at a finite hole doping
\begin{equation}\label{Zex}
  Z_{t\text{-}t^\prime \text{-}J}\equiv \operatorname{Tr} e^{-\beta H_{t\text{-}t^\prime \text{-}J}} =\sum_{C} \tau_{C} W_{t\text{-}t^\prime \text{-}J} [C],
\end{equation}
where $\beta$ is the inverse temperature, $W_{t\text{-}t^\prime \text{-}J} [C]\geqslant 0$ denotes the positive weight for each closed loop $C$ of all spin-hole coordinates on the square lattice, and a quantum sign associated with the path $C$ is characterized by $\tau_{C}=\pm 1$. Here both $W$ and $\tau$ may be generally determined via the stochastic series expansion (cf. Sec. I. of the Supplemental Materials (SM)~\cite{SM}):
\begin{equation}\label{highZ}
  Z_{t\text{-}t^\prime \text{-}J}=\sum_{n=0}^{\infty} \sum_{\left\{\alpha_{i}\right\}} \frac{\beta^{n}}{n !} \prod_{i=0}^{n-1}\left\langle\alpha_{i}|(-H_{t\text{-}t^\prime \text{-}J})| \alpha_{i+1}\right\rangle,
\end{equation}
where for each $n$, $|\alpha_{n}\rangle=|\alpha_{0}\rangle$ denoting the many-body hole and spin-Ising bases (with $\hat{z}$ as the quantization axis) such that the partition function is characterized by a series of closed loops of step $n$'s, denoted by $C$, which include both hopping and superexchange processes. The total sign collected by $\tau_{C}$ is precisely given by~\cite{SM}  
\begin{equation}\label{tau}
  \tau_{C}\equiv \tau^0_{C}  \times  (-1)^{N^h_{\downarrow}} 
\end{equation}
with $  \tau^0_{C}\equiv (-1)^{N^h_{\mathrm{ex}}}\times \left[\operatorname{sgn} \left(t^{\prime}\right) \right]^{N^h_{t'}} $, in which $N^h_{\mathrm{ex}}$ denotes the total number of exchanges between the identical holes, i.e., the usual Fermi statistical sign structure of the doped holes like in a semiconductor, and $N^h_{t'}$ counts the total steps of the NNN hopping of the holes in the path $C$, resulting in a geometric Berry phase at $t'<0$.  In Eq. \eqref{tau}, the NN hopping integral is assumed to be always positive for simplicity, i.e., $t>0$.

%%%%%%
\begin{figure}[t]
	\includegraphics[width=1.0\linewidth]{model.pdf}
	\caption{Phase diagrams of the $t$-$t'$-$J$ model (a) and $\sigma t$-$t'$-$J$ model (b), determined on the $L_{y}=6$ cylinder. At the range of $-0.2 \leq t^\prime / t \leq 0.2$ and doping level $1/12\leq \delta \leq 1/8$, the CDW and SC phases present in the $t$-$t'$-$J$ model are replaced by a ubiquitous FL-like state in the $\sigma t$-$t'$-$J$ model. The topology of large Fermi surfaces in the CDW and SC phases in (a) are reduced to that of small hole pockets in the FL-like phase in (b), as illustrated by the cartoon plots of the momentum distribution (the darker regions indicate the electron occupancy with higher probability). Here the $\sigma t$-$t'$-$J$ model is distinct from the $t$-$t'$-$J$ model only by a spin-dependent sign in the NN hopping integral $\sigma t$ with the same NNN hopping $t'$ and NN superexchange $J$, as indicated in the left inset of (b) with the arrows (circles) denoting electrons (holes). Following Ref.~\cite{Gong_PRL_2021}, the phase boundary in the $t$-$t'$-$J$ model is determined by examining charge density profile and comparing different correlation functions.
 		}
	\label{fig:model}
\end{figure}

%$N_{t^\prime}$ denotes the total number of NNN exchanges between spin and holon, regardless of spin direction, leading to the asymmetry of the opposite sign of $t^\prime$. Also, The sign structure  \eqnref{tau} from the expansion of partition function is precisely The ``no quantum frustration” condition requires that all loops have non-negative weight, since ``quantum frustration” actually measures the nontrivial accumulation of the quantum phase in the Hamiltonian. If so, up to local $\mathbb{Z}_{2}$ transform to add a negative sign, all elements in the Hamiltonian of the system without frustration are non-positive, i.e., $\langle\alpha|H| \beta\rangle \leq 0$ if $|\alpha\rangle \neq|\beta\rangle$, which is also called ``sign-free” condition\cite{Ye.Wang.2014}. Physically, from the perspective of path integral, a violation of this indicates that the contributions from distinct paths connecting site $i$ and site $j$ have different phases. The existence of such destructive interference can greatly deviate the physical behavior of the frustrated system from the classical case. In the following text, we will show the non-trivial sign structure encoded in the $t$-$t^{\prime}$-$J$ model.

The sign factor $(-1)^{N^h_{\downarrow}}$ in Eq.~\eqref{tau} is known as the phase-string~\cite{Weng.Sheng.1996,Ting.Weng.1997,Zaanen.Wu.2008}, in which $N^h_{\downarrow}$ denotes the total mutual exchanges between the holes and down-spins at the NN hoppings, which has been identified in the $t$-$J$ model at $t'=0$~\cite{Zaanen.Wu.2008} as a novel long-range entanglement or mutual statistics between the doped holes and the spins~\cite{Ting.Weng.1997,IJMPB2007}. 
The phase-string Berry phase is thus generally present in the $t$-$t'$-$J$ model. 
To explicitly see its effect, we point out that if the hopping term $H_{t-t^{\prime}}$ is changed to 
\begin{equation}\label{sigma t}
  H_{\sigma t-t^{\prime}}\equiv - t \sum_{\langle i j\rangle \sigma} \sigma c_{i, \sigma}^{\dagger} c_{j, \sigma}-t^{\prime} \sum_{\langle\langle i j\rangle\rangle\sigma} c_{i, \sigma}^{\dagger} c_{j, \sigma}+\text {h.c.},
\end{equation}
with the superexchange term $H_J$ unchanged, which is to be called the $\sigma t$-$t'$-$J$ model below, the $\tau_{C}$ in Eq.~\eqref{tau} will reduce to the sole sign structure in the partition function~\cite{SM}
\begin{equation}\label{Zsigma}
  Z_{\sigma t\text{-}t^\prime \text{-}J} =\sum_{C} \tau^0_{C} W_{t\text{-}t^\prime \text{-}J} [C]~.
\end{equation}
In other words, one can precisely switch off the phase-string in the $\sigma t$-$t'$-$J$ model such that the sign structure becomes a conventional $ \tau^0_{C}$ with the \emph{same} amplitude $W_{t\text{-}t^\prime \text{-}J} [C]$~\cite{SM}. 
The above sign structures for both models are exact at any finite size, arbitrary doping, and temperature.


{\it Phase diagrams obtained from DMRG.---}
In the following, we shall accurately determine the ground states of both models using DMRG calculations~\cite{White_DMRG_1992} to reveal the important role played by the phase-string component of the sign structure in hole pairing.
We choose $t/J = 3$ to mimic a large Hubbard $U/t = 12$ and tune the hopping ratio in the region of $-0.2 \leq t^\prime / t \leq 0.2$. 
We focus on two doping levels at $\delta = N_h / N = 1/12$ and $1/8$, where $N_h$ is the hole number and $N$ is the total site number.
We study the models on cylinder geometry with the periodic boundary condition along the circumference direction ($y$) and open boundary along the axis direction ($x$), and use $L_y$ and $L_x$ to denote the site numbers along the two directions, respectively.
For the $t$-$t'$-$J$ model with spin SU(2) symmetry, we keep the bond dimensions up to $12000$ SU(2) multiplets~\cite{McCulloch2002} (equivalent to about $36000$ U(1) states). 
For the $\sigma t$-$t'$-$J$ model, we use the U(1) symmetry and keep the bond dimensions up to $15000$ states.
We obtain accurate local measurements and correlation functions on the $L_y = 4, 6$ cylinders with the truncation error near $1\times 10^{-6}$.

 
We illustrate the phase diagrams obtained on 6-leg cylinder in Fig.~\ref{fig:model}. 
With the phase-string, the $t$-$t'$-$J$ model shows the CDW and SC phases as a function of $t'/t$ [Fig.~\ref{fig:model}(a)], both of which have large Fermi surfaces.
By contrast, by switching off the phase-string, the $\sigma t$-$t'$-$J$ model exhibits a FL-like phase with small Fermi pockets in the whole studied range of $t'/t$ [Fig.~\ref{fig:model}(b)], which is dictated by the Fermi signs in $ \tau^0_{C}$. 


%%%%%%
\begin{figure}[t]
	\includegraphics[width=1\linewidth]{binding_energy.pdf}
	\caption{Binding energy in the $t$-$t'$-$J$ and $\sigma t$-$t'$-$J$ models. (a) and (b) show the binding energies on the 4-leg cylinders at $\delta=1/12$ and $\delta=1/8$, respectively. The results of the $t$-$t'$-$J$ and $\sigma t$-$t'$-$J$ models are marked with solid and open symbols, respectively. We present two sizes ($L_{x}=24$ and $48$) for the $t$-$t'$-$J$ model to show the good convergence of the binding energy with $L_x$. For the $\sigma t$-$t'$-$J$ model, the binding energies exhibit a pronounced finite-size effect, and we show the results after the size extrapolation to the infinite-$L_{x}$ limit. (c) Extrapolations of binding energies versus system length for the 4-leg $\sigma t$-$t'$-$J$ model at $\delta=1/12$, which are fitted by a second-order polynomial function $\mathcal{C} \left(1/L_x \right)=\mathcal{C} \left(0\right)+a/L_x +b/L_x^2$. (d) Binding energies for the two models on the $L_y = 6, L_x = 32$ cylinder at $\delta=1/12$. The results of the $\sigma t$-$t'$-$J$ model are obtained using $12000$ bond dimensions. For the $t$-$t'$-$J$ model, the binding energies are calculated using the energies $E(N_h, S)$ after the extrapolation to the infinite bond dimension limit (see SM Sec. III.~\cite{SM}).}
	\label{fig:binding}
\end{figure}



%%%%%%
{\it Binding energy.---}
An important result we find in this work is that in the $t$-$t'$-$J$ model the holes form pairs not only in the SC phase, but also in the CDW phase.
Following the usual definition, we calculate the binding energy $E_b$ as
\begin{equation} \label{eq:binding}
E_b \equiv E(N_h, 0) + E(N_h-2, 0) - 2 E(N_h-1, 1/2),
\end{equation}
where $E(N_h, S)$ denotes the lowest-energy in the sector with hole number $N_h$ and total spin quantum number $S$ (total spin-$z$ component $S$) for the $t$-$t'$-$J$ ($\sigma t$-$t'$-$J$) model.
The negative $E_b$ characterizes hole binding.
On the 4-leg cylinder, we obtain well converged binding energy.
In the $t$-$t'$-$J$ model, although the system exhibits different phases including the Luther-Emery liquids, stripe, and phase separation with tuning parameters~\cite{Jiang_PRR_2020}, the binding energies, which converge quickly with system length, are always negative [Figs.~\ref{fig:binding}(a) and \ref{fig:binding}(b)]. 
On the other hand, for the $\sigma t$-$t'$-$J$ model, the binding energies are positive and strongly size dependent, which  are clearly extrapolated to zero for $L_x \to \infty$ [Fig.~\ref{fig:binding}(c) and see SM Sec. II.~\cite{SM} for the binding energy scaling at $\delta = 1/8$]. 

To explore the binding energy towards two dimensions, we also calculate the binding energies on the $L_y = 6, L_x = 32$ cylinder as presented in Fig.~\ref{fig:binding}(d), showing the qualitatively the same conclusion. 
For computing the binding energy in the $t$-$t'$-$J$ model, the energies $E(N_h, S)$ in Eq.~\eqref{eq:binding} have all been extrapolated to the infinite bond dimension limit (SM Sec. III.~\cite{SM}). 
For the $\sigma t$-$t'$-$J$ model, the obtained binding energies, which are almost independent of the kept bond dimension, have small positive values around $0.1$, much smaller than the values for the same $L_x$ on the 4-leg cylinder.
Remarkably, while the pairing correlation decays exponentially in the CDW phase of the 6-leg $t$-$t'$-$J$ model [Fig.~\ref{fig:correlation}(c)], the hole pairing still exists, which is quite consistent with the characteristic of a pseudogap phase~\cite{IJMPB2007}. 
Next, we will further discuss this point from the perspective of correlation functions. 
Our results show that the hole pairing is indeed diminished no matter in what phase if one switches off the phase-string precisely.
  


%%%%%%
\begin{figure}[t]
	\includegraphics[width=1.0\linewidth]{nk6.pdf}
	\caption{Electron distribution in the momentum space $n(\bf k)$ on 6-leg cylinder with $\delta = 1/12$.
		(a) and (b) show the results of the $t$-$t'$-$J$ model with a large Fermi surface. The Fermi surface topology is different in the CDW and SC phases. (c) and (d) show the results of $\sigma t$-$t'$-$J$ model. The electrons of spin-up and spin-down in $n(\bf k)$ are displaced by $(\pi,\pi)$.}
	\label{fig:nk}
\end{figure}


%%%%%%
{\it Fermi surface.---} 
The electron momentum distributions, $n\left(\mathbf{k}\right)=\frac{1}{N}\sum_{i,j,\sigma} \langle {\hat{c} }_{i,\sigma}^{\dagger } {\hat{c} }_{j,\sigma} \rangle e^{i\mathbf{k}\cdot \left({\mathbf{r}}_i -{\mathbf{r}}_j \right)}$, are demonstrated in Fig.~\ref{fig:nk}. 
The $t$-$t'$-$J$ model always exhibits a large Fermi surface with hole binding as given above, but the topology of Fermi surface has a strong $t'/t$ dependence.
By contrast, the $\sigma t$-$t'$-$J$ model shows two small Fermi pockets at $\mathbf{k}=\left(\pi ,\pi \right)$ and $\left(0 ,0 \right)$, which have a weak $t'/t$ dependence and are contributed from the spin-up and spin-down propagators, respectively (see SM Sec. IV. for the results of other parameters~\cite{SM}). 
For both spin components, the quasi-long-range single-particle correlations $G_{\sigma}(r) \equiv \langle c^\dagger_{(x,y),\sigma}  c_{(x+r,y),\sigma}\rangle \sim r^{-K_{\mathrm{G}}}$ with $K_{\mathrm{G}} \approx 1$ [Fig.~\ref{fig:correlation}(d)] resemble a FL-like state, which smoothly persists to a finite $t^\prime/t$ (cf. Table \ref{Table I}).
Such a non-pairing state, which has been reported at $t'/t=0$ in the 2-leg and 4-leg systems~\cite{Jiang_Chen_Weng_2020}, appears to be insensitive to $t'/t$ and stable on larger system size.


{\it CDW profile and correlation functions.---}
Next, we characterize the two systems through charge density profile and correlation functions.
We measure the average charge density distribution in the column $x$ as $n(x) = \sum_{y=1}^{L_y} \langle \hat{n}(x,y)\rangle / L_y$.
For the $t$-$t'$-$J$ model on 6-leg cylinder, one can see a clear charge density oscillation with a period of $4 / (L_y \delta)$ in the CDW phase and uniform charge distribution in the SC phase [Fig.~\ref{fig:correlation}(a)].
On the other hand, the charge density distribution has no apparent dependence on $t'/t$ in the $\sigma t$-$t'$-$J$ model, and the charge oscillation is quite weak and without a well-defined periodicity by switching off the phase-string [Fig.~\ref{fig:correlation}(b)].



\begin{table*}
   \caption{Summary of the quantum phases for the $t$-$t'$-$J$ and $\sigma t$-$t'$-$J$ model on the 6-leg cylinder with doping ratio $\delta=1/12$. The decay exponents of correlation functions are presented at $t'/t = -0.1, 0, 0.2$. The pairing correlation $P_{yy}(r)$ and single-particle Green's function $G(r)$ have been defined in the text. Here we also show the decay exponents of the density correlation function $D(r) = \langle {\hat{n} }(x,y) {\hat{n} }(x+r,y) \rangle -\langle {\hat{n} }(x,y) \rangle \langle {\hat{n} }(x+r,y) \rangle$ and spin correlation function $F(r) = \langle {\mathbf{S}}(x,y) \cdot {\mathbf{S}}(x+r,y) \rangle$. The correlation length of exponential fitting is denoted as $\xi$, and the power exponent of algebraic fitting is denoted by $K$. The fitted correlation functions for the $t$-$t'$-$J$ and $\sigma t$-$t'$-$J$ models are obtained by keeping $12000$ SU(2) multiplets and $15000$ U(1) states, respectively. For the FL-like phase in the $\sigma t$-$t'$-$J$ model, the pairing correlation $P_{yy}(r)$ will behave as a product of two Green's functions and thus also follows an algebraic decay, as shown in Fig.~\ref{fig:correlation}(d). Nonetheless, this algebraic decay does not characterize a quasi-long-range SC order. Therefore, we do not fit the power exponent $K_{\rm sc}$ in the FL-like phase.}
     \begin{ruledtabular}\label{Table I}
        \begin{tabular}{c l c c c c c}
         Models & Parameters & Phase & $P_{yy}(r)$ & $D(r)$ & $G(r)$ & $F(r)$ \\
         \hline
                                              & $t^{\prime}/t=-0.1$ & CDW & $\xi_{\mathrm{sc}} \approx 3.02$ & $\xi_{\mathrm{c}} \approx 4.72$ & $\xi_{\mathrm{G}} \approx 2.73$ & $\xi_{\mathrm{s}} \approx 5.91$ \\
 $t$-$t^{\prime}$-$J$                                & $t^{\prime}/t=0$ & CDW & $\xi_{\mathrm{sc}} \approx 2.95$ & $\xi_{\mathrm{c}} \approx 4.96$ & $\xi_{\mathrm{G}} \approx 2.03$ & $\xi_{\mathrm{s}} \approx 6.13$\\
                                              & $t^{\prime}/t=0.2$ & SC + CDW & $K_{\mathrm{sc}} \approx 0.55$ & $K_{\mathrm{c}} \approx 1.56$ & $\xi_{\mathrm{G}} \approx 1.97$ & $\xi_{\mathrm{s}} \approx 3.32$ \\
          \hline                                    
                                              & $t^{\prime}/t=-0.1$ & FL-like & --- & $K_{\mathrm{c}} \approx 1.93$ & $K_{\mathrm{G}} \approx 0.96$ & $K_{\mathrm{s}} \approx 1.83$ \\
$\sigma t$-$t^{\prime}$-$J$        & $t^{\prime}/t=0$ & FL-like & --- & $K_{\mathrm{c}} \approx 1.55$ & $K_{\mathrm{G}} \approx 0.81$ & $K_{\mathrm{s}} \approx 1.69$\\
                                              & $t^{\prime}/t=0.2$ & FL-like & --- & $K_{\mathrm{c}} \approx 1.82$ & $K_{\mathrm{G}} \approx 0.94$ & $K_{\mathrm{s}} \approx 1.95$\\
         \end{tabular}
      \end{ruledtabular}
\end{table*}


%%%%%%
\begin{figure}[t]
	\includegraphics[width=1\linewidth]{correlation_w6_dope12.pdf}
	\caption{Charge density profile and correlation functions on the 6-leg cylinder with $L_x = 48$ and $\delta=1/12$. (a) and (b) show the charge density profiles $n(x)$ for the $t$-$t'$-$J$ and $\sigma t$-$t'$-$J$ models, respectively. (c) and (d) are the double logarithmic plots of the pairing correlation $P_{yy}(r)$ and single-particle Green's function $G(r)$ for the $t$-$t'$-$J$ and $\sigma t$-$t'$-$J$ models. Note that we only show the $\sigma$ component of the single-particle Green's function in the $\sigma t$-$t'$-$J$ model, due to the absence of spin SU(2) symmetry~\cite{Jiang_Chen_Weng_2020}. The results of more correlation functions and the fitted decay exponents are summarized in Table~\ref{Table I}.
	}
	\label{fig:correlation}
\end{figure}



We further compare different correlation functions.
We examine the spin-singlet pairing correlation between the vertical bonds $P_{y,y}(r) \equiv \langle \hat \Delta^\dagger_{y}(x,y) \hat \Delta_{y}(x+r,y)\rangle$, 
where the pairing operator is defined for two NN sites $(x,y)$ and $(x,y+1)$, i.e. $\hat \Delta_{y}(x,y) = (c_{(x,y),\uparrow}c_{(x,y+1),\downarrow} - c_{(x,y),\downarrow}c_{(x,y+1),\uparrow})/\sqrt{2}$.
In the $t$-$t'$-$J$ model [Fig.~\ref{fig:correlation}(c)], the pairing correlation decays algebraically with strong magnitudes in the SC phase but is suppressed to decay exponentially in the CDW phase.
Remarkably, the weakened pairing correlation in the CDW phase is still much stronger than two single-particle correlator $G^{2}(r)$, which could be consistent with a pseudogap phase with hole binding but lacking phase coherence due to the strong CDW. 
Such pseudogap-like behaviors have also been observed in the triangular-lattice $t$-$J$ model~\cite{Huang_2022}, which also sits near a SC phase and may be common in doped Mott insulators.


%%%%%%

For the $\sigma t$-$t'$-$J$ model, we find that all the correlations exhibit a nice algebraic decay (see Fig.~\ref{fig:correlation}(d) and Table~\ref{Table I}), and the correlations behave smoothly and consistently as a function of $t'/t$ by switching off the phase-string.
The pairing correlation behaving as $G^2(r)$ [Fig.~\ref{fig:correlation}(d)] agrees with the prediction of a FL-like state and confirms no quasi-long-range SC order. 
Our results indicate no phase transition in the $\sigma t$-$t'$-$J$ model and only a FL-like phase exists.
In particular, the quasi-long-ranged Green's function with $K_{\mathrm{G}} \approx 1.0$ is consistent with the description of the Landau Fermi liquid theory. 


%%%%%%
{\it Conclusion.---}
By using DMRG calculation, we have unveiled that the hole pairs constitute the basic building blocks not only in the SC but also in the CDW phase of the $t$-$t^\prime$-$J$ model. 
We have also identified the precise sign structure of the model, which is composed of the Fermi statistics between the doped holes, a geometric Berry phase depending on the sign of $t^\prime$, and the phase-string mutual statistics between charge and spin degrees of freedom. 
In particular, a mere geometric Berry phase at $t^\prime<0$, with the same amplitude $W_{t\text{-}t^\prime \text{-}J} [C]$ [cf. Eqs.~\eqref{Zex} and \eqref{tau}], may stabilize the stripe over SC order. 
By turning off the phase-string, the hole pairing gets diminished to result in a FL-like phase with no more SC and stripe orders, and the corresponding Fermi surface also drastically reconstructs to become small pockets.
This FL-like phase is no longer sensitive to the sign of $t^\prime$. 
The phase-string brings in a strong correlation effect that is responsible for not only the hole pairing~\cite{Weng.Zhao.2022} but also restoring a \emph{large} Fermi surface via ``momentum shifting''~\cite{PhysRevB.98.035129, Zhao2022, Zhang.Weng_2022}. 
Namely the one-to-one correspondence principle of the Landau paradigm, which works for weak interaction in the conventional FL/BCS description, is generally violated here. 

%Here a self-consistent treatment of the mutual influence between spin and charge degrees of freedom must be in a non-perturbative way .





%%%%%%
{\it Acknowledgments.---}
We acknowledge stimulating discussions with Zheng Zhu and Hong-Chen Jiang. X.~L. and S.~S.~G. were supported by the National Natural Science Foundation of China (Grants No. 12274014 and No. 11834014). 
J.~X.~Z. and Z.~Y.~W. were supported by MOST of China (Grant No. 2017YFA0302902).  D.~N.~S. was supported by the U.S. Department of Energy, Office of Basic Energy Sciences under Grant No. DE-FG02-06ER46305 for studying unconventional superconductivity.

%%%%%% reference
\bibliography{refs}

\clearpage
%%%%%%
\appendix
\widetext
\begin{center}
	\textbf{\large Supplementary Materials for: ``Sign structure in the square-lattice $t$-$t^\prime$-$J$ model and numerical consequences''}
\end{center}

\vspace{1mm}

\renewcommand\thefigure{\thesection S\arabic{figure}}
\renewcommand\theequation{\thesection S\arabic{equation}}

\setcounter{figure}{0} 
\setcounter{equation}{0} 


%%%%%%
In the following supplementary materials, we provide more analytical and numerical results to support the conclusions presented in the main text. In Sec.~I., we give a detailed derivation of the precise sign structure of the $t$-$t^\prime$-$J$ model, in which the phase-string sign structure, in addition to the conventional Fermi statistics between the doped holes and the geometric Berry phase due to the sign of $t'/t$, will be identified, together with the analytic expression of the positive weight function. Furthermore, we also present an alternative derivation of the sign structure and demonstrate its applicability to non-bipartite lattices, such as triangular lattices or systems with the next-nearest-neighbor (NNN) superexchange interactions. In Sec.~II., we show the extrapolation of binding energy versus system length for the 4-leg $\sigma t$-$t^{\prime}$-$J$ model at $1/8$ doping. In Sec.~III., we show the extrapolations of the ground-state energies in different sectors versus the bond dimension for the 6-leg $t$-$t^{\prime}$-$J$ model at $1/12$ doping. In Sec.~IV., we show the electron densities for the 6-leg systems at $1/8$ doping and also for the 4-leg systems.


%%%%%


\section{I. The exact sign structure of the $t$-$t^\prime$-$J$ Hamiltonian}

In this section, we give a rigorous proof of the partition function in Eq.~\eqref{Zex} and the sign structure given in Eq.~\eqref{tau}. We shall start with the slave-fermion formalism, in which the electron operator is defined as $c_{i\sigma}=f_{i}^{\dagger} b_{i\sigma}$, with $f_{i}^{\dagger}$ denoting the fermionic holon operator and $b_{i\sigma}$ denoting the bosonic spinon operator, which satisfies the constraint $f_{i}^{\dagger} f_{i}+\sum_{\sigma} b_{i\sigma}^{\dagger} b_{i \sigma}=1$. To clarify the sign structure of this model, we explicitly incorporate the Marshall sign into the $S_z$-spin representation by implementing the substitution
\begin{equation}\label{Marshall}
b_{i \sigma}\rightarrow (-\sigma)^i b_{i \sigma}
\end{equation}
such that
\begin{equation}\label{frac}
c_{i \sigma}=(-\sigma)^i f_i^{\dagger} b_{i \sigma}.
\end{equation}
Then, the $t$-$t^\prime$-$J$ model can be expressed under this transformation as follows:
\begin{equation}
  H_{t-t^{\prime}-J}=-t\left(P_{o \uparrow}-P_{o \downarrow}\right)-t^{\prime} T_{o}-\frac{J}{4}\left(Q+P_{\uparrow \downarrow}\right),
\end{equation}
where
\begin{eqnarray}
  P_{o \sigma}&=&\sum_{\langle ij\rangle} b_{i \sigma}^{\dagger} b_{j \sigma} f_{j}^{\dagger} f_{i}+\text {h.c.}\\
  T_{o}&=&\sum_{\langle\langle ij\rangle\rangle\sigma}  b_{i \sigma}^{\dagger} b_{j \sigma} f_{j}^{\dagger} f_{i}+\text {h.c.}\\
  P_{\uparrow \downarrow}&=&\sum_{\langle i j\rangle} b_{i \uparrow}^{\dagger} b_{j \downarrow}^{\dagger} b_{i \downarrow} b_{j \uparrow}+\text {h.c.} \\
  Q&=&\sum_{\langle ij\rangle}\left(n_{i \uparrow} n_{j \downarrow}+n_{i \downarrow} n_{j \uparrow}\right).
\end{eqnarray}
Here $P_{o \sigma}$($T_{o}$) denotes the hole-spin NN (NNN) exchange operator, $P_{\uparrow \downarrow}$ the NN spin superexchange operator, and $Q$ describes a potential term between NN spins. By making the high-temperature series expansion [cf. Eq.~\eqref{highZ}] of the partition function up to all orders \cite{Zaanen.Wu.2008}
\begin{eqnarray}
  Z_{t\text{-}t^\prime \text{-}J}&=&\operatorname{Tr} e^{-\beta H_{t-t^{\prime}-J}}=\operatorname{Tr} \sum_{n=0}^{\infty} \frac{\beta^{n}}{n !}\left(-H_{t-t^{\prime}-J}\right)^{n}\notag\\
  &=&\sum_{n=0}^{\infty}  \frac{(J \beta / 4)^{n}}{n !} \operatorname{Tr}\left[\sum \ldots\left(\frac{4 t}{J} P_{o \uparrow}\right) \ldots P_{\uparrow \downarrow} \ldots\left(\frac{-4 t}{J} P_{o \downarrow}\right) \ldots\left(\frac{4 t^{\prime}}{J} T_{o}\right) \ldots Q \ldots\right]_{n}\notag\\
  &=&\sum_{n=0}^{\infty}(-1)^{N_{ \downarrow}^h}\left(\operatorname{sgn} t^{\prime}\right)^{N_{t^{\prime}}^h} \frac{(J \beta / 4)^{n}}{n !} \operatorname{Tr}\left[\sum \ldots\left(\frac{4 t}{J} P_{o \uparrow}\right) \ldots P_{\uparrow \downarrow} \ldots\left(\frac{4 t}{J} P_{o \downarrow}\right) \ldots\left(\frac{4 \left| t^{\prime}\right|}{J} T_{o} \right) \ldots Q \ldots\right]_{n}
\end{eqnarray}
where the NN hopping integral is assumed to always be positive for simplicity, that is $t>0$. The notation $[\sum \ldots]_n$ indicates the summation over all $n$-block production, and because of the trace, the initial and final hole and spin configurations should be the same such that all contributions to $Z_{t\text{-}t^\prime \text{-}J}$ can be characterized by closed loops of holes and spins. Here $N_{ \downarrow}^h$ denotes the number of NN exchanges between down-spins and holes, as well as $N_{t^\prime}^h$ denotes the number of NNN exchanges between spins and holes, regardless of spin direction. Inserting complete Ising basis with holes
\begin{equation}
  \sum_{\phi\left\{l_{h}\right\}}\left|\phi ;\left\{l_{h}\right\}\right\rangle\left\langle\phi ;\left\{l_{h}\right\}\right|=1
\end{equation}
between the operator inside the trace with $\phi$ specifying the spin configuration and $\left\{l_{h}\right\}$ denoting the positions of holes. Then, all the elements inside the trace are positive and the partition function can arrive at a compact expression:
\begin{figure}[t]
	\includegraphics[scale=0.5]{tworep.pdf}
	\caption{Illustration of an example of a closed loop for the doped square lattice, with matrix elements of the interaction process ($-H$) labeled by thick red lines. The black (gray) $\pm1$ markers indicate the sign of the matrix elements before (after) the Marshall sign transformation Eq.~\eqref{frac}. Additionally, the grey circles represent holes.}
	\label{fig_sign}
\end{figure}
\begin{equation}
  Z_{t\text{-}t^\prime \text{-}J}=\sum_{C} \tau_{C}W_{t\text{-}t^\prime \text{-}J} [C],
\end{equation}
where all the sign information is captured by  %^{t\text{-}t^\prime %\text{-}J}$:
\begin{equation}\label{tautotal}
  \tau_{C}\equiv \tau_C^0 \times(-1)^{N_{\downarrow}^h},
\end{equation}
with
\begin{equation}\label{tau0}
  \tau_C^0 \equiv(-1)^{N_{\mathrm{ex}}^h} \times\left[\operatorname{sgn}\left(t^{\prime}\right)\right]^{N_{t^{\prime}}^h},
\end{equation}
which is consistent with Eq.~\eqref{tau} in the main text. Here, $N_{\mathrm{ex}}^h$ denotes the number of exchanges between holes due to fermionic statistics of holon $f$. Such a sign structure is precisely described at arbitrary doping, temperature, and finite-size for $t\text{-}t^\prime \text{-}J$ model, which has been previously identified at $t^\prime=0$ in Ref. \onlinecite{Zaanen.Wu.2008}. In addition, the non-negative weight $W[C]$ for closed loop $C$ is given by:
\begin{equation}
  W_{t\text{-}t^\prime \text{-}J} [C]=\left(\frac{4 t}{J}\right)^{M_{t}[C]}\left(\frac{4 |t^{\prime}|}{J}\right)^{M_{t^{\prime}}[C]} \sum_{n} \frac{(J \beta / 4)^{n}}{n !} \delta_{n, M_{t}+M_{t^\prime}+M_{\uparrow \downarrow}+M_Q} \geq 0,
\end{equation}
in which $M_t$ and $M_{t^\prime}$ represent the total steps of the hole NN and NNN “hoppings” along the closed loops for a given path $C$ with length $n$, respectively. Also, $M_{\uparrow \downarrow}$ represents the steps of NN spin exchange process,  while $M_Q$ represents the total number of down spins interacting with up spins via the $S_z$ components of the superexchange term.

In summary, the sign structure of the $t$-$t^\prime$-$J$ model comprises not only the effects of NNN hopping and the traditional fermionic statistics encoded in $\tau_C^0$, but also a significant component of $(-1)^{N_\downarrow^h}$ originating from the NN hole hopping process. This component is depicted in Fig. \ref{fig_sign}(a) by blue $\pm 1$ on the arrows. From the viewpoint of the original representation, i.e., before the Marshall sign transformation in Eq. \eqref{frac}, each spin flip results in a negative sign under the Ising basis, since $\langle\downarrow_i \uparrow_j|J S_i^{-} S_j^{+}| \uparrow_i \downarrow_j\rangle >0$. Therefore, in the presence of hole hopping, an odd number of spin flips can occur in the closed loop of a bipartite lattice, as illustrated in Fig. \ref{fig_sign}(a) by black $\pm 1$ on the arrows.


%NN hops and NNN of the holes associated with a particular closed path $C$ with length $n$, respectively, $M_{\uparrow \downarrow}$ represents NN hops of the down spins, while $M_Q$ represents the total steps of counts the total number of down spins interacting with up spins via the Ising part of the spin–spin interaction.


Furthermore, by introducing the $\sigma  t\text{-}t^\prime \text{-}J$ model, in which the original kinetic energy term Eq.\eqref{Ht} is replaced by:
\begin{equation}
  H_{\sigma t-t^{\prime}}=-\sigma t \sum_{\langle i j\rangle} c_{i, \sigma}^{\dagger} c_{j, \sigma}-t^{\prime} \sum_{\langle\langle i j\rangle\rangle} c_{i, \sigma}^{\dagger} c_{j, \sigma}+\text {h.c.},
\end{equation}
where an extra spin dependent sign $\sigma$ is inserted into the NN hopping term that cancels the ``$-$'' sign in front of the $P_{o \downarrow}$. Consequently, $\sigma  t\text{-}t^\prime \text{-}J$ model under the representation of Eq. \eqref{frac} can be rewritten as:
\begin{equation}
  H_{t-t^{\prime}-J}=-t\left(P_{o \uparrow}+P_{o \downarrow}\right)-t^{\prime} T_{o}-\frac{J}{4}\left(Q+P_{\uparrow \downarrow}\right),
\end{equation}
with the partition function under the high-temperature series expansion:
\begin{eqnarray}
  Z_{\sigma t\text{-}t^\prime \text{-}J} = \sum_{n=0}^{\infty}\left(\operatorname{sgn} t^{\prime}\right)^{N_{t^{\prime}}^h} \frac{(J \beta / 4)^{n}}{n !} \operatorname{Tr}\left[\sum \ldots\left(\frac{4 t}{J} P_{o \uparrow}\right) \ldots P_{\uparrow \downarrow} \ldots\left(\frac{4 t}{J} P_{o \downarrow}\right) \ldots\left(\frac{4 \left| t^{\prime}\right|}{J} T_{o} \right) \ldots Q \ldots\right]_{n}.
\end{eqnarray}
Consequently, the sign structure for $\sigma t\text{-}t^\prime \text{-}J$ model is given by:
\begin{equation}
  \tau_{C}^{\sigma t\text{-}t^\prime \text{-}J}= \tau_{C}^0,
\end{equation}
where $\tau_{C}^0$ is given by Eq. \eqref{tau0} and the positive weight $W[C]$ for each path remains unchanged, leading to Eq. \eqref{Zsigma}.

\subsection{Extension of exact sign structure to non-bipartite lattices }

We have established a rigorous proof of the exact sign structure for the $t$-$t^\prime$-$J$ model. In this subsection, we provide an alternative derivation of the sign structure and demonstrate its applicability to non-bipartite lattices, such as triangular lattices or systems with next-nearest-neighbor (NNN) superexchange interactions. 

\begin{figure}[t]
	\includegraphics[scale=0.8]{gaugechoice.pdf}
	\caption{(a) Illustration the geometric phase $\phi_{ij}$, where the red arrow indicates the non-zero contribution of the points on the selected closed loop (black line), with the total inner angle sum denoted as $\Theta_C$. (b) In the $t$-$t^\prime$-$J$ model, one possible gauge choice for $\phi_{ij}$ is shown here, with black bonds representing nearest-neighbor (NN) links having $e^{i\phi_{ij}}=+1$ and gray bonds representing next-nearest-neighbor (NNN) links having $e^{i\phi_{ij}}=-1$.}
	\label{fig_gauge}
\end{figure}

Here we use the extended $t$-$J$ model on an \emph{arbitrary} lattice as an example, with the Hamiltonian given by
\begin{equation}\label{extended}
    H = - T\sum_{ij,\sigma} c_{i\sigma}^\dagger c_{j\sigma}+h.c.+ K \sum_{ij} \left( \mathbf{S}_i \cdot \mathbf{S}_j-\frac{1}{4} n_i n_j \right).
\end{equation}
Unlike the main text, where the sum indexes $\langle ij \rangle$ represent only the NN links, here the sum includes all allowed links that can connect any two sites. To begin, instead of applying the Marshall basis transformation as described in Eq. \eqref{Marshall}, which relies on the $A$-$B$ sublattice division, we propose a redefinition of the up-spinon operator as follows: 
\begin{equation}\label{tran}
b_{i \uparrow} \rightarrow b_{i \uparrow} e^{-i \Phi_{i}}
\end{equation}
while keeping the down-spinon operator unchanged, such that
\begin{eqnarray}
    c_{i \uparrow}&=&f_i^{\dagger} b_{i \uparrow} e^{-i \Phi_{i}} \notag \\
    c_{i \downarrow}&=&f_i^{\dagger} b_{i \downarrow},
\end{eqnarray}
where
\begin{equation}\label{tran}
\Phi_{i} \equiv \sum_{l \neq i} \theta_{i}(l)=\sum_{l \neq i} \operatorname{Im} \ln \left(z_{i}-z_{l}\right),
\end{equation}
with $z_i$ as the complex coordinate at site $i$. Hence, by simply using the relations $\theta_{i}(j)-\theta_{j}(i)=\pm \pi$, the extended $t$-$J$ model in \label{extended} become:
\begin{eqnarray}
  H_{t\text{-}t^{\prime}}&=&-\sum_{i j, \sigma} \sigma T b_{i \sigma}^{\dagger} b_{j \sigma} f_{j}^{\dagger} f_{i} e^{\frac{i(\sigma+1)}{2} \phi_{ij}}+h.c.\notag\\
  &\;&-\frac{K}{4} \sum_{\langle ij\rangle \sigma}\left(n_{i \sigma} n_{j-\sigma}+b_{i \sigma}^{\dagger} b_{j-\sigma}^{\dagger} b_{i-\sigma} b_{j \sigma} e^{i \sigma \phi_{i j}}\right)\label{gauge},
  %&\;&-\sum_{\langle\langle i j\rangle\rangle \sigma} \sigma t^{\prime} b_{i \sigma}^{\dagger} b_{j \sigma} f_{j}^{\dagger} f_{i} e^{\frac{i(\sigma+1)}{2} \phi_{i j}}+\text{h.c.}  \label{Httfrac}\\
  %H_{J}&=&-\frac{J}{4} \sum_{\langle ij\rangle \sigma}\left(n_{i \sigma} n_{j-\sigma}+b_{i \sigma}^{\dagger} b_{j-\sigma}^{\dagger} b_{i-\sigma} b_{j \sigma} e^{i \sigma \phi_{i j}}\right), \label{HJbnew}
\end{eqnarray}
where 
\begin{equation}
   \phi_{ij}=\sum_{l \neq i j}\left[\theta_{i}(l)-\theta_{j}(l)\right]
\end{equation}
acting like a gauge potential with a gauge invariant strength given by $\sum_{C} \phi_{ij}=\Theta_C$ for a closed loop $C$ on the lattice. The symbol $\Theta_C$ denotes the interior angle sum of the closed-loop $C$ depicted in Fig.~\ref{fig_gauge}(a). It is evident that any point that is situated outside or inside the loop contributes $0$ or $\pm 2\pi$ to $\Theta_C$, respectively. Only the sites $l$ on the closed loop $C$ contribute nontrivial values to the interior angle, such that the sum of interior angles over all the points $l \in C$ corresponds to the total contribution, as illustrated in Fig.~\ref{fig_gauge}(a). Importantly, the phase $\phi_{ij}$ presented here is universally applicable to arbitrary lattices, irrespective of whether they are bipartite or not. Notably, for a bipartite system such as the $t$-$J$ model on a square lattice with $t^\prime=0$, $\Theta_C$ is equal to $2\pi \mathbb{Z}$ for any closed loop, and the phase $\phi_{ij}$ can be completely gauged away.

As the result, in Eq.~\eqref{gauge}, the hidden sign structure of extended $t$-$J$ model on an arbitrary lattice is explicitly decomposed into the geometry phase $\phi_{ij}$ and the extra $\sigma$-sign in the hopping term, which is commonly referred to as the ``phase string effect''. To derive the sign structure Eq.~\eqref{tautotal} of $t$-$t^\prime$-$J$ model on a square lattice, a proper gauge can be selected, as shown in Fig.~\ref{fig_gauge}(b), where NN links with $e^{i\phi_{ij}}=+1$ and NNN links with $e^{i\phi_{ij}}=-1$, to combine these two components of frustration, yielding Eq.~\eqref{tautotal}.

%%In the last section, we give a straightforward proof of the exact sign structure for $t$-$t^\prime$ model. Furthermore, in this section, an alternative derivation will given and such procedures are also applicable to the system on all other non-bipartite lattices, such as triangular lattices, or the introduction of NNN superexchange interaction.

%Note that we introduce  $t^\prime$ term and illustrate the sign structure of generalized $t\text{-}J$ model. Note that the procedures in this work are also applicable to the system on all other non-bipartite lattices, such as triangular lattices, or the introduction of NNN superexchange interaction.
%Compared to conventional Fermi liquid, $(-1)^{N_{\downarrow}^h}$ in $t$-$t^\prime$-$J$ model is an additional sign structure. Physically, from the perspective of path integral, this indicates that the phases contributed from distinct paths with different spin configurations connecting site $i$ and site $j$ fluctuate strongly. The existence of such destructive interference results in the nontrivial interplay between the hopping holes and the spin background, which is indeed the key to understand the striking results in $t$-$t^{\prime}$-$J$ model as shown in the main text.

%Now, the $\sigma$-sign in front of the $t^\prime$ term is canceled by lattice frustration gauge field $\phi_{ij}$, so only hole-spin NN exchange processes $P_{o \sigma}$ have such a spin-direction-dependent singular sign. 






%In addition to the geometry phase $\phi_{ij}$, another more important thing is the quantum frustration caused by the extra $\sigma$-sign in Eq.\ref{Httfrac} even on a bipartite lattice, which is described as the so-called ``phase string effect''\cite{Weng.Sheng.1996,Ting.Weng.1997}. From the view of the original representation, i.e. before the local $U(1)$ transformation Eq.\ref{tran}, since $\langle\downarrow_i \uparrow_j|J S_i^{-} S_j^{+}| \uparrow_i \downarrow_j\rangle >0$, each spin flip gives negative sign under the Ising basis, which can occur the odd number of times in the closed loop of a bipartite lattice with the existence of hole hopping process, as illustrated in Fig.\ref{fig_sign}(b).

%\begin{figure}[t]
%	\includegraphics[scale=0.8]{sign1.pdf}
%	\caption{(a) Illustration of geometry phase $\phi_{ij}$. Only points staying at the selected closed loop(black line) will lead to a non-zero contribution(red arrow), and the total of them is the inner angle sum $\Theta_C$ of the closed loop $C$. (b)Examples of a closed loop for the doped square lattice. The $\pm1$ marked outside(inside) the arrows denotes the sign of matrix elements of interaction process ($-H$) labeled by thick red lines before(after) the transformation Eq.\ref{tran}. The grey circles denote holes.}
%	\label{fig_sign}
%\end{figure}




%Next, we will select the proper gauge to combine these two components of frustration and obtain 


%Note that we introduce  $t^\prime$ term and illustrate the sign structure of generalized $t\text{-}J$ model. Note that the procedures in this work are also applicable to the system on all other non-bipartite lattices, such as triangular lattices, or the introduction of NNN superexchange interaction.


\section{II. Extrapolation of binding energy for the 4-leg $\sigma t$-$t^{\prime}$-$J$ model at $1/8$ doping}

In the main text, we have shown the binding energies for the 4-leg $\sigma t$-$t^{\prime}$-$J$ model at $\delta=1/12$ doping as shown in Fig.~\ref{fig:binding}(c). Here, we supplement with the binding energies at $\delta=1/8$ as shown in Fig.~\ref{fig:BE_8_extrap}. Similar to the results in Fig.~\ref{fig:binding}(c), the binding energies also can be extrapolated to zero in the infinite-$L_x$ limit.


\begin{figure}[t]
	\includegraphics[width=0.4\linewidth]{BE_8_extrap.pdf}
	\caption{Extrapolations of binding energies versus system length for the 4-leg $\sigma t$-$t'$-$J$ model at $\delta=1/8$. The binding energies are fitted by a second-order polynomial function $\mathcal{C} \left(1/L_x \right)=\mathcal{C} \left(0\right)+a/L_x +b/L_x^2$.}
	\label{fig:BE_8_extrap}
\end{figure}


%%%%%%
\section{III. Extrapolations of the energies versus bond dimension for the 6-leg $t$-$t'$-$J$ model}

\begin{figure}[htp]
\includegraphics[width=0.9\linewidth]{ground_energy.pdf}
\caption{Extrapolations of the ground-state energies $E(N_h, S)$ for the 6-leg $t$-$t'$-$J$ model on the size $L_{y} \times L_{x}=6 \times 32$. (a1-c1) show the bond dimension scaling of the ground-state energy at $t^{\prime}/t=-0.1$, $0$, and $0.2$ with the hole number $N_{h}=16$. (a2-c2) and (a3-c3) show the similar results for the hole numbers $N_{h}=15$ and $N_{h}=14$, respectively. We keep the bond dimensions $D$ up to $10000 - 15000$ SU(2) multiplets. The energies are extrapolated by a linear function $\mathcal{C} \left(1/D \right)=a/D + \mathcal{C} \left(0\right)$ to give the energy $E_{\infty}$ in the infinite bond dimension limit.
}
\label{ground_energy}
\end{figure}

For calculating the binding energy on the 4-leg systems, we can obtain the fully converged energies. 
For the wider 6-leg $t$-$t'$-$J$ model we need to carefully check the convergence of the energies in the three sectors $E(N_h, S)$. 
For this purpose, we obtain the energies $E(N_h, S)$ by keeping different bond dimensions, and extrapolate these energies to the infinite bond dimension limit. 
Then we compute the binding energy using the extrapolated results.
In Fig.~\ref{ground_energy}, we show the bond-dimension dependence of the obtained energies $E(N_h, S)$ for $t'/t = -0.1, 0, 0.2$ on the $L_x = 32, L_y = 6$ cylinder at $1/12$ doping.
We keep the bond dimensions up to $10000 - 15000$ SU(2) multiplets to ensure a good linear extrapolation behavior of the data.
We implement the linear fitting $\mathcal{C} \left(1/D \right)=a/D + \mathcal{C} \left(0\right)$ to obtain the energy in the infinite bond dimension limit $E_{\infty}$, from which we calculate the binding energies shown in Fig.~\ref{fig:binding}(d).

For the $\sigma t$-$t'$-$J$ model on the 6-leg cylinder, we find that although the energies $E(N_h, S)$ get improved with increasing bond dimension, the binding energy is almost independent of bond dimension in our calculation. Therefore, we do not extrapolate the energies but use the results by keeping the largest bond dimension.




%%%%%%
\section{IV. Electron momentum distribution}


We have shown the electron momentum distributions $n\left(\mathbf{k}\right)=\frac{1}{N}\sum_{i,j,\sigma} \langle {\hat{c} }_{i,\sigma}^{\dagger } {\hat{c} }_{j,\sigma} \rangle e^{i\mathbf{k}\cdot \left({\mathbf{r}}_i -{\mathbf{r}}_j \right)}$ for the 6-leg cylinder at $\delta=1/12$ in the main text. Here, we supplement with similar results for $\delta=1/8$ doping as shown in Fig.~\ref{supfig:nk6-8}. In the $t$-$t'$-$J$ model, a large Fermi surface is visible, and the topology of Fermi surface shows difference in the CDW and SC phases. By contrast, in the $\sigma t$-$t'$-$J$ model, two small Fermi pockets appear at $\mathbf{k}=\left(\pi ,\pi \right)$ and $\mathbf{k}=\left(0 ,0 \right)$, which have a weak $t'/t$ dependence and are contributed from the spin-up propagator and spin-down propagator, respectively. 
We also present the results for the 4-leg $\sigma t$-$t'$-$J$ model in Fig.~\ref{supfig:nk4}, which show the consistent features with the results on the 6-leg cylinder.


\begin{figure}[htp]
\includegraphics[width=0.5\linewidth]{nk8_8.pdf}
\caption{Electron momentum distribution $n(\bf k)$.
		(a) and (b) show the results of the $t$-$t'$-$J$ model, where an open and closed Fermi surface emerge in the CDW and SC phases. (c) and (d) show the results of the $\sigma t$-$t'$-$J$ model, where the electrons of spin-up and spin-down in $n(\bf k)$ are displaced by $(\pi,\pi)$. Here $L_y = 6$, $t^\prime/t = -0.1$ and $0.2$ at the doping level $\delta = 1/8$.
}
\label{supfig:nk6-8}
\end{figure}

%%%%%%


\begin{figure}[htp]
\includegraphics[width=0.7\linewidth]{nk4.pdf}
\caption{Electron momentum distribution $n(\bf k)$.
(a-c) show the results of the $\sigma t$-$t'$-$J$ model at $\delta = 1/12$.
(d-f) show the similar results at $\delta = 1/8$. The electrons of spin-up and spin-down in $n(\bf k)$ are displaced by $(\pi,\pi)$. Here $L_y = 4$ and $t^\prime/t = -0.1, 0, 0.2$.
}
\label{supfig:nk4}
\end{figure}
%%%%%%






\end{document}
%%%%%%

