


Coarse correlated equilibria are a concept of equilibria for games with many players which allows for correlation between players' strategies, thus generalizing the notion of Nash equilibria.
In this paper, we propose a notion of coarse correlated equilibria for a class of continuous time symmetric stochastic differential games and study the corresponding mean field formulation as the number of players $N$ goes to infinity.

Mean field games (MFGs) have been an active theme of research for almost two decades, started in the mid 2000's from the seminal works of Lasry and Lions \cite{lasry_lions} and of Huang, Malham\'e and Caines \cite{huang_malhame_caines}.
Roughly speaking, MFGs arise as the limit formulation of symmetric stochastic $N$-player games with mean field interactions between the players.
Thanks to the mean field interaction and propagation of chaos type results, one expects that the empirical distribution of players' states converges to the law of some representative player.
In the limit, the concept of Nash equilibrium translates into a fixed point problem in the space of flows of measures.
For a probabilistic approach to MFGs, we refer to the two-volume book by Carmona and Delarue \cite{librone_vol1,librone_vol2}.
The relation between the MFG and the $N$-player game is commonly understood in two ways: on the one hand, a solution of the MFG allows to construct approximate Nash equilibria for the corresponding $N$-player games, if $N$ is sufficiently large, see, e.g., \cite{campi18absorption,car_del_probabilistic,huang_malhame_caines}.
On the other hand, approximate Nash equilibria can be shown to converge to solutions of the corresponding MFG.
The choice of admissible strategies, while always important, is crucial for results of this kind: see \cite{fischer2017connection, lacker2016} for earlier results in open loop strategies and Cardaliaguet et al. \cite{cardaliaguet2019master_equation}, Lacker \cite{lacker2020convergence}, and Lacker and Le~Flem \cite{lacker_leflem2022} for convergence in closed loop strategies.


\bigskip
The notion of coarse correlated equilibrium (CCE) makes its first appearances implicitly in Hannan's work \cite{hannan1957} and explicitly in Moulin's and Vial's \cite{moulin_vial1978}.
The idea of CCEs can be summarized as follows: The game includes a correlation device or a mediator, who picks a strategy profile randomly according to some probability distribution over the set of strategy profiles, which is assumed to be common knowledge among the players.
Each player must decide whether to commit or not to the strategies selected for her by the mediator \emph{before} the mediator runs the lottery. If a player deviates, she will do so without any information on the outcome of the lottery.
If a player commits, the mediator informs her of her own recommendation, without revealing the recommendation to any other player.
In equilibrium, it is
best to commit to the anticipated outcome of the lottery if one believes that every other player is doing the same.
This notion of equilibrium is weaker than that of correlated equilibrium (CE) \`a la Aumann (see \cite{aumann1970,aumann1987}), where players decide whether to accept the mediator's recommendation \emph{after} having been informed (in private) of the strategies extracted for them.
When the distribution used by the mediator is a product distribution, CCEs reduce to usual Nash equilibria in mixed strategies, because in this case the mediator's recommendations do not carry any additional information over what is common knowledge. 
Among the nice features of CCEs, we notice the fact that they may lead to higher payoffs than Nash equilibria, even when true CEs do not exist (see Moulin et al. \cite{moulinraysengupta2014,Moulin2014Coarse} for an example in a two-person static linear quadratic game), and they naturally arise from a learning procedure of the players, such as the so called regret-based dynamics (see, e.g., Hart and Mas-Colell \cite{hart_mascolell_regret_based} and Roughgarden \cite[Section 17.4]{roughgarden_2016}).

\bigskip
Recently, correlation between players' strategy choices has been considered in the context of mean field games, although only for a class of symmetric games with discrete time and finite states and finite actions.
Campi and Fischer \cite{campifischer2021} establish the existence of symmetric CEs in such a class of games, give a definition of CE in the mean field limit and provide both approximation and convergence results.
In their formulation, players are allowed to use only restricted strategies, that is, feedback strategies that depend only on time and each player's private state.
As usual, the mean field interaction in the $N$-player game is modeled via the empirical measure of players' private states.
In the mean field limit, they propose a notion of correlated MFG solution, which is defined as a probability distribution over all the pairs of strategies and flows of measures.
Remarkably, in the mean field limit, the flow of measures is naturally stochastic, as the aggregation of individual behaviours preserves the stochasticity of the correlation device.
The subsequent work with Bonesini \cite{bonesiniCE} extends the notion of correlated solution for the MFG to allow for progressive deviations and presents a new formulation based on open loop stochastic controls.
In a second group of papers by M\"{u}ller et al. \cite{muller2021learning,muller2022learningCE}, notions of both CCEs and CEs are studied for a class of symmetric games with discrete time, finite states and finite actions, in a setting close to the one in \cite{campifischer2021,bonesiniCE}.
Interestingly, they propose a different definition of CE for the mean field game from that in \cite{campifischer2021}, and in \cite{muller2022learningCE} the authors discuss on how the two definitions of CE can be seen as equivalent.
In addition, \cite{muller2021learning,muller2022learningCE} contain an extensive discussion of learning algorithms for approximating Nash equilibria, CEs and CCEs in the mean field limit.

\bigskip
In the $N$-player game, the state dynamics follow stochastic differential equations (SDEs), driven by independent standard Brownian motions representing additive idiosyncratic noise, and the interaction between players is given through the empirical distribution of their states, which appears both in the drift of the SDEs and in each player's payoff functional.
Players' strategies are assumed to be open loop, i.e., the correlation device would recommend the players to use strategies adapted to the filtration generated by noises and initial data. More precisely, the correlation device, or recommendation to the $N$ players, is modeled as a random variable taking values in the set of open loop strategy profiles; we require it to be independent of the random shocks and the initial states which determine players' states' evolution.
We deal very carefully with the measurability properties the recommendation has to fulfill so that the players' states are well-defined and the recommended strategies are implementable by the players.
In the mean field limit, the notion of coarse correlated solution we present corresponds to a pair given by a recommendation with values in the set of open loop strategies for the representative player and a random flow of measures fulfilling the following two properties: 
\begin{itemize}[label= --]
    \item \textit{Optimality}: the representative player has no incentive to deviate from the recommended strategy \emph{before} the extraction has happened.
    \item \textit{Consistency}: the flow of measures at any time $t$ equals the marginal law of the representative player's state conditioned on the $\sigma$-algebra generated by the whole flow of measures up to terminal time.
\end{itemize}
Through the study of a simple example, our notion of coarse correlated solution to the MFG is compared to the more usual notion of MFG solution, (as defined, e.g., in \cite{car_del_probabilistic}) and the notion of weak MFG solution of \cite{lacker2016}.
Our main contributions are as follows:
\begin{itemize}[label= --]

    \item We justify our notion of coarse correlated solution for the MFG by showing that any coarse correlated solution for the MFG induces a sequence of approximate CCEs in the $N$-player game, with vanishing error as $N$ goes to infinity.

    \item Under an additional convexity assumption, we prove the existence of a coarse correlated solution for the mean field game.

\end{itemize}
Both results will be established using a genuinely probabilistic approach.
To prove existence, we associate a zero-sum game to the search of a coarse correlated solution for the MFG, inspired by the works of Hart and Schmeilder \cite{hart_schmeidler} (for static games), Nowak \cite{nowak1992correlated,Nowak1993} (for continuous time dynamic games) and Bonesini \cite[Appendix 1.B]{bonesini_thesis} (for mean field games with discrete time and finite states and actions, in the setting of \cite{campifischer2021}), which require us to apply a minimax theorem.
Therefore, compactness arguments are exploited, adapting some of the techniques used in Lacker's works \cite{lacker2015martingale,lacker2020convergence}.
On the other hand, the approximation relies on propagation of chaos arguments, which are reminiscent of \cite{car_del_probabilistic}.

\bigskip
The rest of the paper is organised as follows: in Section \ref{sezione_notations_assumptions} we collect some notations and state the main assumptions, which will be in force throughout the whole paper.
In Section \ref{sezione_formulazione_N_giocatori}, we define the $N$-player game and present a notion of CCE; coarse correlated solutions for the MFG are defined in Section \ref{sezione_formulazione_mfg}.
The approximation and existence results are presented in Sections \ref{sezione_approssimazione} and \ref{sezione_existence}, respectively.
In Section \ref{sezione_esempio}, we consider a simple class of games, already discussed in the literature (see \cite{bardi_fischer2019,lacker2016,lacker2020convergence}): we show that it has coarse correlated solutions which are different from the classical MFG solution, and we compare them with the notions of solution of the previous literature.
Finally, some auxiliary technical results are gathered in the Appendix.