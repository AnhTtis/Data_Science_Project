

Taking inspiration from the work of Bardi and Fischer \cite{bardi_fischer2019} and Lacker's papers \cite{lacker2016,lacker2020convergence}, we provide a simple example of a mean field game possessing mean field CCEs with non-deterministic flow of measures $\mu$.

\medskip
Let $d=1$.
Set $A=[a,b]$, with $a < 0 < b$, and $\nu=\delta_0$.
For $m \in \probmeasures{\R}$, denote by $\overline{m}$ its mean $\int_{\R} y m(dy)$.
Consider the following coefficients and cost functions:
\begin{equation*}
    \begin{aligned}
    & b(t,x,m,a)=a, && f(t,x,m,a)=0, &&& g(x,m)=cx\overline{m},
    \end{aligned}
\end{equation*}
with $c>0$ a positive constant.
Observe that they satisfy the requirements of Assumption \ref{standing_assumptions}.
We want to find a coarse correlated solution for the mean field game whose payoff functional, to be maximized, is given by
\begin{equation}\label{esempio_payoff}
    \J(\Lambda,\mu)=\attesa{cX_T \overline{\mu}_T},
\end{equation}
under the constraint
\begin{equation}\label{esempio_dinamiche}
    X_t = \int_0^t \lambda_s ds + W_t, \quad 0\leq t \leq T,
\end{equation}
where $\lambda$ is the strategy associated to an admissible recommendation $\Lambda$ in the sense of \eqref{mf:controllo_indotto}.

\medskip
Set $\Omega^0=\{1,2\}^2$, $\F^{0-}=2^{\Omega^0}$ the power set and, given some probability measure $\prob^0 \in \mathcal{P}(\Omega^0,\F^{0-})$, we set $\prob^0((i,j))=p_{i,j}$, so that $p_{i,j}\geq 0$ for all $i,j$ and $\sum_{i,j=1}^2 p_{i,j}=1$.
Consider the following open loop strategies and flows of measures:
\begin{equation}
    \begin{aligned}
        u^+_t(\omega_*)\equiv b, \qquad \mu^+ = (\prob^* \circ (tb + W_t)^{-1})_{t \in [0,T]}; \\
        u^-_t(\omega_*)\equiv a, \qquad \mu^- = (\prob^* \circ (ta + W_t)^{-1})_{t \in [0,T]}.
    \end{aligned}
\end{equation}
It was shown in \cite{bardi_fischer2019} that the pairs $(u^+,\mu^+)$ and $(u^-,\mu^-)$ are two non-equivalent open-loop solutions of the mean field game, with initial distribution $\nu=\delta_{0}$, where by ``non-equivalent'' we mean that the flows of measures $\mu^+$ and $\mu^-$ do not coincide.
We point out that this result holds for more general initial distributions $\nu\in\mathcal{P}(\R)$, see \cite[Definition 3.1 and Theorem 3.1]{bardi_fischer2019}.
Choose $a_1,a_2 \in [0,1]$, and set:
\begin{equation}\label{esempio:def_strategie_flussi}
    \begin{aligned}
        \mu^1=\tonde{a_1 \mu^+_t + (1-a_1)\mu^-_t}_{t \in [0,T]}, \\
        \mu^2=\tonde{a_2 \mu^+_t + (1-a_2)\mu^-_t}_{t \in [0,T]}.
    \end{aligned}
\end{equation}
Define $(\Lambda,\mu)$ in the following way:
\begin{equation}\label{esempio_corr_sol}
    (\Lambda,\mu)\tonde{(i,j)}=\begin{cases}
    (u^+,\mu^1) \quad (i,j)=\tonde{1,1} \\
    (u^+,\mu^2) \quad (i,j)=\tonde{1,2} \\
    (u^-,\mu^1) \quad (i,j)=\tonde{2,1} \\
    (u^-,\mu^2) \quad (i,j)=\tonde{2,2}
    \end{cases}
\end{equation}
We claim that, as long as $a<0<b$, for every $T,c>0$ there exists a probability measure $(p_{i,j})_{i,j=1,2}$ and a suitable choice of the parameters $(a_i)_{i=1,2}$ so that the tuple $((\Omega^0,\F^{0-},\prob^0),\Lambda,\mu)$ is a coarse correlated solution of the mean field game according to Definition \ref{def_mean_field_sol}.

\medskip
First of all, as shown in Example \ref{mf:example:admissible_recommendations} in Section \ref{sezione_formulazione_mfg}, since $\Lambda$ takes only two values, it is admissible.
Therefore, the tuple $((\Omega^0,\F^{0-},\prob^0),\Lambda,\mu)$ is a correlated flow.

\medskip
Let us begin with the consistency condition.
We first observe that, when the state equation is controlled by $u^+$ (respectively, $u^-$), the law of the state process at time $t$, $X_t$, is exactly $\mu^+_t$ (respectively, $\mu^-_t$), for every time $t \in [0,T]$.

Suppose that $p_{1,1}+p_{2,1}$ and $p_{1,2}+p_{2,2}$ are both strictly positive.
Then, observe that
\begin{equation*}
    \prob(X_t \in \cdot \; \vert \; \mu)(\omega)=\prob(X_t \in \cdot \; \vert \; \mu)(\omega_0)=\begin{cases}
    \prob(X_t \in \cdot \; \vert \;  \mu=\mu^1) \quad   \text{if } \mu(\omega_0)=\mu^1, \\
    \prob(X_t \in \cdot \; \vert \;  \mu=\mu^2) \quad  \text{if } \mu(\omega_0)=\mu^2.
    \end{cases}
\end{equation*}
We can compute explicitly such a conditional probability. Fix $A\in \boreliani{\R^d}$:
\begin{equation*}
    \begin{cases}
    \prob(X_t \in A \; \vert \; \mu=\mu^1) \\
    \prob(X_t \in A \; \vert \; \mu=\mu^2)
    \end{cases}  = \quad \begin{cases}
    \frac{p_{1,1}}{p_{1,1}+p_{2,1}}\prob(X^+_t \in A) +  \frac{p_{2,1}}{p_{1,1}+p_{2,1}}\prob(X^-_t \in A) \quad \text{if } \mu(\omega_0)=\mu^1, \\
    \frac{p_{1,2}}{p_{1,2}+p_{2,2}}\prob(X^+_t \in A) +  \frac{p_{2,2}}{p_{1,2}+p_{2,2}}\prob(X^-_t \in A) \quad  \text{if } \mu(\omega_0)=\mu^2.
    \end{cases}
\end{equation*}
In order to satisfy the consistency condition, it must hold
\begin{equation*}
    \begin{cases}
    \frac{p_{1,1}}{p_{1,1}+p_{2,1}}\prob(X^+_t \in A) +  \frac{p_{2,1}}{p_{1,1}+p_{2,1}}\prob(X^-_t \in A)=\mu^1_t(A) \\
    \frac{p_{1,2}}{p_{1,2}+p_{2,2}}\prob(X^+_t \in A) +  \frac{p_{2,2}}{p_{1,2}+p_{2,2}}\prob(X^-_t \in A) = \mu^2_t(A)
    \end{cases}
\end{equation*}
for every $A\in \boreliani{\R^d}$.
By definition of $\mu^1$ and $\mu^2$,
\begin{equation*}
    \begin{cases}
    \frac{p_{1,1}}{p_{1,1}+p_{2,1}}\mu^+_t(A) +  \frac{p_{2,1}}{p_{1,1}+p_{2,1}}\mu^-_t(A)=a_1\mu^+_t(A) + (1-a_1)\mu^-_t(A), \\
    \frac{p_{1,2}}{p_{1,2}+p_{2,2}}\mu^+_t(A) +  \frac{p_{2,2}}{p_{1,2}+p_{2,2}}\mu^-_t(A) = a_2 \mu^+_t(A) + (1-a_2)\mu^-_t(A)
    \end{cases}
\end{equation*}
which holds if and only if
\begin{equation}\label{esempio_condizione_flussi_misure}
    \begin{cases}
    \frac{p_{1,1}}{p_{1,1}+p_{2,1}} = a_1, \\
    \frac{p_{1,2}}{p_{1,2}+p_{2,2}} = a_2.
    \end{cases}
\end{equation}
We can regard \eqref{esempio_condizione_flussi_misure} as the consistency condition.

\medskip
We now turn our attention to the optimality condition.
Set $\gamma = \prob \circ (\Lambda,\mu)^{-1} = \prob^0 \circ (\Lambda,\mu)^{-1}$ and $\rho=\prob\circ\mu^{-1}=\prob^0\circ\mu^{-1}$.
As described in Remark \ref{mf:remark_condizione_integrale}, since $\Lambda$ takes only two values, we can  rewrite the optimality condition using disintegration of measures as
\begin{equation*}
    \int_{\A\times\pwassspace{2}{\contrd}}  \J(\alpha,m) \gamma(d\alpha,dm) \geq \int_{\pwassspace{2}{\contrd}} \J(\beta,m) \rho(dm)
\end{equation*}
under the constraint
\begin{equation*}
\begin{aligned}
    & X_t =  \int_0^t \theta_s ds + W_t, \quad 0\leq t \leq T,
\end{aligned}
\end{equation*}
for $\theta=\alpha$ in $\Lambda(\Omega^0):=\{u^+,u^-\} \subseteq \A$ on the left-hand side of the inequality above and $\theta=\beta$ in $ \A$ on the right-hand side. We rewrite explicitly the inequality as
\begin{equation}\label{esempio_funzionale2}
    \begin{aligned}
    \J(\Lambda,\mu) - \J(\beta,\mu) & =  p_{1,1}\tonde{\J(u^+,\mu^1) - \J(\beta,\mu^1)}  + p_{1,2}\tonde{\J(u^+,\mu^2) - \J(\beta,\mu^2)} \\ 
    & + p_{2,1}\tonde{\J(u^-,\mu^1) - \J(\beta,\mu^1)} + p_{2,2}\tonde{\J(u^-,\mu^2) - \J(\beta,\mu^2)} \geq 0.
    \end{aligned}
\end{equation}
Therefore, using \eqref{esempio_payoff}, we have
\begin{equation*}
    \begin{aligned}
    \J(\Lambda,\mu) - \J(\beta,\mu) & = p_{1,1}\tonde{ cT^2b(a_1b + (1-a_1)a)  - c M(\beta)T(a_1b + (1-a_1)a)}  \\ 
    & +   p_{1,2}\tonde{ cT^2 b(a_2b + (1-a_2)a)  - c M(\beta)T(a_2b + (1-a_2)a)} \\
    & + p_{2,1}\tonde{ cT^2 a(a_1b + (1-a_1)a)  - c M(\beta)T(a_1b + (1-a_1)a)} \\
    & +  p_{2,2}\tonde{ cT^2a(a_2b + (1-a_2)a)  - c M(\beta)T(a_2b + (1-a_2)a)},
    \end{aligned}
\end{equation*}
where $M(\beta):=\E[\int_0^T\beta_t dt]=\E[X^\beta_T]$.
We can set $m(\beta):=\sfrac{1}{T}M(\beta)=\sfrac{1}{T}\E[\int_0^T\beta_t dt]$. Observe that $m(\beta) \in [a,b]$, being the mean of an $A$-valued process, and $m(\A)=[a,b]$, since for every $c \in [a,b]$ the constant process $\beta\equiv c$ belongs to $\A$.
We divide by $cT^2$ to obtain the following condition:
\begin{equation}\label{esempio_funzionale3}
    \begin{aligned}
    p_{1,1} & \tonde{ b(a_1b + (1-a_1)a)  - m(\beta)(a_1b + (1-a_1)a)}  \\ 
    &+  p_{1,2}\tonde{ b(a_2b + (1-a_2)a)  - m(\beta)(a_2b + (1-a_2)a)} \\
    &+ p_{2,1}\tonde{a(a_1b + (1-a_1)a)  - m(\beta) (a_1b + (1-a_1)a)} \\
    &+  p_{2,2}\tonde{ a(a_2b + (1-a_2)a)  - m(\beta) (a_2b + (1-a_2)a)} \\
    & \geq 0.
    \end{aligned}
\end{equation}
The condition above can be seen as a positivity condition for a real affine function of $g(m)$, $m \in [a,b]$, i.e.
\begin{equation}\label{esempio_condizione_inf}
    \begin{aligned}
        & \inf_{m \in [a,b]} g(m)=\inf_{m \in [a,b]} h((p_{i,j})_{i,j=1,2} , (a_i)_{i=1,2};a,b) m + k((p_{i,j})_{i,j=1,2} , (a_i)_{i=1,2};a,b) \geq 0 \\
        & \begin{cases}
        h((p_{i,j})_{i,j=1,2} , (a_i)_{i=1,2};a,b)= & - \left\{p_{1,1}(a_1b+(1-a_1)a) + p_{1,2}(a_2 b+(1-a_2)a) \right.\\ 
        & \left. + p_{2,1}(a_1b+(1-a_1)a) + p_{2,2}(a_2b+(1-a_2)a)  \right\}, \\
        k((p_{i,j})_{i,j=1,2} , (a_i)_{i=1,2};a,b) =  & p_{1,1}b(a_1b+(1-a_1)a) + p_{1,2}b(a_2 b+(1-a_2)a)\\ 
        & + p_{2,1}a(a_1b+(1-a_1)a) + p_{2,2}a(a_2b+(1-a_2)a) . \\
        \end{cases}
    \end{aligned}
\end{equation}
We now impose the consistency condition \eqref{esempio_condizione_flussi_misure} to get:
\begin{equation}\label{esempio_coefficienti}
    \begin{aligned}
    h((p_{i,j})_{i,j=1,2} ;a,b)  & = - b \tonde{\frac{p_{1,1}^2 + p_{2,1}p_{1,1}}{p_{1,1} + p_{2,1}} + \frac{p_{1,2}^2 + p_{1,2}p_{2,2}}{p_{1,2} + p_{2,2}} }  \\
    &\quad  -a \tonde{\frac{p_{2,1}^2 + p_{2,1}p_{1,1}}{p_{1,1} + p_{2,1}} + \frac{p_{2,2}^2 + p_{1,2}p_{2,2}}{p_{1,2} + p_{2,2}} }, \\
    k((p_{i,j})_{i,j=1,2} ;a,b) & =  b^2 \tonde{\frac{p_{1,1}^2}{p_{1,1} + p_{2,1}} + \frac{p_{1,2}^2}{p_{1,2} + p_{2,2}} } +a^2 \tonde{\frac{p_{2,1}^2}{p_{1,1} + p_{2,1}} + \frac{p_{2,2}^2 }{p_{1,2} + p_{2,2}} } \\
    & \quad + 2ab\tonde{\frac{p_{1,1}p_{2,1}}{p_{1,1}+p_{2,1}} + \frac{p_{1,2}p_{2,2}}{p_{1,2}+p_{2,2}}}.
    \end{aligned}
\end{equation}
Observe that imposing the consistency condition \eqref{esempio_condizione_flussi_misure} reduces the number of parameters but makes the problem nonlinear in the probabilities $(p_{i,j})_{i,j=1,2}$.

\bigskip
Looking at \eqref{esempio_condizione_inf} and \eqref{esempio_coefficienti}, we observe that it covers the case treated in \cite{bardi_fischer2019}, for any choices of $a < 0 < b$.
Consider a probability measures $\prob^0=(p_{i,j})_{i,j=1,2}$ so that $p_{1,2}=p_{2,1}=0$ and, therefore, $p_{2,2}=1-p_{1,1}$.
Equations \eqref{esempio_coefficienti} take the simpler form
\begin{equation}\label{esempio_coefficienti_diagonale}
\begin{aligned}
    h((p_{1,1},0,0,1-p_{1,1}); a,b) & = - bp_{1,1} -a(1-p_{1,1}),  \\
    k((p_{1,1},0,0,1-p_{1,1}); a,b) &= b^2p_{1,1} +a^2(1-p_{1,1}).
\end{aligned}
\end{equation}
Sending $p_{1,1}$ to $1$, by continuity, we get
$h((1,0,0,0);a,b)=-b$ and $k((1,0,0,0);a,b)=b^2$, so that the condition \eqref{esempio_condizione_inf} becomes
\begin{equation*}
    \inf_{ m \in [a,b]} -bm + b^2 \geq 0,
\end{equation*}
which is satisfied for every $b>0$.
On the other hand, by sending $p_{1,1}$ to $0$, we get $h((0,0,0,1);a,b)=-a$ and $k((0,0,0,1);a,b)=a^2$ and condition \eqref{esempio_condizione_inf} takes the form
\begin{equation*}
    \inf_{m \in [a,b]} -am+a^2 \geq 0,
\end{equation*}
which is satisfied for every $a<0$.
Observe that, when $p_{1,1}=1$, $\mu\equiv\mu^1=\mu^+$, while, when $p_{2,2}=1$, $\mu\equiv\mu^2=\mu^-$.
This shows that the deterministic correlated flows $(\Lambda,\mu)\equiv (u^+,\mu^+)$ and $(\Lambda,\mu)\equiv (u^-,\mu^-)$ are indeed mean field CCE in the sense of Definition \ref{def_mean_field_sol}.

\medskip
Turning to more interesting cases, consider $[a,b]=[-1,1]$.
The choice of a symmetric interval is not necessary, but it has been made to ease the comparison with previous results in the literature (see the next subsection).
Figure \ref{fig:esempio_immagine} shows the existence of coarse correlated mean field equilibria as the probability measure $(p_{i,j})_{i,j=1,2}$ varies.
In particular, it shows the existence of infinitely many coarse correlated mean field equilibria for the system.


\begin{figure}
	\centering
	\includegraphics[width=\textwidth]{sections/esempio_immagine.jpg}
	\caption{Existence of correlated equilibria as probability measure $(p_{i,j})_{i,j=1,2}$ varies for $[a,b]=[-1,1]$.
	White points correspond to the values of $(p_{i,j})$ so that $\inf_{m\in[a,b]} g(m) \geq 0$, black points to the other ones.
	\\
	The probability measure has been generated in different ways above and below the dashed diagonal.
	Above the diagonal, $p_{1,1}$ and $p_{2,2}$ vary from $0$ to $1$ as indicated by the axis.
	We set $p_{1,2}=\alpha(1-p_{1,1}-p_{2,2})$, $p_{2,1}=(1-\alpha)(1-p_{1,1}-p_{2,2})$ for some value $\alpha \in [0,1]$.
	\\
	Under the  diagonal, we have a symmetric choice of probability: again, $p_{1,1}$ and $p_{2,2}$ vary from $0$ to $1$, although in the opposite directions respect to above the diagonal, and we set $p_{1,2}=(1-\alpha)(1-p_{1,1}-p_{2,2})$, $p_{2,1}=\alpha(1-p_{1,1}-p_{2,2})$, for the same choice of $\alpha \in [0,1]$ as for above the diagonal. Different choices of $\alpha$ are considered. \\
    } \label{fig:esempio_immagine}
\end{figure}


White spots in Figure \ref{fig:esempio_immagine} refer to those probability measures on $(\Omega^0,\F^{0-})$ so that $(\Lambda,\mu)$ is indeed a mean field CCE.
Observe that, on the dashed diagonals, it always holds $p_{1,1}+p_{2,2}=1$, which implies that the coarse correlated solution $(\Lambda,\mu)$ is a randomization of the open loop MFG solutions $(u^+,\mu^+)$ and $(u^-,\mu^-)$.
On the other hand, there exist infinitely many coarse correlated solutions of the mean field game so that $\Lambda$ is not a deterministic function of $\mu$, i.e., they are not a randomization of the solutions $(u^+,\mu^+)$ and $(u^-,\mu^-)$.


\subsection*{Comparison with Lacker's notion of weak mean field game solution without common noise of \cite{lacker2016}}

Consider $A=[-1,1]$, $T=2$.
With this choice of control actions and time horizon, the example we proposed matches the setting of Lacker's \say{illuminating example} of \cite[Section 3.3]{lacker2016}.
We show that there exists a coarse correlated solution of the MFG which is not a weak MFG solution without common noise as defined in Definition 3.1 therein.
In particular, the most important feature is the fact that the recommendation $\Lambda$ can not be expressed as a deterministic function of the flow of measures.

\medskip
To be consistent with the notation and the setup of Lacker's paper, we use the notion of relaxed controls, which are used extensively in Section \ref{sezione_existence} (see, in particular, Section \ref{existence:sezione_crtl_rilassati} for definitions, notation and some important properties).
Let $(p_{i,j})_{i,j=1,2}$ be so that $p_{1,1}+p_{2,1}$ and $p_{1,2}+p_{2,2}$ are strictly positive.
We introduce the relaxed controls $\delta^+=(\delta^+_t)_{t \in [0,T]}$ and $\delta^-=(\delta^-_t)_{t \in [0,T]}$, by setting
\begin{equation*}
\begin{aligned}
    & \delta^+_t(\omega_*;da)=\delta_{u^+_t(\omega_*)}(da) \equiv \delta_{1}(da), \quad \forall t \in [0,T], \omega_* \in \Omega^*,\\
    & \delta^-_t(\omega_*;da)=\delta_{u^-_t(\omega_*)}(da) \equiv \delta_{-1}(da), \quad \forall t \in [0,T], \omega_* \in \Omega^*.
\end{aligned}
\end{equation*}
Consider the correlated flow $(\Lambda,\mu)$ defined by \eqref{esempio_corr_sol}
and observe that the strategy $\lambda=(\lambda_t)_{t \in [0,T]}$ associated to the admissible recommendation $\Lambda$ can be rewritten as a relaxed control as
\begin{equation}\label{raccomandazione_come_lacker}
    \mathfrak{r}_t(\omega;da)= \mathfrak{r}_t(\omega_0,\omega_*;da)=\1_{\{\Lambda=u^+\}}(\omega_0)\delta^+_t(da) + \1_{\{\Lambda=u^-\}}(\omega_0)\delta^-_t(da).
\end{equation}
We point out that $\mathfrak{r}$ does not depend on $\omega_*$ since $\delta^+$ and $\delta^-$ do not depend on $\omega_*$.
Starting from $(\Lambda,\mu)$, we define a random variable $\Tilde{\mu}$ with values in $\mathcal{P}(\contrd \times \mathcal{V} \times \contrd)$ by setting
\begin{equation}\label{esempio_comparison_flusso}
    \Tilde{\mu}(\cdot)=\prob((W,\mathfrak{r},X) \in \cdot \; \vert \; \mu ).
\end{equation}
We observe that $\sigma(\mu)=\sigma(\Tilde{\mu})$: we have $\sigma(\Tilde{\mu}) \subseteq \sigma(\mu)$ since, by definition of regular conditional probability, $\Tilde{\mu}$ must be $\sigma(\mu)$ measurable; to get the opposite inclusion, for every $t \in [0,T]$, let $\Tilde{\mu}^x_t$ be the push forward of $\Tilde{\mu}$ through the map $\contrd \times \mathcal{V} \times \contrd \ni (w,q,x) \mapsto x_t \in \R^d$.
Then, by exploiting the consistency condition \eqref{def_mean_field_sol:cons}, we have 
\begin{equation*}
    \Tilde{\mu}^x_t(A)=\Tilde{\mu}(\{x \in \contrd: \; x_t \in A\})=\prob(X_t \in A \; \vert \; \mu ) = \mu_t(A),
\end{equation*}
for every $A \in \boreliani{\R^d}$, i.e. $\Tilde{\mu}^x_t=\mu_t$ $\prob$-a.s, for every $t \in [0,T]$.
Let $(\mathcal{B}_{t,\contrd})_{t \in [0,T]}$ be the natural filtration of the identity process on $\contrd$, i.e. $\mathcal{B}_{t,\contrd}=\sigma(\contrd \ni x \mapsto x_s \in \R^d: \; 0 \leq s \leq t)$, and let $(\F^{\Tilde{\mu}}_t)_{t \in [0,T]}$ be the natural filtration of $\Tilde{\mu}$, that is
\begin{equation*}
    \F^{\Tilde{\mu}}_t=\sigma(\Tilde{\mu}\tonde{C}: \; C \in \mathcal{B}_{t,\contrd}\otimes \F^{\mathcal{V}}_t \otimes \mathcal{B}_{t,\contrd}).
\end{equation*}
We observe that, for every $t \in (0,T]$, we have $\F^{\Tilde{\mu}}_t=\sigma(\mu)$.
To see this, observe that
\begin{equation*}
    \sigma(\mu) \supseteq \F^{\Tilde{\mu}}_t \supseteq \sigma(\Tilde{\mu}^x_s: \; s \leq t) = \sigma(\mu_s: \; s \leq t) = \sigma(\mu),
\end{equation*}
where the last equality holds for every $t>0$, as can be verified by explicit calculations.
Finally, for $t=0$, we have $\F^{\Tilde{\mu}}_0=\{\emptyset,\Omega^0\}$.
Having established the relations between such $\sigma$-algebras, it is straighforward to verify that the tuple $((\Omega,\F,\mathbb{F},\prob),W,\Tilde{\mu},\mathfrak{r},X)$ satisfies properties (1-4) and (6) of \cite[Definition 3.1]{lacker2016}.
Now, pick a probability measure $\prob^0$ so that $\min(p_{1,2},p_{2,1}) > 0$ and $\overline{\mu}^1_T>0$, $\overline{\mu}^2_T<0$.
Figure \ref{fig:esempio_immagine} shows that such a choice is possible (actually, there exist infinitely many measures $\prob^0$ with the desired property).
For such a choice of $\prob^0$, the relaxed control $\mathfrak{r}$ does not satisfy the optimality condition (5) of \cite[Definition 3.1]{lacker2016}, since, as shown in \cite[Section 3.3]{lacker2016}, every optimal control $\mathfrak{r}^*$ must be of the form $\mathfrak{r}^*_t(da)(\omega)=\delta_{\alpha^*_t(\omega)}(da)$ for $Leb_{[0,T]}$-a.e. $t$, with
\begin{equation*}
    \alpha^*_t = \text{sign}\tonde{\attesa{\Tilde{\mu}^x_T \; \vert \; \F^{\Tilde{\mu}}_t}}.
\end{equation*}
Here, $\text{sign}\tonde{0}=0$.
Since $\F^{\Tilde{\mu}}_0$ is trivial and $\F^{\Tilde{\mu}}_t=\sigma(\mu)$ for $t > 0$, the optimal control $\alpha^*_t$ must be equal to
\begin{equation}\label{esempio_comparison_ctrl_ottimo}
    \alpha^*_t(\omega)=\alpha^*_t(\omega_0)=\begin{cases}
    -1 \quad \text{ if } \overline{\mu}_T(\omega_0)<0, \\
    1 \qquad \text{ if } \overline{\mu}_T(\omega_0)>0,
    \end{cases} \quad 0 < t \leq T,
\end{equation}
and equal to an arbitrary value at $t=0$.
In particular, observe that such a control is a deterministic function of the measure $\Tilde{\mu}$.
For every $\prob^0$ so that $p_{1,2}+p_{2,1}>0$, this is not the case of the correlated flow $(\Lambda,\mu)$ defined in \eqref{esempio_corr_sol}, since $\Lambda$ is not a deterministic function of $\mu$.

\medskip
The essential reason for the lack of optimality, in the sense of Lacker, of the relaxed control $\mathfrak{r}$ defined by \eqref{raccomandazione_come_lacker} resides in the differences between allowed deviations: on one hand, for weak mean field games solutions in the sense of \cite{lacker2016}, all adapted compatible controls $\mathfrak{b}=(\mathfrak{b}_t)_{t \in [0,T]}$ are allowed, where ``compatible'' means that $\sigma(\mathfrak{b}_s: s \leq t)$ is conditionally independent of $\F^{\xi,\Tilde{\mu},W}_T$ given $\F^{\xi,\Tilde{\mu},W}_t$ for every $t$, which leads to a very rich class of controls.
On the other hand, for coarse correlated solution of the MFG, only $\mathbb{F}^*$-progressively measurable strategies are allowed as deviations.
Therefore, many more solutions exist.

\medskip
More generally, one can not compare weak MFG solutions without common noise of \cite{lacker2016} and mean field CCEs, due to the difference between the respective consistency conditions.
Nevertheless, we can make an additional assumption on the random measure $\Tilde{\mu}$ which makes it possible to define a mean field CCE starting from a weak MFG solution.
Let $\Tilde{\mu}$ be a weak MFG solution without common noise.
Let $\mu_t$ be the push forward of $\Tilde{\mu}$ through the map $\contrd \times \mathcal{V} \times \contrd \ni (w,q,x) \mapsto x_t \in \R^d$.
Define a random flow of measures by setting $\mu=(\mu_t)_{t \in [0,T]}$.
Assume that the flow of measures $\mu$ carries the same information as the random measure $\Tilde{\mu}$, i.e.
\begin{equation}\label{esempio:hp_filtrazioni}
    \sigma(\mu_s: \; 0 \leq s \leq t)=\F^{\Tilde{\mu}}_t, \quad \forall t \in [0,T].
\end{equation}
If a weak MFG solution $\Tilde{\mu}$ satisfies condition \eqref{esempio:hp_filtrazioni}, then $\Tilde{\mu}$ does induce a mean field CCE.
Indeed, set $\rho=\prob\circ\mu^{-1}$.
By \eqref{esempio:hp_filtrazioni}, we have $\mu_t=\prob(X \in \cdot \; \; \vert \; \Tilde{\mu}) = \prob(X \in \cdot \; \; \vert \; \mu)$, i.e. consistency condition \eqref{def_mean_field_sol:cons} is satisfied.
Moreover, the assumption on equality of the filtrations ensures that there exists a progressively measurable function $\varphi:[0,T]\times\contpdue \to A$ so that
\begin{equation*}
    \alpha^*_t=\text{sign}\tonde{\E\quadre{\mu_T \; \vert \; \F^{\Tilde{\mu}}_t}}=\varphi\tonde{t,\mu}.
\end{equation*}
Then, we define $(\Omega^0,\F^{0-},\prob^0)$ and $(\Lambda^*,\mu^*)$ as
\begin{equation}
    \begin{aligned}
        & (\Omega^0,\F^{0-},\prob^0)=\tonde{\contpdue,\boreliani{\contpdue},\rho}, \\
        & \mu^*=\text{Id}:\tonde{\contpdue,\boreliani{\contpdue},\rho} \to \tonde{\contpdue,\boreliani{\contpdue},\rho}, \\
        & \begin{aligned}
        \Lambda^*: \tonde{\contpdue,\boreliani{\contpdue},\rho} & \to (\A,\boreliani{\A}) \\
        m & \mapsto \Lambda^*(m)=(\varphi(t,m))_{t \in [0,T]}.
        \end{aligned}
    \end{aligned}
\end{equation}
By Lemma \ref{esempi:lemma_misurabile}, the tuple $((\Omega^0,\F^{0-},\prob^0),\Lambda^*,\mu^*)$ is a correlated flow.
Let $X^*$ be the solution of \eqref{esempio_dinamiche} on the product probability space $(\Omega,\F,\mathbb{F},\prob)$ defined in point \ref{mf:condizione_ammissibilita} of Definition \ref{mf:admissible_recommendation}.
Since uniqueness in law holds by Theorem \ref{teorema_di_unicita_legge}, it follows that $(X^*,\mu^*)$ has the same joint law as $(X,\mu)$, which implies that the consistency condition \eqref{def_mean_field_sol:cons} is satisfied.
Since $\lambda^*_t=\varphi(t,\mu^*)$, $(\Lambda^*,\mu^*)$ satisfies optimality condition \eqref{def_mean_field_sol:opt} as well and therefore it is a mean field CCE.


We observe that the additional assumption on the filtrations \eqref{esempio:hp_filtrazioni} is satisfied both by the weak MFG solution exhibited in \cite[Proposition 3.7]{lacker2016} and in our case, as shown above.
We point out that this CCE has been already considered: suppose that the flow of measures as law $\rho=a\delta_{\mu^+} + (1-a)\delta_{\mu^-}$, $a \in (0,1)$, for $\mu^+$ and $\mu^-$ given by \eqref{esempio:def_strategie_flussi}.
Then, the correlated flow $(\Lambda^*,\mu^*)$ corresponds to the white spots on the dashed diagonal of Figure \ref{fig:esempio_immagine}, i.e. to the probability measures $\prob^0$ so that $p_{1,2}=p_{2,1}=0$, $p_{1,1}=a$ and $p_{2,2}=1-a$.
Roughly speaking, it correspond to the case when $\Lambda^*=\phi(\mu^*)$ $\prob^0$-a.s., for some deterministic measurable $\phi$.