


Here, we collect the most frequent notations that occur in this work and state the assumptions.

For a metric space $(E,d_E)$, we denote by $\boreliani{E}$ the Borel $\sigma$-algebra generated by the topology of $E$. When the context allows, we will drop the dependence upon $E$, and just denote it by $\mathcal{B}$.
We denote by $\cbounded{E}$ the set of continuous bounded function $f:E \to \R$.

We will denote by $\mathcal{P}(E)$ the set of probability measures on $(E,\boreliani{E})$.
For $p \geq 1$, we denote by $\mathcal{P}^p(E)$ the set of probability measures $m \in \mathcal{P}(E)$ so that, for some point $x_0 \in E$, and thus for any, the $p$-moment of $m$ is finite:
\begin{equation*}
    m \in \mathcal{P}^p(E) \iff m \in \mathcal{P}(E) \text{ and } \int_E d^p_E(x,x_0)m(dx) < \infty.
\end{equation*}
Let $\mathcal{W}_{p,E}(m_1,m_2)$ denote the $p$-Wasserstein distance on $\mathcal{P}^p(E)$, defined as
\begin{equation*}
    \pwassmetric{p}{E}{p}(m_1,m_2)=\inf\graffe{ \int_{E \times E} d^p_E(x,y)\pi(dx,dy):\; \pi \in \mathcal{P}(E\times E), \text{ $\pi$ has marginals $m_1$, $m_2$}}.
\end{equation*}
Any time we will be given two metric spaces $(E,d_E)$ and $(E,d_{E'})$, we will regard $E \times E'$ as a metric space itself, with the distance $d((e,f),(e',f'))=d_E(e,f) + d_E(e',f')$.
The $p$-Wasserstein distance on $\mathcal{P}^{p}(E \times E')$ will always be meant with respect to such distance on $E \times E'$.

We recall some well-known equivalences for convergence in $p$-Wasserstein distance, which will be used extensively:
\begin{prop}[\cite{villani2003}, Theorem 7.12]\label{wass:equivalenze_convergenza}
Let $(E,d_E)$ be a metric space, and suppose $\mu^n,\mu \in \mathcal{P}^p(E)$. Then the following are equivalent:
\begin{enumerate}[label=(\arabic*)]
    \item $\pwassmetric{p}{E}{}(\mu^n,\mu)\to 0$.
    \item $\mu^n \to \mu$ weakly and for some (and thus any) $x_0 \in E$ we have
    \begin{equation}\label{wass:uniforme_integrabilita}
    \lim_{r \to \infty}\sup_n\int_{\insieme{x:\;d_E^p(x,x_0)\geq r}}d_E^p(x,x_0)\mu_n(dx)=0
    \end{equation}
    \item $\int\phi(x)\mu_n(dx) \to \int\phi(x) \mu(dx)$ for all continuous functions $\phi : E \to \R$ such that there exist $x_0 \in E$ and $c > 0$ so that $\vert \phi(x)\vert \leq c(1+d_E^p(x,x_0))$ for all $x \in E$.
\end{enumerate}
In particular, a sequence $(\mu_n)_n \subset \mathcal{P}^p(E)$ is relatively compact if and only if it is tight and satisfies \eqref{wass:uniforme_integrabilita}.
\end{prop}
For $T>0$ fixed, we denote by $\contrd$ the set of continuous functions from $[0,T]$ in $\R^d$, $d \in \N$, i.e. $\contrd=\conttraj{\R^d}$. We endow $\contrd$ with the norm $\lVert x \rVert_{\contrd}=\sup_{ s \in [0,T]}\vert x_s \vert $.
Occasionally, we will use the semi-norm $\lVert x \rVert_{t,\contrd}=\sup_{s \in [0,t]}\vert x_s \vert$, for $x \in \contrd$.
We will denote as $\wienermeasure \in \mathcal{P}(\contrd)$ the law of a standard $d$-dimensional Brownian motion, and by $\contpdue$ the set of continuous functions from $[0,T]$ in $\mathcal{P}^2(\R^d)$, i.e. $\contpdue=\mathcal{C}([0,T];\mathcal{P}^2(\R^d))$, where $\mathcal{P}^2(\R^d)$ is endowed with the $2$-Wasserstein distance.
We endow $\contpdue$ with the supremum distance $\sup_{t \in [0,T]}\pwassmetric{2}{\mathcal{P}^2(\R^d)}{}(m^1_t,m^2_t)$, for any $m^1=(m^1_t)_{t \in [0,T]}$ and $m^2=(m^1_t)_{t \in [0,T]}$ in $\contpdue$.

When given a filtered probability space $(\Omega,\F,(\mathcal{G}_t)_t,\prob)$, we regard as the $\prob$-augmentation of the filtration $(\mathcal{G}_t)_t$ the filtration $\mathbb{F}=(\F_t)_t$, where $\F_t=\cap_{\eps > 0} \sigma(\mathcal{G}_{t + \eps},\mathcal{N})$ and $\mathcal{N}$ stands for the $\prob$-null sets of $\Omega$. Such a filtration satisfies the usual assumptions.

\medskip
We end this section by stating our standing assumptions on the state dynamics and on the costs of the players in both the $N$-player game and the limit game.
We are given a finite time horizon $T >0$, a control actions space $A$, an initial state distribution $\nu \in \probmeasures{\R^d}$, and the following functions:
\begin{equation*}
\begin{aligned}
    (b,f):[0,T]\times\R^d\times\mathcal{P}^2(\R^d)\times A\to\R^d, \\
    g:\R^d\times\mathcal{P}^2(\R^d)\to\R^d,
\end{aligned}
\end{equation*}
which will be referred to, respectively, as the drift function, the running cost and the terminal cost.

The following Assumptions \ref{standing_assumptions} will be in force throughout the whole manuscript.

\begin{customassumptions}{\textbf{A}}\label{standing_assumptions}
\begin{enumerate}[label=\normalfont(A.\arabic*)]
    \item[]
    \item $A\subseteq \R^l$, for some $l \geq 1$, is a compact set.
    \item $\nu \in \mathcal{P}^{\overline{p}}(\R^d)$, for some $\overline{p} > 4$.
    \item The functions $b$, $f$ and $g$ are jointly measurable in $(t,x,m,a)$.
    \item $b(t,x,m,a)$ is Lipschitz in $a \in A$, $m \in \mathcal{P}^2(\R^d)$ and $x \in \R^d$, uniformly in $t$:
    \begin{equation*}
        \vert b(t,x,m,a)-b(t,x',m',a') \vert \leq L \tonde{\vert a-a' \vert  + \vert x-x' \vert  + \pwassmetric{2}{\R^d}{}(m,m')}
    \end{equation*}
    for every $t \in [0,T]$, $(x,m,a)$ and $(x',m',a')$ in $\R^d \times \mathcal{P}^2(\R^d)\times A$.
    \item The functions $[0,T] \ni t \mapsto (b,f)(t,0,\delta_0,a_0)$ are bounded, for some $a_0 \in A$.
    \item $f$ and $g$ are locally Lipschitz in $(x,m,a)$ for every fixed $t \in [0,T]$ with at most quadratic growth, i.e., there exists a positive constant $L>0$ so that
    \begin{equation*}
    \begin{aligned}
        \big\vert (f,g) & (t,x,m,a)-(f,g)(t,x',m',a') \big\vert  \\
        \leq & \, L\left( 1 + \abs{x}+\abs{x'} + \tonde{\int_{\R^d} \abs{y}^2 m(dy)}^\frac{1}{2} + \tonde{\int_{\R^d} \abs{y}^2 m'(dy)}^\frac{1}{2} +\abs{a}+\abs{a'}\right) \\
        & \cdot \tonde{\vert x-x' \vert +\pwassmetric{2}{\R^d}{}(m,m')+\vert a-a' \vert },
    \end{aligned}
    \end{equation*}
    for every $t \in [0,T]$, $(x,m,a)$ and $(x',m',a')$ in $\R^d \times \mathcal{P}^2(\R^d)\times A$.
\end{enumerate}
\end{customassumptions}

