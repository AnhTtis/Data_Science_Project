\documentclass[a4paper,11pt,oneside]{article}


\usepackage[utf8]{inputenc}
\usepackage[T1]{fontenc}
\usepackage[english]{babel}

\usepackage{answers}
\usepackage{setspace}
\usepackage{graphicx}
\usepackage{enumitem}
\usepackage{multicol}
\usepackage{mathrsfs}
\usepackage{amsmath,amsthm,amssymb}
\usepackage{bbm}
\usepackage{esdiff}
\usepackage{braket}
\usepackage{tikz}
\usepackage{dsfont}
\usepackage{hyperref}
\usepackage{bm}
\usepackage{comment}
\usepackage{dirtytalk}
\usepackage{mathtools,tikz-cd}

\usepackage{color,soul}

\usepackage{cleveref}

\usepackage{nicefrac, xfrac}

\newtheorem{innercustomgeneric}{\customgenericname}
\providecommand{\customgenericname}{}
\newcommand{\newcustomtheorem}[2]{%
  \newenvironment{#1}[1]
  {%
   \renewcommand\customgenericname{#2}%
   \renewcommand\theinnercustomgeneric{##1}%
   \innercustomgeneric
  }
  {\endinnercustomgeneric}
}

\newcustomtheorem{customassumptions}{Assumptions}
\newcustomtheorem{customassumption}{Assumption}

\makeatletter
\newtheorem*{rep@theorem}{\rep@title}
\newcommand{\newreptheorem}[2]{%
\newenvironment{rep#1}[1]{%
 \def\rep@title{#2 \ref{##1}}%
 \begin{rep@theorem}}%
 {\end{rep@theorem}}}
\makeatother


\newreptheorem{theorem}{Theorem}
\newreptheorem{lemma}{Lemma}



\usepackage[toc,page]{appendix}

\newcommand{\nocontentsline}[3]{}
\newcommand{\tocless}[2]{\bgroup\let\addcontentsline=\nocontentsline#1{#2}\egroup}


\usepackage[a4paper,top=3cm,bottom=3cm,left=2.5cm,right=2.5cm]{geometry}
\setlength{\parindent}{1em}


\numberwithin{equation}{section}


\newcommand{\de}{\mathrm{d}}
\newcommand{\N}{\mathbb{N}}
\newcommand{\Z}{\mathbb{Z}}
\newcommand{\C}{\mathbb{C}}
\newcommand{\R}{\mathbb{R}}
\newcommand{\Q}{\mathbb{Q}}
\newcommand{\E}{\mathbb{E}}


\newcommand{\attesa}[1]{\mathbb{E}\quadre{{#1}}}
\newcommand{\J}{\mathfrak{J}}
\newcommand{\A}{\mathbb{A}}
\newcommand{\F}{\mathcal{F}}
\newcommand{\prob}{\mathbb{P}}
\newcommand{\wienermeasure}{\mathbb{W}^d}


\newcommand{\legge}[1]{\mathcal{L}\tonde{{#1}}}
\newcommand{\probmeasures}[1]{\mathcal{P}(#1)}
\newcommand{\contrd}{\mathcal{C}^d}
\newcommand{\conttraj}[1]{\mathcal{C}\tonde{[0,T];#1}}
\newcommand{\cbounded}[1]{\mathcal{C}_b(#1)}

\newcommand{\contpdue}{\mathcal{C}(\mathcal{P}^2)}


\newcommand{\boreliani}[1]{\mathcal{B}_{#1}}
\newcommand{\pwassspace}[2]{
\mathcal{P}^{#1}\tonde{#2}}
\newcommand{\pwassmetric}[3]{
\mathcal{W}_{#1,#2}^{#3}}


\newcommand{\1}{\mathds{1}}
\newcommand{\tonde}[1]{\left({#1}\right)}
\newcommand{\quadre}[1]{\left[{#1}\right]}
\newcommand{\graffe}[1]{\left\lbrace{#1}\right\rbrace}
\newcommand{\abs}[1]{\left\lvert{#1}\right\rvert}
\newcommand{\norm}[1]{\left\lVert{#1}\right\rVert}
\newcommand{\wto}{\rightharpoonup}
\newcommand{\eps}{\varepsilon}

\newcommand{\insieme}[1]{\left\lbrace{#1}\right\rbrace}
\newcommand{\Span}[1]{\overline{\mathrm{span}}\left\lbrace{#1}\right\rbrace}


\DeclareMathOperator{\sech}{sech}
\DeclareMathOperator{\csch}{csch}
\DeclareMathOperator{\supp}{supp}
\DeclareMathOperator{\diver}{div}
\DeclareMathOperator{\rot}{rot}

\theoremstyle{plain}
\newtheorem{thm}{Theorem}[section]
\newtheorem{lemma}[thm]{Lemma}
\newtheorem{prop}[thm]{Proposition}
\newtheorem{corollary}{Corollary}[thm]
\newtheorem{claim}{Claim}
\newtheorem{hp}{Assumption}
\newtheorem{notazione}{Notation}

\theoremstyle{definition}
\newtheorem{definition}{Definition}
\newtheorem{example}{Example}
\newtheorem{problema}{Problem}[section]
\newtheorem{remark}{Remark}


\title{Coarse correlated equilibria for continuous time mean field games in open loop strategies}

\author{Luciano Campi\thanks{Department of Mathematics "Federigo Enriques", University of Milan, Via Saldini 50, 20133, Milan, Italy.} \and Federico Cannerozzi\footnotemark[1] \and Markus Fischer\thanks{Department of Mathematics "Tullio Levi-Civita", University of Padua, via Trieste 63, 35121, Padova, Italy.}}

\date{\today}

\begin{document}

\maketitle

\begin{abstract}
In the framework of continuous time symmetric stochastic differential games in open loop strategies, we introduce a generalization of mean field game solution, called coarse correlated solution. This can be seen as the analogue of a coarse correlated equilibrium in the $N$-player game, where a moderator randomly generates a strategy profile and asks the players to pre-commit to such strategies before disclosing them privately to each one of them; such a profile is a coarse correlated equilibrium if no player has an incentive to unilaterally deviate. We justify our definition by showing that a coarse correlated solution for the mean field game induces a sequence of approximate coarse correlated equilibria with vanishing error for the underlying $N$-player games. Existence of coarse correlated solutions for the mean field game is proved by means of a minimax theorem. An example with explicit solutions is discussed as well.
\end{abstract}


\tableofcontents


\section{Introduction}
\section{Introduction}
\label{sec:introduction}
% \begin{itemize}
%     % Diffusion of FL
%     \item {\st{Diffusion of FL}}
%     % Security threats to FL
%     \item {\st{Security threats to FL with particular focus on model poisoning}}
%     % Limitations of existing countermeasures
%     \item {\st{Current countermeasures (e.g., KRUM) and their limitations}}
%     % Proposed method and its advantages
%     \item {\st{Intuitive description of the proposed method and its difference (i.e., advantages) w.r.t. state of the art}}
%     % Main contributions
%     \item {\st{Summary of the main contributions of this work}}
%     % Paper's structure and organization
%     \item {\st{Paper's structure and organization}}
% \end{itemize}

% Diffusion of FL
Recently, {\em federated learning} (FL) has emerged as the leading paradigm for training distributed, large-scale, and privacy-preserving machine learning (ML) systems~\cite{mcmahan2017googleai,mcmahan2017aistats}. 
The core idea of FL is to allow multiple edge clients to collaboratively train a shared, global model without disclosing their local private training data.
%Specifically, an FL system consists of a central server and many edge clients; 
A typical FL round involves the following steps: {\em(i)} the server randomly picks some clients and sends them the current, global model; {\em(ii)} each selected client locally trains its model with its own private data; then, it sends the resulting local model to the server;\footnote{Whenever we refer to global/local model, we mean global/local model {\em parameters}.} {\em(iii)} the server updates the global model by computing an \emph{aggregation function}, usually the average (FedAvg), on the local models received from clients.
% \begin{enumerate}
%     \item[{\em(i)}] the server sends the current, global model to the clients and appoints some of them for training;
%     \item[{\em(ii)}] each selected client locally trains its copy of the global model with its own private data; then, it sends the resulting local model back to the server;\footnote{Whenever we refer to global/local model, we mean global/local model {\em parameters}.}
%     \item[{\em(iii)}] the server updates the global model by computing an \emph{aggregation function} on the local models received from clients (by default, the average, also referred to as FedAvg~\cite{mcmahan2017aistats}).
% \end{enumerate}
This process goes on until the global model converges. %(e.g., after a certain number of rounds or other similar stopping criteria).
%\\
% The advantages of FL over the traditional, centralized learning paradigm are undoubtedly clear in terms of flexibility/scalability (clients can join/disconnect from the FL network dynamically), network communications (only model weights\footnote{We will use \textit{parameters} and \textit{weights} interchangeably.} are exchanged between clients and server), and privacy (each client's private training data is kept local at the client's end and not uploaded to the server).
\\
% Security threats to FL
%However, the growing adoption of FL also raises security concerns~\cite{costa2022covert}, particularly about its confidentiality, integrity, and availability.
Although its advantages over standard ML, FL also raises security concerns~\cite{costa2022covert}. %, particularly about its confidentiality, integrity, and availability~\cite{costa2022covert}.
% OLD, LONG VERSION
% Indeed, some work deals with privacy leakage that may expose the local data of some clients~\cite{melis2019sp}. 
% A large body of work, instead, investigates attacks that usually aim to detriment the predictive accuracy of the learned global model. For instance, \emph{data poisoning} attacks achieve this goal by letting an adversary pollute the training set of some corrupt FL clients with maliciously crafted examples~\cite{jagielski2018sp}.
% Similarly, in \emph{model poisoning} the attacker attempts to tweak the global model weights~\cite{bhagoji2019pmlr} by directly perturbing the local model's weights of some infected FL clients before these are sent to the central server for aggregation, usually via so-called Byzantine attacks. 
% It turns out that Byzantine model poisoning attacks severely impact standard FedAvg; therefore, more robust aggregation functions must be designed to make FL systems secure.
Here, we focus on \emph{untargeted model poisoning} attacks~\cite{bhagoji2019pmlr}, where an adversary attempts to tweak the global model weights %\footnote{We will use the terms \textit{parameters} and \textit{weights} interchangeably.} 
by directly perturbing the local model's parameters of some infected clients before these are sent to the central server for aggregation.
In doing so, the adversary aims to jeopardize the global model \textit{indiscriminately} at inference time.
Such model poisoning attacks severely impact standard FedAvg; therefore, more robust aggregation functions must be designed to secure FL systems.
\\
% In this paper, we focus on designing a novel robust aggregation scheme at the server's end to contrast the effect of Byzantine model poisoning attacks.
%
% Current countermeasures and their limitations
%Several countermeasures have been proposed in the literature to combat model poisoning attacks on FL systems.
% Some methods use simple statistics more robust than plain average to smooth the impact of malicious updates (e.g., Trimmed Mean and FedMedian~\cite{yin2018icml}). 
% Other defenses implement outlier detection techniques to discard malicious updates from the aggregation performed at the server's end. Those are either based on heuristics (e.g., Krum/Multi-Krum~\cite{blanchard2017nips} and Bulyan~\cite{mhamdi2018pmlr}) or data-driven approaches (e.g., K-means clustering~\cite{shen2016acm} or DnC via spectral analysis~\cite{shejwalkar2021ndss}). 
% Finally, some strategies rely on a centralized ``source of trust'' to spot potential malicious updates (e.g., FLTrust~\cite{cao2020fltrust}).
% Several countermeasures have been proposed in the literature to combat model poisoning attacks on FL systems, i.e., to discard possible malicious local updates from the aggregation performed at the server's end. 
% These techniques range from simple statistics more robust than plain average (e.g., Trimmed Mean and FedMedian~\cite{yin2018icml}) to outlier detection heuristics (e.g., Krum/Multi-Krum~\cite{blanchard2017nips} and Bulyan~\cite{mhamdi2018pmlr}) or data-driven approaches (e.g., spectral analysis via K-means clustering~\cite{shen2016acm} or spectral analysis), or methods based on ``source of trust'' (e.g., FLTrust~\cite{cao2020fltrust}).
% OLD, LONG VERSION
%Several countermeasures have been proposed in the literature to combat Byzantine model poisoning attacks on FL systems.
% Descriptive statistics
% For example, Trimmed Mean and FedMedian aggregate local model updates using more robust statistics than standard average~\cite{yin2018icml}.
%
% % Heuristics for outlier detection
% Many existing Byzantine-resilient strategies implement some outlier detection heuristics to discard the model updates sent by potentially malicious clients from the input of the aggregation function.
% One of the most popular heuristics is Krum~\cite{blanchard2017nips}.
% This strategy tries to mitigate the impact of Byzantine attacks by selecting as a global model the local model with the smallest sum of Euclidean distances to {\em all} the other local models.
% Although powerful, Krum requires the server to know (or, at least, estimate) the number of malicious FL clients upfront, which is generally impossible in a realistic attack scenario. %
% Moreover, Krum may become ineffective for complex, high-dimensional model parameter spaces due to the curse of dimensionality.
% Bulyan~\cite{mhamdi2018pmlr} tries to overcome this issue by combining Krum with a variant of Trimmed Mean.
% % Data-driven outlier detection
% Other strategies use data-driven outlier detection techniques -- e.g., via K-means clustering~\cite{shen2016acm} -- to spot potential malicious local model updates. 
% %For instance, Shen et al. propose to cluster local model updates with K-means and thus identify outliers.
%
% % Other techniques
% As far as the server is concerned, any local model received can be from a potential malicious client. 
% FLTrust~\cite{cao2020fltrust} assumes the server acts as a client, i.e., trains a local model on an additional {\em trustworthy} dataset at the server's end and compares it against all the local models from other clients. 
% This way, the server can rely on some ``source of trust'' when discarding potentially malicious clients.
%\\
% Limitations of existing Byzantine-resilient strategies
Unfortunately, existing defense mechanisms either rely on simple heuristics (e.g., Trimmed Mean and FedMedian by~\cite{yin2018icml}) or need strong and unrealistic assumptions to work effectively (e.g., foreknowledge or estimation of the number of malicious clients in the FL system, as for Krum/Multi-Krum~\cite{blanchard2017nips} and Bulyan~\cite{mhamdi2018pmlr}, which, however, cannot exceed a fixed threshold).
Furthermore, outlier detection methods using K-means clustering~\cite{shen2016acm} or spectral analysis like DnC~\cite{shejwalkar2021ndss} do not directly consider the temporal evolution of local model updates received.
Finally, strategies like FLTrust~\cite{cao2020fltrust} require the server to collect its own dataset and act as a proper client, thereby altering the standard FL protocol.
\\
% OLD, LONG VERSION
% Overall, existing Byzantine-resilient strategies are either simple heuristics (e.g., FedMedian) or, if they are more complex, they rely on strong and unrealistic assumptions to work effectively (e.g., knowing the number of malicious clients in the FL system in advance, as for Krum and alike).
% Furthermore, data-driven outlier detection methods do not consider the temporary evolution of local model updates received (e.g., K-means clustering). 
% Finally, strategies like FLTrust requires the server to collect its own dataset and act as a proper client, thereby altering the standard FL protocol.
%
% Description of the proposed method
This work introduces a novel pre-aggregation \textit{filter} robust to untargeted model poisoning attacks. Notably, this filter $(i)$ operates without requiring prior knowledge or constraints on the number of malicious clients and $(ii)$ inherently integrates temporal dependencies. 
The FL server can employ this filter as a preprocessing step before applying \textit{any} aggregation function, be it standard like FedAvg or robust like Krum or Bulyan.
Specifically, we formulate the problem of identifying corrupted updates as a multidimensional (i.e., matrix-valued) time series anomaly detection task. 
The key idea is that legitimate local updates, resulting from well-calibrated iterative procedures like stochastic gradient descent (SGD) with an appropriate learning rate, show \textit{higher predictability} compared to malicious updates. This hypothesis stems from the fact that the sequence of gradients (thus, model parameters) observed during legitimate training exhibit regular patterns, as validated in Section~\ref{subsec:intuition}. %until convergence. 
%This regularity may be more pronounced for smooth convex loss functions, but it can still be captured within an appropriate time window, even for more complex and convoluted loss surfaces. 
%We provide evidence of this claim in Appendix~B, where we show that the average mutual information (i.e., ``predictability''), calculated over pairs of legitimate model updates sent at different FL rounds, is significantly higher than the corresponding computation for a malicious client.
\\
Inspired by the matrix autoregressive (MAR) framework for multidimensional time series forecasting~\cite{chen2021je}, we propose the FLANDERS ({\em \textbf{F}ederated \textbf{L}earning meets \textbf{AN}omaly \textbf{DE}tection for a \textbf{R}obust and \textbf{S}ecure}) filter.
The main advantages of FLANDERS over existing strategies like FLDetector~\cite{zhao2020multivariate} are its resilience to large-scale attacks, where $50\%$ or more FL participants are hostile, and the capability of working under realistic non-iid scenarios.
We attribute such a capability to two key factors: $(i)$ FLANDERS works without knowing a priori the ratio of corrupted clients, and $(ii)$ it embodies temporal dependencies between intra- and inter-client updates, quickly recognizing local model drifts caused by evil players. Below, we summarize our main contributions:

\begin{itemize}
\item[{\em(i)}]
We provide empirical evidence that the sequence of models sent by legitimate clients is more predictable than those of malicious participants performing untargeted model poisoning attacks.
\\
\item[{\em(ii)}] 
We introduce FLANDERS, the first pre-aggregation filter for FL robust to untargeted model poisoning based on multidimensional time series anomaly detection.
\\
\item[{\em(iii)}] 
We integrate FLANDERS into Flower,\footnote{\scriptsize{\url{https://flower.dev/}}} a popular FL simulation framework for reproducibility.
\\
\item[{\em(iv)}] 
We show that FLANDERS improves the robustness of the existing aggregation methods under multiple settings: different datasets, client's data distribution (non-iid), models, and attack scenarios.
\\
\item[{\em(v)}] 
We publicly release all the implementation code of FLANDERS along with our experiments.\footnote{\scriptsize{\url{https://anonymous.4open.science/r/flanders_exp-7EEB}}}
\end{itemize}

% Paper's structure and organization
The remainder of the paper is structured as follows. %some related work and the current state-of-the-art solutions to security issues that FL entails. 
Section~\ref{sec:background} covers background and preliminaries. 
In Section~\ref{sec:related}, we discuss related work.
Section~\ref{sec:problem} and Section~\ref{sec:method} describe the problem formulation and the method proposed. % to tackle it. 
Section~\ref{sec:experiments} gathers experimental results. %, and Section~\ref{sec:limitations} discusses some limitations of this work.
Finally, we conclude in Section~\ref{sec:conclusion}.
 %discusses the limitations of this work and draws future research directions.
%reports conclusions and draws perspectives for future research directions.

%%%%%%% OLD %%%%%%%
%to overcome the resilience of Byzantine failures in distributed Stochastic Gradient Descent computations. 
% The strength of Krum is its time complexity, which is linear in the gradient dimension. 
% However, the robustness of the approach is guaranteed for gradient-based learning applications only when the majority of the clients are not compromised. 
% Besides, the aggregation mechanism of Krum, as well as that of similar methods, is robust from a coarse-grained perspective and does not provide solutions to errors and perturbations that may occur at inference time.
%A related approach to~\cite{blanchard2017nips} is the work of Su et al.~\cite{su2016dc}. Here, the authors propose an iterated approximate agreement to tackle a multi-layer scenario attacked by Byzantine agents. 
%However, the method works efficiently on the sole discrete context and it is inapplicable to continuous state environments.
%\gabri{Maybe, we should just talk about the main limitations of existing countermeasures without digging into their details (or, we can just mention Krum as this is the most popular one). I will move the description of all these methods to the Related Work section.}


\section{Notations and standing assumptions}\label{sezione_notations_assumptions}



Here, we collect the most frequent notations that occur in this work and state the assumptions.

For a metric space $(E,d_E)$, we denote by $\boreliani{E}$ the Borel $\sigma$-algebra generated by the topology of $E$. When the context allows, we will drop the dependence upon $E$, and just denote it by $\mathcal{B}$.
We denote by $\cbounded{E}$ the set of continuous bounded function $f:E \to \R$.

We will denote by $\mathcal{P}(E)$ the set of probability measures on $(E,\boreliani{E})$.
For $p \geq 1$, we denote by $\mathcal{P}^p(E)$ the set of probability measures $m \in \mathcal{P}(E)$ so that, for some point $x_0 \in E$, and thus for any, the $p$-moment of $m$ is finite:
\begin{equation*}
    m \in \mathcal{P}^p(E) \iff m \in \mathcal{P}(E) \text{ and } \int_E d^p_E(x,x_0)m(dx) < \infty.
\end{equation*}
Let $\mathcal{W}_{p,E}(m_1,m_2)$ denote the $p$-Wasserstein distance on $\mathcal{P}^p(E)$, defined as
\begin{equation*}
    \pwassmetric{p}{E}{p}(m_1,m_2)=\inf\graffe{ \int_{E \times E} d^p_E(x,y)\pi(dx,dy):\; \pi \in \mathcal{P}(E\times E), \text{ $\pi$ has marginals $m_1$, $m_2$}}.
\end{equation*}
Any time we will be given two metric spaces $(E,d_E)$ and $(E,d_{E'})$, we will regard $E \times E'$ as a metric space itself, with the distance $d((e,f),(e',f'))=d_E(e,f) + d_E(e',f')$.
The $p$-Wasserstein distance on $\mathcal{P}^{p}(E \times E')$ will always be meant with respect to such distance on $E \times E'$.

We recall some well-known equivalences for convergence in $p$-Wasserstein distance, which will be used extensively:
\begin{prop}[\cite{villani2003}, Theorem 7.12]\label{wass:equivalenze_convergenza}
Let $(E,d_E)$ be a metric space, and suppose $\mu^n,\mu \in \mathcal{P}^p(E)$. Then the following are equivalent:
\begin{enumerate}[label=(\arabic*)]
    \item $\pwassmetric{p}{E}{}(\mu^n,\mu)\to 0$.
    \item $\mu^n \to \mu$ weakly and for some (and thus any) $x_0 \in E$ we have
    \begin{equation}\label{wass:uniforme_integrabilita}
    \lim_{r \to \infty}\sup_n\int_{\insieme{x:\;d_E^p(x,x_0)\geq r}}d_E^p(x,x_0)\mu_n(dx)=0
    \end{equation}
    \item $\int\phi(x)\mu_n(dx) \to \int\phi(x) \mu(dx)$ for all continuous functions $\phi : E \to \R$ such that there exist $x_0 \in E$ and $c > 0$ so that $\vert \phi(x)\vert \leq c(1+d_E^p(x,x_0))$ for all $x \in E$.
\end{enumerate}
In particular, a sequence $(\mu_n)_n \subset \mathcal{P}^p(E)$ is relatively compact if and only if it is tight and satisfies \eqref{wass:uniforme_integrabilita}.
\end{prop}
For $T>0$ fixed, we denote by $\contrd$ the set of continuous functions from $[0,T]$ in $\R^d$, $d \in \N$, i.e. $\contrd=\conttraj{\R^d}$. We endow $\contrd$ with the norm $\lVert x \rVert_{\contrd}=\sup_{ s \in [0,T]}\vert x_s \vert $.
Occasionally, we will use the semi-norm $\lVert x \rVert_{t,\contrd}=\sup_{s \in [0,t]}\vert x_s \vert$, for $x \in \contrd$.
We will denote as $\wienermeasure \in \mathcal{P}(\contrd)$ the law of a standard $d$-dimensional Brownian motion, and by $\contpdue$ the set of continuous functions from $[0,T]$ in $\mathcal{P}^2(\R^d)$, i.e. $\contpdue=\mathcal{C}([0,T];\mathcal{P}^2(\R^d))$, where $\mathcal{P}^2(\R^d)$ is endowed with the $2$-Wasserstein distance.
We endow $\contpdue$ with the supremum distance $\sup_{t \in [0,T]}\pwassmetric{2}{\mathcal{P}^2(\R^d)}{}(m^1_t,m^2_t)$, for any $m^1=(m^1_t)_{t \in [0,T]}$ and $m^2=(m^1_t)_{t \in [0,T]}$ in $\contpdue$.

When given a filtered probability space $(\Omega,\F,(\mathcal{G}_t)_t,\prob)$, we regard as the $\prob$-augmentation of the filtration $(\mathcal{G}_t)_t$ the filtration $\mathbb{F}=(\F_t)_t$, where $\F_t=\cap_{\eps > 0} \sigma(\mathcal{G}_{t + \eps},\mathcal{N})$ and $\mathcal{N}$ stands for the $\prob$-null sets of $\Omega$. Such a filtration satisfies the usual assumptions.

\medskip
We end this section by stating our standing assumptions on the state dynamics and on the costs of the players in both the $N$-player game and the limit game.
We are given a finite time horizon $T >0$, a control actions space $A$, an initial state distribution $\nu \in \probmeasures{\R^d}$, and the following functions:
\begin{equation*}
\begin{aligned}
    (b,f):[0,T]\times\R^d\times\mathcal{P}^2(\R^d)\times A\to\R^d, \\
    g:\R^d\times\mathcal{P}^2(\R^d)\to\R^d,
\end{aligned}
\end{equation*}
which will be referred to, respectively, as the drift function, the running cost and the terminal cost.

The following Assumptions \ref{standing_assumptions} will be in force throughout the whole manuscript.

\begin{customassumptions}{\textbf{A}}\label{standing_assumptions}
\begin{enumerate}[label=\normalfont(A.\arabic*)]
    \item[]
    \item $A\subseteq \R^l$, for some $l \geq 1$, is a compact set.
    \item $\nu \in \mathcal{P}^{\overline{p}}(\R^d)$, for some $\overline{p} > 4$.
    \item The functions $b$, $f$ and $g$ are jointly measurable in $(t,x,m,a)$.
    \item $b(t,x,m,a)$ is Lipschitz in $a \in A$, $m \in \mathcal{P}^2(\R^d)$ and $x \in \R^d$, uniformly in $t$:
    \begin{equation*}
        \vert b(t,x,m,a)-b(t,x',m',a') \vert \leq L \tonde{\vert a-a' \vert  + \vert x-x' \vert  + \pwassmetric{2}{\R^d}{}(m,m')}
    \end{equation*}
    for every $t \in [0,T]$, $(x,m,a)$ and $(x',m',a')$ in $\R^d \times \mathcal{P}^2(\R^d)\times A$.
    \item The functions $[0,T] \ni t \mapsto (b,f)(t,0,\delta_0,a_0)$ are bounded, for some $a_0 \in A$.
    \item $f$ and $g$ are locally Lipschitz in $(x,m,a)$ for every fixed $t \in [0,T]$ with at most quadratic growth, i.e., there exists a positive constant $L>0$ so that
    \begin{equation*}
    \begin{aligned}
        \big\vert (f,g) & (t,x,m,a)-(f,g)(t,x',m',a') \big\vert  \\
        \leq & \, L\left( 1 + \abs{x}+\abs{x'} + \tonde{\int_{\R^d} \abs{y}^2 m(dy)}^\frac{1}{2} + \tonde{\int_{\R^d} \abs{y}^2 m'(dy)}^\frac{1}{2} +\abs{a}+\abs{a'}\right) \\
        & \cdot \tonde{\vert x-x' \vert +\pwassmetric{2}{\R^d}{}(m,m')+\vert a-a' \vert },
    \end{aligned}
    \end{equation*}
    for every $t \in [0,T]$, $(x,m,a)$ and $(x',m',a')$ in $\R^d \times \mathcal{P}^2(\R^d)\times A$.
\end{enumerate}
\end{customassumptions}




\section{Formulation of the $N$-player game}\label{sezione_formulazione_N_giocatori}


Consider the following canonical space
\begin{equation}\label{canonical_setup}
\begin{aligned}
    \Omega^1=\bigtimes_{1}^\infty \R^d \times \contrd, \quad \F^1=\bigotimes_{1}^\infty \boreliani{\R^d} \otimes \boreliani{\contrd}, \quad \prob^1=\bigotimes_{1}^\infty \nu \otimes \wienermeasure.
\end{aligned}
\end{equation}
We define a sequence of random variables $(\xi^i)_{i \geq 1}$ and of Brownian motions $(W^i)_{i \geq 1}$, by taking the projections:
\begin{equation}\label{canonical_browian_motions}
    \xi^i(\omega_1)=\xi^i((x^j,w^j)_{j \geq 1})=x^i, \quad W^i_t(\omega_1)=W^i_t((x^j,w^j)_{j \geq 1})=w^i_t, \; t \in [0,T].
\end{equation}
By definition of $\prob^1$, $(\xi^i)_{i\geq 1}$ and $(W^i)_{i \geq 1}$ are mutually independent, $(\xi^i)_{i\geq 1}$ are independent and identically distributed with law $\nu \in \mathcal{P}^{\overline{p}}(\R^d)$ and $(W^i)_{i \geq 1}$ are independent $d$-dimensional standard Brownian motions.


\medskip
Let $N\in\N$, $N\geq 2$, be the number of players.
We define the filtration $\mathbb{F}^{1,N}$ as the $\prob^1$-augmentation of the filtration generated by the first $N$ random variables $(\xi^i)_{i=1}^N$ and Brownian motions $(W^i)_{i=1}^N$.
Therefore, for the $N$-player game, we work on the space
\begin{equation}\label{finite_players:canonical_space}
    (\Omega^1,\F^1,\mathbb{F}^{1,N},\prob^1).
\end{equation}
We stress that, for every $N \geq 2$, we keep the probability space $(\Omega^1,\F^1,\prob^1)$ fixed while the filtration $\mathbb{F}^{1,N}$ varies.

\medskip
Consider the set $\A_N$ of $\mathbb{F}^{1,N}$-progressively measurable processes taking values in $A$:
\begin{equation}\label{finite_players:strategie_open_loop}
    \A_N = \insieme{\alpha:[0,T]\times\Omega^1\to A \; \Big\vert \;  \text{$\alpha$ is $\mathbb{F}^{1,N}$-progressively measurable } }.
\end{equation}
Provided that we identify processes which are equal $Leb_{[0,T]}\otimes\prob^1$-a.e., we can regard $\A_N$ as
\begin{equation*}
    \A_N=L^2 \tonde{[0,T]\times\Omega^1,\mathcal{P}^{1,N},Leb_{[0,T]} \otimes \prob^1;A},
    \end{equation*}
where $\mathcal{P}^{1,N}$ stands for the progressive $\sigma$-algebra on $[0,T]\times\Omega^1$, using the filtration $\mathbb{F}^{1,N}$.
We call any element $\alpha\in\A_N$ an open loop strategy for the $N$-player game.
We regard a vector $(\alpha^1,\dots,\alpha^N) \in \A_N^N=\bigtimes_1^N \A_N$ as an open loop strategy profile for the $N$ players, which will be
occasionally denoted by $\bm{\alpha}$.
We endow such a space $\A$ with the norm
\begin{equation}\label{finite_players:semi_norm}
    \norm{\alpha}_{L^2}=\E^{\prob^1}\quadre{\int_0^T \abs{\alpha_t}^2 dt}^\frac{1}{2}
\end{equation}
and consider the Borel $\sigma$-algebra $\boreliani{\A_N}$ associated to that.
We observe that, since $([0,T]\times\Omega^1,\boreliani{[0,T]\times\Omega^1})$ is Polish and $A$ is closed, $\A_N$ is a separable Banach space.
In the following, we will make no distinction between an $\mathbb{F}^{1,N}$-progressively measurable process $\alpha$ and any other process $\alpha'$ which is equal to it $Leb_{[0,T]}\otimes\prob^1$-almost everywhere.

\begin{definition}[Admissible recommendation profile and correlated strategy profile]\label{def_correlated_strategy}

We call \emph{admissible recommendation profile} to the $N$ players a pair $((\Omega^0,\F^{0-},\prob^0),\Lambda)$ so that the following holds:
\begin{enumerate}
    
    \item $(\Omega^0,\mathcal{F}^{0-}$, $\prob^0)$ is a complete probability space; $\Omega^0$ is a Polish space and $\F^{0-}$ is its corresponding Borel $\sigma$-algebra.

    \item $\Lambda =(\Lambda^1,\dots,\Lambda^N)$ is a random vector with values in $\A_N^N$:
    \begin{equation}\label{raccomandazione}
    \begin{aligned}
    \Lambda: (\Omega^0,\mathcal{F}^{0-},\prob^0) & \longrightarrow (\A_N^N,\mathcal{B}_{\A_N^N}) \\
    \omega_0 & \longmapsto \Lambda(\omega_0)=(\alpha^1,\dots,\alpha^N):[0,T]\times\Omega^1\to A^N.
    \end{aligned}
    \end{equation}

    \item \label{finite_players:condizione_ammissibilita} $\Lambda$ is admissible in the following sense: Let $(\Omega,\mathcal{F},\prob)$ be the product space of $(\Omega^0,\mathcal{F}^{0-},\prob^0)$ and $(\Omega^1,\mathcal{F}^1,\prob^1)$:
    \begin{equation*}
        (\Omega,\mathcal{F},\prob)= (\Omega^0 \times \Omega^1,\mathcal{F}^{0-}\otimes\F^1,\prob^0\otimes\prob^1).
    \end{equation*}
    We complete the $\sigma$-algebra $\F$ with the $\prob$-null sets and endow the product probability space with the $\prob$-augmentation of the filtration
    \begin{equation*}
    \mathbb{F}= \F^{0-} \otimes \mathbb{F}^{1,N} = (\F^{0-}\otimes\F^{1,N}_t)_{t\in[0,T]}.
    \end{equation*}
    Given an $(\A_N^N,\boreliani{\A_N^N})$-valued random variable $\Lambda=(\Lambda^1,\dots,\Lambda^N)$ as in \eqref{raccomandazione}, we say that $\Lambda$ is admissible if there exists a process $\lambda=(\lambda^1_t,\dots,\lambda^N_t)_{t\in[0,T]}$ with values in $A^N$, defined on $(\Omega,\F,\prob)$ and $\mathbb{F}$-progressively measurable, so that, for every $i=1,\dots,N$, for $\prob^0$-a.e. $\omega_0 \in \Omega^0$, the section $(\lambda^i_t(\omega_0,\cdot))_{t \in [0,T]}$ is equal to $\Lambda^i(\omega_0)$ in $\A_N$:
        \begin{equation}\label{finite_players:uguaglianza_ammissibilita}
            \norm{(\lambda^i_t(\omega_0,\cdot))_{t \in [0,T]}-\Lambda^i(\omega_0)}_{L^2([0,T] \times \Omega^1)}=0, \quad \prob^0\text{-a.s.,} \: i=1,\dots,N.
        \end{equation}
\end{enumerate}
If $\Lambda=(\Lambda^1,\dots,\Lambda^N)$ is an admissible recommendation to the $N$ players, we write
\begin{equation}\label{controllo_indotto}
    \lambda^i_t(\omega)= \lambda^i_t(\omega_0,\omega_1)=\Lambda^i(\omega_0)_t(\omega_1), \quad i=1,\dots,N,
\end{equation}
where equality is to be understood in the sense of \eqref{finite_players:uguaglianza_ammissibilita}.
We call \emph{correlated strategy profile} associated to the admissible recommendation profile $((\Omega^0,\F^{0-}$, $\prob^0),\Lambda)$ the $\mathbb{F}$-progressively measurable process $\lambda=(\lambda_t)_{t\in[0,T]}$ satisfying \eqref{controllo_indotto}.
\end{definition}
We remark that, by Proposition \ref{esempi:unicita_strategia_associata}, given any admissible recommendation to the $N$ players $((\Omega^0,\F^{0-},\prob^0),\Lambda)$, the correlated strategy profile $\lambda$ associated to it is unique $Leb_{[0,T]}\otimes\prob$-almost everywhere.
We point out that in general, for instance when a recommendation profile $\Lambda$ takes uncountably many values, we cannot recover the progressive measurability property of the strategy $\lambda$ associated to the recommendation $\Lambda$.
The essential reason is that we cannot deduce the measurability of a set in the product $\sigma$-algebra from the measurability of its sections, as shown, e.g., in \cite[p.\,5]{stoyanov}.
Therefore, the admissibility requirement on $((\Omega^0,\F^{0-},\prob^0),\Lambda)$ is necessary.
Nevertheless, we give some examples of admissible recommendations in Example \ref{mf:example:admissible_recommendations} in the following Section \ref{sezione_formulazione_mfg}.


\begin{remark}\label{finite_players:remark_estensioni}
As usual, we can extend random variables defined on $\Omega^1$ to random variables defined on $\Omega$. Indeed, suppose $X:(\Omega^1,\mathcal{F}^1)\to (E,\mathcal{E})$ is a random variable with values in some measurable space $(E,\mathcal{E})$. We can then regard  $X$ as defined on the space $(\Omega,\mathcal{F})$ via the identification $\Tilde{X}(\omega_0,\omega_1)=X(\omega_1)$, and analogously for $\Omega^0$.
In this sense, via the identification $(\Tilde{\xi}^i,\Tilde{W}^i)(\omega_0,\omega_1)=(\xi^i,W^i)(\omega_1)$ for every $i=1,\dots,N$, we can regard the Brownian motions and initial data as defined on $\Omega$; we observe that $(W^i)_{i=1}^N$ are independent standard Brownian motions with respect to the filtration $\mathbb{F}$ as well.
Moreover, we can identify each process $\alpha\in\A_N$, which is defined on $\Omega^1$, with a process $\Tilde{\alpha}$ defined on $\Omega$ via the identification $\Tilde{\alpha}(\omega_0,\omega_1)=\alpha(\omega_1)$.
Such a process is progressively measurable with respect to the filtration $\mathbb{F}$ and independent of $\F^{0-}$.
\end{remark}

We interpret the admissible recommendation as follows: A correlation device or a mediator runs a lottery over open loop strategy profiles according to some publicly known distribution $\prob^0$ and communicates privately to each player a strategy according to the selected profile.
The extraction of the strategy profile happens before the game starts and it is independent of the idiosyncratic shocks that determine the random evolution of players' states.
These features are captured by the construction of the underlying probability space as a product of $(\Omega^0,\F^{0-},\prob^0)$, which contains the information used to correlate players' strategies, and $(\Omega^1,\F^1,\prob^1)$, where noises and open loop strategies are defined, and by the choice of the filtration $\mathbb{F}$, since $\F^{0-} \subseteq \F_t$ for every $t \in [0,T]$.
We stress that, by definition, the realization $\Lambda^i(\omega_0)$ is an $\mathbb{F}^{1,N}$-progressively measurable process in $\A_N$, for any scenario $\omega_0 \in \Omega^0$ and $i=1,\dots,N$.
Observe that, even though $\Lambda$ and $(\xi^j,W^j)_{j=1}^N$ are independent, the correlated strategy profile $\lambda=(\lambda_t)_{t \in [0,T]}$ is in general not independent of either of them, since it is the result of both the recommendation profile and the random shocks and initial data.

\medskip
Let $((\Omega^0,\F^{0-},\prob^0),\Lambda)$ be an admissible recommendation profile.
On the space $(\Omega,\F,\mathbb{F},\prob)$ defined at point \ref{finite_players:condizione_ammissibilita} of Definition \ref{def_correlated_strategy}, we assign players state dynamics and define the cost functionals.
If all players follow the recommendation $\Lambda$, players' state dynamics are given by the following system of stochastic differential equations:
\begin{equation}\label{dinamiche_no_deviazione}
    \begin{cases}
    dX^j_t = b(t,X^j_t,\mu^N_t,\lambda^j_t)dt +dW^j_t, \qquad 0\leq t \leq T, \\
    X^j_0 = \xi^j,
    \end{cases}
\end{equation}
for every $j\in\insieme{1,\dots,N}$, where $\mu^N_t$ is the empirical measure of the state processes of all players at time $t$:
\begin{equation}\label{misura_empirica}
    \mu^N_t=\frac{1}{N}\sum_{\substack{j=1}}^N \delta_{X^j_t}.
\end{equation}


Suppose player $i$ deviates, while the other players follow the recommendations they receive from the mediator.
The deviating player will pick instead an open loop strategy $\beta\in\A_N$.
In other words, at every time $t$ and for every scenario $\omega$, player $i$ plays the action $\Tilde{\beta}_t(\omega)=\beta_t(\omega_1)$ instead of playing the recommended action $\lambda^i_t(\omega)=\Lambda^i(\omega_0)(\omega_1)$.
Then, players' state dynamics are given by the following system of stochastic differential equations:
\begin{equation}\label{dinamiche_deviazione}
    \begin{cases}
    dX^j_t = b(t,X^j_t,\mu^N_t,\lambda^{j}_t)dt +dW^j_t, \qquad   0\leq t \leq T, \quad X^j_0=\xi^j, \quad j\neq i \\
    dX^i_t = b(t,X^i_t,\mu^N_t,\beta_t)dt +dW^i_t, \qquad 0\leq t \leq T, \quad X^i_0=\xi^i, \end{cases}
\end{equation}
where $\mu^N_t$ is defined as in \eqref{misura_empirica}. Assumptions \ref{standing_assumptions} ensure that there always exists an $\mathbb{F}$-adapted continuous solution to both equations \eqref{dinamiche_no_deviazione} and \eqref{dinamiche_deviazione} so that $\E[\sup_{t \in [0,T]}\max_{1 \leq j \leq N}\vert X^j_t\vert^2]$ is finite.
Moreover, pathwise uniqueness holds so that, by Theorem \ref{teorema_di_unicita_legge}, uniqueness in law holds as well.


\begin{remark}\label{finite_players:remark_deviazioni}
We notice that there is an asymmetry between the information available to the mediator and the deviating player: if a player deviates, she chooses her strategy on her own, ignoring the information contained in mediator's recommendation, since, by definition of the process $\beta$ on the product space, deviating player's strategy $\beta$ is independent of the recommendation profile $\Lambda$.
We can interpret player $i$'s deviation in the following way: either the player commits to the moderator ex-ante or she does not; if she does not, she will not exploit any of the additional information the mediator would give away when communicating the recommended strategies to the players.
\end{remark}

As for the cost functional, let $((\Omega^0,\F^{0-},\prob^0),\Lambda)$ be an admissible recommendation profile.
If all players follow the recommendation, then the cost functional of each player $j=1,\dots,N$ is given by
\begin{equation*}
    \J^N_j(\Lambda) =\E\quadre{\int_0^T f(t,X^j_t,\mu^N_t,\lambda^j_t)dt + g(X^j_T,\mu^N_T)},
\end{equation*}
with dynamics given by \eqref{dinamiche_no_deviazione}.
If instead player $i$ does not play according to the recommendation $\Lambda^i$ and plays a different strategy $\beta \in \A$, while the other players stick to the recommendation profile $\Lambda^{-i}=(\Lambda^1,\dots,\Lambda^{i-1},\Lambda^{i+1},\dots,\Lambda^N)$, we define the cost functional of each player $j$ as
\begin{equation*}
\begin{aligned}
    \J^N_j(\Lambda) & =\E\quadre{\int_0^T f(t,X^j_t,\mu^N_t,\lambda^j_t)dt + g(X^j_T,\mu^N_T)}, \quad j \neq i \\ 
    \J^N_i(\Lambda^{-i},\beta) & =\E\quadre{\int_0^T f(t,X^i_t,\mu^N_t,\beta_t)dt + g(X^i_T,\mu^N_T)},
\end{aligned}
\end{equation*}
where the dynamics are given by \eqref{dinamiche_deviazione}.
We stress that the expectation in the cost functional is taken with respect to the product probability measure $\prob = \prob^0 \otimes \prob^1$, although we omit this dependence for conciseness.
Finally, we give the notion of $\eps$-coarse correlated equilibrium:
\begin{definition}[$\eps$-coarse correlated equilibrium]\label{def_CCE}
Let $\eps\geq0$. An admissible recommendation profile $((\Omega^0,\F^{0-},\prob^0),\Lambda)$ is an \emph{$\eps$-coarse correlated equilibrium} for the $N$-player game ($\eps$-CCE) if
\begin{equation}\label{def_CCE:optimality}
    \J^N_i(\Lambda)\leq \J^N_i(\Lambda^{-i},\beta) + \eps
\end{equation}
for all open loop strategies $\beta \in \A_N$ and all players $i=1,\dots,N$.
We call an admissible recommendation profile $((\Omega^0,\F^{0-},\prob^0),\Lambda)$ a \emph{coarse correlated equilibrium} for the $N$-player game if it is an $\eps$-coarse correlated equilibrium with $\eps=0$.
\end{definition}


The usual notion of Nash equilibrium in open loop strategies is consistent with the definition of coarse correlated equilibrium: Suppose we are given an $\eps$-Nash equilibrium $(\alpha^1,\dots,\alpha^N)$ in open loop strategies.
We choose $(\Omega^0,\F^{0-},\prob^0)$ as the trivial probability space and $\Lambda$ as constant and equal to $(\alpha^1,\dots,\alpha^N)$.
It is then straightforward to see that the triple $((\Omega^0,\F^{0-},\prob^0),\Lambda)$ is an $\eps$-CCE according to Definition \ref{def_CCE}.


\medskip
Notice that a Nash equilibrium in open loop strategies $(\alpha^1,\dots,\alpha^N)$ is $\mathbb{F}^{1,N}$-progressively measurable, while a correlated strategy profile $\lambda$ associated to an admissible recommendation $\Lambda$ contains the information carried by $\Lambda$ itself, which is the information the mediator uses to randomize players' strategies.
Moreover, while in both cases the deviating player will use an open loop strategy, CCEs present a certain asymmetry between the information available to the mediator and the deviating player, as pointed out in Remark \ref{finite_players:remark_deviazioni}, while, when dealing with Nash equilibria, 
the deviating player has access to the same information of the other players, since they all use $\mathbb{F}^{1,N}$-progressively measurable strategies.

\begin{remark}[Role of the probability space $(\Omega^0,\mathcal{F}^{0-},\prob^0)$]
According to Definition \ref{def_correlated_strategy}, the probability space $(\Omega^0,\mathcal{F}^{0-},\prob^0)$ is part of the definition of admissible recommendation.
The natural interpretation is that the mediator chooses the auxiliary space he uses to correlate players' strategies.
Moreover, according to equations \eqref{dinamiche_no_deviazione} and \eqref{dinamiche_deviazione}, it determines the probability space on which state processes are defined.
In order to keep the notation as simple as possible, by abuse of notation, we mostly refer only to $\Lambda$ as the admissible recommendation instead of the pair $((\Omega^0,\F^{0-},\prob^0),\Lambda)$.
\end{remark}

\begin{remark}[Relationship with correlated equilibria of \cite{bonesiniCE, campifischer2021}]
It is worth to briefly compare our notion of coarse correlated equilibria with the notion of correlated equilibria of \cite{campifischer2021} and \cite{bonesiniCE}.
Besides the fact that the sets of times, individual states and control actions are finite, therein strategies recommended to the players are in restricted closed loop form, that is, they are Markovian functions of each player's private state, hence of the form $(u^j(t,X^j_t))_{t \in [0,T]}$, with $u^j:[0,T]\times \R^d \to A$ measurable, for each $j=1,\dots,N$.
Recommendations thus take values in the set of such functions of time and state.
Most importantly, in their framework, the deviating player reacts to the recommended strategy: she observes the recommendation the mediator gives her and decides whether or not to play accordingly after receiving it.
Therefore, using our notation, the deviating player $i$ has access to the information carried by $\Lambda^i$.
This feature is not present in our model, as previously discussed.
\end{remark}




\section{Formulation of the mean field game}\label{sezione_formulazione_mfg}


Consider the following canonical space
\begin{equation}\label{mf:canonical_space}
    \Omega^*=\R^d \times \contrd, \quad \F^*=\boreliani{\R^d} \otimes \boreliani{\contrd},\quad \prob^*=\nu\otimes\wienermeasure.
\end{equation}
Define $\xi$ and $W=(W_t)_{t \in [0,T]}$ as
\begin{equation}
    \xi(\omega_*)=\xi(x,w)=x, \quad W_t(\omega_*)=W_t(x,w)=w_t.
\end{equation}
By definition of $\prob^*$, $\xi$ and $W$ are independent, $\xi$ is an $\R^d$-valued random variable with law $\nu$ and $W$ is a standard Brownian motion.
Define the filtration $\mathbb{F}^*$ as the $\prob^*$-augmentation of the filtration generated by $\xi$ and $W$.

\medskip
Consider the set $\A$ of $\mathbb{F}^*$-progressively measurable processes taking values in $A$:
\begin{equation}\label{mf:strategie_open_loop}
    \A = \insieme{\alpha:[0,T]\times\Omega^*\to A \; \Big\vert \;  \text{$\alpha$ is $\mathbb{F}^*$-progressively measurable } }.
\end{equation}
Provided that we identify processes which are equal $Leb_{[0,T]}\otimes\prob^*$-a.e., we can regard $\A$ as
\begin{equation*}
    \A=L^2 \tonde{[0,T]\times\Omega^*,\mathcal{P}^*,Leb_{[0,T]} \otimes \prob^*;A},
    \end{equation*}
where $\mathcal{P}^*$ stands for the progressive $\sigma$-algebra on $[0,T]\times\Omega^*$, using the filtration $\mathbb{F}^*$.
We call any element $\alpha\in\A$ an open loop strategy for the mean field game.
We endow such a space $\A$ with the norm
\begin{equation}\label{mf:semi_norm}
    \norm{\alpha}_{L^2}=\E^{\prob^*}\quadre{\int_0^T \abs{\alpha_t}^2 dt}^\frac{1}{2}
\end{equation}
and consider the Borel $\sigma$-algebra $\boreliani{\A}$ associated to that.
We observe that, since $([0,T]\times\Omega^*,\boreliani{[0,T]\times\Omega^*})$ is Polish and $A$ is closed, $\A$ is a separable Banach space.
Finally, we will make no distinction between an $\mathbb{F}^*$-progressively measurable process $\alpha$ and any other process $\alpha'$ which is equal to it $Leb_{[0,T]}\otimes\prob^*$-almost everywhere.

\begin{definition}[Admissible recommendation for the mean field game]\label{mf:admissible_recommendation}
We call \emph{admissible recommendation} a pair $((\Omega^0,\F^{0-},\prob^0),\Lambda)$ where:
\begin{enumerate}
    
    \item $(\Omega^0,\mathcal{F}^{0-}$, $\prob^0)$ is a complete probability space; $\Omega^0$ is a Polish space and $\F^{0-}$ is its corresponding Borel $\sigma$-algebra.

    \item $\Lambda$ is a random variable with values in $\A$:
    \begin{equation}\label{mf:raccomandazione}
    \begin{aligned}
    \Lambda: (\Omega^0,\mathcal{F}^{0-},\prob^0) & \longrightarrow (\A,\boreliani{\A}) \\
    \omega_0 & \longmapsto \Lambda(\omega_0)=\alpha:[0,T]\times\Omega^*\to A.
    \end{aligned}
    \end{equation}

    \item \label{mf:condizione_ammissibilita} $\Lambda$ is admissible, in the following sense: let $(\Omega,\F,\prob)$ be the product space of $(\Omega^0,\F^{0-},\prob^0)$ and $(\Omega^*,\F^*,\prob^*)$:
    \begin{equation*}
        (\Omega,\F,\prob)= (\Omega^0 \times \Omega^*,\F^{0-}\otimes\F^*,\prob^0\otimes\prob^*).
    \end{equation*}
    We complete the $\sigma$-algebra $\F$ with the $\prob$-null sets and endow the product probability space with the $\prob$-augmentation of the filtration
    \begin{equation*}
    \mathbb{F}= \F^{0-} \otimes \mathbb{F}^* = (\F^{0-}\otimes\F^*_t)_{t\in[0,T]}.
    \end{equation*}
    Given an $(\A,\boreliani{\A})$-valued random variable $\Lambda$ as in \eqref{mf:raccomandazione}, we say that $\Lambda$ is admissible if there exists an $A$-valued process $\lambda=(\lambda_t)_{t\in[0,T]}$, defined on $(\Omega,\F,\prob)$ and $\mathbb{F}$-progressively measurable, so that, for $\prob^0$-a.e. $\omega_0 \in \Omega^0$, the section $(\lambda_t(\omega_0,\cdot))_{t \in [0,T]}$ is equal to $\Lambda(\omega_0)$ in $\A$:
    \begin{equation}\label{mf:uguaglianza_ammissibilita}
        \norm{(\lambda_t(\omega_0,\cdot))_{t \in [0,T]}-\Lambda(\omega_0)}_{L^2([0,T] \times \Omega^*)}=0, \quad \prob^0\text{-a.s.}
    \end{equation}
\end{enumerate}
If $\Lambda$ is an admissible recommendation, we write
\begin{equation}\label{mf:controllo_indotto}
    \lambda_t(\omega)= \lambda_t(\omega_0,\omega_*)=\Lambda(\omega_0)_t(\omega_*),
\end{equation}
where equality is to be understood in the sense of \eqref{mf:uguaglianza_ammissibilita}.
We call \emph{strategy associated to the admissible recommendation} $((\Omega^0,\F^{0-},\prob^0),\Lambda)$ the $\mathbb{F}$-progressively measurable process $\lambda=(\lambda_t)_{t\in[0,T]}$ satisfying \eqref{mf:controllo_indotto}.
\end{definition}
We remark that, by Proposition \ref{esempi:unicita_strategia_associata}, given any admissible recommendation $((\Omega^0,\F^{0-},\prob^0),\Lambda)$, the strategy $\lambda$ associated to it is unique $Leb_{[0,T]}\otimes\prob$-almost everywhere.

\begin{definition}[Correlated flow]\label{def_correlated_flow}
A \emph{correlated flow} is a triple $((\Omega^0,\F^{0-},\prob^0),\Lambda,\mu)$ where:
\begin{enumerate}
    \item $((\Omega^0,\F^{0-},\prob^0),\Lambda)$ is an admissible recommendation.

    \item $\mu:(\Omega^0,\F^{0-},\prob^0)\to (\contpdue,\boreliani{\contpdue})$ is a random continuous flow of measures in $\mathcal{P}^2(\R^d)$.
\end{enumerate}
\end{definition}
The same considerations as in Remark \ref{finite_players:remark_estensioni} about the extension of random variables on the product space $(\Omega,\F,\prob)$ hold for correlated flows as well.

\medskip
Let $((\Omega^0,\F^{0-},\prob^0),\Lambda,\mu)$ be a correlated flow.
On the product probability space $(\Omega,\mathcal{F},\prob)$ defined at point \ref{mf:condizione_ammissibilita} of Definition \ref{mf:admissible_recommendation}, we assign state dynamics.
If the representative player decides to play according to the admissible recommendation $\Lambda$, the dynamics is given by the following SDE:
\begin{equation}\label{dinamica_MF_no_deviazione}
    \begin{cases}
    dX_t = b(t,X_t,\mu_t,\lambda_t)dt +dW_t, \qquad  0\leq t \leq T, \\
    X_0 = \xi.
    \end{cases}
\end{equation}
If instead the representative player decides to ignore the mediator's recommendation and to use a possibly different strategy $\beta\in\A$, the dynamics is given by the following SDE:
\begin{equation}\label{dinamica_MF_deviation}
    \begin{cases}
    dX_t = b(t,X_t,\mu_t,\beta_t)dt +dW_t, \qquad 0\leq t \leq T, \\
    X_0 = \xi.
    \end{cases}
\end{equation}
By Assumptions \ref{standing_assumptions}, on any space $(\Omega,\F,\mathbb{F},\prob)$ there exists a solution to equation \eqref{dinamica_MF_no_deviazione} and pathwise uniqueness holds.
By Theorem \ref{teorema_di_unicita_legge}, uniqueness in law holds.
Analogous considerations apply to equation \eqref{dinamica_MF_deviation}.


\medskip
Let $((\Omega^0,\F^{0-},\prob^0),\Lambda,\mu)$ be a correlated flow.
The cost functionals for the representative player and the deviating player, whose state dynamics follow \eqref{dinamica_MF_no_deviazione} and \eqref{dinamica_MF_deviation}, respectively, are given by:
\begin{equation}\label{costi_mfg}
\begin{aligned}
    & \J(\Lambda,\mu)=\E\quadre{\int_0^T f(t,X_t,\mu_t,\lambda_t)dt + g(X_T,\mu_T)}, \\
    & \J(\beta,\mu)=\E\quadre{\int_0^T f(t,X_t,\mu_t,\beta_t)dt + g(X_T,\mu_T)}.
\end{aligned}
\end{equation}
As in the $N$-player game, the expectation in the cost functional is taken with respect to the product probability measure $\prob = \prob^0 \otimes \prob^*$; in particular, it depends on the mediator's randomization $\prob^0$ also when the representative player deviates. Finally, we give the definition of coarse correlated solution of the mean field game:
\begin{definition}[Coarse correlated solution]\label{def_mean_field_sol}
A correlated flow $((\Omega^0,\F^{0-},\prob^0),\Lambda,\mu)$ is a \emph{coarse correlated solution} of the mean field game if the following properties hold:
\begin{enumerate}[label=(\roman*)]
    \item Optimality: for every deviation $\beta \in \A$, it holds
    \begin{equation}\label{def_mean_field_sol:opt}
        \J(\Lambda,\mu)\leq \J(\beta,\mu).
    \end{equation}
    \item Consistency: for every time $t \in [0,T]$, $\mu_t$ is a version of the conditional law of $X_t$ given $\mu$, that is,
    \begin{equation}\label{def_mean_field_sol:cons}
        \mu_t(\cdot)=\prob(X_t \in\cdot \; \vert \; \mu ) \quad \prob\text{-a.s.} \;\; \forall t \in [0,T].
    \end{equation}
\end{enumerate}
We will refer to coarse correlated solutions of the mean field game as coarse correlated mean field solutions and mean field coarse correlated equilibria (CCE) as well.
\end{definition}

\begin{remark}[Role of $(\Omega^0,\F^{0-},\prob^0)$]
Analogously as in the $N$-player game, although the probability space $(\Omega^0,\mathcal{F}^{0-},\prob^0)$ is part of the definitions of admissible recommendation and correlated flow, when it is clear from the context we refer to $\Lambda$ and $(\Lambda,\mu)$, instead of the pair $((\Omega^0,\F^{0-},\prob^0),\Lambda)$ and the triple $((\Omega^0,\F^{0-},\prob^0),\Lambda,\mu)$, as admissible recommendation and correlated flow, respectively.
\end{remark}

As in \cite{bonesiniCE,campifischer2021}, the consistency condition \eqref{def_mean_field_sol:cons} should be read in the following way: the mediator imagines what the flow of measures will be, up to the terminal horizon $T$, before the game starts, and gives a recommendation to each player according to his idea. Since the flow of measures is expected to be stochastic as a result of the mediator's randomization only, we request it to be measurable with respect to $\F^{0-}$, and, since the randomization is performed before the game starts, we have $ \F^{0-} \subseteq \F_t$ for any $t \geq 0$. If all players commit to the mediator's lottery for generating recommendations, then the flow of measures should arise from aggregation of the individual behaviors, consistently with what imagined by the mediator. Since the generation of the recommendation is performed on the basis of the whole flow of measures, we formulate consistency condition \eqref{def_mean_field_sol:cons} with respect to conditioning on the whole flow.
Regarding the strategy of the deviating player, as in the $N$-player game, if the player deviates, she chooses her strategy on her own, without using any of the information carried by $\Lambda$ or $\mu$:
the only information she has about $\Lambda$ or $\mu$ comes from the knowledge of the distribution $\prob^0$, which is assumed to be known by the representative player, in analogy to the $N$-player game.



\begin{remark}[Relation with MFGs with common noise]
Given the consistency condition \eqref{def_mean_field_sol:cons}, it is worth comparing coarse correlated solutions to the MFG and solutions to MFGs with common noise (see, e.g., \cite{carmona2016commonnoise,lacker2016,lacker_leflem2022} and in \cite[Volume II]{librone}).
In the latter, the flow of measures is stochastic due to a common noise that equally impacts the state dynamics of all players in the underlying $N$-player game.
As a consequence, the flow of measures is expected to be adapted to the filtration generated by the common noise (the so called \emph{strong solutions}); if this is not the case, compatibility conditions between the noises and the flow of measures itself are needed in order to guarantee that the flow $\mu$ picks into the future in a minimal way (the so called \emph{weak solutions}).
In the case of a coarse correlated solution to the MFG, on the other hand, the flow of measures is expected to be stochastic as a result of the mediator's randomization only, which is generated before the beginning of the game. More formally, this implies that the flow of measure is $\F^{0-}$-measurable with $\F^{0-} \subseteq \F_t$ for any $t \geq 0$. Recommendations to the representative player are given according to the mediator's idea of the whole flow, which leads to the consistency condition with conditioning with respect to the whole flow up to terminal time. In this sense, the mediator sees into the future, and consequently no compatibility condition is needed.

One might be tempted to regard the randomness driving the mediator's lottery for selecting recommendations as a common noise that affects the state dynamics only through the control. There are at least two major differences though: First, such a common noise will have no impact on the controls of a deviating player. To put it differently, only the pre-committing players' dynamics are directly affected by the mediator's lottery over strategy profiles.
Second, such a common noise would not be exogenous; instead, it is built into the correlation device used by the mediator, as represented by the auxiliary probability space $(\Omega^0,\F^{0-},\prob^0)$, and as such is part of the solution.
\end{remark}


\begin{example}[Admissible recommendations]\label{mf:example:admissible_recommendations}
Fix a complete probability space $(\Omega^0,\F^{0-},\prob^0)$.
We provide some simple examples of random variables $\Lambda:(\Omega^0,\F^{0-},\prob^0) \to (\A,\boreliani{\A})$ which are admissible recommendations in the sense of Definition \ref{mf:admissible_recommendation}.

\begin{enumerate}[wide]
\item \label{mf:example:finitely_many_values}
Suppose that $\Lambda$ takes only finitely many values, say $\alpha^1,\dots,\alpha^k \in \A$, $k \geq 1$, i.e. $\prob^0(\Lambda=\alpha^i)=p_i$, with $p_i \geq 0$ for every $i=1,\dots,k$, $\sum_{i=1}^k p_i=1$.
We can easily define the associated strategy $(\lambda_t)_{t \in [0,T]}$ as
\begin{equation*}
    \lambda_t(\omega_0,\omega_*)=\sum_{i=1}^k \1_{\insieme{\Lambda=\alpha^i}}(\omega_0)\alpha^i_t(\omega_*).
\end{equation*}
We explicit the dependence upon the scenario $\omega=(\omega_0,\omega_*)$:
\begin{equation*}
    \Lambda(\omega_0)_t(\omega_*) = \sum_{i=1}^k \1_{\insieme{\alpha^i}}(\Lambda(\omega_0))\alpha^i_t(\omega_*).
\end{equation*}
By the same line of reasoning of Remark \ref{finite_players:remark_estensioni}, we have that this process is $\mathbb{F}$-progressively measurable, since the processes $\alpha^i$, $i=1,\dots,k$, are $\mathbb{F}$-progressively measurable and the $\F^{0-}$-measurable real-valued random variables $\1_{\{\alpha^1\}}(\Lambda(\omega_0))$ can be regarded as defined on the product space $\Omega^0\times\Omega^*$ and $\F^{0-}\otimes \{\emptyset, \Omega^*\}$-measurable, therefore $\mathbb{F}$-progressively measurable.
Finally, condition \eqref{mf:uguaglianza_ammissibilita} is satisfied by $\lambda$ itself.

\item Suppose $\Lambda$ takes at most countably many values.
We can define $\lambda=(\lambda_t(\omega_0,\omega_*))_{t \in [0,T]}$ as
\begin{equation*}
    \lambda_t(\omega_0,\omega_*)= \sum_{i=1}^\infty\1_{\insieme{\Lambda=\alpha^i}}(\omega_0)\alpha^i_t(\omega_*).
\end{equation*}
Set $\lambda^n_t(\omega_0,\omega_*)=\sum_{i=1}^n \1_{\{\Lambda=\alpha^i\}}(\omega_0)\alpha^i_t(\omega_*)$ and observe that, by the same argument of the previous point, $\lambda^n_t$ is an $\mathbb{F}$-progressively measurable process for each $n\geq 1$.
Furthermore, for each $(t,\omega_0,\omega_*) \in [0,T]\times\Omega^0\times\Omega^*$, the sequence $\lambda^n_t(\omega_0,\omega_*)$ is eventually constant, being $(\{\Lambda=\alpha^i\})_{i\geq 1}$ a partition of $\Omega^0$.
Therefore, the sequence $\lambda^n$ converges pointwise to $\lambda=(\lambda_t)_{t \in [0,T]}$.
Being $\lambda$ the pointwise limit of $\lambda^n$, we deduce that $\lambda$ is a progressively measurable process with values in $A$ which satisfies \eqref{mf:uguaglianza_ammissibilita}, so that $\Lambda$ is admissible.

\item Let $(\Omega^0,\F^{0-},\prob^0)$ be a complete probability space, with $\Omega^0$ Polish and $\F^{0-}$ the corresponding Borel $\sigma$-algebra, and let $(\lambda_t)_{t \in [0,T]}$ be an $A$-valued process defined on the $\prob^0\otimes\prob^*$-completion of the product space $(\Omega^0\times\Omega^*,\F^{0-}\otimes\F,\prob^0\otimes\prob^*)$ with values in $A$.
Assume that it is progressively measurable with respect to the $\prob^0\otimes\prob^*$-augmentation of the filtration $\mathbb{F}=(\F^{0-}\otimes\F^*_t)_{t \in [0,T]}$.
We can define a function $\Lambda:\Omega^0\to \A$ by setting
\begin{equation}\label{raccomandazione_indotta}
\begin{aligned}
    \Lambda(\omega_0)=\left \{ \: \begin{aligned}
        & \begin{aligned}
            (\lambda_{t}(\omega_0,\cdot))_{t \in [0,T]} &  :\space [0,T] \times \Omega^* \to A \\
        & (t,\omega_*) \to \lambda_t(\omega_0,\omega_*), 
        \end{aligned} &&  \omega^0 \in \Omega^0\setminus N, \\ 
        & a_0 && \omega_0 \in N.
    \end{aligned} \right.
\end{aligned}
\end{equation}
where $N\subset \Omega^0$ is a $\prob^0$-null set and $a_0$ is an arbitrary point in $A$.
By Lemma \ref{esempi:lemma_misurabile} in Appendix \ref{appendix_recommendations}, the pair $((\Omega^0,\F^{0-},\prob^0),\Lambda)$ is an admissible recommendation, with strategy associated to the recommendation $\Lambda$ given by the process $\lambda$ itself.
\end{enumerate}
\end{example}



\section{Approximate $N$-player coarse correlated equilibria}\label{sezione_approssimazione}



The next result shows how to construct a sequence of approximate $N$-player coarse correlated equilibria with approximation error tending to zero as $N \to \infty$, provided we have a coarse correlated solution to the mean field game.

\begin{thm}\label{thm_approssimazione}
Let $((\Omega^0,\F^{0-}$, $\prob^0),\Lambda^*,\mu^*)$ be a coarse correlated solution of the mean field game.
For each $N \geq 2$, there exist:
\begin{enumerate}[label=(\roman*)]
    \item an admissible recommendation to the $N$ players $((\Omega^{0,N},\F^{0-,N},\prob^{0,N}),\Lambda^N)$;

    \item a real valued $\eps_N \geq 0$, with $\eps_N \to 0$ as $N \to \infty$,
\end{enumerate}
so that $(\mathfrak{B}_N,(\Omega^{0,N},\F^{0-,N},\prob^{0,N}),\Lambda^N)$ is an $\eps_N$-coarse correlated equilibrium for the $N$-player game.
\end{thm}


\subsection{Construction of the admissible recommendation profiles to the $N$-player game}\label{sezione_approssimazione:sezione_costruzione_raccomandazioni}


With respect to the probability space $(\Omega^1,\F^1,\prob^1)$ defined in \eqref{finite_players:canonical_space}, let us denote by $\mathbb{F}^{(i)}$ the $\prob^1$-augmentation of the filtration generated by $(\xi^i,W^i)$.
Let us introduce the following set of strategies:
\begin{equation}\label{approximation:strategy_sets}
\begin{aligned}
    & \A_{(i)}=\insieme{\alpha \in \A_N \space \; \vert \; \alpha \text{ is $\mathbb{F}^{(i)}$ progressively measurable}}.
\end{aligned}
\end{equation}
We stress that, by construction, for each $N \geq 2$, open loop strategies for the $N$-player game are defined on the same probability space $(\Omega^1,\F^1,\prob^1)$ and we have the inclusions $\A_N\subseteq \A_{N+1}$ and $\A_{(i)}\subseteq \A_N$ for every $i \leq N$.

\bigskip
We build a probability space $(\overline{\Omega},\overline{\F},\overline{\prob})$ large enough to carry any sequence $(\Lambda^i)_{i \geq 1}$ of admissible recommendations such that
\begin{enumerate}
    \item for every $i$, $\Lambda^i$ is supported on the set $\A_{(i)}$;
    \item for every $i$, $\Lambda^i$ has the same distribution as $\Lambda^*$,
    \item for every $N \geq 2$, $(\Lambda^1,\dots,\Lambda^N)$ is exchangeable.
\end{enumerate}
Then, we define the probability space $(\Omega^{0,N},\F^{0-,N},\prob^{0,N})$ as $(\overline{\Omega},\F^{0-,N},\overline{\prob})$, where $\F^{0-,N}$ is a suitable sub $\sigma$-field of $\overline{\F}$ that will be defined below.

\bigskip
Let us denote by $\rho \in \probmeasures{\contpdue}$ the distribution of $\mu^*$.
Let
\begin{equation*}
    K:\F^{0-} \times\contpdue \to [0,1]
\end{equation*}
be the regular conditional probability of $\prob^0$ given $\mu^*$, which exists and is unique since both $(\Omega^0,\F^{0-},\prob^0)$ and $(\contpdue,\mathcal{B})$ are Polish spaces.
Here and in the following, $\mathcal{B}$ stands for the Borel $\sigma$-algebra on $\contpdue$.
Let $\gamma$ denote the joint law of $(\Lambda^*,\mu^*)$ under $\prob^0$, let $\kappa$ be a version of the regular conditional probability of $\gamma$ given $\mu^*$, that is, the stochastic kernel $\kappa:\boreliani{\A}\times\contpdue \to [0,1]$ so that it holds
\begin{equation}\label{approximation:legge_soluzione_disintegrate}
    \prob^0\tonde{(\Lambda^*,\mu^*) \in C \times B}=\int_B \kappa(C,m)\rho(dm) \quad \forall C \in \boreliani{\A}, \; \forall B \in \mathcal{B}.
\end{equation}
Define the probability space $\tonde{\overline{\Omega},\overline{\F}}$ in the following way:
\begin{equation}\label{approximation:spazio_raccomandazioni}
\begin{aligned}
    & \overline{\Omega}= \tonde{ \bigtimes_{1}^\infty \Omega^0} \times \contpdue,
    && \overline{\F}= \tonde{ \bigotimes_{1}^\infty \F^{0-} } \otimes \mathcal{B},
\end{aligned}
\end{equation}
and define $\overline{\prob}$ so that, for every cylinder $R$ with basis $A_1 \times \cdots \times A_N \times B $, with $A_i \in \F^{0-}$ for every $i = 1, \dots, N$, $N \geq 2$, $B \in \mathcal{B}$, it holds
\begin{equation}\label{approximation:legge_spazio_raccomandazioni}
    \overline{\prob}\tonde{ R }=\int_{B} \prod_{i=1}^N K\tonde{A_i,m}\rho(dm).
\end{equation}
We complete the space $(\overline{\Omega},\overline{\F},\overline{\prob})$ with the $\overline{\prob}$-null sets.
Let $\overline{\omega}=((\omega_0^i)_{i \geq 1},m)$ denote a scenario in $\overline{\Omega}$.
Let $\mu: (\overline{\Omega},\overline{\F},\overline{\prob}) \to (\contpdue,\mathcal{B})$ be the projection on $\contpdue$, that is
\begin{equation}\label{approximation:approx_misura}
    \mu\tonde{\overline{\omega}}=m.
\end{equation}

\begin{lemma}\label{approximation:proprieta_raccomandazioni_N_player}
There exists a sequence of recommendations $(\Lambda^i)_{i \geq 1}$ from $(\overline{\Omega},\overline{\F},\overline{\prob})$ to $\bigtimes_{N=1} ^\infty \A_N$ so that, for each $i \geq 1$, the following holds:
\begin{enumerate}[label=(\alph*)]
    \item \label{approximation:proprieta_raccomandazioni_N_player:ammissibilita} $\Lambda^i$ is an admissible recommendation, and it takes values in $\A_{(i)}$.
    \item \label{approximation:proprieta_raccomandazioni_N_player:legge_congiunta_profilo} The joint law of $(\Lambda^{1},\dots,\Lambda^{N})$ under $\overline{\prob}$ is supported on $\bigtimes_{i=1}^N\A_{(i)}\subseteq \A_N^N$ and it is given by
    \begin{equation}\label{legge_eps_equilibrio}
        \gamma_N\tonde{d\alpha^1,\dots,d\alpha^N}= \int_{\contpdue} \bigotimes_{i=1}^N \kappa\tonde{d\alpha^i,m}\rho(dm).
     \end{equation}
     As a consequence, for every $i \geq 1$, $(\Lambda^i,\mu)$ has the same distribution as $(\Lambda^*,\mu^*)$ and  $(\Lambda^i)_{i \geq 1}$ are conditionally independent given $\mu$.
\end{enumerate}
\end{lemma}

\begin{proof}
Recall from \eqref{finite_players:canonical_space} and \eqref{mf:canonical_space} the definitions of the spaces $(\Omega^1,\F^1,\prob^1)$ and $(\Omega^*,\F^*,\prob^*)$.
Observe that, up to completion, it holds
\begin{equation}\label{approximation:spazio_rumori}
    \tonde{\Omega^1,\F^1,\prob^1}= \bigotimes_{1}^\infty \tonde{\Omega^*,\F^*,\prob^*}
\end{equation}
so that a scenario $\omega_1 \in \Omega^1$ can be written as $\omega_1=(\omega^j_*)_{j \geq 1}$.
Moreover, by definition of $(\xi^j,W^j)_{j \geq 1}$ in \eqref{canonical_browian_motions}, for every $i \geq 1$ it holds
\begin{equation*}
    (\xi^i,W^i)(\omega_1)=(x^i,w^i)=(\xi^*,W^*)(\omega^i_*),
\end{equation*}
so that $(\xi^i,W^i)_{i\geq 1}$ can be seen as a sequence of independent copies of $(\xi^*,W^*)$.
Define the filtered probability space $(\Omega,\F,\mathbb{F},\prob)$ as in point \ref{finite_players:condizione_ammissibilita} of Definition \ref{def_correlated_strategy}.
Let $\lambda^i=(\lambda^i_t)_{t \in [0,T]}$, $i \geq 1$, be independent copies of $\lambda^*=(\lambda^*_t)_{t \in [0,T]}$, the strategy associated to the admissible recommendation $\Lambda^*$ according to \eqref{mf:controllo_indotto}, so that
\begin{equation*}
    \lambda^i_t(\overline{\omega},\omega_1)= \lambda^i_t((\omega^j_0)_{j \geq 1},m,(\omega^j_*)_{j \geq 1})=\lambda^*_t\tonde{\omega_0^i,\omega^i_*}.
\end{equation*}
For every $i$, $\lambda^i$ is $\mathbb{F}$-progressively measurable: indeed, since by definition the measures $\prob$ and $\prob^0 \otimes \prob^*$ coincide on the cylinders $A_i$ of the form
\begin{equation}\label{approximation:cilindro_iesimo}
    A_i=\insieme{ (\overline{\omega},\omega_1)=((\omega_0^j)_{j \geq 1} ,m,(\omega_*^j)_{j \geq 1}) \in \overline{\Omega}\times\Omega^1 \; \vert \; (\omega_0^i,\omega_*^i) \in G},
\end{equation}
for any $G \in \F^{0-} \otimes \F^1$, every $\prob^0 \otimes \prob^*$-null set $N$ can be identified with a $\prob$-null cylinder $A_i$ of the form \eqref{approximation:cilindro_iesimo} with basis $N$.
Therefore, for every $t \in [0,T]$, the $\prob$-augmentation of the filtration $\F^{0-}\otimes\F^i_t$ contains all the cylinders with basis 
\begin{equation*}
    A_i=\insieme{ (\overline{\omega},\omega_1)=((\omega_0^j)_{j \geq 1} ,m,(\omega_*^j)_{j \geq 1}) \in \overline{\Omega}\times\Omega^1 \; \vert \; (\omega_0^i,\omega_*^i) \in G},
\end{equation*}
for any $G$ in the $\prob^0 \otimes \prob^*$-augmentation of $\F^{0-} \otimes \F^*_t$. 
This is enough to conclude that $\lambda^i$ is progressively measurable with respect to the $\prob$-augmentation of $\F^{0-} \otimes \F^i_t$, and so with respect to the filtration $\mathbb{F}$ as well.
We define $\Lambda^i$ as in \eqref{raccomandazione_indotta}, that is
\begin{equation}\label{approximation:approx_raccomandazione}
\begin{aligned}
    \Lambda^i(\overline{\omega})=\left \{ \: \begin{aligned}
        & \begin{aligned}
            (\lambda^i_{t}(\overline{\omega},\cdot))_{t \in [0,T]} &  :\space [0,T] \times \Omega^1 \to A \\
        & (t,\omega_1) \to \lambda^i_t(\overline{\omega},\omega_1), 
        \end{aligned} &&  \overline{\omega} \in \overline{\Omega}\setminus \mathcal{N}, \\
        & a_0 && \overline{\omega} \in \mathcal{N},
    \end{aligned} \right.
\end{aligned}
\end{equation}
where $\mathcal{N} \subseteq \overline{\Omega}$ is a $\overline{\prob}$-null set and $a_0$ is an arbitrary point in $A$.
By Lemma \ref{esempi:lemma_misurabile}, $\Lambda^i$ is an admissible recommendation from $(\overline{\Omega},\overline{\F},\overline{\prob})$ to $(\A_N,\boreliani{\A_N})$, for every $N \geq i$.
Since the associated strategies coincide pointwise, it holds $\Lambda^i(\overline{\omega})=\Lambda^*(\omega_0^i)$ 
$\overline{\prob}$-a.s., as ensured by Proposition \ref{esempi:unicita_strategia_associata}.
In particular, this implies that $\Lambda^i$ only takes values in $\A_{(i)}$, since for every fixed $\overline{\omega}$ the control process $(\lambda^i_{t}(\overline{\omega},\cdot))_{t \in [0,T]}$ is $\mathbb{F}^{(i)}$-progressively measurable.
This proves point \ref{approximation:proprieta_raccomandazioni_N_player:ammissibilita}.

As for point \ref{approximation:proprieta_raccomandazioni_N_player:legge_congiunta_profilo}, for every $N \geq 2$, $(\Lambda^1,\dots,\Lambda^N)$ takes values in $\bigtimes_{j=1}^N\A_{\tonde{j}}$ by construction.
Hence, we may restrict the attention to Borel sets $C_j \subseteq \A_{\tonde{j}}$, for every $j=1,\dots,N$.
Let $B \in \mathcal{B}$.
Since  $\Lambda^j(\overline{\omega})=\Lambda^*(\omega^j_0)$ for every $j=1,\dots,N$ $\overline{\prob}$-a.s., by definition of $\overline{\prob}$, we have
\begin{equation*}
\begin{aligned}
    \overline{\prob} & \tonde{\Lambda^{1} \in C_1, \dots, \Lambda^N \in C_N, \mu \in B} = \overline{\prob}\tonde{\bigtimes_{j=1}^N\insieme{\omega^j_0: \; \Lambda^*(\omega_0^j) \in C_j} \; \times \bigtimes_{j=N+1}^\infty \Omega^0 \; \times \;  \contpdue } \\
    & = \int_B \prod_{j=1}^N K\tonde{\insieme{\omega^j_0:\;\Lambda^* (\omega_0^j) \in C_j},m} \rho(dm) = \int_B \prod_{j=1}^N K\tonde{\insieme{\omega_0:\;\Lambda^*\tonde{\omega_0} \in C_j},m} \rho(dm)  \\
    & = \int_{B} \prod_{j=1}^N \kappa\tonde{C_j, m} \rho(dm).
\end{aligned}
\end{equation*}
This shows also that $(\Lambda^i,\mu)$ are identically distributed as $(\Lambda^*,\mu^*)$ and that $(\Lambda^i)_{i\geq 1}$ are conditionally i.i.d. given $\mu$.
\end{proof}
For each $N\geq 2$, set $\F^{0-,N}=\sigma(\Lambda^1,\dots,\Lambda^N)$ and $(\Omega^{0,N},\F^{0-,N},\prob^{0,N})=(\overline{\Omega},\F^{0-,N},\overline{\prob})$.
Then,  
\begin{equation*}
    \tonde{\Lambda^1,\dots,\Lambda^N}:\tonde{\overline{\Omega},\F^{0-,N},\overline{\prob}} \to \tonde{\A_N^N, \boreliani{\A_N^N}}
\end{equation*}
is the candidate $\eps_N$-coarse correlated equilibrium to the $N$-player game, with $\eps_N$ to be determined.

\begin{remark}
The construction of the probability spaces $(\Omega^{0,N},\F^{0-,N},\prob^{0,N})$ is rather involved, but has the advantage of making the admissible recommendation to the $N$-players $(\Lambda^1,\dots,\Lambda^N)$ easy to define.
Besides this technical reason, we notice that, both in the $N$-player game and in the mean field game, the mediator may choose the space $(\Omega^0,\F^{0-},\prob^0)$ he uses to randomize players' strategies, as already pointed out in Sections \ref{sezione_formulazione_N_giocatori} and \ref{sezione_formulazione_mfg}.
Then, it is natural to use the same space $(\Omega^0,\F^{0-},\prob^0)$ on which the coarse correlated solution to the mean field game is defined to randomize players' strategies in the $N$-player game as well.
\end{remark}


\subsection{Proof of Theorem \ref{thm_approssimazione}}
By symmetry, let us consider only possible deviations of player $i=1$.
For every $N\geq 2,$ let $\eps_N$ be given by
\begin{equation}
    \eps_N :=\sup_{\beta \in \A_N}\tonde{\J^N_1(\Lambda^N) - \J^N_1(\Lambda^{N,-1},\beta)}=\J^N_1(\Lambda^N) - \inf_{\beta \in \A_N}\J_N^1(\Lambda^{N,-1},\beta).
\end{equation}
$\Lambda^N$ is an $\eps_N$-coarse correlated equilibrium for every $N \geq 2$.
We must show that $\eps_N \overset{N \to \infty}{\longrightarrow} 0$.
For each $N \geq 2$, choose $\beta^N \in \A_N$ so that
\begin{equation*}
    \J^N_1(\Lambda^{N,-1},\beta^N) \leq  \inf_{\beta \in \A_N}\J^N_1(\Lambda^{N,-1},\beta) - \frac{1}{N}.
\end{equation*}
Let $Z=(Z_t)_{t \in [0,T]}$ be the solution of
\begin{equation}\label{approximation:equazione_Z}
    dZ_t=b(t,Z_t,\mu_t,\beta^N_t)dt + dW^1_t, \quad Z_0=\xi^1,
\end{equation}
and define the corresponding cost as
\begin{equation*}
    \J(\beta^N,\mu)=\E\quadre{\int_0^T f(s,Z_s,\mu_s,\beta^N_s)ds + g(Z_T,\mu_T)}.
\end{equation*}
Let $X$ be the solution of 
\begin{equation}\label{equazione_ottima}
    dX_t=b(t,X_t,\mu_t,\lambda^1_t)dt + dW^1_t, \quad X_0=\xi^1,
\end{equation}
with associated cost
\begin{equation*}
    \J(\Lambda^1,\mu)=\E\quadre{\int_0^T f(s,X_s,\mu_s,\lambda^1_s)ds + g(X_T,\mu_T)}.
\end{equation*}
Observe that, by construction,  $(\xi^1,W^1,\mu,\lambda^1)$ under $\prob$ is distributed as $(\xi,W,\mu^*,\lambda^*)$ under $\prob^*$.
Therefore, by Theorem \ref{teorema_di_unicita_legge} the joint distribution of $(X,\lambda^1,\mu)$ under $\prob$ is the same as $(X^*,\lambda^*,\mu^*)$ under $\prob^*$, where $X$* denotes the state process resulting from the mean field CCE $(\Lambda^*,\mu^*)$.
Moreover, note that by construction $\lambda^1$ is $\mathbb{F}^{\mu,\xi^1,W^1}$-progressively measurable, where
\begin{equation*}
    \F^{\mu,\xi^1,W^1}_t= \sigma(\mu)\vee\sigma(\xi^1)\vee\sigma(W^1_s: \; s \leq t), \; t \in [0,T].
\end{equation*}
By the Lipschitz continuity of $b$, $X$ may be taken to be $\mathbb{F}^{\mu,\xi^1,W^1}$-adapted as well.

\bigskip
To prove the theorem, the following must be shown:
\begin{subequations}
\label{conv_costi:optim}
\begin{align}
    & \J(\Lambda^1,\mu) \leq \J(\beta^N,\mu) \label{conv_costi:minimo_limite}, \\
    & \lim_{N \to \infty} \J_N^1(\Lambda^N)=\J(\Lambda^1,\mu),  \label{conv_costi:convergenza_lambda} \\
    & \lim_{N \to \infty} \abs{\J_N^1(\Lambda^{N,-1},\beta^N) - \J(\beta^N,\mu)} = 0,  \label{conv_costi:convergenza_beta}.
\end{align}
\end{subequations}
If these hold, then we can conclude as follows:
\begin{equation*}
    \begin{aligned}
        \eps_N &=  \J^1_N(\Lambda^N) - \inf_{\beta \in \A_N}\J_N^1(\Lambda^{N,-1},\beta) \leq  \J^1_N(\Lambda^N) - \J^1_N(\Lambda^{N,-1},\beta_N) - \frac{1}{N} \\
        &\leq  \J^1_N(\Lambda^N) - \J(\Lambda^1,\mu) +\J(\Lambda^1,\mu) - \J(\beta^N,\mu) + \J(\beta^N,\mu) - \J^1_N(\Lambda^{N,-1},\beta^N) - \frac{1}{N} \\
        & \leq  \J^1_N(\Lambda^N) - \J(\Lambda^1,\mu) + \J(\beta^N,\mu) - \J^1_N(\Lambda^{N,-1},\beta^N) 
    \end{aligned}
\end{equation*}
and, taking the $\limsup$, we have $\limsup_N \eps_N  \leq 0$, which proves $\eps_N \to 0$ as $N \to \infty$.

\bigskip
We start by proving \eqref{conv_costi:minimo_limite}.
We observe that we cannot just deduce it from the optimality property \eqref{def_mean_field_sol:opt} of $(\Lambda^*,\mu^*)$: since $\beta^N$ may belong to $\A_N \setminus \A_{(1)}$, it may not be identifiable with an open loop strategy for the MFG, for which inequality \eqref{conv_costi:minimo_limite} would hold.
Instead, we prove it by using the regular conditional probability of $\prob$ given $(\xi^i,W^i)_{i=2}^N$.
Denote by $(\bm{x},\bm{w})$ a point $(x^i,w^i)_{i=2}^N \in (\R^d \times \contrd)^{N-1}$, and let $P_\nu=\bigotimes_1^{N-1}(\nu \otimes \wienermeasure)$ denote the joint law of $(\xi^i,W^i)_{i=2}^N$ under $\prob$.
Let $\prob^{\bm{x},\bm{w}}$ be a version of the regular conditional probability of $\prob$ given $(\xi^i,W^i)_{i=2}^N=(x^i,w^i)_{i=2}^N$.
We rewrite \eqref{conv_costi:minimo_limite} as
\begin{equation}\label{conv_costi:int_disintegrato}
\begin{aligned}
    \J & (\Lambda^1,\mu)-\J(\beta^N,\mu) =\\
     & = \E \quadre{\tonde{\int_0^T f(s,X_s,\mu_s,\lambda^1_s)ds + g(X_T,\mu_T)} - \tonde{\int_0^T f(s,Z_s,\mu_s,\beta^N_s)ds + g(Z_T,\mu_T) }} \\
    & =  \E  \left[ \E\left[ \tonde{\int_0^T f(s,X_s,\mu_s,\lambda^1_s)ds + g(X_T,\mu_T)} \right. \right. \\
    & \quad \left. \left. - \tonde{\int_0^T f(s,Z_s,\mu_s,\beta^N_s)ds + g(Z_T,\mu_T)} \Big \vert (\xi^i,W^i)_{i=2}^N \right] \right] \\
    & = \int_{(\R^d \times \contrd)^{N-1}} \left( \E^{\prob^{\bm{x},\bm{w}}}\left[\int_0^T f(s,X_s,\mu_s,\lambda^1_s)ds + g(X_T,\mu_T) \right] \right. \\
    & \quad - \left. \E^{\prob^{\bm{x},\bm{w}}}\left[ \int_0^T f(s,Z_s,\mu_s,\beta^N_s)ds + g(Z_T,\mu_T) \right] \right) P_\nu(d\bm{x},d\bm{w}).
\end{aligned}
\end{equation}
We analyse separately the two terms in the last equality.
Let us start with the term depending upon $\lambda^1$.
Since $\mu$, $\lambda^1$, $W^1$ and $\xi^1$ are independent of $(\xi^i,W^i)_{i=2}^N$  under $\prob$ and $X$ is $\mathbb{F}^{\mu,\xi^1,W^1}$-adapted, $X$ is independent of $(\xi^i,W^i)_{i=2}^N$ as well.
We deduce that, under $\prob^{\bm{x},\bm{w}}$, $W$ is an $\mathbb{F}$-Brownian motion, $X$ solves equation \eqref{equazione_ottima} and 
$\prob^{\bm{x},\bm{w}}\circ(X,\lambda^1,\mu)^{-1}=\prob\circ(X,\lambda^1,\mu)^{-1}=\prob^*\circ(X^*,\lambda^*,\mu^*)^{-1}$, for $P_\nu$-a.e. $(\bm{x},\bm{w}) \in (\R^d \times \contrd)^{N-1}$.
In particular, this implies that
\begin{equation}\label{funzionale_condizionato1}
\begin{aligned}
    \E^{\prob^{\bm{x},\bm{w}}}& \quadre{\int_0^T f(s,X_s,\mu_s,\lambda^1_s)ds + g(X_T,\mu_T)}=\E^{\prob^*}\quadre{\int_0^T f(s,X^*_s,\mu^*_s,\lambda^*_s)ds + g(X^*_T,\mu^*_T)}\\
    & =\J(\Lambda^*,\mu^*)
\end{aligned}
\end{equation}
for $P_\nu$-a.e. $(\bm{x},\bm{w})\in (\R^d \times \contrd)^{N-1}$.

\bigskip
As for the term depending upon $\beta^N$, we note that, since $\beta^N \in \A_N$, there exists a progressively measurable functional $\hat{\beta}:[0,T] \times (\R^d \times \contrd)^N \to A$ so that
\begin{equation*}
    \beta^N_t=\hat{\beta}\tonde{t,\xi^1,W^1,\dots,\xi^N,W^N}.
\end{equation*}
Under $\prob^{\bm{x},\bm{w}}$, it holds
\begin{equation}\label{strategia_condizionata1}
    \beta^N_t= \hat{\beta}\tonde{t,\xi^1,W^1,x^2,w^2,\dots,x^N,w^N} \; \forall t \in [0,T] \quad \prob^{\bm{x},\bm{w}}\text{-a.s.},
\end{equation}
since $(\xi^i,W^i)_{i=1}^2$ are almost surely constant under $\prob^{\bm{x},\bm{w}}$.
Since the joint law of $\mu$, $W^1$ and $\xi^1$ is the same under both $\prob$ and $\prob^{\bm{x},\bm{w}}$, \eqref{strategia_condizionata1} implies that $Z$ satisfies
\begin{equation}\label{conv_costi:differenziale_zeta}
    dZ_t=b(t,Z_t,\mu_t,\hat{\beta}^N_t(\bm{x},\bm{w}))dt + dW^1_t, \quad Z_0=\xi^1,
\end{equation}
under $\prob^{\bm{x},\bm{w}}$, with $\hat{\beta}^N(\bm{x},\bm{w})$ $\mathbb{F}^{(1)}$-progressively measurable.
For every $(\bm{x},\bm{w}) \in (\R^d \times \contrd)^{N-1}$, define the strategy
\begin{equation}
    \Tilde{\beta}\tonde{\bm{x},\bm{w}}=\tonde{ \hat{\beta}\tonde{t,\xi,W,x^2,w^2,\dots,x^N,w^N}}_{t \in [0,T]}.
\end{equation}
Then $\Tilde{\beta}(\bm{x},\bm{w})$ belongs to $\A$  for every $(\bm{x},\bm{w})\in(\R^d\times\contrd)^{N-1}$, and it depends measurably upon $(\bm{x},\bm{w})$.
For every $(\bm{x},\bm{w})$, let $\Tilde{Z}$ be the solution of 
\begin{equation*}
    d\Tilde{Z}_t=b(t,\Tilde{Z}_t,\mu^*_t,\Tilde{\beta}_t(\bm{x},\bm{w}))dt + dW_t, \quad \Tilde{Z}_0=\xi.
\end{equation*}
Since $\prob^{\bm{x},\bm{w}}\circ(Z,\beta,\mu)^{-1}=\prob^{\bm{x},\bm{w}}\circ(Z,\hat{\beta}^N(\bm{x},\bm{w}),\mu)^{-1}=\prob^*\circ(\Tilde{Z},\Tilde{\beta}(\bm{x},\bm{w}),\mu^*)^{-1}$, it follows that 
\begin{equation}\label{funzionale_condizionato2}
\begin{aligned}
    \E^{\prob^{\bm{x},\bm{w}}} & \quadre{\int_0^T f(s,Z_s,\mu_s,\beta^N_s)ds + g(Z_T,\mu_T)}  = \E^{\prob^{\bm{x},\bm{w}}}\quadre{\int_0^T f(s,Z_s,\mu_s,\hat{\beta}^N_s(\bm{x},\bm{w})) ds + g(Z_T,\mu_T)}\\
    & = \E^{\prob^*} \quadre{\int_0^T f(s,\hat{Z}_s,\mu^*_s,\Tilde{\beta}_s(\bm{x},\bm{w}))ds + g(\hat{Z}_T,\mu^*_T) } =\J(\Tilde{\beta}(\bm{x},\bm{w}),\mu^*).
\end{aligned}
\end{equation}
We note that the left-hand side of \eqref{funzionale_condizionato1} depends measurably upon $(\bm{x},\bm{w})$ due to a monotone class argument.
Being $(\Lambda^*,\mu^*)$ a mean field CCE by assumption, \eqref{funzionale_condizionato1} and \eqref{funzionale_condizionato2} imply that
\begin{equation*}
\begin{aligned}
    \J(\Lambda^1,\mu)-\J(\beta^N,\mu) = & \int  \left( \E^{\prob^{\bm{x},\bm{w}}}\left[\int_0^T f(s,X_s,\mu_s,\lambda^1_s)ds + g(X_T,\mu_T) \right] \right. \\
    & - \left. \E^{\prob^{\bm{x},\bm{w}}}\left[ \int_0^T f(s,Z_s,\mu_s,\beta^N_s)ds + g(Z_T,\mu_T) \right] \right) P_\nu(d\bm{x},d\bm{w}) \\
    = & \int  \left( \J(\Lambda^*,\mu^*) -\J(\Tilde{\beta}(\bm{x},\bm{w}),\mu^*) \right) P_\nu(d\bm{x},d\bm{w}) \leq 0,
\end{aligned}
\end{equation*}
which yields \eqref{conv_costi:minimo_limite}.

\bigskip
As for \eqref{conv_costi:convergenza_lambda} and \eqref{conv_costi:convergenza_beta}, they must be handled by continuity arguments on the cost functions and propagation of chaos as stated in Lemma \ref{lemma_poc}.
We give the details only for \eqref{conv_costi:convergenza_lambda}, since \eqref{conv_costi:convergenza_beta} is analogous.
We have:
\begin{equation*}
\begin{aligned}
    \abs{\J^1_N(\Lambda^N) - \J(\Lambda^1,\mu)} & \leq \E\left[\int_0^T \abs{f(t,X^{1,N}_t,\mu^N_t,\lambda^1_t) - f(t,X_t,\mu_t,\lambda^1_t) } ds \right. \\
     & \quad +  \abs{g(X^N_T,\mu_T^N) - g(X_T,\mu_T)} \bigg ] = \attesa{\Delta f + \Delta g}.
\end{aligned}
\end{equation*}
For $\Delta f$, Assumptions \ref{standing_assumptions} ensure that $f$ is locally Lipschitz with at most quadratic growth.
Therefore, by straightforward estimates, we have:
\begin{equation*}
\begin{aligned}
    \E & \quadre{\Delta f} \leq L \E \left[ \int_0^T \tonde{1 + \abs{X^{1,N}_t} + \tonde{\int_{\R^d}\abs{y}^2\mu^N_t(dy)}^\frac{1}{2} + \abs{X_t} + \tonde{\int_{\R^d}\abs{y}^2\mu_t(dy)}^\frac{1}{2} + 2 \abs{\lambda^1_t} }\right. \\
    & \quad \cdot \E \left[ \int_0^T \tonde{\abs{X^N_t - X_t} + \mathcal{W}_{2,\R^d}(\mu^N_t,\mu_t) }^2 dt \right] \\
    & \leq C \E \left[ \norm{X^{1,N}}_{\contrd}^2 + \norm{X}_{\contrd}^2 +  \int_0^T \int_{\R^d}\abs{y}^2 \mu_t(dy)dt + \int_0^T \int_{\R^d}\abs{y}^2 \mu^N_t(dy)dt + 2 \int_0^T \abs{\lambda^1_t} dt \right]^\frac{1}{2} \\
    & \quad \cdot \E\left[  \tonde{\norm{X^{1,N} - X}^2_{\contrd}  + \int_0^T\pwassmetric{2}{\R^d}{2}(\mu^N_t,\mu_t)dt}  \right]^\frac{1}{2} \\
    & \leq  C \tonde{1 + \max_{k=1,\dots,N}\E \quadre{ \norm{X^{1,N}}_{\contrd}^2}^\frac{1}{2} + \E \quadre{\norm{X}_{\contrd}^2}^\frac{1}{2}  }\\
    & \quad \cdot \tonde{ \E\quadre{ \norm{X^{1,N} - X}_{\contrd}^2 }^\frac{1}{2}+ \sup_{t \in [0,T]}\attesa{\pwassmetric{2}{\R^d}{2}(\mu^N_t,\mu_t)}^{\frac{1}{2}} }. \\
\end{aligned}
\end{equation*}
By Lemma \ref{lemma_poc}, the right-hand side tends to $0$ as $N$ goes to infinity.
The convergence of $\E[\Delta g]$ is shown analogously.

\section{Existence of a coarse correlated solution of the mean field game}\label{sezione_existence}

Taking inspiration from \cite{hart_schmeidler, nowak1992correlated, Nowak1993} and \cite[Appendix 1.B]{bonesini_thesis}, we associate a zero-sum game to the search of a mean field CCE.
Loosely speaking, the game should be of the following type: player A, the maximizer, chooses a correlated flow $(\Lambda,\mu)$, while player B chooses a deviating strategy $\beta \in \A$.
The payoff functional is the following:
\begin{equation}\label{existence:payoff_informale}
    F\quadre{(\Lambda,\mu),\beta}=\J(\beta,\mu)-\J(\Lambda,\mu).
\end{equation}
Player A aims at maximizing $F$, while player B chooses her strategy in order to minimize $F$.
In order to get an equilibrium, one should restrict to correlated flows $(\Lambda,\mu)$ so that the consistency condition \eqref{def_mean_field_sol:cons} is satisfied.
If we could show that the game has a positive value and player A has an optimal strategy $(\Lambda^*,\mu^*)$ , then we would have established that such a strategy would satisfy the optimality property \eqref{def_mean_field_sol:opt} as well, and therefore $(\Lambda^*,\mu^*)$ would be a mean field CCE.
In order to get a convenient structure for the sets of strategies and good continuity and convexity properties of the payoff functionals, we embed our auxiliary problem in a more general zero-sum game which, roughly speaking, extends the payoff functional in equation \eqref{existence:payoff_informale}.
Care is needed in dealing with the term depending both on $\beta$ and $\mu$, since it must reflect independent strategy choices of the opponents. Using Fan's minimax theorem, we will show that the auxiliary game has positive value and admits an optimal strategy for the maximizing player.
Finally, such an optimal strategy is used to induce a coarse correlated solution of the mean field game.

\subsection{Relaxed controls}\label{existence:sezione_crtl_rilassati}
Since we are going to use compactness arguments, it is useful to provide some information about relaxed controls before introducing the auxiliary zero-sum game in the next section.
Relaxed controls have a long history in control theory (see, e.g., \cite{el_karoui_compactification} and \cite{haussmann1990sicom}), and also in mean field games (see the series of works \cite{delarue16commonnoise,lacker2015martingale,lacker2016,lacker2020convergence}, or \cite{campi18absorption} in a slightly different framework).
We will use them in a similar way.
\medskip

Denote by $\mathcal{V}$ the set of positive measures $q$ on $[0,T] \times A$ so that the time marginal is equal to the Lebesgue measure, i.e., $q ([s,t]\times A)=t-s$ for every $0 \leq s \leq t \leq T$.
We endow $\mathcal{V}$ with the topology of weak convergence of measures, which makes $\mathcal{V}$ a Polish space.
It is a well known fact that, when the set $A$ is compact, $\mathcal{V}$ is compact as well.
Moreover, for every measure $q \in \mathcal{V}$, there exists a measurable map $[0,T] \ni t \mapsto q_t \in \mathcal{P}(A)$ so that $q(da,dt)=q_t(da)dt$, with $(q_t)_{t \in [0,T]}$ unique up to $Leb_{[0,T]}$-a.e. equality.
We can equip the measurable space $(\mathcal{V},\boreliani{\mathcal{V}})$ with the filtration $(\F^{\mathcal{V}}_t)_{t \in [0,T]}$ defined by
\begin{equation*}
    \F^{\mathcal{V}}_t = \sigma( \mathcal{V} \ni q \mapsto q(C): \; C \in \boreliani{[0,t]\times A}).
\end{equation*}
We observe that $\F^{\mathcal{V}}_t$ is countably generated for every $t \in [0,T]$, by reasoning as in the proof of \cite[Proposition 7.25]{bertsekas_shreve}.
Finally, one can prove that there exists an $\mathcal{F}_t^{\mathcal{V}}$-predictable process $\overline{q}:[0,T]\times\mathcal{V}\to\mathcal{P}(A)$ such that, for each $q \in \mathcal{V}$, $\overline{q}(t,q)=q_t$ for a.e. $t \in [0,T]$ (see, e.g., \cite[Lemma 3.2]{lacker2015martingale}).
By an abuse of notation, we write $q_t(da)=\overline{q}(t,q)(da)$.


\medskip
Consider a filtered probability space $(\Omega,\F,\mathbb{F},\prob)$.
A relaxed control $\mathfrak{r}$ is a $\mathcal{V}$-valued random variable.
We say that $\mathfrak{r}$ is $\mathbb{F}$-adapted if $\mathfrak{r}(C)$ is a real valued $\F_t$-measurable random variable for every $C \in \boreliani{[0,t]\times A}$.
Observe that every $A$-valued progressively measurable process $\alpha=(\alpha_t)_{t \in [0,T]}$, which is often referred to as strict control, induces a relaxed control by setting
\begin{equation*}
    \mathfrak{r}_t(da)dt=\delta_{\alpha_t}(da)dt.
\end{equation*}
Finally, using the map $\overline{q}$ described above, we can safely identify every $\mathbb{F}$-adapted relaxed control $\mathfrak{r}$ with the unique (up to $Leb_{[0,T]}$-a.e. equality) $\mathbb{F}$-progressively measurable process $(\mathfrak{r}_t)_{t \in [0,T]}$ with values in $\mathcal{P}(A)$ so that
\begin{equation*}
    \prob(\mathfrak{r}(da,dt)=\mathfrak{r}_t(da)dt)=1.
\end{equation*}
In the following, we will use mostly the notation $(\mathfrak{r}_t)_{t \in [0,T]}$ for a relaxed control and will make no distinction between a $\mathcal{V}$-valued random variable and a $\mathcal{P}(A)$-valued process.

\bigskip
\subsection{The auxiliary zero-sum game}
We now formally define and study the auxiliary zero-sum game.
\begin{definition}[Strategies for player A]\label{existence:strategie_max}
A \emph{strategy for player A} is a probability measure $\Gamma \in \mathcal{P}(\contrd \times \mathcal{V} \times \contpdue)$ so that there exists a tuple $\mathfrak{U}=((\Omega,\F,\mathbb{F},\prob),\xi,W,\mu,\mathfrak{r})$ with the following properties:
\begin{enumerate}[label=(\roman*)]
    \item $(\Omega,\F,\mathbb{F},\prob)$ is a filtered probability space satisfying the usual assumptions; $\Omega$ is Polish and $\F$ is its corresponding Borel $\sigma$-algebra.
    \item $W$ is an $\mathbb{F}$-Brownian motion and $\xi$ is an $\F_0$-measurable independent $\R^d$-valued random variable with law $\nu$.
    \item $\mu$ is an $\F_0$-measurable random variable with values in $\contpdue$; it is independent of both $\xi$ and $W$.
    \item $\mathfrak{r}$ is an $\mathbb{F}$-progressively measurable relaxed control $\mathfrak{r}=(\mathfrak{r}_t)_{t \in [0,T]}$ with values in $A$.
    \item Let $X$ be the solution of
    \begin{equation}\label{existence:eq_processo_K}
        dX_t=\int_A b(t,X_t,\mu_t,a)\mathfrak{r}_t(da)dt + dW_t, \; t \in [0,T], \quad X_0=\xi.
    \end{equation}
    Then $\mu_t(\cdot)=\prob(X_t \in \cdot \; \vert \; \mu)$ $\prob$-a.s for every $t \in [0,T]$.
    
    \item $\Gamma$ is the joint law under $\prob$ of  $X$, $\mu$ and $\mathfrak{r}$: $\Gamma=\prob\circ(X,\mathfrak{r},\mu)^{-1}$.
\end{enumerate}
We denote by $\mathcal{K}$ the set of strategies for player A.
\end{definition}
We observe that, by Assumptions \ref{standing_assumptions}, there exists a unique solution to equation \eqref{existence:eq_processo_K} for every tuple $\mathfrak{U}$ satisfying properties (i-iv).

\begin{definition}[Strategies for player B]\label{existence:strategie_min}
A stochastic kernel $\Sigma$ from $\contpdue$ to $\contrd \times \mathcal{V}$ is a \emph{strategy for player B} if there exists a tuple $\mathfrak{U}=((\Omega,\F,\mathbb{F},\prob),\xi,W,\mathfrak{r})$ so that
\begin{enumerate}[label=(\roman*)]
    \item $(\Omega,\F,\mathbb{F},\prob)$ is a filtered probability space satisfying the usual assumptions; $\Omega$ is Polish and $\F$ is its corresponding Borel $\sigma$-algebra.
    \item $W$ is an $\mathbb{F}$-Brownian motion and $\xi$ is an $\F_0$-measurable independent $\R^d$-valued random variable with law $\nu$.
    \item $\mathfrak{r}$ is an $\mathbb{F}$-progressively measurable relaxed control $\mathfrak{r}=(\mathfrak{r}_t)_{t \in [0,T]}$ with values in $A$.
    \item For every $m \in \contpdue$, $\Sigma(\cdot, m) \in \mathcal{P}(\contrd \times \mathcal{V})$ is the joint law under $\prob$ of $(X^m,\mathfrak{r})$, where $X^m$ is the solution to
    \begin{equation}\label{existence:eq_processo_Q}
    dX^m_t=\int_A b(t,X^m_t,m_t,a)\mathfrak{r}_t(da)dt + dW_t, \quad X_0=\xi,
    \end{equation}
    that is:
    \begin{equation}
        \Sigma(B,m)=\prob((X^m,\mathfrak{r}) \in B) \quad \forall m \in \contpdue, B \in \boreliani{\contrd} \otimes \boreliani{\mathcal{V}}.
    \end{equation}
\end{enumerate}
We denote by $\mathcal{Q}$ the set of strategies for player B.
\end{definition}
By Lemma \ref{existence:lemma_kernels}, the set of strategies $\mathcal{Q}$ for player B is well defined in the sense that
the map $\Sigma$ is truly a stochastic kernel.

\bigskip
We now define the payoff functional $\mathfrak{p}$ for the zero-sum game.
Let us introduce the function $\mathfrak{F}: \contrd \times \mathcal{V}\times \contpdue \to  \R$ defined by
\begin{equation}\label{existence:funzione_f}
\begin{aligned}
    \mathfrak{F} (y,q,m)=\int_0^T\int_A f(t,y_t,m_t,a)q_t(da)dt + g(y_T,m_T).
\end{aligned}
\end{equation}
\begin{definition}[Auxiliary zero-sum game]\label{existence:def_zerosum}
The \emph{auxiliary zero-sum game} is a zero-sum game where:
\begin{itemize}
    \item The set of strategies for player A, the maximizer, is the set $\mathcal{K}$ introduced in Definition \ref{existence:strategie_max}.
    \item The set of strategies for player B, the minimizer, is the set $\mathcal{Q}$ introduced in Definition \ref{existence:strategie_min}.
    \item The payoff functional is the function $\mathfrak{p}:\mathcal{K}\times\mathcal{Q}\to \R$ defined as
    \begin{equation}\label{existence:payoff_functional}
    \begin{aligned}
        \mathfrak{p}(\Gamma,\Sigma) = & \int_{\contrd \times \mathcal{V} \times \contpdue} \mathfrak{F} (y,q,m) \Sigma(dy,dq,m)\rho(dm)\\
        & - \int_{\contrd \times \mathcal{V} \times \contpdue} \mathfrak{F} (y,q,m) \Gamma(dy,dq,dm),
    \end{aligned}
    \end{equation}
    where $\rho$ denotes the marginal of $\Gamma$ on $\contpdue$.
\end{itemize}
We denote the lower and upper values of the game as, respectively, $v^A$ and $v^B$: 
\begin{equation*}
\begin{aligned}
    & v^A=\sup_{\Gamma \in \mathcal{K}} \inf_{\Sigma \in \mathcal{Q}} \mathfrak{p}(\Gamma,\Sigma), && \quad 
    v^B= \inf_{\Sigma \in \mathcal{Q}} \sup_{\Gamma \in \mathcal{K}} \mathfrak{p}(\Gamma,\Sigma).
\end{aligned}
\end{equation*}
If the lower and upper values of the game are equal, we set $v=v^A=v^B$ and call $v$ the value of the game.
We say that a strategy $\Gamma^* \in \mathcal{K}$ is optimal for player A if
\begin{equation*}
    \inf_{\Sigma \in \mathcal{Q}}\mathfrak{p}(\Gamma^*,\Sigma)=\max_{\Gamma \in \mathcal{K}}\inf_{\Sigma \in \mathcal{Q}}\mathfrak{p}(\Gamma,\Sigma).
\end{equation*}
\end{definition}

\subsection{Relationship between the zero-sum game and the mean field game}
The goal of this section is to show how to use an optimal strategy for the maximizing player of the auxiliary game to induce a coarse correlated solution to the mean field game.
The next proposition shows that, for every correlated flow $(\Lambda,\mu)$ so that consistency condition \eqref{def_mean_field_sol:cons} is satisfied and every deviation $\beta \in \A$, there exists a pair of strategies $(\Gamma,\Sigma) \in \mathcal{K}\times\mathcal{Q}$ so that the following equality holds:
\begin{equation}\label{existence:prop_relazione_mfg:equazione_payoffs}
        \mathfrak{p}(\Gamma,\Sigma)=\J(\Lambda,\mu)-\J(\beta,\mu)=F[(\Lambda,\mu),\beta].
\end{equation}

\begin{prop}\label{existence:prop_relazione_mfg}
Let $((\Omega^0,\F^{0-}$, $\prob^0),\Lambda,\mu)$ be a correlated flow.
Denote by $\rho$ the law of $\mu$.
Let $\lambda=(\lambda_t)_{t \in [0,T]}$ be the strategy associated to the admissible recommendation $\Lambda$ and let $\beta \in \A$.
\begin{enumerate}[label=(\roman*)]
    \item     Let $X$ be the solution to \eqref{dinamica_MF_no_deviazione}.
    Suppose that consistency condition \eqref{def_mean_field_sol:cons} is satisfied.
    For every $t \in [0,T]$, set $\mathfrak{r}_t(da)dt=\delta_{\lambda_t}(da)dt$.
    Then, the probability measure $\Gamma=\prob\circ(X,\mathfrak{r},\mu)^{-1}$ belongs to the set $\mathcal{K}$.

    \item \label{existence:prop_relazione_mfg:deviazioni}
    For every $t \in [0,T]$, set $\mathfrak{b}_t(da)dt=\delta_{\beta_t}(da)dt$.
    Denote by $Y$ the solution to \eqref{dinamica_MF_deviation}.
    Then, there exists $\Sigma \in \mathcal{Q}$ so that
    \begin{equation}\label{existence:prop_relazione_mfg:decomposizione}
        \prob((Y,\mathfrak{b},\mu)\in B \times S)=\int_S \Sigma(B,m)\rho(dm), \quad \forall B \in \boreliani{\contrd \times \mathcal{V}}, \; S \in \boreliani{\contpdue}.
    \end{equation}
    
    \item \label{existence:prop_relazione_mfg:relazione_payoffs}
    The pair of strategies $(\Gamma,\Sigma)$ satisfies equation \eqref{existence:prop_relazione_mfg:equazione_payoffs}.
\end{enumerate}
\end{prop}
\begin{proof}
In the following, we work on the probability space $(\Omega,\F,\mathbb{F},\prob)$ defined in point \eqref{mf:condizione_ammissibilita} of Definition \ref{mf:admissible_recommendation}.
Recall that, as pointed out in Remark \ref{finite_players:remark_estensioni}, we can think of $W$, $\xi$ and $\mu$ as independent random variables, each of them defined on the same probability space $(\Omega,\F,\mathbb{F},\prob)$.
Observe that the $\mathcal{P}(A)$-valued process $\mathfrak{r}=(\delta_{\lambda_t})_{t \in [0,T]}$ is $\mathbb{F}$-progressively measurable since $\Lambda$ is admissible by assumption.
Let $X$ be the solution to equation \eqref{dinamica_MF_no_deviazione}.
Since $X$ obviously satisfies \eqref{existence:eq_processo_K} for such a process $\mathfrak{r}$ and the condition $\mu_t(\cdot)=\prob(X_t \in \cdot \; \vert \; \mu)$ holds by assumption, $\Gamma=\prob\circ(X,\mu,\mathfrak{r})^{-1}$ belongs to $\mathcal{K}$.


As for point \ref{existence:prop_relazione_mfg:deviazioni}, recall from Remark \ref{finite_players:remark_estensioni} that we can regard $\beta$ as defined on the product space $(\Omega,\F,\prob)$, and that $\beta$ and $\mu$ are mutually independent by construction.
Therefore, the $\mathcal{P}(A)$-valued process $\mathfrak{b}=(\delta_{\beta_t}(da))_{t \in [0,T]}$ is independent of $\mu$.
Let $Y$ be the solution of equation \eqref{dinamica_MF_deviation}.
By Lemma \ref{existence:lemma_decomposizione} in Appendix \ref{appendix_existence}, equation \eqref{existence:prop_relazione_mfg:decomposizione} holds.

Finally, since $X$ and $Y$ are defined on the same filtered probability space $(\Omega,\F,\mathbb{F},\prob)$, we can write the integrals in $\mathfrak{p}$ as expectations:
\begin{equation*}
\begin{aligned}
    & \int \mathfrak{F} (y,q,m) \Gamma(dy,dq,dm) = \E\quadre{\int_0^T f(t,X_t,\mu_t,\lambda_t)dt + g(X_T,\mu_T)} = \J(\Lambda,\mu), \\
    & \int \mathfrak{F} (y,q,m) \Sigma(dy,dq,m)\rho(dm) = \E\quadre{\int_0^T  f(t,Y_t,\mu_t,\beta_t)dt + g(Y_T,\mu_T)}=\J(\beta,\mu).
\end{aligned}
\end{equation*}
This proves \eqref{existence:prop_relazione_mfg:equazione_payoffs}.
\end{proof}

The next result ensures existence of an optimal strategy for the maximizing player:
\begin{thm}[Existence of the value of the game and of an optimal strategy for the maximizing player]\label{existence:thm_esistenza_strategia_ottima}
Consider the game described in Definition \ref{existence:def_zerosum}.
The following holds:
\begin{enumerate}[label=(\roman*)]
    \item \label{existence:thm_esistenza_strategia_ottima:existence_value} The game has a value, i.e. $v^A=v^B$.
    
    \item \label{existence:thm_esistenza_strategia_ottima:optimal_strategy} There exists a strategy $\Gamma^* \in \mathcal{K}$ which is optimal for player $A$.

    \item \label{existence:thm_esistenza_strategia_ottima:positive_value} The value $v$ of the game is non negative: $v \geq 0$.

\end{enumerate}
\end{thm}
The proof of this theorem is deferred to Section \ref{existence:sezione_opt_strat_auxiliary}.
The following result is some sort of mimicking result: it will allow us to find, given any measure $\Gamma \in \mathcal{K}$, a probability measure $\hat{\Gamma}$ so that $\hat{\Gamma}$ and $\Gamma$ share the same payoff for every opponent's strategy $\Sigma \in \mathcal{Q}$ and $\hat{\Gamma}$ is induced by a correlated flow, as in Proposition \ref{existence:prop_relazione_mfg}:
\begin{lemma}\label{existence:lemma_mimicking}
Let $\Gamma \in \mathcal{K}$. There exists a measure $\hat{\Gamma} \in \mathcal{K}$ so that the following holds:
\begin{itemize}
    \item The marginal distributions of $\Gamma$ and $\hat{\Gamma}$ on $\contpdue$ are the same: $\Gamma(\contrd \times \mathcal{V} \times \cdot)=\hat{\Gamma}(\contrd \times \mathcal{V} \times \cdot)$.
    
    \item Let $(X,\mathfrak{r},\mu)$ be such that $\hat{\Gamma}=\prob\circ(X,\mathfrak{r},\mu)^{-1}$.
    Then $\mathfrak{r}$ is of the form $\mathfrak{r}_t=\hat{q}_t(X_t,\mu)$, where $\hat{q}:[0,T]\times\R^d\times\contpdue \to \mathcal{P}(A)$ is a measurable function.
    \item For every $\Sigma \in \mathcal{Q}$, it holds
    \begin{equation*}
        \mathfrak{p}(\Gamma,\Sigma)=\mathfrak{p}(\hat{\Gamma},\Sigma).
    \end{equation*}
\end{itemize}
\end{lemma}
The proof of this lemma is postponed to Appendix \ref{appendix_existence}.
In order to prove Theorem \ref{existence:main_theorem}, we need the  following additional assumption, which are standard when dealing with relaxed controls (see, e.g., \cite{haussmann1990sicom}):
\begin{customassumption}{\textbf{B}}\label{convexity_assumptions}
For every $(t,x,m) \in [0,T] \times \R^d \times \pwassspace{2}{\R^d}$, the set
\begin{equation}\label{convexity_assumptions:insieme_K}
    K(t,x,m)=\insieme{ (b(t,x,m,a),z): \; a \in A, \; f(t,x,m,a) \leq z} \subseteq \R^d \times \R
\end{equation}
is closed and convex.
\end{customassumption}

Finally, we prove the existence of a coarse correlated solution to the mean field game:
\begin{thm}[Existence of a coarse correlated solution of the MFG]\label{existence:main_theorem}
In addition to Assumptions \ref{standing_assumptions}, suppose that Assumption \ref{convexity_assumptions} holds.
Then there exists a coarse correlated solution of the mean field game.
\end{thm}
\begin{proof}
Let $\Gamma^* \in \mathcal{K}$ be an optimal strategy for player A, which exists by Theorem \ref{existence:thm_esistenza_strategia_ottima}.
Consider the strategy $\hat{\Gamma}^*$ given by Lemma \ref{existence:lemma_mimicking}, so that it holds
\begin{equation}\label{existence:main_thm:condizione_min_max}
    \inf_{\Sigma \in \mathcal{Q}}\mathfrak{p}(\hat{\Gamma}^*,\Sigma)=\inf_{\Sigma \in \mathcal{Q}}\mathfrak{p}(\Gamma^*,\Sigma)=\max_{\Gamma \in \mathcal{K}}\inf_{\Sigma \in \mathcal{Q}}\mathfrak{p}(\Gamma,\Sigma) \geq 0.
\end{equation}
Let $\mathfrak{U}=((\hat{\Omega},\hat{\F},\hat{\mathbb{F}},\hat{\prob}),\hat{\xi},\hat{W},\hat{\mu},\hat{\mathfrak{r}})$ 
be as in Definition \ref{existence:strategie_max}, so that $\hat{\Gamma}^*=\prob\circ(\hat{X},\hat{\mathfrak{r}},\hat{\mu})^{-1}$.
Recall that, by Lemma \ref{existence:lemma_mimicking}, $\hat{\mathfrak{r}}_t(da)dt=\hat{q}_t(\hat{X}_t,\hat{\mu})(da)dt$ $Leb_{[0,T]} \otimes \hat{\prob}$-a.s..
By Assumption \ref{convexity_assumptions}, the set $K(t,x,m_t)$ defined by \eqref{convexity_assumptions:insieme_K} is convex for every $(t,x,m) \in [0,T] \times \R^d \times \contpdue$.
Therefore, by a well known measurable selection argument (see, e.g., \cite[Lemma A.9]{haussmann1990sicom}) there exists a measurable function $\hat{\alpha}:[0,T]\times\R^d\times \contpdue \to A$ so that
\begin{equation}\label{existence:main_theorem:strict_ctrl}
\begin{aligned}
    \int_A b(t,x,m_t,a)\hat{q}_t(x,m)(da)=b(t,x,m_t,\hat{\alpha}(t,x,m)), \\
    f(t,x,m_t,\hat{\alpha}(t,x,m)) \leq \int_A f(t,x,m_t,a)\hat{q}_t(x,m)(da).
\end{aligned}
\end{equation}
It follows that $\hat{X}$ is a solution to equation
\begin{equation}\label{existence:main_theorem:opt_ctrled_eq}
    d\hat{X}_t=b(t,\hat{X}_t,\hat{\mu}_t,\hat{\alpha}(t,\hat{X}_t,\hat{\mu}))dt + d\hat{W}_t, \quad \hat{X}_0=\hat{\xi}
\end{equation}
as well, and the consistency condition \eqref{def_mean_field_sol:cons} is still satisfied.
By Lemma \ref{existence:mimicking:lemma_strong_existence}, we deduce that the solution $\hat{X}$ to equation \eqref{existence:main_theorem:opt_ctrled_eq} can be taken adapted to the $\hat{\prob}$-augmentation of the filtration $(\sigma(\hat{\mu}) \vee \sigma(\hat{\xi}) \vee \sigma(\hat{W}_s: \; s \leq t))_{t \in [0,T]}$, and therefore there exists a progressively measurable function $\Phi:\contpdue\times\R^d\times\contrd\to\contrd$ so that
\begin{equation}
    \hat{X}= \Phi(\hat{\mu},\hat{\xi},\hat{W}) \quad \hat{\prob}\text{-a.s.}
\end{equation}
Set
\begin{equation}\label{existence:main_theorem:ctrl_indotto}
\begin{aligned}
    & \begin{aligned}
        \hat{\lambda}:[0,T]\times\contpdue\times \R^d \times \contrd & \to A \\
    (t,m,x,w)&\mapsto \hat{\lambda}_t(m,x,w)=\hat{\alpha}_t(\Phi_t(m,x,w),m_t);
    \end{aligned} \\
    & \lambda =(\lambda_t)_{t \in [0,T]}=(\hat{\lambda}_t(\hat{\mu},\hat{\xi},\hat{W}))_{t \in [0,T]}.
\end{aligned}
\end{equation}
Then, the progressively measurable processes $(\hat{\alpha}_t(\hat{X}_t,\hat{\mu}))_{t \in [0,T]}$ and $(\hat{\lambda}_t(\hat{\mu},\hat{\xi},\hat{W}))_{t \in  [0,T]}$ are equal $Leb_{[0,T]}\otimes\hat{\prob}$-a.s., which implies that $\hat{X}$ solves 
\begin{equation*}
    d\hat{X}_t=b(t,\hat{X}_t,\hat{\mu}_t,\hat{\lambda}_t(\hat{\mu},\hat{\xi},\hat{W}))dt + d\hat{W}_t, \quad \hat{X}_0=\hat{\xi}
\end{equation*}
as well, and the consistency condition is still satisfied.
Set $(\Omega^0,\F^{0-},\prob^0)=(\contpdue,\boreliani{\contpdue},\rho)$.
By Lemma \ref{esempi:lemma_misurabile}, there exists a $\prob^0$-null set $N \subset \Omega^0$ so that the pair $(\Lambda^*,\mu*)$ defined by
\begin{equation}
\begin{aligned}
    \Lambda^*: \tonde{\contpdue,\boreliani{\contpdue},\rho} & \to (\A,\boreliani{\A}) \\
    m & \mapsto \Lambda^*(m)= \left\{ \begin{aligned}
        & (\hat{\lambda}_t(m,\cdot,\cdot))_{t \in [0,T]}, && m \in \Omega^0 \setminus N, \\
        & a_0  && m \in N,
        \end{aligned} \right. \\
    \mu^*=\text{Id}:\tonde{\contpdue,\boreliani{\contpdue},\rho} & \to \tonde{\contpdue,\boreliani{\contpdue},\rho}
\end{aligned}
\end{equation}
is a correlated flow, where $a_0$ is an arbitrary point in $A$.
Let $X^*$ be the solution of \eqref{dinamica_MF_deviation} on the product probability space $(\Omega,\F,\mathbb{F},\prob)$ defined in point \ref{mf:condizione_ammissibilita} of Definition \ref{mf:admissible_recommendation}.
Note that the strategy associated to the admissible recommendation $\Lambda^*$ strategy $\lambda^*$ is equal to $\hat{\lambda}_t(\mu^*,\xi,W)$ $Leb_{[0,T]}\otimes\prob$-almost surely.
Since uniqueness in law holds by Theorem \ref{teorema_di_unicita_legge}, it follows that
\begin{equation}\label{existence:main_thm:uguaglianza_leggi}
    \prob\circ(X^*,(\delta_{\lambda^*_t}(da))_{t \in [0,T]},\mu^*)^{-1}= \hat{\prob}\circ(\hat{X},(\delta_{\hat{\alpha}(t,\hat{X}_t,\hat{\mu})}(da))_{t \in [0,T]},\hat{\mu})^{-1},
\end{equation}
which implies that the consistency condition \eqref{def_mean_field_sol:cons} is satisfied.

\bigskip
Finally, we verify that the correlated flow just defined satisfy the optimality condition \eqref{def_mean_field_sol:opt}.
For any $\beta \in \A$, let $\Sigma \in \mathcal{Q}$ be as in point \ref{existence:prop_relazione_mfg:deviazioni} of Proposition \ref{existence:prop_relazione_mfg}.
Then, by \eqref{existence:main_thm:uguaglianza_leggi}, \eqref{existence:main_theorem:strict_ctrl} and \eqref{existence:main_thm:condizione_min_max}, for every $\Sigma \in \mathcal{Q}$ it holds
\begin{equation*}
\begin{aligned}
    \J & (\Lambda^*,\mu^*) = \E^{\hat{\prob}}\quadre{\int_0^T f(t,\hat{X}_t,\hat{\mu}_t,\hat{\alpha}(t,\hat{X}_t,\hat{\mu})) dt + g(\hat{X}_T,\hat{\mu}_T)} \\
    & \leq \E^{\hat{\prob}}\quadre{\int_0^T \int_A f(t,\hat{X}_t,\hat{\mu}_t,a)\hat{q}_t(\hat{X}_t,\hat{\mu}) dt + g(\hat{X}_T,\hat{\mu}_T)} = \int \mathfrak{F} (y,q,m) \hat{\Gamma}^*(dy,dq,dm) \\
    & \leq \int \mathfrak{F} (y,q,m) \Sigma(dy,dq,m)\rho(dm) = \J(\beta,\mu^*),
\end{aligned}
\end{equation*}
which proves that $(\Lambda^*,\mu^*)$ satisfies the optimality condition and therefore is a mean field CCE.
\end{proof}

\subsection{Proof of Theorem \ref{existence:thm_esistenza_strategia_ottima}.}
\label{existence:sezione_opt_strat_auxiliary}
The main instrument is the following Minimax Theorem, due to K. Fan:
\begin{thm}[ \cite{fan1953minimax}, Theorem 2 ] \label{thm_minimax}
Let $X$ be a compact Hausdorff space and $Y$ an arbitrary set (not topologized). Let $f:X\times Y\to \R$ be a real-valued function such that, for every $y \in Y$, $x \mapsto f(x,y)$ is lower semi-continuous on $X$.
If $f(\cdot,y)$ is concave on $X$ for every $y \in Y$ and $f(x,\cdot)$ convex on $Y$ for every $x \in X$, then
\begin{equation}
    \max_{x \in X} \inf_{y \in Y} f(x,y) = \inf_{y \in Y} \max_{x \in X} f(x,y).
\end{equation}
\end{thm}

The following results aims at verifying that the auxiliary zero-sum game in Definition \ref{existence:def_zerosum} satisfies the assumptions of Theorem \ref{thm_minimax}.
We start with some useful moment estimates for the solution to \eqref{existence:eq_processo_K}:
\begin{lemma}[Estimates]\label{lemma_stima_a_priori}
Let $\Gamma \in \mathcal{K}$, let $\mathfrak{U}=((\Omega,\F,\mathbb{F},\prob),\xi,W,\mu,\mathfrak{r})$ be the tuple associated to $\Gamma$, as in Definition \ref{existence:strategie_max}, and let $X$ be the solution to \eqref{existence:eq_processo_K}.
Then, for every $2 \leq p \leq \overline{p}$, there exists a constant $C=C(p,T,\nu,b,A)$ so that
\begin{equation}\label{lemma_stima_a_priori:stima}
    \attesa{\norm{X}_{\contrd}^p}\leq C.
\end{equation}
\end{lemma}
The proof is omitted as it is just a straightforward application of Gronwall's lemma.

\begin{lemma}\label{existence:lemma_tightness}
$\mathcal{K}$ is pre-compact in $(\mathcal{P}^2(\contrd \times \mathcal{V}\times \contpdue),\pwassmetric{2}{\contrd \times \mathcal{V}\times \contpdue}{})$.
\end{lemma}
\begin{proof}
Let $(\Gamma^n)_{n \geq 1}$ be a sequence in $\mathcal{K}$, let us show that it is pre-compact, which, by Lemma \ref{wass:equivalenze_convergenza} is equivalent to show that $(\Gamma^n)_{n \geq 1}$ is tight and condition \eqref{wass:uniforme_integrabilita} is satisfied.
Moreover, by \cite[Lemma A.2]{lacker2015martingale}, relative compactness of the sequence $(\Gamma^n)_{n \geq 1}$ is equivalent to the relative compactness of each sequence of marginals on $\contrd$, $\contpdue$ and $\mathcal{V}$.

Since $A$ is compact by Assumption \ref{standing_assumptions}, the space $\mathcal{V}$ is compact as well.
Then, we automatically get both tightness of the sequence of the marginals on $\mathcal{V}$ of $(\Gamma^n)_{n\geq 1}$ and property \eqref{wass:uniforme_integrabilita}.

In the following, for every $n\geq 1$, let $\mathfrak{U}^n=((\Omega^n,\F^n,\mathbb{F}^n,\prob^n),\xi^n,W^n,\mu^n,\mathfrak{r}^n)$ and $X^n$ be as in Definition \ref{existence:strategie_max}, so that $\Gamma^n=\prob^n\circ(X^n,\mu^n,\mathfrak{r}^n)^{-1}$.
Let $\Gamma^n_1$ be the law of $X^n$ under $\prob^n$.
We prove the tightness by means of Kolmogorov-\v{C}entsov criterion, as stated, e.g., in \cite[Corollary 16.9]{kallenberg_foundations}.
Let $2< p \leq \overline{p}$, $0\leq s < t \leq T$. We have:
\begin{equation*}
\begin{aligned}
    \E^n&\quadre{\abs{X^n_t - X^n_s}^p} \leq C\E^n\quadre{\int_s^t \int_A \abs{b(u,X^n_u,\mu^n_u,a)}^p \mathfrak{r}^n_u(da)du + \abs{W_t - W_s}^p} \\
    & \leq C\tonde{\abs{t-s}^{p-1}\int_s^t \E^n\quadre{ \int_A \abs{b(u,X^n_u,\mu^n_u,a)}^p \mathfrak{r}^n_u(da)}du + \abs{t-s}^\frac{p}{2}},
\end{aligned}
\end{equation*}
for some positive constant $C$ which is updated from line to line.
For every $u \in [0,T]$, we have
\begin{equation}\label{lemma_tightness:bound_uniforme}
\begin{aligned}
    \E^n&\quadre{ \int_A \abs{b(u,X^n_u,\mu^n_u,a)}^p \mathfrak{r}^n_u(da)} \\
    & \leq C\E^n\quadre{\abs{X^n_u}^p + \tonde{\int_{\R^d}\abs{y}^2 \mu^n_u(dy)}^\frac{p}{2} + \int_A \abs{a-a_0}^p \mathfrak{r}^n_u(da) + \abs{b(u,0,\delta_0,a_0)}^p } \\
    & \leq C\tonde{1 + \E^n\quadre{\abs{X^n_u}^p + \int_{\R^d}\abs{y}^p \mu^n_u(dy) } } = C\tonde{ 1 + \E^n\quadre{\abs{X^n_u}^p +\E^n\quadre{ \abs{X^n_u}^p \big \vert \mu^n }}} \\
    & = C\tonde{ 1 + 2\E^n\quadre{ \abs{X^n_u}^p } } \leq C \tonde{ 1 + \E^n\quadre{\sup_{u \in [0,T]} \abs{X^n_u}^p } } \leq  C,
\end{aligned}
\end{equation}
where the last inequality follows from Lemma \ref{lemma_stima_a_priori}, with $C$ independent of $n \geq 1$.
Such a uniform bound implies that
\begin{equation*}
\begin{aligned}
    \E^n&\quadre{\abs{X^n_t - X^n_s}^p} \leq C\tonde{\abs{t-s}^{p-1}\int_s^t \E^n\quadre{ \int_A \abs{b(u,X^n_u,\mu^n_u,a)}^p \mathfrak{r}^n_u(da)}du + \abs{t-s}^\frac{p}{2}} \\
    & \leq C\tonde{\abs{t-s}^{p-1}\abs{t-s}C(p,T,\nu,b,A) + \abs{t-s}^\frac{p}{2}} \leq C\tonde{\abs{t-s}^p + \abs{t-s}^{\frac{p}{2}}} \leq C \abs{t-s}^{\frac{p}{2}}.
\end{aligned}
\end{equation*}
Set $\beta=\sfrac{p}{2}-1$, so that we get
\begin{equation}\label{existence:lemma_tightness:stima_momenti}
    \E^n\quadre{\abs{X^n_t-X^n_s}^p}\leq C \abs{t-s}^{1+\beta},
\end{equation}
with $p,\beta>0$.
Since $\prob^n\circ(X_0^n)^{-1}= \nu \in \mathcal{P}^p(\R^d)$ for every $n\geq 1$, we have the tightness of the initial laws as well.
This concludes of the proof of the tightness of $(\Gamma^n_1)_{n \geq 1}$.
As for condition \eqref{wass:uniforme_integrabilita}, we have:
\begin{equation*}
    \begin{aligned}
        \lim_{r \to \infty} & \sup_n \int_{\insieme{y: \; \norm{y}_{\contrd}^2 > r}} \norm{y}_{\contrd}^2 \Gamma^n_1(dy)=\lim_{r \to \infty} \sup_n \E^n\quadre{\norm{X^n}_{\contrd}^2\1_{\insieme{\norm{X^n}_{\contrd}^2 > r}}} \\
        & \leq \lim_{r \to \infty} \sup_n \tonde{\E^n\quadre{\norm{X^n}_{\contrd}^{4}}}^\frac{1}{2}\prob^n\tonde{\norm{X^n}_{\contrd}^2 > r}^\frac{1}{2}\leq C \lim_{r \to \infty} \sup_n \prob^n\tonde{\norm{X^n}_{\contrd}^2 > r}^\frac{1}{2}
    \end{aligned}
\end{equation*}
for some positive constant $C$ independent of $n$.
By Markov's inequality and estimate \eqref{lemma_stima_a_priori:stima} again, we get
\begin{equation*}
    \lim_{r \to \infty} \sup_n \int_{\insieme{y: \; \norm{y}_{\contrd}^2 > r}} \norm{y}_{\contrd}^2 \Gamma^n_1(dy) \leq C \lim_{r \to \infty} \sup_n \E^n\quadre{\norm{X^n}_{\contrd}^2}^\frac{1}{2}r^{-\frac{1}{2}}=0.
\end{equation*}

Finally, we turn to the sequence $(\rho^n)_{n \geq 1}$, where $\rho^n=\prob^n\circ(\mu^n)^{-1}$.
Let $\prob^{n,m}(\cdot)=\prob^n( \cdot \; \vert \; \mu = m)$ be the regular conditional distribution of $\prob^n$ given $\mu^n=m$.
Then, $\mu^n_t=m_t$ $\prob^{n,m}$-a.s. and $\prob^{n,m} \circ (X^n_t)^{-1}= m_t$ $\rho$-a.e. for every $t \in [0,T]$, which implies that, for every $s,t \in [0,T]$, we have 
\begin{equation*}
    \E^{n,m}\quadre{\pwassmetric{2}{\R^d}{p}(\mu^n_t,\mu^n_s)}\leq \E^{n,m} \quadre{\abs{ X^n_t - X^n_s}^p }
\end{equation*}
for $\rho$-a.e. $m \in \contpdue$.
Integrating with respect to $\rho$ yields
\begin{equation*}
    \E^{n}\quadre{\pwassmetric{2}{\R^d}{p}(\mu^n_t,\mu^n_s)}\leq \E^{n} \quadre{\abs{ X^n_t - X^n_s}^p } \leq C \abs{t-s}^{1 + \beta}
\end{equation*}
where the last inequality follows from \eqref{existence:lemma_tightness:stima_momenti} with $\beta=\sfrac{p}{2}-1$.
Since $\prob^{n}\circ(\mu^n_0)^{-1}=\delta_{\nu}$, it is enough to apply again Kolmogorov-\v{C}entsov criterion and deduce the tightness of $(\rho^n)_{n \geq 1}$.
Finally, we verify condition \eqref{wass:uniforme_integrabilita}.
To this extent, we note that, for every $n \geq 1$, there exists a continuous modification of the process $(\E^n[\vert X^n_t \vert ^2\; \vert \; \mu^n])_{t \in [0,T]}$, so that it holds
\begin{equation*}
    \sup_{t \in [0,T]} \int_{\R^d} \abs{y}^2 \mu^n_t(dy) = \sup_{t \in [0,T]}\E^n\quadre{\abs{X^n_t}^2 \; \big\vert \; \mu^n_t } \quad \prob^n\text{-a.s.}
\end{equation*}
Indeed, estimate \eqref{existence:lemma_tightness:stima_momenti} on the moments of $X^n$ implies that the process $(\E^n[\vert X^n_t \vert ^2\; \vert \; \mu^n])_{t \in [0,T]}$ satisfies
\begin{equation*}
\begin{aligned}
    \E^n & \quadre{\abs{    \E^n\quadre{\abs{X^n_t}^2\; \vert \; \mu^n} - \E^n\quadre{\abs{X^n_s}^2\; \vert \; \mu^n}}^p} \leq \E^n \quadre{ \abs{ \abs{X^n_t}^2 - \abs{X^n_s}^2 } ^ p } \\
    & = \E^n \quadre{ \abs{X^n_t - X^n_s}^p \abs{X^n_t + X^n_s}^p } \leq \E^n \quadre{  \abs{X^n_t - X^n_s}^{2p}}^\frac{1}{2} \E^n \quadre{ \abs{X^n_t + X^n_s}^{2p} }^\frac{1}{2} \leq C\abs{t-s}^\frac{p}{2},
\end{aligned}
\end{equation*}
where we have used Cauchy-Schwartz inequality, \eqref{lemma_stima_a_priori:stima} and  \eqref{existence:lemma_tightness:stima_momenti} to bound $\E^n [ \vert X^n_t - X^n_s \vert ^{2p} ]^{\sfrac{1}{2}}$ and $\E^n [ \vert X^n_t + X^n_s \vert ^{2p} ]^{\sfrac{1}{2}}$, respectively.
Therefore, by choosing $2 < p < \sfrac{\overline{p}}{2}$ and $\beta=\sfrac{p}{2}-1$ as above, we deduce from \cite[Theorem 3.3]{kallenberg_foundations}, that there exists a continuous modification of $(\E^n[\vert X^n_t \vert ^2\; \vert \; \mu^n])_{t \in [0,T]}$.
Then, observe that
\begin{equation*}
    \int_{\R^d} \abs{y}^2 \mu^N_t (dy) = \E^n\quadre{\abs{X^n_t}^2\; \vert \; \mu^n} \quad \forall t \in [0,T] \cap \Q, \; \prob^n\text{-a.s.}
\end{equation*}
Since both processes are almost surely continuous, we can take the supremum over every $t \in [0,T]$ to conclude that
\begin{equation}\label{existence:lemma_tightness:stima_sup}
    \sup_{t \in [0,T]}\int_{\R^d} \abs{y}^2 \mu^N_t (dy) = \sup_{t \in [0,T]} \E^n\quadre{\abs{X^n_t}^2\; \vert \; \mu^n} \leq \E^n\quadre{\sup_{t \in [0,T]} \abs{X^n_t}^2  \; \bigg \vert \; \mu^n }  \quad \prob^n\text{-a.s.}
\end{equation}
We are now ready to show that \eqref{wass:uniforme_integrabilita} holds for $(\rho^n)_{n \geq 1}$: by applying \eqref{existence:lemma_tightness:stima_sup} in the first inequality, Cauchy-Schwartz and Markov inequalities, we have
\begin{equation*}
\begin{aligned}
    & \lim_{r \to \infty} \sup_n \int_{\insieme{m: \; \sup_{t \in [0,T]} \int_{\R^d} \abs{y}^2 m_t(dy) > r}} \sup_{t \in [0,T]} \int_{\R^d} \abs{y}^2 m_t(dy) \rho^n(dm) \\
    & \leq \lim_{r \to \infty} \sup_n \E^n\quadre{ \E^n \quadre{ \norm{X}_{\contrd}^2 \; \vert \; \mu^n } \1_{\insieme{\E^n \quadre{ \norm{X}_{\contrd}^2 \; \vert \; \mu^n } > r}}} \\
    & \leq \lim_{r \to \infty} \frac{1}{r^\frac{1}{2}} \sup_n \E^n\quadre{\norm{X^n}_{\contrd}^{4}}^\frac{1}{2}\E^n\quadre{\norm{X^n}_{\contrd}^{2}}^\frac{1}{2}  \leq \lim_{r \to \infty} C r^{-\frac{1}{2}}=0, 
    \end{aligned}
\end{equation*}
since the suprema over $n \geq 1$ are finite by Lemma \ref{lemma_stima_a_priori}.
\end{proof}


\begin{lemma}\label{existence:lemma_closedness}
$\mathcal{K}$ is closed in $(\mathcal{P}^{2}(\contrd \times \mathcal{V}\times \contpdue),\pwassmetric{2}{\contrd \times \mathcal{V}\times \contpdue}{})$.
\end{lemma}
\begin{proof}
It is enough to prove that, for every sequence $(\Gamma^n)_{n \geq 1}\subseteq \mathcal{K}$ converging to $\Gamma$ as $n \to \infty$ in $\pwassmetric{2}{\contrd \times \mathcal{V} \times \contpdue}{}$, we have $\Gamma \in \mathcal{K}$.
We work on the following canonical space: let $(\overline{\Omega},\overline{\mathcal{G}})$ be given by 
\begin{equation*}
    (\overline{\Omega},\overline{\mathcal{G}})=\tonde{\contrd \times \contpdue \times \mathcal{V} ,\boreliani{\contrd} \otimes \boreliani{\mathcal{V}} \otimes \boreliani{\contpdue}}.
\end{equation*}
We equip such a space with the filtration $\mathbb{G}=(\mathcal{G}_t)_{t \in [0,T]}$ given by
\begin{equation*}
    \mathcal{G}_t=\mathcal{B}_{t,\contrd} \otimes \F_t^{\mathcal{V}} \otimes \boreliani{\contpdue},
\end{equation*}
where $\mathcal{B}_{t,\contrd}=\sigma(\contrd \ni x \mapsto x_s: \; s \leq t)$.
Let $x$, $m$ and $q$ denote the projection from $\overline{\Omega}$ in $\contrd$, $\contpdue$ and $\mathcal{V}$, respectively.
Define the process $w=(w_t)_{t \in [0,T]}$ as
\begin{equation}\label{existence:lemma_closedness:brownian_motion}
    w_t=w_t(x,q,m)=x_t-x_0-\int_0^t\int_A b(s,x_s,m_s,a)q_s(da)ds.
\end{equation}
Observe that $w$ is a continuous process on $(\overline{\Omega},\overline{\F})$ and, by \cite[Corollary A.5]{lacker2015martingale}, for every $t \in [0,T]$ $w_t$ is a continuous with at most linear growth function of $(x,q,m)$.

For every $n\geq 1$, let $\mathfrak{U}^n=((\Omega^n,\F^n,\mathbb{F}^n,\prob^n),\xi^n,W^n,\mu^n,\mathfrak{r}^n)$ and $X^n$ be as in Definition \ref{existence:strategie_max}, so that $\Gamma^n=\prob^n\circ(X^n,\mu^n,\mathfrak{r}^n)^{-1}$.
Since $\Gamma^n \circ (x_0,w,m,q,x)^{-1}=\prob^n\circ(\xi^n,W^n,\mu^n,\mathfrak{r}^n,X^n)^{-1}$, we have that the tuple $\mathfrak{U}^n=((\overline{\Omega},\overline{\F},\overline{\mathbb{F}}^{\Gamma^n},\Gamma^n),x_0,w,m,q)$ satisfies the requirements of Definition \ref{existence:strategie_max}, where $\overline{\mathbb{F}}^{\Gamma^n}$ denotes the $\Gamma^n$-augmentation of the filtration $\mathbb{G}$.
We show that the tuple $\mathfrak{U}=((\overline{\Omega},\overline{\F},\overline{\mathbb{F}}^{\Gamma},\Gamma),x_0,w,m,q)$ satisfies the requirements of Definition \ref{existence:strategie_max}, which implies $\Gamma \in \mathcal{K}$.

We start by the independence property of $w$, $m$ and $q$ under $\Gamma$.
Let $(t^i)_{i=1}^{k} \subseteq [0,T]$, $\varphi^i \in \cbounded{\R^d}$ for $i=1,\dots,n$, $\psi \in \cbounded{\contpdue}$, $\phi \in \cbounded{\R^d}$ be bounded continuous functions.
Since $W^n$, $\mu^n$ and $\xi^n$ are independent under $\prob^n$ and $\Gamma^n \to \Gamma$ weakly, we have
\begin{equation*}
\begin{aligned}
    \E^{\Gamma^n} & \quadre{\prod_{i=1}^k\varphi^i\tonde{w_{t^i}(x,q,m)}\psi(m)\varphi\tonde{x_0}} \\
    & = \E^{\prob^n}\quadre{\prod_{i=1}^k\varphi^i\tonde{W^n_{t^i}}\psi(\mu^n)\varphi\tonde{\xi^n}}= \E^{\prob^n}\quadre{\prod_{i=1}^k\varphi^i\tonde{W^n_{t^i}}}\E^{\prob^n}\quadre{\psi(\mu^n)}\E^{\prob^n}\quadre{\varphi\tonde{\xi^n}}  \\
    & = \E^{\Gamma^n}\quadre{\prod_{i=1}^k\varphi^i\tonde{w_{t^i}(x,q,m)}}\E^{\Gamma^n}\quadre{\psi(m)}\E^{\Gamma^n}\quadre{\varphi\tonde{x_0}}
\end{aligned}
\end{equation*}
where the first equality holds since $w_t(X^n,\mathfrak{r}^n,\mu^n)=W^n_t$ for every $t \in [0,T]$ $\prob^n$-a.s..
Then, since $\phi^i \circ w_{t^i}$ is a continuous function of $(x,q,m)$ for every $i$, weak convergence implies that
\begin{equation}\label{existence:lemma_closedness:convergenze}
\begin{aligned}
    \lim_{n \to \infty} & \E^{\Gamma^n}\quadre{\prod_{i=1}^k\varphi^i\tonde{w_{t^i}(x,q,m)}\psi(m)\varphi\tonde{x_0}} = \E^{\Gamma}\quadre{\prod_{i=1}^k\varphi^i\tonde{w_{t^i}(x,q,m)}\psi(m)\varphi\tonde{x_0}}, \\
    \lim_{n \to \infty} & \E^{\Gamma^n}\quadre{\prod_{i=1}^k\varphi^i\tonde{w^n_{t^i}(x,q,m)}}\E^{\Gamma^n}\quadre{\psi(m)}\E^{\Gamma^n}\quadre{\varphi\tonde{\xi^n}} \\
    & = \E^{\Gamma}\quadre{\prod_{i=1}^k\varphi^i\tonde{w_{t^i}(x,q,m)}}\E^{\Gamma}\quadre{\psi(m)}\E^{\Gamma}\quadre{\varphi\tonde{x_0}}.
\end{aligned}
\end{equation}
This is enough to ensure the mutual independence under $\Gamma$ of $(w_{t^i})_{i=1,\dots,k}$, $x_0$ and $m$ for every $(t^i)_{i=1}^{k}\subset [0,T]$, which yields the independence of $w$, $x_0$ and $m$.
Moreover, by taking $\psi$ and $\phi$ identically equal to $1$, equation
\eqref{existence:lemma_closedness:convergenze} implies that $w$ is natural Brownian motion under $\Gamma$, since the finite dimensional distributions of $w$ coincide with the ones of a Brownian motion. 
Let us verify the independence of increments properties.
Let $s > t$, $\varphi \in \cbounded{\contrd}$ $\mathcal{B}_{t,\contrd}$-measurable, $\chi \in \cbounded{\mathcal{V}}$ $\F_t^{\mathcal{V}}$-measurable, $\psi \in \cbounded{\contpdue}$ and $\phi \in \cbounded{\R^d}$.
Then, we have:
\begin{equation*}
\begin{aligned}
    \E^{\Gamma} & \quadre{\phi\tonde{w_s-w_t}\varphi\tonde{x}\chi\tonde{q}\psi(m)} = \E^{\Gamma}  \quadre{\phi\tonde{w_s(x,q,m)-w_t(x,q,m)}\varphi\tonde{x}\chi\tonde{q}\psi(m)} \\
    & =\lim_{n \to \infty}\E^{\Gamma^n}\quadre{\phi\tonde{w_s(x,q,m)-w_t(x,q,m)}\varphi\tonde{x}\chi\tonde{q}\psi(m)} \\
    & = \lim_{n \to \infty}\E^{\prob^n}\quadre{\phi\tonde{w_s(X^n,\mathfrak{r}^n,\mu^n)-w_t(X^n,\mathfrak{r}^n,\mu^n)}\varphi(X^n)\chi\tonde{\mathfrak{r}^n}\psi(\mu^n)} \\
    & = \lim_{n \to \infty}\E^{\prob^n}\quadre{\phi\tonde{W^n_s-W^n_t}\varphi(X^n)\chi\tonde{\mathfrak{r}^n}\psi(\mu^n)} = 0,
\end{aligned}
\end{equation*}
where the last equality holds since $W^n$ is a $\mathbb{F}^n$-Brownian motion under $\prob^n$, $\mu^n$ is $\F_0^n$-measurable and $X^n$ and $\mathfrak{r}^n$ are both $\mathbb{F}^n$-adapted.
By working with an approximating sequence, this holds also for bounded measurable $\varphi$, $\chi$, $\psi$ and $\phi$, which is enough to conclude the independence of increments.
Finally, since $w$ is $\mathbb{G}$-Brownian motion, it remains so under the $\Gamma$-augmentation of $\mathbb{G}$.

Since $\Gamma^n \circ x_0^{-1}\equiv \nu$, we have that $\Gamma \circ x_0 = \nu$ as well.
Moreover, since $w$ is a $\overline{\mathbb{F}}^{\Gamma}$-Brownian motion and $x$ is $\overline{\mathbb{F}}^{\Gamma}$-adapted by definition of the filtration, equation \eqref{existence:lemma_closedness:brownian_motion} implies that $x$ is a solution to \eqref{existence:eq_processo_K}.

As for the consistency condition, observe that, for every $t \in [0,T]$, $\varphi \in \cbounded{\R^d}$, $\psi \in \cbounded{\contpdue}$, we have 
\begin{align*}
    \E^{\Gamma^n}\quadre{\int_{\R^d}\varphi(y)m_t(dy)\psi(m)} & = \E^{\prob^n}\quadre{\int_{\R^d}\varphi(y)\mu^n_t(dy)\psi(\mu^n)}\\
     &=\E^{\prob^n}\quadre{\E^{\prob^n}\quadre{\varphi(X^n_t)\psi(\mu^n)\; \vert \;  \mu^n}}\\
     & =\E^{\prob^n}\quadre{\varphi(X^n_t)\psi(\mu^n)}= \E^{\Gamma^n}\quadre{\varphi\tonde{x_t}\psi(m)},
\end{align*}
since $\mu^n_t$ is a version of the conditional law under $\prob^n$ of $X^n_t$ given $\mu^n$.
Therefore, by weak convergence we have both
\begin{equation*}
\begin{aligned}
    & \lim_{n \to \infty}\E^{\Gamma^n}\quadre{\varphi\tonde{x_t}\psi(m)} = \E^{\Gamma}\quadre{\varphi\tonde{x_t}\psi(m)}, \\
    & \lim_{n \to \infty}\E^{\Gamma^n}\quadre{\int_{\R^d}\varphi(y)m_t(dy)\psi(m)} = \E^{\Gamma}\quadre{\int_{\R^d}\varphi(y)m_t(dy)\psi(m)}
\end{aligned}
\end{equation*}
where the second limit holds since the function $m \mapsto \int_{\contrd}\varphi(y)m_t(dy) \in \cbounded{\contpdue}$, which implies
\begin{equation*}
    \E^{\Gamma}\quadre{\int_{\R^d}\varphi(y)m_t(dy)\psi(m)}=\E^{\Gamma}\quadre{\varphi\tonde{x_t}\psi(m)}.
\end{equation*}
This is enough to conclude that $m_t = \Gamma(x_t \in \cdot \; \vert \; m)$ $\Gamma$-a.s for every $t \in [0,T]$, since the random element $(x_t,m)$ takes values in a Polish space.
\end{proof}


\begin{lemma}[Convexity]\label{existence:lemma_cvx_strategie}
$\mathcal{K}$ and $\mathcal{Q}$ are convex.
\end{lemma}
\begin{proof}
We start by proving that $\mathcal{K}$ is convex.
Let $\Gamma^i$, $i=1,2$, be in $\mathcal{K}$, and let $\alpha \in (0,1)$.
Let $\mathfrak{U}^i=((\Omega^i,\F^i,\mathbb{F}^i,\prob^i),\xi^i,W^i,\mu^i,\mathfrak{r}^i)$ be as in Definition \ref{existence:strategie_max}, so that $\Gamma^i=\prob^i\circ(X^i,\mathfrak{r}^i,\mu^i)^{-1}$.
Set $\Xi^i=(\xi^i,W^i,\mu^i,\mathfrak{r}^i,X^i)$.
Without loss of generality, we can suppose that the tuples are defined on the same probability space $(\Omega,\F,\mathbb{F},\prob)$ which supports also a Bernoulli random variable $\eta \sim B(\alpha)$, so that $\eta$ and $(\Xi^i)_{i=1,2}$ are mutually independent.
If needed, we can enlarge the filtration so that $\eta$ is $\F_0$-measurable.
Let us consider the following random variables:
\begin{equation}\label{existence:lemma_cvx_strategie:combinazioni_variabili}
    \begin{aligned}
        & \xi^\alpha=\eta\xi^1 + \tonde{1-\eta}\xi^2, && \mu^\alpha=\eta\mu^1 + \tonde{1-\eta}\mu^2, \\
        & W^\alpha=\eta W^1 + \tonde{1-\eta}W^2, \; && \mathfrak{r}^\alpha=\eta \mathfrak{r}^1 + \tonde{1-\eta}\mathfrak{r}^2, \\
        & X^\alpha=\eta X^1 + \tonde{1-\eta}X^2.
    \end{aligned}
\end{equation}
Set $\Xi^\alpha=(\xi^\alpha,W^\alpha,\mu^\alpha,\mathfrak{r}^\alpha,X^\alpha)$ and $\Gamma^\alpha=\prob\circ(X^\alpha,\mathfrak{r}^\alpha,\mu^\alpha)^{-1}$.
Observe that the law of $\Xi^1$ under $\prob$ is the same as the law of $\Xi^\alpha$ conditionally to $\eta=1$, as the two tuples coincide on the set $\{\eta=1\}$, and analogously for $\eta=0$.
Therefore, for every Borel set $B$, we have
\begin{equation}\label{existence:lemma_cvx:combinazione_cvx}
\begin{aligned}
    \prob\tonde{\Xi^\alpha\in B}
    & = \prob\tonde{\Xi^\alpha \in B \big \vert  \eta =1}\prob\tonde{\eta=1} + \prob\tonde{ \Xi^\alpha \in B \big \vert  \eta =0}\prob\tonde{\eta=0} \\
    & = \prob\tonde{ \Xi^1 \in B }\prob\tonde{\eta=1} + \prob\tonde{\Xi^2 \in B }\prob\tonde{\eta=0} \\
    & = \alpha\prob\tonde{ \Xi^1 \in B } + (1-\alpha)\prob\tonde{ \Xi^2 \in B }.
\end{aligned}
\end{equation}
In particular, \eqref{existence:lemma_cvx:combinazione_cvx} implies that $\Gamma^\alpha=\alpha\Gamma^1 + (1-\alpha)\Gamma^2$.
Let us show that the tuple $\Xi^\alpha$ satisfies the requirements of Definition \ref{existence:strategie_max}.
By \eqref{existence:lemma_cvx:combinazione_cvx}, $\xi^\alpha$ has law $\nu$ and $W^\alpha$ is a natural Brownian motion.
To see that it is an $\mathbb{F}$-Brownian motion, let $s<t$, $G \in \F_s$, $B \in \boreliani{\R^d}$: then
\begin{equation*}
\begin{aligned}
    \E\quadre{\1_{B}\tonde{W^\alpha_t-W^\alpha_s}\1_G}  = & \E\quadre{\1_{B}\tonde{W^\alpha_t-W^\alpha_s}\1_G \1_{\insieme{\eta =1}}} + \E\quadre{\1_{B}\tonde{W^\alpha_t-W^\alpha_s}\1_G\1_{\insieme{\eta =0}}} \\
     = &  \E\quadre{\1_B\tonde{\eta \tonde{W^1_t-W^1_s} + \tonde{1-\eta}\tonde{W^2_t-W^2_s} \in B}\1_G\1_{\insieme{\eta =0}}} \\
    & + \E\quadre{\1_B\tonde{\eta \tonde{W^1_t-W^1_s} + \tonde{1-\eta}\tonde{W^2_t-W^2_s} \in B}\1_G\1_{\insieme{\eta =1}} }\\
     = &  \E\quadre{\1_B\tonde{W^1_t-W^1_s}\1_{G \cap\insieme{\eta =1}}} + \E\quadre{\1_B\tonde{W^1_t-W^1_s}\1_{G \cap\insieme{\eta =0}} }= 0,
\end{aligned}
\end{equation*}
since $\eta$ is $\F_0$-measurable by assumption.
As for the mutual independence of $\xi^\alpha$, $\mu^\alpha$ and $W^\alpha$, we have that the joint law factorizes in the product of the marginals: by using \eqref{existence:lemma_cvx:combinazione_cvx}, since $(\xi^i,W^i)_{i=1,2}$ share the same joint law, one gets 
\begin{equation*}
\begin{aligned}
    \prob (\mu^\alpha & \in A, W^\alpha \in B,\xi^\alpha \in C) \\
    = & \alpha\prob(\mu^1 \in A, W^1  \in B, \xi^1 \in C)  + (1-\alpha)\prob(\mu^2 \in A, W^2  \in B, \xi^2 \in C) \\
    = & \alpha\prob(\mu^1 \in A)\prob( W^1  \in B)\prob(\xi^1 \in C)  + (1-\alpha)\prob(\mu^2 \in A)\prob(W^2\in B)\prob(\xi^2 \in C) \\
    = & \left(\alpha\prob(\mu^1 \in A)  +(1-\alpha) \prob(\mu^2 \in A) \right)\wienermeasure(B)\nu(C) = \prob(\mu^\alpha \in A)\prob(W^\alpha\in B)\prob(\xi^\alpha \in C).
\end{aligned}
\end{equation*}
With similar arguments, one can show that for every $t \in [0,T]$, $g:\R^d \to \R$, $f:\contpdue \to \R$ bounded and measurable, it holds
\begin{equation*}
\begin{aligned}
    \E \quadre{ g\tonde{X^\alpha_t} f\tonde{\mu^\alpha}} = \E\quadre{ \int_{\contrd} g\tonde{ y } \mu^\alpha_t (dy) f\tonde{\mu^\alpha}  },
\end{aligned}
\end{equation*}
which implies that $\mu^\alpha_t$ is a version of the conditional distribution of $X^\alpha_t$ given $\mu^\alpha$.
Finally, consider the set
\begin{equation*}
    \Omega^1=\insieme{X^1_t = \xi^1 + \int_0^t \int_A b(s,X^1_s,\mu^1_s,a)\mathfrak{r}^1_s(da)ds + W^1_t \;\; \forall t \in [0,T]}\cap \insieme{\eta=1},
\end{equation*}
and define analogously $\Omega^2$.
We have that $\Omega^1 \cap \Omega^2 = \emptyset$ and $\prob(\Omega^1)=\alpha$, since $X^1$ satisfies the equation above $\prob$-a.s., and analogously $\prob(\Omega^2)=1-\alpha$, so that $\prob(\Omega^1 \cup \Omega^2)=1$.
On such a set, $X^\alpha$ satisfies the equation
\begin{equation*}
    X^\alpha_t = \xi^\alpha + \int_0^t \int_A b(s,X^\alpha_s,\mu^\alpha_s,a)\mathfrak{r}^\alpha_s(da)ds + W^\alpha_t, \; t \in [0,T].
\end{equation*}
Since $X^\alpha$ is $\mathbb{F}$-adapted, $X^\alpha$ is a solution to equation \eqref{existence:eq_processo_K}, which concludes this part of the proof.

\medskip
Let us turn to the convexity of the set $\mathcal{Q}$. Let $\Sigma^i$, $i=1,2$, be in $\mathcal{Q}$, and $\alpha \in (0,1)$.
Let $\mathfrak{U}^i=((\Omega^i,\F^i,\mathbb{F}^i,\prob^i)$, $\xi^i,W^i,\mathfrak{r}^i)$ be as in Definition \ref{existence:strategie_min} so that $\Sigma^i(\cdot,m)=\prob^i((X^{m,i},\mathfrak{r}^i)\in \cdot )$, where $X^{m,i}$ is the solution to equation \eqref{existence:eq_processo_Q} on $(\Omega^i,\F^i,\mathbb{F}^i,\prob^i)$ when $b$ is evaluated at $m \in \contpdue$.
Let $\Theta^i=\prob^i\circ(\xi^i,W^i,\mathfrak{r}^i)^{-1}$, and consider the maps $\mathcal{I}_{\Theta^i}$ defined by 
\begin{equation}
\begin{aligned}
    \mathcal{I}_{\Theta^i}: \contpdue  & \longrightarrow \mathcal{P}(\R^d \times \contrd \times \contrd \times \mathcal{V}) \\
    m & \longmapsto \mathcal{I}_{\Theta^i}(m)=\prob^i\circ(\xi^i,W^i,X^{m,i},\mathfrak{r}^i)^{-1}.
\end{aligned}
\end{equation}
Similarly as for the set $\mathcal{K}$, suppose that the tuples are defined on the same probability space $(\Omega,\F,\mathbb{F},\prob)$ supporting also a Bernoulli random variable $\eta \sim B(\alpha)$, so that $\eta$ and $(\xi^i,W^i,\mathfrak{r}^i)_{i=1,2}$ are mutually independent.
If needed, we can enlarge the filtration so that $\eta$ is $\F_0$-measurable.
Let $\xi^\alpha$, $W^\alpha$ and $\mathfrak{r}^\alpha$ be as in \eqref{existence:lemma_cvx_strategie:combinazioni_variabili}, and, for every $m \in \contpdue$, define
\begin{equation*}
    \begin{aligned}
        X^{\alpha,m}=\eta X^{1,m} + \tonde{1-\eta}X^{2,m}.
    \end{aligned}
\end{equation*}
Let $\Theta^\alpha=\prob^\alpha \circ ( \xi^\alpha,W^\alpha,\mathfrak{r}^\alpha)^{-1}$ and consider the map $\mathcal{I}_{\Theta^\alpha}$, defined analogously to above.
By point \ref{existence:lemma_kernels:kernel_ben_definito} of Lemma \ref{existence:lemma_kernels}, it induces a stochastic kernel $\Sigma^\alpha \in \mathcal{Q}$.
By working in the same way as in the case of $\mathcal{K}$, we can show that 
$\mathcal{I}_{\Theta^\alpha}(m) = \alpha\mathcal{I}_{\Theta^1}(m) + (1-\alpha)\mathcal{I}_{\Theta^2}(m)$ for each $m \in \contpdue$, which implies that $\Sigma^\alpha=\alpha \Sigma^1 + (1-\alpha)\Sigma^2 \in \mathcal{Q}$.
\end{proof}

\begin{prop}\label{existence:prop_payoff_continuo}
The map $\mathcal{K} \times \mathcal{Q} \ni (\Gamma,\Sigma) \mapsto \mathfrak{p}(\Gamma,\Sigma)$ is bilinear.
Moreover, $\mathcal{K} \ni \Gamma \mapsto \mathfrak{p}(\Gamma,\Sigma)$ is continuous for every $\Sigma \in \mathcal{Q}$.
\end{prop}
\begin{proof}
Bilinearity is clear, hence we focus on the continuity of $\mathfrak{p}(\cdot,\Sigma)$ for fixed $\Sigma$.
Take $(\Gamma^n)_{n\geq 1}$, $\Gamma$ in $\mathcal{K}$ and suppose $\Gamma^n \to \Gamma$ in the 2-Wasserstein distance.
We treat separately the term depending just upon $\Gamma \in \mathcal{K}$ and the term depending also upon $\Sigma \in \mathcal{Q}$ in \eqref{existence:payoff_functional}.

\medskip
By Proposition \ref{wass:equivalenze_convergenza}, $\Gamma^n \to \Gamma$ in 2-Wasserstein metrics if and only if
\begin{equation}
\int_{\contrd \times\mathcal{V} \times \contpdue}\psi(y,q,m)\Gamma^n(dy,dq,dm) \to \int_{\contrd \times\mathcal{V} \times \contpdue}\psi(y,q,m)\Gamma(dy,dq,dm),
\end{equation}
for every $\psi$ continuous with at most quadratic growth, hence we just need to show that the functional $\mathfrak{F}$ defined in \eqref{existence:funzione_f} is continuous with at most quadratic growth.
By Assumptions \ref{standing_assumptions} and \cite[Corollary A.5]{lacker2015martingale}, we have that $\mathfrak{F}(y,q,m)$ is continuous.
It is straightforward to verify that $\mathfrak{F}$ has at most quadratic growth, in the sense that
\begin{equation*}
    \mathfrak{F}(y,q,m) \leq C\tonde{1 + \norm{y}_{\contrd}^2 + \sup_{t \in [0,T]} \int_{\R^d} \abs{y}^2 m_t(dy) + \int_0^T \int_A \abs{a-a_0}^2 q_s(da)ds }.
\end{equation*}
Therefore, we get continuity of the term depending only upon $\Gamma$.

\medskip
Denote by $\rho^n$ and $\rho$ the marginal of $\Gamma^n$ and $\Gamma$ on $\contpdue$.
We can manipulate the term depending both upon $\Gamma$ and $\Sigma$ as
\begin{equation*}
    \begin{aligned}
        & \int_{\contrd\times \mathcal{V}\times \contpdue}  \mathfrak{F} (y,q,m) \Sigma(dy,dq,m)\rho(dm) \\
        & \: = \int_{\contpdue}\tonde{\int_{\contrd\times \mathcal{V}} \mathfrak{F} (y,q,m) \Sigma(dy,dq,m)}\rho(dm)= \int_{\contpdue} g(m)\rho(dm)
\end{aligned}
\end{equation*}
where we set
\begin{equation*}
    g(m)=\int_{\contrd\times \mathcal{V}} \mathfrak{F} (y,q,m) \Sigma(dy,dq,m).
\end{equation*}
We must show that $g: \contpdue\to\R$ is continuous with at most quadratic growth with respect to the 2-Wasserstein distance.
As for the growth condition, estimate \eqref{existence:lemma_continuita:bound_uniformi} in Lemma \ref{existence:lemma_kernels} proves that $g$ has at most quadratic growth in $m \in \contpdue$.
As for the continuity, let $(m^n)_{n \geq 1},m \in \contpdue$ so that $m^n \to m$ in $\pwassmetric{2}{\contrd}{}$.
Note that $\Sigma(dy,dq,m^n) \to \Sigma(dy,dq,m)$ in $\pwassmetric{2}{\contrd \times \mathcal{V}}{}$, as implied by Lemma \ref{existence:lemma_kernels}.
Define $\phi^n(y,q)=\mathfrak{F}(y,q,m^n)$.
Since the cost functions are locally Lipschitz, we have that $\phi^n$ converges to $\phi$ uniformly on bounded sets of $\contrd \times \mathcal{V}$.
This is enough to conclude that
\begin{equation*}
    g(m^n)=\int_{\contrd\times \mathcal{V}} \phi^n (y,q) \Sigma(dy,dq,m^n) \to \int_{\contrd\times \mathcal{V}} \phi (y,q) \Sigma(dy,dq,m)=g (m) \quad \text{ as } m^n \to m.
\end{equation*}
\end{proof}

We can now prove points \ref{existence:thm_esistenza_strategia_ottima:existence_value} and \ref{existence:thm_esistenza_strategia_ottima:optimal_strategy} of Theorem \ref{existence:thm_esistenza_strategia_ottima}: take $X=\mathcal{K}$, $Y=\mathcal{Q}$ and $f(x,y)=\mathfrak{p}(\Gamma,\Sigma)$ in the statement of Theorem \ref{thm_minimax}.
By Lemmata \ref{existence:lemma_tightness} and \ref{existence:lemma_closedness}, $\mathcal{K}$ is compact with the topology of convergence in 2-Wasserstein distance and both sets $\mathcal{K}$ and $\mathcal{Q}$ are convex by Lemma \ref{existence:lemma_cvx_strategie}.
By Proposition \ref{existence:prop_payoff_continuo}, the payoff $\mathfrak{p}$ is both concave and continuous in $\Gamma$ for every fixed $\Sigma \in \mathcal{Q}$ and convex in $\Sigma$ for every fixed $\Gamma$.
Therefore, Theorem \ref{thm_minimax} yields the existence of both the value $v$ of the auxiliary zero-sum game and an optimal strategy for player A. 
The next proposition proves point \ref{existence:thm_esistenza_strategia_ottima:positive_value}, concluding the proof of Theorem \ref{existence:thm_esistenza_strategia_ottima}.

\begin{prop}[Positivity of the value of the auxiliary zero-sum game]\label{existence:prop_positive_value}
Let $v$ be the value of the zero-sum game defined in Definition \ref{existence:def_zerosum} has a value $v$.
Then $v \geq 0$.
\end{prop}
\begin{proof}
We show that, for every $\Sigma \in \mathcal{Q}$ there exists a strategy $\Gamma_{\Sigma} \in \mathcal{K}$ so that $\mathfrak{p}(\Gamma_{\Sigma},\Sigma)=0$.
Fix $\Sigma \in \mathcal{Q}$, let $\mathfrak{U}=((\Omega,\F,\mathbb{F},\prob),\xi,W,\mathfrak{b})$ be a tuple as in Definition \ref{existence:strategie_min} so that $\Sigma(\cdot,m)=\prob((X^m,\mathfrak{b})\in\cdot)$, for every $m \in \contpdue$.
On this probability space, consider the following stochastic differential equation of McKean-Vlasov type:
\begin{equation}\label{existence:positivity_MKV_SDE}
   \begin{cases}
   dY_t=\int_A b(t,Y_t,p_t,a)\mathfrak{b}_t(da)dt + dW_t, \; t \in [0,T], \quad Y_0=\xi; \\
   \mathcal{L}(Y_t)=p_t, \: t \in [0,T], \quad p=(p_t)_{t \in [0,T]} \in \contpdue.
   \end{cases}
\end{equation}
Under Assumptions \ref{standing_assumptions}, there exists a unique pair $(Y,p)$ satisfying \eqref{existence:positivity_MKV_SDE}, where $Y=(Y_t)_{t \in [0,T]}$ is an $\mathbb{F}$-adapted continuous process so that $\E[\sup_{t \in [0,T]}\vert Y_t \vert ^2]<\infty$, as ensured by, e.g., \cite[Theorem 4.21]{librone_vol1}, which implies that $p$ actually belongs to $\contpdue$.
Define the deterministic flow of measures $\mu$ by setting $\mu=p$.
Define $\Gamma_{\Sigma}$ as $\prob\circ(Y,\mu,\mathfrak{b})^{-1}$.
Since $\mu=p$ is deterministic and $(Y,p)$ is a solution to \eqref{existence:positivity_MKV_SDE}, $\mu$ is $\F_0$-measurable and independent of $\xi$ and $W$, and consistency condition holds trivially.
This implies that $\Gamma_{\Sigma}$ belongs to $\mathcal{K}$.
By writing the integrals in $\mathfrak{p}$ as expectations, we have:
\begin{equation*}
\begin{aligned}
    \mathfrak{p}& (\Gamma_{\Sigma},\Sigma) =  \int_{\contrd\times \mathcal{V}\times \contpdue}  \mathfrak{F} (y,q,m) \Sigma(dy,dq,m)\rho_{\Sigma}(dm) - \int_{\contrd\times \mathcal{V}\times \contpdue} \mathfrak{F} (y,q,m)  \Gamma_{\Sigma}(dy,dq,dm) \\
    & =  \attesa{\int_0^T \int_A f(t,Y_t,p_t,a)\mathfrak{b}_t(da)dt + g(Y_T,p_T)} - \attesa{\int_0^T \int_A f(t,Y_t,p_t,a)\mathfrak{b}_t(da)dt + g(Y_T,p_T)} \\
    & = 0,
\end{aligned}
\end{equation*}
where $\rho_{\Sigma}(\cdot)=\delta_{p}(\cdot)$ denotes the marginal law of $\Gamma_{\Sigma}$ on $\contpdue$.
Since such a construction holds for every $\Sigma \in \mathcal{Q}$, we have 
\begin{equation*}
    \sup_{ \Gamma \in \mathcal{K}} \mathfrak{p}(\Gamma,\Sigma)  \geq \mathfrak{p}(\Gamma_{\Sigma},\Sigma) = 0 \quad \forall \Sigma \in \mathcal{Q}.
\end{equation*}
Taking the infimum with respect to $\Sigma \in \mathcal{Q}$,  we have 
\begin{equation*}
    \inf_{\Sigma \in \mathcal{Q}} \sup_{ \Gamma \in \mathcal{K}} \mathfrak{p}(\Gamma,\Sigma)  \geq \mathfrak{p}(\Gamma_{\Sigma},\Sigma)  \geq 0,
\end{equation*}
which shows that $v^B$ is non-negative.
Since $v^A=v^B=v$, this proves that the value of the auxiliary zero-sum game is non-negative.
\end{proof}


\section{An example of coarse correlated solution to the mean field game}\label{sezione_esempio}


Taking inspiration from the work of Bardi and Fischer \cite{bardi_fischer2019} and Lacker's papers \cite{lacker2016,lacker2020convergence}, we provide a simple example of a mean field game possessing mean field CCEs with non-deterministic flow of measures $\mu$.

\medskip
Let $d=1$.
Set $A=[a,b]$, with $a < 0 < b$, and $\nu=\delta_0$.
For $m \in \probmeasures{\R}$, denote by $\overline{m}$ its mean $\int_{\R} y m(dy)$.
Consider the following coefficients and cost functions:
\begin{equation*}
    \begin{aligned}
    & b(t,x,m,a)=a, && f(t,x,m,a)=0, &&& g(x,m)=cx\overline{m},
    \end{aligned}
\end{equation*}
with $c>0$ a positive constant.
Observe that they satisfy the requirements of Assumption \ref{standing_assumptions}.
We want to find a coarse correlated solution for the mean field game whose payoff functional, to be maximized, is given by
\begin{equation}\label{esempio_payoff}
    \J(\Lambda,\mu)=\attesa{cX_T \overline{\mu}_T},
\end{equation}
under the constraint
\begin{equation}\label{esempio_dinamiche}
    X_t = \int_0^t \lambda_s ds + W_t, \quad 0\leq t \leq T,
\end{equation}
where $\lambda$ is the strategy associated to an admissible recommendation $\Lambda$ in the sense of \eqref{mf:controllo_indotto}.

\medskip
Set $\Omega^0=\{1,2\}^2$, $\F^{0-}=2^{\Omega^0}$ the power set and, given some probability measure $\prob^0 \in \mathcal{P}(\Omega^0,\F^{0-})$, we set $\prob^0((i,j))=p_{i,j}$, so that $p_{i,j}\geq 0$ for all $i,j$ and $\sum_{i,j=1}^2 p_{i,j}=1$.
Consider the following open loop strategies and flows of measures:
\begin{equation}
    \begin{aligned}
        u^+_t(\omega_*)\equiv b, \qquad \mu^+ = (\prob^* \circ (tb + W_t)^{-1})_{t \in [0,T]}; \\
        u^-_t(\omega_*)\equiv a, \qquad \mu^- = (\prob^* \circ (ta + W_t)^{-1})_{t \in [0,T]}.
    \end{aligned}
\end{equation}
It was shown in \cite{bardi_fischer2019} that the pairs $(u^+,\mu^+)$ and $(u^-,\mu^-)$ are two non-equivalent open-loop solutions of the mean field game, with initial distribution $\nu=\delta_{0}$, where by ``non-equivalent'' we mean that the flows of measures $\mu^+$ and $\mu^-$ do not coincide.
We point out that this result holds for more general initial distributions $\nu\in\mathcal{P}(\R)$, see \cite[Definition 3.1 and Theorem 3.1]{bardi_fischer2019}.
Choose $a_1,a_2 \in [0,1]$, and set:
\begin{equation}\label{esempio:def_strategie_flussi}
    \begin{aligned}
        \mu^1=\tonde{a_1 \mu^+_t + (1-a_1)\mu^-_t}_{t \in [0,T]}, \\
        \mu^2=\tonde{a_2 \mu^+_t + (1-a_2)\mu^-_t}_{t \in [0,T]}.
    \end{aligned}
\end{equation}
Define $(\Lambda,\mu)$ in the following way:
\begin{equation}\label{esempio_corr_sol}
    (\Lambda,\mu)\tonde{(i,j)}=\begin{cases}
    (u^+,\mu^1) \quad (i,j)=\tonde{1,1} \\
    (u^+,\mu^2) \quad (i,j)=\tonde{1,2} \\
    (u^-,\mu^1) \quad (i,j)=\tonde{2,1} \\
    (u^-,\mu^2) \quad (i,j)=\tonde{2,2}
    \end{cases}
\end{equation}
We claim that, as long as $a<0<b$, for every $T,c>0$ there exists a probability measure $(p_{i,j})_{i,j=1,2}$ and a suitable choice of the parameters $(a_i)_{i=1,2}$ so that the tuple $((\Omega^0,\F^{0-},\prob^0),\Lambda,\mu)$ is a coarse correlated solution of the mean field game according to Definition \ref{def_mean_field_sol}.

\medskip
First of all, as shown in Example \ref{mf:example:admissible_recommendations} in Section \ref{sezione_formulazione_mfg}, since $\Lambda$ takes only two values, it is admissible.
Therefore, the tuple $((\Omega^0,\F^{0-},\prob^0),\Lambda,\mu)$ is a correlated flow.

\medskip
Let us begin with the consistency condition.
We first observe that, when the state equation is controlled by $u^+$ (respectively, $u^-$), the law of the state process at time $t$, $X_t$, is exactly $\mu^+_t$ (respectively, $\mu^-_t$), for every time $t \in [0,T]$.

Suppose that $p_{1,1}+p_{2,1}$ and $p_{1,2}+p_{2,2}$ are both strictly positive.
Then, observe that
\begin{equation*}
    \prob(X_t \in \cdot \; \vert \; \mu)(\omega)=\prob(X_t \in \cdot \; \vert \; \mu)(\omega_0)=\begin{cases}
    \prob(X_t \in \cdot \; \vert \;  \mu=\mu^1) \quad   \text{if } \mu(\omega_0)=\mu^1, \\
    \prob(X_t \in \cdot \; \vert \;  \mu=\mu^2) \quad  \text{if } \mu(\omega_0)=\mu^2.
    \end{cases}
\end{equation*}
We can compute explicitly such a conditional probability. Fix $A\in \boreliani{\R^d}$:
\begin{equation*}
    \begin{cases}
    \prob(X_t \in A \; \vert \; \mu=\mu^1) \\
    \prob(X_t \in A \; \vert \; \mu=\mu^2)
    \end{cases}  = \quad \begin{cases}
    \frac{p_{1,1}}{p_{1,1}+p_{2,1}}\prob(X^+_t \in A) +  \frac{p_{2,1}}{p_{1,1}+p_{2,1}}\prob(X^-_t \in A) \quad \text{if } \mu(\omega_0)=\mu^1, \\
    \frac{p_{1,2}}{p_{1,2}+p_{2,2}}\prob(X^+_t \in A) +  \frac{p_{2,2}}{p_{1,2}+p_{2,2}}\prob(X^-_t \in A) \quad  \text{if } \mu(\omega_0)=\mu^2.
    \end{cases}
\end{equation*}
In order to satisfy the consistency condition, it must hold
\begin{equation*}
    \begin{cases}
    \frac{p_{1,1}}{p_{1,1}+p_{2,1}}\prob(X^+_t \in A) +  \frac{p_{2,1}}{p_{1,1}+p_{2,1}}\prob(X^-_t \in A)=\mu^1_t(A) \\
    \frac{p_{1,2}}{p_{1,2}+p_{2,2}}\prob(X^+_t \in A) +  \frac{p_{2,2}}{p_{1,2}+p_{2,2}}\prob(X^-_t \in A) = \mu^2_t(A)
    \end{cases}
\end{equation*}
for every $A\in \boreliani{\R^d}$.
By definition of $\mu^1$ and $\mu^2$,
\begin{equation*}
    \begin{cases}
    \frac{p_{1,1}}{p_{1,1}+p_{2,1}}\mu^+_t(A) +  \frac{p_{2,1}}{p_{1,1}+p_{2,1}}\mu^-_t(A)=a_1\mu^+_t(A) + (1-a_1)\mu^-_t(A), \\
    \frac{p_{1,2}}{p_{1,2}+p_{2,2}}\mu^+_t(A) +  \frac{p_{2,2}}{p_{1,2}+p_{2,2}}\mu^-_t(A) = a_2 \mu^+_t(A) + (1-a_2)\mu^-_t(A)
    \end{cases}
\end{equation*}
which holds if and only if
\begin{equation}\label{esempio_condizione_flussi_misure}
    \begin{cases}
    \frac{p_{1,1}}{p_{1,1}+p_{2,1}} = a_1, \\
    \frac{p_{1,2}}{p_{1,2}+p_{2,2}} = a_2.
    \end{cases}
\end{equation}
We can regard \eqref{esempio_condizione_flussi_misure} as the consistency condition.

\medskip
We now turn our attention to the optimality condition.
Set $\gamma = \prob \circ (\Lambda,\mu)^{-1} = \prob^0 \circ (\Lambda,\mu)^{-1}$ and $\rho=\prob\circ\mu^{-1}=\prob^0\circ\mu^{-1}$.
As described in Remark \ref{mf:remark_condizione_integrale}, since $\Lambda$ takes only two values, we can  rewrite the optimality condition using disintegration of measures as
\begin{equation*}
    \int_{\A\times\pwassspace{2}{\contrd}}  \J(\alpha,m) \gamma(d\alpha,dm) \geq \int_{\pwassspace{2}{\contrd}} \J(\beta,m) \rho(dm)
\end{equation*}
under the constraint
\begin{equation*}
\begin{aligned}
    & X_t =  \int_0^t \theta_s ds + W_t, \quad 0\leq t \leq T,
\end{aligned}
\end{equation*}
for $\theta=\alpha$ in $\Lambda(\Omega^0):=\{u^+,u^-\} \subseteq \A$ on the left-hand side of the inequality above and $\theta=\beta$ in $ \A$ on the right-hand side. We rewrite explicitly the inequality as
\begin{equation}\label{esempio_funzionale2}
    \begin{aligned}
    \J(\Lambda,\mu) - \J(\beta,\mu) & =  p_{1,1}\tonde{\J(u^+,\mu^1) - \J(\beta,\mu^1)}  + p_{1,2}\tonde{\J(u^+,\mu^2) - \J(\beta,\mu^2)} \\ 
    & + p_{2,1}\tonde{\J(u^-,\mu^1) - \J(\beta,\mu^1)} + p_{2,2}\tonde{\J(u^-,\mu^2) - \J(\beta,\mu^2)} \geq 0.
    \end{aligned}
\end{equation}
Therefore, using \eqref{esempio_payoff}, we have
\begin{equation*}
    \begin{aligned}
    \J(\Lambda,\mu) - \J(\beta,\mu) & = p_{1,1}\tonde{ cT^2b(a_1b + (1-a_1)a)  - c M(\beta)T(a_1b + (1-a_1)a)}  \\ 
    & +   p_{1,2}\tonde{ cT^2 b(a_2b + (1-a_2)a)  - c M(\beta)T(a_2b + (1-a_2)a)} \\
    & + p_{2,1}\tonde{ cT^2 a(a_1b + (1-a_1)a)  - c M(\beta)T(a_1b + (1-a_1)a)} \\
    & +  p_{2,2}\tonde{ cT^2a(a_2b + (1-a_2)a)  - c M(\beta)T(a_2b + (1-a_2)a)},
    \end{aligned}
\end{equation*}
where $M(\beta):=\E[\int_0^T\beta_t dt]=\E[X^\beta_T]$.
We can set $m(\beta):=\sfrac{1}{T}M(\beta)=\sfrac{1}{T}\E[\int_0^T\beta_t dt]$. Observe that $m(\beta) \in [a,b]$, being the mean of an $A$-valued process, and $m(\A)=[a,b]$, since for every $c \in [a,b]$ the constant process $\beta\equiv c$ belongs to $\A$.
We divide by $cT^2$ to obtain the following condition:
\begin{equation}\label{esempio_funzionale3}
    \begin{aligned}
    p_{1,1} & \tonde{ b(a_1b + (1-a_1)a)  - m(\beta)(a_1b + (1-a_1)a)}  \\ 
    &+  p_{1,2}\tonde{ b(a_2b + (1-a_2)a)  - m(\beta)(a_2b + (1-a_2)a)} \\
    &+ p_{2,1}\tonde{a(a_1b + (1-a_1)a)  - m(\beta) (a_1b + (1-a_1)a)} \\
    &+  p_{2,2}\tonde{ a(a_2b + (1-a_2)a)  - m(\beta) (a_2b + (1-a_2)a)} \\
    & \geq 0.
    \end{aligned}
\end{equation}
The condition above can be seen as a positivity condition for a real affine function of $g(m)$, $m \in [a,b]$, i.e.
\begin{equation}\label{esempio_condizione_inf}
    \begin{aligned}
        & \inf_{m \in [a,b]} g(m)=\inf_{m \in [a,b]} h((p_{i,j})_{i,j=1,2} , (a_i)_{i=1,2};a,b) m + k((p_{i,j})_{i,j=1,2} , (a_i)_{i=1,2};a,b) \geq 0 \\
        & \begin{cases}
        h((p_{i,j})_{i,j=1,2} , (a_i)_{i=1,2};a,b)= & - \left\{p_{1,1}(a_1b+(1-a_1)a) + p_{1,2}(a_2 b+(1-a_2)a) \right.\\ 
        & \left. + p_{2,1}(a_1b+(1-a_1)a) + p_{2,2}(a_2b+(1-a_2)a)  \right\}, \\
        k((p_{i,j})_{i,j=1,2} , (a_i)_{i=1,2};a,b) =  & p_{1,1}b(a_1b+(1-a_1)a) + p_{1,2}b(a_2 b+(1-a_2)a)\\ 
        & + p_{2,1}a(a_1b+(1-a_1)a) + p_{2,2}a(a_2b+(1-a_2)a) . \\
        \end{cases}
    \end{aligned}
\end{equation}
We now impose the consistency condition \eqref{esempio_condizione_flussi_misure} to get:
\begin{equation}\label{esempio_coefficienti}
    \begin{aligned}
    h((p_{i,j})_{i,j=1,2} ;a,b)  & = - b \tonde{\frac{p_{1,1}^2 + p_{2,1}p_{1,1}}{p_{1,1} + p_{2,1}} + \frac{p_{1,2}^2 + p_{1,2}p_{2,2}}{p_{1,2} + p_{2,2}} }  \\
    &\quad  -a \tonde{\frac{p_{2,1}^2 + p_{2,1}p_{1,1}}{p_{1,1} + p_{2,1}} + \frac{p_{2,2}^2 + p_{1,2}p_{2,2}}{p_{1,2} + p_{2,2}} }, \\
    k((p_{i,j})_{i,j=1,2} ;a,b) & =  b^2 \tonde{\frac{p_{1,1}^2}{p_{1,1} + p_{2,1}} + \frac{p_{1,2}^2}{p_{1,2} + p_{2,2}} } +a^2 \tonde{\frac{p_{2,1}^2}{p_{1,1} + p_{2,1}} + \frac{p_{2,2}^2 }{p_{1,2} + p_{2,2}} } \\
    & \quad + 2ab\tonde{\frac{p_{1,1}p_{2,1}}{p_{1,1}+p_{2,1}} + \frac{p_{1,2}p_{2,2}}{p_{1,2}+p_{2,2}}}.
    \end{aligned}
\end{equation}
Observe that imposing the consistency condition \eqref{esempio_condizione_flussi_misure} reduces the number of parameters but makes the problem nonlinear in the probabilities $(p_{i,j})_{i,j=1,2}$.

\bigskip
Looking at \eqref{esempio_condizione_inf} and \eqref{esempio_coefficienti}, we observe that it covers the case treated in \cite{bardi_fischer2019}, for any choices of $a < 0 < b$.
Consider a probability measures $\prob^0=(p_{i,j})_{i,j=1,2}$ so that $p_{1,2}=p_{2,1}=0$ and, therefore, $p_{2,2}=1-p_{1,1}$.
Equations \eqref{esempio_coefficienti} take the simpler form
\begin{equation}\label{esempio_coefficienti_diagonale}
\begin{aligned}
    h((p_{1,1},0,0,1-p_{1,1}); a,b) & = - bp_{1,1} -a(1-p_{1,1}),  \\
    k((p_{1,1},0,0,1-p_{1,1}); a,b) &= b^2p_{1,1} +a^2(1-p_{1,1}).
\end{aligned}
\end{equation}
Sending $p_{1,1}$ to $1$, by continuity, we get
$h((1,0,0,0);a,b)=-b$ and $k((1,0,0,0);a,b)=b^2$, so that the condition \eqref{esempio_condizione_inf} becomes
\begin{equation*}
    \inf_{ m \in [a,b]} -bm + b^2 \geq 0,
\end{equation*}
which is satisfied for every $b>0$.
On the other hand, by sending $p_{1,1}$ to $0$, we get $h((0,0,0,1);a,b)=-a$ and $k((0,0,0,1);a,b)=a^2$ and condition \eqref{esempio_condizione_inf} takes the form
\begin{equation*}
    \inf_{m \in [a,b]} -am+a^2 \geq 0,
\end{equation*}
which is satisfied for every $a<0$.
Observe that, when $p_{1,1}=1$, $\mu\equiv\mu^1=\mu^+$, while, when $p_{2,2}=1$, $\mu\equiv\mu^2=\mu^-$.
This shows that the deterministic correlated flows $(\Lambda,\mu)\equiv (u^+,\mu^+)$ and $(\Lambda,\mu)\equiv (u^-,\mu^-)$ are indeed mean field CCE in the sense of Definition \ref{def_mean_field_sol}.

\medskip
Turning to more interesting cases, consider $[a,b]=[-1,1]$.
The choice of a symmetric interval is not necessary, but it has been made to ease the comparison with previous results in the literature (see the next subsection).
Figure \ref{fig:esempio_immagine} shows the existence of coarse correlated mean field equilibria as the probability measure $(p_{i,j})_{i,j=1,2}$ varies.
In particular, it shows the existence of infinitely many coarse correlated mean field equilibria for the system.


\begin{figure}
	\centering
	\includegraphics[width=\textwidth]{sections/esempio_immagine.jpg}
	\caption{Existence of correlated equilibria as probability measure $(p_{i,j})_{i,j=1,2}$ varies for $[a,b]=[-1,1]$.
	White points correspond to the values of $(p_{i,j})$ so that $\inf_{m\in[a,b]} g(m) \geq 0$, black points to the other ones.
	\\
	The probability measure has been generated in different ways above and below the dashed diagonal.
	Above the diagonal, $p_{1,1}$ and $p_{2,2}$ vary from $0$ to $1$ as indicated by the axis.
	We set $p_{1,2}=\alpha(1-p_{1,1}-p_{2,2})$, $p_{2,1}=(1-\alpha)(1-p_{1,1}-p_{2,2})$ for some value $\alpha \in [0,1]$.
	\\
	Under the  diagonal, we have a symmetric choice of probability: again, $p_{1,1}$ and $p_{2,2}$ vary from $0$ to $1$, although in the opposite directions respect to above the diagonal, and we set $p_{1,2}=(1-\alpha)(1-p_{1,1}-p_{2,2})$, $p_{2,1}=\alpha(1-p_{1,1}-p_{2,2})$, for the same choice of $\alpha \in [0,1]$ as for above the diagonal. Different choices of $\alpha$ are considered. \\
    } \label{fig:esempio_immagine}
\end{figure}


White spots in Figure \ref{fig:esempio_immagine} refer to those probability measures on $(\Omega^0,\F^{0-})$ so that $(\Lambda,\mu)$ is indeed a mean field CCE.
Observe that, on the dashed diagonals, it always holds $p_{1,1}+p_{2,2}=1$, which implies that the coarse correlated solution $(\Lambda,\mu)$ is a randomization of the open loop MFG solutions $(u^+,\mu^+)$ and $(u^-,\mu^-)$.
On the other hand, there exist infinitely many coarse correlated solutions of the mean field game so that $\Lambda$ is not a deterministic function of $\mu$, i.e., they are not a randomization of the solutions $(u^+,\mu^+)$ and $(u^-,\mu^-)$.


\subsection*{Comparison with Lacker's notion of weak mean field game solution without common noise of \cite{lacker2016}}

Consider $A=[-1,1]$, $T=2$.
With this choice of control actions and time horizon, the example we proposed matches the setting of Lacker's \say{illuminating example} of \cite[Section 3.3]{lacker2016}.
We show that there exists a coarse correlated solution of the MFG which is not a weak MFG solution without common noise as defined in Definition 3.1 therein.
In particular, the most important feature is the fact that the recommendation $\Lambda$ can not be expressed as a deterministic function of the flow of measures.

\medskip
To be consistent with the notation and the setup of Lacker's paper, we use the notion of relaxed controls, which are used extensively in Section \ref{sezione_existence} (see, in particular, Section \ref{existence:sezione_crtl_rilassati} for definitions, notation and some important properties).
Let $(p_{i,j})_{i,j=1,2}$ be so that $p_{1,1}+p_{2,1}$ and $p_{1,2}+p_{2,2}$ are strictly positive.
We introduce the relaxed controls $\delta^+=(\delta^+_t)_{t \in [0,T]}$ and $\delta^-=(\delta^-_t)_{t \in [0,T]}$, by setting
\begin{equation*}
\begin{aligned}
    & \delta^+_t(\omega_*;da)=\delta_{u^+_t(\omega_*)}(da) \equiv \delta_{1}(da), \quad \forall t \in [0,T], \omega_* \in \Omega^*,\\
    & \delta^-_t(\omega_*;da)=\delta_{u^-_t(\omega_*)}(da) \equiv \delta_{-1}(da), \quad \forall t \in [0,T], \omega_* \in \Omega^*.
\end{aligned}
\end{equation*}
Consider the correlated flow $(\Lambda,\mu)$ defined by \eqref{esempio_corr_sol}
and observe that the strategy $\lambda=(\lambda_t)_{t \in [0,T]}$ associated to the admissible recommendation $\Lambda$ can be rewritten as a relaxed control as
\begin{equation}\label{raccomandazione_come_lacker}
    \mathfrak{r}_t(\omega;da)= \mathfrak{r}_t(\omega_0,\omega_*;da)=\1_{\{\Lambda=u^+\}}(\omega_0)\delta^+_t(da) + \1_{\{\Lambda=u^-\}}(\omega_0)\delta^-_t(da).
\end{equation}
We point out that $\mathfrak{r}$ does not depend on $\omega_*$ since $\delta^+$ and $\delta^-$ do not depend on $\omega_*$.
Starting from $(\Lambda,\mu)$, we define a random variable $\Tilde{\mu}$ with values in $\mathcal{P}(\contrd \times \mathcal{V} \times \contrd)$ by setting
\begin{equation}\label{esempio_comparison_flusso}
    \Tilde{\mu}(\cdot)=\prob((W,\mathfrak{r},X) \in \cdot \; \vert \; \mu ).
\end{equation}
We observe that $\sigma(\mu)=\sigma(\Tilde{\mu})$: we have $\sigma(\Tilde{\mu}) \subseteq \sigma(\mu)$ since, by definition of regular conditional probability, $\Tilde{\mu}$ must be $\sigma(\mu)$ measurable; to get the opposite inclusion, for every $t \in [0,T]$, let $\Tilde{\mu}^x_t$ be the push forward of $\Tilde{\mu}$ through the map $\contrd \times \mathcal{V} \times \contrd \ni (w,q,x) \mapsto x_t \in \R^d$.
Then, by exploiting the consistency condition \eqref{def_mean_field_sol:cons}, we have 
\begin{equation*}
    \Tilde{\mu}^x_t(A)=\Tilde{\mu}(\{x \in \contrd: \; x_t \in A\})=\prob(X_t \in A \; \vert \; \mu ) = \mu_t(A),
\end{equation*}
for every $A \in \boreliani{\R^d}$, i.e. $\Tilde{\mu}^x_t=\mu_t$ $\prob$-a.s, for every $t \in [0,T]$.
Let $(\mathcal{B}_{t,\contrd})_{t \in [0,T]}$ be the natural filtration of the identity process on $\contrd$, i.e. $\mathcal{B}_{t,\contrd}=\sigma(\contrd \ni x \mapsto x_s \in \R^d: \; 0 \leq s \leq t)$, and let $(\F^{\Tilde{\mu}}_t)_{t \in [0,T]}$ be the natural filtration of $\Tilde{\mu}$, that is
\begin{equation*}
    \F^{\Tilde{\mu}}_t=\sigma(\Tilde{\mu}\tonde{C}: \; C \in \mathcal{B}_{t,\contrd}\otimes \F^{\mathcal{V}}_t \otimes \mathcal{B}_{t,\contrd}).
\end{equation*}
We observe that, for every $t \in (0,T]$, we have $\F^{\Tilde{\mu}}_t=\sigma(\mu)$.
To see this, observe that
\begin{equation*}
    \sigma(\mu) \supseteq \F^{\Tilde{\mu}}_t \supseteq \sigma(\Tilde{\mu}^x_s: \; s \leq t) = \sigma(\mu_s: \; s \leq t) = \sigma(\mu),
\end{equation*}
where the last equality holds for every $t>0$, as can be verified by explicit calculations.
Finally, for $t=0$, we have $\F^{\Tilde{\mu}}_0=\{\emptyset,\Omega^0\}$.
Having established the relations between such $\sigma$-algebras, it is straighforward to verify that the tuple $((\Omega,\F,\mathbb{F},\prob),W,\Tilde{\mu},\mathfrak{r},X)$ satisfies properties (1-4) and (6) of \cite[Definition 3.1]{lacker2016}.
Now, pick a probability measure $\prob^0$ so that $\min(p_{1,2},p_{2,1}) > 0$ and $\overline{\mu}^1_T>0$, $\overline{\mu}^2_T<0$.
Figure \ref{fig:esempio_immagine} shows that such a choice is possible (actually, there exist infinitely many measures $\prob^0$ with the desired property).
For such a choice of $\prob^0$, the relaxed control $\mathfrak{r}$ does not satisfy the optimality condition (5) of \cite[Definition 3.1]{lacker2016}, since, as shown in \cite[Section 3.3]{lacker2016}, every optimal control $\mathfrak{r}^*$ must be of the form $\mathfrak{r}^*_t(da)(\omega)=\delta_{\alpha^*_t(\omega)}(da)$ for $Leb_{[0,T]}$-a.e. $t$, with
\begin{equation*}
    \alpha^*_t = \text{sign}\tonde{\attesa{\Tilde{\mu}^x_T \; \vert \; \F^{\Tilde{\mu}}_t}}.
\end{equation*}
Here, $\text{sign}\tonde{0}=0$.
Since $\F^{\Tilde{\mu}}_0$ is trivial and $\F^{\Tilde{\mu}}_t=\sigma(\mu)$ for $t > 0$, the optimal control $\alpha^*_t$ must be equal to
\begin{equation}\label{esempio_comparison_ctrl_ottimo}
    \alpha^*_t(\omega)=\alpha^*_t(\omega_0)=\begin{cases}
    -1 \quad \text{ if } \overline{\mu}_T(\omega_0)<0, \\
    1 \qquad \text{ if } \overline{\mu}_T(\omega_0)>0,
    \end{cases} \quad 0 < t \leq T,
\end{equation}
and equal to an arbitrary value at $t=0$.
In particular, observe that such a control is a deterministic function of the measure $\Tilde{\mu}$.
For every $\prob^0$ so that $p_{1,2}+p_{2,1}>0$, this is not the case of the correlated flow $(\Lambda,\mu)$ defined in \eqref{esempio_corr_sol}, since $\Lambda$ is not a deterministic function of $\mu$.

\medskip
The essential reason for the lack of optimality, in the sense of Lacker, of the relaxed control $\mathfrak{r}$ defined by \eqref{raccomandazione_come_lacker} resides in the differences between allowed deviations: on one hand, for weak mean field games solutions in the sense of \cite{lacker2016}, all adapted compatible controls $\mathfrak{b}=(\mathfrak{b}_t)_{t \in [0,T]}$ are allowed, where ``compatible'' means that $\sigma(\mathfrak{b}_s: s \leq t)$ is conditionally independent of $\F^{\xi,\Tilde{\mu},W}_T$ given $\F^{\xi,\Tilde{\mu},W}_t$ for every $t$, which leads to a very rich class of controls.
On the other hand, for coarse correlated solution of the MFG, only $\mathbb{F}^*$-progressively measurable strategies are allowed as deviations.
Therefore, many more solutions exist.

\medskip
More generally, one can not compare weak MFG solutions without common noise of \cite{lacker2016} and mean field CCEs, due to the difference between the respective consistency conditions.
Nevertheless, we can make an additional assumption on the random measure $\Tilde{\mu}$ which makes it possible to define a mean field CCE starting from a weak MFG solution.
Let $\Tilde{\mu}$ be a weak MFG solution without common noise.
Let $\mu_t$ be the push forward of $\Tilde{\mu}$ through the map $\contrd \times \mathcal{V} \times \contrd \ni (w,q,x) \mapsto x_t \in \R^d$.
Define a random flow of measures by setting $\mu=(\mu_t)_{t \in [0,T]}$.
Assume that the flow of measures $\mu$ carries the same information as the random measure $\Tilde{\mu}$, i.e.
\begin{equation}\label{esempio:hp_filtrazioni}
    \sigma(\mu_s: \; 0 \leq s \leq t)=\F^{\Tilde{\mu}}_t, \quad \forall t \in [0,T].
\end{equation}
If a weak MFG solution $\Tilde{\mu}$ satisfies condition \eqref{esempio:hp_filtrazioni}, then $\Tilde{\mu}$ does induce a mean field CCE.
Indeed, set $\rho=\prob\circ\mu^{-1}$.
By \eqref{esempio:hp_filtrazioni}, we have $\mu_t=\prob(X \in \cdot \; \; \vert \; \Tilde{\mu}) = \prob(X \in \cdot \; \; \vert \; \mu)$, i.e. consistency condition \eqref{def_mean_field_sol:cons} is satisfied.
Moreover, the assumption on equality of the filtrations ensures that there exists a progressively measurable function $\varphi:[0,T]\times\contpdue \to A$ so that
\begin{equation*}
    \alpha^*_t=\text{sign}\tonde{\E\quadre{\mu_T \; \vert \; \F^{\Tilde{\mu}}_t}}=\varphi\tonde{t,\mu}.
\end{equation*}
Then, we define $(\Omega^0,\F^{0-},\prob^0)$ and $(\Lambda^*,\mu^*)$ as
\begin{equation}
    \begin{aligned}
        & (\Omega^0,\F^{0-},\prob^0)=\tonde{\contpdue,\boreliani{\contpdue},\rho}, \\
        & \mu^*=\text{Id}:\tonde{\contpdue,\boreliani{\contpdue},\rho} \to \tonde{\contpdue,\boreliani{\contpdue},\rho}, \\
        & \begin{aligned}
        \Lambda^*: \tonde{\contpdue,\boreliani{\contpdue},\rho} & \to (\A,\boreliani{\A}) \\
        m & \mapsto \Lambda^*(m)=(\varphi(t,m))_{t \in [0,T]}.
        \end{aligned}
    \end{aligned}
\end{equation}
By Lemma \ref{esempi:lemma_misurabile}, the tuple $((\Omega^0,\F^{0-},\prob^0),\Lambda^*,\mu^*)$ is a correlated flow.
Let $X^*$ be the solution of \eqref{esempio_dinamiche} on the product probability space $(\Omega,\F,\mathbb{F},\prob)$ defined in point \ref{mf:condizione_ammissibilita} of Definition \ref{mf:admissible_recommendation}.
Since uniqueness in law holds by Theorem \ref{teorema_di_unicita_legge}, it follows that $(X^*,\mu^*)$ has the same joint law as $(X,\mu)$, which implies that the consistency condition \eqref{def_mean_field_sol:cons} is satisfied.
Since $\lambda^*_t=\varphi(t,\mu^*)$, $(\Lambda^*,\mu^*)$ satisfies optimality condition \eqref{def_mean_field_sol:opt} as well and therefore it is a mean field CCE.


We observe that the additional assumption on the filtrations \eqref{esempio:hp_filtrazioni} is satisfied both by the weak MFG solution exhibited in \cite[Proposition 3.7]{lacker2016} and in our case, as shown above.
We point out that this CCE has been already considered: suppose that the flow of measures as law $\rho=a\delta_{\mu^+} + (1-a)\delta_{\mu^-}$, $a \in (0,1)$, for $\mu^+$ and $\mu^-$ given by \eqref{esempio:def_strategie_flussi}.
Then, the correlated flow $(\Lambda^*,\mu^*)$ corresponds to the white spots on the dashed diagonal of Figure \ref{fig:esempio_immagine}, i.e. to the probability measures $\prob^0$ so that $p_{1,2}=p_{2,1}=0$, $p_{1,1}=a$ and $p_{2,2}=1-a$.
Roughly speaking, it correspond to the case when $\Lambda^*=\phi(\mu^*)$ $\prob^0$-a.s., for some deterministic measurable $\phi$.


\appendix
\addcontentsline{toc}{section}{Appendices}
\section*{Appendix}
\section{Weak and strong existence for controlled equations}\label{appendix_weak_strong_sol}



We state and prove a Yamada-Watanabe type result for stochastic differential equations with random coefficients as the ones encountered so far.
Recall from Section \ref{existence:sezione_crtl_rilassati} the definition of the space $\mathcal{V}$ and of relaxed controls.

Let $\mathfrak{U}=((\Omega,\F,\mathbb{F},\prob),\xi,W,\mu,\mathfrak{r})$ be a tuple composed by a filtered probability space satisfying usual assumptions, an $\mathbb{F}$-Brownian motion $W$, an $\R^d$-valued $\F_0$-measurable random variable $\xi$, an $\F_0$-measurable random flow of measures $\mu$ taking values in  $\contpdue$ and an $\mathbb{F}$-adapted $\mathcal{V}$-valued random variable $\mathfrak{r}$, in the sense that the random variables $\mathfrak{r}(C)$ are $\F_t$-measurable for every $C \in \boreliani{[0,t]\times A}$.
Let us consider the following stochastic differential equation:
\begin{equation}\label{weak_uniqueness:sde}
    dX_t=G(t,X_t,\mu,\mathfrak{r})dt + dW_t, \quad X_0=\xi.
\end{equation}
where $G:[0,T] \times \R^d \times \contpdue \times \mathcal{V} \to \R^d$ is jointly measurable and progressively measurable in $\mathcal{V}$; progressive measurability must be understood in the following sense: for every $q,q' \in \mathcal{V}$, for every $(t,x,m) \in [0,T] \times \R^d \times \contpdue$, it holds:
\begin{equation*}
    q(C)=q'(C) \; \;  \forall C \in \boreliani{[0,t] \times A} \; \Longrightarrow \; G(t,x,m,q)=G(t,x,m,q').
\end{equation*}



\begin{definition}[Strong solution and pathwise uniqueness]\label{weak_uniqueness:def_pathwise_uniqueness}
Let $\mathfrak{U}=((\Omega,\F,\mathbb{F},\prob),\xi,W,\mu,\mathfrak{r})$ be a tuple as above.
A strong solution to equation \eqref{weak_uniqueness:sde} on $\mathfrak{U}$ is a continuous $\mathbb{F}$-adapted process $X=(X_t)_{t \in [0,T]}$ adapted to the $\prob$-augmentation of $\mathbb{F}$ so that
\begin{equation}
    X_t=\xi+\int_0^t G(s,X_s,\mu,\mathfrak{r})ds + W_t, \quad 0 \leq t \leq T,
\end{equation}
holds $\prob$-almost surely.

Patwhise uniqueness holds for equation \eqref{weak_uniqueness:sde} if, given two strong solutions $X$ and $X'$ to \eqref{weak_uniqueness:sde} on $\mathfrak{U}$, they are indistinguishable:
\begin{equation*}
    \prob(X_t=X'_t \;\; \forall t \in [0,T])=1.
\end{equation*}
\end{definition}


\begin{definition}[Weak solution and uniqueness in law]\label{weak_uniqueness:def_uniqueness_in_law}
A weak solution to equation \eqref{weak_uniqueness:sde} is a tuple $\mathfrak{U}=((\Omega,\F,\mathbb{F},\prob),\xi,W,\mu,\mathfrak{r})$ as above so that there exists a continuous $\mathbb{F}$-adapted process $X=(X_t)_{t \in [0,T]}$ satisfying equation \eqref{weak_uniqueness:sde}.

Weak uniqueness holds for equation \eqref{weak_uniqueness:sde} if for any two weak solution of \eqref{weak_uniqueness:sde} $\mathfrak{U}^i$, $i=1,2$, so that $\prob^1\circ(\xi^1,W^1,\mu^1,\mathfrak{r}^1)^{-1}=\prob^2\circ(\xi^2,W^2,\mu^2,\mathfrak{r}^2)^{-1}$, it holds
\begin{equation*}
    \prob^1\circ(X^1,\xi^1,W^1,\mu^1,\mathfrak{r}^1)^{-1}=\prob^2\circ(X^2,\xi^2,W^2,\mu^2,\mathfrak{r}^2)^{-1},
\end{equation*}
where $X^i$ are the continuous $\mathbb{F}^i$-adapted processes that satisfy equation \eqref{weak_uniqueness:sde} on $\mathfrak{U}^i$, $i=1,2$.
\end{definition}


\begin{thm}\label{teorema_di_unicita_legge}
Suppose pathwise uniqueness holds for equation \eqref{weak_uniqueness:sde}, in the sense of Definition \ref{weak_uniqueness:def_pathwise_uniqueness}.
Then, uniqueness in law in the sense of Definition \ref{weak_uniqueness:def_uniqueness_in_law} holds as well.
\end{thm}

\begin{proof}
Let $\mathfrak{U}^1$ and $\mathfrak{U}^2$ be two weak solutions of equation \eqref{weak_uniqueness:sde} in the sense of Definition \ref{weak_uniqueness:def_uniqueness_in_law} above.
Since pathwise uniqueness holds for equation \eqref{weak_uniqueness:sde} by assumption, our goal is to bring together the solution on the same filtered probability space.
Let us define the following probability measures:
\begin{equation*}
\begin{aligned}
    & \hat{\Q}^i = \prob^i\circ(\xi^i,W^i,\mu^i,\mathfrak{r}^i,X^i)^{-1} \in \mathcal{P}(\R^d \times \contrd \times \contpdue \times \mathcal{V} \times \contrd), \quad i=1,2, \\
    & \Q = \prob^i\circ(\xi^i,W^i,\mu^i,\mathfrak{r}^i)^{-1} \in \mathcal{P}(\R^d \times \contrd \times \contpdue \times \mathcal{V}), \\
    & \Tilde{\Q} = \prob^i\circ(\xi^i,W^i,\mu^i)^{-1} \in \mathcal{P}(\R^d \times \contrd \times \contpdue).
\end{aligned}
\end{equation*}
Observe that $\Q$ and $\Tilde{\Q}$ are well defined, since $(\xi^i,W^i,\mu^i,\mathfrak{r}^i)$ share the same joint law by assumption.
Let us consider the following space:
\begin{equation*}
\begin{aligned}
    \Omega^{can} & =\contrd \times \contrd \times \R^d \times \contrd \times \contpdue \times \mathcal{V}; \\
    \F^{can} & =\boreliani{\contrd} \otimes \boreliani{\contrd } \otimes \boreliani{\R^d } \otimes \boreliani{ \contrd } \otimes \boreliani{\contpdue } \otimes \boreliani{\mathcal{V}}; \\
    \mathcal{G}_t^{can} &  = \mathcal{B}_{t,\contrd} \otimes \mathcal{B}_{t,\contrd} \otimes \boreliani{\R^d} \otimes \mathcal{B}_{t,\contrd} \otimes \boreliani{\contpdue} \otimes \F_t^{\mathcal{V}},
\end{aligned}
\end{equation*}
where
\begin{equation*}
\begin{aligned}
    & \mathcal{B}_{t,\contrd}=\sigma(\contrd \ni x \mapsto x_s \in \R^d: \; s \leq t), && \F^{\mathcal{V}}_t=\sigma( \mathcal{V} \ni q \mapsto q(C) \in \R: \; C \in \boreliani{[0,t]\times A}).
\end{aligned}
\end{equation*}
In order to equip the space $(\Omega^{can},\F^{can},(\mathcal{G}_t^{can})_{t \in [0,T]})$ with a probability measure, we disintegrate the measures $\hat{\Q}^i$, $i=1,2$, in the following way:
let $K^i:\boreliani{\contrd} \times \R^d \times \contrd \times \contpdue \times \mathcal{V} \to [0,1]$ be a regular conditional probability of $\hat{\Q}^i$ for $\boreliani{\contrd}$ given $(x,w,m,q)$, so that it holds
\begin{equation*}
    \hat{\Q}^i( A \times B ) = \int_B K^i(A , x,m,w,q)P(dx,dm,dw,dq),
\end{equation*}
for every $A \in \boreliani{\contrd}$, $B \in \boreliani{\R^d} \otimes \boreliani{\contrd} \otimes \boreliani{\contpdue} \otimes \boreliani{\mathcal{V}}$, or more briefly
\begin{equation*}
    \hat{Q}^i(dx,dw,dm,dq,dy)=K^i(dy,x,m,q,w)P(dx,dw,dm,dq), \: i=1,2.
\end{equation*}

Then, we set 
\begin{equation*}
    \overline{\Q}(dy^1,dy^2,dx,dm,dw,dq)=K^1(dy^1,x,m,q,w)K^2(dy^2,x,m,q,w)\Q (dx,dm,dw,dq).
\end{equation*}
Observe that the joint law under $\overline{\Q}$ of the coordinate projections $y^1$, $x$, $m$, $w$ and $q$ is exactly $\hat{\Q}^1$, and analogously when considering the coordinate process $y^2$ instead of $y^1$.
Finally, complete the $\sigma$-algebra $\F^{can}$ with the $\overline{\Q}$-null sets $\mathcal{N}^{\overline{\Q}}$ and consider the complete right continuous filtration $(\F_t^{can})_{t \in [0,T]}$ given by
\begin{equation*}
    \F_t^{can}=\bigcap_{\eps > 0} \sigma\tonde{\mathcal{G}_{t+\eps}, \mathcal{N}^{\overline{\Q}}}.
\end{equation*}
By Lemma \ref{lemma_moto_browniano}, the coordinate process $w$ is a $(\F^{can}_t)_{t \in [0,T]}$-Brownian motion under $\overline{\Q}$.
Furthermore, it holds
\begin{equation*}
\begin{aligned}
    y^i_t=x + \int_0^t G(s,y^i_s,m,q)ds + w_t, \; \forall t \in [0,T], \; \overline{\Q}\text{-a.s.}
\end{aligned}
\end{equation*}
for $i=1,2$.
Since pathwise uniqueness in the sense of Definition \ref{weak_uniqueness:def_pathwise_uniqueness} holds by assumption, it follows that $y^1$ and $y^2$ are indistinguishable under $\overline{\Q}$, which implies $\hat{\Q}^1=\hat{\Q}^2$.
This proves the desired result.
\end{proof}


\begin{lemma}\label{lemma_moto_browniano}
In the construction of Theorem \ref{teorema_di_unicita_legge}, $w=(w_s)_{s \in [0,T]}$ is a Brownian motion under $\overline{\Q}$ with respect to the filtration $(\F^{can}_s)_{s \in [0,T]}$.
\end{lemma}
\begin{proof}
Observe that $w$ is a natural Brownian motion under $\overline{\Q}$.
In order to show that it is a Brownian motion with respect to the filtration $(\mathcal{G}_t^{can})_{t \in [0,T]}$, we just need to prove that its increments are independent, and the conclusion follows.

Fix $A_1,A_2 \in \mathcal{B}_{t,\contrd}$, $B \in \boreliani{\R^d}$, $C \in \mathcal{B}_{t,\contrd}$, $D \in \boreliani{\contpdue}$ and $F \in \F_t^{\mathcal{V}}$.
By Cauchy-Schwartz inequality, we have, for every $H \in \boreliani{\R^d}$ and $s>t$:
\begin{equation*}
\begin{aligned}
    \E^{\overline{\Q}} & \quadre{\1_H(w_s-w_t)\1_{A_1 \times A_2 \times B \times D \times C \times F}(y^1,y^2,x,m,w,q)}^2 \\
    \leq & \E^{\overline{\Q}}\Big[ \1_H(w_s-w_t)\1_{A_1 \times B \times D \times C \times F}(y^1,x,m,w,q)\Big] \\
    & \cdot \E^{\overline{\Q}}\Big[\1_H(w_s-w_t)\1_{A_2 \times B \times D \times C \times F}(y^2,x,m,w,q) \Big].
\end{aligned}
\end{equation*}
Therefore, it suffices to show that 
\begin{equation}\label{lemma_moto_browniano:indipendenza_incrementi_singoli}
\begin{aligned}
    \E^{\overline{\Q}}\Big[ \1_H(w_s-w_t)\1_{A_1 \times B \times D \times C \times F}(y^1,x,m,w,q)\Big]=0.
\end{aligned}
\end{equation}
Since the integrand does not depend upon $y^2$, we may rewrite such an expectation only with respect to $\hat{\Q}^1$:
\begin{equation*}
\begin{aligned}
    \E^{\overline{\Q}} & \Big[ \1_H(w_s-w_t)\1_{A_1 \times B \times D \times C \times F}(y^1,x,m,w,q)\Big] \\
    = & \int \1_H(w_s-w_t) \1_{A_1 \times B \times D \times C \times F}(y^1,x,m,w,q)\hat{\Q}^1(dy^1,dx,dm,dw,dq).
\end{aligned}
\end{equation*}
Then, we introduce another disintegration of the measure $\hat{\Q}^1$: let $\Theta^1$ be a regular conditional probability for $\boreliani{\contrd}\otimes \boreliani{\mathcal{V}}$ given $(x,w,m)$:
\begin{equation}\label{lemma_moto_browniano:dinsintegrazione_senza_controllo}
    \hat{\Q}^1\tonde{ A \times B \times C \times D \times F } = \int_{ B \times C \times D } \Theta^1(A\times F, x,m,w)\Tilde{\Q}(dx,dm,dw),
\end{equation}
for every $A \in \boreliani{\contrd}$, $B \in \boreliani{\R^d}$, $C \in \boreliani{\contrd}$, $D \in \boreliani{\contpdue}$ and $F \in \boreliani{\mathcal{V}}$, or more briefly
\begin{equation*}
    \hat{\Q}^1(dy^1,dq,dx,dw,dm)= \Theta^1(dy^1,dq,x,m,w)\Tilde{\Q}(dx,dw,dm).
\end{equation*}
As in \cite[Lemma IV.1.1]{ikeda_watanabe1981sdes}, it can easily be shown that,
for every $A\times F \in \mathcal{B}_{s,\contrd} \otimes \F_s^{\mathcal{V}}$, the map
\begin{equation*}
    (x,m,w) \mapsto \Theta^1(A\times F,x,m,w)  
\end{equation*}
is $\boreliani{\R^d} \otimes \boreliani{\contpdue}\otimes \mathcal{B}_{s,\contrd}$-measurable, for every $s \in [0,T]$.
Therefore, we can compute the left-hand side of \eqref{lemma_moto_browniano:indipendenza_incrementi_singoli}:
\begin{equation*}
\begin{aligned}
    \E^{\overline{\Q}} & \Big[ \1_H(w_s-w_t)\1_{A_1 \times B \times D \times C \times F}(y^1,x,m,w,q)\Big] \\
    = & \int \1_H(w_s-w_t) \1_{A_1 \times B \times D \times C \times F}(y^1,x,m,w,q)\hat{\Q}^1(dy^1,dx,dm,dw,dq) \\
    = & \int \1_H(w_s-w_t) \Theta^1(A_1\times F, x,m,w)\1_{B \times D \times C}(x,m,w)\Tilde{\Q}^1(dx,dm,dw) \\
    = & \E^{\prob^{1}}\quadre{ \1_H(W^1_s-W^1_t) \Theta^1(A_1\times F, \xi^1,\mu^1,W^1)\1_{B \times D \times C}(\xi^1,\mu^1,W^1)} = 0,
\end{aligned}
\end{equation*}
since $\Theta^1(A_1\times F, \xi^1,\mu^1,W^1)\1_{B \times D \times C}(\xi^1,\mu^1,W^1)$ is $\F^1_s$-measurable and $W^1$ is an $\mathbb{F}^1$-Brownian motion under $\prob^1$ by assumption.
\end{proof}


\section{On admissible recommendations}\label{appendix_recommendations}



\begin{lemma}\label{esempi:lemma_misurabile}
Let $(\Omega^0,\F^{0-},\prob^0)$ be a complete probability spaces and $(\Omega^*,\F^*,\mathbb{F}^*,\prob^*)$ be a filtered probability space satisfying the usual assumptions.
Fix a bounded $A$-valued process $(\lambda_t)_{t \in [0,T]}$ defined on the completion of the product space $(\Omega^0\times\Omega^*,\F^{0-}\otimes\F,\prob^0\otimes\prob^*)$.
Assume that it is progressively measurable with respect to the filtration $\mathbb{F}=(\F_t)_{t \in [0,T]}$, where $\mathbb{F}$ is the $\prob^0\otimes\prob^*$-augmentation of the filtration $(\F^{0-}\otimes\F^*_t)_{t \in [0,T]}$.
Define a function $\Lambda:\Omega^0\to \A$ by setting
\begin{equation}\label{appendix:raccomandazione_indotta}
\begin{aligned}
    \Lambda(\omega_0)=\left \{ \: \begin{aligned}
        & \begin{aligned}
            (\lambda_{t}(\omega_0,\cdot))_{t \in [0,T]} &  :\space [0,T] \times \Omega^* \to A \\
        & (t,\omega_*) \to \lambda_t(\omega_0,\omega_*), 
        \end{aligned} &&  \omega^0 \in \Omega^0\setminus N, \\
        & a_0 && \omega_0 \in N.
    \end{aligned} \right.
\end{aligned}
\end{equation}
where $N\subset \Omega^0$ is a $\prob^0$-null set and $a_0$ is some point in $A$.
The function $\Lambda$ defined in \eqref{appendix:raccomandazione_indotta} is an admissible recommendation.
\end{lemma}
\begin{proof}
Take any bounded $(\F)_{t \in [0,T]}$-progressively measurable process $(\lambda_t)_{t \in [0,T]}$ defined on the product space $(\Omega^0\times\Omega^*,\F^{0-}\otimes\F^*,\prob^0\otimes\prob^*)$ taking values in $\R$, and not necessarily in $A$.

\medskip
Observe that it is always possible to define a function $\Lambda$ from $(\Omega^0,\F^{0-},\prob^0)$ to $(L^2([0,T]\times \Omega^*;\mathcal{P}^*,Leb_{[0,T]}\otimes\prob^*),\boreliani{L^2})$ as in \eqref{appendix:raccomandazione_indotta}, where $\mathcal{P}^*$ denotes the progressive $\sigma$-algebra associated to the filtration $\mathbb{F}^*$.
Indeed, by construction of the filtration $\mathbb{F}$, since $\lambda$ is $\mathbb{F}$-progressively measurable, there exists a set $N \subset \Omega^0$, $\prob^0(N)=0$, so that the section $(\lambda_t(\omega_0,\cdot))_{t \in [0,T]}$ is $\mathcal{P}^*$-measurable for every $\omega_0 \in \Omega^0 \setminus N$.
Set $\Lambda(\omega_0)=(\lambda_t(\omega_0,\cdot)_{t \in [0,T]})$ for $\omega_0 \in \Omega^0 \setminus N$ and $\Lambda(\omega_0) \equiv a_0$ for $\omega_0 \in N$, where $a_0$ is any point in $\R$, which is exactly \eqref{appendix:raccomandazione_indotta}.

\medskip
Let $\mathcal{H}$ be the set of bounded progressively measurable processes $\lambda$ so that the function $\Lambda$ defined according to \eqref{appendix:raccomandazione_indotta} is a $\F^{0-}\setminus\boreliani{L^2}$ measurable random variable.
We show that $\mathcal{H}$ is a monotone class which contains the set $\mathcal{E}$ of progressively measurable processes $\lambda:[0,T]\times\Omega^0\times\Omega^* \to \R $ of the form
\begin{equation}\label{esempi:processi_semplici}
    \lambda_t=\sum_{i=1}^n \zeta^i\1_{[t_i,t_{i+1})}(t),
\end{equation}
where $n \geq 1$, $t_i \in [0,T]$, $t_i<t_{i+1}$ for every $i=1,\dots,n$, $\zeta^i$ are bounded $\F_{t_i}$-measurable random variables.
Having established such properties, we apply monotone class theorem (as stated, e.g., in \cite[Theorem II.3.1]{rogerswilliams_vol1}) to conclude that $\mathcal{H}$ contains the set of $\mathbb{F}$-progressively measurable bounded processes defined on the product space $\Omega^0\times\Omega^*$.


\medskip
To see that $\mathcal{H}$ is a monotone class, observe that $\mathcal{H}$ is clearly a vector space and contains all processes $\lambda$ so that $\lambda_t\equiv c$ for every $t \in [0,T]$, for all $c \in \R$.
Let $(\lambda^n)_{n \geq 1} \subseteq \mathcal{H}$, with $\lambda^n \uparrow \lambda$ as $n$ goes to infinity, $\lambda^n$ positive and bounded by the same constant $C\geq 0$ for every $n$.
By monotone convergence, $\lambda$ is bounded and $\F^{0-} \otimes \mathcal{P}^*$-measurable as well, so that we can define $\Lambda$ as in \eqref{appendix:raccomandazione_indotta}, as previously discussed.
Let $\Lambda^n$ be the $L^2$-valued random variables defined starting from $\lambda^n$ according to \eqref{appendix:raccomandazione_indotta}, which are $\F^{0-} \setminus \boreliani{L^2}$ measurable since $\lambda^n$ belongs to $\mathcal{H}$ for every $n \geq 1$, by assumption.
Without loss of generality, we can suppose that the $\prob^0$-null set $N$ appearing in the definition of $\Lambda^n$ and $\Lambda$ is the same for every $n \geq 1$.
Notice that, for every $\omega_0 \in \Omega^0 \setminus N$, the sections $(\lambda^n_t(\omega_0,\cdot)_{t \in [0,T]}) \uparrow (\lambda_t(\omega_0,\cdot)_{t \in [0,T]})$ for every $(t,\omega_*) \in [0,T] \times \Omega^*$.
Therefore, by monotone convergence, it holds 
\begin{equation}\label{esempi:approssimazione_raccomdandazione}
\begin{aligned}
    \norm{\Lambda^n(\omega_0)-\Lambda(\omega_0)}^2_{L^2} & = \norm{(\lambda^n_t(\omega_0,\cdot)_{t \in [0,T]}) -(\lambda_t(\omega_0,\cdot)_{t \in [0,T]})}^2_{L^2} \\
    & =\E^{\prob^*}\quadre{\int_0^T \abs{\lambda^n(t,\omega_0,\omega_*)-\lambda(t,\omega_0,\omega_*)}^2 dt}\to 0
\end{aligned}
\end{equation}
for every $\omega_0 \in \Omega^0\setminus N$, i.e. $\Lambda=\lim_{n\to\infty}\Lambda^n$ $\prob^0$-a.s., which implies that $\Lambda$ is $\F^{0-}\setminus\boreliani{L^2}$ measurable, since the probability space is complete and $L^2([0,T]\times\Omega^*,\mathcal{P}^*,Leb_{[0,T]}\otimes\prob^*)$ is a complete norm space.
Finally, $\Lambda$ is admissible, since the process $\lambda$ obviously satisfies \eqref{mf:uguaglianza_ammissibilita}, choosing the same $\prob^0$-null set $N$ used in the definition $\Lambda$.

\medskip
To see that $\mathcal{E} \subseteq \mathcal{H}$, suppose first that $\lambda$ is of the form 
\begin{equation*}
    \lambda_t=\sum_{i=1}^n \1_{A_i}(\omega_0)\1_{B_i}(\omega_*)\1_{[t_i,t_{i+1})}(t),
\end{equation*}
where $n \geq 1$, $t_i \in [0,T]$, $t_i<t_{i+1}$ for every $i=1,\dots,N$, $A_i\in\F^{0-}$ and $B_i\in\F^*_{t_i}$.
We can regard each variable $\1_{B_i}(\omega_*)\1_{[t_i,t_{i+1})}(t)$ as a bounded progressively measurable process $\alpha^i$.
Therefore, \eqref{appendix:raccomandazione_indotta} takes the following form:
\begin{equation*}
    \Lambda(\omega_0)=\begin{cases}
    \alpha^i \; & \omega_0 \in  A_i, \qquad i=1,\dots,N, \\
    0 \; & \omega_0 \in \tonde{\cup_{i=1}^n A_i}^c
    \end{cases}
\end{equation*}
which shows that $\Lambda$ is $\F^{0-}\setminus\boreliani{L^2}$-measurable.
By Dynkin Lemma, conclusion holds true for progressively measurable simple processes of the form
\begin{equation}\label{esempi:processo_intermedio}
    \lambda_t=\sum_{i=1}^n \1_{C_i}(\omega_0,\omega_*)\1_{[t_i,t_{i+1})}(t),
\end{equation}
where $n \geq 1$, $t_i \in [0,T]$, $t_i<t_{i+1}$ for every $i\in\insieme{1,\dots,n}$, $C_i\in\F^{0-}\otimes\F^*_{t_i}$.
Finally, let $\lambda$ be of the form \eqref{esempi:processi_semplici}.
Thanks to the boundedness assumption on $\zeta^i$, we can find a sequence of simple processes $(\lambda^n)_{n\geq 1}$ of the form \eqref{esempi:processo_intermedio} so that $\abs{\lambda^n}\leq \abs{\lambda}$ and $\lambda^{n}_t(\omega_0,\omega_*)\to\lambda_t(\omega_0,\omega_*)$ pointwise for every $(t,\omega_0,\omega_*)$.
Let $\Lambda^n$ and $\Lambda$ be defined according to \eqref{appendix:raccomandazione_indotta} starting by the processes $\lambda^n$.
Due to point 2.b) above, conclusion holds true for each $\Lambda^n$.
Using dominated convergence, we can prove that \eqref{esempi:approssimazione_raccomdandazione} holds for $\prob^0$-a.e. $\omega_0 \in \Omega^0$, so that $\Lambda$ is the a.s. pointwise limit  of $\Lambda^n$, which implies that $\Lambda$ is $\F^{0-}\setminus \boreliani{L^2}$ measurable. 
\end{proof}

\begin{prop}\label{esempi:unicita_strategia_associata}
\begin{enumerate}[label=\roman*), wide]
Let $(\Omega^0,\F^{0-},\prob^0)$ be a complete probability space.
\item \label{esempi:unicita_strategia_associata:unicita_associata}  Let $\Lambda:(\Omega^0,\F^{0-},\prob^0)\to(\A,\boreliani{\A})$ be an admissible recommendation.
Let $\lambda^1$ and $\lambda^2$ be two $\mathbb{F}$-progressively measurable processes with values in $A$ so that \eqref{mf:uguaglianza_ammissibilita} holds.
Then $\lambda^1=\lambda^2$ $Leb_{[0,T]}\otimes\prob$-almost surely.
\item \label{esempi:unicita_strategia_associata:unicita_raccomandazione} Let $\Lambda,\Gamma:(\Omega^0,\F^{0-},\prob^0)\to(\A,\boreliani{\A})$ be admissible recommendations; let $\lambda,\gamma$ be the strategies associated to $\Lambda,\Gamma$, according to \eqref{mf:uguaglianza_ammissibilita}.
Suppose that $\lambda=\gamma$ $Leb_{[0,T]}\otimes \prob$-almost surely.
Then, $\Lambda=\Gamma$ $\prob^0$-a.s.
\end{enumerate}
\end{prop}
\begin{proof}
As for point \ref{esempi:unicita_strategia_associata:unicita_associata}, let $N^i$, $i=1,2$, be two $\prob^0$-null sets so that, for every $\omega_0 \in \Omega_0\setminus N^i$ the sections $(\lambda^i_t(\omega_0,\cdot))_{t \in [0,T]}$ are $\mathbb{F}^*$-progressively measurable processes and equation \eqref{mf:uguaglianza_ammissibilita} holds true.
Without loss of generality, we can assume that $N^1=N^2=N$.
Since for every $\omega_0 \in \Omega^0 \setminus N$ it holds $\norm{\Lambda - (\lambda^i_t(\omega_0,\cdot))_{t \in [0,T]}}_{L^2}=0$, $i=1,2$, we deduce that
\begin{equation*}
    \norm{(\lambda^1_t(\omega_0,\cdot))_{t \in [0,T]} - (\lambda^2_t(\omega_0,\cdot))_{t \in [0,T]}}^2_{L^2} = 0
\end{equation*}
for every $\omega_0 \in \Omega^0 \setminus N$.
Therefore, by taking the integral with respect to $\prob^0$, we obtain
\begin{equation*}
\begin{aligned}
    0 & = \E^{\prob^0} \quadre{ \norm{(\lambda^1_t(\omega_0,\cdot))_{t \in [0,T]} - (\lambda^2_t(\omega_0,\cdot))_{t \in [0,T]}}^2_{L^2} } \\
    & = \E^{\prob^0} \quadre{ \E^{\prob^*} \quadre{ \int_0^T \vert \lambda^1_s(\omega_0,\omega_*) - \lambda^2_s(\omega_0,\omega_*) \vert^2 ds  } } = \E\quadre{ \int_0^T \vert \lambda^1_s(\omega_0,\omega_*) - \lambda^2_s(\omega_0,\omega_*) \vert^2 ds  }
\end{aligned}
\end{equation*}
by Fubini's theorem.
This is enough to conclude that $\lambda^1=\lambda^2$ $Leb_{[0,T]}\otimes\prob$-a.s.

\medskip
As for point \ref{esempi:unicita_strategia_associata:unicita_raccomandazione}, by the same line of reasoning, if $\lambda=\gamma$ $Leb_{[0,T]}\otimes\prob$-a.s., the the sections $(\lambda_t(\omega_0,\cdot))_{t \in [0,T]}$ and $(\gamma_t(\omega_0,\cdot))_{t \in [0,T]}$ are $Leb_{[0,T]}\otimes\prob^*$-almost everywhere equal, which implies that
\begin{equation*}
\begin{aligned}
    \norm{\Lambda(\omega_0)-\Gamma(\omega_0)}^2_{L^2} & = \norm{(\lambda_t(\omega_0,\cdot))_{t \in [0,T]} - (\gamma_t(\omega_0,\cdot))_{t \in [0,T]}}^2_{L^2} \\
    & = \E^{\prob^*}\quadre{\int_0^T \vert \lambda_t(\omega_0,\omega_*) - \gamma_t(\omega_0,\omega_*) \vert^2 dt } = 0
\end{aligned}
\end{equation*}
$\prob^0$-a.s., so that $\Lambda=\Gamma$ $\prob^0$-a.s..
\end{proof}


\section{Propagation of chaos}



Here, we prove the propagation of chaos type result which is needed in the proof of Theorem \ref{thm_approssimazione}.
The probability spaces and the random variables we use here are defined in Section \ref{sezione_approssimazione:sezione_costruzione_raccomandazioni}.

We work on the product probability space
\begin{equation*}
    \tonde{\Omega,\F,\prob}= \tonde{\overline{\Omega},\overline{\F},\overline{\prob}} \otimes \tonde{\Omega^1, \F^1, \prob^1},
\end{equation*}
with $(\overline{\Omega},\overline{\F},\overline{\prob})$ defined by \eqref{approximation:spazio_raccomandazioni} and \eqref{approximation:legge_spazio_raccomandazioni} and $(\Omega^1, \F^1, \prob^1)$ by \eqref{canonical_setup} or, equivalently, by \eqref{approximation:spazio_rumori}.
Consider the random measure flow $\mu$ defined by \eqref{approximation:approx_misura} and the recommendations $(\Lambda^i)_{i \geq 1}$ defined by \eqref{approximation:approx_raccomandazione}, which we recall are conditionally i.i.d. given $\mu$ under $\overline{\prob}$.
We endow such a probability space with the filtration $\mathbb{F}$ given by the $\prob$-augmentation of the filtration generated by $\overline{\F}$, the initial data $(\xi^i)_{i\geq 1}$ and the Brownian motions $(W^i)_{i\geq 1}$.
We observe that for every $N\geq 2$, each $\beta \in \A_N$ is also $\mathbb{F}$-progressively measurable and, for every $i \geq 1$, each strategy $\lambda^i$ associated to the admissible recommendation $\Lambda^i$ is $\mathbb{F}$-progressively measurable as well.


\medskip
Fix $N\geq 2$, $\beta \in \A_N$ and $1 \leq i \leq N$.
Let $X=X [\Lambda^{N,-i},\beta]=\tonde{X^j[\Lambda^{N,-i},\beta]}_{j = 1}^N$ be the solution of
\begin{equation*}
    \begin{cases}
    dX^j_t=b(t,X^j_t,\mu^N_t,\lambda^{j}_t)dt + dW^j_t, \quad X^j_0=\xi^j, \quad j \neq i, \\
    dX^i_t=b(t,X^i_t,\mu^N_t,\beta_t)dt + dW^i_t, \quad X^i_0=\xi^i.
    \end{cases}
\end{equation*}
The process $X[\Lambda^{N,-i},\beta]$ is the state process of the $N$-players when every player $j \neq i$ follows the recommendation $\Lambda^{i}$ and player $i$ deviates by picking the strategy $\beta$, where $\mu^N_t$ denotes the empirical measure of the $N$-players' states at time $t$ defined in \eqref{misura_empirica}.
Let us introduce also the empirical measure of the processes $X=X[\Lambda^{N,-i},\beta]$:
\begin{equation}
    \mu^N[\Lambda^{N,-i},\beta]=\frac{1}{N}\sum_{j=1}^N\delta_{X^j[\Lambda^{N,-i},\beta]} \in \contpdue.
\end{equation}
Let us denote $X=X[\Lambda]=X[\Lambda^{N,-i},\Lambda^{i}]$ the state process of the $N$ players when every player $i=1,\dots,N$ follows the recommendation $\Lambda^{i}$.
Then, let us consider the following auxiliary processes: let $\tonde{Z^j[\Lambda^{N,-i},\beta]}_{j = 1}^N$ be the solution of
\begin{equation*}
    \begin{cases}
    dZ^j_t=b(t,Z^j_t,\mu_t,\lambda^{j}_t)dt + dW^j_t, \quad Z^j_0=\xi^j, \quad j \neq i, \\
    dZ^i_t=b(t,Z^i_t,\mu_t,\beta_t)dt + dW^i_t, \quad Z^i_0=\xi^i
    \end{cases}
\end{equation*}
and $\nu^N[\Lambda^{-i},\beta]$ be the empirical measure of the processes $Z[\Lambda^{-i},\beta]$:
\begin{equation*}
    \nu^N[\Lambda^{-i},\beta]=\frac{1}{N}\sum_{j=1}^N\delta_{Z^j[\Lambda^{N,-i},\beta]} \in \contpdue.
\end{equation*}

\begin{lemma}\label{lemma_poc}
Let $\beta$ be either an open-loop strategy in $\A_{K}$ for some $K \geq 2$, or be equal to $\lambda^{i}$, the strategy associated to the admissible recommendation $\Lambda^i$ to player $i$.
It holds:
\begin{align}
    & \sup_{t \in [0,T]}\attesa{\pwassmetric{2}{\R^d}{2}\tonde{\mu^N_t[\Lambda^{-i},\beta],\mu_t}} \overset{N \to \infty}{\longrightarrow} 0,  \label{lemma_poc:conv_wasserstein} \\
    & \max_{1 \leq j \leq N} \attesa{ \norm{X^{j,N}[\Lambda^{-i},\beta]-Z^j[\Lambda^{-i},\beta]}^2_{\contrd}} \overset{N \to \infty}{\longrightarrow} 0, \label{lemma_poc:conv_norma} \\
    & \sup_{n \geq 2} \max_{1 \leq j \leq N}\attesa{\norm{X^{j,N}[\Lambda^{-i},\beta]}^2_{\contrd} + \norm{Z^j[\Lambda^{-i},\beta]}^2_{\contrd}} < \infty. \label{lemma_poc:finitezza_momenti}
\end{align}
\end{lemma}
\begin{proof}
Because of the symmetry properties of the systems of SDEs, we can suppose $i=1$.
Throughout the proof, to make notation as simple as possible, we omit the dependence upon $[\Lambda^{-1},\beta]$.
For the same reason, define, for each $j \geq 1$, the following process $\gamma^j$:
\begin{equation*}
    \gamma^j_t=\begin{cases}
    \beta_t \qquad j=1, \\
    \lambda^j_t \qquad j \geq 2.
    \end{cases}
\end{equation*}
Obviously, in the case that $\beta$ is $\lambda^1$, we have $\gamma^j \equiv \lambda^j$ for every $j$.
Moreover, let us introduce the following auxiliary processes: let $(Y^j)_{j \geq 1}$ be the solution of
\begin{equation*}
    dY^j_t=b(t,Y^j_t,\mu_t,\lambda^j_t)dt + dW^j_t, \quad Y^j_0=\xi^j.
\end{equation*}
Let $\eta^N$ be the empirical measure of the processes $Y^j$:
\begin{equation*}
    \eta^N=\frac{1}{N}\sum_{j=1}^N\delta_{Y^j} \in \contpdue.
\end{equation*}
Denote by $X^*$ the state process resulting from the coarse correlated solution of the MFG, i.e.
\begin{equation*}
    dX^*_t=b\tonde{t,X^*_t,\mu^*_t,\lambda^*_t}dt + dW^*_t, \quad X^*_0=\xi^*.
\end{equation*}
Since, for every $j \geq 1$, $(\xi^j,W^j,\mu,\lambda^j)$ are distributed as $(\xi^*,W^*,\mu^*,\lambda^*)$, by Theorem \ref{teorema_di_unicita_legge} the processes $(Y^j)_{j \geq 1}$ are identically distributed copies of $X^*$; moreover, the joint distribution of $(Y^j,\mu)$ under $\prob$ is the same of $(X^*,\mu^*)$ under $\prob^*$, which, by marginalizing at every time $t \in [0,T]$, implies that $Y^j$ satisfies the consistency condition \eqref{def_mean_field_sol:cons} as well.

\medskip
For every fixed $t \in [0,T]$, by the triangular inequality, it holds
\begin{equation}\label{lemma_poc:intermedio1}
\begin{aligned}
    \E\quadre{\pwassmetric{2}{\R^d}{2}(\mu^N_t,\mu_t)} & \leq C \E \quadre{ \pwassmetric{2}{\R^d}{2}\tonde{\mu^N_t,\nu^N_t} + \pwassmetric{2}{\R^d}{2}\tonde{\nu^N_t,\eta^N_t} + \pwassmetric{2}{\R^d}{2}\tonde{\eta^N_t,\mu_t}}.
\end{aligned}
\end{equation}
We start from the third term in \eqref{lemma_poc:intermedio1}: let $\prob^m$ be a version of the regular conditional probability of $\prob$ given $\mu=m$, and denote by $\E^m[\cdot]$ the expectation with respect to the measure $\prob^m$.
By construction, the strategies $(\lambda^j)_{j\geq 1}$ associated to the admissible recommendations $(\Lambda^j)_{j \geq 1}$ are i.i.d. under $\prob^m$, for $\rho$-a.e. $m \in \contpdue$.
Since $\mu$ is independent of $(W^j)_{j \geq 1}$ and $(\xi^j)_{j \geq 1}$ under $\prob$, the processes $(Y^j)_{j \geq 1}$ are independent under $\prob^m$.
Moreover, since $\mu_t(\cdot)=\prob( Y^j_t \in \cdot \; \vert \; \mu )$ $\prob$-a.s. for every $t \in [0,T]$ and $\mu_t=m_t$ $\prob^m$-a.s. for $\rho$-a.e. $m \in \contpdue$, we have
\begin{equation}\label{lemma_poc:consistency_disintegrate}
    m_t=\prob^m \circ (Y^j_t)^{-1}, \quad \rho\text{-a.e.}, \; \forall t \in [0,T],
\end{equation}
for every $j \geq 1$.
We can conclude that the processes $(Y^j_t)_{j \geq 1}$ are independent and identically distributed square integrable random variables with law $m_t$ under $\prob^m$, for every $t$ and for $\rho$-a.e. $m$.
Therefore, as ensured, e.g., by \cite[Volume I, (5.19)]{librone}, it holds
\begin{equation*}
    \lim_{N \to \infty} \E^m\quadre{\pwassmetric{2}{\R^d}{2}\tonde{\eta^N_t,\mu_t}} = 0, \quad \rho\text{-a.e.}, \; \forall t \in [0,T].
\end{equation*}
We observe that there exists a function $g: \contpdue \to \R$ so that $g \in L^1(\rho)$ and $\E^m[\pwassmetric{2}{\R^d}{2}(\mu^N_t,\mu_t) ] \leq g (m)$, $\rho$-a.e., for every $t$: indeed, since, under $\prob^m$, $Y^j_t$ are i.i.d with law $m_t$ and $\mu_t=m_t$ a.s., we have 
\begin{equation*}
\begin{aligned}
    \E^m & \quadre{\pwassmetric{2}{\R^d}{2}\tonde{\eta^N_t,\mu_t}} \leq 2 \E^m\quadre{\pwassmetric{2}{\R^d}{2}\tonde{\eta^N_t,\delta_0} + \pwassmetric{2}{\R^d}{2}\tonde{\delta_0,\mu_t}} \\
    & \leq 2 \tonde{\frac{1}{N}\sum_{k=1}^N\E^m\quadre{\abs{Y^k_t}^2} + \int_{\R^d}\abs{y}^2m_t(dy)}  \leq 2 \tonde{\frac{1}{N}\sum_{k=1}^N\E^m\quadre{\abs{Y^1_t}^2} + \E^m\quadre{\abs{Y^1_t}^2}} \\
    & \leq 4 \E^m\quadre{\abs{Y^1_t}^2} \leq 4 \E^m\quadre{\norm{Y^1}^2_{\contrd}}.
\end{aligned}
\end{equation*}
The function $g(m)=\E^m\quadre{\lVert Y^1 \rVert_{\contrd}}$ belongs to $L^1(\rho)$, since
\begin{equation}\label{lemma_poc:bound_uniforme}
    \int_{\contpdue} g(m)\rho(dm) = \E\quadre{ \E\quadre{\norm{Y^1}^2_{\contrd}\; \vert \; \mu}}= \E\quadre{\norm{Y^1}^2_{\contrd}} < \infty.
\end{equation}
Therefore, by dominated convergence theorem, we have 
\begin{equation}\label{lemma_poc:convergenza_wass_puntuale}
    \lim_{N \to \infty} \E\quadre{\pwassmetric{2}{\R^d}{2}\tonde{\eta^N_t,\mu_t}} = 0
\end{equation}
for every $t \in [0,T]$.
The convergence in \eqref{lemma_poc:convergenza_wass_puntuale} is actually uniform in time.
Indeed, fix $t,s \in [0,T]$: then
\begin{equation*}
\begin{aligned}
    \E & \quadre{\pwassmetric{2}{\R^d}{2}\tonde{\eta^N_t,\mu_t} - \pwassmetric{2}{\R^d}{2}\tonde{\eta^N_s,\mu_s}} \\
    & = \E \quadre{\tonde{\pwassmetric{2}{\R^d}{}\tonde{\eta^N_t,\mu_t} - \pwassmetric{2}{\R^d}{}\tonde{\eta^N_s,\mu_s}}\tonde{\pwassmetric{2}{\R^d}{}\tonde{\eta^N_t,\mu_t} + \pwassmetric{2}{\R^d}{}\tonde{\eta^N_s,\mu_s}}} \\
    & \leq C \E\quadre{\norm{Y^1}^2_{\contrd}}^\frac{1}{2} \E \quadre{\tonde{\pwassmetric{2}{\R^d}{}\tonde{\eta^N_t,\mu_t} - \pwassmetric{2}{\R^d}{}\tonde{\eta^N_s,\mu_s}}^2}^{\frac{1}{2}},
\end{aligned}
\end{equation*}
where we used Cauchy-Schwartz inequality together with the uniform in time bound given by \eqref{lemma_poc:bound_uniforme}.
By triangulating with $\eta^N_s$ and $\mu_s$, we get
\begin{equation*}
\begin{aligned}
    \E & \quadre{\tonde{\pwassmetric{2}{\R^d}{}\tonde{\eta^N_t,\mu_t} - \pwassmetric{2}{\R^d}{}\tonde{\eta^N_s,\mu_s}}^2} \leq C \tonde{\E \quadre{\pwassmetric{2}{\R^d}{2}\tonde{\eta^N_t,\eta^N_s}} + \E\quadre{\pwassmetric{2}{\R^d}{2}\tonde{\mu_t,\mu_s}}} \\
    & \leq C \tonde{\E\quadre{\frac{1}{N}\sum_{k=1}^N \abs{Z^k_t - Z^k_s}^2} + \E \quadre{ \E\quadre{\pwassmetric{2}{\R^d}{2}\tonde{\mu_t,\mu_s}\; \vert \; \mu }} } \\
    & \leq C \tonde{\E\quadre{\frac{1}{N}\sum_{k=1}^N \abs{Z^k_t - Z^k_s}^2} + \E\quadre{\abs{Y^1_t - Y^1_s}^2} },
\end{aligned}
\end{equation*}
where in the last inequality we used \eqref{lemma_poc:consistency_disintegrate} and tower property.
By using Lipschitz continuity of $b$, the triangular inequality and $\E[\lVert Y^1 \rVert_{\contrd}]<\infty$, it is straightforward to see that $\E [ \lVert Z^k \rVert_{\contrd} ] \leq C$ for every $k \geq 1$, for some positive constant $C$ independent of $k$.
By the same arguments, we have 
\begin{equation*}
\begin{aligned}
    \E \quadre{\abs{Z^k_t - Z^k_s}^2 } & \leq C \E\quadre{\int_s^t \abs{ b(u,Z^k_u,\mu_u,\gamma^k_u) }^2 du } \\
    & \leq C \E\quadre{\int_s^t \tonde{ 1 + \abs{Z^k_u}^2 + \int_{\R^d}\abs{y}^2\mu_u(dy) + \abs{\gamma^k_u}^2} du }\\
    & \leq C \E\quadre{\int_s^t \tonde{ 1 + \norm{Z^k}^2_{\contrd} + \norm{Y^1}^2_{\contrd} + \abs{\gamma^k_u}^2} du } \leq C\abs{t-s},
\end{aligned}
\end{equation*}
where the constant $C$ depends upon $T$, $b$, $\E[\lVert Y^1 \rVert ^2 _{\contrd}] < \infty$ and $\text{diam}(A)$, which is a finite quantity since the set $A$ of actions is compact by Assumption \ref{standing_assumptions}.
Analogously holds for $\E[\vert Y^1_t - Y^1_s \vert ]$, which implies that 
\begin{equation}\label{lemma_poc:convergenza_uniforme}
\begin{aligned}
    \abs{\E \quadre{\pwassmetric{2}{\R^d}{2}\tonde{\eta^N_t,\mu_t} - \pwassmetric{2}{\R^d}{2}\tonde{\eta^N_s,\mu_s}}} \leq C \abs{t-s}^\frac{1}{2}.
\end{aligned}
\end{equation}
This is enough to conclude, by Arzel\`a-Ascoli theorem, that the convergence in \eqref{lemma_poc:convergenza_wass_puntuale} is uniform in time.


\medskip
Remind from Section \ref{sezione_notations_assumptions} that $\norm{x}_{t,\contrd}=\sup_{s \in [0,t]}\abs{x_s}$, $t \in [0,T]$.
To handle the second term in \eqref{lemma_poc:intermedio1}, we use the coupling of $\nu^N$ and $\eta^N$ given by $\frac{1}{N}\sum_{k=1}^N\delta_{(Z^k,Y^k)}$, together with the Lipschitz continuity of $b$:
\begin{equation*}
\begin{aligned}
    \E\quadre{\norm{Z^k-Y^k}^2_{t,\contrd}} & =\E\quadre{\sup_{0 \leq s \leq t} \tonde{\int_0^s \tonde{b\tonde{u,Z^k_u,\mu_u,\gamma^k_u} - b\tonde{u,Y^k_u,\mu_u,\lambda^k_u}}du}^2} \\
    & \leq C \left( \int_0^t \E\quadre{ \sup_{0 \leq u \leq s} \abs{Z^k_u - Y^k_u }^2}ds +  \int_0^t \E\quadre{ \abs{\lambda^k_t - \gamma^k_t}^2} ds \right).
\end{aligned}
\end{equation*}
By definition of $(\gamma^k)_{k \geq 1}$, we have
\begin{equation*}
    \int_0^T \E\quadre{\abs{\lambda^k_u-\gamma^k_u}^2}du=\begin{cases}
    & \int_0^T \E\quadre{\abs{\lambda^1_u-\beta_u}^2}du \qquad k = 1,\\
    & 0 \qquad \qquad \qquad \qquad  \qquad \quad k \geq 2.
    \end{cases}
\end{equation*}
Therefore, by Gronwall's lemma, we sum over $k=1,\dots,N$ to obtain the estimate
\begin{equation}\label{stime_poc_controlli}
\begin{aligned}
    \sup_{t \in [0,T]} & \E\quadre{\pwassmetric{2}{\R^d}{2}\tonde{\eta^N_t,\nu^N_t}} \leq \E\quadre{\pwassmetric{2}{\contrd}{2}\tonde{\eta^N,\nu^N}} \leq \frac{C}{N} \sum_{k=1}^N \E\quadre{\norm{Y^k - Z^k}_{\contrd}} \\
    & \leq \frac{C}{N} \sum_{j=1}^N \int_0^T \E\quadre{\abs{\lambda^j_u-\gamma^j_u}^2}du = \frac{C}{N} \int_0^T \E\quadre{\abs{\lambda^1_u-\beta_u}^2}du  \leq \frac{C}{N} \overset{N \to \infty}{\longrightarrow} 0,
\end{aligned}
\end{equation}
where the constant $C$ depends only upon $T$, $b$ and $\text{diam}(A)$.

\medskip
Finally, for the first term of \eqref{lemma_poc:intermedio1}, we use the coupling of $\mu^N_t$ and $\nu^N_t$ given by $\frac{1}{N}\sum_{k=1}^N\delta_{(X^{k,N}_t,Z^k_t)}$, together with the Lipschitz continuity of $b$:
\begin{equation*}
\begin{aligned}
    \E & \quadre{\norm{X^{k,N}-Z^k}^2_{t,\contrd}} \leq C \int_0^t \left( \E\quadre{ \sup_{0 \leq u \leq s} \abs{X^{k,N}_u - Z^k_u}^2} + \E\quadre{ \pwassmetric{2}{\R^d}{2}(\mu^N_s,\mu_s) } \right) ds \\
    & \leq C  \int_0^t \left( \E\quadre{ \sup_{0 \leq u \leq s} \abs{X^{k,N}_u - Z^k_u}^2} + \E\quadre{ \pwassmetric{2}{\R^d}{2}(\mu^N_s,\nu^N_s)} + \E\quadre{ \pwassmetric{2}{\R^d}{2}(\nu^N_s,\mu_s) } \right) ds \\
    & \leq C \int_0^t \left( \E\quadre{ \sup_{0 \leq u \leq s} \abs{X^{k,N}_u - Z^k_u}^2} + \frac{1}{N}\sum_{j=1}^N \E\quadre{ \abs{X^{j,N}_s - Z^j_s}^2 } + \E\quadre{ \pwassmetric{2}{\R^d}{2}(\nu^N_s,\mu_s) }\right) ds \\
    & \leq  C \left( \int_0^t \max_{k=1,\dots,n} \E\quadre{ \sup_{0 \leq u \leq s} \abs{X^{k,N}_u - Z^k_u}^2} ds + \sup_{s \in [0,t]}\E\quadre{ \pwassmetric{2}{\R^d}{2}(\nu^N_s,\mu_s) }\right).
\end{aligned}
\end{equation*}
By taking the maximum over $k=1,\dots,N$ on the left-hand side and applying Gronwall's lemma, we get
\begin{equation*}
    \max_{k=1,\dots,N} \E \quadre{\norm{X^{k,N}-Z^k}^2_{\contrd}} \leq C \sup_{t \in [0,T]}\E\quadre{ \pwassmetric{2}{\R^d}{2}(\nu^N_t,\mu_t) } \to 0,
\end{equation*}
by \eqref{lemma_poc:convergenza_wass_puntuale}, \eqref{lemma_poc:convergenza_uniforme} and \eqref{stime_poc_controlli}, which proves \eqref{lemma_poc:conv_norma}.
Coming back to \eqref{lemma_poc:conv_wasserstein}, we have
\begin{equation*}
\begin{aligned}
    \sup_{t \in [0,T]} \E\quadre{ \pwassmetric{2}{\R^d}{2}\tonde{\mu^N_t,\nu^N_t} } \leq \sup_{ t \in [0,T] } \frac{1}{N} \sum_{k=1}^N \E\quadre{\abs{X^{k,N}_t - Z^k_t }^2} \leq \max_{k=1,\dots,N}\E\quadre{ \norm{X^{k,N} - Z^k}_{\contrd}^2} \to 0.
\end{aligned}
\end{equation*}
This estimate together with estimates \eqref{lemma_poc:convergenza_wass_puntuale}, \eqref{lemma_poc:convergenza_uniforme} and \eqref{stime_poc_controlli} implies \eqref{lemma_poc:intermedio1} and therefore \eqref{lemma_poc:conv_wasserstein}.
Finally, \eqref{lemma_poc:finitezza_momenti} follows from the above calculations.
\end{proof}


\section{Auxiliary results for the existence of mean field CCE}\label{appendix_existence}



In this section, we state and prove some auxiliary results that were used in Section \ref{sezione_existence} to prove the existence of a mean field CCE.
In particular, Lemmata \ref{existence:lemma_kernels} and \ref{existence:lemma_decomposizione} provide the technical instruments we used in Proposition \ref{existence:prop_relazione_mfg} to show that, for every deviating strategy $\beta \in \A$ and random flow of measures $\mu$, we can represent the joint law of $\mu$, $\beta$ and deviating player's state process in terms of a strategy for player B in the zero-sum game \ref{existence:def_zerosum} and the the law of $\mu$.
Lemmata \ref{existence:lemma_mimicking} and \ref{existence:mimicking:lemma_strong_existence} were needed in the proof of Theorem \ref{existence:main_theorem} in order to define a mean field CCE starting from an optimal strategy for player A in the zero-sum game \ref{existence:def_zerosum}.


\medskip
Consider any tuple $\mathfrak{U}=((\Omega,\F,\mathbb{F},\prob),\xi,W,\mu,\mathfrak{r})$, composed of a filtered probability space satisfying usual assumptions, a $d$-dimensional $\mathbb{F}$-Brownian motion, an $\R^d$-valued $\F_0$-measurable random variable, an $\F_0$-measurable
random continuous flow of measures in $\mathcal{P}^2(\R^d)$ and an $\mathbb{F}$-progressively measurable $\mathcal{P}(A)$-valued process.
Assume that $\mu$, $W$ and $\xi$ are mutually independent.
Let us consider the following equations:
\begin{align}
    dX_t & =\int_A b(t,X_t,\mu_t,a)\mathfrak{r}_t(da)dt + dW_t, \quad X_0=\xi, \label{existence:eq_misura_aleatoria} \\
    dX^m_t & =\int_A b(t,X^m_t,m_t,a)\mathfrak{r}_t(da)dt + dW_t, \quad X_0=\xi, \label{existence:eq_misura_deterministica}
\end{align}
where $m$ is a point of $\contpdue$.
In order to stress the dependence upon the deterministic flow of measures $m$, we write $X^m$ for the solution of \eqref{existence:eq_misura_deterministica}.

By Assumptions \ref{standing_assumptions}, on any such tuple $\mathfrak{U}$ there exists a solution to equation \eqref{existence:eq_misura_aleatoria} and pathwise uniqueness holds.
If needed, we can suppose that the filtration $\mathbb{F}$ on $(\Omega,\F,\prob)$ is the $\prob$-augmentation of the filtration $\mathbb{F}^{\xi,W,\mu,\mathfrak{r}}$, given by
\begin{equation}\label{existence:filtrazione_forte_ctrl}
    \F_t^{\xi,W,\mu,\mathfrak{r}}=\sigma(\xi)\vee\sigma(\mu)\vee\sigma(W_s: s\leq t)\vee\sigma(\mathfrak{r}(C): C \in \boreliani{[0,t]\times A}).
\end{equation}
By Theorem \ref{teorema_di_unicita_legge}, uniqueness in law holds.
Analogous reasoning holds for equation \eqref{existence:eq_misura_deterministica} as well, for every $m \in \contpdue$.

\begin{lemma}\label{existence:lemma_kernels}
Let $\mathfrak{U}=((\Omega,\F,\mathbb{F},\prob),\xi,W,\mu,\mathfrak{r})$ be as above, let $\Theta \in \mathcal{P}(\R^d \times \contrd \times \mathcal{V})$ be the joint law of $\xi$, $W$ and $\mathfrak{r}$.
Let us define the map
\begin{equation}\label{existence:mappa_strat_min_continua}
\begin{aligned}
    \mathcal{I}_{\Theta}: \contpdue  & \longrightarrow \mathcal{P}(\R^d \times \contrd \times \contrd \times \mathcal{V}) \\
    m & \longmapsto \mathcal{I}_{\Theta}(m)=\prob\circ(\xi,W,X^m,\mathfrak{r})^{-1},
\end{aligned}
\end{equation}
where $X^m$ is the solution to equation \eqref{existence:eq_misura_deterministica}.
\begin{enumerate}[label=(\roman*)]
    \item \label{existence:lemma_kernels:mappa_continua}
    The map $\mathcal{I}_{\Theta}$ is continuous, in the sense that
    \begin{equation*}
        \sup_{t \in [0,T]}\pwassmetric{2}{\R^d}{}(m^n_t,m_t) \to 0 \text{ as } n \to \infty \; \Longrightarrow \; \mathcal{I}_{\Theta}(m^n) \overset{n \to \infty}{\longrightarrow} \mathcal{I}_{\Theta}(m) \text{ in } \pwassmetric{2}{\R^d \times \contrd \times  \mathcal{V} \times \contrd}{}.
    \end{equation*}
    
    \item \label{existence:lemma_kernels:kernel_ben_definito}
    The map $\mathcal{I}_{\Theta}$ induces a stochastic kernel $\Sigma$ from $\contpdue$ to $\contrd \times \mathcal{V}$, by setting
    \begin{equation*}
        \Sigma(B,m)=\prob((X^m,\mathfrak{r}) \in B)=\mathcal{I}_{\Theta}(m)(\R^d \times \contrd \times B) \quad \forall m \in \contpdue, \; B \in \boreliani{\contrd}\otimes\boreliani{\mathcal{V}}.
    \end{equation*}
    $\Sigma$ is a strategy for player B, as described in Definition \ref{existence:strategie_min}.
\end{enumerate}
\end{lemma}
\begin{proof}
Note that, by Theorem \ref{teorema_di_unicita_legge}, $\mathcal{I}_{\Theta}(m)$ is the unique weak solution of \eqref{existence:eq_misura_deterministica} when the joint law of $\xi$, $W$ and $\mathfrak{r}$ is given by $\Theta$ and $b$ is evaluated at $m \in \contpdue$.
Let $\Sigma \in \mathcal{Q}$, let $(m^n)_{n\geq 1} \subset \contpdue$ so that $m^n \to m$.
For every $n\geq 1$, denote by $X$ and $X^n$ the solution to equation \eqref{existence:eq_misura_deterministica} when $b$ is evaluated at $m$ and $m^n$, respectively.
For every $2 \leq p \leq \overline{p}$, by Lipschitz continuity of $b$, we have:
\begin{align}
    \E\quadre{\sup_{0 \leq s \leq T}\abs{X^n_s-X_s}^p}\leq C \sup_{t \in [0,T]}\pwassmetric{p}{\R^d}{2}(m^n_t,m_t), \label{existence:lemma_continuita:stime_norme} \\ 
    \sup_{n \geq 1}\attesa{\sup_{t \in [0,T]}\abs{X^n_t}^p}\leq C\tonde{1 + \sup_{n \geq 1} \sup_{t \in [0,T]} \tonde{\int_{\R^d}\abs{y}^2 m^n_t(dy)}^{\frac{p}{2}}}. \label{existence:lemma_continuita:bound_uniformi}
\end{align}
Therefore, for $p=2$, we have that $m^n \to m$ in $\contpdue$ implies that $\lVert  X^n-X \rVert_{\contrd}^2 \to 0$ in expectation, which in turns implies that $(\xi,W,\mathfrak{r},X^n) \longrightarrow (\xi,W,\mathfrak{r},X)$ in distribution.
In order to have convergence in 2-Wasserstein metrics, it is enough to check uniform integrability, according to  \eqref{wass:uniforme_integrabilita}.
Since $(\mathcal{I}_{\Theta}(m^n))_n$ have the same marginals on $\R^d\times\contrd\times\mathcal{V}$, we just need to check \eqref{wass:uniforme_integrabilita} for the laws of $(X^n)_n$: for every $n \geq 1$, $r > 0$, we have
\begin{equation*}
\begin{aligned}
    \E\quadre{\norm{X^n}_{\contrd}^2\1_{\insieme{\norm{X^n}^2_{\contrd}>r}}} & \leq  \tonde{\E\quadre{\norm{X^n}_{\contrd}^{4}}}^\frac{1}{2}\tonde{\E\quadre{\1_{\insieme{\norm{X^n}^2_{\contrd}>r}}}}^\frac{1}{2}  \\
    & \leq \tonde{\E\quadre{\norm{X^n}_{\contrd}^{4}}}^\frac{1}{2} \E\quadre{\norm{X^n}_{\contrd}^2}^\frac{1}{2}r^{-\frac{1}{2}} \leq C r^{-\frac{1}{2}}
\end{aligned}
\end{equation*}
by using Cauchy-Schwartz inequality, Markov inequality, \eqref{existence:lemma_continuita:stime_norme} and \eqref{existence:lemma_continuita:bound_uniformi}.
By taking the limit as $r \to \infty$, we get condition \eqref{wass:uniforme_integrabilita} satisfied and so point \ref{existence:lemma_kernels:mappa_continua} is proved.

\medskip
As for point \ref{existence:lemma_kernels:kernel_ben_definito}, let $\pi:\R^d\times\contrd\times\contrd\times\mathcal{V}\to\contrd\times\mathcal{V}$ be the projection on the last two components.
Note that $(\mathcal{I}_{\Theta}\circ \pi^{-1})(m)=\prob \circ (X^m,\mathfrak{r})^{-1}$, which shares the same continuity properties of the map $\mathcal{I}_{\Theta}$.
Therefore, in particular, it is Borel measurable, where $\mathcal{P}(\contrd \times \mathcal{V})$ is endowed with the usual Borel $\sigma$-algebra associated with the topology of weak convergence.
Then, the thesis follows from the fact that, for a Polish space $E$, the usual Borel $\sigma$-field on $\mathcal{P}(E)$ coincide with the $\sigma$-field generated by the maps $\mathcal{P}(E) \ni m \mapsto m(S)$, with $S \in \boreliani{E}$, (see, e.g., \cite[Corollary 7.29.1]{bertsekas_shreve}).
\end{proof}


\begin{lemma}\label{existence:lemma_decomposizione}
Let $\mathfrak{U}=((\Omega,\F,\mathbb{F},\prob),\xi,W,\mu,\mathfrak{r})$ be a tuple so that $(\xi,W,\mathfrak{r})$ and $\mu$ are independent.
Denote by $\rho \in \mathcal{P}(\contpdue)$ the law of $\mu$ under $\prob$.
Suppose without loss of generality that $\mathbb{F}$ is the $\prob$-augmentation of the filtration $(\F^{\xi,W,\mu,\mathfrak{r}}_t)_t$ defined by \eqref{existence:filtrazione_forte_ctrl}.
Let $X$ be the unique solution of \eqref{existence:eq_misura_aleatoria} on the tuple $\mathfrak{U}$.
Then, the following decomposition of measure holds
\begin{equation*}
    \prob((X,\mathfrak{r},\mu) \in B \times S)= \int_S \Sigma(B,m)\rho(dm), \quad \forall B \in \boreliani{\contrd \times \mathcal{V}}, \; S \in \boreliani{\contpdue}.
\end{equation*}
In particular, $\Sigma(B,m)=\prob((X,\mathfrak{r}) \in B \; \vert \; \mu=m)=\prob((X^m,\mathfrak{r}) \in B )$ for every $B \in \boreliani{\contrd \times \mathcal{V}}$, $\rho$-a.e. $m \in \contpdue$.
\end{lemma}
\begin{proof}
Let $\prob(\cdot \; \vert \; \mu)$ denote the regular conditional probability of $\prob$ given $\mu$.
Set $\prob^m(\cdot)=\prob(\cdot \; \vert \; \mu=m)$.
Since $(\xi,W,\mathfrak{r})$ and $\mu$ are independent by assumption, we have that $\prob^m\circ(\xi,W,\mathfrak{r})^{-1}=\prob\circ(\xi,W,\mathfrak{r})^{-1}$ for $\rho$-a.e. $m \in \contpdue$.
Therefore, it is enough to prove that $X$ is a solution to \eqref{existence:eq_misura_deterministica} on the tuple $\mathfrak{U}=((\Omega,\F,\mathbb{F},\prob^m),\xi,W,\mathfrak{r})$ for $\rho$-a.e. $m \in \contpdue$.
Then, since uniqueness in law holds for \eqref{existence:eq_misura_deterministica}, we deduce that $\prob^m\circ(\xi,W,\mathfrak{r},X)^{-1}=\mathcal{I}_{\Theta}(m)$ $\rho$-a.s.
Observe that, since the joint law of  $(\xi,W,\mathfrak{r})$ is the same under $\prob$ and $\prob^m$ for $\rho$-a.e $m$, $W$ is a natural Brownian motion under $\prob^m$ as well.
By definition of the filtration $\mathbb{F}$, it can be easily verified that
\begin{equation*}
    \E^\prob[\1_A(W_t-W_s)g \; \vert \;  \mu]=0 \quad \prob\text{-a.s}
\end{equation*}
for every $0 \leq s < t \leq T$, $A \in \boreliani{\R^d}$, $g$ bounded and $\F_s$-measurable.
This implies that
\begin{equation*}
    E^{\prob^m}[\1_A(W_t-W_s)g]=\E^{\prob}[\1_A(W_t-W_s)g\; \vert \;  \mu=m]=0
\end{equation*}
$\rho$-a.s., for every $g$ bounded and $\F_s$-measurable.
By working with a countable measure determining class of sets, which is possible since the $\sigma$-algebra $\F^{\xi,W,\mu,\mathfrak{r}}_t$ is countably generated for every $t \in [0,T]$, the equality holds for every $g$ bounded and $\F_s$-measurable, for $\rho$-a.e. $m \in \contpdue$, which in turn implies that $W$ remains an $\mathbb{F}$-Brownian motion under $\prob^m$ as well.
Under $\prob^m$ one has 
\begin{equation*}
    \prob^{m}\tonde{\int_A b(t,x,\mu_t,a)\mathfrak{r}_t(da)=\int_A b(t,x,m_t,a)\mathfrak{r}_t(da) \quad \forall x \in \R^d}=1 \quad Leb_{[0,T]}\text{-a.e. } t \in [0,T]
\end{equation*}
and therefore $X$ solves \eqref{existence:eq_misura_deterministica} for $b$ evaluated at $m \in \contpdue$.
The thesis follows from marginalizing as in the proof of point \ref{existence:lemma_kernels:kernel_ben_definito} in \ref{existence:lemma_kernels}.
\end{proof}

\medskip
We turn our attention to the proof of Lemma \ref{existence:lemma_mimicking}, which for convenience we restate below.

\begin{replemma}{existence:lemma_mimicking}
Let $\Gamma \in \mathcal{K}$. There exists a measure $\hat{\Gamma} \in \mathcal{K}$ so that the following holds:
\begin{itemize}
    \item The marginal distributions of $\Gamma$ and $\hat{\Gamma}$ on $\contpdue$ are the same: $\Gamma(\contrd \times \mathcal{V} \times \cdot)=\hat{\Gamma}(\contrd \times \mathcal{V} \times \cdot)$.
    
    \item Let $(X,\mathfrak{r},\mu)$ be such that $\hat{\Gamma}=\prob\circ(X,\mathfrak{r},\mu)^{-1}$.
    Then $\mathfrak{r}$ is of the form $\mathfrak{r}_t=\hat{q}_t(X_t,\mu)$, where $\hat{q}:[0,T]\times\R^d\times\contpdue \to \mathcal{P}(A)$ is a measurable function.
    \item For every $\Sigma \in \mathcal{Q}$, it holds
    \begin{equation*}
        \mathfrak{p}(\Gamma,\Sigma)=\mathfrak{p}(\hat{\Gamma},\Sigma).
    \end{equation*}
\end{itemize}
\end{replemma}


\begin{proof}
In the following, for a metric space $(E,d_E)$, $\phi:E \to \R$ continuous and bounded and $m \in \probmeasures{E}$, we set $\langle \phi, m \rangle = \int_E \phi(e) m(de)$.

Let $\mathfrak{U}=((\Omega,\F,\mathbb{F},\prob),\xi,W,\mu,\mathfrak{r})$ be as in Definition \ref{existence:strategie_max}, so that $\Gamma=\prob\circ(X,\mathfrak{r},\mu)^{-1}$.
As ensured by \cite[Lemma C.2]{lacker2020convergence}, we have that, by choosing $Y_t=(X_t,\mu)$ taking values in $\R^d \times \contpdue$ as conditioning process, there exists a jointly measurable function $\hat{q}:[0,T] \times \R^d \times \contpdue \to \mathcal{P}(A)$ so that, for every $\phi:[0,T] \times \R^d \times \contpdue \times A \to \R$ bounded and measurable it holds
\begin{equation}\label{existence:lemma_mimicking:markovian_ctrl}
    \int_{A}\phi(X_t,\mu,a)\hat{q}_t(X_t,\mu)(da)=\attesa{\int_{A}\phi(X_t,\mu,a)\mathfrak{r}_t(da) \big\vert X_t,\mu} \quad \prob\text{-a.s.}, \; \text{a.e. } t \in [0,T],
\end{equation}
which we abbreviate as
\begin{equation*}
    \hat{q}_t(X_t,\mu)(da)=\attesa{\mathfrak{r}_t(da) \big\vert X_t,\mu} \quad \prob\text{-a.s.}, \; \text{a.e. } t \in [0,T].
\end{equation*}
Next, we manipulate the term of the functional $\mathfrak{p}$ in \eqref{existence:payoff_functional} which depends only upon $\Gamma$:
\begin{equation}\label{existence:lemma_mimicking:uguaglianze1}
\begin{aligned}
    \int \: &  \mathfrak{F} (y,q,m) \Gamma(dy,dm,dq) = \attesa{\int_0^T\int_A f(t,X_t,\mu_t,a)\mathfrak{r}_t(da)dt + g(X_T,\mu_T)} \\
    = & \int_0^T \attesa{\attesa{ \int_A f(t,X_t,\mu_t,a)\mathfrak{r}_t(da) \Big \vert X_t,\mu }}dt + \attesa{g(X_T,\mu_T)} \\
    = & \int_0^T\attesa{ \int_A f(t,X_t,\mu_t,a)\hat{q}_t(X_t,\mu)(da) }dt + \attesa{g(X_T,\mu_T)} \\
    = & \int_0^T\attesa{\attesa{ \int_A f(t,X_t,\mu_t,a)\hat{q}_t(X_t,\mu)(da) \Big \vert \mu }}dt + \attesa{\attesa{g(X_T,\mu_T) \Big \vert \mu }} \\
    = & \int_0^T\attesa{ \left \langle \int_A f(t,\cdot,\mu_t,a)\hat{q}_t(\cdot,\mu)(da), \mu_t \right \rangle } dt + \attesa{\left \langle g(\cdot,\mu_T),  \mu_T \right \rangle }.
\end{aligned}
\end{equation}
Second equality holds by Fubini's theorem and tower property of conditional expectation,
third equality holds by definition of the control \eqref{existence:lemma_mimicking:markovian_ctrl}, third and fourth equalities hold by tower property again, and fifth equality holds since, by consistency condition, $\mu_t(\cdot)=\prob(X_t \in \cdot \; \vert \; \mu)$.

\medskip
We observe that, by choosing $\phi(t,x,m,a)=b(t,x,m_t,a)$ in \eqref{existence:lemma_mimicking:markovian_ctrl}, we have 
\begin{equation*}
    \int_A b(t,X_t,\mu_t,a)\hat{q}_t(X_t,\mu)(da)=\attesa{\int_A b(t,X_t,\mu_t,a)\mathfrak{r}_t(da) \big\vert X_t,\mu} \quad \prob\text{-a.s.}, \; \text{a.e. } t \in [0,T].
\end{equation*}
This is enough to apply \cite[Theorem~3.6]{brunick_shreve_mimicking}: indeed, in its terminology, we can take $\mathcal{E}=\R^d \times  \contpdue$, $\Phi: \mathcal{E} \times \contrd_0  \to \conttraj{\mathcal{E}}$ defined by $\Phi_t(x,m,y)=(x_t + y,m) \in \R^d \times \contpdue$, where $\contrd_0=\{ y \in \contrd: \; x_0=0\}$.
Set $Z_t=\Phi(X_t-X_0,X_0,\mu)=(X_t,\mu)$, 
where we note that the second component of $Z$ is constant in time as it is equal to the whole flow $\mu=(\mu_s)_{s \in [0,T]}$.
Then, such a result ensures that there exists a probability space $(\hat{\Omega},\hat{\F},\hat{\mathbb{F}},\hat{\prob})$, with $\hat{\Omega}$ Polish and $\hat{\F}$ its corresponding Borel $\sigma$-algebra, supporting an $\hat{\mathbb{F}}$-Brownian motion $\hat{W}$, a continuous $\mathcal{E}$-valued process $\hat{Z}$ so that there exists an $\hat{\mathbb{F}}$-adapted process $\hat{X}$ that satisfies
\begin{equation*}
    \hat{X}_t=\hat{X_0} + \int_0^T\int_A b(s,\hat{X}_s,\hat{\mu}_s,a)\hat{q}_s(\hat{X}_s,\hat{\mu}_s)(da)ds + \hat{W}_t, \quad \hat{Z}=\Phi(\hat{X}_t-\hat{X}_0,\hat{X}_0,\hat{\mu})
\end{equation*}
so that for every $t \in [0,T]$ it holds $\prob\circ Z_t^{-1}=\hat{\prob}\circ\hat{Z}_t^{-1}$.
This implies both that $\hat{\mu}$ and $\mu$ have the same law $\rho$ and that the consistency condition is satisfied, since $\hat{\prob}\circ(\hat{X}_t,\hat{\mu})^{-1}=\prob \circ (X_t,\mu)^{-1}=m_t(dx)\rho(dm)$.
Finally, since $Z$ is $\hat{\mathbb{F}}$-adapted, we deduce that $\hat{X}_0$ and $\hat{\mu}$ are $\F_0$-measurable and therefore $\hat{W}$, $\hat{X}_0$ and $\hat{\mu}$ are mutually independent.

Set $\hat{\Gamma}=\hat{\prob}\circ(\hat{X},\hat{\mathfrak{r}},\hat{\mu})^{-1}$.
Since the last term in the chain of equalities \eqref{existence:lemma_mimicking:uguaglianze1} depends only upon $\mu$ and $\mu$ and $\hat{\mu}$ share the same law, we can exploit the fact that $\hat{\mu}$ and $\hat{X}$ satisfy the consistency condition as well to get
\begin{equation*}
\begin{aligned}
     \: \int &  \mathfrak{F} (y,q,m) \Gamma(dy,dm,dq) =\int_0^T \E \quadre{  \left \langle \int_A f(t,\cdot,\mu_t,a)\hat{q}_t(\cdot,\mu)(da), \mu_t \right \rangle } dt + \E \quadre{ \left \langle g(\cdot,\mu_T),  \mu_T \right \rangle }\\
     = & \int_0^T \E^{\hat{\prob}} \quadre{  \left \langle \int_A f(t,\cdot,\hat{\mu}_t,a)\hat{q}_t(t,\cdot,\hat{\mu}_t,a)(da), \hat{\mu}_t \right \rangle} dt + \E^{\hat{\prob}} \quadre{ \left \langle g(\cdot,\hat{\mu}_T),  \hat{\mu}_T \right \rangle } \\
     = & \int_0^T \E^{\hat{\prob}} \quadre{ \E^{\hat{\prob}} \quadre{\int_A f(t,\hat{X}_t,\hat{\mu}_t,a)\hat{q}_t(\hat{X}_t,\hat{\mu})(da)} dt + \E^{\hat{\prob}} \quadre{ g(\hat{X}_T,\hat{\mu}_T) \Big \vert \hat{\mu}} } \\
     = & \int_0^T \E^{\hat{\prob}} \quadre{ \int_A f(t,\hat{X}_t,\hat{\mu}_t,a)\hat{q}_t(\hat{X}_t,\hat{\mu})(da) } dt + \E^{\hat{\prob}} \quadre{ g(\hat{X}_T,\hat{\mu}_T) } \\
     = & \int \mathfrak{F} (y,q,m) \hat{\Gamma}(dy,dm,dq).
\end{aligned}
\end{equation*}
Analogously, for every $\Sigma \in \mathcal{Q}$, we have
\begin{equation*}
    \int \mathfrak{F} (y,q,m)  \Sigma(dy,dq,m)\rho(dm) = \int \mathfrak{F} (y,q,m)  \Sigma(dy,dq,m)\hat{\rho}(dm),
\end{equation*}
which proves the desired statement about the payoff functional $\mathfrak{p}$.
\end{proof}

Finally, we show that it is always possible to find a strong solution to equation \eqref{existence:eq_processo_K} in the case of a feedback in state control process $\hat{q}_t(x,m)$, as given by Lemma \ref{existence:lemma_mimicking}:
\begin{lemma}[Strong solutions for feedback in state controls]\label{existence:mimicking:lemma_strong_existence}
Let $(\Omega,\F,\mathbb{F},\prob)$ be a filtered probability space satisfying the usual assumptions, with $\Omega$ Polish and $\F$ its Borel $\sigma$-algebra, supporting a $d$-dimensional $\mathbb{F}$-Brownian motion $W$, an $\F_0$-measurable $\R^d$-valued random $\xi$ with law $\nu$ and a $\F_0$-measurable random flow of measures $\mu$ in $\contpdue$ with law $\rho$.
Assume that $\xi$, $W$ and $\mu$ are mutually independent.
Let $\hat{q}:[0,T]\times \R^d \times \contpdue \to \mathcal{P}(A)$ be a measurable function, and suppose that there exists a solution of the SDE
\begin{equation}
    dX_t=\int_A b(t,X_t,\mu_t,a)\hat{q}_t(X_t,\mu)(da)dt + dW_t,
\end{equation}
so that it holds
\begin{equation*}
    \mu_t(\cdot)=\prob( X_t \in \cdot \; \vert \; \mu) \quad \prob\text{-a.s.}
\end{equation*}
for every $t \in [0,T]$.
Then, $X$ may be taken adapted to the $\prob$-augmentation of the filtration $\mathbb{F}^{\xi,\mu,W}=\sigma(\xi) \vee \sigma(\mu) \vee \mathbb{F}^W$.
In particular, there exists a progressively measurable function $\Phi:\contpdue \times \R^d \times \contrd \to \contrd$ so that $\Phi(\mu,\xi,W)=X$ $\prob$-a.s.
\end{lemma}
\begin{proof}
Set $B(t,x,m)=\int_A b(t,x,m_t,a)\hat{q}_t(x,m)(da)$.
$B$ is jointly measurable in $(t,x,m) \in [0,T] \times \R^d \times\contpdue$ with at most linear growth in $(x,m) \in  \R^d \times \contpdue$ for every $t \in [0,T]$.
The following hold:
\begin{enumerate}
    \item \label{mimicking:lemma_strong_existence:punto_eq_deterministica}
    For every $m \in \contpdue$, equation
    \begin{equation} \label{mimicking:lemma_strong_existence:eq_deterministica}
        dX^m_t=B(t,X^m_t,m)dt +dW_t, \quad X^m_0=\xi.
    \end{equation}
    admits a unique strong solution.
    Moreover, let $P^m=\prob \circ (X^m)^{-1}$.
    Then, the map $\contpdue \ni m \mapsto P^m \in \probmeasures{\contrd}$ is measurable.
    \item \label{mimicking:lemma_strong_existence:esistenza_sol_forte}
    There exist a continuous $\mathbb{F}$-adapted process $X$ solution to
    \begin{equation}\label{mimicking:lemma_strong_existence:eq_aleatoria}
        dX_t=B(t,X_t,\mu)dt + dW_t, \quad X_0=\xi.
    \end{equation}
    $X$ is adapted to the $\prob$-augmentation of the filtration $\mathbb{F}^{\xi,\mu,W}$.
    \item \label{mimicking:lemma_strong_existence:pathwise_uniqueness}
    Pathwise uniqueness holds, in the following sense: 
    suppose there exists a pair of continuous $\mathbb{F}$-adapted processes $(X^1,X^2)$ which satisfy equation \eqref{mimicking:lemma_strong_existence:eq_aleatoria} so that $(X^1_s,X^2_s)_{s \leq t}$ is conditionally independent of $\F^{\xi,\mu,W}_T$ given $\F^{\xi,\mu,W}_t$ for every $t \in [0,T]$.
    Then, $\prob(X^1_t=X^2_t, \; 0 \leq t \leq T)=1$.
    
    \item \label{mimicking:lemma_strong_existence:punto_legge_soluzione}
    The joint law of $X$ and $\mu$ is given by 
    \begin{equation*}
        \prob\circ(X,\mu)^{-1}=P^m(dx)\rho(dm).
    \end{equation*}
\end{enumerate}
This properties can be proven with the same methods of \cite[Appendix~A]{lacker2020convergence} and \cite[Appendix~A]{lacker_leflem2022}.
We just point out that the results therein do not hold automatically in our case, since $B$ is not progressively measurable in the measure flow $m$, in the sense of \cite{lacker2020convergence,lacker_leflem2022}.
Nevertheless, since we require $\mu$ to be $\F_0$-measurable, the same arguments lead to the results above.

\medskip
Let $X$ be as in the statement of the lemma.
We first show that the joint law of $X$ and $\mu$ is given by $P^m(dx)\rho(dm)$.
Let $\prob^m(\cdot)=\prob(\cdot \vert \mu=m)$ be a version of the regular conditional probability of $\prob$ given $\mu=m$.
Then, since $\xi$, $W$ and $\mu$ are mutually independent, 
$\prob^m\circ(\xi,W)^{-1}=\prob\circ(\xi,W)^{-1}$ for $\rho$-a.e. $m$, and, by exploiting the fact the $\F^{\xi,\mu,W,X}_t$ is countably generated for every $t$, $W$ is an $\mathbb{F}^{\xi,\mu,W,X}$-Brownian motion under $\prob^m$ as well.
Therefore, $X$ satisfies equation \eqref{mimicking:lemma_strong_existence:eq_deterministica} on $(\Omega,\F,\mathbb{F}^{\xi,\mu,W,X},\prob^m)$ for $\rho$-a.e. $m \in \contpdue$.
By point \ref{mimicking:lemma_strong_existence:punto_eq_deterministica}, $\prob^m \circ X^{-1}=P^m$ for $\rho$-a.e. $m \in \contpdue$, which implies that $\prob\circ(X,\mu)^{-1}=P^m(dx)\rho(dm)$.

\medskip
It can be shown by straightforward calculations that $(X_s)_{s\leq t}$ is conditionally independent of $\F^{\xi,\mu,W}_T$ given $\F^{\xi,\mu,W}_t$, for every $t \in [0,T]$.
Since pathwise uniqueness holds by point \ref{mimicking:lemma_strong_existence:pathwise_uniqueness}, this implies that $X$ is indistinguishable from an $\mathbb{F}^{\xi,\mu,W}$-adapted solution to equation \eqref{mimicking:lemma_strong_existence:eq_aleatoria}.
\end{proof}


\addcontentsline{toc}{section}{References}

\bibliographystyle{abbrv}
\bibliography{biblio}


\end{document}
