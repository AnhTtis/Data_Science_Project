


We state and prove a Yamada-Watanabe type result for stochastic differential equations with random coefficients as the ones encountered so far.
Recall from Section \ref{existence:sezione_crtl_rilassati} the definition of the space $\mathcal{V}$ and of relaxed controls.

Let $\mathfrak{U}=((\Omega,\F,\mathbb{F},\prob),\xi,W,\mu,\mathfrak{r})$ be a tuple composed by a filtered probability space satisfying usual assumptions, an $\mathbb{F}$-Brownian motion $W$, an $\R^d$-valued $\F_0$-measurable random variable $\xi$, an $\F_0$-measurable random flow of measures $\mu$ taking values in  $\contpdue$ and an $\mathbb{F}$-adapted $\mathcal{V}$-valued random variable $\mathfrak{r}$, in the sense that the random variables $\mathfrak{r}(C)$ are $\F_t$-measurable for every $C \in \boreliani{[0,t]\times A}$.
Let us consider the following stochastic differential equation:
\begin{equation}\label{weak_uniqueness:sde}
    dX_t=G(t,X_t,\mu,\mathfrak{r})dt + dW_t, \quad X_0=\xi.
\end{equation}
where $G:[0,T] \times \R^d \times \contpdue \times \mathcal{V} \to \R^d$ is jointly measurable and progressively measurable in $\mathcal{V}$; progressive measurability must be understood in the following sense: for every $q,q' \in \mathcal{V}$, for every $(t,x,m) \in [0,T] \times \R^d \times \contpdue$, it holds:
\begin{equation*}
    q(C)=q'(C) \; \;  \forall C \in \boreliani{[0,t] \times A} \; \Longrightarrow \; G(t,x,m,q)=G(t,x,m,q').
\end{equation*}



\begin{definition}[Strong solution and pathwise uniqueness]\label{weak_uniqueness:def_pathwise_uniqueness}
Let $\mathfrak{U}=((\Omega,\F,\mathbb{F},\prob),\xi,W,\mu,\mathfrak{r})$ be a tuple as above.
A strong solution to equation \eqref{weak_uniqueness:sde} on $\mathfrak{U}$ is a continuous $\mathbb{F}$-adapted process $X=(X_t)_{t \in [0,T]}$ adapted to the $\prob$-augmentation of $\mathbb{F}$ so that
\begin{equation}
    X_t=\xi+\int_0^t G(s,X_s,\mu,\mathfrak{r})ds + W_t, \quad 0 \leq t \leq T,
\end{equation}
holds $\prob$-almost surely.

Patwhise uniqueness holds for equation \eqref{weak_uniqueness:sde} if, given two strong solutions $X$ and $X'$ to \eqref{weak_uniqueness:sde} on $\mathfrak{U}$, they are indistinguishable:
\begin{equation*}
    \prob(X_t=X'_t \;\; \forall t \in [0,T])=1.
\end{equation*}
\end{definition}


\begin{definition}[Weak solution and uniqueness in law]\label{weak_uniqueness:def_uniqueness_in_law}
A weak solution to equation \eqref{weak_uniqueness:sde} is a tuple $\mathfrak{U}=((\Omega,\F,\mathbb{F},\prob),\xi,W,\mu,\mathfrak{r})$ as above so that there exists a continuous $\mathbb{F}$-adapted process $X=(X_t)_{t \in [0,T]}$ satisfying equation \eqref{weak_uniqueness:sde}.

Weak uniqueness holds for equation \eqref{weak_uniqueness:sde} if for any two weak solution of \eqref{weak_uniqueness:sde} $\mathfrak{U}^i$, $i=1,2$, so that $\prob^1\circ(\xi^1,W^1,\mu^1,\mathfrak{r}^1)^{-1}=\prob^2\circ(\xi^2,W^2,\mu^2,\mathfrak{r}^2)^{-1}$, it holds
\begin{equation*}
    \prob^1\circ(X^1,\xi^1,W^1,\mu^1,\mathfrak{r}^1)^{-1}=\prob^2\circ(X^2,\xi^2,W^2,\mu^2,\mathfrak{r}^2)^{-1},
\end{equation*}
where $X^i$ are the continuous $\mathbb{F}^i$-adapted processes that satisfy equation \eqref{weak_uniqueness:sde} on $\mathfrak{U}^i$, $i=1,2$.
\end{definition}


\begin{thm}\label{teorema_di_unicita_legge}
Suppose pathwise uniqueness holds for equation \eqref{weak_uniqueness:sde}, in the sense of Definition \ref{weak_uniqueness:def_pathwise_uniqueness}.
Then, uniqueness in law in the sense of Definition \ref{weak_uniqueness:def_uniqueness_in_law} holds as well.
\end{thm}

\begin{proof}
Let $\mathfrak{U}^1$ and $\mathfrak{U}^2$ be two weak solutions of equation \eqref{weak_uniqueness:sde} in the sense of Definition \ref{weak_uniqueness:def_uniqueness_in_law} above.
Since pathwise uniqueness holds for equation \eqref{weak_uniqueness:sde} by assumption, our goal is to bring together the solution on the same filtered probability space.
Let us define the following probability measures:
\begin{equation*}
\begin{aligned}
    & \hat{\Q}^i = \prob^i\circ(\xi^i,W^i,\mu^i,\mathfrak{r}^i,X^i)^{-1} \in \mathcal{P}(\R^d \times \contrd \times \contpdue \times \mathcal{V} \times \contrd), \quad i=1,2, \\
    & \Q = \prob^i\circ(\xi^i,W^i,\mu^i,\mathfrak{r}^i)^{-1} \in \mathcal{P}(\R^d \times \contrd \times \contpdue \times \mathcal{V}), \\
    & \Tilde{\Q} = \prob^i\circ(\xi^i,W^i,\mu^i)^{-1} \in \mathcal{P}(\R^d \times \contrd \times \contpdue).
\end{aligned}
\end{equation*}
Observe that $\Q$ and $\Tilde{\Q}$ are well defined, since $(\xi^i,W^i,\mu^i,\mathfrak{r}^i)$ share the same joint law by assumption.
Let us consider the following space:
\begin{equation*}
\begin{aligned}
    \Omega^{can} & =\contrd \times \contrd \times \R^d \times \contrd \times \contpdue \times \mathcal{V}; \\
    \F^{can} & =\boreliani{\contrd} \otimes \boreliani{\contrd } \otimes \boreliani{\R^d } \otimes \boreliani{ \contrd } \otimes \boreliani{\contpdue } \otimes \boreliani{\mathcal{V}}; \\
    \mathcal{G}_t^{can} &  = \mathcal{B}_{t,\contrd} \otimes \mathcal{B}_{t,\contrd} \otimes \boreliani{\R^d} \otimes \mathcal{B}_{t,\contrd} \otimes \boreliani{\contpdue} \otimes \F_t^{\mathcal{V}},
\end{aligned}
\end{equation*}
where
\begin{equation*}
\begin{aligned}
    & \mathcal{B}_{t,\contrd}=\sigma(\contrd \ni x \mapsto x_s \in \R^d: \; s \leq t), && \F^{\mathcal{V}}_t=\sigma( \mathcal{V} \ni q \mapsto q(C) \in \R: \; C \in \boreliani{[0,t]\times A}).
\end{aligned}
\end{equation*}
In order to equip the space $(\Omega^{can},\F^{can},(\mathcal{G}_t^{can})_{t \in [0,T]})$ with a probability measure, we disintegrate the measures $\hat{\Q}^i$, $i=1,2$, in the following way:
let $K^i:\boreliani{\contrd} \times \R^d \times \contrd \times \contpdue \times \mathcal{V} \to [0,1]$ be a regular conditional probability of $\hat{\Q}^i$ for $\boreliani{\contrd}$ given $(x,w,m,q)$, so that it holds
\begin{equation*}
    \hat{\Q}^i( A \times B ) = \int_B K^i(A , x,m,w,q)P(dx,dm,dw,dq),
\end{equation*}
for every $A \in \boreliani{\contrd}$, $B \in \boreliani{\R^d} \otimes \boreliani{\contrd} \otimes \boreliani{\contpdue} \otimes \boreliani{\mathcal{V}}$, or more briefly
\begin{equation*}
    \hat{Q}^i(dx,dw,dm,dq,dy)=K^i(dy,x,m,q,w)P(dx,dw,dm,dq), \: i=1,2.
\end{equation*}

Then, we set 
\begin{equation*}
    \overline{\Q}(dy^1,dy^2,dx,dm,dw,dq)=K^1(dy^1,x,m,q,w)K^2(dy^2,x,m,q,w)\Q (dx,dm,dw,dq).
\end{equation*}
Observe that the joint law under $\overline{\Q}$ of the coordinate projections $y^1$, $x$, $m$, $w$ and $q$ is exactly $\hat{\Q}^1$, and analogously when considering the coordinate process $y^2$ instead of $y^1$.
Finally, complete the $\sigma$-algebra $\F^{can}$ with the $\overline{\Q}$-null sets $\mathcal{N}^{\overline{\Q}}$ and consider the complete right continuous filtration $(\F_t^{can})_{t \in [0,T]}$ given by
\begin{equation*}
    \F_t^{can}=\bigcap_{\eps > 0} \sigma\tonde{\mathcal{G}_{t+\eps}, \mathcal{N}^{\overline{\Q}}}.
\end{equation*}
By Lemma \ref{lemma_moto_browniano}, the coordinate process $w$ is a $(\F^{can}_t)_{t \in [0,T]}$-Brownian motion under $\overline{\Q}$.
Furthermore, it holds
\begin{equation*}
\begin{aligned}
    y^i_t=x + \int_0^t G(s,y^i_s,m,q)ds + w_t, \; \forall t \in [0,T], \; \overline{\Q}\text{-a.s.}
\end{aligned}
\end{equation*}
for $i=1,2$.
Since pathwise uniqueness in the sense of Definition \ref{weak_uniqueness:def_pathwise_uniqueness} holds by assumption, it follows that $y^1$ and $y^2$ are indistinguishable under $\overline{\Q}$, which implies $\hat{\Q}^1=\hat{\Q}^2$.
This proves the desired result.
\end{proof}


\begin{lemma}\label{lemma_moto_browniano}
In the construction of Theorem \ref{teorema_di_unicita_legge}, $w=(w_s)_{s \in [0,T]}$ is a Brownian motion under $\overline{\Q}$ with respect to the filtration $(\F^{can}_s)_{s \in [0,T]}$.
\end{lemma}
\begin{proof}
Observe that $w$ is a natural Brownian motion under $\overline{\Q}$.
In order to show that it is a Brownian motion with respect to the filtration $(\mathcal{G}_t^{can})_{t \in [0,T]}$, we just need to prove that its increments are independent, and the conclusion follows.

Fix $A_1,A_2 \in \mathcal{B}_{t,\contrd}$, $B \in \boreliani{\R^d}$, $C \in \mathcal{B}_{t,\contrd}$, $D \in \boreliani{\contpdue}$ and $F \in \F_t^{\mathcal{V}}$.
By Cauchy-Schwartz inequality, we have, for every $H \in \boreliani{\R^d}$ and $s>t$:
\begin{equation*}
\begin{aligned}
    \E^{\overline{\Q}} & \quadre{\1_H(w_s-w_t)\1_{A_1 \times A_2 \times B \times D \times C \times F}(y^1,y^2,x,m,w,q)}^2 \\
    \leq & \E^{\overline{\Q}}\Big[ \1_H(w_s-w_t)\1_{A_1 \times B \times D \times C \times F}(y^1,x,m,w,q)\Big] \\
    & \cdot \E^{\overline{\Q}}\Big[\1_H(w_s-w_t)\1_{A_2 \times B \times D \times C \times F}(y^2,x,m,w,q) \Big].
\end{aligned}
\end{equation*}
Therefore, it suffices to show that 
\begin{equation}\label{lemma_moto_browniano:indipendenza_incrementi_singoli}
\begin{aligned}
    \E^{\overline{\Q}}\Big[ \1_H(w_s-w_t)\1_{A_1 \times B \times D \times C \times F}(y^1,x,m,w,q)\Big]=0.
\end{aligned}
\end{equation}
Since the integrand does not depend upon $y^2$, we may rewrite such an expectation only with respect to $\hat{\Q}^1$:
\begin{equation*}
\begin{aligned}
    \E^{\overline{\Q}} & \Big[ \1_H(w_s-w_t)\1_{A_1 \times B \times D \times C \times F}(y^1,x,m,w,q)\Big] \\
    = & \int \1_H(w_s-w_t) \1_{A_1 \times B \times D \times C \times F}(y^1,x,m,w,q)\hat{\Q}^1(dy^1,dx,dm,dw,dq).
\end{aligned}
\end{equation*}
Then, we introduce another disintegration of the measure $\hat{\Q}^1$: let $\Theta^1$ be a regular conditional probability for $\boreliani{\contrd}\otimes \boreliani{\mathcal{V}}$ given $(x,w,m)$:
\begin{equation}\label{lemma_moto_browniano:dinsintegrazione_senza_controllo}
    \hat{\Q}^1\tonde{ A \times B \times C \times D \times F } = \int_{ B \times C \times D } \Theta^1(A\times F, x,m,w)\Tilde{\Q}(dx,dm,dw),
\end{equation}
for every $A \in \boreliani{\contrd}$, $B \in \boreliani{\R^d}$, $C \in \boreliani{\contrd}$, $D \in \boreliani{\contpdue}$ and $F \in \boreliani{\mathcal{V}}$, or more briefly
\begin{equation*}
    \hat{\Q}^1(dy^1,dq,dx,dw,dm)= \Theta^1(dy^1,dq,x,m,w)\Tilde{\Q}(dx,dw,dm).
\end{equation*}
As in \cite[Lemma IV.1.1]{ikeda_watanabe1981sdes}, it can easily be shown that,
for every $A\times F \in \mathcal{B}_{s,\contrd} \otimes \F_s^{\mathcal{V}}$, the map
\begin{equation*}
    (x,m,w) \mapsto \Theta^1(A\times F,x,m,w)  
\end{equation*}
is $\boreliani{\R^d} \otimes \boreliani{\contpdue}\otimes \mathcal{B}_{s,\contrd}$-measurable, for every $s \in [0,T]$.
Therefore, we can compute the left-hand side of \eqref{lemma_moto_browniano:indipendenza_incrementi_singoli}:
\begin{equation*}
\begin{aligned}
    \E^{\overline{\Q}} & \Big[ \1_H(w_s-w_t)\1_{A_1 \times B \times D \times C \times F}(y^1,x,m,w,q)\Big] \\
    = & \int \1_H(w_s-w_t) \1_{A_1 \times B \times D \times C \times F}(y^1,x,m,w,q)\hat{\Q}^1(dy^1,dx,dm,dw,dq) \\
    = & \int \1_H(w_s-w_t) \Theta^1(A_1\times F, x,m,w)\1_{B \times D \times C}(x,m,w)\Tilde{\Q}^1(dx,dm,dw) \\
    = & \E^{\prob^{1}}\quadre{ \1_H(W^1_s-W^1_t) \Theta^1(A_1\times F, \xi^1,\mu^1,W^1)\1_{B \times D \times C}(\xi^1,\mu^1,W^1)} = 0,
\end{aligned}
\end{equation*}
since $\Theta^1(A_1\times F, \xi^1,\mu^1,W^1)\1_{B \times D \times C}(\xi^1,\mu^1,W^1)$ is $\F^1_s$-measurable and $W^1$ is an $\mathbb{F}^1$-Brownian motion under $\prob^1$ by assumption.
\end{proof}