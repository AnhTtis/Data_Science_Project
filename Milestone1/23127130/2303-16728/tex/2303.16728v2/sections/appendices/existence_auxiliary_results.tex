


In this section, we state and prove some auxiliary results that were used in Section \ref{sezione_existence} to prove the existence of a mean field CCE.
In particular, Lemmata \ref{existence:lemma_kernels} and \ref{existence:lemma_decomposizione} provide the technical instruments we used in Proposition \ref{existence:prop_relazione_mfg} to show that, for every deviating strategy $\beta \in \A$ and random flow of measures $\mu$, we can represent the joint law of $\mu$, $\beta$ and deviating player's state process in terms of a strategy for player B in the zero-sum game \ref{existence:def_zerosum} and the the law of $\mu$.
Lemmata \ref{existence:lemma_mimicking} and \ref{existence:mimicking:lemma_strong_existence} were needed in the proof of Theorem \ref{existence:main_theorem} in order to define a mean field CCE starting from an optimal strategy for player A in the zero-sum game \ref{existence:def_zerosum}.


\medskip
Consider any tuple $\mathfrak{U}=((\Omega,\F,\mathbb{F},\prob),\xi,W,\mu,\mathfrak{r})$, composed of a filtered probability space satisfying usual assumptions, a $d$-dimensional $\mathbb{F}$-Brownian motion, an $\R^d$-valued $\F_0$-measurable random variable, an $\F_0$-measurable
random continuous flow of measures in $\mathcal{P}^2(\R^d)$ and an $\mathbb{F}$-progressively measurable $\mathcal{P}(A)$-valued process.
Assume that $\mu$, $W$ and $\xi$ are mutually independent.
Let us consider the following equations:
\begin{align}
    dX_t & =\int_A b(t,X_t,\mu_t,a)\mathfrak{r}_t(da)dt + dW_t, \quad X_0=\xi, \label{existence:eq_misura_aleatoria} \\
    dX^m_t & =\int_A b(t,X^m_t,m_t,a)\mathfrak{r}_t(da)dt + dW_t, \quad X_0=\xi, \label{existence:eq_misura_deterministica}
\end{align}
where $m$ is a point of $\contpdue$.
In order to stress the dependence upon the deterministic flow of measures $m$, we write $X^m$ for the solution of \eqref{existence:eq_misura_deterministica}.

By Assumptions \ref{standing_assumptions}, on any such tuple $\mathfrak{U}$ there exists a solution to equation \eqref{existence:eq_misura_aleatoria} and pathwise uniqueness holds.
If needed, we can suppose that the filtration $\mathbb{F}$ on $(\Omega,\F,\prob)$ is the $\prob$-augmentation of the filtration $\mathbb{F}^{\xi,W,\mu,\mathfrak{r}}$, given by
\begin{equation}\label{existence:filtrazione_forte_ctrl}
    \F_t^{\xi,W,\mu,\mathfrak{r}}=\sigma(\xi)\vee\sigma(\mu)\vee\sigma(W_s: s\leq t)\vee\sigma(\mathfrak{r}(C): C \in \boreliani{[0,t]\times A}).
\end{equation}
By Theorem \ref{teorema_di_unicita_legge}, uniqueness in law holds.
Analogous reasoning holds for equation \eqref{existence:eq_misura_deterministica} as well, for every $m \in \contpdue$.

\begin{lemma}\label{existence:lemma_kernels}
Let $\mathfrak{U}=((\Omega,\F,\mathbb{F},\prob),\xi,W,\mu,\mathfrak{r})$ be as above, let $\Theta \in \mathcal{P}(\R^d \times \contrd \times \mathcal{V})$ be the joint law of $\xi$, $W$ and $\mathfrak{r}$.
Let us define the map
\begin{equation}\label{existence:mappa_strat_min_continua}
\begin{aligned}
    \mathcal{I}_{\Theta}: \contpdue  & \longrightarrow \mathcal{P}(\R^d \times \contrd \times \contrd \times \mathcal{V}) \\
    m & \longmapsto \mathcal{I}_{\Theta}(m)=\prob\circ(\xi,W,X^m,\mathfrak{r})^{-1},
\end{aligned}
\end{equation}
where $X^m$ is the solution to equation \eqref{existence:eq_misura_deterministica}.
\begin{enumerate}[label=(\roman*)]
    \item \label{existence:lemma_kernels:mappa_continua}
    The map $\mathcal{I}_{\Theta}$ is continuous, in the sense that
    \begin{equation*}
        \sup_{t \in [0,T]}\pwassmetric{2}{\R^d}{}(m^n_t,m_t) \to 0 \text{ as } n \to \infty \; \Longrightarrow \; \mathcal{I}_{\Theta}(m^n) \overset{n \to \infty}{\longrightarrow} \mathcal{I}_{\Theta}(m) \text{ in } \pwassmetric{2}{\R^d \times \contrd \times  \mathcal{V} \times \contrd}{}.
    \end{equation*}
    
    \item \label{existence:lemma_kernels:kernel_ben_definito}
    The map $\mathcal{I}_{\Theta}$ induces a stochastic kernel $\Sigma$ from $\contpdue$ to $\contrd \times \mathcal{V}$, by setting
    \begin{equation*}
        \Sigma(B,m)=\prob((X^m,\mathfrak{r}) \in B)=\mathcal{I}_{\Theta}(m)(\R^d \times \contrd \times B) \quad \forall m \in \contpdue, \; B \in \boreliani{\contrd}\otimes\boreliani{\mathcal{V}}.
    \end{equation*}
    $\Sigma$ is a strategy for player B, as described in Definition \ref{existence:strategie_min}.
\end{enumerate}
\end{lemma}
\begin{proof}
Note that, by Theorem \ref{teorema_di_unicita_legge}, $\mathcal{I}_{\Theta}(m)$ is the unique weak solution of \eqref{existence:eq_misura_deterministica} when the joint law of $\xi$, $W$ and $\mathfrak{r}$ is given by $\Theta$ and $b$ is evaluated at $m \in \contpdue$.
Let $\Sigma \in \mathcal{Q}$, let $(m^n)_{n\geq 1} \subset \contpdue$ so that $m^n \to m$.
For every $n\geq 1$, denote by $X$ and $X^n$ the solution to equation \eqref{existence:eq_misura_deterministica} when $b$ is evaluated at $m$ and $m^n$, respectively.
For every $2 \leq p \leq \overline{p}$, by Lipschitz continuity of $b$, we have:
\begin{align}
    \E\quadre{\sup_{0 \leq s \leq T}\abs{X^n_s-X_s}^p}\leq C \sup_{t \in [0,T]}\pwassmetric{p}{\R^d}{2}(m^n_t,m_t), \label{existence:lemma_continuita:stime_norme} \\ 
    \sup_{n \geq 1}\attesa{\sup_{t \in [0,T]}\abs{X^n_t}^p}\leq C\tonde{1 + \sup_{n \geq 1} \sup_{t \in [0,T]} \tonde{\int_{\R^d}\abs{y}^2 m^n_t(dy)}^{\frac{p}{2}}}. \label{existence:lemma_continuita:bound_uniformi}
\end{align}
Therefore, for $p=2$, we have that $m^n \to m$ in $\contpdue$ implies that $\lVert  X^n-X \rVert_{\contrd}^2 \to 0$ in expectation, which in turns implies that $(\xi,W,\mathfrak{r},X^n) \longrightarrow (\xi,W,\mathfrak{r},X)$ in distribution.
In order to have convergence in 2-Wasserstein metrics, it is enough to check uniform integrability, according to  \eqref{wass:uniforme_integrabilita}.
Since $(\mathcal{I}_{\Theta}(m^n))_n$ have the same marginals on $\R^d\times\contrd\times\mathcal{V}$, we just need to check \eqref{wass:uniforme_integrabilita} for the laws of $(X^n)_n$: for every $n \geq 1$, $r > 0$, we have
\begin{equation*}
\begin{aligned}
    \E\quadre{\norm{X^n}_{\contrd}^2\1_{\insieme{\norm{X^n}^2_{\contrd}>r}}} & \leq  \tonde{\E\quadre{\norm{X^n}_{\contrd}^{4}}}^\frac{1}{2}\tonde{\E\quadre{\1_{\insieme{\norm{X^n}^2_{\contrd}>r}}}}^\frac{1}{2}  \\
    & \leq \tonde{\E\quadre{\norm{X^n}_{\contrd}^{4}}}^\frac{1}{2} \E\quadre{\norm{X^n}_{\contrd}^2}^\frac{1}{2}r^{-\frac{1}{2}} \leq C r^{-\frac{1}{2}}
\end{aligned}
\end{equation*}
by using Cauchy-Schwartz inequality, Markov inequality, \eqref{existence:lemma_continuita:stime_norme} and \eqref{existence:lemma_continuita:bound_uniformi}.
By taking the limit as $r \to \infty$, we get condition \eqref{wass:uniforme_integrabilita} satisfied and so point \ref{existence:lemma_kernels:mappa_continua} is proved.

\medskip
As for point \ref{existence:lemma_kernels:kernel_ben_definito}, let $\pi:\R^d\times\contrd\times\contrd\times\mathcal{V}\to\contrd\times\mathcal{V}$ be the projection on the last two components.
Note that $(\mathcal{I}_{\Theta}\circ \pi^{-1})(m)=\prob \circ (X^m,\mathfrak{r})^{-1}$, which shares the same continuity properties of the map $\mathcal{I}_{\Theta}$.
Therefore, in particular, it is Borel measurable, where $\mathcal{P}(\contrd \times \mathcal{V})$ is endowed with the usual Borel $\sigma$-algebra associated with the topology of weak convergence.
Then, the thesis follows from the fact that, for a Polish space $E$, the usual Borel $\sigma$-field on $\mathcal{P}(E)$ coincide with the $\sigma$-field generated by the maps $\mathcal{P}(E) \ni m \mapsto m(S)$, with $S \in \boreliani{E}$, (see, e.g., \cite[Corollary 7.29.1]{bertsekas_shreve}).
\end{proof}


\begin{lemma}\label{existence:lemma_decomposizione}
Let $\mathfrak{U}=((\Omega,\F,\mathbb{F},\prob),\xi,W,\mu,\mathfrak{r})$ be a tuple so that $(\xi,W,\mathfrak{r})$ and $\mu$ are independent.
Denote by $\rho \in \mathcal{P}(\contpdue)$ the law of $\mu$ under $\prob$.
Suppose without loss of generality that $\mathbb{F}$ is the $\prob$-augmentation of the filtration $(\F^{\xi,W,\mu,\mathfrak{r}}_t)_t$ defined by \eqref{existence:filtrazione_forte_ctrl}.
Let $X$ be the unique solution of \eqref{existence:eq_misura_aleatoria} on the tuple $\mathfrak{U}$.
Then, the following decomposition of measure holds
\begin{equation*}
    \prob((X,\mathfrak{r},\mu) \in B \times S)= \int_S \Sigma(B,m)\rho(dm), \quad \forall B \in \boreliani{\contrd \times \mathcal{V}}, \; S \in \boreliani{\contpdue}.
\end{equation*}
In particular, $\Sigma(B,m)=\prob((X,\mathfrak{r}) \in B \; \vert \; \mu=m)=\prob((X^m,\mathfrak{r}) \in B )$ for every $B \in \boreliani{\contrd \times \mathcal{V}}$, $\rho$-a.e. $m \in \contpdue$.
\end{lemma}
\begin{proof}
Let $\prob(\cdot \; \vert \; \mu)$ denote the regular conditional probability of $\prob$ given $\mu$.
Set $\prob^m(\cdot)=\prob(\cdot \; \vert \; \mu=m)$.
Since $(\xi,W,\mathfrak{r})$ and $\mu$ are independent by assumption, we have that $\prob^m\circ(\xi,W,\mathfrak{r})^{-1}=\prob\circ(\xi,W,\mathfrak{r})^{-1}$ for $\rho$-a.e. $m \in \contpdue$.
Therefore, it is enough to prove that $X$ is a solution to \eqref{existence:eq_misura_deterministica} on the tuple $\mathfrak{U}=((\Omega,\F,\mathbb{F},\prob^m),\xi,W,\mathfrak{r})$ for $\rho$-a.e. $m \in \contpdue$.
Then, since uniqueness in law holds for \eqref{existence:eq_misura_deterministica}, we deduce that $\prob^m\circ(\xi,W,\mathfrak{r},X)^{-1}=\mathcal{I}_{\Theta}(m)$ $\rho$-a.s.
Observe that, since the joint law of  $(\xi,W,\mathfrak{r})$ is the same under $\prob$ and $\prob^m$ for $\rho$-a.e $m$, $W$ is a natural Brownian motion under $\prob^m$ as well.
By definition of the filtration $\mathbb{F}$, it can be easily verified that
\begin{equation*}
    \E^\prob[\1_A(W_t-W_s)g \; \vert \;  \mu]=0 \quad \prob\text{-a.s}
\end{equation*}
for every $0 \leq s < t \leq T$, $A \in \boreliani{\R^d}$, $g$ bounded and $\F_s$-measurable.
This implies that
\begin{equation*}
    E^{\prob^m}[\1_A(W_t-W_s)g]=\E^{\prob}[\1_A(W_t-W_s)g\; \vert \;  \mu=m]=0
\end{equation*}
$\rho$-a.s., for every $g$ bounded and $\F_s$-measurable.
By working with a countable measure determining class of sets, which is possible since the $\sigma$-algebra $\F^{\xi,W,\mu,\mathfrak{r}}_t$ is countably generated for every $t \in [0,T]$, the equality holds for every $g$ bounded and $\F_s$-measurable, for $\rho$-a.e. $m \in \contpdue$, which in turn implies that $W$ remains an $\mathbb{F}$-Brownian motion under $\prob^m$ as well.
Under $\prob^m$ one has 
\begin{equation*}
    \prob^{m}\tonde{\int_A b(t,x,\mu_t,a)\mathfrak{r}_t(da)=\int_A b(t,x,m_t,a)\mathfrak{r}_t(da) \quad \forall x \in \R^d}=1 \quad Leb_{[0,T]}\text{-a.e. } t \in [0,T]
\end{equation*}
and therefore $X$ solves \eqref{existence:eq_misura_deterministica} for $b$ evaluated at $m \in \contpdue$.
The thesis follows from marginalizing as in the proof of point \ref{existence:lemma_kernels:kernel_ben_definito} in \ref{existence:lemma_kernels}.
\end{proof}

\medskip
We turn our attention to the proof of Lemma \ref{existence:lemma_mimicking}, which for convenience we restate below.

\begin{replemma}{existence:lemma_mimicking}
Let $\Gamma \in \mathcal{K}$. There exists a measure $\hat{\Gamma} \in \mathcal{K}$ so that the following holds:
\begin{itemize}
    \item The marginal distributions of $\Gamma$ and $\hat{\Gamma}$ on $\contpdue$ are the same: $\Gamma(\contrd \times \mathcal{V} \times \cdot)=\hat{\Gamma}(\contrd \times \mathcal{V} \times \cdot)$.
    
    \item Let $(X,\mathfrak{r},\mu)$ be such that $\hat{\Gamma}=\prob\circ(X,\mathfrak{r},\mu)^{-1}$.
    Then $\mathfrak{r}$ is of the form $\mathfrak{r}_t=\hat{q}_t(X_t,\mu)$, where $\hat{q}:[0,T]\times\R^d\times\contpdue \to \mathcal{P}(A)$ is a measurable function.
    \item For every $\Sigma \in \mathcal{Q}$, it holds
    \begin{equation*}
        \mathfrak{p}(\Gamma,\Sigma)=\mathfrak{p}(\hat{\Gamma},\Sigma).
    \end{equation*}
\end{itemize}
\end{replemma}


\begin{proof}
In the following, for a metric space $(E,d_E)$, $\phi:E \to \R$ continuous and bounded and $m \in \probmeasures{E}$, we set $\langle \phi, m \rangle = \int_E \phi(e) m(de)$.

Let $\mathfrak{U}=((\Omega,\F,\mathbb{F},\prob),\xi,W,\mu,\mathfrak{r})$ be as in Definition \ref{existence:strategie_max}, so that $\Gamma=\prob\circ(X,\mathfrak{r},\mu)^{-1}$.
As ensured by \cite[Lemma C.2]{lacker2020convergence}, we have that, by choosing $Y_t=(X_t,\mu)$ taking values in $\R^d \times \contpdue$ as conditioning process, there exists a jointly measurable function $\hat{q}:[0,T] \times \R^d \times \contpdue \to \mathcal{P}(A)$ so that, for every $\phi:[0,T] \times \R^d \times \contpdue \times A \to \R$ bounded and measurable it holds
\begin{equation}\label{existence:lemma_mimicking:markovian_ctrl}
    \int_{A}\phi(X_t,\mu,a)\hat{q}_t(X_t,\mu)(da)=\attesa{\int_{A}\phi(X_t,\mu,a)\mathfrak{r}_t(da) \big\vert X_t,\mu} \quad \prob\text{-a.s.}, \; \text{a.e. } t \in [0,T],
\end{equation}
which we abbreviate as
\begin{equation*}
    \hat{q}_t(X_t,\mu)(da)=\attesa{\mathfrak{r}_t(da) \big\vert X_t,\mu} \quad \prob\text{-a.s.}, \; \text{a.e. } t \in [0,T].
\end{equation*}
Next, we manipulate the term of the functional $\mathfrak{p}$ in \eqref{existence:payoff_functional} which depends only upon $\Gamma$:
\begin{equation}\label{existence:lemma_mimicking:uguaglianze1}
\begin{aligned}
    \int \: &  \mathfrak{F} (y,q,m) \Gamma(dy,dm,dq) = \attesa{\int_0^T\int_A f(t,X_t,\mu_t,a)\mathfrak{r}_t(da)dt + g(X_T,\mu_T)} \\
    = & \int_0^T \attesa{\attesa{ \int_A f(t,X_t,\mu_t,a)\mathfrak{r}_t(da) \Big \vert X_t,\mu }}dt + \attesa{g(X_T,\mu_T)} \\
    = & \int_0^T\attesa{ \int_A f(t,X_t,\mu_t,a)\hat{q}_t(X_t,\mu)(da) }dt + \attesa{g(X_T,\mu_T)} \\
    = & \int_0^T\attesa{\attesa{ \int_A f(t,X_t,\mu_t,a)\hat{q}_t(X_t,\mu)(da) \Big \vert \mu }}dt + \attesa{\attesa{g(X_T,\mu_T) \Big \vert \mu }} \\
    = & \int_0^T\attesa{ \left \langle \int_A f(t,\cdot,\mu_t,a)\hat{q}_t(\cdot,\mu)(da), \mu_t \right \rangle } dt + \attesa{\left \langle g(\cdot,\mu_T),  \mu_T \right \rangle }.
\end{aligned}
\end{equation}
Second equality holds by Fubini's theorem and tower property of conditional expectation,
third equality holds by definition of the control \eqref{existence:lemma_mimicking:markovian_ctrl}, third and fourth equalities hold by tower property again, and fifth equality holds since, by consistency condition, $\mu_t(\cdot)=\prob(X_t \in \cdot \; \vert \; \mu)$.

\medskip
We observe that, by choosing $\phi(t,x,m,a)=b(t,x,m_t,a)$ in \eqref{existence:lemma_mimicking:markovian_ctrl}, we have 
\begin{equation*}
    \int_A b(t,X_t,\mu_t,a)\hat{q}_t(X_t,\mu)(da)=\attesa{\int_A b(t,X_t,\mu_t,a)\mathfrak{r}_t(da) \big\vert X_t,\mu} \quad \prob\text{-a.s.}, \; \text{a.e. } t \in [0,T].
\end{equation*}
This is enough to apply \cite[Theorem~3.6]{brunick_shreve_mimicking}: indeed, in its terminology, we can take $\mathcal{E}=\R^d \times  \contpdue$, $\Phi: \mathcal{E} \times \contrd_0  \to \conttraj{\mathcal{E}}$ defined by $\Phi_t(x,m,y)=(x_t + y,m) \in \R^d \times \contpdue$, where $\contrd_0=\{ y \in \contrd: \; x_0=0\}$.
Set $Z_t=\Phi(X_t-X_0,X_0,\mu)=(X_t,\mu)$, 
where we note that the second component of $Z$ is constant in time as it is equal to the whole flow $\mu=(\mu_s)_{s \in [0,T]}$.
Then, such a result ensures that there exists a probability space $(\hat{\Omega},\hat{\F},\hat{\mathbb{F}},\hat{\prob})$, with $\hat{\Omega}$ Polish and $\hat{\F}$ its corresponding Borel $\sigma$-algebra, supporting an $\hat{\mathbb{F}}$-Brownian motion $\hat{W}$, a continuous $\mathcal{E}$-valued process $\hat{Z}$ so that there exists an $\hat{\mathbb{F}}$-adapted process $\hat{X}$ that satisfies
\begin{equation*}
    \hat{X}_t=\hat{X_0} + \int_0^T\int_A b(s,\hat{X}_s,\hat{\mu}_s,a)\hat{q}_s(\hat{X}_s,\hat{\mu}_s)(da)ds + \hat{W}_t, \quad \hat{Z}=\Phi(\hat{X}_t-\hat{X}_0,\hat{X}_0,\hat{\mu})
\end{equation*}
so that for every $t \in [0,T]$ it holds $\prob\circ Z_t^{-1}=\hat{\prob}\circ\hat{Z}_t^{-1}$.
This implies both that $\hat{\mu}$ and $\mu$ have the same law $\rho$ and that the consistency condition is satisfied, since $\hat{\prob}\circ(\hat{X}_t,\hat{\mu})^{-1}=\prob \circ (X_t,\mu)^{-1}=m_t(dx)\rho(dm)$.
Finally, since $Z$ is $\hat{\mathbb{F}}$-adapted, we deduce that $\hat{X}_0$ and $\hat{\mu}$ are $\F_0$-measurable and therefore $\hat{W}$, $\hat{X}_0$ and $\hat{\mu}$ are mutually independent.

Set $\hat{\Gamma}=\hat{\prob}\circ(\hat{X},\hat{\mathfrak{r}},\hat{\mu})^{-1}$.
Since the last term in the chain of equalities \eqref{existence:lemma_mimicking:uguaglianze1} depends only upon $\mu$ and $\mu$ and $\hat{\mu}$ share the same law, we can exploit the fact that $\hat{\mu}$ and $\hat{X}$ satisfy the consistency condition as well to get
\begin{equation*}
\begin{aligned}
     \: \int &  \mathfrak{F} (y,q,m) \Gamma(dy,dm,dq) =\int_0^T \E \quadre{  \left \langle \int_A f(t,\cdot,\mu_t,a)\hat{q}_t(\cdot,\mu)(da), \mu_t \right \rangle } dt + \E \quadre{ \left \langle g(\cdot,\mu_T),  \mu_T \right \rangle }\\
     = & \int_0^T \E^{\hat{\prob}} \quadre{  \left \langle \int_A f(t,\cdot,\hat{\mu}_t,a)\hat{q}_t(t,\cdot,\hat{\mu}_t,a)(da), \hat{\mu}_t \right \rangle} dt + \E^{\hat{\prob}} \quadre{ \left \langle g(\cdot,\hat{\mu}_T),  \hat{\mu}_T \right \rangle } \\
     = & \int_0^T \E^{\hat{\prob}} \quadre{ \E^{\hat{\prob}} \quadre{\int_A f(t,\hat{X}_t,\hat{\mu}_t,a)\hat{q}_t(\hat{X}_t,\hat{\mu})(da)} dt + \E^{\hat{\prob}} \quadre{ g(\hat{X}_T,\hat{\mu}_T) \Big \vert \hat{\mu}} } \\
     = & \int_0^T \E^{\hat{\prob}} \quadre{ \int_A f(t,\hat{X}_t,\hat{\mu}_t,a)\hat{q}_t(\hat{X}_t,\hat{\mu})(da) } dt + \E^{\hat{\prob}} \quadre{ g(\hat{X}_T,\hat{\mu}_T) } \\
     = & \int \mathfrak{F} (y,q,m) \hat{\Gamma}(dy,dm,dq).
\end{aligned}
\end{equation*}
Analogously, for every $\Sigma \in \mathcal{Q}$, we have
\begin{equation*}
    \int \mathfrak{F} (y,q,m)  \Sigma(dy,dq,m)\rho(dm) = \int \mathfrak{F} (y,q,m)  \Sigma(dy,dq,m)\hat{\rho}(dm),
\end{equation*}
which proves the desired statement about the payoff functional $\mathfrak{p}$.
\end{proof}

Finally, we show that it is always possible to find a strong solution to equation \eqref{existence:eq_processo_K} in the case of a feedback in state control process $\hat{q}_t(x,m)$, as given by Lemma \ref{existence:lemma_mimicking}:
\begin{lemma}[Strong solutions for feedback in state controls]\label{existence:mimicking:lemma_strong_existence}
Let $(\Omega,\F,\mathbb{F},\prob)$ be a filtered probability space satisfying the usual assumptions, with $\Omega$ Polish and $\F$ its Borel $\sigma$-algebra, supporting a $d$-dimensional $\mathbb{F}$-Brownian motion $W$, an $\F_0$-measurable $\R^d$-valued random $\xi$ with law $\nu$ and a $\F_0$-measurable random flow of measures $\mu$ in $\contpdue$ with law $\rho$.
Assume that $\xi$, $W$ and $\mu$ are mutually independent.
Let $\hat{q}:[0,T]\times \R^d \times \contpdue \to \mathcal{P}(A)$ be a measurable function, and suppose that there exists a solution of the SDE
\begin{equation}
    dX_t=\int_A b(t,X_t,\mu_t,a)\hat{q}_t(X_t,\mu)(da)dt + dW_t,
\end{equation}
so that it holds
\begin{equation*}
    \mu_t(\cdot)=\prob( X_t \in \cdot \; \vert \; \mu) \quad \prob\text{-a.s.}
\end{equation*}
for every $t \in [0,T]$.
Then, $X$ may be taken adapted to the $\prob$-augmentation of the filtration $\mathbb{F}^{\xi,\mu,W}=\sigma(\xi) \vee \sigma(\mu) \vee \mathbb{F}^W$.
In particular, there exists a progressively measurable function $\Phi:\contpdue \times \R^d \times \contrd \to \contrd$ so that $\Phi(\mu,\xi,W)=X$ $\prob$-a.s.
\end{lemma}
\begin{proof}
Set $B(t,x,m)=\int_A b(t,x,m_t,a)\hat{q}_t(x,m)(da)$.
$B$ is jointly measurable in $(t,x,m) \in [0,T] \times \R^d \times\contpdue$ with at most linear growth in $(x,m) \in  \R^d \times \contpdue$ for every $t \in [0,T]$.
The following hold:
\begin{enumerate}
    \item \label{mimicking:lemma_strong_existence:punto_eq_deterministica}
    For every $m \in \contpdue$, equation
    \begin{equation} \label{mimicking:lemma_strong_existence:eq_deterministica}
        dX^m_t=B(t,X^m_t,m)dt +dW_t, \quad X^m_0=\xi.
    \end{equation}
    admits a unique strong solution.
    Moreover, let $P^m=\prob \circ (X^m)^{-1}$.
    Then, the map $\contpdue \ni m \mapsto P^m \in \probmeasures{\contrd}$ is measurable.
    \item \label{mimicking:lemma_strong_existence:esistenza_sol_forte}
    There exist a continuous $\mathbb{F}$-adapted process $X$ solution to
    \begin{equation}\label{mimicking:lemma_strong_existence:eq_aleatoria}
        dX_t=B(t,X_t,\mu)dt + dW_t, \quad X_0=\xi.
    \end{equation}
    $X$ is adapted to the $\prob$-augmentation of the filtration $\mathbb{F}^{\xi,\mu,W}$.
    \item \label{mimicking:lemma_strong_existence:pathwise_uniqueness}
    Pathwise uniqueness holds, in the following sense: 
    suppose there exists a pair of continuous $\mathbb{F}$-adapted processes $(X^1,X^2)$ which satisfy equation \eqref{mimicking:lemma_strong_existence:eq_aleatoria} so that $(X^1_s,X^2_s)_{s \leq t}$ is conditionally independent of $\F^{\xi,\mu,W}_T$ given $\F^{\xi,\mu,W}_t$ for every $t \in [0,T]$.
    Then, $\prob(X^1_t=X^2_t, \; 0 \leq t \leq T)=1$.
    
    \item \label{mimicking:lemma_strong_existence:punto_legge_soluzione}
    The joint law of $X$ and $\mu$ is given by 
    \begin{equation*}
        \prob\circ(X,\mu)^{-1}=P^m(dx)\rho(dm).
    \end{equation*}
\end{enumerate}
This properties can be proven with the same methods of \cite[Appendix~A]{lacker2020convergence} and \cite[Appendix~A]{lacker_leflem2022}.
We just point out that the results therein do not hold automatically in our case, since $B$ is not progressively measurable in the measure flow $m$, in the sense of \cite{lacker2020convergence,lacker_leflem2022}.
Nevertheless, since we require $\mu$ to be $\F_0$-measurable, the same arguments lead to the results above.

\medskip
Let $X$ be as in the statement of the lemma.
We first show that the joint law of $X$ and $\mu$ is given by $P^m(dx)\rho(dm)$.
Let $\prob^m(\cdot)=\prob(\cdot \vert \mu=m)$ be a version of the regular conditional probability of $\prob$ given $\mu=m$.
Then, since $\xi$, $W$ and $\mu$ are mutually independent, 
$\prob^m\circ(\xi,W)^{-1}=\prob\circ(\xi,W)^{-1}$ for $\rho$-a.e. $m$, and, by exploiting the fact the $\F^{\xi,\mu,W,X}_t$ is countably generated for every $t$, $W$ is an $\mathbb{F}^{\xi,\mu,W,X}$-Brownian motion under $\prob^m$ as well.
Therefore, $X$ satisfies equation \eqref{mimicking:lemma_strong_existence:eq_deterministica} on $(\Omega,\F,\mathbb{F}^{\xi,\mu,W,X},\prob^m)$ for $\rho$-a.e. $m \in \contpdue$.
By point \ref{mimicking:lemma_strong_existence:punto_eq_deterministica}, $\prob^m \circ X^{-1}=P^m$ for $\rho$-a.e. $m \in \contpdue$, which implies that $\prob\circ(X,\mu)^{-1}=P^m(dx)\rho(dm)$.

\medskip
It can be shown by straightforward calculations that $(X_s)_{s\leq t}$ is conditionally independent of $\F^{\xi,\mu,W}_T$ given $\F^{\xi,\mu,W}_t$, for every $t \in [0,T]$.
Since pathwise uniqueness holds by point \ref{mimicking:lemma_strong_existence:pathwise_uniqueness}, this implies that $X$ is indistinguishable from an $\mathbb{F}^{\xi,\mu,W}$-adapted solution to equation \eqref{mimicking:lemma_strong_existence:eq_aleatoria}.
\end{proof}