\documentclass[aps,prl,reprint,twocolumn,floatfix,superscriptaddress,longbibliography]{revtex4-2}

%%%%%%%%%%%%%%%%%%%%%%%%%%%%%%% packages
\usepackage{standalone}
\usepackage{amsthm}
\usepackage{lipsum}
\usepackage{amsfonts, amsmath, amssymb}
\usepackage{graphicx}
\usepackage{dcolumn}
\usepackage{bm}
\usepackage{dsfont}
\usepackage[normalem]{ulem}
\usepackage{color}
\usepackage[utf8]{inputenc}
\usepackage[colorlinks,citecolor=blue,linkcolor=blue,urlcolor=blue]{hyperref}
\usepackage{tikz}
\usepackage{braket}
\usepackage{physics}
%\graphicspath{{img/}}
\usepackage{color}
\usepackage{soul}
\usepackage{mathtools}
\usepackage{float}
\usepackage{esvect}
\usepackage{subfigure}
\usepackage{stmaryrd} 
\usepackage{enumerate}
\usepackage{bbold}
%\usepackage{xr}
%\externaldocument[S-]{SM}



%%%%%%%%%%%%%%%%%%%%%%%%%%%%%%%%%%%%%%%%%%%%%%%%%%%%%%%%%%
\section{Antimatter from dark-matter annihilation}
\label{sec:physics}
%%%%%%%%%%%%%%%%%%%%%%%%%%%%%%%%%%%%%%%%%%%%%%%%%%%%%%%%%%

%%%%%%%%%%%%%%%%%%%%%%%%%%%%%%%%%%%%%%%%%%%%%%%%%%%%%%%%%%
\subsection{General features of stable antiparticle production in \textsc{Pythia}~8}
%%%%%%%%%%%%%%%%%%%%%%%%%%%%%%%%%%%%%%%%%%%%%%%%%%%%%%%%%%

To properly assess the QCD uncertainties on antimatter spectra in DM indirect searches, we first study their production in a generic annihilation or decay process of a dark-matter candidate with mass in the GeV--TeV range. For this, let us consider the following generic annihilation process\footnote{This discussion applies to the case of decaying dark-matter as well. For instance, $\chi \to {\rm SM}~{\rm SM}$ is theoretically possible in models breaking the $Z_2$ symmetry through {\it e.g.} nonminimal interaction of dark matter with gravity (see {\it e.g.} \cite{Cata:2016dsg, Azri:2020bzl, Shaposhnikov:2020aen} for more details).}
\begin{equation}
    \chi \chi \to X_1 X_2 \ldots X_N \to \bigg(\displaystyle\prod_{i=1}^{a_1} Y_{1i} \bigg) \bigg(\displaystyle\prod_{j=1}^{a_2} Y_{2j} \bigg) \ldots \bigg(\displaystyle\prod_{z=1}^{a_N} Y_{Nz} \bigg).
    \label{eq:annihilation}
\end{equation}
We have assumed the narrow-width approximation which enables us to factorise the process into a production part $\chi \chi \to \prod_{i=1}^{N} X_i$ and a decay part $X_i \to \prod_{k=1}^{a_i} Y_{ik}$. In equation \ref{eq:annihilation}, $X_i$ refers to any parton-level SM particle which could be a resonance such as the Higgs boson, $W/Z$-bosons, or the top quark or a non-resonant state like gluons or light quarks. The $X_i$ states are assumed to produce $a_i$ states, through $X_i \to \prod_{k=1}^{a_i} Y_{ik}$, after a complex sequence of processes such as QED bremsstrahlung, QCD parton showering, hadronisation and hadron decays. The produced antiparticles are, therefore, part of this final state and can be detected in indirect detection experiments. Unlike the case of gamma rays, both the total rate as well as the shape of the antiparticle spectra are slightly affected by QED bremsstrahlung. This process occurs if $X_i$ or $Y_i$ contains electrically charged particles or photons and will lead to production of additional photons and/or charged fermions. Besides QED bremsstrahlung, coloured fermions produced in DM annihilation, either promptly or through the decay of heavy resonances, will undergo QCD bremsstrahlung wherein additional coloured particles are produced. This phenomenon is characterised by an enhancement of probabilities for emissions of soft and/or collinear gluons, with the latter being suppressed if the produced fermions are heavy. Furthermore, the rates of $g\to q\bar{q}$ are enhanced at low gluon virtualities: $Q^2 = (p_q + p_{\bar{q}})^2 \to 0$. The rate of QCD radiation processes is mainly controlled by the effective value of the strong coupling constant $\alpha_S$ which is evaluated at a scale proportional to the shower evolution variable\footnote{In \textsc{Pythia}~8, the shower evolution scale is the transverse momentum of the branching parton.}. We note that the value of $\alpha_S(M_Z)$ in \textsc{Pythia}~8 is larger than $\alpha_S(M_Z)^{\overline{{\rm MS}}}$ %from that derived in the $\overline{\mathrm{MS}}$-scheme by about $20\%$ 
for two reasons: {\it (i)} the so-called Catani-Webber-Marchesini (CMW) scheme involves a set of universal corrections in the soft limit \cite{Catani:1990rr} which is equivalent to a net $10\%$ increase in the value of $\alpha_S(M_Z)$ and {\it (ii)} an agreement between data and theory in experimental measurements of $3$-jets observables in $e^+ e^-$ collisions at LEP energies is reached if $\alpha_S(M_Z)$ is increased by a further $\sim10\%$ \cite{Skands:2010ak,Skands:2014pea}.  

Finally, at a scale $Q_{\rm IR} \simeq \mathcal{O}(1)~{\rm GeV}$, any coloured particle must hadronise to produce a set of colourless hadrons. This process, called fragmentation is modeled within \textsc{Pythia}~8 with the Lund string model \cite{Andersson:1983ia, Sjostrand:1982fn, Sjostrand:1984ic}. The longitudinal part of the description of the hadronisation process is given by the left-right symmetric {\it fragmentation function}, $f(z)$, which gives the probability for a hadron to take a fraction $z \in [0, 1]$ of the remaining energy at each step of the (iterative) string fragmentation process. The general form can be written as 
\begin{equation}
        f(z,m_{\perp h}) \propto N \frac{(1-z)^a}{z}\exp\left(\frac{-b m_{\perp h}^2}{z}\right)~,
\label{eq:fz}
\end{equation}
where $N$ is a normalisation constant that guarantees the distribution to be normalised to unit integral, and $m_{\perp h}= \sqrt{m_h^2 + p_{\perp h}^2}$ is called the ``transverse mass'', with $m_h$ the mass of the produced hadron and $p_{\perp h}$ its momentum transverse to the string direction, and $a$ and $b$ are tunable parameters which are denoted in \textsc{Pythia}~8 by \texttt{StringZ:aLund} and \texttt{StringZ:bLund} respectively. As was pointed in a previous work \cite{Amoroso:2018qga}, the $a$ and $b$ parameters are highly correlated (in the context of fits to measurements) since the former controls the high-$z$ tail, while the latter mainly controls the low-$z$ one, while most of the data is sensitive to the average $z$ which is given by a combination of the two which does not have a simple analytical expression. A new reparametrisation of the fragmentation function exists for which the $b$ parameter is replaced by $\langle z_\rho \rangle$ which represents the average longitudinal momentum fraction taken by mainly the $\rho$ mesons, {\it i.e.} 
\begin{equation}
\left<z_\rho\right> = \int_0^1 \mathrm{d}z \ z f(z,\left<m_{\perp\rho}\right>)~.
\label{eq:zrho}
\end{equation}

This equation is solved numerically for $b$ at the initialisation stage (which requires that the option \texttt{StringZ:deriveBLund = on}) where the following parameters:
\begin{eqnarray}
\left<m_{\perp\rho}\right>^2 & = & 
m^2_\rho + 2( \mbox{\texttt{StringPT:sigma}})^2~,
\\
\left<z_\rho\right> & = &\mbox{\texttt{StringZ:avgZLund}}~,
\end{eqnarray}
are used. 

 In the string-fragmentation picture, baryons are produced similarly to mesons, by allowing the breaking of strings by the production of diquarks-antidiquark pairs; these can be thought as bound states of two quarks (in an antiduqark) or two antiquarks (in a diquark). This basic picture entails a very strong (anti)correlation of the produced baryons in both flavour and phase space, due to the fact that a baryon originating from a diquark produced in a string breaking is associated with an anti-baryon as the new end-point of the residual string. Experimental measurements of $\Lambda^0 \bar{\Lambda}^0$ correlations by the \textsc{Opal} collaboration \cite{Abbiendi:1998ux} do not find such strong correlations. To address this, the so-called \emph{pop-corn} mechanism was suggested \cite{Andersson:1984af, Eden:1996xi}, in which baryons are produced such that the string breaking occurs with the production of one or more $q\bar{q}$ pairs ``in between'' the diquark-antidiquark pair. This picture enables the production of one or more mesons between two baryons and therefore decrease the correlations between them. Note that in the context of DM indirect searches, the correlation between baryons is not relevant as the produced protons travel for long distances before they reach the detector. In \textsc{Pythia8}, baryon production is controlled by an additional parameter denoted by $a_{\rm Diquark}$ such that the $a$ parameter in $f(z)$ is modified as $a \to a + a_{\rm Diquark}$. The extra parameter relevant for baryon production is denoted in \textsc{Pythia}~8 by \verb|StringZ:aExtraDiquark|. 


%%%%%%%%%%%%%%%%%%%%%%%%%%%%%%%%%%%%%%%%%%%%%%%%%%%%%%%%%%%%%%%%%%%%%%%%%%%
\subsection{The origin of positrons, antineutrinos and antiprotons in a generic dark-matter annihilation process}
%%%%%%%%%%%%%%%%%%%%%%%%%%%%%%%%%%%%%%%%%%%%%%%%%%%%%%%%%%%%%%%%%%%%%%%%%%%
After discussing the general features of antiparticle production in a generic dark-matter annihilation process in \textsc{Pythia}~8, we turn into a detailed investigation of the origin of these particle species. For this task, we consider a generic dark matter with mass of $1000$ GeV and focus on four annihilation channels: $q\bar{q}:~q=u,d,s,b$, $t\bar{t}$, $VV:~V=W,Z$ and $HH$. The reason is that at this mass value, the universality of the fragmentation function implies that all these annihilation channels have approximately the same features. In  the following we assume that $\sigma(\chi\chi \to WW) = \sigma(\chi\chi \to ZZ)$ without any loss of generality.

\begin{figure}[!t]
\centering
\includegraphics[width=0.9\linewidth]{Figures/Source-ElectronAntineutrinos-MDM-1000GeV.pdf}
\caption{The mean contribution to $\bar{\nu}_e$ production in DM annihilation with $M_{\chi} = 10000~{\rm GeV}$. From the left to the right we show the $q\bar{q}$, $t\bar{t}$, $VV$ and $HH$ channels. One shows $\bar{\nu}_e$ produced from the decay of muons (red), $K_L$ (turquoise), $n, \bar{n}$ (olive), $D^0$ (dark blue), $D^+$ (violet), $B^0$ (hot pink), $Z^0$ (light blue) and $W^\pm$ (cyan).}
\label{fig:origin:nue} 
\end{figure}

\begin{figure}[!t]
\centering
\includegraphics[width=0.9\linewidth]{Figures/Source-MuonAntineutrinos-MDM-1000GeV.pdf}
\caption{Same as figure \ref{fig:origin:nue} but for $\bar{\nu}_\mu$. Here, $\bar{\nu}_\mu$ are produced from the decay of muons (red), $\pi^\pm$ (turquoise), $D^0$ (olive), $B^+$ (dark blue), $Z^0$ (purple) and $W^\pm$ (hot pink).}
\label{fig:origin:num}
\end{figure}

\begin{figure}[!h]
\centering
\includegraphics[width=0.9\linewidth]{Figures/Source-TauAntineutrinos-MDM-1000GeV.pdf}
\caption{Same as figure \ref{fig:origin:nue} but for $\bar{\nu}_\tau$. The $\bar{\nu}_\tau$ produced from the decay of $\tau^\pm$, $B^+$, $B_s^0$, $B^0$, $D_s^+$, $W^\pm$ and $Z^0$ are shown in red, turquoise, olive, dark blue, purple, hot pink, and cyan respectively.}
\label{fig:origin:nuta}
\end{figure}



%%%%%%%%%%%%%%%%%%%%%%%%%%%%%%%%%%%%
\subsubsection{Antineutrinos}
%%%%%%%%%%%%%%%%%%%%%%%%%%%%%%%%%%%%

We start with the spectra of antineutrinos and we split them into electron antineutrinos ($\bar{\nu}_e$), muon antineutrinos ($\bar{\nu}_\mu$) and tau antineutrinos ($\bar{\nu}_\tau$) and they are shown in figures \ref{fig:origin:nue}--\ref{fig:origin:nuta}. 

First, most of the $\bar{\nu}_e$ are coming from the decay of $\mu^\pm$ in $\mu^- \to e^- \bar{\nu}_e \nu_\mu$. The contribution of the muons to the rate of $\bar{\nu}_e$ is $90\%$ irrespective of the annihilation channel. This is followed by the contribution of (anti)-neutrons through $n \to p e^- \bar{\nu}_e$ which is about $4\%$--$5\%$ depending on the annihilation channel. The other contributions are small (below $\simeq 2\%$); one note among others $K_L$, $D^0$, $D^+$, $B^0$, $W^- \to e^- \bar{\nu}_e$ and $Z^0 \to \nu_e \bar{\nu}_e$. The last two contributions ($Z\to \nu_e$ and $W^+ \to \nu_e$) are possible in the $t\bar{t}$, $VV$ and $HH$ channels. We note that most of the muons that leads to the $\bar{\nu}_e$ are coming from the decays of charged pions, {\it i.e.}, $\pi^- \to \mu^- \bar{\nu}_\mu$. Charged pions are produced in abundance through fragmentation of quarks/gluons and/or decay of heavier hadrons \cite{Amoroso:2018qga}. Therefore, one can find a direct connection between the modeling of gamma rays and electron antineutrinos. 

The situation is slightly different for $\bar{\nu}_\mu$ where can see that both $\mu^+$ and $\pi^+$ contribute to about $47$--$50\%$ of the total rate while the other sources give negligible contributions (see figure \ref{fig:origin:num}). Similar to $\bar{\nu}_e$, most of the muons that give rise to $\bar{\nu}_\mu$ are coming from the decay of $\pi^+$. Finally, one note that $\bar{\nu}_\tau$ have extremely different type of sources -- the total rate of $\bar{\nu}_\tau$ is negligibly small as compared to $\bar{\nu}_e$ and $\bar{\nu}_\mu$ --. $\bar{\nu}_\tau$ are produced from $\tau^+$, $D_s^0$, $B_s^0$, $B^+$, $B^0$, $W^\pm$ and $Z^0$ with average contributions of about $1\%$--$50\%$ depending on the annihilation channel (see figure \ref{fig:origin:nuta}).

\begin{figure}[!t]
\centering
\includegraphics[width=0.95\linewidth]{Figures/Source-Positrons-MDM-1000GeV.pdf}
\caption{Same as figure \ref{fig:origin:nue} but for $e^+$. Here, $e^+$ are produced from the decay of muons (red), $K_L$ (turquoise), $n$ (olive), $D^0$ (dark blue), $D^+$ (purple), $B^0$ (hot pink), $Z^0$ (light blue) and $W^\pm$ (cyan).}
\label{fig:origin:e}
\end{figure}

%%%%%%%%%%%%%%%%%%%%%%%%%%%%%%%%%
\subsubsection{Positrons}
%%%%%%%%%%%%%%%%%%%%%%%%%%%%%%%%%

In figure \ref{fig:origin:e}, we show the mean contribution of different particles to the rate of $e^+$ in generic DM annihilation for DM mass of $1000$~GeV. We can see that, similarly to $\bar{\nu}_e$ (see figure \ref{fig:origin:nue}), most of $e^+$ are produced from the decay of $\mu^+$ independently of the annihilation channel. The other contributions are similar to the case of $\bar{\nu}_e$ production. 

%%%%%%%%%%%%%%%%%%%%%%%%%%%%%%%%%%%
\subsubsection{Antiprotons}
%%%%%%%%%%%%%%%%%%%%%%%%%%%%%%%%%%%

\begin{figure}[!t]
    \centering
    \includegraphics[width=0.95\linewidth]{Figures/Source-Antiprotons-MDM-1000GeV.pdf}
    \caption{Same as in figure \ref{fig:origin:nue} but for $\bar{p}$. Here, one shows $p/\bar{p}$ produced from QCD fragmentation (red), decay of $\Lambda^0$ (turquoise), $n$ (olive), $\Sigma^+$ (dark blue), $\Delta^+$ (purple), $\Delta^{++}$ (hot pink) and $\Delta^0$ (light blue).}
    \label{fig:origin:p}
\end{figure}

In general (anti-)protons can be split into two categories: \emph{(i)} primary (anti-)protons produced directly from the string fragmentation of quarks and gluons and \emph{(ii)} secondary (anti-)protons produced from the decay of heavier baryons. In figure \ref{fig:origin:p}, we display the mean contributions to the (anti-)proton yields in DM annihilation for $M_\chi = 1000$ GeV. We can see that about $22\%$ of the produced (anti)protons are emerging from $q/g$-fragmentation. On the other hand, the dominant large fraction ($\simeq 50\%$) of (anti)-protons is originated from the decay of (anti-)neutrons in decays $\bar{n} \to \bar{p} e^+ \nu_e$ as can be clearly seen in figure \ref{fig:origin:p}. This is followed by the contribution of five baryons which mainly contribute to secondary protons: $\Lambda^0$, $\Sigma^+$ and $\Delta(1232)$ -- summing contributions from $\Delta^+, \Delta^{0}$ and $\Delta^{++}$ -- contribute to about $2\%$--$7\%$ of the produced (anti)-protons respectively. \\

We conclude that the spectra of antiparticles are correlated to each other with the dominant hadrons to be modeled properly are charged pions, protons and $\Lambda^0$ baryons. We note that in ref. \cite{Amoroso:2018qga}, we have discussed in detail the modeling of pion spectra at LEP and therefore we will not discuss it anymore here (although we do include these measurements in our fits). In the next section, we discuss in detail the measurements of baryon spectra at LEP and assess the agreement between the corresponding theory predictions and experimental measurements for three multi-purpose MC event generators.  

\newcommand{\wy}[1]{{\leavevmode\color{brown}#1}} % by W. Yao
\newcommand{\geva}[1]{{\leavevmode\color{cyan}#1}}
\newcommand{\phg}[1]{{\leavevmode\color{orange}#1}}

\begin{document}
%\onecolumn




\title{Electron-photon Chern number in cavity-embedded 2D moir\'e materials}
\author{Danh-Phuong Nguyen}
\affiliation{Universit\'{e} Paris Cit\'e, CNRS, Mat\'{e}riaux et Ph\'{e}nom\`{e}nes Quantiques, 75013 Paris, France}
\author{Geva Arwas}
\affiliation{Universit\'{e} Paris Cit\'e, CNRS, Mat\'{e}riaux et Ph\'{e}nom\`{e}nes Quantiques, 75013 Paris, France}
\author{Zuzhang Lin}
\affiliation{Department of Physics, The University of Hong Kong, Hong Kong, China}
\affiliation{HKU-UCAS Joint Institute of Theoretical and Computational Physics at Hong Kong, China} 
\author{Wang Yao}
\affiliation{Department of Physics, The University of Hong Kong, Hong Kong, China}
\affiliation{HKU-UCAS Joint Institute of Theoretical and Computational Physics at Hong Kong, China}
\author{ Cristiano Ciuti}
\affiliation{Universit\'{e} Paris Cit\'e, CNRS, Mat\'{e}riaux et Ph\'{e}nom\`{e}nes Quantiques, 75013 Paris, France}




\begin{abstract}
\noindent 

We explore theoretically how the topological properties of 2D materials can be manipulated by cavity quantum electromagnetic fields for both resonant and off-resonant electron-photon coupling, with a focus on van der Waals moir\'{e} superlattices. We investigate an electron-photon topological Chern number for the cavity-dressed energy minibands that is well defined for any degree of hy- bridization of the electron and photon states. While an off-resonant cavity mode can renormalize electronic topological phases that exist without cavity coupling, we show that when the cavity mode is resonant to electronic miniband transitions, new and higher electron-photon Chern numbers can emerge.



\end{abstract}
\date{\today}
\maketitle
%\section{Introduction}
In recent years van der Waals heterostructures combining atomic-thin layers of 2D materials such as graphene or transition metal dichalcogenides (TMD) have attracted a great deal of interest \cite {Geim2013,Ajayan2016,Novoselov2016,Song2018,valosOvando2019,Yankowitz2019}. Indeed, these systems present rich and controllable physical properties already at the single-particle level due to multiple quantum degrees of freedom, namely the electron spin, the valley and layer pseudospins. This class of 2D materials exhibits a wide variety of interesting electronic properties including semimetallic, semiconducting, superconducting and magnetic phases. One prominent example is magic angle twisted bilayer graphene, exhibiting quasi-flat electronic bands \cite{Bistritzer2011} and remarkable superconducting properties \cite{Cao2018}. Another notable class of moir\'{e} heterostructures is based on TMD materials, which, thanks to their semiconductor properties, have particularly interesting optical properties \cite{Seyler2019,Tran2019,Jin2019}.

A growing interest is emerging for the manipulation of materials with cavity vacuum fields \cite{FornDaz2019, FriskKockum2019,GarciaVidal2021,Schlawin2022}. In particular, metallic split-ring terahertz electromagnetic resonators are remarkable for their deeply sub-wavelength photon mode confinement \cite{Keller2017,Scalari2012, ParaviciniBagliani2018} with mode volumes that can be as small as $10^{-10}\lambda_0^3$, being $\lambda_0$ the free-space electromagnetic wavelength corresponding to the resonator frequency.  
Studies on GaAs 2D electron gases have shown that electronic transport in mesoscopic quantum Hall bars can be greatly modified by the coupling to electromagnetic resonators even without illumination \cite{Appugliese2022}, as a result of  cavity-mediated electron hopping \cite{Ciuti2021,Arwas2023}, which results in a breakdown of the Hall resistance quantization associated to the topological properties of the electronic Landau states.

Recent theoretical works have investigated how to exploit cavity QED to control topological properties of systems, such as 1D tight-binding chains described by the SSH model \cite{Dmytruk2022}. Concerning 2D systems, a recent work has studied 2D bulk materials \cite{Wang2019,Li2022} where the standard electron Chern number is computed by considering an effective electronic Hamiltonian obtained by adiabatic elimination of the photon degrees of freedom. A letter exploring single-sheet graphene ribbons \cite{Guerci_PRL_2020} has studied  electron Chern numbers computed once the cavity field is approximated in a classical coherent state. Another investigation has explored a generic single-electron problem in the ultrastrong light-matter coupling regime \cite{Masuki2023} and focused on the topological control in the configuration where the cavity photon mode frequency is much larger than the relevant electronic transition frequencies. A key point that has not been addressed is the behavior of genuine electron-photon topological invariants that are associated to the interacting quantum electron-photon system when the photon mode is resonant to electronic transitions.  In the past, this has been done only for classical exciton-polariton normal modes where a bosonic exciton field is strongly coupled to a cavity photon field \cite{Karzig_PRX_2015,Ozawa_RMP_2019} (the bare cavity photon and exciton bands are topologically trivial, but the hybrid light-matter polariton bands      are not). However, for a fermionic particle coupled to a quantized cavity field, to the best of our knowledge, the study of  electron-photon topological invariants for the resonant light-matter coupling has been overlooked.


In this letter, we explore the properties of electron-photon Chern numbers, focusing on cavity-embedded 2D moir\'e materials. We investigate different regimes of coupling (off-resonant versus resonant cavity mode), for different cavity geometries (mode with in-plane or out-of-plane polarization) and explore the topological transitions characterized by such an electron-photon Chern number with realistic values for state-of-the-art split-ring resonators and TMD materials. We show that in the case of resonant electron-photon coupling, new topological phases and high Chern numbers can emerge.

\begin{figure}
\centering
\includegraphics[width = 0.95\hsize]{manu_fig_1_1.pdf} \\ 
\includegraphics[width = 0.49\hsize]{manu_fig_1_2.pdf}
\includegraphics[width = 0.49\hsize]{manu_fig_1_3.pdf}
\caption{Top panel: sketch of a twisted bilayer system with its moir\'e pattern inside the gap region of a split-ring resonator. Bottom panel: on the left, the first Brillouin zone with Dirac points of the bottom layer and corresponding wavevectors; on the right, the Brillouin zones of both the bottom and rotated top layer together with the moir\'{e} mini Brillouin zone (shown in red) for the corresponding moir\'e superlattice.}\label{fig:TMD-mBZ}
 
\end{figure}

{\it Cavity QED Hamiltonian ---} Let us consider a bilayer moir\'e system consisting of two twisted TMD layers embedded in a single-mode resonator (see the sketch in Fig. \ref{fig:TMD-mBZ}). Each of them is a honeycomb lattice defined by two primitive translation vectors $\mathbf{a}_1 = a_0\sqrt{3}/2(1, \sqrt{3}, 0)$ and $\mathbf{a}_2 = a_0\sqrt{3}/2(-1, \sqrt{3}, 0)$, where $a_0$ is the monolayer TMD lattice constant. We consider in-plane parallel stacking (R-stacking) with the layer on top rotated by a small angle $\theta$ in order to create a long wavelength moir\'{e} pattern. The distance between two layers is $d$. The moir\'{e} unit cell is defined by $\mathbf{L}_{i = 1,2} = \left[\mathbb{1} - R(\theta)^{-1}\right]^{-1}\mathbf{a}_i$, where $\mathbb{1}$ and $R(\theta)$ are respectively the identity and the rotation matrix corresponding to the rotation angle $\theta$ about the $z$-axis. We denote the moir\'{e} lattice constant $\left|\mathbf{L}_i\right|$ as $a_M$, where for small angles $a_{M} \simeq a_0/\theta$. In the following, for numerical applications we will use parameters of the $\text{MoTe}_2$ system \cite{Wu2019}. For the photonic part, thanks to the approximately flat mode inside the capacitive gap of complementary split-ring resonators \cite{Appugliese2022}, we will consider a single-mode cavity with a spatially homogeneous field described by the vector potential $\hat{\mathbf{A}} = A_0 \, \mathbf{u}(\hat{a} +\hat{a}^{\dagger})$ with $\hat{a}^{\dagger}$ ($\hat{a}$) the photonic creation (annihilation) operator. The vacuum field amplitude is $A_{\rm 0} = \sqrt{\hbar/(2\omega_c\epsilon_0V_{mode}})$ with $\omega_c$ the mode frequency, $V_{mode} = \eta\lambda_c^3$ the effective mode volume related to the compression factor $\eta$ and $\lambda_c$ the wavelength in free space that corresponds to $\omega_c$. The orientation of the cavity mode polarization is described by the unit vector $\mathbf{u} = (u_x, u_y, u_z)$.


Our treatment is based on the four-band continuum model for small angle twisted bilayer TMD \cite{Wu2019,Zhai2020}. The cavity QED Hamiltonian can be written as
\begin{equation}
\label{eq:Haldane-H4}
\hat{H} = \hbar\omega_c\hat{a}^{\dagger}\hat{a} + \begin{pmatrix}
        \hat{H}_{t} & 0 \\ 0 & \hat{H}_{b}
    \end{pmatrix} + \begin{pmatrix}
        \hat{V}_t & \hat{U} \\ \hat{U}^{\dagger} & \hat{V}_b
    \end{pmatrix},
\end{equation}
where the $2\cross2$ operator matrix $\hat{H}_{t}$ ($\hat{H}_{b}$)  corresponds to the conduction and valence bands of the top (bottom) layer. The moir\'{e} potentials are given by the last term.
%
The light-matter coupling is introduced via Peierls factors \footnote{See Supplementary Material for additional details about the cavity QED Hamiltonian, the purity of electronic reduced density matrix, the spectral functions and spatial distribution of edge states.} in the intralayer ($\hat{H}_{t}$, $\hat{H}_{b}$) and interlayer ($\hat{U}$) hopping terms.
We quantify the interaction strength by the dimensionless constant $g = eA_0 a_0/\hbar$.
For $g \ll 1$ and in the low-energy sector, each diagonal block matrix $\nu = t, b$ and interlayer coupling $\hat{U}$ can be approximated as follows:
\begin{equation}
\label{eq:TMD-each H}
\begin{aligned}
    &\hat{H}_{\nu} = \Delta \sigma_z + v_F\boldsymbol{\sigma}_{xy} \cdot \left[\mathbf{\hat{p}} -\hbar\mathbf{K}_{\nu} + e\mathbf{A}^{(xy)}(\hat{a}+\hat{a}^{\dagger})\right], \\
    &\hat{U} = e^{-i\frac{eA^{(z)}d}{\hbar}(\hat{a}+\hat{a}^{\dagger})}\hat{U}_0,
    \end{aligned}
\end{equation}
with $\mathbf{K}_{t} = \boldsymbol{\kappa}_-$, $ \mathbf{K}_{b} = \boldsymbol{\kappa}_+$ shown in Fig. \ref{fig:TMD-mBZ}, $\boldsymbol{\sigma}_{xy} = (\sigma_x, \sigma_y, 0)$ a vector of Pauli matrices, $\mathbf{A}^{(xy)}$ is the in-plane projection of $A_0 {\mathbf u}$ and $A^{(z)}$ is its $z$-component. Note that the interlayer distance $d$ is much  larger than the in-plane lattice constant $a_0$. Hence, the argument of the Peierl's  exponential for the $U$ term is much larger than its $t$ counterpart (in-plane coupling). We will consider a regime of realistic parameters, where it turns out that the exponential of the $t$-term can be linearized, while the out-of-plane Peierls term must be kept in exponential form. The moir\'{e} potentials $\hat{V}_{\nu}$ and $\hat{U}_0$ are the same as the ones in \cite{Wu2019,Zhai2020}. Due to the large energy gap between the conduction and valence band in the TMD material, we can restrict our description to the latter, for which non-trivial topological properties were studied \cite{Wu2019,yu_giant_2020} in the absence of a cavity field. Finally, we represent the Hamiltonian in the hole picture. The corresponding Hamiltonian reads:
\begin{widetext}
\begin{equation}
\label{eq:TMD-H1}
\begin{aligned}
    \hat{\mathcal{H}} = \hbar \omega_c \hat{a}^{\dagger}\hat{a}
    +\frac{1}{2m^{\star}}\begin{pmatrix}
   (\hat{\mathbf{p}} + \hbar\boldsymbol{\kappa}_-  - e\mathbf{A}^{(xy)}(\hat{a}+\hat{a}^{\dagger}))^2 & 0 \\
    0 & (\hat{\mathbf{p}} + \hbar\boldsymbol{\kappa}_+ -e\mathbf{A}^{(xy)}(\hat{a}+\hat{a}^{\dagger}))^2
    \end{pmatrix} 
    -  
    \begin{pmatrix}
    \hat{V}^v_t & \hat{U}_0^{v\dagger} \\ \hat{U}_0^{v} & \hat{V}^v_b    \end{pmatrix} - i\frac{\omega_ceA^{(z)}d}{2}(\hat{a}-\hat{a}^{\dagger})\tau_z
\end{aligned}
\end{equation}
\end{widetext}
where $m^{\star} = \Delta/v_F^2$ is the effective mass of the valence band, $\tau_z$ is the $z$-axis Pauli matrix with respect to the layer pseudospin. We have considered minimal coupling in the Coulomb gauge for the in-plane motion, while the dipole gauge resulting from a Power-Zienau-Woolley (PZW) transformation is used for the out-of-plane part. The expression for the moir\'{e} potentials $\hat{V}_t^v$, $\hat{V}_b^v$, $\hat{U}_0^v$ is the same as in \cite{Wu2019,Zhai2020} and is reported in the Supplemental Material. To account for the interlayer bias $V_z$ (that breaks the symmetry between $\boldsymbol{\kappa}_{+}$ and $\boldsymbol{\kappa}_-$ \cite{Wu2019}), the term $-V_z/2 \cross \tau_z$ has to be added to the Hamiltonian in Eq.  (\ref{eq:TMD-H1}). As the moir\'{e} potential is periodic with respect to $\mathbf{L}_{i=1,2}$ translations, the electronic part can be block-diagonalized in momentum space (Bloch theorem). Each block $\mathbf{k}$ belongs to the moir\'{e} mini Brillouin zone (mBZ) shown in Fig. \ref{fig:TMD-mBZ}, spanned by moir\'{e} reciprocal vectors $\mathbf{G}_1$ and $\mathbf{G}_2$ satisfying $\mathbf{G}_i\cdot\mathbf{L}_j = 2\pi\delta_{ij}$. 
%\section{Electron-photon Chern number}
\begin{figure}[t!]
\includegraphics[width = 0.95\hsize]{manu_fig_2_1.pdf}
\includegraphics[width = 0.95\hsize]{manu_fig_2_2.pdf}
  \caption{Electron-photon Chern numbers of the first three (top panel) and two (bottom panel) moir\'{e} minibands of the twisted TMD bilayer system (MoTe$_2$ parameters). The dashed lines depict the phase boundary  predicted by an effective electronic Hamiltonian obtained by adiabatically eliminating the photonic degrees of freedom \cite{Arwas2023} and calculating the electron Chern number. Top panel: Chern numbers versus the twisting angle $\theta$ and the dimensionless cavity coupling $g$ (see definition in the text), with no inter-layer bias ($V_z = 0$). Bottom panel: Chern numbers versus $V_z$ and the dimensionless cavity coupling $g$ for a fixed angle $\theta = 1.2^{\circ}$. Parameters: cavity photon energy $\hbar\omega_c = 20$ meV (top panel) and $6$ meV (bottom panel), cavity mode polarization vector $\mathbf{u} = (1,0,0)$.}
\label{fig:TMD-phase diagram}
\end{figure}

{\it Definition of the electron-photon Chern number ---} 
Let us consider a single fermion interacting with a cavity quantized field. This is equivalent to injecting a fermion into empty minibands.  If the fermion wavevector $\mathbf{k}$ is a good quantum number, then we can diagonalize the Hamiltonian  $\hat{\mathcal{H}}_{\mathbf{k}} = e^{-i\mathbf{k}\cdot\hat{\mathbf{r}}}\hat{\mathcal{H}}e^{i\mathbf{k}\cdot\hat{\mathbf{r}}}$ and obtain electron-photon eigenstates of the form $\vert \Psi^{{\mathrm{(e-p})}}_{n\mathbf{k}}\rangle$ with  corresponding electron-photon energy bands $\mathcal{E}^{{\mathrm{(e-p})}}_{n\mathbf{k}}$. Given the form of the system eigenstates, we can introduce the following electron-photon Chern number:
%\begin{widetext}
\begin{figure}[t!]
\includegraphics[scale = 0.23]{manu_fig_3_1.pdf}
\includegraphics[scale = 0.23]{manu_fig_3_2.pdf}
  \caption{Electron-photon energy-momentum dispersions $\mathcal{E}^{(e-p)}_{n \mathbf{k}}$ for $g = 0.0$ (left panel) and $0.05$ (right panel) for a ribbon geometry (periodic boundary conditions along the long direction). Parameters: cavity photon energy $\hbar\omega_c = 6$ meV, $V_z = 0.5$ meV, $\theta = 1.2^{\circ}$, ribbon width $D_1/a_M = 10$ and cavity mode polarization vector $\mathbf{u} = (1,0,0)$. Here, the photonic component of the electron-photon eigenstates is small (see text).}
\label{fig:TMD-off res band}
\end{figure}
\begin{equation}
    \mathcal{C}^{{\mathrm{(e-p})}}_n = \int\frac{d^2k}{2\pi}i\sum_{\mu,\nu}\epsilon_{\mu\nu}\braket{\smash{\partial_{k_{\mu}}\Psi^{{\mathrm{(e-p})}}_{n\mathbf{k}}}}{\smash{\partial_{k_{\nu}}\Psi^{{\mathrm{(e-p})}}_{n\mathbf{k}}}},
    \label{eq:Cn}
    \end{equation}
%\end{widetext}
where $\epsilon_{\mu\nu}$ is the two-dimensional Levi-Civita tensor. Importantly, the electron-photon Chern number in Eq. (\ref{eq:Cn}) is well defined for any arbitrary hybridization between the single electron and the cavity quantum field. If we work in the hole picture, the particle-hole transformation results in a simple change of sign of such Chern number. Numerically the Chern number in Eq. (\ref{eq:Cn}) has been calculated by numerical diagonalization of the Hamiltonian (\ref{eq:TMD-H1}) and using the technique reported in Ref. \cite{Zhao2020}.

%\section{Results and discussion}
{\it Results and discussions ---} In what follows, we
use the electron-photon Chern number to investigate topological properties of cavity-embedded moir\'e systems, focusing on twisted MoTe$_{2}$ bilayers. We first consider the scenario of high photon frequency, where the photon is off-resonant with respect to the relevant miniband electronic transitions. In Fig. \ref{fig:TMD-phase diagram}, we report the electron-photon topological Chern numbers associated to the first three moir\'e minibands of the twisted bilayer TMD system. Here, we consider a cavity mode with in-plane polarization $\mathbf{u} = (1,0,0)$. For $g = 0$ (no cavity coupling), the system is known to exhibit a topological transition as a function of the twisting angle $\theta$ (top panel) and as a function of the inter-layer bias $V_z$ (bottom panel). In the top panel we consider the case with no bias ($V_z = 0$) for which a topological transition occurs by increasing the twisting angle $\theta$ above a critical value ($\approx 1.75^{\circ}$ at zero $g$), with Chern numbers changing from $(+1,-1,0)$ to $(+1,+1,-2)$. The bottom panel corresponds to a situation with a fixed angle $\theta = 1.2^{\circ}$ for which a transition from topologically nontrivial Chern numbers $(+1,-1,0)$ to the trivial $(0,0,0)$  is achieved when $V_z$ is increased above a critical value ($\approx 0.63$ meV at zero $g$). 
\begin{figure}[t!]
\includegraphics[width = 0.9\hsize]{manu_fig_4.pdf}
  \caption{Electron-photon Chern numbers of the first four moir\'{e} minibands for the considered cavity-embedded twisted TMD bilayer system. Note that this topological diagram cannot be predicted by the effective electronic model used in Fig. \ref{fig:TMD-phase diagram}. Other parameters: cavity photon energy $\hbar\omega_c = 10$ meV, $V_z = 0$ meV, and cavity mode polarization $\mathbf{u} = (0,0,1)$.}
\label{fig:TMD-on phase diagram}
\end{figure}
In both situations considered in Fig. \ref{fig:TMD-phase diagram}, a finite cavity coupling modifies the transition boundary significantly for a range of dimensionless coupling $g$, which is accessible with deeply sub-wavelength resonators. The energy-momentum dispersions with and without the cavity are compared for $V_z = 0.5$ meV in Fig. \ref{fig:TMD-off res band} for a ribbon geometry (periodic boundary conditions along the long direction). The left panel shows the presence of topological edge states that cross in energy for $g=0$, while the right panel shows the opening of an edge gap in the presence of a cavity with $g = 0.05$. In other words, Fig. \ref{fig:TMD-off res band} shows the consequence of the topological transition with increasing cavity coupling $g$ depicted in the diagram of the bottom panel in Fig. \ref{fig:TMD-phase diagram}. 

Note that by tracing out the photonic degrees of freedom, we can obtain an electronic reduced density matrix, namely $\hat{\rho}_{n\mathbf{k}} = \text{Tr}_{\text{phot}}\left(\ket{\smash{\Psi_{n\mathbf{k}}^{(e-p)}}}\bra{\smash{\Psi_{n\mathbf{k}}^{(e-p)}}}\right)$. The electronic purity of such reduced density matrix is defined as ${\mathbb P}_{n\mathbf{k}} = \text{Tr}_{\text{el}}(\hat{\rho}_{n\mathbf{k}}^2)$ \cite{CohenTannoudji2020} and is equal to $1$ for a purely electronic state. For $g = 0.1$, the minimum of the electronic purity (the purity depends on $\mathbf{k}$) is about $75\%$ for the top panel of Fig. \ref{fig:TMD-off res band} and $88\%$ for the bottom one. In such configuration, one might describe the system with an effective electronic Hamiltonian \cite{Arwas2023,Dmytruk2022,Wang2019,Li2022,Guerci_PRL_2020}. Indeed, we as shown in of Fig. \ref{fig:TMD-phase diagram} the diagrams are qualitatively reproduced by an effective electronic Hamiltonian approach (see Supplementary Material).
However our electron-photon Chern number introduced in Eq. (\ref{eq:Cn}) and based on the exact light-matter energies $\mathcal{E}^{(e-p)}_{n \mathbf{k}}$ is defined also for low electronic purity, i.e., for any arbitrary light-matter hybridization. 

\begin{figure}[t!]
\includegraphics[scale=0.235]{manu_fig_5_1.pdf}
\includegraphics[scale=0.235]{manu_fig_5_2.pdf}\\
\includegraphics[scale=0.235]{manu_fig_5_3.pdf}
\includegraphics[scale=0.235]{manu_fig_5_4.pdf}
  \caption{Top panels: electron-photon miniband dispersions $\mathcal{E}^{(e-p)}_{n \mathbf{k}}$ for $g_{\perp} = 0$ (left panel) and $0.35$ (right panel), where three edge states emerge (thicker red lines). Bottom panels: electronic purity (left panel) and electronic spectral function (right panel, normalized by its maximum value)  for $g_{\perp} = 0.35$. We have considered a ribbon geometry with width $D_1/a_M = 10$ and periodic boundary conditions along the long direction. Other parameters: cavity photon energy $\hbar\omega_c = 10$ meV, $V_z = 0$ meV, $\theta = 1.8^{\circ}$, cavity mode polarization $\mathbf{u} = (0,0,1)$. }
\label{fig:res-band}
\end{figure}
\label{discuss}


\label{res}
Let us now consider the situation where the cavity mode has a relatively low frequency and is resonant with electronic miniband transitions. Note that we consider here out-of-plane cavity polarization $\mathbf{u} = (0,0,1)$ and the corresponding dimensionless coupling constant is defined as $g_{\perp} = eA_0d/\hbar$. The corresponding results are depicted in Fig. \ref{fig:TMD-on phase diagram} and \ref{fig:res-band}. In the non-interacting case ($g_{\perp}=0$), the photon energy creates replicas of the electronic minibands with non-zero photon numbers, as shown in Fig. \ref{fig:res-band}. The bare electron-photon spectrum becomes degenerate when a photon replica crosses the original electronic miniband with zero photons.
However, a finite coupling lifts such degeneracies, leading to new electron-photon energy minibands for which we can compute the electron-photon Chern number.
Fig. \ref{fig:TMD-on phase diagram} shows the topological diagram characterized by such an electron-photon Chern numbers for the first four minibands. The cavity photon energy is resonant to the transition  between the first and third electronic minibands, depicted in Fig. \ref{fig:res-band}.  The bulk-edge correspondence for the electron-photon Chern number is  satisfied: indeed, we observe three edge states shown by the thicker red line in Fig. \ref{fig:res-band}. These states have low electronic purity (as low as 50\%) and cannot be captured by an effective electronic Hamiltonian where the photon degrees of freedom are eliminated. Moreover, such high Chern numbers, created by the hybridization with the cavity photon replicas, are absent in the bare electronic system. These multiple new edge states corresponding to high Chern numbers can be observed by measuring the electronic spectral function via Angle-Resolved Photoemission Spectroscopy (ARPES) \cite{Noguchi2021}, Scanning Tunneling Microscopy (STM) \cite{Tao2011} or Microwave Impedance Microscopy \cite{Barber2021}.
%\subsection{Discussion}

{\it Conclusions ---} In this letter, we explored the new topological phases characterized  by the electron-photon Chern number, a topological invariant defined in terms of the exact eigenstates for the Hamiltonian describing a fermionic particle coupled to a quantized electromagnetic field. This is a topological invariant for any arbitrary hybridization of the electron-photon eigenstates. 
Note that electron-photon Chern numbers can also defined for disordered and non-homogeneous systems \cite{Prodan2010,Bianco2011,Zhang2013,Caio2019}, so the present topological approach can also be generalized to situations where the cavity mode is not spatially homogeneous or in the presence of electronic disorder. The results of this letter are exact when a single fermion is injected in empty minibands. A future and intriguing problem to explore is how these topological states evolve when such electron-photon states are partially filled.  





\acknowledgements
{We acknowledges financial support from the French agency ANR through the project NOMOS (ANR-18-CE24-0026), and TRIANGLE (ANR-20-CE47-0011), CaVdW under the ANR/RGC Joint Research Scheme sponsored by ANR and RGC of Hong Kong SAR China (ANR-21-CE30-0056-01, A-HKU705/21), and from the Israeli Council for Higher Education - VATAT. 
Z.L. and W.Y. also acknowledge support by RGC of Hong Kong SAR (HKU SRFS2122-7S05, AoE/P-701/20).
}

%% CVPR 2023 Paper Template
% based on the CVPR template provided by Ming-Ming Cheng (https://github.com/MCG-NKU/CVPR_Template)
% modified and extended by Stefan Roth (stefan.roth@NOSPAMtu-darmstadt.de)

\documentclass[10pt,twocolumn,letterpaper]{article}

%%%%%%%%% PAPER TYPE  - PLEASE UPDATE FOR FINAL VERSION
\usepackage{cvpr}      % To produce the REVIEW version
%\usepackage{cvpr}              % To produce the CAMERA-READY version
%\usepackage[pagenumbers]{cvpr} % To force page numbers, e.g. for an arXiv version

% Include other packages here, before hyperref.
\usepackage{graphicx}
\usepackage{amsmath}
\usepackage{amssymb}
\usepackage{booktabs}
\usepackage[ruled,vlined]{algorithm2e}
\usepackage{color}
\usepackage{amsthm}
\newtheorem{theorem}{Theorem}
\newtheorem{proposition}[theorem]{Proposition}
\newtheorem{lmm}{Theorem}
\newtheorem{rmk}{Theorem}
\newtheorem{lemma}[lmm]{Lemma}
\newtheorem{remark}[rmk]{Remark}
\setlength\parskip{0pt}
\newcommand{\PyComment}[1]{\ttfamily\textcolor{commentcolor}{\# #1}}  % add a "#" before the input text "#1"
\newcommand{\PyCode}[1]{\ttfamily\textcolor{black}{#1}} % \ttfamily is the code font
\definecolor{commentcolor}{RGB}{110,154,155}
% It is strongly recommended to use hyperref, especially for the review version.
% hyperref with option pagebackref eases the reviewers' job.
% Please disable hyperref *only* if you encounter grave issues, e.g. with the
% file validation for the camera-ready version.
%
% If you comment hyperref and then uncomment it, you should delete
% ReviewTempalte.aux before re-running LaTeX.
% (Or just hit 'q' on the first LaTeX run, let it finish, and you
%  should be clear).
\usepackage[pagebackref,breaklinks,colorlinks]{hyperref}


% Support for easy cross-referencing
\usepackage[capitalize]{cleveref}
\crefname{section}{Sec.}{Secs.}
\Crefname{section}{Section}{Sections}
\Crefname{table}{Table}{Tables}
\crefname{table}{Tab.}{Tabs.}


%%%%%%%%% PAPER ID  - PLEASE UPDATE
\def\cvprPaperID{9799} % *** Enter the CVPR Paper ID here
\def\confName{CVPR}
\def\confYear{2023}


\begin{document}
	
	%%%%%%%%% TITLE - PLEASE UPDATE
	\title{Transforming  Radiance Field with Lipschitz Network for 
		
		Photorealistic 3D Scene Stylization
  
  \textemdash\textemdash CVPR 2023 Supplementary Material}
	
	 \author{%
 	Zicheng Zhang\textsuperscript{1}%\thanks{Work done during an internship at JD AI Research.}
	\quad
	Yinglu Liu\textsuperscript{2}
	\quad
	Congying Han\textsuperscript{1}
	\quad
        Yingwei Pan\textsuperscript{2}
        \quad
	Tiande Guo\textsuperscript{1}
	\quad
	Ting Yao\textsuperscript{2}
	\quad
	\\ 
	\textsuperscript{1}University of Chinese Academy of Sciences
	\quad
	\textsuperscript{2}JD AI Research
	\quad
	\vspace{.5em} 
	\\
	\tt\small zhangzicheng19@mails.ucas.ac.cn
	\quad
	liuyinglu1@jd.com
	\quad
	hancy@ucas.ac.cn
	\\
	\tt\small
     panyw.ustc@gmail.com
	\quad
	tdguo@ucas.ac.cn
	\quad
	tingyao.ustc@gmail.com
}
	\maketitle
	\appendix

	\section{Proofs}
    \begin{proposition}\label{proposition 1}
		Considering  $f(\boldsymbol{c}) = \boldsymbol{A}\boldsymbol{c}+\boldsymbol{b}$, $\boldsymbol{A}\in \mathbb{R}^{3\times3}$, $\boldsymbol{b}\in \mathbb{R}^{3\times1}$, if $\mathbf{F}_{app}' = f\circ\mathbf{F}_{app}$, $\sum_{i=1}^{T}w_{i} = 1$ and  $vrr(\mathbf{r}_1,\mathbf{r}_2; \mathbf{F})<\epsilon$, we have  $vrr(\mathbf{r}_1,\mathbf{r}_2; \mathbf{F}')<K\epsilon$, where  $K =\left\| \boldsymbol{A} \right\|_2$ is the Lipschitz constant of $f$. 
	\end{proposition}
	
	\begin{proof} 
	\begin{small}
	\begin{equation}\label{eq: vrr}
		\begin{aligned}
			vrr(\mathbf{r}_1,\mathbf{r}_2; \mathbf{F}') &= \left \| {C}(\mathbf{r}_1;\mathbf{F}') -{C}(\mathbf{r}_2;\mathbf{F}')  \right \| \\ 
			&=\left \| \sum\limits_{i=1}^{T} w^{\mathbf{r}_1}_{i}f(\boldsymbol{c}^{\mathbf{r}_1}_i) -\sum\limits_{i=1}^{T} w^{\mathbf{r}_2}_{i}f(\boldsymbol{c}^{\mathbf{r}_2}_i)  \right \|\\
			&= \left \| \sum\limits_{i=1}^{T} w^{\mathbf{r}_1}_{i}(\boldsymbol{A}\boldsymbol{c}^{\mathbf{r}_1}_i+\boldsymbol{b}) -\sum\limits_{i=1}^{T} w^{\mathbf{r}_2}_{i}(\boldsymbol{A}\boldsymbol{c}^{\mathbf{r}_2}_i+\boldsymbol{b})  \right \| \\
			&= \left \| \sum\limits_{i=1}^{T} w^{\mathbf{r}_1}_{i}\boldsymbol{A}\boldsymbol{c}^{\mathbf{r}_1}_i -\sum\limits_{i=1}^{T} w^{\mathbf{r}_2}_{i}\boldsymbol{A}\boldsymbol{c}^{\mathbf{r}_2}_i  \right \| \\
			&= \left \| \boldsymbol{A} \left( \sum\limits_{i=1}^{T} w^{\mathbf{r}_1}_{i}\boldsymbol{c}^{\mathbf{r}_1}_i -\sum\limits_{i=1}^{T} w^{\mathbf{r}_2}_{i}\boldsymbol{c}^{\mathbf{r}_2}_i \right) \right \| \\
			&\leq \left\| \boldsymbol{A} \right\| \left \|  \sum\limits_{i=1}^{T} w^{\mathbf{r}_1}_{i}\boldsymbol{c}^{\mathbf{r}_1}_i -\sum\limits_{i=1}^{T} w^{\mathbf{r}_2}_{i}\boldsymbol{c}^{\mathbf{r}_2}_i \right \|
			\\
			&=\left\| \boldsymbol{A} \right\| vrr(\mathbf{r}_1,\mathbf{r}_2; \mathbf{F}) \\
			&< K \epsilon \notag
		\end{aligned}
	\end{equation}	
	\end{small}
	\end{proof}

	
	
	\begin{lemma}\label{lemma 1}
	Given $f=f_l \circ\cdots \circ f_1$,  $f_j(x) = \boldsymbol{A}_jx+\boldsymbol{b}$ if $j = l$ and  $\sigma(\boldsymbol{A}_{j}x)$ otherwise, where $\sigma$ is a $1$-Lipschitz function. Then $K = \Pi^l_{j=1}\left\| \boldsymbol{A}_{j} \right\|_{2} $ is the Lipschitz constant of $f$.
	\end{lemma}
	\begin{proof}
		Suppose that inputs $x$, $y$ belong to the domain of $f_{j}$, 
		\begin{equation}
			\begin{aligned}
			\left\| f_j(x) - f_{j}(y) \right\| &\leq  \left\| \sigma(\boldsymbol{A}_{j}x) - \sigma(\boldsymbol{A}_{j}y) \right\|  \\	
			 &\leq \left\|  \boldsymbol{A}_jx  - \boldsymbol{A}_jy \right\| \\
				&\leq \left\| \boldsymbol{A}_j \right\| \left\|  x  -  y \right\|.
			\end{aligned}
		\end{equation}
		When $l = 2$, the claim is clearly valid. The remaining cases can be easily proved by induction.
%		\begin{equation}
%			\begin{aligned}
%			\left\| f(x) - f(y) \right\| &\leq \left\| \boldsymbol{A}_l \right\| \left\| f^{l-1}(x) - f^{l-1}(y) \right\| \\
%			 &\leq 	\left\| \boldsymbol{A}_l \right\| \left\| f_{i}(x) - f_{i}(y) \right\| \\ 
%			 &\leq 	\left\| \boldsymbol{A}_l \right\| \left\| f_{i}(x) - f_{i}(y) \right\| \\
%%				&\leq \left\| \boldsymbol{A}_2\sigma(\boldsymbol{A}_1\boldsymbol{x}) - 		\boldsymbol{A}_2\sigma(\boldsymbol{A}_1\boldsymbol{y}) \right\| \\
%%				&\leq \left\| \boldsymbol{A}_2 \right\| \left\| \sigma(\boldsymbol{A}_1\boldsymbol{x})  - 		\sigma(\boldsymbol{A}_1\boldsymbol{y}) \right\| \\
%%				&\leq \left\| \boldsymbol{A}_2 \right\| \left\|  \boldsymbol{A}_1\boldsymbol{x}  - 	\boldsymbol{A}_1\boldsymbol{y} \right\|
%			\end{aligned}
%		\end{equation}
	\end{proof}

	\begin{proposition}\label{proposition 2}
		Considering $f=f_l \circ\cdots \circ f_1$,  $f_j(x) = \boldsymbol{A}_jx+\boldsymbol{b}$ if $j = l$ and  $\sigma(\boldsymbol{A}_{j}x)$ otherwise, where $\sigma=\max(0,x)$. If $\mathbf{F}_{app}' = f\circ\mathbf{F}_{app}$, $\sum_{i=1}^{T}w_{i} = 1$ and $\max_{i=1,\dots,T}\left\|  w^{\mathbf{r}_1}_{i}\boldsymbol{c}^{\mathbf{r}_1}_i - w^{\mathbf{r}_2}_{i}\boldsymbol{c}^{\mathbf{r}_2}_i\right\| < \epsilon/T$,  we have $vrr(\mathbf{r}_1,\mathbf{r}_2; \mathbf{F}')<K\epsilon$, where $K = \Pi^l_{j=1}\left\| \boldsymbol{A}_{j} \right\|_{2} $ is the Lipschitz constant of $f$. 
	\end{proposition}
	\begin{proof}
    \begin{small}
    Note that $\forall a\in \mathbb{R}^{+}$ and $1\leq j < l$, $af_j(x)=a\sigma(\boldsymbol{A}_{j}x)=\sigma(a\boldsymbol{A}_{j}x)=f_j(ax)$. Denoting $f^{j}=f_j \circ\cdots \circ f_1$, we can get the following derivation: 
    \begin{equation}
        \begin{aligned}
            af^{j}(x)&=a\sigma(\boldsymbol{A}_{j}f^{j-1}(x)) = \sigma(a\boldsymbol{A}_{j}f^{j-1}(x)) \\
            &=  \sigma(a\boldsymbol{A}_{j}\sigma(\boldsymbol{A}_{j-1}f^{j-2}(x))) \\
            &=  \sigma(\boldsymbol{A}_{j}\sigma(a\boldsymbol{A}_{j-1}f^{j-2}(x))) \\
            &\cdots \\
            & = f^{j}(ax).
        \end{aligned}
    \end{equation}
    Because the weights are always non-negative in the volume rendering integral,
    we further have
		\begin{equation}\label{eq: vrr}
		\begin{aligned}
			vrr(\mathbf{r}_1,\mathbf{r}_2; \mathbf{F}') &= \left \| {C}(\mathbf{r}_1;\mathbf{F}') -{C}(\mathbf{r}_2;\mathbf{F}')  \right \| \\ 
			&=\left \| \sum\limits_{i=1}^{T} w^{\mathbf{r}_1}_{i}f(\boldsymbol{c}^{\mathbf{r}_1}_i) -\sum\limits_{i=1}^{T} w^{\mathbf{r}_2}_{i}f(\boldsymbol{c}^{\mathbf{r}_2}_i)  \right \|\\
			&=\left \| \sum\limits_{i=1}^{T} w^{\mathbf{r}_1}_{i}\boldsymbol{A}_{l}f^{l-1}(\boldsymbol{c}^{\mathbf{r}_1}_i) -\sum\limits_{i=1}^{T} w^{\mathbf{r}_2}_{i}\boldsymbol{A}_{l}f^{l-1}(\boldsymbol{c}^{\mathbf{r}_2}_i)  \right \|\\
    		&=\left \| \sum\limits_{i=1}^{T} \boldsymbol{A}_{l}f^{l-1}(w^{\mathbf{r}_1}_{i}\boldsymbol{c}^{\mathbf{r}_1}_i) -\sum\limits_{i=1}^{T} \boldsymbol{A}_{l}f^{l-1}(w^{\mathbf{r}_2}_{i}\boldsymbol{c}^{\mathbf{r}_2}_i)  \right \|\\
			&\leq \sum\limits_{i=1}^{T}\left \|  \boldsymbol{A}_{l}f^{l-1}(w^{\mathbf{r}_1}_{i}\boldsymbol{c}^{\mathbf{r}_1}_i) - \boldsymbol{A}_{l}f^{l-1}(w^{\mathbf{r}_2}_{i}\boldsymbol{c}^{\mathbf{r}_2}_i)  \right \|.
			\\
		\end{aligned}
	\end{equation}	
    Based on above inequality and Lemma \ref{lemma 1}, we have
\begin{equation}\label{eq: vrr}
		\begin{aligned}
			&\ \ \ \ \left\|  \boldsymbol{A}_{l}f^{l-1}(w^{\mathbf{r}_1}_{i}\boldsymbol{c}^{\mathbf{r}_1}_i) - \boldsymbol{A}_{l}f^{l-1}(w^{\mathbf{r}_2}_{i}\boldsymbol{c}^{\mathbf{r}_2}_i) \right\|  \\
		&\leq \left\| \boldsymbol{A}_{l} \right\| \left\|  f^{l-1}(w^{\mathbf{r}_1}_{i}\boldsymbol{c}^{\mathbf{r}_1}_i) - f^{l-1}(w^{\mathbf{r}_2}_{i}\boldsymbol{c}^{\mathbf{r}_2}_i) \right\| \\
			&\leq \prod^{l}_{j=1} \left\| \boldsymbol{A}_{i} \right\| \ \left\|  w^{\mathbf{r}_1}_{i}\boldsymbol{c}^{\mathbf{r}_1}_i - w^{\mathbf{r}_2}_{i}\boldsymbol{c}^{\mathbf{r}_2}_i \right\|\\
			&= K \ \left\|  w^{\mathbf{r}_1}_{i}\boldsymbol{c}^{\mathbf{r}_1}_i - w^{\mathbf{r}_2}_{i}\boldsymbol{c}^{\mathbf{r}_2}_i \right\|.
		\end{aligned} 
	\end{equation}
	Therefore, 
	\begin{equation}\label{eq: vrr}
		\begin{aligned}
			vrr(\mathbf{r}_1,\mathbf{r}_2; \mathbf{F}') &\leq \sum\limits_{i=1}^{T} K \left\|  w^{\mathbf{r}_1}_{i}\boldsymbol{c}^{\mathbf{r}_1}_i - w^{\mathbf{r}_2}_{i}\boldsymbol{c}^{\mathbf{r}_2}_i \right\| \\
			&< K \sum\limits_{i=1}^{T} \epsilon/T = K\epsilon \\
		\end{aligned} 
	\end{equation}	
    \end{small}
	\end{proof}

		\begin{lemma} \label{lemma 2}
		$\mathbf{F}_{app}({\boldsymbol{x}, \boldsymbol{d}})$ = $\mathbf{F}_{sh}(\boldsymbol{x})\Gamma(\boldsymbol{d})+\boldsymbol{v}$, where $\Gamma(\boldsymbol{d}):\mathbb{R}^{2\times1} \rightarrow \mathbb{R}^{\ell\times1}$ is the spherical harmonic basis function, $\mathbf{F}_{sh}(\boldsymbol{x}) :\mathbb{R}^{3\times1} \rightarrow \mathbb{R}^{3\times \ell}$ is the coefficient function, and $\boldsymbol{v} \in \mathbb{R}^{3\times1}$. Given $\boldsymbol{A} \in \mathbb{R}^{3\times3}$, $\boldsymbol{b} \in \mathbb{R}^{3\times1}$, then
		$\boldsymbol{A}\mathbf{F}_{app}({\boldsymbol{x}, \boldsymbol{d}})+\boldsymbol{b} \Leftrightarrow 
		\boldsymbol{A}\mathbf{F}_{sh}({\boldsymbol{x}})+2\sqrt{\pi}[\boldsymbol{A}\boldsymbol{v}+\boldsymbol{b}-\boldsymbol{v},\boldsymbol{0}]$.
	\end{lemma}
	\begin{proof}
		\begin{equation}\label{eq: sh}
			\begin{aligned}
				\boldsymbol{A}\mathbf{F}_{app}({\boldsymbol{x}, \boldsymbol{d}})+\boldsymbol{b} &= 
				\boldsymbol{A}(\mathbf{F}_{sh}(\boldsymbol{x})\Gamma(\boldsymbol{d})+\boldsymbol{v}) +\boldsymbol{b} \\
				&= \boldsymbol{A}\mathbf{F}_{sh}(\boldsymbol{x})\Gamma(\boldsymbol{d})+\boldsymbol{A}\boldsymbol{v} +\boldsymbol{b} \\
				&= \boldsymbol{A}\mathbf{F}_{sh}(\boldsymbol{x})\Gamma(\boldsymbol{d})+\boldsymbol{A}\boldsymbol{v} +\boldsymbol{b}. \\
			\end{aligned}
		\end{equation}
		Because the first component of the spherical harmonic basis function outputs a constant value $\frac{1}{2\sqrt{\pi}}$,  we have
		\begin{equation}
			\begin{aligned}
				&(\boldsymbol{A}\mathbf{F}_{sh}({\boldsymbol{x}})+2\sqrt{\pi}[\boldsymbol{A}\boldsymbol{v}+\boldsymbol{b}-\boldsymbol{v},\boldsymbol{0}])\Gamma(\boldsymbol{d})+\boldsymbol{v}\\
				& = \boldsymbol{A}\mathbf{F}_{sh}\Gamma(\boldsymbol{d}) + \boldsymbol{A}\boldsymbol{v}+\boldsymbol{b}-\boldsymbol{v} + \boldsymbol{v} \\
				& =  \boldsymbol{A}\mathbf{F}_{sh}\Gamma(\boldsymbol{d}) + \boldsymbol{A}\boldsymbol{v}+\boldsymbol{b}
			\end{aligned}	
		\end{equation}
	\end{proof}
	\noindent\textbf{Remark of Lemma \ref{lemma 2}}. Similarly, it  can prove $\boldsymbol{A}\mathbf{F}_{sh}({\boldsymbol{x}})+[\boldsymbol{b},\boldsymbol{0}] \Leftrightarrow 
	 \boldsymbol{A}\mathbf{F}_{app}({\boldsymbol{x}, \boldsymbol{d}})+\frac{\boldsymbol{b}}{2\sqrt{\pi}} + \boldsymbol{v}-\boldsymbol{A}\boldsymbol{v}$.
	 
	 \begin{proposition}\label{proposition 3}
	 	Considering  $f(x) = \boldsymbol{A}x+\boldsymbol{b}$, $\boldsymbol{A}\in \mathbb{R}^{3\times \ell}$, $\boldsymbol{b}\in \mathbb{R}^{3\times \ell}$, if $\mathbf{F}_{sh}' = f\circ\mathbf{F}_{sh}$, $\sum_{i=1}^{T}w_{i} = 1$, $vrr(\mathbf{r}_1,\mathbf{r}_2; \mathbf{F})<\epsilon_1$ and $\left\| \Gamma(\boldsymbol{d^{\textbf{r}_{1}}}) -\Gamma(\boldsymbol{d^{\textbf{r}_{2}}}) \right\|<\epsilon_2$, we have  $vrr(\mathbf{r}_1,\mathbf{r}_2; \mathbf{F}')<K_1\epsilon_1$ + $K_2\epsilon_2$, where  $K_1 =\left\| \boldsymbol{A} \right\|_2$ and $K_2 =\left\| \boldsymbol{b} \right\|_2$. Moreover, if $\boldsymbol{b}$ vanishes except for the first column (\textit{i.e.}, the form in above remark), $vrr(\mathbf{r}_1,\mathbf{r}_2; \mathbf{F}')<K_1\epsilon_1$.
	 \end{proposition}
	
		\begin{proof} 
		\begin{small}
			\begin{equation}\label{eq: vrr}
				\begin{aligned}
					&vrr(\mathbf{r}_1,\mathbf{r}_2; \mathbf{F}')\\
					 &= \left \| {C}(\mathbf{r}_1;\mathbf{F}') -{C}(\mathbf{r}_2;\mathbf{F}')  \right \| \\ 
					&= \left \| \sum\limits_{i=1}^{T} w^{\mathbf{r}_1}_{i}\textbf{F}'(\boldsymbol{x}^{\textbf{r}_{1}}_{i})\Gamma(\boldsymbol{d^{\textbf{r}_{1}}}) -\sum\limits_{i=1}^{T} w^{\mathbf{r}_2}_{i}\textbf{F}'(\boldsymbol{x}^{\textbf{r}_{2}}_{i})\Gamma(\boldsymbol{d^{\textbf{r}_{2}}})  \right \| \\
%					&\ \ \ \ + \left \| \sum\limits_{i=1}^{T} w^{\mathbf{r}_2}_{i}\textbf{F}'(\boldsymbol{x}^{\textbf{r}_{2}}_{i})(\Gamma(\boldsymbol{d^{\textbf{r}_{2}}})- \Gamma(\boldsymbol{d^{\textbf{r}_{1}}}))  \right \| \\
					&\leq \left \| \sum\limits_{i=1}^{T} w^{\mathbf{r}_1}_{i}\boldsymbol{A}\textbf{F}(\boldsymbol{x}^{\textbf{r}_{1}}_{i})\Gamma(\boldsymbol{d^{\textbf{r}_{1}}}) -\sum\limits_{i=1}^{T} w^{\mathbf{r}_2}_{i}\boldsymbol{A}\textbf{F}(\boldsymbol{x}^{\textbf{r}_{2}}_{i})\Gamma(\boldsymbol{d^{\textbf{r}_{2}}}) \right \|  \\
					&\ \ \ \ +  \left \| \boldsymbol{b}\Gamma(\boldsymbol{d^{\textbf{r}_{1}}}) -\boldsymbol{b}\Gamma(\boldsymbol{d^{\textbf{r}_{2}}}) \right \| \\
%					&\leq  \left\| \boldsymbol{A} \right\| vrr(\mathbf{r}_1,\mathbf{r}_2; \mathbf{F})  +  \left \| \boldsymbol{b}\Gamma(\boldsymbol{d^{\textbf{r}_{1}}}) -\boldsymbol{b}\Gamma(\boldsymbol{d^{\textbf{r}_{2}}}) \right \|\\
					&\leq  \left\| \boldsymbol{A} \right\| vrr(\mathbf{r}_1,\mathbf{r}_2; \mathbf{F})  +  \left \| \boldsymbol{b}\right \| \left\| \Gamma(\boldsymbol{d^{\textbf{r}_{1}}}) -\Gamma(\boldsymbol{d^{\textbf{r}_{2}}})  \right\|\\
					&<  K_1\epsilon_1 + K_2\epsilon_2.\\
				\end{aligned}
			\end{equation}	
		If $\boldsymbol{b}$ vanishes except for the first column, $ \left \| \boldsymbol{b}\Gamma(\boldsymbol{d^{\textbf{r}_{1}}}) -\boldsymbol{b}\Gamma(\boldsymbol{d^{\textbf{r}_{2}}}) \right \| =0$, thus   $vrr(\mathbf{r}_1,\mathbf{r}_2; \mathbf{F}')<K_1\epsilon_1$.
	\end{small}
	\end{proof}

\noindent\textbf{Remarks}. Prop.~\ref{proposition 3} extends Lipschitz-constrained linear mapping in Prop.~\ref{proposition 1} from appearance representation to spherical harmonics. To prove the bound of Lipschitz MLP applied to spherical harmonics, some fussy assumptions are further required, and the proof will be trivial to repeat the above proving processes. We believe the three propositions have exhibited the intuition and importance of Lipschitz transformations for this task.
\section{More results}
For comprehensive analysis and evaluation, we have supplied a video\footnote{\url{https://www.youtube.com/watch?v=1ft8Mev3RmE}} in the supplementary materials,  which contains the continuous novel views of multiple scenes stylized with various references.  It can be observed that, both $\text{WCT}^2$ and CCPL create noises and disharmony to affect the photorealism of video. In specific, $\text{WCT}^2$ is likely to sharpen the edges excessively that produces artificial boundaries around edges (\textit{e.g.}, the trex and room scenes). It also generates noticeable noises in some stylized scenes. The results of CCPL usually have richer colors that enhances the visual effects. However, the variegated colors acceptable in a still image may be harmful to 3D scenes. For example, in the trex and fortress scenes, the interframe variations of colors results in artifacts and unconsistency of videos. In the flower scene, due to the unconsistency, the colorful leaves and flowers seem to be unrealistic and flickering. In contrast, LipRF can alleviate these downsides to generate more consistent and photorealistic stylized novel views while transferring the color style. The videos of LipRF are more like camera shots to meet the requirement of photorealistic 3D scene stylization. 


%	{\small
%		\bibliographystyle{ieee_fullname}
%		\bibliography{citation}
%	}
	

	
\end{document}

%\bibliography{biblio}
\bibliography{Bib.bib}


	
\end{document}