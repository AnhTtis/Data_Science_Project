\documentclass[article,twocolumn,floatfix,superscriptaddress,longbibliography]{revtex4-1}
%\usepackage[printfigures,blank]{figcaps}
%\usepackage[nomarkers,notables,nolists]{endfloat}
%\usepackage[font=small,  justification=justified,format=plain]{caption}
\usepackage[hidelinks]{hyperref}
\usepackage{physics}
\pagestyle{empty}
\usepackage{xr}
\externaldocument[M-]{main}
\renewcommand{\theequation}{S.\arabic{equation}}
%\usepackage{caption}

%\captionsetup[figure]{font=small,labelsep=period, name={Fig. S}}
%\captionsetup[table]{font=small,labelsep=period, justification=raggedright}

\setcitestyle{super}
\setlength{\parindent}{0in}
\usepackage{graphicx}
\makeatletter
\usepackage{mathtools}
\usepackage[utf8]{inputenc}
\usepackage{mathtools}
\usepackage{bbold}
\newtagform{supplementary}[]()
%\counterwithin*{equation}{section}
\newcommand*{\rom}[1]{\expandafter\@slowromancap\romannumeral #1@}
\renewcommand{\thetable}{S\arabic{table}}  
\renewcommand{\thefigure}{S\arabic{figure}} 
\makeatother
\begin{document}

\title{{\bf Supplementary Material for the article:}\\ ``Electron-photon Chern number in cavity-embedded 2D moir\'{e} materials"}

\author{Danh-Phuong Nguyen}
\affiliation{Universit\'{e} Paris Cit\'e, CNRS, Mat\'{e}riaux et Ph\'{e}nom\`{e}nes Quantiques, 75013 Paris, France}
\author{Geva Arwas}
\affiliation{Universit\'{e} Paris Cit\'e, CNRS, Mat\'{e}riaux et Ph\'{e}nom\`{e}nes Quantiques, 75013 Paris, France}
\author{Zuzhang Lin}
\affiliation{Department of Physics, The University of Hong Kong, Hong Kong, China}
\affiliation{HKU-UCAS Joint Institute of Theoretical and Computational Physics at Hong Kong, China} 
\author{Wang Yao}
\affiliation{Department of Physics, The University of Hong Kong, Hong Kong, China}
\affiliation{HKU-UCAS Joint Institute of Theoretical and Computational Physics at Hong Kong, China}
\author{ Cristiano Ciuti}
\affiliation{Universit\'{e} Paris Cit\'e, CNRS, Mat\'{e}riaux et Ph\'{e}nom\`{e}nes Quantiques, 75013 Paris, France}



\date{\today}
\maketitle
\usetagform{supplementary}
\section{Cavity QED Hamiltonian}
In this section we derive a four-band continuum model \cite{Wu2019,Zhai2020} for small angle twisted bilayer TMD. The electron-photon Hamiltonian can be written as:
\begin{equation}
\label{eq:S-f4}
    \hat{H} = \hbar\omega_c\hat{a}^{\dagger}\hat{a} + \begin{pmatrix}
        \hat{H}_{t} & 0 \\ 0 & \hat{H}_{b}
    \end{pmatrix} + \begin{pmatrix}
        \hat{V}_t & \hat{U} \\ \hat{U}^{\dagger} & \hat{V}_b
    \end{pmatrix},
\end{equation}
where $\hat{a}^{\dagger}$ creates a photon of frequency $\omega_c$, $\hat{H}_{t}$ ($\hat{H}_{b}$) is a $2 \cross 2$ matrix corresponding to the top (bottom) layer describing the TMD material along with the coupling to the cavity photon. The last term is a $4 \cross 4$ matrix describing the moir\'{e} potential. $\hat{V}_{t}$ and $\hat{V}_{b}$ describe the intra-layer interactions, while $\hat{U}$ is the layer-layer interaction. Each layer is a honeycomb lattice with the lattice constant $a_0$ and is defined by the two primitive vectors $\mathbf{a}_1 = a_0\sqrt{3}/2(1, \sqrt{3}, 0)$, $\mathbf{a}_2 = a_0\sqrt{3}/2(-1, \sqrt{3}, 0)$ and their reciprocal counterparts $\mathbf{b}_1 = 2\pi/(3a_0)(\sqrt{3},1,0)$, $\mathbf{b}_2 = 2\pi/(3a_0)(-\sqrt{3},1,0)$. % such that $\mathbf{a}_i\cdot\mathbf{b}_j = 2\pi\delta_{ij}$.
The top layer is rotated in-plane by a small angle $\theta$ in order to create a moir\'{e} pattern. 
%A point $\mathbf{r}$ in the top layer was originally at the point $\mathbf{r}_0 = R^{-1}(\theta)\mathbf{r}$ where $R(\theta)$ is the rotation matrix about the $z$-axis, hence the mismatch between layers is $\boldsymbol{\Delta}(\mathbf{r}) = [\mathbb{1} - R^{-1}(\theta)]\mathbf{r}$. 
The mismatch between layers is given by $\boldsymbol{\Delta}(\mathbf{r}) = [\mathbb{1} - R^{-1}(\theta)]\mathbf{r}$, where $R(\theta)$ is a rotation matrix about the $z$-axis. One moir\'{e} unit cell is defined by $\mathbf{L}_{i=1,2}$ such that $\boldsymbol{\Delta}(\mathbf{L}_i)= \mathbf{a}_i$, hence $\mathbf{L}_i = [\mathbb{1} - R^{-1}(\theta)]^{-1}\mathbf{a}_i$. Similarly, the moir\'{e} reciprocal lattice vectors $\mathbf{G}_i = [\mathbb{1} - R(\theta)]\mathbf{b}_i$, satisfying $\mathbf{L}_i \cdot \mathbf{G}_j = 2\pi \delta_{ij}$. %With a well-defined set of vectors, we will now demonstrate each term in the $4 \cross 4$ matrix in equation (\ref{eq:S-f4}).
\begin{figure}
\centering
%\includegraphics[scale=0.22]{Images/cavity-moire-2.pdf} \\ 
%\centering
\includegraphics[width = 0.49\hsize]{Fig_1_2.pdf}
\includegraphics[width = 0.49\hsize]{Fig_1_3.pdf}
%\includegraphics[scale=0.33]{Images/Main/TMD/all BZ.png}
\caption{First Brillouin zone with Dirac points of the bottom layer (left) and moir\'{e} Brillouin zone (shown in red) of moir\'e superlattice (right).}\label{fig:SM-TMD-mBZ}
\end{figure}

Each layer is described by a tight binding Hamiltonian on the honeycomb lattice, with hopping coupling quantified by $t$. The unit cell consists of two sites $A, B$ having an on-site energy difference of $2\Delta$. The electron-photon interaction is introduced via Peierls substitution. For the bottom layer, we can write the Hamiltonian as:
\begin{equation}
\label{eq:S:H2f}
    \begin{aligned}
        \hat{H}_{b} &= \Delta\sum_{\mathbf{R}}\bigg(\ket{A,\mathbf{R}}\bra{A,\mathbf{R}} - \ket{B,\mathbf{R}}\bra{B,\mathbf{R}}\bigg)\\
        &+ t\sum_{\mathbf{R},j=1,2,3}e^{-ig(\hat{a}+\hat{a}^{\dagger})\mathbf{u}\cdot\mathbf{e}_j}\ket{B,\mathbf{R} + \mathbf{a}_j}\bra{A,\mathbf{R}} + \text{h.c},
    \end{aligned}
\end{equation}
where $g = eA_0a_0/\hbar$, $\mathbf{u}$ is the cavity orientation, $\mathbf{e}_{j=1,2,3} = R\left(j2\pi/3\right)(0,-1,0)$ and $\mathbf{a}_3 = (0,0,0)$. We assume the cavity field $\mathbf{A}$ is uniform, such that the Hamiltonian (\ref{eq:S:H2f}) is block-diagonal in $\mathbf{k}$ space. The Bloch Hamiltonian $H^e_b({\mathbf{k}})$ is given by:
\begin{equation}
\label{eq:S:H3f}
    \begin{aligned}
        \hat{H}_b(\mathbf{k}) &= \Delta\ket{A,\mathbf{k}}\bra{A,\mathbf{k}} - \Delta\ket{B,\mathbf{k}}\bra{B,\mathbf{k}}\\
        &+ t\sum_{j=0,1,2} e^{-ig(\hat{a}+\hat{a}^{\dagger})\mathbf{u}\cdot\mathbf{e}_j}e^{-i\mathbf{k}\cdot\mathbf{a}_j}\ket{B,\mathbf{k}}\bra{A,\mathbf{k}} + \text{h.c},
    \end{aligned}
\end{equation}
where $\mathbf{k}$ is in the Brillouin zone (BZ) as shown in Figure \ref{fig:SM-TMD-mBZ}. Next, we expand around one valley $\mathbf{K}_b$ such that $\mathbf{K}_b\cdot\mathbf{a}_1 = -2\pi/3$, and at the  first order in $g$, we have: 
%
\begin{equation}
\begin{aligned}
\hat{H}_b({\mathbf{k}}) &= \Delta\sigma_z + \frac{3a_0t}{2}\left(\mathbf{k} - \mathbf{K}_b + \frac{e\mathbf{A}^{(xy)}}{\hbar}\left(\hat{a} + \hat{a}^{\dagger}\right)\right)\cdot \boldsymbol{\sigma},
\end{aligned}
\label{SM:eq-Heb}
\end{equation}
where $\boldsymbol{\sigma} = (\sigma_x,\sigma_y,\sigma_z)$ are the Pauli matrices for the sub-lattice degree of freedom, $\mathbf{A}^{(xy)}$ is the in-plane projection of the cavity field $\mathbf{A}$.


The top layer is rotated in real space by an angle $\theta$ such that the lattice vectors in equation (\ref{eq:S:H2f}) are transformed as $\mathbf{a} \to \tilde{\mathbf{a}} = R(\theta)\mathbf{a}$. Following the same procedure and restricting $\mathbf{k}$ to the vicinity of $\mathbf{K}_t = R(\theta)\mathbf{K}_b$ one gets:
\begin{equation}
\label{SM:eq-Het}
\begin{aligned}
\hat{H}_t({\mathbf{k}}) \ &= \ \Delta\sigma_z \\ &+ \frac{3a_0t}{2}R^{-1}(\theta)\left(\mathbf{k} - \mathbf{K}_t + \frac{e\mathbf{A}^{(xy)}}{\hbar}\left(\hat{a} + \hat{a}^{\dagger}\right)\right)\cdot \boldsymbol{\sigma}.
\end{aligned}
\end{equation}

In the presence of the cavity, the interlayer hopping $\hat{U}$ between site $\mathbf{R}_b$ of the bottom and $\mathbf{R}_t$ of the top can be approximated as:
\begin{equation}
    \hat{U} = \sum_{\substack{\mathbf{R}_b,\mathbf{R}_t \\ \sigma = A,B}}e^{-i\Phi_{\mathbf{R}_t,\mathbf{R}_b}(\hat{a}+\hat{a}^{\dagger})}T(\mathbf{R}_t -\mathbf{R}_b)\ket{\sigma,\smash{\mathbf{R}_t}}\bra{\sigma,\mathbf{R}_b} ,
\end{equation}
where $ T(\mathbf{R}_t -\mathbf{R}_b)$ is the hopping coefficient in the absence of a cavity \cite{Wang2017,Bistritzer2011}. We assume that the interlayer hopping is only between sub-lattices of the same type. $\mathbf{R}_t - \mathbf{R}_b = \mathbf{d} + \boldsymbol{\Delta}(\mathbf{r})$ where $\mathbf{d} = d(0,0,1)$ is the distance between two layers. The Peierls phase can be approximated by $\Phi_{\mathbf{R}_t,\mathbf{R}_b} = e\mathbf{A}\cdot(\mathbf{R}_t-\mathbf{R}_b)/\hbar \approx eA^{(z)}d/\hbar$, where $A^{(z)}$ is the $z$-component of the cavity field. Since the Peierls phase does not depend on $\mathbf{r}$, we can write the interlayer coupling as:
\begin{equation}
    \hat{U} = e^{-i\frac{eA^{(z)}d}{\hbar}(\hat{a}+\hat{a}^{\dagger})}\hat{U}_0,
\end{equation}
where $\hat{U}_0$ is a moir\'{e} interlayer hopping in the absence of the cavity. 

Combining all terms and defining $v_F = 3a_0 t/(2\hbar)$ one gets a low-energy Hamiltonian for the twisted bilayer TMD coupled to cavity field:
\begin{widetext}
\begin{equation}
\label{eq:S:H5}
    \widetilde{H} = 
    \begin{pmatrix}
    \Delta \sigma_z + v_F R^{-1}(\theta)\left[\mathbf{\hat{p}} - \hbar\boldsymbol{\kappa}_- +  e\mathbf{A}^{(xy)}(\hat{a}+\hat{a}^{\dagger}) \right]\cdot\boldsymbol{\sigma} + \hat{V}_t & e^{-i\frac{A^{(z)} d}{\hbar} (\hat{a}+\hat{a}^{\dagger})}\hat{U}_0 \\
    e^{i\frac{A^{(z)} d}{\hbar}(\hat{a}+\hat{a}^{\dagger})}\hat{U}_0^{\dagger} &  \Delta \sigma_z + v_F \left[\mathbf{\hat{p}} -\hbar\boldsymbol{\kappa}_+ + e\mathbf{A}^{(xy)}(\hat{a}+\hat{a}^{\dagger})  \right]\cdot\boldsymbol{\sigma}  + \hat{V}_b
    \end{pmatrix} + \hbar \omega_c \hat{a}^{\dagger}\hat{a}.
\end{equation}
%\end{widetext}
Note that here we change the notation as $\mathbf{K}_b$ ($\mathbf{K}_t$) $\rightarrow \boldsymbol{\kappa}_+ $ ($\boldsymbol{\kappa}_-$) to highlight the moir\'{e} mini-BZ (see Fig. \ref{fig:SM-TMD-mBZ}). Next, we apply the unitary transformation $\hat{T}_{\theta} = \text{diag}(e^{-i\frac{\theta}{2}\sigma_z},\mathbb{1})$, leading to
%\begin{widetext}
\begin{equation}
    \hat{T}_{\theta}\widetilde{H}\hat{T}_{\theta}^{\dagger} = \begin{pmatrix}
    \Delta \sigma_z + v_F \left[\mathbf{\hat{p}} - \hbar\boldsymbol{\kappa}_- +  e\mathbf{A}^{(xy)}(\hat{a}+\hat{a}^{\dagger}) \right]\cdot\boldsymbol{\sigma} + \hat{V}_t & e^{-i\frac{A^{(z)} d}{\hbar} (\hat{a}+\hat{a}^{\dagger})}e^{-i\frac{\theta}{2}\sigma_z}\hat{U}_0 \\
    e^{i\frac{A^{(z)} d}{\hbar}(\hat{a}+\hat{a}^{\dagger})}e^{i\frac{\theta}{2}\sigma_z}\hat{U}_0^{\dagger} &  \Delta \sigma_z + v_F \left[\mathbf{\hat{p}} -\hbar\boldsymbol{\kappa}_+ + e\mathbf{A}^{(xy)}(\hat{a}+\hat{a}^{\dagger})  \right]\cdot \boldsymbol{\sigma} + \hat{V}_b
    \end{pmatrix} + \hbar \omega_c \hat{a}^{\dagger}\hat{a} .
\end{equation}
%\end{widetext}
We remove the photonic operators from the off-diagonal terms using the unitary transformation $\hat{T}_{d} = \text{diag}(e^{i\frac{eA^{(z)}d}{2\hbar}\left(\hat{a}+ \hat{a}^{\dagger}\right)},e^{-i\frac{eA^{(z)}d}{2\hbar}\left(\hat{a}+ \hat{a}^{\dagger}\right)})$. The photonic number operator transforms as $\hat{a}^{\dagger}\hat{a} \to \hat{a}^{\dagger}\hat{a} + i\frac{eA^{(z)}d}{2\hbar}\left(\hat{a} - \hat{a}^{\dagger}\right)\tau_z + \text{constant}$, where $\tau_z$ is the Pauli matrix with respect to the layer pseudo-spin. Finally, the four-band Hamiltonian $\hat{H}^{(4)} = \hat{T}_d\hat{T}_{\theta}\widetilde{H}\hat{T}^{\dagger}_{\theta}\hat{T}^{\dagger}_d$ can be written as:
%\begin{widetext}
\begin{equation}
\label{eq:S:Hf}
\begin{aligned}
    \hat{H}^{(4)} &= \begin{pmatrix}
    \Delta \sigma_z + v_F \left[\mathbf{\hat{p}} - \hbar\boldsymbol{\kappa}_- +  e\mathbf{A}^{(xy)}(\hat{a}+\hat{a}^{\dagger}) \right]\cdot\boldsymbol{\sigma} + \hat{V}_t & e^{-i\frac{\theta}{2}\sigma_z}\hat{U}_0 \\
    e^{i\frac{\theta}{2}\sigma_z}\hat{U}_0^{\dagger} &  \Delta \sigma_z + v_F \left[\mathbf{\hat{p}} -\hbar\boldsymbol{\kappa}_+ + e\mathbf{A}^{(xy)}(\hat{a}+\hat{a}^{\dagger})  \right]\cdot \boldsymbol{\sigma} + \hat{V}_b
    \end{pmatrix} \\
    &+ \hbar \omega_c \hat{a}^{\dagger}\hat{a} + i\frac{\omega_c e A^{(z)} d}{2}\left(\hat{a} - \hat{a}^{\dagger}\right)\tau_z.
    \end{aligned}
\end{equation}
\end{widetext}
\section{Effective model for the valence bands}

For the case of large $\Delta$ we can construct an effective model for the valence bands. Let $\Psi = (\Psi_{t}^c, \Psi_{t}^v, \Psi_{b}^c, \Psi_{b}^v)^T$ be an eigenvector of $\hat{H}^{(4)}$ with energy $E$, where $c$ ($v$) corresponds to the conduction (valence) bands. Using (\ref{eq:S:Hf}), we can write
\begin{widetext}
 \begin{equation}
 \label{SM:eq-all 4}
 \begin{aligned}
     \left(\Delta + \hat{V}_t^c + \hbar\omega_c\hat{a}^{\dagger}\hat{a} + \hat{D}\right)\Psi_{t}^c + v_F\left(\hat{\Pi}_{tx} - i\hat{\Pi}_{ty}\right)\Psi_{t}^v + e^{-i\frac{\theta}{2}}_{c}\hat{U}_{0}^c\Psi_{b}^c &= E\Psi_{t}^c,\\
     v_F\left(\hat{\Pi}_{tx} + i\hat{\Pi}_{ty}\right)\Psi_{t}^c + \left(-\Delta + \hat{V}_t^v + \hbar\omega_c\hat{a}^{\dagger}\hat{a} + \hat{D}\right)\Psi_{t}^v + e^{i\frac{\theta}{2}}\hat{U}_{0}^v\Psi_{b}^v &=  E\Psi_{t}^v,\\ 
     \left(\Delta + \hat{V}_b^c + \hbar\omega_c\hat{a}^{\dagger}\hat{a} - \hat{D}\right)\Psi_{b}^c + v_F\left(\hat{\Pi}_{bx} - i\hat{\Pi}_{by}\right)\Psi_{b}^v + e^{i\frac{\theta}{2}}_{c}(\hat{U}^c_{0})^{\dagger}\Psi_{t}^c &= E\Psi_{b}^c,\\
     v_F\left(\hat{\Pi}_{bx} + i\hat{\Pi}_{by}\right)\Psi_{b}^c + \left(-\Delta + \hat{V}_b^v + \hbar\omega_c\hat{a}^{\dagger}\hat{a} - \hat{u}\right)\Psi_{b}^v + e^{-i\frac{\theta}{2}}(\hat{U}^v_{0})^{\dagger}\Psi_{t}^v &=  E\Psi_{b}^v,      
 \end{aligned}
 \end{equation}
 \end{widetext}
where for brevity we introduce the notation $\hat{\boldsymbol{\Pi}}_{b,t} = \hat{\mathbf{p}} - \hbar\boldsymbol{\kappa}_{+,-} + e\mathbf{A}^{(xy)}(\hat{a}+\hat{a}^{\dagger})$ and $\hat{D} = i\frac{\omega_c e A^{(z)} d}{2}\left(\hat{a} - \hat{a}^{\dagger}\right)$.

We consider a TMD material that has a large gap between the conduction and valence bands, and for states localized in the latter, we have $E \approx -\Delta$ and $||\Psi_{t}^c||, ||\Psi_{b}^c|| \ll ||\Psi_{t}^v||, ||\Psi_{b}^v|| $. Additionally, the moir\'{e} potentials and photon energy are also weak such that $||\hat{U}_0||, ||\hat{V}_t||, ||\hat{V}_b||, \hbar\omega_c \ll \Delta$. By neglecting higher order terms, the first two equations of (\ref{SM:eq-all 4}) corresponding to the top layer become:
\begin{widetext}
\begin{equation}
\begin{aligned}
    &2\Delta\Psi_{t}^c + v_F\left(\hat{\Pi}_{tx} - i\hat{\Pi}_{ty}\right)\Psi_{t}^v = 0,\\
    &v_F\left(\hat{\Pi}_{tx} + i\hat{\Pi}_{ty}\right)\Psi_{t}^c + \left(-\Delta + \hat{V}_t^v + \hbar\omega_c\hat{a}^{\dagger}\hat{a} - \hat{D}\right)\Psi_{t}^v + e^{i\frac{\theta}{2}}\hat{U}_{0}^v\Psi_{b}^v =  E\Psi_{t}^v.
\end{aligned}
\end{equation}
%\end{widetext}
Solving the first line for $\Psi_{t}^c$ and defining the effective mass $m^{\star} = \frac{\Delta}{v_F^2}$, the second line can be written as
%\begin{widetext}
\begin{equation}
\left(-\frac{\left[\hat{\mathbf{p}} -\hbar\boldsymbol{\kappa}_-+ e\mathbf{A}^{(xy)}(\hat{a}+\hat{a}^{\dagger})\right]^2}{2m^{\star}} + \hat{V}_t^v - \Delta + \hbar\omega_c\hat{a}^{\dagger}\hat{a} + \hat{D}\right)\Psi_{t}^v + e^{i\frac{\theta}{2}}\hat{U}_{0}^v\Psi_{b}^v = E\Psi_{t}^v.
\end{equation}
%\end{widetext}
Following the same procedure, we obtain another equation for the bottom layer with $\boldsymbol{\kappa}_- \to \boldsymbol{\kappa}_+$, $t \to b$. Combining the two equations we get an effective Hamiltonian for the valence bands: 
%\begin{widetext}
 \begin{equation}
\widetilde{\mathcal{H}} = 
    -\frac{1}{2m^{\star}}\begin{pmatrix}
   (\hat{\mathbf{p}} - \hbar\mathbf{\boldsymbol{\kappa}_-}  + e\mathbf{A}^{(xy)}(\hat{a}+\hat{a}^{\dagger}))^2 & 0 \\
    0 & (\hat{\mathbf{p}} - \hbar\boldsymbol{\kappa}_+ +e\mathbf{A}^{(xy)}(\hat{a}+\hat{a}^{\dagger}))^2
\end{pmatrix} 
    +  
    \begin{pmatrix}
    \hat{V}_t^v & \hat{U}_{0}^v \\ \hat{U}_{0}^{v\dagger} & \hat{V}_b^v
    \end{pmatrix}  + \hbar \omega_c \hat{a}^{\dagger}\hat{a} + i \omega_c\frac{eA^{(z)}d}{2}(\hat{a}-\hat{a}^{\dagger})\tau_z.
    \label{SM:eq-last H}
\end{equation}
 \end{widetext}
 The intralayer moir\'{e} potentials $\hat{V}^{v}_{t}$ and $\hat{V}^{v}_{b}$ can be written as \cite{Wu2019,Zhai2020}:
\begin{equation}
\label{eq:S-Vt}
    V_{t/b}^v(\mathbf{r}) = V_0 \sum_{i=1}^3\text{cos}(\mathbf{K}_i\cdot\boldsymbol{\Delta}(\mathbf{r}) \pm \alpha),
\end{equation}
where $\mathbf{K}_i$ are shown in Fig. \ref{fig:SM-TMD-mBZ}. In order to simplify the calculation, one could use the relation $\mathbf{K}_i\cdot\boldsymbol{\Delta
}(\mathbf{r}) = \mathbf{G}_i\cdot\mathbf{r}$, where $\mathbf{G}_{1,2}$ are reciprocal lattice vectors of the moir\'{e} system defined above and $\mathbf{G}_3 = \mathbf{G}_1 - \mathbf{G}_2$. The interlayer coupling $\hat{U}_0^v$ is given by \cite{Wu2019,Zhai2020}:
\begin{equation}
\label{eq:S:Uvv}
    U_{0}^v(\mathbf{r}) =
   h_0\sum_{j=1}^{3}  e^{i\left(\mathbf{G}_j-\mathbf{G}_1\right)\cdot\mathbf{r}}.
\end{equation}
For the numerical simulations presented in the paper we used the parameters of $\text{MoTe}_2$: $m^{\star} \approx 0.62m_e$ where $m_e$ is the bare electron mass, $d \approx 20$\AA, moir\'{e} potentials $(V_0, \alpha, h_0) \approx (16\text{ meV}, 89.6^{\circ}, -8.5\text{ meV})$ \cite{Wu2019}.

\section{Particle-hole transformation}
In this section we detail the present model (\ref{SM:eq-last H}) in the hole picture. In second quantization, (\ref{SM:eq-last H}) reads:
\begin{widetext}
    \begin{equation}
        \begin{aligned}
            \widetilde{\mathcal{H}} &= \hbar\omega_c\hat{a}^{\dagger}\hat{a}-\frac{\hbar^2}{2m^{\star}}\sum_{\mathbf{q},\nu}\hat{c}_{\nu\mathbf{q}}^{\dagger}\left[\mathbf{q} -\mathbf{K}_{\nu}+ \frac{e\mathbf{A}^{(xy)}}{\hbar}(\hat{a}+\hat{a}^{\dagger})\right]^2\hat{c}_{\nu\mathbf{q}} + \sum_{\nu}\int d\mathbf{r}\hat{c}_{\nu}^{\dagger}(\mathbf{r})V^v_{\nu}(\mathbf{r})\hat{c}_{\nu}(\mathbf{r})\\
            &+ \int d\mathbf{r}\left[\hat{c}_{t}^{\dagger}(\mathbf{r})U_0^v(\mathbf{r})\hat{c}_{b}(\mathbf{r}) + \text{h.c}\right] + i\omega_c\frac{eA^{(z)}d}{2}\left(\hat{a}-\hat{a}^{\dagger}\right)\int d\mathbf{r}\left(\hat{c}^{\dagger}_t(\mathbf{r})\hat{c}_{t}(\mathbf{r}) - \hat{c}^{\dagger}_b\mathbf{r}\hat{c}_{b}\right),
        \end{aligned}
    \end{equation}
%\end{widetext}
where $\nu = t,b$ and $\mathbf{K}_{b,t} = \boldsymbol{\kappa}_{+,-}$. The particle-hole transformation can be defined by a unitary operator $\hat{\Gamma}$, such that $\hat{\Gamma} \hat{c}^{\dagger}_{\nu}(\mathbf{r}) \hat{\Gamma}^{\dagger} = \hat{c}_{\nu}(\mathbf{r})$. Under this transformation the momentum operators transform as $\hat{c}_{\nu\mathbf{q}}^{\dagger} \rightarrow \hat{\Gamma} \hat{c}^{\dagger}_{\nu\mathbf{q}} \hat{\Gamma}^{\dagger} = \hat{c}_{\nu-\mathbf{q}}$. Using the particle-hole transformation, the Hamiltonian $\hat{\mathcal{H}} = \hat{\Gamma}\widetilde{H}\hat{\Gamma}^{\dagger}$ becomes: 
%\begin{widetext}
    \begin{equation}
    \label{SM:eq-2hole}
        \begin{aligned}
            \hat{\mathcal{H}} &= \hbar\omega_c\hat{a}^{\dagger}\hat{a}+\frac{\hbar^2}{2m^{\star}}\sum_{\mathbf{q},\nu}\hat{c}_{\nu\mathbf{q}}^{\dagger}\left[\mathbf{q} +\mathbf{K}_{\nu}- \frac{e\mathbf{A}^{(xy)}}{\hbar}(\hat{a}+\hat{a}^{\dagger})\right]^2\hat{c}_{\nu\mathbf{q}} - \sum_{\nu}\int d\mathbf{r}\hat{c}_{\nu}^{\dagger}(\mathbf{r})V^v_{\nu}(\mathbf{r})\hat{c}_{\nu}(\mathbf{r})\\
            &- \int d\mathbf{r}\left[\hat{c}_{b}^{\dagger}(\mathbf{r})U_0^v(\mathbf{r})\hat{c}_{t}(\mathbf{r}) + \text{h.c}\right] - i\omega_c\frac{eA^{(z)}d}{2}\left(\hat{a}-\hat{a}^{\dagger}\right)\int d\mathbf{r}\left(\hat{c}^{\dagger}_t(\mathbf{r})\hat{c}_{t}(\mathbf{r}) - \hat{c}^{\dagger}_b(\mathbf{r})\hat{c}_{b}(\mathbf{r})\right).
        \end{aligned}
    \end{equation}
\end{widetext}
Equation (\ref{SM:eq-2hole}) results in the change of the sign of the effective mass, the charge and the moir\'{e} potentials. Note that we have disregarded divergent sums that result from the interaction of all valance electrons with the photonic field. This is justified if we assume the light-matter interaction vanishes when all valence mini-bands are completely filled.
\bibliography{Bib.bib}
\end{document}

