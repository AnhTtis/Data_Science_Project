\documentclass[a4paper, 12pt]{report}
\include{begin}

\chapter{Summary of Thesis} \label{Ch_Summary}
We have theoretically investigated the diproton correlation and 
its effect on the two-proton emission. 
These are both the exotic features of proton-rich nuclei which are located  
far from the beta-stability line, and thus are the novel interests 
in current nuclear physics. 
Furthermore, these two phenomena can be strongly connected to each other, 
because they have a common basis of physics, namely the nuclear 
pairing correlation. 
The diproton and dineutron correlations are the intrinsic structures 
characterized by a spatial localization of two nucleons of the same kind with 
a large component of the spin-singlet configuration. 
These are exotic features which cannot be reproduced within the 
pure mean-field theory of nuclei, 
and are strongly related to 
the density-dependence of pairing correlations, 
which is an important prediction from the 
modern theory for nuclear structures. 
Recently, the two-proton decays and emissions have attracted 
much interests as an efficient tool to probe the diproton correlation. 
In the observables of the emitted two protons, information on the diproton 
correlation may be reflected. 
In order to establish this idea, further investigations were 
still necessary with a realistic assumptions for calculations. 
Revealing their fundamental relation is expected to provide another 
way to probe the nuclear pairing phenomena 
and the dinucleon correlations, simultaneously 
a development of the advanced nuclear 
theory which covers both the low and high density-regions, and 
both bound and meta-stable systems. 
Nevertheless, few people have discussed this possibility within a realistic 
assumptions of calculations \cite{08Bertulani,12Maru}. 

For this purpose, we have carried out the quantum three-body model 
calculations in this thesis. 
Our calculations provide semi-microscopic 
descriptions for the nuclear pairing correlations. 
By performing model calculations and analyzing their results, 
we have obtained several important conclusions. 
\vspace{24pt}

%\section{Universality of Dinucleon Correlation}
In the former half of this thesis, we have discussed the diproton correlation 
in bound proton-rich nuclei. 
By calculating $^{18}$Ne, $^{18}$O and $^{17}$Ne nuclei, 
we have confirmed that the diproton correlation exists 
in the ground state of these nuclei, similarly to the dineutron correlation 
in neutron-rich nuclei. 
In these systems, a prominent localization of the two protons and 
neutrons are predicted. 
The spin-singlet configuration takes the major contribution 
to this localization. 
It has also been shown that the Coulomb repulsive force between the two
protons does not affect significantly this correlation. 
Even though this repulsion extends 
the density distribution of two nucleons and 
weakens the binding energy, 
its effect is not sufficiently strong 
to destroy the diproton correlation. 
Our calculations have indicated that the effect 
of the Coulomb force reduces 
the pairing energy gap only by about 10 \%, 
being consistent to 
other theoretical studies. 
We have also found that whether the dinucleon 
correlations exist or not 
is insensitive to the total binding energy 
of a three-body systems. 
Namely, the dinucleon correlations can be 
considered not only in 
loosely bound nuclei, 
but also in deeper bound systems. 

From these results, we can conclude that the dinucleon correlations exist 
almost independently of (i) whether the pair consists of protons or neutrons, 
and of (ii) whether that the pair is loosely or 
deeply bound to the core nucleus. 
Eventually, the dinucleon correlations should be discussed as a 
common property of both stable and unstable nuclei. 
\vspace{24pt}

%\section{Diproton Correlation in Two-proton Emission}
In the latter half of this thesis, 
we have focused on the relation between 
the diproton correlation and the two-proton emissions. 
In two-proton emissions, a pair of protons are emitted directly from 
the parent nucleus. 
This process is a typical meta-stable phenomenon governed by the 
quantum tunneling effect, and can be an promising tool to 
probe the diproton correlation. 
It can be expected that if the valence two protons have 
the diproton correlation 
inside the potential barrier, its effect can be reflected in the 
decay observables. 
However, in order to extract information on the diproton correlation 
from the decay observables, 
one has to treat both the quantum meta-stability and 
the many-body property on an equal footing. 

For this purpose, we have developed 
the time-dependent three-body model. 
In this model, the initial \twop-state is defined as a quasi-bound 
state within a phenomenological confining potential. 
The quantum tunneling process can be treated by solving 
the time-dependent Schr\"{o}dinger equation. 
The sensitivity of \twop-emissions to the diproton correlation 
has been discussed by studying its dependence on the initial 
configuration of the two protons, 
with or without a diproton-like clustering. 
We would like to emphasize that our time-dependent approach has an advantage 
to treat the quantum meta-stable processes, enabling us to distinguish 
the essential cause of phenomena. 
Especially, for two-proton emissions, 
several theoretical works have already 
been done, most of which have been based on 
the time-independent formalism. 
However, the relation between the observables 
in the \twop-emissions and the nuclear 
intrinsic structures, 
including the diproton correlation, 
has not been discussed. 
Thus, our present study provides a novel insight 
into these important problems. 

By applying this model to the $^6$Be nucleus, 
which is the simplest two-proton emitter, 
We have obtained several results suggesting 
that the diproton correlation is 
reflected in the decay observables. 
To be more specific, 
first, we have confirmed that the experimental two-proton 
decay-width of 6Be is well reproduced only 
by assuming the diproton correlation in the initial state. 
Furthermore, the decay width is mostly from the spin-singlet configuration. 
Second, the emitted two protons are expected to have a diproton-like cluster 
at the early stage of emissions, due to the pairing correlations 
(that is, the correlated emission). 
We have also performed the same calculations but 
based on the pure mean-field model, 
completely ignoring the pairing correlations. 
In such calculations, the decay width is severely overestimated, 
and the emission process shows mostly the pure sequential emission, 
differing from the case with the pairing correlations. 
These results suggest the importance of the 
diproton correlation in the \twop-emissions. 
\vspace{24pt}
%We also done the same calculations but without the diproton correlation, 
%to find that the decay width is severely underestimated from 
%the experimental results. 
%Two reasons of this discrepancy have been proposed: (i) the first is 
%the absence of the $(s_{1/2})^2$-channels, which take a major 

%\section{Future Tasks}
At this moment, the strong dependence of the \twop-emission on 
the diproton correlation is suggested. 
It means that the 2p-emission can be an effective tool to evince 
the diproton correlation. 
In order to prove it completely, however, there still remain several 
open problems listed below. 

\begin{enumerate}
\item Final-state interactions: 
Our present results have predicted 
a significant correlated emission, 
including the diproton-like clustering 
in the early stage of the two-proton 
emissions of $^{6}$Be. 
However, on the other hand, there is no significant signal 
of the correlated emission in the experimental angular and 
energy correlation patterns. 
A reason for this discrepancy may be the 
final-state interactions (FSIs). 
In the experimental data, 
there is a strong modification of the correlation pattern 
by the FSIs among all the particles. 
Especially, the long-ranged Coulomb forces can extensively affect the 
two protons during their propagation. 
Consequently, the observed data correspond to the late stage of 
the emission process. 
It has yet to be clarified 
how the diproton correlation at the initial and the earlier stages 
are reflected in the experiments. 
In order to address this question, 
by taking the FSIs sufficiently into account, 
we would need to expand our model-space so that 
a longer time-evolution can be carried out. 
However, at the same time, 
it inevitably leads to a serious 
increment of computing costs, and 
one would need to develop more economic 
procedures for the three-body model calculations 
to resolve this problem. 
Such procedures include, {\it e.g.} 
assuming a sophisticated boundary condition to avoid 
the reflection of wave functions at the edge 
of the box, and employing an 
efficient bases to reduce the dimension 
of the Hamiltonian matrix. 

\item Pairing interaction in the asymptotic region: 
In the present study, we have used the nuclear potential 
between two nucleons, by modifying its parameter 
to reproduce the total Q-value of the \twop-emissions. 
However, such modified potential is not consistent to the 
two-nucleon scattering property in vacuum, 
and may lead to a serious 
error in the calculated results. 
Especially, in order to reproduce the 
angular distributions of the two protons, 
we might have to weaken the pairing attraction in the 
asymptotic region. 
For this purpose, one should install the density-dependence 
into the pairing potential, 
or employ a three-body force which works 
as a short-ranged attraction between three particles. 
The effect of the three-body force on the two-nucleon emissions and 
decays is an important problem, as well as whether such an three-body 
force has an physical meaning or not. 

\item Core excitations: 
For the $^6$Be nucleus, our three-body model well reproduces 
the experimental data of the Q-value and decay-width. 
However, for another light two-proton emitter, $^{16}$Ne, 
our calculations have not been successful in reproducing them consistently: 
the \twop-decay width is considerably underestimated even if we employ the 
appropriate parameters for the total Q-value. 
We note that this problem is not only in our calculations but also in other 
studies based on a similar three-body model to ours \cite{02Gri,09Gri_40}. 
Given this discrepancy, we anticipate a limitation 
of the simple three-body model assuming 
an inert, structure-less core, and the importance of the 
core excitations. 
Similar problems have been reported in other studies of, 
{\it e.g.} the nuclear meta-stable processes 
\cite{83Cald,94Varga,00Esb,00Ferre,02Hagino_1p} and 
the structures of nuclei far from the beta-stability line 
\cite{93Sagawa,95Esb_2,99Tost,01Shyam,08Myo,12Moro,13Moriguchi}, 
suggesting that the core excitation plays an important role 
in these phenomena. 
To discard the assumption of an inert core 
may resolve the discrepancy between the 
calculated and the observed \twop-widths. 
Treating it correctly will be an important task in our future works. 
We also note that the explicit treatment of the core excitations 
would be connected to the role of the tensor force, 
because the tensor force causes 
the 2p2h-type of excitations of the core \cite{08Myo}. 
\end{enumerate}

After these improvements, our time-dependent method will be more 
sophisticated and will be able to 
reveal the essential relation between the diproton correlation and 
the two-proton emissions. 
Moreover, similar time-dependent approaches can be 
applied to describe other quantum meta-stable processes 
in few-body systems. 
Especially, considering the dinucleon correlations, 
the most important one may be the two-neutron ($2n$-) emissions. 
In analogy to the relation between the diproton correlation and 
the \twop-emissions, 
the $2n$-emissions can be a powerful tool 
to examine the dineutron correlation 
in neutron-rich nuclei. 
Because of the absence the long-ranged Coulomb FSIs, 
a theoretical treatment may be easier 
than that for the \twop-emissions, 
although the problems of an asymptotic pairing 
interaction and of the 
core excitations still remain. 
Work towards this direction is also an challenging task in the future. 
\vspace{24pt}

%\section{Closing Remarks}
The quantum meta-stability plays an important role in various 
situations in our world. 
Especially, those of three or more fermions, or with many degrees of 
freedom, have been one of the most important 
subjects in modern physics. 
However, in spite of its importance, 
there remain a lot of unknown aspects 
of the quantum meta-stability. 
The theoretical treatment still needs a further development, 
where the two complementary (or competing) frameworks coexist at present. 
Atomic nuclei, which show various radioactive processes, 
are one of the most suitable fields to discuss these physics. 
The knowledge gained in this field can be extended to other 
quantum meta-stable phenomena with many fermions or 
with strong correlations. 
Understanding the quantum meta-stability 
among different scales will be a great 
benchmark in future physics. 
\include{end}
