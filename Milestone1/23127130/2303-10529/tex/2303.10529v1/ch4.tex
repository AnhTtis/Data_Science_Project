\documentclass[a4paper, 12pt]{report}
\include{begin}

\chapter{Diproton Correlation in Light Nuclei} \label{Ch_Results1}
Before we discuss the two-proton (\twop-) emission, 
we first discuss in this Chapter the pairing and dinucleon correlation 
in particle-bound systems. 
To this end, 
we apply the three-body model to several light nuclei. 
Similar theoretical studies have been carried out 
especially for $^6$He and $^{11}$Li, 
which are well known as $2n$-halo as well as $2n$-Borromean nuclei. 
In these light and weakly bound neutron-rich nuclei, 
it has been shown that the pairing correlation plays an important role 
in generating the dineutron correlation, including 
a spatial concentration of two neutrons and the enhancement of the spin-singlet 
configuration \cite{87Hansen, 91Bert, 93Zhukov, 96Vinh, 97Esb, 05Hagi}. 

It is important to notice that the dineutron correlation itself 
can be considered 
even in deeper bound valence neutrons. 
Based on this idea, in this Chapter, we consider 
the $^{18}$O nucleus, 
in which the three-body picture should be reasonable. 
Additionally, in connection to the two-proton radioactivity, 
we will also discuss light proton-rich nuclei, 
$^{18}$Ne and $^{17}$Ne \cite{04Garrido_01,04Garrido_02,05Gri,10Lay}. 
We particularly discuss the following two points; 
(i) whether the diproton correlation exists similarly to the 
dineutron correlation, and 
(ii) whether the dinucleon correlations are limited only for 
weakly bound nucleons or not. 
For the point (i), the main attention will be paid to 
the effect of the Coulomb 
repulsion between two protons, which may break the diproton-like 
configurations to some extent. 
For the point (ii), 
the universality between strongly and weakly bound nucleons will be 
a key issue. 



\section{Dinucleon Correlation in $^{16}$O+N+N Systems: 
$^{18}$Ne and $^{18}$O} \label{Sec_4q1y3}
We start our discussions with applying our three-body model to 
the ground states of $^{16}$O+N+N nuclei, which are expected to give a 
good testing ground for the dinucleon correlations. 
In the following, we only treat pairs of identical 
nucleons in valence orbits. 
Thus, the corresponding systems are $^{18}$O and $^{18}$Ne, 
with N=n and N=p, respectively. 
In their ground states, these nuclei have the spin-parity of $0^+$. 
The core nucleus, $^{16}$O, consists of eight protons and eight neutrons, 
building the doubly-closed nuclear shell-structure (a double-magic nucleus). 
Because of the double-magic nature, the assumption of a rigid core 
is expected to be reasonable for $^{16}$O, and thus the behaviors of 
the two valence nucleons should be 
well described within the three-body model. 
Indeed, the first excited state of $^{16}$O locates at $6.05$ MeV, 
which is higher enough than the single-nucleon energies of valence orbits in 
$^{17}$O$=^{16}$O+n and $^{17}$F$=^{16}$O+p, namely 
$0.87$ MeV $(2s_{1/2})$ and $5.08$ MeV $(1d_{3/2})$ in $^{17}$O, and 
$0.49$ MeV $(2s_{1/2})$ and $5.00$ MeV $(1d_{3/2})$ in $^{17}$F, measured 
from their ground states with a $(1d_{5/2})$-valence neutron and proton, 
respectively \cite{NNDCHP}. 
We assume that the core is always in its ground state 
and has the spin-parity of $0^+$. 
\begin{table}[tb] \begin{center}
  \catcode`? = \active \def?{\phantom{0}} %define `?' as ' '(one-blank).
  \begingroup \renewcommand{\arraystretch}{1.2}
  \begin{tabular*}{\columnwidth}{ @{\extracolsep{\fill}} ccccc ccc} \hline \hline
                            && \multicolumn{2}{c}{$^{17}$F}         && \multicolumn{2}{c}{$^{17}$O}        & \\ \cline{3-4} \cline{6-7}
                            && calc.     & Exp.\cite{NNDCHP} && calc.     & Exp.\cite{NNDCHP} & \\ \hline
  $\epsilon(1d_{5/2})$ (MeV)  && $-0.601$  & $-0.600$         && $-4.199$  & $-4.143$           & \\
  $\epsilon(2s_{1/2})$ (MeV)  && $-0.106$  & $-0.105$         && $-3.235$  & $-3.273$           & \\
  \hline \hline \end{tabular*}
  \endgroup
  \catcode`? = 12 %initialize `?'.
  \caption{ The energies of the $(2s_{1/2})$ and $(1d_{5/2})$ orbits 
in $^{17}$F and $^{17}$O, 
calculated with the core-nucleon two body model. 
For the comparison, the experimental values are also shown \cite{NNDCHP}. 
All the values are measured from the one-proton or one-neutron 
separation thresholds. 
Note that the experimental errors are only the order of 
$1$ keV or smaller. } \label{tb:A17_18}
\end{center} \end{table}

\subsection{Core-Nucleon Subsystems}
We first solve the core-nucleon two-body states. 
For the core-nucleon interaction, 
we use $r_0=1.22$ fm and $a_{\rm core}=0.65$ fm for the 
Woods-Saxon potential (Eq.(\ref{eq:WSP})). 
The parameters of the potential depth are defined as $V_0 = -55.06$ MeV and 
$V_{ls} = 16.71$ ${\rm MeV \cdot fm^2}$, both for $^{17}$F and $^{17}$O. 
These parameters well reproduce the measured energies of 
the $(2s_{1/2})$ and $(1d_{5/2})$ orbits, as shown in Table \ref{tb:A17_18}. 
In Figure \ref{fig:VCNs_18}, the core-nucleon potentials in $(s_{1/2}),(p_{3/2})$ 
and $(d_{5/2})$ channels are plotted. 
\begin{figure*}[tb]
(a) $V_{\rm c-p}$ for $^{17}$F \\
\fbox{\includegraphics[width=0.4\hsize, scale=1.0, trim= 60 50 0 0]{./FIG4011.eps}}\\
(b) $V_{\rm c-n}$ for $^{17}$O \\
\fbox{\includegraphics[width=0.4\hsize, scale=1.0, trim= 60 50 0 0]{./FIG4012.eps}}\\
%\begin{center}
% \begin{tabular}{c} %switch-off the auto-turning
%  \begin{minipage}{0.48\hsize} \begin{center}
%     (a) $V_{\rm c-p}$ for $^{17}$F \\ \fbox{ \includegraphics[height=50truemm, clip, trim = 10 0 5 5]{./y_18NE/VCN_18Ne.eps}} \\
%  \end{center} \end{minipage}
%  \begin{minipage}{0.48\hsize} \begin{center}
%     (b) $V_{\rm c-n}$ for $^{17}$O \\ \fbox{ \includegraphics[height=50truemm, clip, trim = 10 0 5 5]{./y_18O/VCN_18O.eps}} \\
%  \end{center} \end{minipage}
% \end{tabular}
\caption{The core-nucleon potentials in the $(s_{1/2}),(p_{3/2})$ and $(d_{5/2})$ channels in $^{17}$F$\equiv ^{16}$O+p and $^{17}$O$\equiv ^{16}$O+n. } \label{fig:VCNs_18}
%\end{center}
\end{figure*}

\subsection{Uncorrelated Basis} \label{Sec_4183}
The spin-parity of the ground states of $^{18}$Ne and $^{18}$O is $0^+$. 
On the other hand, as we noted, the core $^{16}$O is assumed to have $0^+$. 
Thus, for the uncorrelated two-nucleon basis, we only need the 
$(J,M)^{\pi}=(0,0)^+$ subspace, 
\beq
 \sum_K \ket{\tilde{\Psi}_K^{(J,M)^{\pi}}} \longrightarrow 
 \sum_K \ket{\tilde{\Psi}_K^{(0,0)^{+}}}, 
\eeq
where $K\equiv \left\{ (nlj)_a,(nlj)_b \right\}$ and $\pi=(-)^{l_a+l_b}$. 
From the basic properties of the angular momenta, 
the condition of $J=0$ and $\pi=+$ leads to $j_a=j_b$ and $l_a=l_b$. 
In other words, apart from the radial quantum numbers, two nucleons must 
have the same angular momenta. 
We represent these bases as $\ket{\tilde{\Psi}_{n_a n_b lj}}$ in the following, 
omitting the superscripts $(0,0)^{+}$ for simplicity. 
Using Eqs.(\ref{eq:Aab}) and (\ref{eq:UNCB}), 
the explicit form of uncorrelated wave functions can be written as 
\beqa
 \tilde{\Psi}_{n_a n_b lj}(\bir_1,\bir_2) 
 &=& \frac{1}{\sqrt{2(1+\delta_{n_a,n_b})}} \left[ \sum_m \cgc{0,0}{j,m:j,-m}\phi_{n_a lj,m}(\bir_1)\phi_{n_b lj,-m}(\bir_2) \right. \nonumber \\
 & & \left. -\sum_{m'} \cgc{0,0}{j,m':j,-m'}\phi_{n_a lj,m'}(\bir_2) \phi_{n_b lj,-m'}(\bir_1) \right] \\
 &=& \tilde{\Pi}_{n_a n_b lj}(r_1,r_2) \sum_m \cgc{0,0}{j,m:j,-m} \nonumber \\
 & & \phantom{00000000000000000} \mathcal{Y}_{lj,m}(\ubir_1 \bis_1) \mathcal{Y}_{lj,-m}(\ubir_2 \bis_2), 
\eeqa
where we defined 
\beq
 \tilde{\Pi}_{n_a n_b lj}(r_1,r_2) \equiv \frac{1}{\sqrt{2(1+\delta_{n_a,n_b})}} 
 \left[ R_{n_a lj}(r_1)R_{n_b lj}(r_2) + R_{n_b lj}(r_1)R_{n_a lj}(r_2) \right]. 
\eeq

In the calculations shown in this Chapter, 
the single particle (s.p.) states are solved within the radial box of 
$R_{\rm box} = 30$ fm, with the radial mesh of $dr = 0.1$ fm. 
We take all the s.p. states up to $l_{\rm max}=5$ into account. 
Namely, we include the uncorrelated partial waves from 
$(l_a j_a)\otimes(l_b j_b)=(s_{1/2})^2$ to $(h_{11/2})^2$. 
In order to truncate the model space, the energy cutoff is also introduced. 
We use $\epsilon_a + \epsilon_b \leq E_{\rm cut} = 30$ MeV, 
where $\epsilon_a$ means the energy of the $a$-th s.p. state. 
According to these constraints, 
we adopt about 360 uncorrelated states in our model space. 
This means that the dimension of the total Hamiltonian matrix is about 
$360\times 360$ for $^{18}$Ne and $^{18}$O. 

\subsection{Parameters for Pairing Interaction}
As introduced in the previous Chapter, we employ the density-dependent 
contact (DDC) interaction for the nuclear part of the pairing interaction, 
\beq
 v_{\rm N-N, Nucl.} (\bir_1,\bir_2) 
 = \delta(\bir_1-\bir_2) 
   \left[ v_0 + \frac{v_{\rho}}{1 + \exp \left( \frac{\abs{(\bir_1+\bir_2)/2}-R_{\rho}}{a_{\rho}} \right)} 
   \right]. \label{eq:DDCP_2}
\eeq
Since $E_{\rm cut}=30$ MeV and $a_{\rm nn}=-18.5$ fm, the parameter $v_0$ is fixed as 
$-875.34$ MeV from Eq.(\ref{eq:v0}). 
For the remaining parameters, we use $a_{\rho}=0.65$ fm 
and $R_{\rho}=1.22\cdot 16^{1/3}\cong 3.07$ fm, 
which are equal to those in the Woods-Saxon function 
of $V_{\rm c-N}$ (see Sec.\ref{Sec_4183}). 
The strength of the phenomenological density-dependent part, $v_{\rho}$, 
is adjusted so that the calculated two-nucleon 
binding energies are consistent to the experimental values, 
$S_{\rm 2p}=4.52$ and $S_{\rm 2n}=12.19$ MeV, for $^{18}$Ne and $^{18}$O, 
respectively. 
This condition yields $v_{\rho}=-1.104v_0$ and $-1.159v_0$ for 
$^{18}$Ne and $^{18}$O, respectively. 

Notice that the density-dependent term decreases the pairing attraction 
inside the core 
($\abs{\bir_1+\bir_2} /2 \lesssim R_{\rho}$), 
compared with the bare pairing attraction 
($\abs{\bir_1+\bir_2} /2 \longrightarrow \infty$). 
It corresponds to taking into account the medium effect on the pairing interaction. 

\subsection{Energy Expectational Values}
We now calculate and diagonalize the matrix elements of the total Hamiltonian 
(Eq.(\ref{eq:H3b})), in the way which was explained in the previous Chapter. 
The obtained wave function for the ground state is given as a 
superposition of the $0^+$ uncorrelated basis, 
\beq
 \Phi_{g.s.}(\bir_1,\bir_2) = 
 \sum_{ab} \alpha_{ab} 
 \tilde{\Psi}_{n_a n_b lj}(\bir_1,\bir_2). \label{eq:gsexp}
\eeq
The two nucleon binding, or equivalently, 
separation energies of $^{18}$Ne and $^{18}$O are 
given as the expectation value of the total Hamiltonian, 
\beq
 -S_{\rm 2N} = \Braket{H_{\rm 3b}} \equiv \Braket{\Phi_{g.s.} |H_{\rm 3b} |\Phi_{g.s.}}. 
\eeq
These values calculated by our parameters are shown in the 
first row of Table \ref{tb:A18_1}. 
\begin{table}[t] \begin{center}
  \catcode`? = \active \def?{\phantom{0}} %define `?' as ' '(one-blank).
  \begingroup \renewcommand{\arraystretch}{1.2}
  \begin{tabular*}{\columnwidth}{ @{\extracolsep{\fill}} c c c c c c} \hline \hline
                                        && $^{18}$Ne && $^{18}$O & \\ \hline
    $\Braket{H_{\rm 3b}}=-S_{\rm 2N}$ (MeV) && $-4.52$      && $-12.19$  & \\
                                        &&          &&        & \\
    $\Braket{v_{\rm N-N}}$ (MeV)          && $-4.25$      && $-4.89$   & \\
    $\Braket{v_{\rm N-N, Nucl.}}$ (MeV)     && $-4.80$      && $-4.89$   & \\
    $\Braket{v_{\rm N-N, Coul.}}$ (MeV)     && $??0.55$      && $??0.??$   & \\
    $\Braket{\rm recoil}$ (MeV)         && $-0.45$      && $-0.56$   & \\
    $\Braket{h_1 + h_2}$ (MeV)           && $??0.18$      && $-6.74$   & \\
    $\Braket{V_{\rm c-N_1}+V_{\rm c-N_2}}$ (MeV)  && $-14.25$   && $-22.03$   & \\
                                        &&          &&        & \\
    $\Braket{h_{\rm N-N}}$ (MeV)          && $??6.17$      && $??6.78$   & \\
    $\Braket{h_{\rm c-NN}}$ (MeV)         && $-10.69$      && $-18.97$   & \\
  \hline \hline \end{tabular*}
  \endgroup
  \catcode`? = 12 %initialize `?'.
  \caption{ The energy-expectation values for $^{18}$Ne $\equiv ^{16}$O+p+p and 
$^{18}$O $\equiv ^{16}$O+n+n, calculated with the three-body model. 
The label ``recoil'' means $\bip_1 \cdot \bip_2 / mA_{\rm c}$. 
The experimental two-nucleon separation energies are 
$S_{\rm 2p}=4.52$ and $S_{\rm 2n}=12.19$ MeV for $^{18}$Ne and $^{18}$O, 
respectively \cite{NNDCHP}. 
Notice that $H_{\rm 3b} = h_1 + h_2 + v_{\rm N-N} + ({\rm recoil})$ 
and $= h_{\rm N-N} + h_{\rm c-NN}$. } \label{tb:A18_1}
\end{center} \end{table}

In Table \ref{tb:A18_1}, we summarize several energy-expectation 
values for the these three-body systems. 
According to Eq.(\ref{eq:H3b}), 
the total energies can be decomposed into 
the expectation values of the uncorrelated Hamiltonian, 
the pairing interaction and the recoil term. 
That is, 
\beq
 \Braket{H_{\rm 3b}} = 
 \Braket{h_1+h_2} + \Braket{v_{\rm N-N}} + \Braket{\frac{\bip_1 \cdot \bip_2}{A_{\rm c} m} }. 
\eeq
Note that the uncorrelated Hamiltonian can further be decomposed as 
\beq
 \Braket{h_1+h_2} = \Braket{V_{\rm c-N_1}+V_{\rm c-N_2}} 
                  + \Braket{\frac{\bip_1^2}{2\mu}+\frac{\bip_2^2}{2\mu}}, 
\eeq
where we show only the potential term in Table \ref{tb:A18_1}. 
It is also useful to decompose the total Hamiltonian into two 
relative components. 
One is the Hamiltonian between the core and a pair of 
nucleons, $h_{\rm c-NN}$, 
whereas the other is that between the two nucleons, $h_{\rm N-N}$. 
That is, 
\beqa
 H_{\rm 3b} &=& h_{\rm c-NN} + h_{\rm N-N} \\
 &=& \left[ \frac{p_{\rm c-NN}^2}{2\mu_{\rm c-NN}} 
     + V_{\rm c-N_1}(\bir_1) + V_{\rm c-N_2}(\bir_2) \right] 
     + \left[ \frac{p_{\rm N-N}^2}{2\mu_{\rm N-N}} 
     + v_{\rm N-N}(\abs{\bir_1-\bir_2}) \right], 
\eeqa
with $\mu_{\rm c-NN} = m A_{\rm c}/(A_{\rm c}+2)$ and $\mu_{\rm N-N}=m/2$. 
Notice that, indeed, there is still a coupling between the core-2N and N-N 
subsystems in $h_{\rm c-NN}$, due to the core-nucleon potentials. 
The relative momenta, $\left\{ \bip_{\rm c-NN}, \bip_{\rm N-N} \right\}$, 
can be related to the original momenta in the V-coordinates, 
$\left\{ \bip_1,\bip_2 \right\}$, by the transformation below. 
\beqa
 \bip_{\rm c-NN} &=& \bip_1+\bip_2, \\
 \bip_{\rm N-N}  &=& (\bip_1-\bip_2)/2. 
\eeqa
The expectational values of $h_{\rm N-N}$ and $h_{\rm c-NN}$ are 
also shown in Table \ref{tb:A18_1}. 

As one see in Table \ref{tb:A18_1}, 
the total binding energies are quite different 
between $^{18}$Ne and $^{18}$O. 
This difference is mainly due to the Coulomb interactions 
in $v_{\rm N-N}$ and $V_{\rm c-N}$. 
These Coulomb repulsions are also affected the $\Braket{h_{\rm N-N}}$ and 
$\Braket{h_{\rm c-NN}}$ values in $^{18}$Ne. 
However, apart from the Coulomb repulsions, $\Braket{v_{\rm N-N,Nucl.}}$ and 
$\Braket{\rm recoil}$ have similar values both in $^{18}$Ne and $^{18}$O. 
It means that the pairing correlations 
caused by the nuclear force and the recoil effect 
are not sensitive to the total binding energy, as long as we consider the same 
valence orbits (In these two nuclei, 
the major valence orbit is $(d_{5/2})^2$, as we will 
discuss in the next subsection). 
It is also notable that the ratio of $\Braket{v_{\rm p-p,Coul.}}$ and 
$\Braket{v_{\rm p-p,Nucl.}}$ is about $-0.11$. 
It shows that the Coulomb repulsion reduces the pairing energy by about 10\%. 
This result is consistent to what has been found with, 
a non-empirical pairing energy-density functional for proton 
pairing gaps \cite{09Lesi}, HFB calculations \cite{09Bert, 11Yama} and the 
three-body model calculations \cite{10Oishi, 10Oishi_err}. 
Accordingly, we can conclude that 
the degrees of pairing correlations, 
indicated by $\Braket{v_{\rm N-N}}+\Braket{\rm recoil}$, 
significantly depend neither on the total binding energy, 
nor the existence of the Coulomb repulsions. 

We can estimate the relative momentum between the two nucleons by using 
\beq
 \Braket{h_{\rm N-N}} = \Braket{\frac{p_{\rm N-N}^2}{2\mu_{\rm N-N}}}
                        + \Braket{v_{\rm N-N}}. 
\eeq
From Table \ref{tb:A18_1}, this equation yields $\sqrt{\Braket{p_{\rm N-N}^2}}=98.9$ 
and $\sqrt{\Braket{p_{\rm N-N}^2}}=104.6$ MeV/c for $^{18}$Ne and $^{18}$O, respectively. 
Because of these similar values of $\sqrt{\Braket{p_{\rm N-N}^2}}$, it is expected 
that the spatial distance between the two nucleons are also similar in $^{18}$Ne and $^{18}$O. 
We will confirm this point in the next subsection. 
One should notice, however, even if the diproton or dineutron correlation is confirmed 
in these nuclei, it does not mean the existence of the 
bound subsystem of the two nucleons, 
since the expectational value of $h_{\rm N-N}$ is positive in both systems. 



\subsection{Density Distributions}
We next study the structural properties of the ground states of 
these three-body systems. 
We summarize the results in Table \ref{tb:A18_2}. 
In this Table, $\Braket{r_i} \equiv \sqrt{\Braket{r^2_i}}$ is the 
expectation value of the averaged distance between 
the core and the $i$-th nucleon. 
Likewise, 
\beqa
 && \sqrt{\Braket{r_{\rm N-N}^2}}  = \sqrt{ \Braket{r_1^2+r_2^2 -2r_1 r_2 \cos \theta_{12}} }, \label{eq:rNtoN} \\
 && \sqrt{\Braket{r_{\rm c-NN}^2}} = \sqrt{ \Braket{r_1^2+r_2^2 +2r_1 r_2 \cos \theta_{12}} }/2, \label{eq:rcto2N} 
\eeqa
mean the mean relative distances between the two nucleons and between the core and 
the center of two nucleons, respectively. 
We also show 
$\theta_{12} \equiv \cos^{-1} \left(\Braket{\cos \theta_{12}} \right)$ 
in the 4th row. 
The probability of each angular channel, 
\beq
 \sum_{n_a,n_b} \abs{\alpha_{n_a n_b lj}}^2, 
\eeq
where $\alpha_{n_a n_b lj}$ are the expansion coefficients given by Eq.(\ref{eq:gsexp}), 
is also computed. 
Those of $(s_{1/2})^2$, $(d_{5/2})^2$, $(p_{3/2})^2$ and $(p_{1/2})^2$ 
channels are listed in the 5-8th rows of Table \ref{tb:A18_2}, 
whereas those of the other channels are summarized as ``others''. 
In the last row, we show the ratio of the spin-singlet configuration of 
the two valence nucleons, which can be calculated as 
\beq
 P(S_{12}=0) = \iint d\bir_1 d\bir_2 \rho_{S_{12}=0}(\bir_1,\bir_2), 
\eeq
where $\rho_{S_{12}=0}$ is the spin-singlet density 
given by Eq.(\ref{eq:rhos01}). 
Of course, the ratio of the spin-triplet configuration is given 
by $P(S_{12}=1)=1-P(S_{12}=0)$. 
\begin{table}[tb] \begin{center}
  \catcode`? = \active \def?{\phantom{0}} %define `?' as ' '(one-blank).
  \begingroup \renewcommand{\arraystretch}{1.2}
  \begin{tabular*}{\columnwidth}{ @{\extracolsep{\fill}} c c c c c c} \hline \hline
                       && $^{18}$Ne && $^{18}$O & \\ \hline
    $\Braket{r_1}=\Braket{r_2}$ (fm) && 3.62 && 3.48 & \\
    $\sqrt{\Braket{r^2_{\rm N-N}}}$ (fm)   && 4.62     && 4.37    & \\
    $\sqrt{\Braket{r^2_{\rm c-NN}}}$ (fm)  && 2.79     && 2.70    & \\
    $\Braket{\theta_{12}}$ (deg) && 81.1     && 79.7    & \\
                       &&          &&         & \\
    $(s_{1/2})^2$ (\%)  && ?6.88    && ?5.75    & \\
    $(d_{5/2})^2$ (\%)  && 86.30    && 86.90    & \\
    $(p_{3/2})^2$ (\%)  && ?0.55    && ?0.54    & \\
    $(p_{1/2})^2$ (\%)  && ?0.14    && ?0.14    & \\
    others,$(l=even)^2$ (\%)  && ?3.36   && ?3.41    & \\
    others,$(l=odd)^2?$ (\%)  && ?2.77   && ?3.26    & \\
                       &&          &&         & \\
    $P(S_{12}=0)$ (\%)  && 79.08    && 78.84   & \\
  \hline \hline \end{tabular*}
  \endgroup
  \catcode`? = 12 %initialize `?'.
  \caption{ The structural properties of $^{18}{\rm Ne}$ and $^{18}{\rm O}$, calculated 
with the two-nucleon wave functions. 
The radius of the core nucleus is assumed to be 
$R_0=1.22\cdot 16^{1/3}\cong 3.074$ in the Woods-Saxon potential. } \label{tb:A18_2}
\end{center} \end{table}

In Figures \ref{fig:4181} and \ref{fig:4182}, 
we exhibit the density distributions, 
$\rho(\bir_1,\bir_2)=\abs{\Phi_{g.s.}(\bir_1,\bir_2)}$, 
obtained from the wave functions for the three-body systems. 
We integrate the density for the spin variables, as explained 
with Eqs.(\ref{eq:3117}) and (\ref{eq:rhod}). 
Because of the symmetry in these systems, 
the angular part of the density depends only on 
the opening angle between the valence nucleons, 
$\theta_{12}=\abs{\ubir_2-\ubir_1}$. 
Therefore, for the plotting purpose, we can fix $\ubir_1=\bar{\bi{z}}$ 
without lacking the general information. 
The integrations for the angular variables are replaced as 
\beq
 \iint d\ubir_{1} d\ubir_{2} \longrightarrow 8\pi^2 \int_{0}^{\pi} \sin \theta_{12} d\theta_{12}. 
\eeq
Thus, the density distribution is normalized as 
\beqa
 1 &=& \iint d\bir_1 d\bir_2 \rho(\bir_1,\bir_2) \\
   &=& \int_{0}^{R_{\rm box}} dr_1 \int_{0}^{R_{\rm box}} dr_2 \int_{0}^{\pi} d\theta_{12} 
       \bar{\rho}(r_1,r_2,\theta_{12}), 
\eeqa
with 
\beqa
 && \bar{\rho}(r_1,r_2,\theta_{12}) = 8\pi^2 r_1^2 r_2^2 \sin \theta_{12} \rho(r_1,r_2,\theta_{12}), \\
 && \rho(r_1,r_2,\theta_{12}) = \abs{\Phi_{g.s.}(r_1,r_2,\theta_{12})}^2. \label{eq:456}
\eeqa
The density distribution, $\abs{\Phi_{g.s.}(r_1,r_2,\theta_{12})}^2$, 
can be decomposed into the spin-singlet and the spin-triplet components. 
After some calculations, we get 
\beqa
 \abs{\Phi_{g.s.}(r_1,r_2,\theta_{12})}^2_{S=0} 
 &=& \frac{1}{4\pi} \sum_{l',j'} \sum_{l,j} Q^*_{l'j'}\cdot Q_{lj}(r_1,r_2) 
     \frac{(-)^{l'+l}}{4} \sqrt{\frac{2j'+1}{2l'+1}} \sqrt{\frac{2j+1}{2l+1}} \nonumber \\
 & & \times 2Y^*_{l',0}(\ubir_2) Y_{l,0}(\ubir_2) 
\eeqa
for the spin-singlet, and 
\beqa
 \abs{\Phi_{g.s.}(r_1,r_2,\theta_{12})}^2_{S=1} 
 &=& \frac{1}{4\pi} \sum_{l',j'} \sum_{l,j} Q^*_{l'j'}\cdot Q_{lj}(r_1,r_2) 
     \frac{(-)^{j'+j}}{4} \sqrt{2-\frac{2j'+1}{2l'+1}} \sqrt{2-\frac{2j+1}{2l+1}} \nonumber \\
 & & \times \left[ Y^*_{l',1}(\ubir_2) Y_{l,1}(\ubir_2) + Y^*_{l',-1}(\ubir_2) Y_{l,-1}(\ubir_2) \right] 
\eeqa
for the spin-triplet\footnote{Notice that these formulas are valid 
only for a state with $J^{\pi}=0^+$.} \cite{91Bert,97Esb}. 
Here, we have defined the radial density for each angular channel, 
$Q_{lj}(r_1,r_2)$, as 
\beq
 Q_{lj}(r_1,r_2) \equiv \sum_{n_a>n_b} \alpha_{n_a n_b lj} \tilde{\Pi}(r_1,r_2). 
\eeq
For the angular part, we can use the following formula. 
\beqa
 Y^*_{l',m}(\ubir_2) Y_{l,m}(\ubir_2) &=& (-)^m \sum_{L=|l'-l|}^{l'+l} 
 \sqrt{\frac{(2l'+1)(2l+1)(2L+1)}{4\pi}} \nonumber \\
 && \phantom{000} \times 
 \left( \begin{array}{ccc} l'&l&L \\ 0&0&0 \end{array} \right)
 \left( \begin{array}{ccc} l'&l&L \\-m&m&0 \end{array} \right) Y_{L,0}(\theta_{12}), 
\eeqa
where it depends only on the opening angle $\theta_{12}$. 
\begin{figure*}[t] \begin{center}
 \begin{tabular}{c} %switch-off the auto-turning
  ``$^{18}$Ne (g.s.)'' \\
  \begin{minipage}{0.48\hsize}
     (a) \\ \fbox{ \includegraphics[height=42truemm,scale=1.0, trim = 60 50 0 0]{FIG4021_dab.eps}}\\ %{./y_18NE/g_dab.eps}} \\
     (c) \\ \fbox{ \includegraphics[height=42truemm,scale=1.0, trim = 50 50 0 0]{FIG4022_ang.eps}} %{./y_18NE/g_ang.eps}}
  \end{minipage}
  \begin{minipage}{0.48\hsize}
     (b) \\ \fbox{ \includegraphics[height=42truemm,scale=1.0, trim = 50 50 0 0]{FIG4023_nnd.eps}}\\ %{./y_18NE/g_nnd.eps}} \\
     (d) \\ \fbox{ \includegraphics[height=42truemm,scale=1.0, trim = 50 50 0 0]{FIG4024_drr.eps}} %{./y_18NE/g_drr.eps}}
  \end{minipage}
 \end{tabular}
 \caption{The two-nucleon density distribution of $^{18}$Ne, 
$\rho$, calculated for the ground state within the three-body model. 
Those are plotted for several sets of coordinates as follows. 
(a) with $r_{\rm N-N}$ and $r_{\rm c-NN}$. 
(b) with $r_{\rm N-N}$, integrated for $r_{\rm c-NN}$. 
(c) with the opening angle $\theta_{12}$ between the valence 
nucleons, integrated for $r_1$ and $r_2$. 
(d) with $r_1$ and $r_2$, integrated for $\theta_{12}$. 
In panel (d), the radial weight $r_1^2r_2^2$ is omitted 
to emphasize the peak(s). } \label{fig:4181}
\end{center} \end{figure*}
\begin{figure*}[t] \begin{center}
 \begin{tabular}{c} %switch-off the auto-turning
  ``$^{18}$O (g.s.)'' \\
  \begin{minipage}{0.48\hsize}
     (a) \\ \fbox{ \includegraphics[height=42truemm,scale=1.0, trim = 60 50 0 0]{FIG4031_dab.eps}} \\%{./y_18O/g_dab.eps}} \\
     (c) \\ \fbox{ \includegraphics[height=42truemm,scale=1.0, trim = 50 50 0 0]{FIG4032_ang.eps}} %{./y_18O/g_ang.eps}}
  \end{minipage}
  \begin{minipage}{0.48\hsize}
     (b) \\ \fbox{ \includegraphics[height=42truemm,scale=1.0, trim = 50 50 0 0]{FIG4033_nnd.eps}} \\%{./y_18O/g_nnd.eps}} \\
     (d) \\ \fbox{ \includegraphics[height=42truemm,scale=1.0, trim = 50 50 0 0]{FIG4034_drr.eps}} %{./y_18O/g_drr.eps}}
  \end{minipage}
 \end{tabular}
 \caption{The same to Figure \ref{fig:4181} but of $^{18}$O. } \label{fig:4182}
\end{center} \end{figure*}

In Figure \ref{fig:4181}, we show the density distribution of $^{18}$Ne, 
plotted within several sets of coordinates. 
In panel (a), $\rho(r_1,r_2,\theta_{12})$ is plotted as a function of the 
relative distances, $r_{\rm N-N}$ and $r_{\rm c-NN}$ 
given by Eqs.(\ref{eq:rNtoN}) and (\ref{eq:rcto2N}). 
In panel (b), this function is integrated for $r_{\rm c-NN}$, and plotted only with $r_{\rm N-N}$. 
Conversely, in panel (c), we integrate $\rho$ for $r_1$ and $r_2$, 
and plot it as a function of the opening angle, $\theta_{12}$. 
We also plot the spin-singlet and triplet components separately in this panel. 
Finally, in panel (d), we integrate $\rho$ for the opening angle, and plot it 
as a function of $r_1$ and $r_2$. 
In this plotting, in order to clarify the peak(s), we omit 
the radial weight, $r_1^2 r_2^2$ in Eq.(\ref{eq:456}). 
We show similar plots for $^{18}$O in Figure \ref{fig:4182}. 

As general aspects, 
from Figures \ref{fig:4181}, \ref{fig:4182} and Table \ref{tb:A18_2}, 
we can see the similarity of the two-nucleon 
configurations in $^{18}$Ne and $^{18}$O, consistently to 
the similarity shown in Table \ref{tb:A18_1}. 
It means that the reduction of pairing energies caused by the 
Coulomb repulsion, which is evaluated as about $-10$\% reduction, 
does not affect significantly the two-nucleon densities. 
Because of the weakly binding due to Coulomb repulsions, 
the density of $^{18}$Ne is slightly extended compared with $^{18}$O. 
This tendency is intuitively understood by comparing Figs. 
\ref{fig:4181}(b),(d) and \ref{fig:4182}(b),(d). 
Correspondingly, the expectation values of distances 
in the 1st-4th rows of Table \ref{tb:A18_2} 
show larger values in the case of $^{18}$Ne. 
It is also shown from the probabilities of the angular channels 
that the $(d_{5/2})^2$ wave is 
dominant in both two cases, 
whereas the $(s_{1/2})^2$ wave has also considerable contributions. 
The distinct three peaks in panels (a) and (c) are mainly due to 
the $(d_{5/2})^2$ component, 
although the mixing of the other waves occurs with the pairing correlations, 
where the Coulomb repulsion plays a minor role. 

We note that the mean distance between the two nucleons, $r_{\rm N-N}$, 
shows a considerably smaller value, compared with the total diameter of 
the whole nucleus, estimated as $\simeq 2r_{\rm c-NN} \simeq 5.5$ fm. 
%Note also that these values are consistent to the relative nucleon-nucleon 
%momenta, $\Braket{p^2_{\rm N-N}}$, estimated as $\simeq 100$ MeV/c 
%from Table. \ref{tb:A18_1}. 

\subsection{Diproton and Dineutron Correlations}
In the ground states of both $^{18}$Ne and $^{18}$O, 
the dinucleon correlation, 
or at least, its tendency can be seen. 
The spatial localization is apparent at 
($r_{\rm N-N},r_{\rm c-NN}$) $\simeq $ (2fm, 3fm) in 
Figs. \ref{fig:4181}(a) and \ref{fig:4182}(a). 
It corresponds to the two nucleons confined in $r_{\rm N-N} \leq 2$ fm, 
which corresponds to the first peak of $\rho_{12}(r_{\rm N-N})$ shown in 
Figs. \ref{fig:4181}(b) and \ref{fig:4182}(b). 
We also find that $\rho_{12}(r_{\rm N-N})$ has almost 
all the components inside $r_{\rm N-N} \leq 8$ fm. 
According to the Ref.\cite{06Mats}, this result 
may be connected to the pairing densities in 
the nuclear matter at $\rho/\rho_0 = 0.1-0.01$ 
(see Fig.\ref{fig:2006Matsuo}), 
even though the assumption of nuclear matters cannot 
be translated directly to the conditions in finite nuclei. 

In Figs. \ref{fig:4181}(c) and \ref{fig:4182}(c), 
the corresponding angular distributions take the asymmetric forms, 
and have the highest peak at the small opening angle, 
$\theta_{12} \simeq \pi/6$. 
Indeed, this asymmetry is an important character of the 
dinucleon correlations: 
it is caused by the mixing of different parities of the core-nucleon 
partial system \cite{84Catara}. 
If we exclude this parity-mixing, 
the angular distributions have the perfect 
symmetric forms. 
We will check this point in the next section. 

We also note that in the asymmetric angular distributions, 
the most localized peak is mainly from the spin-singlet 
configuration, consistently to the definition of the 
dinucleon correlations. 
In both two nuclei, the spin-singlet has a 
probability of about 80\%. 
On the other hand, if we take the naive mean-field approximation, 
the two nucleons have the pure $(d_{5/2})^2$ wave, 
where the contribution from the spin-singlet 
is determined exactly as 60\% from the properties of the 
CG and the $9j$-coefficients. 
Thus, the pairing correlation works to enhance the spin-singlet 
configuration compared to the pure $(d_{5/2})^2$ wave. 

As interim conclusions, we have confirmed that the diproton correlation, 
which is characterized as the spatial localization of two-proton 
density mainly carried out by the spin-singlet configuration, 
is able to occur similarly to the dineutron correlation in 
the mirror nucleus. 
This result is consistent to the minor effect of the Coulomb repulsion, 
estimated by about -10\% reduction of the pairing energy. 
%It is also suggested that structural properties of two nucleons 
%are insensitive to the total binding energies. 
%In order to check this point more clearly, 
%we next try to calculate the similar system but with a lower binding energy. 
%For this purpose, $^{17}$Ne nucleus is the most suitable one. 

\section{Diproton Correlation in $^{17}$Ne}
In the previous section, it has been suggested that the structural 
properties of the two nucleons are insensitive 
to the total binding energy. 
In order to investigate the dependence of the dinucleon correlations 
on the binding energy, 
we next study the 
$^{17}$Ne nucleus, which has been famous as a \twop-Borromean nucleus. 
This nucleus is also a candidate to have the \twop-halo structure, 
due to the loosely bound two protons \cite{95Zhukov,05Gri}. 
In this system, two valence protons are bound with significantly 
small binding energies, $S_{\rm 2p} = 0.93$ MeV \cite{NNDCHP}. 
Therefore, it provides another testing ground to investigate 
the diproton correlation in a weakly bound system, 
in comparison with the diproton correlation in a deeply 
bound nucleus, $^{18}$Ne. 
In this section, 
we will also perform case-studies with different theoretical conditions, 
in order to gain a deeper understanding of the diproton correlation. 
Although these theoretical conditions may not 
correspond to realistic situations, 
those will be helpful to know what is the essential point in the diproton and 
dineutron correlations. 

\subsection{Set up for Calculations}
It is known that there is no bound state in $^{16}$F $\cong ^{15}$O+p, but 
four resonances in the low-lying region. 
These low-lying levels are shown in Figure \ref{fig:17Nelv}. 
The ground state of $^{15}$O has $1/2^-$, whereas its first excited state 
is located at $5.18$ MeV above the ground state. 
Because this excited energy is sufficiently high, 
the first and the second low-lying resonances in $^{16}$F 
at 0.536 and 0.729 MeV can be interpreted 
as the coupled states of $^{15}$O$_{g.s.}(1/2^-)$+$p(s_{1/2})$. 
Likewise, the 3rd and the 4th resonances 
at 0.960 and 1.257 MeV can be interpreted 
as those of $^{15}$O$_{g.s.}(1/2^-)$+$p(d_{5/2})$. 
\begin{figure*}[htbp] \begin{center}
\fbox{\includegraphics[width = 0.4\hsize]{./FIG4041_17Ne.eps}}
\caption{The level scheme of the $^{17}$Ne nucleus and its isotones. 
All the experimental values, printed by black letters, are quoted 
from the database \cite{NNDCHP}, 
except for the first excited state of $^{17}$Ne \cite{06Gri}. 
For the decay widths of $^{16}$F, the experimental data are 
$\Gamma(0^-)=40 \pm 20$ keV and 
$\Gamma(1^-)<40$ keV for the lower two levels, whereas 
$\Gamma(2^-)=40 \pm 30$ keV and 
$\Gamma(3^-)<15$ keV for the upper two levels \cite{NNDCHP}. 
Note that all these levels decay via one-proton emission, where 
the branching ratios to other decay-modes are negligible. 
The values printed by the red and blue letters for $^{16}$F 
indicate the spin-averaged s.p. energies. 
The decay widths of these levels are theoretically computed as 
$\Gamma_0(s_{1/2})=63$ keV and 
$\Gamma_0(d_{5/2})=8.2$ keV, respectively. } \label{fig:17Nelv}
\end{center} \end{figure*}

In this case, we neglect the internal spin of the core, and 
fit the parameters in the core-proton potential 
to the spin-averaged s.p. energies of 
$(s_{1/2})$ and $(d_{5/2})$ states. 
These averaged levels are shown in Figure \ref{fig:17Nelv} 
by the red and blue letters. 
To this end, we calculate the phase shift, $\delta_{lj}(E)$, 
and its derivative for the energy $E$. 
The calculated result is fitted with a function, which consists 
of a pure Breit-Wigner distribution and a smooth background. 
That is, 
\beq
 \frac{d\delta_{lj}(E)}{dE} = \frac{\Gamma_0/2}{\Gamma_0^2/4 + (E_0-E)^2} 
                            + \frac{dC_{lj}(E)}{dE}, \label{eq:sigde}
\eeq
where the right-hand side is the empirical formula. 
How to derive Eq.(\ref{eq:sigde}) and calculate $\delta_{lj}(E)$ 
in the left-hand side is summarized as Appendix \ref{Ap_Scat_2body}. 
%\beq
% \frac{dC_{lj}(E)}{dE} = b E + c, 
%\eeq
%with fitting parameters $b$ and $c$. 

The calculated results and fitted functions are shown in 
Figure \ref{fig:409378}. 
At this moment, the smooth background is neglected. 
By fitting the right-hand side in Eq.(\ref{eq:sigde}) 
to the calculated left-hand side, 
we can extract the resonant energy, $E_0$ and 
the decay width, $\Gamma_0$ of considering resonances. 
With $r_0=1.22$ fm, $a_{\rm core}=0.65$ fm, $V_0 = -53.68$ MeV and 
$V_{ls} = 15.06$ ${\rm MeV \cdot fm^2}$ for the $^{15}$O-proton 
potential, we obtained 
$E_0=0.679$ MeV with $\Gamma_0=63$ keV for the $(s_{1/2})$-resonance, and 
$E_0=1.131$ MeV with $\Gamma_0=8.2$ keV for the $(d_{5/2})$-resonance. 
Obtained values of $E_0$ are consistent 
with the empirical resonant energies shown in Fig. \ref{fig:17Nelv}. 
Notice that the values of $r_0$ and $a_0$ are similar to those 
used for $^{18}$Ne and $^{18}$O. 
The other s.p. states are also solved within this potential. 
\begin{figure*}[t] \begin{center}
 \begin{tabular}{c}
  ``$d\delta_{lj} /dE$ of $^{15}$O-proton'' \\
  \begin{minipage}{0.48\hsize} \begin{center}
     (a) for $(s_{1/2})$ \\ \fbox{ \includegraphics[height=48truemm,scale=1, trim = 50 50 10 0]{FIG4057.eps}} \\%{./y_16F_3/defit_s.eps}} \\
  \end{center} \end{minipage}
  \begin{minipage}{0.48\hsize} \begin{center}
     (b) for $(d_{5/2})$ \\ \fbox{ \includegraphics[height=48truemm,scale=1, trim = 50 50 10 0]{FIG4058.eps}} \\%{./y_16F_3/defit_d.eps}} \\
  \end{center} \end{minipage}
 \end{tabular}
 \caption{The derivative of the phase shift, $\delta_{lj}(E)$ 
for the energy $E$ in the scattering of $^{15}$O-proton. 
The calculated results are shown with symbols, whereas the fitted 
functions given by Eq.(\ref{eq:sigde}) are 
plotted with lines. } \label{fig:409378}
\end{center} \end{figure*}



In order to solve the ground state of $^{17}$Ne, 
we employed the similar setting to that for $^{18}$Ne and $^{18}$O. 
Namely, we take all the s.p. states up to $l_{\rm max}=5$ into account. 
Since the ground state of $^{17}$Ne has the same spin-parity to $^{15}$O, 
which is $(1/2^-)$, 
the two-proton uncorrelated basis can be reduced only to the $0^+$ subspace. 
Consequently, we adopt from $(s_{1/2})^2$ to $(h_{11/2})^2$ partial waves. 
We use $\epsilon_a + \epsilon_b \leq E_{\rm cut} = 30$ MeV as the energy cutoff 
for the uncorrelated basis, providing about 360 bases. 
For the pairing interaction, except for $v_{\rho}$ 
in the density-dependent part, 
we adopt the same parameters as those for $^{18}$Ne and $^{18}$O. 
In order to reproduce the empirical binding energy of the two protons, 
$S_{\rm 2p}=0.93$ MeV \cite{NNDCHP}, we use $v_{\rho}=-1.131v_0$. 

In the setting of the calculations introduced above, 
we treat the pairing correlations as fully as possible, by mixing 
all the uncorrelated bases up to $(h_{11/2})^2$. 
We often refer to this condition as ``full-mixing'' or just ``full'' 
in the following. 
In addition to this ``full-mixing'' case, 
we perform two other sets of calculations with different conditions 
explained below, 
which reveal the essential aspect of the dinucleon correlations. 
\begin{itemize}
 \item {\it Limitation of Core-proton Parity}: \\
As shown in Table \ref{tb:A18_2}, the contributions from 
the partial waves with $(l=odd)^2$ are quite small in $^{18}$Ne and $^{18}$O. 
This situation is expected to occur also in $^{17}$Ne, where the two protons 
may mainly have the $(d_{5/2})^2$ and $(s_{1/2})^2$ configurations. 
Thus, we examine the case by excluding the $(l=odd)^2$ partial waves 
from the calculation, for comparison with the full-mixing case. 
It corresponds to a situation where the parity-mixing in the partial 
core-proton system is prohibited. 
In order to reproduce the \twop-binding energy, we tune the parameter 
$v_{\rho}$ as $v_{\rho}=-0.9981v_0$. 
It meas that, without the parity mixing, we need 
a stronger pairing attraction than that in the full case. 
Notice also that the recoil term, which couples two bases satisfying 
$\abs{l'-l}=1$, do not contribute in this case 
(see Eq.(\ref{eq:3050}) also). 
In the following, we call this setting as ``$(l=even)^2$'' case. 

 \item {\it No Pairing}: \\
In this case, we completely omit the pairing correlations. 
It means that we ignore all the nuclear attraction, 
the Coulomb repulsion and 
the recoil term between the two protons in Eq.(\ref{eq:H3b}). 
Thus, the total Hamiltonian is only 
the uncorrelated Hamiltonian, $h_1+h_2$, 
and the two-proton wave functions becomes identical to 
one of the uncorrelated bases with 
single angular channel. 
We take the $(d_{5/2})^2$ state, which is expected to be the 
major channel in the full-mixing case. 
Because of the lack of the pairing correlations, 
we cannot reproduce the empirical 
binding energy, $S_{\rm 2p}=0.93$ MeV, 
with the original core-proton interaction used in 
the ``full'' and ``$(l=even)^2$ only'' cases. 
Therefore, we inevitably modify $V_{\rm c-N}$. 
We use $V_0=-57.663$ MeV, which yields the {\it bound} s.p. state 
in the $(d_{5/2})$ channel with $\epsilon(d_{5/2})\simeq -0.93/2$ MeV. 
Notice that it is no longer possible to reproduce the Borromean character 
in this case. 
In the following, we call this setting as ``no pairing''. 
\end{itemize}

By comparing among these three cases, 
we will make an attempt to extract the essential character of 
the dinucleon correlation. 
The results and discussions are summarized below. 

\subsection{Energy Expectation Values}
We first discuss the energetic properties tabulated in Table \ref{tb:A17_1}. 
In the full-mixing case, these expectation values show similar 
results to those for $^{18}$Ne shown in Table \ref{tb:A18_1}, 
except for $\Braket{h_1+h_2}$ and $\Braket{h_{\rm c-NN}}$. 
This difference can be interpreted as an effect of the 
weak attractive potential in the core-proton subsystem. 
On the other hand, it is implied that the effects of the pairing 
correlations, 
as well as the relative energy between the two protons, 
are not significantly dependent on the total binding energy. 
The effect of the Coulomb repulsion in the pairing correlation is 
estimated again as a $10\%$ reduction over the nuclear attraction. 
These conclusions are similar to those obtained in Sec. \ref{Sec_4q1y3}. 

In the $(l=even)^2$ case, the situation is significantly different. 
Even though we employ a stronger pairing attraction than in 
the full case, the pairing energy, $\Braket{v_{\rm N-N}}$ has a 
higher value. 
On the other hand, the expectation value of the energy of 
the proton-proton subsystem, $\Braket{h_{\rm N-N}}$ becomes lower. 
Consequently, 
the relative proton-proton kinetic energy, 
$\Braket{p^2_{\rm N-N}/2\mu_{\rm N-N}}=\Braket{h_{\rm N-N}}-\Braket{v_{\rm N-N}}>0$, 
has the lower value than that in the full-mixing case. 
The lower value of $\Braket{p^2_{\rm N-N}/2\mu_{\rm N-N}}$ suggests that 
the spatial distribution between the two protons is 
further expanded, and possibly deviated from a 
diproton-like configuration. 

Finally, in the no pairing case, it is worthwhile to point out that 
$\Braket{h_{\rm N-N}}$ and $\Braket{h_{\rm c-NN}}$ have similar 
values to those in the full case. 
From this result, one may infer that the pairing correlations are 
well mocked up in the mean-field, $V_{\rm c-N}$. 
However, compared with the full-mixing case, 
a discussion on the \twop-configuration is not straight forward, 
because 
the modification of $V_{\rm c-N}$ makes the proton-proton 
subsystem considerably different in the two cases. 
Thus, we will check directly the difference in the spatial 
\twop-distributions in the next subsection. 
\begin{table}[tb] \begin{center}
  \catcode`? = \active \def?{\phantom{0}} %define `?' as ' '(one-blank).
  \begingroup \renewcommand{\arraystretch}{1.2}
  \begin{tabular*}{\columnwidth}{ @{\extracolsep{\fill}} ccccc c} \hline \hline
                                        && \multicolumn{3}{c}{$^{17}$Ne} & \\ \cline{3-5}
                                        && full   & $(l=even)^2$ only & no pairing, deeper $V_{\rm c-p}$ & \\ \hline
    $\Braket{H_{\rm 3b}}=-S_{\rm 2N}$ (MeV) && $-0.93$  & $-0.93$        & $-0.93$  & \\
    &&&&& \\
    $\Braket{v_{\rm N-N}}$ (MeV)          && $-4.00$  & $-3.86$        & $??0.??$  & \\
    $\Braket{v_{\rm N-N, Nucl.}}$ (MeV)     && $-4.53$  & $-4.34$        & $??0.??$  & \\
    $\Braket{v_{\rm N-N, Coul.}}$ (MeV)     && $??0.53$ & $??0.48$       & $??0.??$  & \\
    $\Braket{\rm recoil}$ (MeV)         && $-0.40$  & $??0.??$       & $??0.??$  & \\
    $\Braket{h_1 + h_2}$ (MeV)           && $??3.47$ & $??2.93$       & $-0.93$  & \\
    $\Braket{V_{\rm c-N_1} + V_{\rm c-N_2}}$ (MeV) && $-10.64$ & $-11.75$       & $-12.82$  & \\
    &&&&& \\
    $\Braket{h_{\rm N-N}}$ (MeV)          && $??5.61$ & $??3.02$        & $??5.57$  & \\
    $\Braket{h_{\rm c-NN}}$ (MeV)         && $-6.54$  & $-3.95$         & $-6.50$  & \\
  \hline \hline \end{tabular*}
  \endgroup
  \catcode`? = 12 %initialize `?'.
  \caption{ The energy expectation values for the ground state of 
$^{17}$Ne, calculated with the three-body model of $^{15}$O+p+p. 
See the text for the details of each calculational setting. 
The experimental two-proton separation energy is 
$S_{\rm 2p}=0.93$ MeV \cite{NNDCHP}. 
All quantities are evaluated in the same manner as 
in Table \ref{tb:A18_1}. } \label{tb:A17_1}
\end{center} \end{table}

\subsection{Structural Properties}
The results for the structural properties of $^{17}$Ne are shown 
in Table \ref{tb:A17_2} and 
Figs. \ref{fig:441}, \ref{fig:442} and \ref{fig:443}. 
All the quantities are evaluated and plotted in the same manner as 
those in Table \ref{tb:A18_2} and Figure \ref{fig:4181}. 
\begin{table}[tb] \begin{center}
  \catcode`? = \active \def?{\phantom{0}} %define `?' as ' '(one-blank).
  \begingroup \renewcommand{\arraystretch}{1.2}
  \begin{tabular*}{\columnwidth}{ @{\extracolsep{\fill}} ccccc c} \hline \hline
                       && \multicolumn{3}{c}{$^{17}$Ne}    & \\ \cline{3-5}
                       && full  & $(l=even)^2$ only & no pairing, deeper $V_{\rm c-p}$ & \\ \hline
    $\sqrt{\Braket{r^2_1}}=\sqrt{\Braket{r^2_2}}$ (fm) && 3.92  & 3.70  & 3.67 & \\
    $\sqrt{\Braket{r^2_{\rm N-N}}}$ (fm)        && 4.98  & 5.24  & 5.20 & \\
    $\sqrt{\Braket{r^2_{\rm c-NN}}}$ (fm)       && 3.03  & 2.62   & 2.60 & \\
    $\Braket{\theta_{12}}$ (deg)      && 81.2  & 90.0  & 90.0 & \\
    &&&&& \\
    $(s_{1/2})^2$ (\%)        && 14.63 & 13.24 & ??0.?  & \\
    $(d_{5/2})^2$ (\%)        && 77.97 & 82.64 & 100.?  & \\
    $(p_{3/2})^2$ (\%)        && ?0.79 & ?0.?? & ??0.?  & \\
    $(p_{1/2})^2$ (\%)        && ?0.23 & ?0.?? & ??0.?  & \\
    others,$(l=evev)^2$ (\%) && ?3.87 & ?4.12 & ??0.? & \\
    others,$(l=odd)^2?$ (\%) && ?2.51 & ?0.?? & ??0.? & \\
    &&&&& \\
    $P(S_{12}=0)$ (\%)        && 82.62 & 82.39 & 60.00 & \\
  \hline \hline \end{tabular*}
  \endgroup
  \catcode`? = 12 %initialize `?'.
  \caption{The structural properties in the ground state of 
$^{17}$Ne, calculated with the three-body model of $^{15}$O+p+p. 
See the text for the details of each calculational setting. 
The radius of the core nucleus is estimated 
as $R_0=r_0A_c^{1/3}=1.22\cdot 15^{1/3}\cong 3.009$ fm. } \label{tb:A17_2}
\end{center} \end{table}

In the full-mixing case, we first find that the general features 
of $^{17}$Ne are similar to those of $^{18}$Ne, although the binding energy 
is remarkably smaller. 
This smaller energy yields the sizable extension of 
the \twop-density distribution, 
shown in Figs. \ref{fig:441}(a), (b) and (d). 
Consequently, the expectation values of the radial parameters 
become larger as one can see in Table \ref{tb:A17_2}. 
This extension of the \twop-wave function is consistent to an increment 
of the $(s_{1/2})^2$ wave, which has a long tail outside 
the core-nucleon potential. 
We also stress that the major components in the \twop-wave function are 
$(s_{1/2})^2$ and $(d_{5/2})^2$, 
reflecting the existence of the $(s_{1/2})$- 
and $(d_{5/2})$-resonant states in the core-proton subsystem. 
The dominance of the $(d_{5/2})^2$ wave in the ground state can be 
understood from Eq.(\ref{eq:ME12g}), 
which indicates that the matrix element 
of a coupled operator, $O(\xi_1,\xi_2)$, has a larger value for the 
uncorrelated basis with larger $j$. 
Therefore, to gain a deeper binding energy, the two protons 
tend to occupy the $(d_{5/2})^2$. 
\begin{figure*}[t] \begin{center}
  \begin{tabular}{c} %switch-off the auto-turning
    ``$^{17}$Ne (g.s.), $v_{\rm N-N}=$DDC+Coul., full-mixing'' \\
    \begin{minipage}{0.48\hsize}
     (a) \\ \fbox{ \includegraphics[height=42truemm,scale=1, trim = 60 50 0 0]{FIG4061_dab.eps}}\\ %{./y_17NE/g_dab.eps}} \\
     (c) \\ \fbox{ \includegraphics[height=42truemm,scale=1, trim = 50 50 0 0]{FIG4062_ang.eps}}%{./y_17NE/g_ang.eps}}
    \end{minipage}
    \begin{minipage}{0.48\hsize}
     (b) \\ \fbox{ \includegraphics[height=42truemm,scale=1, trim = 50 50 0 0]{FIG4063_nnd.eps}}\\ %{./y_17NE/g_nnd.eps}} \\
     (d) \\ \fbox{ \includegraphics[height=42truemm,scale=1, trim = 50 50 0 0]{FIG4064_drr.eps}}%{./y_17NE/g_drr.eps}}
    \end{minipage}
  \end{tabular}
\caption{The density distribution of the 
valence two protons in $^{17}$Ne, computed with 
the three-body model of $^{15}$O+p+p. 
In this case, all the uncorrelated bases up to $(h_{11/2})^2$ are 
fully taken into 
account (the full-mixing case). 
The coordinates for all the panels are defined in a 
similar way as Figure \ref{fig:4181}. } 
\label{fig:441} 
\end{center} \end{figure*}

In order to discuss the diproton correlation, it is useful to 
compare the results obtained with the three settings for 
calculations. 
First, in panels (a), (b) and (c) of Fig. \ref{fig:441} for the 
full-mixing case, 
the localization of the density 
at small values of $r_{\rm p-p}$ and of $\theta_{12}$ 
can be seen, 
whereas the localization 
cannot be observed in both 
Fig. \ref{fig:442} for the $(l=even)^2$ only case and 
Fig. \ref{fig:443} for the no-pairing case. 
Notice that this localization in the full-mixing case 
is mainly due to the spin-singlet configuration, 
as shown in Fig. \ref{fig:441}(c). 
This implies the existence of the diproton correlation, 
or at least, its tendency in the ground 
state of weakly bound $^{17}$Ne. 

We also point out that the spatial localization of the two protons 
in $^{17}$Ne is less significant than that in $^{18}$Ne. 
For example, by comparing panels (b) and (d) 
in Figs. \ref{fig:4181} and \ref{fig:441}, 
one can find that the \twop-density distribution shows a larger 
extent in $^{17}$Ne. 
The expectational values, 
$\Braket{r_{N-N}}$ and $\Braket{\theta_{12}}$, 
also have larger values in $^{17}$Ne. 
This result can be interpreted as the effect of the density-dependence 
of the pairing correlation. 
If the density is too low, 
the pairing correlation decreases, 
and eventually vanishes in the zero-density limit \cite{06Mats}. 
In $^{17}$Ne, the valence two protons possibly feel 
the surrounding density, 
which is rather low due to the weakly bound system, and 
causes the diproton correlation to be weaker. 
Whether this tendency leads to the reduction of the diproton correlation 
from unbound systems or not will be a critical point when we analyze 
\twop-emissions. 
We will discuss this point in Chapter \ref{Ch_Results2}. 
\begin{figure*}[t] \begin{center}
  \begin{tabular}{c} %switch-off the auto-turning
    ``$^{17}$Ne (g.s.), $v_{\rm N-N}=$DDC+Coul., $(l=even)^2$ only'' \\
(Figure is hidden in open-print version.)
%    \begin{minipage}{0.48\hsize}
%     (a) \\ \fbox{ \includegraphics[height=42truemm,scale=1, trim = 60 50 0 0]{./y_17NE/even/g_dab_ev.eps}} \\
%     (c) \\ \fbox{ \includegraphics[height=42truemm,scale=1, trim = 50 50 0 0]{./y_17NE/even/g_ang_ev.eps}}
%    \end{minipage}
%    \begin{minipage}{0.48\hsize}
%     (b) \\ \fbox{ \includegraphics[height=42truemm,scale=1, trim = 50 50 0 0]{./y_17NE/even/g_nnd.eps}} \\
%     (d) \\ \fbox{ \includegraphics[height=42truemm,scale=1, trim = 50 50 0 0]{./y_17NE/even/g_drr_ev.eps}}
%    \end{minipage}
  \end{tabular}
\caption{The same as Figure \ref{fig:441} but for the $(l=even)^2$ case. } 
\label{fig:442} 
\end{center} \end{figure*}
\begin{figure*}[htb] \begin{center}
  \begin{tabular}{c} %switch-off the auto-turning
    ``$^{17}$Ne (g.s.), no pairing'' \\
(Figure is hidden in open-print version.)
%    \begin{minipage}{0.48\hsize}
%     (a) \\ \fbox{ \includegraphics[height=42truemm,scale=1, trim = 60 50 0 0]{./y_17NE/none/g_dab_nc.eps}} \\
%     (c) \\ \fbox{ \includegraphics[height=42truemm,scale=1, trim = 50 50 0 0]{./y_17NE/none/g_ang_nc.eps}}
%    \end{minipage}
%    \begin{minipage}{0.48\hsize}
%     (b) \\ \fbox{ \includegraphics[height=42truemm,scale=1, trim = 50 50 0 0]{./y_17NE/none/g_nnd_nc.eps}} \\
%     (d) \\ \fbox{ \includegraphics[height=42truemm,scale=1, trim = 50 50 0 0]{./y_17NE/none/g_drr_nc.eps}}
%    \end{minipage}
  \end{tabular}
  \caption{The same as Figure \ref{fig:441} but without the 
pairing correlations, 
for which a modified core-proton potential is employed. 
Notice that the two protons have a pure $(d_{5/2})^2$ configuration. } 
\label{fig:443} 
\end{center} \end{figure*}

In the $(l=even)^2$ case, 
the density $\rho_{12}(r_{\rm N-N})$ shown in Fig. \ref{fig:442}(b) 
shows a larger extent than that in the full case. 
This is consistent with the larger value of $\sqrt{\Braket{r^2_{\rm N-N}}}$ 
in Table \ref{tb:A17_2}. 
The angular distribution in Fig. \ref{fig:442}(c) has a completely 
symmetric form, yielding $\Braket{\theta_{12}}=90$ (deg). 
From these results, we can conclude that the parity-mixing in the 
core-nucleon subsystem is indispensable to induce the spatial 
localization of two nucleons. 
In other words, if this parity-mixing is forbidden or excessively suppressed, 
two nucleons cannot be localized even with a strong pairing interaction. 

One should remember that, even though the two protons are not 
localized, the $(l=even)^2$ case does not mean the complete 
lack of the pairing correlations. 
In Table \ref{tb:A17_2}, the enhancement of the spin-singlet configuration 
can be seen, as well as in the full-mixing case. 
Comparing Fig. \ref{fig:442}(c) with Fig. \ref{fig:443}(c), 
one can find that the peaks at $\theta_{12} \simeq \pi/6$ and $5\pi/6$ 
become more significant in the $(l=even)^2$ case 
than those in the no pairing case. 
These show that a part of the pairing correlations is taken into account, 
even though the diproton correlation is missing. 



\section{Interaction-Dependence of Diproton Correlation}
It is also useful to check a model-dependence of the results in 
the previous section, which showed the possibility of the dinucleon 
correlations in weakly bound systems. 
%Especially, different descriptions for the pairing interaction 
%may cause the serious disagreement of our results. 
In order to clarify this point, 
in this section, we repeat the same calculations 
for $^{17}$Ne, but employing a different pairing interaction. 

\subsection{Minnesota Potential}
To this end, we adopt the ``Minnesota potential'' for the 
pairing interaction instead of 
the DDC potential. 
This potential was originally proposed by Thompson {\it et.al.}, 
in order to solve nucleon-nucleus scattering problems within the microscopic 
``resonating group method'' \cite{77Thom}. 
For the proton-proton and the neutron-neutron systems, 
the potential is given as 
\beq
 v_{\rm N-N}(r_{12}) = v_0 e^{-b_0 r^2_{12}} - v_1 e^{-b_1 r^2_{12}} \label{eq:Minne}
 + \alpha \hbar c \frac{e^2}{r_{12}} \frac{(1+\hat{t}^{(3)}_{1}) (1+\hat{t}^{(3)}_{2})}{4}, 
\eeq
with $r_{12} \equiv \abs{\bir_1 - \bir_2}$ and $\alpha$ is the 
fine structure constant. 
The Coulomb part is necessary only in the proton-proton case. 
For this interaction, the nuclear part has finite rages, 
in contrast to the zero-range DDC pairing. 
The first term is a phenomenological repulsive part, 
whereas the second term describes the pairing attractions. 
The original parameters in Eq.(\ref{eq:Minne}) were given as 
\beq
 v_0 = 200.~{\rm MeV},~~~~~b_0 = 1.487~{\rm fm}^2, 
\eeq
for the repulsive part, whereas 
\beq
 -v_1 = \left\{ \begin{array}{cc} -178.~{\rm MeV},&(S_{12}=0) \\
                                 -91.85~{\rm MeV},&(S_{12}=1) \end{array} \right. ~~~~~
  b_1 = \left\{ \begin{array}{cc} 0.639~{\rm fm}^2,&(S_{12}=0) \\
                                  0.465~{\rm fm}^2,&(S_{12}=1) \end{array} \right. \label{eq:Minne3}
\eeq
for the attraction, which depends on the total spin of the two nucleons. 
In Eq.(\ref{eq:Minne3}), the parameters for the spin-singlet 
configuration were determined so as to reproduce the proton-proton, 
spin-singlet, $s$-wave scattering properties. 
On the other hand, for the spin-triplet configuration, 
those parameters were determined consistently to the neutron-proton, 
spin-triplet, $s$-wave scattering 
properties \footnote{In the neutron-proton, spin-triplet case, 
Thompson {\it et al.} modified the potential from Eq.(\ref{eq:Minne}) by 
using an additional parameter \cite{77Thom}. 
We do not use this n-p potential in this thesis. }. 
In several theoretical studies based on few-body models for 
finite nuclei, the Minnesota potential has been employed 
with reasonable successes \cite{77Thom,01Myo,04Suzu,07Hagi_03,10Myo}. 
In Figure \ref{fig:vmin}, the potentials for the proton-proton and 
the neutron-neutron channels are shown. 
\begin{figure*}[tb] \begin{center}
\begin{tabular}{c} %switch-off the auto-turning
  \begin{minipage}{0.48\hsize} \begin{center}
     (a) $S_{12}=0$ \\ \fbox{ \includegraphics[height=50truemm,scale=1, trim = 60 50 10 0]{FIG4091.eps}}\\ %{./y_etc/g_vmin00.eps}} \\
  \end{center} \end{minipage}
  \begin{minipage}{0.48\hsize} \begin{center}
     (b) $S_{12}=1$ \\ \fbox{ \includegraphics[height=50truemm,scale=1, trim = 60 50 10 0]{FIG4092.eps}}\\ %{./y_etc/g_vmin01.eps}} \\
  \end{center} \end{minipage}
\end{tabular}
\caption{The original Minnesota potentials in the $S_{12}=0$ 
(the left panel) and the $S_{12}=1$ (the right panel) channels. 
In the proton-proton case, the Coulomb term is also included. 
The dashed curves are for the proton-proton, while the solid 
lines are for the neutron-neutron. } \label{fig:vmin}
\end{center} \end{figure*}

In our calculations for $^{17}$Ne, however, the original set of 
parameters underestimates the empirical \twop-separation energy. 
We thus weaken $v_0$ to $178.1$ MeV, which effectively 
enhances the pairing attraction. 
One should be conscious of that, in this case, a stronger 
pairing attraction is adopted inside nuclei, compared with 
the bare pairing attraction in the vacuum. 
It is quite contrary to the previous case with the DDC pairing interaction, 
where we needed to reduce the pairing attraction inside nuclei to reproduce 
the \twop-binding energy. 
We do not know exactly the origin for the difference, 
but one possibility is due to the range of the pairing attraction. 

To calculate the matrix elements 
of the Minnesota potential based on Eq.(\ref{eq:ME12g}), 
we have to know the multi-pole expansion formula of the Gaussian function. 
This was given by Swiatecki \cite{51Swi}, and thus we do not 
show it here. 
We stress that, except using the Minnesota pairing, our calculations 
were performed within the same assumption to that 
in the full mixing case with DDC pairing. 

\subsection{Results and Comparison}
The results obtained with the Minnesota pairing are summarized in Table 
\ref{tb:17Ne_min} and Figure \ref{fig:461}, 
in the same manner as in the previous section. 
Qualitatively, energy expectational values are independent of the 
choice of the pairing interaction. 
Indeed, $\Braket{v_{\rm N-N,Nucl.}}$ is similar to one another, 
even it is evaluated slightly lower than 
that in the DDC+Coul. case, which 
maybe due to a character of the finite-range potential. 
\begin{table}[tb] \begin{center}
  \catcode`? = \active \def?{\phantom{0}} %define `?' as ' '(one-blank).
  \begingroup \renewcommand{\arraystretch}{1.2}
     \begin{tabular*}{\columnwidth}{c|c} \hline \hline
     \multicolumn{2}{c}{$^{17}$Ne (g.s.), $v_{\rm N-N}=$Minne.+Coul., full-mixing} \\ \hline
     \begin{minipage}{0.45\hsize} \begin{center}
         \begin{tabular*}{\columnwidth}{ @{\extracolsep{\fill}} cc}
         $\Braket{H_{\rm 3b}}=-S_{\rm 2N}$ (MeV) & $-0.93$ \\
         & \\
         $\Braket{v_{\rm N-N}}$ (MeV) & $-3.52$ \\
         $\Braket{v_{\rm N-N, Nucl.}}$ (MeV) & $-4.03$ \\
         $\Braket{v_{\rm N-N, Coul.}}$ (MeV) & $??0.51$ \\
         $\Braket{\rm recoil}$ (MeV) & $-0.31$ \\
         $\Braket{h_1 + h_2}$ (MeV) & $??2.90$ \\
         $\Braket{V_{\rm c-N_1} + V_{\rm c-N_2}}$ (MeV) & $-11.1$ \\
         & \\
         $\Braket{h_{\rm N-N}}$ (MeV) & $??5.41$ \\
         $\Braket{h_{\rm c-NN}}$ (MeV) & $-6.34$ \\ & \\ & \\ \end{tabular*}
     \end{center} \end{minipage} 
     &
     \begin{minipage}{0.45\hsize} \begin{center}
         \begin{tabular*}{\columnwidth}{ @{\extracolsep{\fill}} cc}
         $\sqrt{\Braket{r^2_1}}=\sqrt{\Braket{r^2_2}}$ (fm) & 3.81 \\
         $\sqrt{\Braket{r^2_{\rm N-N}}}$ (fm) & 4.95 \\
         $\sqrt{\Braket{r^2_{\rm c-NN}}}$ (fm) & 2.89 \\
         $\Braket{\theta_{12}}$ (deg) & 83.37 \\
         & \\
         $(s_{1/2})^2$ (\%) & 12.12 \\
         $(d_{5/2})^2$ (\%) & 84.99 \\
         $(p_{3/2})^2$ (\%) & ?0.42 \\
         $(p_{1/2})^2$ (\%) & ?0.10 \\
         others,$(l=evev)^2$ (\%) & ?0.82 \\
         others,$(l=odd)^2?$ (\%) & ?1.55 \\
         & \\
         $P(S_{12}=0)$ (\%) & 72.73 \\ \end{tabular*}
     \end{center} \end{minipage} \\ \hline \hline
     \end{tabular*}
  \endgroup \catcode`? = 12 %initialize `?'.
  \caption{The properties of the ground state of $^{17}$Ne, 
obtained with the Minnesota interaction. 
All the uncorrelated bases up to $(h_{11/2})^2$ are taken into account. 
All the quantities are evaluated in a similar way as in 
Table \ref{tb:A17_1} and \ref{tb:A17_2}. } \label{tb:17Ne_min}
\end{center} \end{table}
%In this thesis, however, we do not discuss in detail such a 
%quantitative differences. 

Even though the relative proton-proton energy is less evaluated 
with the Minnesota potential, 
the tendency of the diproton correlation can be apparent again 
in this case. 
The structural properties shown in Table \ref{tb:17Ne_min} and 
the \twop-density distribution shown in Figure \ref{fig:461} 
exhibit the similar behaviors to those in the DDC+Coul. case. 
The spatial localizations at 
$r_{\rm p-p}\simeq 2$ fm in Fig. \ref{fig:461}(a), 
and at $\theta_{12}\simeq \pi/6$ in Fig. \ref{fig:461}(c) are 
clearly seen 
by taking the core-proton parity-mixing into account. 
The spin-singlet configuration carries the main part of 
this localization as one can see in Fig. \ref{fig:461}(c). 
One may concern the spin-singlet ratio which is slightly smaller 
than that in the DDC+Coul. case. 
However, it is significantly larger than $P(S_{12}=0)=60$ \% 
in the no pairing case, and the enhancement 
of the spin-singlet configuration remains qualitatively 
also with the Minnesota potential. 
\begin{figure*}[t] \begin{center}
  \begin{tabular}{c} %switch-off the auto-turning
    ``$^{17}$Ne (g.s.), $v_{\rm N-N}=$Minne.+Coul., full-mixing'' \\
    \begin{minipage}{0.48\hsize}
     (a) \\ \fbox{ \includegraphics[height=42truemm,scale=1, trim = 60 50 0 0]{FIG4101_dab.eps}}\\ %{./y_17NE/min05/g_dab.eps}} \\
     (c) \\ \fbox{ \includegraphics[height=42truemm,scale=1, trim = 50 50 0 0]{FIG4102_ang.eps}} %{./y_17NE/min05/g_ang.eps}}
    \end{minipage}
    \begin{minipage}{0.48\hsize}
     (b) \\ \fbox{ \includegraphics[height=42truemm,scale=1, trim = 50 50 0 0]{FIG4103_nnd.eps}}\\ %{./y_17NE/min05/g_nnd.eps}} \\
     (d) \\ \fbox{ \includegraphics[height=42truemm,scale=1, trim = 50 50 0 0]{FIG4104_drr.eps}} %{./y_17NE/min05/g_drr.eps}}
    \end{minipage}
  \end{tabular}
  \caption{The same as Figure \ref{fig:441}, 
but obtained with the Minnesota interaction. } \label{fig:461}
\end{center} \end{figure*}

\section{Summary of this Chapter}
We demonstrated the appearance of the dinucleon correlations in 
the ground states of several light nuclei based on the 
core plus two-nucleon model. 
It is found that the Coulomb repulsive force plays a minor 
role in the pairing correlation, and the diproton correlation 
can be realized in proton-rich nuclei, 
in a similar way as the dineutron correlation in neutron-rich nuclei. 
That is, 
our evaluation of the Coulomb effect, which is about a 10\% 
reduction against the nuclear pairing attraction, is 
not sufficient to affect the spatial localization of the two protons. 
We also confirmed that these correlations are not significantly 
dependent on the total binding energy of nuclei. 
In other words, the dinucleon correlations are present 
not only in weakly bound but also in stable nuclei, as long as 
the pairing correlation is sufficiently large. 

As mentioned in Chapter \ref{Ch_2}, 
the dinucleon correlations in the bound, ground states are not 
easy to be directly probed. 
The sensitivity of observables to these intrinsic structures is 
still not evident, although various possibilities 
have been explored. 
Facing on this situation, we propose a possibility to verify 
the diproton correlation with the two-proton emissions and 
radioactive decays. 
Because the valence two protons are spontaneously 
emitted without no disturbance from the external fields, 
observing their wave functions may be a 
direct probe into the diproton correlation in a resonant state. 
Based on this idea, we will extend our analysis to 
a meta-stable three-body system after this Chapter. 
In the next Chapter, we will summarize the historical back-ground of 
the two-proton emissions and radioactive decays. 
We will introduce the time-dependence into our three-body model 
in order to describe the two-proton emissions in Chapter \ref{Ch_TDM}. 
Employing this model, 
we will the diproton correlation associated with 
the two-proton emissions, whose results will be present 
in Chapters \ref{Ch_Results2} and \ref{Ch_Results3}. 

%Correlations between the first and the second protons can be clarified 
%by observing distributions of densities. 
%We assume that the first proton is found at $\bir_1=\bir$. 
%In this situation, the remaining density distribution for 
%the second proton can be given as 
%\beq
% \rho_{\rm p_2} (\bir_2;\bir_1=\bir) = \frac{\rho(\bir_2,\bir)}{ \int d\bir_2 \rho(\bir_2,\bir)} \label{eq:dzx2}
%\eeq
%where it is normalized as $\int d\bir_2 \rho_{\rm p_2} (\bir_2;\bir_1=\bir) \equiv1$. 
%\begin{figure*}[tb] \begin{center}
%  \begin{tabular}{c} %switch-off the auto-turning
%  \catcode`? = \active \def?{\phantom{0}} %define `?' as ' '(one-blank).
%    \fbox{ \begin{minipage}{0.3\hsize} \begin{center} ``full'' \\
%      $z_1=2.4$ fm \\ \includegraphics[width=\hsize, clip, trim = 30 15 15 15]{./y_17NE/g_dzx_24.eps} \\
%      $z_1=3.6$ fm \\ \includegraphics[width=\hsize, clip, trim = 30 15 15 15]{./y_17NE/g_dzx_36.eps} \\
%      $z_1=4.8$ fm \\ \includegraphics[width=\hsize, clip, trim = 30  0 15 15]{./y_17NE/g_dzx_48.eps} \\
%    \end{center} \end{minipage}}
%    \fbox{ \begin{minipage}{0.3\hsize} \begin{center} ``$(l=even)^2$ only'' \\
%      $z_1=2.4$ fm \\ \includegraphics[width=\hsize, clip, trim = 30 15 15 15]{./y_17NE/even/g_dzx_24.eps} \\
%      $z_1=3.6$ fm \\ \includegraphics[width=\hsize, clip, trim = 30 15 15 15]{./y_17NE/even/g_dzx_36.eps} \\
%      $z_1=4.8$ fm \\ \includegraphics[width=\hsize, clip, trim = 30  0 15 15]{./y_17NE/even/g_dzx_48.eps} \\
%    \end{center} \end{minipage}}
%    \fbox{ \begin{minipage}{0.3\hsize} \begin{center} ``no pairing'' \\
%      $z_1=2.4$ fm \\ \includegraphics[width=\hsize, clip, trim = 30 15 15 15]{./y_17NE/none/g_dzx_24.eps} \\
%      $z_1=3.6$ fm \\ \includegraphics[width=\hsize, clip, trim = 30 15 15 15]{./y_17NE/none/g_dzx_36.eps} \\
%      $z_1=4.8$ fm \\ \includegraphics[width=\hsize, clip, trim = 30  0 15 15]{./y_17NE/none/g_dzx_48.eps} \\
%    \end{center} \end{minipage}}
%  \catcode`? = 12 %initialize `?'.
%  \end{tabular}
%  \caption{The density of propabilities of the second proton in $^{17}$Ne (g.s.), when the first 
%proton is fixed at $\bir_1=z_1\bi{e}_z$. 
%All distributions are normalized according to Eq.(\ref{eq:dzx2}). } \label{fig:4501}
%\end{center} \end{figure*}

\include{end}
