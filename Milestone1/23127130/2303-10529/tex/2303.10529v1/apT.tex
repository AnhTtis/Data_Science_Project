\documentclass[a4paper,12pt]{report}
\include{begin}

\chapter{General Formalism of Time-Dependent Method} \label{Ap_TDM}
We briefly introduce the basic formalism of the time-dependent framework for 
quantum meta-stable phenomena in this Chapter. 
We mainly assume the (multi-)particle emissions. 
However, almost all formulas in the following can be generally applied to 
various kinds of quantum meta-stable phenomena, which has been described 
within several structure or reaction models. 

\section{Continuum Expansion}
First we assume the eigen-states of the total Hamiltonian, 
$H$, which is responsible for the time-evolution\footnote{For simplicity, 
the total Hamiltonian is assumed to be static, 
and is not dependent on the wave function self-consistently. 
The similar time-dependent theory with non-static Hamiltonian 
can be considered. 
However, it is over complicated and beyond the coverage of this thesis.}. 
Considering the degeneration, they can be formulated as 
\beqa
 && H \ket{E, i(E)} = E \ket{E, i(E)}, \\
 && \Braket{E',j(E')|E,i(E)} = \delta(E'-E) \delta_{ji}. 
\eeqa
Here the eigen-energy $E$ is real so that we consider the pure 
Hermite space, 
in contrast to other theoretical methods which employ the space with 
complex eigen-energies, such as Berggren space. 
The $i(E)$ identifies one of the degenerating states with the same energy $E$. 
However, as a basic rule in the following, we omit these labels for simplicity. 
If it is necessary to recount the degeneration, 
we remember these labels only for some important formulas. 

Adopting these eigen-states as bases, 
an arbitrary meta-stable state, $\ket{\psi_0}$, can be expanded as 
\beq
  \ket{\psi_0} = \int dE \mu (E) \ket{E}, 
\eeq
where $\left\{ \mu(E) \right\}$ are the expanding coefficients. 
The normalization is represented as 
\beq
  1 = \Braket{\psi_0|\psi_0} = \int dE \abs{\mu (E)}^2. 
\eeq
Physical properties of $\ket{\psi_0}$ are not clear at this moment. 
Those are characterized by the expanding coefficients. 
In the following, we discuss about physics described with $\ket{\psi_0}$. 



\section{Time Evolution}
The quantum meta-stable phenomena, 
including particle(s)-decays and emissions, 
can be treated as the time-developments of meta-stable systems. 
Assuming $\ket{\psi_0}$ as the initial state, 
we consider the time-evolution via $H$. 
\beqa
 \ket{\psi (t)} 
 &=& e^{-itH/\hbar} \ket{\psi_0 } \\
 &=& \int dE \mu (E) e^{-itE/\hbar} \ket{E}. \label{eq:psi_t}
\eeqa
The expectational value of $H$, indicated as $E_0$, 
obviously conserves during the time-evolution. 
\beq
  E_0 \equiv \Braket{\psi_0|H|\psi_0} = \Braket{\psi(t)|H|\psi(t)} = \int dE E \abs{\mu (E)}^2. 
\eeq
This conservation coincide with that the energy-spectrum, 
defined by $\left\{ \abs{\mu(E)}^2 \right\}$, is invariant during 
the time-evolution. 
For a particle(s)-decay or emission, $E_0$ corresponds to the 
Q-value carried out by the emitted particle(s). 

The survival coefficient, $\beta(t)$, is defined as the overlap between 
the initial and the present states. 
\beqa
  \beta(t) &\equiv & \Braket{\psi_0 | \psi(t)} \\
  &=& \int dE' \mu (E') \int dE \mu (E) \Braket{E' | e^{-itE/\hbar} | E} \nonumber \\
  &=& \int dE \abs{\mu (E)}^2 e^{-itE/\hbar}. \label{eq:Krylov}
\eeqa
Note that $\beta(0) = 1$. 
In Eq.(\ref{eq:Krylov}), 
the survival coefficient can be given by the Fourier transformation of 
the invariant energy-spectrum. 
This is nothing but the ``Krylov-Fock theorem'' \cite{47Kry,89Kuku}. 
As one of the important observable properties, 
the survival probability can be given by 
\beq
  P_{\rm surv}(t) = \abs{\beta(t)}^2, 
\eeq
which leads to the decay-rule in this meta-stable process. 
In the next section, we discuss the correspondence between 
the actual decay-rule and the invariant energy-spectrum. 



\section{Exponential Decay-Rule}
The exponential decay-rule has been popular especially in the radioactive processes. 
That is 
\beq
  P(t) = e^{-t/\tau} P(0), 
\eeq
where $P(t)$ means the probability of a radioactive nucleus to survive 
with its characteristic lifetime, $\tau$. 
As the first step to discuss the decay-rule, 
we proof that this exponential decay-rule is equivalent to 
the ideal Breit-Wigner (BW-) distribution in the energy-spectrum. 
The squared expanding coefficients, $\left\{ \abs{\mu (E)}^2 \right\}$, 
are assumed to have the BW-distribution, 
or equivalently, the form of Cauchy-Lorentz function whose center and 
full width at the half maximum (FWHM) are $E_0$ and $\Gamma_0$, respectively. 
That is 
\beq
  \abs{\mu (E)}^2 = \frac{1}{\pi} \frac{(\Gamma_0 /2)}{(E-E_0)^2 + (\Gamma_0 /2)^2} \label{eq:BW1}
\eeq
with $-\infty \leq E \leq \infty$. 
Or equivalently, 
\beq
  \ket{\psi_0} = \int_{-\infty}^{\infty} dE \sqrt{\frac{\Gamma_0}{2\pi}} 
  \frac{e^{ia(E)} }{(E_0 - i\Gamma_0/2) - E} \ket{E}, \label{eq:BW2}
\eeq
where $\left\{ e^{ia(E)} \right\}$ with $a(E) \in \mathbb{R}$ are 
arbitrary phase-factors. 
If we consider the degeneration, Eq.(\ref{eq:BW1}) is modified as 
\beq
  \abs{\mu (E)}^2 = \sum_{i(E)} \abs{\mu (E,i(E))}^2 
  = \frac{1}{\pi} \frac{(\Gamma_0 /2)}{(E-E_0)^2 + (\Gamma_0 /2)^2}. 
\eeq
The normalization is obviously given by 
\beq
 1 = \Braket{\psi (t) | \psi (t)} 
   = \int_{-\infty}^{+\infty} dE \frac{1}{\pi} \frac{(\Gamma_0 /2)}{(E-E_0)^2 + (\Gamma_0 /2)^2}. 
\eeq
For the ideal BW-distribution, however, 
how to define the expectational value of $H$ is not obvious. 
We should be careful for the range of the integration 
which is critical for the 1st moment of BW-distributions. 
At this moment, we assume the isotropic infinite range with 
the central value of $E_0$. 
\beq
  \int_I dE \equiv \lim_{R \rightarrow \infty} \int_{E_0-R}^{E_0+R} dE. 
\eeq
Thus, the 1st moment of the energy is identical to the Cauchy's 
principal value, namely the center of the distribution. 
\beqa
 \Braket{\psi_0 | H | \psi_0} 
 &=& \Braket{\psi (t) | H | \psi (t)} \\
 &=& \int_I dE' \mu (E') \int_I dE \mu (E) 
     \Braket{E' | H | E} \nonumber \\
 &=& \int_I dE' \mu (E') \int_I dE \mu (E) 
     \delta(E'-E) E \nonumber \\
 &=& \int_I dE \abs{\mu (E)}^2 E = E_0, 
\eeqa
In the following, we omit the subscript $I$. 
Substituting Eq.(\ref{eq:BW1}) into Eq.(\ref{eq:Krylov}), 
the survival coefficient can be derived by picking up the residue at 
the pole of $E = E_0 -i\Gamma_0 /2$, namely 
\beqa
 \beta(t) &=& \frac{1}{\pi} \int dE \frac{(\Gamma_0 /2)}{(E-E_0)^2 + (\Gamma_0 /2)^2} e^{-itE/\hbar} = \cdots \nonumber \\
 &=& e^{-it(E_0 -i\Gamma_0/2)/\hbar}. \label{eq:xpdr}
\eeqa
Then the survival probability yields the well-known exponential 
decay-rule, such that 
\beq
 P_{\rm surv}(t) = \abs{\beta(t)}^2 = e^{-t/\tau}, 
\eeq
where the $\tau = \hbar/\Gamma_0$ is the lifetime of this 
meta-stable state \cite{89Bohm, 09Konishi}. 



\section{Practical Problems}
In practice, however, the situation is not so simple. 
First of all, there is the lower limit for the expansion on the energy space, 
consistently to the threshold of the emission. 
Fixing it as $E=0$, we should modify Eq.(\ref{eq:BW2}) as 
\beq
 \int_{-\infty}^{\infty} dE \longrightarrow 
 \int_{0}^{\infty} dE. 
\eeq
Second, the actual energy spectra are not limited 
to have the perfect BW-distributions. 
This discordance leads to the deviation from 
the exponential decay-rule \cite{12Pons}. 
Especially, if the decay width is comparably broad to 
the Q-value: $E_0 \approx \Gamma_0$, 
assuming the BW-distribution may diverge from reality. 

For the numerical calculations, 
we intuitively have to concern two additional affairs. 
The first is the discretization of the continuum space, and 
the second is the energy cutoff, $E_{\rm cut}$. 
Thus, Eq.(\ref{eq:psi_t}) should be modified as 
\beq
 \ket{\psi(t)} = \sum_{N} F_N (0) e^{-itE_N/\hbar} \ket{E_N}, 
\eeq
where $E_N \leq E_{\rm cut}$. 
%This formula is identical to Eqs.(\ref{eq:ex_E}, \ref{eq:excf_E}) in Chapter \ref{Ch_TDM}. 

Finally, we mention the effect of the initial configuration (IC). 
One cannot discuss the meta-stable process without concerning how the initial state 
should be defined. 
The initial state, especially of the particle(s)-emission, 
is usually characterized as, for instance, 
the state where the emitted particles are confined in the narrow region, and/or 
the state which obeys the outgoing boundary condition. 
However, even with these constraints, 
there may be different ICs which follow almost the same decay-rule. 
Possibly, obtained results after the time-evolution may significantly depend on 
the selection of the IC, even though the decay-rule itself hardly changes. 
In this thesis, we employed the phenomenological procedure 
with confining potentials to fix it. 
The more realistic way to fix the IC is, of course, considerable. 
Discussing this effect is, however, beyond the scope of this thesis. 



\include{end}
