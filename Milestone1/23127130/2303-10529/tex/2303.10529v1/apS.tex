\documentclass[a4paper,12pt]{report}
\include{begin}

\chapter{Two-Body Scattering with Spherical Potential} \label{Ap_Scat_2body}
Our goal in this Appendix is to derive the fitting formula for the phase-shift 
of two-body scattering problems. 
For simplicity, we assume that the potential between 
two particles is spherical. 
For quantum resonances in two-body systems, one can usually 
solve the asymptotic waves analytically. 
The phase shift and its derivative can be computed by using these asymptotic 
waves, where it indicates the pole(s) of the S-matrix for the resonance. 
Even if one is interested in the scattering problem with three 
or more particles, 
it is often necessary to solve the partial two-body systems in order to, 
{\it e.g.} prepare the fine two-body interactions. 

\section{Solutions in Asymptotic Region}
Assuming the relative wave function as 
$\phi_{ljm}(\bir,\bis) = R_{lj}(r) \mathcal{Y}_{ljm}(\ubir,\bis)$, 
the radial equation of this problem reads 
\beq
  \left[ -\frac{\hbar^2}{2\mu}\left\{ \frac{d^2}{dr^2} - \frac{l(l+1)}{r^2} \right\} + V_{lj}(r) - E \right] U_{lj} (r,E) = 0, 
\eeq
where we defined $U_{lj}(r,E) \equiv rR_{lj}(r)$ from the radial wave function. 
The relative energy, $E$, for the scattering problem satisfies 
\beq
 E > \lim_{r \rightarrow \infty} V_{lj}(r) \equiv 0. 
\eeq
The equivalent but more convenient radial equation takes the form given by 
\beq
  \left[ \frac{d^2}{d\rho^2} - \frac{l(l+1)}{\rho^2} - \frac{V_{lj}(r)}{E} + 1 \right] 
  U_{lj} (\rho) = 0, \label{eq:apC01}
\eeq
where $\rho \equiv kr$ defined with the relative momentum, $k(E) \equiv \sqrt{2E\mu}/\hbar$. 
In numerical calculations, this type of equations can be solved with, 
{\it e.g.} Numerov method explained in Chapter \ref{Ch_3body}. 

To calculate the phase-shift and also other important quantities, 
asymptotic solutions of Eq.(\ref{eq:apC01}) are often necessary. 
In the following, we note these solutions for two major potentials 
frequently used in nuclear physics. 

\subsection{Short-Range Potential}
Short-range potentials, including nuclear interactions, are characterized as 
\beq
 \lim_{r \rightarrow \infty} V_{lj}(r) < \mathcal{O} (r^{-2}). 
\eeq
The asymptotic condition can be satisfied at $\rho \gg 1$. 
A general solution in this region can be written as 
\beq
 \frac{U_{lj}(\rho)}{\rho} = C_1 j_l(\rho) + C_2 n_l(\rho), 
\eeq
with spherical Bessel and Neumann functions, such as 
\beqa
 j_l (kr) &\longrightarrow & \frac{1}{kr}  \sin \left(kr-l\frac{\pi}{2} \right), \\
 n_l (kr) &\longrightarrow & \frac{-1}{kr} \cos \left(kr-l\frac{\pi}{2} \right). 
\eeqa
Or equivalently, the out-going and in-coming waves can be given as 
\beqa
 h^{(+)}_l (kr) &\equiv & j_l(kr) + i n_l(kr)
    \longrightarrow  \frac{1}{ikr}   e^{ i\left( kr-l\frac{\pi}{2} \right)}, \\
 h^{(-)}_l (kr) &\equiv & j_l(kr) - i n_l(kr)
    \longrightarrow  \frac{-1}{ikr}  e^{-i\left( kr-l\frac{\pi}{2} \right)}. 
\eeqa
Using the coefficients $A_{lj}$ and $B_{lj}$, a general solution takes the form of 
\beqa
 \frac{U_{lj}(kr)}{kr} &=& A_{lj}(E)h^{(+)}_l (kr) + B_{lj}(E)h^{(-)}_l (kr) \nonumber \\
 &=& B_{lj}(E) [S_{lj}(E)h^{(+)}_l (kr) + h^{(-)}_l (kr) ], 
\eeqa
with the S-matrix, $S_{lj}(E) \equiv A_{lj}(E)/B_{lj}(E)$. 
Note that $\abs{S_{lj}(E)}^2 =1$ from the conservation law of the flux. 
Introducing the phase-shift, $\delta_{lj}(E)$ as $S_{lj}(E) \equiv e^{2i\delta_{lj}(E)}$, 
we can get the well-known asymptotic form of $U_{lj}$. 
\beqa
  \frac{U_{lj}(kr)}{kr} &\longrightarrow& \frac{B_{lj}(E)}{ikr} \nonumber 
  \left[ S_{lj}(E) e^{ i\left( kr-l\frac{\pi}{2} \right)} - e^{-i\left( kr-l\frac{\pi}{2} \right)} \right] \\
  && = \frac{B_{lj}(E)e^{i\delta_{lj}(E)}}{ikr} \nonumber 
  \left[ e^{ i\left( kr-l\frac{\pi}{2}+\delta_{lj}(E) \right)} - e^{-i\left( kr-l\frac{\pi}{2}+\delta_{lj}(E) \right)} \right] \\
  && \propto \frac{1}{kr} \sin \left[ kr-l\frac{\pi}{2}+\delta_{lj}(E) \right]. \label{eq:apC05}
\eeqa
Note that $\delta_{lj}(E) \in \mathbb{R}$ since $\abs{S_{lj}(E)}^2 =1$. 


\subsection{Coulomb Potential}
It is formulated as 
\beq
 V_{lj}(r) = V(r) = \alpha \hbar c \frac{Z_1 Z_2}{r}, \phantom{00} 
 \alpha \equiv \frac{e^2}{4\pi \epsilon_0 \cdot \hbar c}. 
\eeq
Defining Sommerfeld parameter, $\eta\equiv Z_1 Z_2 \alpha \mu c/\hbar k$, Eq.(\ref{eq:apC01}) 
can be written as 
\beq
  \left[ \frac{d^2}{d\rho^2} - \frac{l(l+1)}{\rho^2} - \frac{2\eta}{\rho} + 1 \right] U_{l} (\rho,\eta) = 0. 
\eeq
With this Coulomb potential, the asymptotic condition can be satisfied at $\rho \gg 2\eta$. 
A general solution takes the form as 
\beq
 \frac{U_{l}(\rho,\eta)}{\rho} = C_1\frac{F_l(\rho,\eta)}{\rho} + C_2\frac{G_l(\rho,\eta)}{\rho}, 
\eeq
where $F_l$ and $G_l$ are the Coulomb functions \cite{72Abramo}. 
Precise derivations of these functions are found in, {\it e.g.} textbook \cite{07Sasakawa}. 
Their asymptotic forms read 
\beqa
 \frac{1}{kr}F_l(kr,\eta) &\longrightarrow & \frac{1}{kr} \sin \left(kr-l\frac{\pi}{2}-\eta\ln 2kr+a_l(\eta) \right), \\
 \frac{1}{kr}G_l(kr,\eta) &\longrightarrow & \frac{1}{kr} \cos \left(kr-l\frac{\pi}{2}-\eta\ln 2kr+a_l(\eta) \right), 
\eeqa
with $a_l(\eta)=\arg \Gamma(l+1+i\eta)$, which is independent of $kr$. 
There is also an iterative formula for $a_l(\eta)$ as 
\beq
 a_{l+1}(\eta) = a_{l}(\eta) + \tan^{-1} \frac{\eta}{l+1}. 
\eeq
Eliminating these unimportant phases, the outgoing and incoming waves 
can be formulated as \cite{07Sasakawa}, 
\beqa
 u^{(+)}_l (kr,\eta) &\equiv & e^{-ia_l(\eta)} \left[G_l(kr,\eta) + i F_l(kr,\eta)\right]
    \longrightarrow e^{ i\left( kr-l\frac{\pi}{2}-\eta\ln 2kr \right)}, \\
 u^{(-)}_l (kr,\eta) &\equiv & e^{ ia_l(\eta)} \left[G_l(kr,\eta) - i F_l(kr,\eta)\right]
    \longrightarrow e^{-i\left( kr-l\frac{\pi}{2}-\eta\ln 2kr \right)}. 
\eeqa
By using these functions, a general solution can be replaced to 
\beqa
 U_{lj}(\rho,\eta) &=& A_{lj}(E,\eta) u^{(+)}_l (kr,\eta) + B_{lj}(E,\eta) u^{(-)}_l (kr,\eta) \\
 &\propto& \left[ S_{lj}(E,\eta) u^{(+)}_l (kr,\eta) + u^{(-)}_l (kr,\eta) \right], 
\eeqa
where we need an additional variable, $\eta$, in two coefficients. 
The S-matrix, $S_{lj}(E,\eta)$, and the phase-shift, $\delta_{lj}(E,\eta)$, can be defined 
similarly in the case with short-range potentials. 
The asymptotic solution is also given as 
\beq
 U_{lj}(\rho,\eta) \longrightarrow \propto 
 \sin \left[ \rho-l\frac{\pi}{2}-\eta \ln 2\rho + \delta_{lj}(E,\eta) \right]. \label{eq:apC06}
\eeq
In the following, however, we will not use Eqs.(\ref{eq:apC05}) and (\ref{eq:apC06}), 
although those are useful for analytic discussions. 

\section{Fitting Formula for Phase-Shift}
We explain how to compute the S-matrix within the numerical framework. 
First, we consider the position $r=R_b$ at which two particles 
can be separated sufficiently from each other. 
The radial mesh, $dr$, should be enough small compared with $R_b$. 
At this point, we assess the quantity $q$ defined as 
\beq
 q(X) \equiv \frac{U_{lj}(X)}{U_{lj}(X+d)} \label{eq:apC11}
\eeq
with $X\equiv k\cdot R_b$ and $d \equiv k\cdot dr$. 
Remember that the perturbed wave, $U_{lj}(X)$, is computed numerically. 
On the other hand, in the case with Coulomb potential for instance, 
$q(X)$ is also evaluated as 
\beq
 q(X) = \label{eq:apC12}
 \frac{S_{lj}(E,\eta)u^{(+)}_l(X,\eta)+u^{(-)}_l(X,\eta)}{S_{lj}(E,\eta)u^{(+)}_l(X+d,\eta)+u^{(-)}_l(X+d,\eta)}, 
\eeq
where $u_l^{(+)}$ and $u_l^{(-)}$ can be computed independently of $U_{lj}$. 
By solving Eq.(\ref{eq:apC11}) and Eq.(\ref{eq:apC12}) simultaneously for $S_{lj}(E,\eta)$, 
we can get 
\beq
 S_{lj}(E,\eta) = 
 \frac{ U_{lj}(X+d)u_l^{(-)}(X,\eta) - U_{lj}(X)u_l^{(-)}(X+d,\eta) }
      { U_{lj}(X)u_l^{(+)}(X+d,\eta) - U_{lj}(X+d)u_l^{(+)}(X,\eta) }, 
\eeq
and $2i \delta_{lj}(E,\eta) = \ln S_{lj}(E,\eta)$. 
This is the numerical formula for the S-matrix and the phase-shift. 
Notice that the similar formula can be derived in the case with short-range potentials. 
%From the asymptotic formulas, $q_{lj}(E)$ satisfy 
%\beq
% q_{lj}(E) = \frac{\sin[p_{lj}(X,E)+\delta_{lj}(E)] }{\sin[p_{lj}(X-dx,E)+\delta_{lj}(E)] }
%\eeq
%where $p_{lj}(X,E)=X-l\frac{\pi}{2}$ for a short-range potential, 
%whereas $p_{lj}(X,E)=X-l\frac{\pi}{2}-\eta\ln 2X$ for Coulomb potential. 
%Notice that $p_{lj}(X,E)$ can be calculated with unperturbed waves, 
%separately from $U_{lj}(X,E)$. 
%Solving this equation for $\delta_{lj}$ leads to 
%\beq
% \tan\sigma_{lj}(E) = -\frac{\sin p_{lj}(X,E)-q_{lj}(E)\sin p_{lj}(X-dx,E)}{\cos p_{lj}(X,E)-q_{lj}(E)\cos p_{lj}(X-dx,E)}, 
%\eeq



Practically, it is well known that the phase-shift can be fitted by the 
Breit-Wigner distribution. 
That is 
\beq
 \delta_{lj}(E) = \tan^{-1} \left[ \frac{\Gamma_0/2}{E_0-E} \right] + C_{lj}(E), 
\eeq
or equivalently, 
\beq
 \frac{d\delta_{lj}(E)}{dE} = \frac{\Gamma_0/2}{\Gamma_0^2/4 + (E_0-E)^2} + \frac{dC_{lj}(E)}{dE}, \label{eq:apcps}
\eeq
where $C_{lj}(E)$ is a smooth back-ground. 
The central value, $E_0$, and width, $\Gamma_0$, correspond to the complex pole of the S-matrix, 
locating at $E=E_0-i\Gamma_0/2$. 
Accordingly, we have got the fitting formula, which is equivalent to Eq.(\ref{eq:sigde}) in 
Chapter \ref{Ch_Results1}. 

\include{end}
