\documentclass[a4paper,12pt]{report}
\include{begin}

\chapter{Numerov Method} \label{Ap_Numerov}
This is the numerical method to solve an ordinary differential equation 
in which only the zeroth and the second order terms are included, such as 
\beq
 \left[ \frac{d^2}{dx^2} + f(x) \right] U(x) = 0, \label{eq:Num0} 
\eeq 
where $f(x)$ is an arbitrary source function. 
The solution, $U(x)$, is sampled at equidistant points $x_n,(n=0 \sim N)$ 
where the distance between two sampling points is defined as $a$. 
With this method, starting from the solution values at 
two consecutive sampling points, 
namely $U_0 \equiv U(x_0)$ and $U_1 \equiv U(x_1)$, 
we can calculate the remaining solution values as 
\beq
 U_{n+2} = \frac{ (2-5a^2f_{n+1}/6)U_{n+1} - (1+a^2f_{n})U_{n} }
                { 1+a^2f_{n+2}/12 } + \mathcal{O}(a^6), \label{eq:Num1} 
\eeq
where we neglect $\mathcal{O}(a^6)$. 
The derivation of Eq.(\ref{eq:Num1}) is based on the discrete Taylor expansion 
for $U(x)$ until the fifth order. 
Considering the two sampling points, $x_{n-1}=x_n-a$ and $x_{n+1}=x_n+a$, 
Taylor expansions are given as 
\beqa
 U_{n+1} &\equiv& U(x_n+a) \nonumber \\
 &=& U_n+aU'_n+\frac{a^2}{2!}U''_n
        +\frac{a^3}{3!}U^{(3)}_n+\frac{a^4}{4!}U^{(4)}_n
        +\frac{a^5}{5!}U^{(5)}_n+\mathcal{O}(a^6), \\
 U_{n-1} &\equiv& U(x_n-a) \nonumber \\
 &=& U_n-aU'_n+\frac{a^2}{2!}U''_n
        -\frac{a^3}{3!}U^{(3)}_n+\frac{a^4}{4!}U^{(4)}_n
        -\frac{a^5}{5!}U^{(5)}_n+\mathcal{O}(a^6), 
\eeqa
where $U^{(m)}_n \equiv \left. d^mU(x)/dx^m \right|_{x=x_n}$. 
The sum of these two equations gives 
\beq
  U_{n-1} + U_{n+1} = 2U_n + a^2U''_n + \frac{a^4}{12}U^{(4)}_n + \mathcal{O}(a^6). 
\eeq
Solving this equation for $a^2U''_n$ leads to 
\beq
 -a^2U''_n = 2U_n - U_{n-1} - U_{n+1} + \frac{a^4}{12}U^{(4)}_n + \mathcal{O}(a^6). 
 \label{eq:Num4} 
\eeq
In this equation, we can replace $U''_n$ to $-f_n U_n$ because of Eq.(\ref{eq:Num0}). 
Similarly, for the fourth term in the right hand side, we can use 
\beq
 U^{(4)}(x) = \frac{d^2}{dx^2}[-f(x)U(x)]. 
\eeq
The numerical definition of the second derivative is given as 
the second order difference quotient, that is 
\beq
 \frac{d^2}{dx^2}[-f(x)U(x)] \Rightarrow 
 -\frac{f_{n-1}U_{n-1} - 2f_nU_n +f_{n+1}U_{n+1}}{a^2}. 
\eeq
After these replacements, Eq.(\ref{eq:Num4}) is transformed as 
\beq
 a^2f_nU_n = 
 2U_n - U_{n-1} - U_{n+1} - 
 \frac{a^4}{12} \frac{f_{n-1}U_{n-1} - 2f_nU_n +f_{n+1}U_{n+1}}{a^2} + 
 \mathcal{O}(a^6). 
\eeq
Finally, we solve this equation for $U_{n+1}$ to get 
\beq
 U_{n+1} = \frac{ (2-5a^2f_n/6)U_n - (1+a^2f_{n-1})U_{n-1} }
                { 1+a^2f_{n+1}/12 } + \mathcal{O}(a^6), 
\eeq
which is equivalent to Eq.(\ref{eq:Num1}). 

\include{end}
