\documentclass[a4paper,12pt]{report}
\include{begin}

\chapter{Introduction} \label{Ch_Intro}
%\section{Pairing Correlation}
Pairing correlations are characteristic phenomena in many-fermion systems. 
Especially, the nuclear pairing correlation has been a major subject of modern 
nuclear physics \cite{58Bohr,69Bohr,96Doba,05Brink,03Dean_rev,03Bender_rev}. 
After the establishment of the traditional 
mean-field theory for atomic nuclei, 
there have been enormous theoretical, experimental and 
computational developments in this field. 
These developments have led to a deeper insight 
beyond the pure mean-field picture. 

%This mean-field can be characterized by 
%accumulating the interactions from all other nucleons. 
Within the traditional mean-field theory, 
any nucleon inside a nucleus is assumed to be an 
independent particle moving in the mean-field generated by the interactions 
among all the nucleons. 
The traditional mean-field theory was applied to atomic nuclei, 
{\it e.g.} by Mayer {\it et.al.} \cite{49May,49Hax}, 
leading to conclusions about the shell structures and the 
magic numbers, which excellently agree 
with empirical properties of atomic nuclei. 
A more sophisticated definition of the nuclear mean-field is 
given by applying the Hartree-Fock (HF) 
theory \cite{28Hart,30Fock,03Fetter,80Ring} 
\footnote{The HF theory 
itself is a general theory for many-fermion systems. 
As a matter of fact, it was first applied to the electrons in atoms. }. 
Within the HF theory, 
the mean-field for an arbitrary nucleon 
is defined self-consistently by considering 
effective nucleon-nucleon interactions from all the other 
nucleons \footnote{We should notice that this effective interaction 
differs from a nucleon-nucleon interaction in the vacuum, 
which has a strong repulsive core at short distances. 
Except for the Coulomb repulsion, 
the repulsive core is smeared by including the medium effect 
in the effective nucleon-nucleon interaction. }. 
However, this traditional mean-field theory takes into account the 
interaction only on average, and thus misses some parts of the interaction, 
which is called the ``residual interaction''. 

The ``pairing interaction'' is the most important part 
of the residual interaction. 
Taking the pairing interaction into account, 
the traditional mean-field picture is modified to that including 
a collection of two correlated nucleons. 
two nucleons, which are in unnegligible correlations. 
The pairing interaction brings about a significant attraction 
between two nucleons 
when those are coupled to be the spin-singlet state \cite{05Brink,80Ring}. 
Evidences for the pairing correlations can be found, 
for instance, in the fact that 
there is a universal odd-even staggering rule in 
the binding energies. 
That is, even-even nuclei are systematically more bound than the 
neighboring nuclei in the nuclear chart. 
It is also known that 
the even-even nuclei take the spin-parity of $0^+$ in the 
ground state, with no exceptions. 
Similar pairing correlations play important a role not only 
in nuclei, but also 
in several other systems including condensed matters and 
cold fermionic atoms. 

In recent years, the study of nuclear pairing interactions 
and correlations has gained a renewed interest, due to the 
progress of physics of ``unstable nuclei'' 
\footnote{Difference between ``pairing interaction'' and 
``correlation'' is important. 
The pairing interaction means a distinct source of the force 
between nucleons. 
On the other hand, even in the situation where the 
pairing interaction does not exist, 
two nucleons can be kinetically correlated to each other. 
This correlation is mediated by other particles in the system. 
Namely, the paring correlation originates both from the 
pairing interaction and the many-body dynamics. } 
\cite{03Dean_rev,03Bender_rev,03Doba}. 
Unstable nuclei, which have large 
neutron- or proton-excess and locate far from the 
$\beta$-stability line, have been a major topic in 
recent nuclear physics. 
For these nuclei, there are considerably novel features 
which can be connected to the pairing correlation \cite{03Doba}. 
Those include ``dinucleon correlation'', 
which we detail in the next section. 

\section{Dinucleon Correlation}
The diproton and the dineutron correlations are intrinsic 
structural properties of atomic nuclei, caused 
by the pairing interaction. 
As is well known, a diproton or a dineutron is not bound in the vacuum, 
where the only possible bound state of two nucleons is a deuteron. 
However, inside nuclei, 
the situation may be different from that in the vacuum. 
Because of the many-body effect on pairing correlations, 
a possibility of the existence of diproton and dineutron-like 
configurations has been discussed for more than 40 years, 
since the first proposal by Migdal in 1973 \cite{73Mig}. 
These phenomena are called ``dinucleon correlations''. 
The study of dinucleon correlations is expected to provide 
a novel and universal insight into other strongly-correlated 
many-fermion systems. 

Based on the microscopic theory for the pairing correlations, 
{\it e.g.} HF-Bardeen-Cooper-Schrieffer (HF-BCS) or 
HF-Bogoliubov (HFB) theory \cite{80Ring,03Dean_rev,03Bender_rev}, 
it has been known that the pairing correlation 
depends on the surrounding density, $\rho$ 
\cite{05Yama,05Mats,06Mats,07Marg,08Marg,09Yama}. 
For instance, the pairing gap in infinite nuclear matter takes the 
maximum at a density smaller than the normal density. 
This density-dependence is naively due to the many-fermion effects. 
However, the exact origin of this density-dependence has remained unclarified, 
though some candidates have been discussed. 
Those include 
(i) the momentum-dependence of bare nucleon-nucleon forces, 
(ii) the Pauli principle and 
(iii) effects of nuclear three-body forces. 
It should be emphasized that this density-dependence 
causes a variety of pairing correlations. 
In the deeper region of nuclei with normal nuclear matter-density, 
a pair of nucleons is found to be a Cooper pair in the 
HF-BCS theory \cite{05Brink}. 
This pair is in the regime of weak pairing correlations, 
and its spatial distribution is much expanded compared to the typical 
radii of nuclei. 
On the other hand, in the low density region, 
the situation can be altered due to the density-dependence of the 
pairing correlation. 
The effective pairing correlation can be enhanced in that region, 
resulting in the spatial 
localization of two neutrons and protons, 
and also the increase of the probability of the spin-singlet configuration. 
Namely, in this situation, a pair of nucleons plays as a dineutron- or 
diproton-like cluster. 
The existence of this strong correlation can be considered within a wide 
range of the surrounding density, 
$\rho /\rho_0 \sim 0.1 - 0.01$, where $\rho_0 \sim 0.15$ fm$^{-3}$ is the 
nuclear saturation density \cite{06Mats}. 
Finally, if the density becomes infinitesimally small, 
the pairing correlation vanishes and two neutrons or protons 
become unbound. 
This limit is identical to two nucleons in the vacuum. 

From a phenomenological point of view, 
the dinucleon correlation is a kind of phase-crossover 
in many-nucleon systems with dilute 
densities \footnote{Another famous example of similar 
phenomena is the alpha-clustering inside nuclei \cite{13Girod}. 
In this thesis, however, we do not discuss it. }. 
Such a dilute density-situation has been expected to 
occur especially in the valence orbits of weakly-bound neutron-rich nuclei. 
For the past about two decades, 
neutron-rich nuclei with large neutron-excess 
and shorter lifetimes for the beta-decay have been 
extensively studied. 
This is grealy thanks to the experimental achievements which have provided 
the access to these nuclei \cite{85Tani_01, 85Tani_02}. 
In the ground state of these nuclei, 
the valence neutrons should be in the outer orbit far from the core, 
where the surrounding density is not so large that the pairing correlation 
is expected to increase \cite{05Yama,05Mats}. 
This is especially the case for weakly bound nucleons 
\cite{87Hansen,91Bert,05Hagi,07Hagi_01}. 
Various theoretical and experimental studies have been performed to 
investigate the dineutron-correlation in such nuclei, 
in connection to its influence on the nuclear structures and reactions. 
These studies have shown that the correlation may invoke sizable 
effects on 
some phenomena, including the electro-magnetic 
excitations \cite{06Naka,05Hagi,07Bertulani_76}, 
the Coulomb break-up reactions 
\cite{88Koba,04Fuku,01Myo,06Horiuchi,10Kiku,13Kiku} 
and the pair-transfer reactions \cite{73Broglia,91Igarashi,01Oert}. 
On the other hand, for proton-rich nuclei, 
even though a similar diproton correlation can be considered \cite{10Oishi}, 
it has so far been less studied compared to the dineutron correlation. 
%because the Coulomb repulsion plays as an inclusion in pure nuclear systems. 
Whether the Coulomb repulsion disrupts the diproton-like configuration or not 
in proton-rich nuclei is still an remaining question, though its effect 
has been found to be weak compared to the nuclear 
attraction \cite{10Oishi,11Oishi,02Hila,09Lesi,11Yama}. 
For both dineutron and diproton correlations, 
further quantitative and qualitative investigations are still in 
progress today. 

The prediction of the dinucleon correlation is an important 
conclusion from the recent nuclear theory, 
and its detection will give us a strong constraint 
on the basic properties of our nuclear models. 
However, as we will discuss in Chapter \ref{Ch_2}, 
even with various efforts, 
there have been no direct experimental evidences for the 
dinucleon correlation, mainly because it is an intrinsic structure
which is hard to be detected. 
%We detail these situations in Chapter \ref{Ch_2}. 

\section{Two-Proton Decay and Emission}
Given these difficulties mentioned in the previous section, 
``two-proton emission'' and ``two-proton radioactive decay'' 
have been expected to provide a novel way to access the diproton correlation. 
Those are the quantum tunneling phenomena 
that two protons are emitted from 
the proton-rich nuclei beyond the proton-dripline 
\cite{08Blank,12Pfu,09Gri_40}. 
In this process, the decay products can be strongly 
associated with the pairing correlation between two protons. 
The importance of pairing correlations are suggested from, for instance, 
that observed \twop-emitters have the even number of protons 
with no exceptions. 
Thus, emitted two protons are expected to carry information about 
the pairing correlations, probably including the diproton correlations 
in nuclei \cite{07Bertulani_34, 08Bertulani}. 
We focus on these phenomena in the next section. 

The oldest example of the two-proton (\twop-) emitter is the $^6$Be nucleus, 
where its ``alpha+p+p'' resonance has been experimentally 
observed for several decades 
\cite{66Whaling,77Gree,88Ajzen,84Boch,85Boch_377,89Boch,09Gri_677,12Ego}. 
Following $^6$Be, 
similar three-body resonances have been observed in the ground 
state of a few light proton-rich nuclei, such as 
$^{12}$O \cite{78KeKe, 95Kryger} and 
$^{16}$Ne \cite{78KeKe, 83Wood}. 
A typical Q-value and decay-width of these resonances are 
on the order of 100 keV. 
For these nuclei, 
the potential barrier between the core and a proton is mainly 
due to the centrifugal force, 
whereas the Coulomb force is relatively small. 
Because of the low potential barrier, 
the decay width is comparably broad compared to 
the Q-value of these nuclei. 

On the other hand, the \twop-radioactive decay is a novel 
decay-mode of medium-heavy and heavy nuclei outside the 
proton-dripline \footnote{In this thesis, as a criterion of 
``radioactivity'', we adopt a typical lifetime of $10^{-7}$ s \cite{13Olsen}. 
If the considering system or process has a shorter lifetime 
than this criterion, 
we refer to it simply as the \twop-emitter or emission. 
The corresponding decay width to this criterion is about $10^{-14}$ MeV.}. 
A typical lifetime for the \twop-decays of these nuclei is 1-10 ms, 
corresponding to a typical decay width of $10^{-18}$-$10^{-19}$ MeV. 
A typical Q-value is around 1 MeV, similarly to light \twop-emitters. 
The significantly narrow width, compared with light \twop-emitters, 
is due to the higher Coulomb barriers, 
which considerably reduce the tunneling probabilities of two protons. 
In this category, $^{45}$Fe is the most famous example for 
the \twop-radioactivity. 
At the beginning of 2000s, 
the first observation of \twop-radioactivity was made 
for the $^{45}$Fe nucleus \cite{02Pfu,02Gio}. 
After this first discovery, 
the \twop-radioactivity has been confirmed also for $^{54}$Zn and 
possibly for $^{48}$Ni. 

It is also predicted that the \twop-decays and emissions are not limited 
particularly in these nuclides but universally exist along 
the proton-dripline until $Z\leq 82$ \cite{09Gri_40, 13Olsen}. 
Suggested nuclides to have this decay-mode include 
$^{26}_{16}$S \cite{08Blank}, 
$^{30}_{18}$Ar \cite{08Blank, 09Gri_40}, 
$^{34}_{20}$Ca \cite{09Gri_40}, 
$^{38}_{22}$Ti \cite{09Gri_40}, 
$^{41,42}_{24}$Cr \cite{09Gri_40}, 
$^{58}_{32}$Ge \cite{09Gri_40, 13Olsen}, 
$^{62,63}_{34}$Se \cite{09Gri_40, 13Olsen}, 
$^{66}_{36}$Kr \cite{09Gri_40, 13Olsen}, 
$^{102,103}_{52}$Te \cite{13Olsen}, 
$^{109,110}_{56}$Ba \cite{13Olsen}, 
$^{155}_{78}$Pt \cite{13Olsen}, 
$^{159}_{80}$Hg \cite{13Olsen}, and so on. 
Recently, the similar processes but emitting two neutrons, 
namely ``two-neutron emissions or decays'' are 
reported for 
$^{13}_{3}$Li \cite{13Kohley_13Li}, 
$^{16}_{4}$Be \cite{12Spyr} and 
$^{26}_{8}$O \cite{12Lund}. 
Together with the \twop-emitters, 
studies of two-neutron emitters can lead to the universal 
understanding of the two-nucleon radioactivity 
on both proton and neutron-rich sides. 

On the theoretical side, the first prediction of \twop-radioactivity was 
done by Goldansky in 1960 \cite{60Gold, 61Gold}. 
He argued that a ``true \twop-decay'' is allowed only for nuclei 
where the emission of single proton is energetically forbidden. 
The pairing interaction plays an important role to realize this situation, 
by lowering in energy the ground state of even-even parent and daughter 
nuclei with respect of the even-odd intermediate nucleus. 
In this situation, two protons must penetrate the potential 
barriers simultaneously. 
At the earlier stage of study, 
two simple models for the true \twop-radioactivity 
were proposed, namely ``the diproton'' \cite{60Gold, 61Gold, 13Deli} 
and ``the direct decays'' \cite{05Rotu}. 
In these old models, two protons are assumed to decay without passing 
the intermediate core-nucleon resonance. 
The diproton and direct decays correspond to the limits with 
relatively a strong and weak pairing correlations. 
On the other hand, another simple decay-model was also considered 
in different situations. 
It is the ``sequential'', or sometimes called ``cascade \twop-decay'', 
which can exist in nuclei where the one-proton emission is 
energetically available \cite{05Rotu}. 
In this situation, the core-nucleon binary channel becomes dominant, 
whereas the pairing correlations may be not significant. 

However, with various theoretical and experimental developments, 
it has been shown that the actual \twop-decays and emissions are 
more complicated than these simple modes. 
For some \twop-emitters, including $^{45}$Fe and $^6$Be for instance, 
their decaying mechanism cannot be described neither with any of 
these models \cite{07Mie, 09Gri_677, 09Gri_80, 08Muk, 10Muk}. 
It means that the actual \twop-decays and emissions involve several dynamical 
processes in a complicated way. 
%In other words, one has to handle with, at least, three particles and 
%interactions without specially treating an arbitrary two-body subsystem. 
From recent studies, 
the structures of material nuclei and the production mechanism 
of the \twop-emitters 
are also shown to be responsible, as well as 
all the final-state interactions among particles \cite{89Boch, 12Ego, 12Gri}. 
The question whether emitted two protons have the diproton-like 
character or not still remains unsolved, 
that critically relates to the diproton correlation. 

As another interest in \twop-emissions, we here briefly mention 
the quantum entanglement \cite{03Bertulani, 08Bertulani}. 
Since \twop-emissions and decays involve a propagation of two fermions, 
analyzing their wave functions may provide another route to approach, 
{\it e.g.} the Bell's inequality \cite{04Bell} or 
the Einstein-Podolski-Rosen paradox \cite{35Einstein}. 
Observation of two protons in spin-entanglements would become 
an examination of the basic quantum mechanics, 
that is complementary to 
other studies performed in quantum optics and atomic physics. 

Obviously, gaining useful information from 
\twop-emissions depends on our ability to describe the 
multi-fermion property and the quantum meta-stability 
simultaneously \cite{61Bardeen,83Cald,10Deli_text}. 
For these quantum resonances and tunneling phenomena, 
there are mainly two theoretical frameworks; 
namely within 
the time-independent framework \cite{28Gamov_01,28Gamov_02,29Gurney,89Bohm} and 
the time-dependent framework \cite{89Bohm,89Kuku,47Kry}. 
The time-independent one is based on non-Hermite quantum mechanics. 
In this framework, 
one solves, {\it e.g.} a Gamow state \cite{28Gamov_01,28Gamov_02}, 
which is assumed to be a purely outgoing wave outside the potential barrier. 
Generally such state must have a complex eigen-energy, in order to 
satisfy the outgoing boundary condition. 
The imaginary part of the complex energy of the Gamow state is related to 
the decay width, 
while the real part corresponds to the resonance energy or the Q-value. 
An advantage of the time-independent approach is that the decay width 
can be calculated with a high accuracy even when it is extremely 
small \cite{09Gri_40, 97Aberg, 00Davis}. 
On the other hand, 
in the time-dependent framework which we will adopt in this thesis, 
resonances or tunnelings are treated as time-developments 
of quantum meta-stable states. 
An advantage of the time-dependent approach, 
compared with the time-independent one, 
is that it provides an intuitive way to understand the 
tunneling mechanism, even though it is difficult to be applied to 
the situation with an extremely small decay width, 
where it needs very long time-evolutions for the meta-stable 
state to decay out. 
Especially, for light \twop-emitters with relatively the broad widths, 
the time-dependent method is expected to provide a complementary 
studies to the time-independent method. 

\section{Aim of This Thesis}
The aim of this thesis is to investigate theoretically the relation between 
the observables in \twop-emissions and the diproton correlation. 
As we wrote, although there have been various predictions, 
direct experimental evidence of diproton and dineutron 
correlations has not been obtained. 
Recently, on the other hand, two-proton decays and emissions have 
attached much attention in order to provide the direct probe 
into the diproton correlation. 
Nevertheless, the relation between the observed data 
and the nuclear intrinsic structures, including the diproton correlation, 
has been little discussed \cite{08Bertulani,12Ego}. 
Thus, our present study is expected to provide a novel insight into 
these important problems. 

In this thesis, we will employ the three-body model 
consisting of the core (daughter) nucleus and two valence nucleons. 
This model can treat the pairing correlations between the valence nucleons 
based on the semi-microscopic picture. 
In order to take the meta-stability into account for the 
\twop-emissions, 
we will adopt the time-dependent framework. 
Though the time-dependent approach has so far been applied only to two-body 
decaying systems, such as $\alpha$-decays or one-proton decays, 
this framework can bring about an useful mean to explore the mechanism of 
many-particle tunnelings, covering the whole stages of the time-evolution. 
We would like to emphasize that this time-dependent model has 
an advantage to distinguish the effect of pairing correlations 
from other results. 
Especially, it is worthwhile to investigate the evolution of 
\twop-wave function inside and outside the potential barriers, 
which can reflect the effect of the diproton correlation on \twop-emissions. 

The thesis is organized as follows. 
In Chapter \ref{Ch_2}, the history of studies about the dinucleon correlation 
is reviewed, with some connections to unstable nuclei. 
We will mention other exotic features of unstable nuclei, closely relating 
to the dinucleon correlation. 
In Chapter \ref{Ch_3body}, in order to describe the dinucleon correlation, 
we formulate the theoretical three-body model. 
In Chapter \ref{Ch_Results1}, 
we will apply this model to $^{17,18}$Ne and $^{18}$O nuclei, 
and discuss the dinucleon correlations in these nuclei. 
Apart from the beta-decays, 
these nuclei are stable against the neutron-, proton-, and alpha-emissions 
and thus provide good testing grounds for the dinucleon correlations 
in bound many-nucleon systems. 
We also discuss the effect of Coulomb repulsions on the 
nuclear pairing correlations, 
and whether the diproton correlation exists similarly to the 
dineutron correlation. 

In Chapter \ref{Ch_5}-\ref{Ch_Results3}, 
we then discuss the diproton correlations in two-proton emissions. 
In Chapter \ref{Ch_5}, the historical overview of 
two-proton emissions and radioactive decays are summarized. 
Chapter \ref{Ch_TDM} is devoted to a formulation of the time-dependent 
method for the quantum meta-stable systems, 
including two-proton emitters. 
In Chapter \ref{Ch_Results2} and \ref{Ch_Results3}, 
the time-dependent three-body model is applied to analyze \twop-emissions of 
$^6$Be and $^{16}$Ne nuclei, for which the three-body treatment is 
expected to be valid. 
These light proton-rich nuclei have relatively large values of the 
\twop-decay width, which are expected to be well described 
within the time-dependent framework. 
We will discuss whether the diproton correlation can be identified in 
the two-proton emissions. 
%comparing our results with some experimental results. 

Finally, the summary of this thesis is present in Chapter \ref{Ch_Summary}. 
Future works towards the further improvements are also proposed. 
\include{end}
