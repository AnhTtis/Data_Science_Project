\documentclass[a4paper,12pt]{report}
\include{begin}

\chapter{Formalism of Many-Body Coordinates} \label{Ap_3body}
In this Chapter, we introduce the general formalism for the transformation of 
coordinates of many-particle systems. 
For $N$ particles in the three dimensional space, one needs $N$ 
coordinates of space, $\{ \bi{x}_i \},\; i=1 \sim N$. 
In general, these degrees of freedom are separated into the center-of-mass 
coordinate, $\bir _G$, and $N-1$ relative coordinates, 
${\bir_k},\; k=1 \sim N-1$. 
The V-coordinate used in this thesis is just one kind of definitions 
of ${\bir_k},\; k=1,2$, and its definition is detail in the following. 
The derivations of the three-body Hamiltonian in the V-coordinates 
is also introduced. 
For more general formulations and applications, see {\it e.g.} the 
textbook \cite{98Suzuki}. 

\section{Coordinates for Many-Body Problems}
We start our discussions from the original coordinates $\bi{x}_i$ and 
their conjugate momenta $\bi{\pi}_i$. 
These satisfy 
\beq
 [(\bi{x}_i)_{\mu}, (\bi{\pi}_j)_{\nu}] 
 = i\hbar \delta_{ij} \cdot \delta_{\mu \nu} \label{eq:ap101}
 \; \Longrightarrow \; \bi{\pi}_i 
 = -i\hbar \frac{\partial }{\partial \bi{x}_i}, 
\eeq
where $\mu$ and $\nu = x,y,z$. 
Here we define the column vector $\vec{X}$ and $\vec{\Pi}$, whose $i$-th 
component is $\bi{x}_i$ and $\bi{\pi}_i$, respectively. 
\beq
  \vec{X} \equiv 
  \left[ \begin{array}{c} 
  \bi{x}_1 \\ \vdots \\ \bi{x}_N \end{array} \right], \qquad 
  \vec{\Pi} \equiv 
  \left[ \begin{array}{c} 
  \bi{\pi}_1 \\ \vdots \\ \bi{x}_N \end{array} \right]. 
\eeq
Using the transform-matrix $U$, one can define the new set of 
coordinates, $\{ \bir_i \}$, as follows: 
\beq
  \vec{R} \equiv \left[ \begin{array}{c} 
  \bir_1 \\ \vdots \\ \bir_N  \end{array} \right] 
  = U \vec{X} \quad \Longleftrightarrow \quad 
    \vec{X} = U^{-1} \vec{R}, 
\eeq
or equivalently, 
\beq
  \bir_i = \sum_{j=1}^{N} U_{ij} \bi{x}_j \quad \Longleftrightarrow \quad 
  \bi{x}_i = \sum_{j=1}^{N} (U^{-1})_{ij} \bir_j. 
\eeq
The conjugate momenta are also written by applying the chain-rule: 
\beq
  \bi{\pi}_i = -i\hbar \frac{\partial}{\partial \bi{x}_i} 
  = -i\hbar \sum_{j} \frac{\partial \bir_j}{\partial \bi{x}_i} 
    \frac{\partial }{\partial \bir_j} 
  = \sum_{j} U_{ji} \bip_j. 
\eeq
Thus, indicating the transverse matrix of $U$ as ${}^{t}U$, 
the conjugate momenta of $\vec{R}$ can be defined as 
\beq
  \vec{P} \equiv \left[ \begin{array}{c} 
  \bip_1 \\ \vdots \\ \bip_N \end{array} \right] 
  = ({}^{t}U)^{-1} \vec{\Pi} \quad \Longleftrightarrow \quad 
  \vec{\Pi} = {}^{t}U \vec{P}, 
\eeq
or equivalently, 
\beq
  \bip_i = \sum_{j=1}^{N} ({}^{t}U)^{-1}_{ij} \bi{\pi}_j \quad \Longleftrightarrow \quad 
  \bi{\pi}_i = \sum_{j=1}^{N} {}^{t}U_{ij} \bip_j. 
\eeq
Note that the conjugate relation is still satisfied as 
\beq
 [(\bir_i)_{\mu}, (\bip_j)_{\nu}] = i\hbar \delta_{ij} \cdot \delta_{\mu \nu}. 
\eeq
The transform-matrix, $U$, can be chosen arbitrarily and there is no 
mathematical discrimination between different $U$s. 
However, in practice, there are two major coordinates used to solve 
the many-body problems. 
In this thesis, we choose so called ``core-center coordinates'' 
defined by the transform-matrix, 
\beq
 U \equiv \left( \begin{array}{ccccc cc} 
          1 & 0 & 0 & \cdots & 0 & 0 & -1 \\
          0 & 1 & 0 & \cdots & 0 & 0 & -1 \\
          \vdots &&&&&& \\
          0 & 0 & 0 & \cdots & 0      & 1 & -1 \\
          \frac{m_1}{M} & \frac{m_2}{M} & \frac{m_3}{M} & \cdots & \frac{m_{N-2}}{M} & \frac{m_{N-1}}{M} & \frac{m_N}{M} 
          \end{array} \right), 
\eeq
where its inverse matrix is given by 
\beq
 U^{-1} \equiv \left( \begin{array}{ccccc c} 
               1-\frac{m_1}{M} &  -\frac{m_2}{M} &  -\frac{m_3}{M} & \cdots &  -\frac{m_{N-1}}{M} & 1 \\ &&&&& \\
                -\frac{m_1}{M} & 1-\frac{m_2}{M} &  -\frac{m_3}{M} & \cdots &  -\frac{m_{N-1}}{M} & 1 \\ &&&&& \\
                -\frac{m_1}{M} &  -\frac{m_2}{M} & 1-\frac{m_3}{M} & \cdots &  -\frac{m_{N-1}}{M} & 1 \\
               \vdots &&&&& \\
                -\frac{m_1}{M} &  -\frac{m_2}{M} &  -\frac{m_3}{M} & \cdots & 1-\frac{m_{N-1}}{M} & 1 \\ &&&&& \\
                -\frac{m_1}{M} &  -\frac{m_2}{M} &  -\frac{m_3}{M} & \cdots &  -\frac{m_{N-1}}{M} & 1 \\
               \end{array} \right), 
\eeq
with $M\equiv \sum_{i=1}^{N} m_i$. 
In these coordinates, the vector $\bir_N$ indicates the center-of-mass motion, 
whereas $\bir_k$ with $k=1 \sim (N-1)$ indicates the relative motion between the central 
core and the $k$-th particle. 
We schematically indicate these coordinates in the case of three-body systems 
in Figure \ref{fig:A1_1}. 
\begin{figure}[t] \begin{center}
\fbox{ \includegraphics[width = 0.9\hsize]{./FIG9011_VT.eps}}
\caption{(left panel) The original coordinates for the three-body system. 
(right panel) The core-center coordinates. }
\label{fig:A1_1}
\end{center} \end{figure}

We briefly introduce another famous coordinates, 
namely ``Jacobi coordinates''. 
Those are defined by 
\beq
U_{J} \equiv \left( \begin{array}{ccccc cc} 
         1 & -1 & 0 & 0                                                        & \cdots & 0 & 0 \\ &&&&&& \\
         \frac{m_1}{m_{12}}  & \frac{m_2}{m_{12}}  & -1 & 0                    & \cdots & 0 & 0 \\ &&&&&& \\
         \frac{m_1}{m_{123}} & \frac{m_2}{m_{123}} & \frac{m_{23}}{m_{123}} & -1 & \cdots & 0 & 0 \\
         \vdots &&&&&& \\
         \frac{m_1}{m_{12\cdots (N-1)}} & \frac{m_2}{m_{12\cdots (N-1)}} & \cdots & \cdots & \cdots & \frac{m_{N-1}}{m_{12\cdots (N-1)}} & -1 \\
         &&&&&& \\
         \frac{m_1}{M} & \frac{m_2}{M} & \cdots & \cdots & \cdots & \frac{m_{N-1}}{M} & \frac{m_N}{M} 
         \end{array} \right), 
\eeq
and its inverse matrix is given by 
\beq
U_{J}^{-1} \equiv \left( \begin{array}{ccccc c} 
              \frac{m_2}{m_{12}} &     \frac{m_3}{m_{123}} & \frac{m_4}{m_{1234}} & \cdots & \frac{m_N}{M} & 1 \\ &&&&& \\
             -\frac{m_1}{m_{12}} &     \frac{m_3}{m_{123}} & \frac{m_4}{m_{1234}} & \cdots & \frac{m_N}{M} & 1 \\ &&&&& \\
             0                   & -\frac{m_{12}}{m_{123}} & \frac{m_4}{m_{1234}} & \cdots & \frac{m_N}{M} & 1 \\ &&&&& \\
             0 & 0 & -\frac{m_{123}}{m_{1234}} & \cdots & \frac{m_N}{M} & 1 \\
             \vdots &&&&& \\
             0 & 0 & 0 & \cdots & -\frac{m_{12\cdots (N-1)}}{M} & 1 
             \end{array} \right), 
\eeq
with $m_{12\; \cdots N'} \equiv \sum_{i=1}^{N'} m_i$. 
This set of coordinates has been used as well as the core-center coordinates 
for nuclear few-body models \cite{98Suzuki}. 
In this thesis, we use these coordinates only for 
the two relative momenta in the three-body system, 
in order to calculate $\Braket{h_{\rm c-NN}}$ and 
$\Braket{h_{\rm N-N}}$ in Chapters \ref{Ch_Results1} and \ref{Ch_Results2}. 

\section{Hamiltonian of Three-Body System}
In this thesis, we study the quantum three-body systems consisting of 
the core-nucleus and the two valence nucleons. 
We approximately use the same mass for a proton and a neutron. 
Thus, the masses of the core and a valence nucleon are $m_C=A_C m$ and 
$m_1=m_2=m$, respectively, where $A_C$ is the mass-number of the core. 
The total Hamiltonian written in the original coordinates, $\{ \bi{x}_i \}$, 
is given as 
\beqa
H \nonumber 
 &=& \frac{\bi{\pi}_1^2}{2m} + \frac{\bi{\pi}_2^2}{2m} + \frac{\bi{\pi}_C^2}{2A_C m} \\
 & & + V_{C-N_1}(\bi{x}_1 - \bi{x}_C) + V_{C-N_2}(\bi{x}_2 - \bi{x}_C) 
     + v_{N_1-N_2}(\bi{x}_1 - \bi{x}_2), 
\eeqa
where we assigned the third coordinate, ${\bi{x}_3}$, to the core-nucleus. 

To get the Hamiltonian in the core-center, or sometimes called 
``V-coordinates'' for three-body systems, 
we consider the transform-matrix, $U$, defined as 
\beq
U = \left( \begin{array}{ccc} 
    1 & 0 & -1 \\
    0 & 1 & -1 \\
    \frac{m}{M} & \frac{m}{M} & \frac{A_C m}{M} 
    \end{array} \right) 
\quad \Longleftrightarrow \quad 
U^{-1} = \left( \begin{array}{ccc} 
         1-\frac{m}{M} & -\frac{m}{M} & 1 \\
         -\frac{m}{M} & 1-\frac{m}{M} & 1 \\
         -\frac{m}{M}  & -\frac{m}{M} & 1 
         \end{array} \right), 
\eeq
where $M = m_1+m_2+m_C = (A_C+2)m$. 
The transformed coordinates are written as 
\beq
  \left[ \begin{array}{c}
  \bir_1 \\ \bir_2 \\ \bir_3 
  \end{array} \right] 
= U 
  \left[ \begin{array}{c}
  \bi{x}_1 \\ \bi{x}_2 \\ \bi{x}_C 
  \end{array} \right] 
= \left[ \begin{array}{c}
  \bi{x}_1 - \bi{x}_C \\
  \bi{x}_2 - \bi{x}_C \\
  \frac{m}{M}\bi{x}_1 + \frac{m}{M}\bi{x}_2 + \frac{A_C m}{M}\bi{x}_C 
  \end{array} \right], 
\eeq
where $\bir_3$ corresponds to the center-of-mass, $\bir_G$. 
Notice also that $\bi{x}_1-\bi{x}_2 = \bir_1-\bir_2$ and thus 
we can simply replace 
$v_{N_1-N_2}(\bi{x}_1-\bi{x}_2) \longrightarrow v_{N_1-N_2}(\bir_1-\bir_2) $. 
On the other hand, the conjugate momenta are transformed as 
\beq
\left[ \begin{array}{c}
\bi{\pi}_1 \\
\bi{\pi}_2 \\
\bi{\pi}_C 
\end{array} \right] 
= {}^{t}U 
\left[ \begin{array}{c}
\bip_1 \\
\bip_2 \\
\bip_G 
\end{array} \right] 
= 
\left[ \begin{array}{c}
\bip_1 + \frac{m}{M} \bip_G \\
\bip_2 + \frac{m}{M} \bip_G \\
- \bip_1 - \bip_2 + \frac{A_C m}{M} \bip_G 
\end{array} \right]. \label{equ_A1_pi1} 
\eeq
From Eq.(\ref{equ_A1_pi1}), the kinetic terms can be 
re-written as 
\beqa
&& \frac{\bi{\pi}_1^2}{2m} + \frac{\bi{\pi}_2^2}{2m} + \frac{\bi{\pi}_C^2}{2A_C m} \\
&& \qquad = \nonumber 
\frac{1}{2} \left( \frac{1}{m} + \frac{1}{A_C m} \right) \bip_1^2 + 
\frac{1}{2} \left( \frac{1}{m} + \frac{1}{A_C m} \right) \bip_2^2 + 
\frac{1}{A_C m} \bip_1 \cdot \bip_2 + \frac{1}{2M} \bip_G^2 \\
&& \qquad = 
\frac{\bip_1^2}{2\mu} + 
\frac{\bip_2^2}{2\mu} + 
\frac{\bip_1 \cdot \bip_2}{A_C m} + \frac{\bip_G^2}{2M} 
\eeqa
where $\mu = m (A_C+1)/A_C$. 
As the final result, the total Hamiltonian takes the form below. 
\beqa
H &=& \nonumber 
\frac{\bip_1^2}{2\mu} + \frac{\bip_2^2}{2\mu} + 
\frac{\bip_1 \cdot \bip_2}{A_C m} + \frac{\bip_G^2}{2M} \\
& & + V_{C-N_1}(\bir_1) + V_{C-N_2}(\bir_2) + v_{N_1 N_2}(\bir_1-\bir_2). 
\label{equ_A1_H2} 
\eeqa
Notice that in Eq.(\ref{equ_A1_H2}), the center-of-mass motion is separated 
from the three-body relative motion. 
Assuming that $\bip_G = \bi{0}$, 
the three-body Hamiltonian, $H_{\rm 3b}$, in Chapter \ref{Ch_3body} 
is correctly derived. 

\include{end}
