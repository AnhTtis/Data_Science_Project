\documentclass[a4paper,12pt]{book}
\include{begin}

\chapter{Two-Proton Emission of $^{16}$Ne} \label{Ch_Results3}
For further investigation of \twop-emissions in this Chapter, 
we take up another \twop-emitter, $^{16}$Ne. 
To this end, we assume the 
three-body system of $^{14}$O and two valence protons. 
The quantum meta-stability is treated within 
the time-dependent framework. 
In the case of $^6$Be discussed in the previous Chapter, 
our time-dependent three-body model 
well reproduced the experimental $Q_{\rm 2p}$ and $\Gamma_{2p}$ 
simultaneously. 
It suggests that the three-body assumption is valid for this nucleus. 
On the other hand, it has been a serious problem that a similar 
theoretical three-body model within the complex-energy framework 
do not simultaneously reproduce the $Q_{\rm 2p}$ and $\Gamma_{2p}$ of 
other light \twop-emitters \cite{02Gri}. 
Thus, it is worthwhile to 
check whether our model works or not for these nuclei. 
The application to the $^{16}$Ne nucleus to be discussed in this Chapter 
is one example of studies towards this direction. 
\begin{figure}[htb] \begin{center} 
%\includegraphics[width=0.6\hsize]{./y_16NE_a/g_16Ne_lv.eps}
\fbox{\includegraphics[width=0.6\hsize]{./FIG8011_lv.eps}}
\caption{The level scheme of $^{16}$Ne and its isotones. 
The printed values for $^{16}$Ne are taken from 
the Ref. \cite{83Wood}. 
Those for $^{15}$F are calculated with the core-proton 
interaction which is originally introduced in 
the Ref. \cite{10Mukhamed}. 
The color-box of each level indicates its decay width. } \label{fig:801}
\end{center} \end{figure}

In Figure \ref{fig:801}, we show the level scheme of $^{16}$Ne 
and its daughter nuclei after the $1p$- and \twop-emissions, 
measured from the ground state of $^{14}$O. 
Indeed, the $1p$-resonance of the $^{14}$O+p in the $(s_{1/2})$-channel 
is located near the \twop-resonance of $^{16}$Ne. 
Its width, $\Gamma_{\rm 1p} \simeq 500$ keV 
is so large that one may wonder whether the resonance character truly 
exists in this system or not. 
The sequential emission through the core-proton channel is 
expected to be minor, due to its broad width. 
We also stress that there are still ambiguities in the 
experimental data of the first resonance 
of $^{15}$F \cite{78KeKe,03Peters,10Mukhamed,91Ajzen,03Szily,04Goldberg,05Guo}. 
Consequently, it is still unclear whether 
the $^{16}$Ne nucleus is a true \twop-emitter or not. 
In this work, we will assume a relatively low energy and 
narrow width in the $(s_{1/2})$-channel, as detailed in the 
next section. 

\section{Set up for Calculations}
General assumptions for the numerical calculations are similar to those 
for the $^6$Be in the previous Chapter. 
We assume that the core nucleus, $^{14}$O is a structureless 
particle with the spin-parity of $0^+$. 
Because the first resonance state of $^{16}$Ne also has the spin-parity 
of $0^+$, we only need the $0^+$ uncorrelated basis 
for the valence two protons. 
The calculations are performed in the truncated space 
defined by the energy-cutoff: 
$\epsilon_a + \epsilon_b \leq E_{\rm cut} =40$ MeV. 
The continuum states are discretized within a radial box of $R_{\rm box}=80$ fm. 
For the angular momentum channels, we take $l_{\rm max}=5$, that is, 
we include all the partial waves from $(s_{1/2})^2$ to $(h_{11/2})^2$. 
\begin{figure}[t] \begin{center}
$V_{\rm c-p}(r)$ and $V_{\rm c-p}^{conf}(r)$ \\
%\fbox{\includegraphics[width=0.5\hsize, clip, trim = 10 0 5 5]{y_16NE_a/Vcp_15F.eps}}
\fbox{\includegraphics[width=0.5\hsize,scale=1, trim = 50 50 0 0]{FIG8021.eps}}
\caption{The original and confining potentials for the 
$(s_{1/2})$, $(p_{3/2})$ and $(d_{5/2})$ channels in the $^{14}$O-p subsystem. 
The border radius for modifying the potential is 8.5 fm for all the channels. 
%$R_b$ in Eqs. (\ref{eq:8dcv}) and (\ref{eq:8scv}) is set to be 8.5 fm. 
} \label{fig:812}
\end{center} \end{figure}
\begin{figure*}[htb] \begin{center}
  $^{16}$Ne (g.s.), $t=0$, ``full'' \\
  \begin{tabular}{c} %switch-off the auto-turning
     \begin{minipage}{0.48\hsize} \begin{center}
%        \fbox{ \includegraphics[height=45truemm, clip, trim = 20 0 10 20]{./y_16NE_a/g_daba_ini.eps}}
        \fbox{ \includegraphics[height=45truemm,scale=1,trim = 50 50 0 0]{./FIG8031.eps}}
     \end{center} \end{minipage}
     \begin{minipage}{0.48\hsize} \begin{center}
%        \fbox{ \includegraphics[height=45truemm, clip, trim = 10 5 5 5]{./y_16NE_a/g_anga_ini.eps}}
        \fbox{ \includegraphics[height=45truemm,scale=1,trim = 50 50 0 0]{./FIG8032.eps}}
     \end{center} \end{minipage}
  \end{tabular}
  \caption{(The left panel) 
The \twop-density distribution at $t=0$ for the ground state of $^{16}$Ne. 
It is obtained by including all the partial waves up to $(h_{11/2})^2$, 
and plotted as a function of 
$r_{\rm c-pp} = (r_1^2 + r_2^2 + 2r_1r_2\cos \theta_{12})^{1/2}/2$ and 
$r_{\rm p-p} = (r_1^2+r_2^2-2r_1r_2\cos \theta_{12})^{1/2}$. 
(The right panel) The angular distribution at $t=0$ obtained 
by integrating $\bar{\rho}_{2p}(t=0)$ with $r_1$ and $r_2$. } \label{fig:813}
\end{center} \end{figure*}

We assume the Woods-Saxon and Coulomb potentials between 
the core and a proton, similarly to Eq.(\ref{eq:cp_pot6}). 
We employ the same parameters as those in the 
Ref. \cite{10Mukhamed}, in which the authors discussed 
the scattering problem of $^{14}$O+p theoretically. 
The first and second resonances obtained with these parameters are 
shown in Fig. \ref{fig:801}. 
We calculate and fit the derivative of the phase-shift, according to 
Eq.(\ref{eq:apcps}), to get $E_{\rm 1p}$ and $\Gamma_{\rm 1p}$. 
These values are consistent to several experimental 
results \cite{78KeKe,04Goldberg,05Guo}. 
%In order to fit the broad $(s_{1/2})$-resoance, 

For the proton-proton pairing interaction, 
we use the Minnesota potential given by Eq.(\ref{eq:Minne}) 
in this case. 
In order to reproduce the Q-value of \twop-emission, 
$Q_{\rm 2p} \equiv \Braket{H_{\rm 3b}}=1.40$ MeV, 
we adjust the strength of the repulsive part as $v_0=126.2$ MeV. 
The other parameters in the Minnesota potential are fixed to 
the original values in ref.\cite{77Thom}. 

The initial \twop-state for the time-evolution is defined as 
a quasi-bound state obtained with the confining potentials, 
similarly 
to the previous calculations for $^6$Be. 
The confining potentials for $^{16}$Ne are defined as follows. 
In Chapter \ref{Ch_Results1}, we have 
confirmed that in $^{17,18}$Ne nuclei, 
the valence two protons are mainly in the $(d_{5/2})^2$-orbit. 
Because $^{16}$Ne is an isotope of these nuclei, 
the first resonance of $^{16}$Ne is also expected to have a large 
component of $(d_{5/2})^2$-configuration. 
According to this consideration, 
for the single particle (s.p.) $(d_{5/2})$-channel, 
we modify the core-proton potential at 
$t=0$ in order to fix the quasi-bound state as follows. 
\beq
 V_{{\rm c-p},~(d_{5/2})}^{conf}(r) \label{eq:8dcv}
 = \left\{ \begin{array}{cc} 
            V_{{\rm c-p},~(d_{5/2})}(r) & (r \leq R_b), \\
            V_{{\rm c-p},~(d_{5/2})}(R_b) & (r > R_b), \end{array} \right.
\eeq
with $R_b=8.5$ fm in this case. 
For other s.p. channels, we define it as 
\beq
 V_{\rm c-p}^{conf}(r) \label{eq:8scv}
 =\left\{ \begin{array}{cc} 
          V_{\rm c-p}(r) \phantom{0000} & (r \leq R_b), \\
          V_{\rm c-p}(r) + V_b(r) & (r > R_b), \end{array} \right.
\eeq
where $V_b(r) = V_{{\rm c-p},~(d_{5/2})}(R_b) - V_{{\rm c-p},~(d_{5/2})}(r)$. 
The original and confining potentials for the 
$(s_{1/2})$, $(p_{3/2})$ and $(d_{5/2})$ channels are 
shown in Fig. \ref{fig:812}. 

By diagonalizing the modified Hamiltonian including $V_{\rm c-p}^{conf}(r)$, 
we obtain the initial \twop-state. 
In Figure \ref{fig:813}, we show the initial density, 
$\rho_{2p}(t=0)$, and its angular distribution. 
One can clearly see a significant diproton correlation, 
characterized by 
the localization of the two protons in the $S_{12}=0$ configuration. 
The spin-singlet ratio is obtained as $P(S_{12}=0)=87.9$\%. 
The prominent three peaks are due to the dominant $(d_{5/2})^2$ wave, 
with its probability of $49.56$\% in this state. 
In addition, the $(s_{1/2})^2$ wave 
has a comparable probability of $44.70$\% 
in this state. 
The other waves with $(l=odd)^2$ and $(l=even)^2$ carry the 
probabilities of $3.61$\% and $2.13$\%, respectively. 
It is worthwhile to compare this result with that of $^{17}$Ne 
obtained in Chapter \ref{Ch_Results1}. 
By comparing the left panel of Fig. \ref{fig:813} with 
Fig. \ref{fig:461}(a), 
it can be seen that the spatial distribution 
is a little more extended in the $^{16}$Ne nucleus. 
This is consistent with the increment of the $(s_{1/2})^2$ wave, 
which has a long tail outside the core-proton potential. 
While this result can be interpreted as a characteristic difference 
between the bound and meta-stable \twop-states, 
anyway, the diproton correlation is still suggested in the 
initial state of $^{16}$Ne. 
If our time-dependent calculation yields the decay width that 
is consistent with the experiments, 
we can associate the behavior of the emitted two protons 
(at the earlier stage) with the diproton correlation. 
\begin{figure}[tb] \begin{center}
%\fbox{\includegraphics[width=0.5\hsize, clip, trim = 0 0 0 0]{./y_16NE_a/GamNd_S12.eps}}
\fbox{\includegraphics[height=80truemm,scale=1,trim = 50 50 0 0]{./FIG8041_Gam.eps}}
\caption{The decay probabilities and the decay width 
of \twop-emissions from $^{16}$Ne, obtained with the time-dependent method. 
Those for the spin-singlet and triplet configurations are also plotted. 
In calculations, all the partial waves up to $(h_{11/2})^2$ are included. 
The parameters of the pairing interaction are adjusted to reproduce 
the experimental Q-value, $Q_{\rm 2p, Exp.}=1.40(2)$ MeV \cite{83Wood}. 
Note that the experimental decay width, 
$\Gamma_{\rm 2p,Exp.}=110\pm 40$ keV \cite{83Wood} 
is too higher to be indicated 
in the lower panel. } \label{fig:914}
\end{center} \end{figure}

\section{Decay Width}
The results for the decay probability and the decay width are shown 
in Fig. \ref{fig:914}. 
The calculation is carried out up to $ct=2800$ fm 
where the reflection at $R_{\rm box}$ can be neglected. 
Unfortunately, there is a large discrepancy 
between the calculated and the experimental decay widths. 
In Fig. \ref{fig:914}, after a sufficient time-evolution, 
the calculated decay width approximately converges to 
$\Gamma_{\rm 2p,Thr.}\simeq 2-3$ keV, which is underestimated 
against the experimental value, 
$\Gamma_{\rm 2p,Exp.}=110\pm 40$ keV \cite{83Wood}. 
This discrepancy would not be attributed to our small 
model space, or the uncertainties in the time-dependent calculation 
since a similar three-body model calculation also yielded a 
similar discrepancy \cite{02Gri,09Gri_40}. 
Additionally, in the earlier time region, 
the decay probability shows a big bump, causing a large 
oscillation in the decay width, $\Gamma (t)$. 
We do not know exactly whether this bump is just an 
artifact, or originates from the initial configuration, 
including the diproton correlation. 

In order to investigate a possible cause of the underestimated 
decay width, 
we carry out similar calculations but with different values of $v_0$ 
in the Minnesota pairing interaction, 
intuitively discarding the fine set up for the Q-value. 
In Fig. \ref{fig:933}, we show the results obtained with 
$v_0=200.0,168.0$ and $126.2$ MeV. 
The first value is identical to the original parameter \cite{77Thom}, 
whereas the third value is that used in Fig. \ref{fig:914}. 
The decay width is reproduced if we take 
$v_0=168$ MeV, which yields $Q_{\rm 2p,Thr.}=2.04$ MeV. 
In this case, compared with the previous case with $Q_{\rm 2p,Thr.}=1.40$, 
the two protons have a larger energy 
to overcome the potential barriers, and the decay width 
becomes also larger, 
leading to the agreement with the experimental value. 
It means that our naive three-body model leads to 
an over stabilization against the \twop-emission of $^{16}$Ne. 

Notice also that there remains sizable oscillations in $\Gamma(t)$ 
even after a sufficient time-evolution. 
We conjecture that the mixing of the two resonances, namely those in 
the $(s_{1/2})$ and $(d_{5/2})$-channels of the core-proton subsystem, 
is responsible for this result. 
However, we do not explore deeply into this problem here. 
Indeed, we will rather discuss what causes the 
over stabilization of the two protons. 
\begin{figure}[tb] \begin{center}
%\fbox{\includegraphics[width=0.5\hsize, clip, trim = 5 0 0 5]{./y_16NE_a/Gam_vs.eps}}
\fbox{\includegraphics[height=50truemm,scale=1,trim = 50 50 0 0]{./FIG8051_vs.eps}}
\caption{The decay widths obtained for different values of $v_0$ in the 
Minnesota pairing attraction. 
The corresponding Q-values are $Q_{\rm 2p}=2.48$, $2.04$ and $1.40$ MeV for 
$v_0 = 200$, $168$ and $126.2$ MeV, respectively. 
The experimental decay width, 
$\Gamma_{\rm 2p,Exp.} = 110 \pm 40$ keV \cite{83Wood}, 
is also indicated by the shaded area. } \label{fig:933}
\end{center} \end{figure}

\section{Possibilities of Improvements}
The first possible cause of the over stabilization of $^{16}$Ne is 
a lack of core excitations in the present theoretical model. 
In other words, the intrinsic degrees of freedom of the core nucleus 
may not be neglected for $^{16}$Ne. 
One may wonder why the present model works well for $^6$Be, but 
does not for $^{16}$Ne. 
This is due to the stability of the core nuclei, 
namely $alpha$-particle and $^{14}$O for $^6$Be and 
$^{16}$Ne, respectively. 
The first excited state of $\alpha$ is located at $E=20.2$ MeV, 
which is much higher than the first excited state of $^{14}$O 
at $E=5.1$ MeV \cite{NNDCHP}. 
Thus, the core excitation may be relatively important 
in $^{16}$Ne compared to $^{6}$Be. 

Indeed, from recent studies on weakly-bound nuclei, 
it is expected that excitations of the core nucleus play an important 
role in the halo structure and the electro-magnetic excitations of these 
nuclei \cite{93Sagawa,95Esb_2,99Tost,01Shyam,08Myo,12Moro,13Moriguchi}. 
For the meta-stable processes, 
it has been recognized that 
a coupling of the valence particle 
with these degrees of freedom might 
enhance the tunneling probability and thus 
increase the decay width \cite{83Cald,00Esb,00Ferre,02Hagino_1p}. 
The similar effect of the core excitation is expected to exist 
in the \twop-emission, which might restore an discrepancy 
between the calculated and the experimental decay widths. 
The most possible source of these excitations is 
the deformation of the core nucleus. 
However, even if the core is not deformed, 
the considerable component of these excitations exist, 
namely the ``two-particle and two-hole (2p2h-)'' type of excitations 
from the naive shell structure, due to the pairing correlations. 
It includes the 2p2h-excitations by two protons, two neutrons 
and a proton-neutron pair. 
The former two are caused by the ordinary pairing correlations, 
whereas the last one is associated with the tensor force \cite{08Myo}. 
Taking these core excitations into account means 
an extension of the inert-core model, 
and may lead to a relaxation of over 
stabilization of \twop-emissions. 
However, for this purpose, we must expand the model space, 
which would increase the cost of calculations. 
To treat it correctly will be a challenging task in our future works. 

The second possibility is due to a model-dependence of the 
pairing interaction, as partially discussed in the previous Chapter. 
In this thesis, we have adopted the simple Minnesota interaction between 
two protons. 
With other interactions, which include the spin-orbit or the 
momentum-dependence, might reproduce the experimental 
$Q_{\rm 2p}$ and $\Gamma_{\rm 2p}$ simultaneously. 
Furthermore, in order to reproduce the total Q-value, 
we have intuitively modified the parameter of the Minnesota 
potential in this work. 
However, by modifying the parameter, 
two-nucleon scattering property at infinitely far from the core 
is no longer reproduced. 
This deviation may affect the calculated results, 
especially for the meta-stable processes, in which the final-state 
interactions play an important role even far from the core. 
To improve this point, we will have to install the density-dependence 
into the pairing interaction, or employ the phenomenological 
three-body force, which works only when all the particles are 
localized in a small region. 
A work towards this direction is in the progress now. 

\include{end}
