\documentclass[a4paper,12pt]{book}
\include{begin}

\chapter{Two-Proton Emission of $^{6}$Be} \label{Ch_Results2}
We now apply the time-dependent method to the ground state of 
$^{6}$Be nucleus, assuming the three-body structure of $\alpha$+p+p. 
Because the $\alpha$-particle can be well assumed as a rigid core, 
this system provides a good testing ground for our method. 
We also note that, as shown in Figure \ref{fig:054}, 
this is one of the closest systems to a true \twop-emitter: 
the experimental Q-value of the \twop-emission is 
$1.37$ MeV \cite{89Boch,88Ajzen,02Till}, 
which is lower than the one-proton resonance energy in $^5$Li, 
being about $2$ MeV with a broad width \cite{88Ajzen, 02Till}. 
Thus, the sequential process via $\alpha + p$ subsystem is considered 
to be suppressed, and the two protons should penetrate the potential 
barrier simultaneously \footnote{Actually, due to the broad 
width of the $\alpha$-p subsystem, 
there can be a non-negligible possibility of 
the sequential \twop-emission. }. 
In this simultaneous \twop-emission, 
the association between the dinucleon correlations and 
the \twop-emissions, might be clarified. 
\begin{figure*}[htbp] \begin{center}
\fbox{\includegraphics[width=0.6\hsize]{./FIG7011.eps}}%\fbox{\includegraphics[width=0.6\hsize]{./y_06BE_a/g_06Be_lv.eps}}
\caption{The experimental level scheme of $^{6}$Be and its isotones. 
The values for $^{5}$Li are quoted from 
refs.\cite{88Ajzen,09Shirokov}, whereas 
those for $^{6}$Be are quoted from 
refs.\cite{88Ajzen,02Till}. 
The color-box of each level indicates its decay width. } \label{fig:054}
\end{center} \end{figure*}

\section{Set up for Calculations}
Our three-body model consists of $\alpha$-particle 
as a structureless core (daughter) nucleus and two valence protons. 
We employ the V-coordinates similarly to Chapters \ref{Ch_3body} 
and \ref{Ch_Results1}. 
That is, 
\beqa
 H_{\rm 3b} 
 &=& h_1 + h_2 + \frac{\bip_1 \cdot \bip_2}{A_{\rm c} m} + 
     v_{\rm p-p}(\bir_1, \bir_2), \\
 h_i 
 &=& \frac{\bip_i^2}{2\mu} + V_{\rm c-p}(r_i) \; \; \; \; (i=1,2), 
\eeqa
where $h_i$ is the single particle (s.p.) Hamiltonian for the 
relative motion between 
the core and the $i$-th proton. 
We assume that the $\alpha$-p potential is spherical, and independent of 
the spin variables. 
We also assume that the $\alpha$-particle 
always remains in the ground state with the spin-parity of $0^+$. 
Thus, similarly in Chapter \ref{Ch_Results1}, 
we need uncorrelated bases only for the $0^+$ configuration 
since the ground state of $^6$Be also has the spin-parity of $0^+$. 
That is, 
\beqa
 \tilde{\Psi}_{ab} (\xi_1, \xi_2) &\longrightarrow& 
 \tilde{\Psi}^{(0^+)}_{n_a n_b lj} (\xi_1, \xi_2) \nonumber \\
 & & = \frac{1}{\sqrt{2(1+\delta_{n_a,n_b})} } 
       \sum_m \cgc{0,0}{j,m;j,-m} \nonumber \\
 & & \phantom{000} \left[ \phi_{n_a ljm} (\xi_1) \phi_{n_b lj-m} (\xi_2) 
                        + \phi_{n_a ljm} (\xi_2) \phi_{n_b lj-m} (\xi_1) 
                  \right]. \label{eq:601basis_0p} 
\eeqa
In the following, for simplicity, 
we use simplified labels for the uncorrelated bases: 
$\ket{\tilde{\Psi}_K}$ where $K=\left\{ n_a, n_b, l,j \right\}$. 
Then, the time-dependent three-body state, except for 
the center of mass motion of the whole system, 
can be expanded as 
\beq
 \ket{\Phi(t)} = \sum_{K} C_K (t) \ket{\tilde{\Psi}_K}, 
\eeq
where the coefficients $C_K (t)$ are given by 
Eqs. \ref{eq:excf_E} and \ref{eq:excf_UNCB}. 
All our calculations presented below 
are performed in the truncated space 
defined by the energy-cutoff: 
$\epsilon_a + \epsilon_b \leq E_{\rm cut} =40$ MeV. 
We have confirmed that our conclusions do not change 
even if we employ a larger value of $E_{\rm cut}$. 

For the angular momentum channels, 
we include from $(s_{1/2})^2$ to $(h_{11/2})^2$ partial waves, similarly to 
Chapter \ref{Ch_Results1}. 
In order to take into account the effect of the Pauli principle, 
we exclude the bound $(s_{1/2}$ state from $\phi_{nljm}$ in 
Eq.(\ref{eq:601basis_0p}), 
that is occupied by the protons in the core nucleus. 
The continuum states are discretized within the radial box of 
$R_{\rm box}=80$ fm. 
Even though this model space might be not sufficient to fully describe 
the \twop-emission of $^{6}$Be, 
increasing $l_{\rm max}$ or $R_{\rm box}$ causes a serious 
rise of computational costs, which makes our calculations 
practically impossible. 
%Thus, in this thesis, we present the best calculations which can be 
%terminated with our computing resources. 
Also note that $R_{\rm box}$ limits the interval for time-evolution, 
because the wave functions are inevitably reflected 
once it reaches at $r=R_{\rm box}$. 
This reflection as an calculational artifact causes 
the deviation from reality at the 
late-time region. 
A typical maximum time in our time-dependent calculations is 
$ct_{\rm max} \sim 2000$ fm, corresponding to 
the earlier stage of \twop-emissions. 



\subsection{Interactions}
We describe the interaction between $\alpha$ and a valence proton 
using the nuclear Woods-Saxon potential and 
the Coulomb potential, similarly in the previous Chapters. 
That is, 
\beq
 V_{\rm c-p}(r_i) = V_{\rm c-p, Nucl.} (r_i) + V_{\rm c-p, Coul.} (r_i), 
 \label{eq:cp_pot6}
\eeq
where the nuclear and Coulomb terms are 
\beqa
 V_{\rm c-N, Nucl.} (r) &=& 
   \left[ V_0 + V_{ls} r_0^2 (\bi{\ell} \cdot \bi{s}) 
   \frac{1}{r} \frac{d}{dr} \right] f(r), \label{eq:cp_WS} \\
 f(r) &=& 
   \frac{1}{ 1 + \exp \left( \frac{r-R_{\rm core}}{a_{\rm core}} \right) }, 
\eeqa
and 
\beq
 V_{\rm c-p, Coul.} (r) 
 = \left\{ \begin{array}{cc} 
   \frac{Z_{\rm c} e^2}{4\pi \epsilon_0} \frac{1}{r} 
    & (r > R_{\rm core}), \\
   \frac{Z_{\rm c} e^2}{4\pi \epsilon_0} \frac{1}{2R_{\rm core}} \left( 3 - \frac{r^2}{R_{\rm core}^2} \right) 
    & (r \leq R_{\rm core}). 
   \end{array} \right. \label{eq:cp_C}
\eeq
For the Coulomb term, we adopt the one of a uniform-charged sphere 
with the charge radius of the $\alpha$-particle, 
$R_{\rm core} = r_c = 1.68$ fm. 
For the Woods-Saxon part, we use 
$R_{\rm core} = r_c$ and $a_0$ = 0.615 fm, whereas 
strength parameters are fixed as $V_0 = -58.7$ MeV and 
$V_{ls} = 46.3$ ${\rm MeV(fm)^2}$. 
These parameters reproduce the measured resonance energy and width 
for the $(p_{3/2})$-channel of $\alpha$-p scattering \cite{02Till}: 
it yields $E_r (p_{3/2}) = 1.96$ MeV and $\Gamma_r (p_{3/2}) = 1.56$ MeV. 
We calculate and fit the derivative of the phase-shift, according to 
Eq.(\ref{eq:apcps}), to get this result. 
We note that this resonance is quite broad and there are large 
ambiguities in the observed 
decay width \cite{88Ajzen,00Hoef,02Till,09Shirokov}, 
as summarized in Table \ref{tb:709425}.
\begin{figure}[t] \begin{center}
$V_{\rm c-p}(r)$ and $V_{\rm c-p}^{conf}(r)$ \\
%\fbox{ \includegraphics[width=0.5\hsize,clip,trim = 10 0 5 5]{y_06BE_a/Vcp_5li.eps}}
\fbox{\includegraphics[width=0.5\hsize,scale=1,trim = 50 50 0 0]{FIG7021.eps}}
\caption{The original and confining potentials for 
$(s_{1/2})$, $(p_{3/2})$ and $(d_{5/2})$ channels in the $\alpha$-p subsystem. 
The border radius for modifying the potential is 5.7 fm for all the channels. } \label{fig:4001}
\end{center} \end{figure}
\begin{table}[b] \begin{center}
  \catcode`? = \active \def?{\phantom{0}} %define `?' as ' '(one-blank).
  \begingroup \renewcommand{\arraystretch}{1.2}
  \begin{tabular}{ccccc c} \hline \hline
  &                               && $E_r$ (MeV)  & $\Gamma_r$ (MeV) & \\ \hline
  & This work                     && 1.96????    & 1.56??           & \\
  & Ref.\cite{88Ajzen,09Shirokov} && 1.96(5)?    & $\simeq$ 1.5     & \\
  & Ref.\cite{00Hoef}             && 2.90(20)    & 1.0(2)           & \\
  & Ref.\cite{02Till}             && 1.69????    & 1.06??           & \\ \hline \hline
  \end{tabular}
  \endgroup
  \catcode`? = 12 %initialize `?'.
  \caption{The resonant energy and width of the $^{5}$Li 
nucleus in the $(p_{3/2})$-channel. } \label{tb:709425}
\end{center} \end{table}

For the proton-proton interaction, we use the Minnesota potential, 
in which the Coulomb term is explicitly included. 
\beq
 v_{\rm p-p}(r_{12}) = v_0 e^{-b_0 r^2_{12}} - v_1 e^{-b_1 r^2_{12}} 
 + \alpha \hbar c \frac{e^2}{r_{12}}. \label{eq:Minnepp}
\eeq
For $b_0, b_1$ and $v_1$ in Eq.(\ref{eq:Minnepp}), 
we use the same parameters introduced in the original paper \cite{77Thom}. 
On the other hand, the strength of the repulsive core, $v_0$, 
is adjusted so as to reproduce the empirical Q-value 
for two protons, $Q_{2p}=1.37$ MeV \cite{88Ajzen,02Till}. 

\section{Initial State}
As mentioned in Chapter \ref{Ch_TDM}, 
the initial configuration of the two protons is characterized such that 
the density distribution is localized around the core nucleus and 
has almost no amplitude outside the core-proton potential barrier. 
In order to generate such state, we employ 
the confining potential method \cite{87Gurv, 88Gurv, 04Gurv}. 
The confining potential for the initial \twop-state is defined as follows. 
Because the $\alpha$-p subsystem has a resonance at 
$E_0=1.96$ MeV in the $(p_{3/2})$-channel, 
the two protons in $^6$Be are expected to have a large component for 
the $(p_{3/2})^2$ configuration. 
Thus, we first modify the core-proton potential for the 
$(p_{3/2})$-channel at $t=0$ in order to 
generate a quasi-bound state as follows: 
\beq
 V_{{\rm c-p},~(p_{3/2})}^{conf}(r) \nonumber 
 = \left\{ \begin{array}{cc} 
            V_{{\rm c-p},~(p_{3/2})}(r) & (r \leq R_b), \\
            V_{{\rm c-p},~(p_{3/2})}(R_b) & (r > R_b), \end{array} \right.
\eeq
with $R_b=5.7$ fm. 
For the other s.p. channels, we define the confining potential as 
\beq
 V_{\rm c-p}^{conf}(r) \nonumber 
 =\left\{ \begin{array}{cc} 
          V_{\rm c-p}(r) \phantom{0000} & (r \leq R_b), \\
          V_{\rm c-p}(r) + V_b(r) & (r > R_b), \end{array} \right.
\eeq
where $V_b(r) = V_{{\rm c-p},~(p_{3/2})}(R_b) - V_{{\rm c-p},~(p_{3/2})}(r)$. 
The original and confining potentials for the 
$(s_{1/2})$, $(p_{3/2})$ and $(d_{5/2})$ channels are shown in 
Fig. \ref{fig:4001}. 
We note that, for this system, the core-proton barrier 
is mainly due to the centrifugal potential in the 
$(p_{3/2})$ channel, 
rather than the Coulomb potential. 
This situation is quite different from heavy \twop-emitters 
with a large proton-number, such as $^{45}$Fe. 
\begin{figure*}[tb] \begin{center}
  $^{6}$Be (g.s.), $t=0$, ``full'' \\
  \begin{tabular}{c} %switch-off the auto-turning
     \begin{minipage}{0.48\hsize} \begin{center}
%        \fbox{ \includegraphics[height=45truemm, scale=1, trim = 50 50 0 0]{./y_06BE_a/g_daba_t0.eps}}
        \fbox{ \includegraphics[height=45truemm, scale=1, trim = 50 50 0 0]{./FIG7031.eps}}
     \end{center} \end{minipage}
     \begin{minipage}{0.48\hsize} \begin{center}
%        \fbox{ \includegraphics[height=45truemm, scale=1, trim = 50 50 0 0]{./y_06BE_a/g_anga_t0.eps}}
        \fbox{ \includegraphics[height=45truemm, scale=1, trim = 50 50 0 0]{./FIG7032.eps}}
     \end{center} \end{minipage}
  \end{tabular}
  \caption{The left panel: 
The \twop-density distribution at $t=0$ for the ground state of $^6$Be. 
It is obtained by including all the partial waves up to $(h_{11/2})^2$, 
and is plotted as a function of 
$r_{\rm c-pp} = (r_1^2 + r_2^2 + 2r_1r_2\cos \theta_{12})^{1/2}/2$ and 
$r_{\rm p-p} = (r_1^2+r_2^2-2r_1r_2\cos \theta_{12})^{1/2}$. 
The right panel: The corresponding angular distribution 
obtained by integrating $\bar{\rho}_{2p}$ over $r_1$ and $r_2$. } \label{fig:2}
\end{center} \end{figure*}
\begin{figure*}[tb] \begin{center}
  $^{6}$Be (g.s.), $t=0$, ``$(l=odd)^2$ only'' \\
  \begin{tabular}{c} %switch-off the auto-turning
   \begin{minipage}{0.48\hsize} \begin{center}
%     \fbox{ \includegraphics[height=45truemm, scale=1, trim = 50 50 0 0]{./y_06BE_a/Ds4_t0_pp.eps}}
     \fbox{ \includegraphics[height=45truemm, scale=1, trim = 50 50 0 0]{./FIG7041.eps}}
   \end{center} \end{minipage}
   \begin{minipage}{0.48\hsize} \begin{center}
%     \fbox{ \includegraphics[height=45truemm, scale=1, trim = 50 50 0 0]{./y_06BE_a/Dan_t0_pp.eps}}
     \fbox{ \includegraphics[height=45truemm, scale=1, trim = 50 50 0 0]{./FIG7042.eps}}
   \end{center} \end{minipage}
  \end{tabular}
  \caption{The same as Fig. \ref{fig:2} but for the case 
with only the partial waves of $(l=odd)^2$. } \label{fig:3}
\end{center} \end{figure*}

The initial state for the \twop-emission 
is solved by diagonalizing the modified 
Hamiltonian including $V^{conf}_{\rm c-p}(r)$. 
In Fig. \ref{fig:2}, we display the density distribution of the 
initial state obtained in this way. 
\beq
 \bar{\rho}_{2p}(t=0;r_1,r_2,\theta_{12}) 
 = 8\pi^2 r_1^2 r_2^2 \sin \theta_{12} 
   \abs{\Phi(t=0;r_1,r_2,\theta_{12})}. 
\eeq
In the left panel of Fig. \ref{fig:2}, $\bar{\rho}_{2p}$ is plotted as a 
function of the distance between the core and the center of mass 
of two protons: 
$r_{\rm c-pp} = \sqrt{r_1^2 + r_2^2 + 2r_1r_2\cos \theta_{12}}/2$, 
and the relative distance between two protons: 
$r_{\rm pp} = \sqrt{r_1^2 + r_2^2 - 2r_1r_2\cos \theta_{12}}$. 
In the right panel of Fig. \ref{fig:2}, we also show 
the angular distributions obtained by integrating $\bar{\rho}_{2p}$ 
for $r_1$ and $r_2$. 

It is clearly seen that the wave function 
is confined inside the potential barrier at $r \cong 4$ fm 
(see Fig. \ref{fig:4001} again). 
Furthermore, the \twop-density is concentrated 
near $r_{\rm p-p} = 2$ fm, 
corresponding to the diproton correlation in the bound nuclei. 
The corresponding angular distribution becomes asymmetric 
and has the higher peak at the opening angle $\theta_{12} \cong \pi /6$. 
This peak is almost due to the spin-singlet configuration, 
being analogous to the dinucleon correlation. 

As we discussed in Chapter \ref{Ch_Results1}, 
the parity-mixing in the subsystem of the core and a nucleon 
plays an essential role in generating the dinucleon correlation. 
In order to confirm the similar effect in the \twop-emission, 
we have performed the same calculations but only with 
$(p,f,h)^2=(l=odd)^2$ partial waves. 
In this case, pairing correlations are partially taken into 
account only among the s.p. states with the same parity, 
although the parity-mixing in the core-nucleon subsystem 
is perfectly ignored. 
In Fig. \ref{fig:3}, 
we show the initial configuration obtained only 
with $(l=odd)^2$ partial waves. 
In the left panel of Fig. \ref{fig:3}, 
there are two comparable peaks at $r_{\rm p-p} = 2$ and $5$ fm 
whereas, in the right panel, 
the corresponding angular distribution has a symmetric form. 
This result is in contrast with that in the full mixing 
case (see Fig. \ref{fig:2}), 
where the parity-mixing is fully taken into account (``full mixing''). 

The empirical Q-value for the \twop-emission is 1.37 MeV 
for $^6$Be \cite{88Ajzen,02Till}. 
However, the original parameters of the Minnesota potential 
overestimates this value, for instance, by about 50\% in 
the full mixing case. 
We therefore modify the parameter $v_0$ in Eq.(\ref{eq:Minnepp}) 
so as to yield Q = 1.37 MeV. 
Note that we use the modified $v_0$ not only at $t=0$ but also 
during the time-evolutions. 
\begin{table}[tb] \begin{center}
  \catcode`? = \active \def?{\phantom{0}} %define `?' as ' '(one-blank).
  \begingroup \renewcommand{\arraystretch}{1.0}
  \begin{tabular*} {\hsize} { @{\extracolsep{\fill}} ccccc ccc} \hline \hline
  && \multicolumn{3}{c}{$^{6}$Be, $t=0$}         && $^{6}$He & \\ \cline{3-5} \cline{7-7}
  && full & $(l=odd)^2$ only & $(p_{3/2})^2$ only && full    & \\ \hline
   $\Braket{H_{\rm 3b}}$ (MeV) && 1.37 & 1.37 & 1.37 && $-0.975$ & \\
  &&&&& && \\
   $\Braket{r_{\rm N-N}}$ (fm)   && 4.92 & 5.29 & 5.16 && 4.67 & \\
   $\Braket{r_{\rm c-NN}}$ (fm)  && 3.43 & 2.64 & 2.58 && 3.18 & \\
   $\Braket{\theta_{12}}$ (deg) && 75.9 & 90.0 & 90.0 && 78.0 & \\
  &&&&& && \\
   $(p_{3/2})^2$ (\%)        && 88.9  & 97.1 & 100. && 92.7 & \\
   $(p_{1/2})^2$ (\%)        && ?3.1  & ?2.8 & ??0. && ?1.6 & \\
   $(s_{1/2})^2$ (\%)        && ?2.2  & ?0.? & ??0. && ?1.3 & \\
   others,$(l=even)^2$ (\%) && ?5.2  & ?0.? & ??0. && ?4.2 & \\
   others,$(l=odd)^2$ (\%)  && ?0.6  & ?0.1 & ??0. && ?0.2 & \\
  &&&&& && \\
   $P(S_{12}=0)$ (\%) && 82.2 & 80.6 & 66.6 && 78.1 & \\
  &&&&& && \\
   $v_0$ (MeV) && 156.0 & 88.98 & 66.69 && 212.2 & \\ \hline \hline
  \end{tabular*}
  \endgroup
  \catcode`? = 12 %initialize `?'.
  \caption{Calculated properties for the initial state of $^6$Be 
and $^6$He. 
The results with all the uncorrelated basis 
from $(s_{1/2})^2$ to $(h_{11/2})^2$ are 
labeled by ``full''. 
Those obtained only with the $(l=odd)^2$ and $(p_{3/2})^2$ 
bases are also shown. 
The values of $v_0$ (the strength of the repulsive part) for the 
nucleon-nucleon interaction (Eq.(\ref{eq:Minnepp})) 
are tabulated in the last row. 
The original value is $v_0 = 200.$ MeV \cite{77Thom}. } \label{tb:71111}
\end{center} \end{table}
%   $\Braket{v_{\rm N-N}}$ (MeV)      && $-4.26$  & $-5.83$  & $-5.66$  && $-3.53$ & \\
%   $\Braket{v_{\rm N-N, Nucl.}}$ (MeV) && $-4.79$  & $-5.36$  & $-5.20$  && $-3.53$ & \\
%   $\Braket{v_{\rm N-N, Coul.}}$ (MeV) && $??0.52$ & $??0.47$ & $??0.46$ && $??0.??$ & \\
%   $\Braket{\rm recoil}$ (MeV)     && $-1.23$  & $??0.??$ & $??0.??$ && $-1.14$ & \\
%  &&&&& && \\

In Table \ref{tb:71111}, properties of the initial state are summarized. 
In this Table, for comparison, we also perform the same 
calculation but in the 3rd case, namely 
only with the $(p_{3/2})^2$ partial wave. 
The values of $v_0$ in the Minnesota potential 
are tabulated in the last row in Table \ref{tb:71111}. 
It is clearly seen that, 
in the full mixing case, the main component is $(p_{3/2})^2$, 
reflecting that the $(p_{3/2})$ channel has a 
resonance in the $\alpha$-p subsystem. 
The mixing of different partial waves are due to the 
off-diagonal matrix elements of $H_{\rm 3b}$, 
corresponding to the pairing correlations. 
The spin-singlet configuration is remarkably enhanced 
for the full mixing case compared to 
that in the $(p_{3/2})^2$ case. 
On the other hand, in the case only with $(l=odd)^2$, 
a comparable enhancement of the $S_{12}=0$ configuration exists, 
even though there is no localization of the two protons 
as shown in Fig. \ref{fig:3}. 
Notice also that in the $(l=odd)^2$ case, we have to assume a stronger 
pairing attraction in order to reproduce the empirical Q-value, 
as compared to the full-mixing case. 

\subsection{Comparison with $^6$He}
From the point of view of the isobaric symmetry in nuclei, 
it is interesting to compare the initial state of $^6$Be with the 
ground state of its mirror nucleus, $^6$He. 
Assuming the $\alpha$+n+n structure, 
we perform the similar calculation but for the ground state of $^6$He. 
For the $\alpha$-n system, there is an observed 
resonance of $(p_{3/2})$ at 
$E_r=0.735(20)$ MeV with its width, 
$\Gamma_r=0.600(20)$ \cite{NNDCHP,88Ajzen}. 
In order to reproduce this resonance, we exclude the Coulomb term from 
Eq.(\ref{eq:cp_pot6}) and modify the depth parameter 
to $V_0 = -61.25$ MeV in Woods-Saxon potential. 
The pairing interaction is adjusted to reproduce 
$\Braket{H_{\rm 3b}}=-S_{\rm 2n}=-0.975$ MeV \cite{NNDCHP}, 
yielding $v_0=212.2$ MeV in Eq.(\ref{eq:Minne}). 
One may concern the difference of $v_0$ between $^6$Be and $^6$He. 
This might be due to an ambiguity in $V_{\rm c-p}$ for $^6$Be 
originated from a broad resonance in the core-proton subsystem. 
Improving $V_{\rm c-p}$ in $^6$Be can lead to the more consistent 
set of parameters among $V_{\rm c-N}$ and $v_{\rm N-N}$. 
We note that this ambiguity does not affect our qualitative discussions. 
\begin{figure*}[htb] \begin{center}
  $^{6}$He (g.s.), ``full'' \\
  \begin{tabular}{c} %switch-off the auto-turning
     \begin{minipage}{0.48\hsize} \begin{center}
%        \fbox{ \includegraphics[height=45truemm, scale=1, trim = 50 50 0 0]{./y_06HE/g_daba.eps}}
        \fbox{ \includegraphics[height=45truemm, scale=1, trim = 50 50 0 0]{./FIG7051.eps}}
     \end{center} \end{minipage}
     \begin{minipage}{0.48\hsize} \begin{center}
%        \fbox{ \includegraphics[height=45truemm, scale=1, trim = 50 50 0 0]{./y_06HE/g_anga.eps}}
        \fbox{ \includegraphics[height=45truemm, scale=1, trim = 50 50 0 0]{./FIG7052.eps}}
     \end{center} \end{minipage}
  \end{tabular}
\caption{The density distribution of the valence two neutrons, $\bar{\rho}_{2n}$, in the ground state of $^6$He. 
Those are plotted in the same manner as in the left and right panels 
of Figure \ref{fig:2}. 
The partial waves up to $(h_{11/2})^2$ are included. } \label{fig:22}
\end{center} \end{figure*}

In Fig. \ref{fig:22}, the two-neutron density distribution is 
shown in the same manner as in Fig. \ref{fig:2}. 
The energetic and structural properties are tabulated in the last column 
of Table \ref{tb:71111}. 
Obviously, the two-neutron wave function in $^6$He 
has a similar distribution to 
the \twop-wave function in $^6$Be. 
Because the two neutrons are bound in this system, the spatial 
distribution is less expanded in $^6$He. 
This is why both $\Braket{r_{\rm N-N}}$ and $\Braket{r_{\rm c-NN}}$ have 
smaller values than those of $^{6}$Be. 
The dinucleon correlation is present 
also in $^6$He, characterized as the spatial localization with the 
enhanced spin-singlet component \cite{05Hagi}. 
Consequently, the confining potential which we employ 
provides the initial state 
of $^6$Be, which can be interpreted as the isobaric analogue state of $^6$He. 
\begin{figure}[tb] \begin{center}
% \fbox{ \includegraphics[width=0.5\hsize, scale=1, trim = 50 50 0 0]{./y_06BE_a/Nd_Gam_vs.eps}}
 \fbox{ \includegraphics[width=0.5\hsize, scale=1, trim = 50 50 0 0]{./FIG7061_Gam.eps}}
 \caption{The decay probabilities and the decay widths 
for the \twop-emissions from $^6$Be, obtained with the time-dependent method. 
The result with all the partial waves in the model-space 
(full mixing) is plotted by the solid line. 
For c$t \geq 1000$ (fm), the decay width well converges to a constant 
value of $88.2$ keV in the full-mixing case. 
The experimental value, $\Gamma_{\rm Exp} = 92 \pm 6$ keV \cite{02Till}, 
is marked by the bold line. 
The results obtained only with $(p_{3/2})^2$ (dotted line) 
and $(l=odd)^2$ (broken line) partial waves are also shown, 
where the calculated decay widths are clearly 
underestimated. } \label{fig:694}
\end{center} \end{figure}
%Apart from the Coulomb interaction, the time-dependent analysis can 
%regard not only the diproton but also 
%the dineutron correlations in bound nuclei. 

\section{Decay Width}
Starting from the initial state obtained in the previous section, 
we perform the time-evolving calculations for the first $0^+$ 
resonance of $^{6}$Be. 
We show the results of the decay-component $N_d(t)$ 
and width $\Gamma (t)$ 
(see the Eqs.(\ref{eq:603DComp}) and (\ref{eq:width}) in 
the previous Chapter) 
obtained with the time-evolution in Fig. \ref{fig:694}. 

In Fig. \ref{fig:694}, the calculation is carried out up to 
$ct = 0-1400$ fm. 
We have confirmed that the artifact due to the reflection 
at $r=R_{\rm box}$ is negligible in this time-interval. 
One can clearly see that, 
after a sufficient time-evolution, the decay width converges 
to a constant value for all the cases, 
and the exponential decay-rule is realized. 
Furthermore, the result for the full case yields the saturated value 
of $\Gamma (t) \cong 88.2$keV, which 
reproduces the experimental decay width, 
$\Gamma = 92 \pm 6$ keV \cite{88Ajzen,02Till}. 
On the other hand, 
the limitation of the partial waves only to $(l=odd)^2$ or $(p_{3/2})^2$ 
significantly underestimates the decay width. 
This is caused by an increase of the pairing attraction: 
With the $(l=odd)^2$ or $(p_{3/2})^2$ waves only, 
to reproduce the empirical Q-value, 
we needed a stronger pairing attraction (see Table \ref{tb:71111}). 
The two protons are then strongly bound to each other 
and are difficult to go outside, even they have a similar energy release 
of that in the full mixing case. 

From these studies, we can conclude that the parity-mixing 
in the core-proton subsystem is indispensable in order to 
reproduce simultaneously the Q-value and the decay 
width of the \twop-emission. 
This result supports the assumption of the diproton 
correlation at $t=0$. 
\begin{figure*}[htb] \begin{center}
  \begin{tabular}{c} %switch-off the auto-turning
     \begin{minipage}{0.48\hsize} \begin{center}
     (a) full \\ 
%     \fbox{ \includegraphics[height=42truemm, scale=1, trim = 50 50 0 0]{./y_06BE_b/fgs/Gam_S12.eps}}
     \fbox{ \includegraphics[height=42truemm, scale=1, trim = 50 50 0 0]{./FIG7071.eps}}
     \end{center} \end{minipage}
     \begin{minipage}{0.48\hsize} \begin{center}
     (b) $(l=odd)^2$ only \\ 
%     \fbox{ \includegraphics[height=42truemm, scale=1, trim = 50 50 0 0]{./y_06BE_b/fgs_pp/Gam_S12_pp.eps}}
     \fbox{ \includegraphics[height=42truemm, scale=1, trim = 50 50 0 0]{./FIG7072.eps}}
     \end{center} \end{minipage}
  \end{tabular}
  \caption{(a) The total and the partial decay widths for the spin-singlet 
and triplet configurations of $^6$Be. 
The partial waves from $(s_{1/2})^2$ to $(h_{11/2})^2$ are fully included. 
(b) The same as panel (a) but for the case with 
only $(l=odd)^2$ partial waves. } \label{fig:5-12}
\end{center} \end{figure*}

For the cases of full mixing and only $(l=odd)^2$ partial waves, 
we also calculate the partial decay widths for the 
spin-singlet and triplet configurations. 
The corresponding formula to Eq.(\ref{eq:pwidth}) is given as 
\beq
 \Gamma_{S_{12}}(t) \equiv \frac{\hbar }{1-N_d(t)} \frac{d}{dt} N_{d,S_{12}}(t), 
\eeq
where 
\beqa
 N_{d,S_{12}}(t) &\equiv& \Braket{\Phi_{d,S_{12}}(t) | \Phi_{d,S_{12}}(t)} \nonumber \\
 &=& \int_0^{R_{\rm box}} dr_1 \int_0^{R_{\rm box}} dr_2 \int_{0}^{\pi} d\theta_{12} 
     8\pi^2 r_1^2 r_2^2 \sin \theta_{12} \abs{\Phi_{d,S_{12}} (t;r_1,r_2,\theta_{12})}^2, \\
 \ket{\Phi_d (t)} &\equiv& 
 \ket{\Phi (t)} - \beta(t) \ket{\Phi (0)}, \label{eq:dst}
\eeqa
with $\beta (t) = \Braket{\Phi (0) | \Phi (t)}$. 
The results are shown in Fig. \ref{fig:5-12}. 
Clearly, the spin-singlet configuration almost exhausts 
the decay width in the full mixing case shown in Fig. \ref{fig:5-12}(a). 
This suggests that the emitted two protons from the ground state of $^6$Be 
have mostly the configuration of $S_{12}=0$, like a diproton. 
On the other hand, from Fig. \ref{fig:5-12}(b), 
one can see that 
the spin-triplet configuration occupies a considerable amount 
of the total decay width when we exclude $(s,d,g)^2$ partial waves. 
\begin{table}[htb] \begin{center}
  \catcode`? = \active \def?{\phantom{0}} %define `?' as ' '(one-blank).
  \begingroup \renewcommand{\arraystretch}{1.2}
  \begin{tabular*}{\hsize} { @{\extracolsep{\fill}} ccccc cc} \hline \hline
    & & full & $(l=odd)^2$ only & $(l=odd)^2 \oplus (s_{1/2})^2$ & no pairing & exp.data \\
    & & & & & ($ct=3000$ fm) & \cite{88Ajzen,02Till} \\ \hline
    & $\Gamma_{\rm total}$ (keV) & 88.2 & 12.5 & 35.8 & 348. & 92(6) \\
    & $\Gamma_{S_{12}=0}$ (keV) & 87.1 & 10.7 & 34.3 & 232. & - \\
    & $\Gamma_{S_{12}=1}$ (keV) & ?1.1 & ?1.8 & ?1.5 & 116. & - \\ \hline \hline
  \end{tabular*}
  \endgroup
  \catcode`? = 12 %initialize `?'.
  \caption{The contributions from the spin-singlet and triplet 
configurations to the decay width of $^6$Be. 
The values are evaluated at $ct=1200$ fm, except for the ``no pairing'' case, 
whose values are evaluated at $ct=3000$ fm. 
Note that in the all cases, the Q-value of the \twop-emission 
is reproduced consistently to the experimental value, 
$1.37$ MeV \cite{88Ajzen,02Till}. } \label{tb:6473}
\end{center} \end{table}

In the 2nd and the 3rd columns of Table \ref{tb:6473}, 
we tabulate the total and partial widths in the full and 
the $(l=odd)^2$ cases, respectively. 
The values are estimated at $ct=1200$ fm, where the total widths 
sufficiently converge. 
Clearly, there is a significant increase of the spin-singlet width 
in the full mixing case, 
by about one-order magnitude larger than that in the case of $(l=odd)^2$. 
On the other hand, we get similar values of the spin-triplet width 
in the full and $(l=odd)^2$ cases. 
From this result, 
we can conclude that the core-nucleon parity-mixing is responsible 
for the enhancement of the spin-singlet emission, 
although the dominance of the spin-singlet configuration 
in the initial state is apparent in both the 
two cases (see Table \ref{tb:71111}). 

The dominance of the spin-singlet 
configuration is due to the $(s_{1/2})^2$ channel. 
Considering the coupled orbit, $L_{12} \equiv l_1 \oplus l_2$, 
from the coupling rule to the spin-parity of $0^+$, 
the $(s_{1/2})^2$ channel leads to $S_{12} = L_{12} = 0$ 
with $l_1 = l_2 = 0$. 
Because there is no centrifugal barrier in this channel, 
the spin-singlet emission can be dominant. 
On the other hand, for the spin-triplet configuration, 
the only $L_{12} = 1$ are permitted in order to have the 
total angular momentum $0^+$. 
Thus, there is a centrifugal barrier for all the channels in 
the spin-triplet configuration. 
Consequently, apart from the reduction due to the stronger 
pairing attraction, 
the spin-triplet width has similar values to each other 
in the full-mixing and $(l=odd)^2$ cases. 
\begin{figure*}[htb] \begin{center}
  \begin{tabular}{c} %switch-off the auto-turning
     $(l=odd)^2 \oplus (s_{1/2})^2$ \\ 
%     \fbox{ \includegraphics[height=42truemm, scale=1, trim = 50 50 0 0]{./y_06BE_b/fgs_bb/Gam_S12_bb.eps}}
     \fbox{ \includegraphics[height=42truemm, scale=1, trim = 50 50 0 0]{./FIG7081.eps}}
  \end{tabular}
  \caption{The same as Fig. \ref{fig:5-12} but for the 
$(l=odd)^2$ plus $(s_{1/2})^2$ case. } \label{fig:7ljfg}
\end{center} \end{figure*}

In order to check this effect of the $(s_{1/2})^2$ waves directly, 
we perform the same calculation but including the 
uncorrelated bases with $(l=odd)^2$ and 
$(s_{1/2})^2$ configurations. 
Namely, we add only the $(s_{1/2})^2$ waves to the $(l=odd)^2$ case. 
In this case, we use the same parameters for the calculation as those 
for the full-mixing and the $(l=odd)^2$ cases, 
except for the $v_0$ in the Minnesota potential: 
we use $v_0=99.14$ MeV in order to reproduce the Q-value, 
$Q_{2p}=1.37$ MeV for $^6$Be. 
The result is shown in Figure \ref{fig:7ljfg} and in the 4th column 
in Table \ref{tb:6473} in the same manner as the former two cases. 
One can see that the spin-singlet width is significantly increased 
due to the existence of the $(s_{1/2})^2$ channel, whereas the 
spin-triplet width has a similar value to those in the 
full-mixing and $(l=odd)^2$ cases. 
This result supports our former speculation about the role 
of the $(s_{1/2})^2$ channel in the \twop-emission. 
Notice also that, because of the stronger pairing attraction, 
the total width in Figure \ref{fig:7ljfg} is still underestimated 
than the experimental data. 

\section{Time-Evolution of Decay State}
In order to discuss the emission process, 
we show the density distribution of the decay state, 
\beqa
 && \bar{\rho}_{d}(t) = 8\pi^2 r_1^2 r_2^2 \sin \theta_{12} \rho_d(t), \\
 && \rho_d(t) = \abs{\Phi_{d} (t;r_1,r_2,\theta_{12})}^2. 
\eeqa
The most of the amplitude of the decay state exists 
outside the potential barrier, 
because we prepare the initial state, which is orthogonal to the decay state, 
so as to have no amplitude in that region. 
For the presentation, 
we renormalize the $\bar{\rho}_d(t)$ so that 
its integration become unity at each time: 
\beq
  \bar{\rho}_d (t) \longrightarrow \frac{\bar{\rho}_d (t)}{N_d (t)}, 
\eeq
where $N_d (t)$ is the decay probability given by Eq.(\ref{eq:603DComp}). 
We adopt three sets of radial coordinates in the following. 
(i) The first set includes 
$r_{\rm c-pp} = (r_1^2 + r_2^2 + 2r_1r_2\cos \theta_{12})^{1/2}/2$ and 
$r_{\rm p-p} = (r_1^2+r_2^2-2r_1r_2\cos \theta_{12})^{1/2}$, 
similarly to the left panel of Figure \ref{fig:2}. 
(ii) In the second set, we integrate $\bar{\rho}_d$ with respect to 
the opening angle, $\theta_{12}$, 
and plot it as a function of $r_1$ and $r_2$. 
In order to see the peak-structure clearly, 
we omit the radial weight $r_1^2 r_2^2$ in 
$\bar{\rho}_d$ in the second setting. 
(iii) Within the third set, on the other hand, 
we integrate $\bar{\rho}_d(t)$ over radial distances, 
and plot it as a function of $\theta_{12}$. 
\begin{figure*}[tp] \begin{center}
  \begin{tabular}{c} %switch-off the auto-turning
     \begin{minipage}{0.48\hsize} \begin{center}
        (a) diproton \\
     \fbox{ \includegraphics[height=48truemm, clip, trim = 0 20 0 10]
 {FIG7091.eps}} \vspace{10pt} \\% {y_maps/map1.eps}} \vspace{10pt} \\
        (c) simultaneous, $\theta_{12}=0$ \\
     \fbox{ \includegraphics[height=48truemm, clip, trim = 0 20 0 10]
 {FIG7092.eps}} \vspace{10pt} \\% {y_maps/map2.eps}} \vspace{10pt} \\
        (e) correlated \\
     \fbox{ \includegraphics[height=48truemm, clip, trim = 0 20 0 10]
 {FIG7093.eps}} \vspace{10pt} \\% {y_maps/map7.eps}} \vspace{10pt} \\
     \end{center} \end{minipage}

     \begin{minipage}{0.48\hsize} \begin{center}
        (b) simultaneous, $\theta_{12}=\pi/2$ \\
     \fbox{ \includegraphics[height=48truemm, clip, trim = 0 20 0 10]
 {FIG7094.eps}} \vspace{10pt} \\% {y_maps/map3.eps}} \vspace{10pt} \\
        (d) one-proton \\
     \fbox{ \includegraphics[height=48truemm, clip, trim = 0 20 0 10]
 {FIG7095.eps}} \vspace{10pt} \\% {y_maps/map4.eps}} \vspace{10pt} \\
        (f) sequential \\
     \fbox{ \includegraphics[height=48truemm, clip, trim = 0 20 0 10]
 {FIG7096.eps}} \vspace{10pt} \\% {y_maps/map8.eps}} \vspace{10pt} \\
     \end{center} \end{minipage}
  \end{tabular}
  \caption{Schematic illustrations for the trajectories of 
different \twop-emission modes. } \label{fig:80}
\end{center} \end{figure*}

Before we show the results of the actual calculations, 
we schematically illustrate the dynamic of the 
\twop-emissions in Fig. \ref{fig:80}. 
From the geometry, one can distinguish two modes: 
``simultaneous two-proton'' and ``one-proton ($1p-$)'' emissions. 
The diproton emission is a special case in the first category. 
The second category corresponds to the case 
where only one proton penetrate the barrier. 
The trajectories of three simultaneous \twop- and a $1p$-emissions 
are schematically shown in 
Fig. \ref{fig:80}(a), (b), (c) and (d). 

In the simultaneous emissions, 
two protons are emitted simultaneously with their opening 
angle remaining from 
$\theta_{12}=0$ to $\pi$, where $\theta_{12}=0$ corresponds to 
the diproton emission. 
Fig. \ref{fig:80}(a), (b) and (c) correspond to $\theta_{12}=0,\pi/2$ 
and $\pi$. respectively. 
In these cases, the density in the $(r_1,r_2)$-plane shows the same 
patterns in these figures, and is concentrated along $r_1\cong r_2$. 
The simultaneous emissions with different opening angles 
 can be distinguished only in the $(r_{\rm p-p}, r_{\rm c-pp})$-plane: 
for instance, in the diproton emission, 
the probability shows mainly along the line with 
$r_{\rm c-pp} \gg r_{\rm p-p}$, 
while it is along the line with $r_{\rm c-pp}=0$ for $\theta_{12}=\pi$. 
In the one-proton emission shown in Fig.\ref{fig:80}(d), 
only one of the two protons goes through while 
the other proton remains inside the core nucleus. 
This is seen as the increment along $r_{\rm c-pp} \cong r_{\rm p-p}/2$ 
and $r_1$ or $r_2 \cong 0$ lines. 
\begin{figure*}[tp] \begin{center}
  \begin{tabular}{c} %switch-off the auto-turning
   \begin{minipage}{0.32\hsize} \begin{center}
     \fbox{ \includegraphics[height=31truemm, scale=0.9, trim = 60 55 0 0]
{FIG7101a.eps}} \\%{./y_06BE_b/fgs/ds4_dt0100.eps}} \\
     \fbox{ \includegraphics[height=31truemm, scale=0.9, trim = 60 55 0 0]
{FIG7101b.eps}} \\%{./y_06BE_b/fgs/ds4_dt0200.eps}} \\
     \fbox{ \includegraphics[height=31truemm, scale=0.9, trim = 60 55 0 0]
{FIG7101c.eps}} \\%{./y_06BE_b/fgs/ds4_dt0600.eps}} \\
     \fbox{ \includegraphics[height=31truemm, scale=0.9, trim = 60 55 0 0]
{FIG7101d.eps}} \\%{./y_06BE_b/fgs/ds4_dt1000.eps}} \\
   \end{center} \end{minipage}

   \begin{minipage}{0.32\hsize} \begin{center}
     \fbox{ \includegraphics[height=31truemm, scale=0.9, trim = 60 55 0 0]
{FIG7102a.eps}} \\%{./y_06BE_b/fgs/ds3_dt0100.eps}} \\
     \fbox{ \includegraphics[height=31truemm, scale=0.9, trim = 60 55 0 0]
{FIG7102b.eps}} \\%{./y_06BE_b/fgs/ds3_dt0200.eps}} \\
     \fbox{ \includegraphics[height=31truemm, scale=0.9, trim = 60 55 0 0]
{FIG7102c.eps}} \\%{./y_06BE_b/fgs/ds3_dt0600.eps}} \\
     \fbox{ \includegraphics[height=31truemm, scale=0.9, trim = 60 55 0 0]
{FIG7102d.eps}} \\%{./y_06BE_b/fgs/ds3_dt1000.eps}} \\
   \end{center} \end{minipage}

   \begin{minipage}{0.32\hsize} \begin{center}
     \fbox{ \includegraphics[height=31truemm, scale=0.7, trim = 60 55 0 0]
{FIG7103a.eps}} \\%{./y_06BE_b/fgs/dng_dt0100.eps}} \\
     \fbox{ \includegraphics[height=31truemm, scale=0.7, trim = 60 55 0 0]
{FIG7103b.eps}} \\%{./y_06BE_b/fgs/dng_dt0200.eps}} \\
     \fbox{ \includegraphics[height=31truemm, scale=0.7, trim = 60 55 0 0]
{FIG7103c.eps}} \\%{./y_06BE_b/fgs/dng_dt0600.eps}} \\
     \fbox{ \includegraphics[height=31truemm, scale=0.7, trim = 60 55 0 0]
{FIG7103d.eps}} \\%{./y_06BE_b/fgs/dng_dt1000.eps}} \\
   \end{center} \end{minipage}
  \end{tabular}
\caption{The \twop-density distribution for the decay states, 
$\bar{\rho}_d(t)$, obtained with the time-evolving calculations. 
All the uncorrelated partial waves up to $(h_{11/2})^2$ are included. 
(The left column): These distributions are plotted as a function of 
$r_{\rm c-pp} = (r_1^2 + r_2^2 + 2r_1r_2\cos \theta_{12})^{1/2}/2$ and 
$r_{\rm p-p} = (r_1^2+r_2^2-2r_1r_2\cos \theta_{12})^{1/2}$. 
(The middle column): The same as the left column but 
as a function of $r_1$ and $r_2$, 
obtained by integrating $\bar{\rho}_d$ for $\theta_{12}$. 
In order to clarify the peak(s), 
the radial weight $r_1^2 r_2^2$ is omitted. 
(The right column): The angular distributions of the decay state 
plotted as a function of the opening angle $\theta_{12}$ 
between the two protons. 
It is obtained by integrating $\bar{\rho}_d(t)$ 
for the radial coordinates. 
Beside the total distribution, the spin-singlet and triplet 
components are also plotted. } \label{fig:81}
\end{center} \end{figure*}

In Fig. \ref{fig:80}(e) and (f), 
we additionally illustrate the two hybrid processes. 
The first one is a ``correlated emission'', 
shown in Fig. \ref{fig:80}(e). 
In the correlated emission, 
the two protons are emitted simultaneously 
to almost the same direction, 
holding the diproton-like configuration. 
In this mode, at the earlier stage of tunneling, 
the density distribution 
has a larger amplitude in the region with 
$r_1 \cong r_2$ and small $\theta_{12}$. 
In the $(r_{\rm p-p}, r_{\rm c-pp})$-plane, 
It corresponds to the increment of the probability 
in the region of $r_{\rm p-p} \ll r_{\rm c-pp}$. 
After the barrier penetration, the 
two protons separate from each other 
mainly due to the Coulomb repulsion, 
increasing $r_{\rm p-p}$. 

The second hybrid process is a ``sequential emission'', 
which is shown in Fig. \ref{fig:80}(f). 
In this mode, 
there is a large possibility of that one proton is emitted 
whereas the other proton remains around the core. 
The density distribution shows high peaks along 
$r_1 \gg r_2$ and $r_1 \ll r_2$. 
In the $(r_{\rm p-p}, r_{\rm c-pp})$-plane, it corresponds to 
the increment along the line of 
$r_{\rm c-pp} \cong r_{\rm p-p}/2$. 
Being different from the pure one-proton emission, 
the remaining proton eventually goes 
through the barrier also 
when the core-proton subsystem is unbound. 
\begin{figure*}[tp] \begin{center}
  \begin{tabular}{c} %switch-off the auto-turning
%   \begin{minipage}{0.32\hsize} \begin{center}
%     \fbox{ \includegraphics[height=31truemm, scale=0.9, trim = 60 55 0 0]
%{./y_06BE_b/fgs_pp/ds4_pp_dt0100.eps}} \\
%     \fbox{ \includegraphics[height=31truemm, scale=0.9, trim = 60 55 0 0]
%{./y_06BE_b/fgs_pp/ds4_pp_dt0200.eps}} \\
%     \fbox{ \includegraphics[height=31truemm, scale=0.9, trim = 60 55 0 0]
%{./y_06BE_b/fgs_pp/ds4_pp_dt0600.eps}} \\
%     \fbox{ \includegraphics[height=31truemm, scale=0.9, trim = 60 55 0 0]
%{./y_06BE_b/fgs_pp/ds4_pp_dt1000.eps}} \\
%   \end{center} \end{minipage}
%%---
%   \begin{minipage}{0.32\hsize} \begin{center}
%     \fbox{ \includegraphics[height=31truemm, scale=0.9, trim = 60 55 0 0]
%{./y_06BE_b/fgs_pp/ds3_pp_dt0100.eps}} \\
%     \fbox{ \includegraphics[height=31truemm, scale=0.9, trim = 60 55 0 0]
%{./y_06BE_b/fgs_pp/ds3_pp_dt0200.eps}} \\
%     \fbox{ \includegraphics[height=31truemm, scale=0.9, trim = 60 55 0 0]
%{./y_06BE_b/fgs_pp/ds3_pp_dt0600.eps}} \\
%     \fbox{ \includegraphics[height=31truemm, scale=0.9, trim = 60 55 0 0]
%{./y_06BE_b/fgs_pp/ds3_pp_dt1000.eps}} \\
%   \end{center} \end{minipage}
%%---
%   \begin{minipage}{0.32\hsize} \begin{center}
%     \fbox{ \includegraphics[height=31truemm, scale=0.7, trim = 60 55 0 0]
%{./y_06BE_b/fgs_pp/dng_dt0100_pp.eps}} \\
%     \fbox{ \includegraphics[height=31truemm, scale=0.7, trim = 60 55 0 0]
%{./y_06BE_b/fgs_pp/dng_dt0200_pp.eps}} \\
%     \fbox{ \includegraphics[height=31truemm, scale=0.7, trim = 60 55 0 0]
%{./y_06BE_b/fgs_pp/dng_dt0600_pp.eps}} \\
%     \fbox{ \includegraphics[height=31truemm, scale=0.7, trim = 60 55 0 0]
%{./y_06BE_b/fgs_pp/dng_dt1000_pp.eps}} \\
%   \end{center} \end{minipage}
(Figure is hidden in open-print version.)    
  \end{tabular}
\caption{The same as Fig.\ref{fig:81} but for the case with 
only $(l=odd)^2$ waves. 
Notice a different scale in the left column 
from that in Fig.\ref{fig:81}. } \label{fig:82}
\end{center} \end{figure*}



\subsection{Full-Mixing Case}
We now show the results of the time-dependent calculations 
for the \twop-emission of $^6$Be. 
We first discuss the full-mixing case which is the closest 
assumption to reality. 
The density distribution for the decay state 
along the time-evolution is shown in Fig. \ref{fig:81}. 
The left, middle and right columns correspond to the coordinate sets 
(i), (ii) and (iii), respectively. 
The 1st to 4th panels in each column show the decay-density at 
$ct=100,200,600$ and $1000$ fm, respectively. 
For a presentation purpose, we normalize $\bar{\rho}_d$ 
at any step of time. 

In the left and middle columns of Fig. \ref{fig:81}, 
it can be seen that the process in this case 
is likely the correlated emission shown in Fig. \ref{fig:80}(e). 
Contributions from the other modes shown 
in Fig. \ref{fig:80} are small. 
In the middle column of  Fig. \ref{fig:81}, 
during the time-evolution, 
there is a significant increment of 
$\bar{\rho}_d$ along the line with $r_1 \cong r_2$. 
The corresponding peak in the left column is at 
$r_{\rm p-p} \ll r_{\rm c-pp} \cong 10$ fm, which means 
a small value of $\theta_{12}$. 
It should also be noted that, after the barrier penetration, 
the two protons lose their diproton-like configuration 
due to the Coulomb repulsion 
increasing $r_{p-p}$. 
Thus, for $r_{\rm c-pp} \geq 10$ fm which is 
a typical position 
of the potential barrier from the core, 
the density distribution extends around 
the $r_{\rm c-pp} \cong r_{\rm p-p}$ region. 
In this process, the pairing correlation plays an important role 
to generate the significant diproton-like configuration before 
the end of the barrier penetration, 
similarly to the dinucleon correlations. 

In the right column of Fig. \ref{fig:81}, the distributions 
are also displayed as a function of 
the opening angle, $\theta_{12}$. 
We can clearly see that the decay state has a high peak at 
$\theta_{12} \cong \pi/6$, and thus 
the emitted two protons should show the opening angles 
close to this value. 
However, this result may appear 
somewhat inconsistent to the experiments, 
in which the correlation is much weaker in the observed 
angular distribution of $^{6}$Be \cite{09Gri_80, 09Gri_677} 
(see Fig. \ref{fig:2009Gri_06Be}). 
A reason for this discrepancy is due to the final-state 
interactions (FSIs) at the late stage of propagation of 
the two protons. 
In the experiments, 
the observed spectra and the correlation patterns 
correspond to those at the 
late-time region, 
where the two protons have been much 
influenced by FSIs. 
On the other hand, in this thesis, we mainly discuss the 
earlier stage of the \twop-emission with a 
small value of $R_{\rm box}$. 
%Thus, the inconsistency itself is not a unnatural result, because the 
%interested region of time is different. 
By taking the FSIs into account at the late stage, 
we expect that we achieve a better agreement 
between the theoretical and experimental results. 
For this purpose, however, we would have to 
expand the model space defined 
with $R_{\rm box}$ and $l_{\rm max}$, 
which would severely increase the 
computational costs. 

\subsection{Case of $(l=odd)^2$ Waves}
We next discuss the case only with $(l=odd)^2$ waves 
(Fig. \ref{fig:82}). 
Even though the experimental $Q_{\rm 2p}$ and $\Gamma_{\rm 2p}$ 
are not simultaneously represented in this case 
(see Fig. \ref{fig:5-12}), 
it is still useful to discuss the density distribution in order 
to know what happens when the pairing correlation 
between the parity-plus and minus states in the core-proton 
subsystem is absent. 
In Fig. \ref{fig:82}, 
the decay density shows strong patterns as the sequential emission 
introduced in Fig. \ref{fig:80}(f): 
significant increments occur 
along the lines with $r_{\rm c-pp} \cong r_{\rm p-p}/2$ and 
$r_1 \gg r_2$ or $r_1 \ll r_2$. 
Notice that the contribution from the simultaneous 
emissions also exists, especially in the earlier time region. 
As a result, the decay state has widely spread amplitudes 
as a mixture of these emission modes. 
However, the simultaneous mode is quite minor compared with 
the full mixing case. 
Notice that the character of a true \twop-emitter exists also 
in this case: 
the core-proton resonance is located at $1.96$ MeV 
which is above $Q_{\rm 2p}1.37$ MeV. 
Even with the strong pairing attraction and the energy 
condition of the true \twop-emitter, 
the process hardly becomes the correlated emission when the 
parity-mixing is forbidden or extensively suppressed. 
On the other hand, 
the angular distribution shows exactly the symmetry form, 
and is almost invariant during the time-evolution. 
In this calculation, 
we exclude the pairing correlation 
between the parity-plus and minus states in the core-proton, 
not only at $t=0$ but also during the time-evolution. 
In other words, there are almost no FSIs to alter the shape of 
the angular distribution. 

\subsection{Without Pairing Correlation}
For a comparison with the above two cases, 
we also perform similar calculations but by 
completely neglecting the pairing correlation. 
In this case, we only consider the uncorrelated 
Hamiltonian, $h_1 + h_2$. 
Because of the absence of the non-diagonal components 
in the Hamiltonian matrix, 
it can be proved that, if the s.p. resonance is at 
an energy $\epsilon_0$ with its width $\gamma_0$, 
the \twop-resonance should be at $2 \epsilon_0$ with its 
width $2 \gamma_0$ since there are no 
couplings between the two protons. 
The \twop-wave function is expanded on the uncorrelated 
basis with a single set of angular quantum numbers. 
Namely, 
\beq
 \ket{\Phi_{(lj)}(t)} = \sum_{n_a,n_b} C_{n_a,n_b,l,j} 
                        \ket{\tilde{\Psi}_{n_a,n_b,l,j}}, 
\eeq
where $(lj)=(p_{3/2})$ for $^6$Be. 
In order to reproduce the empirical Q-value of $^6$Be, 
we inevitably modify the core-proton potential. 
We employ $V_0=-68.65$ MeV instead of that in the full mixing case 
to yield the s.p. resonance at $\epsilon_0(p_{3/2}) = 1.37/2 = 0.685$ MeV, 
with which the core-proton scattering data are not reproduced and 
the character of a true \twop-emitter disappears. 
With this potential, we get the s.p. resonance with 
a broad width: $\gamma_0(p_{3/2}) \cong 170$ keV. 
\begin{figure}[t] \begin{center}
%    \fbox{ \includegraphics[height=55truemm, scale=1, trim = 50 50 0 0]{./y_06BE_b/fgs_none/Gam_S12_nc.eps}}
    \fbox{ \includegraphics[height=55truemm, scale=1, trim = 50 50 0 0]{./FIG7121.eps}}
\caption{The same as Fig. \ref{fig:5-12} but for the case 
without the pairing correlations. } \label{fig:5-3}
\end{center} \end{figure}

The result for the \twop-decay width is shown in Fig. \ref{fig:5-3} 
and in the last column of Table \ref{tb:6473}. 
To get the saturated result, 
we somewhat need a relatively longer time-evolution than that 
in the full mixing and the $(l=odd)^2$ cases. 
Thus, in Table \ref{tb:6473}, we estimate the decay 
at $ct3000$ fm by width the result well converges 
in this case. 
We also expand the radial box to $R_{\rm box}=200$ fm in order 
to neglect the artifact due to the reflection 
in the longer time-evolution. 
After a sufficient time-evolution, 
the total decay width, $\Gamma(t)$, converges to about $340$ keV 
which is consistent to that expected from 
the s.p. resonance, $\gamma_0(p_{3/2})$. 
During the time-interval shown in Fig. \ref{fig:5-3}, 
there still remain some oscillations in $\Gamma(t)$. 
This is a characteristic behavior of the broad resonance, 
namely the oscillatory deviation from the exponential decay-rule. 
For the spin-singlet and triplet configurations, 
their contributions have exactly the ratio of $2:1$. 
This result is simply due to including 
only $(p_{3/2})^2$ partial waves, 
and is proved by calculating the coefficient $D_J$ in Eq.(\ref{eq:3iarg}). 
\begin{figure*}[tbp] \begin{center}
\begin{tabular}{c} %switch-off the auto-turning
%   \begin{minipage}{0.32\hsize} \begin{center}
%     \fbox{ \includegraphics[height=31truemm, scale=0.9, trim = 60 55 0 0]
%{./y_06BE_b/fgs_none/ds4_dt0200_nc.eps}} \\
%     \fbox{ \includegraphics[height=31truemm, scale=0.9, trim = 60 55 0 0]
%{./y_06BE_b/fgs_none/ds4_dt0400_nc.eps}} \\
%     \fbox{ \includegraphics[height=31truemm, scale=0.9, trim = 60 55 0 0]
%{./y_06BE_b/fgs_none/ds4_dt1200_nc.eps}} \\
%     \fbox{ \includegraphics[height=31truemm, scale=0.9, trim = 60 55 0 0]
%{./y_06BE_b/fgs_none/ds4_dt2000_nc.eps}} \\
%   \end{center} \end{minipage}
%%---
%   \begin{minipage}{0.32\hsize} \begin{center}
%     \fbox{ \includegraphics[height=31truemm, scale=0.9, trim = 60 55 0 0]
%{./y_06BE_b/fgs_none/ds3_dt0200_nc.eps}} \\
%     \fbox{ \includegraphics[height=31truemm, scale=0.9, trim = 60 55 0 0]
%{./y_06BE_b/fgs_none/ds3_dt0400_nc.eps}} \\
%     \fbox{ \includegraphics[height=31truemm, scale=0.9, trim = 60 55 0 0]
%{./y_06BE_b/fgs_none/ds3_dt1200_nc.eps}} \\
%     \fbox{ \includegraphics[height=31truemm, scale=0.9, trim = 60 55 0 0]
%{./y_06BE_b/fgs_none/ds3_dt2000_nc.eps}} \\
%   \end{center} \end{minipage}
%%---
%   \begin{minipage}{0.32\hsize} \begin{center}
%     \fbox{ \includegraphics[height=31truemm, scale=0.7, trim = 60 55 0 0]
%{./y_06BE_b/fgs_none/dng_dt0200_nc.eps}} \\
%     \fbox{ \includegraphics[height=31truemm, scale=0.7, trim = 60 55 0 0]
%{./y_06BE_b/fgs_none/dng_dt0400_nc.eps}} \\
%     \fbox{ \includegraphics[height=31truemm, scale=0.7, trim = 60 55 0 0]
%{./y_06BE_b/fgs_none/dng_dt1200_nc.eps}} \\
%     \fbox{ \includegraphics[height=31truemm, scale=0.7, trim = 60 55 0 0]
%{./y_06BE_b/fgs_none/dng_dt2000_nc.eps}} \\
%   \end{center} \end{minipage}
(Figure is hidden in open-print version.)
\end{tabular}
\caption{The same as Fig. \ref{fig:81} but for the case 
without the pairing correlations 
and a deeper $V_{\rm c-p}$. } \label{fig:83}
\end{center} \end{figure*}

By comparing the results with those in 
the full mixing case, where the pairing correlations are fully 
taken into account, 
we can clearly see a decisive role of the pairing correlations 
in \twop-emissions. 
Assuming the empirical Q-value, 
if we explicitly consider the pairing correlations, 
the decay width becomes narrow and agrees with 
the experimental data. 
On the other hand, in the no pairing case, 
we need a modified core-proton interaction to reproduce the 
empirical Q-value, and 
the core-proton resonance properties become inconsistent 
with the experimental data. 
Even though the Q-value is adjusted in this way, 
the calculated \twop-decay width 
is significantly overestimated in this case. 
Namely, we cannot simultaneously reproduce 
the experimental Q-value and the decay width with 
the no pairing assumption. 
If one is focused to reproduce them simultaneously, 
one may need unphysical assumptions 
for the core-proton interactions. 
In the next Section, we will present further investigations about 
this problem. 

In Fig. \ref{fig:83}, we show the density distribution 
of the decay state during the time-evolution. 
Obviously, the process is the sequential or, moreover, like 
the one-proton emission in this case. 
There is a significant increase of the density 
along the lines with $r_{\rm c-pp} \cong r_{\rm p-p}/2$ and, consistently, 
with $r_1 \gg r_2$ and $r_1 \ll r_2$ (see Fig. \ref{fig:80} again). 
On the other hand, the probability for the simultaneous and 
correlated emissions are 
negligibly small. 
We emphasize that this is quite different from that in the 
full mixing case, where the correlated emission is apparent. 
Notice that, 
with a disagreement with the experimental decay width, 
this result should not correspond to 
the \twop-emission of $^6$Be in reality. 
This situation can be interpreted as the limit where the 
core-proton resonance plays an excessively dominant role. 

\section{Role of Pairing Correlation}
In order to discuss the role of the pairing correlations
in the \twop-emission, 
we calculate the \twop-decay width for different Q-values, 
for the full-mixing and the no pairing cases. 
%Of course, this is the theoretical attempts including 
%unrealistic parameters. 
%However, it would reveal the common physics in the many-body 
%unstable states with unperturbative interactions. 

To this end, the Q-value is 
varied by modifying the parameter $V_0$ in 
the core-proton potential (Eq.(\ref{eq:cp_WS})). 
In the previous calculations, 
we used $V_0=-58.7$ and $V_0=-68.65$ MeV 
in the full mixing and the no pairing cases, respectively. 
These original values yield the empirical Q-value, $Q_{\rm 2p}=1.37$ MeV. 
In addition to these original values, 
we change the value of $V_0$ as 
$V_0 \pm 0.5$ and $V_0 \pm 1.0$ MeV. 
%Then, time-evolving calculations have been performed in these 10 cases. 
The calculated decay widths are well converged after 
a sufficient time-evolution in all the cases. 
We note that, in the full mixing case, we adopt the same pairing 
interaction as in the previous calculation. 
\begin{figure*}[htbp] \begin{center}
  \begin{tabular}{c} %switch-off the auto-turning
%\fbox{\includegraphics[width=0.5\hsize, scale=1, trim = 50 50 0 0]{./y_06BE_a/plqvc.eps}}
\fbox{\includegraphics[width=0.5\hsize, scale=1, trim = 50 50 0 0]{./FIG7141_plqvc.eps}}
  \end{tabular}
\caption{The calculated decay widths for the \twop-emission of $^{6}$Be, as a function of the Q-value. 
The Q-value is varied by modifying the core-proton potential. 
The experimental values are indicated as the point at 
$Q_{\rm 2p}=1.37$ and $\Gamma_{\rm 2p}=0.092(6)$ MeV. } \label{fig:5qp}
\end{center} \end{figure*}

In Fig. \ref{fig:5qp}, 
the decay width is plotted as a function of the Q-value. 
The decay width in each case is evaluated at 
$ct=1200$ and $3000$ fm in the full mixing and 
the no pairing cases, respectively. 
Clearly, the no pairing calculation overestimates the decay width, 
in all the region of $Q_{\rm 2p}$. 
Namely, the three-body system becomes easier to decay without 
the pairing correlations compared to the full-mixing case, 
even if we consider the same value of the total energy (Q-value). 
In other words, the pairing correlation plays an essential role in the 
meta-stable state, stabilizing it against particle emissions. 
Moreover, as we have confirmed in the previous section, 
the emission modes with and without the 
pairing correlations are essentially different to each other: 
the correlated emission is suggested if the pairing correlation 
is fully considered, 
whereas omitting it yields the sequential emission. 
Of course, this result can be associated with the character of $^6$Be 
as a true \twop-emitter. 
Consequently, we conclude that the pairing correlation must be 
treated explicitly in the meta-stable states, 
or one would miss the essential effect on the 
dynamical phenomena. 

\section{Summary of this Chapter}
We have applied the time-dependent three-body model 
to $^6$Be, which has a close 
character to the true \twop-emitter. 
The initial state of $^6$Be has the diproton correlation, 
similarly to the 
dineutron correlation in the ground state of $^6$He. 
The empirical relation between 
$Q_{\rm 2p}$ and $\Gamma_{\rm 2p}$ is well 
reproduced by fully including the pairing correlation 
(in the full mixing case). 
We have also showed that the decay process at 
its earlier stage is mainly 
the correlated emission, 
in which the two protons are emitted 
to the same direction with $S_{12}=0$, like a diproton. 
The dominance of the spin-singlet decay width is 
explained as the effect of the $(s{1/2})^2$ wave. 

We have performed the calculations by switching off a part of 
the pairing correlation in order to study 
its role in the \twop-emissions. 
First, we excluded the parity-mixing in the core-proton subsystem, 
equivalently to forbidding the diproton correlation 
in particle-bound nuclei. 
Notice that the character of a true \twop-emitter exists 
also in this assumption. 
In this case, the decay width and its spin-singlet ratio are 
remarkably underestimated compared to the full mixing case. 
The decay process has a large component of the sequential emission, 
which is quite different from the correlated emission. 
From this result, we can infer that the diproton correlation 
is essential in describing the \twop-emission. 
Second, we completely omitted the pairing correlation, and 
adjusted the mean-field between the core and a proton to 
reproduce the Q-value of the emitted two protons. 
The character of a true \twop-emitter no longer exists in this case. 
It was shown that the pairing correlation plays an essential role 
in the meta-stable states: 
omitting the pairing correlation leads to a largely overestimated 
decay width, and almost the perfect sequential emission which scarcely 
exists in the full mixing case. 

At this moment, the dependence of \twop-emissions on the initial 
diproton correlation is strongly suggested, 
but this has not yet been proved. 
Indeed, the FSIs must be taken into account at the late stage 
of the time-evolution, 
in order to probe the diproton correlation with the 
experimental observables. 
Towards this goal, we plan to expand our model space defined with 
$R_{\rm box}$ and $l_{\rm max}$, enabling us to perform the longer 
time-evolution where the FSIs play a dominant role. 
The sensitivity to the diproton correlation is translated to the 
initial-configuration dependence of observables. 
If the $Q_{\rm 2p}$ and $\Gamma_{\rm 2p}$ are by no means 
reproduced simultaneously by excluding 
the diproton correlation at $t=0$, 
we will be able to conclude the presence of the 
diproton correlation. 
Possibly, for instance, we will also infer that the observed 
signals associated with the diproton-emission \cite{09Gri_80, 09Gri_677} 
reflect the survived components originally emerged at $t=0$. 
The time-dependent method, which can distinguish the cause and 
the effect in the observables, 
will be a powerful tool in these discussions. 
We also mention that the other approaches within 
complex-energy framework 
is hard to separately discuss the early and late time regions, or 
equivalently, the cause and the effect. 
Therefore, our studies will produce a complementary point of view to 
the \twop-emission and possibly the diproton correlation. 

The expansion of the model space, however, will lead to a serious increment 
of computational costs. 
To overcome this difficulty, we will have to adopt an improved boundary 
condition which does not emerge the reflection of the wave function at 
the edge of the radial box, 
or/and more efficient bases which can reduce the dimension 
of the Hamiltonian matrix. 
Additionally, we should also concern the pairing interaction. 
The pairing interaction employed in this thesis 
should be regarded as an effective interaction, 
since it is inconsistent to the scattering 
problem of \twop in vacuum due to our modification of $v_0$. 
Within further expanded model space, it may cause the unphysical result. 
One may also introduce a three-body force, 
which works only if three particles 
are close to each other \cite{95Aoyama,01Myo,10Kiku}. 
The effect of this three-body force on decay processes is an 
important topic, in regard to whether such an interaction is 
really just a phenomenological one 
or has a physical meaning beyond the two-body force. 
\include{end}
