\documentclass[a4paper,12pt]{report}
\include{begin}

\chapter{Scattering Problem with Contact Potential} \label{Ap_Scat_Contact}
In this Chapter, we discuss the nucleon-nucleon scattering 
problem with the phenomenological contact potential, 
$V(\bir)=V_0 \delta(\bir)$. 
The contact potential can provide physical meanings only within the 
truncated space defined by the energy cutoff, $E_{\rm cut}$. 
With an arbitrary $E_{\rm cut}$ value, one can determine $V_0$ to reproduce 
the scattering character, namely the scattering length or the phase shift 
at the lower energy limit. 
The \Schr equation of this scattering problem is written as 
\beq
\left[ -\frac{\hbar^2}{2\mu} \nabla^2_{\bir} + V_0\delta(\bir) \right] \phi (\bir) 
= E \phi (\bir), 
\eeq
with the incident energy $E$. 
Here $\mu=m/2$ is the relative mass for the 
two-nucleon system where $m$ is the one-nucleon mass. 



We define the relative momentum, $k \equiv \frac{\sqrt{2\mu E}}{\hbar}$, and 
the converted potential, $v(\bir) \equiv \frac{2\mu}{\hbar^2} V_0 \delta(\bir) = v_0 \delta(\bir)$. 
Using these symbols, we can modify the \Schr equation as below. 
\beq
 \left[ \nabla^2_{\bir} + k^2 \right] \phi (\bir,k) = v(\bir) \phi (\bir,k). 
\eeq
The outgoing solution of this equation is formally represented with a Green's 
function \cite{03Fetter}: 
\begin{eqnarray}
  G^{(+)}(\bir,\bir',k) &\equiv & \label{equ_A2_G1} 
  \lim_{\eta \rightarrow 0} \int \frac{d^3\bip}{(2\pi)^3} 
  \frac{e^{i \bip \cdot (\bir-\bir')}}{p^2-k^2-i\eta} \\
  &=& \label{equ_A2_G2} \frac{1}{4\pi} \frac{e^{ik |\bir - \bir'|}}{|\bir - \bir'|}, 
\end{eqnarray}
which satisfies 
\beqa
 \left[ \nabla^2_{\bir} + k^2 \right] G^{(+)} (\bir,\bir',k) 
   &=& -\int \frac{d^3\bip}{(2\pi)^3} e^{i \bip \cdot (\bir-\bir')} \\
   &=& -\delta({\bf r}-{\bf r}'). 
\eeqa
The scattered wave function within the outgoing boundary condition, 
$\phi^{(+)}$, is formulated as 
\beq
 \label{equ_A2_phi1} \phi^{(+)}(\bir,k) = \phi^{(+)}_0(\bir,k) 
 - \int d^3\bir' G^{(+)} (\bir,\bir',k) v(\bir') \phi^{(+)}(\bir',k), 
\eeq
where $\phi^{(+)}_0(\bir,k)$ indicates the outgoing plane-wave. 
Substituting $v(\bir') = v_0 \delta(\bir')$, we can solve the 
$\phi^{(+)}(\bir',k)$ as 
\beqa
  \phi^{(+)}(\bir,k) 
  &=& \phi^{(+)}_0(\bir,k) - \int d^3\bir' G^{(+)} (\bir,\bir',k) v_0 \delta(\bir') 
      \phi^{(+)}(\bir',k), \\
  &=& \phi^{(+)}_0(\bir,k) - G^{(+)} (\bir,\bi{0},k) 
      v_0 \phi^{(+)}(\bi{0},k), \\
  &=& \phi^{(+)}_0(\bir,k) - \frac{v_0}{4\pi} 
      \phi^{(+)}(\bi{0},k) \frac{e^{ikr}}{r}. 
\eeqa
Here we used Eq.(\ref{equ_A2_G2}) to get the last formula. 
Now we can derive the well-known formula for the scattered wave, that is 
\beq
 \phi^{(+)}(\bir,k) = \phi^{(+)}_0(\bir,k) + f(k) \frac{e^{ikr}}{r}, 
\eeq
by defining the ``form factor'', $f(k)$, as 
\beq
 f(k) = - \frac{v_0}{4\pi} \phi^{(+)}(\bi{0},k). \label{eq:A2_f1} 
\eeq
It means that the scattered wave was disrupted only at $\bir=\bi{0}$, 
consistently to the infinitesimal range of the contact interaction. 

On the other hand, substituting Eq.(\ref{equ_A2_G1}) into 
Eq.(\ref{equ_A2_phi1}), we can derive an alternative formula for the scattered wave: 
\beq
  \phi^{(+)}(\bir,k) = \phi^{(+)}_0(\bir,k) - v_0 \phi^{(+)}(\bi{0},k) 
  \lim_{\eta \rightarrow 0} \int \frac{d^3\bip}{(2\pi)^3} 
  \frac{e^{i \bip \cdot \bir}}{p^2-k^2-i\eta}. 
\eeq
At $\bir = \bi{0}$, assuming the energy cutoff, 
$E_{C} \longleftrightarrow k_C$, we can apparently solve this equation. 
That is 
\beqa
 \phi^{(+)}(\bi{0},k) &=& \phi^{(+)}_0(\bi{0},k) - v_0 \phi^{(+)}(\bi{0},k) 
 \frac{1}{2\pi^2} \int_0^{k_C} dp \frac{p^2}{p^2-k^2} \\
 &=& 1 - \frac{v_0}{2\pi^2} \phi^{(+)}(\bi{0},k) 
 \left[ k_C+\frac{k}{2} \ln \left| \frac{k_C-k}{k_C+k} \right| \right]. 
\eeqa
Thus we get 
\beq
 \phi^{(+)}(\bi{0},k) = \left( 1 + \frac{v_0}{2\pi^2} 
 \left[ k_C+\frac{k}{2} \ln \left| \frac{k_C-k}{k_C+k} \right| \right] 
 \right)^{-1} \label{equ_A2_phi2}, 
\eeq
which is the complementary equation to Eq.(\ref{eq:A2_f1}). 

\section{Low Energy Limit}
In the following, we consider the s-wave at the 
low energy limit ($k \rightarrow 0$). 
As well known, the form factor can be written as 
\beq
 f_s(k) = \frac{1}{k} e^{i\delta_s} \sin \delta_s 
\eeq
where $\delta_s $ is the phase shift. 
It leads to an asymptotic formula below: 
\beqa
 |f_s(k)|^2 &=& \frac{\sin^2 \delta_s}{k^2} = \frac{1}{k^2 + k^2\cot^2 \delta_s} \\
 \Longrightarrow k\cot \delta_s &=& \left( \frac{1}{|f_s(k)|^2 }-k^2 \right)^{1/2} 
 \simeq \frac{1}{|f_s(k)|} \left( 1 - \frac{k^2|f_s(k)|^2}{2} \right). \label{equ_A2_kcotd1} 
\eeqa
From Eqs.(\ref{eq:A2_f1}), (\ref{equ_A2_phi2}) and (\ref{equ_A2_kcotd1}), 
the phase shift of the s-wave can be approximated as 
\beq
  k\cot \delta_s \simeq -\frac{4\pi}{v_0} 
  \left( 1 + \frac{v_0}{2\pi^2} \left[ k_C+\frac{k}{2} \ln \left| \frac{k_C-k}{k_C+k} \right| \right] 
  \right). \label{equ_A2_kcotd3} 
\eeq
Note that the logarithmic term is expanded as a polynomial of $k$: 
\beq
 \ln \left| \frac{k-k_C}{k+k_C} \right| \simeq 
 -2\frac{k}{k_C} + \mathcal{O}\left( \left( \frac{k}{k_C} \right)^3 \right), 
\eeq
where there are no terms on the order of $k^0$. 
On the other hand, 
there is an well-known empirical formula for $k\cot \delta_s$ at $k \rightarrow 0$, 
such as 
\beq
 k\cot \delta_s \simeq - \frac{1}{a_{\rm nn}} + \frac{r_{\rm nn}}{2} k^2. \label{equ_A2_kcotd2} 
\eeq
Here $a_{\rm nn}$ is the nucleon-nucleon scattering length whose empirical 
value is $-18.5$ fm, whereas the $r_{\rm nn}$ is the effective range. 
Comparing the leading terms in Eqs. (\ref{equ_A2_kcotd3}) and 
(\ref{equ_A2_kcotd2}), the strength $V_0$ can be defined within 
the energy-cutoff $k_C$ to reproduce the scattering length. 
That is 
\beqa
 -\frac{4\pi}{v_0} \left(1 + \frac{v_0 k_C}{2\pi^2} \right) 
 &=& -\frac{1}{a_{\rm nn}} \nonumber \\
 \Longrightarrow v_0 
 &=& 4\pi \left( \frac{1}{a_{\rm nn}} - \frac{2}{\pi} k_C \right)^{-1} 
     = 4\pi \left( \frac{\pi a_{\rm nn}}{\pi -2 a_{\rm nn} k_C} \right). 
\eeqa
Remembering $v_0/2 = \mu V_0/\hbar^2$, consequently we get the 
fitting formula for the contact potential: 
\beqa
  V_0 &=& \label{equ_A2_V03} \frac{\hbar^2}{2\mu} 
          \left( \frac{4\pi^2 a_{\rm nn}}{\pi -2 a_{\rm nn} k_C} \right). 
\eeqa
It should be noted that Eq. (\ref{equ_A2_V03}) is valid at $k \rightarrow 0$ limit. 
However, in practical cases, the parameter defined with 
$a_{\rm nn} = -18.5$ fm may be too strong especially for the valence nucleons. 
Thus the lower value, {\it e.g.} $a_{\rm nn}=15.0$ fm, is also used to 
prepare the appropriate paring attraction as often as the original value. 
In this thesis, we confirmed that our conclusions do not change even if 
we employ $a_{\rm nn} = 15.0$ fm instead of $a_{\rm nn} = 18.5$ fm. 
%\beq
% \frac{r_{\rm nn}}{2} k^2 = \frac{2}{\pi k_C} k^2 
%\eeq

\include{end}
