\documentclass[a4paper,12pt]{book}
\include{begin}

\chapter{Time-Dependent Method} \label{Ch_TDM}
We extend our three-body model to the time-dependent (TD) one, 
in order to treat a meta-stable state 
of the three-body system and its decay 
via the emission of the two valence particles. 
General formalism of the time-dependent method for 
quantum meta-stable phenomena is summarized in Appendix \ref{Ap_TDM}. 
Therefore, in the following, we mainly describe how to apply 
the time-dependent method to the 
two-proton (\twop-) decays and emissions. 


Within our TD three-body model, the two-proton decays and emissions 
can be described as 
dynamical processes driven by the static Hamiltonian, $H_{\rm 3b}$. 
Furthermore, in many cases, a proton in the three-body system does not 
have a sufficient energy to get over the potential barriers from 
other particle(s). 
Thus, the quantum tunneling effect plays an 
essential role in these processes. 
We emphasize that this tunneling effect can be naturally taken 
into account by solving 
the time-dependent Schr\"{o}dinger equation. 
Our formalism will not assume whether the 
two protons are either emitted sequentially or simultaneously. 
In other words, our method includes all the possible configurations in the 
emission process on equal footing. 

\section{Discretized Continuum Space}
Assuming the \twop-emission as a time-dependent process, 
we carry out the time-evolving calculations for 
the three-body system. 
%First we have to mention that, even if the time-evolution can be correctly 
%performed, it does not guarantee the validity of our calculations: 
First we have to prepare the initial state, 
$\ket{\Phi (0)}$, defined consistently to 
the realistic emissions in order for our calculations to be valid. 
Phenomenologically, the initial state, $\ket{\Phi (0)}$, should reflect 
the configuration of two protons confined inside 
the potential barrier. 
The \twop-density for such initial state should 
have almost no amplitudes 
outside the potential barrier. 
For this purpose, we employ ``confining potential method'' 
in this thesis. 
A concrete form of the confining potential will be given 
in the next Chapter, because the definition of it 
is critical to our final results. 
%However, we need further advanced frameworks to fully include these effects, 
%{\it e.g.} $\alpha p + p + n$ model for the one-neutron knockout reaction from $^7$Be. 
%Instead of considering these complex affairs rigorously, 
%we use an altrenative way to arrange the initial configuration. 

%Because of the numerical calculations, we have to discretize the 
%continuum energy space. 
Before we do this definition, however, we would like to introduce 
some formulas which will be used in the actual calculations 
of the \twop-emissions. 
In performing the time-dependent calculations, we discretize 
the continuum energy space. 
Let us expand the initial state as a confined wave-packet 
on the discretized continuum eigen-space of 
the Hamiltonian, namely 
\beq
 \ket{\Phi (0)} = \sum_N F_N (0) \ket{E_N}, \label{eq:expand_E} 
\eeq
where $H_{\rm 3b} \ket{E_N} = E_N \ket{E_N}$. 
We note that the more general formalism of 
the time-dependent calculation without 
the continuum-descretization is summarized in Appendix \ref{Ap_TDM}. 
That formalism is, however, not useful for 
the numerical calculations. 

In Eq.(\ref{eq:expand_E}), one can obtain the discretized 
eigen-states $\ket{E_N}$ by, {\it e.g.} solving 
the three-body Hamiltonian within a large box. 
We stress that all the eigen-energies are real numbers: 
$E_N \in \mathbb{R}$. 
Namely, we consider within the pure Hermit framework, in contrast to other 
non-Hermite frameworks frequently used for quantum meta-stable 
processes \cite{10Deli_text,28Gamov_01,28Gamov_02,71Aguilar,71Balslev,06Aoyama,09Gri_40,09Gri_80}. 
Each eigenstate, $\ket{E_N}$, is expanded on 
the anti-symmetrized uncorrelated basis 
given by Eq.(\ref{eq:uncorrebasis}). 
Namely, 
\beq
 \ket{E_N} = \sum_K U_{NK} \ket{\tilde{\Psi}_K}, 
\eeq
where we use simplified labels as 
$K \equiv \left\{ n_a,l_a,j_a,n_b,l_b,j_b \right\}$. 
Note that the expansion coefficients, $\left\{ U_{NK} \right\}$, 
are obtained by diagonalizing the Hamiltonian matrix, 
$\left\{ \Braket{\tilde{\Psi}_{K'}| H_{\rm 3b} |\tilde{\Psi}_K} \right\}$, 
and are independent of time, $t$. 

The state at an arbitrary time, $t$, can be expanded on 
the uncorrelated basis, or equivalently, 
on the eigen-states of the Hamiltonian, $H_{\rm 3b}$. 
Those are represented as 
\beqa
 \ket{\Phi (t)} 
 &=& \exp \left[ -it \frac{H_{\rm 3b}}{\hbar} \right] \ket{\Phi (0)} \nonumber \\
 &=& \sum_{N} F_N (t) \ket{E_N}, \label{eq:ex_E} \\
 &=& \sum_{K} C_K (t) \ket{\tilde{\Psi}_K}, 
 \label{eq:tdse}
\eeqa
with the expansion coefficients given by 
\beqa
 F_N (t) &=& e^{-itE_N/\hbar} F_N(0) \label{eq:excf_E} , \\
 C_K (t) &=& \sum_N F_N (t) U_{NK} \label{eq:excf_UNCB} . 
\eeqa
The Q-value of the \twop-emission is given as the expectation 
value of the total Hamiltonian. 
From Eq.(\ref{eq:expand_E}) and (\ref{eq:excf_E}), 
it is shown that the Q-value is conserved during the time-evolution, 
that is, 
\beq
 Q_{2p} 
 \equiv \Braket{\Phi (t) | H_{\rm 3b} | \Phi (t)} 
 = \sum_N E_N \abs{F_N(t)}^2 
 = \sum_N E_N \abs{F_N(0)}^2. \label{eq:Q_conserve} 
\eeq
We also note that the norm of the \twop-state is normalized 
at any time: 
\beq
 \Braket{\Phi(0) | \Phi(0)} = \Braket{\Phi(t) | \Phi(t)} = 
 \sum_N \abs{F_N(0)}^2 = 1. 
\eeq

\section{Decay State and Width}
In order to extract the information on the emission, 
it is useful to  define the ``{\it decay state}'', $\ket{\Phi_d (t)}$, 
by projecting out to the initial state \cite{08Bertulani}. 
That is, 
\beq
 \ket{\Phi_d (t)} \equiv 
 \ket{\Phi (t)} - \beta(t) \ket{\Phi (0)}, \label{eq:decaystate}
\eeq
where $\beta (t) = \Braket{\Phi (0) | \Phi (t)}$ is the 
overlap coefficient. 
Because we prepare the initial state which has almost no amplitude 
outside the potential barrier, 
the decay state is mostly an outgoing wave, and its density has 
non-zero values almost only outside the potential barrier. 
The decay probability is given by its norm, 
\beq
 N_d (t) \label{eq:603DComp}
 = \Braket{\Phi_d (t) | \Phi_d (t)} = 1 - \abs{\beta (t)}^2 . 
\eeq
Notice that $N_d(0) = 0$ since $\beta (0) = 1$. 
Noticing that the quantity $\abs{\beta (t)}^2$ is identical to 
the survival probability for the initial state, 
the decay width can be defined from 
$N_d (t)$ \cite{94Serot,98Talou,99Talou_60,00Talou}. 
That is 
\beqa
 \Gamma (t) 
 &\equiv & -\hbar \frac{d}{dt} \ln \left[ 1-N_d(t) \right] \\
 &=& \frac{\hbar }{1-N_d(t)} \frac{d}{dt} N_d(t). \label{eq:width} 
\eeqa
If the time-evolution converges to the well-known exponential 
decay process, such that 
\beq
 \left[ 1-N_d(t) \right] = e^{-t/\tau}, 
\eeq
then $\Gamma(t)$ obviously corresponds to the lifetime, 
$\Gamma = \hbar / \tau$. 
This is the situation in which the energy spectrum, 
defined with $\{ \abs{F_N(0)}^2 \}$ in the discrete continuum space, 
is well approximated by the Breit-Wigner 
distribution \cite{89Kuku, 89Bohm, 09Konishi}. 
For the relation between the exponential decay-rule and 
the Breit-Wigner distribution, 
see also Appendix \ref{Ap_TDM}. 

It is also helpful to define the ``{\it partial decay width}'' 
to discuss the tunneling properties. 
The purpose is to decompose the total decay width into the 
widths for partial components labeled by $s$, such as 
\beq
 \Gamma (t) = \sum_s \Gamma_s (t). 
\eeq
The corresponding expansion for the decay state on the 
partial components, $\{ \ket{s} \}$, can be defined as 
\beq
 \ket{\Phi_d (t)} = \sum_s a_s(t) \ket{s}, 
 \label{eq:part_expand} 
\eeq
where all the partial components are orthogonal 
to each other: $\Braket{s' | s} = \delta_{s's}$. 
Using Eq.(\ref{eq:width}), we can write 
\beq
 \Gamma_s (t) \equiv \frac{\hbar }{1-N_d(t)} \frac{d}{dt} N_{d,s}(t). 
 \label{eq:pwidth} 
\eeq
where $N_{d,s} = \abs{a_s(t)}^2$. 

We note that Eqs.(\ref{eq:part_expand}) and (\ref{eq:pwidth}) 
can be defined generally for any choice of the partial components 
as long as they are orthogonal. 
For example, one can employ the components which have different 
energies, angular momenta or spin-parities. 
In the next Chapter, we will apply these formulas in order to 
calculate the spin-singlet and triplet widths in the \twop-emission of $^6$Be. 
We also emphasize that our formulas themselves in this subsection 
are not limited to the three-body framework, but can be extended to 
further complex systems. 

\section{Time-Dependent Density Distribution}
By integrating over the spin-variables, 
similarly to Chapters \ref{Ch_3body} and \ref{Ch_Results1}, 
we can obtain the spatial density distribution, 
parametrized by the radial 
distances $\{ r_1,r_2 \}$ and the opening angle $\theta_{12}$ 
from the symmetry. 
It is formulated as 
\beqa
 && \bar{\rho}_{2p} (t;r_1,r_2,\theta_{12}) = 8\pi^2 r_1^2 r_2^2 \sin \theta_{12} \rho_{2p} (t;r_1,r_2,\theta_{12}), \\
 && \rho_{2p} (t;r_1,r_2,\theta_{12}) = \abs{\Phi (t;r_1,r_2,\theta_{12})}^2, 
\eeqa
where $\bar{\rho}_{2p}$ is normalized at any time as 
\beq
 \int_0^{R_{\rm box}} dr_1 \int_0^{R_{\rm box}} dr_2 \int_{0}^{\pi} d\theta_{12} 
 \bar{\rho}_{2p}(t) = 1. 
\eeq
However, for the emission process, 
it is often more useful to discuss the density of the decay state 
defined by Eq.(\ref{eq:decaystate}). 
This is given by 
\beq
 \bar{\rho}_d (t;r_1,r_2,\theta_{12}) 
 = 8\pi^2 r_1^2 r_2^2 \sin \theta_{12} 
   \abs{\Phi_d (t;r_1,r_2,\theta_{12})}^2. 
\eeq
Because of the definition of the decay state, 
this quantity represents the components which 
have penetrated the potential barrier. 
For a presentation purpose, 
we often renormalize the $\rho_d(t)$ so that 
its integration become unity at each time: 
\beq
 \bar{\rho}_d (t) \longrightarrow \frac{\bar{\rho}_d (t)}{N_d (t)}, 
\eeq
where $N_d (t)$ is the decay probability given by Eq.(\ref{eq:603DComp}). 

In our discussions after this Chapter, 
we make full use of this decay density, in order to investigate, 
{\it e.g.} the effect of diproton correlations, 
the competition between the true and the sequential emissions, 
the spatial distributions of two protons and so on. 
It will provide a great advantage to intuitively understand 
the \twop-emissions and its relation to the diproton correlation. 

\section{Test of Time-Dependent Method: One-Proton Emission}
To check that the time-dependent method can correctly describe the 
decay of a quantum meta-stable state, 
we apply it to a problem of the one-proton ($1p$-) emission. 
This is a two-body problem of a core nucleus and a valence proton, 
with a spherical potential, $V_{lj}(r)$. 
Thus, the Hamiltonian is given by 
\beq
 h = \frac{\bip^2}{2\mu} + V_{lj}(r). \label{eq:sph61}
\eeq
Taking the relative wave function as 
$\Psi_{ljm}(\bir,\bis) = U_{lj}(r)/r \cdot \mathcal{Y}_{ljm}(\ubir,\bis)$, 
the Schr\"{o}dinger equation is given as 
\beq
  \left[ -\frac{\hbar^2}{2\mu}\left\{ \frac{d^2}{dr^2} 
         -\frac{l(l+1)}{r^2} \right\} + V_{lj}(r) - E \right] U_{lj} (r) 
  = 0, %$U_{lj}(r,E) \equiv rR_{lj}(r)$
\eeq
with the relative energy larger than the threshold of 
the $1p$-emission in this system: 
\beq
 E > \lim_{r \rightarrow \infty} V_{lj}(r) \equiv 0. 
\eeq
We adopt the Woods-Saxon plus the Coulomb 
potential of a uniform-charged sphere for $V_{lj}(r)$. 
That is, 
\beqa
  V_{lj}(r) &=& V_{lj,{\rm Nucl.}}(r) + V_{{\rm Coul.}}(r) \\
  &=& \left[ V_0 + V_{lj} r_0^2 (\bi{\ell} \cdot \bi{s}) 
      \frac{1}{r} \frac{d}{dr} \right] f(r) + V_{{\rm Coul.}}(r), \label{eq:vorig64}
\eeqa
with
\beq
    f(r) = \frac{1}{ 1 + \exp \left( \frac{r-R_c}{a_c} \right)}, 
\eeq
and 
\beq
 V_{\rm Coul.} (r) 
 = \left\{ \begin{array}{cc} 
   \frac{Z_{\rm c} e^2}{4\pi \epsilon_0} \frac{1}{r} & (r > R_c), \\
   \frac{Z_{\rm c} e^2}{4\pi \epsilon_0} \frac{1}{2R_c} \left( 3 - \frac{r^2}{R_c^2} \right) & (r \leq R_c). 
   \end{array} \right. 
\eeq
The parameters are taken to be $A_c=4$, $Z_c=2$, 
$r_0=1.12$ fm, $R_c=r_0\cdot A_c^{1/3}$ fm, 
$a_c=0.755$ fm, $V_0 =-58.7$ MeV, and 
$V_{lj} r_0^2 = 51.68$ MeV$\cdot {\rm fm}^2$. 
%Although these parameters have no corresponding system in reality, 
We will use the similar parameters 
to study the $^{6}$Be nucleus in the next Chapter. 
For the angular channel, we only discuss the $(p_{3/2})$-channel. 
As we will show, this channel has a resonant state within 
the present core-proton potential. 
\begin{figure}[tb] \begin{center}
\fbox{\includegraphics[width=0.5\hsize, scale=1, trim = 50 50 0 0]{./FIG6011.eps}}%\includegraphics[width=0.6\hsize, clip, trim = 10 0 0 0]{./y_05LI_3t/05LI_3t_dps.eps}
\caption{The calculated energy-derivative of the phase-shift, $\delta_{p_{3/2}}$(E). 
For the fitting purpose, the pure Breit-Wigner distribution, 
$L(E)=\frac{\Gamma_0/2}{\Gamma_0^2/4 + (E_0-E)^2}$ is assumed. } \label{fig:05LI_3t_1}
\end{center} \end{figure}

By calculating and fitting the phase-shift according to 
the formalism in Appendix \ref{Ap_Scat_2body}, 
we obtain the resonant energy and width as 
$E_0=308$ keV and $\Gamma_0=41$ keV, respectively. 
These values are obtained using 
the fitting function as the pure Breit-Wigner distribution: 
\beq
 \frac{d\delta_{lj}(E)}{dE} \equiv 
 \frac{\Gamma_0/2}{\Gamma_0^2/4 + (E_0-E)^2}, \label{eq:psch64}
\eeq
for $l=1$ and $j=3/2$, based on the two-body scattering formalism. 
The calculated result and its fit are presented in Fig. \ref{fig:05LI_3t_1}. 

On the other hand, we can calculate the resonant energy and width 
by another method, namely by the time-dependent method. 
With this method, we first have to prepare 
the initial state for the $1p$-emission. 
For this purpose, we adopt the ``confining potential'' method. 
That is, we assume the modified Hamiltonian, 
\beq
 h_{(p_{3/2})}^{conf} = \frac{\bip^2}{2\mu} + V_{(p_{3/2})}^{conf}(r), 
\eeq
with the confining potential for the $(p_{3/2})$-channel as 
\beq
  V_{(p_{3/2})}^{conf}(r) \label{eq:vconf64} 
 = \left\{ \begin{array}{cc} 
            V_{(p_{3/2})}(r) & (r \leq R_b), \\
            V_{(p_{3/2})}(R_b) & (r > R_b), \end{array} \right. 
\eeq
with $R_b = 8.2$ fm. 
The original and confining potentials 
are shown in Fig. \ref{fig:05LI_3t_2}(b). 
The initial state, $\Phi(t=0;\bir,\bis)$, is defined as an 
eigen-state of this modified Hamiltonian, namely, 
\beq
 h_{(p_{3/2})}^{conf} \ket{\Phi(0)} = E^{conf} \ket{\Phi(0)}. 
\eeq
On the other hand, we also consider the eigen-states of the original 
Hamiltonian as 
\beq
 h_{(p_{3/2})} \ket{E_N} = E_N \ket{E_N}, 
\eeq
with the discretized continuum energies, $\{ E_N \}$. 
In order to discretize the continuum, we assume the radial box of 
$R_{\rm box}=120$ fm in this case, and 
impose a boundary condition that 
the wave function satisfies $U_{(p_{3/2})}(r=R_{\rm box}) = 0$. 
The energy cutoff is employed as $E_{\rm cut} = 40$ MeV. 
As a result, we have $N_{\rm max}=47$ bases. 
\begin{figure*}[tb] \begin{center}
  \begin{tabular}{c} %switch-off the auto-turning
     \begin{minipage}{0.48\hsize} \begin{center}
%        (a) \\ \fbox{ \includegraphics[height=50truemm, clip, trim = 10 0 0 0]{./y_05LI_3t/05LI_3t_g.eps}}
        (a) \\ \fbox{ \includegraphics[height=44truemm, scale=1, trim = 50 50 0 0]{./FIG6021_g.eps}}
     \end{center} \end{minipage}
     \begin{minipage}{0.48\hsize} \begin{center}
%        (b) \\ \fbox{ \includegraphics[height=50truemm, clip, trim = 10 0 0 0]{./y_05LI_3t/05LI_3t_wf.eps}}
        (b) \\ \fbox{ \includegraphics[height=44truemm, scale=1, trim = 50 50 0 0]{./FIG6022_wf.eps}}
     \end{center} \end{minipage}
  \end{tabular}
\caption{(Left panel) The calculated decay width within the confining potential, Eq.(\ref{eq:vconf64}). 
(Right panel) The original and confining core-proton potentials given by 
Eqs.(\ref{eq:vorig64}) and (\ref{eq:vconf64}). 
The Q-value, $E_0=0.339$ MeV, and the density distributions at 
$ct=0$ and $1200$ fm are also shown. } \label{fig:05LI_3t_2}
\end{center} \end{figure*}

By diagonalizing the $47\times 47$ matrix, 
$\{ \Braket{E_M | h_{(p_{3/2})}^{conf} | E_N} \}$, 
the initial state can be represented as the expansion on the 
original eigen-states. 
That is, 
\beq
 \ket{\Phi(0)} = \sum_N F_N(0) \ket{E_N}. 
\eeq
This equation has an identical form to Eq.(\ref{eq:expand_E}). 
Thus, by performing the calculations according to Eqs.(\ref{eq:ex_E}), 
(\ref{eq:Q_conserve}) and (\ref{eq:width}), 
we can determine 
the Q-value, $E_0 = \sum_N E_N \abs{F_N(0)}^2$ and 
the decay width, $\Gamma(t)$. 
We obtain $E_0=339$ keV. 
The calculated decay width is shown in Figure \ref{fig:05LI_3t_2}(a). 
After a sufficient time-evolution, the decay width well converges to a 
constant value, corresponding to the exponential decay-rule. 
We have obtained $\Gamma_0=37$ keV at $ct=1200$ fm, when the 
decay width is sufficiently converged. 
These results are consistent to those obtained by calculating the 
two-body scattering phase-shift, 
justifying the time-dependent approach. 
%To this end, we have confirmed that the time-dependent method can 
%well describe the core-proton meta-stable state, as well as the 
%two-body scattering formalism. 
Furthermore, 
in Figure \ref{fig:05LI_3t_2}, we also show the density distribution, 
$\abs{U_{(p_{3/2})}(r)}^2$ at $ct = 0$ and $1200$ fm. 
Although these two functions have almost the same form, 
the later one shows a larger amplitude outside the potential barrier. 
This indicates the penetration of the valence proton. 
Consequently, we can observe the time-development of the emitted 
particle(s) based on this method. 
This will be a great advantage providing an intuitive view 
of the decay process, 
when we will apply this method to 
the two-proton emissions, in order to discuss the effect of 
the diproton correlation. 
\include{end}
