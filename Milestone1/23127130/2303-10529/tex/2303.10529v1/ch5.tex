\documentclass[a4paper,12pt]{book}
\include{begin}

\chapter{Review of Two-Proton Decay and Emission} \label{Ch_5}
The two-proton decay and emission are characteristic decay-modes of 
nuclei beyond the proton-dripline. 
We review in this Chapter the theoretical and experimental studies 
of these phenomena, with their relevant topics. 

In nuclear physics, 
there have been five major radioactive processes in which 
one or several nucleons are emitted from the parent nuclei 
\footnote{There are also known radioactive processes by 
the weak or the electro-magnetic interactions. 
In this thesis, however, we push aside these topics. }. 
Those are (i) alpha decay, 
(ii) one-proton and one-neutron decays, 
(iii) two-proton and two-neutron decays, 
(iv) heavier cluster decay, and 
(v) nuclear fission. 
All these processes belong to the quantum meta-stable phenomena by 
the nuclear interaction. 

Needless to say, the alpha-decay is one of the most famous 
nuclear radioactive processes. 
In many standard textbooks of nuclear physics, 
this problems is discussed as an tunneling problem of a point-like 
alpha-particle. 
However, it is also known that the emitted alpha-particle is 
a composite system of four nucleons. 
Therefore, to describe the alpha-decay properly, 
one would need a microscopic framework including 
many-body effects. 
There have been several theoretical studies based this 
consideration \cite{70Sasa,79Tono,94Varga,12Betan}. 
However, mainly because of a difficulty to handle with many-body 
correlations, there have been no quantitatively successful works yet. 
We also note that the physics of meta-stable states with intrinsic 
degrees of freedom, or of many particles 
are one of the major subjects in modern physics. 
They occupy the essential positions not only in nuclear physics, 
but also in molecular, condensed matter and astro-nuclear physics. 
Famous examples include, e.g. nuclear fissions and fusions, 
resonances of cold atoms and Jossefson effects. 
A unified study of multi-fermion meta-stable systems in different 
scales might be useful in gaining a deeper 
understanding of our world. 

In the following, we mainly focus on the \twop-decay and emission, 
whereas other processes will be briefly or never mentioned. 
In these processes, two protons are emitted simultaneously or 
sequentially from the parent nucleus with an even-number of protons. 
Because of the remarkable developments in the experimental 
techniques \cite{08Blank,09Blank,09Gri_40,12Pfu}, for recent about 10 years, 
two-proton (\twop-) emitters have been one of the main topics in radioactive 
nuclear physics, and knowledge about \twop-emissions and 
radioactive decays have been accumulated. 
Recently, furthermore, 
the two-proton (\twop-) emission has attracted 
much attention as an useful tool to probe the diproton correlation. 
We detail this history in the following. 

\section{History of Two-Proton Decay and Emission}
Comparing with alpha-decays and fissions, 
the two-proton emissions and decays are much simpler. 
Moreover, those are the dynamical phenomena, including 
the pairing correlations which are unique in multi-fermion systems. 
By studying \twop-decays, it has been expected to provide 
the benchmark for the quantum meta-stability of many fermions. 
\begin{figure*}[t] \begin{center}
  \begin{tabular}{c} %switch-off the auto-turning
     \begin{minipage}{0.48\hsize} \begin{center}
        (a) true \twop-emitter \\
        \fbox{ \includegraphics[width=0.9\hsize, clip, trim = 0 0 0 0]{FIG5011.eps}}%{./y_etc/g_true_2p.eps}}
     \end{center} \end{minipage}
     \begin{minipage}{0.48\hsize} \begin{center}
        (b) sequential \twop-emitter \\
        \fbox{ \includegraphics[width=0.9\hsize, clip, trim = 0 0 0 0]{FIG5012.eps}}%{./y_etc/g_cascade_2p.eps}}
     \end{center} \end{minipage}
  \end{tabular}
\caption{Schematic figures for the energy conditions of the true \twop-emitters (the left panel) and 
the sequential \twop-emitters (the right panel). 
The parent nucleus is indicated as $^AZ$. 
Each level is measured from the ground state of 
the daughter nucleus, $^{A-2}(Z-2)$. 
It is assumed that there are no bound $1p$- and $2p$-states in the 
$^{A-1}(Z-1)$ and $^AZ$ nuclei, respectively. 
The symbol $\Delta_{\rm pair}$ means the pairing energy gap due to the 
pairing attraction. 
The decay widths are not considered in these schematic figures. } \label{fig:51}
\end{center} \end{figure*}

\subsection{Early Studies: before the 21th Century}
The first prediction of \twop-decays was done by 
V. I. Goldansky \cite{60Gold,61Gold}. 
In his theoretical study, 
he considered two different, 
simple situations for \twop-emissions: ``true'' and ``sequential'' 
\twop-emitters. 
In Fig. \ref{fig:51}, we schematically describe these two situations, 
as the energy conditions for a parent nucleus 
with respect to the nuclei after $1p$- and \twop-emissions. 

In the true \twop-emitter, 
the energy pf the \twop-resonance of a parent nucleus 
is lower than the $1p$-resonance of its isotone after the $1p$-emission. 
In this situation indicated in Fig. \ref{fig:51}(a), 
only the simultaneous \twop-emission is allowed, whereas the emission 
of single proton is forbidden. 
In the pure mean-field approximation, 
this situation is never realized as long as 
the intermediate nucleus, $^{A-1}(Z-1)$, 
has no bound single-proton states: 
the \twop-resonance has the energy of $2\epsilon_r$, 
where the $\epsilon_r>0$ is the single-proton resonance energy in the 
intermediate nucleus. 
Thus, in order to realize the true \twop-emission, 
the pairing energy gap must be large enough to pull down 
the \twop-resonance under the single-proton resonance. 
This energy gap is, of course, caused by the pairing interaction. 
Based on the quasi-classical formulas, Goldansky showed that, 
in the true \twop-emission, the decay width is sharply reduced 
from that of common binary decays \cite{60Gold,61Gold}. 

In Goldansky's pioneer works, he proposed two types of 
the true \twop-emissions. 
The first one is ``diproton'' emission: 
if the pairing attraction is much stronger than the 
Coulomb repulsion, 
two protons are emitted almost as a diproton. 
He showed that, in the case of two protons restricted to the relative 
$(s_{1/2})$-orbit from the core, the penetrability is identical to that 
of a diproton with the total spin $S=0$. 
This observation has been a basic idea of the diproton emission. 
On the other hand, he also proposed ``direct'' emission, 
where the pairing correlation is rather weak. 
In the direct \twop-emission, the observables are mainly governed by the 
core-proton interactions, where the disruption by the pairing 
interaction can be negligible 
(Notice that even if the pairing energy gap is large enough, 
the effect of pairing correlations may 
be minor for the emitted particles). 
Consequently, the diproton and direct decays or emissions 
correspond to the extreme conditions with relatively 
strong and weak pairing correlations, respectively. 
The binary channel of the proton-proton and the core-proton plays 
a dominant role in the diproton and the direct emissions, respectively. 

However, it is important to notice that 
the actual true \twop-emissions are not so simple that 
they usually show an intermediate character between the diproton and 
direct emissions. 
It is often necessary to deal with three particles without 
any discrimination. 
This process is referred to as ``democratic'' emission 
\cite{89Boch,09Gri_40,12Ego}, 
where there are no autocrat binary channels. 
Namely, the actual true \twop-emissions are essentially the three-body 
problems, and they cannot be approximated as the product 
of partial two-body problems. 

In the sequential \twop-emitter indicated in Fig. \ref{fig:51}(b), 
on the other hand, 
the $1p$-emission becomes dominant in a parent nucleus. 
This situation can be understood by the mean-field plus a 
small pairing gap, which is insufficient to realize the true 
\twop-emission. 
Thus, the decay process to the final daughter nucleus 
occurs as a sequence of two $1p$-emissions. 
According to this picture, sometimes the process is 
also called ``cascade'' decay. 
In the sequential \twop-emissions, of course, the pairing correlations 
should be minor. 
Observables are almost explained only with the core-proton interactions. 
The total penetrability should be well approximated as 
the bi-product of the penetrabilities for the first and second protons. 
However, one should also notice that the direct \twop-emission 
can takes place also in this situation, if the first and second 
$1p$-emissions occur at the same time. 

After Goldanski's works, these three types of $\twop$-emissions 
have been the major assumptions in almost all the theoretical 
works \cite{64Gali,91Brown,96Nazare,97Woods}. 
We also note that there were less theoretical 
works before 2000. 
The reason for this poor crop may be a shortage of observed 
examples of \twop-emitters, which can be analyzed within the 
theoretical models in those days. 

In the experiments, 
two categories of $\twop$-emissions, namely 
the \twop-emissions from the ground state nuclei and the beta-delayed 
\twop-emissions, have been known. 
The first one includes the $^{6}$Be nucleus, which is the 
simplest \twop-emitter interpreted as the $\alpha$-particle 
with two valence protons 
\cite{66Whaling,77Gree,88Ajzen,84Boch,85Boch_377,87Boch,88Boch,89Boch,92Boch}. 
Its decay width, $\Gamma_{\rm 2p} \cong 92$ keV, was measured 
more than 30 years ago. 
Especially, the works performed by Bochkarev {\it et.al.} have 
presented the benchmark results 
\cite{84Boch,85Boch_377,87Boch,88Boch,89Boch,92Boch}. 
In their results, it was already suggested that the assumption of the 
diproton emission is not valid: 
it leads to an unrealistic property that 
the relative proton-proton kinetic energy is extremely small. 
Thus, the necessity of considering the democratic emission has been 
extensively discussed. 
It is worthwhile to note that the $\alpha$+N+N three-body decay 
from the $2^+$ excited states of $^{6}$He and $^{6}$Li has been 
also observed as well as the $^{6}$Be \cite{85Boch_374,87Boch,94Boch}. 
For these 2N-emissions, the isobaric symmetry in the meta-stable states 
has been discussed. 
This is still an open problem even at present. 

We also mention that similar three-body resonances have been known 
in $^{12}$O \cite{78KeKe,95Kryger} and 
$^{16}$Ne nuclei \cite{83Wood}. 
Compared with $^{6}$Be, these nuclei have comparable or larger 
decay widths of the order of $100$ keV. 
Phenomenologically, investigations of these nuclei is important 
for the universal understanding of \twop-emissions along the 
proton-dripline. 
Nevertheless, they have been less studied in the past. 
Recently, several improvements have been done both 
in the theoretical and experimental studies of these 
nuclei \cite{02Gri,09Gri_40,08Muk,10Muk,12Jager}. 
However, at the same time, 
a new and serious problem has been realized 
that the decay widths of these nuclides are 
too broad to be reproduced 
within the simple three-body model \cite{02Gri}. 
This is still an open problem at present. 
We will mention again this discrepancy in Chapter \ref{Ch_Results3}, 
where the $^{16}$Ne nucleus will be treated within our model. 
\begin{figure}[tb] \begin{center}
%\fbox{\includegraphics[width = 0.6\hsize]{./y_quoted/1984Cable_9.eps}}
(Figure is hidden in open-print version.)
\caption{Figure 14 in Ref.\cite{84Cable}. 
The decay scheme of $^{22}$Al, which is one of the beta-delayed 
\twop-emitters. } \label{fig:1984Cable}
\end{center} \end{figure}

The second category of the \twop-emissions 
is that after the $\beta$-decay 
of a parent nuclei (the beta-delayed \twop-emission) 
\cite{84Cable,91Detraz,92Moltz,98Mukha,00Fynbo,03Fynbo}. 
The most famous example may be the $^{22}$Al \cite{84Cable}. 
In Figure \ref{fig:1984Cable} taken from the Ref.\cite{84Cable}, 
the decay scheme of $^{22}$Al is shown. 
The ground state of this nuclide undergoes the $\beta^+$-decay 
to the $4^+$ state of the 
$^{22}$Mg, which has 12 protons. 
Note that the branching ratio for this decay is very small, being 
about a few percent \cite{84Cable}. 
Apart from the de-excitations and $\alpha$-decays, 
the generated $4^+$ state is unstable against the $1p$- and 
the associated sequential $2p$-emissions. 
We must pay attention to that, in this decay scheme, there are several 
intermediate resonances in the $^{20}$Ne-proton binary channel 
for the $1p$-emission. 
Furthermore, the $^{21}$Na nucleus has the {\it bound} 
single-proton states. 
Thus, there are two destinations of the proton(s)-emissions from the 
$^{22}$Mg: 
one is the bound $1p$-states of the $^{21}$Na 
reached by the $1p$-emission, whereas 
the other is the bound \twop-states of $^{20}$Ne through 
the sequential $2p$-emission. 
Similar complexities in the decay scheme are also in other 
$\beta$-delayed \twop-emitters, such as 
$^{26}$P \cite{84Cable,91Detraz}, $^{31}$Ar \cite{98Mukha,00Fynbo} 
and $^{39}$Ti \cite{92Moltz}. 
Therefore, the theoretical treatments are not simple for these nuclides. 
The effect of pairing correlations may not be significant, 
due to the dominant core-proton binary channels. 
Because of these complexities, 
we do not treat the beta-delayed \twop-emissions in this thesis. 
\begin{table}[t] \begin{center}
  \begingroup \renewcommand{\arraystretch}{1.3}
  \begin{tabular*}{\columnwidth}{ @{\extracolsep{\fill}} ccccc c} \hline \hline
  nuclide           & decay & $E^*$ (keV) & $Q_{\rm 2N}$ (keV) & $\Gamma_{\rm 2N}$ (keV) & other refs. \\ \hline
  $^{6}_{4}$Be$(0^+)$ & $\alpha$+2p & g.s. & 1371(5) & 92(6) \cite{88Ajzen,02Till} & \cite{89Boch},\cite{09Gri_80,09Gri_677}$^c$ \\
  $^{12}_{8}$O$(0^+)$ & $^{10}$C+2p & g.s. & 1790(40) & 578(205) \cite{95Kryger} & \cite{78KeKe,12Jager} \\
  $^{16}_{10}$Ne$(0^+)$ & $^{14}$O+2p & g.s. & 1400(20) & 110(40) \cite{83Wood} & \cite{78KeKe},\cite{08Muk,10Muk}$^c$ \\
  $^{17}_{10}$Ne$(3/2^-)$ & $^{15}$O+2p & 1288(8) & 344(8) & $7.6^{+4.9}_{-3.7}\times 10^{-6}$ \cite{97Chro} & \cite{93Till} \\
  $^{19}_{12}$Mg$(1/2^-)$ & $^{17}$Ne+2p & g.s. & 750(50) & $1.1^{+1.4}_{-0.25}\times 10^{-7}$ \cite{07Muk} & \cite{08Muk}$^c$ \\
  $^{45}_{26}$Fe$(3/2^+)$ & $^{43}$Cr+2p & g.s. & 1154(16) & [$2.8^{+1.0}_{-0.7}$ ms] \cite{05Doss} & \cite{02Pfu,02Gio},\cite{07Mie}$^c$ \\
  $^{48}_{28}$Ni$(0^+)$ & $^{46}$Fe+2p & g.s. & 1350(20) & [$8.4^{+12.8}_{-7.0}$ ms] \cite{05Doss} & \cite{11Pomo} \\
  $^{54}_{30}$Zn$(0^+)$ & $^{52}$Ni+2p & g.s. & 1480(20) & [$3.7^{+2.2}_{-1.0}$ ms] \cite{05Blank} & \cite{11Asch} \\
  &&&&& \\
  $^{6}_{2}$He$(2^+)$ & $\alpha$+2n & 1797(25) & 825 & 113(20) \cite{94Boch} & \cite{87Boch} \\
  $^{16}_{4}$Be$(0^+)$ & $^{14}$Be+2n & g.s. & 1350(100) & $800^{+100}_{-200}$ \cite{12Spyr} & \cite{03Audi} \\
  $^{26}_{8}$O$(0^+)$ & $^{24}$C+2n & g.s. & 150$^{+50}_{-150}$ & ? \cite{12Lund} & \cite{13Kohley_26O,13Caesar} \\ \hline \hline \end{tabular*}
  \endgroup
  \caption{ Table of nuclides in which two-nucleon emissions or 
radioactive decays have been experimentally observed. 
Similar tables can be found in the Refs.\cite{09Gri_40,12Pfu}. 
The 1st column is for the parent nucleus and its spin-parity in the 
reference state. 
The 2nd column indicates decay-modes. 
The 3th column is for the excited energy of the corresponding state, 
measured from the ground state. 
The 4th and 5th columns are for the Q-value and the decay width, 
respectively. 
Q-values are respect to the ground states of the daughter nuclides. 
For some long-lived nuclides, their lifetimes are shown 
instead of the decay widths. 
The 6th column lists the references other than that listed in the 5th column. 
Those which report the \twop-correlation measurements are indicated 
by the superscript $c$. } \label{tb_ch5_1}
\end{center} \end{table}

Even with several examples introduced above, however, there had not 
any observed nuclides, which have a long lifetimes enough to 
characterize the \twop-radioactivity. 
The breakthrough was made at the beginning of the 21th century in the 
experimental side, as we will review in the next subsection. 

\subsection{Modern Studies: after the 21th Century}
At the beginning of 21th century, a great development was made in the 
study of \twop-radioactivity. 
In 2002, the first observation of the true \twop-radioactivity in the 
$^{45}$Fe nucleus was made independently by two experimental groups, 
headed by M. Pf\"{u}tzner \cite{02Pfu} and by B. Blank \cite{02Gio}. 
In these experiments, the $^{45}$Fe nucleus was created by the 
projectile fragmentation with the primary beam of $^{58}$Ne. 
The decay products, namely $^{43}$Cr and two protons, 
were implanted into silicon detectors, where the total energy 
release in the decay can be determined experimentally. 
On the other hand, the identification of $^{43}$Cr was done by 
means of the energy-loss and the time-of-flight measurements. 
From the measured distributions of the energy release, 
the half-life of $^{45}$Fe was determined 
as $T_{1/2} = 3.2^{+2.6}_{-1.0}$ ms \cite{02Pfu} and 
$4.7^{+3.4}_{-1.4}$ ms \cite{02Gio}, which is long enough 
to be characterized as the \twop-radioactivity. 
We also note that, for these experiments, 
theoretical works \cite{91Brown,96Ormand,96Cole,97Ormand} 
played an helpful roles to infer the candidates of the true 
\twop-radioactive nuclides. 
\begin{table}[t] \begin{center}
  \begingroup \renewcommand{\arraystretch}{1.3}
  \begin{tabular*}{\columnwidth}{ @{\extracolsep{\fill}} ccccc c} \hline \hline
   nuclide           & decay & $J^{\pi}_{\rm core}$ & $Q_{\rm 1N}$ (keV) & $\Gamma_{\rm 1N}$ (keV) & other refs. \\ \hline
  $^{5}_{3}$Li$(3/2^-)$ & $\alpha$+p & $0^+$ & 1960(50) & $\simeq 1500$ \cite{88Ajzen} & \cite{00Hoef,02Till} \\
  $^{11}_{7}$N$(1/2^-)$ & $^{10}$C+p & $0^+$ & 2200(100) & 740(100) \cite{95Kryger} & \cite{78KeKe} \\
  $^{15}_{9}$F$(1/2^+)$ & $^{14}$O+p & $0^+$ & 1370(180) & 530(300) \cite{78KeKe} & \cite{03Peters,10Mukhamed} \\
  $^{16}_{9}$F$(0^-)$   & $^{15}$O+p & $1/2^-$ & 535(8) & 40(20) \cite{93Till} & \\
  $^{18}_{11}$Na$(1^-)$ & $^{17}$Ne+p & $1/2^-$ & 1250(110) & $\simeq 700$ \cite{NNDCHP} & \\
  $^{44}_{25}$Mn$(2^-)$ & $^{43}$Cr+p & $3/2^-$ & 1700(600) & [$<151$ ns] \cite{NNDCHP} & \\
  $^{47}_{27}$Co$(?)$    & $^{46}$Fe+p & $0^+$ & 2000(9000) & ? \cite{NNDCHP} & \\
  $^{53}_{29}$Cu$(3/2^-)$ & $^{52}$Ni+p & $0^+$ & $>350$ & [$<188$ ns] \cite{13Blank_49} & \\
  &&&&& \\
  $^{5}_{2}$He$(3/2^-)$ & $\alpha$+n & $0^+$ & 735(20) & 600(20) \cite{88Ajzen} & \cite{02Till} \\
  $^{15}_{4}$Be$(3/2^+)$ & $^{14}$Be+n & $2^+$(?) & $>1540$ & ? \cite{11Spyr} & \cite{03Audi} \\
  $^{25}_{8}$O$(3/2^+)$ & $^{24}$O+n & $0^+$ & $770^{+10}_{-10}$ & 172(30) \cite{08Hoffman} & \cite{13Caesar} \\ \hline \hline \end{tabular*}
  \endgroup
  \caption{ The core-nucleon subsystems of the two-nucleon emitters listed 
in the Table \ref{tb_ch5_1} are summarized. 
All the listed states are the ground states as 1N-resonances. 
The 1st column is for the core-nucleon system and its spin-parity 
in the reference state. 
The 2nd column indicates decay-modes. 
The 3th column indicates the spin-parity of the core nucleus. 
The 4th and 5th columns are for the Q-value and the decay width. 
Q-values are respect to the ground states of the daughter nuclides. 
For some long-lived nuclides, 
their lifetimes are shown instead of the decay widths. 
The 6th column is for the references other than that listed 
in the 5th column. } \label{tb_ch5_2}
\end{center} \end{table}

Since the memorable works for $^{45}$Fe, experimental efforts 
have been continued, in order to detect other \twop-radioactive 
nuclides and also to increase the accuracy of data. 
The novel \twop-emitters observed in this period include 
$^{19}$Mg, $^{48}$Ni, $^{54}$Zn and so on. 
In Table. \ref{tb_ch5_1}, we summarize the up-to-date properties 
of the observed \twop-emitters, and also of $2n$-emitters. 
As the additional information of the parent nuclei 
shown in Table \ref{tb_ch5_1}, 
we tabulate the properties of their core-nucleon subsystems 
in Table \ref{tb_ch5_2}. 
By comparing the corresponding 2N- and 1N-resonance energies, 
one can infer whether the interested nucleus is a true 
2N-emitter or not (the resonance energies are indicated as $Q_{\rm 2N}$ 
and $Q_{\rm 1N}$ in these Tables). 
For instance, in the case of $^{6}$Be, the data show 
that the \twop-resonance energy of $^6$Be is lower than the $1p$-resonance 
energy of its core-proton subsystem, $^5$Li. 
Thus, $^6$Be is expected to be a true \twop-emitter. 

For the \twop-emitters listed in Table. \ref{tb_ch5_1}, one can find that 
there is a broad gap between the lifetimes of the lighter and 
the heavier nuclides. 
%This feature is due to whether the Coulomb barrier is dominant or not. 
In the lighter \twop-emitters, such as $^6$Be, $^{12}$O and $^{16}$Ne, 
the Coulomb barrier plays a minor role and the resonance is 
mainly stabilized by the centrifugal barriers between the core and 
the valence protons. 
Consequently, their typical decay widths are on the same order 
among those nuclei, namely about $100$ keV. 
On the other hand, in the heavier \twop-emitters, the 
Coulomb barrier is higher, which reduces the penetrability of two protons, 
and the lifetimes become considerably longer. 
In recent studies, searching intermediate long-lived 
\twop-emitters, which may locates between $14\leq Z \leq 24$, 
has been a challenging task. 
Also notice that there have been no heavier $2n$-radioactive nuclides 
observed than those listed in Table \ref{tb_ch5_2}. 
Whether the $2n$-radioactive nuclide with 20 or more neutrons 
exists or not is still an open question. 
\begin{figure}[t] \begin{center}
(Figure is hidden in open-print version.)%\includegraphics[width = 0.55\hsize]{./y_quoted/2007Miernik_2.eps}
\caption{Figure 1 in Ref.\cite{07Mie}. 
A photograph of the \twop-radioactive decay of $^{45}$Fe 
obtained with the optical time-projection chamber. 
A track of a $^{45}$Fe ion entering the chamber from left is seen. 
The two bright, short tracks are protons emitted after 
the implantation of $^{45}$Fe on the detector. } \label{fig:2007Mie}
\end{center} \end{figure}

It is worthwhile to mention that the kinematics of the emitted 
two protons has been measured in the recent experiments, 
especially owing to the time-projection chamber. 
This device yields the 
photographs of the \twop-decays in a real-time 
regime, and the complete kinematics in most cases can 
be reconstructed \cite{07Mie_TPC,07Mie,07Gio,08Blank}. 
The photograph in Figure \ref{fig:2007Mie} displays 
this kinematics. 
\begin{figure}[t] \begin{center}
%  \fbox{\includegraphics[width = 0.55\hsize]{./y_quoted/2009Gri_06Be.eps}}
(Figure is hidden in open-print version.)
\caption{Figure 1 in Ref.\cite{09Gri_677}. 
The energy-angular correlation pattern of the 
\twop-emission from the $^{6}$Be nucleus. 
The upper two panels show the theoretical results in two different 
coordinates, whereas the lower two panels show the experimental results. 
See the original paper \cite{09Gri_677} for the definition of 
the variables. } \label{fig:2009Gri_06Be}
\end{center} \end{figure}

It should be noticed that, 
for the two-particle decay processes including the alpha-decays and 
the $1p$-emissions, the kinetic properties is completely determined with the 
total energy release. 
On the other hand, for three or more particle decays, 
even if the total energy is identified, 
one needs additional degrees of freedom to fully understand the process. 
In the studies of the \twop-emissions, as a tradition, 
(i) the total energy release and 
(ii) the opening angle between two relative momenta, 
corresponding to the two relative coordinates in the three-body system, 
have been often employed for this 
purpose \cite{01Gri_I,10Gri_V,09Gri_80,09Gri_677,12Ego}. 
Owing to the recent developments of experimental techniques, 
for several nuclides, their energy-angular distributions 
have been measured \cite{09Gri_80,09Gri_677,08Muk,10Muk,07Mie}. 
The observed distributions usually 
show the characteristic correlation pattern of the two protons, 
which can be interpreted as the dynamical character of each nuclide. 
In Figures \ref{fig:2009Gri_06Be} and \ref{fig:2009Gri_45Fe} taken from 
the Ref.\cite{09Gri_677}, for instance, the correlations in the 
decay of $^{6}$Be 
and $^{45}$Fe in the energy-angle plane are shown. 
In the 6th column of Table \ref{tb_ch5_1}, we list the references 
which report these measurements. 
It has also been expected that the qualitative information 
during the decay process can be extracted from 
these correlation patters. 
To investigate them could clarify, 
for instance, the density-dependence of 
the nuclear force, the pairing correlations in loosely or quasi-bound systems, 
and possibly the diproton correlation \cite{08Bertulani}. 
\begin{figure}[t] \begin{center}
%  \fbox{\includegraphics[width = 0.55\hsize]{./y_quoted/2009Gri_45Fe.eps}}
(Figure is hidden in open-print version.)
\caption{Figure 2 in Ref.\cite{09Gri_677}. 
The same as Figure 1 in Ref.\cite{09Gri_677}, but for the \twop-radioactive decay from the $^{45}$Fe nucleus. } \label{fig:2009Gri_45Fe}
\end{center} \end{figure}

As shown in Fig. \ref{fig:2009Gri_45Fe}, 
in the \twop-radioactivity of $^{45}$Fe, the measured correlation 
pattern suggests that there are considerable probabilities 
for the diproton-decay, characterized by the strong correlation 
between the emitted two protons. 
On the other hand, in $^6$Be shown in Fig. \ref{fig:2009Gri_06Be}, 
the diproton decay is less significant, 
and the correlation pattern shows a more extended 
and complicated distributions. 
It means that all the interactions in the final state take comparable 
contributions in this system to each other. 
Consequently, in light \twop-emitters, the observed 
quantities may be strongly affected by the final state interactions. 
It remains an open question how to extract the information on the 
diproton correlation from the experimental observables of the 
\twop-decays, which should be addressed by theoretical approaches. 

On the theoretical side, making synergy with the experiments, 
there have been remarkable developments established. 
The theoretical works by L. V. Grigorenko {\it et. al.} should be especially 
introduced \cite{93Zhukov,01Gri_I,03Gri_II,07Gri_III,07Gri_IV,10Gri_V,02Gri_15,02Gri,03Gri,09Gri_80,09Gri_677,09Gri_40,12Ego,12Gri}. 
Their works until 2009 are well summarized in the Ref.\cite{09Gri_40}. 
They have investigated \twop-decays based on 
the Three-body scattering equation, 
which is the basic formalism of scattering problems in 
quantum three-body systems. 
To solve the three-body Three-body scattering equation, they have developed 
the Hyper-spherical Harmonics (HH-) method within the 
non-Hermite framework. 
The HH-functions were originally proposed as an efficient basis to 
solve the quantum few-body problems \cite{93Zhukov}. 
Additionally, they have carefully treated the asymptotic properties 
of the Coulomb three-body problems. 
Notice that even the asymptotic solutions cannot be analytically 
obtained for this problem. 
Thus, within an approximate asymptotic conditions, they have 
employed an enormously large model space, 
which guarantees the saturation of results. 
Up to date, their calculations have been remarkably successful in 
reproducing the experimental results of both the decay widths (the lifetime) 
and the \twop-correlation patterns for several nuclides 
\cite{09Gri_40,09Gri_80,09Gri_677} (see Figs. \ref{fig:2009Gri_06Be} and 
\ref{fig:2009Gri_45Fe}). 
They have also shown that the final-state interactions lead to a crucial 
effect on the democratic \twop-emission, especially for $^{6}$Be 
\cite{09Gri_80,09Gri_677,12Ego}. 
Recently, they have also discussed the effect of 
the initial configurations of two protons before 
the barrier penetration \cite{12Ego,12Gri}. 
However, the relations between the \twop-emission and the 
diproton correlation is still not investigated. 

Other theoretical efforts based on the microscopic 
picture of \twop-decays 
have also been devoted \cite{05Rotu,06Rotu,13Deli,13Olsen}. 
As a notable progress, it was predicted that 
the \twop-radioactive nuclides can exist widely along 
the proton-dripline, up to the proton number of $Z\leq82$ \cite{13Olsen} 
(the upper limit of $Z$ is owing to the dominance of 
the $\alpha$-decay). 
This prediction is an inspiring work towards the further 
exploration of the \twop-emitters. 
Today, predicting and discovering the novel information of 
\twop-emissions are hot interests in both theoretical 
and experimental sides. 
We also mention that, if a full microscopic theory of 
\twop-radioactivity is established, it can be naturally extended to 
other processes, such as $2n$- and $\alpha$-decays \cite{12Betan}. 
However, in these microscopic models, the equal treatment of both 
the true and sequential processes is a challenging task. 
Furthermore, there remains a serious problem, 
associated with the computational resources. 

\section{Theoretical Frameworks for Quantum Meta-stability}
Theoretically, there are two main frameworks for quantum 
meta-stability. 
One is the non-Hermite, time-independent framework, 
whereas the other is the time-dependent framework. 
In this section, we detail the advantages and shortcomings of 
these methods, 
regarding their applications to the \twop-emissions. 
From a theoretical point of view, the \twop-emissions are 
quantum-mechanical phenomena, dominated critically by 
the tunneling effect coupled to the continuum region. 
Additionally, in contrast to the two-body decay processes 
including the alpha-decays and $1p$-emissions from the spherical 
parent nuclei, 
the many-body properties with the nuclear and Coulomb interactions 
must be treated on equal footing in the \twop-decays. 

\subsection{Time-Independent Framework}
Up to present, almost all the theoretical works of 
\twop-emissions have been based on the time-independent, or 
equivalently on the non-Hermite framework. 
The original idea of this method was proposed by Gamow 
to understand the $\alpha$-particle 
decays \cite{28Gamov_01,28Gamov_02,29Gurney,89Bohm}. 
With the time-independent method, 
a meta-stable state is solved as a time-independent 
eigen-state of the Hamiltonian with a {\it complex} eigen-energy, 
corresponding to the boundary condition that the wave function 
should be asymptotically connected to the out-going wave. 
In actual calculations, one solves this non-Hermite 
eigen-state by, {\it e.g.} complex-scaling the coordinates 
in the wave function so as to yield the complex eigen-energy 
in the continuum region \cite{06Aoyama}. 
The imaginary part of the eigen-energy corresponds to the decay width, 
while the real part corresponds to the total energy release of the 
decay (the Q-value). 
An advantage of this method is that one can solve the meta-stable 
states in almost the same way as the stable states. 
The decay width can be calculated with a high accuracy even if 
it is extremely small \cite{97Aberg,00Davis}. 

As already introduced in the previous section, for \twop-emissions, 
the results by Grigorenko {\it et.al.} within the 
time-independent method show the excellent agreement with the 
experiments for the observed momentum and angular correlations. 
On the other hand, this method is somewhat difficult to extract the 
essential cause of phenomena. 
The correspondence between the wave functions with complex 
energies and the 
real phenomena is not completely recognized. 
Although the obtained results have well reproduced the experimental 
data for the \twop-decay, the mechanism to yield this agreement 
has not been sufficiently clarified. 
The connection between the diproton correlation and the decay 
observables has yet to be revealed. 

\subsection{Time-Dependent Framework}
In contrast to the time-independent method, 
the time-dependent method treats the quantum 
resonances or tunnelings as the temporal developments 
of meta-stable states, maintaining the Hermiticy in the 
framework \cite{89Bohm,89Kuku,47Kry}. 
These approaches have been applied to several quantum tunneling phenomena 
\cite{94Serot, 98Talou, 99Talou_60, 00Talou, 11Garc, 11Campo, 12Pons}, 
with an advantage that it can provide an intuitive way to understand 
the tunneling mechanism. 
However, there have been no applications of this method 
to the \twop-decays, except for that for the dynamics 
in the classically allowed region after 
the tunneling stage \cite{08Bertulani}. 

In applications of the time-dependent method to \twop-emissions, 
the initial \twop-state should be defined as a quasi-bound 
state inside the Coulomb and centrifugal potential barriers. 
For instance, one modifies the potential barrier at $t=0$ so that 
the initial state can be prepared as a quasi-bound state of 
the original Hamiltonian. 
The modified potential is then suddenly changed to the original one, 
and the initial state evolves in time 
to the final state where all the particles 
are separated along the time-evolution with the original Hamiltonian. 
The decay width can be determined from 
the survival probability of the initial state. 
Furthermore, the tunneling process can be intuitively 
understood by monitoring the time-development of the wave function 
and thus of the density distribution. 
The sensitivity of the \twop-emissions to, {\it e.g.} the diproton 
correlation, can be translated to the dependence on 
the initial configuration of two protons inside a parent nucleus. 
A drawback of this method is that it does not practically work 
when the decay width is extremely small. 
It often needs a great amount of computational resources to 
obtain the final results. 

In this thesis, from a complementary point of view to the works 
in past based on the time-independent method, 
we employ the time-dependent method. 
We only focus on the light \twop-emitters with comparably short 
lifetimes, to which the time-dependent method practically works. 
This method can be a powerful tool to reveal 
the relation between the diproton correlation and the \twop-emission 
by making full use of its intuitive nature. 
We stress that these problems have seldom been studied in literature 
in the past, and our present study is expected to provide 
a novel insight into the multi-nucleon meta-stable 
systems and their decays. 
\include{end}
