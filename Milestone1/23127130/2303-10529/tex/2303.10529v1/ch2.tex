\documentclass[a4paper,12pt]{report}
\include{begin}

\chapter{Review of Dinucleon Correlation} \label{Ch_2}
In this Chapter, we briefly summarize the history of studies on the 
dinucleon correlation, and also of some related topics. 
We do not include the two-nucleon emissions and radioactive decays here, 
which will be detailed in Chapter \ref{Ch_5}. 

\section{Dinucleon Correlation in Stable Nuclei} \label{Sec_2_1} 
The first proposal of the dinucleon correlation was made by A.B. Migdal 
for two neutrons inside nuclei \cite{73Mig}. 
He argued that, even a dineutron is not bound in the vacuum, 
there can be a bound state of two neutrons near the surface of 
atomic nuclei, due to the nuclear meanfield confining those. 
After his proposal, several theoretical 
studies have been performed regarding the dineutron correlations. 
The dineutron correlation can be characterized as 
the special localization of two neutrons with, 
a compact distance compared to the total radius of the whole nucleus, 
and a large component of the spin-singlet configuration. 
For the spin-singlet character, 
it has been known from, {\it e.g.} the characteristic 
odd-even staggering of binding energies, that two nucleons 
in the same orbit tend to couple into the spin-singlet state 
due to the pairing correlation. 

Various efforts have been devoted to investigating 
the spatial correlation between two nucleons 
associated with the pairing interaction. 
The paper by Catara {\it et.al.} is worthwhile to be 
mentioned \cite{84Catara}. 
In this paper, the authors discussed the two-neutron spatial correlation 
caused by the pairing interaction in the ground and excited $0^+$ states 
of $^{206}$Pb, based on shell model with a schematic pairing interaction. 
It was shown that the parity-mixing in the partial core-neutron system 
is indispensable to occur the spatial localization of the two neutrons in the 
ground state (see Figure \ref{fig:1984Catara}). 
This parity-mixing is due to the scattering effect due to the pairing 
interaction inside nuclei. 
It was also suggested that the pairing interaction is responsible 
not only for localization of two neutrons, 
but also for an increase of the spin-singlet 
configuration, which cannot be explained within the pure shell 
(mean-field) model. 
At the same time, the authors raised the alarm that contributions 
of the pairing interaction ($\sim 1$ MeV) to the relative distribution of two 
neutrons are not sufficiently 
large to overcome the dominant shell structure. 
They argued that a two-neutron cluster cannot have a 
$\delta$-function-like distribution, 
even if an enormously large model-space is employed. 
\begin{figure*}[h] \begin{center}
%  \fbox{\includegraphics[width = 0.98\hsize]{./y_quoted/1984Catara_2.eps}}
(Figure is hidden in open-print version.)
\caption{Figure 2 in Ref.\cite{84Catara}. 
In panels (a) and (b), authors show the density distributions of 
two neutrons without configuration mixing of different parities. 
In panel (c), on the other hand, they show the result 
with configuration mixing, where the localization of two neutrons 
can be seen. } \label{fig:1984Catara}
\end{center} \end{figure*}

Similar calculations but based on different theoretical models 
have also been performed, where their conclusions agree 
with each other \cite{89Lotti, 91Bert, 97Esb, 05Mats, 05Hagi, 11KEnyo}: 
the pairing interaction causes the spatial 
localization with the enhanced spin-singlet configuration, 
which is absent in the pure mean-field model. 
We also touch on the paper \cite{05Mats} by Matsuo and his collaborators. 
In this paper, based on the Hartree-Fock-Bogoliubov theory, 
the authors discussed the pairing and dineutron correlations in 
medium-heavy neutron-rich nuclei. 
It was shown that the mixing of, not only the core-nucleon parities, 
but also higher core-nucleon angular momenta, $l$, are indispensable 
to invoke the spatial localization of two neutrons. 
%According to their calculations, at least, one should take up to $l=5$ 
%into account to discuss the dinucleon correlation. 
%These situations are either in the weakly-bound nuclei, 
%which will be mentioned in the section \ref{Sec_23}. 

We also refer to the connection between the dineutron correlations 
and the pair-transfer reactions. 
It has been actively discussed that the dinucleon correlation may 
enhance the cross sections for the simultaneous two-nucleon 
transfer reactions. 
The simplest probe is given with $(t,p)$ and 
$(p,t)$ reactions. 
The pair-transfer strength of nuclides differing by two units 
have been studied extensively in the 
experiments using these reactions \cite{68Bjerr,69Bjerr,73Broglia,01Oert}. 
As a result, the significant increase of transfer cross sections 
for nuclei with even-number nucleons has been found. 
A detailed theoretical studies was also performed in 
\cite{91Igarashi} by Igarashi {\it et.al.} for Pb isotopes. 
They showed that the cross sections of $(p,t)$ reactions are 
increased due to the configuration mixing caused by the pairing interaction, 
that is consistent with the experimental data. 
Following these simple cases, a similar enhancement in 
collisions of two heavy-ions (HIs) has also been predicted 
and observed 
\cite{71Diet,71Kleber,11Shim,13Potel,13Shim_arx,91Speer,99Peter}. 
The enhanced pair-transfer cross sections can be naively understood 
as arising from the transferred dineutron-like cluster, which 
can be associated with collective features, {\it e.g.} the pair-vibrational 
or/and the pair-rotational excitations. 
However, the pair-transfer reaction itself is not only from 
the one-step transfer of spatially localized two nucleons, 
but also from the sequential two-step transfers. 

Thus, in discussing the dinucleon correlations, the second mechanism 
has to be handled with good care. 
%if one hopes to ensure the dinucleon correlations, 
%showing the dominance of the 1st mechanism must be the key point. 
Even with many experimental data, 
whether one can extract useful information on the dinucleon correlations 
depends on the theoretical ability to describe its collective effect on 
the pair-transfer reactions in heavy-ion collisions \cite{01Oert}. 
In theoretical calculations, one should treat a change 
of coordinates associated with transferred two nucleons to evaluate 
the reaction cross sections. 

It considerably complicates a theoretical formulation of two-neutron 
transfer reactions, 
if one treats it rigorously. 
At the same time, the results sensitively depend on the wave functions 
of two colliding nuclei, which should be computed by taking 
the pairing correlations into account. 
In order to get a sufficient accuracy, there still remain several 
problems for nuclear structure calculations, 
including the nuclear tensor forces, the core excitations and so on, 
in addition to a theoretical modeling of a complicated pair-transfer 
reaction. 
It is expected that theoretical improvements overcoming these 
difficulties will provide an evidence for the dinucleon correlations. 

\section{Unstable Nuclei}
The dineutron correlation has been attracted a renewed interest 
due to the establishment of the unstable nuclear physics. 
For neutron-rich unstable nuclei, 
the idea of the dinucleon correlations 
has been frequently discussed as one of the exotic features 
associated with the pairing correlation in weakly bound systems. 
%In the next section, we briefly review the famous aspects in unstable nuclei, 
%which can be regarded with dinucleon correlations. 

The frontier of nuclei in the nuclear chart 
has been expanded enormously for the recent decades. 
This is mainly thanks to the experimental developments enabling one 
to access ``unstable'' nuclei. 
These nuclei have a large proton- or neutron-excess, 
locate far from the $\beta$-stability valley, and 
are significantly short-lived compared with traditional 
radioactive nuclei close to the beta-stability line. 
For any unstable nuclide, 
one should be careful of ``what makes it to be unstable''. 
Most unstable nuclei known today are, in fact, stable against the 
nucleon emission. 
The main source of this instability is thus the weak interactions, 
not the strong interactions. 
On the other hand, by increasing the proton or neutron-excess, 
one can find many nuclides which are unstable against the 
nucleon emission. 
These nuclides define the proton- and neutron- driplines. 
Nuclei near and beyond these driplines can be considered as 
novel and exotic regions in nuclear physics. 
For the past decades, 
studies of these exotic nuclei have brought about deeper insights 
into nuclear physics, 
even though those scarcely exist on earth. 

\subsection{Neutron-Rich Nuclei}
Historically, the earlier interests were focused on 
the neutron-rich side. 
Especially, since the seminal experiments with radioactive isotope (RI) 
beams performed in 1980's \cite{85Tani_01,87Hansen,88Tani}, 
several exotic features in neutron-rich unstable nuclei have been discovered. 
These exotic features mainly due to the weakly binding of valence neutron(s). 
We list them below. 

\begin{enumerate}
\item Dineutron correlation: As mentioned in Chapter \ref{Ch_Intro}, for 
neutron-rich nuclei, the strong pairwise correlation between two neutrons 
has been predicted. 
Its source is the density-dependence of the pairing correlation, and it 
may lead to the dineutron-like clustering inside nuclei. 
We introduce this topic more in detail later. 

\item Halo and skin structures: A Large extension of the density distribution 
has been found for several neutron-rich nuclei, 
which are referred to as ``halo'' or ``skin'' 
nuclei \cite{85Tani_01,88Tani,87Hansen}. 
Famous examples include $^{6}$He and $^{11}$Li. 
For these nuclei, significantly large reaction cross sections 
were observed. 
By analyzing these experimental data with the Glauber model \cite{63Glauber}, 
their neutron radii were shown to be significantly larger than other
isotopes (see Figure \ref{fig:19852006exp}(a)). 
The neutron density was shown to have a long tail from the core nucleus. 
The weakly bound neutron(s) in the valence $(s_{1/2})$- or 
$(p_{3/2,1/2})$-orbit can generate this tail, 
like the halo or the skin around the core. 
With neutron-removal reactions, 
the corresponding narrow momentum distributions have been observed 
in such nuclei \cite{88Koba,90Anne,95Shimoura}. 
Studying these structures can lead to the understanding of 
the loosely bound or the dilute density region of nuclear systems. 

\item Soft multi-pole excitations: A significant increase 
of the probability for the electro-magnetic excitations at the 
lower energies has been observed for several nuclei 
\cite{93Ieki,95Shimoura,99Aumann,04Fuku,06Naka}. 
Especially, as shown in Figure \ref{fig:19852006exp}(b), 
the $E1$-transition strength of $^{11}$Li has a remarkable 
increase at excitation energies around $1$ MeV only. 
This is in marked contrast against normal nuclei, which show 
the $E1$-response at $E=10-20$ MeV due to the 
giant dipole resonance \cite{05Paar,05Terasaki,06Sil}. 
Theoretically, It has been considered that the soft multi-pole excitations 
are due to the relative motion 
between the core and the loosely bound neutron(s) \cite{10Ikeda}. 
Especially, for nuclei with two or more loosely bound neutrons, 
it is expected that the excitation spectra reflect not only the 
core-neutron motion but also the relative motion of two 
neutrons \cite{95Shimoura,06Naka,91Esb,92Esb,95Esb,95Sagawa,97Bona,03Myo,05Mats,05Hagi,06Horiuchi,06Gri,07Hagi_SDE,07Bertulani_76,10Kiku,11Oishi,13Kiku}. 
Geometry of the ingredient particles inside nuclei 
may be also revealed by analyzing these excitations. 
Especially, the opening angle between the valence neutrons is an important 
quantity, which is intimately related to the dineutron 
correlation \cite{06Naka,05Hagi,07Bertulani_76}. 

\item Borromean character: For several nuclides, so called ``Borromean 
character'' has also been discussed \cite{91Bert,93Zhukov,95Esb,05Hagi}. 
A Borromean nucleus is defined as a three-body bound system 
in which any two-body subsystem does not bound alone. 
Famous two-neutron Borromean nuclei are $^6$He $\cong \alpha +n+n$ 
and $^{11}$Li $\cong$ $^{9}$Li $+n+n$, 
where $^5$He, $^{10}$Li and a dineutron have no bound states. 
The pairing interaction between the valence nucleons plays an 
essential role in stabilizing these nuclei \cite{91Bert}. 
A similar character exists in proton-rich nuclei, namely 
a two-proton Borromean nucleus, $^{17}$Ne 
\cite{95Zhukov,96Tim,04Garrido_01,04Garrido_02,05Gri,06Gri}. 
The Borromean character deeply associates with the halo 
structure and the soft multi-pole excitations. 
For $^{6}$He or $^{11}$Li, as mentioned above, 
there have been enormous experiments which 
suggest the extended density-distribution or the enhancement of 
low-lying excitations. 
%Further studies across these subjects should be performed. 

\item Two-neutron emission: Recently, as we touched on Chapter \ref{Ch_Intro}, 
two-neutron emissions from the ground states have been observed 
in several neutron-rich nuclei \cite{12Spyr,13Kohley_13Li,13Kohley_26O}. 
Because there are no Coulomb barriers for neutrons, 
the main source of these resonances is the centrifugal barriers 
between the core (daughter) nucleus and valence neutrons. 
Similarly to \twop-emissions, two-neutron emissions are promising phenomena 
which can provide the useful means to investigate the 
dineutron correlations. 
In this thesis, however, we do not discuss the two-neutron emissions in detail. 
\end{enumerate}

Of course, these listed properties are entangled to each other. 
Our main interest in this thesis is the dineutron and, as mentioned later, 
the diproton correlation. 
However, except for nuclei with only one weakly bound nucleon, 
we can overlook all of the above properties from a common point 
of view: ``pairing correlation''. 
Therefore, a deep understanding of the dinucleon correlation is 
expected to reveal not only a novel aspect of the pairing correlations, 
but also an universal property covering all the subjects listed above. 
Furthermore, these research achievements may be exported to other 
multi-fermion systems. 
\begin{figure*}[tb] \begin{center}
(Figure is hidden in open-print version.)
% \begin{tabular}{c} %switch-off the auto-turning
%  \begin{minipage}[t]{0.38\hsize} \begin{center}
%     (a) \\ \fbox{\includegraphics[height=68truemm, clip, trim = 0 0 0 0]{./y_quoted/1985Tanihata_2.eps}} \\
%  \end{center} \end{minipage}
%  \begin{minipage}[t]{0.61\hsize} \begin{center}
%     (b) \\ \fbox{\includegraphics[height=68truemm, clip, trim = 0 0 0 0]{./y_quoted/2006Nakamura_2.eps}} \\
%  \end{center} \end{minipage}
% \end{tabular}
\caption{Figure 3 in Ref.\cite{85Tani_01} in the left panel; Figure 3 in Ref.\cite{06Naka} in the right panel.
The left panel: The root-mean-square matter-radii determined from experimental 
data of reaction cross sections. 
Large radii of $^6$He, $^8$He and $^{11}$Li can be seen. 
The right panel: The $E1$-strength distribution 
observed with the Coulomb break-up of $^{11}$Li. 
An enhancement of the strength in lower energy region is present. } 
\label{fig:19852006exp}
\end{center} \end{figure*}

\subsection{Proton-Rich Nuclei}
We also summarize supplementary information unique to the proton-rich side. 
In fact, the exotic features listed in the previous subsection 
can be considered almost equally 
for the proton-rich unstable nuclei. 
For example, the $^{17}$Ne nucleus is a 
\twop-Borromean nucleus 
\cite{95Zhukov,96Tim,04Garrido_01,04Garrido_02,05Gri,06Gri}, and 
also is a famous candidate to have 
the \twop-halo \cite{94Ozawa,95Zhukov,96Tim,05Gri,06Gri} and 
the diproton correlation \cite{04Garrido_02,10Oishi,11Oishi}. 
Nevertheless, compared to the neutron-rich side, 
the proton-rich unstable nuclei have been less studied so far. 
The characteristic problem in proton-rich nuclei is, of course, 
the Coulomb repulsion between the valence protons. 
As a natural consequence of the Coulomb repulsion, 
proton-rich nuclei have less binding energies than those of 
their mirror neutron-rich nuclei. 
Furthermore, even if its mirror partner can be bound, 
a proton-rich nucleus may become unstable against 
proton(s)-emissions. 
Thus, if we restrict our interests in nuclei which are 
stable against nucleon emissions, 
proton-rich side may be, in a sense, ``barren land''. 
This is a symbolic property of the breaking of the mirror-symmetry. 
However, abandoning this restriction, 
breaking of the mirror-symmetry can be interpreted as an useful 
property which produces a variety of phenomena of atomic nuclei, 
some of which can be observed only on the 
proton-rich side \cite{95Doba,96Cole,96Doba}. 

Concerning the pairing properties, it has been 
frequently discussed whether the Coulomb repulsion 
strongly affects the nuclear pairing attraction or not. 
Recent studies suggest that the effect of the Coulomb repulsion on 
binding energies of nuclei is minor, 
and the effect is roughly estimated as an about $10\%$ reduction over 
the nuclear attractions. 
This conclusion can be deduced from several theoretical and experimental 
analysis \cite{02Hila,09Lesi,09Bert,11Yama}. 
Moreover, in our previous studies \cite{10Oishi,11Oishi}, 
it was also suggested that 
the diproton correlation can exist in proton-rich nuclei similarly to 
the dineutron correlation in neutron-rich nuclei, 
due to the minor role of the Coulomb repulsion. 
If the diproton correlation really exists, 
breaking of the mirror-symmetry can provide another route to probe it, 
namely ``two-proton (\twop-) radioactivity''. 
This idea is the basis of this thesis, and we will detail it 
in Chapter \ref{Ch_5}. 

\section{Dinucleon Correlation in Unstable Nuclei} \label{Sec_23}
Because of the recent theoretical and computational developments, 
it has become possible to perform much reliable calculations for 
nuclear pairing correlations. 
This development brought us a point of view to discuss the dinucleon 
correlations in connection to the density-dependence 
of pairing correlations. 
\begin{figure}[tbp] \begin{center}
%\fbox{\includegraphics[width = 0.6\hsize, clip, trim = 0 0 0 0]{./y_quoted/2006Matsuo_4}}
(Figure is hidden in open-print version.)
\caption{Figure 2 in Ref.\cite{06Mats}. 
The pairing gap in the symmetric and the pure-neutron 
nuclear matters as functions of the density, $\rho$. 
Those are calculated based on the HF-BCS theory with 
some different models of the nuclear interaction. } \label{fig:2006Matsuo_gap}
\end{center} \end{figure}

For this purpose, it is useful to discuss nuclear matter at first. 
There have been considerable studies which reports the 
significant density-dependence of nuclear pairing correlations 
in the nuclear matter \cite{06Mats,07Marg,10XGuang,13Sun}. 
We especially refer to the Ref.\cite{06Mats}, where the author 
applied the HF-BCS approach to the symmetric and pure-neutron 
nuclear matters. 
According to their results, as shown in Figure \ref{fig:2006Matsuo_gap}, 
the pairing gap in both 
symmetric and pure-neutron matters significantly 
depends on the density, $\rho$. 
It takes the maximum value within the 
densities of $\rho/\rho_0 \simeq 0.1 - 0.01$, 
where $\rho_0 \sim 0.15$ fm$^{-3}$ is the nuclear saturation density. 
Furthermore, as shown in Figure \ref{fig:2006Matsuo}, 
it is found that the spatial distribution of the spin-singlet Cooper pair 
of nucleons within a wide range of $\rho/\rho_0 \simeq 0.5 - 10^{-4}$ 
is well localized with a typical distance of $r_{\rm N-N} \leq 5$ fm. 
They also found a compact root-mean-square (rms) radii, $\xi_{\rm rms} \leq 5$ 
fm of two nucleons, 
suggesting the dinucleon correlations in nuclear matters. 
On the other hand, in the saturated or the infinitesimal density-region, 
a Cooper pair loses the dinucleon correlations. 
This result is, of course, the coincidence to the weakening of the 
pairing correlations in the saturated and 
the infinitesimally dilute densities. 
We also note that this variety of the pairing correlations as a function 
of the density
can be connected to the BCS-BEC crossover 
in nuclear matters. 
In the paper \cite{06Mats}, it was suggested that 
the region of $\rho/\rho_0 \simeq 0.1 - 10^{-4}$ 
corresponds to the domain of the BCS-BEC crossover. 
The similar conclusions have been obtained from other studies, 
although there are some quantitative differences in the appropriate 
value of $\rho$ at which the dinucleon correlation and 
the BCS-BEC crossover appear \cite{07Marg,10XGuang,13Sun}. 
\begin{figure*}[tb] \begin{center}
%\fbox{\includegraphics[width=0.95\hsize]{./y_quoted/2006Matsuo_6.eps}}
(Figure is hidden in open-print version.)
\caption{Figure 4 in Ref.\cite{06Mats}. 
The wave function of a Cooper neutron-pair in the symmetric and 
pure-neutron matters, calculated 
with the HF-BSC theory. } \label{fig:2006Matsuo}
\end{center} \end{figure*}

The similar studies have been performed 
for finite nuclei, where 
some of those were already introduced in Sec. \ref{Sec_2_1}. 
Furthermore, for unstable nuclei with weakly-bound nucleons, 
the dinucleon correlations have been discussed in connection to 
other exotic features listed in the previous section. 
Especially, $^6$He, $^{11}$Li and $^{17}$Ne nuclei have attracted 
much attentions. 
Theoretical studies in 
Refs.\cite{91Bert,92Esb,93Zhukov,96Tim,96Vinh,97Esb,04Garrido_01,04Garrido_02,05Gri,05Hagi,06Gri,07Bertulani_76,07Hagi_01,07Hagi_SDE,07Hagi_03,08Hagi,10Oishi,10Oishi_err,11Hagi_01,11KEnyo}, 
were dedicated for these problems. 
A popular model, on which almost all of these theoretical studies were based, 
is the nuclear three-body model, where 
one can describe a pair of nucleons in the mean-field generated 
by the core nucleus. 
The density-dependence of pairing correlations is usually taken into account 
in a phenomenological way, such as modifying the pairing interaction 
from that in vacuum. 
According to these mean-field plus pairing model calculations, 
a strong localization of the valence nucleons inside the ground states of 
these nuclei has been predicted 
\cite{91Bert,93Zhukov,96Vinh,04Garrido_01,04Garrido_02,05Hagi,07Bertulani_76,07Hagi_01,07Hagi_03,10Oishi,10Oishi_err}. 
As an example, Figure \ref{fig:2007Hagino} taken from Ref.\cite{07Hagi_01} 
shows this localization. 
This localization often occurs together with an enhancement of 
the spin-singlet configuration, 
identically to the dinucleon correlations. 
We also note that nuclei with weakly bound nucleons are expected 
to be good testing grounds for 
the BCS-BEC crossover in finite nuclei \cite{07Hagi_01} and 
the anti-halo effect of pairing correlations 
\cite{00Benn,11Hagi_AHE,13Sun_arx}. 
These topics are, however, beyond the scope of this thesis. 
\begin{figure*}[tbp] \begin{center}
(Figure is hidden in open-print version.)
%  \fbox{\includegraphics[width = 0.2\hsize, clip, trim = 0 0 0 10]{./y_quoted/hagino2.eps}
%        \includegraphics[width = 0.6\hsize, clip, trim = 5 5 0 15]{./y_quoted/hagino1.eps}}
\caption{Figure 1 in Ref.\cite{07Hagi_01}. 
The density distribution of the valence two neutrons in $^{11}$Li, 
obtained with $^9$Li$+n+n$ model calculations. 
A localization with $r\cong 2$ fm and $R\cong 3$ fm can be seen. 
} \label{fig:2007Hagino}
\end{center} \end{figure*}

\section{Possible Means to Probe Dinucleon Correlation}
Although there have been various theoretical predictions, 
there have been so far no direct evidences 
for the dinucleon correlation. 
The most serious difficulty is that 
the diproton and dineutron correlations are intrinsic 
phenomena, and are hard to be probed by popular means of 
experiments. 
Especially, for the dinucleon correlations in the bound state, 
it is in principle impossible to probe those without 
disruptions by an external field. 
Thus, we have to change our view to 
``how well do we extract the information on the dinucleon correlations''. 

For the purpose towards this direction, 
a lot of possibilities have been discussed. 
The first one is analyzing 
the pair-transfer reaction in heavy-ion collisions. 
Its basic idea, history and the present difficulties 
have already been introduced in Sec.\ref{Sec_2_1}. 
We should also note that, for unstable nuclei, a theoretical 
analysis for the pair-transfer reactions may become 
even more complicated due to their exotic structures. 
If these problems can be resolved, 
the pair-transfer reaction will be one of the most powerful tools to 
investigate the dinucleon correlations in both stable and 
unstable nuclei. 

The second candidate is using excitations by electro-magnetic interactions. 
The soft multi-pole excitations and the Coulomb break-up reactions belong to 
this category. 
For instance, the momentum distributions observed in 
Coulomb break-up reactions have been discussed frequently 
associated with the dinucleon correlations 
\cite{95Shimoura,06Naka,07Hagi_SDE,07Bertulani_76,10Kiku,13Kiku}. 
However, these experiments are performed by perturbing 
the ground state properties \cite{01Myo,03Myo,10Kiku}. 
Consequently, the experimental results depend not only on the ground 
state, but also on the excited states. 
From recent theoretical studies, 
it is concerned that this inclusion of excited states suppress 
the sensitivity to the dinucleon correlation, bringing a 
serious drawback to the direct access to it \cite{10Kiku}. 
Furthermore, even if there will be a significant signal of 
the dinucleon correlations in the experimental data, 
one must distinguish whether it reflects the dinucleon correlations 
in the ground or in the excited states. 

Another possibility to probe the dinucleon correlation 
is to observe two-nucleon decays and emissions. 
These attempts have been performed intensively since 
the beginning of 2000s, mainly due to the remarkable 
developments in the experimental techniques \cite{08Blank,09Blank,12Pfu}. 
However, the connection between the two-nucleon emissions and 
the dinucleon correlations has not yet been clarified 
\cite{01GCampo,07Bert,08Bertulani,12Maru}. 
In Chapter \ref{Ch_5}, we will summarize 
the history and backgrounds of these topics. 

\section{Summary of this Chapter}
We have introduced the historical background of dinucleon correlations 
and its relevant topics in this Chapter. 
Although it is still a theoretical prediction, 
the dinucleon correlation is one of the important characters of 
multi-nucleon systems, and 
is essentially connected to the nuclear pairing correlations. 
If the dinucleon correlations will be directly detected, it will 
provide strong constraints on the nuclear interactions 
and on the framework for multi-nucleon problems. 
Furthermore, we may extract an universal knowledge in 
other multi-fermion systems from these observations. 

In Chapters \ref{Ch_3body} and \ref{Ch_Results1}, 
we discuss how the dinucleon correlations are theoretically predicted. 
For this purpose, we will employ the three-body model, similarly to Refs. 
\cite{91Bert,92Esb,93Zhukov,96Tim,96Vinh,97Esb,04Garrido_01,04Garrido_02,05Gri,05Hagi,06Gri,07Bertulani_76,07Hagi_01,07Hagi_SDE,07Hagi_03,10Oishi,10Oishi_err}. 
The next Chapter will be dedicated to the formalism of this model. 
Applying this model to several nuclei, we will discuss 
the dinucleon correlations in finite nuclei. 
Those will be summarized in Chapter \ref{Ch_Results1}. 
\include{end}
