\documentclass[a4paper,12pt]{report}
\include{begin}

\chapter{Quantum Three-Body Model} \label{Ch_3body}
In this Chapter, we introduce in detail the three-body model of the 
core nucleus + nucleon + nucleon, which we use to describe the 
dinucleon correlations. 
This model is identical to the three-body version of the 
Core-Orbital Shell Model (COSM) \cite{88Suzuki_COSM}. 
With COSM, one starts from considering the core nucleus as a source 
of the mean-field. 
Then one adds one or more valence nucleon(s) around the core. 
In this thesis, we do not care about the core-excitations and thus 
the core plays simply as an inert particle. 
The pairing correlations between the valence nucleons 
can be explicitly included in this model. 
The deviations from the pure mean-field approximation can also 
be discussed, providing the semi-microscopic point of view of 
the pairing correlations. 

In our formalism, 
the coordinates and the spin variables of each nucleon are indicated as 
$\bir_i$ and $\bis_i \phantom{0}(i=1,2)$, respectively. 
We also use $\xi_i \equiv \{ \bir_i, \bis_i \}$ 
for a shortened notation. 
\footnote{Because we only treat systems with the core plus 
two nucleons of the same kind in this thesis, 
the isospin variables are not necessary. } 
The angular variable, that is equivalent to the radial unit vector, 
are indicated by $\ubir$. 
The orbital and the spin-coupled angular momenta are indicated by 
$\bi{l} = \bir \times \bip / \hbar $ and 
$\bi{j} = \bi{l} + \bis$, respectively. 

\section{Three-Body Hamiltonian}
We define the V-coordinates for three-body systems, 
similarly to other papers \cite{91Bert,97Esb,96Vinh}. 
The vector $\bir_i$ indicates the relative coordinates between the core 
nucleus and the $i$-th valence nucleon (see Fig.\ref{fig_3bs}). 
We subtract the center of mass motion of the whole system. 
Thus, apart from the spin variables, 
we need two vectors, $\bir_1$ and $\bir_2$, to fully describe the system. 
The total Hamiltonian reads 
\beqa
 H_{\rm 3b} 
 &=& h_1 + h_2 + \frac{\bip_1 \cdot \bip_2}{A_{\rm c} m} + 
     v_{\rm N-N}(\xi_1, \xi_2 ), \label{eq:H3b} \\
 h_i 
 &=& \frac{\bip_i^2}{2\mu} + V_{\rm c-N}(\xi_i), 
\eeqa
where $h_i$ is the single particle (s.p.) Hamiltonian 
for the relative motion between 
the core and the $i$-th nucleon. 
$\mu \equiv m A_{\rm c} / (A_{\rm c}+1)$ is the 
reduced mass, where $m$ and $A_{\rm c}$ indicate the one-nucleon mass and 
the number of nucleons in the core, respectively. 
The diagonal component of the kinetic energy of the core is 
included in the s.p. Hamiltonians, $h_1+h_2$. 
On the other hand, the off-diagonal component, 
referred to as ``recoil term'' in the following, 
is taken into account as the third term in Eq.(\ref{eq:H3b}) \cite{97Esb,05Hagi}. 
See Appendix \ref{Ap_3body} for a derivation of this Hamiltonian. 

In the Hamiltonian, 
$V_{\rm c-N}$ is the interaction for the core-nucleon subsystem. 
On the other hand, $v_{\rm N-N}$ indicates the pairing interaction for 
the two valence nucleons \footnote{In this thesis, 
we use the subscript N to indicate 
``nucleon'' generally, whereas $p$ and $n$ mean 
``proton'' and ``neutron'', respectively.} \footnote{In this thesis, 
we do not treat a phenomenological three-body force.}. 
It should be mentioned that, 
even if the pairing interaction is zero, 
the pairing correlation does not vanish because of the recoil term, 
$\bip_1 \cdot \bip_2 /A_{\rm c} m$ 
\footnote{Phenomenologically, this correlation can be interpreted as that 
mediated by the core nucleus. }. 
We give explicit forms of these interactions in the next section. 
\begin{figure*}[h] \begin{center}
\fbox{\includegraphics[width = 0.8\hsize]{./FIG3011.eps}}
\caption{A schematic figure for the three-body model defined 
in the V-coordinates. } \label{fig_3bs}
\end{center} \end{figure*}



\section{Interactions}
In this thesis, we assume that the core-nucleon potential, 
$V_{\rm c-N}$, is spherical and does not depend on the spin variables. 
Apart from the Coulomb interaction for a valence proton, 
we employ the Woods-Saxon potential including the spin-orbit coupling term. 
\beqa
 V_{\rm c-N, Nucl.} (r) 
 &=& \left[ V_0 + V_{ls} r_0^2 (\bi{\ell} \cdot \bi{s}) 
     \frac{1}{r} \frac{d}{dr} \right] f(r), \label{eq:WSP} \\
 &=& \left[ V_0 + V_{ls} r_0^2 
     \left( \frac{j(j+1)-l(l+1)-3/4}{2} \right) 
     \frac{1}{r} \frac{d}{dr} \right] f(r)
\eeqa
with 
\beq
  f(r) = \frac{1}{ 1 + \exp \left( \frac{r-R_{\rm core}}{a_{\rm core}} \right) }, 
\eeq
where $R_{\rm core} = r_0 A_{\rm c}^{1/3}$ is the radius of the core nucleus. 
In the core-proton case, in addition, 
the Coulomb potential of a uniform-charged sphere, whose 
radii and charge are $R_{\rm core}$ and $Z_{\rm c} e$, respectively, is 
also employed. 
\beq
 V_{\rm c-p, Coul.} (r) 
 = \left\{ \begin{array}{cc} 
   \frac{Z_{\rm c} e^2}{4\pi \epsilon_0} \frac{1}{r} & (r > R_{\rm core}) \\
   \frac{Z_{\rm c} e^2}{4\pi \epsilon_0} \frac{1}{2R_{\rm core}} \left( 3 - \frac{r^2}{R_{\rm core}^2} \right) & (r \leq R_{\rm core}) 
   \end{array} \right. \label{eq:cpCoul}
\eeq
Thus, the total core-proton potential is given as 
\beq
 V_{{\rm c-p},lj} = V_{\rm c-p, Nucl.} (r) + V_{\rm c-p, Coul.} (r). 
\eeq
There are four parameters in the core-nucleon potential, 
namely $V_0, V_{ls}, r_0$ and $a_{\rm core}$. 
We determine the values of these parameters for each system, 
as we will explain in 
Chapter \ref{Ch_Results1} and \ref{Ch_Results2}. 

On the other hand, for the nucleon-nucleon pairing interaction, 
we adopt the phenomenological ``density-dependent contact (DDC)'' 
interaction. 
It is formulated as 
\beq
 v_{\rm N-N, Nucl.} (\bir_1,\bir_2) 
 = \delta(\bir_1-\bir_2) 
   \left[ v_0 + 
          \frac{v_{\rho}} 
               {1 + \exp \left( \frac{\abs{(\bir_1+\bir_2)/2}-R_{\rho}}{a_{\rho}} \right)} 
   \right]. \label{eq:DDCP}
\eeq
The first term, $v_0 \delta(\bir_1-\bir_2)$, indicates the nucleon-nucleon 
interaction in vacuum, which is 
approximated to have the zero range. 
The second term is a phenomenological density-dependent part 
which is assumed as the Woods-Saxon form. 
This type of pairing interaction has been employed 
in several nuclear structural 
calculations, with a great advantage that it can dramatically 
reduce the computational cost. 
These calculations have provided reasonable 
results \cite{84Catara,91Bert,97Esb,05Hagi,07Hagi_01}, 
despite the simple form of the pairing interaction. 
Especially, within the three-body model with 
DDC pairing, 
the binding energies and the Borromean properties explained in 
the previous Chapter have been well reproduced 
for $^6$He and $^{11}$Li \cite{91Bert,97Esb,05Hagi}. 
In the case with two protons, 
we also take the Coulomb potential into account. 
\beq
  v_{\rm p-p, Coul.} (\bir_1,\bir_2) \label{eq:ppCoul}
  = \frac{e^2}{4\pi \epsilon_0}  \frac{1}{\abs{\bir_1-\bir_2}}. 
\eeq
For the nuclear part of the pairing interaction, 
there are four parameters in Eq.(\ref{eq:DDCP}), namely 
$v_0, v_{\rho}, R_{\rho}, a_{\rho}$. 
The strength of the bare nucleon-nucleon potential, $v_0$, can be 
defined by solving the nucleon-nucleon scattering problem with 
the bare contact interaction, $v_0 \delta(\bir_1-\bir_2)$. 
As well known, this contact interaction must be treated in a 
truncated space defined with the energy cutoff, 
$\epsilon_{\rm cut}$, or 
it loses physical meanings. 
The strength of the bare interaction, $v_0$, can be 
determined so as to reproduce the empirical scattering length 
$a_{\rm N-N}$ in the nucleon-nucleon scattering \cite{97Esb}. 
For a given cutoff $\epsilon_{\rm cut}$, 
this is formulated as 
\beq
 v_0 = \frac{2\pi^2\hbar^2}{m}\frac{2a_{\rm N-N}}{\pi-2a_{\rm N-N} k_{\rm cut}}, \label{eq:v0}
\eeq
where the relative maximum momentum of two nucleons, $k_{\rm cut}$, is defined as 
\beq
 k_{\rm cut} = \sqrt{{m \epsilon_{\rm cut}}/{\hbar^2}}. 
\eeq
A discussion and derivation of Eq.(\ref{eq:v0}) are summarized as 
Appendix \ref{Ap_Scat_Contact}. 
The empirical scattering length for a neutron-neutron scattering is 
$a_{\rm n-n}=-18.5$ fm \cite{05Baumer}, 
while that for a proton-proton scattering is $a_{\rm p-p}=-7.81$ 
fm \cite{88Berg}. 
The difference between $a_{\rm n-n}$ and $a_{\rm p-p}$ is mainly due to 
the Coulomb repulsion in a two-proton system. 
Since we explicitly include the Coulomb interaction in our calculations, 
we use the neutron-neutron scattering length $a_{\rm n-n}$ to determine the 
strength of the bare interaction, Eq.(\ref{eq:v0}), assuming the 
charge independence of nuclear force. 

Once $v_0$ is determined in this way, the remaining parameters in the 
density-dependent term, $v_\rho, R_\rho$, and $a_\rho$ are adjusted 
to reproduce the three-body binding energy of the considering system. 
The concrete values of these parameters 
used in the actual calculations 
are given in Chapter \ref{Ch_Results1} and \ref{Ch_Results2}. 



\section{Single-Particle States}
In order to describe an arbitrary wave function, 
the basis expansion is a popular method. 
We use this method in our three-body problems. 
As the first step, 
we solve the partial core-nucleon wave functions. 
Because we assumed that the core-nucleon potential, 
$V_{\rm c-N}$, is spherical and does not depend on the spin variables, 
the corresponding \Schr equation reads 
\beq
 h_i \phi_{nljm} (\xi_i) = \epsilon_{nlj} \phi_{nljm} (\xi_i). 
 \label{eq:speq} 
\eeq
Here we indicate the radial quantum numbers, 
quantum numbers of 
orbital angular momenta and of 
coupled angular momenta as 
$n,l$ and $j$, respectively. 
We also need $m$ to indicate the magnetic quantum number. 
The wave function of the single nucleon can be separated 
into the radial and the angular parts as 
\beq
 \phi_{nljm} (\xi) = R_{nlj}(r) \mathcal{Y}_{ljm} (\ubir, \bis) 
 = \frac{U_{nlj} (r)}{r} \mathcal{Y}_{ljm} (\ubir, \bis). 
\eeq
where $\ubir \equiv (\theta, \phi)$. 
The function $\mathcal{Y}_{ljm}$ indicates the composite angular part 
of $\bi{j} = \bi{l} + \bis$ coupled to $(j,m)$, that is 
\beq
 \hat{\bi{j}}^2 \mathcal{Y}_{ljm} = j(j+1) \mathcal{Y}_{ljm}, ~~~~~~ 
      \hat{j_z} \mathcal{Y}_{ljm} = m \mathcal{Y}_{ljm}. 
\eeq
Using the \CG coefficients, its explicit form is given as 
\beqa
 \mathcal{Y}_{ljm} (\ubir, \bis) 
 &\equiv & \Braket{ \ubir, \bis \mid (l \oplus 1/2)\phantom{0} j,m} \\
 &=&       \sum_h \sum_v \cgc{j,m}{l,h;1/2,v} Y_{lh} (\ubir) \chi_v (\bis), 
\eeqa
where $Y_{lh}$ and $\chi_{v}$ satisfy 
\beqa
 \hat{\bi{l}}^2 Y_{lh} (\ubir) = l(l+1) Y_{lh}(\ubir), 
 && ~~ \hat{l}_z Y_{lh} (\ubir) = h Y_{lh} (\ubir), \\
 \hat{\bis}^2 \chi_v (\bis) = \frac{3}{4} \chi_v (\bis), 
 && ~~ \hat{s}_z \chi_v (\bis) = v \chi_v (\bis), 
\eeqa
with $h = -l\sim l$ and $v = \pm 1/2$. 
Then Eq.(\ref{eq:speq}) can be reduced to the equation only for the 
radial part $R_{nlj}$. 
That is 
\beq
 \left[ -\frac{\hbar^2}{2\mu} 
        \left( \frac{1}{r} \frac{d^2}{dr^2} r - \frac{l(l+1)}{r^2} \right)
 + V_{{\rm c-N},lj}(r) \right] 
 R_{nlj}(r) = \epsilon_{nlj} R_{nlj}(r), 
\eeq
or equivalently for $U_{nlj}(r)=rR_{nlj}$, 
\beq
 \left[ \frac{d^2}{dr^2} - \frac{l(l+1)}{r^2} 
      - \frac{2\mu}{\hbar^2} 
        \left(  V_{{\rm c-N},lj}(r) - \epsilon_{nlj} \right) 
 \right] U_{nlj}(r) = 0. 
\eeq
In this thesis, 
we solve the radial part $U_{nlj}$ numerically within a discrete 
variable-domain. 
Assuming a radial box with its size $R_{\rm box}$, 
sampling points are distributed in the interval $0 \sim R_{\rm box}$ 
where the distance between two consecutive points is $dr = r_{n-1}-r_n$. 
For the continuum s.p. states with 
$\epsilon_{nlj} > V_{{\rm c-N},lj}(r \to \infty) \equiv 0$, 
we assume the boundary condition with a vanishing wave function 
at $r=R_{\rm box}$. 
That is 
\beq
 U_{nlj} (r=R_{\rm box}) = 0. 
\eeq
Because of this boundary condition, 
the continuum energy spectrum is discretized. 
Either for the bound and the discretized continuum s.p. states, 
their radial wave functions can be calculated numerically. 
The numerical method we employ in this thesis is ``Numerov method'', 
which was developed by B. V. Numerov \cite{93Hairer}. 
A detailed introduction of this method is separately 
summarized as Appendix \ref{Ap_Numerov}. 

\section{Uncorrelated Basis for Two Nucleons}
Using s.p. wave functions $\{ \phi_{nljm} \}$ obtained 
in the previous section, 
the ``uncorrelated basis'' for two-nucleon states can be constructed. 
If two nucleons are coupled to the spin $(J,M)$, 
the uncorrelated states are formulated as 
\beqa
 \Psi^{(J,M)}_{ab} (\xi_1, \xi_2) &=& \Psi^{(J,M)}_{(nlj)_a (nlj)_b} (\xi_1, \xi_2) 
 \equiv \left[ \phi_{(nlj)_a} (\xi_1) \otimes 
                  \phi_{(nlj)_b} (\xi_2) \right]^{(J,M)} \\
 &=& R_{(nlj)_a}(r_1) R_{(nlj)_b}(r_2) \cdot \label{eq:uncbs_03} 
     W^{(J,M)}_{ab} (\ubir_1 \bis_1, \ubir_2 \bis_2), 
\eeqa
where we define the shortened subscripts $(nlj)_a \equiv (n_a,l_a,j_a)$. 
The coupled angular part, $W^{(J,M)}_{ab}$, is defined as 
\beqa
 && W^{(J,M)}_{ab} (\ubir_1 \bis_1, \ubir_2 \bis_2) \label{eq:couplang} 
    \equiv \Braket{ \ubir_1 \bis_1, \ubir_2 \bis_2 \mid (j_a \oplus j_b) ~ J,M } \\
 && ~~~~ = \sum_{m_a,m_b} \cgc{J,M}{j_a,m_a;j_b,m_b} 
           \mathcal{Y}_{(ljm)_a} (\ubir_1,\bis_1) \mathcal{Y}_{(ljm)_b} (\ubir_2,\bis_2). 
\eeqa
This function means that the first and the second valence nucleons 
are in the core-nucleon orbits labeled by 
$(n_a,l_a,j_a)$ and $(n_b,l_b,j_b)$, respectively. 
In actual calculations, we also add another constraint of the 
total parity, by including only those configurations with the 
same value of 
$\pi = (-)^{l_a+l_b}$ in defining basis. 
For two nucleons of the same kind, 
we have to take the anti-symmetrization into account. 
That is 
\beq
 \tilde{\Psi}^{(J,M)}_{ab} (\xi_1, \xi_2) 
 \equiv A_{ab} \label{eq:uncorrebasis}
        \left[ \Psi^{(J,M)}_{ab} (\xi_1,\xi_2) 
             - \Psi^{(J,M)}_{ab} (\xi_2,\xi_1) \right], 
\eeq
where $A_{ab}$ is the normalization factor. 
This is given as 
\beq
 A_{ab} = \label{eq:Aab} \left\{ \begin{array}{cc} 
                1/2 & (n_a=n_b \cap l_a=l_b \cap j_a=j_b) \\
         1/\sqrt{2} & ({\rm otherwise}) \end{array} \right. 
\eeq
If we write it explicitly, Eq.(\ref{eq:uncorrebasis}) is given as 
\beqa
 \tilde{\Psi}^{(J,M)}_{ab} (\xi_1, \xi_2) 
 &=& A_{ab} \left[ R_{(nlj)_a}(r_1) R_{(nlj)_b}(r_2) \cdot W^{(J,M)}_{ab} (\ubir_1 \bis_1, \ubir_2 \bis_2) \right. \nonumber \\
 & & \; \; -\left. R_{(nlj)_a}(r_2) R_{(nlj)_b}(r_1) \cdot W^{(J,M)}_{ab} (\ubir_2 \bis_2, \ubir_1 \bis_1) \right] . \label{eq:53548}
\eeqa
Using the formula of the \CG coefficients; 
\beq
 \cgc{J,M}{j_a,m_a;j_b,m_b} = (-)^{j_a+j_b-J} \cgc{J,M}{j_b,m_b;j_a,m_a}, 
\eeq
the coupled angular part of the second term in Eq.(\ref{eq:53548}) 
can be transformed to 
\beqa
 && W^{(J,M)}_{ab} (\ubir_2 \bis_2, \ubir_1 \bis_1) \nonumber \\
 && ~~~~ = \sum_{m_a,m_b} \cgc{J,M}{j_a,m_a;j_b,m_b} 
    \mathcal{Y}_{(ljm)_a} (\ubir_2,\bis_2) \mathcal{Y}_{(ljm)_b} (\ubir_1,\bis_1) \nonumber \\
 && ~~~~ = (-)^{j_a+j_b-J} \sum_{m_a,m_b} \cgc{J,M}{j_b,m_b;j_a,m_a} 
    \mathcal{Y}_{(ljm)_a} (\ubir_2,\bis_2) \mathcal{Y}_{(ljm)_b} (\ubir_1,\bis_1) \nonumber \\
 && ~~~~ = (-)^{j_a+j_b-J} W^{(J,M)}_{ba} (\ubir_1 \bis_1, \ubir_2 \bis_2). 
\eeqa
Thus we obtain another formula for $\tilde{\Psi}^{(J,M)}_{ab}$. 
\beq
 \tilde{\Psi }^{(J,M)}_{ab} (\xi_1, \xi_2) \label{eq:UNCB}
 = A_{ab} \left[ \Psi^{(J,M)}_{nlj(a),nlj(b)} (\xi_1,\xi_2) 
        - B_{ab} \Psi^{(J,M)}_{nlj(b),nlj(a)} (\xi_1,\xi_2) \right] 
\eeq
with $B_{ab} \equiv (-)^{j_a+j_b-J}$. 
Notice that $\tilde{\Psi}^{(J,M)}_{ab} $ is an eigenstate of the uncorrelated 
Hamiltonian, $h_1 + h_2$. 
Its eigen-equation reads 
\beq
 (h_1 + h_2) \tilde{\Psi}^{(J,M)}_{ab} = (\epsilon_a + \epsilon_b) \tilde{\Psi}^{(J,M)}_{ab}, 
\eeq
where $\epsilon_a$ and $\epsilon_b$ are the eigen-energies of the first and the second 
orbits, respectively. 
%\beq
% A_{ab} = \frac{1}{\sqrt{2}} \left[ 1 - (-)^{j_a+j_b-J} \delta_{nlj(a),nlj(b)} \right]^{-1/2}. 
%\eeq
%In the following, we often use the simplified labels for the uncorrelated basis, 
%namely, 
%\beq
% \tilde{\Psi}^{(J,M)}_{P} \equiv \tilde{\Psi}^{(J,M)}_{ab} 
% = \tilde{\Psi}^{(J,M)}_{(nlj)_a, (nlj)_b}. 
%\eeq

We can now expand an arbitrary two-nucleon state 
with $(J,M)$ on the uncorrelated basis. 
That is, 
\beq
 \Phi^{(J,M)} (\xi_1,\xi_2) = \label{eq:exp_ucb} 
 \sum_{a\leq b} \alpha_{ab} \tilde{\Psi}^{(J,M)}_{ab} (\xi_1,\xi_2). 
\eeq
where our model-space is truncated by the cutoff energy for the 
uncorrelated basis, defined as 
$E_{\rm cut}=\epsilon_{\rm cut}(A_{\rm c}+1)/A_{\rm c}$ \cite{97Esb}. 
In practice, we have to introduce also 
the cutoff angular momentum, $l_{\rm cut}$, 
in addition to $E_{\rm cut}$. 
Notice that Matsuo {\it et.al.} have shown 
that the spatial localization cannot 
be reproduced theoretically unless one includes a sufficient number 
of angular momenta. 
Referring to their results, we would have to employ the model-space 
with, at least, up to $l_{\rm max}=5$ in order to take the dinucleon 
correlations into account. 

In the following, for simplicity, 
we omit the subscripts $(J,M)$ unless it is needed. 
For the eigenstates of $H_{\rm 3b}$, namely 
$H_{\rm 3b} \ket{\Phi_N } = E_N \ket{\Phi_N }$, 
the expansion coefficients $\{ \alpha_{ab} \}$ can be obtained 
by diagonalizing the Hamiltonian matrix. 
In the next section, we detail how to calculate these matrix elements. 





\section{Matrix Elements with Uncorrelated Basis}
First, for the uncorrelated Hamiltonian, $h_1 + h_2$, 
the matrix elements are trivially given as 
\beq
 \Braket{ \tilde{\Psi}_{cd} \mid (h_1+h_2) \mid \tilde{\Psi}_{ab} } 
 = (\epsilon_a + \epsilon_b) \delta_{cd,ab} , 
\eeq
where $\epsilon_a \equiv \epsilon_{n_a l_a j_a}$. 
For the other parts of the Hamiltonian, 
we need much complicated calculations in general. 
A matrix element (ME) of an arbitrary operator, $\oprt{O}$, 
is decomposed into four terms, 
\beqa
 \Braket{\tilde{\Psi}_{cd} \mid \oprt{O} \mid \tilde{\Psi}_{ab}} 
 &=& A_{cd}A_{ab} \left[ \Braket{ \Psi_{cd} \mid \oprt{O} \mid \Psi_{ab}} 
     + B_{cd}B_{ab} \Braket{ \Psi_{dc} \mid \oprt{O} \mid \Psi_{ba}} \right. \nonumber \\
 & & \left. -B_{ab} \Braket{ \Psi_{cd} \mid \oprt{O} \mid \Psi_{ba}} 
            -B_{cd} \Braket{ \Psi_{dc} \mid \oprt{O} \mid \Psi_{ab}} \right] \label{eq:ME01}
\eeqa
where we have applied Eq.(\ref{eq:UNCB}). 
In the following subsections, we explain how to calculate MEs of several 
important operators. 

\subsection{Single Particle Operators}
This kind of operators is characterized as 
$\oprt{O} = O(\xi_1) + O(\xi_2)$. 
These include the core-nucleon interaction $V_{\rm c-N}(\xi_i)$, 
the s.p. kinetic energy $\bip^2_i / 2\mu = h_i - V_{\rm c-N}(\xi_i)$, 
the radial distance $\abs{\bir_i}^2$, and so on. 
For the operator $\oprt{O}(1) = O(\xi_1)$ which acts only 
on the first particle, 
the first term in Eq.(\ref{eq:ME01}) has the form of 
\beqa
 \Braket{ \Psi_{cd} \mid \oprt{O}(1) \mid \Psi_{ab}} 
 &=& \int d\xi_1 \int d\xi_2 
     \Psi_{cd}^* (\xi_1,\xi_2) O(\xi_1) \Psi_{ab} (\xi_1,\xi_2) \\
 &=& \delta_{d,b} \sum_{m_c,m_a} \nonumber 
     \cgc{J,M*}{j_d,M-m_c;j_c,m_c} \cgc{J,M}{j_b,M-m_a;j_a,m_a} \\
 & & \phantom{000} \times 
     \int d\xi_1 \phi^*_{(nljm)_c}(\xi_1) O(\xi_1) \phi_{(nljm)_a}(\xi_1), 
\eeqa
which vanishes if $n_b \neq n_d \cup l_b \neq l_d \cup j_b \neq j_d$. 
Thus, the only quantity we have to calculate is the integration 
in the last sentence. 
The other terms in Eq.(\ref{eq:ME01}) can be calculated similarly. 
Defining the following symbol; 
\beqa
 O_{ca}^{(1)} &\equiv& \sum_{m_c,m_a} \nonumber 
 \cgc{J,M*}{j_d,M-m_c;j_c,m_c} \cgc{J,M}{j_b,M-m_a;j_a,m_a} \\
 & & \phantom{000} \times \label{eq:spME0}
     \int d\xi_1 \phi^*_{(nljm)_c}(\xi_1) O(\xi_1) \phi_{(nljm)_a}(\xi_1), 
\eeqa
we can represent the matrix element of $\oprt{O}(1) = O(\xi_1)$ 
after the anti-symmetrization as follows. 
\beqa
 \Braket{\tilde{\Psi}_{cd} \mid \oprt{O}(1) \mid \tilde{\Psi}_{ab}} 
 &=& A_{cd}A_{ab} \left[ \delta_{db} O_{ca}^{(1)} 
          + B_{cd}B_{ab} \delta_{ca} O_{bd}^{(1)} \right. \nonumber \\
 & & \left.      -B_{ab} \delta_{cb} O_{da}^{(1)} 
                 -B_{cd} \delta_{da} O_{cb}^{(1)} \right]. \label{eq:spME1}
\eeqa
We also derive the similar formula for the summation of 
$\oprt{O}(1)$ and $\oprt{O}(2)$. 
The result reads 
\beqa
 & & \Braket{\tilde{\Psi}_{cd} \mid \oprt{O}(1)+\oprt{O}(2) \mid \tilde{\Psi}_{ab}} \nonumber \\
 & & \phantom{000} = A_{cd}A_{ab} \nonumber 
     \left[  \delta_{db} (O_{ca}^{(1)} + B_{cd}B_{ab}O_{ca}^{(2)})
            +\delta_{ca} (B_{cd}B_{ab}O_{db}^{(1)} + O_{db}^{(2)}) \right. \\
 & & \phantom{000000000000} 
     \left. -\delta_{cb} (B_{cd}O_{da}^{(1)} + B_{ab}O_{da}^{(2)}) 
            -\delta_{da} (B_{ab}O_{cb}^{(1)} + B_{cd}O_{cb}^{(2)}) \right]. 
\eeqa
We will use this formula to calculate, {\it e.g.} those of $h_1+h_2$ 
or $V_{\rm c-N}(\xi_1)+V_{\rm c-N}(\xi_2)$. 



If the operator is spherical; $O(\xi_1)=O(r_1)$, Eq.({\ref{eq:spME0}}) can be 
reduced as the integration only for the radial distance. 
\beqa
 O_{ca}^{(1),spherical} \nonumber 
 &=& \delta_{j_c j_a} \delta_{l_c l_a} \int dr_1 r_1^2 R_{(nlj)_c}(r_1) O(r_1) R_{(nlj)_a}(r_1), \\
 &=& \delta_{j_c j_a} \delta_{l_c l_a} \int dr_1 U_{(nlj)_c}(r_1) O(r_1) U_{(nlj)_a}(r_1), 
\eeqa
where the product of coupled angular parts is given as 
$\delta_{j_c,j_a} \delta_{l_c,l_a} $. 



\subsection{Two-Particle Operators}
The Operators in this category are given as 
$\oprt{O} = O(\xi_1,\xi_2)$. 
These include, for instance, the pairing interaction, $v_{\rm N-N}$, 
the recoil term, $\bip_1 \cdot \bip_2 / A_{\rm c} m$, and the 
opening angle between 
two nucleons, $\cos \theta_{12}$. 
In order to calculate Eq.(\ref{eq:ME01}), it is often necessary to know 
the following quantity. 
\beq
 \Braket{\mathcal{Y}_{(ljm)_c} \mid Y_{lh} \mid \mathcal{Y}_{(ljm)_a}} 
 = \int d\ubir \int d\bis \mathcal{Y}^*_{(ljm)_c}(\ubir,\bis) Y_{lh}(\ubir) \mathcal{Y}_{(ljm)_a}(\ubir,\bis)
\eeq
For this purpose, one can use the Wigner-Eckart 
theorem found in, {\it e.g.} the textbook 
by Edmonds \cite{60Edm}, 
\beqa
 && \Braket{\mathcal{Y}_{(ljm)_c} \mid Y_{lh} \mid \mathcal{Y}_{(ljm)_a}} \nonumber \\
 && ~~~ = (-)^{j_a-m_a-2h} \frac{\cgc{l,h}{j_c,m_c;j_a,m_a} }{\sqrt{2l+1}} 
    \Braket{j_c(l_c,1/2) \Vert Y_l \Vert j_a(l_a,1/2)}, 
\eeqa
where the reduced matrix element is written with the $6j$-symbols as 
\beqa
 \Braket{j_c(l_c,1/2) \Vert Y_l \Vert j_a(l_a,1/2)} &=& (-)^{l_c+l_a+j_a+l} \sqrt{(2j_c+1)(2j_a+1)} \nonumber \\
 & & \times \left\{ \begin{array}{ccc} l_c & j_c & 1/2 \\ j_a & 1/2 & l \end{array} \right\} 
     \Braket{l_c \Vert Y_l \Vert l_a}, \nonumber \\
 \Braket{l_c \Vert Y_l \Vert l_a} 
 &=& (-)^{l_c+l} \frac{\sqrt{(2l_c+1)(2l_a+1)}}{4\pi} \nonumber 
     \cgc{l,0}{l_c,0;l_a;0}. 
\eeqa
When the operator is scalar and does not include spin variables, 
it can be generally represented by the multi-pole expansion. 
Namely, 
\beq
 O(\bir_1,\bir_2) = \sum_{l=0}^{\infty} O_l(r_1,r_2) \sum_{h=-l}^{l} 
 Y_{l,h} (\ubir_1) \cdot (-)^h Y_{l,-h} (\ubir_2). \label{eq:39634}
\eeq
Then, we can formulate each component in Eq.(\ref{eq:ME01}). 
For the $l$-th term in Eq.(\ref{eq:39634}), 
the radial part becomes 
\beqa
 && rad.part_{(l)} \left[ \Braket{\Psi_{cd} \mid \oprt{O}(1,2) \mid \Psi_{ab}} \right] \nonumber \\
 && ~= \iint dr_1 dr_2 
       R^*_{(nlj)_c}(r_1) R^*_{(nlj)_d}(r_2) O_l(r_1,r_2) R_{(nlj)_a}(r_1) R_{(nlj)_b}(r_2), 
\eeqa
whereas the angular part is given as 
\beqa
 && ang.part_{(l)} \left[ \Braket{\Psi_{cd} \mid \oprt{O}(1,2) \mid \Psi_{ab}} \right] \nonumber \\
 && ~~~= \left< (j_c\oplus j_d)J,M \mid \sum_h Y_{lh}(1) (-)^h Y_{l,-h}(2) \mid (j_a\oplus j_b)J,M \right> \nonumber \\
 && ~~~= \sum_{all~m} \cgc{*J,M}{j_c,m_c;j_d,m_d} \cgc{J,M}{j_a,m_a;j_b,m_b} \nonumber \\ 
 && ~~~~~~~ \sum_h \Braket{\mathcal{Y}_{(ljm)_c} \mid Y_{l, h} \mid \mathcal{Y}_{(ljm)_a}} 
            (-)^h  \Braket{\mathcal{Y}_{(ljm)_d} \mid Y_{l,-h} \mid \mathcal{Y}_{(ljm)_b}}. \label{eq:ME12a} 
\eeqa
By performing a few calculations for the angular momenta, 
Eq.(\ref{eq:ME12a}) can be simplified as 
\beqa
 && ang.part_{(l)} \left[ \Braket{\Psi_{cd} \mid \oprt{O}(1,2) \mid \Psi_{ab}} \right] \nonumber \\
 && ~~ = (-)^{j_a+j_d-J} 
    \left\{ \begin{array}{ccc} j_a & l & j_c \\ j_d & J & j_b \end{array} \right\} \nonumber \\
 && \phantom{00000} \times \Braket{j_c(l_c,1/2) \Vert Y_l \Vert j_a(l_a,1/2)} \cdot \Braket{j_d(l_d,1/2) \Vert Y_l \Vert j_b(l_b,1/2)}, 
\eeqa
where the summations over the magnetic quantum numbers do not appear \cite{60Edm}. 
Consequently, we can write down the general formula for the ME of 
a two-particle operator as 
\beqa
 && \Braket{\Psi_{cd} \mid \oprt{O}(1,2) \mid \Psi_{ab}} \nonumber \\
 && ~~ = \sum_l \iint dr_1 dr_2 \nonumber 
       R^*_{(nlj)_c}(r_1) R^*_{(nlj)_d}(r_2) O_l(r_1,r_2) R_{(nlj)_a}(r_1) R_{(nlj)_b}(r_2) \\
 && \phantom{00000} \times (-)^{j_a+j_d-J} \left\{ \begin{array}{ccc} j_a & l & j_c \\ j_d & J & j_b \end{array} \right\} \nonumber \\
 && \phantom{0000000} \Braket{j_c(l_c,1/2) \Vert Y_l \Vert j_a(l_a,1/2)} \cdot \Braket{j_d(l_d,1/2) \Vert Y_l \Vert j_b(l_b,1/2)}. \label{eq:ME12g}
\eeqa
We mention that the orbital angular momenta, $l$, 
must be truncated in actual calculations. 
Thus, the summation over $l$ is also truncated as 
$\sum_{l=0}^{\infty} \rightarrow \sum_{l=0}^{l_{\rm max}}$. 

We also mention how to derive the $O_{l}(r_1,r_2)$. 
For the pairing interaction, the two-particle operator 
depends only on the relative distance, 
\beq
 r_{12}=\abs{\bir_1-\bir_2}=\sqrt{r_1^2+r_2^2-2r_1r_2\cos \theta_{12}}. 
\eeq
The multi-pole expansion for an arbitrary function of $r_{12}$ satisfies 
\beqa
 f(r_{12}) &=& \sum_{l=0}^{l_{\rm max}} (2l+1) g_l (r_1,r_2) P_{l} (\cos \theta_{12}) \\
 &=& \sum_{l=0}^{l_{\rm max}} g_l (r_1,r_2) 4\pi \sum_{h=-l}^{l} 
     Y_{l,h}(\ubir_1) (-)^h Y_{l,-h}(\ubir_2), 
\eeqa
with 
\beq
  g_l (r_1,r_2) = \frac{1}{2} \int_0^{\pi} f(r_{12}) P_{l} (\cos \theta_{12}) \sin \theta_{12} d\theta_{12}, 
\eeq
where $P_l$ is the Legendre polynomial. 
We list below concrete forms of the functions used for 
the pairing interaction. 
\begin{enumerate}
\item a delta function; 
      \beq
        f(r_{12}) = \delta(\abs{\bir_1-\bir_2}) = \frac{\delta(r_1-r_2)}{r_1 r_2} \sum_l \sum_{h=-l}^{l} Y_{l,h}(\ubir_1) (-)^h Y_{l,-h}(\ubir_2). 
      \eeq
\item an inverse function; 
      \beq
        f(r_{12}) = \frac{1}{\abs{\bir_1-\bir_2}} = \sum_l \frac{r_<^l}{r_>^{l+1}} \frac{4\pi}{2l+1} \sum_{h=-l}^{l} Y_{l,h}(\ubir_1) (-)^h Y_{l,-h}(\ubir_2), 
      \eeq
      where $r_>$ ($r_<$) indicates the larger (smaller) one between $r_1$ and $r_2$. 
\end{enumerate}
On the other hand, for the recoil term; $\bip_1 \cdot \bip_2 / A_{\rm c} m$, 
we first use the formula of the spatial differentiation, that is 
\beq
 \frac{1}{2} \left[ \nabla^2, \bir \right] = \nabla \Longleftrightarrow 
 \nabla = \ubir \left( \frac{d}{dr} + \frac{1}{r} \right) - \frac{1}{2r} \left[ \hat{\bi{l}}^2,\ubir \right]. 
\eeq
Thus, for the product $\nabla_1 \cdot \nabla_2$, 
its ME before the anti-symmetrization takes the form of 
\beqa
 && \Braket{\Psi_{cd} | \nabla_1 \cdot \nabla_2 | \Psi_{ab}} \nonumber \\
 && ~~~ = \left< R_c R_d \mid \left\{ (\frac{d}{dr_1}+\frac{1}{r_1}) - \frac{1}{2r_1}(l_c(l_c+1)-l_a(l_a+1)) \right\} \right. \nonumber \\
 && ~~~~~~~~ \left. \left\{ (\frac{d}{dr_2}+\frac{1}{r_2}) - \frac{1}{2r_2}(l_d(l_d+1)-l_b(l_b+1)) \right\} \mid R_a R_b \right> \nonumber \\
 && ~~~~~ \times \Braket{W_{cd} \mid \ubir_1 \cdot \ubir_2 \mid W_{ab}}, 
\eeqa
where the radial part can be calculated with the first derivatives. 
In the angular part, we expand the function, $\ubir_1 \cdot \ubir_2$, 
as follows. 
\beq
 \ubir_1 \cdot \ubir_2 = \cos \theta_{12} = P_{l=1}(\cos \theta_{12}) 
 = \frac{4\pi}{3} \sum_{h=-1}^{1} Y_{1,h} (\ubir_1) (-)^h Y_{1,-h} (\ubir_2). 
\eeq
Consequently, the MEs of this operator can be calculated 
by means of the dipole expansion. 
\beqa
 && \Braket{W_{cd} \mid \ubir_1 \cdot \ubir_2 \mid W_{ab}} \nonumber \\
 && ~~ = \frac{4\pi}{3} (-)^{j_a+j_d-J} 
    \left\{ \begin{array}{ccc} j_a & 1 & j_c \\ j_d & J & j_b \end{array} \right\} \nonumber \\
 && \phantom{00000} \times \Braket{j_c(l_c,1/2) \Vert Y_1 \Vert j_a(l_a,1/2)} \cdot 
    \Braket{j_d(l_d,1/2) \Vert Y_1 \Vert j_b(l_b,1/2)}. \label{eq:3050}
\eeqa
Obviously, the recoil term mixes the uncorrelated basis which satisfy $\abs{l_c-l_a}=1$. 
If we limit the model space with $(-)^{l_a}=odd$ or $even$ only, 
the recoil term does not contribute. 


\section{Density Distribution}
We also derive the formulas for the density distributions of two nucleons. 
With our uncorrelated basis, the two-nucleon state can be expanded as 
Eq.(\ref{eq:exp_ucb}). 
Its density distribution is obviously given by 
\beqa
\rho (\xi_1,\xi_2) 
&=& \abs{\Phi(\xi_1,\xi_2)}^2 
    = \sum_{34} \sum_{12} \alpha_{34}^* \alpha_{12} 
      \tilde{\Psi}_{34}^* \cdot \tilde{\Psi}_{12} (\xi_1,\xi_2) \\
&=& \sum_{34} \sum_{12} \alpha_{34}^* \alpha_{12} A_{34} A_{12} \nonumber \\
& & \times \left[ \pi_{34,12} (\xi_1,\xi_2) + B_{34} B_{12} \pi_{43,21} 
  - B_{34} \pi_{43,12} -                      B_{12} \pi_{34,21} \right], \label{eq:rho_expand} 
\eeqa
with $B_{ab} \equiv (-)^{j_a+j_b-J}$. 
Each component $\pi_{cd,ab}(\xi_1,\xi_2)$ can be written as 
\beqa
  \pi_{34,12} (\xi_1,\xi_2) &\equiv& \Psi_{34}^* \cdot \Psi_{12} (\xi_1,\xi_2) \label{eq:denspi} \\
  &=& \left[ \sum_{m_3} \cgc{J,M}{j_3,m_3;j_4,M-m_3} 
            \phi_{(nljm)_3} (\xi_1) \phi_{(nljm)_4} (\xi_2) \right]^* \times \nonumber \\
  & & \left[ \sum_{m_1} \cgc{J,M}{j_1,m_1;j_2,M-m_1} 
            \phi_{(nljm)_1} (\xi_1) \phi_{(nljm)_2} (\xi_2) \right] \nonumber \\
  &=& R^*_{(nlj)_3}(r_1) R^*_{(nlj)_4}(r_2) \cdot R_{(nlj)_1}(r_1) R_{(nlj)_2}(r_2) \nonumber \\
  & & \times ~ W_{34}^{*(J,M)}(\ubir_1 \bis_1, \ubir_2 \bis_2) \cdot W_{12}^{(J,M)}(\ubir_1 \bis_1, \ubir_2 \bis_2). 
\eeqa
where we used the Eqs(\ref{eq:uncbs_03}) and (\ref{eq:couplang}). 



\section{Spin-Orbit Decomposition}
It will be also helpful to formulate the decomposition of two-nucleon states 
into those of the spin-singlet and triplet configurations. 
For this purpose, at first, we have to discuss some mathematics of angular momenta. 
In Eq.(\ref{eq:couplang}), to fix the final angular momentum, $(J,M)$, 
we first couple $\bi{l}_a$ and $\bis_1$ to $\bi{j}_a$, 
and then couple $\bi{j}_a$ and $\bi{j}_b$ to $\bi{J}$. 
That is, 
\beq
  \left. \begin{array}{c} 
    \bi{l}_a \oplus \bis_1 = \bi{j}_a \\ 
    \bi{l}_b \oplus \bis_2 = \bi{j}_b 
  \end{array} \right\} \longrightarrow \bi{j}_a \oplus \bi{j}_b = \bi{J}, 
\eeq
where $s_i=\abs{\bis_i}=1/2$. 
Within this coupling scheme, we got the coupled angular part, 
$W_{ab}^{(J,M)}=W_{l_a l_b j_a j_b}^{(J,M)}(\ubir_1 \bis_1, \ubir_2 \bis_2)$. 
On the other hand, another coupling scheme 
can be considered as 
\beq
  \left. \begin{array}{c} 
    \bi{l}_a \oplus \bi{l}_b = \bi{L} \\ 
      \bis_1 \oplus \bis_2   = \bi{S} 
  \end{array} \right\} \longrightarrow \bi{L} \oplus \bi{S} = \bi{J}.
\eeq
Those two coupling schemes can be related to each other by 
the unitary transformation. 
Namely, we can write down 
\beqa
 & & W_{l_a l_b j_a j_b}^{(J,M)}(\ubir_1 \bis_1, \ubir_2 \bis_2) \nonumber \\
 & & ~~~~~~~ = 
     \sum_{L=|l_a-l_b|}^{l_a+l_b} \sum_{S=0,1} 
     D_J (j_a j_b;l_a l_b s_1 s_2;LS) 
     \cdot \Xi_{l_a l_b LS}^{(J,M)}(\ubir_1 \bis_1, \ubir_2 \bis_2), 
\eeqa
with the $LS$-coupled angular part; 
\beqa
 & & \Xi_{l_a l_b LS}^{(J,M)}(\ubir_1 \bis_1, \ubir_2 \bis_2) 
     = \sum_{M_S = \pm 1} \cgc{J,M}{L,M-V ; S,M_S} \nonumber \\
 & & ~~~~~~~~~~~~~ \times 
     \left[ Y_{l_a}(\ubir_1) \otimes Y_{l_b} (\ubir_2) \right]^{(L,M-V)} 
     \left[ \chi(\bis_1) \otimes \chi(\bis_2) \right]^{(S,V)}, 
\eeqa
and the expansion coefficients including the $9j$-symbol; 
\beqa
 && D_J (j_a j_b;l_a l_b s_1 s_2;LS) \nonumber \\
 && ~~~~~~~ \equiv \sqrt{(2L+1)(2S+1)(2j_a+1)(2j_b+1)} 
    \left\{ \begin{array}{ccc} 
         l_a & l_b & L \\ 
         s_1 & s_2 & S \\
         j_a & j_b & J \end{array} \right\}. \label{eq:3iarg}
\eeqa
Using these formulas, the anti-symmetrized uncorrelated basis can be 
decomposed into the spin-singlet and triplet configurations as follows. 
\beqa
 \tilde{\Psi}_{ab} (\xi_1,\xi_2) &=& 
 \tilde{\Psi}_{ab,S=0} (\xi_1,\xi_2) + \tilde{\Psi}_{ab,S=1} (\xi_1,\xi_2), \\
 \tilde{\Psi}_{ab,S} (\xi_1,\xi_2) &=& 
     A_{ab} \left[ \Psi_{ab,S}(\xi_1,\xi_2) - B_{ab} \Psi_{ba,S}(\xi_1,\xi_2) \right], 
\eeqa
with 
\beqa
 && \Psi_{ab,S}(\xi_1,\xi_2) = R_{(nlj)_a}(r_1) R_{(nlj)_b}(r_2) \nonumber \\
 && ~~~~~~ \times \sum_{L=|l_a-l_n|}^{l_a+l_b} D_J (j_a j_b;l_a l_b s_1 s_2;LS) \cdot \Xi_{l_a l_b LS}^{(J,M)}(\ubir_1 \bis_1, \ubir_2 \bis_2). 
\eeqa
Notice that the normalization of each basis function reads 
\beqa
 1 &=& \int d\xi_1  \int d\xi_2  \abs{\tilde{\Psi}_{ab} (\xi_1,\xi_2)}^2 \\
   &=& \int d\bir_1 \int d\bir_2 \left\{ \abs{\tilde{\Psi}_{ab,S=0} (\bir_1,\bir_2)}^2 + \abs{\tilde{\Psi}_{ab,S=1} (\bir_1,\bir_2)}^2 \right\} \nonumber \\
   &=& \abs{A_{ab}}^2 \sum_S \sum_L \left\{ \abs{D_J (j_a j_b;l_a l_b s_1 s_2;LS)}^2 + \abs{D_J (j_a j_b;l_a l_b s_1 s_2;LS)}^2 \right. \nonumber \\
   & & -B_{ab} D^*_J(j_b j_a;l_b l_a s_2 s_1;LS) D_J(j_a j_b;l_a l_b s_1 s_2;LS) \delta_{n_b,n_a} \nonumber \\
   & & \left. -B_{ab} D^*_J(j_a j_b;l_a l_b s_1 s_2;LS) D_J(j_b j_a;l_b l_a s_2 s_1;LS) \delta_{n_b,n_a} \right\}, 
\eeqa
where the radial integrations in the cross terms become $\delta_{n_b,n_a}$. 
We also introduce a similar decomposition for the density distribution. 
Namely, Eq.(\ref{eq:denspi}) can be decomposed as 
\beqa
  \pi_{34,12} (\xi_1,\xi_2) 
  &\equiv & \Psi_{34}^* \cdot \Psi_{12} (\xi_1,\xi_2) \nonumber \\
  &=& R^*_{(nlj)_3}(r_1) R^*_{(nlj)_4}(r_2) \cdot R_{(nlj)_1}(r_1) R_{(nlj)_2}(r_2) \nonumber \\
  & & \times \sum_{L',S'} \left[ D_J(j_3 j_4;l_3 l_4 s_3 s_4 ;L'S') \cdot \Xi_{l_3 l_4 L'S'}^{(J,M)}(\ubir_1 \bis_1, \ubir_2 \bis_2) \right]^* \nonumber \\
  & & \times \sum_{L, S } \left[ D_J(j_1 j_2;l_1 l_2 s_1 s_2 ;L S ) \cdot \Xi_{l_1 l_2 L S }^{(J,M)}(\ubir_1 \bis_1, \ubir_2 \bis_2)\right], 
\eeqa
where $s_1 \sim s_4 = 1/2$. 
Substituting this equation into Eq.(\ref{eq:rho_expand}), 
we can also decompose 
the total density into the spin-singlet and triplet terms. 
The cross terms of the spin-singlet and triplet components are, indeed, 
irrelevant because those can be vanished by integrating over 
the spin variables. 
We use this technique in order to derive the spin-integrated density 
as we show in the next subsection. 

\subsection{Spin-Integrated Density}
In practice, we often need to integrate the density 
over the spin variables. 
From the orthogonality between the spin-singlet and 
triplet configurations, 
\beqa
 \Braket{S',M'_S|S,M_S} &=& \int d\bis_1 \int d\bis_2 
 \left[ \chi(\bis_1) \otimes \chi(\bis_2) \right]^{(S',M'_S) \dagger} 
 \left[ \chi(\bis_1) \otimes \chi(\bis_2) \right]^{(S ,M_S )} \nonumber \\
 &=& \delta_{S'S} \delta_{M'_S M_S}, 
\eeqa
a component of the spin-integrated density, 
$d_{34,12}(\bir_1,\bir_2)$, can be represented as 
\beqa
 && d_{34,12}(\bir_1,\bir_2) \equiv \int d\bis_1 \int d\bis_2 \pi_{34,12}(\xi_1,\xi_2), \label{eq:3117} \\
 && = R^*_{(nlj)_3}(r_1) R^*_{(nlj)_4}(r_2) \cdot R_{(nlj)_1}(r_1) R_{(nlj)_2}(r_2) \sum_{S=0,1} \sum_{M_S = -S}^{S} \nonumber \\
 && ~ \times \sum_{L'} \left[ D_J(j_3 j_4;l_3 l_4 s_3 s_4 ;L'S) \cgc{J,M}{L',M-M_S;S,M_S} \left[ Y_{l_3}(\ubir_1) \otimes Y_{l_4} (\ubir_2) \right]^{(L',M-M_S)} \right]^* \nonumber \\
 && ~ \times \sum_{L } \left[ D_J(j_1 j_2;l_1 l_2 s_1 s_2 ;L S) \cgc{J,M}{L ,M-M_S;S,M_S} \left[ Y_{l_1}(\ubir_1) \otimes Y_{l_2} (\ubir_2) \right]^{(L, M-M_S)} \right]. \nonumber \\ && \\
 && = d_{34,12,S=0}(\bir_1,\bir_2) + d_{34,12,S=1}(\bir_1,\bir_2). 
\eeqa
Therefore, we can finally formulate the spin-integrated density, 
$\rho(\bir_1,\bir_2)$, as below. 
\beqa
 \rho(\bir_1,\bir_2) &\equiv& \int d\bis_1 \int d\bis_2 \rho(\xi_1,\xi_2) \label{eq:rhod} \\
 &=& \int d\bis_1 \int d\bis_2 \sum_{cd} \sum_{ab} \alpha_{cd}^* \alpha_{ab} \tilde{\Psi}^*_{cd} \cdot \tilde{\Psi}_{ab} (\xi_1,\xi_2) \nonumber \\
 &=& \sum_{cd} \sum_{ab} \alpha_{cd}^* \alpha_{ab} A_{cd} A_{ab} \nonumber \\
 & & ~ \sum_{S=0,1} \left[ d_{cd,ab,S}(\bir_1,\bir_2) + B_{cd} B_{ab} d_{dc,ba,S}(\bir_1,\bir_2) \right. \nonumber \\
 & & ~~~ \left.   - B_{cd} d_{dc,ab,S}(\bir_1,\bir_2) -        B_{ab} d_{cd,ba,S}(\bir_1,\bir_2) \right], \\
 &=& \rho_{S=0}(\bir_1,\bir_2) + \rho_{S=1}(\bir_1,\bir_2) \label{eq:rhos01}. 
\eeqa
Note that the normalization is given as 
\beq
 \int d\bir_1 \int \bir_2 \rho(\bir_1,\bir_2) 
 = \sum_{ab} \abs{\alpha_{ab}}^2 = 1, 
\eeq
since $\int d\bir_1 \int \bir_2 d_{cd,ab}(\bir_1,\bir_2)=\delta_{ca} \delta_{db}$. 

\section{Matrix Diagonalization}
In this Chapter, 
we have derived the basic formulas for the three-body model. 
With the uncorrelated basis, one can represent the 
eigen-states of the Hamiltonian with a spin $(J,M)$, namely 
$H_{\rm 3b} \ket{E_N ^{(J,M)}} = E_N \ket{E_N ^{(J,M)}}$, as follows. 
\beq
 \ket{E_N ^{(J,M)}} = \sum_K U_{NK} \ket{\tilde{\Psi}_K^{(J,M)}}, 
\eeq
where $K \equiv \left\{ (nlj)_a (nlj)_b \right\}$. 
In this expansion, there are also continuum basis with 
$\epsilon_a + \epsilon_b > 0$. 
One should notice that, even for a bound three-body state with $E_N \leq 0$, 
the wave function includes continuum s.p. states. 
The expansion coefficients $\{ U_{NK} \}$ can be obtained 
by diagonalizing the Hamiltonian matrix, 
$\Braket{\tilde{\Psi}_{K'} | H_{\rm 3b} | \tilde{\Psi}_K}$. 
Since we consider the pure Hermite space, all the MEs are real numbers. 
Thus, in order to diagonalize the Hamiltonian matrix, 
we employ ``Jacobi method'' 
for real, symmetric matrices \cite{95Kelley}. 
A typical dimension of our Hamiltonian is about 
from $100 \times 100$ to $1000 \times 1000$. 
The dimension actually depends on the cutoff 
parameters which we will explain later. 

%Indeed, there are two remaining works after this Chapter. 
%First, we have to fix the parameters in the core-nucleon and 
%the nucleon-nucleon interactions. 
%Parameters of $V_{\rm c-N}$ will be defined so as to reproduce 
%the energetic or scattering 
%properties of the partial core-nucleon system. 
%On the other hand, for remaining parameters in $v_{\rm N-N}$, 
%we will take usual values 
%adopted in other papers. 
%One of these parameters is, however, used as the fitting parameter in order to 
%reproduce the binding energies of considering systems. 
%Precise discussions about parameters will be present in the next Chapter. 
%Second, for \twop-emissions, 
%we must formulate the time-dependent framework for the quantum 
%meta-stable processes. 
%We stack aside this task until Chapter \ref{Ch_TDM}. 

In the next Chapter, we will apply the formalism presented in this Chapter 
to the pairing and dinucleon correlations in particle-bound nuclei, 
whereas an application to \twop-emitters will be discussed in 
Chapter \ref{Ch_Results2} and \ref{Ch_Results3}. 
\include{end}
