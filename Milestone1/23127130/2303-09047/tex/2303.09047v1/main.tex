\documentclass[%
 reprint,
%superscriptaddress,
%groupedaddress,
%unsortedaddress,
%runinaddress,
%frontmatterverbose, 
%preprint,
%preprintnumbers,
%nofootinbib,
%nobibnotes,
%bibnotes,
 amsmath,amssymb,
 aps,
 prl,
%prb,
%rmp,
%prstab,
%prstper,
%floatfix,
]{revtex4-1}

\usepackage{graphicx}% Include figure files
\usepackage{dcolumn,color}% Align table columns on decimal point
\usepackage[english]{babel}
%\usepackage{bm}% bold math
%\usepackage[caption=false]{subfig}
%\usepackage{subcaption}
%\usepackage{hyperref}% add hypertext capabilities
\usepackage{bbm}
\usepackage{amsmath}
\usepackage{mathtools}
%\usepackage[mathlines]{lineno}% Enable numbering of text and display math
%\linenumbers\relax % Commence numbering lines

%\usepackage[showframe,%Uncomment any one of the following lines to test 
%%scale=0.7, marginratio={1:1, 2:3}, ignoreall,% default settings
%%text={7in,10in},centering,
%%margin=1.5in,
%%total={6.5in,8.75in}, top=1.2in, left=0.9in, includefoot,
%%height=10in,a5paper,hmargin={3cm,0.8in},
%]{geometry}
\def \cm#1{ \textcolor{red}{#1} } 

\begin{document}

\preprint{APS/123-QED}

\title{Response Theory Identifies Reaction Coordinates and Explains Critical Phenomena in Noisy Interacting Systems}% 
\author{Niccol\`o Zagli}
 \email{niccolo.zagli@su.se}
\affiliation{Nordita, Stockholm University and KTH Royal Institute of Technology, Hannes Alfvéns väg 12, SE-106 91 Stockholm, Sweden}
 \affiliation{Centre for the Mathematics of Planet Earth, University of Reading, Reading, RG6 6AX, UK}
 
 \author{Valerio Lucarini}
\affiliation{Department of Mathematics and Statistics, University of Reading, Reading, RG6 6AX, UK}
 \affiliation{Centre for the Mathematics of Planet Earth, University of Reading, Reading, RG6 6AX, UK}
 
 \author{Grigorios A. Pavliotis}
 \affiliation{Department of Mathematics, Imperial College London, London, SW7 2AZ, UK}


\date{\today}% It is always \today, today,
             %  but any date may be explicitly specified

\begin{abstract}
We consider a nonequilibrium system of interacting agents with pairwise interactions and quenched disorder in the dynamics. Such systems are routinely used to model collective emergent behaviour in multiple areas of social and natural sciences as they exhibit, in the thermodynamic limit, both continuous and discontinuous phase transitions.
The goal of this contribution  is twofold. Firstly, we provide a framework to identify suitable reaction coordinates for the system starting from the microscopic properties of the interaction structure among the agents. We state conditions on the interactions that lead to a dimension reduction of the system in terms of a finite number of reaction coordinates.
Secondly, we show that such reaction coordinates prove to be proper nonequilibrium thermodynamic variables as they carry information on correlation and memory properties and provide the natural way to probe the sensitivity of the system to external perturbations. In particular, we prove that the investigation of response properties of the observables corresponding to the reaction coordinates allows to identify, pinpoint and characterise phase transitions of the system as these manifest themselves as singularities in the corresponding susceptibility function. We apply our theory, corroborated by numerical evidence, to the investigation of static and dynamic phase transitions of the paradigmatic Kuramoto model. Our approach allows us to prove that the dynamic phase transition corresponds to an uncommon scenario of resonance originating from a non-vanishing real part of the residue associated to the singular part of the susceptibility. In this setting, the system exhibits a coherent behaviour characterised by a spontaneously acquired intrinsic phase, leading to the emergence of a phase shift of the response to sinusoidal forcings at the critical frequency.
\end{abstract}
\maketitle
Interacting agent models are at the basis of the microscopic description of the rich variety of collective emergent phenomena that high dimensional complex systems often exhibit. Common applications of such models range from synchronisation of nonlinear oscillators \cite{SakaguchiRossler,Pikovsky2003,Pecora1998,Pecora2015,Eroglu2017,OTT200229,Kuramoto,VlasovBiHarmonicPotential,PolitiClusellaBiHarmonic}, see also the recent special issue \cite{PikovskySpecialIssue2023}, phase transitions in complex energy landscapes \cite{Gomes,GomesPavliotis2017} to opinion dynamics and consensus formation \cite{HasgelmannKrause,Goddard2022}, socio-economic siences \cite{NaldiParentiToscani,toscani2014}, life sciences and neural dynamics \cite{DaiPra,ColletDaiPraFormentin}, formation of swarms \cite{Carrillo2014,Carrillo2010reviewSwarming}, dynamical networks \cite{Carrillo:2020aa}, self-gravitating systems \cite{Chavanis2014,SelfGravitating} and algorithms for optimisation and training of neural networks \cite{borovykh2020stochastic,reich2020,RotskoffVandenEijnden2022}.
In the thermodynamic limit, such systems can exhibit phase transitions resulting from the interplay between the interaction among the agents, their internal dynamics and the noise. The investigation of these critical phenomena relies on the identification of a suitable set of reaction coordinates acting as thermodynamic variables by providing a low dimensional description of the macroscopic feature of the system. While order parameters can in many cases be deduced for equilibrium systems using, \textit{e.g.}, symmetry arguments, the identification of reaction coordinates in nonequilibrium settings is far less trivial \cite{Rogal2021,Ma2005}. In this paper we provide a constructive way to identify a set of reaction coordinates $\{ A_i \}$ starting from the microscopic interaction structure among the agents. Moreover, inspired by the success of response theory in explaining critical phenomena in a class of interacting systems \cite{FirstPaper,ZagliLucariniPavliotis,ZagliPavliotisLucarini2023,Topaj2001}, we show that, according to this perspective, the $\{ A_i \}$ act as proper nonequilibrium thermodynamic variables as they  not only carry information on correlation properties and  sensitivity of the system to perturbations but also fully characterise the development of a critical phenomenon. 
\\ \\
We consider an ensemble of $N$  interacting $M-$dimensional systems $\{ \mathbf{x}^k \}_{k=1}^N$ whose dynamics is described by the following stochastic differential equations
\begin{equation}
\label{eq: N particle system quenched}
    \mathrm{d}\mathbf{x}^k = \mathbf{\mathbf{F}}_\alpha(\mathbf{x}^k; \mathbf{h}^k) \mathrm{d}t - \frac{\theta}{N}\sum_{j=1}^N  \mathbf{K}(\mathbf{x}^k,\mathbf{x}^j) \mathrm{d}t+ \sigma \mathbf{s}(\mathbf{x^k})\mathrm{d}\mathbf{W}^{(k)} 
\end{equation}The smooth vector field $\mathbf{F}_\alpha(\mathbf{x}; \cdot) : \mathbb{R}^M \to \mathbb{R}^M$, possibly depending on a set of parameters $\alpha$, defines the local dynamics of each agent corresponding to, in general, nonequilibrium conditions. The (pair-wise) interactions among the systems are introduced via the function $\mathbf{K} : \mathbb{R}^M \times \mathbb{R}^M \to \mathbb{R}^M$ with $\theta$ representing the coupling strength. The function $\mathbf{K}$ is assumed to be antisymmetric in its arguments, $\mathbf{K}(\mathbf{x},\mathbf{y}) = -\mathbf{K}(\mathbf{y},\mathbf{x})$, to model Newton's third law of motion for the internal forces acting between the systems. The volatility matrix $\mathbf{s}: \mathbb{R}^M \to \mathbb{R}^{M \times M}$ determines the state dependent noise term, with $\sigma$ being its amplitude and $\mathrm{d}\mathbf{W} = (\mathrm{d}\mathbf{W}^{(1)}, \dots, \mathrm{d}\mathbf{W}^{(N)})$ representing a $M \times N$ Brownian motion. In  the following we will adopt the Itô's convention. Furthermore, we introduce a source of quenched disorder by letting the function $\mathbf{F}_\alpha$ depend on a vector of parameters $\mathbf{h}^k \in \mathbb{R}^m$ drawn from a fixed known distribution $\mu$, that is $\mathbf{h}^k \sim \mu$ $\forall k = 1, \dots, N$. The microscopic architecture of the system given by $\mu$ can be interpreted either as an intrinsic property of the system, such as the natural frequencies of an ensemble of oscillators, or as a model error feature arising from partial knowledge of the microscopic properties of the agents.
\\
In the thermodynamic limit $N \to + \infty$, the mean field nature of the coupling allows to obtain a macroscopic hydrodynamic description of the ensemble of agents  \cite{McKean2,Snitz,ChaintronDiez2022}. More specifically, we define the empirical density of the $N$ particle system as $\rho_N(\mathbf{x},\mathbf{h},t) = \frac{1}{N}\sum_{j=1}^N \delta(\mathbf{x} - \mathbf{x}^j)\delta(\mathbf{h} - \mathbf{h}^j)$, where $\delta(\cdot)$ represents the Dirac delta distribution. Under general conditions, it is possible to show \cite{DaiPra,PraHollander} that, fixed a time interval $[0,\mathrm{T}]$ and given a chaotic initial condition $\mathbf{x}^k(t=0) \sim \hat{\rho}(\mathbf{x})$ $\forall k=1,\dots,N$ , the empirical density $\rho_N$ converges weakly to $\rho(\mathbf{x},t;\mathbf{h})\mu(\mathbf{h})$ where the the one-particle distribution $\rho(\mathbf{x},t;\mathbf{h})$ satisfies the nonlinear Fokker Planck Equation (NFPE) 
\begin{equation}
\begin{split}
\label{eq: General Equation quenched quenched}
\partial_t \rho &= - \nabla \cdot \left[\left(\mathbf{\mathbf{F}}_\alpha(\mathbf{x},\mathbf{h}) - \theta \int \left(\mathbf{K}\star_2 \rho\right) \mu(\mathbf{h})\mathrm{d}\mathbf{h} \right) \rho \right] + \\
&+ \frac{\sigma^2}{2} \nabla^2 : \left( \mathbf{s}(\mathbf{x})\mathbf{s}^T(\mathbf{x}) \rho \right) 
\end{split}
\end{equation}
where $\rho(\mathbf{x},0;\cdot)  =\hat{\rho}(\mathbf{x})$ and $t \in [0,\mathrm{T}]$. In equation \eqref{eq: General Equation quenched quenched} the drift term depends on the probability distribution $\rho$ itself through $(\mathbf{K} \star_2 \rho)(\mathbf{x};\mathbf{h}) = \int \mathbf{K}(\mathbf{x},\mathbf{y}) \rho(\mathbf{y},t;\mathbf{h}) \mathrm{d}\mathbf{y}$ originating from the coupling among the agents and $\nabla^2 : \left( \mathbf{s}\mathbf{s}^T \rho \right)  = \sum_{i,j,k=1}^M\partial^2_{ij}\left( s_{ik}(\mathbf{x})s_{jk}(\mathbf{x})\rho  \right)$ represents the state dependent diffusive part. We remark that the microscopic architecture of the system manifest itself in equation \eqref{eq: General Equation quenched quenched} as an expectation value over the distribution $\mu$. In particular, the homogeneous case where no quenched disorder is present is obtained by setting $\mu(\mathbf{h}) = \delta(\mathbf{h}-\mathbf{h}_0)$ where $\mathbf{h}_0$ is a constant parameter. We observe that equation \eqref{eq: NLFPEquenched} can support multiple coexisting stationary measures. Moreover, exchange of stability of such measures, as the parameters of the system are varied, correspond to critical phenomena for the system \cite{FrankBook,Carrillo:2020aa}. 
\subsection{Identification of Reaction Coordinates} 
\noindent
The characterisation of suitable reaction coordinates for these systems is strictly related to the properties of the interaction kernel $\mathbf{K}$. We consider a separable interaction kernel, that is, admitting a decomposition
\begin{equation}
\label{eq: separable kernel}
K_i(\mathbf{x},\mathbf{y}) = \sum_{l = 1}^{d} k_{i;l}(\mathbf{x}) a_{i;l}(\mathbf{y})
\end{equation}where $d \in \mathbb{N}$, $d < + \infty$. We remark that it is always possible to assume that the functions in the above decomposition are linearly independent \cite{Kress2014lie}. In this case, the effect of the nonlinearity in equation  \eqref{eq: General Equation quenched quenched} can be written in terms of a finite number of suitable observables rather than on the full probability distribution $\rho$ since, using  \eqref{eq: separable kernel}, the interaction term results in
\begin{widetext}
\begin{equation}
\label{eq: sep kernel2}
\int \left(K_i \star_2 \rho\right)(\mathbf{x};\mathbf{h}) \mu(\mathbf{h})\mathrm{d}\mathbf{h} = \int \int  K_i(\mathbf{x},\mathbf{y}) \rho(\mathbf{y},t;\mathbf{h}) \mu(\mathbf{h})\mathrm{d}\mathbf{h}\mathrm{d}\mathbf{y} = \sum_{l=1}^d k_{i;l}(\mathbf{x}) \langle \langle a_{i;l} \rangle \rangle(t) \coloneqq \mathcal{K}_i(\mathbf{x};\{\langle \langle a_{i;l} \rangle \rangle \}_l)
\end{equation}
\end{widetext}
where $\langle \langle \cdot \rangle \rangle$ represents the double expectation value with respect to $\rho(\mathbf{x},t; \mathbf{h})$ and $\mu$. Equation \eqref{eq: General Equation quenched quenched} can now be written as  
\begin{equation}
\begin{split}
\label{eq: NLFPEquenched}
    \partial_t \rho &= - \nabla \cdot \bigg( \big(\mathbf{\mathbf{F}}_\alpha(\mathbf{x},\mathbf{h}) - \theta \boldsymbol{\mathcal{K}}\left(\mathbf{x}, \langle \langle \mathbf{A} \rangle \rangle\right)  \big) \rho\bigg) + \\
&+ \frac{\sigma^2}{2} \nabla^2 : \left( \mathbf{s}(\mathbf{x})\mathbf{s}^T(\mathbf{x}) \rho \right) 
     \coloneqq \mathcal{L}_{\langle\langle \mathbf{A} \rangle\rangle} \rho
\end{split}
\end{equation}
where $\langle \langle \mathbf{A}(\mathbf{x})\rangle \rangle = \{ \langle \langle a_{i;l}(\mathbf{x})\rangle \rangle \}^{N,d}_{i=1,l=1} \in \mathbb{R}^p$ is a finite vector ($p$ not necessarily equals to $d$, see later discussion) of observables  and $A_j(\mathbf{x})$ $j=1,\dots,p$ are linearly independent functions. We have also defined the nonlinear operator $\mathcal{L}_{ \langle \langle \mathbf{A} \rangle \rangle }$ depending on the probability distribution through the vector $\langle \langle \mathbf{A} \rangle \rangle$.  Following the terminology of \cite{FrankBook}, this scenario corresponds to a finite nonlinearity dimension equals to $p$ for equation \eqref{eq: General Equation quenched quenched} and it corresponds to a natural dimension reduction of the ensemble of agents in terms of the variables $\langle \langle \mathbf{A} \rangle \rangle $. 
\\
Separable kernels $\mathbf{K}(\mathbf{x},\mathbf{y})= \nabla \mathcal{U}(\mathbf{x}-\mathbf{y})$ arising from interaction potentials are routinely used in applications, with radial potentials of the form $\mathcal{U}(\mathbf{x})= u(|\mathbf{x}|)$ being a common choice. A paradigmatic example is represented by quadratic interactions  $u(x) = \frac{x^2}{2}$, yielding a separable kernel $\mathbf{K}(\mathbf{x},\mathbf{y}) = \mathbf{x} - \mathbf{y}$. This corresponds to linear forces among the particles trying to synchronise them towards their centre of mass $\bar x = \frac{1}{N} \sum_j{\mathbf{x}}_j(t)$ and has numerous applications, see our previous work \cite{FirstPaper,ZagliLucariniPavliotis} on this. In this case, $\forall i=1,\dots,M$, the expansion \eqref{eq: separable kernel} of the interaction kernel holds with $d=2$ and  $(f_{i;1},g_{i;1},f_{i;2},g_{i;2})=(x,0,0,-y)$  resulting in a non linear dimension $p=
M$ for equation \eqref{eq: NLFPEquenched} given by the observables $\langle \langle \mathbf{A}(\mathbf{x}) \rangle \rangle = \langle \langle \mathbf{x} \rangle \rangle$. Similarly, higher order polynomials $u(x) = \sum_{k=2}^n u_k x^k$ with $n \geq 2$ give rise to separable kernels and a finite nonlinearity dimension. As a further example, we mention separable kernels provided by trigonometric interactions of any order with applications to synchronisation phenomena, see \cite{Battle1977,PolitiClusellaBiHarmonic} and the section below on the Kuramoto model. We refer the reader to the conclusions section for a discussion regarding non separable kernels. 
\\ 
We remark that equation \eqref{eq: NLFPEquenched} can be interpreted as a thermodynamic formulation of the system with the $\langle \langle \mathbf{A} \rangle \rangle$s being a generalisation of thermodynamic variables to nonequilibrium settings \cite{Zubarev1996}. Firstly, stationary distributions $\rho_0 = \rho_0(\mathbf{x};\mathbf{h})$ of \eqref{eq: NLFPEquenched} can be parametrised in terms of the stationary values of the thermodynamic variables $\langle \langle \mathbf{A} \rangle \rangle_0$ according to the self consistency equation given by the stationary condition $\mathcal{L}_{0} \rho_0 =0$ where $\mathcal{L}_0= \mathcal{L}_{\langle \langle \mathbf{A}\rangle\rangle_0}$ is the operator defined in \eqref{eq: NLFPEquenched} evaluated at stationarity. Secondly, the $ \langle \langle \mathbf{A}(\mathbf{x}) \rangle \rangle$s govern time dependent properties of the ensemble of agents, such as its response, correlation and memory features, as we elucidate in the next section.
% This is better understood in the case of homogeneous equilibrium systems, characterised by a gradient dynamics $\mathbf{F}_{\alpha}(\mathbf{x})= - \nabla V_\alpha(\mathbf{x})$ and thermal noise $\mathbf{s}(\mathbf{x}) = \mathbf{1}_{M\times M}$. In such systems, stationary solutions of \eqref{eq: General Equation quenched quenched} satisfy the self consistency equation \cite{Carrillo:2020aa} $\rho_{eq}(\mathbf{x}) = \frac{1}{Z}\exp\left(-\frac{2}{\sigma^2}\left( V_\alpha(\mathbf{x}) - \mathcal{U}\star \rho_{eq} \right) \right)$ where $Z$ is a normalisation constant and we have assumed that the interaction kernel derives from an interaction potential $\mathcal{U}(\mathbf{x})$, i.e., it can be written as $\mathbf{K}(\mathbf{x},\mathbf{y}) = \nabla \mathcal{U}(\mathbf{x}-\mathbf{y})$. Assuming a separable kernel, equilibrium stationary solutions are parametrised by a finite number of observables $\langle \mathbf{A} \rangle_{eq}$ since they can be written as $\rho_{eq}\left(\mathbf{x};\langle \mathbf{A} \rangle_{eq} \rangle\right) = \frac{1}{Z}\exp\left(-\frac{2}{\sigma^2}\left( V_\alpha(\mathbf{x}) - f(\mathbf{x},\langle \mathbf{A} \rangle_{eq}) \right)\right)$ where $f$ is a suitable function deriving by the convolution product $\mathcal{U}\star \rho_{eq}$, similarly to equation \eqref{eq: sep kernel2}. 
\subsection{Response Formulas}
\label{sec: Response Formulas}
%We denote one invariant measure of equation \eqref{eq: NLFPEquenched} as $\rho_0(\mathbf{x};\mathbf{h})$. We remark that the stationary state is characterised by the constant set of observables $\langle \langle \mathbf{A} \rangle \rangle_0$, where the subscript denotes that the expectation value is taken with respect to $\rho_0$ (and $\mu$). 
We perturb a stationary state $\rho_0$ of equation \eqref{eq: NLFPEquenched} by letting $\mathbf{F}_{\alpha}(\mathbf{x};\mathbf{h}) \rightarrow \mathbf{F}_\alpha(\mathbf{x};\mathbf{h}) + \varepsilon T(t)\mathbf{X}(\mathbf{x})$ and write $\rho(\mathbf{x},t;\mathbf{h}) \approx \rho_0(\mathbf{x};\mathbf{h}) + \varepsilon \rho_1(\mathbf{x},t;\mathbf{h})$. We observe that, since $\langle \langle \mathbf{A} \rangle \rangle \approx \langle \langle  \mathbf{A} \rangle \rangle_0 + \varepsilon \langle \langle \mathbf{A} \rangle \rangle_1(t)$, where the last subscript denotes the expectation value with respect to $\rho_1$, the perturbation to the nonlinear part of the drift term can be written as
\begin{equation}
\boldsymbol{\mathcal{K}}\left(\mathbf{x}, \langle \langle\mathbf{A} \rangle \rangle\right) \approx \boldsymbol{\mathcal{K}}\left(\mathbf{x}, \langle\langle  \mathbf{A} \rangle\rangle_0 \right) + \varepsilon \mathbf{J}\left(\mathbf{x} \right)\langle\langle \mathbf{A} \rangle\rangle_1(t)
\end{equation}where the matrix $\mathbf{J} \in \mathbb{R}^{M \times p}$ provides the information on how the drift term changes, in a linear approximation, with respect to the observables $\langle \langle \mathbf{A} \rangle \rangle$ and is defined as
\begin{equation}
\label{eq: matrix J}
J_{ij}\left( \mathbf{x} \right) = J_{ij}(\mathbf{x},\langle\langle \mathbf{A} \rangle\rangle_0) = \frac{\partial \mathcal{K}_i}{\partial \langle\langle A_j \rangle\rangle}\left( \mathbf{x}, \langle\langle \mathbf{A} \rangle\rangle_0 \right)
\end{equation}
\begin{figure*}[t]
    \centering
\includegraphics[scale=0.5]{Combined_figure.pdf}
    \caption{Real part of the susceptibility $\tilde{\chi}_1(\omega)$ associated to the reaction coordinate  $ A_1(x) $. Panel (a) refers to the homogeneous case $\mu(h) = \delta(0)$ and panel (b) to a  bimodal distribution $\mu_{h_0}(h) = \frac{1}{2} (\delta(h_0) + \delta(-h_0) )$. As the thermodynamic limit is approached, $\mathbf{Re}\tilde{\chi}_1(\omega)$ develops a diverging behaviour for $\omega = \omega_0 = 0$ (panel (a)) and $\omega = \omega^\star = \sqrt{h_0^2 - D^2}$ (panel (b)). The investigation of $\mathbf{Im} \tilde{\chi}_1$ (bottom insets) provides information on the residue associated to the pole $\omega_0=0$ of the susceptibility for the homogeneous case. The top insets show the macroscopic Green function $\tilde{G}_1(t)$.
    % As explained in the main text, the response function $\tilde{G}_1(t)$ for the bimodal case $\mu_{h_0}$ undergoes an initial linear behaviour (not shown in the figure) with positive slope proportional to the real part of the residue.
    Red (black) lines refer to settings at (away from) the phase transition. The numerical response experiments have been performed with $N = 6,12,24,48 \times 10^3$, see more details in the Supplementary Material.}
    \label{fig:my_label}
\end{figure*}
From \eqref{eq: NLFPEquenched} we obtain an equation for the perturbation to the stationary measure
\begin{equation}
\begin{split}
    \label{eq: correction to invariant measure}
    \rho_1(\mathbf{x},t;\mathbf{h}) &= \int_{0}^t T(s) e^{\left(t-s\right)\mathcal{L}_{0}  }\mathcal{L}_p \rho_0 \mathrm{d}s +\\
    &+\theta \sum_{j=1}^{p} \int_{0}^t \langle \langle A_j \rangle\rangle_1(s) \sum_{k=1}^M e^{\left( t-s\right) \mathcal{L}_{0}} \partial_{x_k} \left( J_{kj}\left(\mathbf{x} \right)\rho_0\right) \mathrm{d}s
\end{split}
\end{equation}
where $\mathcal{L}_p \cdot \coloneqq - \nabla \cdot \left( \mathbf{X}(\mathbf{x}) \quad \!\!\!\! \cdot \quad \!\!\!\!  \right)$ is the perturbation operator.
%$\mathcal{L}_0=\mathcal{L}_{\langle \langle \mathbf{A}\rangle\rangle_0}$ is the operator defined in \eqref{eq: NLFPEquenched} evaluated at the stationary state
We now evaluate the perturbation to the observable $A_i(\mathbf{x})$ by taking the expected value with respect to $\rho_1$ and $\mu$ to obtain
\begin{equation}
\begin{split}
\label{eq: expected value 1 quenched}
\langle \langle A_i \rangle \rangle_1(t) = \left( G_i \star T \right)(t) + \theta \sum_{j=1}^p \left(  Y_{ij} \star \langle \langle A_j \rangle \rangle_1  \right)(t)
%&= \int_{-\infty}^{+\infty} T(s) G_i(t-s) \mathrm{d}s + \\
%&+ \theta \sum_{j=1}^p\int_{-\infty}^{+\infty} \langle \langle A_j \rangle \rangle(s) Y_{ij}(t-s) \mathrm{d}s
\end{split}
\end{equation}
where $\star$ denotes the convolution product and we have defined the microscopic Green's functions %$G_i(t) = \int G_i(t;\mathbf{h})\mu(\mathbf{h})\mathrm{d}\mathbf{h}$ and $Y_{ij}(t)  = \int Y_{ij}(t;\mathbf{h})\mu(\mathbf{h})\mathrm{d}\mathbf{h} $ obtained from
\begin{subequations}
\begin{align}
G_i(t) &=  \Theta(t)\int \mathrm{d}\mathbf{h}\mu(\mathbf{h})\int \mathrm{d}\mathbf{x} A_i(\mathbf{x}) e^{t\mathcal{L}_{0} } \mathcal{L}_p \rho_0,
\\
\label{eq: microscopic Green Function Y}
Y_{ij}(t) &= \Theta(t)\int \mathrm{d}\mathbf{h}\mu(\mathbf{h}) \int \mathrm{d}\mathbf{x} A_i(\mathbf{x}) e^{t\mathcal{L}_{_0} } \sum_{k=1}^M \partial_{x_k} \left( J_{kj}\left(\mathbf{x}\right) \rho_0\right). 
\end{align}
\end{subequations}
We observe that $G_i$ depends on the perturbation whereas $Y_{ij}$ does not, representing thus an intrinsic property of the system describing its internal feedbacks, see \cite{FirstPaper}.  
The coupling among the agents manifests itself as a memory term in \eqref{eq: expected value 1 quenched} conveyed by the variables $\langle \langle A_j\rangle \rangle$ through the microscopic Green's Function $Y_{ij}$. 
Taking the Fourier transform of \eqref{eq: expected value 1 quenched} we obtain
\begin{equation}
\label{eq: response in frequency}
\sum_{j=1}^p P_{ij}(\omega) \langle \langle A_j \rangle \rangle_1(\omega) = T(\omega) \chi_{i}(\omega)
\end{equation}where we have defined the $p \times p$ matrix 
\begin{equation}
\label{eq: renormalisation matrix}
P_{ij}(\omega) = \delta_{ij} - \theta Y_{ij}(\omega)
\end{equation}and $\chi_i(\omega)$  ($Y_{ij}(\omega)$) are the Fourier transform of the microscopic Green functions $G_i(t)$ ($Y_{ij}(t)$) respectively. Equations \eqref{eq: response in frequency} and \eqref{eq: renormalisation matrix} completely characterise the response of statistical properties of the system to perturbations in terms of the observables $\langle \langle \mathbf{A} \rangle \rangle$. When $\chi_i(\omega)$ is an analytic function in the upper complex $\omega$ plane and $P_{ij}(\omega)$ is invertible the response of the system is smooth and the macroscopic susceptibility of the system is $\tilde{\boldsymbol{\chi}}(\omega) = \mathbf{P}^{-1}(\omega)\boldsymbol{\chi}(\omega)$. In this case, $\tilde{G}_i(t)$, the inverse Fourier transform of $\tilde{\chi}_i(\omega)$, defines a causal macroscopic Green's function describing the response properties of the system as $\langle \langle A_i(t) \rangle\rangle_1(t) = (\tilde{G}_i \star T)(t)$. The response of the system breaks down, as a result of a pole in $\tilde{\boldsymbol{\chi}}$, when either $\chi_i(\omega)$ develops a singularity, corresponding to a destabilisation of each agent in the system \cite{FirstPaper}, or when $P_{ij}(\omega)$ becomes noninvertible. Equation \eqref{eq: renormalisation matrix} shows that the latter case arises due to endogenous instabilities associated with positive feedbacks resulting from the coupling among the agents. This identifies the occurrence of a phase transition in the system as it is intimately related to taking the thermodynamic limit \cite{FirstPaper}. 
\subsection{Critical phenomena for the Kuramoto model}
\label{sec: Kuramoto model}
We elucidate the above results by investigating critical phenomena in the Kuramoto model \cite{AcebronBonilla2005,GuptaCampaRuffo2014}, the paradigmatic example of synchronisation phenomena of phase oscillators, described by the following SDEs
\begin{equation}
\label{eq : Kuramoto model}
    \mathrm{d}x_k = \big[ h_k -\frac{\theta}{N} \sum_{j=1}^N \sin\left( x_k - x_j \right) \big] \mathrm{d}t +\sigma \mathrm{d}W_k
\end{equation}
where the quenched disorder in this case corresponds to the presence of a non-trivial distribution of intrinsic frequencies across the oscillators, so that $h_k \sim \mu$.
The interaction kernel $K(x,y) = \sin(x-y)=\sin(x)\cos(y) - \cos(x)\sin(y)$ is separable and leads, in the thermodynamic limit, to the following NFPE of nonlinearity dimension $p=2$
\begin{equation}
\label{eq: NLFPE Kuramoto}
\partial_t \rho = - \partial_x \big( h - \mathcal{K}(x,\langle \langle \mathbf{A} \rangle \rangle) \big) \rho + \frac{\sigma^2}{2} \partial_{xx}\rho = \mathcal{L}_{\langle \langle \mathbf{A} \rangle \rangle} \rho
\end{equation}
where $\mathcal{K}(x,\langle \langle \mathbf{A} \rangle \rangle) = \sin(x) \langle \langle A_1(x) \rangle \rangle - \cos(x) \langle \langle A_2(x)\rangle \rangle $ and $\mathbf{A}(x) = \left( \cos(x),\sin(x)\right)$ represents the set of reaction coordinates for the system. The uniform distribution $\rho_0 =\frac{1}{2\pi}$ is always a solution of equation  \eqref{eq: NLFPE Kuramoto} corresponding to a disordered, non synchronised state characterised by a set of reaction coordinates $\langle \langle \mathbf{A} \rangle \rangle_0 = (0,0)$. From  \eqref{eq: matrix J}, we evaluate $\mathbf{J} = \left( \sin(x),-\cos(x)\right)$ yielding - see \footnote{We refer the reader to the Supplementary Material for the details regarding the calculations regarding this section} - the microscopic Green's Function $Y_{ij}(t)$:
\begin{equation}
    Y_{ij}(t) = -\Theta(t) \int \mathrm{d}
    h \mu(h)\frac{D}{h^2+D^2} \frac{\mathrm{d}}{\mathrm{d}t}C_{ij}(t;h),
\end{equation}
where $D= \frac{\sigma^2}{2}$ and $C_{ij}(t;h)$ is the correlation function in the stationary state between observable $A_i(x)$ and $\psi_j(x;h) = A_j(x) - \frac{h}{D}\partial_x A_j(x)$. If we consider an even distribution of intrinsic frequencies $\mu(-h)=\mu(h)$ the matrix \eqref{eq: renormalisation matrix} is diagonal $P_{ij}(\omega) = P(\omega)\delta_{ij}$ and
\begin{equation}
\label{eq: function P)}
    P(\omega) = 1 - \frac{\theta D}{2}\int \frac{\mu(h)}{D^2+h^2}\left( 1 + 2i \omega C\left(\omega;h\right)\right) \mathrm{d}h
\end{equation}
where $C(\omega;h) = C_{R}(\omega;h) + i C_{I}(\omega;h)$ is the Fourier Transform of the (diagonal part of the) correlation function $C_{ij}(t;h)$ and the explicit expressions for $C_R$ and $C_I$ are in the Supplementary Material. Singularities in the response of the reaction coordinate $A_i(x)$, due to a phase transition, are characterised by a value $\omega^* \in \mathbb{R}$ such that the matrix $P_{ij}(\omega^*)$ is not invertible, or, equivalently, such that $P(\omega^*) = 0$. Separating real and imaginary part of $P(\omega^*) = 0$ results respectively in
\begin{subequations}
    \begin{align}
    \label{eq: first condition}
        &\frac{D\theta}{2} \int \frac{\mu(h)}{h^2+D^2}\left( 1 - 2\omega^* C_I(\omega^*;h) \right) \mathrm{d}h = 1,
        \\
        \label{eq: second condition}
        & \omega^*\int \frac{\mu(h)}{h^2+D^2} C_R(\omega^*;h) \mathrm{d}h = 0.
    \end{align}
\end{subequations} 
Equation \eqref{eq: second condition} determines the critical value $\omega^* \in \mathbb{R}$ whereas \eqref{eq: first condition} identifies the corresponding transition point in the parameter space $(\theta,D)$.
An immediate solution of \eqref{eq: second condition} is $\omega^*=\omega_0 = 0$, which takes place, according to \eqref{eq: first condition}, at the critical coupling strength $ \theta^{stat}_c = 2 [ \int \frac{D}{h^2+D^2}\mu(h) \mathrm{d}h ]^{-1}$ in agreement with \cite{StrogatzMirollo1991}.
This setting corresponds to a static phase transition, characterised by a vanishing pole $\omega_0=0$ of the susceptibility and a lack of oscillations in the system, see panel (a) of Figure \ref{fig:my_label} where we have investigated this scenario for $\mu_0(h)= \delta(h)$. The introduction of a distribution of natural frequencies $\mu$ brings the system to a nonequilibrium regime and it is interesting to investigate whether a dynamical phase transition, characterised by a pair of poles $\omega_1 = - \omega_2 = \omega^* > 0$, can arise in the system. This involves the study of the dynamical properties encoded in $C(\omega;h)$. 
\begin{figure}
    \centering
    \includegraphics[scale=0.03]{CriticalPhaseShift.jpeg}
    \caption{Top panel: Response of the reaction coordinate $A_1$ to a sinusoidal forcing $T(t)=\sin(\omega^* t)$ . The dashed line corresponds to $T(t)$, the continuous one to the analytical result and the red dots to the numerical response evaluated as $\tilde{G}_1 \star T$ with $\tilde{G}_1(t)$ obtained from the numerical simulations for $N=48 \times 10^3$. The amplitude of the numerical response has been re-scaled by its maximum value. Bottom panel: Correlation function (CF) between $T(t)$ and the numerical response as a function of the angle $\omega^* t$. Its maximum value yields an estimate of the phase shift $\phi'$. The vertical dashed line corresponds to the analytical value of $\phi'$. Red dots correspond to multiples of $\frac{\pi}{2}$ and are just a visual aid.} 
    \label{fig:phase shift critical}
\end{figure}
Further insight can be gained by observing that, if \eqref{eq: second condition} is satisfied for $\omega^* >0$, the following equation holds, see details in the Supplementary Material,
\begin{equation}
\label{eq: real part Kuramoto}
     \int \frac{\omega^* \pm h}{D^2 + (\omega^* \pm h)^2} \mu(h)\mathrm{d}h = 0,
\end{equation}
which is a known result for the Kuramoto model \cite{GuptaCampaRuffo2014}. In particular, no dynamical phase transition can take place if $\mu(h)$ is unimodal as no $\omega^* \neq 0$ satisfying \eqref{eq: real part Kuramoto} exists. 
We here consider, as in \cite{Bonilla1992}, a bimodal distribution $\mu_{h_0}(h) = \frac{1}{2}(\delta(h-h_0)+\delta(h+h_0))$. In this setting, equation \eqref{eq: second condition} admits, further to the static transition scenario, a new solution $\omega^*= \sqrt{h_0^2 - D^2}$ when $h_0>D$. According to \eqref{eq: first condition}, this dynamical phase transition scenario takes place at the transition point $\theta_{dyn} = 4D$. Panel (b) of Figure \ref{fig:my_label} confirms that, at the phase transition, response properties break down as the susceptibility develops a singular resonant behaviour for  $\omega = \omega^*$ as $N \to \infty$, resulting in greatly amplified and long lasting oscillations of the macroscopic Green's function $\tilde{G}_i(t)$. The susceptibility can be written, at the phase transition, as 
\begin{equation}
\label{eq: mollified susceptibility}
     \tilde{\chi}_1(\omega) = 
     \begin{cases} 
    \frac{\kappa}{\omega + i \gamma(N) } + \psi(\omega) & \text{if $\omega^*=0$}
    \\
    \frac{\kappa}{\omega- \omega^* + i \gamma(N)}  - \frac{\kappa^*}{\omega+ \omega^* + i \gamma(N)} + \psi(\omega) & \text{if $\omega^* >0$}
    \end{cases}
\end{equation} where $\psi(\omega)$ is an analytic function in the upper complex $
\omega$ plane and the residue $\kappa$ characterises the strength of the divergence. We remark that the finiteness of the system results in a mollifying effect of the singularity into a resonant behaviour, where the poles are slightly shifted as $\pm \omega^* \to \pm \omega^* + i \gamma(N)$, with $\gamma(N) \to 0$ as $N\to +\infty$. In the homogeneous case the residue turns out to be completely imaginary and equal to $\kappa = \frac{i}{2}$, whereas for the bimodal case it results in $\kappa = \kappa_R + i \kappa_I = - \frac{D}{4 \omega^*} + \frac{i}{4}$. In the former case, one expects that, as $N \to +\infty$, the function $\omega \mathbf{Im}\tilde{\chi}_1(\omega)$ will be constant and equal to $|\kappa| = \frac{1}{2}$ in the proximity of the pole and to quickly drop to zero for $\omega =0$. A visual inspection of the bottom inset of panel (a) of Figure \ref{fig:my_label} provides a clear evidence of such theoretical prediction. A fundamentally different situation occurs in the latter case as the residue has a non-vanishing real part leading to a distortion of the resonance at $\omega^*$ as $\mathbf{Im}\tilde{\chi}(\omega^*) \neq 0$. In order to further investigate this non typical situation, we consider a perturbation $T(t) = \sin(\omega^* t)$. Using \eqref{eq: mollified susceptibility} and performing an inverse Fourier transform, we evaluate the response $\langle \langle A_1 \rangle \rangle_1(t) = \frac{|\kappa|^2}{\gamma(N)}\sin\left(\omega^* t + \phi' \right)$ where $\phi'= \phi - \frac{\pi}{2}$ and $\phi = \arctan(\frac{\kappa_I}{\kappa_R}) = \arctan(\frac{\omega^*}{D})$. We observe that the non vanishing real part of the residue yields a spontaneous phase shift of the response to perturbations as confirmed by Figure \ref{fig:phase shift critical}. At the phase transition, the system shows an emergent behaviour characterised by its own intrinsic phase.
 %Moreover, dispersion relations and sum rules for the susceptibility represent general constraints, deriving from the principle of causality, for the linear response of any physical system and have been routinely used to test the quality of the output of models or experimental data \cite{Lucarini2005}. Moreover, they are intimately linked with the short time behaviour of response functions or, equivalently, with the asymptotic behaviour of the susceptibility for $\omega \to \infty$ \cite{Lucarini2005,LucariniSarno2011}. Equation \eqref{eq: mollified susceptibility} shows that asymptotically $\tilde{\chi}_1(\omega) = i a / \omega -  b/ \omega^2 + o(1/\omega^3)$ where $a=2\kappa_I = 1/2$ and $b = - 2 \kappa_R \omega^* + 2 \kappa_I \gamma(N) = (D+\gamma(N))/2$. This implies that it is possible to envision a much less computationally demanding numerical approach to the investigation of the singular behaviour of the susceptibility at the phase transition obtained by solely observing the short time $t \to 0^+$ behaviour of the Green's function of the system that, according to the susceptibility's asymptotic behaviour, is expected to behave as $\tilde{G}_1(t) = \Theta(t) \left(  a  +  bt \right) + o(t^2)$ \cite{LucariniSarno2011}, from which the properties of the residue $\kappa$ are easily inferred.  
\subsection{Conclusions}
\label{sec: conclusion}
In this paper we have considered the relevant issue of the identification of a set of reaction coordinates $\{ A_i \}$ for the thermodynamic limit of ensembles of interacting agents in the general case of pairwise interaction and quenched disorder in the dynamics. Given a general class of interaction structure among the agents, we have provided a framework to define the $\{ A_i \}s$ and construct a nonequilibrium thermodynamical formulation of the macroscopic description of the system. These reaction coordinates provide a parametrization of the stationary measure and fully characterise the linear response of the system to perturbations. Such a model reduction strategy is particularly useful as it allows to study not only smooth regimes of the response but also critical behaviours, characterised as an emergence of singular poles of the complex valued susceptibility of the system. The analysis in terms of the reaction coordinates of the paradigmatic model of the Kuramoto ensemble of oscillators elucidates the development of both static and dynamic phase transitions.
\\
Our results are exact in the case of separable interactions. However, non separable kernels are also found in applications. If $\mathbf{K}(\mathbf{x},\mathbf{y})$ is a smooth function, an expansion as in \eqref{eq: sep kernel2} still holds by using a Schauder decomposition \cite{LindenstraussTzafriri1977} but it contains in general an infinite number of terms, resulting in an infinite number of observables $\{  A_i \} $ with a clear loss of  dimension reduction of the system. However, we there exist numerous rigorous techniques to approximate in a controlled manner a general kernel $\mathbf{K}(\mathbf{x},\mathbf{y})$ in terms of suitable separable kernels, see \cite[Ch. 11]{Kress2014lie}\cite{Providas2021,DELLWO1995}. Such approximating methods would be the basis for the development of phenomenological theories of interacting systems with suitable separable interactions for which the identification of a finite number of observables can be performed exactly. Future research in this direction will involve the generalisation of this framework to systems with a microscopic network structure \cite{QuininaoTouboul2015}, to the physically relevant cases where the dynamics is not assumed to be overdamped, in the direction of understanding nonequilibrium ensembles in statistical mechanics \cite{G14} and to non-Markovian interacting particles~\cite{DuongPavliotis2018}, where appropriate Markovian closures have to be considered \cite{pavliotisbook2014,santos2021}.
% It is worth investigating how this general framework relates to other dimension reduction strategies, such as machine learning techniques, \textit{e.g.}, variational auto-encoders \cite{Kingma2014}, or collective coordinates \cite{Gottwald2017,SmithGottwald2020,YueSmithGottwald2020} and Ott-Attonsen-like \cite{OttAntonsen,CestnikPikovsky2022} approaches for phase oscillators.


\begin{acknowledgments}
 
VL acknowledges the support received by the European Union’s Horizon 2020 research and innovation program through the project TiPES (Grant Agreement No. 820970) and Marie Curie ITN CriticalEarth (Grant Agreement No. 956170). The work of GP was partially funded by the EPSRC, grant number EP/P031587/1, and by J.P. Morgan Chase \& Co through a Faculty Research Award 2019 and 2021. NZ has been supported by the Wallenberg Initiative on Networks and Quantum Information (WINQ). Nordita is supported in part by NordForsk.
\end{acknowledgments}

\begin{thebibliography}{53}%
\makeatletter
\providecommand \@ifxundefined [1]{%
 \@ifx{#1\undefined}
}%
\providecommand \@ifnum [1]{%
 \ifnum #1\expandafter \@firstoftwo
 \else \expandafter \@secondoftwo
 \fi
}%
\providecommand \@ifx [1]{%
 \ifx #1\expandafter \@firstoftwo
 \else \expandafter \@secondoftwo
 \fi
}%
\providecommand \natexlab [1]{#1}%
\providecommand \enquote  [1]{``#1''}%
\providecommand \bibnamefont  [1]{#1}%
\providecommand \bibfnamefont [1]{#1}%
\providecommand \citenamefont [1]{#1}%
\providecommand \href@noop [0]{\@secondoftwo}%
\providecommand \href [0]{\begingroup \@sanitize@url \@href}%
\providecommand \@href[1]{\@@startlink{#1}\@@href}%
\providecommand \@@href[1]{\endgroup#1\@@endlink}%
\providecommand \@sanitize@url [0]{\catcode `\\12\catcode `\$12\catcode
  `\&12\catcode `\#12\catcode `\^12\catcode `\_12\catcode `\%12\relax}%
\providecommand \@@startlink[1]{}%
\providecommand \@@endlink[0]{}%
\providecommand \url  [0]{\begingroup\@sanitize@url \@url }%
\providecommand \@url [1]{\endgroup\@href {#1}{\urlprefix }}%
\providecommand \urlprefix  [0]{URL }%
\providecommand \Eprint [0]{\href }%
\providecommand \doibase [0]{http://dx.doi.org/}%
\providecommand \selectlanguage [0]{\@gobble}%
\providecommand \bibinfo  [0]{\@secondoftwo}%
\providecommand \bibfield  [0]{\@secondoftwo}%
\providecommand \translation [1]{[#1]}%
\providecommand \BibitemOpen [0]{}%
\providecommand \bibitemStop [0]{}%
\providecommand \bibitemNoStop [0]{.\EOS\space}%
\providecommand \EOS [0]{\spacefactor3000\relax}%
\providecommand \BibitemShut  [1]{\csname bibitem#1\endcsname}%
\let\auto@bib@innerbib\@empty
%</preamble>
\bibitem [{\citenamefont {Sakaguchi}(2000)}]{SakaguchiRossler}%
  \BibitemOpen
  \bibfield  {author} {\bibinfo {author} {\bibfnamefont {H.}~\bibnamefont
  {Sakaguchi}},\ }\href {\doibase 10.1103/PhysRevE.61.7212} {\bibfield
  {journal} {\bibinfo  {journal} {Phys. Rev. E}\ }\textbf {\bibinfo {volume}
  {61}},\ \bibinfo {pages} {7212} (\bibinfo {year} {2000})}\BibitemShut
  {NoStop}%
\bibitem [{\citenamefont {Pikovsky}\ \emph {et~al.}(2003)\citenamefont
  {Pikovsky}, \citenamefont {Kurths}, \citenamefont {Rosenblum},\ and\
  \citenamefont {Kurths}}]{Pikovsky2003}%
  \BibitemOpen
  \bibfield  {author} {\bibinfo {author} {\bibfnamefont {A.}~\bibnamefont
  {Pikovsky}}, \bibinfo {author} {\bibfnamefont {J.}~\bibnamefont {Kurths}},
  \bibinfo {author} {\bibfnamefont {M.}~\bibnamefont {Rosenblum}}, \ and\
  \bibinfo {author} {\bibfnamefont {J.}~\bibnamefont {Kurths}},\ }\href
  {https://books.google.co.uk/books?id=FuIv845q3QUC} {\emph {\bibinfo {title}
  {Synchronization: A Universal Concept in Nonlinear Sciences}}},\ Cambridge
  Nonlinear Science Series\ (\bibinfo  {publisher} {Cambridge University
  Press},\ \bibinfo {year} {2003})\BibitemShut {NoStop}%
\bibitem [{\citenamefont {Pecora}(1998)}]{Pecora1998}%
  \BibitemOpen
  \bibfield  {author} {\bibinfo {author} {\bibfnamefont {L.~M.}\ \bibnamefont
  {Pecora}},\ }\href@noop {} {\bibfield  {journal} {\bibinfo  {journal} {Phys.
  Rev. E}\ }\textbf {\bibinfo {volume} {58}},\ \bibinfo {pages} {347} (\bibinfo
  {year} {1998})}\BibitemShut {NoStop}%
\bibitem [{\citenamefont {Pecora}\ and\ \citenamefont
  {Carroll}(2015)}]{Pecora2015}%
  \BibitemOpen
  \bibfield  {author} {\bibinfo {author} {\bibfnamefont {L.~M.}\ \bibnamefont
  {Pecora}}\ and\ \bibinfo {author} {\bibfnamefont {T.~L.}\ \bibnamefont
  {Carroll}},\ }\href {\doibase 10.1063/1.4917383} {\bibfield  {journal}
  {\bibinfo  {journal} {Chaos: An Interdisciplinary Journal of Nonlinear
  Science}\ }\textbf {\bibinfo {volume} {25}},\ \bibinfo {pages} {097611}
  (\bibinfo {year} {2015})},\ \Eprint
  {http://arxiv.org/abs/https://doi.org/10.1063/1.4917383}
  {https://doi.org/10.1063/1.4917383} \BibitemShut {NoStop}%
\bibitem [{\citenamefont {Eroglu}\ \emph {et~al.}(2017)\citenamefont {Eroglu},
  \citenamefont {Lamb},\ and\ \citenamefont {Pereira}}]{Eroglu2017}%
  \BibitemOpen
  \bibfield  {author} {\bibinfo {author} {\bibfnamefont {D.}~\bibnamefont
  {Eroglu}}, \bibinfo {author} {\bibfnamefont {J.~S.~W.}\ \bibnamefont {Lamb}},
  \ and\ \bibinfo {author} {\bibfnamefont {T.}~\bibnamefont {Pereira}},\ }\href
  {\doibase 10.1080/00107514.2017.1345844} {\bibfield  {journal} {\bibinfo
  {journal} {Contemporary Physics}\ }\textbf {\bibinfo {volume} {58}},\
  \bibinfo {pages} {207} (\bibinfo {year} {2017})},\ \Eprint
  {http://arxiv.org/abs/https://doi.org/10.1080/00107514.2017.1345844}
  {https://doi.org/10.1080/00107514.2017.1345844} \BibitemShut {NoStop}%
\bibitem [{\citenamefont {Ott}\ \emph {et~al.}(2002)\citenamefont {Ott},
  \citenamefont {So}, \citenamefont {Barreto},\ and\ \citenamefont
  {Antonsen}}]{OTT200229}%
  \BibitemOpen
  \bibfield  {author} {\bibinfo {author} {\bibfnamefont {E.}~\bibnamefont
  {Ott}}, \bibinfo {author} {\bibfnamefont {P.}~\bibnamefont {So}}, \bibinfo
  {author} {\bibfnamefont {E.}~\bibnamefont {Barreto}}, \ and\ \bibinfo
  {author} {\bibfnamefont {T.}~\bibnamefont {Antonsen}},\ }\href {\doibase
  https://doi.org/10.1016/S0167-2789(02)00663-2} {\bibfield  {journal}
  {\bibinfo  {journal} {Physica D: Nonlinear Phenomena}\ }\textbf {\bibinfo
  {volume} {173}},\ \bibinfo {pages} {29} (\bibinfo {year} {2002})}\BibitemShut
  {NoStop}%
\bibitem [{\citenamefont {Acebr\'on}\ \emph
  {et~al.}(2005{\natexlab{a}})\citenamefont {Acebr\'on}, \citenamefont
  {Bonilla}, \citenamefont {P\'erez~Vicente}, \citenamefont {Ritort},\ and\
  \citenamefont {Spigler}}]{Kuramoto}%
  \BibitemOpen
  \bibfield  {author} {\bibinfo {author} {\bibfnamefont {J.~A.}\ \bibnamefont
  {Acebr\'on}}, \bibinfo {author} {\bibfnamefont {L.~L.}\ \bibnamefont
  {Bonilla}}, \bibinfo {author} {\bibfnamefont {C.~J.}\ \bibnamefont
  {P\'erez~Vicente}}, \bibinfo {author} {\bibfnamefont {F.}~\bibnamefont
  {Ritort}}, \ and\ \bibinfo {author} {\bibfnamefont {R.}~\bibnamefont
  {Spigler}},\ }\href {\doibase 10.1103/RevModPhys.77.137} {\bibfield
  {journal} {\bibinfo  {journal} {Rev. Mod. Phys.}\ }\textbf {\bibinfo {volume}
  {77}},\ \bibinfo {pages} {137} (\bibinfo {year}
  {2005}{\natexlab{a}})}\BibitemShut {NoStop}%
\bibitem [{\citenamefont {Vlasov}\ \emph {et~al.}(2015)\citenamefont {Vlasov},
  \citenamefont {Komarov},\ and\ \citenamefont
  {Pikovsky}}]{VlasovBiHarmonicPotential}%
  \BibitemOpen
  \bibfield  {author} {\bibinfo {author} {\bibfnamefont {V.}~\bibnamefont
  {Vlasov}}, \bibinfo {author} {\bibfnamefont {M.}~\bibnamefont {Komarov}}, \
  and\ \bibinfo {author} {\bibfnamefont {A.}~\bibnamefont {Pikovsky}},\ }\href
  {\doibase 10.1088/1751-8113/48/10/105101} {\bibfield  {journal} {\bibinfo
  {journal} {Journal of Physics A: Mathematical and Theoretical}\ }\textbf
  {\bibinfo {volume} {48}},\ \bibinfo {pages} {105101} (\bibinfo {year}
  {2015})}\BibitemShut {NoStop}%
\bibitem [{\citenamefont {Clusella}\ and\ \citenamefont
  {Politi}(2017)}]{PolitiClusellaBiHarmonic}%
  \BibitemOpen
  \bibfield  {author} {\bibinfo {author} {\bibfnamefont {P.}~\bibnamefont
  {Clusella}}\ and\ \bibinfo {author} {\bibfnamefont {A.}~\bibnamefont
  {Politi}},\ }\href {\doibase 10.1103/PhysRevE.95.062221} {\bibfield
  {journal} {\bibinfo  {journal} {Phys. Rev. E}\ }\textbf {\bibinfo {volume}
  {95}},\ \bibinfo {pages} {062221} (\bibinfo {year} {2017})}\BibitemShut
  {NoStop}%
\bibitem [{\citenamefont {Pikovsky}\ and\ \citenamefont
  {Rosenblum}(2023)}]{PikovskySpecialIssue2023}%
  \BibitemOpen
  \bibfield  {author} {\bibinfo {author} {\bibfnamefont {A.}~\bibnamefont
  {Pikovsky}}\ and\ \bibinfo {author} {\bibfnamefont {M.}~\bibnamefont
  {Rosenblum}},\ }\href {\doibase 10.1063/5.0139277} {\bibfield  {journal}
  {\bibinfo  {journal} {Chaos: An Interdisciplinary Journal of Nonlinear
  Science}\ }\textbf {\bibinfo {volume} {33}},\ \bibinfo {pages} {010401}
  (\bibinfo {year} {2023})},\ \Eprint
  {http://arxiv.org/abs/https://doi.org/10.1063/5.0139277}
  {https://doi.org/10.1063/5.0139277} \BibitemShut {NoStop}%
\bibitem [{\citenamefont {Gomes}\ \emph {et~al.}(2019)\citenamefont {Gomes},
  \citenamefont {Kalliadasis}, \citenamefont {Pavliotis},\ and\ \citenamefont
  {Yatsyshin}}]{Gomes}%
  \BibitemOpen
  \bibfield  {author} {\bibinfo {author} {\bibfnamefont {S.~N.}\ \bibnamefont
  {Gomes}}, \bibinfo {author} {\bibfnamefont {S.}~\bibnamefont {Kalliadasis}},
  \bibinfo {author} {\bibfnamefont {G.~A.}\ \bibnamefont {Pavliotis}}, \ and\
  \bibinfo {author} {\bibfnamefont {P.}~\bibnamefont {Yatsyshin}},\ }\href
  {\doibase 10.1103/PhysRevE.99.032109} {\bibfield  {journal} {\bibinfo
  {journal} {Phys. Rev. E}\ }\textbf {\bibinfo {volume} {99}},\ \bibinfo
  {pages} {032109} (\bibinfo {year} {2019})}\BibitemShut {NoStop}%
\bibitem [{\citenamefont {Gomes}\ and\ \citenamefont
  {Pavliotis}(2018)}]{GomesPavliotis2017}%
  \BibitemOpen
  \bibfield  {author} {\bibinfo {author} {\bibfnamefont {S.}~\bibnamefont
  {Gomes}}\ and\ \bibinfo {author} {\bibfnamefont {G.}~\bibnamefont
  {Pavliotis}},\ }\href@noop {} {\bibfield  {journal} {\bibinfo  {journal} {J.
  Nonlin. Sci.}\ }\textbf {\bibinfo {volume} {28}},\ \bibinfo {pages} {905}
  (\bibinfo {year} {2018})}\BibitemShut {NoStop}%
\bibitem [{\citenamefont {Wang}\ \emph {et~al.}(2017)\citenamefont {Wang},
  \citenamefont {Li}, \citenamefont {E},\ and\ \citenamefont
  {Chazelle}}]{HasgelmannKrause}%
  \BibitemOpen
  \bibfield  {author} {\bibinfo {author} {\bibfnamefont {C.}~\bibnamefont
  {Wang}}, \bibinfo {author} {\bibfnamefont {Q.}~\bibnamefont {Li}}, \bibinfo
  {author} {\bibfnamefont {W.}~\bibnamefont {E}}, \ and\ \bibinfo {author}
  {\bibfnamefont {B.}~\bibnamefont {Chazelle}},\ }\href {\doibase
  10.1007/s10955-017-1718-x} {\bibfield  {journal} {\bibinfo  {journal}
  {Journal of Statistical Physics}\ }\textbf {\bibinfo {volume} {166}},\
  \bibinfo {pages} {1209} (\bibinfo {year} {2017})}\BibitemShut {NoStop}%
\bibitem [{\citenamefont {Goddard}\ \emph {et~al.}(2022)\citenamefont
  {Goddard}, \citenamefont {Gooding}, \citenamefont {Short},\ and\
  \citenamefont {Pavliotis}}]{Goddard2022}%
  \BibitemOpen
  \bibfield  {author} {\bibinfo {author} {\bibfnamefont {B.~D.}\ \bibnamefont
  {Goddard}}, \bibinfo {author} {\bibfnamefont {B.}~\bibnamefont {Gooding}},
  \bibinfo {author} {\bibfnamefont {H.}~\bibnamefont {Short}}, \ and\ \bibinfo
  {author} {\bibfnamefont {G.~A.}\ \bibnamefont {Pavliotis}},\ }\href {\doibase
  10.1093/imamat/hxab044} {\bibfield  {journal} {\bibinfo  {journal} {IMA J.
  Appl. Math.}\ }\textbf {\bibinfo {volume} {87}},\ \bibinfo {pages} {80}
  (\bibinfo {year} {2022})}\BibitemShut {NoStop}%
\bibitem [{\citenamefont {Naldi}\ \emph {et~al.}(2010)\citenamefont {Naldi},
  \citenamefont {Pareschi},\ and\ \citenamefont
  {Toscani}}]{NaldiParentiToscani}%
  \BibitemOpen
  \bibfield  {author} {\bibinfo {author} {\bibfnamefont {G.}~\bibnamefont
  {Naldi}}, \bibinfo {author} {\bibfnamefont {L.}~\bibnamefont {Pareschi}}, \
  and\ \bibinfo {author} {\bibfnamefont {G.}~\bibnamefont {Toscani}},\ }\href
  {\doibase 10.1007/978-0-8176-4946-3} {\emph {\bibinfo {title} {Mathematical
  Modeling of Collective Behavior in Socio-Economic and Life Sciences}}}\
  (\bibinfo  {publisher} {Birkh{\"a}user Basel},\ \bibinfo {year}
  {2010})\BibitemShut {NoStop}%
\bibitem [{\citenamefont {Pareschi}\ and\ \citenamefont
  {Toscani}(2013)}]{toscani2014}%
  \BibitemOpen
  \bibfield  {author} {\bibinfo {author} {\bibfnamefont {L.}~\bibnamefont
  {Pareschi}}\ and\ \bibinfo {author} {\bibfnamefont {G.}~\bibnamefont
  {Toscani}},\ }\href@noop {} {\emph {\bibinfo {title} {Interacting multiagent
  systems: kinetic equations and Monte Carlo methods}}}\ (\bibinfo  {publisher}
  {OUP Oxford},\ \bibinfo {year} {2013})\BibitemShut {NoStop}%
\bibitem [{\citenamefont {Dai~Pra}(2019)}]{DaiPra}%
  \BibitemOpen
  \bibfield  {author} {\bibinfo {author} {\bibfnamefont {P.}~\bibnamefont
  {Dai~Pra}},\ }in\ \href@noop {} {\emph {\bibinfo {booktitle} {Stochastic
  Dynamics Out of Equilibrium}}},\ \bibinfo {editor} {edited by\ \bibinfo
  {editor} {\bibfnamefont {G.}~\bibnamefont {Giacomin}}, \bibinfo {editor}
  {\bibfnamefont {S.}~\bibnamefont {Olla}}, \bibinfo {editor} {\bibfnamefont
  {E.}~\bibnamefont {Saada}}, \bibinfo {editor} {\bibfnamefont
  {H.}~\bibnamefont {Spohn}}, \ and\ \bibinfo {editor} {\bibfnamefont
  {G.}~\bibnamefont {Stoltz}}}\ (\bibinfo  {publisher} {Springer International
  Publishing},\ \bibinfo {address} {Cham},\ \bibinfo {year} {2019})\ pp.\
  \bibinfo {pages} {3--27}\BibitemShut {NoStop}%
\bibitem [{\citenamefont {Collet}\ \emph {et~al.}(2015)\citenamefont {Collet},
  \citenamefont {Dai~Pra},\ and\ \citenamefont
  {Formentin}}]{ColletDaiPraFormentin}%
  \BibitemOpen
  \bibfield  {author} {\bibinfo {author} {\bibfnamefont {F.}~\bibnamefont
  {Collet}}, \bibinfo {author} {\bibfnamefont {P.}~\bibnamefont {Dai~Pra}}, \
  and\ \bibinfo {author} {\bibfnamefont {M.}~\bibnamefont {Formentin}},\ }\href
  {\doibase 10.1007/s00030-015-0331-4} {\bibfield  {journal} {\bibinfo
  {journal} {Nonlinear Differential Equations and Applications NoDEA}\ }\textbf
  {\bibinfo {volume} {22}},\ \bibinfo {pages} {1461} (\bibinfo {year}
  {2015})}\BibitemShut {NoStop}%
\bibitem [{\citenamefont {Carrillo}\ \emph {et~al.}(2014)\citenamefont
  {Carrillo}, \citenamefont {Choi},\ and\ \citenamefont
  {Hauray}}]{Carrillo2014}%
  \BibitemOpen
  \bibfield  {author} {\bibinfo {author} {\bibfnamefont {J.~A.}\ \bibnamefont
  {Carrillo}}, \bibinfo {author} {\bibfnamefont {Y.-P.}\ \bibnamefont {Choi}},
  \ and\ \bibinfo {author} {\bibfnamefont {M.}~\bibnamefont {Hauray}},\
  }\enquote {\bibinfo {title} {The derivation of swarming models: Mean-field
  limit and wasserstein distances},}\ in\ \href {\doibase
  10.1007/978-3-7091-1785-9_1} {\emph {\bibinfo {booktitle} {Collective
  Dynamics from Bacteria to Crowds: An Excursion Through Modeling, Analysis and
  Simulation}}}\ (\bibinfo  {publisher} {Springer Vienna},\ \bibinfo {address}
  {Vienna},\ \bibinfo {year} {2014})\ pp.\ \bibinfo {pages} {1--46}\BibitemShut
  {NoStop}%
\bibitem [{\citenamefont {Carrillo}\ \emph {et~al.}(2010)\citenamefont
  {Carrillo}, \citenamefont {Fornasier}, \citenamefont {Toscani},\ and\
  \citenamefont {Vecil}}]{Carrillo2010reviewSwarming}%
  \BibitemOpen
  \bibfield  {author} {\bibinfo {author} {\bibfnamefont {J.~A.}\ \bibnamefont
  {Carrillo}}, \bibinfo {author} {\bibfnamefont {M.}~\bibnamefont {Fornasier}},
  \bibinfo {author} {\bibfnamefont {G.}~\bibnamefont {Toscani}}, \ and\
  \bibinfo {author} {\bibfnamefont {F.}~\bibnamefont {Vecil}},\ }\enquote
  {\bibinfo {title} {Particle, kinetic, and hydrodynamic models of swarming},}\
  in\ \href {\doibase 10.1007/978-0-8176-4946-3_12} {\emph {\bibinfo
  {booktitle} {Mathematical Modeling of Collective Behavior in Socio-Economic
  and Life Sciences}}}\ (\bibinfo  {publisher} {Birkh{\"a}user Boston},\
  \bibinfo {address} {Boston},\ \bibinfo {year} {2010})\ pp.\ \bibinfo {pages}
  {297--336}\BibitemShut {NoStop}%
\bibitem [{\citenamefont {Carrillo}\ \emph {et~al.}(2020)\citenamefont
  {Carrillo}, \citenamefont {Gvalani}, \citenamefont {Pavliotis},\ and\
  \citenamefont {Schlichting}}]{Carrillo:2020aa}%
  \BibitemOpen
  \bibfield  {author} {\bibinfo {author} {\bibfnamefont {J.~A.}\ \bibnamefont
  {Carrillo}}, \bibinfo {author} {\bibfnamefont {R.~S.}\ \bibnamefont
  {Gvalani}}, \bibinfo {author} {\bibfnamefont {G.~A.}\ \bibnamefont
  {Pavliotis}}, \ and\ \bibinfo {author} {\bibfnamefont {A.}~\bibnamefont
  {Schlichting}},\ }\href {\doibase 10.1007/s00205-019-01430-4} {\bibfield
  {journal} {\bibinfo  {journal} {Archive for Rational Mechanics and Analysis}\
  }\textbf {\bibinfo {volume} {235}},\ \bibinfo {pages} {635} (\bibinfo {year}
  {2020})}\BibitemShut {NoStop}%
\bibitem [{\citenamefont {Chavanis}(2014)}]{Chavanis2014}%
  \BibitemOpen
  \bibfield  {author} {\bibinfo {author} {\bibfnamefont {P.-H.}\ \bibnamefont
  {Chavanis}},\ }\href {\doibase 10.1140/epjb/e2014-40586-6} {\bibfield
  {journal} {\bibinfo  {journal} {The European Physical Journal B}\ }\textbf
  {\bibinfo {volume} {87}},\ \bibinfo {pages} {120} (\bibinfo {year}
  {2014})}\BibitemShut {NoStop}%
\bibitem [{\citenamefont {Tatekawa}\ \emph {et~al.}(2005)\citenamefont
  {Tatekawa}, \citenamefont {Bouchet}, \citenamefont {Dauxois},\ and\
  \citenamefont {Ruffo}}]{SelfGravitating}%
  \BibitemOpen
  \bibfield  {author} {\bibinfo {author} {\bibfnamefont {T.}~\bibnamefont
  {Tatekawa}}, \bibinfo {author} {\bibfnamefont {F.}~\bibnamefont {Bouchet}},
  \bibinfo {author} {\bibfnamefont {T.}~\bibnamefont {Dauxois}}, \ and\
  \bibinfo {author} {\bibfnamefont {S.}~\bibnamefont {Ruffo}},\ }\href
  {\doibase 10.1103/PhysRevE.71.056111} {\bibfield  {journal} {\bibinfo
  {journal} {Phys. Rev. E}\ }\textbf {\bibinfo {volume} {71}},\ \bibinfo
  {pages} {056111} (\bibinfo {year} {2005})}\BibitemShut {NoStop}%
\bibitem [{\citenamefont {Borovykh}\ \emph {et~al.}(2021)\citenamefont
  {Borovykh}, \citenamefont {Kantas}, \citenamefont {Parpas},\ and\
  \citenamefont {Pavliotis}}]{borovykh2020stochastic}%
  \BibitemOpen
  \bibfield  {author} {\bibinfo {author} {\bibfnamefont {A.}~\bibnamefont
  {Borovykh}}, \bibinfo {author} {\bibfnamefont {N.}~\bibnamefont {Kantas}},
  \bibinfo {author} {\bibfnamefont {P.}~\bibnamefont {Parpas}}, \ and\ \bibinfo
  {author} {\bibfnamefont {G.~A.}\ \bibnamefont {Pavliotis}},\ }\href {\doibase
  10.1016/j.physd.2021.132844} {\bibfield  {journal} {\bibinfo  {journal}
  {Phys. D}\ }\textbf {\bibinfo {volume} {418}},\ \bibinfo {pages} {Paper No.
  132844, 21} (\bibinfo {year} {2021})}\BibitemShut {NoStop}%
\bibitem [{\citenamefont {Garbuno-Inigo}\ \emph {et~al.}(2020)\citenamefont
  {Garbuno-Inigo}, \citenamefont {N\"{u}sken},\ and\ \citenamefont
  {Reich}}]{reich2020}%
  \BibitemOpen
  \bibfield  {author} {\bibinfo {author} {\bibfnamefont {A.}~\bibnamefont
  {Garbuno-Inigo}}, \bibinfo {author} {\bibfnamefont {N.}~\bibnamefont
  {N\"{u}sken}}, \ and\ \bibinfo {author} {\bibfnamefont {S.}~\bibnamefont
  {Reich}},\ }\href {\doibase 10.1137/19M1304891} {\bibfield  {journal}
  {\bibinfo  {journal} {SIAM J. Appl. Dyn. Syst.}\ }\textbf {\bibinfo {volume}
  {19}},\ \bibinfo {pages} {1633} (\bibinfo {year} {2020})}\BibitemShut
  {NoStop}%
\bibitem [{\citenamefont {Rotskoff}\ and\ \citenamefont
  {Vanden-Eijnden}(2022)}]{RotskoffVandenEijnden2022}%
  \BibitemOpen
  \bibfield  {author} {\bibinfo {author} {\bibfnamefont {G.}~\bibnamefont
  {Rotskoff}}\ and\ \bibinfo {author} {\bibfnamefont {E.}~\bibnamefont
  {Vanden-Eijnden}},\ }\href {\doibase https://doi.org/10.1002/cpa.22074}
  {\bibfield  {journal} {\bibinfo  {journal} {Communications on Pure and
  Applied Mathematics}\ }\textbf {\bibinfo {volume} {75}},\ \bibinfo {pages}
  {1889} (\bibinfo {year} {2022})}\BibitemShut {NoStop}%
\bibitem [{\citenamefont {Rogal}(2021)}]{Rogal2021}%
  \BibitemOpen
  \bibfield  {author} {\bibinfo {author} {\bibfnamefont {J.}~\bibnamefont
  {Rogal}},\ }\href {\doibase 10.1140/epjb/s10051-021-00233-5} {\bibfield
  {journal} {\bibinfo  {journal} {The European Physical Journal B}\ }\textbf
  {\bibinfo {volume} {94}},\ \bibinfo {pages} {223} (\bibinfo {year}
  {2021})}\BibitemShut {NoStop}%
\bibitem [{\citenamefont {Ma}\ and\ \citenamefont {Dinner}(2005)}]{Ma2005}%
  \BibitemOpen
  \bibfield  {author} {\bibinfo {author} {\bibfnamefont {A.}~\bibnamefont
  {Ma}}\ and\ \bibinfo {author} {\bibfnamefont {A.~R.}\ \bibnamefont
  {Dinner}},\ }\bibfield  {booktitle} {\emph {\bibinfo {booktitle} {The Journal
  of Physical Chemistry B}},\ }\href {\doibase 10.1021/jp045546c} {\bibfield
  {journal} {\bibinfo  {journal} {The Journal of Physical Chemistry B}\
  }\textbf {\bibinfo {volume} {109}},\ \bibinfo {pages} {6769} (\bibinfo {year}
  {2005})}\BibitemShut {NoStop}%
\bibitem [{\citenamefont {Lucarini}\ \emph {et~al.}(2020)\citenamefont
  {Lucarini}, \citenamefont {Pavliotis},\ and\ \citenamefont
  {Zagli}}]{FirstPaper}%
  \BibitemOpen
  \bibfield  {author} {\bibinfo {author} {\bibfnamefont {V.}~\bibnamefont
  {Lucarini}}, \bibinfo {author} {\bibfnamefont {G.~A.}\ \bibnamefont
  {Pavliotis}}, \ and\ \bibinfo {author} {\bibfnamefont {N.}~\bibnamefont
  {Zagli}},\ }\href {https://doi.org/10.1098/rspa.2020.0688} {\bibfield
  {journal} {\bibinfo  {journal} {Proc. R. Soc. A.}\ }\textbf {\bibinfo
  {volume} {476}} (\bibinfo {year} {2020})},\ \Eprint
  {http://arxiv.org/abs/https://doi.org/10.1098/rspa.2020.0688}
  {https://doi.org/10.1098/rspa.2020.0688} \BibitemShut {NoStop}%
\bibitem [{\citenamefont {Zagli}\ \emph {et~al.}(2021)\citenamefont {Zagli},
  \citenamefont {Lucarini},\ and\ \citenamefont
  {Pavliotis}}]{ZagliLucariniPavliotis}%
  \BibitemOpen
  \bibfield  {author} {\bibinfo {author} {\bibfnamefont {N.}~\bibnamefont
  {Zagli}}, \bibinfo {author} {\bibfnamefont {V.}~\bibnamefont {Lucarini}}, \
  and\ \bibinfo {author} {\bibfnamefont {G.~A.}\ \bibnamefont {Pavliotis}},\
  }\href {\doibase 10.1063/5.0053558} {\bibfield  {journal} {\bibinfo
  {journal} {Chaos: An Interdisciplinary Journal of Nonlinear Science}\
  }\textbf {\bibinfo {volume} {31}},\ \bibinfo {pages} {061103} (\bibinfo
  {year} {2021})},\ \Eprint
  {http://arxiv.org/abs/https://doi.org/10.1063/5.0053558}
  {https://doi.org/10.1063/5.0053558} \BibitemShut {NoStop}%
\bibitem [{\citenamefont {Zagli}\ \emph {et~al.}(2023)\citenamefont {Zagli},
  \citenamefont {Pavliotis}, \citenamefont {Lucarini},\ and\ \citenamefont
  {Alecio}}]{ZagliPavliotisLucarini2023}%
  \BibitemOpen
  \bibfield  {author} {\bibinfo {author} {\bibfnamefont {N.}~\bibnamefont
  {Zagli}}, \bibinfo {author} {\bibfnamefont {G.~A.}\ \bibnamefont
  {Pavliotis}}, \bibinfo {author} {\bibfnamefont {V.}~\bibnamefont {Lucarini}},
  \ and\ \bibinfo {author} {\bibfnamefont {A.}~\bibnamefont {Alecio}},\ }\href
  {\doibase 10.1103/PhysRevResearch.5.013078} {\bibfield  {journal} {\bibinfo
  {journal} {Phys. Rev. Res.}\ }\textbf {\bibinfo {volume} {5}},\ \bibinfo
  {pages} {013078} (\bibinfo {year} {2023})}\BibitemShut {NoStop}%
\bibitem [{\citenamefont {Topaj}\ \emph {et~al.}(2001)\citenamefont {Topaj},
  \citenamefont {Kye},\ and\ \citenamefont {Pikovsky}}]{Topaj2001}%
  \BibitemOpen
  \bibfield  {author} {\bibinfo {author} {\bibfnamefont {D.}~\bibnamefont
  {Topaj}}, \bibinfo {author} {\bibfnamefont {W.-H.}\ \bibnamefont {Kye}}, \
  and\ \bibinfo {author} {\bibfnamefont {A.}~\bibnamefont {Pikovsky}},\ }\href
  {\doibase 10.1103/PhysRevLett.87.074101} {\bibfield  {journal} {\bibinfo
  {journal} {Phys. Rev. Lett.}\ }\textbf {\bibinfo {volume} {87}},\ \bibinfo
  {pages} {074101} (\bibinfo {year} {2001})}\BibitemShut {NoStop}%
\bibitem [{\citenamefont {McKean~Jr.}(1967)}]{McKean2}%
  \BibitemOpen
  \bibfield  {author} {\bibinfo {author} {\bibfnamefont {H.}~\bibnamefont
  {McKean~Jr.}},\ }\href@noop {} {\bibfield  {journal} {\bibinfo  {journal}
  {Stochastic Differential Equations Lecture Series in Differential Equations,
  Session 7, Catholic Univ.}\ } (\bibinfo {year} {1967})}\BibitemShut {NoStop}%
\bibitem [{\citenamefont {Sznitman}(1989)}]{Snitz}%
  \BibitemOpen
  \bibfield  {author} {\bibinfo {author} {\bibfnamefont {A.}~\bibnamefont
  {Sznitman}},\ }\href@noop {} {\emph {\bibinfo {title} {Topics in propagation
  of chaos.}}},\ \bibinfo {series} {Hennequin PL. (eds) Ecole d'Et{\'e} de
  Probabilit{\'e}s de Saint-Flour XIX --- 1989. Lecture Notes in Mathematics},
  Vol.\ \bibinfo {volume} {1464}\ (\bibinfo  {publisher} {Springer, Berlin,
  Heidelberg},\ \bibinfo {year} {1989})\BibitemShut {NoStop}%
\bibitem [{\citenamefont {Chaintron}\ and\ \citenamefont
  {Diez}(2022)}]{ChaintronDiez2022}%
  \BibitemOpen
  \bibfield  {author} {\bibinfo {author} {\bibfnamefont {L.-P.}\ \bibnamefont
  {Chaintron}}\ and\ \bibinfo {author} {\bibfnamefont {A.}~\bibnamefont
  {Diez}},\ }\href {\doibase 10.48550/ARXIV.2203.00446} {\enquote {\bibinfo
  {title} {Propagation of chaos: a review of models, methods and applications.
  i. models and methods},}\ } (\bibinfo {year} {2022})\BibitemShut {NoStop}%
\bibitem [{\citenamefont {Pra}\ and\ \citenamefont
  {Hollander}(1996)}]{PraHollander}%
  \BibitemOpen
  \bibfield  {author} {\bibinfo {author} {\bibfnamefont {P.~D.}\ \bibnamefont
  {Pra}}\ and\ \bibinfo {author} {\bibfnamefont {F.~d.}\ \bibnamefont
  {Hollander}},\ }\href {\doibase 10.1007/BF02179656} {\bibfield  {journal}
  {\bibinfo  {journal} {Journal of Statistical Physics}\ }\textbf {\bibinfo
  {volume} {84}},\ \bibinfo {pages} {735} (\bibinfo {year} {1996})}\BibitemShut
  {NoStop}%
\bibitem [{\citenamefont {Frank}(2005)}]{FrankBook}%
  \BibitemOpen
  \bibfield  {author} {\bibinfo {author} {\bibfnamefont {T.~D.}\ \bibnamefont
  {Frank}},\ }\href@noop {} {\emph {\bibinfo {title} {Nonlinear Fokker-Planck
  Equations Fundamentals and Applications}}}\ (\bibinfo  {publisher} {Springer,
  Berlin, Heidelberg},\ \bibinfo {year} {2005})\BibitemShut {NoStop}%
\bibitem [{\citenamefont {Kress}(2014)}]{Kress2014lie}%
  \BibitemOpen
  \bibfield  {author} {\bibinfo {author} {\bibfnamefont {R.}~\bibnamefont
  {Kress}},\ }\href@noop {} {\emph {\bibinfo {title} {Linear Integral
  Equations}}},\ \bibinfo {edition} {3rd}\ ed.\ (\bibinfo  {publisher}
  {Springer New York},\ \bibinfo {address} {New York, NY},\ \bibinfo {year}
  {2014})\BibitemShut {NoStop}%
\bibitem [{\citenamefont {Battle}(1977)}]{Battle1977}%
  \BibitemOpen
  \bibfield  {author} {\bibinfo {author} {\bibfnamefont {G.~A.}\ \bibnamefont
  {Battle}},\ }\href {\doibase 10.1007/BF01614553} {\bibfield  {journal}
  {\bibinfo  {journal} {Comm. Math. Phys.}\ }\textbf {\bibinfo {volume} {55}},\
  \bibinfo {pages} {299} (\bibinfo {year} {1977})}\BibitemShut {NoStop}%
\bibitem [{\citenamefont {Zubarev}\ \emph {et~al.}(1996)\citenamefont
  {Zubarev}, \citenamefont {Morozov},\ and\ \citenamefont
  {R\"{o}pke}}]{Zubarev1996}%
  \BibitemOpen
  \bibfield  {author} {\bibinfo {author} {\bibfnamefont {D.}~\bibnamefont
  {Zubarev}}, \bibinfo {author} {\bibfnamefont {V.}~\bibnamefont {Morozov}}, \
  and\ \bibinfo {author} {\bibfnamefont {G.}~\bibnamefont {R\"{o}pke}},\
  }\href@noop {} {\emph {\bibinfo {title} {Statistical mechanics of
  nonequilibrium processes. {V}ol. 1}}}\ (\bibinfo  {publisher} {Akademie
  Verlag, Berlin},\ \bibinfo {year} {1996})\ p.\ \bibinfo {pages} {375},\
  \bibinfo {note} {basic concepts, kinetic theory}\BibitemShut {NoStop}%
\bibitem [{\citenamefont {Acebr\'on}\ \emph
  {et~al.}(2005{\natexlab{b}})\citenamefont {Acebr\'on}, \citenamefont
  {Bonilla}, \citenamefont {P\'erez~Vicente}, \citenamefont {Ritort},\ and\
  \citenamefont {Spigler}}]{AcebronBonilla2005}%
  \BibitemOpen
  \bibfield  {author} {\bibinfo {author} {\bibfnamefont {J.~A.}\ \bibnamefont
  {Acebr\'on}}, \bibinfo {author} {\bibfnamefont {L.~L.}\ \bibnamefont
  {Bonilla}}, \bibinfo {author} {\bibfnamefont {C.~J.}\ \bibnamefont
  {P\'erez~Vicente}}, \bibinfo {author} {\bibfnamefont {F.}~\bibnamefont
  {Ritort}}, \ and\ \bibinfo {author} {\bibfnamefont {R.}~\bibnamefont
  {Spigler}},\ }\href {\doibase 10.1103/RevModPhys.77.137} {\bibfield
  {journal} {\bibinfo  {journal} {Rev. Mod. Phys.}\ }\textbf {\bibinfo {volume}
  {77}},\ \bibinfo {pages} {137} (\bibinfo {year}
  {2005}{\natexlab{b}})}\BibitemShut {NoStop}%
\bibitem [{\citenamefont {Gupta}\ \emph {et~al.}(2014)\citenamefont {Gupta},
  \citenamefont {Campa},\ and\ \citenamefont {Ruffo}}]{GuptaCampaRuffo2014}%
  \BibitemOpen
  \bibfield  {author} {\bibinfo {author} {\bibfnamefont {S.}~\bibnamefont
  {Gupta}}, \bibinfo {author} {\bibfnamefont {A.}~\bibnamefont {Campa}}, \ and\
  \bibinfo {author} {\bibfnamefont {S.}~\bibnamefont {Ruffo}},\ }\href
  {\doibase 10.1088/1742-5468/14/08/r08001} {\bibfield  {journal} {\bibinfo
  {journal} {J. Stat. Mech. Theory Exp.}\ ,\ \bibinfo {pages} {R08001, 61}}
  (\bibinfo {year} {2014})}\BibitemShut {NoStop}%
\bibitem [{Note1()}]{Note1}%
  \BibitemOpen
  \bibinfo {note} {We refer the reader to the Supplementary Material for the
  details regarding the calculations regarding this section}\BibitemShut
  {NoStop}%
\bibitem [{\citenamefont {Strogatz}\ and\ \citenamefont
  {Mirollo}(1991)}]{StrogatzMirollo1991}%
  \BibitemOpen
  \bibfield  {author} {\bibinfo {author} {\bibfnamefont {S.~H.}\ \bibnamefont
  {Strogatz}}\ and\ \bibinfo {author} {\bibfnamefont {R.~E.}\ \bibnamefont
  {Mirollo}},\ }\href {\doibase 10.1007/BF01029202} {\bibfield  {journal}
  {\bibinfo  {journal} {J. Statist. Phys.}\ }\textbf {\bibinfo {volume} {63}},\
  \bibinfo {pages} {613} (\bibinfo {year} {1991})}\BibitemShut {NoStop}%
\bibitem [{\citenamefont {Bonilla}\ \emph {et~al.}(1992)\citenamefont
  {Bonilla}, \citenamefont {Neu},\ and\ \citenamefont {Spigler}}]{Bonilla1992}%
  \BibitemOpen
  \bibfield  {author} {\bibinfo {author} {\bibfnamefont {L.~L.}\ \bibnamefont
  {Bonilla}}, \bibinfo {author} {\bibfnamefont {J.~C.}\ \bibnamefont {Neu}}, \
  and\ \bibinfo {author} {\bibfnamefont {R.}~\bibnamefont {Spigler}},\ }\href
  {\doibase 10.1007/BF01049037} {\bibfield  {journal} {\bibinfo  {journal} {J.
  Statist. Phys.}\ }\textbf {\bibinfo {volume} {67}},\ \bibinfo {pages} {313}
  (\bibinfo {year} {1992})}\BibitemShut {NoStop}%
\bibitem [{\citenamefont {Lindenstrauss}\ and\ \citenamefont
  {Tzafriri}(1977)}]{LindenstraussTzafriri1977}%
  \BibitemOpen
  \bibfield  {author} {\bibinfo {author} {\bibfnamefont {J.}~\bibnamefont
  {Lindenstrauss}}\ and\ \bibinfo {author} {\bibfnamefont {L.}~\bibnamefont
  {Tzafriri}},\ }\href@noop {} {\emph {\bibinfo {title} {Classical {B}anach
  spaces. {I}}}},\ Ergebnisse der Mathematik und ihrer Grenzgebiete, Band 92\
  (\bibinfo  {publisher} {Springer-Verlag, Berlin-New York},\ \bibinfo {year}
  {1977})\ pp.\ \bibinfo {pages} {xiii+188}\BibitemShut {NoStop}%
\bibitem [{\citenamefont {Providas}(2021)}]{Providas2021}%
  \BibitemOpen
  \bibfield  {author} {\bibinfo {author} {\bibfnamefont {E.}~\bibnamefont
  {Providas}},\ }\href
  {http://iclibezp1.cc.ic.ac.uk/login?url=https://www.proquest.com/scholarly-journals/unified-formulation-analytical-numerical-methods/docview/2584303993/se-2}
  {\bibfield  {journal} {\bibinfo  {journal} {Algorithms}\ }\textbf {\bibinfo
  {volume} {14}},\ \bibinfo {pages} {293} (\bibinfo {year} {2021})}\BibitemShut
  {NoStop}%
\bibitem [{\citenamefont {Dellwo}(1995)}]{DELLWO1995}%
  \BibitemOpen
  \bibfield  {author} {\bibinfo {author} {\bibfnamefont {D.~R.}\ \bibnamefont
  {Dellwo}},\ }\href {\doibase https://doi.org/10.1016/0377-0427(93)E0273-O}
  {\bibfield  {journal} {\bibinfo  {journal} {Journal of Computational and
  Applied Mathematics}\ }\textbf {\bibinfo {volume} {58}},\ \bibinfo {pages}
  {135} (\bibinfo {year} {1995})}\BibitemShut {NoStop}%
\bibitem [{\citenamefont {Qui{\~n}inao}\ and\ \citenamefont
  {Touboul}(2015)}]{QuininaoTouboul2015}%
  \BibitemOpen
  \bibfield  {author} {\bibinfo {author} {\bibfnamefont {C.}~\bibnamefont
  {Qui{\~n}inao}}\ and\ \bibinfo {author} {\bibfnamefont {J.}~\bibnamefont
  {Touboul}},\ }\href {\doibase 10.1007/s10440-014-9945-5} {\bibfield
  {journal} {\bibinfo  {journal} {Acta Applicandae Mathematicae}\ }\textbf
  {\bibinfo {volume} {136}},\ \bibinfo {pages} {167} (\bibinfo {year}
  {2015})}\BibitemShut {NoStop}%
\bibitem [{\citenamefont {Gallavotti}(2014)}]{G14}%
  \BibitemOpen
  \bibfield  {author} {\bibinfo {author} {\bibfnamefont {G.}~\bibnamefont
  {Gallavotti}},\ }\href@noop {} {\emph {\bibinfo {title} {Nonequilibrium and
  irreversibility}}}\ (\bibinfo  {publisher} {Springer},\ \bibinfo {address}
  {New York},\ \bibinfo {year} {2014})\BibitemShut {NoStop}%
\bibitem [{\citenamefont {Duong}\ and\ \citenamefont
  {Pavliotis}(2018)}]{DuongPavliotis2018}%
  \BibitemOpen
  \bibfield  {author} {\bibinfo {author} {\bibfnamefont {M.}~\bibnamefont
  {Duong}}\ and\ \bibinfo {author} {\bibfnamefont {G.}~\bibnamefont
  {Pavliotis}},\ }\href {\doibase 10.4310/CMS.2018.v16.n8.a7} {\bibfield
  {journal} {\bibinfo  {journal} {Communications in Mathematical Sciences}\
  }\textbf {\bibinfo {volume} {16}},\ \bibinfo {pages} {2199} (\bibinfo {year}
  {2018})}\BibitemShut {NoStop}%
\bibitem [{\citenamefont {Pavliotis}(2014)}]{pavliotisbook2014}%
  \BibitemOpen
  \bibfield  {author} {\bibinfo {author} {\bibfnamefont {G.~A.}\ \bibnamefont
  {Pavliotis}},\ }\href {\doibase 10.1007/978-1-4939-1323-7} {\emph {\bibinfo
  {title} {Book}}},\ Vol.~\bibinfo {volume} {60}\ (\bibinfo  {publisher}
  {Springer, New York},\ \bibinfo {year} {2014})\BibitemShut {NoStop}%
\bibitem [{\citenamefont {Santos~Guti\'errez}\ \emph
  {et~al.}(2021)\citenamefont {Santos~Guti\'errez}, \citenamefont {Lucarini},
  \citenamefont {Chekroun},\ and\ \citenamefont {Ghil}}]{santos2021}%
  \BibitemOpen
  \bibfield  {author} {\bibinfo {author} {\bibfnamefont {M.}~\bibnamefont
  {Santos~Guti\'errez}}, \bibinfo {author} {\bibfnamefont {V.}~\bibnamefont
  {Lucarini}}, \bibinfo {author} {\bibfnamefont {M.~D.}\ \bibnamefont
  {Chekroun}}, \ and\ \bibinfo {author} {\bibfnamefont {M.}~\bibnamefont
  {Ghil}},\ }\href {\doibase 10.1063/5.0039496} {\bibfield  {journal} {\bibinfo
   {journal} {Chaos: An Interdisciplinary Journal of Nonlinear Science}\
  }\textbf {\bibinfo {volume} {31}},\ \bibinfo {pages} {053116} (\bibinfo
  {year} {2021})},\ \Eprint
  {http://arxiv.org/abs/https://doi.org/10.1063/5.0039496}
  {https://doi.org/10.1063/5.0039496} \BibitemShut {NoStop}%
\end{thebibliography}%

\end{document}

