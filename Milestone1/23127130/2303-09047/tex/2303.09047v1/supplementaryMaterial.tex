\documentclass{article}


\usepackage[english]{babel}


\usepackage[letterpaper,top=2cm,bottom=2cm,left=3cm,right=3cm,marginparwidth=1.75cm]{geometry}

% Useful packages
\usepackage{mathtools}
\usepackage{amsmath}
\usepackage{graphicx}
\usepackage{url}
\usepackage{multicol}
\usepackage{wrapfig}
\usepackage{graphicx}
\usepackage[table,xcdraw]{xcolor}
\usepackage{verbatim}
\usepackage{xr}
\usepackage{amsmath}
\usepackage{amsfonts}
\usepackage{subcaption}
\usepackage{dirtytalk}
\usepackage{bbm}
\usepackage[colorlinks=true, allcolors=blue]{hyperref}

\title{Supplementary Material}
\author{}
\date{}
\def \cm#1{ \textcolor{red}{#1} } 
\begin{document}
\maketitle

\section{Dynamical Properties of the Kuramoto model}
We provide here the detailed calculations leading to the results reported in the main text regarding the Kuramoto model. This model describes an ensemble of one dimensional  ($M=1$) phase oscillators on the torus $\mathbb{T}^1 = [0,2\pi]$ whose dynamics is determined by the following Stochastic Differential Equations
\begin{equation}
\label{eq : Kuramoto model}
    \mathrm{d}x_k = \big[ h_k -\frac{\theta}{N} \sum_{j=1}^N \sin\left( x_k - x_j \right) \big] \mathrm{d}t +\sigma \mathrm{d}W_k, \quad k = 1, \dots, N
\end{equation}
where $h_k \sim \mu$, with $\mu(-h)= \mu(h)$, represents the intrinsic frequencies of each oscillator and the symbols are defined as in the main text. The interaction kernel $K(x,y) = \sin(x-y)=\sin(x)\cos(y) - \cos(x)\sin(y)$ is separable and leads, in the thermodynamic limit $N \to + \infty$ \cite{DaiPra,PraHollander}, to the following nonlinear nonlocal Fokker Planck equation of nonlinearity dimension $p=2$
\begin{equation}
    \label{eq appendix: NLFPE Kuramoto}
\partial_t \rho = - \partial_x \big( h - \mathcal{K}(x,\langle \langle \mathbf{A} \rangle \rangle) \big) \rho + \frac{\sigma^2}{2} \partial_{xx}\rho = \mathcal{L}_{\langle \langle \mathbf{A} \rangle \rangle} \rho
\end{equation}
where $\mathcal{K}(x,\langle \langle \mathbf{A} \rangle \rangle) = \sin(x) \langle \langle A_1(x) \rangle \rangle - \cos(x) \langle \langle A_2(x)\rangle \rangle $ where $\mathbf{A}(x) = \left( \cos(x),\sin(x)\right)$ represents the set of reaction coordinates for the Kuramoto model. It is easy to verify that the uniform solution $\rho_0 =\frac{1}{2\pi}$ is always a stationary solution of equation  \eqref{eq appendix: NLFPE Kuramoto} corresponding to a disordered state characterised by a set of observables $\langle \langle \mathbf{A} \rangle \rangle_0 = (0,0)$. Moreover, this stationary solution $\rho_0$ satisfies $\mathcal{L}_{0} \rho_0 = 0$ where 
\begin{equation}
\label{eq: parametrised operator}
\mathcal{L}_{0} \coloneqq \mathcal{L}_{\langle \langle \mathbf{A}\rangle \rangle_0}   = -h \partial_x + \frac{\sigma^2}{2}\partial_{xx}
\end{equation}
Below we investigate the response properties of the disordered state and detect phase transitions of the system by looking at the singularities of the susceptibility of the system originating from non invertibility properties of the matrix $P_{ij}(\omega)$, see equation $(12)$ in the main text. 
We evaluate the matrix $\mathbf{J} \in \mathbb{R}^2$ (here a vector since $M=1$)  
\begin{equation}
\mathbf{J} = \frac{\partial \mathcal{K}}{\partial \langle \langle \mathbf{A} \rangle \rangle} = \left(\sin(x), - \cos(x)\right)
\end{equation}Considering that $\partial \mathbf{J}(x) = \mathbf{A}(x)$, the microscopic Green's function $Y_{ij}(t)$ can be written as $Y_{ij}(t) = \int \mathrm{d}h  \mu(h) Y_{ij}(t;h)$ where
\begin{equation}
Y_{ij}(t;h) = \Theta(t)\int_0^{2\pi} \frac{\mathrm{d}x }{2\pi}A_i(x)e^{t \mathcal{L}_{0}} A_j(x)
\end{equation}Now, it is easy to verify that 
\begin{equation}
\mathbf{A} = - \frac{D}{h^2+D^2}\mathcal{L}_{0} \boldsymbol{\psi}
\end{equation}where $D=\frac{\sigma^2}{2}$ and 
\begin{equation}
\boldsymbol{\psi}(x;h) = 
\begin{pmatrix}
\cos(x) + \frac{h}{D}\sin(x) \\
\sin(x) - \frac{h}{D} \cos(x)
\end{pmatrix} = \mathbf{A}(x) - \frac{h}{D}\partial_x \mathbf{A}(x)
\end{equation}
The microscopic Green Function can then be written as 
\begin{equation}
\label{eq: intrinsic susceptibility}
\begin{split}
Y_{ij}(t;h) &= - \frac{D}{h^2+D^2}\Theta(t) \int_{0}^{2\pi} \frac{\mathrm{d}x}{2\pi} A_i(x) e^{t \mathcal{L}_{0}} \mathcal{L}_{0} \psi_j(x;h) =\\ 
& = -\frac{D}{h^2+D^2} \Theta(t) \frac{\mathrm{d}}{\mathrm{d}t} \int_{0}^{2\pi} \frac{\mathrm{d}x}{2\pi} A_i(x) e^{t \mathcal{L}_{0}} \psi_j(x;h) = \\
&\coloneqq - \frac{D}{h^2+D^2}\Theta(t) \frac{\mathrm{d}}{\mathrm{d}t}C_{ij}(t;h)
\end{split}
\end{equation}where $C_{ij}(t;h)$ is the correlation function between observable $A_i(x)$ and $\psi_j(x;h)$. The corresponding microscopic susceptibility is then given by 
\begin{equation}
Y_{ij}(\omega;h) = \int_{-\infty}^{+\infty} e^{i\omega t}Y_{ij}(t)\mathrm{d}t=\frac{D}{h^2+D^2} \left( C_{ij}(t=0;h)+i\omega C_{ij}(\omega;h)\right)
\end{equation}
where $C_{ij}(\omega;h) = \int_{0}^{+\infty} e^{i\omega t}C_{ij}(t;h)\mathrm{d}t $ is the (one sided) Fourier Transform of $C_{ij}(t;h)$. Since the correlation function at time $t=0$ is $C_{ij}(t=0;h) = \frac{1}{2}\delta_{ij} - \frac{h}{2D}(1-\delta_{ij})$ the microscopic susceptibility associated to $Y_{ij}(t)$ is
\begin{equation}
\label{eq: intrinsic susc Kuramoto}
 Y_{ij}(\omega) = \int Y_{ij}(\omega;h)\mu(h)\mathrm{d}h = \frac{D}{2}\delta_{ij} \int \frac{\mu(h)}{D^2+h^2}\mathrm{d}h + i \omega D \int  \frac{C_{ij}(\omega;h)}{h^2+D^2} \mu(h)\mathrm{d}h
\end{equation}where we have used the fact that the distribution of frequencies is even. The matrix $P_{ij}(\omega)$ describing the response to the observable $\langle \langle A_i(x) \rangle \rangle$ is given by 
\begin{equation}
\label{eq: renormalisation Kuramoto}
P_{ij}(\omega) = \delta_{ij} - \theta Y_{ij}(\omega) = \left( 1 - \frac{\theta D}{2} \int \frac{\mu(h)}{D^2+h^2}\mathrm{d}h \right)\delta_{ij} - i \theta \omega D \int \frac{{C}_{ij}(\omega;h)}{h^2+D^2} \mu(h)\mathrm{d}h
\end{equation}
\subsubsection{Evaluation of the Correlation Function}
The correlation function $C_{ij}(t;h)$ defined in \eqref{eq: intrinsic susceptibility} is completely determined by the spectral properties of the operator $\mathcal{L}_{0}$ which is a linear differential operator with constant coefficients, see equation \eqref{eq: parametrised operator}. As such, its spectral features on the space of functions that are square integrable with respect to the invariant measure $L^2(\mathbb{T};\rho_0)$ is easily found to be given by the eigenfunctions $\phi_k = e^{ikx}$ with relative eigenvalue $\lambda_k = - ihk - k^2 D$ where $k=0, \pm 1, \pm 2, \dots$ . We remark that the eigenfunctions $\phi_k$ are orthonormal in $L^2(\mathbb{T};\rho_0)$ since $\langle \phi_k | \phi_{k'} \rangle_0 = \int_0^{2\pi} \phi_k^*(x) \phi_{k'}(x) \rho_0 = \frac{1}{2\pi}\int_0^{2\pi} e^{i (k' - k)x} = \delta_{kk'}$ where $\langle \cdot | \cdot \rangle_0$ represents the $L^2(\mathbb{T},\rho)$ scalar product. 
We can then obtain a spectral decomposition \cite{EngelNagel2000,ChekrounJSPI} of the operator $e^{t \mathcal{ L}_0}$ as
\begin{equation}
    e^{t\mathcal{L}_0} = \sum_{k = - \infty}^{+\infty} e^{t \lambda_k} \Pi_k
\end{equation}
where $   \Pi_k = | \phi_k \rangle \langle \phi_k | $  is the projector onto the eigenspace in $L^2(\mathbb{T};\rho_0)$ relative to the eigenvalue $\lambda_k$. We then obtain 
\begin{equation}
\begin{split}
C_{ij}(t;h) &= \sum_{k = - \infty}^{+ \infty} e^{t \lambda_k} \int \mathrm{d}x A_i(x) \Pi_k \psi_j(x;h) =
 \sum_{k = - \infty}^{+ \infty} e^{t \lambda_k} \langle \frac{A_i}{\rho_0} | \phi_k \rangle_0 \langle \phi_k | \psi_j \rho_0 \rangle_0 \coloneqq \sum_{k = - \infty}^{+ \infty} e^{t \lambda_k}  \alpha_{ij}^{(k)}(h)
\end{split}
\end{equation}
Since $\lambda_{-k} = \lambda_k^*$ and $\phi_{-k}= \phi_k^*$,  the coefficients $\alpha_{ij}^{(k)}$ satisfy $\alpha_{ij}^{(-k)} = \left(\alpha_{ij}^{(k)}\right)^*$ and the correlation function can be written as 
\begin{equation}
C_{ij}(t;h) = 2 \mathbf{Re} \sum_{k=1}^{+\infty} e^{\lambda_k t}\alpha_{ij}^{(k)} 
\end{equation}where the mode $k=0$ does not contribute to the correlation function, as $\Pi_0$ projects onto the invariant measure $\rho_0 = \frac{1}{2\pi}$ yielding $a_{ij}^{(0)} = 0$. Given the particular structure of the coupling kernel $K(x,y) = sin(x-y)$ selecting the reaction coordinates $\mathbf{A}(x)$, it is possible to show that only the first $k=1$ contributes to the correlation functions, since $\alpha_{ij}^{(k)} = 0$ for $k=2,3,\dots$ . Indeed a simple but lengthy calculation shows that 
\begin{equation}
\label{eq: coefficients}
\alpha_{ij}^{(k)} = \frac{\delta_{k1}}{4} \left(\delta_{ij}  - \frac{h}{D}\left(1 - \delta_{ij} \right) S_{ij} + i \left(  \left(  \left(1- \delta_{ij} \right) A_{ij} - \frac{h}{D}\delta_{ij}\right) \right) \right)
\end{equation}where $\mathbf{S} = \begin{pmatrix} 0 & 1 \\ 1 & 0\end{pmatrix}$ and $\mathbf{A}= \begin{pmatrix} 0 & -1 \\ 1 & 0
\end{pmatrix}$ are two antidiagonal matrices describing the interaction between the observables $A_i$ and $\psi_j$ with $i \neq j$. 
The correlation function can then be written as 
\begin{equation}
C_{ij}(t;h) =  2 \mathbf{Re} \left( e^{\lambda_1 t} \alpha_{ij}^{(1)} \right) = \frac{\delta_{ij}}{2} e^{-Dt} \left( \cos(ht) - \frac{h}{D}\sin(ht) \right) + c_{ij}(t;h)
\end{equation}where $c_{ij}(t;h)$ is an off diagonal contribution and is an odd function of the intrinsic frequency, $c_{ij}(t;-h)=-c_{ij}(t;h)$. Since we have assumed that the intrinsic frequency distribution is even, we observe that equation \eqref{eq: renormalisation Kuramoto} shows that $c_{ij}(t;h)$ provides a vanishing contribution to the matrix $P_{ij}(\omega)$ and in the following calculations it will be neglected. 
By taking the (one sided) Fourier transform we obtain
\begin{equation}
\label{eq: FT correlation transform Kuramoto}
C_{ij}(\omega;h) = \frac{\delta_{ij}}{2} \frac{-D+h^2/D + i \omega}{\omega^2 - (D^2+h^2)+2i \omega D} \coloneqq  C(\omega;h)\delta_{ij},
\end{equation}where we have defined
\begin{subequations}
\begin{align}
\label{eq: def of C}
C(\omega;h) &=  C_R(\omega;h) + i C_I(\omega;h) 
\\
\label{eq: real part C}
C_R(\omega;h) &= \frac{D^2+h^2}{2D} \frac{\omega^2+D^2-h^2}{\left(\omega^2 - \left( D^2 + h^2 \right)  \right)^2 + 4 \omega^2 D^2}, \\
\label{eq: imaginary part C}
C_I (\omega;h)  &= \frac{ \omega}{2}  \frac{\omega^2 +D^2- 3h^2}{\left(\omega^2 - \left( D^2 + h^2 \right)  \right)^2 + 4 \omega^2 D^2}.
\end{align}
\end{subequations}
From equation \eqref{eq: renormalisation Kuramoto} and \eqref{eq: FT correlation transform Kuramoto} we note that the renormalisation matrix is diagonal $P_{ij}(\omega) = P(\omega)\delta_{ij}$, with
\begin{equation}
\label{eq: equation for P}
     P(\omega) = 1 - \frac{\theta D}{2}\int \frac{\mu(h)}{D^2+h^2}\left( 1 + 2i \omega C\left(\omega;h\right)\right) \mathrm{d}h.
\end{equation}
\subsubsection{Phase transitions for the Kuramoto model}
As explained in the main text, phase transitions are associated to settings where the matrix $P_{ij}(\omega^*)$ is not invertible for $\omega^* \in \mathbb{R}$. Given equation \eqref{eq: equation for P}, this is equivalent to the condition $P(\omega^*) = 0$. Now we separate real and imaginary part of the above equation and, considering  \eqref{eq: def of C} and \eqref{eq: equation for P} we get, respectively,
\begin{subequations}
    \begin{align}
        \label{eq: first condition}
\frac{D\theta}{2}& \int \frac{\mu(h)}{h^2+D^2}\left( 1 - 2\omega^* C_I(\omega^*;h) \right) \mathrm{d}h = 1,\\
\label{eq: second condition}
\omega^* &\int \frac{\mu(h)}{h^2+D^2} C_R(\omega^*;h) \mathrm{d}h = 0.
    \end{align}
\end{subequations}
Equation \eqref{eq: second condition} provides the value of $\omega^*$ in terms of the parameters of the system and the properties of the quenched disorder distribution $\mu$ whereas \eqref{eq: first condition} gives the associated transition point, that is a relationship among the parameters of the system $(\theta,D)$ and the properties of $\mu$. In general, we can identify two typical scenarios of phase transitions: static phase transitions associated to a single simple pole $\omega^*=0$ and dynamic phase transitions associated to an opposite pair of pole $\omega_1 = - \omega_2 = \omega^* > 0$. 
\subsubsection{Static phase transitions}
It is immediate to see that $\omega^* = \omega_0 = 0$ is a solution of \eqref{eq: second condition}. Evaluating \eqref{eq: first condition} for $\omega^*=\omega_0 = 0$ yields
\begin{equation}
1 - \frac{\theta D}{2}\int \frac{\mu(h)}{h^2+D^2} \mathrm{d}h = 0
\end{equation}
from which we obtain the critical value of the coupling strength
\begin{equation}
\label{eq: critical strength}
\theta^{stat}_c = 2 \bigg[ \int \frac{D}{h^2 + D^2}\mu(h)\mathrm{d}h \bigg]^{-1} 
\end{equation}We remark that this phase transition is characterised by a vanishing simple pole $\omega_0 =0$, resulting in a static transition where the static susceptibility $\tilde{ \boldsymbol{\chi}}(0) = \left( \mathbf{P}^{-1}\boldsymbol{\chi} \right)\left( 0\right) $ diverges.
\subsubsection{Dynamic phase transitions: properties of $\mu$}
The investigation of the onset of a dynamical phase transition is more complicated as it requires the study of frequency dependent properties of $C(\omega;h)$. Since we are considering a dynamic phase transition characterised by $\omega^* \neq 0$, equation \eqref{eq: second condition} can be written as
\begin{equation}
\label{eq: real part dyn}
    \int \frac{\mu(h)}{h^2+D^2} C_R(\omega^*;h) \mathrm{d}h = \frac{1}{2D}\int \frac{\omega^2+D^2-h^2}{\left(\omega^2 - \left( D^2 + h^2 \right)  \right)^2 + 4 \omega^2 D^2} \mu(h) = 0.
\end{equation}
where in the last equality we have used \eqref{eq: real part C}. 
We can prove that \eqref{eq: real part dyn} implies a known result for the Kuramoto model by noticing that such equation contains the notion that, if $\omega^* \neq 0$ is a solution, then $-\omega^*$ is also a solution. Indeed it is easy to show that for $\omega \neq 0$ the following identity holds
\begin{equation}
 \frac{\omega^2+D^2-h^2}{\left(\omega^2 - \left( D^2 + h^2 \right)  \right)^2 + 4 \omega^2 D^2} = \frac{1}{2\omega} \left(  \frac{\omega+h}{D^2 + (\omega+h)^2} + \frac{\omega - h}{D^2 + (\omega -h )^2}\right)
\end{equation}
so that \eqref{eq: real part dyn} implies that
\begin{equation}
\int  \left( \frac{\omega+h}{D^2 + (\omega+h)^2} + \frac{\omega - h}{D^2 + (\omega -h )^2} \right)  \mu(h)\mathrm{d}h = 0.
\end{equation}
Considering that $\mu(h)$ is an even function the previous equation implies that
\begin{equation}
\label{eq: Kuramoto condition}
    \int \frac{\omega^* \pm h}{D^2 + (\omega^*\pm h)^2} \mu(h) \mathrm{d}h = 0,
\end{equation}
which is a known formula for the Kuramoto model \cite{GuptaCampaRuffo2014}. The above equation shows that the existence of a nonvanishing singularity $\omega^*$ is closely related to the properties of the intrinsic frequency distribution $\mu(h)$. In particular, if $\mu(h)$ is a unimodal function then there does not exist a $\omega^* \neq 0$ satisfying \eqref{eq: Kuramoto condition}. In other words, an even unimodal distribution cannot support the development of a dynamic phase transition.
\subsubsection{Dynamic phase transition for a bimodal frequency distribution}
The simplest setting that can support a dynamic phase transition scenario is represented by a bimodal distribution
\begin{equation}
\mu_{h_0}(h) = \frac{\delta(h+h_0)+\delta(h-h_0)}{2}.
\end{equation}
In this case, \eqref{eq: real part dyn} becomes simply
\begin{equation}
\label{eq: condition bimodal}
    C_R(\omega^*;h_0) = 0
\end{equation}
that, given \eqref{eq: real part C}, yields a solution $\omega^* = \sqrt{h_0^2 - D^2} $. This solution $\omega^*$ is real only if $h_0 > D$, that is, a sufficiently large separation of the peaks of the bimodal distribution is needed, as observed in \cite{Bonilla1992,AcebronBonilla2005}. 
The associated phase transition point is given by \eqref{eq: first condition}, which reads for a bimodal distribution
\begin{equation}
\label{eq: bimodal phase trans cond}
    \frac{\theta D}{2} \left(1 - 2\omega^* C_I(\omega^*;h_0)  \right) = h_0^2 +D^2
\end{equation}
We first evaluate the denominator of $C_I(\omega^*;h_0)$
\begin{equation}
    \left( \omega_*^2 - \left( D^2 + h_0^2 \right)  \right)^2 + 4 \omega_*^2 D^2 =  (2D^2)^2 + 4 D^2 \omega_*^2 = 4D^2(D^2+\omega_*^2)
\end{equation}
 where for the sake of notation clarity we have defined $\omega_*^2 \coloneqq (\omega^*)^2 = h_0^2 -D^2$. We can then write from \eqref{eq: imaginary part C} 
 \begin{equation}
 \label{eq: condition bimodal on imaginary part C}
     C_I(\omega^*;h_0) = \frac{\omega^*}{2} \frac{\omega_*^2 + D^2 -3h_0^2}{ \left( \omega_*^2 - \left( D^2 + h_0^2 \right)  \right)^2 + 4 \omega_*^2 D^2} = -  \omega^* \frac{h_0^2}{4D^2(D^2+\omega_*^2)} = - \frac{\omega^*}{4D^2}
 \end{equation}
 where in the last equality we have used the fact that by definition of $\omega^*$ we have that $D^2 + \omega_*^2 = h_0^2$. Finally, equation \eqref{eq: bimodal phase trans cond} results in 
 \begin{equation}
     \begin{split}
    \frac{\theta D}{2} \left(1 + \frac{\omega^2}{2D^2} \right) &= h_0^2 + D^2, \\
     \frac{\theta D}{2} \frac{2D^2+ \omega_*^2}{2D^2} &= h_0^2 +D^2, \\
     \frac{\theta D}{2} \frac{h_0^2+D^2}{2D^2} &= h_0^2 +D^2, \\
     \frac{\theta D}{2}\frac{1}{2D^2} &= 1,
     \end{split}
 \end{equation}
 yielding eventually the critical phase transition point $\theta_c^{dyn} = 4D$.
\subsection{Characterisation of phase tranisitions: residues}
 The response properties of the system to external perturbations are characterised by the total susceptibility 
\begin{equation}
    \tilde{\chi}_i(\omega) =  \left( \mathbf{P}^{-1} \boldsymbol{\chi} \right)_i\left(\omega\right) = P^{-1}(\omega) \chi_{i}(\omega),
\end{equation}
where in the last equality we have used the fact that $P_{ij}(\omega) = P(\omega) \delta_{ij}$, with $P(\omega)$ given by \eqref{eq: equation for P}, when the distribution of frequencies $\mu(h)$ is an even function. Since the microscopic Green's Function $G_i(t)$ is given by, see equation $(10a)$ in the main text, 
\begin{equation}
    G_i(t) = \int \mathrm{d}h \mu(h)\int \mathrm{d}x A_i(x) e^{t\mathcal{L}_{0} } \mathcal{L}_p \rho_0,
\end{equation}
where $\mathcal{L}_p \rho_0 = - \partial_x \left( X(x) \rho_0 \right)$, we see that a homogeneous perturbation $X(x) = const$ would result in a null response $G_i(t)=0$, and consequently $\chi_i(\omega)=0$, given that the stationary measure is the uniform distribution $\rho=\frac{1}{2\pi}$. We have chosen to perform the numerical response simulations with $X(x) = - \sin(x) = A_2(x)$, so that the microscopic Green's Function is  $G_i(t) = Y_{i1}(t)$ and the corresponding susceptibility is $\chi_i(\omega) = Y_{i1}(\omega)$. 
\subsubsection{Residue for static phase transitions}
A static phase transition is characterised by a simple pole at $\omega=\omega^*=0$, so that the critical response of the reaction coordinate $A_1(x)$ can be written as 
\begin{equation}
\label{eq: susc as pole}
\tilde{\chi}_1(\omega) = \frac{Y_{11}(\omega)}{P(\omega)} =  \frac{Y_{11}(0)}{P'(0)\omega} + \psi(\omega) =  \frac{1/\theta}{P'(0)\omega} + \psi(\omega) \coloneqq \frac{\kappa}{\omega} + \psi(\omega),
\end{equation}
where $\psi(\omega)$ is an analytic function in the upper complex $\omega^*$ plane and we have used the fact that, at a static phase transition, $P(0) = 0$ and $Y_{11}(0)= (1-P(0))/\theta = 1/\theta$. The quantity $\kappa =  \frac{1/\theta}{P'(0)}$ represents the residue of the pole $\omega^*=0$ and determines the properties of the singularity. From \eqref{eq: equation for P} we evaluate the derivative of $P(\omega)$
\begin{equation}
\label{eq: derivative of P}
    P'(\omega) = - i \theta D \int \frac{\mu(h)}{D^2+h^2} \left( C(\omega;h) + \omega C'(\omega;h)\right) \mathrm{d}h
\end{equation}
For a static phase transition we are interested in 
\begin{equation}
    P'(0) = -  i \theta D \int \frac{\mu(h)}{D^2+h^2}C(0;h) \mathrm{d}h = - \frac{i \theta}{2} \int \mu(h) \frac{D^2-h^2}{\left( D^2+h^2 \right)^2} \mathrm{d}h
\end{equation}
where in the last equality we have used that $C(0;h)=  C_R(0;h) +i C_I(0;h) = C_R(0;h) = \frac{1}{2D}\frac{D^2-h^2}{D^2+h^2}$. 
The residue is thus 
\begin{equation}
\label{eq: static residue}
    \kappa = \frac{2i}{\theta^2 \int \mu(h) \frac{D^2-h^2}{\left( D^2+h^2 \right)^2} \mathrm{d}h} = i \frac{\left( \int \frac{D}{h^2+D^2}\mu(h)\mathrm{d}h\right)^2}{2\int  \frac{D^2-h^2}{\left( D^2+h^2 \right)^2}\mu(h) \mathrm{d}h }.
\end{equation}
where we have taken into account that, at the phase transition, $\theta = \theta_c^{stat}$ where $\theta_c^{stat}$ is given by \eqref{eq: critical strength}. The above expression shows that the residue for a static phase transition is completely imaginary.
\\
In the main text, we have investigated the static phase transition arising in the homogeneous Kuramoto model characterised by a delta-distribution of frequency $\mu(h) = \delta(h)$. It is easy to see from \eqref{eq: static residue} that, in this case, the residue is independent of the parameter of the system and equals to $\kappa = \frac{i}{2}$.
\subsubsection{Residue for dynamic phase transitions}
We here investigate the residue for dynamic phase transitions characterised by the development of singularities of the susceptibility for a pair of opposite real frequencies $\omega_1 = - \omega_2 = \omega^* >0$.
\\
A similar argument of the previous section holds but it has to be modified to take into account both poles. We have that
\begin{equation}
    \tilde{\chi}_1(\omega) = \frac{Y_{11}(\omega)}{P(\omega)} = \frac{Y_{11}(\omega^*)}{P'(\omega^*)\left(\omega - \omega^* \right)} +  \frac{Y_{11}(-\omega^*)}{P'(-\omega^*)\left(\omega + \omega^* \right)} + \psi(\omega)
\end{equation}
where $\psi(\omega)$ is an analytic function in the upper complex $\omega$ plane. As before we define the residue $\kappa = \frac{Y_{11}(\omega)}{P'(\omega^*)}$ and we observe that, since the susceptibility, being the Fourier transform of a real function, has to satisfy the condition $\tilde{\chi}_i(-\omega) = \tilde{\chi}_i^*(\omega)$. This implies that $\frac{Y_{11}(-\omega^*)}{P'(-\omega^*)} = - \kappa^*$ 
so that the previous equation can be written as
\begin{equation}
\label{eq: proper equation for poles}
    \tilde{\chi}_1(\omega) = \frac{\kappa}{\omega-\omega^*} - \frac{\kappa^*}{\omega+\omega^*} + \psi(\omega),
\end{equation}
Similarly to the previous section $Y_{11}(\omega^*)=1/\theta$ and the residue is
\begin{equation}
\label{eq: residue dynamic transition formula}
    \kappa = \frac{1}{\theta P'(\omega^*)}
\end{equation}
As opposed to the static phase transition, the evaluation of the residue now requires a more careful study of $P'(\omega^*)$ where $\omega^* >0$. 
\\
We are here interested in the evaluation of the residue for the dynamic phase transition arising from the bimodal distribution $\mu_0$ discussed in the previous sections. We have seen that such transition is associated to a pole $\omega^* = \sqrt{h_0^2 - D^2}$ arising at the transition point $\theta_c^{dyn} = 4D$.
Evaluating \eqref{eq: derivative of P} for a bimodal distribution $\mu_0$ and separating real and imaginary part we obtain the following equations
\begin{subequations}
    \begin{align}
    P'(\omega)&=P_R'(\omega)+i P_I'(\omega) \\
P_R'(\omega^*) &= \frac{\theta D}{\left( D^2+h_0^2\right) }\left(C_I(\omega^*;h_0) + \omega^* C_{I}'(\omega^*;h_0) \right) \\
 P_I'(\omega^*) &= -\frac{\theta D}{\left( D^2 + h_0^2\right)}\left(C_R(\omega^*;h_0) + \omega^* C_R'(\omega^*;h_0) \right) =-\frac{\theta D}{\left( D^2 + h_0^2\right)} \omega^* C_R'(\omega^*;h_0) 
    \end{align}
\end{subequations}
where we have taken into account \eqref{eq: def of C} and in the last equality we have imposed condition \eqref{eq: condition bimodal}. We can evaluate the derivative of $C(\omega;h_0)$ from equations \eqref{eq: def of C},\eqref{eq: real part C} and \eqref{eq: imaginary part C} with similar arguments. We omit the details, as it is just algebra, but we mention that we make extensive use of the fact that $\omega^2_* \coloneqq (\omega^*)^2 = h_0^2 + D^2$. The result is
\begin{subequations}
    \begin{align}
        C'(\omega^*;h_0) &= C_R'(\omega^*;h_0) + i C_I'(\omega^*;h_0) \\
        C_R'(\omega^*;h_0) &= \frac{\omega^*}{4D^3}\frac{D^2+h_0^2}{D^2+\omega_*^2} \\
        \label{eq: derivative of C imaginary bimodal}
        C_I'(\omega^*;h_0) &= - \frac{1}{4\left( D^2+\omega_*^2\right)}
    \end{align}
\end{subequations}
From the above equations it is immediate to evaluate 
\begin{equation}
    P_I'(\omega^*) = - \frac{\omega_*^2}{D^2+\omega_*^2} \frac{\theta}{4D^2} = - \frac{1}{D}\frac{\omega_*^2}{D^2+\omega_*^2}
\end{equation}
where we have used that $\theta = \theta_c^{dyn} = 4D$. We can then proceed to calculate $P'_R(\omega^*)$. Taking into account \eqref{eq: condition bimodal on imaginary part C} and \eqref{eq: derivative of C imaginary bimodal} we get
\begin{equation}
    P'_{R}(\omega^*) = \frac{\theta D}{ \left( D^2 + h_0^2\right)}\left(-\frac{\omega^*}{4D^2} - \frac{\omega^*}{4 \left( D^2 + \omega_*^2\right)}\right) = - \frac{\theta \omega^* \left(2D^2 + \omega_*^2 \right)}{4D\left(D^2 + h_0^2 \right)\left(D^2 + \omega_*^2 \right)} = - \frac{\omega_*}{D^2+\omega_*^2}
\end{equation}
where in the last equality we have used that, by the definition of $\omega^*$, $2D^2+\omega_*^2 = D^2+h_0^2$ and that $\theta = \theta_{c}^{dyn} = 4D$. As a result, we can write that 
\begin{equation}
    P'(\omega^*) = P_{R}'(\omega^*) + P_I'(\omega^*) = - \frac{\omega_*}{D^2+\omega_*^2}\left(1 + i \frac{\omega_*}{D} \right) 
\end{equation}
Since $\theta = \theta_c^{dyn} = 4D$, the residue \eqref{eq: residue dynamic transition formula} turns out to be for the dynamic phase transition associated to the bimodal frequency distribution
\begin{equation}
    \kappa = \kappa_R + i \kappa_I = - \frac{D}{4\omega^*} + \frac{i}{4}
\end{equation}
where $D$ satisfies the property $\theta = 4 D$ identifying the phase transition in the parameter space $(\theta,D)$. As opposed to the static transition, the residue for the dynamic phase transition is not completely imaginary but has also a real part that depends on both the strength of the noise and the critical frequency. Consequently, such singular behaviour is fundamentally different from the static transition as will be highlighted in the next section. 
\subsection{Response for finite $N$: singularities and resonances}
In order to numerically evaluate the response properties of the finite system we perform, following \cite{Marconi2008},  $n$ simulations of equations \eqref{eq appendix: NLFPE Kuramoto} where the initial conditions are sampled according to the unperturbed invariant measure $\rho_0 = \frac{1}{2 \pi}$ and where at time $t = 0$ we apply a perturbation proportional to a Dirac $\delta(t)$ function. We remark that this is a macroscopic perturbation where each agent is perturbed according to $F(x;h) \to F(x;h) + \varepsilon X(x)\delta(t)$, where $F(x;h)=h$ for the Kuramoto model. The average of the response $\langle \langle A_i \rangle \rangle = \langle \langle A_i \rangle \rangle_0 + \varepsilon \langle \langle A_i \rangle \rangle_1 = \varepsilon \langle \langle A_i \rangle \rangle_1$ over the $n$ simulations gives an estimate $\tilde{G}_i$ of the macroscopic response function defined since
\begin{equation}
\label{eq: time convolution}
  \langle \langle A_i \rangle \rangle_1(t) = \left(\tilde{G}_i \star T \right)(t) =  \left(\tilde{G}_i \star \delta \right)(t) = \tilde{G}_i(t)
\end{equation}
where we remark that $\langle\langle A_i \rangle\rangle_0 = 0$ for the Kuramoto model.  Convergence to the linear response regime has been assessed by performing simulations with different values of $\varepsilon$. The total susceptibility $\tilde{\chi}_i(\omega)$ of the system is obtained as 
\begin{equation}
    \tilde{\chi}_i (\omega) = \int_{- \infty}^{+\infty}\tilde{G}_i(t)e^{i \omega t}\mathrm{d}t
\end{equation}
\begin{figure}
    \centering
    \includegraphics[scale=0.5]{PhaseShifts.pdf}
    \caption{Top panels: Response $\langle \langle A_1 \rangle \rangle_1(t)$ in the time domain to a sinusoidal perturbation. The left (right) column refers to a sinusoidal forcing at (double) the critical frequency $\omega^*$. Solid blue lines represent theoretical prediction for the response whereas the red dots represents the numerical response ($N=48000$) obtained as a time convolution according to \eqref{eq: time convolution}. Dashed blue lines represent the forcing $T(t;\hat \omega)$. The amplitude of the numerical response function has been rescaled to unity. Bottom panels: Correlation function (CF) between the sinusoidal forcing and $\langle \langle A_1 \rangle \rangle_1(t)$. Red dots correspond to multiples of $\frac{\pi}{2}$ and the vertical dashed line represents the theoretical value for the phase shift $\varphi ' = \varphi - \frac{\pi}{2}$ given by \eqref{eq: shift}. }
    \label{fig:my_label2}
\end{figure}
For finite $N$ we observe a mollifying effect of the singular behaviour of the susceptibility \cite{ZagliLucariniPavliotis} where the poles $\omega^*$ gets slightly shifted towards the lower half of the complex plane, namely $\pm \omega^* \to \pm \omega^* - i \gamma(N)$ where $\gamma(N) \to 0$ as $N \to + \infty$. Singularities of the susceptibility in the thermodynamic limit correspond to resonances for finite $N$ characterised by 
\begin{equation}
\label{eq: resonances}
    \tilde{\chi}_1(\omega) =  \frac{\kappa}{\omega-\omega^*+ i \gamma(N)} +\psi(\omega) = \frac{\kappa}{\omega-\omega^*+ i \gamma(N)} - \frac{\kappa^*}{\omega+\omega^*+ i \gamma(N)} + \psi '(\omega)
\end{equation}
The resonant behaviour of the susceptibility becomes singular in the thermodynamic limit as, for $N \to + \infty$,
\begin{equation}
\label{eq: asymptotic behaviour}
    \lim_{N \to \infty} \tilde{\chi}_N(\omega) = - i \pi \kappa \delta(\omega-\omega^*) + \kappa \mathcal{P} \left( \frac{1}{\omega-\omega^*} \right) + \psi(\omega).
\end{equation}
where $\mathcal{P}(\frac{1}{\omega})$ represents the Principal Value distribution. For the static transition, the residue is completely imaginary so that $\mathbf{Re}\tilde{\chi}_i(\omega)$ provides the $\delta-$diverging behaviour whereas the imaginary part yields the Principal Value behaviour. In particular, the function $\omega \mathbf{Im}\tilde{\chi}_i(\omega)$ is constant and equals to $|\kappa|$ in a neighborhood of $\omega^*=0$. A visual inspection of Panel $(a)$ (bottom inset) of Figure $1$ in the main text clearly shows that numerical experiments agree with the theoretical prediction $|\kappa| = \frac{1}{2}$ for the homogeneous case $\mu(h) = \delta(h)$.
\\
As for the dynamic transition, the residue has a real part too and the above identification of the two different behaviours of the susceptibility is not as straightforward. More importantly, the non vanishing real part of the residue has strong consequences on the response properties of the system, with the development of a spontaneous phase shift of the response with respect to the forcing. In order to clarify this feature, we evaluate \eqref{eq: resonances} at the critical frequency $\omega=\omega^*$ and neglect non diverging terms in $\gamma(N)$
\begin{equation}
    \tilde{\chi}_1(\omega^*) = \frac{\kappa}{i \gamma(N)} = \frac{\kappa_I}{\gamma(N)} -  i \frac{\kappa_R}{\gamma(N)}.
\end{equation}
We now consider a pure sinusoidal time dependent forcing $T(t;\hat \omega) = \sin(\hat \omega t)$ and evaluate the response  $\langle \langle A_1 \rangle \rangle_1$ through the response formula 
\begin{equation}
\label{eq: basic linear response}
    \langle \langle A_1 \rangle \rangle_1(\omega) = \tilde{\chi}_1(\omega)T(\omega) 
\end{equation}
Given that $T(\omega;\hat \omega)=- i \pi \left( \delta(\omega+\hat \omega) - \delta(\omega - \hat \omega ) \right)$, by taking the anti-Fourier Transform of \eqref{eq: basic linear response} one obtains the response in the time domain
\begin{equation}
\label{eq: response to sinusoidal}
    \langle \langle A_1 \rangle \rangle_1(t;\hat \omega) = - \mathbf{Im}\left( e^{-i\hat \omega t} \tilde{\chi}(\hat \omega)\right) = - \mathbf{Im} \tilde{\chi}(\hat \omega) \cos(\hat \omega t) + \mathbf{Re} \tilde{\chi}(\hat \omega) \sin(\hat \omega t) 
\end{equation}
The response to perturbations at the critical frequency $\omega^*$ is then
\begin{equation}
\label{eq: critical response at critical frequency}
    \langle \langle A_1 \rangle \rangle_1(t;\omega^*) = \frac{1}{\gamma(N)} \left( \kappa_R \cos(\omega^* t) + \kappa_I \sin(\omega^* t) \right) = \frac{|\kappa|^2}{\gamma(N)}\sin\left( \omega^*t + \varphi ' \right)
\end{equation}
where  $\varphi' = \varphi - \frac{\pi}{2}$ and $\varphi$ is related to the ratio of the imaginary and real part of the residue as 
\begin{equation}
\label{eq: shift}
\varphi = \arctan\left(- \frac{\kappa_I}{\kappa_R} \right) = \arctan\left(\frac{\omega^*}{D} \right)  
\end{equation}
Firstly, as expected, the amplitude of the oscillation diverges in the thermodynamic limit as $\gamma(N) \to 0$. Secondly, given that $\kappa_R \neq 0$, the response spontaneously develop a phase shift $\varphi ' \neq 0$ with respect to the external sinusoidal forcing. 

We have investigated the development of the phase shift by comparing the numerical response $\langle \langle A_1 \rangle \rangle_1(t)$ obtained as the convolution product \eqref{eq: time convolution} and the exact theoretical result \eqref{eq: critical response at critical frequency}.  A quantitative estimation of the phase shift is obtained by looking at the correlation function between the forcing and the response. The values at which the correlation function has its maxima correspond to the phase shift. A visual inspection of figure \ref{fig:my_label2} clearly indicates a very good agreement between numerical results and theory. We have also investigated the high frequency behaviour of the critical susceptibility of the system. We expect that $\mathbf{Re}\tilde{\chi}_i(\omega) \approx 0$ for $\omega \to \infty$ so that, from \eqref{eq: response to sinusoidal}, the response will be in quadrature with the forcing $\langle \langle A_1 \rangle \rangle (t) \propto \cos(\omega t)$, as clearly shown in the right column of figure \ref{fig:my_label2} (here we have chosen a frequency $\hat \omega = 2 \omega^*$).
\subsubsection{Parameters for the numerical experiments}
We here report the critical and off-critical parameters for the numerical experiments. 
\begin{itemize}
    \item Static Transition $\mu(h) = \delta(h)$: the coupling strength is fixed$\theta = 1$. The value of the noise strength at the phase transition is $\sigma_c = 1$, whereas away from the transition is $\sigma = 1.1$. 
    \item Dynamic Transition $\mu_{h_0}(h) = \frac{1}{2}\left( \delta(h-h_0) + \delta(h+h_0)\right)$: $h_0= 2$, $\theta = 2$ are fixed. The value of the noise strength at the transition is $\sigma_c = 1$, away from the transition is $\sigma = 1.4$. As a result, the critical frequency is $\omega^* = \sqrt{h_0^2 - D^2} \approx 1.93$ and the phase shift $\varphi ' = \varphi - \frac{\pi}{2}  \approx -0.25$, see equation \eqref{eq: shift}.
\end{itemize}
\begin{thebibliography}{1}

\bibitem{DaiPra}
P.~Dai~Pra, ``Stochastic mean-field dynamics and applications to life
  sciences,'' in {\em Stochastic Dynamics Out of Equilibrium} (G.~Giacomin,
  S.~Olla, E.~Saada, H.~Spohn, and G.~Stoltz, eds.), (Cham), pp.~3--27,
  Springer International Publishing, 2019.

\bibitem{PraHollander}
P.~D. Pra and F.~d. Hollander, ``Mc{K}ean-{V}lasov limit for interacting random
  processes in random media,'' {\em Journal of Statistical Physics}, vol.~84,
  no.~3, pp.~735--772, 1996.

\bibitem{EngelNagel2000}
K.-J. Engel and R.~Nagel, {\em One-parameter semigroups for linear evolution
  equations}, vol.~194 of {\em Graduate Texts in Mathematics}.
\newblock Springer-Verlag, New York, 2000.
\newblock With contributions by S. Brendle, M. Campiti, T. Hahn, G. Metafune,
  G. Nickel, D. Pallara, C. Perazzoli, A. Rhandi, S. Romanelli and R.
  Schnaubelt.

\bibitem{ChekrounJSPI}
M.~D. Chekroun, A.~Tantet, H.~A. Dijkstra, and J.~D. Neelin,
  ``Ruelle--pollicott resonances of stochastic systems in reduced state space.
  part i: Theory,'' {\em Journal of Statistical Physics}, 2020.

\bibitem{GuptaCampaRuffo2014}
S.~Gupta, A.~Campa, and S.~Ruffo, ``Kuramoto model of synchronization:
  equilibrium and nonequilibrium aspects,'' {\em J. Stat. Mech. Theory Exp.},
  no.~8, pp.~R08001, 61, 2014.

\bibitem{Bonilla1992}
L.~L. Bonilla, J.~C. Neu, and R.~Spigler, ``Nonlinear stability of incoherence
  and collective synchronization in a population of coupled oscillators,'' {\em
  J. Statist. Phys.}, vol.~67, no.~1-2, pp.~313--330, 1992.

\bibitem{AcebronBonilla2005}
J.~A. Acebr\'on, L.~L. Bonilla, C.~J. P\'erez~Vicente, F.~Ritort, and
  R.~Spigler, ``The kuramoto model: A simple paradigm for synchronization
  phenomena,'' {\em Rev. Mod. Phys.}, vol.~77, pp.~137--185, Apr 2005.

\bibitem{Marconi2008}
U.~M.~B. Marconi, A.~Puglisi, L.~Rondoni, and A.~Vulpiani,
  ``Fluctuation-dissipation: Response theory in statistical physics,'' {\em
  Phys. Rep.}, vol.~461, p.~111, 2008.

\bibitem{ZagliLucariniPavliotis}
N.~Zagli, V.~Lucarini, and G.~A. Pavliotis, ``Spectroscopy of phase transitions
  for multiagent systems,'' {\em Chaos: An Interdisciplinary Journal of
  Nonlinear Science}, vol.~31, no.~6, p.~061103, 2021.

\end{thebibliography}

\end{document}




