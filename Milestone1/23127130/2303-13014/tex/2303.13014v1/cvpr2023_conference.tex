% CVPR 2023 Paper Template
% based on the CVPR template provided by Ming-Ming Cheng (https://github.com/MCG-NKU/CVPR_Template)
% modified and extended by Stefan Roth (stefan.roth@NOSPAMtu-darmstadt.de)

\documentclass[10pt,twocolumn,letterpaper]{article}

%%%%%%%%% PAPER TYPE  - PLEASE UPDATE FOR FINAL VERSION
% \usepackage[review]{cvpr}      % To produce the REVIEW version
% \usepackage{cvpr}              % To produce the CAMERA-READY version
\usepackage[pagenumbers]{cvpr} % To force page numbers, e.g. for an arXiv version
% \usepackage[accsupp]{axessibility}
% Include other packages here, before hyperref.
\usepackage{graphicx}
\usepackage{amsmath}
\usepackage{amssymb}
\usepackage{booktabs}
\usepackage{multirow}
\usepackage{enumerate}
% It is strongly recommended to use hyperref, especially for the review version.
% hyperref with option pagebackref eases the reviewers' job.
% Please disable hyperref *only* if you encounter grave issues, e.g. with the
% file validation for the camera-ready version.
%
% If you comment hyperref and then uncomment it, you should delete
% ReviewTempalte.aux before re-running LaTeX.
% (Or just hit 'q' on the first LaTeX run, let it finish, and you
%  should be clear).
\usepackage[pagebackref,breaklinks,colorlinks]{hyperref}
% \newcolumntype{H}{>{\setbox0=\hbox\bgroup}c<{\egroup}@{}}
\makeatletter
\def\@fnsymbol#1{\ensuremath{\ifcase#1\or \dagger\or \ddagger\or
   \mathsection\or \mathparagraph\or \|\or **\or \dagger\dagger
   \or \ddagger\ddagger \else\@ctrerr\fi}}
\makeatother

% Support for easy cross-referencing
\usepackage[capitalize]{cleveref}
\crefname{section}{Sec.}{Secs.}
\Crefname{section}{Section}{Sections}
\Crefname{table}{Table}{Tables}
\crefname{table}{Tab.}{Tabs.}


%%%%%%%%% PAPER ID  - PLEASE UPDATE
\def\cvprPaperID{6314} % *** Enter the CVPR Paper ID here
\def\confName{CVPR}
\def\confYear{2023}
\newcommand{\yueqi}[1]{{\color{red}[Yueqi: #1]}}
\newcommand{\mymethod}{\text{S-Ray }}
\begin{document}

%%%%%%%%% TITLE - PLEASE UPDATE
\title{Semantic Ray: Learning a Generalizable Semantic Field with Cross-Reprojection Attention}

% \author{First Author\\
% Institution1\\
% Institution1 address\\
% {\tt\small firstauthor@i1.org}
% % For a paper whose authors are all at the same institution,
% % omit the following lines up until the closing ``}''.
% % Additional authors and addresses can be added with ``\and'',
% % just like the second author.
% % To save space, use either the email address or home page, not both
% \and
% Second Author\\
% Institution2\\
% First line of institution2 address\\
% {\tt\small secondauthor@i2.org}
% }
\author{Fangfu Liu$^{1,3}$, Chubin Zhang$^{2}$, Yu Zheng$^{2}$, Yueqi Duan$^{1\dagger}$\\
$^{1}$Department of Electronic Engineering, Tsinghua University\\
$^{2}$Department of Automation, Tsinghua University\\
$^{3}$Beijing National Research Center for Information Science and Technology\\
{\tt\small \{liuff19,zhangcb19,zhengyu19\}@mails.tsinghua.edu.cn, }
{\tt\small duanyueqi@tsinghua.edu.cn}}

\makeatletter
\let\@oldmaketitle\@maketitle% Store \@maketitle
\renewcommand{\@maketitle}{\@oldmaketitle
\begin{minipage}{\textwidth}
\centering
\vspace{-0.8cm}
\includegraphics[width=1\linewidth]{cvpr2023/Figures/large_teaser_v7.pdf}
\vspace{-6mm}
\captionof{figure}{\textbf{Top}: Comparisons between Semantic-NeRF~\cite{semantic-nerf} and our method Semantic-Ray. Semantic-NeRF (S-NeRF for short) needs to train one specific model for each scene, while our Semantic-Ray (S-Ray for short) trains one unified model on multiple scenes and generalizes to unseen scenes. \textbf{Bottom}: Experimental comparisons between S-Ray and S-NeRF on generalization ability. We observe that our network S-Ray can effectively \textit{fast generalize} across diverse unseen scenes while S-NeRF fails in a new scene. Moreover, our result can be improved by fine-tuning on more images for only 10 min ($2k$ iterations), which achieves comparable quality with the Semantic-NeRF's result for $100k$ iterations per-scene optimization.}
\vspace{-0.2cm}
\label{fig:teaser}
\end{minipage}
\bigskip
}% ... an image

\maketitle

\newcommand\blfootnote[1]{% 
\begingroup 
\renewcommand\thefootnote{}\footnote{#1}% 
\addtocounter{footnote}{-1}% 
\endgroup 
}
\blfootnote{\textsuperscript{\dag}Corresponding author.}
\begin{abstract}
As models continue to grow in size, the development of memory optimization methods (MOMs) has emerged as a solution to address the memory bottleneck encountered when training large models. To comprehensively examine the practical value of various MOMs, we have conducted a thorough analysis of existing literature from a systems perspective. 
% Furthermore, we have evaluated the most widely adopted MOMs employed in mainstream frameworks for both vision and language models.
Our analysis has revealed a notable challenge within the research community: the absence of standardized metrics for effectively evaluating the efficacy of MOMs. The scarcity of informative evaluation metrics hinders the ability of researchers and practitioners to compare and benchmark different approaches reliably. Consequently, drawing definitive conclusions and making informed decisions regarding the selection and application of MOMs becomes a challenging endeavor.
To address the challenge, this paper summarizes the scenarios in which MOMs prove advantageous for model training. We propose the use of distinct evaluation metrics under different scenarios. By employing these metrics, we evaluate the prevailing MOMs and find that their benefits are not universal. We present insights derived from experiments and discuss the circumstances in which they can be advantageous.

\end{abstract}
%%%%%%%%% BODY TEXT
% \section{Introduction}


Recent years have witnessed the rise of human digitization~\cite{habermannDeepCapMonocularHuman2020,alexanderCREATINGPHOTOREALDIGITAL,pengNeuralBodyImplicit2021,alldieckDetailedHumanAvatars2018, rajANRArticulatedNeural2020}. This technology greatly impacts the entertainment, education, design, and engineering industry.
There is a well-developed industry solution for this task.
High-fidelity reconstruction of humans can be achieved either with full-body laser scans~\cite{saitoSCANimateWeaklySupervised2021}, dense synchronized multi-view cameras~\cite{xiangModelingClothingSeparate2021a,xiangDressingAvatarsDeep2022a}, or light stages~\cite{alexanderCREATINGPHOTOREALDIGITAL}.
However, these settings are expensive and tedious to deploy and consist of a complex processing pipeline, preventing the technology's democratization.

Another solution is to view the problem as inverse rendering and learn digital humans directly from custom-collected data.
Traditional approaches directly optimize explicit mesh representation~\cite{loperSMPLSkinnedMultiperson2015, fangRMPERegionalMultiperson2018, pavlakosExpressiveBodyCapture2019} which suffers from the problems of smooth geometry and coarse textures~\cite{prokudinSMPLpixNeuralAvatars2020,alldieckVideoBasedReconstruction2018}. Besides, they require professional artists to design human templates, rigging, and unwrapped UV coordinates.
Recently, with the help of volumetric-based implicit representations~\cite{mildenhallNeRFRepresentingScenes2020, parkDeepSDFLearningContinuous2019, meschederOccupancyNetworksLearning2019} and neural rendering~\cite{laineModularPrimitivesHighPerformance2020, liuSoftRasterizerDifferentiable2019, thiesDeferredNeuralRendering2019}, 
one can easily digitize a quality-plausible human avatar from video footage~\cite{jiangNeuManNeuralHuman2022,wengHumanNeRFFreeviewpointRendering}.
Particularly, volumetric-based implicit representations~\cite{mildenhallNeRFRepresentingScenes2020, pengNeuralBodyImplicit2021} can reconstruct scenes or objects with much higher fidelity against previous neural renderer~\cite{thiesDeferredNeuralRendering2019,prokudinSMPLpixNeuralAvatars2020}, and is more user-friendly as it does not need any human templates, pre-set rigging, or UV coordinates.
Captured visual footage and corresponding skeleton tracking are enough for training.
However, better reconstructions and more friendly usability are at the expense of the following factors.
1) \textbf{Inefficiency:}
They require longer optimization times (typically tens of hours or days) and inference slowly.
Volume rendering~\cite{mildenhallNeRFRepresentingScenes2020,lombardiNeuralVolumesLearning2019} formulates images by querying the densities and colors of millions of spatial coordinates. 
In the training stage, due to memory constraints, only a small fraction of points are sampled which leads to slow convergence speed.
2) \textbf{Entangled representations}:
The geometry, materials, and motion dynamics are entangled in the neural networks. 
Due to the implicit nature of neural nets, one can hardly edit one property without touching the others~\cite{yuanNeRFEditingGeometryEditing2022a,liuEditingConditionalRadiance2021}.
3) \textbf{Graphics incompatibility}:
Volume rendering is incompatible with the current popular graphic pipeline,
which renders triangular/quadrilateral meshes efficiently with the rasterization technique.
Many downstream applications require mesh rasterization in their workflow (\eg, editing~\cite{foundationBlenderOrgHome}, simulation~\cite{benderPositionBasedSimulationMethods2015}, real-time rendering~\cite{akenine2019real}, ray-tracing~\cite{waldRTXRayTracing}).
Although there are approaches~\cite{lorensenMarchingCubesHigh,labelleIsosurfaceStuffingFast2007} can convert volumetric fields into meshes, the gaps from discrete sampling degrade the output quality in terms of both meshes and textures.


To address these issues, we present \textbf{EMA}, a method based on \textbf{E}fficient \textbf{M}eshy neural fields to reconstruct animatable human \textbf{A}vatars.
Our method enjoys flexibility from implicit representations and efficiency from explicit meshes, yet still maintains high-fidelity reconstruction quality.
Given video sequences and the corresponding pose tracking, our method digitizes humans in terms of canonical triangular meshes, physically-based rendering (PBR) materials, and skinning weights \textit{w.r.t.} skeletons.
We jointly learn the above components via inverse rendering~\cite{laineModularPrimitivesHighPerformance2020,chenDIBRLearningPredict2021,chenLearningPredict3D2019} in an end-to-end manner.
Each of them is derived from a separate neural field, which relaxes the requirements of a preset human template, rigging, or UV coordinates.
Specifically, we predict a canonical mesh out of a signed distance field (SDF) by differentiable marching tetrahedra~\cite{shenDeepMarchingTetrahedra2021,gaoGET3DGenerativeModel,gaoLearningDeformableTetrahedral2020,munkbergExtractingTriangular3D2022}, then we extend the marching tetrahedra~\cite{shenDeepMarchingTetrahedra2021} for spatial-varying materials by utilizing a neural field to predict PBR materials \textit{on the mesh surfaces} after rasterization~\cite{munkbergExtractingTriangular3D2022,hasselgrenShapeLightMaterial2022,laineModularPrimitivesHighPerformance2020}.
To make the canonical mesh animatable, we take another neural field to model the forward linear blend skinning for the meshes. 
Given a posed skeleton, the canonical mesh is then transformed into the corresponding poses.
Finally, we shade the mesh with a rasterization-based differentiable renderer~\cite{laineModularPrimitivesHighPerformance2020} and train our models with a photo-metric loss.
After training, we export the mesh with materials and discard the neural fields.

\looseness=-1
There are several merits of our method design.
1) \textbf{Efficiency}:
Powered by efficient mesh rendering, our method can render in real-time.
Besides, the training speed is boosted as well, 
since we compute loss holistically on the whole image and the gradients only flow on the mesh surface. In contrast, volume rendering takes limited pixels for loss computation and back-propagates the gradients in the whole space.
Our method only needs about an hour of training and minutes of optimization are enough for plausible avatar reconstruction.
2) \textbf{Disentangled representations}:
Our shape, materials, and motion modules are disentangled naturally by design, which facilitates editing. 
Besides, Canonical meshes with forward skinning modeling handle the out-of-distribution poses better.
3) \textbf{Graphics compatibility}:
Our derived mesh representation is compatible with 
the prominent graphic pipeline, which leads to instant downstream applications (\eg, the shape and materials can be edited directly in design software~\cite{foundationBlenderOrgHome}).
To further improve reconstruction quality, we additionally optimize image-based environment lights and non-rigid motions.


We conduct extensive experiments on standards benchmarks H36M~\cite{ionescuHuman36MLarge2014b} and ZJU-MoCap~\cite{pengNeuralBodyImplicit2021}.
Our method achieves very competitive performance for novel view synthesis, generalizes better for novel poses, 
and significantly improves both training time and inference speed against previous arts.
Our research-oriented code reaches real-time inference speed (100+ FPS for rendering $512\times512$ images).
We in addition showcase applications including novel pose synthesis, material editing, and relighting.
\section{Introduction}

Recently, Neural Radiance Field (NeRF) \cite{NeRF}, a new novel view synthesis method with implicit representation, has taken the field of computer vision by storm \cite{NeRF_review2022}. NeRF and its variants \cite{NeRF,nerf++, mipnerf, pixelnerf} adopt multi-layer perceptrons (MLPs) to learn continuous 3D representations and utilize multi-view images to render unseen views with fine-grained details. NeRF has shown state-of-the-art visual quality, produced impressive demonstrations, and inspired many subsequent works~\cite{mvsNeRF, GeoNeRF, code_nerf, IBRnet, pixelnerf}.

While the conventional NeRFs have achieved great success in low- and middle-level vision tasks such as neural scene rendering, image synthesis, and multi-view reconstruction \cite{mvsNeRF, neuray, kilo_nerf, Fast_nerf, grf, GIRAFFE, UNISURF}, it is interesting to explore their more possibilities in high-level vision tasks and applications. Learning high-level semantic information from 3D scenes is a fundamental task of computer vision with a wide range of applications \cite{dl_for_medical_img_seg, autonomous_driving_seg, garcia2017review, robotic_semseg}. For example, a comprehensive semantic understanding of scenes enables intelligent agents to plan context-sensitive actions in their environments. One notable attempt to learn interpretable semantic understanding with the NeRF structure is Semantic-NeRF \cite{semantic-nerf}, which regresses a 3D-point semantic class together with radiance and density. Semantic-NeRF shows the potential of NeRF in various high-level tasks, such as scene-labeling and novel semantic view synthesis.

However, Semantic-NeRF follows the vanilla NeRF by estimating the semantic label from a single ray with a new semantic head. While this operation is reasonable to learn low-level information including color and density, a single ray fails to provide rich semantic patterns -- we can tell the color from observing a single pixel, but not its semantic label. To deal with this, Semantic-NeRF heavily relies on positional encoding to learn semantic features, which is prone to overfit the current scene and only applicable to novel \textit{views} within the same scene~\cite{NeSF}. As a result, Semantic-NeRF has to train one model from scratch for every scene independently or provides very limited novel scene generalization by utilizing other pretrained models to infer 2D segmentation maps as training signals for unseen scenes. This significantly limits the range of applications in real-world scenarios.

In this paper, we propose a neural semantic representation called \textbf{Semantic Ray} (S-Ray) to build a generalizable semantic field, which is able to learn from multiple scenes and directly infer semantics on novel viewpoints across novel scenes as shown in Figure~\ref{fig:teaser}. 
% To our best knowledge, this is the first work to learn a generalizable semantic field in real-world scenes. 
As each view provides specific high-level information for each ray regarding of viewpoints, occlusions, etc., we design a Cross-Reprojection Attention module in S-Ray to fully exploit semantic information from the reprojections on multiple views, so that the learned semantic features have stronger discriminative power and generalization ability. While directly performing dense attention over the sampled points on each reprojected ray of multiple views would suffer from heavy computational costs, we decompose the dense attention into intra-view radial and cross-view sparse attentions to learn comprehensive relations in an efficient manner.

More specifically, for each query point in a novel view, different from Semantic-NeRF that directly estimates its semantic label with MLP, we reproject it to multiple known views. It is worth noting that since the emitted ray from the query point is virtual, we cannot obtain the exact reprojected point on each view, but a reprojected ray that shows possible positions. Therefore, our network is required to simultaneously model the uncertainty of reprojection within each view, and comprehensively exploit semantic context from multiple views with their respective significance. To this end, our Cross-Reprojection Attention consists of an intra-view radial attention module that learns the relations among sampled points from the query ray, and a cross-view sparse attention module that distinguishes the same point in different viewpoints and scores the semantic contribution of each view. As a result, our S-Ray is aware of the scene prior with rich patterns and generalizes well to novel scenes. We evaluate our method quantitatively and qualitatively on synthetic scenes from the Replica dataset \cite{replica} and real-world scenes from the ScanNet dataset \cite{scannet}. Experiments show that our S-Ray successfully learns from multiple scenes and generalizes to unseen scenes. By following Semantic-NeRF \cite{semantic-nerf}, we design competitive baselines based on the recent MVSNeRF \cite{mvsNeRF} and NeuRay \cite{neuray} architectures for generalizable semantic field learning. Our S-Ray significantly outperforms these baselines which demonstrates the effectiveness of our cross-reprojection attention module.

% 如果实验结果有时间优势,一定放到这个里面!!!!!

\section{Related Work}
\textbf{Unanswerable Questions. } Unanswerable questions in MRC draw much attention from the research community after the publication of SQuAD 2.0 \cite{rajpurkar-etal-2018-know}. Following the guidelines proposed by \citet{rajpurkar-etal-2018-know}, unanswerable questions in MRC are introduced in MRC of other languages such as French in FQuAD 2.0 \cite{fquad20} and Vietnamese in UIT-ViQuAD 2.0 \cite{viquad20}. The research community commonly refers to unanswerable questions in SQuAD, FQuAD, and UIT-ViQuAD as "artificial unanswerable questions" because annotators are instructed to intentionally create questions that cannot be answered using the information provided in the given context. On the other hand, unanswerable questions that naturally arise are also introduced recently in Natural Questions \cite{kwiatkowski-etal-2019-natural} and TyDi QA \cite{clark-etal-2020-tydi}, in which the evidence documents are provided after the questions are written by annotators.\\
\textbf{Multilingual versus Monolingual Models. } \citet{vulic-etal-2020-probing} probe an empirical analysis on monolingual BERTs and mBERT across six languages and five different lexical tasks. They show that Monolingual BERT encodes significantly more lexical information than mBERT.

Besides, \citet{rust-etal-2021-good} compare pre-trained multilingual language models with monolingual counterparts regarding their monolingual task performances in nine languages and five tasks to reveal the reason for the gap between the performances of monolingual models and multilingual models. This comprehensive analysis later reveals that while pre-training data size played a vital role in the performances of language models on downstream tasks, the monolingual tokenizers designed by native speakers are also an important reason for the high performances of monolingual models in single-language settings. Results from this analysis show that \citet{nguyen-tuan-nguyen-2020-phobert} significantly contributed to the development of Vietnamese language models with a high-quality tokenizer that is suitable for the unique linguistic features of Vietnamese.
\begin{table*}[h]
\centering
\begin{tabular}{@{}llllcc@{}}
\toprule
 &          & EM(\%)    & F1(\%)   & Recall\textsubscript{unanswerable}(\%)  & Recall\textsubscript{answerable}(\%)  \\ \cmidrule(l){2-6} 
\multirow{2}{*}{monolingual}  & WikiBERT & 46.51 & 55.84 & 50.68 & 74.37 \\
 & PhoBERT     & 63.52          & 75.87          & 73.37          & \textbf{89.21} \\ \cmidrule(l){2-6} 
\multirow{3}{*}{multilingual} & mBERT\textsubscript{our}    & 57.66 & 66.84 & 65.84 & 80.47        \\
& mBERT\textsubscript{VLSP}    & 53.55 & 63.03 & - & -        \\
 & XLM-RoBERTa & \textbf{67.84} & \textbf{78.15} & \textbf{75.86} & 88.81 \\ \bottomrule
\end{tabular}
\caption{Performance of models on the UIT-ViQuAD 2.0 Development set}
\label{overall-performane}
\end{table*}

\vspace{-0.3em}
\section{Method}
\vspace{-0.3em}

Our sensitivity-aware visual parameter-efficient fine-tuning consists of two stages. In the first stage, SPT measures the task-specific sensitivity for the pre-trained parameters (Section~\ref{subsec:sensitivity}). Based on the parameter sensitivity and a given parameter budget, SPT then adaptively allocates trainable parameters to task-specific important positions (Section~\ref{subsec:SPT}).

\vspace{-0.3em}
\subsection{Task-specific Parameter Sensitivity}
\label{subsec:sensitivity}
\vspace{-0.3em}

Recent research has observed that pre-trained backbone parameters exhibit varying feature patterns~\cite{raghu2021vision,naseer2021intriguing} and criticality~\cite{zhang2019all,chatterji2019intriguing} at distinct positions. 
Moreover, when transferred to downstream tasks, their efficacy varies depending on how much pre-trained features are reused and how well they adapt to the specific domain gap~\cite{yosinski2014transferable,kumar2022finetuning,neyshabur2020being}. Motivated by these observations, we argue that not all parameters contribute equally to the performance across different tasks in PEFT and propose a new criterion to measure the sensitivity of the parameters in the pre-trained backbone for a given task.

Specifically, given the training dataset $\gD_t$ for the $t$-th task and the pre-trained model weights $\vw=\left\{w_1, w_2, \ldots, w_N\right\}\in \sR^N$ where $N$ is the total number of parameters, the objective for the task is to minimize the empirical risk: $\min_{\vw} E(\gD_t, \vw)$.
We denote the parameter sensitivity \bohan{set} as $\gS=\{s_1, \ldots, s_N\}$ and the sensitivity $s_n$ for parameter $w_n$ is measured by the empirical risk difference when tuning it:
\begin{equation}
\vspace{-0.3em}
    \begin{aligned}
        s_n = E(\gD_t, \vw)-E(\gD_t, \vw\mid w_n=w_n^*),
    \end{aligned}
\label{eq:sensitivity}
\end{equation}
where $w_n^*=\underset{w_n}{\rm argmin}(E(\gD_t, \vw))$. We can reparameterize the tuned parameters as  $w_n^*=w_n+\Delta_{w_n}$, where $\Delta_{w_n}$ denotes the update for $w_n$ after tuning. Here we individually measure the sensitivity of each parameter, which is reasonable given that most of the parameters are frozen during fine-tuning in PEFT. However, it is still computationally intensive to compute Eq.~(\ref{eq:sensitivity}) for two reasons. Firstly, getting the empirical risk for $N$ parameters requires forwarding the entire network $N$ times, which is time-consuming. Secondly, it is challenging to derive $\Delta_{w_n}$, as we have to tune each individual $w_n$ until convergence.

{\begin{algorithm}[t!]
\caption{\label{alg:tps} Computing task-specific parameter sensitivities}
\begin{algorithmic}
    \STATE \textbf{Input:} Pre-trained model with network parameters $\vw$, training set $\gD_t$ for the $t$-th task, and number of training samples $C$ used to calculate the parameter sensitivities
    \STATE \textbf{Output:} Sensitivity set $\gS=\{s_1, \ldots, s_N\}$
    \STATE Initialize $\gS=\{0\}^N$
    \FOR{$i\in\{1,\ldots,C\}$}
        \STATE Get the $i$-th training sample of $\gD_t$
	    \STATE Compute loss $E$
		\STATE Compute gradients $\vg$
		\FOR{$n\in\{1,\ldots,N\}$}
                \STATE Update sensitivity for the $n$-th parameter: $s_{n} = s_{n} + g_n^2$
		    \ENDFOR
    \ENDFOR
\end{algorithmic}
\end{algorithm}}


\begin{figure*}[t]
\begin{center}
    \includegraphics[width=\linewidth]{main_figure.pdf}
\end{center}\vspace{-2em}
\caption{Overview of our trainable parameter allocation strategy. With the parameter sensitivity \bohan{set} $\gS$, we first get the top-$\tau$ sensitive parameters. Instead of directly tuning these sensitive parameters, we also boost the representational capability by replacing unstructured tuning with structured tuning at sensitive weight matrices that have a large number of sensitive parameters, which can be implemented by an existing structured tuning method, \eg, LoRA~\cite{hu2022lora} and Adapter~\cite{houlsby2019parameter}. Red lines and blocks represent trainable parameters and modules, while blue lines represent frozen parameters.}
\label{fig:main}
\vspace{-1.5em}
\end{figure*}


To overcome the first barrier, we simplify the empirical loss by approximating $s_n$ in the vicinity of $\vw$ by its first-order Taylor expansion
\vspace{-0.3em}
\begin{equation}
\vspace{-0.5em}
    \begin{aligned}
        s_n^{(1)} = -g_n\Delta_{w_n},
    \end{aligned}
\label{eq:first-order}
\end{equation}
where the gradients $\vg=\partial E/\partial\vw$, and $g_n$ is the gradient of the $n$-th element of $\vg$. 
To address the second barrier, following~\cite{liu2018darts,cai2018proxylessnas}, we take the one-step unrolled weight as the surrogate for $w_n^*$ and approximate $\Delta_{w_n}$ in Eq.~(\ref{eq:first-order}) with a single step of gradient descent. We can accordingly get $s_n^{(1)} \approx g_n^2\epsilon$,
where $\epsilon$ is the learning rate. Since $\epsilon$ is the same for all parameters, we can eliminate it when comparing the sensitivity with the other parameters and finally get 
\vspace{-0.5em}
\begin{equation}
\vspace{-0.3em}
    \begin{aligned}
        s_n^{(1)} \approx g_n^2.
    \end{aligned}
\label{eq:first-order-simp}
\end{equation}
Therefore, the sensitivity of a parameter can be efficiently measured by its potential to reduce the loss on the target domain. Note that although our criterion draws inspiration from pruning work~\cite{molchanov2019importance}, it is distinct from it. \cite{molchanov2019importance} measures the parameter importance by the squared change in loss when removing them, \ie, $\left( E(\gD_t, \vw)-E(\gD_t, \vw\mid w_n=0) \right)^2$ and finally derives the parameter importance by $\left( g_n w_n \right)^2$, which is different from our formulations in Eqs.~(\ref{eq:sensitivity}) and~(\ref{eq:first-order-simp}).

In practice, we accumulate $\gS$ from a total number of $C$ training samples ahead of fine-tuning to generate accurate sensitivity as shown in Algorithm~\ref{alg:tps}, where $C$ is a pre-defined hyper-parameter. In Section~\ref{subsec:abl}, we show that employing only 400 training samples is sufficient for getting reasonable parameter sensitivity, which requires only 5.5 seconds with a single GPU for any VTAB-1k dataset with ViT-B/16 backbone~\cite{vit}.

\vspace{-0.3em}
\subsection{Adaptive Trainable Parameters Allocation}
\label{subsec:SPT}
\vspace{-0.2em}

Our next step is to allocate trainable parameters based on the obtained parameter sensitivity set $\gS$ and a desired parameter budget $\tau$. A straightforward solution is to directly tune the top-$\tau$ most sensitive unstructured connections (parameters) \rev{while keeping the rest frozen}, which we name unstructured tuning. Specifically, we select the top-$\tau$ most sensitive weight connections in $\gS$ to form the sensitive weight connection set $\gT$. Then, for \rev{a} weight matrix $\mW\in \sR^{d_{\rm in}\times d_{\rm out}}$, we can get a binary mask $\mM\in \sR^{d_{\rm in}\times d_{\rm out}}$ computed by
\vspace{-0.5em}
\begin{equation}
\vspace{-0.5em}
    {\begin{array}{ll}
    \small
    \begin{aligned}
    \mM^j =
    \left\{\begin{array}{ll} 
    1 ~~~~~ \mW^j \in \gT \\
    0 ~~~~~ \mW^j \notin \gT
    \end{array}\right.
    \end{aligned},
    \small
    \end{array}}
\label{eq:mask}
\end{equation}
where $\mW^j$ and $\mM^j$ are the $j$-th element in $\mW$ and $\mM$, respectively. Accordingly, we can train the sensitive parameters by gradient descent and the updated weight matrix can be formulated as $\mW'\leftarrow \mW - \epsilon\vg_{\mW}\odot\mM$, where $\vg_{\mW}$ is the gradient for $\mW$.

However, considering PEFT approaches generally limit the proportion of trainable parameters to less than 1\%, tuning only a small number of unstructured weight connections might not have enough representational capability to handle the downstream datasets with large domain gaps from the source pre-training data. Therefore, to improve the representational capability, we propose to replace unstructured tuning with structured tuning at the sensitive weight matrices that have a high number of sensitive parameters. To preserve the parameter budget, we can implement structured tuning with an existing efficient structured tuning PEFT method~\cite{hu2022lora,chen2022adaptformer,houlsby2019parameter,jie2022convolutional} that learns to directly adjust \rev{all hidden dimensions at once}. We depict an overview of our trainable parameter allocation strategy in Figure~\ref{fig:main}. For example, we can employ the low-rank reparameterization trick LoRA~\cite{hu2022lora} to the sensitive weight matrices \rev{and the one-step update for $\mW$ can be formulated as}
\vspace{-0.4em}
\begin{equation}
\vspace{-0.4em}
    {\begin{array}{ll}
    \small
    \begin{aligned}
    \mW' = \left\{\begin{array}{ll} 
    \mW + \mW_{\rm down}\mW_{\rm up} & ~~ \text { if } ~~ \sum_{j=0}^{d_{\rm in}\times d_{\rm out}} \mM^j \geq \sigma_{\rm opt} \\
    \mW - \epsilon\vg_{\mW}\odot\mM & ~~ {\rm otherwise}
    \end{array}\right.
    \end{aligned},
    \small
    \end{array}}
\label{eq:weight_updat}
\end{equation}
where $\mW_{\rm down}\in \sR^{d_{\rm in}\times r}$ and $\mW_{\rm up}\in \sR^{r\times d_{\rm out}}$ are two learnable low-rank matrices to approximate the update of $\mW$ and rank $r$ is a hyper-parameter where $r \ll {\rm min}(d_{\rm in},d_{\rm out})$. In this way, we perform structured tuning on $\mW$ when its number of sensitive parameters exceeds $\sigma_{\rm opt}$, whose value depends on the pre-defined type of structured tuning method. For example, since implementing structured tuning with LoRA requires $2\times d_{\rm in} \times d_{\rm out} \times r$ trainable parameters for each sensitive weight matrix, we set $\sigma_{\rm LoRA} \leftarrow 2\times d_{\rm in} \times d_{\rm out} \times r$ to ensure that the number of trainable parameters introduced by structured tuning is always equal to or lower than the number of sensitive parameters.

In this way, our SPT adaptively incorporates both structured and unstructured tuning granularities to enable higher flexibility and stronger representational power, simultaneously. In Section~\ref{subsec:abl}, we show that structured tuning is important for the downstream tasks with larger domain gaps and both unstructured and structured tuning contribute clearly to the superior performance of our SPT.
\begin{table*}[t!]
\begin{minipage}{0.175\linewidth}
\centering
% \hspace{1.8mm}
\captionof{table}{\small Datasets statistics \label{graph_datasets}}
\begin{tiny}
\begin{tabular}{c||c|c}
      {\bf Graph} & {\bf \#nodes} & {\bf \#edges} \\ \hline
      {\em FL} & 80\,513     & 5\,899\,882 \\
      {\em YT} & 1\,138\,499 & 2\,990\,443 \\
      {\em LJ} & 2\,238\,731 &14\,608\,137 \\
      {\em OR} & 3\,072\,441 & 117\,185\,083 \\
      {\em TW} & 41\,652\,230 & 1\,468\,365\,182 \\
\end{tabular}
\end{tiny}
\end{minipage}%
 \quad
 \begin{minipage}{.265\linewidth}
\centering
\tabcolsep=0.05cm

\captionof{table}{\small Avg. memory footprint (GB) of {\sf DistGER} and {\sf KnightKing} on each machine, where $\sigma$ is the standard deviation}
\label{Memory_usage}
\begin{tiny}
\newcommand{\tabincell}[2]{\begin{tabular}{@{}#1@{}}#2\end{tabular}}
  % \caption{\small {\color{blue} Avg. memory footprint (GB) of {\sf DistGER} and {\sf KnightKing} on each machine, where $\sigma$ is the standard deviation.}}
  \begin{tabular}{c|cc|cc}
    %\hline
    { }&\multicolumn{2}{c|}{\bfseries{ Sampling}}&\multicolumn{2}{c}{\bfseries{Training}}\\
    \hline
    {\bf{Graph}} &{\sf KnightKing} &{\sf DistGER} &{\sf KnightKing} &{\sf DistGER} \\
    \hline
     {\em FL} & 0.66($\pm$0.06)	&{\bf 0.41($\pm$0.02)}	&1.31($\pm$0.17) 	&{\bf 0.86($\pm$0.06)} 	\\

     {\em YT} &4.11($\pm$0.55)	&{\bf 1.36($\pm$0.23)} 	&4.73($\pm$0.72) 	&{\bf 4.26($\pm$0.63)} \\

     {\em LJ} & 7.65($\pm$0.82)	&{\bf 1.95($\pm$0.16)}	&6.38($\pm$0.97) 	&{\bf 5.49($\pm$0.85)} 	\\

     {\em CO} &10.98($\pm$1.03)	&{\bf 3.27($\pm$0.79)} 	&8.52($\pm$1.01) 	&{\bf 6.86($\pm$0.69)} 	\\

     {\em TW} & out-of-memory	&{\bf 20.18($\pm$3.62)} 	&out-of-memory 	& {\bf 67.16($\pm$5.18)} 	\\
  %\hline
\end{tabular}
\end{tiny}

\end{minipage}
\quad
\begin{minipage}{.25\linewidth}
    \centering
    \includegraphics[width= 1.85in, height = 1.2in]{./Figures/Dist_total_time_partition.eps}%
    \captionof{figure}
      {\small Efficiency: {\sf PBG} \cite{PBG_2019}, {\sf DistDGL} \cite{DistDGL_2020}, {\sf KnightKing} \cite{KnighKing_2019}, {\sf HuGE-D} (baseline), {\sf DistGER} (ours)
        \label{overall_performance}
      }
\end{minipage}%\hfill
\quad
\begin{minipage}{.25\linewidth}
    \centering
    \includegraphics[width= 1.85in, height = 1.2in]{./Figures/Dist_scalability_partition.eps}%
    \captionof{figure}
      {\small Scalability: {\sf PBG} \cite{PBG_2019}, {\sf DistDGL} \cite{DistDGL_2020}, {\sf KnightKing} \cite{KnighKing_2019}, {\sf HuGE-D} (baseline), {\sf DistGER} (ours)
        \label{Dist_scalability}
      }
\end{minipage}
\end{table*}


\section{Experimental Results}
\label{sec:experiments}
We evaluate the efficiency (\S \ref{sec:overall}) and scalability (\S \ref{sec:scalability}) of our proposed method, {\sf DistGER}
by comparing with {\sf HuGE-D} (baseline),
{\sf KnightKing} \cite{KnighKing_2019}, {\sf PyTorch-BigGraph} ({\sf PBG}) \cite{PBG_2019}, and {\sf Distributed DGL}
({\sf DistDGL}) \cite{DistDGL_2020}. We also compare the effectiveness (\S \ref{sec:effectiveness}) of generated embeddings
on link prediction.
% and multi-label classification tasks. 
Finally, we analyze efficiency due to individual
parts of {\sf DistGER} (\S \ref{sec:individual})
and the generality of {\sf DistGER} for other random walk-based embeddings (\S \ref{sec:generality}).
Our codes and datasets are at \cite{code}.
%
\subsection{Experimental Setup}
\label{sec:setup}
%
\spara{Environment.} We conduct experiments on a cluster of 8 machines with 2.60GHz Intel $^\circledR$ Xeon $^\circledR$ Gold 6240 CPU with 72 cores (hyper-threading)
in a dual-socket system, and each machine is equipped with 192GB DDR4 memory and connected by a 100Gbps network.
The machines run Ubuntu 16.04 with Linux kernel 4.15.0. We use GCC v9.4.0 for compiling {\sf DistGER}, {\sf KnightKing}, and {\sf HuGE-D},
and use Python v3.6.15 and torch v1.10.2 as the backend deep learning framework for {\sf Pytorch-BigGraph} and {\sf DistDGL}.

\spara{Datasets. } We employ five widely-used, real-world graphs
(Table~\ref{graph_datasets}): {\em Flickr} (FL) \cite{Flickr_Youtube_Graph},
{\em Youtube} (YT) \cite{Flickr_Youtube_Graph},
{\em LiveJournal} (LJ) \cite{BlogCatalog_Twitter_LiveJournal_Graph},
{\em Com-Orkut} (OR) \cite{com-orkut_2012}, and {\em Twitter} (TW) \cite{twitter_2010}.
The first two graphs are selected for multi-label node classification with distinct number of node labels 195 and 47, respectively, %in {\em Flickr} and {\em Youtube},
where labels in {\em Flickr} represent interest groups of users, and {\em Youtube}'s labels represent groups of viewers that enjoy common video genres. The last four graphs are used in link prediction. We also use synthetic graphs \cite{RMAT_2004} (up to 1 billion nodes, 10 billion edges) and a real-world {\em UK graph} \cite{BSVLTAG} (100M nodes, 3.7B edges) to assess the scalability of {\sf DistGER}.
Considering the default settings of popular random walk-based methods (e.g., Deepwalk, node2vec, HuGE), we use their undirected version.

\spara{Competitors.} We compare {\sf DistGER} against three state-of-the-art distributed graph embedding frameworks: the distributed random walk engine, {\sf KnightKing} {\scriptsize\url{https://github.com/KnightKingWalk/KnightKing}}
\cite{KnighKing_2019}; the distributed multi-relations based graph embedding system, {\sf PyTorch-BigGraph} ({\sf PBG})
{\scriptsize\url{https://github.com/facebookresearch/PyTorch-BigGraph}} \cite{PBG_2019} -- designed by Facebook; and
the distributed graph neural networks-based system, {\sf DistDGL} {\scriptsize\url{https://github.com/dmlc/dgl}}
\cite{DistDGL_2020} -- recently proposed by Amazon. We also implement {\sf HuGE-D}, a distributed version of
information-centric random walk-based graph embedding ({\sf HuGE} \cite{HuGE_2021}), on top of {\sf KnightKing},
served as our baseline. Since {\sf KnightKing} and {\sf HuGE-D} provide distributed support only for
random walk without that for embedding learning, we generate their node embeddings using
{\sf Pword2vec} {\scriptsize\url{https://github.com/IntelLabs/pWord2Vec}} \cite{Pword2vec_2019},
the most popular distributed {\sf Skip-Gram} system released by Intel.
%We find that {\sf pSGNScc} \cite{pSGNSCC_2017} (\S \ref{sec:learning})
%only provides a single-machine implementation, thus we do not include it in our distributed experiments.

\spara{Parameters.} For {\sf DistGER} and {\sf HuGE-D} random walks, we set
parameters $\mu$=0.995, $\delta$=0.001 based on information measurements (\S \ref{sec:preliminaries}),
while {\sf KnightKing} uses $L$=80 and $r$=10 that are routine configurations in the traditional
random walk-based graph embedding \cite{node2vec_2016, DeepWalk_2014, KnighKing_2019}. For {\sf DistGER}, {\sf KnightKing}, and {\sf HuGE-D} training,
we set the sliding window size $w$=10, number of negative samples $K$=5, and synchronization period=0.1 sec \cite{Pword2vec_2019},
and additionally, multi-windows number=2, $\gamma$=2 for {\sf DisrGER}.
%For {\sf Pytorch-BigGraph} ({\sf PBG}), we set the number of partitions to 16 following \cite{PBG_2019}, that is, using $2m$ partitions for the number of machines $m$ = 8
%in our case. %For {\sf DistDGL}, the deployed {\sf GaphSAGE} model uses three graph convolutional layers.
For fair comparison across all systems, %the efficiency performance of all systems involved in the experiments,
we set the embedding dimension $d$=128 that is commonly used \cite{HuGE_2021,node2vec_2016,DeepWalk_2014,Line_2015,Verse_2018,ProNE_2019},
and report the average running time for each epoch. For task effectiveness evaluations,
we find the best results from a grid search over learning rates from 0.001-0.1, \# epochs from 1-30,
and \# dimensions from 128-512.


%
\eat{
\begin{table}
\newcommand{\tabincell}[2]{\begin{tabular}{@{}#1@{}}#2\end{tabular}}
  \caption{\small Avg. memory footprint (GB) of {\sf DistGER} and {\sf KnightKing} on each machine, where $\sigma$ is the standard deviation.}
  \label{Memory_usage}
  \begin{center}
   \footnotesize
  \begin{tabular}{c|cc|cc}
    %\hline
    { }&\multicolumn{2}{c|}{\bfseries{ Sampling}}&\multicolumn{2}{c}{\bfseries{Training}}\\
    \hline
    {\bf{Graph}} &{\sf KnightKing} &{\sf DistGER} &{\sf KnightKing} &{\sf DistGER} \\
    \hline
     {\em Flickr} & 0.66($\pm$0.06)	&{\bf 0.41($\pm$0.02)}	&1.31($\pm$0.17) 	&{\bf 0.86($\pm$0.06)} 	\\

     {\em Youtube} &4.11($\pm$0.55)	&{\bf 1.36($\pm$0.23)} 	&4.73($\pm$0.72) 	&{\bf 4.26($\pm$0.63)} \\

     {\em LiveJournal} & 7.65($\pm$0.82)	&{\bf 1.95($\pm$0.16)}	&6.387($\pm$0.97) 	&{\bf 5.49($\pm$0.85)} 	\\

     {\em Com-Orkut} &10.98($\pm$1.03)	&{\bf 3.27($\pm$0.79)} 	&8.52($\pm$1.01) 	&{\bf 6.86($\pm$0.69)} 	\\

     {\em Twitter} & out-of-memory	&{\bf 37.1($\pm$5.28)} 	&out-of-memory 	& {\bf 79.5($\pm$7.27)} 	\\
  %\hline
\end{tabular}
\end{center}
\end{table}
%
}
%


%
\subsection{Efficiency and Memory Use w.r.t. Competitors}
\label{sec:overall}
%\begin{figure}
%  \centering
%  \includegraphics[width= 3 in]{Dist_total_time.eps}
%  \caption{\small Overall performance of PBG, DistDGL, KnightKing, HuGE-D and DistGER for generating embeddings on different read-word graphs, {\color{blue}for Twitter graph, DistDGL cannot finish in one day, and KnightKing fails to perform due to memory issue, where the y axis is in log-scale.}}
%  \label{overall_performance}
%\end{figure}
%
We report the end-to-end running times of {\sf PBG}, {\sf DistDGL}, {\sf KnightKing}, {\sf HuGE-D}, and {\sf DistGER}
on five real-world graphs with the cluster of 8 machines in Figure~\ref{overall_performance}.
The reported end-to-end time includes the running time of partitioning, random walks (for random walk-based frameworks), and training procedures.
%{\color{blue} Noted that the reported end-to-end time in our experiments excludes the partition time for all evaluated frameworks due to the all used partition schemes are executed as a preprocessing component, and we separately evaluate the partition efficiency in Section 6.5, thus the end-to-end time refers to the running time of random walk (only for random walk-based framework) and training procedure.}
{\sf DistGER} significantly outperforms the competitors
on all these graphs, achieving a speedup ranging from 2.33$\times$ to 129$\times$. %, by an average acceleration of $39.78 \times$.
Recall that {\sf DistGER} is a similar type of system as {\sf KnightKing} and {\sf HuGE-D},
and our key improvements are discussed in \S \ref{sec:DistGER} and in \S \ref{sec:learning}.
Analogously, Figure~\ref{overall_performance} exhibits that our system, %designs are more effective (see more evaluation details in \S 6.3),
{\sf DistGER} achieves an average speedup of 9.25$\times$ and 6.56$\times$ compared with {\sf KnightKing} and {\sf HuGE-D}.
Notice that we fail to run {\sf KnightKing} on the largest {\em Twitter} dataset
because its routine random walk strategy requires more main memory space.
%Although Huge-D achieves comparable performance,
The advantage of information-centric random walk in {\sf HuGE} is almost wiped out in {\sf HuGE-D}
due to on-the-fly information measurements and the higher communication costs in a distributed setting.
The multi-relation-based {\sf PBG} leverages a parameter server to synchronize embeddings between clients,
resulting in more load on the communication network. As a result, {\sf PBG} is on average
26.22$\times$ slower than {\sf DistGER}. For graph neural network-based system {\sf DistDGL},
due to the long running time of graph sampling (e.g., taking 80\% of the overhead for the {\sf GraphSAGE}),
it is highly inefficient than other systems. For the billion-edge {\em Twitter} graph, it does not terminate in 1 day.
%
%Considering the resource consumption that affects scalability,
{Table ~\ref{Memory_usage}} shows {\sf DistGER}'s average memory footprint on each machine of the 8-machine cluster. %from a cluster of 8 machines.
%and the standard deviation %($\sigma$) %of the results
%in 
%Table ~\ref{Memory_usage}. 
Compared
to %other methods, %with the 
same type of system
{\sf KnightKing}, 
% that is of the same system type, 
{\sf DistGER} requires less memory for sampling and training.


\subsection{Scalability w.r.t. Competitors}
\label{sec:scalability}
%
%\begin{figure}
%  \centering
%  \includegraphics[width= 3.2 in]{Dist_scalability.eps}
%  \caption{\small Scalability comparison on LiveJournal graph, where the y axis is in log-scale.}
% \label{Dist_scalability}
%\end{figure}
%
Figure~\ref{Dist_scalability} shows end-to-end running times of all competing
systems on the {\em LiveJournal} graph, as we increase \# machines
from 1 to 8 to evaluate scalability. {\sf DistGER} achieves better scalability than the other
four distributed systems.
%Due to space limitation, we omit results on other graph datasets,
%which exhibit similar trends.
{\sf PBG} leverages a parameter server and a shared network filesystem
to synchronize the parameters in the distributed model. %The edges are partitioned into $m^2$ buckets
%and training can be performed in parallel using up to $m/2$ machines. After one bucket completes
%the training, it needs to communicate with the parameter server.
When the number of machines increases, {\sf PBG} puts more load
on the communications network, resulting in poor scalability. Likewise, {\sf DistDGL}
is bounded by the synchronization overhead for gradient updates,
limiting its scalability.
%Since {\sf DistDGL} uses mini-batches for sampling, %features %for GraphSAGE,
%if the mini-batch samples cannot be generated on time, the trainer will be delayed on the forward pass, and all other
%machines need to wait before starting the backward pass. Thus, increasing the number of
%machines also affects the efficiency of backward pass. %Being the random walk-based distributed systems,
Both {\sf KnightKing} and {\sf HuGE-D} suffer from higher communication costs during random walks,
due to their only workload-balancing partitioning scheme (\S \ref{sec:dRand}, \S \ref{sec:individual}).
%Their scalability is relatively poor as the number of machines increases.
%{\sf KnightKing} partitions the graph by a workload-balancing scheme, inevitably introducing higher
%cross-machine communications due to the randomness inherent in the random walking procedure (\S \ref{sec:dRand}, \S \ref{sec:individual}).
%With more machines, the inefficiency of the partitioning scheme is further magnified.
Since {\sf HuGE-D} is implemented on top of {\sf KnigtKing},
it exhibits worse scalability due to high communication costs and on-the-fly information measurements in a distributed setting (\S \ref{sec:HUGED}).
%In contrast, its performance is much better than all the competitors in a single machine.
In comparison, {\sf DistGER} incorporates multi-proximity-aware streaming graph partitioning and incremental computations
to reduce both communication and computation costs, it also employs hotness-block based parameters synchronization
during training to dramatically reduce the pressure on network bandwidth. Hence, {\sf DistGER} achieves better scalability than other systems.
Due to space limitations, we omit {\sf DistGER}'s scalability results on other graphs, which exhibit similar trends. On {\em Twitter}, the end-to-end running times {\sf DistGER} on 1, 2, 4, and 8 machines are 3090s, 1739s, 1197s, and 746s, respectively,
while on {\em Com-Orkut}, the results are 304s, 204s, 149s, and 89s, respectively. 
The results show a good linear relationship.
% The results demonstrate a desired scalability with the increase of the machines.

\begin{table}
\quad
\begin{minipage}{0.46\linewidth}
    \centering
    \includegraphics[width= 1.6 in]{./Figures/Dist_scalability_datasize.eps}%
    \captionof{figure}
      {\small {Scalability of {\sf DistGER} on synthetic graphs, where Y-axis is in log-scale}}
      %The lines depict the running time required for random walk (blue line) and training (red line), respectively. Pentagrams show the time cost of six real-world graphs,
        \label{Dist_scalability_data}
      
\end{minipage}\hfill
\quad
\begin{minipage}{.46\linewidth}
    \centering
    \includegraphics[width= 1.6 in]{./Figures/Dist_time_auc.eps}%
    \captionof{figure}
      {\small {The influence of running time on embedding quality for {\sf DistGER} and competitors}}
        \label{Dist_time_auc}
\end{minipage}
\end{table}


To further assess the scalability of {\sf DistGER}, we generate synthetic graphs \cite{RMAT_2004} with a fixed node degree of 10 and the number of nodes from $10^5$ to $10^9$. Figure~\ref{Dist_scalability_data} presents the running times for random walks and training on these synthetic graphs using a cluster of 8 machines, suggesting that the running time increases linearly with the size of a graph, and {\sf DistGER} has the capability to handle even billion-node graphs. Moreover, the running times for six real-world graphs (including the {\em UK graph} with $|E|=3.7B$, $|V|=100M$, for which the competing systems do not terminate in 1 day or crash due to hardware and memory limitation) are inserted into the plot, which is consistent with the trend on synthetic data.

%
%
\subsection{Effectiveness w.r.t. Competitors}
\label{sec:effectiveness}
%
\spara{Link prediction.} To perform link prediction on a given graph $G$, following \cite{HuGE_2021,node2vec_2016,Verse_2018,NRP_2020},
we first uniformly at random remove 50\% edges as positive test edges, and the rest are used as positive training edges.
We also provide negative training and test edges by considering those node pairs between which no edge exists in $G$.
We ensure that the positive and negative set sizes are similar. %For a pair of nodes $(u, v)$, let $\varphi(u)$ and
%$\varphi(v)$ be the vectors learned by embedding methods.
The link prediction is conducted as a classification task
based on the similarity of $u$ and $v$, i.e., $\varphi(u)\cdot\varphi(v)$.
The effectiveness of link prediction is measured via the $AUC$ (Area Under Curve) score \cite{AUC_kdd} -- the higher the better.
We repeat this procedure 50 times to offset the randomness of edge removal and report the average $AUC$ in
Table~\ref{AUC_results}.
%shows $AUC$ for all the methods on five real-world graphs.
%, respectively, where a ``$-$'' indicates that the method fails due to the limitation of computing resources or because its running time exceeds 1 day.
{\sf DistGER} outperforms all competitors on these graphs, except for {\sf PBG} on {\em Com-Orkut}, where {\sf DistGER} ranks second.
On average, {\sf DistGER} has an 11.7\% higher $AUC$ score compared with the other three systems, thanks to our
information-centric random walks. {\sf PBG} is the best on {\em Com-Orkut} because this graph is much denser
and is friendly to the multi-relationship-based model in {\sf PBG}.
Figure~\ref{Dist_time_auc} exhibits accuracy-efficiency tradeoffs of {\sf DistGER} and competitors, i.e., their $AUC$ convergence curves w.r.t. increasing running times of random walks and training, over {\em LiveJournal}, further indicating
that {\sf DistGER} has better efficiency and effectiveness than the competitors.
%As a system of the same type, DistGER achieves better accuracies on all graphs than KnightKing which leverages the routine random walk configuration, thanks to its information-centric random walk strategies. We do not report the effectiveness of HuGE-D here because it uses the same random walk model as DistGER.
%
\begin{table}[h!]
\newcommand{\tabincell}[2]{\begin{tabular}{@{}#1@{}}#2\end{tabular}}
  \caption{\small $AUC$ scores of {\sf DistGER} and competitors for link prediction}
  \label{AUC_results}
  \begin{center}
  \footnotesize
  \begin{tabular}{cccccc}
%    \hline
    {Method}&\tabincell{c}{Youtube}&{LiveJournal}&\tabincell{c}{Com-Orkut}&{ Twitter}\\
    \hline
    {\sf PBG}        & 0.753           &0.882            &\bfseries{0.955} &0.912\\

    {\sf DistDGL}    &0.894            &0.718            &0.815            & running time $>$ 1 day \\

    {\sf KnightKing} &0.904            &0.963            & $0.918$         & out-of-memory\\

    {\sf DistGER}    &\bfseries{0.966} &\bfseries{0.976} &0.921            &\bfseries{0.919}\\
%  \hline
\end{tabular}
\end{center}
\end{table}

% \eat{
\spara{Multi-label node classification.}
This task predicts one or more labels for each graph node and has applications in %modern applications ranging from
text categorization \cite{zhang2006multilabel} and bioinformatics \cite{zhang2018ontological}.
We use embedding vectors and a one-vs-rest logistic regression classifier
with L2 regularization \cite{MLC_LIBLINEAR_2008}, %(using the LIBLINEAR library),
then evaluate the effectiveness by micro-averaged F1 ($Micro-F1$) and macro-averaged F1 ($Macro-F1$) \cite{WangC016}
scores, where $Micro-F1$ gives equal weight to each test instance and $Macro-F1$ assigns equal weight to each label category \cite{keikha2018community}.
%To train a classifier, nodes are uniformly at random split into training and test sets.
Following \cite{HuGE_2021,node2vec_2016,DeepWalk_2014,Line_2015,Verse_2018},
we select 10\% to 90\% training data ratio on {\em Flickr}, and 1\% to 9\% training ratio on {\em Youtube}.
%and the remaining nodes for testing.
We report the averaged $Macro-F1$ and $Micro-F1$ scores from 50 trials in Figure~\ref{Dist_MLC_mac_mic_F1}.
% shows the $Macro-F1$ and $Micro-F1$ scores achieved by each system as a
%function of the training ratio variation, respectively.
We find that {\sf DistGER} has better $Macro-F1$ and $Micro-F1$ scores
than existing frameworks, %on these graphs, %. In particular, compared with the KnightKing,
%DistGER consistently outperforms the other random walk-based systems on all graphs in $Macro-F1$ and $Micro-F1$ scores,
gaining 9.2\% and 3.3\% average improvements, respectively, due to its more effective information-centric random walks.
%Definition of $Macro-F1$ and $Micro-F1$ are as the following:

\begin{figure}[h!]
  \centering
  \includegraphics[width= 3.45 in]{./Figures/Dist_MLC_mac_mic_F1_1.eps}
  \caption{\small $Macro-F1$ (a1, b1) and $Micro-F1$ (a2, b2) scores for multi-label node classification. $X$-axis: training data ratio}
  \label{Dist_MLC_mac_mic_F1}
\end{figure}

% }
%\begin{equation}
%Precision = \frac{\sum\nolimits_{i}^{K}TP(i)}{\sum\nolimits_{i}^{K}(TP(i)+FP(i))}
%\end{equation}
%
%\begin{equation}
%Recall = \frac{\sum\nolimits_{i}^{K}TP(i)}{\sum\nolimits_{i}^{K}(TP(i)+FN(i))}
%\end{equation}
%
%\begin{equation}
%Micro-F1 = \frac{2\times Precision\times Recall}{Precision+Recall}
%\end{equation}
%
%\begin{equation}
%Macro-F1 = \frac{\sum\nolimits_{i}^{K}Micro-F1(i)}{|K|}
%\end{equation}
%
%where $TP(i)$, $FP(i)$ and $FN(i)$ are the number of true positives, false positives and false negatives in the instances which are predicted as $i$, respectively. Suppose $K$ is the overall label set, $Micro-F1$($i$) and $Macro-F1$ are the measure of $Micro-F1$ and $Macro-F1$ for the label $i$, respectively.
%\begin{table}
%\setlength{\abovecaptionskip}{0.cm}
%\setlength{\belowcaptionskip}{-0.cm}
%\newcommand{\tabincell}[2]{\begin{tabular}{@{}#1@{}}#2\end{tabular}}
%  \caption{$Macro-F1$ and $Micro-F1$ for multi-label classification on Flickr and Youtube graph, where train ratio is 0.5.}
%  \label{cluster_results}
%  \begin{center}
%  \small
%  \begin{tabular}{ccccc}
%    \hline
%    { }&\multicolumn{2}{c}{\bfseries{ \scriptsize Flickr}}&\multicolumn{2}{c}{\bfseries{\scriptsize Youtube}}\\
%    \hline
%    { }&Macro-F1 &Micro-F1&Macro-F1 &Micro-F1\\
%
%    \hline
%    \small PBG & 0.225	&0.387 	&0.295 	&0.406 	\\
%
%    \small DistDGL &0.205 	&0.378 	&0.283 	&0.403 	\\
%
%    \small KnightKing &0.239    &0.386 &0.285 	&0.402 	\\
%
%    \small DistGER  &\bfseries{0.277} &\bfseries{0.409}&\bfseries{0.298} &\bfseries{0.417}\\
%
%  \hline
%\end{tabular}
%\end{center}
%\end{table}
%

\begin{figure}
  \centering
  \includegraphics[width= 3.2 in]{./Figures/Dist_sampling_training_Mpad_efficiency.eps}
  \caption{\small {(a) Random walk efficiency, (b) training efficiency, (c) \# cross-machine messages, (d) random walk efficiency for {\sf MPGP} (ours) and workload-balancing scheme ({\sf KnightKing})}}
  \label{Dist_efficiency_sampling_training_MPGP}
\end{figure}

\begin{table*}[t!]
\begin{minipage}{0.275\linewidth}
\centering
\renewcommand\arraystretch{1.2}
\captionof{table}{\small Performance evaluation of partitioning for {\sf DistGER} and Competitors } %{\sf PBG} and {\sf DistDGL}
\label{Partition_sechme_overhead}
\begin{scriptsize}
\begin{tabular}{ccccc}
    \multicolumn{5}{c}{{\bfseries (a) Partitioning time for {\sf DistGER} and competitors }} \\
    \hline
    {\bf graph} & {\sf PBG} & {\sf DistDGL} & {\sf DistGER}\\
                &           & ({\sf METIS}) &  ({\sf MPGP}) \\
    \hline
    {\sf FL} & 383.28 s & 127.72 s & \bfseries{15.96 s} \\
    {\sf YT} & 349.15 s & 116.30 s & \bfseries{13.56 s} \\
    {\sf LJ} & 458.52 s & 425.19 s & \bfseries{36.42 s} \\
    {\sf OR} & 2662.62 s & 2761.25 s &\bfseries{294.68 s}\\
    {\sf TW} & 22 hour s & $>$ 1 day &\bfseries{9 hours}\\
    \hline
%    \multicolumn{5}{c}{} \\
    \multicolumn{5}{c}{{\bfseries (b) Evaluation of {\sf Parallel MPGP} }} \\
    \hline
    {\bf graph} & {\sf Streaming} & {\sf Partitioning} & {\sf Walking}\\
    \hline
  %  {\sf MPGP}   &DFS+deg  & 9 hours & \bfseries{575.22 s} \\
    \multirow{2}{*}{\sf LJ} &DFS+deg & 21.86 s & \bfseries{23.78 s} \\
           & BFS+deg & \bfseries{21.25 s} & 24.79 s \\
    \multirow{2}{*}{\sf OR} &DFS+deg & \bfseries{151.29 s} & 77.12 s \\
           & BFS+deg & 156.37 s & \bfseries{46.55 s} \\
    \multirow{2}{*}{\sf TW} &DFS+deg & \bfseries{1940.65} s & 683.81 s \\
           & BFS+deg & 2034.21 s & \bfseries{590.36 s}
\end{tabular}
\end{scriptsize}
\end{minipage}%\hfill
\quad
\begin{minipage}{.3\linewidth}
    \centering
    \includegraphics[width= 2.5in, height = 1.45 in]{./Figures/Dist_Mpad_streaming_vertex_time.eps}%
    \captionof{figure}
      {\small The distribution of local computations and cross-machine communications for different streaming orders on {\em LiveJournal}. The top table reports their running times for partitioning and random walks
        \label{Dist_MPaD_streaming}
      }
\end{minipage}%\hfill
\qquad
\begin{minipage}{.37\linewidth}
    \centering
    \includegraphics[width= 2.5 in, height = 1.45 in]{./Figures/Dist_generality_table_HuGE+.eps}%
    \captionof{figure}
      {\small Generality of {\sf DistGER} vs. {\sf KnightKing}. The bars show random walk efficiency ($-R$) and training efficiency ($-T$) for {\sf Deepwalk} ({\sf DW}), {\sf node2vec} ({\sf n2v}) and {\sf HuGE+}. The top table shows the ratio $\frac{\text{{\em AUC} for {\sf DistGER}}}{\text{{\em AUC} of {\sf KnightKing}}}$, with {\sf DW} and {\sf n2v}, task: link prediction
        \label{Dist_generality}
      }
\end{minipage}
\end{table*}


\subsection{Efficiency due to Individual Parts of DistGER}
\label{sec:individual}
\spara{Random walk and training efficiency.}
To evaluate the system design of {\sf DistGER} (\S \ref{sec:DistGER}, \S \ref{sec:learning}),
we first compare the efficiency of random walks and training with those of {\sf KnighKing} and {\sf HuGE-D}.
%For fair comparison, the running times that we reported for {\sf KnightKing} and {\sf HuGE-D} exclude the
%time of vocabulary table construction, since it is a serial process in {\sf Pwode2vec}, while {\sf DistGER}
%pipelines the construction during random walks.
For random walks (Figure~\ref{Dist_efficiency_sampling_training_MPGP}(a)),
{\sf DistGER} significantly outperforms {\sf KnightKing} and {\sf HuGE-D} on all our graph
datasets, achieving an average speedup of $3.32\times$ and $3.88\times$, respectively.
Although {\sf HuGE-D} implements information-oriented random walks on {\sf KnightKing},
due to additional computation and communication overheads during on-the-fly information
measurements (\S \ref{sec:HUGED}), its efficiency can be lower than that of {\sf KnightKing}.
We also notice that the random walk lengths ($L$) and the number of random walks ($r$) reduce (on average)
63.2\% and 18\%, respectively, in our information-oriented random walks, compared to {\sf KnightKing}'s
routine random walk configuration.
%which supports the traditional
%random walk methods. %To provide a straightforward adaptation for the information-oriented approach, DistGER leverages the incremental information-centric computation mechanism to mitigate the redundant computation and high communication cost in HuGE-D, then it achieves an average speedup of $3.32\times$ and $3.88\times$ in random walk procedure compared to KnightKing and HuGE-D.

Another benefit of information-centric random walks is that it generates concise and effective corpus to improve 
training efficiency. Compared to {\sf KnightKing}, {\sf DistGER} achieves $17.37\times$-$27.95\times$ acceleration
in training over all our graphs. Next, considering the same corpus size, we compare the training efficiency of {\sf Pword2vec} and {\sf DSGL}
(trainer in {\sf DistGER}). Figure~\ref{Dist_efficiency_sampling_training_MPGP}(b) shows that {\sf DSGL} achieves $4.31\times$ average speedup
compared to {\sf Pword2vec}. We also notice that the average throughput (number of nodes processed per second) for {\sf DSGL} is up to 49.5 million/s,
while that of {\sf Pword2vec} is only up to 16.1 million/s. These results indicate that our distributed {\sf Skip-Gram} learning model (\S \ref{sec:learning})
is more efficient than {\sf Pword2vec}.
%
%\begin{figure}
%  \centering
%  \includegraphics[width= 2.5 in]{Dist_Mpad_efficiency.eps}
%  \caption{\small (a) exhibits the number of cross-machine computation for DistGER on workload-balancing and MPGP partition scheme, respectively, and (b) shows the random walk time of DistGER on the two schemes, where y axis is in log-scale.}
%  \label{Dist_efficiency_MPaD}
%\end{figure}

\spara{Partitioning efficiency.} Considering the %large number cross-machine computing introduced by the
randomness inherent in random walks, the partitioning scheme is critical to overall efficiency. %of the distributed framework.
%To validate the efficiency of our multi-proximity-aware streaming graph partitioning (MPGP),
%we deploy the workload balancing scheme used in KnightKing and MPGP on DistGER,
%respectively,
%and report the number of cross-machine computations during the random walk procedure for the two schemes.
%We also present the efficiency performance of MPGP compared with the workload-balancing scheme.
For {\sf DistGER},
Figure~\ref{Dist_efficiency_sampling_training_MPGP}(c) exhibits that our multi-proximity-aware streaming graph partitioning ({\sf MPGP})
significantly reduces (avg. reduction $45\%$) the number of cross-machine messages than the workload-balancing partition of {\sf KnightKing}
on five graphs. Moreover, it improves the efficiency by 38.9\% for the random walking procedure
(Figure~\ref{Dist_efficiency_sampling_training_MPGP}(d)) over the same set of walks.
We report in Table~\ref{Partition_sechme_overhead}(a) the time required for graph partitioning in competing systems,
where {\sf DistDGL} uses the {\sf METIS} algorithm \cite{METIS_1998} for partitioning.
The results show that {\sf MPGP} performs partitioning with very little overhead in most cases, and
the partitioning efficiency is on average $25.1\times$ faster than competitors.
In Figure~\ref{Dist_MPaD_streaming}, we exhibit the distribution of local computations and cross-machine communications
on four machines for different streaming orders, and the top table reports their running times for partitioning and random walks.
For sequential {\sf MPGP}, we find that the {\sf DFS+degree}-based streaming order (\S \ref{sec:partition}) is more efficient than other streaming orders,
and it also strikes the best balance between cross-machine communications reduction and workload balancing.
Table~\ref{Partition_sechme_overhead}(b) exhibits the performance evaluation of {\sf parallel MPGP} on the small- ({\em LiveJournal}), medium- ({\em Com-Orkut}) and large-scale ({\em Twitter}) graphs. The results show that {\sf DFS+Degree} in {\sf parallel MPGP} is still the best or comparable in terms of partition time, due to the same reason as stated in our third optimization scheme (\S \ref{sec:partition}). On the other hand, {\sf BFS+Degree} in {\sf parallel MPGP} works the best in terms of random walk time due to preserving the locality of the graph structure (our fourth optimization scheme in \S \ref{sec:partition}).
%as using its streaming order to parallel partitioning can reduce the influence of relevance between each segment.
We ultimately recommend {\sf BFS+Degree} for {\sf parallel MPGP}, since it reduces the partition time greatly, while the random walk time is comparable to that obtained from sequential {\sf MPGP}.
%
%\begin{figure}
%  \centering
%  \includegraphics[width= 3 in]{Dist_Mpad_streaming_vertex_time.eps}
%  \caption{\small The distribution of local computations and cross-machine communications for different streaming orders on {\em LiveJournal}. The top table reports their running times for partitioning and random walks.}
%  \label{Dist_MPaD_streaming}
%\end{figure}
%
%\begin{table}
%\setlength{\abovecaptionskip}{0.cm}
%\setlength{\belowcaptionskip}{-0.cm}
%\newcommand{\tabincell}[2]{\begin{tabular}{@{}#1@{}}#2\end{tabular}}
%  \caption{\small Time execution time (seconds) of the partition scheme in PBG, DistDGL, and DistGER, ``$-$'' means the scheme fails under constrains of computation resource.}
%  \label{Partition_sechme_overhead}
%  \begin{center}
%  \small
%  \begin{tabular}{ccccc}
%    \hline
%    {Graph}&{PBG}&{DistDGL(METIS)}&{DistGER}\\
%    \hline
%    Flickr& 383.28 &127.72 &\bfseries{15.96} \\
%
%    Youtube& 349.15 &116.30&\bfseries{13.56} \\
%
%    LiveJournal& 458.52 &425.19 &\bfseries{36.42} \\
%
%    Com-Orkut& 2662.62 &2761.25 &\bfseries{294.68}\\
%
%    Twitter&78986.85 &$-$&\bfseries{35500.41}\\
    %\hline
%    \multicolumn{5}{l}{* HuGE+ generates the smallest corpus size for training among all methods tested.} \\
%
%  \hline
%\end{tabular}
%\end{center}
%\end{table}
%


\subsection{Generality of DistGER}
\label{sec:generality}
%\begin{figure}
%  \centering
%  \includegraphics[width= 3 in, height= 1.65 in]{Dist_generality_table.eps}
%  \caption{\small Generality comparison for DistGER and KnightKing, %on real-word graphs,
%  The bars display random walk (denoted as $-R$) and training efficiency (denoted as $-T$) for {\sf Deepwalk} (DW) and {\sf node2vec} (n2v), respectively.%, and the y axis is in log-scale.
%  Top table shows the ratio $\frac{\text{{\em AUC} for {\sf DistGER}}}{\text{{\em AUC} of {\sf KnightKing}}}$, both with Deepwalk and node2vec, respectively, considering link prediction.}
%  \label{Dist_generality}
%\end{figure}
%
%Since our proposed information-oriented random walk framework DistGER aims to address the redundant computations and high communication cost introduced by the effectiveness measurement of the generated walking information in distributed setting, it provides a good systematic support for the information-centric approach HuGE as shown by the previous experimental results. A natural question arises: can DistGER also support the traditional random-walk-based methods?
To demonstrate the generality of {\sf DistGER}, we deploy {\sf Deepwalk} \cite{DeepWalk_2014}, {\sf node2vec} \cite{node2vec_2016} and {
\sf HuGE+} \cite{HuGE+_2022}
on {\sf DistGER}. While the original {\sf Deepwalk} and {\sf node2vec} follow
traditional random walks, in {\sf DistGER} the walk length and the number of walks are decided via information-centric measurements.
Next, we also deploy both {\sf Deepwalk} and {\sf node2vec} on {\sf KnightKing} which supports the routine configuration random walk.
Figure~\ref{Dist_generality} illustrates that {\sf DistGER} reduces the random walks time by 41.1\% and 51.6\% on average for
{\sf Deepwalk} and {\sf node2vec}, respectively. For training, {\sf DistGER} is on average $17.7\times$ and $21.3\times$ faster than {\sf KnightKing}+{\sf Pword2vec}
for {\sf Deepwalk} and {\sf node2vec}, respectively.
Moreover, we also show the {\em AUC} ratio of {\sf DistGER} and {\sf KnightKing}, considering {\sf Deepwalk} and {\sf node2vec}, for link prediction.
% tasks, where performing multi-label classification on Flickr graph  and link prediction on other graphs, the accuracy metric for the two task are $Miro-F1$ and $AUC$ score, respectively, it can be found from
Our results depict that {\sf DistGER} has comparable (in most cases, higher) {\em AUC} scores, while it improves the efficiency significantly
even for traditional random walk-based graph embedding methods.
{\sf HuGE+} is an extension of {\sf HuGE}, and it uses the same {\sf HuGE} information-centric method to determine the walk length and the number of walks per node. Figure~\ref{Dist_generality} exhibits the compatibility of {\sf HuGE+} on {\sf DistGER} via its general API.
%%%%%%%%%%%%%%%%%%%%%%%%%%%%%%%%%%%%%%%%%%%%%%%%%

\section{Conclusion} 
We introduce \algname{}, a novel object detection method that achieves highly efficient inference speed while also improving zero-shot generalization compared with existing methods. The prompt-based decoding approach reduces the computational burden of object queries. The RoI-based masked attention and RoI pruning techniques allow us to efficiently leverage a large ViT-based CLIP model, enhancing detection performance through classification prediction ensembling. Comprehensive experiments show that \algname{} is $21.2$ times faster than OV-DETR while achieving comparable or higher APs on base and novel classes compared to two-stage OVD methods. %We believe that our work will inspire future work to explore the benefits of using Transformers.


%\paragraph{Ethics Statement.} 
%The focus of this paper is on open-vocabulary object detection. Our approach involves the integration of Transformer-based object detector and CLIP. We have not identified any foreseeable negative social impact associated with our work to share our findings with the scientific community. Nonetheless, we will continue to monitor and consider any potential concerns that may arise. 


%%%%%%%%% REFERENCES
{\small
\bibliographystyle{ieee_fullname}
\bibliography{cvpr2023_conference}
}
\section*{Appendix}
\section*{Additional implementation details}
\noindent \textbf{Training details.} 
At the training time, we first project the query ray instead of a single point to each source view and fetch the corresponding ray-based feature, which contains rich contextual information in each intra-view. 
For pre-training, we train on a single NVIDIA RTX3090-Ti GPU with 24GB memory. On this hardware, we train our S-Ray for 260k iterations in 60 different scenes of ScanNet~\cite{scannet} (real-world data) and 100k iterations in 12 different scenes of Replica~\cite{replica}  (synthetic data). For finetuning, we only require 10min finetuning time corresponding to 2k iterations. This finetuning result is comparable and even better than 100k optimizations of Semantic-NeRF \cite{semantic-nerf} from each independent scene.

We do not show the specific details of the semantic loss design in the paper. In code implementation, we apply two-stage (coarse and fine) ray sampling as done in NeRF~\cite{NeRF}. Therefore, our semantic loss is actually computed as
\begin{equation}
    L_{sem}=-\sum_{\mathbf{r} \in \mathcal{R}}\left[\sum_{l=1}^L p^l(\mathbf{r}) \log \hat{p}_c^l(\mathbf{r})+\sum_{l=1}^L p^l(\mathbf{r}) \log \hat{p}_f^l(\mathbf{r})\right]
\end{equation}

where $\mathcal{R}$ are the set of sample rays within a training batch, $1\leq l\leq L$ is the class index, and $p^l, \hat{p}^l_c, \hat{p}^l_f$ are the multi-class probability at class $l$ of the ground truth, coarse semantic logits and fine semantic logits for the query ray $\mathbf{r}$. Actually, for fair comparison in Section 4.2 of our paper, we adopt the same training loss with Semantic-NeRF~\cite{semantic-nerf} as:
\begin{equation}
    \mathcal{L}_{total} = \lambda_1 \mathcal{L}_{sem} + \lambda_2 \mathcal{L}_{photometric}, 
\end{equation}
where the color head is from the geometry aware network with photometric loss same as~\cite{semantic-nerf}. Like Semantic-NeRF, we also set $\lambda_1 = \lambda_2 =1$ in Section 4.2 and set $\lambda_1 = 0, \lambda_2 = 1$ as NeRF for ablation study in Table 2 of the paper. 

\noindent \textbf{Data split.} Our training data consists of both synthetic data and real data. For real data training, we choose 60 different scenes from ScanNet~\cite{scannet} as training datasets and use the image resolution of $320 \times 240$. We then choose 10 unseen novel scenes as test datasets to evaluate the generalizability of S-Ray in real data. For synthetic data, we choose 12 different scenes (\ie, 2 rooms, 2 offices, 7 apartments, 1 hotel) from Replica~\cite{replica} for the training set and the remains (\ie, 2 apartments, 3 offices, 1 room) as test set with the image resolution of $640\times 480$. 
For each test scene, we select 20 nearby views; we then select 8 views as source input views, 8 as additional input for per-scene fine-tuning, and take the remaining 4 as testing views. Our training data includes various camera setups and scene types, which allows our method to generalize well to unseen semantic scenarios.


\section*{Additional experiments and analysis}
\noindent \textbf{More discussion of loss function.} When adding color rendering, it is interesting to see the effect of the weighting factor, thus conducting the following experiments in Table~\ref{tab:weight-factor}. We observe that color rendering can benefit semantics but color rendering is not sensitive to semantics. Furthermore, Table~\ref{tab:weight-factor} shows that the semantic loss alone can also drive our model to learn reasonable contextual geometry for semantic information as visualized in Figure~\ref{fig: vis}.

\begin{table}[!h]
\centering
\resizebox{\columnwidth}{!}{%
\begin{tabular}{cccccc}
\hline
$\lambda_1/\lambda_2$ & 1/0   & 0.75/0.25 & 0.5/0.5 & 0.25/0.75 & 0/1   \\ \hline
PSNR                  & 17.49 & 25.26     & 25.35   & 26.24     & 26.57 \\
mIoU(\%)              & 55.10 & 56.51     & 57.15   & 58.12     & 3.62  \\ \hline
\end{tabular}%
}
\vspace{-3mm}
\caption{Different weighting factors effect under ScanNet~\cite{scannet} generalization settings. }
\label{tab:weight-factor}
\end{table}

\begin{figure}[!h]
    \centering
    \includegraphics[width=\linewidth]{cvpr2023/Figures/rebuttal.fig1.pdf}
    \vspace{-6mm}
    \caption{Visualization of 2D CNN features from ResUnet and intra-view attention map. It shows that our ResUnet can help S-Ray learn reasonable geometry for contextual semantics and the intra-view attention map is closely related to the visibility.}
    \label{fig: vis}
\end{figure}

\noindent \textbf{Effectiveness of the CRA module.} To further validate the computational effectiveness of our Cross-Reprojection Attention (CRA) module, we provide the comparisons with Dense Attention in FLOPs and Memory usage. 



\begin{table}[!h]
    \centering
    \resizebox{\linewidth}{!}{
    \begin{tabular}{lcccc}
        \toprule
         Description & GFLOPs & mIoU(\%) &  Total Acc(\%)\\
         \midrule
        %  NeRF~\cite{mildenhall2020nerf} & 10k  & $\sim$30min &       \\
        w/o CRA      & 0 & 76.30  & 86.02 & \\
         Dense Attention &  10.25 & 90.46  & 94.52  & \\
         only intra-view Att  & 3.05  & 81.24 & 89.58 & \\
         only cross-view Att  & 2.35  & 87.01 & 93.34 & \\
         full CRA     & \textbf{5.40} & \textbf{91.08}  & \textbf{98.20} & \\
         \bottomrule
    \end{tabular}
    }
    \vspace{-3mm}
    \caption{Performance on real data \cite{scannet} for different settings of Cross-Reprojection Attention module (CRA). FLOPs increments are estimated for the input of $1024 \times 64 \times 8 \times 32$. }
    \label{tab:flop_compare}
\end{table}

\begin{table}[!h]
    \centering
    \resizebox{\linewidth}{!}{
    \begin{tabular}{lcccc}
        \toprule
         Description & Memory(MB) & mIoU(\%) &  Total Acc(\%)\\
         \midrule
        %  NeRF~\cite{mildenhall2020nerf} & 10k  & $\sim$30min &       \\
        w/o CRA      & 0 &  76.30 &86.02  & \\
         Dense Attention &  17647 & 90.46 & 94.52 & \\
         only intra-view Att  & 3899  & 81.24 & 89.58 & \\
         only cross-view Att  &  1583 & 87.01 & 93.34 & \\
          full CRA     & \textbf{4143} & \textbf{91.08}  & \textbf{98.20} & \\
         \bottomrule
    \end{tabular}
    }
    \vspace{-3mm}
    \caption{Performance on real data \cite{scannet} for different settings of Cross-Reprojection Attention module (CRA). Memory increments are estimated for an input of $1024 \times 64 \times 8 \times 32$.}
    \label{tab:mem_compare}
\end{table}

Table~\ref{tab:flop_compare} and Table~\ref{tab:mem_compare} show the computational performance of real data by adopting different settings of our Cross-Reprojection Attention (CRA) module. We observe that directly applying the dense attention over multi-view reprojected rays suffers from heavy computational cost and high GPU memory. In contrast, our CRA module can achieve the comparable performance of dense attention with friendly GPU memory and high computational efficiency. Specifically, the design of CRA can improve the performance by $47.3\%$ in FlOPs and $76.5\%$ in GPU memory. These results prove that the proposed cross-reprojection attention can achieve high mIoU and total accuracy by capturing dense and global contextual information with computational efficiency.
\newline
\newline
\noindent \textbf{Semantic ray construction.} To construct the final semantic ray in Section 3.4 of our paper, we assign distinct weights to different source views and compute the semantic ray with semantic consistency. Specifically, we design the \textit{Semantic-aware Weight Network} to rescore the significance of each view with a hyperparameter $\tau$, as 
\begin{equation}
     \mathbf{w} \in \mathbb{C}(\tau):=\left\{\mathbf{w}: 0<\frac{\tau}{m}<w_i<\frac{1}{\tau m}, \sum_{i=1}^m w_i=1\right\},
\end{equation}
where $\mathbf{w}$ is the view reweighting vector with length $m$ indicating the importance of source views. Instead of mean aggregation which ignores the different significance of different source views, the hyperparameter $\tau$ controls the semantic awareness of each view. The effectiveness of $\tau$ can be seen in Table~\ref{tab:tau}.

\begin{table}[!h]
    \centering
    \resizebox{\linewidth}{!}{
    \begin{tabular}{lcccc}
        \toprule
         hyperparameter $\tau$ & mIoU(\%) & Total Accuracy(\%)  & Average Accuracy(\%)\\
         \midrule
        %  NeRF~\cite{mildenhall2020nerf} & 10k  & $\sim$30min &       \\
        1    & 54.21 &   77.13  &  59.05\\
         0.8 &  56.33 & 78.01 & 60.37& \\
         0.7  & \textbf{57.15} & \textbf{78.24} & \textbf{62.55} & \\
         0.5 & 55.70  & 76.64 & 60.80 & \\
          0.2  &54.03  & 77.25 &  61.34 & \\
         \bottomrule
    \end{tabular}
    }
    \vspace{-3mm}
    \caption{Performance on real data \cite{scannet} for different settings of hyperparameter $\tau$ in test set.}
    \label{tab:tau}
\end{table}

From Table~\ref{tab:tau}, we observe that we can improve the performance of semantic segmentation by assigning different weights to each source view with hyperparameter $\tau$. Note that $\tau = 1$ means the mean aggregation operation.
\newline
\newline
\noindent \textbf{Training process.}  Given multiple views of a scene, we construct a training pair of source and query view (\ie, target view) by first randomly selecting a target view, and sampling 8 nearby but sparse views as source views. We follow \cite{neuray} to build our sampling strategy. The performance of our method in different training iterations can be found in Table~\ref{tab:training_detail}. The results show that we only require 260k iterations for 20 hours to train our S-Ray over 60 different real scenes, which demonstrates the efficiency and effectiveness of our network design.
\newline
\newline
\noindent \textbf{More comparisons with Semantic-NeRF.} To further show our strong and fast generalizability in a novel unseen scene, we compare our performance with Semantic-NeRF~\cite{semantic-nerf} in per-scene optimization. The results are shown in Table~\ref{tab:compare_semanticnerf}. While Semantic-NeRF~\cite{semantic-nerf} needs to train one independent model for an unseen scene, we observe that our network S-Ray can effectively generalize across unseen scenes. What's more, our direct result can be improved by fine-tuning on more images for only 10 min (2k iterations), which achieves comparable quality with Semantic-NeRF for 100k iterations per-scene optimization. Moreover, Semantic-NeRF shows very limited generalizability by first generating pseudo semantic labels for an unseen scene with a pretrained model, and then training on this scene with the pseudo labels. In this way, Semantic-NeRF is able to apply to new scenes without GT labels. In contrast, our S-Ray provides stronger generalization ability by enabling directly test on unseen scenes. We provide additional experiments in Table~\ref{tab:pseudo}.
\newline
\newline
\noindent \textbf{Comparison with GPNR.} The recent work GPNR~\cite{GPNR} also generates novel views from unseen scenes by enabling cross-view communication through the attention mechanism, which makes it a bit similar to our S-Ray. To further justify the motivation and novelty, we summarize several key differences as follows. \textbf{Tasks:} GPNR utilizes fully attention-based architecture for color rendering while our S-Ray focuses on learning a generalizable semantic field for semantic rendering. \textbf{Embeddings:} GPNR applies three forms of positional encoding to encode the information of location, camera pose, view direction, etc. In contrast, our proposed S-Ray only leverages image features with point coordinates without any handcrafted feature engineering. In this sense, our S-Ray enjoys a simpler design in a more efficient manner. \textbf{Training cost.} While GPNR requires training 24 hours on 32 TPUs, S-Ray only needs a single RTX3090-Ti GPU with similar training time.
\begin{table}[!h]
\centering
\resizebox{\columnwidth}{!}{%
\begin{tabular}{cccccccc}
\hline
       & w/o ft & ft 5k(p) & ft 5k(gt) & ft 50k(p) & ft 50k(gt) & ft converge(p) & ft converge(gt) \\ \hline
S-NeRF & N/A    & 78.59    & 86.32     & 85.64     & 91.33      & 92.10          & 95.24           \\
S-Ray  & 77.82  & 88.07    & 93.40     & 91.25     & 95.15      & 92.43          & 95.39           \\ \hline
\end{tabular}%
}
\vspace{-3mm}
\caption{More mIoU comparisons with SemanticNeRF(S-NeRF) in the scene0160-01 from ScanNet. Same with S-NeRF, we choose pretrained DeepLabV3+~\cite{deeplabv3} to generate pseudo semantic labels for finetuning. ``p'' means finetuning with pseudo labels, and ``gt'' means finetuning with ground truth.}

\label{tab:pseudo}
\end{table}
\newline
\newline
\noindent \textbf{More discussion for reconstruction quality.} To further demonstrate the reconstruction quality and generalizability of S-Ray, we evaluate S-Ray with NeuRay~\cite{neuray}, MVSNeRF~\cite{mvsNeRF}, and IBR-Net~\cite{IBRnet} on two typical benchmark datasets (\ie, Real Forward-facing~\cite{mildenhall2019local} and Realistic Synthetic 360$^\circ$~\cite{NeRF}) in Table~\ref{tab:benchmark}. In general, Table~\ref{tab:benchmark} shows our Cross-Reprojection Attention module is also useful for generalizable NeRFs with out semantic supervision.
\begin{table}[!h]
\centering
\resizebox{\columnwidth}{!}{%
\begin{tabular}{ccccccc}
\hline
            & \multicolumn{3}{c}{Realistic Synthetic 360°}        & \multicolumn{3}{c}{Real Forward-facing}             \\ \hline
Method      & PSNR$\uparrow$ & SSIM$\uparrow$ & LPIPS$\downarrow$ & PSNR$\uparrow$ & SSIM$\uparrow$ & LPIPS$\downarrow$ \\ \hline
MVSNeRF & 23.46 & 0.851 & 0.172 & 22.93 & 0.794 & 0.260  \\
IBRNet  & 24.52 & 0.874 & 0.158 & 24.17 & 0.802 & 0.215 \\
NeuRay  & 26.73 & 0.908 & 0.123 & 25.35 & 0.824 & 0.198 \\
S-Ray(Ours) & \textbf{26.84} & \textbf{0.917} & \textbf{0.115}    & \textbf{25.68} & \textbf{0.833} & \textbf{0.180}     \\ \hline
\end{tabular}%
}
\vspace{-3mm}
\caption{Quantitative comparisons of scene rendering in the generalization setting. All generalization methods including our method are pretrained on the same scenes and tested on unseen test scenes. }
\label{tab:benchmark}
\end{table}
While the three mentioned methods in Table~\ref{tab:benchmark} and our method are image-based rendering, the main difference lies in how to extract useful features: (a) MVS-NeRF leverages cost volume to extract geometry features, which benefits the acquisition of density; IBRNet performs feature attention on rays in 3D space and NeuRay further extracts the occlusion-aware features by explicitly modeling occlusion. Their features are sparse in 3D space but sufficient for color rendering. (b) Our method goes back to the 2D reprojection space and obtains dense attention by cascading two sparse attention modules, thus extracting rich semantic and geometric features. A key point is that we apply a ResUnet segmentation network fro context feature extraction to get semantic priors, which is not present in the previous methods.
\newline
\newline
\noindent \textbf{Disscusion of the number of source views.} 
\begin{table}[!h]
    \centering
    \resizebox{\linewidth}{!}{
    \begin{tabular}{lccccc}
        \toprule
         $N_s$ & mIoU(\%) & Total Acc(\%)  & Avg Acc(\%) & PSNR & SSIM\\
         \midrule
        %  NeRF~\cite{mildenhall2020nerf} & 10k  & $\sim$30min &       \\
        1    & 67.55 &  86.15   & 73.73 & 26.47 & 0.9077 \\
         4&  75.41 & 90.51  & 81.06 & 30.90 & 0.9368 \\
         8  & {79.97} & {93.06} & {84.92}  &  \textbf{29.52} & \textbf{0.9106}\\
         12 & {83.21} & 93.89 & 88.07 & 28.57 & 0.8969 \\
          16  & \textbf{84.84}& \textbf{94.33} & \textbf{89.78} & 27.85 & 0.8859 \\
         \bottomrule
    \end{tabular}
    }
    \vspace{-3mm}
    \caption{Performance(mIoU, Total accuracy, Average accuracy, PSNR, SSIM) on the real data scene~\cite{scannet} wiht different source view numbers $N_s$. } 
    \label{tab:Ns_views}
\end{table}
We observe that using more source views on our S-Ray model can improve semantic rendering quality. The results are shown in Table~\ref{tab:Ns_views}. The reason is that adding more reference views in training means leveraging more contextual information for semantic feature learning to build a larger 3D contextual space and reconstruct the final semantic ray, which improves the view consistency and accuracy of semantic segmentation.
\newline
\newline
\noindent \textbf{Disccusion of semantic-aware weight.} In semantic ray construction, we learn the view reweighting vector $\mathbf{w}$ to rescore the significance of each source view. To further demonstrate the effectiveness of this rescore strategy, we show the example in Figure~\ref{fig:new_weight}. The results show that $\mathbf{w}$ can distinct the different significance of different source views to the query semantic ray.


\section*{Network architecture}
\noindent \textbf{Semantic feature extraction.} Given input views and a query ray, we project the ray to each input view and apply the semantic feature extraction module in Table~\ref{table:feature_extraction} to learn contextual features and build an initial 3D contextual space. The details can be found in Table~\ref{table:feature_extraction} and Section 3.2 in the paper.
\begin{table*}[!h]
\centering

\begin{tabular}{lcccc}
\toprule
Type & Size/Channels & Activation & Stride & Normalization\\
\midrule
Input 1: RGB images & - & - & - & -\\
Input 2: View direction differences & - & - & - & -\\
L1: Conv $7\times 7$ & $3, 16$ & ReLU & $2$ & Instance\\
% layer1
L2: ResBlock $3\times 3$ & $16, 32, 32$ & ReLU & $2, 1$ & Instance\\
% layer2
L3: ResBlock $3\times 3$ & $32, 64, 64$ & ReLU & $2, 1$ & Instance\\
L4: ResBlock $3\times 3$ & $64, 64, 64$ & ReLU & $1, 1$ & Instance\\
% layer3
L5: ResBlock $3\times 3$ & $64, 128, 128$ & ReLU & $2, 1$ & Instance\\
L6: ResBlock $3\times 3$ & $128, 128, 128$ & ReLU & $1, 1$ & Instance\\
L7: ResBlock $3\times 3$ & $128, 128, 128$ & ReLU & $1, 1$ & Instance\\
L8: ResBlock $3\times 3$ & $128, 128, 128$ & ReLU & $1, 1$ & Instance\\
L9: ResBlock $3\times 3$ & $128, 128, 128$ & ReLU & $1, 1$ & Instance\\
L10: ResBlock $3\times 3$ & $128, 128, 128$ & ReLU & $1, 1$ & Instance\\

L11: Conv $3\times3$ & $128, 64$ & - & $1$ & Instance\\
L12: Up-sample 2$\times$ & - & - & - & -\\
L13: Concat (L12, L4) & - & - & - & -\\
L14: Conv $3\times3$ & $128, 64$ & - & $1$ & Instance\\

L15: Conv $3\times3$ & $64, 32$ & - & $1$ & Instance\\
L16: Up-sample $2\times$ & - & - & - & -\\
L17: Concat (L16, L2) & - & - & - & -\\
L18: Conv $3\times3$ & $64, 32$ & - & $1$ & Instance\\
L19: Conv $1\times1$ & $32, 32$ & - & $1$ & Instance\\

L20: Reprojection\\
L21: MLP (Input 2) & $4, 16, 32$ & ELU & - & -\\
L22: Add (L21, L20) & - & - & - & -\\

\bottomrule
\end{tabular}
\vspace{-3mm}
\caption{Semantic feature extraction.}
\label{table:feature_extraction}
\end{table*}




\noindent \textbf{Cross-Reprojection Attention.} To model full semantic-aware dependencies from the 3D contextual space with computational efficiency, we design the Cross-Reprojection Attention module in Table~\ref{table:CRA_module} to learn dense and global contextual information, which can finally benefit the performance of semantic segmentation. The details of architecture and design can be found in Table~\ref{table:CRA_module} and Section 3.3 in the paper.

\begin{table*}[!h]
\centering

\begin{tabular}{lccc}
\toprule
Type & Feature dimension & Activation \\
\midrule
Input: Initial 3D contextual space & - & -\\
L1: Transpose (Input) & - & -\\
L2: Position Embeddings & - & -\\
L3: Add (L1, L2) & - & -\\
L4: Multi-head Attention (nhead=4) (L3) & 32 & ReLU \\
L5: Transpose (L4) & - & -\\
L6: Multi-head Attention (nhead=4) (L5) & 32 & ReLU \\
\bottomrule
\end{tabular}
\vspace{-3mm}
\caption{Cross-Reprojection Attention module.}
\label{table:CRA_module}
\end{table*}

\begin{table*}[!h]
\centering
\begin{tabular}{lccc}
\toprule
Type & Feature dimension & Activation \\
\midrule
Input 1: Initial 3D contextual space & - & -\\
Input 2: View direction differences & - & -\\
L1: Concat (Input 1, Input 2) & - & - \\
L2: MLP (L1) & $37, 16, 8, 1$ & ELU\\
L3: Sigmoid (L2) & - & -\\
\bottomrule
\end{tabular}
\vspace{-3mm}
\caption{Semantic-aware weight network.}
\label{table:semantic_aware_weight_net}
\end{table*}


\noindent \textbf{Semantic-aware weight network.} To construct the final semantic ray from refined 3D contextual space and learn the semantic consistency along the ray, we introduce the semantic-aware weight network in Table~\ref{table:semantic_aware_weight_net} to rescore the significance of each source view. More experiments about the semantic-aware weight net can be found in Table~\ref{tab:tau}, and we show architecture details in Table~\ref{table:semantic_aware_weight_net} and Section 3.4 of the paper.

\noindent \textbf{Geometry-aware network.} To build our generalizable semantic field, we adopt a geometry-aware network to predict density $\sigma$ and render the final semantic field with semantic logits. Moreover, we also leverage this network to produce the radiance and render a radiance field to show our rendering quality. We show the details of this network in Table~\ref{table:geometry_aware_net} and Section 3.4 of the paper.









\begin{figure*}[!h]
    \centering
    \includegraphics[width=\linewidth]{cvpr2023/Figures/new_weight.pdf}
    \caption{Different significance weight of source view. Given the query ray, we apply the semantic-aware weight network to learn the significance weight $\mathbf{w}$ to restore each source view. Note that the greater weight will be assigned to the more important source view. }
    \label{fig:new_weight}
\end{figure*}


\begin{table*}[!t]
\centering
\begin{tabular}{lccc}
\toprule
Type & Feature dimension & Activation\\
\midrule
Input: Initial 3D contextual space & - & -\\
L1: MLP (Input) & $32, 32$ & ELU\\
L2: MLP (Input) & $32, 1$ & ELU\\
L3: Sigmoid (L2) & - & -\\
L4: Dot-product (L1, L3) & - & -\\
L5: Cross-view Mean (L4) & - & - \\
L6: Cross-view Varience (L4) & - & -\\
L7: Concat (L5, L6) & - & - \\
L8: MLP (L7) & $64, 32, 16$ & ELU\\
L9: Multi-head Attention (nhead=4) (L8) & 16 & ReLU \\
L10: MLP (L9) & $16$ & ELU\\
L11: MLP (L10) & $1$ & ReLU\\
\bottomrule
\end{tabular}
\vspace{-3mm}
\caption{Geometry-aware network.}
\label{table:geometry_aware_net}
\end{table*}

% \begin{figure*}[!t]
%     \centering
%     \includegraphics[width=\linewidth]{cvpr2023/Figures/weight.pdf}
%     \caption{fddasfasd }
%     \label{fig:signifi}
% \end{figure*}

% \begin{figure*}[!t]
%     \centering
%     \includegraphics[width=\linewidth]{cvpr2023/Figures/semantic_render.pdf}
%     \caption{\textbf{Semantic rendering quality comparison}. On the left, we show direct semantic rendering results of our method and NeuRay~\cite{neuray} with semantic head. Limited by insufficient generalization, NeuRay with semantic head has difficulty to render semantics in unseen scenes and fails to capture contextual structure, while our method is able to learn the semantic structural prior, thus showing strong generalizability across different scenes. On the right, we show the experimental comparison between our S-Ray with $2k$ iterations finetuning ($\sim$10min) and Semantic-NeRF \cite{semantic-nerf} with $100k$ iterations. }
%     \label{fig:result}
% \end{figure*}


\begin{figure*}[!t]
    \centering
    \includegraphics[width=\linewidth]{cvpr2023/Figures/additional_vis_final.pdf}
    \caption{{Additional semantic rendering quality comparison}. More qualitative comparisons between our method S-Ray and non-generalizable method Semantic-NeRF~\cite{semantic-nerf} (S-NeRF for short) for semantic rendering in real data~\cite{scannet}. }
    \label{fig:more_semantic_result}
\end{figure*}

\begin{figure*}[!t]
    \centering
    \includegraphics[width=\linewidth]{cvpr2023/Figures/big_visrender.pdf}
    \caption{Qualitative results of scene rendering for generalization (w/o ft) and fine-tuning settings (ft) in real data~\cite{scannet}. Adding a color head from the geometry-aware network, We compare our method S-Ray with the generalizable rendering method NeuRay~\cite{neuray} and Valina NeRF~\cite{NeRF} with 200k iterations. }
    \label{fig:more_render}
\end{figure*}

\begin{table*}[!h]

\centering

\begin{tabular}{ccccccc}
\hline
\multirow{2}{*}{Method} & \multicolumn{3}{c}{Validation Set} & \multicolumn{3}{c}{Test Set} \\ \cline{2-7} 
                        & mIoU(\%)   & Total Acc(\%)    & Avg Acc(\%)   & mIoU(\%) & Total Acc(\%)  & Avg Acc(\%) \\ \hline
S-Ray (10k iters)       & 63.70   & 85.70        & 71.86     & 48.53 & 74.75      & 56.55   \\
S-Ray (50k iters)       & 72.85   & 88.72        & 79.52     & 52.32 & 77.31      & 59.38   \\
S-Ray (100k iters)      & 81.25   & 94.80        & 86.84     & 54.27 & 79.13      & 61.76   \\
S-Ray (200k iters)      & 89.31   & 97.91        & \textbf{91.10}     & 54.44 & 76.46      & 60.63   \\
\textbf{S-Ray (260k iters)} & \textbf{89.57} & \textbf{98.54} & 91.02 & \textbf{57.15} & \textbf{78.24} & \textbf{62.55}\\
S-Ray (300k iters)      & 88.99   & 98.40        & 90.39     & 55.84 & 77.67      & 62.15   \\ \hline
\end{tabular}%
\vspace{-3mm}
\caption{Quantitative results (mIoU, total accuracy, average accuracy) of our method (S-Ray) in training process from multiple scenes in real dataset~\cite{scannet}. }
\label{tab:training_detail}
\end{table*}

\begin{table*}[!h]
\centering
\begin{tabular}{cccccc}
\hline
Steps                 & Method        & mIoU(\%) &    Average Accuracy(\%) &    Total Accuracy(\%) & PSNR \\ \hline
\multirow{2}{*}{0}    & Ours          &  77.22   &        81.68          &      88.53         &   29.47   \\
                      & Semantic NeRF &   -   &              -    &            -    &  -    \\ \hline
\multirow{2}{*}{2k}   & Ours          &  92.66    &       98.73           &      98.73          &   29.80   \\
                      & Semantic NeRF &   78.32  &          82.69        &        85.81        &     20.62 \\ \hline
\multirow{2}{*}{4k}   & Ours          &   93.40   &        98.97          &       98.97         &   29.86   \\
                      & Semantic NeRF &   86.97   &        86.61          &        87.48        &    21.85  \\ \hline
\multirow{2}{*}{6k}   & Ours          &   94.17   &          99.06        &        99.06        &   29.92   \\
                      & Semantic NeRF &   87.08   &        87.85          &     88.01           &  22.62    \\ \hline
\multirow{2}{*}{8k}   & Ours          &  94.59    &        99.15          &      99.15          &    29.95  \\
                      & Semantic NeRF &  88.78    &        88.57          &        89.67        &   22.94   \\ \hline
\multirow{2}{*}{30k}  & Ours          &   95.10   &         99.43         &          99.42      &   30.05   \\
                      & Semantic NeRF &    91.78  &          94.86        &         95.78       &    24.78  \\ \hline
\multirow{2}{*}{100k} & Ours          &    -  &       -           &       -         &   -   \\
                      & Semantic NeRF &  95.05    &           98.73       &        99.02        &    29.97  \\ \hline
\end{tabular}%
\vspace{-3mm}
\caption{Performance of per-scene optimization in ScanNet~\cite{scannet}. We compare our method S-Ray with Semantic-NeRF~\cite{semantic-nerf} in per-scene optimization to show our fast generalizability in real data. Specifically, we choose the unseen scene0160\_02 for comparison. }
\label{tab:compare_semanticnerf}
\end{table*}



\end{document}
