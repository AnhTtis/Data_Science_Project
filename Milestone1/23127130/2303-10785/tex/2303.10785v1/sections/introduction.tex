\section{Introduction}\label{sec-1}
% 

%\noindent \textbf{Extension to Curvilinear space:} Though the main governing equations of curvilinear were widely discussed in the literature, the topics such as eigen structure of these governing equations, Riemann solvers, boundary conditions, freestream preservation, viscous fluxes are not given enough attention in the literature in the context of curvilinear coordinates. In this paper we detail everything in a self contained fashion for the reader to be able to implement our method. \\

% Although RANS based methods have gained significant attention and usage in the industry, they fail to resolve vortex-dominated flows, flows with large flow separations, computing aeroacoustic problems, etc.


%The two classes of fourth/sixth order accurate schemes implemented in the solver either use Explicit or implicit gradients of primitive variables to estimate inviscid fluxes, diffusion fluxes and other physical quantities (vorticity, Enstrophy etc.) involved in the post-processing, thus enhancing the computational efficiency. 

% monotonocity preserving inviscid flux discretization and $\alpha$-damping based viscous discretization schemes

% However, the issues concerning application of these methods to non-uniform curvilinear meshes have not been discussed.

% Numerical simulation of high speed flows of practical interest require both good shock-capturing and eddy resolving capabilities. The properties of a numerical scheme to satisfy both these requirements are quite opposing, since eddy resolving requires minimum dissipation while shock capturing on the other hand requires some form of dissipation to avoid oscillations.

Accurate numerical simulations of high-speed flows demand methods that exhibit good shock-capturing and eddy-resolving capabilities. On this front, several high-resolution numerical methods and improvements are proposed in the literature every year \cite{chamarthi2021high,li2021low}. Many of the proposed methods are generally developed and tested on Cartesian grids and employing canonical test cases. However, extending these methods and testing their applicability on the non-uniform meshes is also important to enable complex flow simulations of practical interest. In the previous paper from our group by Chamarthi \cite{chamx}, a new gradient based reconstruction algorithm and $\alpha$-damping \cite{Nishikawa2010} viscous flux discretization scheme were proposed. In this paper, we extend the proposed numerical schemes of Chamarthi \cite{chamx} to a generalized curvilinear coordinate system and perform demonstrations on highly skewed and dynamically deforming meshes. Moreover, we also demonstrate the method's parallelizability and efficacy in resolving the physics of noise due to an under-expanded supersonic jet.


One of the first widely recognized high-order shock-capturing schemes in the literature was the Weighted Essentially Non-Oscillatory (WENO) class of schemes first introduced by Liu et al. \cite{liu1994weighted} and later improved by Jiang and Shu \cite{jiang1996efficient}. The performance of WENO schemes and the issues concerning its extension to simulate flows over non-uniform curvilinear grids were addressed in various studies \cite{cai2008performance,nonomura2010freestream,shadab2019fifth}. It was then followed by the wide usage of the WENO schemes to study the physics of numerous high-speed flow problems of practical interest \cite{cao2019gortler,cheng2005numerical}. Although the WENO class of schemes has gained significant popularity in the literature with several enhancements proposed over the years \cite{martin2006bandwidth,Henrick2005,Fu2016}, alternative high-order shock-capturing schemes with even better spectral properties were also studied and applied. A few examples include the limiter-based Monotonocity Preserving (MP) approach of Suresh and Huynh \cite{suresh1997accurate}, filtering/artificial dissipation based approaches \cite{kawai2008localized, kakumani2022use}, and hybrid approaches such as Boundary Variation Diminishing (BVD) algorithm \cite{sun2016boundary}. 
Efforts were also made by researchers to mitigate the dissipation coming from the shock capturing scheme. This was achieved through hybrid-central schemes, where a non-dissipative central scheme in smooth regions is combined with a shock capturing scheme near discontinuities. Examples of such attempts include the Hybrid Central-WENO schemes by Costa \& Don \cite{costa2007high} and Karami et al. \cite{karami2019high}. In addition, there have been attempts to design non-linearly stable numerical methods that retain the energy properties of the governing equations in a discrete sense. These methods, known as energy consistent schemes, have been found to enhance stability and eliminate numerical dissipation \cite{pirozzoli2011stabilized}. The inviscid scheme in the present study proposed by Chamarthi \cite{chamx} fundamentally operates on a novel reconstruction polynomial that employs explicit or implicit gradients of primitive variables to enhance the solution accuracy and an improved MP limiter for shock capturing. The two versions of the algorithm either use the standard central sixth-order explicit (E6) gradients or the fourth-order implicit gradients (IG4) (specific details of the gradient scheme will be presented in Sec. \ref{sec:disc}). The resultant schemes were named MEG6 and MIG4 where `M' stands for Monotonocity preserving, and `EG6' and `IG4' refer to sixth-order Explicit Gradients and fourth-order Implicit Gradients, respectively. \\

Historically, viscous flux discretization has received significantly less attention in the literature than the discretization of inviscid terms. Many of the widely used viscous flux discretization schemes suffer from the odd-even decoupling phenomenon and poor second derivative spectral properties \cite{chamarthi2022importance,sainadh2022spectral}. To counter this problem, Chamarthi \cite{chamx} has also proposed a new $\alpha$-damping discretization with superior spectral properties. One of the key benefits of the method is that the same cell-center primitive variable gradients used in inviscid scheme are used in the viscous flux discretization algorithm. This gradient-sharing strategy was demonstrated to balance the computational efficiency and solution resolution well. The new inviscid and viscous discretization schemes proposed in Ref. \cite{chamx} were primarily tested using two-dimensional test cases on Cartesian grids. However, the fine details of extending the MEG6/MIG4 and the new $\alpha$-damping schemes to curvilinear, stationary, and dynamic meshes have not been addressed thus far and require attention. This establishes the motivation for the first objective of the current work.

\noindent \textbf{Objective 1:} To adapt MEG6/MIG4 and $\alpha$-damping schemes to simulate flows on stationary and moving curvilinear grids. \\



Mesh non-uniformities and low-quality grid cells are unavoidable and difficult to control while designing grids for complex geometries. Often, such low-quality grids can affect the simulation results, mainly when the interest of the simulation is to capture sensitive flow features such as small amplitude acoustic waves, eddies, or instability waves. The freestream preservation property is vital while solving such flow problems on non-uniform grids (containing grid stretching and local skewness) since the errors associated with non-preserved freestream can grow to become comparable to the magnitude of essential flow features of interest. Visbal and Gaitonde \cite{Visbal2002} were the first to address this issue in the context of solving the compressible Navier-Stokes equations on curvilinear grids using high-order finite difference schemes. They employed the conservative metric term formulations of Thomas and Lombard \cite{thomas1979geometric} and maintained a consistent discretization between inviscid flux terms and metric terms to achieve freestream preservation on stationary and dynamically moving three-dimensional curvilinear meshes. Later, in 2010, Nonomura et al. \cite{nonomura2010freestream} pointed out that employing the same procedure for the WENO class of shock capturing schemes does not preserve freestream due to the nature of WENO reconstruction. They resolved this issue in the same work by proposing an alternative discretization approach. In the later years, other methods were proposed in the literature for freestream preservation in the context of the WENO class of schemes \cite{nonomura2015new,nonomura2010freestream,zhu2019free}. In addition to objective 1 stated above, the present work also explores the freestream preservation nature of the MEG6 and MIG4 schemes using theory and example numerical flow simulations. This motivates the second objective of this work.

\noindent \textbf{Objective 2:} To elucidate the required procedure to guarantee freestream preservation for the present methods on stationary and moving curvilinear grids. \\


%In the current work we show that along with the the use of conservative metrics and consistent discretization of metric terms, it is also important to interpolate the metric terms consistently with the primitive variable reconstruction scheme involved in the inviscid flux estimation. In the current study we demonstrate the freestream preservation property using various test cases on both stationary and dynamic meshes.


% a para about jet noise

% For instance, such errors can lead to nonphysical events such as false transition, vortex decay, or over/under-amplified fluctuation fields.

% In the process of adapting high-order methods to non-uniform (stretched/skewed) grids, the discretization and interpolation of the metric terms has to be treated with care in order to satisfy geometric conservation law (GCL) and volume conservation Law (VCL) [Ref]. Failing to do so can yield inaccuracies in the computed solution which may lead to unphysical solutions.

% general introduction
% Large Eddy Simulations (LES) and Direct Numerical Simulations (DNS) are the current state of the art high-fidelity simulation strategies that enable us to understand the physics of fundamental and applied flow problems [Refs]. In this front, high-order methods offer superior accuracy and numerical efficiency compared to their low-order counter parts by producing superior accuracy at comparatively low grid count. 


% about shock capturing schemes and benifits of MEG6 and MIG4 schemes
% Numerical simulation of high speed flows require both good shock-capturing and eddy resolving capabilities. While the eddy resolving demands high resolution dissipation free numerical discretization, a discontinuity capturing needs dissipation of some form. The dissipation is generally added through artificial viscosity, solution limiting or by choosing an upwind stencil for discretization. Designing a scheme that only does the discontinuity capturing only at the discontinuities and keep the high resolution properties elsewhere and being robust for wide range of simulation scenarios is very challenging. 

% Historically the viscous flux discretization has received lesser attention compared to the inviscid flux discretization. Recent studies by Chamarthi and group [Ref] have demonstrated that they can significantly effect the 


% doing freestrean and extending the scheme to curv is not is not very clear, so...


% about curvilinear grids and how they are useful

% Contrary to other methods, the MEG6 and MIG4 approaches employ gradients in their algorithm which can be reused to compute the viscous fluxes and for other tasks such as post processing; this promotes computational efficiency.  


% But with several high-order methods being proposed every year it is also important to adapt those state-of-the-art strategies to geometrically complex domains which are of interest in practically relevant engineering applications. In this paper we focus on the use of gradient based reconstruction schemes and $\alpha$-damping scheme of Chamarthi et al. \cite{chamx} to simulate flows over stationary and moving grids in curvilinear coordinates. 

%

% Although the RANS based methods have been widely used to study flow problems, they fail to resolve vortex-dominated flows, flows with large flow separations, computing aeroacoustic problems, etc. 

% Although the RANS based methods have been widely used to study high speed flow problems previously, they fail to resolve aeroacoustic vortex-dominated flows, flows with large separations, computing aeroacoustic problems, etc. 
% Powell in 1953 has discovered that non-ideally expanded jets emit sound waves of distinct frequency at very high amplitudes, a phenomenon termed as `screech'.

The present study is part of the broader investigation initiated in the CFDLAB group at Technion to understand the fundamental and applied aspects of noise due to supersonic jets \cite{kakumani2023gpu}. Supersonic jet noise is one of the primary concerns in the aerospace industry. Although numerous experimental and computational studies have been conducted since the 1950s \cite{powell1953mechanism}, some of the fundamental aspects of supersonic jet noise remain unsolved. Of all the supersonic jet noise components, the noise due to aeroacoustic resonance (commonly referred to as screech) can be harmful due to its discrete tonal nature and requires a through physical understanding and control. The review article by Edgington \cite{edgington2019aeroacoustic} summarizes the current understanding of the fundamental aspects of aeroacoustic resonance loop in shock containing supersonic jets. It was believed that the high-accuracy methods (MEG6/MIG4 and $\alpha$-damping schemes) adapted in the current work could offer superior fidelity in resolving the three-dimensional physical aspects of supersonic jet noise that are otherwise challenging to capture even through state-of-the-art experimental diagnostics. This leads to the third objective of the current study.

\noindent \textbf{Objective 3:} To explore the efficacy of MEG6 and MIG4 schemes in resolving the screech tones and unsteady aspects of a supersonic jet. \\


% The current paper attempts to extend the MEG6/MIG4 and $\alpha$-damping schemes to enable Monotonocially Integrated LES [Ref] simulations of practical importance.

% about GPUs
Large Eddy Simulations (LES) and Direct Numerical Simulations (DNS) are the current state-of-the-art high-fidelity simulation strategies that enable us to understand the physics of fundamental and applied flow problems. However, the grid resolution requirements for a wall resolved LES roughly scales with the $\left(\frac{13}{7}\right)^{\text{th}}$ power of the flow Reynolds number \cite{choi2012grid}. As a result, the primary bottleneck in performing LES (particularly of high Reynolds number flows) is the computational time. Although the computing power of CPUs has been growing continuously over the years (owing to their increasing transistor concentration), significant speedup gains in recent times were achieved through dedicated accelerator cards, especially from Graphics Processing Units (GPUs). In recent years, there has been an increasing interest in the use of GPUs to accelerate Computational Fluid Dynamics (CFD) applications. A number of popular open-source GPU-based CFD solvers have been developed, such as PyFR \cite{witherden2014pyfr}, STREAmS \cite{bernardini2021streams}, ZEFR \cite{romero2020zefr}, and HTR \cite{di2021htr}. These solvers utilize some of the latest generations of high-order algorithms to simulate a range of flow regimes including incompressible \cite{witherden2014pyfr}, compressible \cite{bernardini2021streams,romero2020zefr}, and turbulent reacting flows \cite{di2021htr}. Additionally, several non-open-source GPU accelerated codes have also been developed, such as CharLES \cite{goc2021large} and COMP-SQUARE \cite{nampelly2022surface}. A study by Konrad et al. \cite{goc2021large} has shown that utilizing 96 NVIDIA V100 GPU cards can result in up to $26$ times faster performance than using 2000 CPU cores for the same task. This effectively means that each GPU can replace approximately 540 CPU cores. The exceptional acceleration capabilities of GPUs can be attributed to their unique hardware architecture, which is comprised of hundreds of weaker processing elements, referred to as threads, rather than a smaller number of more powerful processor cores as in CPUs. This allows GPUs to excel at processing data when a limited number of straightforward instructions are to be executed on large datasets, a requirement commonly seen in CFD algorithms. Research has also shown that GPUs are more energy efficient than CPUs \cite{huang2009energy,vspetko2021dgx}, which has led to the widespread adoption of GPU cards in the new exascale data center supercomputers and desktop-grade workstations. To take advantage of this increased computing power, it is important to adapt parallelization models in CFD applications to suit GPU accelerators. In this study, we focus on accelerating the MEG6/MIG4 + $\alpha$-damping algorithm using OpenACC. While most of the GPU accelerated CFD applications in the literature primarily use the CUDA based programming model \cite{bres2022gpu,goc2021large,terrana2020gpu,laufer2022gpu,cernetic2022high}, we investigate the performance gains that could be achieved by using OpenACC a directive-based programming language, which requires less development time due to its high-level nature. The fourth objective of the present study is stated below.

\noindent \textbf{Objective 4:} To accelerate the MEG6/MIG4 algorithms on GPUs and explore their computational efficiency and performance statistics on the latest generations of data center GPUs. \\

% puri2017evaluation
% Objectives
% \noindent The distinct objectives of the present work are:
% \begin{enumerate}
%     \item To adapt MEG6/MIG4 and $\alpha$-damping schemes to simulate flows through stationary and moving curvilinear grids.
%     \item To elucidate the required procedure and demonstrate the freestream and vortex preservation nature of MEG6 and MIG4 schemes on stationary and moving grids.
%     \item Explore the efficacy of MEG6 and MIG4 schemes in resolving the screech tones of a supersonic jet.
%     \item To accelerate the MEG6/MIG4 algorithms on single and multi-GPU platforms explore the computational efficiency. \\
% \end{enumerate}

% paper structure
The rest of the article is organized as follows. Firstly, the governing equations are presented. This is followed by a brief description of the conservative finite difference discretization approach in section \ref{sec:coner-FD+algo}. The freestream preserving conservative metric formulations are presented in section \ref{conser-metrics}. The discretization approach employed for inviscid and viscous terms is presented in Section \ref{sec:disc}. The freestream preservation nature of MEG6 and MIG4 schemes is discussed in section \ref{sec:FP}. Next, the results and demonstrations are presented using a suite of standard and practical test cases (section \ref{sec:results}). In section \ref{sec:gpu-accel} the GPU acceleration models adapted for single and multi-GPU simulations to accelerate the MEG6 and MIG4 algorithms are presented with detailed analyses. Concluding remarks are laid out in the last section.


% Although GPUs were originally designed to carry out graphical operations, they were also found to offer an excellent parallel architecture for general-purpose computing. A few areas where GPUs are widely being used are scientific computing, machine learning, data analytics, and image processing.

% When it comes to low speed compressible flows which are generally free of shocks and other flow discontinuities, the complexity of numerical scheme that is required is relatively less as one doesn't need to worry about resolving any flow discontinuities.

% ideas
% TGV on deformed and wavy meshes, compare enstrophy and dissipation rate with uniform meshes
% 