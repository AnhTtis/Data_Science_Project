\subsection{Mach 1.35 under-expanded supersonic round jet} \label{sec:JNapplication}
Shock containing supersonic jets emit intense aeroacoustic noise at distinct frequencies due to the phenomenon termed `screech'. The nearfield noise levels due to screech in a free single jet configuration typically range between 120dB-180dB. Although the physical mechanism of screech tone production is still not completely understood, it is a well-established fact that it occurs as the consequence of aeroacoustic resonance \cite{edgington2019aeroacoustic,edgington2021generation}. According to the classical feedback loop theory of Powell \cite{powell1953mechanism,raman1999supersonic}, firstly, a series of small flow disturbances are initiated at the nozzle lip region. These disturbances travel downstream (in the form of Kelvin-Helmholtz instabilities) towards the shock cells to interact with the oblique shocks. As the disturbances interact with the shocks (generally after the second or third shock cells), they undergo strong amplification and radiate intense acoustic waves, which drive the entire jet plume to resonate and consequently oscillate. Some of the produced acoustic waves travel upstream towards the nozzle lip through the atmosphere outside the jet to close the feedback loop. The screech is sustained with the continuous excitation of the shear layer (receptivity process) from these emission waves. Therefore, since the screech depends on several flow features, such as shock cells, shear layer, vortical structures, and acoustic waves, it is crucial to capture all these features with sufficient fidelity in order to simulate its effect through computations. The main objective of this section is to test the efficacy of MEG6 and MIG4 schemes in resolving `screech' and `unsteady jet oscillation mode' of an under-expanded supersonic jet. Simulations are also performed using WENO-Z \cite{borges2008improved} scheme for comparison against the present scheme.

\begin{figure}[h!]
    \centering
    \includegraphics[width=160mm]{Images/jet-domain-mesh.pdf}
    \caption{(a) Computational domain used, (b) Butterfly mesh topology adapted to avoid singularity at the axis, visualized on $x=2D$ plane.}
    \label{jet-domain-mesh}
\end{figure}

An under-expanded supersonic jet ejecting from a converging, choked nozzle operating at a Nozzle Pressure Ratio (NPR) of $2.97$ is considered as the test case. The flow simulated in the present computations is compared with the nearfield microphone data of Ponton et al. \cite{ponton1997near}. The experimental setup used in their work consists of a settling chamber attached to an axisymmetric converging nozzle with an exit diameter (D) equal to one inch. The lip thickness of the nozzle exit is $0.6D$. The flow diagnostics include a microphone placed on the nozzle exit plane at a radial location of two nozzle diameters away from the nozzle axis. Fig. \ref{jet-domain-mesh}a shows the dimensions of the computational domain used to simulate the flow. The domain is axisymmetric and consists of four blocks. The jet Reynolds number based on the nozzle diameter and the jet exit conditions is $Re_D=1.144 \times 10^6$. The flow is non-dimensionalized based on the ambient atmospheric conditions ($T_{\infty}=293\text{ K}$, $p_{\infty} = 101325\text{ Pa}$). The nozzle exit conditions are theoretically computed using Eqns. \ref{ini-cons}. The computed inlet conditions are directly provided as an inlet boundary condition on the nozzle exit plane. The effect of the boundary layer and turbulent fluctuations developed inside the annular region are not considered. Although this does not replicate the experimental conditions accurately, the effect of annular fluctuations in the present case was believed to show little effect on the screech feedback loop based on the study by \cite{ahn2021numerical,ahn2018supersonic}. To prevent spurious reflections from entering in to the computational domain, sponge zones are employed at the far-field boundaries in the regions marked in Fig. \ref{jet-domain-mesh}. These zones are implemented by incorporating a dissipative source term into the governing equations as outlined in the methodology of Bodony \cite{bodony2006analysis}.

\begin{equation} \label{ini-cons}
    \begin{aligned}
    &\frac{p_e}{p_{\infty}}=\frac{1}{\gamma}\left[\frac{2+(\gamma-1) M_j^2}{\gamma+1}\right]^{\frac{\gamma}{\gamma-1}} \\
    &\frac{\rho_e}{\rho_{\infty}}=\frac{\gamma(\gamma+1) p_e}{2 (T_n/T_{\infty})} \\
    &\frac{u_e}{a_{\infty}}=\sqrt{\frac{2 (T_n/T_{\infty})}{\gamma+1}}, \quad v_e = w_e = 0
\end{aligned}
\end{equation}

In this test case, three grids were utilized with resolutions of $10$, $13$, and $20$ million cells to examine the independence of the results from the choice of grid. To prevent a singularity at the nozzle axis, the butterfly topology was employed, as illustrated in Fig. \ref{jet-domain-mesh}b. The grid was clustered towards the axis and exit of the nozzle, where the key flow features are located. Along the axial direction of the geometry, a minimum grid spacing of $\Delta x = 0.0085D$, $0.0075D$, and $0.0065D$ was maintained for the $10$, $13$, and $20$ million grids, respectively. The grid spacing was linearly increased along the $x$-direction, with the maximum grid spacing not exceeding $0.1D$ in all three grids. The grid was uniformly distributed along the azimuthal direction, with 80, 100, and 120 divisions for the 10, 13, and 20 million grids, respectively. The grid distribution along the radial direction was consistent across all three meshes used in the study. Along the radial direction, a minimum spacing of $0.0075D$ was used close to the axis, with a growth rate of $\approx 4\%$, resulting in a total of 120 cells. Computations were performed with a $\Delta t a_{\infty}/D = 2\times10^{-3}$ until an end-time of $t a_{\infty}/D = 800$ ($400,000$ iterations). All the statistics were collected after the flow reached a statistically steady state, which is after $t a_{\infty}/D = 400$. All the simulations were run in parallel on a single Nvidia A100 GPU. The simulations corresponding to 13 million grid were completed in a span of $34.0$ and $41.7$ hours using MEG6 and MIG4 schemes, respectively. More details about the GPU acceleration will be discussed in the next section.


Given the high Reynolds number in the present case, Large Eddy Simulations (LES) were conducted on course grids. In conventional Large Eddy Simulation (LES) approaches, the effect of unresolved sub-grid-scale (SGS) flow features is modeled through an explicit SGS model while the larger scales are resolved by the grid \cite{germano1991dynamic}. However, in the present simulations, we solve the unfiltered Navier-Stokes equations directly and utilize the inherent numerical dissipation of the MEG6/MIG4 schemes to implicitly mimic the effects of SGS models. This approach is commonly referred to as ``implicit-LES'' or, specifically in the present case, ``Monotonically integrated Implicit LES (MILES)'' \cite{fureby1999monotonically} because of the use of the Monotonocity Preserving limiter in the inviscid flux discretization and its high spectral resolution. A similar LES strategy was also employed in the study performed by Ahn et al. \cite{ahn2021numerical} on twin-jet configurations.

\subsubsection{Instantaneous and average flow features}
Fig. \ref{jet-inst-ave}a-b shows the time-averaged axial density and $x$-velocity profiles on three different grid resolutions. The peaks and troughs in the profiles suggest the presence of shock cells. The solution has converged sufficiently at a grid resolution of $13$ million cells, especially in the region corresponding to the first four shock cells. It can be noted that the solution corresponding to the 10 million grid fails to capture the peaks and troughs of the density and velocity profiles between the range of $x/D=0$ to $4$, which is critical for the screech phenomenon. Thus, the results corresponding to $10$ million grid are not considered for the analysis. Fig. \ref{jet-inst-ave}c-d shows the instantaneous and mean contours of density and Mach number. Various flow features such as shocks, expansion fans, shear-layer, vortices, and induced density fluctuations can be noted in the pictures. The first two shock cells appear clearly defined, but progressively downstream the structure of shock cells become less distinct due to the growth of instability waves and turbulent mixing. The non-dimensional shock cell spacing in the jet is a crucial length scale in supersonic jets as it dictates the location of effective screech noise source \cite{powell1953mechanism}. The first shock cell spacing denoted by $\lambda_1$, can be calculated theoretically using the formulation proposed by Pack \cite{pack1950note} through Eqn. \ref{firstShockSpace} approximately.

\begin{equation}\label{firstShockSpace}
    \lambda_1=2.695 \sqrt{\left(\mathrm{NPR}^{0.291}-1.205\right)}
\end{equation}

Plugging in NPR=$2.97$ in to the equation, results in $\lambda_1 \approx 1.1$. On the other hand in the present simulations, as can be seen in the time-averaged density and velocity plots shown in Fig. \ref{jet-inst-ave}(a,b), the first shock cell spacing was obtained to be $\lambda_1 = 1.15$. A good agreement is observed between the theoretical calculations and the numerical results. Subsequently the shock cell spacing downstream was noted to slowly decrease with  $\lambda_2=1.15$, $\lambda_3=1.0$, $\lambda_4=1.0$, and $\lambda_5=0.9$. In Fig. \ref{jet-inst-ave}(c,d) the shear layer instabilities initiated near the nozzle lip can be seen to grow progressively in scale as they move downstream to interact with the shock cells. Consequently, local acoustic disturbances are produced, which travel upstream to close the feedback loop causing screech, as explained before. To better visualize this phenomenon, movies of density field are provided in the supplementary materials section \ref{sec:supply}. From the time-resolved data shown in Movie-1 of Sec. \ref{sec:supply}, the evolution of Kelvin-Helmholtz instabilities, the interaction of shear layer vortices with the shock cells, emitted acoustic radiation, and the consequent jet oscillation behavior can be remarked.

%  The movies are particularly saved at very high temporal resolution (XX Hz)) that is generally difficult to achieve on the modern high speed cameras.
% The average shock cell spacing is noted to be close to $1.2D$.

A supersonic jet has multiple noise sources \cite{tam1995supersonic,bailly2016high,edgington2019aeroacoustic}. The above-mentioned acoustic resonance, shock shear layer interaction, and the jet mixing are part of the generation mechanisms behind different the various noise components. The tones corresponding to screech noise are predominant and can be characterized in the upstream region of the jet \cite{gojon2019antisymmetric,powell1953mechanism,davies1962tones,westley1969near}. Fig. \ref{jet-spl}(a) presents a snapshot of the instantaneous 3-D jet surface visualized through density iso-surface, along with the pressure fluctuation field surrounding it. The figure illustrates the progressive amplification of three-dimensional hydrodynamic disturbances on the jet surface close to the nozzle exit. In Fig. \ref{jet-spl}(b), the instantaneous pressure fluctuation contours and the mean Sound Pressure Levels (SPL) are displayed on the $z=0$ plane. The pressure fluctuation field immediately close to the jet surface appears choppy and seemingly random; however, a clear sinusoidal-type pattern can be observed in the upstream region of the jet just above the nozzle, as indicated by the dashed white box in Fig. \ref{jet-spl}. Given the pronounced sinusoidal nature of the wave pattern and the prevalence of screech in the upstream location, it can be inferred that these fluctuations correspond to screech. The lower half of Fig. \ref{jet-spl} displays isolines of SPL, with SPL near the lip of the nozzle observed to be close to $150$dB. The solutions presented in Fig. \ref{jet-inst-ave} correspond to the MIG4 scheme; however, similar results were also observed with the MEG6 scheme.

\begin{figure}[h!]
    \centering
    \includegraphics[width=150mm]{Images/jet-inst-ave.pdf}
    \caption{Time averaged axial profiles of density (a) and x-velocity (b) at various grid resolutions. The length of first shock cell spacing $\lambda_1$ is also indicated in the figures (a) and (b)}. Instantaneous and mean density contours (c) and local Mach number (d) contours of $M_{j}=1.35$ jet solution plotted on $z/D = 0$ plane employing a grid resolution of 20M. All the plots shown here are based on MIG4 scheme.
    \label{jet-inst-ave}
\end{figure}


\begin{figure}[h!]
    \centering
    \includegraphics[width=\textwidth]{Images/jet-spl.pdf}
    \caption{(a) Visualization of the jet surface using an iso-surface of density corresponding to $\rho/\rho_{\infty}=1.2$. The figure also illustrates the circular pressure ripples surrounding the jet planes $y=0$ and $x=-2D$. (b) Instantaneous pressure fluctuations (upper half) and overall sound pressure levels (lower half) are shown.}
    \label{jet-spl}
\end{figure}

\subsubsection{Comparison of acoustic data with experiments} \label{freqs}

\begin{figure}[h!]
    \centering
    \includegraphics[width=\textwidth]{Images/jet-freqs.pdf}
    \caption{Comparison of pressure spectra of probe data collected at various azimuthal angles on the nozzle exit plane $2D$ away from the nozzle axis, using (a,b) MEG6, (c,d) MIG4, and (e,f) WENO-Z scheme, with the experimental data of Ponton et al. \cite{ponton1997near}. The figure also includes (g) a plot illustrating the variation of the amplitude of the fundamental tone with azimuthal angle, using data from twenty probes and the various schemes.}
    \label{jet-freqs}
\end{figure}

To evaluate the screech tones and amplitudes captured in the flowfield (Fig. \ref{jet-spl}), a Fourier analysis of the nearfield pressure data is conducted and compared to the experiments of Ponton et al. \cite{ponton1997near}. Given the non-axisymmetric nature of screech pressure fluctuations at the present jet Mach number as previously reported in studies \cite{ahn2021numerical,gojon2019antisymmetric}, time-series data is gathered at multiple azimuthal locations on the jet exit plane. A total of 20 equally spaced pressure probes (computational) are placed at two nozzle diameters from the nozzle axis on the jet exit plane, as shown in Fig. \ref{jet-freqs}e. In contrast, the experiments of Ponton et al. \cite{ponton1997near} only utilized one microphone placed at an arbitrary azimuthal location two diameters away from the nozzle axis on the jet exit plane. The Fourier analysis of the time-series data collected at various azimuthal locations is then compared to the pressure spectra obtained in the experiments \cite{ponton1997near}.

Fig. \ref{jet-freqs}(a-d) presents pressure spectra of probe data using the MEG6, MIG4, and WENO-Z schemes on a 13 million cell grid at various azimuthal locations. The accuracy of capturing the fundamental screech frequency and the first harmonic at all azimuthal locations is observed in computations performed on both 13 and 20 million cell grids. As an illustration, four sample pressure spectra are displayed at different azimuthal angles, with the peak of the fundamental tone indicated by a dashed circle. The Strouhal number remains constant across all azimuthal angles, as summarized in Table \ref{jet-table} for the fundamental screech tone at different grid resolutions using the MEG6 and MIG4 schemes, showing a good agreement with experimental results. In addition to the screech, the plots also reveal the presence of the first harmonic at around St$\approx 0.62$ and mixing noise component at St$\approx 0.2$. However, the WENO-Z scheme in Fig. \ref{jet-freqs}(e,f) exhibited a fundamental tone at St$=0.43$ which is significantly off from the experimental value. The inaccuracy can be attributed to the relatively poor spectral properties of the scheme.

Despite the constant Strouhal number across various azimuthal probe locations, variations in amplitude can be observed in Fig. \ref{jet-freqs}a-d. The screech amplitudes recorded at different azimuthal angles ($\theta$) using the MEG6, MIG4, and WENO-Z schemes are presented in Fig. \ref{jet-freqs}g. The amplitude of the WENO-Z scheme maintains a consistent level, while the MEG6 and MIG4 schemes exhibit a fluctuating wavy pattern with a difference in amplitude between peaks and troughs greater than 15dB. The physical reasoning behind this observation is due to the flapping mode of the jet which will be discussed next in Sec. \ref{sec:jet-oscill}. Table \ref{jet-table} presents the azimuthal average of screech amplitudes for different grid resolutions using the MEG6 and MIG4 schemes, and a good agreement with experimental values is observed. This supports the effectiveness of both MEG6 and MIG4 schemes in resolving supersonic jet screech `frequencies' and `amplitudes'.

% Given the axisymmetric nature of the geometry, this may seem counterintuitive. But, given the existence of a flapping plane and non-uniform distribution of pressure fluctuations along the azimuthal direction, the amplitude variation of screech tones along the azimuthal direction is expected.

\begin{table}[h!]
    \centering
    \includegraphics[width=120mm]{Images/jet-table.pdf}
    \caption{Summary of Strouhal numbers and average amplitudes of fundamental screech tones obtained at various grid resolutions employing MEG6, MIG4 and WENO-Z (only at 13M cell count) schemes. The screech amplitude corresponding to experiments presented in the table does not correspond to azimuthal average since only one microphone is employed in the experiments.}
    \label{jet-table}
\end{table}

\subsubsection{Unsteady jet oscillation behaviour} \label{sec:jet-oscill}

% In order to understand the jet oscillation behavior, pressure contours are visualized on a Y-Z plane situated two diameters upstream of the nozzle exit ($x=-2D$). The lateral oscillations normal to the $x$-axis can be observed on this plane. Contours of instantaneous pressure fluctuations on the plane are shown in Fig. \ref{jet-precess}. The contours show that the pressure fluctuations are spread out along a particular orientation in a polarized fashion. Such a distribution suggests that the jet is undergoing a flapping motion. Furthermore it can be noted that the pressure fluctuation magnitude is out of phase along the flapping indicated with a green dashed line. This particular oscillation mode where the jet primarily undergoes the flapping along a preferred plane of oscillation can be categorized into flapping `mode-B' according to the literature \cite{powell1992observations,umeda2001sound}.

% This particular oscillation mode where the jet is undergoing simultaneous flapping and precession can be categorized as flapping `mode-D' according to the literature \cite{powell1992observations,umeda2001sound}.

% \begin{figure}[h!]
%     \centering
%     \includegraphics[width=150mm]{Images/jet-precess.pdf}
%     \caption{\textcolor{magenta}{(a) Instantaneous pressure fluctuation contours on the $x=-2D$ and $y=0$ planes, overlaid with an iso-surface of density corresponding to $\frac{\rho}{\rho_{\infty}}=1.2$. (b) Contours of instantaneous pressure fluctuations on the $x=-2D$ plane, indicating the approximate orientation of the flapping plane.}}
%     \label{jet-precess}
% \end{figure}

\begin{figure}[h!]
    \centering
    \includegraphics[width=\textwidth]{Images/jet-line-fft.pdf}
    \caption{Pressure spectra (a,c) and phase variation of fundamental frequency $St=0.308$ (b,d) along a circular line two diameters away from the nozzle axis on the $x=-2D$ plane using MEG6 and WENO-Z schemes. The location of the fundamental frequency is marked with a red arrow on the top in figures (a) and (c). The phase shift of $\pi$ in figure (b) illustrates the flapping nature of the jet as captured by the MEG6 scheme and in the experiments of Ponton et al. \cite{ponton1997near}. Black arrows in figure (a) indicate the azimuthal location at which this phase shift occurs.}
    \label{new1}
\end{figure}

Previously in Sec \ref{freqs}, the amplitude variation depicted in Fig. \ref{jet-freqs} indicated the presence of azimuthally asymmetric jet oscillations. However, the specific oscillation mode cannot be determined from those plots alone as they correspond to a data from single point. Previous research by Powell et al. \cite{powell1992observations} has determined that the jet should exhibit flapping B mode oscillations at the current Nozzle Pressure Ratio of 2.97. In order to accurately identify the jet oscillation mode in the present simulations, Fourier analysis was conducted on two-dimensional nearfield pressure data on plane $x=-2D$, which is a jet normal plane located two diameters upstream of the nozzle exit. The results from the MEG6 and WENO-Z methods are compared to each other along with experimental observations \cite{ponton1997near,powell1992observations}.

In Fig. \ref{new1}(a,c), the frequency and amplitude of pressure data collected along a circular line located two diameters away from the nozzle axis on the $x=-2D$ plane are presented, utilizing the MEG6 and WENO-Z schemes. The ordinate of the plot represents the azimuthal angle $\theta$ along the circle. The data reveals the presence of fundamental tones (marked with red arrows on top) at Strouhal numbers of $0.308$ and $0.432$, which are consistent with the observations made in the Fourier analysis presented in Sec. \ref{freqs}. The spectrum for the WENO-Z scheme in Fig. \ref{new1}(c) appears fairly uniform at all azimuthal angles, whereas the MEG6 results in Fig. \ref{new1}(a) exhibit two notches separated by an angular distance of $\pi$ radians which are indicated by two black arrows. This suggests the presence of a plane of symmetry on the $x=-2D$ surface. The phase angle (the argument of the Fourier transformation) at the fundamental tone for the MEG6 and WENO-Z schemes is plotted in Fig. \ref{new1}(b,c). The MEG6 results show a step-like distribution with a step height equal to $\pi$ radians, which is indicative of sinuous flapping behavior of the jet, where pressure fluctuations are separated by a phase difference of $\pi$ radians about a plane of symmetry. In contrast, the WENO-Z results shown in Fig. \ref{new1}(d) exhibited a constant phase value at all azimuthal locations, indicating an axisymmetric or toroidal oscillation mode (or A1 mode) of the jet, which ideally leads to axisymmetric pressure fluctuations. To better understand these aspects noted from the 1D line data, the Fourier amplitude and phase fields of 2D data from the plane at $x = -2D$ are presented next.

\begin{figure}[h!]
    \centering
    \includegraphics[width=\textwidth]{Images/jet-amp-phase.pdf}
    \caption{Fourier amplitude fields (a,c) and Fourier phase fields (b,d), corresponding to the fundamental frequency $St=0.308$ on $x=-2D$ plane using MEG6 and WENO-Z schemes. The contour distributions depicted in these figures indicate the presence of the flapping B-mode and A1-toroidal modes, captured by the MEG6 and WENO-Z schemes, respectively.}
    \label{new2}
\end{figure}

Fig. \ref{new2} illustrates the Fourier amplitudes and phase values computed on the $x=-2D$ plane at the fundamental tone frequency, as determined by the MEG6 and WENO-Z schemes. It can be observed that the results obtained using the WENO-Z scheme exhibit a clear axisymmetric distribution of both amplitude and phase, which is not consistent with the flapping behavior observed in experiments \cite{powell1992observations}. In contrast, the amplitude levels for the MEG6 scheme, as shown in Fig. \ref{new2}(a), displays a plane of symmetry, about which the jet flapping occurs. Additionally, a spiraling phase distribution, as seen in Fig. \ref{new2}(b) for the MEG6 scheme, is a clear indication of the sinuous flapping mode. This conclusively confirms the efficacy of the MEG6 scheme in resolving the Flapping B mode observed by Powell et al. \cite{powell1992observations} in their experiments.


\subsubsection{Proper Orthogonal Decomposition of density field}

Proper Orthogonal Decomposition (POD) was performed on the fluctuations of the 2D density field collected on the $xy$-plane, within a window of $x/D=[0,8]$ and $y/D=[-4,4]$, to identify the dominant unsteady coherent structures within the jet that contribute to the screech. The POD analysis in this Sec. is based on the snapshot technique described by Weiss in Ref-\cite{weiss2019tutorial} and follows a similar approach to the analysis conducted by Chandravamsi et al. \cite{chandravamsi2023control}. Firstly, a time series of `$N$' ($N=1600$ in the present case) evenly spaced instantaneous 2D density fluctuation field data (separated by $\Delta t a_{\infty}/D=0.015$) is collected and reshaped to form a 2D solution matrix named $\mathbf{A} = [\rho^{'}(x,y,t_1), \rho^{'}(x,y,t_2), .... \rho^{'}(x,y,t_N)]$. The data is reshaped in such a way that each column in $\mathbf{A}$ corresponds to a `time-instance' and each row `spatial location'. The covariance matrix $\mathbf{C}=\mathbf{A}^{T}\mathbf{A}$ was then subjected to eigenvalue decomposition to find its eigenvalues `$\Lambda_i$' and eigenvectors $\mathbf{E_i}$, forming an orthogonal basis in the eigenspace where each eigenvalue represents the variance along the corresponding eigenvector. Finally, the POD modes ($\mathbf{\Phi}_i$) and their corresponding time coefficients ($\Tilde{\mathbf{a}}$) are evaluated using the eigenvectors and input flow data ($\rho^{'}(x,y,t)$) using the following relations:

% The eigen values $\Lambda_i$ and eigen vectors $\mathbf{E}_i$ are sorted in the increasing order of the variance (magnitude of eigen value).

\begin{equation}
    \boldsymbol{\Phi}_i=\frac{\sum_{j=1}^N E_i^j \rho^{'}(x,y,t_j)}{\left\|\sum_{j=1}^N E_i^j \rho^{'}(x,y,t_j)\right\|}, \quad \text{and} \quad \Tilde{\mathbf{a}}_i = (\mathbf{\Phi}_i)^{T}\mathbf{A}, \quad i=1, 2 \ldots, N.
\end{equation}

The index `$i$' here represents the mode number. The modes are ordered based on their relative mode energy, calculated as $RE_i = \Lambda_i \times \left(\sum_{i=1}^N \Lambda_i\right)^{-1}$. As depicted in Fig. \ref{POD}(a), amongst the first eight dominant modes displayed, the first mode itself accounts for $\approx 45\%$ of the total energy. A noticeable decrease in energy can be observed following the first mode, implying that only a single dominant mode exists, and the majority of the jet's unsteady dynamics are contained within mode-1. The normalized mode-1 shape is displayed in Fig. \ref{POD}(b), along with annotations describing the locations of shock cells by white dashed lines. The shock cell spacings ($\lambda$), as determined from Fig. \ref{jet-inst-ave}(b), are indicated at the top. The mode shape exhibits symmetry about the jet axis, with a reversal in sign, indicating that the jet oscillations of the most energetic portion are oriented along the $y$-axis. The dominant fluctuations are observed in the region corresponding to the third, fourth, and fifth shock cells, suggesting that the shear layer structures along the jet surface experience significant amplification after their interaction with the third shock cell. Therefore, the effective source of the screech noise can be considered to be situated between the third and fifth shock cells. As depicted in Fig. \ref{POD}(c), the Fourier analysis of the mode-1 time coefficients revealed a ``pure tone'' that precisely matches the screech frequency ($St = 0.308$) determined from the acoustic data in Sec. \ref{freqs}. This suggests that the oscillation frequency of the oblique shocks within the jet plume and the turbulent structures between the third and fifth shock cells are all operating at the same frequency as the screech.

\begin{figure}[h!]
    \centering
    \includegraphics[width=\textwidth]{Images/POD.pdf}
    \caption{(a) Mode energies of first eight POD modes, (b) Mode-1 shape with shock cell locations and spacing indicated, and (c) Power Spectral Density of mode-1 POD
coefficients.}
    \label{POD}
\end{figure}

