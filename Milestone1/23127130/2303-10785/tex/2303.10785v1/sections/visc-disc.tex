\subsection{Viscous flux discretization - $\alpha$-damping scheme}  \label{sec:viscDisc}

Many viscous schemes such as those proposed in \cite{Visbal2002,shen2010large} are prone to odd-even decoupling and inaccurate spectral representation particularly in the high wave-number region. In the recent past, efforts have been made in the development of numerical algorithms to compute diffusion fluxes without odd-even decoupling and good spectral properties over a broader wavenumber range \cite{Nishikawa2013, chamarthi2022importance,sainadh2022spectral}. The current work employs the fourth order accurate version of viscous fluxes proposed by Chamarthi et al. \cite{chamarthi2022importance}. The gradients required for the viscous fluxes are not re-computed here since they are already computed while estimating inviscid flux residuals (Eqns-\ref{E6_grads} and \ref{IG4_grads}). A step-by-step procedure to evaluate the interface viscous fluxes according to this algorithm in curvilinear coordinates is depicted in Fig. \ref{visc_algo}. For simplicity, the steps are described only for the $\xi$ direction but the same can be extrapolated to other directions, as well. \\

\begin{figure}[h!]
    \centering
    \includegraphics[width=140mm]{Images/visc_algo.pdf}
    \caption{Various stages involved in estimating the viscous flux residual using the $\alpha$-damping scheme. Applies for both E6 and IG4 gradients of $\bm{\mathit{\mathscr{P}}}$, yielding 4th order accuracy for both \cite{chamx}.}
    \label{visc_algo}
\end{figure}

\noindent \textbf{Step-1, Compute interface gradients:} Interface gradients of $\bm{\mathit{\mathscr{P}}} = [u,v,w,T]^{T}$ in $\xi$ $\eta$ and $\zeta$ directions are first computed using Eqn. \ref{AD-eqn1}, where $\bm{\mathit{\mathscr{P}}}$ is a vector of variables required to evaluate the viscous fluxes. For simplicity, the equations are presented only for the variable $u$ here. The same should be repeated for other variables of $\bm{\mathit{\mathscr{P}}}$ as well.

\begin{equation} \label{AD-eqn1}
    \left(\frac{\partial u}{\partial \xi}\right)_{i+\frac{1}{2}}=\underbrace{\frac{1}{2}\left[\left(\frac{\partial u}{\partial \xi}\right)_{i}+\left(\frac{\partial u}{\partial \xi}\right)_{i+1}\right]}_{\text {Consistent term }}+\underbrace{\frac{\alpha^D}{2}\left(u_{R}-u_{L}\right)}_{\text {Damping term }},
\end{equation}

Where the left and right states of $u$ are defined using the following interpolation polynomial:
\begin{eqnarray} \label{AD-eqn2}
u_{L}=u_{i}+ 0.5 \left(\frac{\partial u}{\partial \xi}\right)_{i}+\beta^D\left(u_{i+1}-2 u_{i}+u_{i-1}\right) \\
\quad u_{R}=u_{i+1}- 0.5 \left(\frac{\partial u}{\partial \xi}\right)_{i+1}+\beta^D\left(u_{i+2}-2 u_{i+1}+u_{i}\right) .
\end{eqnarray}

$\alpha^D=4$ and $\beta^D=0$ are the scheme coefficients based on Ref. \cite{chamx} for both E6 and IG4 based gradients of $\bm{\mathit{\mathscr{P}}}$. These coefficients produce fourth order accuracy for both E6 and IG4 gradients. The temperature at the $i+\frac{1}{2}$ location is computed from an arithmetic average of $T_L$ and $T_R$ values. From this, the interface dynamic viscosity $\mu_{i+\frac{1}{2}}$ is estimated using the Sutherland's law of viscosity. \\

\noindent \textbf{Step-2, Compute the derivatives in Cartesian coordinates:} Using the $\xi$, $\eta$ and $\zeta$ direction gradients of $[u,v,w,T]$ computed in the previous step, the derivative chain rule is used to evaluate the $x$, $y$, and $z$ derivatives of the same variables. For instance, this is done as follows to compute the interface $x$ direction derivative:

\begin{equation} \label{AD-eqn3}
   \left(\frac{\partial u}{\partial x}\right)_{i+1 / 2} = (\xi_{x})_{i+\frac{1}{2}} \left(\frac{\partial u}{\partial \xi}\right)_{i+\frac{1}{2}} + (\eta_{x})_{i+\frac{1}{2}} \left(\frac{\partial u}{\partial \eta}\right)_{i+\frac{1}{2}} + (\zeta_{x})_{i+\frac{1}{2}} \left(\frac{\partial u}{\partial \zeta}\right)_{i+\frac{1}{2}}
\end{equation}

\noindent Note: Metric values at $i+\frac{1}{2}$ location are interpolated from the cell center metrics using the relations provided in Eqns. \ref{FP-interp} and \ref{FP-interp2}. \\

\noindent \textbf{Step-3, Compute stress terms:} The shear stress and energy diffusion terms $\tau_{ij}$, $\beta_i$ present in the viscous flux vector are computed at $i+\frac{1}{2}$ using Eqns. \ref{eqn:shear-str} and \ref{eqn:thermal-work} provided in Section \ref{sec:gov-eqns} and with the interface gradients computed in the previous step. The required interface velocities are computed via an arithmetic average of the left and right states of the velocities (e.g. $u_{i+\frac{1}{2}} = 0.5(u_L + u_R)$). The shear-stress term $\tau_{xx}$ at $i+\frac{1}{2}$ is evaluated as follows:

\begin{equation}
    (\tau_{xx})_{i+\frac{1}{2}} = 2 \mu_{i+\frac{1}{2}} \left(\frac{\partial u}{\partial x} \right)_{i+\frac{1}{2}} + \lambda_{i+\frac{1}{2}} \left[ \left(\frac{\partial u}{\partial x} \right)_{i+\frac{1}{2}} + 
    \left(\frac{\partial v}{\partial y} \right)_{i+\frac{1}{2}} +
    \left(\frac{\partial w}{\partial z} \right)_{i+\frac{1}{2}}\right]
\end{equation} \\

\noindent \textbf{Step-4, Estimate the interface fluxes:} The viscous fluxes at $i+\frac{1}{2}$ locations are computed using the relations provided in Eqn. \ref{visc-curvi}. For instance, the viscous flux vector in the $\xi$ direction is evaluated as follows:

\begin{equation}
    \hat{F}^{\text{V}}_{i+\frac{1}{2}}= \left[\begin{array}{c}
0 \\
\left(\hat{\xi}_{x}\right)_{i+\frac{1}{2}} (\tau_{x x})_{i+\frac{1}{2}}+\left(\hat{\xi}_{y}\right)_{i+\frac{1}{2}} (\tau_{x y})_{i+\frac{1}{2}}+\left(\hat{\xi}_{z}\right)_{i+\frac{1}{2}} (\tau_{x z})_{i+\frac{1}{2}} \\
\left(\hat{\xi}_{x}\right)_{i+\frac{1}{2}} (\tau_{y x})_{i+\frac{1}{2}}+\left(\hat{\xi}_{y}\right)_{i+\frac{1}{2}} (\tau_{y y})_{i+\frac{1}{2}}+\left(\hat{\xi}_{z}\right)_{i+\frac{1}{2}} (\tau_{y z})_{i+\frac{1}{2}} \\
\left(\hat{\xi}_{x}\right)_{i+\frac{1}{2}} (\tau_{z x})_{i+\frac{1}{2}}+\left(\hat{\xi}_{y}\right)_{i+\frac{1}{2}} (\tau_{z y})_{i+\frac{1}{2}}+\left(\hat{\xi}_{z}\right)_{i+\frac{1}{2}} (\tau_{z z})_{i+\frac{1}{2}} \\
\left(\hat{\xi}_{x}\right)_{i+\frac{1}{2}} (\beta_{x} )_{i+\frac{1}{2}}+\left(\hat{\xi}_{y}\right)_{i+\frac{1}{2}} (\beta_{y} )_{i+\frac{1}{2}}+\left(\hat{\xi}_{z}\right)_{i+\frac{1}{2}} (\beta_{z} )_{i+\frac{1}{2}}
\end{array}\right]
\end{equation}

\noindent \textbf{Step-5, Compute residual:} Finally, the viscous flux residual is computed using the following relation:

\begin{equation}
    \left(\frac{\partial \boldsymbol{F}^v}{\partial \xi}\right)_i \approx \frac{\boldsymbol{F}_{i+\frac{1}{2}}^v-\boldsymbol{F}_{i-\frac{1}{2}}^v}{\Delta \xi}
\end{equation}

After computing the inviscid and viscous flux residuals time marching is performed to compute the solution corresponding to next time-step. The details regarding the time marching scheme and the stable time-step evaluation for both inviscid and viscous flow simulations is detailed in \ref{sec:time-int}.