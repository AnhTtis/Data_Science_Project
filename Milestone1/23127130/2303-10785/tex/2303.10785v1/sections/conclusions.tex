\section{Conclusions}

Various aspects of extending the `MEG6/MIG4' and `$\alpha$-damping' schemes (proposed by Chamarthi (2022) in Ref. \cite{chamx}) to simulate flows over curvilinear, multi-block meshes are presented in this paper. The first part of this paper discusses the extension of MEG6/MIG4 and $\alpha$-damping schemes to the generalized curvilinear coordinates. Various steps required to achieve freestream and vortex preservation on static and dynamic low-quality stretched and skewed grids are presented with demonstrative examples. The theoretical and numerical results showed that the freestream preservation property was satisfied by the use of central schemes to compute metrics that were derived consistently from the corresponding upwind schemes MEG6 and MIG4 (averaging left and right states). The efficacy of MEG6 and MIG4 schemes to run simulations on dynamically deforming meshes is also illustrated using both inviscid and viscous test cases. The results show that employing conservative metric terms that are computed and interpolated consistently using the same scheme as that of the inviscid flux discretization will ensure freestream preservation on both stationary and moving grids. \\


% All the tests were noted to yield highly accurate results with reference to the exact/experimental solutions.
% The results show that by employing conservative metric terms computed and interpolated consistently using the same scheme as that of the inviscid flux discretization will ensure freestream preservation on both stationary and moving grids.

% It was noted that, the use of conservative metric formulation along with derivative and interpolation formulae corresponding to inviscid flux algorithm are to be used for freestream preservation.
% along with suitable demonstrative test cases with and without shocks.

The second part demonstrates the efficacy of MEG6 and MIG4 schemes in resolving supersonic jet screech. Both the schemes were noted to perform well in terms of resolving the essential flow features necessary to simulate the screech. The predicted screech frequencies and screech amplitudes of a round Mach $1.35$ under-expanded jet are noted to be in great agreement with the experimental measurements. Furthermore, aspects concerning the jet oscillation mode and screech amplitude variation along the azimuthal direction are elucidated. In the particular case studied, it was noted that the nearfield screech amplitude can differ by almost $15$dB (varies between $155$dB to $140$dB) depending on the azimuthal location of the observer. \\

The third part of the paper discusses the GPU acceleration model employed for single GPU computations. The benchmark speedup tests conducted on different GPUs show that the simulations can be run up to $\approx 200$ times faster on a single NVIDIA A100 GPU card compared to a single-core Intel Xeon Gold CPU. The achieved speedup statistics were noted be on par with some of the CUDA based implementations reported in the literature. In the context of the MEG6/MIG4 schemes, this study also investigates several aspects of GPU acceleration through the use of example test cases, including the `speedup efficiencies of different GPU hardware,' `compute loads of various functions,' `memory occupancy,' `the impact of cell-count on speedup,' and the `influence of working precision.' The supersonic jet noise simulation (13 million cells, 400,000 iterations, MEG6 scheme) presented in the paper was completed in a reasonable turnaround time of 34.5 hours on a single A100 GPU card.

The current approach however has a limitation in its inviscid scheme which only achieves second-order accuracy for non-linear test cases, as depicted in Ref. \cite{chamx}. Despite efforts to improve the MP limiter, it may still get activated frequently in the regions that doses not correspond to shocks. The ongoing work to address these issues will be reported in a separate publication. The readers should note that, despite its low-order nature, the present method outperforms a truly fifth-order scheme, as shown in Ref. \cite{chamx}.

