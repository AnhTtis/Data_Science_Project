\section{Review of conservative finite difference framework, and an overview of the algorithm} \label{sec:coner-FD+algo}

The current work employs MEG6, MIG4, and $\alpha$-damping schemes \cite{chamx}, which use a conservative finite difference approach to discretize the spatial derivative terms. Like most numerical schemes, these schemes are derived based on uniform stencils. Applying these schemes to non-uniform stencils (in curved/stretched grids) requires a recast of the Cartesian form of governing equations into generalized curvilinear coordinates form. The transformed equations are presented in Section \ref{sec:gov-eqns}. Such a coordinate transformation allows non-uniform skewed input cells of the mesh to be re-stretched into uniform cubical cells in the computational space, as shown in Fig. \ref{FD_grid}(a). Based on this idea, MEG6, MIG4, and $\alpha$-damping schemes are implemented in the solver to compute flow fields over non-uniform meshes.

\begin{figure}[h!]
    \centering
    \includegraphics[width=140mm]{Images/CFDM.pdf}
    \caption{(a) A 3-dimensional grid cell before and after coordinate transformation, (b) A two-dimensional $\xi-\eta$ plane of the computational domain depicting, cell centers, interface centers (located at the centroid of cell interface), and interface fluxes.}
    \label{FD_grid}
\end{figure}

The association between grid, solution, and conservative fluxes is as follows. The input mesh consists of hexahedral cells, identified by the coordinates of cell vertices provided in the grid file. A schematic of the two-dimensional solution domain in transformed coordinates is depicted in Fig. \ref{FD_grid}(b). The figure shows cell centers, interface centers (half locations), and fluxes. The solution $\textbf{Q}$ (conservative variables) is stored at the cell centers, which are located by the centroid of corresponding cell vertices. On the other hand, inviscid and diffusion fluxes ($\mathbf{F_{i+1/2}}, \mathbf{G_{i+1/2}}$ etc.) are computed at cell-interface centers to preserve conservation. The current paper also refers to these interface centers as half locations. After computing the inviscid and diffusion fluxes at all six half locations corresponding to the cell, the right-hand-side (RHS) residual corresponding to that cell denoted by $\mathbf{Res}_{i,j,k}$ is computed as follows:

\begin{equation}
 \frac{\partial}{\partial t}\left(\frac{\mathbf{Q}}{J}\right)=\mathbf{Res}_{i, j, k}=-\frac{\partial \left(\hat{\mathbf{F}} - \hat{\mathbf{F}}^{\mathrm{v}}\right)}{\partial \xi} -\frac{\partial \left(\hat{\mathbf{G}} - \hat{\mathbf{G}}^{\mathrm{v}}\right)}{\partial \eta} -\frac{\partial \left(\hat{\mathbf{H}} - \hat{\mathbf{H}}^{\mathrm{v}}\right)}{\partial \zeta}
\end{equation}

\begin{equation}
    \begin{aligned}
 \frac{\partial \left(\hat{\mathbf{F}} - \hat{\mathbf{F}}^{\mathrm{v}}\right)}{\partial \xi} = &\frac{1}{\Delta \xi}\left[\left(\hat{\mathbf{F}}_{i+\frac{1}{2}, j, k}-\hat{\mathbf{F}}_{i-\frac{1}{2}, j, k}\right)-\left(\hat{\mathbf{F}}_{i+\frac{1}{2}, j, k}^{\mathrm{v}}-\hat{\mathbf{F}}_{i-\frac{1}{2}, j, k}^{\mathrm{v}}\right)\right] \\
\frac{\partial \left(\hat{\mathbf{G}} - \hat{\mathbf{G}}^{\mathrm{v}}\right)}{\partial \eta} = &\frac{1}{\Delta \eta}\left[\left(\hat{\mathbf{G}}_{i, j+\frac{1}{2}, k}-\hat{\mathbf{G}}_{i, j-\frac{1}{2}, k}\right)-\left(\hat{\mathbf{G}}_{i, j+\frac{1}{2}, k}^{\mathrm{v}}-\hat{\mathbf{G}}_{i, j-\frac{1}{2}, k}^{\mathrm{v}}\right)\right] \\
\frac{\partial \left(\hat{\mathbf{H}} - \hat{\mathbf{H}}^{\mathrm{v}}\right)}{\partial \zeta} = &\frac{1}{\Delta \zeta}\left[\left(\hat{\mathbf{H}}_{i, j, k+\frac{1}{2}}-\hat{\mathbf{H}}_{i, j, k-\frac{1}{2}}\right)-\left(\hat{\mathbf{H}}_{i, j, k+\frac{1}{2}}^{\mathrm{v}}-\hat{\mathbf{H}}_{i, j, k-\frac{1}{2}}^{\mathrm{v}}\right)\right].
\end{aligned}
\end{equation}

The above formulation is conservative throughout the computational domain since the residuals are computed based on fluxes at cell interfaces (half-locations). Even though the fluxes are estimated at the cell interfaces, the process involved in computing these fluxes ($\hat{\mathbf{F}}, \hat{\mathbf{G}}, \hat{\mathbf{H}}, \hat{\mathbf{F}}^{\mathrm{v}}, \hat{\mathbf{G}}^{\mathrm{v}}, \hat{\mathbf{H}}^{\mathrm{v}}$) only requires the cell-center solution information and uses finite difference approximations for reconstruction. Hence the approach is classified as a conservative finite difference approach. 

Fig. \ref{algo} shows a flow chart view of key stages involved in the algorithm. The program starts with pre-processing steps and ends with post-processing. Before entering into the main time-loop, the necessary information required to execute the main time-loop is transferred to GPU memory, which will remain and update there without any communication with the CPU until the simulation ends at time $t_{end}$. This allows the computationally expensive parts of the program such as reconstruction, Riemann solvers, viscous fluxes, residuals, and time integration to be executed solely on GPUs. More details about the parallelization model employed and the achieved performance increase will be discussed in Section \ref{sec:gpu-accel}.\\


\begin{figure}[h!]
    \centering
    \includegraphics[width=130mm]{Images/algo.pdf}
    \caption{Flow chart of solver algorithm. The stages outlined in the green boxes are executed fully on the GPU.}
    \label{algo}
\end{figure}

One of the key features of the present Navier-Stokes algorithm is gradient sharing. The gradients of primitive variables are utilized in multiple stages of the algorithm, enhancing efficiency and solution accuracy. This includes the use of gradients in inviscid fluxes (via gradient-based reconstruction - Sec. \ref{Inv-disc}) and viscous flux discretization (via $\alpha$-damping approach - Sec. \ref{sec:viscDisc}), as well as the improvement of shock-capturing through the improved MP-limiter. Additionally, the gradients of primitive variables are used in post-processing calculations such as enstrophy, vorticity, density gradient magnitudes, Q-criterion, and others. As a result, the approach is efficient and produces relatively accurate and well resolved flow solutions.

