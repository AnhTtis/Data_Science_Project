\section{Metric terms} \label{conser-metrics}
\subsection{Metric terms at cell centers and half locations}

Since the current framework stores the solution at cell centers and fluxes at half locations, the solver requires metric terms at both these locations. The metrics belonging to cell centers are used in the following stages of the algorithm:
\begin{enumerate}
    \item To impose boundary conditions (Section \ref{sec:mul-block})
    \item While computing the time-step based on the inviscid and viscous CFL condition (Section \ref{sec:time-int})
    \item For computing cell-interface metrics (Section \ref{sec:FP})
    \item While computing post-processing quantities such as enstrophy, vorticity, etc.
\end{enumerate}

\noindent On the other hand, the metrics corresponding to half locations ($i+\frac{1}{2},j+\frac{1}{2},k+\frac{1}{2}$) are used during the following stages of the algorithm:
\begin{enumerate}
    \item While transforming primitive variables in to characteristic space (to be discussed in Section \ref{Inv-disc})
    \item The approximate Riemann solver (Section \ref{sec:riemann-solver})
    \item While computing viscous fluxes (Section \ref{sec:viscDisc})
\end{enumerate}

The half location metrics required in the above mentioned steps are interpolated from cell-centered metrics employing the same scheme as that of the reconstruction scheme used for inviscid flux computation (Section \ref{Inv-disc}) in order to satisfy the geometric conservation law. A detailed description of the interpolation formula and it's effect on the flow simulations will be discussed in Section \ref{sec:FP}. Firstly, the formulations employed to compute cell-centered metrics will be presented.

\subsection{Conservative metrics for cell centered grid}
The following conservative form of grid metrics given by Thomas and Lombard \cite{thomas1979geometric} were used in this work to satisfy the Geometric Conservation Law (GCL).

\begin{equation} \label{GCL_metrics}
\begin{aligned}
&\hat{\xi}_{x}=\left(y_{\eta} z\right)_{\zeta}-\left(y_{\zeta} z\right)_{\eta}, \quad
 \hat{\xi}_{y}=\left(z_{\eta} x\right)_{\zeta}-\left(z_{\zeta} x\right)_{\eta}, \quad
 \hat{\xi}_{z}=\left(x_{\eta} y\right)_{\zeta}-\left(x_{\zeta} y\right)_{\eta}, \\
&\hat{\eta}_{x}=\left(y_{\zeta} z\right)_{\xi}-\left(y_{\xi} z\right)_{\zeta}, \quad 
 \hat{\eta}_{y}=\left(z_{\zeta} x\right)_{\xi}-\left(z_{\xi} x\right)_{\zeta}, \quad
 \hat{\eta}_{z}=\left(x_{\zeta} y\right)_{\xi}-\left(x_{\xi} y\right)_{\zeta}, \\
&\hat{\zeta}_{x}=\left(y_{\xi} z\right)_{\eta}-\left(y_{\eta} z\right)_{\xi}, \quad
 \hat{\zeta}_{y}=\left(z_{\xi} x\right)_{\eta}-\left(z_{\eta} x\right)_{\xi}, \quad
 \hat{\zeta}_{z}=\left(x_{\xi} y\right)_{\eta}-\left(x_{\eta} y\right)_{\xi}.
\end{aligned}
\end{equation}

By satisfying GCL, metric cancellation errors corresponding to the grid transformation are reduced to machine zero, thus achieving freestream and vortex preservation properties. Ensuring freestream preservation is particularly important while studying flows involving features such as transition, aero-acoustic feedback mechanisms, etc., which are significantly affected by the non-preserved freestream errors.

\noindent While the metric terms in Eqn. (\ref{GCL_metrics}) are only computed once per simulation for stationary grid cases, they are re-computed after each time-step for dynamic meshes. The chain rule is used to evaluate the temporal metric terms as follows,
\begin{equation}
\xi_{t}=-\left  (x_{t}\xi_{x}  +y_{t}\xi_{y}  +z_{t}\xi_{z}  \right), \quad \eta_{t}=-\left (x_{t}\eta_{x} +y_{t}\eta_{y} +z_{t}\eta_{z} \right), \quad \zeta_{t}=-\left(x_{t}\zeta_{x}+y_{t}\zeta_{y}+z_{t}\zeta_{z}\right).
\end{equation}

The grid velocities $x_t$, $y_t$, and $z_t$ could be either predetermined analytically or can also be estimated numerically on the fly as the grid deforms and changes with time. In the current work the test case considered uses predetermined analytical values for $x_t$, $y_t$, and $z_t$. For more details on the use of numerically estimated grid velocities and the associated errors, the reader is referred to  Ref. \cite{Visbal2002}.\\

\noindent The time derivative term in Eqn. (\ref{trans-eqn}), is split as follows,
\begin{equation} \label{time-split}
    \frac{\partial }{\partial t}\left(\frac{\textbf{Q}}{J}\right) = \frac{1}{J} \frac{\partial \textbf{Q}}{\partial t} + \left(\frac{1}{J}\right)_{t} \textbf{Q}
\end{equation}

The first term $1/J$ on the right hand side is the instantaneous Jacobian computed based on instantaneous grid cell positions as follows. 

\begin{equation} \label{Jac-matrix}
    \frac{1}{J}=\left|\begin{array}{lll}
    x_{\xi} & y_{\xi} & z_{\xi} \\
    x_{\eta} & y_{\eta} & z_{\eta} \\
    x_{\zeta} & y_{\zeta} & z_{\zeta}
    \end{array}\right|
\end{equation}

The second term with $\left(\frac{1}{J}\right)_{t}$ is only relevant to the dynamically changing meshes and is computed in accordance with GCL as follows,

\begin{equation} \label{Jacb2}
    \left(\frac{1}{J}\right)_{t}=-\left[(\xi_{t})_{\xi}+(\eta_{t})_{\eta}+(\zeta_{t})_{\zeta}\right]
\end{equation}

%%%%%%%%%%%%%% PART-2 %%%%%%%%%%%%%%%
