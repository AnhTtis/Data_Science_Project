\section{Governing equations}   \label{sec:gov-eqns}
The unsteady three-dimensional compressible Navier-Stokes equations (dimensional) in a generalized curvilinear coordinate system with $\xi$, $\eta$, and $\zeta$ as the spatial coordinate directions and $t$ as the time can be written in the following vector form:
\begin{equation} \label{trans-eqn}
    \frac{\partial}{\partial t}\left(\frac{\mathbf{Q}}{J}\right)+\frac{\partial \mathbf{\hat{F}}}{\partial \xi}+\frac{\partial \mathbf{\hat{G}}}{\partial \eta}+\frac{\partial \mathbf{\hat{H}}}{\partial \zeta}=\frac{\partial \mathbf{\hat{F}_{v}}}{\partial \xi}+\frac{\partial \mathbf{\hat{G}_{v}}}{\partial \eta}+\frac{\partial \mathbf{\hat{H}_{v}}}{\partial \zeta},
\end{equation}\label{NS_TC}

\noindent where $\mathbf{Q}$ represents the vector of conservative variables, i.e. $\mathbf{Q}=[\rho, \rho u, \rho v, \rho w, E]^T$. The vectors of inviscid fluxes ($\mathbf{\hat{F}}$, $\mathbf{\hat{G}}$ and $\mathbf{\hat{H}}$) and viscous fluxes ($\mathbf{\hat{F}^{v}}$, $\mathbf{\hat{G}^{v}}$ and $\mathbf{\hat{H}^{v}}$) are,

\begin{subequations} \label{inv_fluxes}
    \begin{gather}
    \mathbf{\hat{F}}=\left[\begin{array}{c}
    \rho \hat{U} \\
    \rho u \hat{U}+\hat{\xi}_{x} p \\
    \rho v \hat{U}+\hat{\xi}_{y} p \\
    \rho w \hat{U}+\hat{\xi}_{z} p \\
    (E+p) \hat{U} -\hat{\xi}_{t} p
    \end{array}\right], \quad \mathbf{\hat{G}}=\left[\begin{array}{c}
    \rho \hat{V} \\
    \rho u \hat{V}+\hat{\eta}_{x} p \\
    \rho v \hat{V}+\hat{\eta}_{y} p \\
    \rho w \hat{V}+\hat{\eta}_{z} p \\
    (E+p) \hat{V} -\hat{\eta}_{t} p
    \end{array}\right], \quad \mathbf{\hat{H}}=\left[\begin{array}{c}
    \rho \hat{W} \\
    \rho u \hat{W}+\hat{\zeta}_{x} p \\
    \rho v \hat{W}+\hat{\zeta}_{y} p \\
    \rho w \hat{W}+\hat{\zeta}_{z} p \\
    (E+p) \hat{W} -\hat{\zeta}_{t} p
    \end{array}\right].
    \tag{\theequation a-\theequation c}
    \end{gather}
\end{subequations}

\begin{subequations}\label{visc-curvi}
    \begin{gather}
        \mathbf{\hat{F}^{v}}=\left[\begin{array}{c}
    0 \\
    \hat{\xi}_{x} \tau_{x x}+\hat{\xi}_{y} \tau_{x y}+\hat{\xi}_{z} \tau_{x z} \\
    \hat{\xi}_{x} \tau_{y x}+\hat{\xi}_{y} \tau_{y y}+\hat{\xi}_{z} \tau_{y z} \\
    \hat{\xi}_{x} \tau_{z x}+\hat{\xi}_{y} \tau_{z y}+\hat{\xi}_{z} \tau_{z z} \\
    \hat{\xi}_{x} \beta_{x}+\hat{\xi}_{y} \beta_{y}+\hat{\xi}_{z} \beta_{z}
    \end{array}\right], 
        \tag{\theequation a-\theequation b}
        \quad
         \mathbf{\hat{G}^{v}}=\left[\begin{array}{c}
        0 \\
        \hat{\eta}_{x} \tau_{x x}+\hat{\eta}_{y} \tau_{x y}+\hat{\eta}_{z} \tau_{x z} \\
        \hat{\eta}_{x} \tau_{y x}+\hat{\eta}_{y} \tau_{y y}+\hat{\eta}_{z} \tau_{y z} \\
        \hat{\eta}_{x} \tau_{z x}+\hat{\eta}_{y} \tau_{z y}+\hat{\eta}_{z} \tau_{z z} \\
        \hat{\eta}_{x} \beta_{x}+\hat{\eta}_{y} \beta_{y}+\hat{\eta}_{z} \beta_{z}
        \end{array}\right], \\
       \mathbf{\hat{H}^{v}}=\left[\begin{array}{c}
    0 \\
    \hat{\zeta}_{x} \tau_{x x}+\hat{\zeta}_{y} \tau_{x y}+\hat{\zeta}_{z} \tau_{x z} \\
    \hat{\zeta}_{x} \tau_{y x}+\hat{\zeta}_{y} \tau_{y y}+\hat{\zeta}_{z} \tau_{y z} \\
    \hat{\zeta}_{x} \tau_{z x}+\hat{\zeta}_{y} \tau_{z y}+\hat{\zeta}_{z} \tau_{z z} \\
    \hat{\zeta}_{x} \beta_{x} +\hat{\zeta}_{y} \beta_{y} +\hat{\zeta}_{z} \beta_{z}
    \end{array}\right]. \tag{\theequation c}
    \end{gather}
\end{subequations}

The quantity `$J$' here represents the Jacobian of grid transformation. The metric terms in the inviscid and viscous flux terms in Eqns.(\ref{inv_fluxes},\ref{visc-curvi}) with a hat $\hat{(\cdot)}$ denote Jacobian normalized quantities (e.g. $\hat{\xi}_x = \xi_x/J$).\\

\noindent The contravariant velocities $\hat{U}$, $\hat{V}$, and $\hat{W}$ are,

\begin{subequations}
\begin{gather}
    \hat{U}=\hat{\xi}_{t} + \hat{\xi}_{x} u+\hat{\xi}_{y} v+\hat{\xi}_{z} w, \quad
    \hat{V}=\hat{\eta}_{t} +\hat{\eta}_{x} u+\hat{\eta}_{y} v+\hat{\eta}_{z} w, \quad
    \hat{W}=\hat{\zeta}_{t} + \hat{\zeta}_{x} u+\hat{\zeta}_{y} v+\hat{\zeta}_{z} w \tag{\theequation a-\theequation c}
\end{gather}
\end{subequations}

\noindent The energy per unit volume $E$ is,
\begin{equation}
E = \frac{p}{\gamma-1} + \frac{1}{2} \rho (u^2 + v^2 + w^2)
\end{equation}

For stationary meshes, the computation of the terms $\hat{\xi}_{t}$, $\hat{\eta}_{t}$, and $\hat{\zeta}_{t}$, can be skipped as they are zero. The viscous and thermal stress related terms in the equations are defined as,
% \begin{equation*}
% \tau_{x x}=\hat{\lambda} \nabla \cdot \bar{u}+2 \hat{\mu} \frac{\partial u}{\partial x}, \quad \tau_{y y}=\hat{\lambda} \nabla \cdot \bar{u}+2 \hat{\mu} \frac{\partial v}{\partial y}, \quad \tau_{z z}=\hat{\lambda} \nabla \cdot \bar{u}+2 \hat{\mu} \frac{\partial w}{\partial z}
% \end{equation*}

\begin{subequations} \label{eqn:shear-str}
\begin{gather}
\tau_{x x}= \lambda \nabla \cdot \bar{u}+2 \mu \frac{\partial u}{\partial x}, \quad
\tau_{y y}= \lambda \nabla \cdot \bar{u}+2 \mu \frac{\partial v}{\partial y}, \quad
\tau_{z z}= \lambda \nabla \cdot \bar{u}+2 \mu \frac{\partial w}{\partial z},  \tag{\theequation a-\theequation c}\\
\tau_{x y}=\tau_{y x}=\mu\left(\frac{\partial u}{\partial y}+\frac{\partial v}{\partial x}\right), \quad
\tau_{x z}=\tau_{z x}=\mu\left(\frac{\partial u}{\partial z}+\frac{\partial w}{\partial x}\right), \quad 
\tau_{y z}=\tau_{z y}=\mu\left(\frac{\partial v}{\partial z}+\frac{\partial w}{\partial y}\right) \tag{\theequation d-\theequation f}
\end{gather}
\end{subequations}

\begin{subequations} \label{eqn:thermal-work}
    \begin{gather}
    \beta_{x}=u \tau_{x x}+v \tau_{x y}+w \tau_{x z}+ \kappa_t \frac{\partial T}{\partial x}, \quad
    \beta_{y}=u \tau_{y x}+v \tau_{y y}+w \tau_{y z}+ \kappa_t \frac{\partial T}{\partial y}, \quad
    \beta_{z}=u \tau_{z x}+v \tau_{z y}+w \tau_{z z}+ \kappa_t \frac{\partial T}{\partial z} \tag{\theequation a-\theequation c}
    \end{gather}
\end{subequations}

\noindent Where $\bar{u}$ in the normal stress terms denotes the velocity vector. The thermodynamic equation of state $p=\rho RT$ is used to close the governing equations. $\mu$ and $\kappa_t$ denote the fluid's dynamic viscosity and thermal conductivity, respectively, which are temperature-dependent properties. The dynamic viscosity is computed using Sutherland's law \cite{Sutherland1893} while thermal conductivity is calculated using the Prandtl number ($Pr$) and dynamic viscosity. Additionally, bulk viscosity $\lambda=-\frac{2}{3}\mu$ is incorporated into the normal stress terms $\tau_{xx}$, $\tau_{yy}$, and $\tau_{zz}$ based on Stokes' hypothesis. The values of $\gamma$ and $Pr$ are set to 1.4 and 0.71, respectively, for air as the working fluid. Although the governing equations are presented in their dimensional form here, they are solved in their non-dimensional form to facilitate problem parameterization and minimize computational errors. The ambient speed of sound, density, temperature, viscosity, and nozzle exit diameter are used as reference quantities for scaling the flow variables.

% \noindent The equation of state $p= \rho RT$ is used to close the system of equations. The physical quantities: density, velocity, pressure, temperature, dynamic viscosity, length scales and time scales are non dimensionalized with the corresponding reference quantities: $\rho_{\infty}$, $U_{\infty}$, $\rho_{\infty} U_{\infty}^2$, $T_{\infty}$, $\mu_{\infty}$, $L_{ref}$ and $L_{ref}U_{\infty}^{-1}$ respectively. As a result, the Reynolds number $Re=\rho_{\infty} u_{\infty} L_{ref} \mu_{\infty}^{-1}$ and Mach number $M=u_{\infty}(\gamma R_{gas} T_{\infty})^{-1/2}$ appear as parameters in the scaled diffusion coefficients: $\hat{\mu} = \frac{\mu}{\mathrm{Re}}$, $\hat{\lambda} = -\frac{2}{3} \hat{\mu}$ and $\hat{\kappa} = \hat{\mu}(\mathrm{M}^2(\gamma-1)\mathrm{Pr})^{-1}$ in the above equations. $Pr$ stands for the Prandtl number and is taken to be $0.71$ in the present work.

% The simulations presented in the paper utilize Newton's law of viscosity and assumes dynamic viscosity ($\mu$) to vary with temperature according to Sutherland's law \cite{Sutherland1893}. The bulk viscosity $\lambda$ is added in to the normal stress terms $\tau_{xx}$, $\tau_{yy}$ and $\tau_{zz}$ in accordance with Stokes' hypothesis.\\
