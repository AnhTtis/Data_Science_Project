
\section{Freestream preservation} \label{sec:FP}

In this section, firstly we layout the three essential steps to be followed to ensure freestream preservation for MEG6 and MIG4 schemes. Then we present a proof showing how the metric identities are satisfied with the use of MEG6 and MIG4 schemes to discretize and interpolate the conservative metric term formulae (Eqn. \ref{GCL_metrics}). The following are the three steps to be incorporated while computing metric terms in order to achieve freestream preservation:

\begin{enumerate}
    \item \textbf{Conservative metrics:} Conservative metrics presented in Eqn. \ref{GCL_metrics} should be used to ensure geometric and volume conservation \cite{thomas1979geometric}.
    
    \item \textbf{Consistent discretization using a central scheme}:  While computing the cell-centered metric terms, the same scheme as that of the inviscid flux discretization ($\frac{\partial F}{\partial \xi}$) has to be used to compute derivative terms involved in the conservative metric formulae. Furthermore, in order to estimate the metric terms in a linear fashion, the limiting steps (steps 4 and 5 in Fig. \ref{inv_algo}) are skipped and the left and right biased interpolated states should be averaged. 
    
    For instance, to compute the $(x_{\xi})_i$ term (which is required to compute metric terms), firstly, the values of $x$ at $i+\frac{1}{2}$ and $i-\frac{1}{2}$ should be computed by averaging the left and right biased interpolated values obtained using the formulation described in Eqn. \ref{vanleer-poly}. Then, using Eqn. \ref{FP-interp2} the derivative at location $i$ is computed.
    
    \begin{subequations} \label{FP-interp}
    \begin{gather}
    \begin{aligned}
         x_{i+\frac{1}{2}} &= 0.5\left(x_{i+\frac{1}{2}}^L + x_{i+\frac{1}{2}}^R\right) = 0.5 \left[\left(x_{i} + \frac{1}{2} x'_{i}+ \frac{1}{12} x''_{i} \right) + \left(x_{i+1} - \frac{1}{2} x'_{i+1}+ \frac{1}{12} x''_{i+1} \right) \right], \\
        x_{i-\frac{1}{2}} &= 0.5\left(x_{i-\frac{1}{2}}^L + x_{i-\frac{1}{2}}^R\right) = 0.5 \left[\left(x_{i-1} + \frac{1}{2} x'_{i-1}+ \frac{1}{12} x''_{i-1} \right) + \left(x_{i} - \frac{1}{2} x'_{i}+ \frac{1}{12} x''_{i} \right) \right] 
    \end{aligned}
    \tag{\theequation a-\theequation b}
    \end{gather}
    \end{subequations}
    
    \begin{equation} \label{FP-interp2}
        (x_{\xi})_i = \frac{x_{i+\frac{1}{2}} - x_{i-\frac{1}{2}}}{\Delta \xi}
    \end{equation}
    
    In the above equations, the $x'$ and $x''$ terms should be computed using the formulations described in Eqns. \ref{E6_grads} or \ref{IG4_grads} depending on whether the scheme being employed is MEG6 or MIG4. The procedure leads to linear interpolation and, consequently, a linear estimate of all the metric terms. It should be noted that though $x'$ and $x_{\xi}$ have mathematically the same meaning, they are different in the numerical sense.
    
    \item \textbf{Consistent interpolation using a central scheme}: The same formulation used in interpolating the left and right characteristic variable states in the inviscid flux algorithm (Eqns. \ref{legendre_polys}) should be used to interpolate metric terms from cell centers to cell interfaces. Furthermore, while interpolating metric terms of the form $\frac{\xi_x}{J}$ (which are used in the Riemann solver), the quantity should be interpolated as a whole rather than interpolating $\xi_x$ and $\frac{1}{J}$ separately and then multiplying them later. To make the interpolation central, the left and right states should be averaged. For instance, the following expression should be used to interpolate $\frac{\xi_x}{J}$ to the $i+\frac{1}{2}$ interface.
    
    % (used in Riemann solver and viscous flux computations)
    
    \begin{equation}    \label{interp-formulae}
    \begin{aligned}
        \left(\frac{\xi_x}{J}\right)_{i+\frac{1}{2}} &= 0.5 \left\{\left(\frac{\xi_x}{J}\right)_{i+\frac{1}{2}}^L + \left(\frac{\xi_x}{J}\right)_{i+\frac{1}{2}}^R \right\}\\
        &= 0.5 \left\{\left[\left(\frac{\xi_x}{J}\right)_{i} + \frac{1}{2} \left(\frac{\xi_x}{J}\right)'_{i}+ \frac{1}{12} \left(\frac{\xi_x}{J}\right)''_{i} \right] + \left[\left(\frac{\xi_x}{J}\right)_{i+1} - \frac{1}{2} \left(\frac{\xi_x}{J}\right)'_{i+1}+ \frac{1}{12} \left(\frac{\xi_x}{J}\right)''_{i+1} \right] \right\}
    \end{aligned}
    \end{equation}
    
\end{enumerate}

Now that all the steps are detailed, next, we present a proof showing how freestream preservation is ensured by following these steps. To ensure freestream preservation, the interpolated metrics should satisfy the three metric identities listed in Eqn. \ref{metric-Idty} in the discrete sense \cite{thomas1979geometric,nonomura2010freestream}. These metric identities can be obtained by simplifying the momentum equations by considering the freestream flow assumption (i.e., constant $[\rho,u,v,w,p]$ values throughout the domain). For the sake of brevity, the proof is only presented for the metrics corresponding to MEG6. Nevertheless, following the essential steps, the same result can also be obtained even for metrics that are computed with the MIG4 scheme.
    
\begin{subequations}  \label{metric-Idty}
\begin{gather}
    \frac{\partial}{\partial \xi}\left(\hat{\xi_x}\right)+\frac{\partial}{\partial \eta}\left(\hat{\eta_x}\right)+\frac{\partial}{\partial \zeta}\left(\hat{\zeta_x}\right) =0, \\
    \frac{\partial}{\partial \xi}\left(\hat{\xi_y}\right)+\frac{\partial}{\partial \eta}\left(\hat{\eta_y}\right)+\frac{\partial}{\partial \zeta}\left(\hat{\zeta_y}\right) =0, \\
    \frac{\partial}{\partial \xi}\left(\hat{\xi_z}\right)+\frac{\partial}{\partial \eta}\left(\hat{\eta_z}\right)+\frac{\partial}{\partial \zeta}\left(\hat{\zeta_z}\right) =0
\end{gather}
\end{subequations}



\noindent \textbf{Statement:} The three metric identities (Eqn. \ref{metric-Idty}) will be satisfied in discrete sense if the conservative metrics are discretized and interpolated based on MEG6 or MIG4 schemes (as described in steps 2-3 above).

\noindent \textbf{Proof:} The metric terms $\hat{\xi}_{x}$, $\hat{\eta}_{x}$, and $\hat{\zeta}_{x}$ presented in Eqn. \ref{metric-Idty}a can be analytically expressed as,

\begin{equation}
\hat{\xi}_x=\left(y_\eta z\right)_\zeta-\left(y_\zeta z\right)_\eta, \quad \hat{\eta}_x=\left(y_\zeta z\right)_{\xi}-\left(y_{\xi} z\right)_\zeta, \quad \hat{\zeta}_x=\left(y_{\xi} z\right)_\eta-\left(y_\eta z\right)_{\xi}
\end{equation}

\noindent We proceed to discretize $\hat{\xi}_{x}$, $\hat{\eta}_{x}$ and $\hat{\zeta}_{x}$ at $i,j,k$ locations. Firstly the terms $y_\xi z$, $y_\eta z$, and $y_\zeta z$ are computed using the discretization scheme employed for the MEG6 scheme as follows:

\begin{equation}
    \begin{split}
    \left(y_\xi z\right)_{i,j,k} &= \frac{1}{\Delta \xi}\left(y_{i+\frac{1}{2},j,k} + y_{i-\frac{1}{2},j,k}\right) z_{i,j,k} \\
    &= \frac{1}{2 \Delta \xi}\left(y_{i+\frac{1}{2},j,k}^L+y_{i+\frac{1}{2},j,k}^R\right) z_{i,j,k} + \frac{1}{2 \Delta \xi}\left(y_{i-\frac{1}{2},j,k}^L+y_{i-\frac{1}{2},j,k}^R\right) z_{i,j,k} \\
    &= \frac{2282}{2880\Delta \xi}(y_{i + 1, j, k} - y_{i - 1, j, k})z_{i, j, k} + \frac{-546}{2880\Delta \xi}(y_{i + 2, j, k} - y_{i - 2, j, k})z_{i, j, k} \\
    &+ \frac{89}{2880\Delta \xi}(y_{i + 3, j, k} - y_{i - 3, j, k})z_{i, j, k} +\frac{-3}{2880\Delta \xi}(y_{i + 4, j, k} - y_{i - 4, j, k})z_{i, j, k} \\
    &+ \frac{-1}{2880\Delta \xi}(y_{i + 5, j, k} - y_{i - 5, j, k})z_{i, j, k}.
    \end{split}
\end{equation}

\noindent Similar expressions can be derived for $y_\eta z$ and $y_\zeta z$. Next we compute the terms $\left(y_\eta z\right)_\zeta$ and $\left(y_\zeta z\right)_\eta$ as follows.

\begin{equation}    \label{proof1}
    \begin{split}
    \left[\left(y_\eta z\right)_\zeta\right]_{i,j,k} &= \frac{1}{\Delta \zeta} \left[ \left(y_\eta z\right)_{i,j,k+\frac{1}{2}}-\left(y_\eta z\right)_{i,j,k-\frac{1}{2}} \right] \\
    &= \frac{1}{2 \Delta \zeta}\left[\left(y_\eta z\right)_{i,j,k+\frac{1}{2}}^L+\left(y_\eta z\right)_{i,j,k+\frac{1}{2}}^R\right] + \frac{1}{2 \Delta \zeta}\left[\left(y_\eta z\right)_{i,j,k-\frac{1}{2}}^L+\left(y_\eta z\right)_{i,j,k-\frac{1}{2}}^R\right] \\
    &= \frac{2282}{2880\Delta \zeta}\left[\left(y_\eta z\right)_{i , j, k+ 1} - \left(y_\eta z\right)_{i, j, k- 1}\right] + \frac{-546}{2880\Delta \zeta}\left[\left(y_\eta z\right)_{i, j, k + 2} - \left(y_\eta z\right)_{i, j, k - 2}\right] \\
    & + \frac{89}{2880\Delta \zeta}\left[\left(y_\eta z\right)_{i, j, k + 3} - \left(y_\eta z\right)_{i, j, k - 3}\right] + \frac{-3}{2880\Delta \zeta}\left[\left(y_\eta z\right)_{i, j, k + 4} - \left(y_\eta z\right)_{i, j, k - 4}\right] \\
    & + \frac{-1}{2880\Delta \zeta}\left[\left(y_\eta z\right)_{i, j, k + 5} - \left(y_\eta z\right)_{i, j, k - 5}\right].
    \end{split}
\end{equation}

\noindent After computing the term $\left[(y_{\zeta}z)_{\eta}\right]_{i,j,k}$ in a similar fashion, it can be subtracted from $\left[(y_\eta z)_\zeta\right]_{i,j,k}$ to obtain the metric term $\left(\xi_x\right)_{i,j,k}$. Similarly, other metric terms $\left(\eta_x\right)_{i,j,k}$ and $\left(\zeta_x\right)_{i,j,k}$ can also be computed. The next step is to interpolate $\xi_x$, $\eta_x$, and $\zeta_x$ terms from $(i,j,k)$ locations to the half locations and evaluate Eqn. \ref{metric-Idty}a. For instance, the first term of Eqn. \ref{metric-Idty}a is computed as follows. It should be noted that the grid spacing in the computational coordinates is uniform and taken equal to one unit, i.e., $\Delta \xi = \Delta \eta = \Delta \zeta = 1$.

\begin{equation}
    \begin{aligned}
    \frac{\partial}{\partial \xi}\left(\hat{\xi}_x\right) = \frac{(\xi_x)_{i+\frac{1}{2},j,k}-(\xi_x)_{i-\frac{1}{2},j,k}}{\Delta \xi} &= \frac{1}{2 \Delta \xi}\left[(\xi_x)_{i+\frac{1}{2},j,k}^L+(\xi_x)_{i+\frac{1}{2},j,k}^R\right] - \frac{1}{2 \Delta \xi}\left[(\xi_x)_{i-\frac{1}{2},j,k}^L+(\xi_x)_{i-\frac{1}{2},j,k}^R\right] \\
    &= \frac{1}{23887872000}[-y_{i-5,j-5,k-5}z_{i-5,j-5,k} -3y_{i-5,j-5,k-4}z_{i-5,j-5,k} \\
    & +\quad \cdots \quad  + 3y_{i+5,j+5,k+4}z_{i+5,j+5,k} + y_{i+5,j+5,k+5}z_{i+5,j+5,k}]
    \end{aligned}
\end{equation}

\noindent Similarly, the expressions $\frac{(\eta_x)_{i,j+\frac{1}{2},k}-(\eta_x)_{i,j-\frac{1}{2},k}}{\Delta \eta}$ and $\frac{(\zeta_x)_{i,j,k+\frac{1}{2}}-(\zeta_x)_{i,j,k-\frac{1}{2}}}{\Delta \zeta}$ can also be discretized. Adding up all the expressions will result in a sum equal to zero. The zero-sum proves that the first metric identity is satisfied in a discrete sense. The other two metric identities Eqns. \ref{metric-Idty}b-c can also be proved similarly. Due to the large number of terms involved, full length arithmetic operations are not showed here. A detailed, worked-out Mathematica notebook file has been provided in the supplementary section (Suppl. \ref{sec:supply}1) to support the proof. It was also found that the use of any other interpolation scheme for the present discretization approach (MEG4/MIG4), other than the one mentioned in step 3 above, will not satisfy the metric identities. 
\begin{equation}
     \frac{(\xi_x)_{i+\frac{1}{2},j,k}-(\xi_x)_{i-\frac{1}{2},j,k}}{\Delta \xi} +   \frac{(\eta_x)_{i,j+\frac{1}{2},k}-(\eta_x)_{i,j-\frac{1}{2},k}}{\Delta \eta} +  \frac{(\zeta_x)_{i,j,k+\frac{1}{2}}-(\zeta_x)_{i,j,k-\frac{1}{2}}}{\Delta \zeta} = 0
\end{equation}

\section{Results} \label{sec:results}
The results are organized into three parts. The first part (Sec. \ref{sec:FP-stationary}) presents demonstrative examples to show the freestream preserving nature of the MEG6/MIG4 schemes. The second part (Sec. \ref{sec:dynamic}) explores the efficacy of MEG6 and MIG4 schemes to simulate flows through dynamically deforming grids. Finally, in the third part (Sec. \ref{sec:JNapplication}), the resolving capability of MEG6 and MIG4 in capturing screech tones and unsteady aspects of an under-expanded supersonic jet \cite{ponton1997near} are shown employing coarse grid LES. The test cases used in each of the three parts are enumerated below.

\begin{enumerate}
    \item \textbf{Freestream preservation:} Three-dimensional uniform flow, convecting vortex, and Double Mach reflection.
    \item \textbf{Dynamically deforming meshes:} Three-dimensional uniform flow, stationary vortex, and double periodic shear layer.
    \item \textbf{Application to simulate supersonic jet noise:} Mach 1.35 choked under-expanded supersonic jet.
    % \item \textbf{Experimental validation:} Schardin's test and Mach 1.35 choked under-expanded supersonic jet (sections \ref{sec:schardin} and \ref{sec:JNapplication}).
\end{enumerate}

\subsection{Freestream preservation: Demonstration} \label{sec:FP-stationary}

The test cases in this section were chosen to cover two-dimensional, three-dimensional, subsonic, and supersonic flow scenarios. To discern the effect of using a non-freestream preserving approach for computing metric terms, tests were also performed using a non-conservative metric term approach presented in \ref{app:non-FP}. The non-freestream preserving approach employs a non-conservative form of metrics and explicit fourth order central scheme for first and second derivative evaluation instead of Eqns. \ref{E6_grads} or \ref{IG4_grads}. The non-freestream preserving metrics are referred to with the names \textit{Non-Freestream Preserving MEG6} (Non-FP MEG6) and \textit{Non-Freestream Preserving MEG6} (Non-FP MIG4) when MEG6 and MIG4 schemes were used to compute the inviscid fluxes, respectively.

\subsubsection{Three-dimensional uniform flow} \label{3D-UF}

A fluid stream with uniform density, velocity, and pressure flowing through a periodic domain is simulated to test freestream preservation of MEG6 and MIG4 schemes in this test case. The effect of mesh non-uniformity on the solution is assessed by employing freestream preserving and non-freestream preserving metric terms. Two non-uniform meshes are employed in this test case; the first is a wavy sinusoidal domain, and the second is a randomized grid (Fig. \ref{freestream_grid}). The formulation used to generate the sinusoidal grid \cite{jiang2014free} is as follows:

\begin{equation}
   \begin{aligned}
    x_{i, j, k}=x_{\min }+&\Delta x_0\left[(i-1)+ \sin (\pi(j-1) \Delta y_0) \sin(\pi(k-1) \Delta z_0) \right], \\
    y_{i, j, k}=y_{\min }+&\Delta y_0\left[(j-1)+ \sin (\pi(k-1) \Delta z_0) \sin(\pi(i-1) \Delta x_0) \right], \\
    z_{i, j, k}=z_{\min }+&\Delta z_0\left[(k-1)+ \sin (\pi(i-1) \Delta x_0) \sin(\pi(j-1) \Delta y_0) \right]
    \end{aligned}
\end{equation}

where,

\begin{equation} \label{usual-pars}
    \begin{aligned}
        i=1,2, \cdots, NI, \quad & j=1,2, \cdots, NJ, \quad & k=1,2, \cdots, NK, \\
        \Delta x_0=\frac{L_x}{NI}, \quad & \Delta y_0=\frac{L_y}{NJ}, \quad & \Delta z_0=\frac{L_z}{NK}, \\
        x_{\min }=-\frac{L_x}{2}, \quad & y_{\min }=-\frac{L_y}{2}, \quad & z_{\min }=-\frac{L_z}{2}
    \end{aligned}
\end{equation}

The variables $NI$, $NJ$, and $NK$ represent the total number of cells along the $\xi$, $\eta$, and $\zeta$ directions, respectively; all of them are taken as $20$, thus making the grid resolution $20 \times 20 \times 20$. $L_x$, $L_y$, and $L_z$ are the length parameters taken as four units each. On the other hand, the randomized grid shown in Fig. \ref{freestream_grid}b is generated by randomly moving the uniformly distributed grid node positions by an amount of $20$ percent of the local grid spacing in each direction.

\begin{figure}[h!]
    \centering
    \includegraphics[width=90mm]{Images/freestream_grid.pdf}
    \caption{(a) Sinusoidal grid and (b) Randomized grid used to test freestream preservation nature of MEG6 and MIG4 schemes.}
    \label{freestream_grid}
\end{figure}

The initial conditions of the flow are: $(\rho, u, v, w, p)=(1,1,0,0,\frac{1}{\gamma M^2})$. The value of $\gamma$ is taken to as $1.4$. Inviscid calculations were performed with a Mach number of $0.5$. The initial conditions essentially dictate that the $y$ and $z$ velocities should remain zero everywhere in the domain at the end of the simulation. However, if the scheme is not freestream preserving, the errors associated with the metric cancellation will add up and grow. Consequently, they will be seen as non-zero values in the $y$ and $z$ direction velocities. 

The flow is simulated until a non-dimensional time of 10 units. Then, the $L^2$ norm errors associated with the $y$ and $z$ velocities are computed for both freestream and non-freestream preserving metric approaches. All the $L^2$ errors recorded in the simulations are tabulated in table \ref{freestream_table}. The errors associated with non-freestream preserving schemes presented in the first and second rows of the table can be seen to be significantly high. The errors are in the order of $10^{-3}$ and $10^{-1}$ in magnitude on sinusoidal and randomized grids, respectively. On the other hand, the errors associated with freestream preserving metrics are close to machine zero (of double-precision computations) on both sinusoidal and randomized grids; these errors can be treated as zeros. This clearly demonstrates that the current approach is freestream preserving concerning both MEG6 and MIG4 schemes. Experiments were also performed with conservative metrics computed using the fourth-order scheme (Eqn. \ref{4thorder-grads}), which also failed to preserve the freestream values, as expected.

\begin{table}[h!]
    \centering
    \includegraphics[width=110mm]{Images/freestream_table.pdf}
    \caption{$L^2$ norm errors of $y$ and $z$ component velocity in the freestream preservation test on sinusoidal and randomized grids employing non-conservative and conservative metric term approaches.}
    \label{freestream_table}
\end{table}

\subsubsection{Two-dimensional moving vortex}

\begin{figure}[h!]
    \centering
    \includegraphics[width=135mm]{Images/freestream_COVO.pdf}
    \caption{(a) $64\times64$ 2D grid used for the simulation, (b) Plots of swirl velocity along $y=8$ and, (c,d,e,f) Plots of z-vorticity ($\Omega_z$) at $t=16$ using conservative and non-conservative metrics for MEG6 and MIG4 schemes.}
    \label{freestream_COVO}
\end{figure}

In this test case, an isentropic inviscid vortex is advected through a distorted randomized grid of size $64 \times 64$ to test for the freestream and vortex preservation nature of the MEG6 and MIG4 schemes. The test is performed on a periodic domain spanning over square dimensions of $[0,16]\times[0,16]$. With the advection velocity chosen as one unit and the simulation time $16$ non-dimensional time units, the vortex is allowed to travel one full cycle to reach back to its initialized location. The grid randomization is only done in the central portion of the domain spanning over the region given by ${x \in [3,13] \cup y\in [3,13]}$ (Fig. \ref{freestream_COVO}a). Such an abrupt mesh non-uniformity (also an abrupt change in metric term values) is created intentionally to check the robustness of the schemes. The level of grid randomization in the current simulation is also considered twenty percent, similar to the previous case. The mesh resulting non-uniformities created a maximum included angle of about $160$ degrees in the domain. The initial conditions are given by:

\begin{equation}
\begin{aligned}
& \rho = 1 \\
& u =1-\frac{C\left(y-y_{c}\right)}{U_{\infty} R} \exp \left(-r^{2} / 2\right) \\
& v =\frac{C\left(x-x_{c}\right)}{U_{\infty} R^{2}} \exp \left(-r^{2} / 2\right) \\
& p =1-\gamma M^2 \frac{C^{2}}{U_{\infty}^2 R^{2}} \exp \left(-r^{2}\right) \\
& r^{2} =\left(x-x_{c}\right)^{2}+\left(y-y_{c}\right)^{2}
\end{aligned}
\end{equation}

where $(x_c,y_c) = (8,8)$ are the coordinates of vortex core center. $R$ denotes the vortex core radius. The value of non-dimensional vortex core strength $\frac{C}{U_{\infty}R}$ is taken as $0.02$. The freestream Mach number $M$ is set to $0.1$.

Fig. \ref{freestream_COVO}b shows the swirl velocity profiles (y-direction velocity) along $y=8$ computed using freestream preserving and non-freestream preserving metric schemes in comparison to the exact solution. The blue triangles and the black circles can be noted to follow close to the exact solution, unlike the solution corresponding to the non freestream scheme. A similar conclusion can be drawn from the vorticity contours presented in Fig \ref{freestream_COVO}c-d. While MEG6 and MIG4 conservative approaches preserve the vortex shape and its surrounding field, the solution computed using non-conservative compact fourth-order schemes yields an unphysical solution. This test has successfully demonstrated the vortex-preserving nature of the MEG6 and MIG4 conservative metric approaches.

\subsubsection{Double Mach reflection} \label{DMR-case}

% This popular test case is typically used to test shock capturing schemes and the ability of the scheme to resolve shear layers.

Through this test case, the freestream preservation property of the MEG6 and MIG4 schemes is further accessed in the presence of shock waves and shear layers in the solution field. The problem comprises a planar shock wave moving towards a compression ramp. The ramp is inclined at 30 degrees relative to the shock. The moving shock wave interacts with the wall to form two triple points, two reflected shocks, two Mach stems, and a slip-stream. In order to simplify the computational domain and grid, the planar shock wave is inclined while keeping the ramp wall horizontal; this consequently makes the domain rectangular. Simulations are performed on a domain size of $[0,3]\times[0,1]$ at a resolution of $768\times256$. The grid is randomized in the central portion of the domain by randomly moving the grid nodes in both $x$ and $y$ directions by a maximum distance of fifty percent of the uniform grid spacing. The maximum included angle of the resulting mesh in the entire domain is close to $200$ degrees, making the grid highly skewed and non-uniform. A closeup view of the randomized grid is shown in Fig. \ref{freestream_DMR}a. The initial conditions for this test case are:

\begin{equation}
    (\rho, u, v, p)= \begin{cases}\left(8,8.25 \cos 30^{0},-8.25 \sin 30^{0}, 116.5\right), & x<1 / 6+\frac{y}{\tan 60^{0}} \\ (1.4,0,0,1), & x>1 / 6+\frac{y}{\tan 60^{0}}\end{cases}
\end{equation}

\begin{figure}[h!]
    \centering
    \includegraphics[width=135mm]{Images/freestream_DMR.pdf}
    \caption{(a) $768\times256$ 2D grid used for the simulation, (b,c,d,e) Plots of density contours using conservative and non-conservative metrics for MEG6 and MIG4 schemes. Blow-up view of the shear layer corresponding to (d,e) conservative metrics approach are also shown at the bottom of the image. (f) Close-up view of the shear layer region, corresponding to the results obtained from the WENO-Z simulation employing uniform grid.}
    \label{freestream_DMR}
\end{figure}

% include max skewness and aspect ratio, max included angle

The boundary conditions are as follows. Post-shock flow conditions are set at the left boundary, and zero-gradient conditions are imposed on the right boundary. At the bottom boundary, reflecting boundary conditions are used from $x = 1/6$ to $x = 3.0$ and post-shock conditions for rest of the region i.e., $x \in [0, 1/6]$. A time-varying boundary condition is imposed on the top with pre- and post-shock conditions to the left and right of the shock, respectively, to accommodate the shock motion. The flow is simulated until an end time of $t=0.2$. 

Fig. \ref{freestream_DMR}b-d shows density contours at $t=0.2$ obtained using MEG6 and MIG4 schemes with freestream-preserving and non-freestream preserving metrics. The leftover errors corresponding to the non-freestream preserving schemes can be seen in the central region of the solution as rough patches in Figs. \ref{freestream_DMR}b-c. Interestingly the errors corresponding to MIG4 are minimal and significantly less compared to the MEG6 Non-FP approach (Fig. \ref{freestream_DMR}c). This might be attributed to the accuracy of implicit gradient values used in estimating the metric terms of the MIG4-based derivatives and interpolations. On the other hand, with conservative metric terms evaluated using MEG6 and MIG4 schemes, the solution in the central region can be seen unaffected by the grid distortion. Furthermore, the original solution is noted to be preserved even in the presence of shocks. In addition, the bottom right of the figure illustrates the outcome of a WENO-Z simulation for comparison with the current schemes. The simulation was conducted utilizing a uniform grid at the same resolution. The variations in the shear layer characteristics between the current schemes and the WENO-Z scheme demonstrate the enhanced resolving capabilities of the current schemes. This concludes the discussion on demonstrating the freestream and vortex-preserving nature of the MEG6 and MIG4 schemes. The use of these schemes on dynamically deforming meshes is explored next.\\

% %%%%%%%%%%%%%%%%%%%%%%%%%%%%%%%%%%%%%%%%%%%%%%%%%%%%%%%%%%%%%%%%
%    -------------------- DYNAMIC MESHES --------------------
% %%%%%%%%%%%%%%%%%%%%%%%%%%%%%%%%%%%%%%%%%%%%%%%%%%%%%%%%%%%%%%%%

\subsection{Application to dynamically deforming meshes} \label{sec:dynamic}
In this section, we examine the usage of MEG6 and MIG4 schemes to simulate flows through dynamically deforming mesh environments using three canonical test cases, namely `3-D uniform flow', `stationary vortex,' and `doubly periodic shear layer.' The time derivative is split into two parts using the chain rule described in Eqn. \ref{time-split}. To ensure the geometric conservation law (GCL) which is a necessary condition for freestream preservation, the term $\left(\frac{1}{J}\right)_t$ is computed using Eqn. \ref{Jacb2}. The mesh velocities are computed analytically for all the test cases presented in this section. The corresponding expressions are presented while describing the case.

% The use of high order conservative schemes to simulate flows through dynamically deforming meshes especially in the context of compressible flows is quite limited in the literature.

\subsubsection{Three-dimensional uniform flow on deforming sinusoidal grid} \label{3D-moving-FP}
Firstly, we start with the trivial case of three-dimensional uniform flow. The flow is simulated in a dynamically moving mesh environment with periodic boundary conditions on all sides. The initial and boundary conditions for this test case are the same as that of the conditions described in section \ref{3D-UF}. Inviscid calculations are performed using both freestream and non-freestream preserving metric approaches. The time-dependent location of the grid nodes for the simulation is given by: 

\begin{equation}
    \begin{aligned}
        x_{i, j, k}(t)=& x_{\min }+\Delta x_o\left[i+A_x \sin (2 \pi \omega t) \sin \frac{n_{x y} \pi j \Delta y_o}{L_y} \sin \frac{n_{x z} \pi k \Delta z_o}{L_z}\right] \\
        y_{i, j, k}(t)=& y_{\min }+\Delta y_o\left[j+A_y \sin (2 \pi \omega t) \sin \frac{n_{y x} \pi i \Delta x_o}{L_x} \sin \frac{n_{y z} \pi k \Delta z_o}{L_z}\right] \\
        z_{i, j, k}(t)=& z_{\min }+\Delta z_o\left[k+A_z \sin (2 \pi \omega t) \sin \frac{n_{z x} \pi i \Delta x_o}{L_x} \sin \frac{n_{z y} \pi j \Delta y_o}{L_y}\right]
    \end{aligned}
\end{equation}  

Taking the analytical derivative of the above expressions, the following grid time-dependent deformation velocity expressions are obtained:

\begin{equation}
    \begin{aligned}
        x_{i, j, k}^{'}(t)=&  2 \pi \omega \Delta x_o A_x \cos (2 \pi \omega t) \sin \frac{n_{x y} \pi j \Delta y_o}{L_y} \sin \frac{n_{x z} \pi k \Delta z_o}{L_z}, \\
        y_{i, j, k}^{'}(t)=&  2 \pi \omega \Delta y_o A_y \cos (2 \pi \omega t) \sin \frac{n_{y x} \pi i \Delta x_o}{L_x} \sin \frac{n_{y z} \pi k \Delta z_o}{L_z}, \\
        z_{i, j, k}^{'}(t)=&  2 \pi \omega \Delta z_o A_z \cos (2 \pi \omega t) \sin \frac{n_{z x} \pi i \Delta x_o}{L_x} \sin \frac{n_{z y} \pi j \Delta y_o}{L_y},
    \end{aligned}
\end{equation}

where, $i$, $j$, $k$, $\Delta x_o$, $\Delta y_o$, $\Delta z_o$, $x_{\min}$, $y_{\min}$, $z_{\min}$ assume their usual meanings described in Eqn. \ref{usual-pars}. The grid resolution for the simulation is set to $32^3$. The values of the rest of the parameters are: $A_x=A_y=A_z=1.5$, $L_x=L_y=L_z=12$, and $n_{xy}=n_{yz}= \cdots = 4$. The grid deformation oscillation frequency $\omega$ is set to $1$. As the mesh undergoes deformation, the maximum included angle in the domain reaches a maximum value of $\approx 160$.

\begin{figure}[h!]
    \centering
    \includegraphics[width=150mm]{Images/dynamic1.pdf}
    \caption{Snapshots of $20\times20$ three dimensional deforming sinusoidal grid with y-velocity contours on the surface using (a) non conservative (non-FP) metrics and (b) conservative (FP) metrics for MEG6 and MIG4 schemes. (c) $L^2$ norm error in y and z velocities using different different metric schemes.}
    \label{dynamic1}
\end{figure}

Contours of $y$-velocity and the corresponding grid at $t=5.25$ are shown in Fig. \ref{dynamic1}. The contours clearly indicate that the non-conservative metrics (non-FP) approach feeds error into the solution while the conservative metrics (FP) approach does not. Fig. \ref{dynamic1}c shows the $L^2$ norm errors computed using MEG6 and MIG4 schemes. The errors corresponding to the conservative metric formulation can be noted to stay close to machine zero, while the non-conservative approach does not. This clearly demonstrates that the present schemes satisfy the geometric conservation law and are thus freestream preserving even on dynamically deforming meshes.

\subsubsection{Stationary vortex on a three-dimensional deforming mesh}
In this test case, the vortex preservation nature of MEG6 and MIG4 schemes is shown in a moving grid environment. A stationary columnar vortex is initialized at the center of a three-dimensional, cubical domain of dimensions $[-8,8]\times[-8,8]\times[-8,8]$. In each direction, thirty-two cells were used. The grid is set to deform in time according to the same equations presented in the previous test case on \textit{three-dimensional uniform flow} (section \ref{3D-moving-FP}). The oscillation frequency of the grid is set to $\omega=1$.

\begin{figure}[h!]
    \centering
    \includegraphics[width=140mm]{Images/dynamic2.pdf}
    \caption{(a,b) Instantaneous snapshots of vorticity contours collected on the surface corresponding cells on the 16th grid index in the $\zeta$-direction. (c) Swirl velocity plot along NZ/2 and y = 8 at $t=8$.}
    \label{dynamic2}
\end{figure}

Inviscid calculations were performed until an end-time of $t_{end}=8$ using MEG6 and MIG4 schemes. The grid undergoes eight deformation cycles before the end time. Fig. \ref{dynamic2}a-b shows the grid and $z$-vorticity contours computed using the MIG4 scheme (with conservative metrics approach) at two different time instances. The vortex shape can be noted to stay intact and unaffected by the grid motion throughout. Fig. \ref{dynamic2}c shows the computed and exact swirl velocity profiles along the mid-horizontal line. Both MEG6 and MIG4 approaches can be noted to be in excellent agreement with the theoretical solution.

\subsubsection{Double periodic shear-layer on dynamic mesh}
In this test case, the freestream preserving nature of the MEG6 and MIG4 schemes in the presence of viscous stresses are tested. The case consists of two shear layers initially parallel to each other that evolve to produce two large vortices at $t=1$. Viscous forces are dominant near the shear layer in this test case. The non-dimensional parameters concerning the flow are, $\mathrm{Re} = 1 \times 10^4$ and $\mathrm{M} = 0.1$. The value of $\gamma$ is taken as $5/3$. The initial conditions for the flow are:

\begin{subequations}
    \begin{align}
        \rho &= \frac{1}{\gamma \mathrm{M}^2}, \\
        u &= 
        \begin{cases}
            \tanh (80 \times(y-0.25)), & \text{ if } (y \leq 0.5), \\
            \tanh (80 \times(0.75-y)), & \text{ if } (y > 0.5),
        \end{cases} \\
        v &= 0.05 \times \sin (2 \pi(x+0.25)) \\
        T &= 1.
    \end{align}
\end{subequations}

Simulations are performed on a two-dimensional deforming grid of resolution $360 \times 360$ using MEG6 and MIG4 schemes with the freestream preserving metrics. The spatial positions of mesh nodes and their respective velocities in time are given by Eqn. \ref{DPSL-eqn1} and Eqn. \ref{DPSL-eqn2} respectively, as described in Ref-\cite{achu2021entropically}. 

\begin{equation} \label{DPSL-eqn1}
    \begin{aligned}
x_{i, j}(\tau)=x_{\min }+\Delta x_0&\left[(i-1)+A_x \sin (2 \pi \omega \tau) \sin \left(\frac{n_x \pi(j-1) \Delta y_0}{L_y}+\frac{i \phi_x}{I L-1}\right)\right] \\
y_{i, j}(\tau)=y_{\min }+\Delta y_0&\left[(j-1)+A_y \sin (2 \pi \omega \tau) \sin \left(\frac{n_y \pi(i-1) \Delta x_0}{L_x}+\frac{i \phi_y}{J L-1}\right)\right]
\end{aligned}
\end{equation}

\begin{equation} \label{DPSL-eqn2}
    \begin{aligned}
&x_\tau=2 \pi \omega \Delta x_0 A_x \cos (2 \pi \omega \tau) \sin \left(\frac{n_x \pi(j-1) \Delta y_0}{L_y}+\frac{i \phi_x}{I L-1}\right) \\
&y_\tau=2 \pi \omega \Delta y_0 A_y \cos (2 \pi \omega \tau) \sin \left(\frac{n_y \pi(i-1) \Delta x_0}{L_x}+\frac{i \phi_y}{J L-1}\right)
\end{aligned}
\end{equation}

The length parameters $L_x$ and $L_y$ are considered $1$ unit each. The mesh oscillation frequency $\omega$ is set to 1, enabling the mesh to deform through one full cycle during the simulation ($t_{end}=1$).

\begin{equation} \label{KE-Ens}
    \begin{gathered}
E_k=\frac{1}{L_x L_y} \int_0^{L y} \int_0^{L x} \rho \frac{u^2+v^2}{2} \mathrm{~d} x \mathrm{~d} y \\
\mathcal{E}=\frac{1}{L_x L_y} \int_0^{L y} \int_0^{L x} \rho \frac{\boldsymbol{\Omega} \cdot \boldsymbol{\Omega}}{2} \mathrm{~d} x \mathrm{~d} y
\end{gathered}
\end{equation}

\begin{figure}[h!]
    \centering
    \includegraphics[width=150mm]{Images/dynamic3.pdf}
    \caption{(a,b,c,d) Snapshots of instantaneous vorticity contours collected at four different time instances with every 10th point of the grid overlain on the solution computed using MIG4 scheme. Evolution of (e) Kinetic energy (KE) and (f) Enstrophy computed using MEG6 and MIG4 schemes compared against the incompressible solution of Clausen \cite{clausen2013entropically}.}
    \label{dynamic3}
\end{figure}

Fig. \ref{dynamic3} shows the $z$-vorticity field at four equally spaced time instances during the simulation. Two large vortices of opposing angular velocity can be seen to evolve as time progresses. The dynamic mesh motion can be seen to not show any effect on the flow contours (for comparison against the stationary grid DNS solution, please refer to \cite{sainadh2022spectral}). The evolution of Kinetic Energy (KE) and Enstrophy ($\mathcal{E}$) in the entire computational domain is tracked in time for both MEG6 and MIG4 schemes. The expressions to compute these quantities are provided in Eqn. \ref{KE-Ens}. Evolution plots of these quantities are shown in Figs. \ref{dynamic3}e-f in comparison to the solutions corresponding to a stationary grid simulation and the incompressible flow solution of Clausen \cite{clausen2013entropically}, both simulated on a uniform grid with a resolution of on $512 \times 512$. An excellent agreement between the reference solutions and the current calculations can be noted in both KE and Enstrophy plots. This demonstrates that a Navier-Stokes solution computed using the higher accuracy schemes, MEG6 and MIG4, will resist the errors associated with the dynamic grid motion. 

% \subsection{Schardin's test} \label{sec:schardin}

% Schardin's test consists of an unsteady shock wave moving at a speed of Mach 1.3 that interacts with a two-dimensional prism. Due to shock diffraction and reflections, several intricate flow features such as slip-lines, contact discontinuities, acoustic waves, and triple points arise in this problem. The multi-block approach and the shock-capturing strategy of MEG6 and MIG4, as implemented in the current Navier-Stokes solver, are tested through this test case. Fig. \ref{schardin1}a shows domain dimensions and the initial set-up of the problem. First, the shock is initialized at the nose of the prism. The shock then moves towards the right to interact with the inclined surface of the prism, eventually diffracting at the trailing edge. A block-structured body-fitted grid comprising seven blocks with approximately $0.3$ million control volumes is employed for this simulation. The results are compared against the experiments of Schardin \cite{schardin1957high} and fine grid WENO-5 \cite{jiang1996efficient} simulations of Chaudhuri et al. \cite{chaudhuri2011use}. For further details and a complete description of this case, the reader is referred to Ref-\cite{chang2000shock}.

% \begin{figure}[h!]
%     \centering
%         \includegraphics[width=110mm]{Images/schardin1.pdf}
%     \caption{(a) Geometry, problem set-up and initial conditions for the Schardin's test case, (b) Density gradient contours at non-dimensional time $=2.752$ and its comparison with the experimental schlieren data taken from \cite{schardin1957high}.}
%     \label{schardin1}
%     %  (one originated from the top surface of wedge and one from the rightside region.)
% \end{figure}

% \begin{figure}[h!]
%     \centering
%         \includegraphics[width=120mm]{Images/schardin2.pdf}
%     \caption{(a) Trajectories of the two triple points on the upper half of the geometry. (b) Zoomed in view of flow-field close to the prism is shown on the right with a comparison between MEG6 and MIG4.}
%     \label{schardin2}
% \end{figure}

% Fig. \ref{schardin1}b shows the density gradients of the solution at $t = 2.752$ in comparison with the experiments. Several important features visible in the experimental schlieren, such as triple points, Mach stems, and vortices, can be noted in the MIG4 solution reproduced with good fidelity. Fig. \ref{schardin2}a shows the trajectories of two triple points that originate from the upper half of the domain (one from the top surface of the prism and the other from the rear). A good agreement between the present solution computed using MIG4 and the reference solutions of WENO-5 simulations and experiments can be observed. Fig. \ref{schardin2}b shows a comparison between MEG6 and MIG4 solutions. Although both the solutions look almost alike, the vortices at the trailing edge can be noted to have been slightly better resolved in the MIG4 solution. This ensures that the present solver can resolve several transient and intricate flow features with sufficient fidelity on a multi-block curvilinear grid. 

% In the next section, the use of MEG6 and MIG4 schemes to predict screech tones corresponding to a supersonic jet is presented, along with a discussion on the flow physics and grid requirements.
% \newpage

\subsection{Mach 1.35 under-expanded supersonic round jet} \label{sec:JNapplication}
Shock containing supersonic jets emit intense aeroacoustic noise at distinct frequencies due to the phenomenon termed `screech'. The nearfield noise levels due to screech in a free single jet configuration typically range between 120dB-180dB. Although the physical mechanism of screech tone production is still not completely understood, it is a well-established fact that it occurs as the consequence of aeroacoustic resonance \cite{edgington2019aeroacoustic,edgington2021generation}. According to the classical feedback loop theory of Powell \cite{powell1953mechanism,raman1999supersonic}, firstly, a series of small flow disturbances are initiated at the nozzle lip region. These disturbances travel downstream (in the form of Kelvin-Helmholtz instabilities) towards the shock cells to interact with the oblique shocks. As the disturbances interact with the shocks (generally after the second or third shock cells), they undergo strong amplification and radiate intense acoustic waves, which drive the entire jet plume to resonate and consequently oscillate. Some of the produced acoustic waves travel upstream towards the nozzle lip through the atmosphere outside the jet to close the feedback loop. The screech is sustained with the continuous excitation of the shear layer (receptivity process) from these emission waves. Therefore, since the screech depends on several flow features, such as shock cells, shear layer, vortical structures, and acoustic waves, it is crucial to capture all these features with sufficient fidelity in order to simulate its effect through computations. The main objective of this section is to test the efficacy of MEG6 and MIG4 schemes in resolving `screech' and `unsteady jet oscillation mode' of an under-expanded supersonic jet. Simulations are also performed using WENO-Z \cite{borges2008improved} scheme for comparison against the present scheme.

\begin{figure}[h!]
    \centering
    \includegraphics[width=160mm]{Images/jet-domain-mesh.pdf}
    \caption{(a) Computational domain used, (b) Butterfly mesh topology adapted to avoid singularity at the axis, visualized on $x=2D$ plane.}
    \label{jet-domain-mesh}
\end{figure}

An under-expanded supersonic jet ejecting from a converging, choked nozzle operating at a Nozzle Pressure Ratio (NPR) of $2.97$ is considered as the test case. The flow simulated in the present computations is compared with the nearfield microphone data of Ponton et al. \cite{ponton1997near}. The experimental setup used in their work consists of a settling chamber attached to an axisymmetric converging nozzle with an exit diameter (D) equal to one inch. The lip thickness of the nozzle exit is $0.6D$. The flow diagnostics include a microphone placed on the nozzle exit plane at a radial location of two nozzle diameters away from the nozzle axis. Fig. \ref{jet-domain-mesh}a shows the dimensions of the computational domain used to simulate the flow. The domain is axisymmetric and consists of four blocks. The jet Reynolds number based on the nozzle diameter and the jet exit conditions is $Re_D=1.144 \times 10^6$. The flow is non-dimensionalized based on the ambient atmospheric conditions ($T_{\infty}=293\text{ K}$, $p_{\infty} = 101325\text{ Pa}$). The nozzle exit conditions are theoretically computed using Eqns. \ref{ini-cons}. The computed inlet conditions are directly provided as an inlet boundary condition on the nozzle exit plane. The effect of the boundary layer and turbulent fluctuations developed inside the annular region are not considered. Although this does not replicate the experimental conditions accurately, the effect of annular fluctuations in the present case was believed to show little effect on the screech feedback loop based on the study by \cite{ahn2021numerical,ahn2018supersonic}. To prevent spurious reflections from entering in to the computational domain, sponge zones are employed at the far-field boundaries in the regions marked in Fig. \ref{jet-domain-mesh}. These zones are implemented by incorporating a dissipative source term into the governing equations as outlined in the methodology of Bodony \cite{bodony2006analysis}.

\begin{equation} \label{ini-cons}
    \begin{aligned}
    &\frac{p_e}{p_{\infty}}=\frac{1}{\gamma}\left[\frac{2+(\gamma-1) M_j^2}{\gamma+1}\right]^{\frac{\gamma}{\gamma-1}} \\
    &\frac{\rho_e}{\rho_{\infty}}=\frac{\gamma(\gamma+1) p_e}{2 (T_n/T_{\infty})} \\
    &\frac{u_e}{a_{\infty}}=\sqrt{\frac{2 (T_n/T_{\infty})}{\gamma+1}}, \quad v_e = w_e = 0
\end{aligned}
\end{equation}

In this test case, three grids were utilized with resolutions of $10$, $13$, and $20$ million cells to examine the independence of the results from the choice of grid. To prevent a singularity at the nozzle axis, the butterfly topology was employed, as illustrated in Fig. \ref{jet-domain-mesh}b. The grid was clustered towards the axis and exit of the nozzle, where the key flow features are located. Along the axial direction of the geometry, a minimum grid spacing of $\Delta x = 0.0085D$, $0.0075D$, and $0.0065D$ was maintained for the $10$, $13$, and $20$ million grids, respectively. The grid spacing was linearly increased along the $x$-direction, with the maximum grid spacing not exceeding $0.1D$ in all three grids. The grid was uniformly distributed along the azimuthal direction, with 80, 100, and 120 divisions for the 10, 13, and 20 million grids, respectively. The grid distribution along the radial direction was consistent across all three meshes used in the study. Along the radial direction, a minimum spacing of $0.0075D$ was used close to the axis, with a growth rate of $\approx 4\%$, resulting in a total of 120 cells. Computations were performed with a $\Delta t a_{\infty}/D = 2\times10^{-3}$ until an end-time of $t a_{\infty}/D = 800$ ($400,000$ iterations). All the statistics were collected after the flow reached a statistically steady state, which is after $t a_{\infty}/D = 400$. All the simulations were run in parallel on a single Nvidia A100 GPU. The simulations corresponding to 13 million grid were completed in a span of $34.0$ and $41.7$ hours using MEG6 and MIG4 schemes, respectively. More details about the GPU acceleration will be discussed in the next section.


Given the high Reynolds number in the present case, Large Eddy Simulations (LES) were conducted on course grids. In conventional Large Eddy Simulation (LES) approaches, the effect of unresolved sub-grid-scale (SGS) flow features is modeled through an explicit SGS model while the larger scales are resolved by the grid \cite{germano1991dynamic}. However, in the present simulations, we solve the unfiltered Navier-Stokes equations directly and utilize the inherent numerical dissipation of the MEG6/MIG4 schemes to implicitly mimic the effects of SGS models. This approach is commonly referred to as ``implicit-LES'' or, specifically in the present case, ``Monotonically integrated Implicit LES (MILES)'' \cite{fureby1999monotonically} because of the use of the Monotonocity Preserving limiter in the inviscid flux discretization and its high spectral resolution. A similar LES strategy was also employed in the study performed by Ahn et al. \cite{ahn2021numerical} on twin-jet configurations.

\subsubsection{Instantaneous and average flow features}
Fig. \ref{jet-inst-ave}a-b shows the time-averaged axial density and $x$-velocity profiles on three different grid resolutions. The peaks and troughs in the profiles suggest the presence of shock cells. The solution has converged sufficiently at a grid resolution of $13$ million cells, especially in the region corresponding to the first four shock cells. It can be noted that the solution corresponding to the 10 million grid fails to capture the peaks and troughs of the density and velocity profiles between the range of $x/D=0$ to $4$, which is critical for the screech phenomenon. Thus, the results corresponding to $10$ million grid are not considered for the analysis. Fig. \ref{jet-inst-ave}c-d shows the instantaneous and mean contours of density and Mach number. Various flow features such as shocks, expansion fans, shear-layer, vortices, and induced density fluctuations can be noted in the pictures. The first two shock cells appear clearly defined, but progressively downstream the structure of shock cells become less distinct due to the growth of instability waves and turbulent mixing. The non-dimensional shock cell spacing in the jet is a crucial length scale in supersonic jets as it dictates the location of effective screech noise source \cite{powell1953mechanism}. The first shock cell spacing denoted by $\lambda_1$, can be calculated theoretically using the formulation proposed by Pack \cite{pack1950note} through Eqn. \ref{firstShockSpace} approximately.

\begin{equation}\label{firstShockSpace}
    \lambda_1=2.695 \sqrt{\left(\mathrm{NPR}^{0.291}-1.205\right)}
\end{equation}

Plugging in NPR=$2.97$ in to the equation, results in $\lambda_1 \approx 1.1$. On the other hand in the present simulations, as can be seen in the time-averaged density and velocity plots shown in Fig. \ref{jet-inst-ave}(a,b), the first shock cell spacing was obtained to be $\lambda_1 = 1.15$. A good agreement is observed between the theoretical calculations and the numerical results. Subsequently the shock cell spacing downstream was noted to slowly decrease with  $\lambda_2=1.15$, $\lambda_3=1.0$, $\lambda_4=1.0$, and $\lambda_5=0.9$. In Fig. \ref{jet-inst-ave}(c,d) the shear layer instabilities initiated near the nozzle lip can be seen to grow progressively in scale as they move downstream to interact with the shock cells. Consequently, local acoustic disturbances are produced, which travel upstream to close the feedback loop causing screech, as explained before. To better visualize this phenomenon, movies of density field are provided in the supplementary materials section \ref{sec:supply}. From the time-resolved data shown in Movie-1 of Sec. \ref{sec:supply}, the evolution of Kelvin-Helmholtz instabilities, the interaction of shear layer vortices with the shock cells, emitted acoustic radiation, and the consequent jet oscillation behavior can be remarked.

%  The movies are particularly saved at very high temporal resolution (XX Hz)) that is generally difficult to achieve on the modern high speed cameras.
% The average shock cell spacing is noted to be close to $1.2D$.

A supersonic jet has multiple noise sources \cite{tam1995supersonic,bailly2016high,edgington2019aeroacoustic}. The above-mentioned acoustic resonance, shock shear layer interaction, and the jet mixing are part of the generation mechanisms behind different the various noise components. The tones corresponding to screech noise are predominant and can be characterized in the upstream region of the jet \cite{gojon2019antisymmetric,powell1953mechanism,davies1962tones,westley1969near}. Fig. \ref{jet-spl}(a) presents a snapshot of the instantaneous 3-D jet surface visualized through density iso-surface, along with the pressure fluctuation field surrounding it. The figure illustrates the progressive amplification of three-dimensional hydrodynamic disturbances on the jet surface close to the nozzle exit. In Fig. \ref{jet-spl}(b), the instantaneous pressure fluctuation contours and the mean Sound Pressure Levels (SPL) are displayed on the $z=0$ plane. The pressure fluctuation field immediately close to the jet surface appears choppy and seemingly random; however, a clear sinusoidal-type pattern can be observed in the upstream region of the jet just above the nozzle, as indicated by the dashed white box in Fig. \ref{jet-spl}. Given the pronounced sinusoidal nature of the wave pattern and the prevalence of screech in the upstream location, it can be inferred that these fluctuations correspond to screech. The lower half of Fig. \ref{jet-spl} displays isolines of SPL, with SPL near the lip of the nozzle observed to be close to $150$dB. The solutions presented in Fig. \ref{jet-inst-ave} correspond to the MIG4 scheme; however, similar results were also observed with the MEG6 scheme.

\begin{figure}[h!]
    \centering
    \includegraphics[width=150mm]{Images/jet-inst-ave.pdf}
    \caption{Time averaged axial profiles of density (a) and x-velocity (b) at various grid resolutions. The length of first shock cell spacing $\lambda_1$ is also indicated in the figures (a) and (b)}. Instantaneous and mean density contours (c) and local Mach number (d) contours of $M_{j}=1.35$ jet solution plotted on $z/D = 0$ plane employing a grid resolution of 20M. All the plots shown here are based on MIG4 scheme.
    \label{jet-inst-ave}
\end{figure}


\begin{figure}[h!]
    \centering
    \includegraphics[width=\textwidth]{Images/jet-spl.pdf}
    \caption{(a) Visualization of the jet surface using an iso-surface of density corresponding to $\rho/\rho_{\infty}=1.2$. The figure also illustrates the circular pressure ripples surrounding the jet planes $y=0$ and $x=-2D$. (b) Instantaneous pressure fluctuations (upper half) and overall sound pressure levels (lower half) are shown.}
    \label{jet-spl}
\end{figure}

\subsubsection{Comparison of acoustic data with experiments} \label{freqs}

\begin{figure}[h!]
    \centering
    \includegraphics[width=\textwidth]{Images/jet-freqs.pdf}
    \caption{Comparison of pressure spectra of probe data collected at various azimuthal angles on the nozzle exit plane $2D$ away from the nozzle axis, using (a,b) MEG6, (c,d) MIG4, and (e,f) WENO-Z scheme, with the experimental data of Ponton et al. \cite{ponton1997near}. The figure also includes (g) a plot illustrating the variation of the amplitude of the fundamental tone with azimuthal angle, using data from twenty probes and the various schemes.}
    \label{jet-freqs}
\end{figure}

To evaluate the screech tones and amplitudes captured in the flowfield (Fig. \ref{jet-spl}), a Fourier analysis of the nearfield pressure data is conducted and compared to the experiments of Ponton et al. \cite{ponton1997near}. Given the non-axisymmetric nature of screech pressure fluctuations at the present jet Mach number as previously reported in studies \cite{ahn2021numerical,gojon2019antisymmetric}, time-series data is gathered at multiple azimuthal locations on the jet exit plane. A total of 20 equally spaced pressure probes (computational) are placed at two nozzle diameters from the nozzle axis on the jet exit plane, as shown in Fig. \ref{jet-freqs}e. In contrast, the experiments of Ponton et al. \cite{ponton1997near} only utilized one microphone placed at an arbitrary azimuthal location two diameters away from the nozzle axis on the jet exit plane. The Fourier analysis of the time-series data collected at various azimuthal locations is then compared to the pressure spectra obtained in the experiments \cite{ponton1997near}.

Fig. \ref{jet-freqs}(a-d) presents pressure spectra of probe data using the MEG6, MIG4, and WENO-Z schemes on a 13 million cell grid at various azimuthal locations. The accuracy of capturing the fundamental screech frequency and the first harmonic at all azimuthal locations is observed in computations performed on both 13 and 20 million cell grids. As an illustration, four sample pressure spectra are displayed at different azimuthal angles, with the peak of the fundamental tone indicated by a dashed circle. The Strouhal number remains constant across all azimuthal angles, as summarized in Table \ref{jet-table} for the fundamental screech tone at different grid resolutions using the MEG6 and MIG4 schemes, showing a good agreement with experimental results. In addition to the screech, the plots also reveal the presence of the first harmonic at around St$\approx 0.62$ and mixing noise component at St$\approx 0.2$. However, the WENO-Z scheme in Fig. \ref{jet-freqs}(e,f) exhibited a fundamental tone at St$=0.43$ which is significantly off from the experimental value. The inaccuracy can be attributed to the relatively poor spectral properties of the scheme.

Despite the constant Strouhal number across various azimuthal probe locations, variations in amplitude can be observed in Fig. \ref{jet-freqs}a-d. The screech amplitudes recorded at different azimuthal angles ($\theta$) using the MEG6, MIG4, and WENO-Z schemes are presented in Fig. \ref{jet-freqs}g. The amplitude of the WENO-Z scheme maintains a consistent level, while the MEG6 and MIG4 schemes exhibit a fluctuating wavy pattern with a difference in amplitude between peaks and troughs greater than 15dB. The physical reasoning behind this observation is due to the flapping mode of the jet which will be discussed next in Sec. \ref{sec:jet-oscill}. Table \ref{jet-table} presents the azimuthal average of screech amplitudes for different grid resolutions using the MEG6 and MIG4 schemes, and a good agreement with experimental values is observed. This supports the effectiveness of both MEG6 and MIG4 schemes in resolving supersonic jet screech `frequencies' and `amplitudes'.

% Given the axisymmetric nature of the geometry, this may seem counterintuitive. But, given the existence of a flapping plane and non-uniform distribution of pressure fluctuations along the azimuthal direction, the amplitude variation of screech tones along the azimuthal direction is expected.

\begin{table}[h!]
    \centering
    \includegraphics[width=120mm]{Images/jet-table.pdf}
    \caption{Summary of Strouhal numbers and average amplitudes of fundamental screech tones obtained at various grid resolutions employing MEG6, MIG4 and WENO-Z (only at 13M cell count) schemes. The screech amplitude corresponding to experiments presented in the table does not correspond to azimuthal average since only one microphone is employed in the experiments.}
    \label{jet-table}
\end{table}

\subsubsection{Unsteady jet oscillation behaviour} \label{sec:jet-oscill}

% In order to understand the jet oscillation behavior, pressure contours are visualized on a Y-Z plane situated two diameters upstream of the nozzle exit ($x=-2D$). The lateral oscillations normal to the $x$-axis can be observed on this plane. Contours of instantaneous pressure fluctuations on the plane are shown in Fig. \ref{jet-precess}. The contours show that the pressure fluctuations are spread out along a particular orientation in a polarized fashion. Such a distribution suggests that the jet is undergoing a flapping motion. Furthermore it can be noted that the pressure fluctuation magnitude is out of phase along the flapping indicated with a green dashed line. This particular oscillation mode where the jet primarily undergoes the flapping along a preferred plane of oscillation can be categorized into flapping `mode-B' according to the literature \cite{powell1992observations,umeda2001sound}.

% This particular oscillation mode where the jet is undergoing simultaneous flapping and precession can be categorized as flapping `mode-D' according to the literature \cite{powell1992observations,umeda2001sound}.

% \begin{figure}[h!]
%     \centering
%     \includegraphics[width=150mm]{Images/jet-precess.pdf}
%     \caption{\textcolor{magenta}{(a) Instantaneous pressure fluctuation contours on the $x=-2D$ and $y=0$ planes, overlaid with an iso-surface of density corresponding to $\frac{\rho}{\rho_{\infty}}=1.2$. (b) Contours of instantaneous pressure fluctuations on the $x=-2D$ plane, indicating the approximate orientation of the flapping plane.}}
%     \label{jet-precess}
% \end{figure}

\begin{figure}[h!]
    \centering
    \includegraphics[width=\textwidth]{Images/jet-line-fft.pdf}
    \caption{Pressure spectra (a,c) and phase variation of fundamental frequency $St=0.308$ (b,d) along a circular line two diameters away from the nozzle axis on the $x=-2D$ plane using MEG6 and WENO-Z schemes. The location of the fundamental frequency is marked with a red arrow on the top in figures (a) and (c). The phase shift of $\pi$ in figure (b) illustrates the flapping nature of the jet as captured by the MEG6 scheme and in the experiments of Ponton et al. \cite{ponton1997near}. Black arrows in figure (a) indicate the azimuthal location at which this phase shift occurs.}
    \label{new1}
\end{figure}

Previously in Sec \ref{freqs}, the amplitude variation depicted in Fig. \ref{jet-freqs} indicated the presence of azimuthally asymmetric jet oscillations. However, the specific oscillation mode cannot be determined from those plots alone as they correspond to a data from single point. Previous research by Powell et al. \cite{powell1992observations} has determined that the jet should exhibit flapping B mode oscillations at the current Nozzle Pressure Ratio of 2.97. In order to accurately identify the jet oscillation mode in the present simulations, Fourier analysis was conducted on two-dimensional nearfield pressure data on plane $x=-2D$, which is a jet normal plane located two diameters upstream of the nozzle exit. The results from the MEG6 and WENO-Z methods are compared to each other along with experimental observations \cite{ponton1997near,powell1992observations}.

In Fig. \ref{new1}(a,c), the frequency and amplitude of pressure data collected along a circular line located two diameters away from the nozzle axis on the $x=-2D$ plane are presented, utilizing the MEG6 and WENO-Z schemes. The ordinate of the plot represents the azimuthal angle $\theta$ along the circle. The data reveals the presence of fundamental tones (marked with red arrows on top) at Strouhal numbers of $0.308$ and $0.432$, which are consistent with the observations made in the Fourier analysis presented in Sec. \ref{freqs}. The spectrum for the WENO-Z scheme in Fig. \ref{new1}(c) appears fairly uniform at all azimuthal angles, whereas the MEG6 results in Fig. \ref{new1}(a) exhibit two notches separated by an angular distance of $\pi$ radians which are indicated by two black arrows. This suggests the presence of a plane of symmetry on the $x=-2D$ surface. The phase angle (the argument of the Fourier transformation) at the fundamental tone for the MEG6 and WENO-Z schemes is plotted in Fig. \ref{new1}(b,c). The MEG6 results show a step-like distribution with a step height equal to $\pi$ radians, which is indicative of sinuous flapping behavior of the jet, where pressure fluctuations are separated by a phase difference of $\pi$ radians about a plane of symmetry. In contrast, the WENO-Z results shown in Fig. \ref{new1}(d) exhibited a constant phase value at all azimuthal locations, indicating an axisymmetric or toroidal oscillation mode (or A1 mode) of the jet, which ideally leads to axisymmetric pressure fluctuations. To better understand these aspects noted from the 1D line data, the Fourier amplitude and phase fields of 2D data from the plane at $x = -2D$ are presented next.

\begin{figure}[h!]
    \centering
    \includegraphics[width=\textwidth]{Images/jet-amp-phase.pdf}
    \caption{Fourier amplitude fields (a,c) and Fourier phase fields (b,d), corresponding to the fundamental frequency $St=0.308$ on $x=-2D$ plane using MEG6 and WENO-Z schemes. The contour distributions depicted in these figures indicate the presence of the flapping B-mode and A1-toroidal modes, captured by the MEG6 and WENO-Z schemes, respectively.}
    \label{new2}
\end{figure}

Fig. \ref{new2} illustrates the Fourier amplitudes and phase values computed on the $x=-2D$ plane at the fundamental tone frequency, as determined by the MEG6 and WENO-Z schemes. It can be observed that the results obtained using the WENO-Z scheme exhibit a clear axisymmetric distribution of both amplitude and phase, which is not consistent with the flapping behavior observed in experiments \cite{powell1992observations}. In contrast, the amplitude levels for the MEG6 scheme, as shown in Fig. \ref{new2}(a), displays a plane of symmetry, about which the jet flapping occurs. Additionally, a spiraling phase distribution, as seen in Fig. \ref{new2}(b) for the MEG6 scheme, is a clear indication of the sinuous flapping mode. This conclusively confirms the efficacy of the MEG6 scheme in resolving the Flapping B mode observed by Powell et al. \cite{powell1992observations} in their experiments.


\subsubsection{Proper Orthogonal Decomposition of density field}

Proper Orthogonal Decomposition (POD) was performed on the fluctuations of the 2D density field collected on the $xy$-plane, within a window of $x/D=[0,8]$ and $y/D=[-4,4]$, to identify the dominant unsteady coherent structures within the jet that contribute to the screech. The POD analysis in this Sec. is based on the snapshot technique described by Weiss in Ref-\cite{weiss2019tutorial} and follows a similar approach to the analysis conducted by Chandravamsi et al. \cite{chandravamsi2023control}. Firstly, a time series of `$N$' ($N=1600$ in the present case) evenly spaced instantaneous 2D density fluctuation field data (separated by $\Delta t a_{\infty}/D=0.015$) is collected and reshaped to form a 2D solution matrix named $\mathbf{A} = [\rho^{'}(x,y,t_1), \rho^{'}(x,y,t_2), .... \rho^{'}(x,y,t_N)]$. The data is reshaped in such a way that each column in $\mathbf{A}$ corresponds to a `time-instance' and each row `spatial location'. The covariance matrix $\mathbf{C}=\mathbf{A}^{T}\mathbf{A}$ was then subjected to eigenvalue decomposition to find its eigenvalues `$\Lambda_i$' and eigenvectors $\mathbf{E_i}$, forming an orthogonal basis in the eigenspace where each eigenvalue represents the variance along the corresponding eigenvector. Finally, the POD modes ($\mathbf{\Phi}_i$) and their corresponding time coefficients ($\Tilde{\mathbf{a}}$) are evaluated using the eigenvectors and input flow data ($\rho^{'}(x,y,t)$) using the following relations:

% The eigen values $\Lambda_i$ and eigen vectors $\mathbf{E}_i$ are sorted in the increasing order of the variance (magnitude of eigen value).

\begin{equation}
    \boldsymbol{\Phi}_i=\frac{\sum_{j=1}^N E_i^j \rho^{'}(x,y,t_j)}{\left\|\sum_{j=1}^N E_i^j \rho^{'}(x,y,t_j)\right\|}, \quad \text{and} \quad \Tilde{\mathbf{a}}_i = (\mathbf{\Phi}_i)^{T}\mathbf{A}, \quad i=1, 2 \ldots, N.
\end{equation}

The index `$i$' here represents the mode number. The modes are ordered based on their relative mode energy, calculated as $RE_i = \Lambda_i \times \left(\sum_{i=1}^N \Lambda_i\right)^{-1}$. As depicted in Fig. \ref{POD}(a), amongst the first eight dominant modes displayed, the first mode itself accounts for $\approx 45\%$ of the total energy. A noticeable decrease in energy can be observed following the first mode, implying that only a single dominant mode exists, and the majority of the jet's unsteady dynamics are contained within mode-1. The normalized mode-1 shape is displayed in Fig. \ref{POD}(b), along with annotations describing the locations of shock cells by white dashed lines. The shock cell spacings ($\lambda$), as determined from Fig. \ref{jet-inst-ave}(b), are indicated at the top. The mode shape exhibits symmetry about the jet axis, with a reversal in sign, indicating that the jet oscillations of the most energetic portion are oriented along the $y$-axis. The dominant fluctuations are observed in the region corresponding to the third, fourth, and fifth shock cells, suggesting that the shear layer structures along the jet surface experience significant amplification after their interaction with the third shock cell. Therefore, the effective source of the screech noise can be considered to be situated between the third and fifth shock cells. As depicted in Fig. \ref{POD}(c), the Fourier analysis of the mode-1 time coefficients revealed a ``pure tone'' that precisely matches the screech frequency ($St = 0.308$) determined from the acoustic data in Sec. \ref{freqs}. This suggests that the oscillation frequency of the oblique shocks within the jet plume and the turbulent structures between the third and fifth shock cells are all operating at the same frequency as the screech.

\begin{figure}[h!]
    \centering
    \includegraphics[width=\textwidth]{Images/POD.pdf}
    \caption{(a) Mode energies of first eight POD modes, (b) Mode-1 shape with shock cell locations and spacing indicated, and (c) Power Spectral Density of mode-1 POD
coefficients.}
    \label{POD}
\end{figure}



