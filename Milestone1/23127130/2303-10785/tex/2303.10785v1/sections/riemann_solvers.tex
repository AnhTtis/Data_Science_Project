\subsubsection{The HLLC Riemann solver} \label{sec:riemann-solver}
In this section, the HLLC Riemann solver is presented within the framework of curvilinear coordinates. The HLLC Riemann solver is a three-wave approximate solution that accounts for the presence of contact discontinuities. One of the notable advantages of the HLLC fluxes is their ability to minimize dissipation in comparison to other commonly used methods, such as HLL \cite{harten1983upstream} and Rusanov \cite{toro2009riemann}. As a result, HLLC is able to effectively resolve shocks, contact discontinuities, and shear layers. The inviscid flux $\mathbf{F}_{i+\frac{1}{2}}$ in the transformed coordinates through the HLLC approximate Riemann solver is obtained as follows:


% To the best of the authors knowledge, the formulation for the HLLC Riemann solver in in three-dimensional coordinates is not presented in the literature so far.
% After obtaining the left ($L$) and right ($R$) reconstructed states of the primitive variables at each cell interface (after step-5 in Fig. \ref{inv_algo}), the next task is to evaluate the convection fluxes $\mathbf{F}_{i+\frac{1}{2}}, \mathbf{G}_{i+\frac{1}{2}}$, and $\mathbf{H}_{i+\frac{1}{2}}$. 

\begin{equation}
\hat{\mathbf{F}}_{i+\frac{1}{2}}^{\mathrm{HLLC}}=\left\{\begin{array}{ll}
\mathbf{\hat{F}}^{\mathbf{L}} & \text { if } \quad \hat{S}^{L} \geq 0, \\
\mathbf{\hat{F}}^{* \mathbf{L}}=\mathbf{\hat{F}}^{\mathbf{L}}+\hat{S}^{L}\left(\mathbf{Q}^{* \mathbf{L}}-\mathbf{Q}^{\mathbf{L}}\right) & \text { if } \quad \hat{S}^{L} \leq 0 \leq \hat{S}^{*}, \\
\mathbf{\hat{F}}^{* \mathbf{R}}=\mathbf{\hat{F}}^{\mathbf{R}}+\hat{S}^{R}\left(\mathbf{Q}^{* \mathbf{R}}-\mathbf{Q}^{\mathbf{R}}\right) & \text { if } \quad \hat{S}^{*} \leq 0 \leq \hat{S}^{R}, \\
\mathbf{\hat{F}}^{\mathbf{R}} & \text { if } \quad \hat{S}^{R} \leq 0
\end{array}\right.
\end{equation}

The three wave speeds of the local one-dimensional problem are calculated as follows.
\begin{equation*}
    \hat{S}^{L}=\operatorname{Min}\left(\hat{U}^{L}-c^{L} \sqrt{\hat{\xi}_{x}^{2}+\hat{\xi}_{y}^{2}+\hat{\xi}_{z}^{2}}, \quad \hat{U}^{R}-c^{R} \sqrt{\hat{\xi}_{x}^{2}+\hat{\xi}_{y}^{2}+\hat{\xi}_{z}^{2}}\right) 
\end{equation*}

\begin{equation*}
    \hat{S}^{R}=\operatorname{Max}\left(\hat{U}^{L}+c^{L} \sqrt{\hat{\xi}_{x}^{2}+\hat{\xi}_{y}^{2}+\hat{\xi}_{z}^{2}}, \quad \hat{U}^{R}+c^{R} \sqrt{\hat{\xi}_{x}^{2}+\hat{\xi}_{y}^{2}+\hat{\xi}_{z}^{2}}\right)
\end{equation*}

\begin{equation*}
    \hat{S}^{*}=\frac{\rho^{R} \hat{U}^{R}\left(\hat{S}^{R}-\hat{U}^{R}\right)-\rho^{L} \hat{U}^{L}\left(\hat{S}^{L}-\hat{U}^{L}\right)+\left(P_{L}-P_{R}\right)\left(\hat{\xi}_{x}^{2}+\hat{\xi}_{y}^{2}+\hat{\xi}_{z}^{2}\right)}{\rho^{R}\left(\hat{S}^{R}-\hat{U}^{R}\right)-\rho^{L}\left(\hat{S}^{L}-\hat{U}^{L}\right)}
\end{equation*} 

The conservative variables in the star region $\mathbf{Q}^{* \mathbf{K}}$ for $\mathbf{K} = L,R$ are estimated as follows.
% \begin{equation}
% \mathbf{Q}^{* \mathbf{K}}=\rho^{\mathbf{K}}\left(\frac{\hat{S}^{\mathbf{K}}-\hat{U}^{\mathbf{K}}}{\hat{S}^{\mathbf{K}}-\hat{S}^{*}}\right)\left[\begin{array}{c}
% 1 \\
% \frac{\hat{\xi}_{x}\hat{S}^{*}+(\hat{\xi}_{y}^{2}+\hat{\xi}_{z}^{2}) u^{\mathbf{K}}-\hat{\xi}_{y} \hat{\xi}_{x} v^{\mathbf{K}}-\hat{\xi}_{z} \hat{\xi}_{x}w^{\mathbf{K}}}{\hat{\xi}_{x}^{2}+\hat{\xi}_{y}^{2}+\hat{\xi}_{z}^{2}} \\
% \frac{\hat{\xi}_{y}\hat{S}^{*}-\hat{\xi}_{x} \hat{\xi}_{y} u^{\mathbf{K}}+(\hat{\xi}_{x}^{2}+\hat{\xi}_{z}^{2}) v^{\mathbf{K}}-\hat{\xi}_{z} \hat{\xi}_{y}w^{\mathbf{K}}}{\hat{\xi}_{x}^{2}+\hat{\xi}_{y}^{2}+\hat{\xi}_{z}^{2}} \\
% \frac{\hat{\xi}_{z}\hat{S}^{*}-\hat{\xi}_{x} \hat{\xi}_{z} u^{\mathbf{K}}-\hat{\xi}_{y} \hat{\xi}_{z} v^{\mathbf{K}}+(\hat{\xi}_{x}^{2}+\hat{\xi}_{y}^{2})w^{\mathbf{K}}}{\hat{\xi}_{x}^{2}+\hat{\xi}_{y}^{2}+\hat{\xi}_{z}^{2}} \\
% \frac{E^{\mathbf{K}}}{\rho^{\mathbf{K}}}+\left(\hat{S}^{*}-\hat{U}^{\mathbf{K}}\right)\left\{\frac{\hat{S}^{*}}{\left(\hat{\xi}_{x}^{2}+\hat{\xi}_{y}^{2}+\hat{\xi}_{z}^{2}\right)}+\frac{p^{\mathbf{K}}}{\rho^{\mathbf{K}}(\hat{S}^{\mathbf{K}}-\hat{U}^{\mathbf{K}})}\right\}
% \end{array}\right]
% \end{equation}

\begin{equation}
\mathbf{Q}^{* \mathbf{K}}=\rho^{\mathbf{K}}\left(\frac{\hat{S}^{\mathbf{K}}-\hat{U}^{\mathbf{K}}}{\hat{S}^{\mathbf{K}}-\hat{S}^{*}}\right)\begin{bmatrix}
    1 \\
\frac{\hat{\xi}_{x}(\hat{S}^{*}-\hat{\xi}_{t})+(\hat{\xi}_{y}^{2}+\hat{\xi}_{z}^{2}) u^{\mathbf{K}}-\hat{\xi}_{y} \hat{\xi}_{x} v^{\mathbf{K}}-\hat{\xi}_{z} \hat{\xi}_{x}w^{\mathbf{K}}}{\hat{\xi}_{x}^{2}+\hat{\xi}_{y}^{2}+\hat{\xi}_{z}^{2}} \\
\frac{\hat{\xi}_{y}(\hat{S}^{*}-\hat{\xi}_{t})-\hat{\xi}_{x} \hat{\xi}_{y} u^{\mathbf{K}}+(\hat{\xi}_{x}^{2}+\hat{\xi}_{z}^{2}) v^{\mathbf{K}}-\hat{\xi}_{z} \hat{\xi}_{y}w^{\mathbf{K}}}{\hat{\xi}_{x}^{2}+\hat{\xi}_{y}^{2}+\hat{\xi}_{z}^{2}} \\
\frac{\hat{\xi}_{z}(\hat{S}^{*}-\hat{\xi}_{t})-\hat{\xi}_{x} \hat{\xi}_{z} u^{\mathbf{K}}-\hat{\xi}_{y} \hat{\xi}_{z} v^{\mathbf{K}}+(\hat{\xi}_{x}^{2}+\hat{\xi}_{y}^{2})w^{\mathbf{K}}}{\hat{\xi}_{x}^{2}+\hat{\xi}_{y}^{2}+\hat{\xi}_{z}^{2}} \\
\frac{E^{\mathbf{K}}}{\rho^{\mathbf{K}}}+\left(\hat{S}^{*}-\hat{U}^{\mathbf{K}}\right)\left\{\frac{\hat{S}^{*}-\hat{\xi}_{t}}{\left(\hat{\xi}_{x}^{2}+\hat{\xi}_{y}^{2}+\hat{\xi}_{z}^{2}\right)}+\frac{p^{\mathbf{K}}}{\rho^{\mathbf{K}}(\hat{S}^{\mathbf{K}}-\hat{U}^{\mathbf{K}})}\right\}
  \end{bmatrix}
\end{equation}

The $i+\frac{1}{2}$ metrics involved in the above formulation are interpolated consistently with the inviscid flux discretization to prevent metric cancellation errors, which can accumulate in the flow solution. The corresponding interpolation formulae and demonstrative examples concerning metric cancellation errors and their effect on the flow solution will be discussed in Section \ref{sec:FP}. The above formulations only show the flux calculation in the $\xi$ direction. Extending these formulae to other directions is straightforward. For instance, 

\begin{equation}
    \hat{\mathbf{G}}_{i,j+\frac{1}{2},k} = \mathbf{f}_{HLLC} \left\{ \mathbf{P}^L_{i,j+\frac{1}{2},k}, \mathbf{P}^R_{i,j+\frac{1}{2},k}, (\hat{\eta}_t)_{i,j+\frac{1}{2},k}, (\hat{\eta}_x)_{i,j+\frac{1}{2},k}, (\hat{\eta}_y)_{i,j+\frac{1}{2},k}, (\hat{\eta}_z)_{i,j+\frac{1}{2},k} \right\}.
\end{equation}

% It should be noted that all the quantities in the above formulae with a hat on the top $\hat{(\cdot)}$ represent that they are normalized with local Jacobian. 