\appendix

\section{Primitive variables form of 3-D Euler equations and its eigen Structure} \label{app:Eig-structure}
The eigenvectors used for characteristic transformation employed during the reconstruction procedure in the MEG6 and MIG4 schemes (section \ref{Inv-disc}) are presented in this section. Firstly, removing the viscous terms from the Navier-Stokes equations (Eqn. \ref{trans-eqn}) we obtain the following inviscid counterpart.

\begin{equation}
    \frac{\partial}{\partial t}\left(\frac{\mathbf{Q}}{J}\right)+\frac{\partial \hat{\mathbf{F}}}{\partial \xi}+\frac{\partial \hat{\mathbf{G}}}{\partial \eta}+\frac{\partial \hat{\mathbf{H}}}{\partial \zeta}=0
\end{equation}\label{EulerGC}

Following a series of algebraic manipulations, the above equation can be transformed into its primitive variables form with $\mathbf{P}=[\rho,u,v,w,p]^T$ present in the time, and spatial derivative terms \cite{masatsuka2013like}.

\begin{equation}\label{prim-form}
    \frac{\partial \mathbf{P}}{\partial t}+\overline{\mathbf{B}}_{\xi} \frac{\partial \mathbf{P}}{\partial \xi}+\overline{\mathbf{B}}_{\eta} \frac{\partial \mathbf{P}}{\partial \eta}+\overline{\mathbf{B}}_{\zeta} \frac{\partial \mathbf{P}}{\partial \zeta}=0
\end{equation}

where the coefficient matrices $\overline{\mathbf{B}}_{\xi}$, $\overline{\mathbf{B}}_{\eta}$ and $\overline{\mathbf{B}}_{\zeta}$ are,

\begin{equation*}
\overline{\mathbf{B}}_{\xi}=\frac{1}{J}\left[\begin{array}{ccccc}
U & \rho \xi_{x} & \rho \xi_{y} & \rho \xi_{z} & 0 \\
0 & U & 0 & 0 & \frac{\xi_{x}}{\rho} \\
0 & 0 & U & 0 & \frac{\xi_{y}}{\rho} \\
0 & 0 & 0 & U & \frac{\xi_{z}}{\rho} \\
0 & \gamma p \xi_{x} & \gamma p \xi_{y} & \gamma p \xi_{z} & U
\end{array}\right], \quad
\overline{\mathbf{B}}_{\eta}=\frac{1}{J}\left[\begin{array}{ccccc}
V & \rho \eta_{x} & \rho \eta_{y} & \rho \eta_{z} & 0 \\
0 & V & 0 & 0 & \frac{\eta_{x}}{\rho} \\
0 & 0 & V & 0 & \frac{\eta_{y}}{\rho} \\
0 & 0 & 0 & V & \frac{\eta_{z}}{\rho} \\
0 & \gamma p \eta_{x} & \gamma p \eta_{y} & \gamma p \eta_{z} & V
\end{array}\right] \text{and}
\end{equation*}

\begin{equation*}
\overline{\mathbf{B}}_{\zeta}=\frac{1}{J}\left[\begin{array}{ccccc}
W & \rho \zeta_{x} & \rho \zeta_{y} & \rho \zeta_{z} & 0 \\
0 & W & 0 & 0 & \frac{\zeta_{x}}{\rho} \\
0 & 0 & W & 0 & \frac{\zeta_{y}}{\rho} \\
0 & 0 & 0 & W & \frac{\zeta_{z}}{\rho} \\
0 & \gamma p \zeta_{x} & \gamma p \zeta_{y} & \gamma p \zeta_{z} & W
\end{array}\right] .
\end{equation*}

For the characteristic variable projection used in the reconstruction procedure of the inviscid flux computation algorithm (Eqns. \ref{forward-projection} and \ref{rev-projection}), the eigenvectors of the coefficient matrices $\overline{\mathbf{B}}_{\xi}$, $\overline{\mathbf{B}}_{\eta}$ and $\overline{\mathbf{B}}_{\zeta}$ are used. The eigen structure of the coefficient matrices can be found by splitting them using eigen decomposition. The following three eigenvalue problems must be solved to find the corresponding eigen matrices. 

\begin{subequations}
    \begin{gather}
        \overline{\mathbf{B}}_{\xi} \overline{\mathbf{R}}_{\xi}=\overline{\mathbf{R}}_{\xi} \bar{\Lambda}_{\xi}, \quad \overline{\mathbf{B}}_{\eta} \overline{\mathbf{R}}_{\eta}=\overline{\mathbf{R}}_{\eta} \bar{\Lambda}_{\eta}, \text{ and} \quad \overline{\mathbf{B}}_{\zeta} \overline{\mathbf{R}}_{\zeta}=\overline{\mathbf{R}}_{\zeta} \bar{\Lambda}_{\zeta}.
        \tag{\theequation a-\theequation c}
    \end{gather}
\end{subequations}

Where, $\overline{\mathbf{R}}_{\xi}$, $\overline{\mathbf{R}}_{\eta}$, and $\overline{\mathbf{R}}_{\zeta}$ are the right eigen vector matrices and $\bar{\Lambda}_{\xi}$, $\bar{\Lambda}_{\eta}$, and $\bar{\Lambda}_{\zeta}$ are the eigen value matrices which are block diagonal in nature.

% For instance, $\overline{\mathbf{B}}_{\xi}$ can be decomposed by solving the eigen value problem: $\overline{\mathbf{B}}_{\xi} \overline{\mathbf{R}}_{\xi}=\overline{\mathbf{R}}_{\xi} \bar{\Lambda}_{\xi}$. Where $\overline{\mathbf{R}}_{\xi}$ and $\bar{\Lambda}_{\xi}$ are right eigen vector matrix and eigen value matrix of $\overline{\mathbf{B}}_{\xi}$ respectively. Similarly, $\overline{\mathbf{B}}_{\eta}$ and $\overline{\mathbf{B}}_{\zeta}$ can also be decomposed by solving $\overline{\mathbf{B}}_{\eta} \overline{\mathbf{R}}_{\eta}=\overline{\mathbf{R}}_{\eta} \bar{\Lambda}_{\eta}$ and $\overline{\mathbf{B}}_{\zeta} \overline{\mathbf{R}}_{\zeta}=\overline{\mathbf{R}}_{\zeta} \bar{\Lambda}_{\zeta}$. This will result in the following equation.
% 
% \begin{equation}
%     \frac{\partial \mathbf{P}}{\partial t}+ \left( \overline{\mathbf{R}}_{\xi} \bar{\Lambda}_{\xi} \overline{\mathbf{R}}_{\xi}^{-1} \right) \frac{\partial \mathbf{P}}{\partial \xi}+ \left(\overline{\mathbf{R}}_{\eta} \bar{\Lambda}_{\eta} \overline{\mathbf{R}}_{\eta}^{-1}\right) \frac{\partial \mathbf{P}}{\partial \eta}+ \left(\overline{\mathbf{R}}_{\zeta} \bar{\Lambda}_{\zeta} \overline{\mathbf{R}}_{\zeta}^{-1}\right) \frac{\partial \mathbf{P}}{\partial \zeta}=0
% \end{equation}

Solving the above mentioned eigen value problems, the following left and right eigen vector and eigen value matrices are obtained.

\begin{subequations}
    \begin{gather}
        \overline{\boldsymbol{\Lambda}}_{\xi}=\left[\begin{array}{ccccc}
        U-a_{\xi} & 0 & 0 & 0 & 0 \\
        0 & U & 0 & 0 & 0 \\
        0 & 0 & U & 0 & 0 \\
        0 & 0 & 0 & U & 0 \\
        0 & 0 & 0 & 0 & U+a_{\xi}
        \end{array}\right], \quad  \overline{\mathbf{R}}_{\xi}=\left[\begin{array}{ccccc}
        \frac{1}{2 a^{2}} & \frac{\tilde{\xi}_{x}}{a^{2}} & \frac{\tilde{\xi}_{z}}{a^{2}} & -\frac{\tilde{\xi}_{y}}{a^{2}} & \frac{1}{2 a^{2}} \\
        -\frac{\tilde{\xi}_{x}}{2 \rho a} & 0 & -\tilde{\xi}_{y} & -\tilde{\xi}_{z} & \frac{\tilde{\xi}_{x}}{2 \rho a} \\
        -\frac{\tilde{\xi}_{y}}{2 \rho a} & -\tilde{\xi}_{z} & \tilde{\xi}_{x} & 0 & \frac{\tilde{\xi}_{y}}{2 \rho a} \\
        -\frac{\tilde{\xi}_{z}}{2 \rho a} & \tilde{\xi}_{y} & 0 & \tilde{\xi}_{x} & \frac{\tilde{\xi}_{z}}{2 \rho a} \\
        \frac{1}{2} & 0 & 0 & 0 & \frac{1}{2}
        \end{array}\right] \tag{\theequation a-\theequation b} \\
        \overline{\boldsymbol{\Lambda}}_{\eta}=\left[\begin{array}{ccccc}
        V-a_{\eta} & 0 & 0 & 0 & 0 \\
        0 & V & 0 & 0 & 0 \\
        0 & 0 & V & 0 & 0 \\
        0 & 0 & 0 & V & 0 \\
        0 & 0 & 0 & 0 & V+a_{\eta}
        \end{array}\right], \quad \overline{\mathbf{R}}_{\eta}=\left[\begin{array}{ccccc}
        \frac{1}{2 a^{2}} & \frac{\tilde{\eta}_{x}}{a^{2}} & \frac{\tilde{\eta}_{z}}{a^{2}} & -\frac{\tilde{\eta}_{y}}{a^{2}} & \frac{1}{2 a^{2}} \\
        -\frac{\tilde{\eta}_{x}}{2 \rho a} & 0 & -\tilde{\eta}_{y} & -\tilde{\eta}_{z} & \frac{\tilde{\eta}_{x}}{2 \rho a} \\
        -\frac{\tilde{\eta}_{y}}{2 \rho a} & -\tilde{\eta}_{z} & \tilde{\eta}_{x} & 0 & \frac{\tilde{\eta}_{y}}{2 \rho a} \\
        -\frac{\tilde{\eta}_{z}}{2 \rho a} & \tilde{\eta}_{y} & 0 & \tilde{\eta}_{x} & \frac{\tilde{\eta}_{z}}{2 \rho a} \\
        \frac{1}{2} & 0 & 0 & 0 & \frac{1}{2}
        \end{array}\right] \tag{\theequation c-\theequation d} \\
        \overline{\boldsymbol{\Lambda}}_{\zeta}=\left[\begin{array}{ccccc}
        W-a_{\zeta} & 0 & 0 & 0 & 0 \\
        0 & W & 0 & 0 & 0 \\
        0 & 0 & W & 0 & 0 \\
        0 & 0 & 0 & W & 0 \\
        0 & 0 & 0 & 0 & W+a_{\zeta}
        \end{array}\right] , \quad
        \overline{\mathbf{R}}_{\zeta}=\left[\begin{array}{ccccc}
        \frac{1}{2 a^{2}} & \frac{\tilde{\zeta}_{x}}{a^{2}} & \frac{\tilde{\zeta}_{z}}{a^{2}} & -\frac{\tilde{\zeta}_{y}}{a^{2}} & \frac{1}{2 a^{2}} \\
        -\frac{\tilde{\zeta}_{x}}{2 \rho a} & 0 & -\tilde{\zeta}_{y} & -\tilde{\zeta}_{z} & \frac{\tilde{\zeta}_{x}}{2 \rho a} \\
        -\frac{\tilde{\zeta}_{y}}{2 \rho a} & -\tilde{\zeta}_{z} & \tilde{\zeta}_{x} & 0 & \frac{\tilde{\zeta}_{y}}{2 \rho a} \\
        -\frac{\tilde{\zeta}_{z}}{2 \rho a} & \tilde{\zeta}_{y} & 0 & \tilde{\zeta}_{x} & \frac{\tilde{\zeta}_{z}}{2 \rho a} \\
        \frac{1}{2} & 0 & 0 & 0 & \frac{1}{2}
        \end{array}\right] \tag{\theequation e-\theequation f} 
    \end{gather}
\end{subequations}


where,
\begin{equation*}
a_{\xi}=a\|\nabla \xi\|, \quad a_{\eta}=a\|\nabla \eta\|, \quad a_{\zeta}=a\|\nabla \zeta\|.
\end{equation*}\\

The metric terms with a tilde on the top $\tilde{(\cdot)}$ present in the above matrices denote normalized quantities defined as follows.
\begin{equation} \label{norm-metrics}
\begin{array}{ll}
\tilde{\xi}_{x}=\frac{\xi_{x}}{\|\nabla \xi\|}, \quad \tilde{\xi}_{y}=\frac{\xi_{y}}{\|\nabla \xi\|}, \quad \tilde{\xi}_{z}=\frac{\xi_{z}}{\|\nabla \xi\|}, \\
\tilde{\eta}_{x}=\frac{\eta_{x}}{\|\nabla \eta\|}, \quad \tilde{\eta}_{y}=\frac{\eta_{y}}{\|\nabla \eta\|}, \quad \tilde{\eta}_{z}=\frac{\eta_{z}}{\|\nabla \eta\|}, \\
\tilde{\zeta}_{x}=\frac{\zeta_{x}}{\|\nabla \zeta\|}, \quad \tilde{\zeta}_{y}=\frac{\zeta_{y}}{\|\nabla \zeta\|}, \quad \tilde{\zeta}_{z}=\frac{\zeta_{z}}{\|\nabla \zeta\|}.
\end{array}
\end{equation}

where,
\begin{equation*}
\begin{array}{l}
\|\nabla \xi\|=\sqrt{\xi_{x}^{2}+\xi_{y}^{2}+\xi_{z}^{2}}, \quad
\|\nabla \eta\|=\sqrt{\eta_{x}^{2}+\eta_{y}^{2}+\eta_{z}^{2}}, \text{ and} \quad
\|\nabla \zeta\|=\sqrt{\zeta_{x}^{2}+\zeta_{y}^{2}+\zeta_{z}^{2}}
\end{array}
\end{equation*}

\noindent The corresponding inverse matrices or the left eigen vector matrices are,

\begin{subequations}
    \begin{gather}
        \overline{\mathbf{R}}_{\xi}^{-1}=\left[\begin{array}{ccccc}
        0 & -\tilde{\xi}_{x} \rho a & -\tilde{\xi}_{y} \rho a & -\tilde{\xi}_{z} \rho a & 1 \\
        \tilde{\xi}_{x} a^{2} & 0 & -\tilde{\xi}_{z} & \tilde{\xi}_{y} & -\tilde{\xi}_{x} \\
        \tilde{\xi}_{z} a^{2} & -\tilde{\xi}_{y} & \tilde{\xi}_{x} & 0 & -\tilde{\xi}_{z} \\
        -\tilde{\xi}_{y} a^{2} & -\tilde{\xi}_{z} & 0 & \tilde{\xi}_{x} & \tilde{\xi}_{y} \\
        0 & \tilde{\xi}_{x} \rho a & \tilde{\xi}_{y} \rho a & \tilde{\xi}_{z} \rho a & 1
        \end{array}\right], \quad  \overline{\mathbf{R}}_{\eta}^{-1}=\left[\begin{array}{ccccc}
        0 & -\tilde{\eta}_{x} \rho a & -\tilde{\eta}_{y} \rho a & -\tilde{\eta}_{z} \rho a & 1 \\
        \tilde{\eta}_{x} a^{2} & 0 & -\tilde{\eta}_{z} & \tilde{\eta}_{y} & -\tilde{\eta}_{x} \\
        \tilde{\eta}_{z} a^{2} & -\tilde{\eta}_{y} & \tilde{\eta}_{x} & 0 & -\tilde{\eta}_{z} \\
        -\tilde{\eta}_{y} a^{2} & -\tilde{\eta}_{z} & 0 & \tilde{\eta}_{x} & \tilde{\eta}_{y} \\
        0 & \tilde{\eta}_{x} \rho a & \tilde{\eta}_{y} \rho a & \tilde{\eta}_{z} \rho a & 1
        \end{array}\right] \tag{\theequation a-\theequation b} \\
        \overline{\mathbf{R}}_{\zeta}^{-1}=\left[\begin{array}{ccccc}
        0 & -\tilde{\zeta}_{x} \rho a & -\tilde{\zeta}_{y} \rho a & -\tilde{\zeta}_{z} \rho a & 1 \\
        \tilde{\zeta}_{x} a^{2} & 0 & -\tilde{\zeta}_{z} & \tilde{\zeta}_{y} & -\tilde{\zeta}_{x} \\
        \tilde{\zeta}_{z} a^{2} & -\tilde{\zeta}_{y} & \tilde{\zeta}_{x} & 0 & -\tilde{\zeta}_{z} \\
        -\tilde{\zeta}_{y} a^{2} & -\tilde{\zeta}_{z} & 0 & \tilde{\zeta}_{x} & \tilde{\zeta}_{y} \\
        0 & \tilde{\zeta}_{x} \rho a & \tilde{\zeta}_{y} \rho a & \tilde{\zeta}_{z} \rho a & 1
        \end{array}\right]. \tag{\theequation c} 
    \end{gather}
\end{subequations}

%These eigen vector matrices are used in the MEG6 and MIG4 algorithms (Fig. \ref{inv_algo}) to project the primitive variables in to characteristic space and vice versa, using the equations \ref{forward-projection} and \ref{rev-projection}.


\section{Time integration} \label{sec:time-int}
For time marching the solution, the 3rd order explicit Total Variation Diminishing Range-Kutta (TVD RK-3) \cite{gottlieb1998total} time integration scheme is used. To employ this method, firstly, the residual ($Res$) is computed from the spatial discretization procedure described in the previous sections. Then the following three-stage formulation is employed where the solution corresponding to the previous RK step is used to compute the solution corresponding to the next; thus arriving at the final solution corresponding to $t+\Delta t$ at the end.

\begin{equation}
    \begin{aligned}
\hat{\mathbf{Q}}^{(1)} &=\hat{\mathbf{Q}}^{\mathbf{n}}+\Delta t \operatorname{Res}\left(\hat{\mathbf{Q}}^{\mathbf{n}}\right) J \\
\hat{\mathbf{Q}}^{(\mathbf{2})} &=\frac{3}{4} \hat{\mathbf{Q}}^{\mathbf{n}}+\frac{1}{4} \hat{\mathbf{Q}}^{(1)}+\frac{1}{4} \Delta t \operatorname{Res}\left(\hat{\mathbf{Q}}^{(1)}\right) J \\
\hat{\mathbf{Q}}^{\mathbf{n}+\mathbf{1}} &=\frac{1}{3} \hat{\mathbf{Q}}^{\mathbf{n}}+\frac{2}{3} \hat{\mathbf{Q}}^{(\mathbf{2})}+\frac{2}{3} \Delta t \mathbf{R e s}\left(\hat{\mathbf{Q}}^{(\mathbf{2})}\right) J,
\end{aligned}
\end{equation}

The Jacobian (J) used in the above formulation corresponds to the cell center of the grid, the location where time integration is performed.\\

All the simulations using this solver are performed with a Courant Friedrichs Lewy (CFL) number number of 0.2 unless a different value is specified for a particular case. As opposed to maintaining a constant CFL, a constant stable time-step ($\text{CFL}<0.5$) is maintained for supersonic jet flow calculations to facilitate Fourier analysis of pressure signals extracted from the solution. The time-step while performing Euler simulations is computed as follows.

\begin{equation}
    \Delta t=\mathrm{CFL} \times \min _{\text {cells }} \left(\frac{1}{|U|+c \|\nabla \xi\|},                                                            \frac{1}{|V|+c \|\nabla \eta\|},                                                           \frac{1}{|W|+c \|\nabla \zeta\|}\right).
\end{equation}

\noindent The time step for viscous flow computations is computed as follows.

\begin{equation}
\Delta t=\mathrm{CFL} \times \min \left[\min _{\text {cells }}\left(\frac{1}{|U|+c\|\nabla \xi\|}, \frac{1}{|V|+c\|\nabla \eta\|}, \frac{1}{|W|+c\|\nabla \zeta\|}\right)\right. ,\frac{1}{\alpha}\left.\min _{\text {cells }}\left(\frac{1}{\mu\|\nabla \xi\|^2}, \frac{1}{\mu\|\nabla \eta\|^2}, \frac{1}{\mu\|\nabla \zeta\|^2}\right)\right]
\end{equation}

\section{Metric term computation at boundaries} \label{app:one-sided}

As previously discussed in Section \ref{sec:FP}, the lack of geometric information in ghost points poses a limitation on the computation of metric terms using high-order estimates, as outlined in Equations \ref{FP-interp}, \ref{FP-interp2}, and \ref{interp-formulae}. To address this limitation, low-order approximations are employed for interpolation and derivative computation in the calculation of metric terms. An example of this approach is demonstrated in the computation of the term `$x_{\xi}$' at various boundary cell centers and interfaces.

% The unavailability of coordinate information at grid cells close to the block boundaries poses limitation on employing the current high interpolation and derivative formulae, which are required to compute metric terms. In order to compute metric terms at block boundaries, we reduce the order of interpolation gradually towards boundaries 

\begin{figure}[h!]
    \centering
    \includegraphics[width=100mm]{Images/boundary-pts.pdf}
    \caption{Representation of boundary cells and interfaces along the $\xi$ direction, with corresponding indices. Note that the first cell index begins at $1$ and the first interface index starts at `$1/2$'.}
    \label{boundary-pts}
\end{figure}

\begin{equation}
    x_{i+\frac{1}{2}} = \left\{\begin{array}{ll}
x_{i+1}-\frac{1}{2} x^{'}_{i+1} +\frac{1}{12} x^{''}_{i+1} & \text { if } \quad i=1 \\
\frac{1}{2}\left[x_{i}+\frac{1}{2} x^{'}_{i} +\frac{1}{12} x^{''}_i \right] + \frac{1}{2}\left[x_{i+1}-\frac{1}{2} x^{'}_{i+1} +\frac{1}{12} x^{''}_{i+1} \right] & \text { if } \quad i=2 \text { and } 3.
\end{array}\right.
\end{equation}

\begin{equation}
    \left(x^{'}\right)_{i} = \left\{\begin{array}{ll}
-\frac{3}{2}x_i + 2x_{i+1} -\frac{1}{2}x_{i+2} & \text { if } \quad i=1 \text { (2nd order one sided)}, \\
-\frac{1}{4}x_i - \frac{5}{6}x_{i+1} +\frac{3}{2}x_{i+2} -\frac{1}{2}x_{i+3} +\frac{1}{12}x_{i+4} & \text { if } \quad i=2 \text { (4th order partially one sided)}, \\
\frac{2}{3}\left[ x_{i+1} - x_{i-1}\right] - \frac{1}{12}\left[ x_{i+2} - x_{i-2}\right] & \text { if } \quad i=3 \text { (4th order central)}.
\end{array}\right.
\end{equation}

\begin{equation}
    \left(x^{''}\right)_i = \left\{\begin{array}{ll}
-\frac{3}{2} x^{'}_i + 2x^{'}_{i+1} -\frac{1}{2} x^{'}_{i+2} & \text { if } \quad i=1, \\
-\frac{1}{4}x^{'}_i - \frac{5}{6} x^{'}_{i+1} +\frac{3}{2} x^{'}_{i+2} -\frac{1}{2}x^{'}_{i+3} +\frac{1}{12}x^{'}_{i+4} & \text { if } \quad i=2, \\
-\frac{1}{2}\left[ x^{'}_{i+1} - x^{'}_{i-1}\right] - 2\left[ x_{i+1} - 2x_i + x_{i-1}\right]  & \text { if } \quad i=3.
\end{array}\right.
\end{equation}

and finally
\begin{equation}
    \left(x_{\xi}\right)_i = x_{i+\frac{1}{2}} - x_{i-\frac{1}{2}}
\end{equation}


After computing $x_{\xi}$ and the other required terms present in Equation \ref{GCL_metrics}, they are utilized in the computation of the metric terms through the use of conservative formulations detailed in Section \ref{conser-metrics}. However, as primitive variables are retained in the ghost cells, it is not necessary to use one-sided formulae to compute fluxes at boundary points. As a result, the high-order relations discussed in Section \ref{sec:disc} remain valid for both inviscid and viscous flux computation at boundaries.

\section{Non-conservative metric term formulation} \label{app:non-FP}
The non-conservative metric formulations used to compute metrics in the approach mentioned in section \ref{sec:FP} are:

\begin{equation}
    \begin{aligned}
&\hat{\xi}_{x}=\left(y_\eta z_\zeta-y_\zeta z_\eta\right),\quad 
 \hat{\xi}_{y}=\left(z_\eta x_\zeta-z_\zeta x_\eta\right),\quad 
 \hat{\xi}_{z}=\left(x_\eta y_\zeta-x_\zeta y_\eta\right), \\
&\hat{\eta}_{x}=\left(y_\zeta z_{\xi}-y_{\xi} z_\zeta\right), \quad
 \hat{\eta}_{y}=\left(z_\zeta x_{\xi}-z_{\xi} x_\zeta\right), \quad
 \hat{\eta}_{z}=\left(x_\zeta y_{\xi}-x_{\xi} y_\zeta\right), \\
&\hat{\zeta}_{x}=\left(y_{\xi} z_\eta-y_\eta z_{\xi}\right), \quad
 \hat{\zeta}_{y}=\left(z_\zeta x_\eta-z_\eta x_\zeta\right), \quad
 \hat{\zeta}_{z}=\left(x_{\xi} y_\eta-x_\eta y_{\xi}\right).
\end{aligned}
\end{equation}

The Jacobian is computed using Eqn. \ref{Jac-matrix}. To estimate the derivative terms involved in the above equations, the fourth order explicit central scheme is used \cite{lele1992compact,Visbal2002}. For instance, to estimate $x_{\xi}$, 

\begin{equation} \label{4thorder-grads}
    (x_{\xi})_i = \frac{\frac{2}{3}(x_{i+1}-x_{i-1}) - \frac{1}{12}(x_{i+2}-x_{i-2})}{\Delta \xi}
\end{equation}

The interpolation of metrics to cell interfaces is performed using the same formulae listed in Eqn. \ref{interp-formulae}, except the derivative terms involved in it are computed using the above mentioned fourth order central scheme.


\section{Improved MP-limiter} \label{app:MP-limit}

A modified version of the MP-limiter as described in Ref.\cite{chamx} is presented here. After evaluating the unlimited upwind interpolated values of the characteristic variables ($\mathbf{W}_{i+1/2, unlimited}^{L,R}$) in Step-3 described in Sec. \ref{Inv-disc}, the solution is limited in Step-4. For brevity the procedure is only described for the left interpolated value here.

\noindent Check the following condition to determine if a discontinuity exists:
\begin{equation}
    \left(\mathbf{W}_{i+1 / 2}^{L}-\hat{\mathbf{W}}_i\right)\left(\mathbf{W}_{i+1 / 2}^{L}-\mathbf{W}^{M P}\right) \leq 10^{-20}
\end{equation}

where,
\begin{equation}
    \begin{gathered}
\mathbf{W}^{M P}=\hat{\mathbf{W}}_i+\operatorname{minmod}\left[\hat{\mathbf{W}}_{i+1}-\hat{\mathbf{W}}_i, \beta\left(\hat{\mathbf{W}}_i-\hat{\mathbf{W}}_{i-1}\right)\right], \\
\text { and, } \operatorname{minmod}(a, b)=\frac{1}{2}(\operatorname{sign}(a)+\operatorname{sign}(b)) \min (|a|,|b|) .
\end{gathered}
\end{equation}


Compute $\mathbf{W}_{i+\frac{1}{2}}^{\min }$ and $\mathbf{W}_{i+\frac{1}{2}}^{\max }$ as follows:
\begin{equation}
    \begin{aligned}
& d_i=2\left(\hat{\mathbf{W}}_{i+1}-2 \hat{\mathbf{W}}_i+\hat{\mathbf{W}}_{i-1}\right)-0.5 \Delta x\left(\mathbf{W}_{i+1}^{\prime}-\mathbf{W}_{i-1}^{\prime}\right) \\
& d_{i+1 / 2}^M=\operatorname{minmod}\left(d_i, d_{i+1}\right) \\
& \mathbf{W}_{i+\frac{1}{2}}^{M D}=\frac{1}{2}\left(\hat{\mathbf{W}}_i+\hat{\mathbf{W}}_{i+1}\right)-\frac{1}{2} d_{i+\frac{1}{2}}^M \\
& \mathbf{W}_{i+\frac{1}{2}}^{U L}=\hat{\mathbf{W}}_i+\alpha\left(\hat{\mathbf{W}}_i-\hat{\mathbf{W}}_{i-1}\right) \\
& \mathbf{W}_{i+\frac{1}{2}}^{L C}=\frac{1}{2}\left(3 \hat{\mathbf{W}}_i-\hat{\mathbf{W}}_{i-1}\right)+\frac{4}{3} d_{i-\frac{1}{2}}^M \\
& \mathbf{W}_{i+\frac{1}{2}}^{\min }=\max \left[\min \left(\hat{\mathbf{W}}_i, \hat{\mathbf{W}}_{i+1}, \mathbf{W}_{i+\frac{1}{2}}^{M D}\right), \min \left(\hat{\mathbf{W}}_i, \mathbf{W}_{i+1 / 2}^{U L}, \mathbf{W}_{i+1 / 2}^{L C}\right)\right] \\
& \mathbf{W}_{i+\frac{1}{2}}^{\max }=\min \left[\max \left(\hat{\mathbf{W}}_i, \hat{\mathbf{W}}_{i+1}, \mathbf{W}_{i+\frac{1}{2}}^{M D}\right), \max \left(\hat{\mathbf{W}}_i, \mathbf{W}_{i+\frac{1}{2}}^{U L}, \mathbf{W}_{i+\frac{1}{2}}^{L C}\right)\right] \\
&
\end{aligned}
\end{equation}

Finally, the limited value is computed.
\begin{equation}
    \mathbf{W}_{i+\frac{1}{2}}^{\mathrm{limited}}=\mathbf{W}_{i+\frac{1}{2}}^{L}+\operatorname{minmod}\left(\mathbf{W}_{i+\frac{1}{2}}^{\min }-\mathbf{W}_{i+\frac{1}{2}}^{L}, \mathbf{W}_{i+\frac{1}{2}}^{\max }-\mathbf{W}_{i+\frac{1}{2}}^{L}\right).
\end{equation}

