\section{Multi-block approach and boundary conditions} \label{sec:mul-block}
\begin{figure}[h!]
    \centering
    \includegraphics[width=150mm]{Images/multi-block.pdf}
    \caption{(a) Schematic of a block and its ghost points on left, right, bottom and top boundaries. (b) Schematic of ghost point sharing between blocks sharing interface along $\xi$ direction. The letter `G' in figure-b represents ghost cell region.}
    \label{multi-block}
\end{figure}

The multi-block approach enables geometric flexibility to simulate flows through and around domains of various atypical shapes. The multi-block approach adopted in the present work is similar to the approach proposed by Lien et al. \cite{lien1996multiblock}. Any block in the domain can interface one or more of it's boundaries (six in 3-D, four in 2-D) with the blocks surrounding it. Each block possesses its own coordinate system, essentially making the approach block unstructured. To ensure consistency of the block dimensions at the boundaries, the dimensions of the block interface plane should be identical on both home and neighbor blocks. A schematic of a multi-block layout depicting the arrangement of cells and the coordinate systems is illustrated in  Fig. \ref{multi-block}a. Ghost cells are adapted to introduce necessary inter-block information sharing and enable sufficient stencil size at block boundaries to compute high-order fluxes corresponding to MEG6, MIG4, and $\alpha$-damping.

Boundary conditions are imposed at each block boundary through ghost cells. The geometric location of the ghost points is arbitrary as they are physically non-existent. Therefore, metric terms are also non-existent and do not need to be computed in those locations. However, the lack of geometric information at boundaries poses difficulties in applying the usual high order approximations that are employed for cells inside the domain. \ref{app:one-sided} shows the special boundary formulae that were adopted in the present work.

A schematic of a two-dimensional block and the imaginary ghost cells surrounding it's four boundaries is illustrated in Fig. \ref{multi-block}. A total of five ghost cells were used for all the simulations in the current study. Suitable primitive variable values are specified in those ghost cells based on the type of boundary condition adapted at that boundary. A list of expressions to compute the primitive variables in ghost cells corresponding to various boundary conditions is provided in Table \ref{BCs}. All the formulations and concepts discussed in this section can be extrapolated to three dimensions.

\begin{table}[h!]
    \centering
    \includegraphics[width=\textwidth]{Images/BCs.pdf}
    \caption{Expressions for primitive variable values in the ghost cells for various boundary conditions. Where, $N_g$ is number of ghost cells, prefix $(.)_{in}$ represents inflow conditions at the inflow boundary, $NI$ is number of points in $xi$ direction, $U$ is the contravariant velocity in $\xi$ direction, $V$ is the contravariant velocity in $\eta$ direction, Suffix `$H$' and `$N$' denote `Home' and `Neighbour' blocks respectively.}
    \label{BCs}
\end{table}