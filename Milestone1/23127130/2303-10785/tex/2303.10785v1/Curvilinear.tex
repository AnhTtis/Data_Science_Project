\documentclass[a4paper,10pt]{elsarticle}
%\documentclass[preprint,10pt,1p]{elsarticle}
%\documentclass[times,final]{elsarticle}
%% Stylefile to load JCOMP template
%\documentclass[12pt]{elsarticle}
\usepackage{jcomp}
\usepackage{framed,multirow}
\usepackage[utf8]{inputenc}
\usepackage{bm}
\usepackage{amsmath}
\usepackage{mathrsfs}
\usepackage{graphicx}
\usepackage{epsfig, setspace}
\usepackage{url}
\usepackage{graphicx, transparent, color}
\usepackage{algorithm} 
\usepackage{algorithmicx}
\usepackage{etex}
\usepackage{dirtytalk}
\usepackage[mathscr]{euscript}
%\reserveinserts{28}
\usepackage[bookmarks=true]{hyperref}
 %contains integral mean symbol
%\usepackage[margin=3cm]{geometry}
%\usepackage{caption}
%\usetikzlibrary{shapes,calc}
%\usepackage{verbatim}
\usepackage{amsmath}
\usepackage{xcolor, soul}
\usepackage{rotating} % Rotating the figures but keeps the page in portrait mode
\usepackage{pdflscape} % Rotates single page into Landscape mode
\usepackage{silence}
\usepackage{ulem,cancel}
\newtheorem{remark}{\bf Remark}[section]
\newtheorem{example}{Example}[section]
\usepackage{graphicx}
\usepackage{float}
\usepackage{subfigure}% subcaptions for subfigures
\usepackage{subfigmat}% matrices of similar subfigures,
\DeclareMathOperator{\sign}{sign}
\newcommand{\mathd}{\mathrm{d}}
\newcommand{\tmem}[1]{{\em #1\/}}
\usepackage{tikz}
\usepackage{capt-of}
%\journal{Journal of Computational Physics}
\usepackage{setspace,lipsum}
\usepackage{lineno}
\usepackage{gensymb}
\usepackage{amssymb}
\biboptions{sort&compress}
%\renewcommand*{\bibfont}{\footnotesize}
\def\bibfont{\footnotesize}
% \linenumbers
\newcommand{\half}{\frac{1}{2}}
\usepackage{setspace,lipsum}
\usepackage{listings}
\usepackage{xcolor}
\definecolor{aquamarine}{rgb}{0.5, 1.0, 0.83}
\definecolor{OliveGreen}{rgb}{0,0.6,0}
\usepackage[bottom]{footmisc}

\sethlcolor{aquamarine}
%\usepackage[top=18truemm,bottom=18truemm,left=20truemm,right=20truemm]{geometry}%Margin adjustment

\definecolor{codegreen}{rgb}{0,0.6,0}
\definecolor{codegray}{rgb}{0.5,0.5,0.5}
\definecolor{codepurple}{rgb}{0.58,0,0.82}
\definecolor{backcolour}{rgb}{0.95,0.95,0.92}

\lstdefinestyle{mystyle}{
    backgroundcolor=\color{backcolour},   
    commentstyle=\color{codegreen},
    keywordstyle=\color{magenta},
    numberstyle=\tiny\color{codegray},
    stringstyle=\color{codepurple},
    basicstyle=\ttfamily\footnotesize,
    breakatwhitespace=false,         
    breaklines=true,                 
    captionpos=b,                    
    keepspaces=true,                 
    numbers=left,                    
    numbersep=5pt,                  
    showspaces=false,                
    showstringspaces=false,
    showtabs=false,                  
    tabsize=2
}

\lstset{style=mystyle}

\setlength{\skip\footins}{-0.2cm}
% \addtolength\footskip{-1cm}

\begin{document}

\begin{frontmatter}
% \title{A GPU accelerated curvilinear and multi-block gradient-based Navier-Stokes solver and its application in predicting supersonic jet screech}
% Towards GPU Accelerated Flow Simulations on Generalized Curvilinear and Dynamic Mesh Domains using Gradinet Based Reconstruction
% Gradient Based High-Accuracy Conservative Finite-Difference Method for Flow Simulations on Generalized Curvilinear Dynamic Meshes
% On the Application of Gradient Based Reconstruction for Flow Simulations on Generalized Curvilinear and Dynamic Mesh Domains
% Application of Gradient Based Reconstruction for Solving Compressible Navier-Stokes Equations on Generalized Curvilinear and Dynamic Mesh Domains
% Extension of Gradient Based High-Accuracy Conservative Finite-Difference Method for Compressible Flow Simulations on Generalized Curvilinear Dynamic Meshes
\title{On the Application of Gradient Based Reconstruction for Flow Simulations on Generalized Curvilinear and Dynamic Mesh Domains}


\author[]{Hemanth Chandravamsi\footnote{Corresponding author, hemanthgrylls@gmail.com}}
% \author[]{Hemanth Chandravamsi\thanks{Corresponding author, hemanthgrylls@gmail.com}}
\author[]{Amareshwara Sainadh Chamarthi}
\author[]{Natan Hoffmann}
\author[]{Steven H.\ Frankel}
\address{Faculty of Mechanical Engineering, Technion - Israel Institute of Technology, Haifa, Israel}

% in the interest of simulating practically relevant high-speed flow problems such as supersonic jet noise

\begin{abstract}
% Accurate simulation of high-speed flows of practical interest require numerical methods with high-resolution properties. In this paper, we primarily focus on extending and demonstrating the use of the new high-accuracy \say{Gradient-based reconstruction} and \say{$\alpha$-damping} schemes introduced by Chamarthi (2022) \cite{chamx} to generalized curvilinear and dynamic mesh domains with freestream preservation property. The paper is organized into three parts. The first part focuses on the implementation details of Gradient-based reconstruction and $\alpha$-damping schemes. Various steps involved in achieving freestream and vortex preservation properties are addressed through demonstrative test cases. The second part presents the algorithm's efficacy in simulating supersonic jet screech. It was noted that both the implicit and explicit versions of the algorithm are able to resolve the screech tones and characterize the unsteady lateral flapping mode of a Mach $1.35$ under-expanded supersonic jet with good fidelity. On the other hand WENO-Z scheme was noted to fail in resolve the same at the same grid resolution. The third part discusses the GPU parallelization model employed and the corresponding performance statistics achieved on the latest generations of data center GPU hardware. With more than two hundred times speedup compared to a single-core CPU, a turnaround time of 34.5 hrs per simulation was achieved with GPU acceleration for the supersonic jet noise simulation case ran at 13 million grid cell count. 

Accurate high-speed flow simulations of practical interest require numerical methods with high-resolution properties. In this paper, we present an extension and demonstration of the high-accuracy Gradient-based reconstruction and $\alpha$-damping schemes introduced by Chamarthi (2022) \cite{chamx} for simulating high-speed flows in generalized curvilinear and dynamic mesh domains with the freestream preservation property. In the first part of this paper, the algorithms are detailed within the generalized curvilinear coordinate framework, with a focus on demonstration through stationary and dynamic mesh test cases. It has been shown both theoretically and through the use of test cases that the conservative metrics, including their interpolation to cell interfaces, must be numerically computed using a central scheme that is consistent with the inviscid flux algorithm to achieve the freestream preservation property. The second part of the paper illustrates the efficacy of the algorithm in simulating supersonic jet screech by displaying its capability to capture the screech tones and accurately characterize the unsteady lateral flapping mode of a Mach 1.35 under-expanded supersonic jet, in contrast to the WENO-Z scheme which fails to do so at the same grid resolution. In the final part of the paper, the parallelizability of the schemes on GPU architectures is demonstrated and performance metrics are evaluated. A significant speedup of over $200 \times$ (compared to a single core CPU) and a reduction in simulation completion time to 34.5 hours per simulation were achieved for the supersonic jet noise case at a grid resolution of 13 million cells.


% The jet oscillation behaviour and azimuthal variation of screech amplitude is also investigated.
% The key features of the methodology include high-resolution fourth-order inviscid and viscous flux discretization, freestream preservation, dynamic mesh motion, and GPU acceleration.
\end{abstract}

\begin{keyword}
Curvilinear coordinates, Gradient based reconstruction, $\alpha$-damping, Freestream preservation, Dynamic mesh, GPU acceleration
\end{keyword}

\end{frontmatter}



%%%%%%%% Part-1 %%%%%%%% 
\section{Introduction}
\label{sec:introduction}
% \begin{itemize}
%     % Diffusion of FL
%     \item {\st{Diffusion of FL}}
%     % Security threats to FL
%     \item {\st{Security threats to FL with particular focus on model poisoning}}
%     % Limitations of existing countermeasures
%     \item {\st{Current countermeasures (e.g., KRUM) and their limitations}}
%     % Proposed method and its advantages
%     \item {\st{Intuitive description of the proposed method and its difference (i.e., advantages) w.r.t. state of the art}}
%     % Main contributions
%     \item {\st{Summary of the main contributions of this work}}
%     % Paper's structure and organization
%     \item {\st{Paper's structure and organization}}
% \end{itemize}

% Diffusion of FL
Recently, {\em federated learning} (FL) has emerged as the leading paradigm for training distributed, large-scale, and privacy-preserving machine learning (ML) systems~\cite{mcmahan2017googleai,mcmahan2017aistats}. 
The core idea of FL is to allow multiple edge clients to collaboratively train a shared, global model without disclosing their local private training data.
%Specifically, an FL system consists of a central server and many edge clients; 
A typical FL round involves the following steps: {\em(i)} the server randomly picks some clients and sends them the current, global model; {\em(ii)} each selected client locally trains its model with its own private data; then, it sends the resulting local model to the server;\footnote{Whenever we refer to global/local model, we mean global/local model {\em parameters}.} {\em(iii)} the server updates the global model by computing an \emph{aggregation function}, usually the average (FedAvg), on the local models received from clients.
% \begin{enumerate}
%     \item[{\em(i)}] the server sends the current, global model to the clients and appoints some of them for training;
%     \item[{\em(ii)}] each selected client locally trains its copy of the global model with its own private data; then, it sends the resulting local model back to the server;\footnote{Whenever we refer to global/local model, we mean global/local model {\em parameters}.}
%     \item[{\em(iii)}] the server updates the global model by computing an \emph{aggregation function} on the local models received from clients (by default, the average, also referred to as FedAvg~\cite{mcmahan2017aistats}).
% \end{enumerate}
This process goes on until the global model converges. %(e.g., after a certain number of rounds or other similar stopping criteria).
%\\
% The advantages of FL over the traditional, centralized learning paradigm are undoubtedly clear in terms of flexibility/scalability (clients can join/disconnect from the FL network dynamically), network communications (only model weights\footnote{We will use \textit{parameters} and \textit{weights} interchangeably.} are exchanged between clients and server), and privacy (each client's private training data is kept local at the client's end and not uploaded to the server).
\\
% Security threats to FL
%However, the growing adoption of FL also raises security concerns~\cite{costa2022covert}, particularly about its confidentiality, integrity, and availability.
Although its advantages over standard ML, FL also raises security concerns~\cite{costa2022covert}. %, particularly about its confidentiality, integrity, and availability~\cite{costa2022covert}.
% OLD, LONG VERSION
% Indeed, some work deals with privacy leakage that may expose the local data of some clients~\cite{melis2019sp}. 
% A large body of work, instead, investigates attacks that usually aim to detriment the predictive accuracy of the learned global model. For instance, \emph{data poisoning} attacks achieve this goal by letting an adversary pollute the training set of some corrupt FL clients with maliciously crafted examples~\cite{jagielski2018sp}.
% Similarly, in \emph{model poisoning} the attacker attempts to tweak the global model weights~\cite{bhagoji2019pmlr} by directly perturbing the local model's weights of some infected FL clients before these are sent to the central server for aggregation, usually via so-called Byzantine attacks. 
% It turns out that Byzantine model poisoning attacks severely impact standard FedAvg; therefore, more robust aggregation functions must be designed to make FL systems secure.
Here, we focus on \emph{untargeted model poisoning} attacks~\cite{bhagoji2019pmlr}, where an adversary attempts to tweak the global model weights %\footnote{We will use the terms \textit{parameters} and \textit{weights} interchangeably.} 
by directly perturbing the local model's parameters of some infected clients before these are sent to the central server for aggregation.
In doing so, the adversary aims to jeopardize the global model \textit{indiscriminately} at inference time.
Such model poisoning attacks severely impact standard FedAvg; therefore, more robust aggregation functions must be designed to secure FL systems.
\\
% In this paper, we focus on designing a novel robust aggregation scheme at the server's end to contrast the effect of Byzantine model poisoning attacks.
%
% Current countermeasures and their limitations
%Several countermeasures have been proposed in the literature to combat model poisoning attacks on FL systems.
% Some methods use simple statistics more robust than plain average to smooth the impact of malicious updates (e.g., Trimmed Mean and FedMedian~\cite{yin2018icml}). 
% Other defenses implement outlier detection techniques to discard malicious updates from the aggregation performed at the server's end. Those are either based on heuristics (e.g., Krum/Multi-Krum~\cite{blanchard2017nips} and Bulyan~\cite{mhamdi2018pmlr}) or data-driven approaches (e.g., K-means clustering~\cite{shen2016acm} or DnC via spectral analysis~\cite{shejwalkar2021ndss}). 
% Finally, some strategies rely on a centralized ``source of trust'' to spot potential malicious updates (e.g., FLTrust~\cite{cao2020fltrust}).
% Several countermeasures have been proposed in the literature to combat model poisoning attacks on FL systems, i.e., to discard possible malicious local updates from the aggregation performed at the server's end. 
% These techniques range from simple statistics more robust than plain average (e.g., Trimmed Mean and FedMedian~\cite{yin2018icml}) to outlier detection heuristics (e.g., Krum/Multi-Krum~\cite{blanchard2017nips} and Bulyan~\cite{mhamdi2018pmlr}) or data-driven approaches (e.g., spectral analysis via K-means clustering~\cite{shen2016acm} or spectral analysis), or methods based on ``source of trust'' (e.g., FLTrust~\cite{cao2020fltrust}).
% OLD, LONG VERSION
%Several countermeasures have been proposed in the literature to combat Byzantine model poisoning attacks on FL systems.
% Descriptive statistics
% For example, Trimmed Mean and FedMedian aggregate local model updates using more robust statistics than standard average~\cite{yin2018icml}.
%
% % Heuristics for outlier detection
% Many existing Byzantine-resilient strategies implement some outlier detection heuristics to discard the model updates sent by potentially malicious clients from the input of the aggregation function.
% One of the most popular heuristics is Krum~\cite{blanchard2017nips}.
% This strategy tries to mitigate the impact of Byzantine attacks by selecting as a global model the local model with the smallest sum of Euclidean distances to {\em all} the other local models.
% Although powerful, Krum requires the server to know (or, at least, estimate) the number of malicious FL clients upfront, which is generally impossible in a realistic attack scenario. %
% Moreover, Krum may become ineffective for complex, high-dimensional model parameter spaces due to the curse of dimensionality.
% Bulyan~\cite{mhamdi2018pmlr} tries to overcome this issue by combining Krum with a variant of Trimmed Mean.
% % Data-driven outlier detection
% Other strategies use data-driven outlier detection techniques -- e.g., via K-means clustering~\cite{shen2016acm} -- to spot potential malicious local model updates. 
% %For instance, Shen et al. propose to cluster local model updates with K-means and thus identify outliers.
%
% % Other techniques
% As far as the server is concerned, any local model received can be from a potential malicious client. 
% FLTrust~\cite{cao2020fltrust} assumes the server acts as a client, i.e., trains a local model on an additional {\em trustworthy} dataset at the server's end and compares it against all the local models from other clients. 
% This way, the server can rely on some ``source of trust'' when discarding potentially malicious clients.
%\\
% Limitations of existing Byzantine-resilient strategies
Unfortunately, existing defense mechanisms either rely on simple heuristics (e.g., Trimmed Mean and FedMedian by~\cite{yin2018icml}) or need strong and unrealistic assumptions to work effectively (e.g., foreknowledge or estimation of the number of malicious clients in the FL system, as for Krum/Multi-Krum~\cite{blanchard2017nips} and Bulyan~\cite{mhamdi2018pmlr}, which, however, cannot exceed a fixed threshold).
Furthermore, outlier detection methods using K-means clustering~\cite{shen2016acm} or spectral analysis like DnC~\cite{shejwalkar2021ndss} do not directly consider the temporal evolution of local model updates received.
Finally, strategies like FLTrust~\cite{cao2020fltrust} require the server to collect its own dataset and act as a proper client, thereby altering the standard FL protocol.
\\
% OLD, LONG VERSION
% Overall, existing Byzantine-resilient strategies are either simple heuristics (e.g., FedMedian) or, if they are more complex, they rely on strong and unrealistic assumptions to work effectively (e.g., knowing the number of malicious clients in the FL system in advance, as for Krum and alike).
% Furthermore, data-driven outlier detection methods do not consider the temporary evolution of local model updates received (e.g., K-means clustering). 
% Finally, strategies like FLTrust requires the server to collect its own dataset and act as a proper client, thereby altering the standard FL protocol.
%
% Description of the proposed method
This work introduces a novel pre-aggregation \textit{filter} robust to untargeted model poisoning attacks. Notably, this filter $(i)$ operates without requiring prior knowledge or constraints on the number of malicious clients and $(ii)$ inherently integrates temporal dependencies. 
The FL server can employ this filter as a preprocessing step before applying \textit{any} aggregation function, be it standard like FedAvg or robust like Krum or Bulyan.
Specifically, we formulate the problem of identifying corrupted updates as a multidimensional (i.e., matrix-valued) time series anomaly detection task. 
The key idea is that legitimate local updates, resulting from well-calibrated iterative procedures like stochastic gradient descent (SGD) with an appropriate learning rate, show \textit{higher predictability} compared to malicious updates. This hypothesis stems from the fact that the sequence of gradients (thus, model parameters) observed during legitimate training exhibit regular patterns, as validated in Section~\ref{subsec:intuition}. %until convergence. 
%This regularity may be more pronounced for smooth convex loss functions, but it can still be captured within an appropriate time window, even for more complex and convoluted loss surfaces. 
%We provide evidence of this claim in Appendix~B, where we show that the average mutual information (i.e., ``predictability''), calculated over pairs of legitimate model updates sent at different FL rounds, is significantly higher than the corresponding computation for a malicious client.
\\
Inspired by the matrix autoregressive (MAR) framework for multidimensional time series forecasting~\cite{chen2021je}, we propose the FLANDERS ({\em \textbf{F}ederated \textbf{L}earning meets \textbf{AN}omaly \textbf{DE}tection for a \textbf{R}obust and \textbf{S}ecure}) filter.
The main advantages of FLANDERS over existing strategies like FLDetector~\cite{zhao2020multivariate} are its resilience to large-scale attacks, where $50\%$ or more FL participants are hostile, and the capability of working under realistic non-iid scenarios.
We attribute such a capability to two key factors: $(i)$ FLANDERS works without knowing a priori the ratio of corrupted clients, and $(ii)$ it embodies temporal dependencies between intra- and inter-client updates, quickly recognizing local model drifts caused by evil players. Below, we summarize our main contributions:

\begin{itemize}
\item[{\em(i)}]
We provide empirical evidence that the sequence of models sent by legitimate clients is more predictable than those of malicious participants performing untargeted model poisoning attacks.
\\
\item[{\em(ii)}] 
We introduce FLANDERS, the first pre-aggregation filter for FL robust to untargeted model poisoning based on multidimensional time series anomaly detection.
\\
\item[{\em(iii)}] 
We integrate FLANDERS into Flower,\footnote{\scriptsize{\url{https://flower.dev/}}} a popular FL simulation framework for reproducibility.
\\
\item[{\em(iv)}] 
We show that FLANDERS improves the robustness of the existing aggregation methods under multiple settings: different datasets, client's data distribution (non-iid), models, and attack scenarios.
\\
\item[{\em(v)}] 
We publicly release all the implementation code of FLANDERS along with our experiments.\footnote{\scriptsize{\url{https://anonymous.4open.science/r/flanders_exp-7EEB}}}
\end{itemize}

% Paper's structure and organization
The remainder of the paper is structured as follows. %some related work and the current state-of-the-art solutions to security issues that FL entails. 
Section~\ref{sec:background} covers background and preliminaries. 
In Section~\ref{sec:related}, we discuss related work.
Section~\ref{sec:problem} and Section~\ref{sec:method} describe the problem formulation and the method proposed. % to tackle it. 
Section~\ref{sec:experiments} gathers experimental results. %, and Section~\ref{sec:limitations} discusses some limitations of this work.
Finally, we conclude in Section~\ref{sec:conclusion}.
 %discusses the limitations of this work and draws future research directions.
%reports conclusions and draws perspectives for future research directions.

%%%%%%% OLD %%%%%%%
%to overcome the resilience of Byzantine failures in distributed Stochastic Gradient Descent computations. 
% The strength of Krum is its time complexity, which is linear in the gradient dimension. 
% However, the robustness of the approach is guaranteed for gradient-based learning applications only when the majority of the clients are not compromised. 
% Besides, the aggregation mechanism of Krum, as well as that of similar methods, is robust from a coarse-grained perspective and does not provide solutions to errors and perturbations that may occur at inference time.
%A related approach to~\cite{blanchard2017nips} is the work of Su et al.~\cite{su2016dc}. Here, the authors propose an iterated approximate agreement to tackle a multi-layer scenario attacked by Byzantine agents. 
%However, the method works efficiently on the sole discrete context and it is inapplicable to continuous state environments.
%\gabri{Maybe, we should just talk about the main limitations of existing countermeasures without digging into their details (or, we can just mention Krum as this is the most popular one). I will move the description of all these methods to the Related Work section.}
\section{Governing equations}   \label{sec:gov-eqns}
The unsteady three-dimensional compressible Navier-Stokes equations (dimensional) in a generalized curvilinear coordinate system with $\xi$, $\eta$, and $\zeta$ as the spatial coordinate directions and $t$ as the time can be written in the following vector form:
\begin{equation} \label{trans-eqn}
    \frac{\partial}{\partial t}\left(\frac{\mathbf{Q}}{J}\right)+\frac{\partial \mathbf{\hat{F}}}{\partial \xi}+\frac{\partial \mathbf{\hat{G}}}{\partial \eta}+\frac{\partial \mathbf{\hat{H}}}{\partial \zeta}=\frac{\partial \mathbf{\hat{F}_{v}}}{\partial \xi}+\frac{\partial \mathbf{\hat{G}_{v}}}{\partial \eta}+\frac{\partial \mathbf{\hat{H}_{v}}}{\partial \zeta},
\end{equation}\label{NS_TC}

\noindent where $\mathbf{Q}$ represents the vector of conservative variables, i.e. $\mathbf{Q}=[\rho, \rho u, \rho v, \rho w, E]^T$. The vectors of inviscid fluxes ($\mathbf{\hat{F}}$, $\mathbf{\hat{G}}$ and $\mathbf{\hat{H}}$) and viscous fluxes ($\mathbf{\hat{F}^{v}}$, $\mathbf{\hat{G}^{v}}$ and $\mathbf{\hat{H}^{v}}$) are,

\begin{subequations} \label{inv_fluxes}
    \begin{gather}
    \mathbf{\hat{F}}=\left[\begin{array}{c}
    \rho \hat{U} \\
    \rho u \hat{U}+\hat{\xi}_{x} p \\
    \rho v \hat{U}+\hat{\xi}_{y} p \\
    \rho w \hat{U}+\hat{\xi}_{z} p \\
    (E+p) \hat{U} -\hat{\xi}_{t} p
    \end{array}\right], \quad \mathbf{\hat{G}}=\left[\begin{array}{c}
    \rho \hat{V} \\
    \rho u \hat{V}+\hat{\eta}_{x} p \\
    \rho v \hat{V}+\hat{\eta}_{y} p \\
    \rho w \hat{V}+\hat{\eta}_{z} p \\
    (E+p) \hat{V} -\hat{\eta}_{t} p
    \end{array}\right], \quad \mathbf{\hat{H}}=\left[\begin{array}{c}
    \rho \hat{W} \\
    \rho u \hat{W}+\hat{\zeta}_{x} p \\
    \rho v \hat{W}+\hat{\zeta}_{y} p \\
    \rho w \hat{W}+\hat{\zeta}_{z} p \\
    (E+p) \hat{W} -\hat{\zeta}_{t} p
    \end{array}\right].
    \tag{\theequation a-\theequation c}
    \end{gather}
\end{subequations}

\begin{subequations}\label{visc-curvi}
    \begin{gather}
        \mathbf{\hat{F}^{v}}=\left[\begin{array}{c}
    0 \\
    \hat{\xi}_{x} \tau_{x x}+\hat{\xi}_{y} \tau_{x y}+\hat{\xi}_{z} \tau_{x z} \\
    \hat{\xi}_{x} \tau_{y x}+\hat{\xi}_{y} \tau_{y y}+\hat{\xi}_{z} \tau_{y z} \\
    \hat{\xi}_{x} \tau_{z x}+\hat{\xi}_{y} \tau_{z y}+\hat{\xi}_{z} \tau_{z z} \\
    \hat{\xi}_{x} \beta_{x}+\hat{\xi}_{y} \beta_{y}+\hat{\xi}_{z} \beta_{z}
    \end{array}\right], 
        \tag{\theequation a-\theequation b}
        \quad
         \mathbf{\hat{G}^{v}}=\left[\begin{array}{c}
        0 \\
        \hat{\eta}_{x} \tau_{x x}+\hat{\eta}_{y} \tau_{x y}+\hat{\eta}_{z} \tau_{x z} \\
        \hat{\eta}_{x} \tau_{y x}+\hat{\eta}_{y} \tau_{y y}+\hat{\eta}_{z} \tau_{y z} \\
        \hat{\eta}_{x} \tau_{z x}+\hat{\eta}_{y} \tau_{z y}+\hat{\eta}_{z} \tau_{z z} \\
        \hat{\eta}_{x} \beta_{x}+\hat{\eta}_{y} \beta_{y}+\hat{\eta}_{z} \beta_{z}
        \end{array}\right], \\
       \mathbf{\hat{H}^{v}}=\left[\begin{array}{c}
    0 \\
    \hat{\zeta}_{x} \tau_{x x}+\hat{\zeta}_{y} \tau_{x y}+\hat{\zeta}_{z} \tau_{x z} \\
    \hat{\zeta}_{x} \tau_{y x}+\hat{\zeta}_{y} \tau_{y y}+\hat{\zeta}_{z} \tau_{y z} \\
    \hat{\zeta}_{x} \tau_{z x}+\hat{\zeta}_{y} \tau_{z y}+\hat{\zeta}_{z} \tau_{z z} \\
    \hat{\zeta}_{x} \beta_{x} +\hat{\zeta}_{y} \beta_{y} +\hat{\zeta}_{z} \beta_{z}
    \end{array}\right]. \tag{\theequation c}
    \end{gather}
\end{subequations}

The quantity `$J$' here represents the Jacobian of grid transformation. The metric terms in the inviscid and viscous flux terms in Eqns.(\ref{inv_fluxes},\ref{visc-curvi}) with a hat $\hat{(\cdot)}$ denote Jacobian normalized quantities (e.g. $\hat{\xi}_x = \xi_x/J$).\\

\noindent The contravariant velocities $\hat{U}$, $\hat{V}$, and $\hat{W}$ are,

\begin{subequations}
\begin{gather}
    \hat{U}=\hat{\xi}_{t} + \hat{\xi}_{x} u+\hat{\xi}_{y} v+\hat{\xi}_{z} w, \quad
    \hat{V}=\hat{\eta}_{t} +\hat{\eta}_{x} u+\hat{\eta}_{y} v+\hat{\eta}_{z} w, \quad
    \hat{W}=\hat{\zeta}_{t} + \hat{\zeta}_{x} u+\hat{\zeta}_{y} v+\hat{\zeta}_{z} w \tag{\theequation a-\theequation c}
\end{gather}
\end{subequations}

\noindent The energy per unit volume $E$ is,
\begin{equation}
E = \frac{p}{\gamma-1} + \frac{1}{2} \rho (u^2 + v^2 + w^2)
\end{equation}

For stationary meshes, the computation of the terms $\hat{\xi}_{t}$, $\hat{\eta}_{t}$, and $\hat{\zeta}_{t}$, can be skipped as they are zero. The viscous and thermal stress related terms in the equations are defined as,
% \begin{equation*}
% \tau_{x x}=\hat{\lambda} \nabla \cdot \bar{u}+2 \hat{\mu} \frac{\partial u}{\partial x}, \quad \tau_{y y}=\hat{\lambda} \nabla \cdot \bar{u}+2 \hat{\mu} \frac{\partial v}{\partial y}, \quad \tau_{z z}=\hat{\lambda} \nabla \cdot \bar{u}+2 \hat{\mu} \frac{\partial w}{\partial z}
% \end{equation*}

\begin{subequations} \label{eqn:shear-str}
\begin{gather}
\tau_{x x}= \lambda \nabla \cdot \bar{u}+2 \mu \frac{\partial u}{\partial x}, \quad
\tau_{y y}= \lambda \nabla \cdot \bar{u}+2 \mu \frac{\partial v}{\partial y}, \quad
\tau_{z z}= \lambda \nabla \cdot \bar{u}+2 \mu \frac{\partial w}{\partial z},  \tag{\theequation a-\theequation c}\\
\tau_{x y}=\tau_{y x}=\mu\left(\frac{\partial u}{\partial y}+\frac{\partial v}{\partial x}\right), \quad
\tau_{x z}=\tau_{z x}=\mu\left(\frac{\partial u}{\partial z}+\frac{\partial w}{\partial x}\right), \quad 
\tau_{y z}=\tau_{z y}=\mu\left(\frac{\partial v}{\partial z}+\frac{\partial w}{\partial y}\right) \tag{\theequation d-\theequation f}
\end{gather}
\end{subequations}

\begin{subequations} \label{eqn:thermal-work}
    \begin{gather}
    \beta_{x}=u \tau_{x x}+v \tau_{x y}+w \tau_{x z}+ \kappa_t \frac{\partial T}{\partial x}, \quad
    \beta_{y}=u \tau_{y x}+v \tau_{y y}+w \tau_{y z}+ \kappa_t \frac{\partial T}{\partial y}, \quad
    \beta_{z}=u \tau_{z x}+v \tau_{z y}+w \tau_{z z}+ \kappa_t \frac{\partial T}{\partial z} \tag{\theequation a-\theequation c}
    \end{gather}
\end{subequations}

\noindent Where $\bar{u}$ in the normal stress terms denotes the velocity vector. The thermodynamic equation of state $p=\rho RT$ is used to close the governing equations. $\mu$ and $\kappa_t$ denote the fluid's dynamic viscosity and thermal conductivity, respectively, which are temperature-dependent properties. The dynamic viscosity is computed using Sutherland's law \cite{Sutherland1893} while thermal conductivity is calculated using the Prandtl number ($Pr$) and dynamic viscosity. Additionally, bulk viscosity $\lambda=-\frac{2}{3}\mu$ is incorporated into the normal stress terms $\tau_{xx}$, $\tau_{yy}$, and $\tau_{zz}$ based on Stokes' hypothesis. The values of $\gamma$ and $Pr$ are set to 1.4 and 0.71, respectively, for air as the working fluid. Although the governing equations are presented in their dimensional form here, they are solved in their non-dimensional form to facilitate problem parameterization and minimize computational errors. The ambient speed of sound, density, temperature, viscosity, and nozzle exit diameter are used as reference quantities for scaling the flow variables.

% \noindent The equation of state $p= \rho RT$ is used to close the system of equations. The physical quantities: density, velocity, pressure, temperature, dynamic viscosity, length scales and time scales are non dimensionalized with the corresponding reference quantities: $\rho_{\infty}$, $U_{\infty}$, $\rho_{\infty} U_{\infty}^2$, $T_{\infty}$, $\mu_{\infty}$, $L_{ref}$ and $L_{ref}U_{\infty}^{-1}$ respectively. As a result, the Reynolds number $Re=\rho_{\infty} u_{\infty} L_{ref} \mu_{\infty}^{-1}$ and Mach number $M=u_{\infty}(\gamma R_{gas} T_{\infty})^{-1/2}$ appear as parameters in the scaled diffusion coefficients: $\hat{\mu} = \frac{\mu}{\mathrm{Re}}$, $\hat{\lambda} = -\frac{2}{3} \hat{\mu}$ and $\hat{\kappa} = \hat{\mu}(\mathrm{M}^2(\gamma-1)\mathrm{Pr})^{-1}$ in the above equations. $Pr$ stands for the Prandtl number and is taken to be $0.71$ in the present work.

% The simulations presented in the paper utilize Newton's law of viscosity and assumes dynamic viscosity ($\mu$) to vary with temperature according to Sutherland's law \cite{Sutherland1893}. The bulk viscosity $\lambda$ is added in to the normal stress terms $\tau_{xx}$, $\tau_{yy}$ and $\tau_{zz}$ in accordance with Stokes' hypothesis.\\

\section{Review of conservative finite difference framework, and an overview of the algorithm} \label{sec:coner-FD+algo}

The current work employs MEG6, MIG4, and $\alpha$-damping schemes \cite{chamx}, which use a conservative finite difference approach to discretize the spatial derivative terms. Like most numerical schemes, these schemes are derived based on uniform stencils. Applying these schemes to non-uniform stencils (in curved/stretched grids) requires a recast of the Cartesian form of governing equations into generalized curvilinear coordinates form. The transformed equations are presented in Section \ref{sec:gov-eqns}. Such a coordinate transformation allows non-uniform skewed input cells of the mesh to be re-stretched into uniform cubical cells in the computational space, as shown in Fig. \ref{FD_grid}(a). Based on this idea, MEG6, MIG4, and $\alpha$-damping schemes are implemented in the solver to compute flow fields over non-uniform meshes.

\begin{figure}[h!]
    \centering
    \includegraphics[width=140mm]{Images/CFDM.pdf}
    \caption{(a) A 3-dimensional grid cell before and after coordinate transformation, (b) A two-dimensional $\xi-\eta$ plane of the computational domain depicting, cell centers, interface centers (located at the centroid of cell interface), and interface fluxes.}
    \label{FD_grid}
\end{figure}

The association between grid, solution, and conservative fluxes is as follows. The input mesh consists of hexahedral cells, identified by the coordinates of cell vertices provided in the grid file. A schematic of the two-dimensional solution domain in transformed coordinates is depicted in Fig. \ref{FD_grid}(b). The figure shows cell centers, interface centers (half locations), and fluxes. The solution $\textbf{Q}$ (conservative variables) is stored at the cell centers, which are located by the centroid of corresponding cell vertices. On the other hand, inviscid and diffusion fluxes ($\mathbf{F_{i+1/2}}, \mathbf{G_{i+1/2}}$ etc.) are computed at cell-interface centers to preserve conservation. The current paper also refers to these interface centers as half locations. After computing the inviscid and diffusion fluxes at all six half locations corresponding to the cell, the right-hand-side (RHS) residual corresponding to that cell denoted by $\mathbf{Res}_{i,j,k}$ is computed as follows:

\begin{equation}
 \frac{\partial}{\partial t}\left(\frac{\mathbf{Q}}{J}\right)=\mathbf{Res}_{i, j, k}=-\frac{\partial \left(\hat{\mathbf{F}} - \hat{\mathbf{F}}^{\mathrm{v}}\right)}{\partial \xi} -\frac{\partial \left(\hat{\mathbf{G}} - \hat{\mathbf{G}}^{\mathrm{v}}\right)}{\partial \eta} -\frac{\partial \left(\hat{\mathbf{H}} - \hat{\mathbf{H}}^{\mathrm{v}}\right)}{\partial \zeta}
\end{equation}

\begin{equation}
    \begin{aligned}
 \frac{\partial \left(\hat{\mathbf{F}} - \hat{\mathbf{F}}^{\mathrm{v}}\right)}{\partial \xi} = &\frac{1}{\Delta \xi}\left[\left(\hat{\mathbf{F}}_{i+\frac{1}{2}, j, k}-\hat{\mathbf{F}}_{i-\frac{1}{2}, j, k}\right)-\left(\hat{\mathbf{F}}_{i+\frac{1}{2}, j, k}^{\mathrm{v}}-\hat{\mathbf{F}}_{i-\frac{1}{2}, j, k}^{\mathrm{v}}\right)\right] \\
\frac{\partial \left(\hat{\mathbf{G}} - \hat{\mathbf{G}}^{\mathrm{v}}\right)}{\partial \eta} = &\frac{1}{\Delta \eta}\left[\left(\hat{\mathbf{G}}_{i, j+\frac{1}{2}, k}-\hat{\mathbf{G}}_{i, j-\frac{1}{2}, k}\right)-\left(\hat{\mathbf{G}}_{i, j+\frac{1}{2}, k}^{\mathrm{v}}-\hat{\mathbf{G}}_{i, j-\frac{1}{2}, k}^{\mathrm{v}}\right)\right] \\
\frac{\partial \left(\hat{\mathbf{H}} - \hat{\mathbf{H}}^{\mathrm{v}}\right)}{\partial \zeta} = &\frac{1}{\Delta \zeta}\left[\left(\hat{\mathbf{H}}_{i, j, k+\frac{1}{2}}-\hat{\mathbf{H}}_{i, j, k-\frac{1}{2}}\right)-\left(\hat{\mathbf{H}}_{i, j, k+\frac{1}{2}}^{\mathrm{v}}-\hat{\mathbf{H}}_{i, j, k-\frac{1}{2}}^{\mathrm{v}}\right)\right].
\end{aligned}
\end{equation}

The above formulation is conservative throughout the computational domain since the residuals are computed based on fluxes at cell interfaces (half-locations). Even though the fluxes are estimated at the cell interfaces, the process involved in computing these fluxes ($\hat{\mathbf{F}}, \hat{\mathbf{G}}, \hat{\mathbf{H}}, \hat{\mathbf{F}}^{\mathrm{v}}, \hat{\mathbf{G}}^{\mathrm{v}}, \hat{\mathbf{H}}^{\mathrm{v}}$) only requires the cell-center solution information and uses finite difference approximations for reconstruction. Hence the approach is classified as a conservative finite difference approach. 

Fig. \ref{algo} shows a flow chart view of key stages involved in the algorithm. The program starts with pre-processing steps and ends with post-processing. Before entering into the main time-loop, the necessary information required to execute the main time-loop is transferred to GPU memory, which will remain and update there without any communication with the CPU until the simulation ends at time $t_{end}$. This allows the computationally expensive parts of the program such as reconstruction, Riemann solvers, viscous fluxes, residuals, and time integration to be executed solely on GPUs. More details about the parallelization model employed and the achieved performance increase will be discussed in Section \ref{sec:gpu-accel}.\\


\begin{figure}[h!]
    \centering
    \includegraphics[width=130mm]{Images/algo.pdf}
    \caption{Flow chart of solver algorithm. The stages outlined in the green boxes are executed fully on the GPU.}
    \label{algo}
\end{figure}

One of the key features of the present Navier-Stokes algorithm is gradient sharing. The gradients of primitive variables are utilized in multiple stages of the algorithm, enhancing efficiency and solution accuracy. This includes the use of gradients in inviscid fluxes (via gradient-based reconstruction - Sec. \ref{Inv-disc}) and viscous flux discretization (via $\alpha$-damping approach - Sec. \ref{sec:viscDisc}), as well as the improvement of shock-capturing through the improved MP-limiter. Additionally, the gradients of primitive variables are used in post-processing calculations such as enstrophy, vorticity, density gradient magnitudes, Q-criterion, and others. As a result, the approach is efficient and produces relatively accurate and well resolved flow solutions.


\subsection{DARPA Program Metrics} \label{sec_metrics}

The methods in this paper, developed with support and oversight of the DARPA Computable Models Disruption Opportunity \cite{DARPACOMPMods}, demonstrate several measurable advancements over the state-of-the-art. Here, we summarize them in terms of the relevant program metrics, i.e., modeling accuracy and numerical efficiency. As previously discussed, upscaling theory by multiple scale expansions ensures that the modeling error of coarse-grained approximations is \emph{a priori} bounded under appropriate dynamic conditions expressed ontherms of dimensionless numbers. When such conditions are locally (in space and/or time violated), it is therefore important that any further strategy (numerical or analytical) that aims at coupling fine-scale models with their continuum-scale counterpart in the same simulation domain be bounded by the oforementioned upscaling error. In this regard, the accuracy of any proposed hybrid scheme can be directly assessed against such an a priori error. In Sections~\ref{sec:acc-hc} and~\ref{sec:xhc-acc}, we show that both coupling schemes satisfy the requested accuracy. An additional important metric is that the computation cost associated with the iterative coupling between fine- and coarse-scale models does not overcome the cost of full fine-scale simulations over the microscopic domain (here considered the benchmark for both accuracy and cost). In Section~\ref{sec:efficiency}, we provide both an extensive analysis of the cost-accuracy tradeoffs as well as guidelines for the efficient adoption of hybrid algorithms in large-scale domains.
\section{Discretization}    \label{sec:disc}

In this section the discretization approaches used to compute inviscid and viscous fluxes are detailed. The MEG6/MIG4 algorithm of Chamarthi \cite{chamx}, the HLLC Riemann solver, and the $\alpha$-damping approach are explained in the context of generalized curvilinear coordinates with applicability to both stationary and moving grids.

\subsection{Inviscid flux discretization - MEG6 and MIG4 schemes}    \label{Inv-disc}
Monotonocity preserving Explicit sixth order Gradient scheme (MEG6) and the Monotonocity Preserving Implicit 4th order Gradient scheme (MIG4) proposed by Chamarthi \cite{chamx} are a class of gradient-based algorithms to discretize inviscid flux terms in the compressible Navier-Stokes equations. They are primarily driven by the gradients of primitive variables ($\mathbf{P}_{\xi},\mathbf{P}_{\eta}$, and $\mathbf{P}_{\zeta}$) and the Monotonicity Preserving limiter of Suresh and Huynh \cite{suresh1997accurate} for shock capturing. For interpolating the variables to the half locations, the following general form containing the functional values and their corresponding first and second derivatives at cell centers is employed \cite{van1977towards}:

\begin{equation}
   \phi(\xi)=\phi_i+\left( \frac{\partial \phi}{\partial \xi} \right)_i \left(\xi-\xi_i\right)+3 \kappa \left( \frac{\partial^2 \phi}{\partial \xi^2} \right)_i\left[\left(\xi-\xi_i\right)^2-\frac{\Delta \xi_i^2}{12}\right] \quad \text{for} \quad \xi_{i-\frac{1}{2}} \le \xi \le \xi_{i+\frac{1}{2}}
\end{equation}

where $\phi$ is the variable of interest to be interpolated and $\kappa=\frac{1}{3}$ \cite{van1977towards}. Since the above formulation contains functional values and its derivatives that are belonging to location $i$, by substituting $\xi = \pm \frac{\Delta \xi}{2}$ the left ($L$) and right ($R$) biased interpolation formulae for $\phi$ at locations $i+\frac{1}{2}$ and $i-\frac{1}{2}$ are obtained. Simplifying the resulting equations with $\Delta \xi =1$ (since unit grid spacing convention is followed for transformed coordinates) yields the following left and right biased states of $\phi$ at $i+\frac{1}{2}$ location.

\begin{equation} \label{vanleer-poly}
    \begin{aligned}
&\phi_{i+\frac{1}{2}}^L= \phi_i+\frac{1}{2} \left( \frac{\partial \phi}{\partial \xi} \right)_i+\frac{1}{12} \left( \frac{\partial^2 \phi}{\partial \xi^2} \right)_i \\
&\phi_{i+\frac{1}{2}}^R= \phi_{i+1}-\frac{1}{2} \left( \frac{\partial \phi}{\partial \xi} \right)_{i+1}+\frac{1}{12} \left( \frac{\partial^2 \phi}{\partial \xi^2} \right)_{i+1}
\end{aligned}
\end{equation}

The above two polynomials are used to reconstruct the variables in MEG6 and MIG4 schemes. The two methods MEG6 and MIG4 differ from each other only in the scheme that is used to compute the derivatives of primitive variables present in the above equations (Eqn. \ref{vanleer-poly}); the rest of the algorithm (which will be detailed soon) remains the same. Despite the different gradients employed, both MEG6 and MIG4 schemes yield fourth-order accurate estimates of inviscid flux derivatives $\frac{\partial \mathbf{\hat{F}}}{\partial \xi}, \frac{\partial \mathbf{\hat{G}}}{\partial \eta}$, and $\frac{\partial \mathbf{\hat{H}}}{\partial \zeta}$. The theoretical details regarding order of accuracy can be found in Ref. \cite{chamx}. The schemes however differ in their dispersion and dissipation properties. Readers are referred to Fig. 3 of Ref. \cite{chamx} which shows the spectral properties of MEG6 and MIG4 schemes in comparison with other well known discretization approaches.

Fig. \ref{inv_algo} depicts a flow chart view of different steps in computing the inviscid flux residual at half locations. A detailed step-by-step procedure is shown. For simplicity, the procedure is described for $\mathbf{F}_{i+\frac{1}{2}}$ fluxes only. Extending the same to other directions is straightforward. Also, for the sake of simplicity the `$j$' and `$k$' subscripts corresponding to $\eta$ and $\zeta$ directions are omitted in all the expressions to be presented now; the formulations remain the same and are independent of `$j$' and `$k$' indices. \\

% MEG6 uses the standard sixth-order explicit gradients computed using Eqn. \ref{E6_grads}. On the other hand, MIG4 uses a fourth-order implicit scheme as per Eqn. \ref{IG4_grads} with $\alpha=\frac{5}{14}$.

\begin{figure}[h!]
    \centering
    \includegraphics[width=140mm]{Images/inviscid_algo.pdf}
    \caption{Various stages while estimating inviscid flux residual using the MEG/MIG schemes.}
    \label{inv_algo}
\end{figure}

\noindent \textbf{Step-1, Compute gradients:} Compute first and second gradients of primitive variables $\mathbf{P}=[\rho,u,v,w,p]^T$ at cell centers in the computational coordinate system using the following expressions:

\begin{equation}\label{E6_grads}
    \text{For MEG6:}\quad \left(\frac{\partial u}{\partial \xi}\right)_i = \frac{3}{4}\left( u_{i+1} + u_{i-1}\right) - \frac{3}{20}\left(u_{i+2} + u_{i-2}  \right) + \frac{1}{60}\left(u_{i+3} + u_{i-3}  \right)
\end{equation}

\begin{equation}\label{IG4_grads}
    \text{For MIG4:}\quad \alpha \left(\frac{\partial u}{\partial \xi}\right)_{i-1} + \left(\frac{\partial u}{\partial \xi}\right)_{i} + \alpha \left(\frac{\partial u}{\partial \xi}\right)_{i+1} = \frac{2 (2 + \alpha)}{6}\left( u_{i+1} + u_{i-1}\right) + \frac{-1+4 \alpha}{12}\left(u_{i+2} + u_{i-2} \right)
\end{equation}

\begin{equation}
    \text{For both MEG6 and MIG4:}\quad \left(\frac{\partial^{2} u}{\partial \xi^{2}}\right)_{i}=2\left(\hat{u}_{i+1}-2 \hat{u}_{i}+\hat{u}_{i-1}\right)-0.5\left(u_{i+1}^{\prime}-u_{i-1}^{\prime}\right)
\end{equation}

$\alpha=\frac{5}{14}$ is considered for the implicit scheme in Eqn.\ref{IG4_grads}. The above mentioned explicit sixth-order (E6) gradients are used for the MEG6 scheme, and implicit gradients of fourth-order (IG4) are used for the MIG4 scheme.\\

\noindent \textbf{Step-2, Perform characteristic transformation:} The cell center primitive variables $\mathbf{P}=[\rho,u,v,w,p]^T$, their first derivatives, $\frac{\partial \mathbf{P}}{\partial \xi}$ and second derivatives, $\frac{\partial^2 \mathbf{P}}{\partial \xi^2}$ are transformed into characteristic space $\mathbf{W}$, $\frac{\partial \mathbf{W}}{\partial \xi}$ and, $\frac{\partial^2 \mathbf{W}}{\partial \xi^2}$ by multiplying them with the left-eigenvectors of the flux Jacobian matrix, $\frac{\partial \mathbf{F}}{\partial \mathbf{P}}$, using the following relations Eqn. \ref{forward-projection}. Reconstructing the resulting characteristic variables will result in cleaner results without any oscillations near the discontinuities. This is since the Euler equations resemble linear wave equation in characteristic form for which the up-winding in the Riemann solver is designed for (will be discussed in Sec. \ref{sec:riemann-solver}).

\begin{subequations} \label{forward-projection}
    \begin{gather}
    \mathbf{W} = \overline{\mathbf{R}}_{\xi}^{-1} \mathbf{P}, \quad
    \frac{\partial \mathbf{W}}{\partial \xi}     = \overline{\mathbf{R}}_{\xi}^{-1}  \frac{\partial \mathbf{P}}{\partial \xi}, \quad \text{and }
    \frac{\partial^2 \mathbf{W}}{\partial \xi^2} = \overline{\mathbf{R}}_{\xi}^{-1} \frac{\partial^2 \mathbf{P}}{\partial \xi^2}
    \tag{\theequation a-\theequation c}
    \end{gather}
\end{subequations}

\noindent The eigenvector matrices employed in these equations are provided in \ref{app:Eig-structure}.\\

\noindent \textbf{Step-3, Compute left and right biased states of $\mathbf{W}$:} Evaluate the unlimited left and right biased states of characteristic variables at half locations employing Eqns. \ref{vanleer-poly}. Fig. \ref{recon} illustrates the location of these quantities on the interpolation stencil.

\begin{eqnarray} \label{legendre_polys}
    \left(\mathbf{W}_{i+1/2}^L\right)_{unlimited} = \mathbf{W}_{i} + \frac{1}{2} \left(\frac{\partial \mathbf{W}}{\partial \xi}\right)_{i} + \frac{1}{12} \left(\frac{\partial^2 \mathbf{W}}{\partial \xi^2}\right)_{i} \\
    \left(\mathbf{W}_{i+1/2}^R\right)_{unlimited} = \mathbf{W}_{i+1} - \frac{1}{2} \left(\frac{\partial \mathbf{W}}{\partial \xi}\right)_{i+1} + \frac{1}{12} \left(\frac{\partial^2 \mathbf{W}}{\partial \xi^2}\right)_{i+1} 
\end{eqnarray}

\begin{figure}[h!]
    \centering
    \includegraphics[width=50mm]{Images/reconstruct.pdf}
    \caption{An illustration depicting left and right biased states of $W$ on the grid stencil.}
    \label{recon}
\end{figure}

\noindent \textbf{Step-4, Modify $\mathbf{W}_{i+1/2}^{L,R}$ through limiting:} An improved version of the Monotonicity Preserving fifth order limiting algorithm (MP5) \cite{suresh1997accurate} (see \ref{app:MP-limit}) proposed in Ref. \cite{chamx} is employed to check for discontinuities and limit the left and right reconstructed states of the characteristic variables ($\mathbf{W}_{i+1/2}^{L,R}$). This step essentially captures flow discontinuities by limiting the interpolated unlimited value from the previous step and avoids unphysical oscillations. One of the key favourable attributes of the MP5 limiter is that it allows an effective capturing of shock and acoustic waves simultaneously compared to the WENO based strategies \cite{Shu1997}. The limiting procedure is detailed in \ref{app:MP-limit}.

% The limiting procedure is detailed in Ref. \cite{chamx} and is being skipped here to avoid repetition and maintain brevity.\\

\noindent \textbf{Step-5, Map characteristic variables back to primitive variables:} The reconstructed left and right characteristic variable states computed in the previous step are now transformed back to primitive variables by multiplying them with the right-eigenvector matrices provided in \ref{app:Eig-structure}. 

\begin{subequations} \label{rev-projection}
    \begin{gather}
    (\mathbf{P}^L)_{i+1/2} = \overline{\mathbf{R}}_{\xi} (\mathbf{W}^L_{i+1/2})_{limited} \\
    (\mathbf{P}^R)_{i+1/2} = \overline{\mathbf{R}}_{\xi} (\mathbf{W}^R_{i+1/2})_{limited}
    \end{gather}
\end{subequations}

\noindent \textbf{Step-6, Compute interface fluxes:} Now that the flow states on both sides of the interface are known, the mathematical setting at each $i+\frac{1}{2}$ location reduces to a Riemann problem. In this step, the local one-dimensional Riemann problem is solved, and the resultant flux $\textbf{F}_{i+\frac{1}{2}}$ is estimated at each interface. The HLLC approximate riemann solver (Harten–Lax–van Leer with Contact restoration) \cite{toro2009riemann} is employed to estimate the fluxes in the present work. As shown in Eqn. \ref{hllc-function}, the only inputs required in order to compute the fluxes using the HLLC approach are the local left and right reconstructed states of the primitive variables and the interpolated metric terms at the $i+\frac{1}{2}$ locations. The required relations to compute the fluxes via the HLLC Riemann solver in the context of curvilinear coordinates is provided in a dedicated section on Riemann solvers in Section \ref{sec:riemann-solver}.

\begin{equation} \label{hllc-function}
    \hat{\mathbf{F}}_{i+\frac{1}{2}} = \mathbf{f}_{HLLC} \left\{ \mathbf{P}^L_{i+\frac{1}{2}}, \mathbf{P}^R_{i+\frac{1}{2}}, (\hat{\xi}_t)_{i+\frac{1}{2}}, (\hat{\xi}_x)_{i+\frac{1}{2}}, (\hat{\xi}_y)_{i+\frac{1}{2}}, (\hat{\xi}_z)_{i+\frac{1}{2}} \right\}
\end{equation}

% \begin{equation}
%     (\mathbf{P}^L)_{i+1/2}, (\mathbf{P}^R)_{i+1/2}  \xrightarrow{\quad \text{Riemann solver (section \ref{sec:riemann-solver})} \quad} \boldsymbol{F}_{i+\frac{1}{2}}
% \end{equation} \\

% Based on the local wave speeds, the Riemann solver essentially estimates some kind of upwinded solution at each interface.

% The reconstructed left and right biased primitive variables are sent to the HLLC Riemann solver routine to compute the final upwinded fluxes at the interfaces.

\noindent \textbf{Step-7, Compute residual:} Using the interface fluxes estimated through the Riemann solver, the net flux entering into each cell is computed via Eqn. \ref{res-inv}. It should be noted that if the flow problem is linear in nature, the accuracy of this term theoretically yields fourth-order using both E6 and IG4 gradients in step-1 (Eqns. \ref{E6_grads} and \ref{IG4_grads}). More theoretical details including demonstrative tests regarding the order of accuracy can be found in \cite{chamx}. \\

\begin{equation} \label{res-inv}
    \left(\frac{\partial \boldsymbol{F}}{\partial \xi}\right)_i = \frac{ \boldsymbol{F}_{i+\frac{1}{2}}- \boldsymbol{F}_{i-\frac{1}{2}}}{\Delta \xi}
\end{equation}

% Contrary to the conventional LES where the effect of sub-grid-scale (SGS) flow features is approximated using an explicit SGS model (that adds dissipation to filter out the large eddies) \cite{mathew2003explicit}, the present algorithm solves the unfiltered Navier-Stokes equations directly and utilizes the numerical dissipation of the MEG6 and MIG4 schemes to implicitly mimic the effect of SGS models. Such an approach is popularly termed `implicit-LES' or specifically in the present case `Mononically integrated Implicit LES (MILES)' \cite{fureby1999monotonically}, since the MP5 algorithm adapted in the MEG6/MIG4 schemes is high resolution and monotonocity preserving.


% The current approach is featured by high-resolution solutions that are also stable and monotonically converge with grid refinement (this will be demonstrated in section \ref{sec:JNapplication}), as a result, the authors believe that the current approach possess the required capabilities to be treated as an implicit-LES flow solver. This argument is further strengthened in section \ref{sec:results}, by comparing and validating the computed flow solutions with the experimental measurements.

\subsubsection{The HLLC Riemann solver} \label{sec:riemann-solver}
In this section, the HLLC Riemann solver is presented within the framework of curvilinear coordinates. The HLLC Riemann solver is a three-wave approximate solution that accounts for the presence of contact discontinuities. One of the notable advantages of the HLLC fluxes is their ability to minimize dissipation in comparison to other commonly used methods, such as HLL \cite{harten1983upstream} and Rusanov \cite{toro2009riemann}. As a result, HLLC is able to effectively resolve shocks, contact discontinuities, and shear layers. The inviscid flux $\mathbf{F}_{i+\frac{1}{2}}$ in the transformed coordinates through the HLLC approximate Riemann solver is obtained as follows:


% To the best of the authors knowledge, the formulation for the HLLC Riemann solver in in three-dimensional coordinates is not presented in the literature so far.
% After obtaining the left ($L$) and right ($R$) reconstructed states of the primitive variables at each cell interface (after step-5 in Fig. \ref{inv_algo}), the next task is to evaluate the convection fluxes $\mathbf{F}_{i+\frac{1}{2}}, \mathbf{G}_{i+\frac{1}{2}}$, and $\mathbf{H}_{i+\frac{1}{2}}$. 

\begin{equation}
\hat{\mathbf{F}}_{i+\frac{1}{2}}^{\mathrm{HLLC}}=\left\{\begin{array}{ll}
\mathbf{\hat{F}}^{\mathbf{L}} & \text { if } \quad \hat{S}^{L} \geq 0, \\
\mathbf{\hat{F}}^{* \mathbf{L}}=\mathbf{\hat{F}}^{\mathbf{L}}+\hat{S}^{L}\left(\mathbf{Q}^{* \mathbf{L}}-\mathbf{Q}^{\mathbf{L}}\right) & \text { if } \quad \hat{S}^{L} \leq 0 \leq \hat{S}^{*}, \\
\mathbf{\hat{F}}^{* \mathbf{R}}=\mathbf{\hat{F}}^{\mathbf{R}}+\hat{S}^{R}\left(\mathbf{Q}^{* \mathbf{R}}-\mathbf{Q}^{\mathbf{R}}\right) & \text { if } \quad \hat{S}^{*} \leq 0 \leq \hat{S}^{R}, \\
\mathbf{\hat{F}}^{\mathbf{R}} & \text { if } \quad \hat{S}^{R} \leq 0
\end{array}\right.
\end{equation}

The three wave speeds of the local one-dimensional problem are calculated as follows.
\begin{equation*}
    \hat{S}^{L}=\operatorname{Min}\left(\hat{U}^{L}-c^{L} \sqrt{\hat{\xi}_{x}^{2}+\hat{\xi}_{y}^{2}+\hat{\xi}_{z}^{2}}, \quad \hat{U}^{R}-c^{R} \sqrt{\hat{\xi}_{x}^{2}+\hat{\xi}_{y}^{2}+\hat{\xi}_{z}^{2}}\right) 
\end{equation*}

\begin{equation*}
    \hat{S}^{R}=\operatorname{Max}\left(\hat{U}^{L}+c^{L} \sqrt{\hat{\xi}_{x}^{2}+\hat{\xi}_{y}^{2}+\hat{\xi}_{z}^{2}}, \quad \hat{U}^{R}+c^{R} \sqrt{\hat{\xi}_{x}^{2}+\hat{\xi}_{y}^{2}+\hat{\xi}_{z}^{2}}\right)
\end{equation*}

\begin{equation*}
    \hat{S}^{*}=\frac{\rho^{R} \hat{U}^{R}\left(\hat{S}^{R}-\hat{U}^{R}\right)-\rho^{L} \hat{U}^{L}\left(\hat{S}^{L}-\hat{U}^{L}\right)+\left(P_{L}-P_{R}\right)\left(\hat{\xi}_{x}^{2}+\hat{\xi}_{y}^{2}+\hat{\xi}_{z}^{2}\right)}{\rho^{R}\left(\hat{S}^{R}-\hat{U}^{R}\right)-\rho^{L}\left(\hat{S}^{L}-\hat{U}^{L}\right)}
\end{equation*} 

The conservative variables in the star region $\mathbf{Q}^{* \mathbf{K}}$ for $\mathbf{K} = L,R$ are estimated as follows.
% \begin{equation}
% \mathbf{Q}^{* \mathbf{K}}=\rho^{\mathbf{K}}\left(\frac{\hat{S}^{\mathbf{K}}-\hat{U}^{\mathbf{K}}}{\hat{S}^{\mathbf{K}}-\hat{S}^{*}}\right)\left[\begin{array}{c}
% 1 \\
% \frac{\hat{\xi}_{x}\hat{S}^{*}+(\hat{\xi}_{y}^{2}+\hat{\xi}_{z}^{2}) u^{\mathbf{K}}-\hat{\xi}_{y} \hat{\xi}_{x} v^{\mathbf{K}}-\hat{\xi}_{z} \hat{\xi}_{x}w^{\mathbf{K}}}{\hat{\xi}_{x}^{2}+\hat{\xi}_{y}^{2}+\hat{\xi}_{z}^{2}} \\
% \frac{\hat{\xi}_{y}\hat{S}^{*}-\hat{\xi}_{x} \hat{\xi}_{y} u^{\mathbf{K}}+(\hat{\xi}_{x}^{2}+\hat{\xi}_{z}^{2}) v^{\mathbf{K}}-\hat{\xi}_{z} \hat{\xi}_{y}w^{\mathbf{K}}}{\hat{\xi}_{x}^{2}+\hat{\xi}_{y}^{2}+\hat{\xi}_{z}^{2}} \\
% \frac{\hat{\xi}_{z}\hat{S}^{*}-\hat{\xi}_{x} \hat{\xi}_{z} u^{\mathbf{K}}-\hat{\xi}_{y} \hat{\xi}_{z} v^{\mathbf{K}}+(\hat{\xi}_{x}^{2}+\hat{\xi}_{y}^{2})w^{\mathbf{K}}}{\hat{\xi}_{x}^{2}+\hat{\xi}_{y}^{2}+\hat{\xi}_{z}^{2}} \\
% \frac{E^{\mathbf{K}}}{\rho^{\mathbf{K}}}+\left(\hat{S}^{*}-\hat{U}^{\mathbf{K}}\right)\left\{\frac{\hat{S}^{*}}{\left(\hat{\xi}_{x}^{2}+\hat{\xi}_{y}^{2}+\hat{\xi}_{z}^{2}\right)}+\frac{p^{\mathbf{K}}}{\rho^{\mathbf{K}}(\hat{S}^{\mathbf{K}}-\hat{U}^{\mathbf{K}})}\right\}
% \end{array}\right]
% \end{equation}

\begin{equation}
\mathbf{Q}^{* \mathbf{K}}=\rho^{\mathbf{K}}\left(\frac{\hat{S}^{\mathbf{K}}-\hat{U}^{\mathbf{K}}}{\hat{S}^{\mathbf{K}}-\hat{S}^{*}}\right)\begin{bmatrix}
    1 \\
\frac{\hat{\xi}_{x}(\hat{S}^{*}-\hat{\xi}_{t})+(\hat{\xi}_{y}^{2}+\hat{\xi}_{z}^{2}) u^{\mathbf{K}}-\hat{\xi}_{y} \hat{\xi}_{x} v^{\mathbf{K}}-\hat{\xi}_{z} \hat{\xi}_{x}w^{\mathbf{K}}}{\hat{\xi}_{x}^{2}+\hat{\xi}_{y}^{2}+\hat{\xi}_{z}^{2}} \\
\frac{\hat{\xi}_{y}(\hat{S}^{*}-\hat{\xi}_{t})-\hat{\xi}_{x} \hat{\xi}_{y} u^{\mathbf{K}}+(\hat{\xi}_{x}^{2}+\hat{\xi}_{z}^{2}) v^{\mathbf{K}}-\hat{\xi}_{z} \hat{\xi}_{y}w^{\mathbf{K}}}{\hat{\xi}_{x}^{2}+\hat{\xi}_{y}^{2}+\hat{\xi}_{z}^{2}} \\
\frac{\hat{\xi}_{z}(\hat{S}^{*}-\hat{\xi}_{t})-\hat{\xi}_{x} \hat{\xi}_{z} u^{\mathbf{K}}-\hat{\xi}_{y} \hat{\xi}_{z} v^{\mathbf{K}}+(\hat{\xi}_{x}^{2}+\hat{\xi}_{y}^{2})w^{\mathbf{K}}}{\hat{\xi}_{x}^{2}+\hat{\xi}_{y}^{2}+\hat{\xi}_{z}^{2}} \\
\frac{E^{\mathbf{K}}}{\rho^{\mathbf{K}}}+\left(\hat{S}^{*}-\hat{U}^{\mathbf{K}}\right)\left\{\frac{\hat{S}^{*}-\hat{\xi}_{t}}{\left(\hat{\xi}_{x}^{2}+\hat{\xi}_{y}^{2}+\hat{\xi}_{z}^{2}\right)}+\frac{p^{\mathbf{K}}}{\rho^{\mathbf{K}}(\hat{S}^{\mathbf{K}}-\hat{U}^{\mathbf{K}})}\right\}
  \end{bmatrix}
\end{equation}

The $i+\frac{1}{2}$ metrics involved in the above formulation are interpolated consistently with the inviscid flux discretization to prevent metric cancellation errors, which can accumulate in the flow solution. The corresponding interpolation formulae and demonstrative examples concerning metric cancellation errors and their effect on the flow solution will be discussed in Section \ref{sec:FP}. The above formulations only show the flux calculation in the $\xi$ direction. Extending these formulae to other directions is straightforward. For instance, 

\begin{equation}
    \hat{\mathbf{G}}_{i,j+\frac{1}{2},k} = \mathbf{f}_{HLLC} \left\{ \mathbf{P}^L_{i,j+\frac{1}{2},k}, \mathbf{P}^R_{i,j+\frac{1}{2},k}, (\hat{\eta}_t)_{i,j+\frac{1}{2},k}, (\hat{\eta}_x)_{i,j+\frac{1}{2},k}, (\hat{\eta}_y)_{i,j+\frac{1}{2},k}, (\hat{\eta}_z)_{i,j+\frac{1}{2},k} \right\}.
\end{equation}

% It should be noted that all the quantities in the above formulae with a hat on the top $\hat{(\cdot)}$ represent that they are normalized with local Jacobian. 
\subsection{Viscous flux discretization - $\alpha$-damping scheme}  \label{sec:viscDisc}

Many viscous schemes such as those proposed in \cite{Visbal2002,shen2010large} are prone to odd-even decoupling and inaccurate spectral representation particularly in the high wave-number region. In the recent past, efforts have been made in the development of numerical algorithms to compute diffusion fluxes without odd-even decoupling and good spectral properties over a broader wavenumber range \cite{Nishikawa2013, chamarthi2022importance,sainadh2022spectral}. The current work employs the fourth order accurate version of viscous fluxes proposed by Chamarthi et al. \cite{chamarthi2022importance}. The gradients required for the viscous fluxes are not re-computed here since they are already computed while estimating inviscid flux residuals (Eqns-\ref{E6_grads} and \ref{IG4_grads}). A step-by-step procedure to evaluate the interface viscous fluxes according to this algorithm in curvilinear coordinates is depicted in Fig. \ref{visc_algo}. For simplicity, the steps are described only for the $\xi$ direction but the same can be extrapolated to other directions, as well. \\

\begin{figure}[h!]
    \centering
    \includegraphics[width=140mm]{Images/visc_algo.pdf}
    \caption{Various stages involved in estimating the viscous flux residual using the $\alpha$-damping scheme. Applies for both E6 and IG4 gradients of $\bm{\mathit{\mathscr{P}}}$, yielding 4th order accuracy for both \cite{chamx}.}
    \label{visc_algo}
\end{figure}

\noindent \textbf{Step-1, Compute interface gradients:} Interface gradients of $\bm{\mathit{\mathscr{P}}} = [u,v,w,T]^{T}$ in $\xi$ $\eta$ and $\zeta$ directions are first computed using Eqn. \ref{AD-eqn1}, where $\bm{\mathit{\mathscr{P}}}$ is a vector of variables required to evaluate the viscous fluxes. For simplicity, the equations are presented only for the variable $u$ here. The same should be repeated for other variables of $\bm{\mathit{\mathscr{P}}}$ as well.

\begin{equation} \label{AD-eqn1}
    \left(\frac{\partial u}{\partial \xi}\right)_{i+\frac{1}{2}}=\underbrace{\frac{1}{2}\left[\left(\frac{\partial u}{\partial \xi}\right)_{i}+\left(\frac{\partial u}{\partial \xi}\right)_{i+1}\right]}_{\text {Consistent term }}+\underbrace{\frac{\alpha^D}{2}\left(u_{R}-u_{L}\right)}_{\text {Damping term }},
\end{equation}

Where the left and right states of $u$ are defined using the following interpolation polynomial:
\begin{eqnarray} \label{AD-eqn2}
u_{L}=u_{i}+ 0.5 \left(\frac{\partial u}{\partial \xi}\right)_{i}+\beta^D\left(u_{i+1}-2 u_{i}+u_{i-1}\right) \\
\quad u_{R}=u_{i+1}- 0.5 \left(\frac{\partial u}{\partial \xi}\right)_{i+1}+\beta^D\left(u_{i+2}-2 u_{i+1}+u_{i}\right) .
\end{eqnarray}

$\alpha^D=4$ and $\beta^D=0$ are the scheme coefficients based on Ref. \cite{chamx} for both E6 and IG4 based gradients of $\bm{\mathit{\mathscr{P}}}$. These coefficients produce fourth order accuracy for both E6 and IG4 gradients. The temperature at the $i+\frac{1}{2}$ location is computed from an arithmetic average of $T_L$ and $T_R$ values. From this, the interface dynamic viscosity $\mu_{i+\frac{1}{2}}$ is estimated using the Sutherland's law of viscosity. \\

\noindent \textbf{Step-2, Compute the derivatives in Cartesian coordinates:} Using the $\xi$, $\eta$ and $\zeta$ direction gradients of $[u,v,w,T]$ computed in the previous step, the derivative chain rule is used to evaluate the $x$, $y$, and $z$ derivatives of the same variables. For instance, this is done as follows to compute the interface $x$ direction derivative:

\begin{equation} \label{AD-eqn3}
   \left(\frac{\partial u}{\partial x}\right)_{i+1 / 2} = (\xi_{x})_{i+\frac{1}{2}} \left(\frac{\partial u}{\partial \xi}\right)_{i+\frac{1}{2}} + (\eta_{x})_{i+\frac{1}{2}} \left(\frac{\partial u}{\partial \eta}\right)_{i+\frac{1}{2}} + (\zeta_{x})_{i+\frac{1}{2}} \left(\frac{\partial u}{\partial \zeta}\right)_{i+\frac{1}{2}}
\end{equation}

\noindent Note: Metric values at $i+\frac{1}{2}$ location are interpolated from the cell center metrics using the relations provided in Eqns. \ref{FP-interp} and \ref{FP-interp2}. \\

\noindent \textbf{Step-3, Compute stress terms:} The shear stress and energy diffusion terms $\tau_{ij}$, $\beta_i$ present in the viscous flux vector are computed at $i+\frac{1}{2}$ using Eqns. \ref{eqn:shear-str} and \ref{eqn:thermal-work} provided in Section \ref{sec:gov-eqns} and with the interface gradients computed in the previous step. The required interface velocities are computed via an arithmetic average of the left and right states of the velocities (e.g. $u_{i+\frac{1}{2}} = 0.5(u_L + u_R)$). The shear-stress term $\tau_{xx}$ at $i+\frac{1}{2}$ is evaluated as follows:

\begin{equation}
    (\tau_{xx})_{i+\frac{1}{2}} = 2 \mu_{i+\frac{1}{2}} \left(\frac{\partial u}{\partial x} \right)_{i+\frac{1}{2}} + \lambda_{i+\frac{1}{2}} \left[ \left(\frac{\partial u}{\partial x} \right)_{i+\frac{1}{2}} + 
    \left(\frac{\partial v}{\partial y} \right)_{i+\frac{1}{2}} +
    \left(\frac{\partial w}{\partial z} \right)_{i+\frac{1}{2}}\right]
\end{equation} \\

\noindent \textbf{Step-4, Estimate the interface fluxes:} The viscous fluxes at $i+\frac{1}{2}$ locations are computed using the relations provided in Eqn. \ref{visc-curvi}. For instance, the viscous flux vector in the $\xi$ direction is evaluated as follows:

\begin{equation}
    \hat{F}^{\text{V}}_{i+\frac{1}{2}}= \left[\begin{array}{c}
0 \\
\left(\hat{\xi}_{x}\right)_{i+\frac{1}{2}} (\tau_{x x})_{i+\frac{1}{2}}+\left(\hat{\xi}_{y}\right)_{i+\frac{1}{2}} (\tau_{x y})_{i+\frac{1}{2}}+\left(\hat{\xi}_{z}\right)_{i+\frac{1}{2}} (\tau_{x z})_{i+\frac{1}{2}} \\
\left(\hat{\xi}_{x}\right)_{i+\frac{1}{2}} (\tau_{y x})_{i+\frac{1}{2}}+\left(\hat{\xi}_{y}\right)_{i+\frac{1}{2}} (\tau_{y y})_{i+\frac{1}{2}}+\left(\hat{\xi}_{z}\right)_{i+\frac{1}{2}} (\tau_{y z})_{i+\frac{1}{2}} \\
\left(\hat{\xi}_{x}\right)_{i+\frac{1}{2}} (\tau_{z x})_{i+\frac{1}{2}}+\left(\hat{\xi}_{y}\right)_{i+\frac{1}{2}} (\tau_{z y})_{i+\frac{1}{2}}+\left(\hat{\xi}_{z}\right)_{i+\frac{1}{2}} (\tau_{z z})_{i+\frac{1}{2}} \\
\left(\hat{\xi}_{x}\right)_{i+\frac{1}{2}} (\beta_{x} )_{i+\frac{1}{2}}+\left(\hat{\xi}_{y}\right)_{i+\frac{1}{2}} (\beta_{y} )_{i+\frac{1}{2}}+\left(\hat{\xi}_{z}\right)_{i+\frac{1}{2}} (\beta_{z} )_{i+\frac{1}{2}}
\end{array}\right]
\end{equation}

\noindent \textbf{Step-5, Compute residual:} Finally, the viscous flux residual is computed using the following relation:

\begin{equation}
    \left(\frac{\partial \boldsymbol{F}^v}{\partial \xi}\right)_i \approx \frac{\boldsymbol{F}_{i+\frac{1}{2}}^v-\boldsymbol{F}_{i-\frac{1}{2}}^v}{\Delta \xi}
\end{equation}

After computing the inviscid and viscous flux residuals time marching is performed to compute the solution corresponding to next time-step. The details regarding the time marching scheme and the stable time-step evaluation for both inviscid and viscous flow simulations is detailed in \ref{sec:time-int}.
\section{Multi-block approach and boundary conditions} \label{sec:mul-block}
\begin{figure}[h!]
    \centering
    \includegraphics[width=150mm]{Images/multi-block.pdf}
    \caption{(a) Schematic of a block and its ghost points on left, right, bottom and top boundaries. (b) Schematic of ghost point sharing between blocks sharing interface along $\xi$ direction. The letter `G' in figure-b represents ghost cell region.}
    \label{multi-block}
\end{figure}

The multi-block approach enables geometric flexibility to simulate flows through and around domains of various atypical shapes. The multi-block approach adopted in the present work is similar to the approach proposed by Lien et al. \cite{lien1996multiblock}. Any block in the domain can interface one or more of it's boundaries (six in 3-D, four in 2-D) with the blocks surrounding it. Each block possesses its own coordinate system, essentially making the approach block unstructured. To ensure consistency of the block dimensions at the boundaries, the dimensions of the block interface plane should be identical on both home and neighbor blocks. A schematic of a multi-block layout depicting the arrangement of cells and the coordinate systems is illustrated in  Fig. \ref{multi-block}a. Ghost cells are adapted to introduce necessary inter-block information sharing and enable sufficient stencil size at block boundaries to compute high-order fluxes corresponding to MEG6, MIG4, and $\alpha$-damping.

Boundary conditions are imposed at each block boundary through ghost cells. The geometric location of the ghost points is arbitrary as they are physically non-existent. Therefore, metric terms are also non-existent and do not need to be computed in those locations. However, the lack of geometric information at boundaries poses difficulties in applying the usual high order approximations that are employed for cells inside the domain. \ref{app:one-sided} shows the special boundary formulae that were adopted in the present work.

A schematic of a two-dimensional block and the imaginary ghost cells surrounding it's four boundaries is illustrated in Fig. \ref{multi-block}. A total of five ghost cells were used for all the simulations in the current study. Suitable primitive variable values are specified in those ghost cells based on the type of boundary condition adapted at that boundary. A list of expressions to compute the primitive variables in ghost cells corresponding to various boundary conditions is provided in Table \ref{BCs}. All the formulations and concepts discussed in this section can be extrapolated to three dimensions.

\begin{table}[h!]
    \centering
    \includegraphics[width=\textwidth]{Images/BCs.pdf}
    \caption{Expressions for primitive variable values in the ghost cells for various boundary conditions. Where, $N_g$ is number of ghost cells, prefix $(.)_{in}$ represents inflow conditions at the inflow boundary, $NI$ is number of points in $xi$ direction, $U$ is the contravariant velocity in $\xi$ direction, $V$ is the contravariant velocity in $\eta$ direction, Suffix `$H$' and `$N$' denote `Home' and `Neighbour' blocks respectively.}
    \label{BCs}
\end{table}

%%%%%%%% Part-2 %%%%%%%% 
\section{Results}
\label{results}

\begin{figure*}[ht]
    \centering
    \includegraphics[scale=0.15,trim={0 2.5cm 0 5cm},clip]{images/aoi-single_burst}
    \caption{The time average peak Age of Information with burst and \gls{soa} loss values against the dynamic reliability logic for different network topologies.}
    \label{fig:aoi_burst}\vspace{-0.4cm}
\end{figure*}


This paper focuses on both transport layer and application layer metrics to determine the feasibility of dynamic reliability. For this, we have selected the session packet volume, as transmitted, retransmitted, lost and backlogged packets as \glspl{kpi} for the transport layer; while focusing on the \gls{aoi} for the application layer. The \gls{aoi} was chosen as a crucial indicator for the freshness of packets in real-time applications. More specifically, this work adopts the time average peak \gls{aoi} equation \cite{aoi_equation} depicted in Eq. \ref{aoi}, where $\Delta(r_{i+1})$ is the $i$th update at the time it was received at the server, for a session time period of $\tau$.

\begin{equation}
    \label{aoi}
    \gls{aoi}_\tau = \frac{1}{n-1}\sum_{i=1}^{n-1} \Delta(r_{i+1})
\end{equation}

We include a comparison between the vanilla QUIC implementation which does not enjoy the dynamic reliability extension, with a number of dynamic reliability policies. The tests were run a number of times for statistical significance, with the mean value of vanilla implementation used as a baseline for comparison. The topology utilised both random loss and bursty loss to explore the bounds of dynamic reliability. The \gls{soa} loss in the figures correspond to the loss values presented in Table. \ref{tab:path_char}, for ease of comparison between bursty and random loss scenarios.

\subsection{Transport-Layer KPIs}

To analyse the performance gain at the transport layer due to dynamic reliability, the volume of transmitted and backlogged packets is examined. The figures are in the form of boxplots, which take the vanilla implementation as a benchmark, depicted as the red dashed line.

As seen in Fig. \ref{fig:sent_burst}, the loss plays a crucial role in the performance of the reliability policies. The policies under random loss did incredibly well for the networks with a larger capacity, namely \gls{mmwave} and Sub-6~GHz, whereas for burst loss, the lower network capacities had a larger packet reduction. With the increase in burst loss, the behaviour of the set split reliable policies became unpredictable, if a reliable assignment happened to coincide with a burst loss, the number of transmitted packets increases, and vice versa. On the other hand, in smarter policies, such as Loss-Aware, the performance lightly matched the vanilla baseline, as the reliable assignment dominated the session to compensate for a higher burst loss. Not only that but, the burst loss also impacted the variance of the transmitted packets for the policies.

Unsurprisingly, the unreliable focused policy, 80-20 split, outperformed other policies for all topologies in random and bursty loss scenarios, with an approximate reduction of 80\%. That being said, the majority of the policies reduced the transmitted packets on the link by approximately 70\% for random loss, while the reduction started at $\approx 15\%$ and decreased as the loss increased for the burst loss scenario.

The retransmitted and lost packets, not shown due to space limitations, followed the same trend as the transmitted packets for the random loss scenarios. However, for the burst loss scenarios, the larger capacity networks had a lower reduction in the retransmitted and lost packets. This can be seen as a favorable outcome since the lower capacity networks are scarce on resources. It is important to note that the Loss-Aware policy mimicked the vanilla approach as the burst loss increased, signifying the overwhelming appointment of reliable packets in adapting to the harsh burst loss conditions.
 
Alternatively, Fig. \ref{fig:backlog_burst} clearly shows a stark comparison between the policies and loss scenario in the reduction of the backlogged packets. The Loss-Aware policy for random loss scenario reduced the backlogged packets by up to 50\%, beating all other policies by approximately 30\%. Furthermore, it is clear that the unreliability focused policies resulted in the lowest backlog for the session. In comparison, we notice that the burst loss and the backlogged frequency have a positive correlation, where the maximum reduction of the backlogged packets for the policies is at most 20\%. Much like the transmitted packets, the probability of a burst loss occurrence plays a vital role in the number of retransmissions sent and by extension the number of backlogged packets. Thus, we can conclude that the stress placed on the buffer is a result of the reliable packets which is tightly coupled with the congestion on the session. Whereas, unreliable focused policies did not encounter such a phenomenon regardless if it was experiencing a burst loss.


\subsection{Application-Layer KPIs}

The feasibility of dynamic reliability for real-time applications can be determined by the \gls{aoi}, with comparison across different topologies and policies. If we take a strict approach and consider anything below $10$~ms is real-time \cite{real-time}, then all the reliability policies passed that requirement, which is attractive for real-time applications, as shown in Fig. \ref{fig:aoi_burst}. Utilising the median as an estimate of the runs, the policies in the WLAN and Sub-6~GHz topology with random loss floated around $4-5$~ms with negligible difference, while the \gls{aoi} for \gls{mmwave} was $\approx 2-3$~ms. It is clear that the \gls{aoi} and the network capacity have a negative correlation, as the network capacity decreases, the \gls{aoi} increases. The same correlation is extended to the bursty loss scenarios, where \gls{mmwave} dominated the other topologies. That being said, it is crucial to note that the \gls{aoi} for the reliability policies is often slightly better than or equal to the \gls{aoi} of the vanilla implementation, proving that dynamic reliability reduces the congestion of the session at no cost to the \gls{aoi}.

\newpage
\section{GPU acceleration model and speedup analysis} \label{sec:gpu-accel}

This section discusses the GPU parallelization model employed in the present work to implement the MEG6 and MIG4 schemes along with the performance results achieved on various GPUs. In presenting the performance details, a particular emphasis was laid on subjects such as `the contribution of various functions to the overall computational time', `the effect of cell count', `memory occupancy versus cell count', and `the effect of working precision'.

In order to specify the parallelism on GPUs, the directive based `OpenACC' programming language \cite{openacc-web1} is used in the present work. The use of OpenACC directives over the more popular `CUDA' programming language is motivated by less code development time associated with OpenACC while still being able to achieve beneficial speedup results that are comparable to CUDA, as will be demonstrated soon. The details of OpenACC implementation are being skipped here for brevity. Interested readers are refered to the OpenACC API guide \cite{openacc-web2} and other resources available in their website \cite{openacc-web1}.

\begin{figure}[h!]
    \centering
    \includegraphics[width=\textwidth]{Images/gpu-strategy1.pdf}
    \caption{Schematic diagram of the parallelization model employed for single device GPU computations. Shown in the left of the figure, is an example multi-block computational domain (of an airfoil) whose geometrical and initial conditions data is distributed to the host memory. The three steps involved in the parallelization model are shown in blue circles with numbers.}
    \label{gpu-strategy1}
\end{figure}

A high-level view of the computing model adapted for single GPU computations is pictorially illustrated in Fig. \ref{gpu-strategy1}. It consists of a CPU-GPU pair connected via a PCI Express bus or an NVLink interconnect; the latter option which is an NVIDIA proprietary bus is designed to offer a higher bandwidth. The program is hosted by the CPU (or called `host'), which initializes the compute environment, reads the geometry, mesh, and other input variables. The compute routines of the program are off-loaded onto the GPU; GPU is refereed to as `device'. Fig. \ref{gpu-strategy1}, shows the three elemental steps involved in performing a single device GPU simulation. The first step before starting the main time loop involves creating a copy of all the necessary data, such as mesh, initial conditions, and the flow parameters on the device memory. Since scientific computing based programs such as present are mainly bandwidth bounded, this step is done as a one time act to restrict the communication between the host and device during the simulation. It is followed by executing all the routines that are part of the main time loop listed in Fig. \ref{algo} until the simulation reaches its end-time. Once the computations inside the main time loop are completed, the solution is transferred to the host memory (step-3 in Fig. \ref{gpu-strategy1}), from where it can be viewed and post-processed in the end. But often in many cases, the user requires to store the solution corresponding to intermediate time-steps for temporal analysis. In such a scenario, only the primitive variable data is transferred to the host whenever necessary. Such communication during the middle of a simulation, if carried out just once per hundred iterations or more, was noted to add only a negligible amount of computational time to the overall simulation. 

% Since the amount of memory is limited on the GPU, it consequently limits the size of the simulation that can be run on the GPU without memory over-subscription. The higher the GPU memory the larger grid one can accommodate on the GPU.

% In multi-GPU computations, the compute environment consists of multiple pairs of hosts and devices (CPUs and GPUs). A schematic of the decomposed computational domain and the parallelization model adapted in the present work for multi-GPU computations is shown in Fig. \ref{gpu-strategy1}b. Firstly, an MPI environment is initialized among the CPU cores, virtually creating an individual memory space for each processor. Then each CPU processor is assigned with a single GPU privately. The information corresponding to decomposed blocks, such as grid, initial conditions, boundary conditions, etc., are fed into the corresponding host memory spaces. The computations on each host and device pair in this multi-GPU setup are performed through a similar strategy described above for single GPU computations. This essentially means that the computations belonging to a device are considered independent of others and carried out as standalone individual simulations except for the communication required at the block boundaries.

% Firstly, as a pre-processing step, if the domain does not already have enough blocks to be distributed amongst the available GPUs, the computational domain is decomposed into multiple blocks.

% \begin{figure}[h!]
%     \centering
%     \includegraphics[width=150mm]{Images/gpu-strategy2.pdf}
%     \caption{Schematics of the halo-region data flow routes in multi-GPU computations. (a) Through host-1 and host-2 (option-1), and (b) CUDA-Aware MPI based direct communication between GPUs (option-2). }
%     \label{gpu-strategy2}
% \end{figure}

% At the end of each time-step the halo-region data has to be shared with the neighbouring processors. To accomplish this, the data from the halo regions (ghost cells) is packed from the `device memory of the source GPU' and transferred to the `device memory of the destination GPU' via one of the two communication options presented in Fig. \ref{gpu-strategy2}. The first option represented as `path-1' in Fig. \ref{gpu-strategy2}a takes a longer route via host CPUs passing through a total of three communication bridges. The second option denoted as `path-2' in Fig. \ref{gpu-strategy2}b takes a direct route between the GPUs and is often preferred if a communication bus exists between the GPUs. The second option can be enabled in Nvidia GPUs having an `NVLink', using the \texttt{DCUDA\_AWARE\_MPI} environment variable available in the OpenACC compilers. Due to the lack of NVLink interconnect amongst some of our GPU resources, the first option (path-1) is used as the inter-GPU communication model in all the computations presented in the current study to maintain consistency. It is worth noting that, since MPI is employed here, the strategy remains unchanged even when multiple CPU/GPU `nodes' are employed. One can also note that, if the GPUs are taken out of the computing environment and computations are chosen to be performed only on the CPUs (taking out the GPUs out the model), such an environment will yield MPI parallel computations on CPUs. This essentially means the same code can be used for both CPU and GPU parallel computations, which makes this strategy highly portable.

Now that the parallelization strategy is outlined, the performance details achieved employing both MEG6 and MIG4 schemes will be presented. All the computations are performed using double precision by default unless any other working precision is specified. The high performance `Intel(R) Xeon(R) Gold 5115 CPU @ 2.40GHz' is used for all the CPU computations. For GPU computations, three generations of data center GPUs, namely, NVIDIA Quadro A6000, NVIDIA V100, and NVIDIA A100 are employed to present a comparative study of their efficiencies. These GPUs are referred to as A6000, V100 and A100 for short in this paper. The acceleration offered by a GPU is mainly dependent on three key hardware attributes which are listed in table \ref{gpuspecs}.

% It can be noted that A100 GPU has both highest Streaming Multiprocessor (SM) count, double precision processing speed, and memory bandwidth.

\begin{table}[h!]
    \centering
    \includegraphics[width=120mm]{Images/gpu-specs.pdf}
    \caption{Key hardware specifications of GPUs used in the current study.}
    \label{gpuspecs}
\end{table}

% \begin{table}[]
% \caption{Key hardware specifications of GPUs used in the current study.}
% {%
% \begin{tabular}{@{}cccccc@{}}
% \toprule
% \textbf{GPU} & \textbf{SMs} & \textbf{Memory} & \textbf{Memory Bandwidth} & \textbf{Processing speed (DP)} & \textbf{Processing speed (SP)} \\ \midrule
% A6000 & 84  & 48 GB & 768 GB/s  & 1.2 TFLOPS          & 38.7 TFLOPS \\
% V100  & 80  & 32 GB & 1134 GB/s & 7 TFLOPS   & 14 TFLOPS   \\
% A100  & 128 & 40 GB & 2039 GB/s & 9.7 TFLOPS & 19.5 TFLOPS \\ \bottomrule
% \end{tabular}%
% }
% \label{gpuspecs}
% \end{table}

\subsection{Speedup comparison to single core CPU} % correction - three test cases were used and not two
Tables \ref{gpu_table1} and \ref{gpu_table2} show the elapsed simulation times per hundred iterations on the CPU and various GPUs. To access the performance, three test cases are employed. The first is the two-dimensional Inviscid Double Mach Reflection (DMR) case (described in section \ref{DMR-case}) consisting of 3 million cells. The second is the three-dimensional Viscous Taylor Greene Vortex (TGV) case (details of the test case can be found in \cite{sainadh2022spectral}) consisting of 27 million cells. The relative difference in the grid resolution between the two cases and the fact that the viscous TGV test case has extra routines to be executed (thus higher exposed parallelism) is planned intentionally to note the performance variation between two such contrasting compute scenarios. To also understand the speedup on a practically relevant multi-block test case, the supersonic jet noise case discussed in Sec. \ref{sec:jet} is simulated at a grid resolution of 13 million.

\begin{table}[h!]
    \centering
    \includegraphics[width=155mm]{Images/gpu_table1.pdf}
    \caption{Comparison of time taken per 100 iterations on CPU versus various Nvidia GPUs employing MEG6 scheme.}
    \label{gpu_table1}
\end{table}

\begin{table}[h!]
    \centering
    \includegraphics[width=155mm]{Images/gpu_table2.pdf}
    \caption{Comparison of time taken per 100 iterations on CPU versus various Nvidia GPUs employing MIG4 scheme.}
    \label{gpu_table2}
\end{table}

\begin{table}[h!]
    \centering
    \includegraphics[width=155mm]{Images/gpu_table3.pdf}
    \caption{Comparison of time taken per 100 iterations on CPU versus various Nvidia GPUs employing WENO-Z scheme.}
    \label{gpu_table3}
\end{table}

The results presented in Tables \ref{gpu_table1}, \ref{gpu_table2}, and \ref{gpu_table3} suggest that the computations performed on V100 and A100 are about two orders of magnitude faster than the calculations performed on CPU. These speedup results can be noted to be on par with the CUDA based parallelization models implemented in other studies, for instance Refs-\cite{cernetic2022high,crespo2015dualsphysics,laufer2022gpu}. Amongst the GPUs, the A100 can be noted to perform the best by producing a maximum speedup of greater than $200\times$ for both MEG6 and MIG4 schemes. V100 stands next in terms of performance, followed by A6000 in the last position. This is consistent with the hardware specifications presented in table \ref{gpuspecs}; the double precision processing speed and memory bandwidth of Nvidia A100 is the highest which is followed by V100 and A6000. The effect of superior single precision speed corresponding to A6000 on the simulation times will be discussed later in section \ref{effect-WP}. 

The results also indicate that the speedup of the simulation is influenced by its parallelism and size. The viscous TGV simulation at 27 million cell count was found to run efficiently compared to the 3 million cell count inviscid DMR test case. The MEG6 scheme was the most efficient among the three schemes tested, due to its explicit derivative computations as opposed to implicit derivatives being used in MIG4 and WENO-Z tests. Since WENO-Z does not use gradients for inviscid flux calculations, for the inviscid DMR case, the WENO-Z scheme was the fastest. However, for the viscous cases (viscous TGV and supersonic jet), the computational time of WENO-Z and MIG4 were similar, with MIG4 being slightly more efficient. Although the computational time of WENO-Z and MIG4 are close for viscous cases, the solution resolution is significantly different, with MIG4 being superior.

% In summary the efficiency is dependent on the nature of scheme (explicit/implicit), size of the simulation and of course the GPU hardware.
% max limited to ~30M on 40gb a6000
% not fair comparison but gives an idea

\subsection{Contribution of various routines to the overall computational time}
Fig. \ref{gpu1}a-d shows a pie-chart view representing the contribution of various routines to the overall computational time in performing one RK-iteration of MEG6 and MIG4 algorithms on both CPU (Intel Xeon Gold 5115) and GPU (Nvidia A100). All the tests were run using the Viscous Taylor-Greene-Vortex case at a grid resolution of $256^3$. The plots suggest that more than 50\% of the time on both CPU and GPU is spent just performing the `reconstruction routine' (indicated with number `1') in all the scenarios. This is mainly because of the multiple sub-steps involved in the reconstruction routine described in \ref{inv_algo}. Furthermore, the contribution of primitive variable gradients (indicated with the number `6') can be noted to occupy a more significant portion of the pie in the MIG4 scheme compared to MEG6. This is due to the extra tri-diagonal matrix inversion step involved in the MIG4 algorithm. Apart from reconstruction and gradient computation routines, the other routines can be noted to occupy a relatively smaller portion of the pie due to the relatively simple nature of the corresponding kernels.

\begin{figure}[h!]
    \centering
    \includegraphics[width=\textwidth]{Images/gpu1.pdf}
    \caption{Contribution of various routines to the total computational time while executing one RK-time step, corresponding to (a) MEG-6, (b) MIG-4 schemes tested using Viscous Taylor Greene Vortex test case ($256\times256\times256$) on Nvidia A100 GPU.}
    \label{gpu1}
\end{figure}

To understand the efficiency of parallelization corresponding to each routine, the individual speedup values for each routine achieved for both MEG6 and MIG4 are computed and tabulated in Fig. \ref{gpu1}e. Interestingly, the speedup values corresponding to MEG6 and MIG4 are nearly similar for all the routines, including the `primitive variable gradient computation'. The `Riemann solver' routine can be noted to run with highest efficiency achieving a speedup of about $\approx 420 \times$ in both MEG6 and MIG4 approaches. On the other hand, the least speedup is noted for the residual computations, time integration, and the boundary condition routines.

\subsection{Effect of cell count on the speedup}
Variation of speedup with increasing cell count is studied relative to the computations on a single core CPU. The test runs were performed on the Nvidia A100 using the Viscous Taylor Greene Vortex test case. The speedup was measured by taking the ratio of time taken per hundred iterations on a single GPU to that of a single core CPU. To avoid GPU memory over-subscription (40GB - A100), the simulations were not carried out beyond 27 Million cells. Fig. \ref{gpu3} shows a monotonically increasing trend in the efficiency for all the three schemes tested. However, due to the implicit nature of the derivative computation involved in the MIG4 and WENO-Z schemes, their parallel efficiency turned out to be slightly lower compared to MEG6. In comparison, WENO-Z was noted to produce the least amount of efficiency, which can be acknowledged by comparing the curves in Fig. \ref{gpu3} near the 27 million range.

%While the speedup has raised from 0 to $150\times$ in the 0 to 5 millions cell count range, the speedup has only increased from $150\times$ to $200\times$ in the 5 Million and 27 Million range.

\begin{figure}[h!]
    \centering
    \includegraphics[width=100mm]{Images/gpu3.pdf}
    \caption{Speedup efficiencies of MEG6, MIG4, and WENO-Z schemes with increasing grid size on the Nvidia A100 GPU with reference to single-core CPU.}
    \label{gpu3}
\end{figure}

The curves shown in Fig. \ref{gpu3} suggest that the rate of rise in speedup is considerably high in the initial range of $0$ to $2.5$ million cells. Although the speedup still increases beyond the $0-2.5$ million range, the curve gets progressively plateaued with increasing cell count. This trend can be explained as follows. Initially, when the cell count is increased, the threads in the GPU are progressively occupied, resulting in a steady and almost linear rise in the speedup until all the GPU threads are fully occupied. This maximum GPU occupancy reaches at close to two million cells in the specific case studied here. However, when the cell count is further increased, the kernel divides the program loops into multiple sets to run each of them in parallel resulting in multiple computation cycles within the kernel. This effect manifests as a slow rise in the parallel efficiency after a certain cell count limit. Despite this plateauing effect, keeping the GPU unit as occupied as possible is always a good practice to extract the highest possible efficiency, especially keeping in view the net power consumption when multiple GPUs are employed.


\subsection{Memory occupancy of the solver on GPU} \label{effect-WP}

\begin{figure}[h!]
    \centering
    \includegraphics[width=92mm]{Images/GPU-mem-consumption.pdf}
    \caption{The GPU memory consumption with respect to variation in the grid size of the simulation, as demonstrated by tests performed on an NVIDIA A100 with 40GB RAM using the Viscous Taylor-Green Vortex test case. The gray dashed line corresponds to the approximate estimate of memory consumption based on the number of global variables defined in the solver.}
    \label{mem_cons}
\end{figure}

The capability to run a simulation of given size on a GPU is largely determined by the available RAM on the device. Memory utilization is a crucial factor for finite difference algorithms based on curvilinear coordinates, as they require multiple variables representing metric terms to be defined at each cell center and interface. The memory footprint of a three-dimensional solver scales approximately as $O(N^3)$, where $N$ represents the number of cells in each direction, and for simplicity, it is assumed to be equal in all directions.

Fig. \ref{mem_cons} displays the memory occupancy ($\mathcal{M}$) of the flow solver at different grid resolutions. The plot also includes a rough estimate of the expected memory consumption ($\mathcal{M}=\left[117(N+8)^3\right] \times 64 \text { bits }$) derived based on the number of variables in the solver. The solver uses $117$ global 3D double-precision arrays, each of size $(N+8)^3$ (including ghost cells), excluding small scalars and arrays. The discrepancy between the blue line and gray dashed line in Fig. \ref{mem_cons} is due to arrays/variables defined in each function that consume memory on GPU RAM. For the present flow solver (employing MEG6 or MIG4), tests show that a Nvidia A100 (40GB) can handle simulations with up to 34 million cells without any decrease in performance or memory over-subscription issues. Among the various data variables, metric terms consume about $60\%$ of the total memory as they are needed at cell centers and six-cell interfaces surrounding each cell. $71$ out of a total $117$ variables belong to the metric terms such as $(\xi_{x})_i$, $(\xi_{x})_{i+\frac{1}{2}}$, $(\hat{\xi}_{x})_{i+\frac{1}{2}}$, and $(\Tilde{\xi}_{x})_{i+\frac{1}{2}}$.


\subsection{Effect of working precision on the speedup} \label{effect-WP}

The floating point representation in computer calculations can influence the computational time as it can directly affect the number of arithmetic operations and the overall memory movement required to complete a task. A 32-bit/single-precision memory allocation (FP32) is generally used for tasks where the accuracy of calculations is not critical and the tasks are bounded by limitations in allocated memory. On the other hand, a 64-bit/double-precision (FP64) allocation is usually employed for tasks like scientific calculations where accuracy is needed. Although FP64 is widely used for CFD simulations, a few studies show that the working precision does not significantly influence the solution and the corresponding physics. One can also improve the simulation turnaround times by employing FP32 for the transient phase of simulation (where the flowfield is not of interest to the user) and switching back to FP64 for the rest of the flowfield calculations when the accuracy of the solution is essential.

\begin{figure}[h!]
    \centering
    \includegraphics[width=150mm]{Images/gpu-SP-DP.pdf}
    \caption{Comparison of elapsed times using Single Precision (FP32) and Double Precision (FP64) data variables.}
    \label{gpu-SP-DP}
\end{figure}

Fig. \ref{gpu-SP-DP} compares FP32 and FP64 computational times for both MEG6 and MIG4 schemes on various GPUs. The tests were performed by running one hundred iterations of the Viscous Taylor Greene vortex test case at $256^3$ resolution. The simulations that use FP32 can be noted to run faster than FP64 by a factor of $\approx 1.3\times$ to $1.4\times$ contrary to the anticipated $2\times$ speedup that is suggestive from Table \ref{gpuspecs}. There is a high discrepancy between the FP32 processing speed of A6000 displayed in table \ref{gpuspecs} and the FP32 speedup achieved here. Furthermore, the FP32 calculations on A6000 can be noted to only run almost as fast as FP64 calculations on the V100. This suggests that rather than the processing speed, the computations are primarily bounded by the memory bandwidth of GPU. Which means that the overall simulation time is not only spent in performing the floating point operations but also on other essential tasks such as initializing the kernels, moving data across GPU RAM and various levels of cache memory storage, etc. It is worth noting that scientific computing applications like the present solver are generally bounded by memory bandwidth as opposed to arithmetic operations. Therefore, the working precision does not hugely affect the performance, as demonstrated here. Nevertheless, the speedup from the employment of FP32 can still be used to accelerate the simulations.


%%%%%%%% ending %%%%%%%%
\section{Conclusions}
We consider the phase-extraction problem, and we showed that, given a unitary $U = e^{i\pi H}$ and its inverse $U^{\dag}$, we could implement a block-encoding of $\phi(H)$ for some smooth function $\phi(x)$. The word `smooth' here means existence and continuity of the derivatives: the higher the number of continuous derivatives that a function has, the faster its Fourier sum (and thus the Laurent polynomial on the eigenphases) uniformly converges to that function. We are confident this can have many more applications beyond what is shown in this work. It is also worth remarking that Jackson showed that the convergence rate of a Fourier series is almost-optimal, in the sense that no trigonometric (or, equivalently, complex exponential) series can approximate the desired function faster, up to that $\log d$ factor~\cite[p.\ 21]{jacksonTheoryApproximation1930a}. Also remember that `smoothing' a function, i.e., replacing its derivative with a continuous function, does not give faster convergence for free in general, as its derivative will become steep in the points where we smooth out discontinuities, and this translates to a high Lipschitz constant: a~clear example is given by Eq.~\ref{eq:lipschitz-constant-recurrence-solution}, but in that case, fortunately, nothing depends on the size of the input $N$, and thus does not influence the asymptotic query complexity of Algorithm~\ref{alg:prop-sampling-qsp}, although the constant factor can become large even for $p = 20$. From a theoretical point of view, this work shows that, for any $\eta > 0$, there is an algorithm with query complexity 
$$\Tilde{\bigO}\left(\frac{1}{\bar{c}^{\frac{1}{2} + \eta}} \frac{1}{\epsilon^\eta} \right)$$
solving the proportional-sampling problem. This statement seems to suggest there exists an algorithm which directly solves the problem with $\eta = 0$, and an open question would be to find such algorithm.


It is also interesting to remark that Theorems~\ref{thm:haah-construction},~\ref{thm:haah-completion} indeed allow the construction for any $\phi$, even complex-valued, provided that its absolute value is reciprocal.

One could think that, in Section~\ref{sec:prop-sampling}, instead of using the linear function in the phase-extraction subroutine, we could approximate the square root and then apply the transformation directly on $e^{i \pi c(x)}$. However, in the case of proportional sampling this would be inconvenient, as the derivative of the square root function has a discontinuity with an infinite jump around 0, and we could not choose a constant $\delta$ if we had values of the oracle that are too close to $0$.
\chapter*{Acknowledgement}
\addcontentsline{toc}{chapter}{Acknowledgement}
The authors thank Andrzej Kupsc, Sergey Barsuk, Olivier Callot and Wolfgang K{\"u}hn for their contribution on the CDR draft.
%The authors thank the international review committee XXX for their great effort in reading the CDR draft and providing valuable suggestions. 
The STCF working group thanks all 
the colleagues in the world-wide community for many profitable discussions
and expresses gratitude to the Hefei Comprehensive National Science Center for their strong support.  This work is supported by: international 
partnership program of the Chinese Academy of Sciences Grant No. 211134KYSB20200057.
% \section*{Credit author statement}
\textbf{Hemanth Chandravamsi:} Conceptualization, Methodology, Software, Validation, Formal Analysis, Visualization, Writing Original Draft. \textbf{Amareshwara Sainadh Chamarthi:} Supervison, Conceptualization, Writing - Review \& Editing. \textbf{Natan Hoffmann:} Resources, Writing - Review \& Editing. \textbf{Steven H. Frankel:} Funding acquisition, Supervison, Resources, Writing - Review \& Editing.
\section{Appendix for Proofs}

\paragraph{Proof of Theorem \ref{thm:main}.}

\begin{proof}
\label{proof:main}
Our proof has two steps. In Step 1, we will show that SimCLR is equivalent to minimizing the cross entropy loss defined in Eqn.~(\ref{eqn:cross-entropy}). 
In Step 2, we will show  that minimizing the cross-entropy loss 
is equivalent to spectral clustering on $\bfpi$. 
Combining the two steps together, we have proved our theorem. 

\textbf{Step 1: } SimCLR is equivalent to minimizing the cross entropy loss.

The cross-entropy loss takes expectation over 
$\bfW_\bfX\sim \mathbb{P}(\cdot ; \bfpi)$, 
which means $\bfW_\bfX$ has exactly one non-zero entry in each row $i$. By Lemma~\ref{lem:multinomial}, we know every row $i$ of $\bfW_\bfX$ is independent of other rows. Moreover, 
$\bfW_{\bfX,i}\sim \mathcal{M}(1, \bfpi_i/\sum_j \bfpi_{i,j})=\mathcal{M}(1, \bfpi_i)$, because $\bfpi_i$ itself is a probability distribution.
Similarly, we know $\bfW_\bfZ$ also has the row-independent property by sampling over $\mathbb{P}(\cdot;\bfK_\bfZ)$.
Therefore, by Lemma~\ref{lem:cross_split}, we know Eqn.~(\ref{eqn:cross-entropy}) is equivalent to:
\[
 -\sum_{i=1}^n \mathbb{E}_{\bfW_{\bfX,i}}[\log \mathbb{P}(\bfW_{\bfZ,i}=\bfW_{\bfX,i};\bfK_\bfZ)],
\]

This expression takes expectation over $\bfW_{\bfX,i}$ for the given row $i$. Notice that 
$\bfW_{\bfX,i}$ has exactly one non-zero entry, which equals $1$ (same for $\bfW_{\bfZ,i}$). 
As a result
we expand the above expression to be:
\begin{equation}
 -\sum_{i=1}^n \sum_{j\neq i} \Pr(\bfW_{\bfX,i,j}=1)\log \Pr(\bfW_{\bfZ,i,j}=1).
\label{eqn:detailed-expansion}    
\end{equation}


By Lemma~\ref{lem:multinomial}, $\Pr(\bfW_{\bfZ,i,j}=1)=\bfK_{\bfZ,i,j}/\|\bfK_{\bfZ,i}\|_1$ for $j\neq i$. Recall that $\bfK_\bfZ=(k(\bfZ_i-\bfZ_j))_{(i,j)\in[n]^2}$, which means 
$\bfK_{\bfZ,i,j}/\|\bfK_{\bfZ,i}\|_1=\frac{\exp(-\|\bfZ_i-\bfZ_j\|^2/{2\tau})}{\sum_{k\neq i}
\exp(-\|\bfZ_i-\bfZ_k\|^2/{2\tau})
}$ for $j\neq i$, when $k$ is the Gaussian kernel with variance $\tau$. 

Notice that $\bfZ_i=f(\bfX_i)$, so we know
\begin{equation}
-\log \Pr(\bfW_{\bfZ,i,j}=1)=
-\log \frac{\exp(-\|f(\bfX_i)-f(\bfX_j)\|^2/{2\tau})}{\sum_{k\neq i}
\exp(-\|f(\bfX_i)-f(\bfX_k)\|^2/{2\tau}),
}
\label{eqn:infonce-equivalence}    
\end{equation}


The right hand side is exactly the InfoNCE loss defined in Eqn.~(\ref{eqn:infonce}).
Inserting Eqn.~(\ref{eqn:infonce-equivalence}) into Eqn.~(\ref{eqn:detailed-expansion}), we get the SimCLR algorithm, which first samples augmentation pairs $(i,j)$ with $\Pr(\bfW_{\bfX,i,j}=1)$ for each row $i$, and then optimize the InfoNCE loss. 

\textbf{Step 2: } minimizing the cross entropy loss 
is equivalent to spectral clustering on $\bfpi$.


By Lemma~\ref{lem:convert_to_spectral}, we may further convert the loss to 
\begin{equation}
\label{eqn:main-theorem-repul-attr}
\min_{\bfZ}
-\sum_{(i,j)\in [n]^2} \mathbf{P}_{i,j}
\log k (\bfZ_i-\bfZ_j)+\log \mathbf{R}(\bfZ).
\end{equation}
Since $k$ is the Gaussian kernel, this reduces to \[
\min_\bfZ \mathrm{tr}(\bfZ^\top \mathbf{L}(\bfpi) \bfZ)
+\log \mathbf{R}(\bfZ),
\]

where we use the fact that $\mathbb{E}_{\bfW_\bfX\sim \mathbb{P}(\cdot; \bfpi)}[\mathbf{L}(\bfW_\bfX)]
=\mathbf{L}(\bfpi)
$, because the Laplacian operator is linear and $
\mathbb{E}_{\bfW_\bfX\sim \mathbb{P}(\cdot; \bfpi)}(\bfW_\bfX)=\bfpi
$.
\end{proof}

\paragraph{Proof of Theorem \ref{thm:clip}.}
\begin{proof}
Since $\bfW_\bfX\sim \mathbb{P}(\cdot;\bfpi_{\mathbf{A}, \mathbf{B}})$, we know 
$\bfW_\bfX$ has exactly one non-zero entry in each row, denoting the pair that got sampled. 
A notable difference compared to the previous proof is we now have $n_\mathcal{A}+n_\mathcal{B}$ objects in our graph. CLIP deals with this by taking a mini-batch of size $2N$, 
such that $n_\mathcal{A}=n_\mathcal{B}=N$, and adding the $2N$ InfoNCE losses together. We label the objects in $\mathcal{A}$ as $[n_\mathcal{A}]$, and the objects in $\mathcal{B}$ as $\{n_\mathcal{A}+1, \cdots, n_\mathcal{A}+n_\mathcal{B}\}$. 

Notice that $\bfpi_{\mathbf{A}, \mathbf{B}}$ is a bipartite graph, so the edges of objects in $\mathcal{A}$ will only connect to object in $\mathcal{B}$ and vice versa. We can define the similarity matrix in $\cZ$ as $\bfK_\bfZ$, 
where $\bfK_\bfZ(i, j+n_\mathcal{A})=\bfK_\bfZ(j+n_\mathcal{A},i)= k(\bfZ_i-\bfZ_j)$ for $i\in [n_\mathcal{A}], j\in [n_\mathcal{B}]$, and otherwise we set $\bfK_\bfZ(i,j)=0$. 
The rest is same as the previous proof. 
\end{proof}

\paragraph{Proof of Theorem \ref{thm:exponential}.}

\begin{proof}
\label{proof:exponential}
Since the objective function consists of a linear term combined with an entropy regularization, which is a strongly concave function, the maximization problem is a convex optimization problem. Owing to the implicit constraints provided by the entropy function, the problem is equivalent to having only the equality constraint. We then introduce the Lagrangian multiplier $\lambda$ and obtain the following relaxed problem:

$$
\widetilde{E}(\boldsymbol{\alpha})=\psi_{1}-\sum_{i=1}^n \alpha_{i} \psi_{i}+\tau \sum_{i=1}^n \alpha_{i}\log \alpha_{i}+\lambda\left(\boldsymbol{\alpha}^{\top} \mathbf{1}_n-1\right).
$$

As the relaxed problem is unconstrained, taking the derivative with respect to $\alpha_{i}$ yields

$$
\frac{\partial \widetilde{E}(\boldsymbol{\alpha})}{\partial \alpha_{i}}=-\psi_{i}+\tau\left(\log \alpha_{i}+\alpha_{i} \frac{1}{\alpha_{i}}\right)+\lambda=0.
$$

Solving the above equation implies that $\alpha_{i}$ takes the form
$
\alpha_{i}=\exp \left(\frac{1}{\tau} \psi_{i}\right) \exp \left(\frac{-\lambda}{\tau}-1\right).
$ Since $\alpha_{i}$ lies on the probability simplex, the optimal $\alpha_{i}$ is explicitly given by
$
\alpha^{*}_{i}=\frac{\exp \left(\frac{1}{\tau} \psi_{i}\right)}{\sum_{i^{\prime}=1}^n \exp \left(\frac{1}{\tau} \psi_{i^{\prime}}\right)} .
$ Substituting the optimal point into the objective function, we obtain
$$
\begin{aligned}
E\left(\boldsymbol{\alpha}^*\right)  &=\psi_1-\sum_{i=1}^n \frac{\exp \left(\frac{1}{\tau} \psi_{i}\right)}{\sum_{i^{\prime}=1}^n \exp \left(\frac{1}{\tau} \psi_{i^{\prime}}\right)} \psi_{i}+\tau \sum_{i=1}^n \frac{\exp \left(\frac{1}{\tau} \psi_{i}\right)}{\sum_{i^{\prime}=1}^n \exp \left(\frac{1}{\tau} \psi_{i^{\prime}}\right)}\log \frac{\exp \left(\frac{1}{\tau} \psi_{i}\right)}{\sum_{i^{\prime}=1}^n \exp \left(\frac{1}{\tau} \psi_{i^{\prime}}\right)} \\
& =\psi_1 - \tau \log \left(\sum_{i=1}^n \exp \left(\frac{1}{\tau} \psi_{i}\right)\right).
\end{aligned}
$$
Thus, the Lagrangian dual function is given by
\begin{equation*}
-E\left(\boldsymbol{\alpha}^*\right)= -\tau \log \frac{\exp \left(\frac{1}{\tau} \psi_{1}\right)}{\sum_{i=1}^n \exp \left(\frac{1}{\tau} \psi_{i}\right)}.\qedhere
\end{equation*}
\end{proof}



\section{More on Experiments} \label{section: experiment_details}

\paragraph{CIFAR-10 and CIFAR-100} CIFAR-10 ~\citep{krizhevsky2009learning} and CIFAR-100 ~\citep{krizhevsky2009learning} are well-known classic image classification datasets. Both CIFAR-10 and CIFAR-100 contain a total of 60k $32 \times 32$ labeled images of different classes, with 50k for training and 10k for testing. CIFAR-10 is similar to CIFAR-100, except there are 10 different classes in CIFAR-10 and 100 classes in CIFAR-100.

\paragraph{TinyImageNet} TinyImageNet ~\citep{le2015tiny} is a subset of ImageNet ~\citep{deng2009imagenet}. There are 200 different object classes in TinyImageNet, with 500 training images, 50 validation images, and 50 test images for each class. All the images in TinyImageNet are colored and labeled with a size of $64 \times 64$.

\textbf{Pseudo-code.} Algorithm \ref{alg:Training Procedure} presents the pseudo-code for our empirical training procedure.

\begin{algorithm}[!htbp]
\caption{Training Procedure}
\label{alg:Training Procedure}
\begin{algorithmic}[1]
\REQUIRE trainable encoder network $f$, batch size $N$, augmentation strategy \textit{aug}, loss function $L$ with hyperparameters \textit{args}
\FOR {sampled minibatch ${x_i}_{i=1}^N$}
\FORALL{$i \in { 1, ..., N }$}
\STATE draw two augmentations $t_i = \textit{aug}\left(x_i\right) $, $t_i' = \textit{aug}\left(x_i\right) $
\STATE $z_i = f\left(t_i\right)$, $z_i' = f\left(t_i'\right)$
\ENDFOR
\STATE compute loss $\mathcal{L} = L(N, z, z', \textit{args})$
\STATE update encoder network $f$ to minimize $\mathcal{L}$
\ENDFOR
\STATE \textbf{Return} encoder network $f$
\end{algorithmic}
\end{algorithm}

We also provide the pseudo-code for our core loss function used in the training procedure in Algorithm \ref{alg:Core loss}. The pseudo-code is almost identical to SimCLR's loss function, with the exception of an extra parameter $\gamma$.

\begin{algorithm}[!htbp]
\caption{Core loss function $\mathcal{C}$}
\label{alg:Core loss}
\begin{algorithmic}[1]
\REQUIRE batch size $N$, two encoded minibatches $z_1, z_2$, $\gamma$, temperature $\tau$
\STATE $z = \textit{concat}\left(z_1, z_2\right)$
\FOR {$i \in {1, ..., 2N }, j \in {1, ..., 2N}$ }
\STATE $s_{i,j} = \Vert z_i - z_j \Vert_2^{\gamma}$
\ENDFOR
\STATE \textbf{define} $l(i, j)$ \textbf{as} $l(i, j) = - \log \frac{exp\left(s_{i,j}/\tau \right)}{\sum_{k=1}^{2N} \mathbf{1}{[k \ne i]} exp\left(s{i, j} / \tau \right)} $
\STATE \textbf{Return} $\frac{1}{2N} \sum_{k=1}^N\left[l(i, i+N) + l(i+N, i)\right]$
\end{algorithmic}
\end{algorithm}

Utilizing the core loss function $\mathcal{C}$, we can define all kernel loss functions used in our experiments in Table \ref{table: loss definition}. For all $z_i \in z$ with even dimensions $n$, we define $z_{L_i} = z_i\left[0:n/2\right]$ and $z_{R_i} = z_i\left[n/2:n\right]$.

\begin{table}[ht]
\centering
\begin{tabular}{{@{}l|l@{}}}
Kernel  &  Loss function \\ \midrule
Laplacian & $\mathcal{C}\left(N, z, z', \gamma=1, \tau\right)$\\ \midrule
Sum       & $\lambda * \mathcal{C}\left(N, z, z', \gamma=1, \tau_1\right) + (1-\lambda) * \mathcal{C}\left(N, z, z', \gamma=2, \tau_2\right)$  \\ \midrule
Concatenation Sum&$\lambda * \mathcal{C}\left(N, z_L, z'_L, \gamma=1, \tau_1\right) + (1-\lambda) * \mathcal{C}\left(N, z_R, z'_R, \gamma=2, \tau_2\right)$\\ \midrule
$\gamma = 0.5$ & $\mathcal{C}\left(N, z, z', \gamma=0.5, \tau\right)$          \\ 

\end{tabular}

\caption{Definition of kernel loss functions in our experiments}
\label {table: loss definition}
\end{table}

\textbf{Baselines.} We reproduce the SimCLR algorithm using PyTorch Lightning~\citep{PytorchLightning}.

\textbf{Encoder details.}
The encoder $f$ consists of a backbone network and a projection network. We employ ResNet50~\citep{ResNet} as the backbone and a 2-layer MLP (connected by a batch normalization~\citep{ioffe2015batch} layer and a ReLU \cite{nair2010rectified} layer) with hidden dimensions 2048 and output dimensions 128 (or 256 in the concatenation kernel case).

\textbf{Encoder hyperparameter tuning.}
For each encoder training case, we randomly sample 500 hyperparameter groups (sample details are shown in Table \ref{table: Hyperparameter sample}) and train these samples simultaneously using Ray Tune ~\citep{RayTune}, with the ASHA scheduler~\citep{li2018massively}. Ultimately, the hyperparameter group that maximizes the online validation accuracy (integrated in PyTorch Lightning) within 5000 validation steps is chosen for the given encoder training case.

\begin{table}[ht]
\centering

\begin{tabular}{@{}l|l|l@{}}
\midrule
Hyperparameter  & Sample Range & Sample Strategy \\ \midrule
start learning rate & $\left[10^{-2}, 10\right]$ & log uniform \\ \midrule
$\lambda$       & $\left[0, 1\right]$ & uniform \\ \midrule
$\tau$, $\tau_1$, $\tau_2$ & $\left[0, 1\right]$ & log uniform \\ \midrule
\end{tabular}

\caption{Hyperparameters sample strategy}
\label {table: Hyperparameter sample}
\end{table}

\textbf{Encoder training.} 
We train each encoder using the LARS optimizer~\citep{LARSOptimizer}, LambdaLR Scheduler in PyTorch, momentum 0.9, weight decay $10^{-6}$, batch size 256, and the aforementioned hyperparameters for 400 epochs on a single A-100 GPU.

\textbf{Image transformation.} The image transformation strategy, including augmentation, is identical to the default transformation strategy provided by PyTorch Lightning.

\textbf{Linear evaluation.}
The linear head is trained using the SGD optimizer with a cosine learning rate scheduler, batch size 64, and weight decay $10^{-6}$ for 100 epochs. The learning rate starts at $0.3$ and ends at $0$.

\textbf{Moco Experiments.} We also tested our method based on MoCo~\citep{he2019moco}. The results are summarized in Table \ref{tab:results-moco}. Here we choose ResNet18~\citep{ResNet} as the backbone and set a temperature of $0.1$ as default. For our simple sum kernel, we set $\lambda=0.8$. The results show that our method outperforms the original MoCo method.

\begin{table}[thb]
\centering
\caption{MoCo Experiment Results on CIFAR-10 and CIFAR-100.}
\label{tab:results-moco}
\resizebox{\textwidth}{!}{%
\begin{tabular}{@{}c|ccc|ccc@{}}
\toprule
\multirow{3}{*}{Method} & \multicolumn{3}{c|}{CIFAR-10} & \multicolumn{3}{c}{CIFAR-100} \\ \cmidrule(lr){2-4} \cmidrule(lr){5-7} 
                        & 200 epochs & 400 epochs    & 1000 epochs   & 200 epochs & 400 epochs & 1000 epochs         \\ \midrule
MoCo (repro.)         & $76.41 \pm 0.12$    & $80.01 \pm 0.15$          & $84.45 \pm 0.08$    & $\mathbf{47.02 \pm 0.11}$ & $52.50 \pm 0.07$ & $57.62 \pm 0.15$            \\
\midrule
Laplacian Kernel        & ${78.09 \pm 0.10}$    & $\mathbf{83.85 \pm 0.09}$          & $\mathbf{88.34 \pm 0.16}$    & $46.12 \pm 0.22$   & $53.44 \pm 0.17$ & $59.10 \pm 0.14$        \\
Simple Sum Kernel & $\mathbf{78.12 \pm 0.15}$   & $83.23 \pm 0.18$ & $87.50 \pm 0.20$ & $46.65 \pm 0.06$ & $\mathbf{53.62 \pm 0.19}$ & $\mathbf{59.83 \pm 0.12}$\\
\bottomrule
\end{tabular}
}
\end{table}



\section{More Experiments on Synthetic Data}


Consider a scenario with $n$ clusters, each containing $k$ vertices. Let the probability of vertices $u$ and $v$ from the same cluster belonging to $\bfpi$ be $p$. Conversely, for vertices $u$ and $v$ from different clusters, let the probability of belonging to $\pi$ be $q$. We generate the graph $\bfpi$ randomly, based on $p$ and $q$. We experiment with values of $k=100$ and $n=6$ for ease of visualization, embedding all points in a two-dimensional space. Each vertex's initial position originates from a normal distribution. In each iteration, we sample a subgraph of $\bfpi$ uniformly, ensuring each vertex has an out-degree of $1$. We then optimize the corresponding vectors using InfoNCE loss with an SGD optimizer and iterate until convergence. Our experimental setup consists of an SGD learning rate of $1$, an InfoNCE loss temperature of $0.5$, and a batch size of $50$. We evaluate two scenarios with different $p$ and $q$ values: $p=1$, $q=0$, and $p=0.75$, $q=0.2$. The results of these experiments are visualized in Figure \ref{fig:vis-spectral-cluster}. The obtained embeddings exhibit the hallmark pattern of spectral clustering of graph $\bfpi$.

\begin{figure}[!tb]
\centering
\subfigure{
\includegraphics[width=1\textwidth]{Figures/cluster_pi.png}
\label{fig:vis-cluster}
}
\subfigure{
\includegraphics[width=1\textwidth]{Figures/noised_cluster_pi.png}
\label{fig:vis-noised-cluster}
}
\caption{Visualizations of the optimization process using InfoNCE Loss on the vectors corresponding to $\bfpi$. Points of identical color belong to the same cluster within $\bfpi$. To showcase the internal structure of $\bfpi$, we randomly select 10 vertices from each cluster to display the edge distribution of $\bfpi$.}
\label{fig:vis-spectral-cluster}
\end{figure}



\bibliographystyle{elsarticle-num}
\bibliography{curvilinear}

\end{document}



%% Notes
% Schemes - MP5, MEG, and MIG.
% A short assessment on the effect of grid quality is also performed, which revealed that the implicit algorithms are least affected by the grid skewness and grid distortion compared to the employment of explicit schemes.

% Best practices to predict supersonic jet screech - grid requirements, scheme requirements, 
% Include best practices for GPU acceleration
