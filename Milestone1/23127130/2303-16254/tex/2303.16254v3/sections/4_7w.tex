\begin{figure}[t]
\vspace{-0.4in}
\begin{center}
    \includegraphics[width=\linewidth]{figures/7w_pdb.pdf}
\end{center}
\caption{\textbf{Visualization of PEDV spike protein dataset. } On the left in each pair are our manually modified atomic models (PDB files)  in their intermediate states; on the right are their corresponding converted density fields (MRC files).
}
\vspace{-0.2in}
\label{fig:pdev}
\end{figure} 

\section{PEDV Spike Protein Dataset}
To evaluate CryoFormer and other heterogeneous cryo-EM reconstruction algorithms, we create a synthetic dataset of the spike protein of the \textit{porcine epidemic diarrhea virus} (PEDV).
%
The spike protein is a homotrimer, with each monomer containing a \textit{domain 0} (D0) region that modulates the enteric tropism of PEDV by binding to \textit{sialic acids} (SAs) on the surface of enterocytes~\citep{hou2017deletion} and can exist in both ``up'' and ``down'' states.
%
\citet{huang2022situ} determined the atomic coordinates and deposited them in the Protein Data Bank (PDB)~\citep{berman2000protein} under the accession codes \textit{7W6M} and \textit{7W73}.

%
We utilized \textit{Pymol}~\citep{delano2002pymol} to manually supplement the reasonable process of the movement of the D0 region in the format of intermediate atomic models (Fig.~\ref{fig:pdev} (a)). 
%
We converted these atomic models (PDB files) to discrete potential maps (MRC files) using \textit{pdb2mrc} module from \textit{EMAN2}~\citep{tang2007eman2},  which were then projected into 2D images (Fig.~\ref{fig:pdev} (b)).
%
We then simulate the image formation model as in Eqn.~\ref{eq:formation} at uniformly sampled rotations and in-plane translations.
%
On clean synthetic images, we add a zero-mean white Gaussian noise and apply the PSF.
%
We adjust the noise scale to produce the desired SNR such as $0.1, 0.01$ and $0.001$.
%
We will make the atomic models, density maps, and simulated projections publicly available.

