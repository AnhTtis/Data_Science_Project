\begin{figure*}[t]
\vspace{-0.3in}
\begin{center}
    \includegraphics[width=\linewidth]{figures/7w_fsc.pdf}
\end{center}
\caption{\textbf{Heterogenous reconstruction on PEDV spike dataset.} \textbf{Left:} Ground truth volume and reconstructed 3D volumes with SNR $=0.01$ and SNR $=0.001$. \textbf{Right: }Curves of the Fourier Shell Correlation (FSC) to the ground truth volumes. Our method produces a more refined reconstruction than both cryoDRGN and SFBP, especially in better recovery of the flexible D0 region across both noise scale levels. In addition, our approach yields the highest FSC curves.}
\label{fig:7w}
\vspace{-0.2in}
\end{figure*}
%
\section{Results}
%
In this section, we evaluate the performance of CryoFormer for homogeneous and heterogeneous cryo-EM reconstruction on two synthetic and two real experimental datasets and compare it with the state-of-the-art approaches. We also validate the effectiveness of our building components.
%
Please also kindly refer to our appendix and supplementary video. 

\noindent\textbf{Implementation Details} 
We adopt MLPs that contain 10 hidden layers of width 128 with ReLU activations for both the orientation encoder and the deformation encoder.
%
For the implicit spatial feature volume, we utilized a hash grid with $16$ levels, where the number of features in each level is $2$, the hashmap size is $2^{15}$, and the base resolution is $16$.
%
This hash grid is followed by a tiny MLP with one layer and hidden dimension $16$ to extract final spatial features.
%
For the query-based deformation transformer, we adopt $N=64$ queries with $C=64$ dimensions.
%
For synthetic datasets, we use ground truth poses for all the methods.
%
For real datasets, we use CryoSPARC~\citep{punjani2017cryosparc} for initial pose estimation (following \citet{cryodrgn} and \citet{sfbp}).
%
All experiments including training and testing have been conducted on a single NVIDIA GeForce RTX 3090 Ti GPU. 
%

\noindent\textbf{Reconstruction Metrics. } 
For quantitative evaluations, we employ the Fourier Shell Correlation (FSC) curves, defined as the frequency correlation between two density maps~\citep{harauz1986exact}. 
%
%
A higher FSC curve indicates a better reconstruction result.
%
For synthetic datasets, we compute FSC between the reconstructions and the corresponding ground truths and take the average if there are multiple conformational states. 
%
For real experimental datasets where the ground truth volume is unavailable, we compute FSC between two half-maps, each reconstructed from half the particle dataset.
%
For EMPIAR-10180 (real data with multiple states), we conduct a principal component analysis on the deformation latent space obtained from both reconstructions. By uniform sampling along the principal component axes, we obtain corresponding volumes at the same conformational states and report the average of 10 FSC curves.
%
We report the spatial resolutions of the reconstructed volumes, defined as the inverse of the maximum frequency at which the FSC exceeds a threshold ($0.5$ for synthetic datasets and $0.143$ for experimental datasets). 
%
\subsection{Heterogeneous Reconstruction on Synthetic Datasets}
%
We first evaluate CryoFormer on two synthetic datasets: 1) the synthetic dataset proposed by~\citep{cryodrgn}, generated from an atomic model of a protein complex (cryoDRGN synthetic dataset), and 2) our proposed PEDV spike dataset.
%
We compare CryoFormer against \textbf{cryoDRGN}~\citep{cryodrgn} (MLPs) and Sparse Fourier Backpropagation (\textbf{SFBP})~\citep{sfbp} (voxel grids) as representatives of coordinate-based methods. 
%

\noindent\textbf{CryoDRGN Synthetic Dataset.} This dataset contains $50,000$ images with size $D=128$ (pixel size $= 1.0 $\AA) and SNR $ =0.1 (-10\mathrm{dB})$ from an atomic model of a protein complex containing a 1D continuous motion~\citep{cryodrgn}. 
%
We demonstrate that our reconstruction aligns quantitatively with the ground truth in Tab.\ref{tab:syn}. 
%
Detailed results on the cryoDRGN synthetic dataset can be found in Sec.~\ref{sec:drgndataset}. 

\begin{wrapfigure}{r}{0.65\textwidth}
    \begin{minipage}{\linewidth}
        \centering
        \includegraphics[width=\linewidth]{figures/7w_multiple.pdf}
        \caption{\textbf{Qualitative comparison of the joint visualizations of multiple reconstructed states on PEDV spike dataset.} Our approach recovers all the conformational states and captures fine-grained details. In contrast, the baselines either exhibit lower spatial resolution or fail to capture all the states.}
        \label{fig:7w_multi}
        \vspace{0.1in}
    \end{minipage}
    \begin{minipage}{\linewidth}
        \fontsize{7.4}{8}\selectfont
        \begin{tabularx}{\linewidth}{l*{3}{>{\centering\arraybackslash}X}}
            \toprule
            & \textbf{CryoDRGN Synthetic} & \textbf{PEDV (SNR=0.01)} & \textbf{PEDV (SNR=0.001)} \\
            \midrule
            CryoDRGN & 3.45 & 6.5 & 19.21 \\
            SFBP & 2.18 & 6.06 & 20.8 \\
            Ours & \textbf{2.03} & \textbf{4.6} & \textbf{7.47} \\
            \bottomrule
        \end{tabularx}
        \captionof{table}{\textbf{Quantitative comparison for heterogeneous reconstruction on synthetic datasets.} Spatial resolution (in \AA, $\downarrow$) is quantified by an FSC=0.5 threshold.}
        \label{tab:syn}
    \end{minipage}
\end{wrapfigure}

\noindent\textbf{PEDV Spike Protein Dataset. } 
To further investigate CryoFormer's capability, we use the new PEDV spike protein dataset containing $50,000$ image with size $D=128$ (pixel size $= 1.6$\AA) to create more challenging experiment settings. 
%
These settings involve more complicated 3D density maps and lower signal-to-noise ratios (SNR).
%
%
We conducted experiments in two different levels of noise scale: SNR $ =0.01 (-20\mathrm{ dB})$ and SNR $ =0.001 (-30\mathrm{ dB})$.
%
As is shown in Fig.~\ref{fig:7w} (left panel), our method produces a more refined reconstruction than cryoDRGN and SFBP, with a better-recovered flexible D0 region under both levels of the noise scale. 
%
In Fig.~\ref{fig:7w_multi}, we jointly visualize the reconstructed states from various approaches with SNR \( =0.1 \) (\( -10\mathrm{ dB} \)).
%
Our results not only recover all the conformational states but also capture fine-grained details. In contrast, cryoDRGN's reconstructions exhibit lower spatial resolutions for details, and SFBP fails to capture all the states.
%
The quantitative results from Fig.\ref{fig:7w} (right panel) and Tab.\ref{tab:syn} indicate that our reconstruction outperforms competing approaches in terms of the FSC curve and the spatial resolution, with an exception in a low-frequency region where our curve marginally falls below that of SFBP.



\subsection{Homogeneous Reconstruction on Experimental Datasets}
To demonstrate CryoFormer's reconstruction performance on the real experimental data, we begin with homogeneous reconstruction on an experimental dataset with the ignorable biological motions from EMPIAR-10028~\citep{10028}, consisting of 105,247 images of the 80S ribosome downsampled to $D = 256$ (pixel size $= 1.88$\AA).
%
Our baselines include neural reconstruction approaches \textbf{cryoDRGN}~\citep{cryodrgn} and \textbf{SFBP}~\citep{sfbp} as well as a traditional state-of-the-art method \textbf{cryoSPARC}~\citep{punjani2017cryosparc}.
%
As illustrated in the left panel of Fig.~\ref{fig:10028}, our method manages to recover the shape and integrity of detailed structures like the $\alpha$-helices (as seen in the zoom-in region) in contrast to baseline approaches.
%
The right panel of Fig.~\ref{fig:10028} shows that our FSC curve consistently surpasses those of all the baselines, quantitatively demonstrating the accuracy of our reconstructed details.
%
For the resolution, defined as the inverse of the maximum frequency at which the FSC exceeds $0.143$, our approach achieves the theoretical maximum value of $3.80$\AA. This surpasses CryoDRGN ($3.93$\AA), SFBP ($6.19$\AA), and CryoSPARC ($8.63$\AA).





\begin{figure}[t]
\begin{center}
\vspace{-0.5in}
    \includegraphics[width=\linewidth]{figures/10028_comparison.pdf}
    \vspace{-0.4in}
\end{center}
\caption{\textbf{Homogeneous cryo-EM reconstruction on EMPIAR-10028.} \textbf{Left:} Reconstructed 3D volumes. \textbf{Right: }Curves of FSC between half-maps. Our method recovers detailed structures, such as the $\alpha$-helices in zoom-in regions, more clearly than baselines and achieves the highest FSC curve.}
\label{fig:10028}
\end{figure}
\begin{figure}[t]
\begin{center}
    \includegraphics[width=\linewidth]{figures/10180_comparison.pdf}
    \vspace{-0.3in}
\end{center}
\caption{\textbf{Heterogeneous cryo-EM reconstruction on EMPIAR-10180.} \textbf{Left:} Reconstructed 3D volumes. We display two states for each method. \textbf{Right: }Curves of FSC between two half-maps. Our method manages to recover continuous motions with a clearer outline of the secondary structure and achieve the highest FSC curve.}
\vspace{-0.2in}
\label{fig:10180}
\end{figure}



\subsection{Heterogeneous Reconstruction on Experimental Datasets}

To test CryoFormer's capability of heterogeneous reconstruction on real experimental datasets, we evaluate it on EMPIAR-10180~\citep{10180}, consisting of 327,490 images of a pre-catalytic spliceosome downsampled to $D=128$ (pixel size $= 4.2475$\AA).
%
We compare CryoFormer with \textbf{CryoDRGN}~\citep{cryodrgn}, \textbf{CryoSPARC}~\citep{punjani2017cryosparc} and \textbf{3DFlex}.
%
Notably, we follow the original paper of 3DFlex to use CryoSPARC's reconstructed volume as an input canonical volume reference.
%
As low-quality particles highly decrease CryoSPARC's reconstruction performance, we manually remove particles with lower quality after 2D classification to improve subsequent reconstruction performance and denote this result as \textbf{CryoSPARC*}.
%
In addition, we denote 3DFlex with CryoSPARC*'s reconstruction as the input canonical reference map as \textbf{3DFlex*}.
%
%
As shown on the left side of Fig.~\ref{fig:10180},  CryoSPARC and thus 3DFlex fail to provide reasonable reconstructions. Our method and CryoDRGN, CryoSPARC*, and 3DFlex* manage to maintain structural integrity during dynamic processes, while our reconstructions exhibit a clear outline of the secondary structure.
%
Quantitatively, as depicted on the right side of Fig.~\ref{fig:10180}, our method achieves the highest FSC curve.
%


\subsection{Evaluation}
\label{sec:evaluation}
To validate our architecture designs of CryoFormer, we conduct the following evaluations on our synthetic PEDV spike protein dataset.
%
We generate a dataset with 50,000 projections of the ground truth volume with SNR $=0.1$.
%
We sample particle rotations uniformly from $SO(3)$ space and particle in-plane translations uniformly from $[-10\text{pix.}, 10\text{pix.}]^2$ space. 
%
To simulate imperfect pre-computed poses, we perturb the ground truth rotations using additive noise ($\mathcal{N}(\mathbf{0}, 0.1\mathbf{I})$), and the translations using another uniform distribution $[-10\text{pix.}, 10\text{pix.}]^2$. 
%
We use the perturbed poses as simulated initial coarse estimations for the pre-training of our orientation encoder.
%
We report the resolution at the 0.5 cutoff in \AA\ and the errors of the final pose estimations.
%
We conduct analysis on the orientation encoder and the deformation encoder in Sec.~\ref{sec:analysis}.

\noindent\textbf{Orientation Encoder Refinement.} We test a variant of our method without fine-tuning the orientation encoder using image loss. As seen in Tab.~\ref{tab:ablation}, the refined orientation encoder fixes inaccurate initial estimations, and without refinement, the model cannot reconstruct structures reasonably.


\begin{wrapfigure}{r}{0.45\textwidth}
        \centering
        \vspace{-0.2in}
        \includegraphics[width=\linewidth]{figures/attention_map.pdf}
        \vspace{-0.1in}
        \caption{\textbf{Visualization of PEDV spike protein attention map.}
        %
        We map the attention value to the surface color of the reconstructed volume of PEDV spike protein.
        %
        The highlight (high attention value) reflects its flexible regions.
        %
        Every row shows three different states from the same perspective.
        }
        \vspace{-0.3in}
        \label{fig:attention}
\end{wrapfigure}
\noindent\textbf{Deformation Encoder.} We conducted experiments with CryoFormer without the deformation encoder for extracting deformation features. As evident from Tab.~\ref{tab:ablation}, this variant cannot accurately account for structural motions, resulting in a lower resolution.



\noindent\textbf{Real Domain Representation.} 
We conducted experiments using the Fourier domain variant of our method. As evident from Tab.~\ref{tab:ablation}, real domain reconstruction achieves significantly better resolutions and lower pose errors.


\noindent\textbf{Query-based Deformation Transformer Decoder.} 
%
To verify the effectiveness of our query-based deformation transformer decoder, we experimented with a variant of our method, replacing it with simple concatenation and an MLP. Tab.~\ref{tab:ablation} shows that the replacement will decrease CryoFormer's performance.
%
Furthermore, we can analyze attention maps to locate flexible regions. The 3D attention maps are computed at each coordinate through spatial cross-attention between its spatial feature and deformation-aware queries.
%
For visualization, we map the attention value of one channel to the surface color of the reconstructed volume. 
%
As shown in Fig.~\ref{fig:attention}, the displayed channel of attention map reflects a flexible region of the PEDV spike.

\begin{table}[t]
    \centering
    \vspace{-0.5in}
    \fontsize{7.4}{8}\selectfont

    \resizebox{1\textwidth}{!}{%
    \vspace{-0.5in}
    \begin{tabularx}{\textwidth}{*{7}{>{\centering\arraybackslash}m{1.6cm}}c}
        \toprule
        \textbf{Ori. Encoder Refinement} & \textbf{Deformation Encoder} & \textbf{Spatial Cross-Attention} & \textbf{Domain} & \textbf{Resolution($\downarrow$)} & \textbf{Rot. Error($\downarrow$)} & \textbf{Trans. Error($\downarrow$)} \\
        \midrule
        & \checkmark & \checkmark & Real & 26.1 & 0.144 & 0.138 \\
        \checkmark & & \checkmark & Real & 8.7 & 0.118 & 0.128 \\
        \checkmark & \checkmark & & Real & 6.5 & 0.066 & 0.036 \\
        \checkmark & \checkmark & \checkmark & Fourier & 7.3 & 0.088 & 0.021 \\
        \checkmark & \checkmark & \checkmark & Real & \textbf{4.1} & \textbf{0.030} & \textbf{0.018} \\
        \bottomrule
    \end{tabularx}
    }
    \vspace{-0.1in}
    \caption{\textbf{Quantitative ablation study.} Resolution is reported using the FSC $= 0.5$ criterion, in \AA. Rotation error is the mean square Frobenius norm between predicted and ground truth. Translation error is the mean L2-norm over the image side length.}
    \vspace{-0.2in}
    \label{tab:ablation}
\end{table}



\section{Discussion}

\noindent\textbf{Limitations.} 
As the first trial to achieve continuous heterogeneous reconstruction of protein in real space without the need for a 3D reference map, CryoFormer still suffers from some limitations.
%
Though we use hash encoding for efficient training and inferring, our method still demands a significant amount of computational resources and requires a long training time for full-resolution reconstruction due to we have to query every voxel in implicit feature volume for 3D projection operation, as discussed in Sec.~\ref{sec:time}.
%
Also, our orientation encoder depends on pre-training with initial pose estimation so CryoFormer cannot handle \textit{ab}-initio reconstruction while CryoFormer shows the capability to further refine them during the training.



\noindent\textbf{Conclusion.}
We have introduced CryoFormer for high-resolution continuous heterogeneous cryo-EM reconstruction.
%
Our approach builds an implicit feature volume directly in the real domain as the 3D representation to facilitate the modeling of local flexible regions.
%
Furthermore, we propose a novel query-based deformation transformer decoder to enhance the quality of reconstruction.
%
Our approach can refine pre-computed pose estimations and locate flexible regions.
%
Quantitative and qualitative experiment results show that our approach outperforms traditional methods and recent neural methods on both synthetic datasets and real datasets.
%
In the future, we believe real-domain neural reconstruction methods can play a greater role in cryo-EM applications.
%