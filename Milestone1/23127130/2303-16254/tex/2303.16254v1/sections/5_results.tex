\begin{figure}[t]
\begin{center}
    \includegraphics[width=\linewidth]{figures/flower_comparison.pdf}
\end{center}
\caption{\textbf{Qualitative comparison of different methods on cryoDRGN Synthetic Dataset. } cryoFormer reconstructed volumes qualitatively match the ground truth and baselines' reconstructed structures, with better recovery of details.
%
As shown in the right, \ours is capable of continuously modeling the conformational heterogeneity.
}
\label{fig:fan}
\end{figure}

\begin{figure*}[th]
\begin{center}
    \includegraphics[width=\linewidth]{figures/7w_fsc.pdf}
\end{center}
\caption{\textbf{Qualitative and quantitative comparison of different methods on PEDV spike sataset.} We compare \ours with cryoDRGN and SFBP on PEDV spike dataset, with SNR $=0.01$ and SNR $=0.001$. \textbf{Left:} Reconstructed 3D volumes. Under different levels of the noise intensity, the volumes reconstructed by \ours exhibit better restoration of details than the baseline. \textbf{Right: }The average FSC to each ground truth of for \ours, cryoDRGN, and SFBP, where a higher curve indicates better reconstruction.
}
\label{fig:7w}
\end{figure*}


\section{Results}
In this section, we present qualitative and quantitative evaluations of \ours for heterogeneous cryo-EM reconstruction. We verify \ours on two synthetic datasets by modeling conformational heterogeneity and recovering structural details. Next, we evaluate \ours on a real experimental dataset with minimal structural motions. We then evaluate \ours on a real experimental dataset with significant conformational heterogeneity to test cryoFormer's capability for heterogeneous reconstruction in real data.

We use ground-truth poses for synthetic datasets, and pre-computed poses for real experimental datasets, so we skip the orientation encoder from our architecture for better expositions. 
%
% We also conduct experiments in setting where accurate poses are not available and include these results in the appendix.

\textbf{Implementation details. } 
For the implicit spatial feature volume, we utilized a hash grid with $16$ levels, where the number of features in each level is $2$, the hashmap size is $2^{15}$, and the base resolution is $16$.
This hash grid is followed by a tiny MLP with one layer and hidden dimension $16$.
%
The choice of the query-based deformation transformer's parameters is discussed in Sec.~\ref{sec:eva}.
%
We trained our model on a single NVIDIA GeForce RTX 3090 Ti GPU. The training time depends on the number of images and their resolution in the dataset. It takes approximately 4 hours to train our method for a real dataset of around 100,000 images with a resolution of 128.

We provide additional experimental details and results in the supplementary materials, including video results, and encourage readers to refer to them.


\subsection{Heterogeneous Reconstruction on Synthetic Datasets}
We first evaluate \ours on two synthetic datasets: 1) the synthetic dataset proposed by~\cite{cryodrgn}, generated from an atomic model of a protein complex, and 2) our proposed PEDV spike dataset.

We compare \ours against cryoDRGN~\cite{cryodrgn} and Sparse Fourier Backpropagation (SFBP)~\cite{sfbp} as representatives of VAE-based methods. 
%
% We run smallDRGN with a small encoder and decoder (cryoDRGN-small) on the cryoDRGN synthetic dataset and run smallDRGN with larger networks (cryoDRGN-large) on PEDV Spike Dataset.

\noindent\textbf{CryoDRGN Synthetic Dataset.} This dataset contains $50$k image with size $D=64$ and SNR $ =0.1$ from an atomic model of a protein complex containing a 1D continuous motion~\cite{cryodrgn}. 

In Fig.~\ref{fig:fan}, we show that \ours reconstructed volumes qualitatively match the ground truth and baselines' reconstructed structures.
%
Note that while we only visualize 10 structures sampled along the latent space, we can reconstruct the full continuum of states.
%
CryoDRGN and SFBP also succeeded in continuously modeling the conformational heterogeneity but retaining fewer details than ours.

\noindent\textbf{PEDV Spike Protein Dataset. } We use our proposed PEDV spike dataset containing $20$k image with size $D=128$.
%
To test the performance of the algorithm under different levels of noise scale, we conducted experiments in two settings: SNR $ =0.01$ and SNR $ =0.001$.
%
We also compare the results with the ground truth as it is a synthetic dataset.

As is shown in Fig.~\ref{fig:7w} left, our method produces better reconstruction than cryoDRGN and SFBP, with better-recovered details under different levels of the noise scale.
%
Here we only present one state of heterogeneous reconstruction, and we leave out the presentation of others in the supplementary material.

We measured the Fourier shell correlation (FSC) between the reconstructed and the ground truth structures. 
%
As is done in \cite{sfbp}, the FSC is measured as the normalized complex correlation between Fourier components of the solvent-masked reconstruction and the known ground truth for each frequency band, averaged over 30 samples (three out of each conformation).
%
In  Fig.~\ref{fig:7w} right, we show the FSC curves of our method, cryoDRGN, and SFBP. 
%
A higher curve indicates better performance.
%
Under different levels of noise intensity, our quantitative results outperform all other baselines. 
%
Particularly at SNR = 0.001, the metrics of the baselines decrease significantly because of the failure to restore high-frequency information. 


\begin{figure}[t]
\begin{center}
    \includegraphics[width=\linewidth]{figures/10028_comparison.pdf}
\end{center}
\caption{\textbf{Qualitative comparison for homogeneous cryo-EM reconstruction on EMPIAR-10028.} Our method matches or even outperforms baselines in terms of reconstructing fine details of biological structures. The two rows show the same reconstructed structure viewed from different angles.
}
\label{fig:10028}
\end{figure}

\subsection{Homogeneous Reconstruction on Experimental Datasets}
Next, we present our experimental results of \ours on real experimental datasets. 
%
We begin by testing \ours on a dataset with ignorable biological motion, using it for homogeneous reconstruction. 
%
It is worth noting that although \ours is designed for the more challenging task of heterogeneous reconstruction, we retain the architecture of \ours without modification for homogeneous reconstruction.
%
We use the real dataset from EMPIAR-10028, which consists of 105,247 images of the 80S ribosome downsampled to an image size of $128$.
%
Here we perform homogeneous reconstruction with cryoDRGN~\cite{cryodrgn}, a representative neural representation-based method, and cryoSPARC~\cite{punjani2017cryosparc}, a traditional state-of-the-art method.

As is shown in Fig.~\ref{fig:10028}, reconstructed volumes from our method match or even surpass those from baseline methods, especially in certain details.
%
Since this is real data without available ground truth to quantitatively evaluate performance, we follow typical practice by splitting the full dataset into two halves, reconstructing each half separately, and computing the FSC between them. We included the FSC curves of this experiment in the appendix.


\subsection{Heterogeneous Reconstruction on Experimental Datasets}
To test \ours's capability of heterogeneous reconstruction, we evaluate it on a real experimental dataset containing significant heterogeneity.
%
We use the real dataset from EMPIAR-10076, which consists of 131,899 images of the \textit{E. coli} large ribosomal subunit (LSU) undergoing assembly.

We perform heterogeneous reconstruction using \ours, cryoDRGN~\cite{cryodrgn} and cryoSPARC~\cite{punjani2017cryosparc} (using multiclass refinement).
%
As shown in Fig.~\ref{fig:10076}, when processing data containing multiple conformational states, both cryoDRGN and cryoSPARC can only reconstruct the outline of the structure without retaining finer details.
%
In contrast, our method can successfully model the heterogeneity of the data while keeping fine details.

\begin{figure}[t]
\begin{center}
    \includegraphics[width=\linewidth]{figures/10076_comparison.pdf}
\end{center}
\caption{\textbf{Qualitative comparison for heterogeneous cryo-EM reconstruction on EMPIAR-10076.} Our method outperforms cryoDRGN and cryoSPARC in terms of the quality of heterogeneous reconstruction and is able to model the dynamic structures with finer details.}
\label{fig:10076}
\end{figure}






\subsection{Evaluation}
\label{sec:eva}
To validate our architectural designs, we conduct the following evaluations on the PEDV spike dataset.
%
We use a dataset with 20,000 projections with SNR $=0.01$. In all experiments, we run variants of \ours for 100 epochs to guarantee convergence.
%
We quantify the accuracy of the reconstructed volume by computing the FSC between the reconstruction and the ground truth and reporting the resolution at the 0.5 cutoff in \AA. 

\noindent\textbf{Implicit spatial feature volume.} Different from recent cryo-EM reconstruction using Fourier domain neural representations~\cite{cryodrgn, cryodrgn2, cryoai, cryofire}, \ours uses a spatial feature volume built in the spatial domain with hash grid~\cite{muller2022instant}. We evaluate two variants of \ours with changed neural representation. In Tab.~\ref{table:ablation}, ``positional encoding'' stands for using positional encoding with a large global MLP, and ``Fourier domain'' stands for building the hash grid neural representation in the Fourier domain.
%
Both variants show significantly decreased reconstruction resolution and increased time consumption.

\noindent\textbf{Deformation Transformer.} We design the query-based deformation transformer decoder to bring about the interaction between the deformation features and the coordinate features.
%
We evaluate a variant of \ours, where coordinate features and image deformation features are directly concatenated and fed into a single MLP to output a density value.
%
As in Tab.~\ref{table:ablation} ``w/o transformer'', it is more time-efficient but achieves worse reconstruction ability than ours.
%
We also change the parameter of our deformation transformer decoder and report their reconstruction resolution and time. We take $N=64,C=64$ as our default setting.

\begin{table}[t]
	\centering
	\small{
	\begin{tabular}{lcc}
	\multicolumn{3}{c}{ \colorbox{best1}{best} \colorbox{best2}{second-best} } \\
        \textbf{Method}        & \textbf{Resolution} $\mathbf{(\downarrow)}$ & \textbf{Time} $\mathbf{(\downarrow)}$ \\ \hline
        positional encoding  &6.18 & 8.3h\\
		Fourier domain &  9.99              & 2.11h\\ \hline
		w/o transformer       & 8.9                &\cellcolor{best1} 0.91h \\
	Ours $(N=32, C=32)$      & 5.34 &1.4h\\ 
Ours ($N=64, C=32$ )     & \cellcolor{best2}4.83   & 1.33h\\
Ours ($N=32, C=64$ )     & 4.93  & \cellcolor{best2}1.21h\\\hline
		\textbf{Ours}  $\mathbf{(N=64, C=64)}$         &\cellcolor{best1} 4.74 & 1.39h \\    \hline
    \end{tabular}
    }
\rule{0pt}{0.05pt}
\caption{\textbf{Quantitative evaluation on our design choices.} We evaluate the core design of our implicit spatial feature volume and deformation transformer decoder. When $N = 64, C=64$, our full model in the spatial domain achieves the highest performance of spatial resolution with little sacrifice of training speed.}
\label{table:ablation}
\end{table}

% 16x16:50s
% 32x32:50s
% C64N32 50s
% C32N64 50s
% C64N64 50s

\noindent\textbf{Interpretation on the 3D attention maps.} 3D attention maps are computed at each coordinate by spatial cross-attention between its spatial feature and deformation-aware queries.
%
The value is used as the surface color of the reconstructed volume. 
%
We manually set a threshold, the regions above it are colored in red, and the rest are colored in blue. 
%
As is shown in Fig.~\ref{fig:attention}, the displayed channel of attention map captures information about the regions of the PEDV spike that exhibit noticeable motion.

\begin{figure}[t]
\begin{center}
    \includegraphics[width=\linewidth]{figures/attention_map.pdf}
\end{center}
\caption{\textbf{Visualization of 3D attention map.}
%
We map the value in the attention map to the surface color of the reconstructed volume.
%
The displayed channel of the 3D attention map captures information about the flexible regions of the PEDV spike.
%
The top and bottom rows show the same attention map viewed from different angles.
}
\label{fig:attention}
\end{figure}

\section{Discussion}

\noindent\textbf{Limitations.} 
%
As the first trial to tackle continuous heterogeneous reconstruction of cryo-EM structures in the spatial domain, our approach suffers from some limitations.
%
Though we use hash encoding for efficient training and inferring, our method still demands a significant amount of computational resources and requires a lengthy training process that prevents us from processing raw full-resolution images of cryo-EM data.
%
Also, our pose module~\cite{cryoai, cryofire} cannot fully handle \textit{ab}-initio reconstruction without pre-computed poses. To support it, developing an improved pose module may indicate a future research direction.
% lxh:
% 时间和空间复杂度没有优势,导致不能有效率的运用全分辨率的原始图像
% pose module的设计比较naive 不能够支持ab-initio重建
% 未来可以考虑将pose estimation和ctf correction更好的融合进我们的方法来实现end-to-end的重建



\noindent\textbf{Conclusion.}
%
We have presented cryoFormer, the first approach that uses a transformer-based network for continuous heterogeneous cryo-EM reconstruction in the spatial domain.
%
% Our image feature encoders provide information on conformations and poses.
%
For efficient training and inference, we use hash encoding augmented by a small MLP to reconstruct a feature volume in the spatial domain.
%
Our novel deformation-aware transformer decoder combines conformation features and spatial features to generate the final spatial density volume.
%
Finally, Our transformer-based architecture supports a new capability of highlighting flexible regions of 3D structures.
%
The experimental results show that our method outperforms other methods in both synthetic and real datasets.
%
% We believe that our method can serve as a milestone of continuous heterogeneous reconstruction from Cryo-EM data in the spatial domain.
%
Our approach holds great promise to better understand the fundamental processes of life by revealing the unseen conformations of 3D biomolecules.
