\section{Video}
For better visualization of cryo-EM reconstruction results, we use ChimeraX~\cite{goddard2018ucsf} to create a set of video visualizations where 
reconstructions are rotated, and multiple reconstructed states are played in series. 
%
Please check the file \textit{supplementary\textunderscore video.mp4}.

\section{Cryo-EM}
Cryo-EM is a revolutionary imaging technique used to discover the 3D structure of macromolecules, including proteins and viruses. It enables structural biologists to understand how these miniature biological machines behave~\cite{kuhlbrandt2014resolution, nogales2016development, renaud2018cryo}.
%
In a typical cryo-EM experiment, a purified sample containing many instances of the specimen is obtained.
%
This sample is plunged into a cryogenic liquid, such as liquid ethane, which causes the molecules to freeze within a vitreous ice matrix. 
%
Once frozen, the sample is loaded into a transmission electron microscope and exposed to parallel electron beams (Fig.~\ref{fig:cryoem} (a)), leading to the acquisition of two-dimensional projection images of the Coulomb scattering potential of the molecules (Fig.~\ref{fig:cryoem} (b)). 
%
These raw micrographs are then processed by algorithms to reconstruct the volume.
%
Cryo-EM reconstruction is a challenging task due to several factors. Firstly, in the micrographs, the poses of the individual particles are unknown, and articles can also adopt multiple conformational states. Secondly, micrographs are acquired under low-dose conditions to limit damage to the radiation-sensitive particles, resulting in low signal-to-noise ratios (typically less than 0.1).

In the cryo-EM image formation model, the 3D biological structure is represented as a function  $\sigma: \mathbb{R}^{3} \mapsto \mathbb{R}^{+}$, which expresses the Coulomb potential induced by the atoms.
%
The probing electron beam interacts with the electrostatic potential, and ideally, one can formulate its clean projections without any corruption as
\begin{equation}
    \mathbf{I}_{\text{clean}}(x, y) = \int_{\mathbb{R}} \sigma(\mathbf{R}^{\top} \mathbf{x}+\mathbf{t})\, \mathrm{d}z , \quad \mathbf{x} = (x, y, z)^{\top},
\end{equation}
%
where $\mathbf{R} \in SO(3)$ is an orientation representing the 3D rotation of the molecule and $\mathbf{t} = (t_x, t_y,0)^\top$ is an in-plane translation corresponding to an offset between the center of projected particles and center of the image.
%
The projection is, by convention, assumed to be along the $z$-direction after rotation.
%
In practice, images are intentionally captured under defocus in order to
improve the contrast, which is modeled by convolving the clean images with a point spread function (PSF) $g$. 
%
All sources of noise are modeled with an additional term $\epsilon$ and are usually assumed to be Gaussian white noise. 
%
The image formation model considering PSF and noise is expressed as
\begin{equation}
    \mathbf{I} = g \star \mathbf{I}_{\text{clean}} + \epsilon.
\end{equation}

%
\begin{figure}[t]
\begin{center}
    \includegraphics[width=\linewidth]{figures_supp/cryoem.pdf}
\end{center}
\caption{\textbf{Illustration of a cryo-EM experiment. }(a) A sample containing molecules of interest is frozen in a thin layer of vitreous ice and exposed to parallel electron beams. (b) Two-dimensional projection images of the Coulomb scattering potential of the molecules. For clarity, we present simulated images with a signal-to-noise ratio (SNR) of $0.1$. Note that actual experimental data may suffer from more severe noise.}
\label{fig:cryoem}
\end{figure}

\begin{figure*}[t]
\begin{center}
    \includegraphics[width=\linewidth]{figures_supp/pca_volume.pdf}
\end{center}
\caption{\textbf{Our reconstructed states of PEDV spike protein.} We sample six cluster centers in Fig.~\ref{fig:7w_pca} and visualize the corresponding reconstructed states. The volumes, from left to right, correspond to the following clusters: 5, 6, 4, 2, 8, and 1.} 
\label{fig:7w_pca_vol}
\end{figure*}

\begin{figure}[t]
\begin{center}
    \includegraphics[width=\linewidth]{figures_supp/pca_10.png}
\end{center}
\caption{\textbf{Distribution of image deformation features.} We visualize the distribution of all the images' deformation features in 2D with PCA and color-code the points by their corresponding K-means labels. The cluster centers are annotated.
%
% match the volumes in Fig.~\ref{fig:7w_pca_vol}. The correspondence of point 1 is on the leftmost, and that of point 6 is on the rightmost.
}
\label{fig:7w_pca}
\end{figure}

\section{Baseline Settings}
\noindent\textbf{CryoDRGN~\cite{cryodrgn}.} All cryoDRGN experiments mentioned in this paper were run with its official repository on Github$^\dagger$.
\let\thefootnote\relax\footnote{$\dagger$ The repository is available at \url{https://github.com/zhonge/cryodrgn}.}
%
The setup is that the latent dimension is $8$, and both the encoder and decoder are MLPs with $3$ layers and $1024$ units per layer.

\noindent\textbf{Sparse Fourier Backpropagation~\cite{sfbp}.} 
As there is no open-source code available for SFBP, we re-implement the method. 
%
We followed the same setting as in the original paper, with an encoder consisting of five layers and a decoder consisting of a single linear layer, 256 units per layer. 
%
The number of structural bases used was 16.

\noindent\textbf{CryoSPARC~\cite{punjani2017cryosparc}.} We follow the software's defaults for all the settings.

\noindent\textbf{ResMFN~\cite{ResMFN}.} We use the official repository on Github$^\star$.
We follow the setting of using a 3-layer network with 128 hidden units to represent the 3D structure.
%
The network outputs the structure in 3 hierarchical scales, and  the training process is divided into 3 stages, one for each scale.
%
We spend respectively 25, 25, and 50 epochs optimizing the first to the third scales. 
\let\thefootnote\relax\footnote{$\star$The repository is available at \url{https://github.com/shekshaa/ResidualMFN}.}


\section{Additional Results}
\label{sec:supp_results}
In this section, we provide additional experimental details and results as a complement to Sec. 5.

\noindent\textbf{Heterogeneous reconstruction of PEDV spike protein.}
%In Sec. 5.1 we compare our method with cryoDRGN~\cite{cryodrgn,zhongreconstructing} and SFBP~\cite{sfbp} on PEDV spike protein. 
%
We provide further analysis of the heterogeneous reconstruction performance of cryoFormer on PEDV spike protein dataset. To verify the effectiveness of the deformation encoder, we perform principal component analysis (PCA) on the deformation features of all images and plot their distributions in Fig.~\ref{fig:7w_pca}.
We apply K-means algorithm with 10 clusters and color-code the points with their corresponding K-means labels.
%
We display the reconstructed volumes corresponding to 6 of the cluster centers in Fig.~\ref{fig:7w_pca_vol}.



To demonstrate the ability of different methods to handle conformational heterogeneity as well as their reconstruction performance in data, we qualitatively compared their reconstructions by showing the composition of multiple reconstructed conformations with SNR $=0.1$.
%
% We perform this experiment with SNR $=0.1$.
%
In Fig.~\ref{fig:7w_multi}, each reconstructed conformation is represented by a different color. It can be observed that our results contain the correspondences of all the states in the ground truth as well as the fine-grained details while the results of cryoDRGN and SFBP have lower spatial resolutions and thus poorer conformation correspondences to ground truth.

%The different states in the reconstructions of baselines are hard to distinguish and have poorer details.

% Since this is a synthetic dataset, we can verify if the reconstructed volume generated from the deformation feature of an input image is in the same state as the volume used to create that image.
% %
% We selected $50$ images and found that the reconstructed volumes generated from $44$ of them corresponded to their ground truth states. The accuracy is $88\%$.



\begin{figure*}[t]
\begin{center}
    \includegraphics[width=\linewidth]{figures_supp/7w_multi.pdf}
\end{center}
\caption{\textbf{Different conformational states of reconstruction.} Our method's reconstruction contains all the corresponding states present in the ground truth while maintaining the details. On the other hand, the reconstructions of the baselines are difficult to differentiate between different states and exhibit inferior details.}
\label{fig:7w_multi}
\end{figure*}

\begin{figure}[ht]
\begin{center}
    \includegraphics[width=\linewidth]{figures_supp/halfmap_curve.png}
\end{center}
\caption{\textbf{Quantitative comparison of different methods for reconstruction on real experimental datasets.} The FSC curves are computed between the reconstructions from two halves of the input. A higher curve means a better reconstruction.}
\label{fig:halfmap_curve}
\end{figure}

\noindent\textbf{Quantitative comparison on EMPIAR-10028.}
We quantitatively compare different approaches for cryo-EM reconstruction on the real experimental dataset EMPIAR-10028 by the typical practice of splitting the full dataset into two halves, reconstructing each half separately, and computing the FSC between them. 
%
We display the FSC curves in Fig.~\ref{fig:halfmap_curve}. 
CryoSPARC has the highest half-map FSC curve among the three methods.

\noindent\textbf{Multi-state qualitative comparison on EMPIAR-10076.}
%In Sec.~5.3, we perform heterogeneous reconstruction with our proposed cryoFormer,  cryoDRGN~\cite{cryodrgn,zhongreconstructing} and cryoSPARC~\cite{punjani2017cryosparc} on the real experimental dataset from EMPIAR-10076.
%
We provide a more comprehensive multi-state qualitative comparison on EMPIAR-10076 in Fig.~\ref{fig:10076_sup}.
%
We select three states from the reconstructions obtained by different methods and observe them from two views. 
%
For ours and CryoDRGN's results, we can always find three states while ours has more details.
%
For cryoSPARC's results, we could not find state 2 and it lacks details.
%
\begin{figure*}[t]
\begin{center}
    \includegraphics[width=\linewidth]{figures_supp/10076_supp_figure.pdf}
\end{center}
\caption{\textbf{Qualitative comparison for heterogeneous cryo-EM reconstruction on EMPIAR-10076.} We display three states from the reconstructions obtained by different methods and observe them from two views. Our reconstructions preserve finer details.}
\label{fig:10076_sup}
\end{figure*}



\section{Reconstruction without Accurate Pre-computed Poses}
%Although this work does not claim reconstruction without accurate poses as its main contribution, this section provides some discussion on this setting.
%
Pose refinement serves as a core step for preserving high-frequency details of reconstruction results though jointly estimating accurate poses and reconstructing 3D structures are chicken-egg problems, especially for heterogeneous reconstruction. Thus, we provide some discussions in this section.
%
For the experiments conducted in Sec.~5 and Sec.~\ref{sec:supp_results}, we assume that accurate poses for all images are known for better expositions, and therefore we directly use the given poses when running our approach and other methods. In this section, we compare our approach with ResMFN~\cite{ResMFN} for cryo-EM reconstruction in scenarios where accurate pre-computed poses are not available. 
%
ResMFN is a multi-scale coordinate network  suitable for tasks that involve coarse-to-fine training, and it has been shown to be effective in cryo-EM reconstruction without accurate poses.
%
In line with ResMFN's experimental setup, we initialize each pose parameter to a random value sampled uniformly from a geodesic distance of $45$ to $90$ degrees from the ground truth, which can be considered as a  crude initial estimate for poses.
%
To initialize our image orientation encoder, we use all the image-pose pairs to pre-train it until convergence. 
%
We then evaluate both methods on the PEDV spike dataset with $20000$ images and SNR $=0.1$. 
%
As shown in Fig.~\ref{fig:pose_fsc}, our reconstruction is inferior to ResMFN, which could possibly be attributed to ResMFN's coarse-to-fine training strategy.
%
Note that this experimental setup differs from ab-initio cryo-EM reconstruction, where no form of information regarding poses is given, including coarse estimates.
%
Our method is not capable of performing ab-initio cryo-EM reconstruction like cryoDRGN2~\cite{cryodrgn2}, cryoAI~\cite{cryoai} and cryoFIRE~\cite{cryofire}.
%
To address this limitation, future improvements may involve incorporating more refined pose estimation modules into our method.

\begin{figure}[t]
\begin{center}
    \includegraphics[width=\linewidth]{figures_supp/curve_pose.png}
\end{center}
\caption{\textbf{Quantitative comparison of ours and ResMFN for reconstruction without accurate poses.} FSC curves of cryoFormer and ResMFN for cryo-EM reconstruction without accurate pre-computed poses. A higher curve means better reconstruction.}
\label{fig:pose_fsc}
\end{figure}