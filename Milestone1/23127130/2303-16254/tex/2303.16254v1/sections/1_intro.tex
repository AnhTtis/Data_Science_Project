\section{Introduction}
% BEGIN: ready for review
\begin{figure}[t]
\begin{center}
    \includegraphics[width=\linewidth]{figures/new_teaser.pdf}
\end{center}
\caption{With noisy images as inputs, our method can successfully reconstruct the heterogeneous structures of proteins and highlight the local, flexible motions.}
\label{fig:teaser}
\end{figure}

% 1. The importance of protein movements.
Dynamic objects as giant as planets and as minute as proteins constitute our physical world and produce nearly infinite possibilities of life forms. 
%
While computer vision has largely focused on modeling such object motions and shape variations at a macro scale, it is equally important to conduct a similar analysis on micro-scale proteins and other biomolecules, which helps reveal the fundamental processes of biodynamics as well as facilitates drug development~\cite{kuhlbrandt2014resolution, nogales2016development, renaud2018cryo}.
%
With recent advances in imaging techniques and data processing methods, cryo-electron microscopy (cryo-EM) can now capture hundreds of thousands of high-resolution images of biomolecules of random orientations under varying conformations. 
%
The collected dataset can be potentially used for recovering protein shapes and dynamics at an atom level via the single-particle analysis (SPA) technique.

% 2. major challenges
The latest SPA solutions were unanimous extensions of existing shape recovery techniques designed to handle macro-scale objects. In reality, cryo-EM differs from classical imaging techniques in many ways.
%
First, cryo-EM uses low electron dosage to minimize the damage to biological molecules, resulting in extremely noisy images. 
%
Further, the obtained protein images have unknown orientations that are difficult to calibrate due to a lack of features. 
%
Last, proteins exhibit conformational heterogeneity, i.e., the presence of continuous structural states of biological molecules. 
%
In particular, very few solutions can robustly recover multiple conformations of 3D structures in cryo-EM~\cite{singer2020computational}.
%However, the performance of single-particle analysis is still limited due to several issues, 1) Low electron dosage to minimize the damage of biological molecules, resulting in extremely noisy images, 2) unknown orientations of individual particles, and 3) conformational heterogeneity, i.e., the presence of continuous structural states of biological molecules.
%
% Among these challenges, the heterogeneity of molecules has become a major bottleneck of existing pipelines. The task to recover multiple conformations of 3D structures in Cryo-EM is called \textbf{heterogeneous cryo-EM reconstruction}~\cite{singer2020computational}.
%
% Researchers have developed algorithms that use SPA to visualize and study complex conformations of dynamic biological molecules and extract multiple structures, termed \textbf{heterogeneous cryo-EM reconstruction}~\cite{singer2020computational}.

% 3. Related work
Existing software packages~\cite{scheres2012relion, punjani2017cryosparc} for high-resolution cryo-EM reconstruction typically adopt conventional approaches that may assume the 3D structure to be static or can be classified into a discrete set of conformations, resulting in blurred and low-resolution results of flexible regions with continuous motions.
%
% Non-uniform refinement~\cite{punjani2020non} uses an adaptive regularizer to improve the spatial resolution of flexible regions but still cannot handle continuous motions.
% Recently, advanced methods in computer vision have shown compelling results in reconstructing the non-rigid continuous motion of humans using dynamic neural radiance fields~\cite{zhao2022human}. 
%
Recently, there have been some neural approaches proposed for continuous heterogeneous cryo-EM reconstruction~\cite{cryodrgn, cryodrgn2, cryofire, gupta2020multi, punjani20213d}.
%
Similar to NeRF~\cite{mildenhall2020nerf}, cryoDRGN~\cite{cryodrgn} uses an MLP to learn a continuous volumetric representation in the Fourier domain and additional latent space to model heterogeneity. cryoFIRE~\cite{cryofire} adds a pose encoder for \textit{ab}-initio heterogeneous reconstruction.
%
However, such methods relying on Fourier domain heterogeneous reconstruction are hard to achieve high spatial resolution due to local continuous motions of flexible regions corresponding to a global discontinuous change in the Fourier domain.
%
Flex3D~\cite{punjani20213d} tends to improve the resolution of flexible regions by generating 3D deformation fields in the spatial domain.
%
But it requires an additional 3D canonical map as input.
%and it is hard to model large motions and topological changing~\cite{park2021hypernerf}.
% Following conventional methods, there are many approaches for continuous heterogeneous Cryo-EM reconstruction in the Fourier domain but no one has tried in the spatial domain before.
%
% However, they lack interpretability regarding heterogeneity, as they cannot highlight the dynamic components of macromolecular complexes. Furthermore, their spatial resolution of reconstruction may suffer from imperfect modeling of the highly discontinuous Fourier space.
% However, revealing the continuous heterogeneity of proteins and other biological molecules remains challenging

%TODO
%However, revealing the continuous heterogeneity of proteins and other biological molecules remains challenging due to the extremely comprehensive data collecting and processing pipelines.
%
%In a SPA experiment, a purified specimen containing multiple copies of the target molecule in random orientations is flash-frozen to cryogenic temperatures and exposed to an electron beam in a transmission electron microscope to capture an image (Fig.~\ref{fig:teaser} (a)).
%
%The electron scattering potential of the target molecule, which is a 3D volume, is projected onto the 2D detector to form raw micrographs of particles. After pre-processing and particle-picking procedures, the resulting particle dataset may contain hundreds of thousands of particle images. (Fig.~\ref{fig:teaser} (b)).
%
%These images are then processed by existing conventional pipelines, such as RELION~\cite{scheres2012relion} or cryoSPARC~\cite{punjani2017cryosparc}, to reconstruct a single 3D volume or multiple volumes for discrete conformations of the target molecule.
%%%%%%% place holder
% Jiakai:1. 第一个空间域基于projection模型的动态重建算法,在synthetic数据上达到非常高的分辨率,maybe claim达到neural的sota
% 2. 在真实数据上进行了测试,得到了真实数据的动态序列
% 3. 基于空间域的重建和attention模块enable了全新的visualization ability XXX

In this paper, we propose a novel approach, cryoFormer, that utilizes a transformer-based network architecture for continuous heterogeneous cryo-EM reconstruction. We, for the first time, directly reconstruct continuous conformations
of 3D structures using an implicit feature volume in the 3D spatial domain to model local changes of conformations.  A deformation transformer decoder further improves reconstruction quality and, more importantly, locates and robustly tackles flexible 3D regions caused by conformations, as illustrated in Fig.~\ref{fig:teaser}.

%
To efficiently model continuous changes of 3D local structures in the spatial domain, we first use an implicit spatial feature volume represented by hash encoding~\cite{muller2022instant} augmented by a small MLP to reduce the computational costs while preserving geometric details significantly.  Next, we deploy a pair of image encoders~\cite{cryodrgn, cryoai} to extract the orientation feature and deformation feature, respectively. 
%
To generate the final density volume based on the extracted deformation and spatial features, we introduce a novel query-based deformation transformer decoder where both features are fused into structure queries. The queries, each corresponding to a specific part of the structure, first extract deformation information from an image by deformation-aware cross-attention and then integrate the spatial feature by spatial-wise cross-attention.
%
Compared with directly using a pure MLP architecture to decode density, our transformer-based architecture is more effective and robust in fusing different modality information as well as visualizing flexible regions of 3D structures caused by varying conformations. 


We validate cryoFormer on three public datasets (one synthetic and two experimental) and a new synthetic dataset of PEDV spike protein and achieve state-of-the-art reconstruction performance in terms of spatial resolution on both synthetic and real datasets, surpassing a number of baselines, including popular traditional softwares as well as neural approaches.
%
% While sharing the advantage of other neural representation-based methods in being able to model heterogeneity continuously,  we are able to recover the details of the reconstructed protein better. 
%
Our method has the potential to address the limitations of traditional methods in reconstructing dynamic regions in biological structures, which often imply functional areas.
%
We hope our approach can benefit the understanding of the mechanisms of molecular interactions.

% To summarize, our main contributions are as follows:

% \begin{itemize}
%     \item We propose a novel transformer-based pipeline for cryo-EM continuous heterogeneous reconstruction. It is the first approach that directly reconstructs the continuous conformation of 3D structures in the spatial domain. 
%     \item We show the new capability of highlighting 3D flexible regions for better interpretability with the help of our novel transformer-based architecture design.
%     \item We apply our proposed approach to three public datasets and one newly created synthetic dataset. We achieve state-of-the-art reconstruction performance in terms of spatial resolution on both synthetic and real datasets.
    
%     % novel visualization methods for flexible regions by showing 3D attention maps of the transformer decoder to improve the interp of continuous heterogeneous results.
% \end{itemize}