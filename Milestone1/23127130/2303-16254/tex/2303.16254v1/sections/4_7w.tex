
\begin{figure}[t]
\begin{center}
    \includegraphics[width=\linewidth]{figures/7w_pdb.pdf}
\end{center}
\caption{\textbf{Visualization of PEDV spike protein dataset. } \textbf{(a)} Our manually modified atomic models (PDB files)   in the intermediate states. \textbf{(b)} Their corresponding converted density fields (MRC files).
}
\label{fig:pdev}
\end{figure} 

\section{PEDV Spike Protein Dataset}
To better evaluate algorithms for heterogeneous cryo-EM reconstruction, we create a synthetic dataset based on the spike protein of the porcine epidemic diarrhea virus (PEDV).
%
% The trimeric spike protein of PEDV plays a crucial role in virus-host recognition and membrane fusion and is a primary target for vaccine development and antigen analysis. 
% %
% Studying the structure of the spike protein is a major focus in virus research.
% %
The spike protein is a homotrimer, with each monomer containing a domain 0 (D0) region that modulates the enteric tropism of PEDV by binding to sialic acids (SAs) on the surface of enterocytes~\cite{hou2017deletion} and can exist in both ``up'' and ``down'' states.
%
\cite{huang2022situ} determined the atomic coordinates of PEDV PT52 S and deposited them in the Protein Data Bank (PDB)~\cite{berman2000protein} under the accession codes 7W6M and 7W73 .

% During the 3D classification of the structure, the D0 region is classified into three categories based on its state: ``down'', ``up'', and ``mix''. 
% %
% The ``mix'' and ``down'' states appear to be quite similar, with the ``mix'' state being a combination of many intermediate states. 
% %
% The classification logic fails to address this issue, leading to ambiguity.
%
We utilized Pymol software~\cite{delano2002pymol} to manually supplement the reasonable process of the movement of the D0 region  in the format of intermediate atomic models (Fig.~\ref{fig:pdev} (a)). 
%
We converted these atomic models (PDB files) to discrete potential maps (MRC files) using \textit{pdb2mrc} module from EMAN2 software~\cite{tang2007eman2},  which were then projected into 2D images (Fig.~\ref{fig:pdev} (b)).
%
We then simulate the image formation model as in Eqn.~\ref{eq:formation} at $20$k rotations uniformly sampled on $SO(3)$.
%
We model $\epsilon$ as zero-mean white Gaussian noise and apply the PSF.
%
We adjust the noise scale to produce the required SNR.
%
We will make the atomic models, density maps, and simulated projections publicly available with the aim of advancing cryo-EM reconstruction algorithms.

