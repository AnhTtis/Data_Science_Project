\newpage
\renewcommand\thefigure{\Alph{figure}} 
\renewcommand\thetable{\Alph{table}}  
\setcounter{figure}{0}
\setcounter{table}{0}
\section*{Appendix}
\section{Video}
For better visualization of cryo-EM reconstruction results, we use ChimeraX~\citep{goddard2018ucsf} to create a set of video visualizations with free-viewpoint rendering and multiple conformational states. Please refer to \textit{supplementary\textunderscore video.mp4} for more results and evaluations of CryoFormer.

\section{Imaging Model of Cryo-EM}
\label{sec:cryoem}
Cryo-EM is a revolutionary imaging technique used to discover the 3D structure of biomolecules, including proteins and viruses. 
In a typical cryo-EM experiment, a purified sample containing many instances of the specimen is plunged into a cryogenic liquid, such as liquid ethane.
%
This causes the molecules to freeze within a vitreous ice matrix. 
%
The frozen sample is loaded into a transmission electron microscope and exposed to parallel electron beams (Fig.~\ref{fig:cryoem} (a)), resulting in projections of the Coulomb scattering potential of the molecules (Fig.~\ref{fig:cryoem} (b)). 
%
These raw micrographs can then be processed by algorithms to reconstruct the volume.
%

\begin{figure}[h]
\begin{center}
    \includegraphics[width=0.45\linewidth]{figures_supp/cryoem.pdf}
\end{center}
\caption{\textbf{Illustration of a cryo-EM experiment. }(a) A sample containing molecules of interest is frozen in a thin layer of vitreous ice and exposed to parallel electron beams. (b) Two-dimensional projection images of the Coulomb scattering potential of the molecules.}
\label{fig:cryoem}
\end{figure}

In the cryo-EM image formation model, the 3D biological structure is represented as a function  $\sigma: \mathbb{R}^{3} \mapsto \mathbb{R}^{+}$, which expresses the Coulomb potential induced by the atoms.
%
The probing electron beam interacts with the electrostatic potential, and ideally, one can formulate its clean projections without any corruption as
\begin{equation}
    \mathbf{I}_{\text{clean}}(x, y) = \int_{\mathbb{R}} \sigma(\mathbf{R}^{\top} \mathbf{x}+\mathbf{t})\, \mathrm{d}z , \quad \mathbf{x} = (x, y, z)^{\top},
\end{equation}
%
where $\mathbf{R} \in SO(3)$ is an orientation representing the 3D rotation of the molecule and $\mathbf{t} = (t_x, t_y,0)^\top$ is an in-plane translation corresponding to an offset between the center of projected particles and center of the image.
%
The projection is, by convention, assumed to be along the $z$-direction after rotation.
%
In practice, images are intentionally captured under defocus in order to improve the contrast, which is modeled by convolving the clean images with a point spread function (PSF) $g$. 
%
All sources of noise are modeled with an additional term $\epsilon$ and are usually assumed to be Gaussian white noise. 
%
The image formation model considering PSF and noise is expressed as
\begin{equation}
    \mathbf{I} = g \star \mathbf{I}_{\text{clean}} + \epsilon.
\end{equation}

\section{Structural Deformations in Different Domains}
To elucidate the advantages of reconstructing motions in the real domain, we use two neighbor conformational states of the PEDV spike protein as an illustration. As shown in Fig.~\ref{fig:realfreq}, there are two volumes colored in grey and yellow that represent state 0 and state 1, respectively,  the difference between these two states is manifested as a local motion in the real domain. However, in the Fourier domain, they present a  global difference.

\begin{figure}[h]
\begin{center}
    \includegraphics[width=0.45\linewidth]{figures_supp/diff.pdf}
\end{center}
\caption{Visualization of two conformational states and their differences in the real domain and the Fourier domain.}
\label{fig:realfreq}
\end{figure}

Fourier reconstruction methods such as CryoDRGN~\citep{cryodrgn}, require the decoder to model the global and large value changes in the Fourier domain between two neighbor states, which should have very similar conformational embedding from the image encoder. However, our approach performs reconstruction in real domain so the neural representation only needs to model local and small changes between two neighbor conformational states.

\section{Experiment Details}
\subsection{Datasets}
We adopt a number of synthetic and real experimental datasets in our experiments. We show sample images in Fig.~\ref{fig:dataset} and list parameters in Tab.~\ref{tab:dataset}.
\begin{figure}[h]
\begin{center}
    \includegraphics[width=1.0\linewidth]{figures_supp/dataset_vis_appendix.pdf}
\end{center}
\caption{\textbf{Sample images from synthetic and experimental datasets.}}
\label{fig:dataset}
\end{figure}
\begin{table}[h]
    \centering
    \fontsize{8}{8}\selectfont
    
    \begin{tabular}{lcccc}
        \toprule
        Dataset & Number of Particles & Image Resolution (pixel) & Pixel Size (\AA) & Number of States \\
        \midrule
       PEDV Spike Protein & 50,000 & 128 & 1.60 & 10  \\
        CryoDRGN Synthetic & 50,000 & 128 & 1.00 & 10  \\
        EMPIAR-10028 & 105,247 & 256 & 1.88 & N/A  \\
        EMPIAR-10180 & 327,490 & 128 & 4.25 & N/A  \\
        \bottomrule
    \end{tabular}
    \caption{\textbf{Summary of the parameters for synthetic and experimental datasets.}}
    \label{tab:dataset}
\end{table}

\subsection{Baseline Settings}
\noindent\textbf{CryoDRGN~\citep{cryodrgn}.} We use the official repository\footnote{The repository is available at \url{https://github.com/zhonge/cryodrgn}.}. The number of latent dimensions is $8$, and both the encoder and decoder have $3$ layers and $1024$ units per layer.

\noindent\textbf{Sparse Fourier Backpropagation~\citep{sfbp}.} 
As there is no open-source code available for SFBP, we re-implement the method, following the same setting as in the original paper, with an encoder consisting of five layers and a decoder consisting of a single linear layer, 256 units per layer. The number of structural bases is 16.

\noindent\textbf{CryoSPARC~\citep{punjani2017cryosparc} and 3DFlex~\citep{punjani20213d}.}
%
We use CryoSPARC v4.2.1. For homogenerous dataset, we follow the typical workflow (import particle stacks, perform \textit{ab}-initio reconstruction before homogeneous refinement) with default parameters. For heterogeneous dataset, we change the number of classes parameter in \textit{ab}-initio reconstruction job to 5, and run heterogenous refinement between \textit{ab}-initio reconstruction and homogeneous refinement. We run 3DFlex following the official tutorial with default parameters using CryoSPARC's reconstruction map as input canonical maps.

\section{Experimental Results (Cont'd)}
\subsection{CryoDGRN Synthetic Dataset Result}
\label{sec:drgndataset}
In Fig.\ref{fig:fan} (left and middle panels), we demonstrate that our reconstruction not only aligns qualitatively with the ground truth but is also adept at capturing the dynamics of the structure.
Notably, although we only illustrate 10 structures sampled from various points along the latent space, our approach is capable of reconstructing the full continuous conformational states.
% 
As shown in Fig.~\ref{fig:fan} (right panel), our reconstruction's FSC curve is predominantly higher than that of the baselines, with a singular exception where it falls marginally below cryoDRGN in a low-frequency region. 

\begin{figure}[h]
\begin{center}
    \includegraphics[width=\linewidth]{figures/fan.pdf}
\end{center}
\caption{\textbf{Heterogenous reconstruction on the cryoDRGN synthetic dataset. }\textbf{Left:} The ground truth volume and reconstructions from our approach and baselines. \textbf{Middle:} Multiple conformational states of the ground truth and our reconstruction. \textbf{Right:} Curves of the Fourier Shell Correlation (FSC) to the ground truth volumes. Our reconstruction qualitatively aligns with the ground truth, with an FSC curve predominantly higher than baselines'.}
\label{fig:fan}
\vspace{-0.2in}
\end{figure}

\subsection{Analysis}
\label{sec:analysis}
To better understand the behaviors of our approach and its building components, we conduct the following analysis on the synthesized datasets from the PEDV spike proteins. In this section, we follow the same experimental setting as Sec.~\ref{sec:evaluation}.

\noindent\textbf{Deformation Encoder.} To analyze our deformation encoder, we perform principal component analysis (PCA) on the deformation features of all images and plot their distributions in Fig.~\ref{fig:7w_pca}.
We apply K-means algorithm with 10 clusters and color-code the points with their corresponding K-means labels.
%
We display the reconstructed volumes corresponding to 6 of the cluster centers in Fig.~\ref{fig:7w_pca_vol}.
\begin{figure}[h]
\begin{center}
\vspace{-0.2in}
    \includegraphics[width=0.48\linewidth]{figures_supp/pca_10.png}
\end{center}
\vspace{-0.1in}
\caption{\textbf{Distribution of deformation features.} We visualize the distribution of all the images' deformation features in 2D with PCA and color-code the points by their corresponding K-means labels. The cluster centers are annotated.}
\label{fig:7w_pca}
\end{figure}

\begin{figure}[h]
\begin{center}
\vspace{-0.1in}
    \includegraphics[width=0.9\linewidth]{figures_supp/pca_volume.pdf}
\end{center}
\caption{\textbf{Reconstructed states of PEDV spike protein.} We sample six cluster centers in Fig.~\ref{fig:7w_pca} and visualize the corresponding reconstructed states. The volumes, from left to right, correspond to the following clusters: 5, 6, 4, 2, 8, and 1.} 
\vspace{-0.1in}
\label{fig:7w_pca_vol}
\end{figure}



\noindent\textbf{Orientation Encoder Refinement. }
To analyze our orientation encoder, we visualized the initial pose estimation and the refined pose in Fig.\ref{fig:pose} (left panel) and the video. It can be seen that after the refinement of the orientation encoder via image loss, the previously inaccurate initial pose estimation has been optimized and aligns with the ground truth. In Fig.\ref{fig:pose} (right panel), we compare multiple reconstructed states of the variant without the refinement approach and our full model. It can be observed that if we do not refine the pose estimation, the initial inaccurate pose estimation results in a very low resolution of the reconstruction.
\begin{figure}[h]
\begin{center}
    \includegraphics[width=\linewidth]{figures_supp/pose_supp_map.pdf}
\end{center}
\caption{ \textbf{Left:} Visualization of the initial poses and the predicted poses from orientation encoder after pre-training and refinement. We draw the predicted pose with a smaller size for better distinction. \textbf{Right:} Reconstructed volumes without and with the refinement of the orientation encoder.} 
\label{fig:pose}
\end{figure}

{\noindent\textbf{Running Time. }}
\label{sec:time}
To demonstrate the runtime of our algorithm, we present the execution times on a single NVIDIA GeForce RTX 3090 Ti GPU in Tab.~\ref{tab:time}. We also report the runtime of CryoDRGN as a reference. The reported time corresponds to 20 epochs. The timings for our method include both the pre-training of the orientation encoder (200 epochs) and the training of the system (20 epochs). As can be observed, when the image size is 128, our algorithm and CryoDRGN have comparable runtimes. However, for a larger image size of 256, our approach takes significantly more time. This may be attributed to the increased time complexity of the spatial cross-attention mechanism in our method when processing high-resolution images.
%
\begin{table}[h]
    \centering
    \fontsize{8}{8}\selectfont
    
    \begin{tabular}{lcc}
        \toprule
        Dataset & Ours & CryoDRGN \\
        \midrule
        CryoDRGN Synthetic & 2h22min & 2h48min \\
        PEDV Spike Protein & 3h12min & 3h16min \\
        EMPIAR-10028 & 31h14min & 8h12min \\
        EMPIAR-10180 & 18h34min & 19h50min \\
        \bottomrule
    \end{tabular}
    \caption{\textbf{Comparison of processing times between our method and CryoDRGN.}}
    \label{tab:time}
\end{table}
