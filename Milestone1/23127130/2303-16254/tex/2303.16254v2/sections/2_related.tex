\section{Related Work}
\noindent\textbf{Conventional Cryo-EM Reconstruction.}
Traditional cryo-EM reconstruction involves the creation of a low-resolution initial model~\citep{leschziner2006orthogonal, punjani2017cryosparc} followed by the iterative refinement~\citep{scheres2012relion, punjani2017cryosparc, hohn2007sparx}.
%
These algorithms perform reconstruction in the Fourier domain since this can reduce computational cost via Fourier-slice Theorem~\citep{bracewell1956strip}.
%
When tackling structural heterogeneity, they classify conformational states into several discrete states~\citep{scheres2010maximum, lyumkis2013likelihood}.
%
While this paradigm is sufficient when the structure has only a small number of discrete conformations, it is nearly impossible to individually reconstruct every state of a protein with continuous conformational changes in a flexible region~\citep{10180}.
%
\begin{figure*}[t]
\begin{center}
\vspace{-0.3in}
    \includegraphics[width=1\linewidth]{figures/method-final.pdf}
\end{center}
\caption{\textbf{Pipeline of CryoFormer.}
%
\textbf{1)} Given an input image, our orientation encoder and deformation encoder first extract orientation representations and deformation features. We use pre-computed pose estimations to pre-train the orientation encoder.
%
\textbf{2)} 
We convert the orientation representation into a pose estimation and transformed coordinates are fed into our implicit neural spatial feature volume to produce a spatial feature.
\textbf{3)} 
The spatial feature and the deformation image feature then interact in the deformation transformer decoder to output the density prediction. 
}
\vspace{-0.2in}
\label{fig:method}
\end{figure*}

\noindent\textbf{Dynamic Neural Representations.} 
Neural Radiance Fields (NeRFs)~\citep{mildenhall2020nerf} and their subsequent variants~\citep{muller2022instant, kerbl20233d} have achieved impressive results in novel view synthesis.
%
Numerous studies have introduced extensions of NeRF for dynamic scenes~\citep{xian2021space,li2021neural,li2022neural,park2021hypernerf,yuan2021star,fang2022fast,song2023nerfplayer}. 
%
Most of these dynamic neural representations either construct a static canonical field and use a deformation field to warp this to the arbitrary timesteps~\citep{dnerf, tretschk2021non, zhang2021stnerf, park2021nerfies}, or represent the scene using a 4D space-time grid representation, often with planar decomposition
or hash functions for efficiency~\citep{shao2023tensor4d, attal2023hyperreel, cao2023hexplane, fridovich2023k}. 
%

\noindent\textbf{Neural Representations for Cryo-EM Reconstruction.} Recent work has widely adopted neural representations for cryo-EM reconstruction~\citep{cryodrgn, cryoai, cryofire,ResMFN}.
%
CryoDRGN~\citep{cryodrgn} first proposed a VAE architecture to encode conformational states from images and decode it by an coordinated-based MLP that represents the 3D Fourier volume.
%
Such a design can model continuous heterogeneity of protein and achieve higher spatial resolution compared with traditional methods. 
%
To reduce the computational cost of large MLPs, \citet{sfbp} uses a voxel grid representation. 
%
% \citep{ResMFN} enables coarse-to-fine optimization with control over the frequency spectrum.
%
To enable an end-to-end reconstruction, there are some \textit{ab}-initio neural methods~\citep{cryodrgn2, cryoai, cryofire, chen2023ace} directly reconstruct protein from images without requiring pre-computed poses from traditional methods.
%, as well as cryo-electron tomography (cryo-ET) reconstruction~\citep{drgnET}.
%
CryoFIRE~\citep{cryofire} attempts to use an encoder to estimate poses from input image by minimizing reconstruction loss directly, but the performance is still limited due to the ambiguity of conformation and orientation in the extremely noisy image.

To model the 3D local motion, \citet{punjani20213d} and \citet{chen2021deep}  perform reconstruction in the real domain by using a flow field to model the structural motion, while they require a canonical structure as input.

% 