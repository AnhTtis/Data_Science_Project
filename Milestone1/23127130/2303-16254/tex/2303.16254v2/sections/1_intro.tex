\section{Introduction}

Dynamic objects as giant as planets and as minute as proteins constitute our physical world and produce nearly infinite possibilities of life forms.
%
Their 3D shape, appearance, and movements reflect the fundamental law of nature.
%
Conventional computer vision techniques combine specialized imaging apparatus such as dome or camera arrays with tailored reconstruction algorithms (SfM~\citep{schonberger2016structure} and most recently NeRF~\citep{mildenhall2020nerf}) to capture and model the fine-grained 3D dynamic entities at an object level.
%
Similar approaches have been adopted to recover shape and motion at a micro-scale level.
%
In particular, to computationally determine protein structures, cryo-electron microscopy (cryo-EM) flash-freezes a purified solution that has hundreds of thousands of particles of the target protein in a thin layer of vitreous ice.
%
In a cryo-EM experiment, an electron gun generates a high-energy electron beam that interacts with the sample, and a detector captures scattered electrons during a brief duration, resulting in a 2D projection image that contains many particles.
%
Given projection images, the single particle analysis (SPA) technique iteratively optimizes for recovering a high-resolution 3D protein structure~\citep{kuhlbrandt2014resolution, nogales2016development, renaud2018cryo}. 
%
Applications are numerous, ranging from revealing virus fundamental processes~\citep{yao2020molecular} in biodynamics to unveiling drug-protein interactions~\citep{hua2020activation} in drug development.
%
Compared with conventional shape reconstruction of objects of macro scales, cryo-EM reconstruction is particularly challenging. 
%
First, the images of particles are in the low signal-to-noise ratio (SNR) with unknown orientations. 
%
Such low SNR typically affects orientation estimation due to the severe corruption of the structural signal of particles. 
%
In addition, the flexible region of proteins induces conformational heterogeneity that disrupts orientation estimation and is harder to reconstruct. 
%
Conventional software packages~\citep{scheres2012relion, punjani2017cryosparc}  only reconstruct a small discrete set of conformations to reduce the complexity. 
%
However, such approaches often yield low-resolution reconstructions of flexible regions without guidance from human experts. 
%
\begin{wrapfigure}{r}{0.56\textwidth}
\centering
\vspace{-0.1in}
\includegraphics[width=\linewidth]{figures/teaser_1.pdf}
\vspace{-0.1in}
\caption{\textbf{Overview.} With noisy images and pre-computed poses as inputs, our method continuously reconstructs the heterogeneous structures of proteins. It also enables the identification of local flexible motions through the analysis of 3D attention values.}
\vspace{-0.1in}
\label{fig:teaser}
\end{wrapfigure}
%
Recently, neural approaches exploit coordinate-based representations for heterogeneous cryo-EM reconstruction~\citep{donnat2022deep,cryodrgn, cryodrgn2, cryofire, sfbp}.
%
%
To reduce the computational cost of image projection via the Fourier-slice theorem~\citep{bracewell1956strip}, they perform a 3D Fourier reconstruction.
%
A downside, however, is that it is counter-intuitive to model local density changes between conformations in the Fourier domain. 
%
In contrast, 3DFlex~\citep{punjani20213d} performs reconstruction in the real domain where motion is more naturally parameterized and more interpretable, but it requires an additional 3D canonical map as input.
%
In this paper, we propose {\em CryoFormer} for high-resolution continuous heterogeneous cryo-EM reconstruction (Fig.~\ref{fig:teaser}). 
%
Different from previous Fourier domain approaches~\citep{cryodrgn2, cryoai}, CryoFormer is conducted in the \textbf{real} domain to facilitate the modeling of local flexible regions. 
%
Taking 2D particle images as inputs, our orientation encoder and deformation encoder first extract orientation representations and deformation features, respectively. 
%
Notably, to further disentangle orientation and conformation, we use pre-computed pose estimations to pre-train the orientation encoder. 
%
Next, we build an implicit feature volume in the real domain as the core of our approach to achieve higher resolution and recover continuous conformational states. 
%
Furthermore, we propose a novel query-based transformer decoder to obtain continuous heterogeneous density volume by integrating 3D spatial features with conformational features. 
%
The transformer-based decoder not only can model fine-grained structures but also supports highlighting spatial local changes for interpretability.

In addition, we present a new synthetic dataset of porcine epidemic diarrhea virus (PEDV) trimeric spike protein,
%
which is a primary target for vaccine development and antigen analysis. 
%
Its dynamic movements from up to down in the domain 0 (D0) region modulate the enteric tropism of PEDV via binding to sialic acids (SAs) on the surface of enterocytes. We validate CryoFormer on the PEDV spike protein synthetic dataset and three existing datasets. 
%
Our approach outperforms the state-of-the-art methods including popular traditional software~\citep{punjani2017cryosparc,punjani20213d} as well as recent neural approaches~\citep{cryodrgn,sfbp} in terms of spatial resolution on both synthetic and experimental datasets.
%
Specifically, our method reveals dynamic regions of biological structures of the PEDV spike protein in our synthetic experiment, which implies functional areas but are hardly recovered by other methods.
%
We will release our code and PEDV spike protein dataset.



