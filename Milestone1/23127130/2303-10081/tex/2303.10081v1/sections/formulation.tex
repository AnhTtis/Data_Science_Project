%!TEX root = ../main.tex


\section{A Multilevel POP Formulation for Verification \& Synthesis of Robust CBFs}
\label{sec:formulation}

Consider a control-affine system with additive uncertainty
\bea \label{eq:system}
\dot{x} = f(x) + g(x)u + J(x) \epsilon
\eea
where $x \in \Real{n}$ is the state, $u \in \bbU \subseteq \Real{m}$ is the control, $f(x):\Real{n} \mapsto \Real{n}$, $g(x): \Real{n} \mapsto \Real{n \times m}$, $J(x): \Real{n} \mapsto \Real{n \times d}$ and $\epsilon \in \Real{d}$.
We make the following assumptions for system~\eqref{eq:system}.

\begin{assumption}[Polynomial dynamics] \label{assume:dynamics}
The entries of $f,g,J$ are polynomials in $x$.
\end{assumption}

\begin{assumption}[Convex polynomial control constraints]\label{assume:control}
    $\bbU$ is a \HYedit{compact} convex set defined by finite polynomial inequality constraints, \HYedit{\ie $\bbU = \{ u \in \Real{m} \mid c_{u,i}(u) \leq 0, i=1,\dots,l_u \}$ where $\{c_{u,i}\}_{i=1}^{l_u}$ are convex polynomials. Moreover, there exists a $u_0$ such that $c_{u,i} (u_0) < 0$ for all $i=1,\dots,l_u$.}
\end{assumption}
This assumption is quite general as it includes typical control sets such as polytopes~\cite{dai2022arxiv-clfcbfsynveri}, boxes~\cite{zhao22arxiv-cbfsos}, and ellipsoids.

\begin{assumption}[Bounded uncertainty]\label{assume:epsilon}
$\norm{\epsilon} \leq M_{\epsilon}$.
\end{assumption}

\subsection{Robust Control Barrier Function}
A robust control barrier function is defined as follows.

\begin{definition}[Robust CBF]\label{def:robustcbf}
Let $b(x): \Real{n} \mapsto \Real{}$ be a smooth function and $\calC \doteq \{ x \in \Real{n} \mid b(x) \geq 0 \}$ be its \HYedit{compact} superlevel set. Then, $b(x)$ is a robust CBF for system~\eqref{eq:system} if there exists a class-$K$ function $\alpha$ such that for all $x \in \calC$
\bea\label{eq:rcbforg}
\max_{u \in \bbU} \min_{\norm{\epsilon} \leq M_\epsilon } \dot{b}(x) \geq - \alpha (b(x)).
\eea
In words, if the system~\eqref{eq:system} starts inside $\calC$, then there exists a sequence of control such that the system trajectory remains inside $\calC$, regardless of the value of $\epsilon$. Due to Nagumo‘s Theorem~\cite{blanchini99automatica-set}, \eqref{eq:rcbforg} holds for all $x \in \calC$ if and only if
\bea 
\max_{u \in \bbU} \min_{\norm{\epsilon} \leq M_\epsilon} \dot{b}(x) \geq 0, \quad \forall x \in \partial \calC,
\eea
\ie there exists $u$ to pull the state back to $\calC$ whenever the state lies on the boundary of $\calC$ ($\partial\calC \doteq \{x\in\Real{n} \mid b(x) =0 \}$).
\end{definition}

\subsection{A Hierarchical POP Formulation}
\label{sec:formulation:multilevel}
Given a parametric function $b(x,\theta)$ \HYedit{that is polynomial in $x$ and $\theta$, and assume the parameter $\theta$ belongs to a compact semialgebraic set $\Theta \subseteq \Real{k}$, \ie $\Theta$ is defined by finite polynomial (in-)equalities}. We consider verifying and synthesizing a robust CBF from $b(x,\theta)$. By Definition~\ref{def:robustcbf}, we can formulate the following optimization problems. 

\begin{problem}[Verification]\label{prob:cbfverification} Fix $\theta$, define 
\bea \label{eq:verifyvaluefun}
V(\theta) := \min_{x \in \partial \calC} \max_{u \in \bbU} \min_{\norm{\epsilon} \leq M_\epsilon} \dot{b}(x,\theta).
\eea
If $V(\theta) \geq 0$ for a given $\theta$, then $b(x,\theta)$ is a robust CBF. Otherwise, a global minimizer $x^\star$ of~\eqref{eq:verifyvaluefun} with $V(\theta) < 0$ acts as a witness that $b(x,\theta)$ is not a robust CBF.
\end{problem}

$V(\theta)$ is called the \emph{value function} of the verification POP.

\begin{problem}[Synthesis]\label{prob:cbfsynthesis} Let $V(\theta)$ be as in~\eqref{eq:verifyvaluefun}, define
\bea\label{eq:cbfsynthesis}
V^\star = \max_{\theta \in \Theta} V(\theta), \quad \theta^\star \in \argmax_{\theta \in \Theta}V(\theta).
\eea
If $V^\star \geq 0$, then $b(x,\theta^\star)$ is a robust CBF. Otherwise ($V^\star < 0$),  there does not exist a robust CBF from the family $b(x,\theta)$.
\end{problem}

Note that for verification, any $x \in \partial \calC$ with ``$\max_u \min_\epsilon \dot{b}(x,\theta) < 0$'' (not necessarily an $x^\star$) can refute $b(x,\theta)$ as a valid CBF; and for synthesis, any $\theta$ (not necessarily a $\theta^\star$) with $V(\theta) \geq 0$ leads to a valid CBF $b(x,\theta)$. In fact, our synthesis method can return a set of valid $\theta$ with $V(\theta) \geq 0$. However, we choose to state our definitions as in Problems~\ref{prob:cbfverification}-\ref{prob:cbfsynthesis} to make it easier to streamline our algorithm.


\subsection{Reduction to Single-level and Min-max POP}
\label{sec:formulation:reduction}
Problems~\eqref{eq:verifyvaluefun} and~\eqref{eq:cbfsynthesis} are instances of \emph{hierarchical optimization} problems~\cite{bennett22mp-hierarchical}, which are in general very difficult to analyze and solve. Nonetheless, thanks to Assumptions~\ref{assume:control} and~\ref{assume:epsilon}, we will show that~\eqref{eq:verifyvaluefun} can be reduced to a single-level POP, while~\eqref{eq:cbfsynthesis} can be reduced to a min-max POP.

We start by expanding $\dot{b}(x,\theta)$:
\bea \label{eq:dotbexpand}
\hspace{-2mm}
\dot{b}(x,\theta)\! =\! \underbrace{\frac{\partial b}{\partial x}(x,\theta) f(x)}_{:= L_f b(x,\theta)}\! +\! \underbrace{\frac{\partial b}{\partial x}(x,\theta) g(x)}_{:=L_g b(x,\theta)} u\! +\! \underbrace{\frac{\partial b}{\partial x}(x,\theta) J(x)}_{:=L_J b(x,\theta)} \epsilon.\!\!\!
\eea
With~\eqref{eq:dotbexpand}, we develop $V(\theta)$ in~\eqref{eq:verifyvaluefun}:
\bea 
& \displaystyle\min_{x \in \partial \calC} \max_{u \in \bbU} \min_{\norm{\epsilon} \leq M_\epsilon} L_f b(x,\theta) + L_g b(x,\theta)u + L_J b(x,\theta) \epsilon \nonumber\\
= &  \displaystyle \min_{x \in \partial \calC} \max_{u \in \bbU} \bracket{L_f b(x,\theta)\! +\! L_g b(x,\theta)u\! +\! \min_{\norm{\epsilon}\! \leq M_\epsilon} L_J b(x,\theta) \epsilon  } \label{eq:developVstepone}\\
 = & \displaystyle \min_{x \in \partial \calC} \bracket{ L_f b(x,\theta)\! +\! \underbrace{\max_{u \in \bbU} L_g b(x,\theta) u}_{:=V_u^\star}\! +\! \underbrace{\min_{\norm{\epsilon} \leq M_\epsilon} L_J b(x,\theta) \epsilon}_{:=V_\epsilon^\star} } \label{eq:developVsteptwo}
\eea
where~\eqref{eq:developVstepone} holds because ``$L_f b(x,\theta)$'' and ``$L_g b(x,\theta) u$'' are constants \wrt ``$\min_{\epsilon \leq M_\epsilon}$''; and~\eqref{eq:developVsteptwo} holds because ``$L_f b(x,\theta)$'' and ``$\min_{\norm{\epsilon} \leq M_\epsilon} L_J b(x,\theta) \epsilon$'' are constants \wrt ``$\max_{u \in \bbU}$''. We next show that both $V_\epsilon^\star$ and $V_u^\star$ in~\eqref{eq:developVsteptwo} can be solved in closed form.

{\bf (1) $V_\epsilon^\star$:}
$\norm{\epsilon}\leq M_\epsilon$ defines a $d$-dimensional ball with radius $M_\epsilon$ and it is easy to verify that choosing
\bea 
\epsilon^\star = \begin{cases}
    - M_\epsilon \frac{L_J b(x,\theta)}{ \norm{L_J b(x,\theta)}} & \text{if } \norm{L_J b(x,\theta)} \neq 0 \\
    \text{arbitrary} & \text{otherwise}
\end{cases}
\eea
leads to
\bea \label{eq:Vepsstar}
V^\star_\epsilon = - M_\epsilon \norm{L_J b(x,\theta)}.
\eea 

{\bf (2) $V_u^\star$:} 
according to Assumption~\ref{assume:control}, 
$u^\star$ is optimal for ``$\max_{u \in \bbU} L_g b(x,\theta) u$'' if and only if there exists a dual variable $\zeta^\star \in \Real{l_u}$ such that $(u^\star, \zeta^\star)$ satisfies the following KKT optimality conditions~\cite{boyd04book-convex}:
\begin{subequations}\label{eq:kktconds}
    \begin{eqnarray}
        \hspace{-4mm} \text{\grayout{primal feasibility:} }& c_{u,i}(u) \leq 0,i=1,\dots,l_u \label{eq:kktprimal}\\
        \hspace{-4mm} \text{\grayout{dual feasibility:} }& \zeta_i \geq 0, i=1,\dots,l_u \label{eq:kktdual}\\
        \hspace{-4mm} \text{\grayout{stationarity:} }& - L_g b(x,\theta)\! +\! \displaystyle \sum_{i=1}^{l_u} \zeta_i \frac{\partial c_{u,i}}{\partial u}(u)\! =\! 0 \label{eq:kktstationary} \\
        \hspace{-4mm} \text{\grayout{complementarity:} }& \zeta_i c_{u,i}(u) = 0,i=1,\dots,l_u. \label{eq:kktcomp}
    \end{eqnarray}
\end{subequations}
Observe that~\eqref{eq:kktconds} is a set of polynomial (in-)equalities.
Let $\bbK(x,\theta) \subseteq \Real{m} \times \Real{l_u}$ be the set of optimal $u$ and $\zeta$ defined by~\eqref{eq:kktconds}, we have that
\bea \label{eq:Vustar}
V_{u}^\star = L_g b(x,\theta) u^\star, \quad (u^\star,\zeta^\star) \in \bbK(x,\theta).
\eea 
\begin{remark} The optimal control $u^\star$ is generally a nonsmooth function of $x$ and $\theta$. Consider $c_{u,i} = u_i^2 - 1,i=1,\dots,m$, \ie $\bbU = [-1,1]^m$ is an $m$-D box, $u^\star$ is Bang-Bang in each of its $m$ dimensions, depending on the sign of $L_g b(x,\theta)$. Therefore, (i)  methods assuming $u^\star$ to be smooth (\eg a polynomial~\cite{jarvis03cdc-some}) are conservative; (ii) recent work~\cite{zhao22arxiv-cbfsos} synthesizes CBF by considering all $2^m$ combinations of Bang-Bang controllers. By explicitly introducing dual variables $\zeta$, our derivation uses KKT conditions to bridge nonsmooth control and smooth polynomial optimization.
\end{remark}

Plugging~\eqref{eq:Vepsstar} and~\eqref{eq:Vustar} into~\eqref{eq:developVsteptwo}, we have that 
\bea 
\hspace{-2mm}V(\theta) =\!\!\!\! \min_{\substack{x \in \Real{n}, z \in \Real{} \\ u^\star \in \Real{m}, \zeta^\star \in \Real{l_u}}} & \!\!\!\!\! L_f b(x,\theta) + L_g b(x,\theta) u^\star - M_\epsilon z \label{eq:verifypop}\\
\subject & z^2 = \norm{L_J b(x,\theta)}^2 \label{eq:zlift}\\
        & b(x,\theta) = 0 \label{eq:xcon}\\
        & (u^\star, \zeta^\star) \in \bbK(x,\theta)
\eea
where $x \in \partial\calC$ is explicitly written as~\eqref{eq:xcon}; \eqref{eq:zlift} enforces $z = \pm \norm{L_J b(x,\theta)}$ and the ``$\min$'' in the objective~\eqref{eq:verifypop} will push $z = \norm{L_J b(x,\theta)}$ (and hence~\eqref{eq:Vepsstar} is implicit).

{\bf Standard POP}. Denote $y = [x;z;u^\star;\zeta^\star] \in \Real{N}$ with $N=n+1+m+l_u$, 
% and recall $x \in \bbX$ can be described by polynomial (in-)equalities (Assumption~\ref{assume:state}), 
we can convert the verification problem~\eqref{eq:verifypop} to the following standard POP
\bea \label{eq:standardpop}
V(\theta) = \min_{y \in \Real{N}} \cbrace{ \varphi(y,\theta) \mymid \substack{ \displaystyle h_i(y,\theta) = 0, i=1,\dots,l_h \\[1mm] \displaystyle s_i(y,\theta) \geq 0,i=1,\dots,l_s } }.
\eea
Consequently, the original synthesis problem~\eqref{eq:cbfsynthesis} is equivalent to a min-max POP (where the ``$\max$'' is over $\theta \in \Theta$).

\HYedit{
\begin{remark}[Non-polynomial Dynamics]
Our framework is not restricted to polynomial dynamics. When the original dynamics~\eqref{eq:system} is non-polynomial, our framework still applies as long as the verification problem~\eqref{eq:verifypop} can be converted into a POP via a change of variables. For instance, assume the dynamics~\eqref{eq:system} and the CBF candidate $b(x,\theta)$ contain trigonometric terms in $\bar{x} \in x$, so long as the functions in problem~\eqref{eq:verifypop} are polynomials in $\sin(\bar{x})$ and $\cos(\bar{x})$, creating $\snew = \sin{\bar{x}},\cnew=\cos{\bar{x}}$ can turn~\eqref{eq:verifypop} into a POP with an extra polynomial constraint $\snew^2 + \cnew^2 = 1$. However, we stated Assumption~\ref{assume:dynamics} in the beginning to simplify our presentation. 
\end{remark}    
}



