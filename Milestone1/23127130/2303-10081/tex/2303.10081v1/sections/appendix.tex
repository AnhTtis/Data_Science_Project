%!TEX root = ../main.tex
\appendix
\HYedit{
\section{Proof of Proposition~\ref{prop:boundeddual}}
\label{app:proof:prop:boundeddual}
\begin{proof}
(i) Consider $\bbU$ a polytope with no degenerate extreme points. Observe that ``$\max_{u \in \bbU} L_g b(x,\theta) u$'' is a linear program in $u$ with fixed $(x,\theta)$. Because $\bbU$ is non-degenerate, there are $m' \leq m$ active linearly independent constraints at an optimal $u^\star$, and hence $m'$ nonzero $\zeta_i^\star$'s~\cite{bertsimas97book-lp}. Without loss of generality, let $i=1,\dots,m'$ be such active constraints. The KKT stationarity condition~\eqref{eq:kktstationary} reads: 
\bea
\underbrace{\bmat{ccc}
w_1 & \cdots & w_{m'}
\emat}_{:=W \in \Real{m \times m'}} \zeta^\star = L_g b(x,\theta),
\eea 
which implies $(\zeta^\star)\tran (W\tran W) (\zeta^\star) = \norm{L_g b(x,\theta)}^2$. As $W\tran W \succ 0$ ($\{w_i\}_{i=1}^{m'}$ are linearly independent), we have 
\bea 
\norm{\zeta^\star}^2 \leq \frac{\norm{L_g b(x,\theta)}^2}{\lambda_\min (W\tran W)}.
\eea 
Finally, $\norm{L_g b(x,\theta)}^2$ is bounded because it is smooth over the compact set $\partial \calC \times \Theta$. Thus, $\norm{\zeta^\star}^2$ is bounded.

(ii) Consider $\bbU$ a box with each dimension between $[-w_i,w_i]$. The KKT complementarity condition~\eqref{eq:kktcomp} says either $\zeta_i^\star = 0$ or $\zeta_i^\star \neq 0$ but $u^\star_i = \pm w_i$. In the second case, using the KKT stationarity condition~\eqref{eq:kktstationary}, we have
\bea 
\zeta_i^\star = \frac{[L_g b(x,\theta)]_i}{2u^\star_i} \Rightarrow (\zeta_i^\star)^2 = \frac{[L_g b(x,\theta)]_i^2}{4w_i^2}, i=1,\dots,m
\eea 
where $[L_g b(x,\theta)]_i$ denotes the $i$-th entry of $L_g b(x,\theta)$. Due to the boundedness of $[L_g b(x,\theta)]_i$, $\norm{\zeta^\star_i}^2$ is bounded.

(iii) Consider $\bbU$ an ellipsoid with a single constraint $u\tran W u \leq 1$. The KKT complementarity condition states either $\zeta^\star = 0$ or $\zeta^\star \neq 0$ but $(u^\star)\tran W (u^\star) = 1$. In the second case, the KKT stationarity condition~\eqref{eq:kktstationary} reads
\bea 
(\zeta^\star)^2 = \frac{\norm{L_g b(x,\theta)}^2 }{4 \norm{W u^\star}^2} \leq \frac{\norm{L_g b(x,\theta)}^2}{4 \lambda_\min (W)}.
\eea
The boundedness of $(\zeta^\star)^2$ follows from the boundedness of $\norm{L_g b(x,\theta)}^2$ over $ \partial \calC \times \Theta$.
\hfill \qedsymbol
\end{proof}}

\section{Bounded $y$ for POP~\eqref{eq:standardpop} with system~\eqref{eq:cleanvdpodynamics} and CBF~\eqref{eq:cleanvdpocbf}}
\label{app:bound:y:cleanvdp}
Clearly $\norm{x}^2 \leq \theta, u^2 \leq u_\max^2$ are bounded. 
It remains to show that $\zeta^\star$ is bounded. The KKT complementarity condition~\eqref{eq:kktcomp} states that either (i) $\zeta^\star = 0$, or (ii) when $\zeta^\star \neq 0$, $c_{u,i}(u) = 0$, which means $u^\star= \pm u_\max$. In the second case, the KKT stationarity condition~\eqref{eq:kktstationary} leads to
\bea 
2\zeta^\star u^\star = L_g b(x,\theta) = -2 x_1 x_2.
\eea 
Solving for $\zeta^\star$ results in
\bea
\zeta^\star = \frac{-x_1 x_2}{u^\star} \Rightarrow (\zeta^\star)^2 = \frac{x_1^2 x_2^2}{u_\max^2} \leq \frac{(x_1^2 + x_2^2)^2}{4 u_\max^2} = \frac{\theta^2}{4u_\max^2}.
\eea

\section{Proof of Proposition~\ref{prop:wrongvdpocbf}}
\label{app:proofwrongvdpocbf}
\begin{proof}
    We have $L_f b(x,\theta) = -x_2^2 (1-x_2^2)$, $L_g b(x,\theta) = -2x_1 x_2$, $L_J b(x,\theta) = -2 x_2$. 
    As a result:
    \begin{subequations}
        \begin{eqnarray}
            V_u^\star =& \displaystyle \max_{u^2 \leq u_\max^2} (-2x_1 x_2) u = 2 u_\max \abs{x_1 x_2}  \\
            V_\epsilon^\star =& \hspace{-4mm}\displaystyle \min_{\norm{\epsilon} \leq M_\epsilon} -2x_2 \epsilon\! =\! -2 M_\epsilon \abs{x_2},
        \end{eqnarray}
    \end{subequations}
    and 
    \bea 
    \hspace{-2mm}V(\theta) =  \displaystyle \min_{\norm{x}^2 = \theta} -x_2^2(1-x_2^2) + 2 u_\max \abs{x_1 x_2} - 2 M_\epsilon \abs{x_2}.
    \eea
    Choosing $x_2 = 0,x_1 = \pm \sqrt{\theta}$, we obtain $V(\theta) \leq 0$. \hfill \qedsymbol
    % If $\theta \leq 1$, then with $x_1 = 0$ and $x_2 = \sqrt{\theta}$, we have $V(\theta) \leq - \theta (1-\theta) - 2 M_\epsilon \sqrt{\theta} < 0$. If $\theta > 1$, then 
\end{proof}

\section{Bounded $y$ for POP~\eqref{eq:standardpop} with system~\eqref{eq:vdpodynamics} and CBF~\eqref{eq:ellipsoidalcbf}}
\label{app:bound:y:uncertainvdp}
Clearly, $u^2 \leq u_\max^2$ is bounded.
$x$ is bounded because
\bea 
\max_{x\tran A x = 1} x\tran x \overset{v := A^{1/2}x}{=} \max_{v\tran v = 1} v\tran A\inv v = \frac{1}{\lambda_\min (A)},
\eea
where $\lambda_\min$ and $\lambda_\max$ indicates the minimum and maximum eigenvalue of $A$. Note that 
\bea 
\lambda_\min (\lambda_\max + \lambda_\min) \geq \lambda_\min \lambda_{\max} = \theta_1\theta_2 - \theta_3^2 \Longrightarrow \\
\lambda_\min \geq \frac{\theta_1 \theta_2 - \theta_3^2}{\lambda_\min + \lambda_\max} = \frac{\theta_1 \theta_2 - \theta_3^2}{\theta_1 + \theta_2}.
\eea 
Therefore, $x$ is bounded because
\bea \label{eq:uncertainvdpoxbound}
\norm{x}^2 \leq  \frac{1}{\lambda_\min (A)} \leq \frac{\theta_1 + \theta_2}{\theta_1 \theta_2 - \theta_3^2}.
\eea 
Now consider $z^2 = \norm{L_J b(x,\theta)}^2$:
\bea 
\norm{L_J b(x,\theta)}^2 = \norm{-2x\tran A J(x)}^2 = 4x\tran A J(x)J(x)\tran A x.
\eea 
Note that $J(x)J(x)\tran \preceq \eye$,
which means
\bea  \label{eq:uncertainvdpozbound}
\norm{L_J b(x,\theta)}^2 \leq 4x\tran A^2 x \leq 4\lambda_\max(A) \leq 4(\theta_1 + \theta_2).
\eea
It remains to show $\zeta^\star$ is bounded. Similar to Appendix~\ref{app:bound:y:uncertainvdp}, the KKT conditions tell us either $\zeta^\star = 0$ or 
\bea  
\zeta^\star = \frac{L_g b(x,\theta)}{2u^\star}, \quad (u^\star)^2 = u_\max^2.
\eea
Writing $(L_g b(x,\theta))^2$ as
\bea  
(L_g b(x,\theta))^2 = \norm{-2x\tran A g}^2 = 4x\tran A g g\tran A x
\eea 
with $g g\tran \preceq (x_1^2 + x_2^2) \eye$,
we obtain
\bea  
& \displaystyle (L_g b(x,\theta))^2 \leq 4 \norm{x}^2 x\tran A^2 x \leq 4\frac{(\theta_1 + \theta_2)^2}{\theta_1 \theta_2 - \theta_3^2}, \\
& \Rightarrow \displaystyle (\zeta^\star)^2 = \frac{(L_g b(x,\theta))^2}{4u_\max^2} \leq \frac{(\theta_1 + \theta_2)^2}{u_\max^2(\theta_1 \theta_2 - \theta_3^2)}.\label{eq:uncertainvdpozetabound}
\eea 

\section{Computing Moments}
\label{sec:app:computemoments}
{\bf Clean Van der Pol with circular CBF}. Consider the parameter space $\Theta = [\theta_\min, \theta_\max]$ and let $\Delta_{\theta} = \theta_\max - \theta_\min$ be its length. We compute the moments in \eqref{eq:computemoments} as
\bea 
\gamma_\beta\! =\! \int_{\theta_\min}^{\theta^\max}\!\! \theta^{\beta} d \psi(\theta) \!=\! \frac{1}{\Delta_\theta}  \int_{\theta_\min}^{\theta^\max} \!\! \theta^{\beta} d \theta \!=\! \frac{(\theta_\max^{\beta+1} - \theta_\min^{\beta+1})}{(\beta + 1)\Delta_\theta}.
\eea 

{\bf Uncertain Van der Pol with elliptical robust CBF}.
Consider $\Theta$ as in~\eqref{eq:ellipsoidalparam}, we compute \eqref{eq:computemoments}:
\bea 
& \gamma_\beta = \displaystyle \int_{\Theta} \theta_1^{\beta_1} \theta_2^{\beta_2} \theta_3^{\beta_3} d \psi(\theta)\nonumber \\
= & \displaystyle \int_{\theta_1} \int_{\theta_2} \int_{\theta_3}  \theta_1^{\beta_1} \theta_2^{\beta_2} \theta_3^{\beta_3} \parentheses{\frac{1}{\volume{\Theta}} d\theta_1 d\theta_2 d\theta_3} \nonumber\\
=& \displaystyle \frac{1}{\volume{\Theta}} \int_{\theta_1} \int_{\theta_2} \theta_1^{\beta_1} \theta_2^{\beta_2} d\theta_1 d\theta_2 \parentheses{\int_{ -\xi \sqrt{\theta_1 \theta_2}}^{\xi \sqrt{\theta_1 \theta_2}}  \theta_3^{\beta_3}  d\theta_3} \nonumber\\
= &\!\!\!\!\! \displaystyle \frac{\xi^{\beta_3+1} (1 - (-1)^{\beta_3+1}) }{(\beta_3+1)\volume{\Theta}} \int_{\theta_1} \int_{\theta_2} \theta_1^{\beta_1 + \frac{\beta_3+1}{2}} \theta_2^{\beta_2 + \frac{\beta_3+1}{2}} d\theta_1 d\theta_2 \nonumber\\
=&  \displaystyle \frac{\xi^{\beta_3+1} (1 - (-1)^{\beta_3+1}) }{(\beta_3+1)\volume{\Theta}} \frac{
    \thetaub^{\tldbeta_1} - \thetalb^{\tldbeta_1}
}{\tldbeta_1}
\frac{
    \thetaub^{\tldbeta_2} - \thetalb^{\tldbeta_2}
}{\tldbeta_2}
\eea 
where $\volume{\Theta}$ is the volume of $\Theta$ (a constant that we do not need to compute) and $\tldbeta_1 = \beta_1 + 1 + \frac{\beta_3+1}{2}$, $\tldbeta_2 = \beta_2 + 1 + \frac{\beta_3+1}{2}$.

