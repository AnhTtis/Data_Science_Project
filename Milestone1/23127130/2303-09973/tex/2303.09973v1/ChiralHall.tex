\documentclass
[aps,prl,amsfonts,amssymb,twocolumn,amsmath,preprintnumbers,nofootinbib,floatfix,superscriptaddress,longbibliography]{revtex4-1}%
\usepackage[dvips]{graphics}
\usepackage{graphicx}
\usepackage{bm}
\usepackage{amsmath}
\usepackage{amsfonts}
\usepackage{amssymb}
\usepackage{xcolor}
\usepackage{subfigure}
\usepackage{tabularx}
\usepackage{multirow}
\usepackage{braket}
\usepackage{commath}
\usepackage[colorlinks,linkcolor=blue,anchorcolor=blue,citecolor=blue,urlcolor=blue]%
{hyperref}%

%=====================================================================
% Redefine \maketitle so that it can be used twice (for supplementary)
\makeatletter
\def\maketitle{
	\@author@finish
	\title@column\titleblock@produce
	\suppressfloats[t]}
\makeatother
%=====================================================================

\setcounter{MaxMatrixCols}{30}
%TCIDATA{OutputFilter=latex2.dll}
%TCIDATA{Version=5.50.0.2960}
%TCIDATA{LastRevised=Friday, March 10, 2023 16:49:16}
%TCIDATA{<META NAME="GraphicsSave" CONTENT="32">}
%TCIDATA{<META NAME="SaveForMode" CONTENT="1">}
%TCIDATA{BibliographyScheme=Manual}
%TCIDATA{Language=American English}
%BeginMSIPreambleData
\providecommand{\U}[1]{\protect\rule{.1in}{.1in}}
%EndMSIPreambleData
\def\CTeXPreproc{Created by ctex v0.2.14, don't edit!}
\newcommand{\bk}{\boldsymbol{k}}
\newcommand{\bK}{\boldsymbol{K}}
\newcommand{\br}{\boldsymbol{r}}
\newcommand{\bG}{\boldsymbol{G}}
\newcommand{\bd}{\boldsymbol{d}}
\newcommand{\bsigma}{\boldsymbol{\sigma}}
\begin{document}

\title{Crossed Nonlinear Dynamical Hall Effect in Twisted Bilayers }
\author{Cong Chen}
\thanks{These authors contributed equally to this work.}
\author{Dawei Zhai}
\thanks{These authors contributed equally to this work.}
\author{Cong Xiao}
\email{congxiao@hku.hk}
\author{Wang Yao}
\email{wangyao@hku.hk}
\affiliation{Department of Physics, The University of Hong Kong, Hong Kong, China}
\affiliation{HKU-UCAS Joint Institute of Theoretical and Computational Physics at Hong Kong, China}

\begin{abstract}
We propose an unconventional nonlinear dynamical Hall effect characteristic of
twisted bilayers. The joint action of in-plane and out-of-plane ac electric
fields generates Hall currents $\boldsymbol{j}\sim\boldsymbol{\dot{E}_{\perp}%
}\times\boldsymbol{E}_{\parallel}$ in both sum and difference frequencies,
%leads to both intrinsic dc and double-frequency Hall currents when the two fields have the same frequency,
and when the two orthogonal fields have common frequency their phase
difference controls the on/off, direction and magnitude of the rectified dc
Hall current. This novel intrinsic Hall response has a band geometric origin
in the momentum space curl of interlayer Berry connection polarizability,
arising from layer hybridization of electrons by the twisted interlayer
coupling. The effect allows a unique rectification functionality and a
transport probe of chiral symmetry in bilayer systems. We show sizable effects
in twisted homobilayer transition metal dichalcogenides and twisted bilayer
graphene over broad range of twist angles.
\end{abstract}
\maketitle


Nonlinear Hall-type response to an in-plane electric field in a two
dimensional (2D) system with time reversal symmetry has attracted marked
interests \cite{Fu2015,Ma2019,Kang2019,Lu2021}. Intensive studies have been
devoted to uncovering new types of nonlinear Hall transport induced by quantum
geometry \cite{Lu2021,Lai2021,Zhou2022} and their applications such as
terahertz rectification \cite{Zhang2021} and magnetic information readout
\cite{Shao2020}.
Restricted by symmetry \cite{Fu2015}, the known mechanisms of nonlinear Hall
response in quasi-2D nonmagnetic materials
\cite{Ma2019,Kang2019,He2022,Yao2022} are all of extrinsic nature, sensitive
to fine details of disorders \cite{Nagaosa2010,Du2019}, which have limited
their utilization for practical applications.
Moreover, having a single driving field only, the effect has not unleashed the
full potential of nonlinearity for enabling controlled gate in logic
operation, where separable inputs (i.e., in orthogonal directions) are
desirable. The latter, in the context of Hall effect, calls for control by
both out-of-plane and in-plane electric fields.


A strategy to introduce quantum geometric response to out-of-plane field in
quasi-2D geometry
is made possible in van der Waals (vdW) layered structures with twisted
stacking
\cite{moireReviewEvaMacDonaldNatMater2020,moireReviewNatPhysBalents2020,moireReviewRubioNatPhys2021,moireReviewExptFolksNatRevMat2021,moireReviewJeanieLauNature2022,moireexcitonreviewNature2021,moireexcitonreviewNatRevMater2022}%
. Taking homobilayer as an example, electrons have an active layer degree of
freedom that is associated with an out-of-plane electric dipole
\cite{Pesin2012,Xu2014}, whereas interlayer quantum tunneling rotates this
pseudospin about in-plane axes that are of topologically nontrivial textures
in the twisted landscapes \cite{WuMacDonaldPRL2019,HongyiNSR,Zhai2020PRL}.
Such layer pseudospin structures can underlie novel quantum geometric
properties when coupled with out-of-plane field. Recent studies have found
layer circular photogalvanic effect \cite{Gao2020} and layer-contrasted
time-reversal-even Hall effect \cite{ZhaiLayerHall2022}, arising from band
geometric quantities.

In this work we unveil a new type of nonlinear Hall effect in time-reversal
symmetric twisted bilayers, where an intrinsic Hall current emerges under the
combined action of an in-plane electric field $\boldsymbol{E}_{\parallel}$ and
an out-of-plane ac field $\boldsymbol{E}_{\perp}(t)$: $\boldsymbol{j}%
\sim\boldsymbol{\dot{E}_{\perp}}\times\boldsymbol{E}_{\parallel}$ [see
Fig.~\ref{Fig:tTMD}(a)]. Having the two driving fields (inputs) and the
current response (output) all orthogonal to each other, the effect is dubbed
as the \textit{crossed nonlinear dynamical Hall effect}.
This is also the first nonlinear Hall contribution of an intrinsic nature in
nonmagnetic materials without external magnetic field, determined solely by
the band structures, not relying on extrinsic factors such as disorders and
relaxation times.
The effect arises from the interlayer hybridization of electronic states under
the chiral crystal symmetry characteristic of twisted bilayers, and has a
novel band geometric origin in the momentum space curl of interlayer Berry
connection polarizability (BCP). Having two driving fields of the same
frequency, a \textit{dc} Hall current develops, whose on/off, direction and
magnitude can all be controlled by the phase difference of the two fields,
which does not affect the magnitude of the double-frequency component. Such a
characteristic tunability renders this effect a unique approach to
rectification and transport probe of chiral bilayers. As examples, we show
sizable effects in small angle twisted transition metal dichalcogenides (tTMDs)
and twisted bilayer graphene (tBG), as well as tBG of large angles where
Umklapp interlayer tunneling dominates.

\emph{{\color{blue} Geometric origin of the effect. }}A bilayer system couples
to in-plane and out-of-plane driving electric fields in completely different
ways. The in-plane field couples to the 2D crystal momentum, leading to
Berry-phase effects in the 2D momentum space \cite{Xiao2010}. In comparison,
the out-of-plane field is coupled to the interlayer dipole moment $\hat{p}$ in
the form of $-E_{\perp}\hat{p}$, where $\hat{p}=ed_{0}\hat{\sigma}_{z}$ with
$\hat{\sigma}_{z}$ as the Pauli matrix in the layer index subspace and $d_{0}$
the interlayer distance.
%Note that the out-of-plane field is a driving field of transport effect hence is considered in the transport limit (or termed uniform limit) without changing the layer distribution of electrons. Therefore, it is different in nature from the usually discussed gate field which is considered in the equilibrium limit (or called static limit).
When the system has a more than twofold rotational axis in the $z$ direction,
as in tBG and tTMDs, any in-plane current driven by the out-of-plane field
alone is forbidden. It also prohibits the off-diagonal components of the
symmetric part of the conductivity tensor $\sigma_{ab}=\partial j_{a}/\partial
E_{||,b}$ with respect to the in-plane input and output. Since the
antisymmetric part of $\sigma_{ab}$ is not allowed by the Onsager reciprocity
in nonmagnetic systems, all the off-diagonal components of $\sigma_{ab}$ is
forbidden, irrespective of the order of out-of-plane field. On the other hand,
as we will show, an in-plane Hall conductivity $\sigma_{xy}=-\sigma_{yx}$ can
still be driven by the product of an in-plane field and the time variation
rate of an out-of-plane ac field, which is a characteristic effect of chiral bilayers.

To account for the effect, we make use of the semiclassical theory
\cite{Xiao2010,Gao2014,Xiao2021OM,Xiao2022MBT}. The velocity of an electron in
a bilayer system is given by
\begin{equation}
\boldsymbol{\dot{r}}=\frac{1}{\hbar}\partial_{\boldsymbol{k}}\tilde
{\varepsilon}-\frac{e}{\hbar}\boldsymbol{E}_{\parallel}\times
\boldsymbol{\Omega}_{\boldsymbol{k}}-\boldsymbol{\Omega}_{\boldsymbol{k}%
E_{\perp}}\dot{E}_{\perp},
\end{equation}
where $\hbar\boldsymbol{k}$ is the 2D crystal momentum. Here and hereafter we
suppress the band index for simplicity, unless otherwise noted. The three
contributions in this equation come from the band velocity, the anomalous
velocities induced by the \textit{k}-space Berry curvature $\boldsymbol{\Omega
}_{\boldsymbol{k}}$ and by the hybrid Berry curvature $\boldsymbol{\Omega
}_{\boldsymbol{k}E_{\perp}}$ in the $\left(  \boldsymbol{k},E_{\perp}\right)
$ space.

For the velocity at the order of interest, the $k$-space Berry curvature is
corrected to the first order of the variation rate of out-of-plane field
$\dot{E}_{\perp}$ as
\begin{equation}
\boldsymbol{\Omega}_{\boldsymbol{k}}=\partial_{\boldsymbol{k}}\times
(\boldsymbol{\mathcal{A}}+\boldsymbol{\mathcal{A}}^{\dot{E}_{\perp}}).
\end{equation}
Here $\boldsymbol{\mathcal{A}}=\langle u_{\boldsymbol{k}}|i\partial
_{\boldsymbol{k}}|u_{\boldsymbol{k}}\rangle$ is the unperturbed $k$%
-space\ Berry connection, with $|u_{\boldsymbol{k}}\rangle$ being the
cell-periodic part of the Bloch wave, whereas%
\begin{equation}
\boldsymbol{\mathcal{A}}^{\dot{E}_{\perp}}\left(  \boldsymbol{k}\right)
=\boldsymbol{\mathcal{G}}\left(  \boldsymbol{k}\right)  \dot{E}_{\perp}
\label{k-space}%
\end{equation}
is its gauge invariant correction \cite{Thouless1983,Dimi2006,Xiao2010}, which
can be identified physically as an in-plane positional shift of an electron
\cite{Gao2014,Wang2021,Liu2021} induced by the time evolution of the
out-of-plane field. For a band with index $n$, we have
\begin{equation}
\boldsymbol{\mathcal{G}}^{n}\left(  \boldsymbol{k}\right)  =2\hbar
^{2}\mathrm{{\operatorname{Re}}}\sum_{m\neq n}\frac{p^{nm}\left(
\boldsymbol{k}\right)  \boldsymbol{v}^{mn}\left(  \boldsymbol{k}\right)
}{[\varepsilon_{n}\left(  \boldsymbol{k}\right)  -\varepsilon_{m}\left(
\boldsymbol{k}\right)  ]^{3}}, \label{BCP}%
\end{equation}
whose numerator involves the interband matrix elements of the interlayer
dipole and velocity operators, and $\varepsilon_{n}$ is the unperturbed band
energy. Meanwhile, up to the first order of in-plane field, the hybrid Berry
curvature reads $\boldsymbol{\Omega}_{\boldsymbol{k}E_{\perp}}=\partial
_{\boldsymbol{k}}(\mathfrak{\bm A}+\mathfrak{\bm A}^{E_{||}})-\partial
_{E_{\perp}}(\boldsymbol{\mathcal{A}}+\boldsymbol{\mathcal{A}}^{E_{||}})$.
Here $\boldsymbol{\mathcal{A}}^{E_{||}}$ is the $k$-space Berry connection
induced by $E_{||}$ field \cite{Gao2014,Xiao2022MBT}, which represents an
intralayer positional shift and whose detailed expression is not needed for
our purpose. $\mathfrak{\bm A}=\langle u_{\boldsymbol{k}}|i\partial_{E_{\perp
}}|u_{\boldsymbol{k}}\rangle$ is the $E_{\perp}$-space Berry connection
\cite{Gao2020}, and
\begin{equation}
\mathfrak{\bm A}^{E_{||}}\left(  \boldsymbol{k}\right)  =\frac{e}{\hbar
}\boldsymbol{\mathcal{G}}\left(  \boldsymbol{k}\right)  \cdot\boldsymbol{E}%
_{\parallel} \label{lamda-space}%
\end{equation}
is its first order correction induced by the in-plane field. In addition,
$\tilde{\varepsilon}=\varepsilon+\delta\varepsilon$, where $\delta
\varepsilon=e\boldsymbol{E}_{\parallel}\cdot\boldsymbol{\mathcal{G}}\dot
{E}_{\perp}$ is the field-induced electron energy \cite{Xiao2021OM}.

Given that $\mathfrak{\bm A}^{E_{||}}$ is the $E_{\perp}$-space counterpart of
intralayer shift $\boldsymbol{\mathcal{A}}^{E_{||}}$, and that $E_{\perp}$ is
conjugate to the interlayer dipole moment, we can pictorially interpret
$\mathfrak{\bm A}^{E_{||}}$ as the interlayer shift induced by in-plane field.
It indeed has the desired property of flipping sign under the horizontal
mirror-plane reflection, hence is analogous to the so-called interlayer
coordinate shift introduced in the study of layer circular photogalvanic
effect \cite{Gao2020}, which is nothing but the $E_{\perp}$-space counterpart
of the shift vector well known in the nonlinear optical phenomenon of shift
current. Therefore,
%This argument is in analogy to the proposal of the interlayer coordinate shift in \cite{Gao2020}, which also
%has the dimension of $\mathfrak{\bm A}^{E}$, as a $\lambda$-space cousin of the genuine coordinate shift vector.
the $E_{\perp}$-space BCP $e\boldsymbol{\mathcal{G}}/\hbar$ can be
understood as the interlayer BCP. This picture is further augmented by the
connotation that the interlayer BCP is featured exclusively by
interlayer-hybridized electronic states: According to Eq.~(\ref{BCP}), if the
state $|u_{n}\rangle$ is fully polarized in a specific layer around some
momentum $\boldsymbol{k}$, then $\boldsymbol{\mathcal{G}}\left(
\boldsymbol{k}\right)  $ is suppressed.

With the velocity of individual electrons, the charge current density
contributed by the electron system can be obtained from%
\begin{equation}
\boldsymbol{j}=e\int[d\boldsymbol{k}]f_{0}\boldsymbol{\dot{r}}, \label{sum}%
\end{equation}
where $[d\boldsymbol{k}]$ is shorthand for $\sum_{n}d^{2}\boldsymbol{k}%
/(2\pi)^{2}$, and the distribution function is taken to be the Fermi function
$f_{0}$ as we focus on the intrinsic response. The band geometric
contributions to $\boldsymbol{\dot{r}}$ lead to a Hall current
\begin{equation}
\boldsymbol{j}=\chi^{\text{int}}\boldsymbol{\dot{E}_{\perp}}\times
\boldsymbol{E}_{\parallel}, \label{Hall}%
\end{equation}
where
\begin{equation}
\chi^{\text{int}}=\frac{e^{2}}{\hbar}\int[d\boldsymbol{k}]f_{0}[\partial
_{\boldsymbol{k}}\times\boldsymbol{\mathcal{G}}\left(  \boldsymbol{k}\right)
]_{z} \label{Eq:vorticity}%
\end{equation}
is intrinsic to the band structure. This band geometric quantity measures the
$k$-space curl of the interlayer BCP over the occupied states, and hence
is also a characteristic of layer-hybridized electronic states. Via an
integration by parts, it becomes clear that $\chi^{\text{int}}$ is a Fermi surface
property. Since $\chi^{\text{int}}$ is a time-reversal even pseudoscalar, it
is invariant under rotation, but flips sign under space inversion, mirror
reflection and rotoreflection symmetries. As such, $\chi^{\text{int}}$ is
allowed if and only if the system possesses a chiral crystal structure, which
is the very case of twisted bilayers~\cite{Gao2020,ZhaiLayerHall2022}.
Moreover, since twisted structures with opposite twist angles are mirror
images of each other, whereas the mirror reflection flips the sign of
$\chi^{\text{int}}$, the direction of Hall current can be reversed by
reversing twist direction.

\begin{figure*}[t]
\setlength{\abovecaptionskip}{0.cm}
\includegraphics[width=15 cm]{Fig1_TMD.pdf}\caption{(a) Schematics of
experimental setup. (b, c) Valence band structure and intrinsic Hall conductivity
with respect to in-plane input for tMoTe$_{2}$ at twist angles (b)
$\theta=1.2^{\circ}$ and (c) $\theta=2^{\circ}$ in +K valley. Color coding in
(b) and (c) denotes the layer composition $\sigma_{n}^{z}(\boldsymbol{k})$.}%
\label{Fig:tTMD}%
\end{figure*}

%%%%%%%%%%%%%%%%%%%%%%%%%%%%%%%%%%%%%%%%%%%%%%%%%%%%%%%%%%%%%%%%%%%%%%%%%
%%%%%%%%%%%%%%%%%%%%%%%%%%%%%%%%%%%%%%%%%%%%%%%%%%%%%%%%%%%%%%%%%%%%%%%%%


\emph{{\color{blue} Hall rectification and frequency doubling. }}This effect
can be utilized for the rectification and frequency doubling of an in-plane ac
input $\boldsymbol{E}_{\parallel}=\boldsymbol{E}_{\parallel}^{0}\cos\omega t$,
provided that the out-of-plane field has the same frequency, namely $E_{\perp
}=E_{\perp}^{0}\cos\left(  \omega t+\varphi\right)  $. The phase difference
$\varphi$ between the two fields plays an important role in determining the
Hall current, which takes the form of
\begin{equation}
\boldsymbol{j}=\boldsymbol{j}^{0}\sin\varphi+\boldsymbol{j}^{2\omega}%
\sin(2\omega t+\varphi).
\end{equation}
Here $\omega$ is required to be below the threshold for direct interband
transition in order to validate the semiclassical treatment, and
\begin{equation}
\boldsymbol{j}^{0}=\boldsymbol{j}^{2\omega}=\sigma_{\text{H}}\boldsymbol{\hat
{z}}\times\boldsymbol{E}_{\parallel}^{0}~~\text{with}~~\sigma_{\text{H}}=\frac{1}{2}\omega E_{\perp}^{0}\chi^{\text{int}}.
\end{equation}
%where $\sigma_{\text{H}}=\frac{1}{2}\omega E_{\perp}^{0}\chi^{\text{int}}$ 
$\sigma_{\text{H}}$ has the dimension of conductance and quantifies the Hall response with respect to the in-plane input.
In experiment, the Hall output by the crossed nonlinear dynamic Hall effect
can be distinguished readily from the conventional nonlinear Hall effect
driven by in-plane field alone, as they are odd and even, respectively, in the
in-plane field.

One notes that while the double-frequency component appears for any $\varphi$, the rectified
output is allowed only if the two crossed driving fields are not in-phase or
anti-phase. Its on/off, chirality (right or left), and magnitude are all
controlled by the phase difference of the two fields. Such a unique tunability
provides not only a prominent experimental hallmark of this effect, but also a
controllable route to Hall rectification. In addition, reversing the direction
of the out-of-plane field switches that of the Hall current, which also serves as a
control knob.



%%%%%%%%%%%%%%%%%%%%%%%%%%%%%%%%%%%%%%%%%%%%%%%%%%%%%%%%%%%%%%%%%%%%%%%%%
%%%%%%%%%%%%%%%%%%%%%%%%%%%%%%%%%%%%%%%%%%%%%%%%%%%%%%%%%%%%%%%%%%%%%%%%%


\emph{{\color{blue} Application to tTMDs.}} We now study the effect
quantitatively in tTMDs, using tMoTe$_{2}$ as an
example~\cite{WuMacDonaldPRL2019,HongyiNSR} (see details of the continuum model in \cite{ZhaiLayerHall2022}). For illustrative purposes, we take
$\omega/2\pi=0.1$~THz and $E_{\perp}^{0}d_{0}=10$~mV
\cite{moireReviewEvaMacDonaldNatMater2020,moireexcitonreviewNature2021,moireexcitonreviewNatRevMater2022}
in what follows.

Figures~\ref{Fig:tTMD}(b) and (c) present the electronic band structures along
with the layer composition $\sigma_{n}^{z}(\boldsymbol{k})$ at twist angles
$\theta=1.2^{\circ}$ and $\theta=2^{\circ}$. In both cases, the energy spectra
exhibit isolated narrow bands with strong layer hybridization. At
$\theta=1.2^{\circ}$, the conductivity shows two peaks $\sim0.1e^{2}/h$ at low
energies associated with the first two valence bands. The third band does not
host any sizable conductivity signal. At higher hole-doping levels, a
remarkable conductivity peak $\sim e^{2}/h$ appears near the gap separating
the fourth and fifth bands. At $\theta=2^{\circ}$, the conductivity shows
smaller values, but the overall trends are similar: A peak $\sim
\mathcal{O}(0.01)e^{2}/h$ appears at low energies, while larger responses
$\sim\mathcal{O}(0.1)e^{2}/h$ can be spotted as the Fermi level decreases.

\begin{figure}[t]
\includegraphics[width=3.4 in]{distribution.pdf} \caption{(a) The interlayer
BCP $\boldsymbol{\mathcal{G}}$, and (b) its vorticity $[\partial
_{\boldsymbol{k}}\times\boldsymbol{\mathcal{G}}]_{z}$ on the first valence
band from +K valley of 1.2$^{\circ}$ tMoTe$_{2}$. Background color and arrows
in (a) denote the magnitude and vector flow, respectively. Grey curves in (b)
show energy contours at $1/2$ and $3/4$ of the band width. The black dashed
arrow denotes direction of increasing hole doping level. Black dashed hexagons
in (a, b) denote the boundary of moir\'{e} Brillouin zone (mBZ).}%
\label{Fig:tTMD_distribution}%
\end{figure}

One can understand the behaviors of $\sigma_{\text{H}}$ from the interlayer
BCP in Eq.~(\ref{BCP}). It favors band near-degeneracy regions in
\textit{k}-space made up of strongly layer hybridized electronic states. As
such, the conductivity is most pronounced when the Fermi level is located
around such regions, which directly accounts for the peaks of response in
Fig.~\ref{Fig:tTMD}(b) [and \ref{Fig:tTMD}(c)]. When the Fermi level is
located on the third valence band in
Fig.~\ref{Fig:tTMD}(b), the effect is vanishingly small due to the
large gaps to adjacent bands.

\begin{figure*}[t]
\setlength{\abovecaptionskip}{0.cm}
\includegraphics[width=14 cm]{tBG21.pdf}\caption{(a-c) Three high-symmetry
stacking registries for tBG with a commensurate twist angle $\theta
=21.8^{\circ}$. Lattice geometries with rotation center on an overlapping
atomic site (a, b) and hexagonal center (c). (d) Schematic of the moir\'{e}
pattern when the twist angle slightly deviates from $21.8^{\circ}$, here
$\theta=21^{\circ}.$ Red squares marked by A, B and C are the local regions
that resemble commensurate $21.8^{\circ}$ patterns in (a), (b) and (c),
respectively. (e, f) Low-energy band structures and intrinsic Hall
conductivity of the two geometries [(a) and (b) are equivalent]. The shaded
areas highlight energy windows $\sim\hbar\omega$ around band degeneracies where interband transitions, not considered here, may quantitatively affect the conductivity measured.}%
\label{Fig:tBG21}%
\end{figure*}

Let us take the case of Fermi level being located within the first valence
band in Fig.~\ref{Fig:tTMD}(b) as an example and explain the emergence of the
first peak of conductivity. The \textit{k}-space distributions of
$\boldsymbol{\mathcal{G}}$ and $[\partial_{\boldsymbol{k}}\times
\boldsymbol{\mathcal{G}}]_{z}$ for the first valence band of 1.2$^{\circ}$
tMoTe$_{2}$ are shown in Figs.~\ref{Fig:tTMD_distribution}(a) and
\ref{Fig:tTMD_distribution}(b), respectively. $\boldsymbol{\mathcal{G}}$ is
suppressed around the corners of mBZ, for the states are strongly layer
polarized there. Interlayer hybridization becomes stronger as $\boldsymbol{k}$
moves away from mBZ corners. In this process, the competition between enlarged
$p^{nm}(\boldsymbol{k})$ and \textit{k}-space local gap renders narrow
ring-like structures enclosing the mBZ corners, in which
$\boldsymbol{\mathcal{G}}$ is prominent and points radially inward/outward
around $\kappa$/$\kappa^{\prime}$. The distribution of $\boldsymbol{\mathcal{G}}$ dictates
that of $[\partial_{\boldsymbol{k}}\times\boldsymbol{\mathcal{G}}]_{z}$. The
solid and dashed grey curves in Fig.~\ref{Fig:tTMD_distribution}(b) represent
two energy contours with the former corresponding to a lower doping level. One
observes that $[\partial_{\boldsymbol{k}}\times\boldsymbol{\mathcal{G}}]_{z}$
is negligible at lower energies, and it is dominated by positive values as the
doping increases, thus the conductivity rises initially. When the doping level
is higher, regions with $[\partial_{\boldsymbol{k}}\times
\boldsymbol{\mathcal{G}}]_{z}<0$ start to contribute, thus the conductivity
decreases after reaching a maximum.
%Moreover, as the integral of the vorticity over mBZ vanishes, the conductivity
%finally reduces to zero when the Fermi level is shifted into band gap.


\emph{{\color{blue} Application to tBG.}} The second example is tBG.
%Results for small twist angles are shown in Supplemental Material~\cite{supp}.
We focus on commensurate twist angles in the large angle
limit in the main text~\cite{PRB2013_Moon}, which possess moir{\'{e}}-lattice assisted strong
interlayer tunneling via Umklapp processes~\cite{PRB_Mele2010}. This case is
appealing because the Umklapp interlayer tunneling is a manifestation of
discrete translational symmetry of moir{\'{e} }superlattice, which is
irrelevant at small twist angles and not captured by the continuum model but
plays important roles in physical contexts such as higher order topological
insulator~\cite{Park2019} and moir{\'{e}} excitons
\cite{HongyiPRL2015,Seyler2019,Deng2020}. The Umklapp tunneling is strongest
for the commensurate twist angles of $\theta=21.8^{\circ}$ and $\theta
=38.2^{\circ}$, whose corresponding periodic moir{\'{e}} superlattices have
the smallest lattice constant ($\sqrt{7}$ of the monolayer counterpart).
Such a small moir\'{e} scale implies that the exact crystalline symmetry,
which depends sensitively on fine details of rotation center, has critical
influence on low-energy response properties.


To capture the Umklapp tunneling, we employ the tight-binding
model~\cite{PRB2013_Moon}. Figures~\ref{Fig:tBG21}(a, b) and (c) show two
distinct commensurate structures of tBG at $\theta=21.8^{\circ}$ belonging to
chiral point groups $D_{3}$ and $D_{6}$, respectively. The atomic
configurations in Figs.~\ref{Fig:tBG21}(a, b) are equivalent, which are
constructed by twisting AA-stacked bilayer graphene around an overlapping atom
site, and that in Fig.~\ref{Fig:tBG21}(c) is obtained by rotating around a
hexagonal center.
%~\footnote{Fig.~\ref{Fig:tBG21}(c) can also be obtained from Fig.~\ref{Fig:tBG21}(a) by a translation $\boldsymbol{\delta}$ of the upper layer to recover another commensurate structure~\cite{PRB2013_Moon}.}.
%Note that the second configuration
%can also be constructed by twisting AA-stacked bilayer graphene around a
%hexagonal center without additional translation~\cite{Park2019}.
Band structures of these two configurations are drastically different within a
low-energy window of $\sim10$ meV around the $\kappa$ point~\cite{PRB2013_Moon}.
%At zero energy, the band dispersion of the geometry in Figs.~\ref{Fig:tBG21}(a, b) shows quadratic touching
%[Fig.~\ref{Fig:tBG21}(e)], while that of the configuration in Fig.~\ref{Fig:tBG21}(c) is gapped [Fig.~\ref{Fig:tBG21}(f)].
Remarkably, despite large $\theta$, we still get $\sigma_{\text{H}}$
$\sim\mathcal{O}(0.001)\,e^{2}/h$ ($D_{3}$) and $\sim\mathcal{O}%
(0.1)\,e^{2}/h$ ($D_{6}$), which are comparable to those at small angles (cf.
Fig.~S1 in the Supplemental Material \cite{supp}). Such sizable responses can
be attributed to the strong interlayer coupling enabled by Umklapp
processes~\cite{ZhaiLayerHall2022,HongyiPRL2015,Seyler2019,Deng2020}. Apart
from different intensities, the Hall conductivities in the two stacking
configurations have distinct energy dependence: In Fig.~\ref{Fig:tBG21}(e),
$\sigma_{\text{H}}$ shows a single peak centered at zero energy; In
Fig.~\ref{Fig:tBG21}(f), it exhibits two antisymmetric peaks around zero. The
peaks are centered around band degeneracies, and their profiles can be
understood from the distribution of $\left[  \partial_{\boldsymbol{k}}%
\times\boldsymbol{\mathcal{G}}\right]  _{z}$.
%The drastic contrast in the conductivity line shapes and the large intensities suggest that this intrinsic
%Hall response could be useful for distinguishing the different stacking configurations.


Figure~\ref{Fig:tBG21}(d) illustrates the atomic structure of tBG with a twist
angle slightly deviating from $\theta=21.8^{\circ}$, forming a supermoir\'{e}
pattern. In short range, the local stacking geometries resemble the
commensurate configurations at $\theta=21.8^{\circ}$, while the stacking
registries at different locales differ by a translation. Similar to the
moir\'{e} landscapes in the small-angle limit, there also exist high-symmetry
locales: Regions A and B enclose the $D_{3}$ structure, and region C contains
the $D_{6}$ configuration. Position-dependent Hall response is therefore
expected in such a supermoir\'{e}. As the intrinsic Hall signal from the
$D_{6}$ configuration dominates [see Figs.~\ref{Fig:tBG21}(e) vs (f)], the net
response mimics that in Fig.~\ref{Fig:tBG21}(f).


\emph{{\color{blue} Discussion. }}We have uncovered the crossed nonlinear
dynamical intrinsic Hall effect characteristic of layer hybridized electronic
states in twisted bilayers, and elucidated its geometric origin in the
\textit{k}-space curl of interlayer BCP. It offers a new tool for
rectification and frequency doubling in chiral vdW bilayers, and is
sizable in tTMD and tBG. Here our focus is on the intrinsic effect, which can be
evaluated quantitatively for each material and provides a benchmark for
experiments. There may also be extrinsic contributions, similar to the side
jump and skew scattering ones in anomalous Hall effect. They typically have
distinct scaling behavior with the relaxation time $\tau$ from the intrinsic effect,
hence can be distinguished from the latter in experiments
\cite{Kang2019,Xiao2019scaling,Du2019,Lai2021}. Moreover, they are suppressed
in the clean limit $\omega\tau\gg1$ [$(\omega\tau)^{2}\gg1$, more precisely]
\cite{Du2019}. In high-quality tBG samples, $\tau\sim$ ps at room temperature
\cite{Sun2021}. Much longer $\tau$ can be obtained at lower temperatures. In
fact, a recent theory explaining well the resistivity of tBG predicted
$\tau\sim10^{-8}$ s at 10 K \cite{Sharma2021}. As such, high-quality tBG under
low temperatures and sub-terahertz input ($\omega/2\pi=0.1$ THz) is located in
the clean limit, rendering an ideal platform for isolating the intrinsic effect.

This work paves a new route to driving in-plane response by out-of-plane
dynamical control of layered vdW structures \cite{zhai2022ultrafast}. The
study can be generalized to other observables such as spin current and spin
polarization, and the in-plane driving can be statistical forces, like
temperature gradient. Such orthogonal controls rely critically on the
nonconservation of layer pseudospin degree of freedom endowed by interlayer
coupling, and constitute an emerging research field at the crossing of 2D vdW
materials, layertronics, twistronics and nonlinear electronics.

This work is supported by the Research Grant Council of Hong Kong
(AoE/P-701/20, HKU SRFS2122-7S05), and the Croucher Foundation. W.Y. also
acknowledges support by Tencent Foundation.

\bibliographystyle{apsrev4-1}
\bibliography{ChiralHall_ref}


%%%%%%%%%%%%%%%%%%%%%%%%%%%%%%%%%%%%%%%%%%%%%%%%%%%%%%%%%%%%%%%%%%%%%%%%
% Supplementary

\clearpage

\title{Supplemental Material}

\maketitle
\onecolumngrid

\renewcommand{\thefigure}{S\arabic{figure}}
\renewcommand{\theequation}{S\arabic{equation}}

\begin{center}
	\textbf{Extra figures for tBG at small twist angles}
\end{center}


Figure~\ref{Fig:tBG}(a) shows the band structure of tBG with $\theta
=1.47^{\circ}$ obtained from the continuum model~\cite{KoshinoTBGPRX2018}. The
central bands are well separated from higher ones, and show Dirac points at
$\kappa$/$\kappa^{\prime}$ points protected by valley $U(1)$ symmetry and a
composite operation of twofold rotation and time reversal $C_{2z}\mathcal{T}%
$~\cite{PRL_2019_SongZD}. Degeneracies at higher energies can also be
identified, for example, around $\pm75$~meV at the $\gamma$ point. As the
two Dirac cones from the two layers intersect around the same area, such
degeneracies are usually accompanied by strong layer hybridization [see
the color in the left panel of Fig.~\ref{Fig:tBG}(a)].
Additionally, it is well-known that the two layers are strongly coupled when
$\theta$ is around the magic angle ($\sim1.08^{\circ}$), rendering narrow
bandwidths for the central bands. As discussed in the main text, coexistence of
strong interlayer hybridization and small energy separations is expected to
contribute sharp conductivity peaks near band degeneracies, as shown in
Fig.~\ref{Fig:tBG}(a). In this case, the conductivity peak near the Dirac
point can reach $\sim0.1e^{2}/h$, while the responses around $\pm0.08$~eV are
smaller at $\sim0.01e^{2}/h$.

The above features are maintained when $\theta$
is enlarged, as illustrated in Figs.~\ref{Fig:tBG}(b) and (c) using
$\theta=2.65^{\circ}$ and $\theta=6.01^{\circ}$. Since interlayer coupling
becomes weaker and the bands are more separated at low energies when $\theta$
increases, intensity of the conductivity drops significantly. 

We stress that
$\boldsymbol{\mathcal{G}}$ is not defined at degenerate points, and interband transitions
may occur when energy separation satisfies $|\varepsilon_{n}-\varepsilon_{m}|\sim
\hbar\omega$, the effects of which are not included in the current formulations.
Consequently, the results around band degeneracies within energy $\sim
\hbar\omega$ [shaded areas in Fig.~\ref{Fig:tBG}] should be excluded.

\begin{figure}[th]
	\includegraphics[width=4 in]{tBG.pdf} \caption{Band structure and layer composition $\sigma_{n}^{z}$ in +K valley of tBG (left panel) and the
		intrinsic Hall conductivity (right panel) at three different twist angle $\theta$. The
		shaded areas highlight energy windows $\sim\hbar\omega$ around band
		degeneracies in which the conductivity results should not be considered. Here
		$\sigma_{H}$ should be multiplied by a factor of 2 accounting for spin
		degeneracy.}%
	\label{Fig:tBG}%
\end{figure}


\end{document}
