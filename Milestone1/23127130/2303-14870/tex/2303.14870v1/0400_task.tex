\section{Task Setup}
\label{sec:task_setup}
% \st{Our goal is to study how to obtain controllers for bi-manual magnetic block manipulation through reinforcement learning (RL), such that they can be executed on real-world robotic system.}
Our goal is to study how to obtain controllers for real-world bi-manual manipulation systems. To this end, we design a magnetic block assembly task as a testbed.
    \subsection{Task Description}
        \label{sec:tasks}
        We study the \textit{U-Shape Block Assembly} task (Figure~\ref{fig:three_block_task}) where three blocks with magnetic connection points (Figure~\ref{fig:icra_blocks}) are placed on the ground, and two robotic arms must attach the three blocks to form a U-shape.
        While our long-term objective is to enable assembling any number of blocks, we design this task as a minimal configuration that stresses bi-arm coordination in the real world.
        Despite its minimalism, increasing the number of magnetic blocks can support the creation of arbitrarily complex composed structures, which can lead to many intriguing avenues of future research.
        In our U-shape block assembly task, each trial continues indefinitely until 1) system identifies all three blocks are connected, 2) robots get faulted, 3) robots are manually stopped to prevent breakages (in real only), 4) 50-second timelimit passes (in sim only).
        % any of the following conditions are met:
        % \ps{(We can inline these with 1), 2), 3), 4) instead of bullet points to save space.)}
        % \begin{itemize}
        %     \item system identifies all three blocks are connected (in both sim and real)
        %     \item robots get faulted by safety limit (in real) or a physics error is raised (in sim)
        %     \item robots are manually stopped to prevent breakages by the operator (in real)
        %     \item 50-second timelimit passes (in sim) \yc{I forgot, but do real robots go on forever without time limit?}
        % \end{itemize}
        Determination of success is different between sim and real, according to the different information available in each environment. In simulation, we programmatically verify the connections of the correct magnet pairs to evaluate success. In the real world, at the end of each trial, a human operator verifies that the correct magnetic connections are successfully made. If the robot arms run into unsafe behavior in the real world, the human operator terminates the trial and marks it as a failure.
     
     %\sg figured to intro
      
