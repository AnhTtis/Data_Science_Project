\begin{abstract}

% \yc{ (YS draft of abstract)} Most of the successes in robotic manipulation have been restricted to single-arm robots,
% which limits the range of solvable tasks to pick-and-place, insertion, and object rearrangement. 
% In contrast, dual and multi-arm robot platforms unlock a rich diversity of complex tasks, such as laundry folding or cooking.
% However, multi-arm robots also entail a suite of unique challenges which render controller development much more difficult, 
% e.g. collision avoidance and the need for coordinated behaviors. 
% In this work, we study the efficacy of reinforcement learning (RL) in simulation and simulation-to-real (Sim2Real) 
% as methods to overcome these challenges and develop controllers for bi-manual tasks.
% Our RL approach allows significant simplifications by using real-time (4Hz) joint-space controls 
% and unfiltered observations.

% In addition to designing the control algorithm, another key challenge is designing a fair evaluation task which emphasizes
% effective bimanual coordination and removes orthogonal complicating factors, e.g. high-level perception.
% To this end, we propose a "Connect Task", where the aim is for two robot arms to pick up and attach assembly blocks
% with magnetic connection points in succession. 
% % The task difficulty can be specified with the number of blocks that
% % must be connected, e.g. 2-block connection, 3-block connection. 
% We demonstrate our proposed method and task with two xArm6 robots and 3D printed block 
% with magnetic connection points. Our RL trained controller has a @@@ success rate on task @@@ and @@@ success rate on task @@@. 
% The accompanying project webpage can be found at: \href{https://sites.google.com/view/bimanual-attachment}{sites.google.com/view/bimanual-attachment}\sk{anonymize}

% SG: I recommend rewriting most of the paper, because: (1) real-world assembly is now strong (3 block means: dexterity, combinatorial generalization, + long-horizong planning. significantly more impressive than existing biarm tasks in sim/real (e.g. in https://robosuite.ai/), (2) as biarm paper, it lacks diversity of biarm tasks, such as biarm pick up of heavy object or cloth folding. Therefore, better to be pitched as "real world assembly" paper
% SG: i imagine the paper could have 3 main lists:
%  a) list of why biarm assembly in real world is interesting, novel, and challenging?: e.g. bi-arm coordination and collision avoidance, robust grasping, and long-horizon planning, combinatorial generalization
%. b) list of system components that are critical: task specification, learning algorithm, task-space randomization, direct joint-space control, behavior constraints
%  c) list of emergent behaviors from RL: retries through regrasping and intentional slow-downs during block attachments
% SG: i find "real world biarm assembly of 3 blocks" most impressive. less so about "joint control". not impressive by "bihand pickup" or "evaluation protocols"

% \kg{We can come back to abstract once paper is closer to completion}
Most
% of the successes in learning-based
successes in robotic manipulation have been restricted to single-arm gripper robots, whose low dexterity limits the range of solvable tasks to pick-and-place, insertion, and object rearrangement. More complex tasks such as assembly require dual and multi-arm platforms, but % \yc{("but" instead of "and"?) }
entail a suite of unique challenges such as bi-arm coordination and collision avoidance, robust grasping, and long-horizon planning.
In this work we investigate the feasibility of training deep reinforcement learning (RL) policies in simulation and transferring them to the real world (Sim2Real) as a generic methodology for obtaining performant controllers for real-world bi-manual robotic manipulation tasks.
As a testbed for bi-manual manipulation, we develop the ``U-Shape Magnetic Block Assembly Task", wherein two robots with parallel grippers must connect 3 magnetic blocks to form a ``U"-shape.
% In this work, we develop a system for learning bi-manual block assembly of a 3-block U-shaped blueprint.
Without a manually-designed controller nor human demonstrations, we demonstrate that with careful Sim2Real considerations,
% our policies trained with RL in simulation transfer to the real-world without any additional real-world fine-tuning.
our policies trained with RL in simulation enable two xArm6 robots to solve the U-shape assembly task with a success rate of above 90\% in simulation, and 50\% on real hardware without any additional real-world fine-tuning.
% can assemble 3 printed blocks with magnetic connectors
% Notably, we demonstrate that two xArm6 robots can solve the U-shape assembly task with a success rate of above 90\% in simulation, and achieve 50\% on real hardware with zero-shot transfer.
%\kg{needs a bit more accurate description of what the task actually is}. 
Through careful ablations, we highlight how each component of the system is critical for such simple and successful policy learning and transfer, including task specification, learning algorithm, direct joint-space control, behavior constraints, perception and actuation noises, action delays and action interpolation. %Lastly, since our policy is optimized with joint control using large-scale RL in simulation, it exhibits many emergent behaviors such as robust retries and intentional change-of-pace, all of which are also transferred to real world. 
Our results present a significant step forward for bi-arm capability on real hardware,
% We
and we hope our system can inspire future research on deep RL and Sim2Real transfer of bi-manual policies, drastically scaling up the capability of real-world robot manipulators. The accompanying project webpage and videos can be found at: \href{https://sites.google.com/view/u-shape-block-assembly}{sites.google.com/view/u-shape-block-assembly}.

% \kg{I'm gonna quickly try something since this abstract feels a little out of sync with introduction now}
% - multi-arm allows for much more complex tasks (done)
% - multi-arm is hard (done)
% - use of rl for joint-space control of bimanual is underexplored
% - (done) in this work we investigate the feasability of of training deep RL policies in simulation anad transferring them to the real world (Sime2Real) as a generic methodology for obtaining performant controllers for real-world bi-manual robotic manipulation tasks
% - we develop u-shape assembly as a testbed
% - through detailed ablations we demonstrate that with careful consideration such as ... we can train in sim transfer to real without finetuning
% - our policies are able to solve u-shape
% - our results present a significatn step forward for bi-arm capability on real hardware

% - 



\end{abstract}
