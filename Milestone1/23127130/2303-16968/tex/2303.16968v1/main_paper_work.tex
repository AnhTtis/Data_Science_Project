\documentclass{amsart}
%%%%%%%% CONTROL COMMENTS
\newif\ifcomments
\commentstrue  % change this to false to hide our rainbow comments
%\commentsfalse

\usepackage{amsmath,amsthm,amssymb,amscd,url,enumerate}
\usepackage[utf8]{inputenc}
\usepackage{hyperref}
\usepackage{multirow}
\usepackage{pdflscape}
\usepackage{caption}
\usepackage[noadjust]{cite}
\usepackage{xypic}
\usepackage{ulem}
\usepackage{longtable}
\usepackage{tikz}
\usetikzlibrary{decorations.markings,arrows}
\usepackage{tikz-cd}
\usepackage{amsfonts}
\usepackage{amsmath}
\usepackage{graphicx}
\usepackage{amsthm}
\usepackage{xcolor}
\usepackage{mathrsfs}
\usepackage[margin=1in]{geometry}
\usepackage[ruled,vlined,linesnumbered,algosection]{algorithm2e}
\usepackage{algpseudocode}
\usepackage{csquotes}
\usepackage[all,cmtip]{xy} 
%\usepackage[notref]{showkeys}
\usepackage{float}
\usepackage[caption = false]{subfig}

\newenvironment{solution}
{\begin{proof}[\textbf{Proof}]}
	{\end{proof}}
\usepackage{enumerate}
\usepackage{tcolorbox}
\newtheorem{theorem}{Theorem}
\newtheorem{definition}{Definition}
\newtheorem{corollary}{Corollary}
\newtheorem{lemma}{Lemma}
\newtheorem{proposition}{Proposition}
\newtheorem{example}{Example}
\newtheorem{remark}{Remark}
%\newtheorem{algorithm}{Algorithm}
\usepackage{listings}
\usepackage{color}

\newcommand{\tri}[1]{\left|#1\right|}
\newcommand{\Q}{\mathbb{Q}}
\newcommand{\tron}[1]{\left(#1\right)}
\DeclareMathOperator{\QQ}{\mathbb{Q}} 
\DeclareMathOperator{\RR}{\mathbb{R}} 
\DeclareMathOperator{\ZZ}{\mathbb{Z}} 
\newcommand{\set}[1]{\left\{#1\right\}}
\DeclareMathOperator{\vol}{vol}
\DeclareMathOperator{\Tr}{Tr}
\DeclareMathOperator{\N}{N}
\DeclareMathOperator{\Gal}{Gal}
\DeclareMathOperator{\covol}{covol}


%\newcommand{\HT}[1]{\textcolor{red}{{\sf (Ha:} {\sl{#1})}}}




\title{\textbf{Well-Rounded ideal lattices of cyclic cubic and quartic fields}}


\author{ Dat T. Tran, Nam H. Le, Ha T. N. Tran}
%\address{Department of Mathematical and Physical Sciences, Concordia University of Edmonton, 7128  Ada Blvd NW, Edmonton, AB T5B 4E4, Canada}
%\email{hatran1104@gmail.com}




\keywords{well-rounded ideal, lattices, cyclic cubic field, cyclic quartic field}

\subjclass[2020]{Primary: 
	11R16, %Cubic and quartic extensions
	06B10, %Lattice ideals, congruence relations
	06B99, %None of the above, but in this section
	Secondary: 
	11Y40, %Algebraic number theory computations
	% https://mathscinet.ams.org/msnhtml/msc2020.pdf
}


\thanks{
	The authors would like to thank Amy Feaver for her help to improve the initial version of this manuscript. Ha T. N. Tran was supported by the Natural Sciences and Engineering Research Council of Canada (NSERC) (funding RGPIN-2019-04209 and DGECR-2019-00428). 
}

%\author{hatran1104}
%\date{April 2020}




\begin{document}
	\maketitle
	%%%%%%%%%%%%%%%%%%%%%
		\begin{abstract}
	

In this paper, we find criteria for when cyclic cubic and cyclic quartic fields  have well-rounded ideal lattices. We show that every cyclic cubic field has at least one well-rounded ideal. We also prove that there exist families of cyclic quartic fields which have well-rounded ideals and  explicitly construct their minimal bases.  In addition,  for a given prime number $p$, if a cyclic quartic field has a unique prime ideal above $p$, then we provide the necessary and sufficient conditions for that ideal to be well-rounded.  Moreover, in cyclic quartic fields, we provide the prime decomposition of all odd prime numbers and construct an explicit integral basis for every prime ideal.

\end{abstract}
	
%%%%%%%%%%%%%%%%%%%%%%%%%%%%%%%%%%%%%%%%%%%%%%%%%%%%%%%%%%%%%%%
\section{Introduction}
\section{Introduction}
\label{sec:introduction}
% \begin{itemize}
%     % Diffusion of FL
%     \item {\st{Diffusion of FL}}
%     % Security threats to FL
%     \item {\st{Security threats to FL with particular focus on model poisoning}}
%     % Limitations of existing countermeasures
%     \item {\st{Current countermeasures (e.g., KRUM) and their limitations}}
%     % Proposed method and its advantages
%     \item {\st{Intuitive description of the proposed method and its difference (i.e., advantages) w.r.t. state of the art}}
%     % Main contributions
%     \item {\st{Summary of the main contributions of this work}}
%     % Paper's structure and organization
%     \item {\st{Paper's structure and organization}}
% \end{itemize}

% Diffusion of FL
Recently, {\em federated learning} (FL) has emerged as the leading paradigm for training distributed, large-scale, and privacy-preserving machine learning (ML) systems~\cite{mcmahan2017googleai,mcmahan2017aistats}. 
The core idea of FL is to allow multiple edge clients to collaboratively train a shared, global model without disclosing their local private training data.
%Specifically, an FL system consists of a central server and many edge clients; 
A typical FL round involves the following steps: {\em(i)} the server randomly picks some clients and sends them the current, global model; {\em(ii)} each selected client locally trains its model with its own private data; then, it sends the resulting local model to the server;\footnote{Whenever we refer to global/local model, we mean global/local model {\em parameters}.} {\em(iii)} the server updates the global model by computing an \emph{aggregation function}, usually the average (FedAvg), on the local models received from clients.
% \begin{enumerate}
%     \item[{\em(i)}] the server sends the current, global model to the clients and appoints some of them for training;
%     \item[{\em(ii)}] each selected client locally trains its copy of the global model with its own private data; then, it sends the resulting local model back to the server;\footnote{Whenever we refer to global/local model, we mean global/local model {\em parameters}.}
%     \item[{\em(iii)}] the server updates the global model by computing an \emph{aggregation function} on the local models received from clients (by default, the average, also referred to as FedAvg~\cite{mcmahan2017aistats}).
% \end{enumerate}
This process goes on until the global model converges. %(e.g., after a certain number of rounds or other similar stopping criteria).
%\\
% The advantages of FL over the traditional, centralized learning paradigm are undoubtedly clear in terms of flexibility/scalability (clients can join/disconnect from the FL network dynamically), network communications (only model weights\footnote{We will use \textit{parameters} and \textit{weights} interchangeably.} are exchanged between clients and server), and privacy (each client's private training data is kept local at the client's end and not uploaded to the server).
\\
% Security threats to FL
%However, the growing adoption of FL also raises security concerns~\cite{costa2022covert}, particularly about its confidentiality, integrity, and availability.
Although its advantages over standard ML, FL also raises security concerns~\cite{costa2022covert}. %, particularly about its confidentiality, integrity, and availability~\cite{costa2022covert}.
% OLD, LONG VERSION
% Indeed, some work deals with privacy leakage that may expose the local data of some clients~\cite{melis2019sp}. 
% A large body of work, instead, investigates attacks that usually aim to detriment the predictive accuracy of the learned global model. For instance, \emph{data poisoning} attacks achieve this goal by letting an adversary pollute the training set of some corrupt FL clients with maliciously crafted examples~\cite{jagielski2018sp}.
% Similarly, in \emph{model poisoning} the attacker attempts to tweak the global model weights~\cite{bhagoji2019pmlr} by directly perturbing the local model's weights of some infected FL clients before these are sent to the central server for aggregation, usually via so-called Byzantine attacks. 
% It turns out that Byzantine model poisoning attacks severely impact standard FedAvg; therefore, more robust aggregation functions must be designed to make FL systems secure.
Here, we focus on \emph{untargeted model poisoning} attacks~\cite{bhagoji2019pmlr}, where an adversary attempts to tweak the global model weights %\footnote{We will use the terms \textit{parameters} and \textit{weights} interchangeably.} 
by directly perturbing the local model's parameters of some infected clients before these are sent to the central server for aggregation.
In doing so, the adversary aims to jeopardize the global model \textit{indiscriminately} at inference time.
Such model poisoning attacks severely impact standard FedAvg; therefore, more robust aggregation functions must be designed to secure FL systems.
\\
% In this paper, we focus on designing a novel robust aggregation scheme at the server's end to contrast the effect of Byzantine model poisoning attacks.
%
% Current countermeasures and their limitations
%Several countermeasures have been proposed in the literature to combat model poisoning attacks on FL systems.
% Some methods use simple statistics more robust than plain average to smooth the impact of malicious updates (e.g., Trimmed Mean and FedMedian~\cite{yin2018icml}). 
% Other defenses implement outlier detection techniques to discard malicious updates from the aggregation performed at the server's end. Those are either based on heuristics (e.g., Krum/Multi-Krum~\cite{blanchard2017nips} and Bulyan~\cite{mhamdi2018pmlr}) or data-driven approaches (e.g., K-means clustering~\cite{shen2016acm} or DnC via spectral analysis~\cite{shejwalkar2021ndss}). 
% Finally, some strategies rely on a centralized ``source of trust'' to spot potential malicious updates (e.g., FLTrust~\cite{cao2020fltrust}).
% Several countermeasures have been proposed in the literature to combat model poisoning attacks on FL systems, i.e., to discard possible malicious local updates from the aggregation performed at the server's end. 
% These techniques range from simple statistics more robust than plain average (e.g., Trimmed Mean and FedMedian~\cite{yin2018icml}) to outlier detection heuristics (e.g., Krum/Multi-Krum~\cite{blanchard2017nips} and Bulyan~\cite{mhamdi2018pmlr}) or data-driven approaches (e.g., spectral analysis via K-means clustering~\cite{shen2016acm} or spectral analysis), or methods based on ``source of trust'' (e.g., FLTrust~\cite{cao2020fltrust}).
% OLD, LONG VERSION
%Several countermeasures have been proposed in the literature to combat Byzantine model poisoning attacks on FL systems.
% Descriptive statistics
% For example, Trimmed Mean and FedMedian aggregate local model updates using more robust statistics than standard average~\cite{yin2018icml}.
%
% % Heuristics for outlier detection
% Many existing Byzantine-resilient strategies implement some outlier detection heuristics to discard the model updates sent by potentially malicious clients from the input of the aggregation function.
% One of the most popular heuristics is Krum~\cite{blanchard2017nips}.
% This strategy tries to mitigate the impact of Byzantine attacks by selecting as a global model the local model with the smallest sum of Euclidean distances to {\em all} the other local models.
% Although powerful, Krum requires the server to know (or, at least, estimate) the number of malicious FL clients upfront, which is generally impossible in a realistic attack scenario. %
% Moreover, Krum may become ineffective for complex, high-dimensional model parameter spaces due to the curse of dimensionality.
% Bulyan~\cite{mhamdi2018pmlr} tries to overcome this issue by combining Krum with a variant of Trimmed Mean.
% % Data-driven outlier detection
% Other strategies use data-driven outlier detection techniques -- e.g., via K-means clustering~\cite{shen2016acm} -- to spot potential malicious local model updates. 
% %For instance, Shen et al. propose to cluster local model updates with K-means and thus identify outliers.
%
% % Other techniques
% As far as the server is concerned, any local model received can be from a potential malicious client. 
% FLTrust~\cite{cao2020fltrust} assumes the server acts as a client, i.e., trains a local model on an additional {\em trustworthy} dataset at the server's end and compares it against all the local models from other clients. 
% This way, the server can rely on some ``source of trust'' when discarding potentially malicious clients.
%\\
% Limitations of existing Byzantine-resilient strategies
Unfortunately, existing defense mechanisms either rely on simple heuristics (e.g., Trimmed Mean and FedMedian by~\cite{yin2018icml}) or need strong and unrealistic assumptions to work effectively (e.g., foreknowledge or estimation of the number of malicious clients in the FL system, as for Krum/Multi-Krum~\cite{blanchard2017nips} and Bulyan~\cite{mhamdi2018pmlr}, which, however, cannot exceed a fixed threshold).
Furthermore, outlier detection methods using K-means clustering~\cite{shen2016acm} or spectral analysis like DnC~\cite{shejwalkar2021ndss} do not directly consider the temporal evolution of local model updates received.
Finally, strategies like FLTrust~\cite{cao2020fltrust} require the server to collect its own dataset and act as a proper client, thereby altering the standard FL protocol.
\\
% OLD, LONG VERSION
% Overall, existing Byzantine-resilient strategies are either simple heuristics (e.g., FedMedian) or, if they are more complex, they rely on strong and unrealistic assumptions to work effectively (e.g., knowing the number of malicious clients in the FL system in advance, as for Krum and alike).
% Furthermore, data-driven outlier detection methods do not consider the temporary evolution of local model updates received (e.g., K-means clustering). 
% Finally, strategies like FLTrust requires the server to collect its own dataset and act as a proper client, thereby altering the standard FL protocol.
%
% Description of the proposed method
This work introduces a novel pre-aggregation \textit{filter} robust to untargeted model poisoning attacks. Notably, this filter $(i)$ operates without requiring prior knowledge or constraints on the number of malicious clients and $(ii)$ inherently integrates temporal dependencies. 
The FL server can employ this filter as a preprocessing step before applying \textit{any} aggregation function, be it standard like FedAvg or robust like Krum or Bulyan.
Specifically, we formulate the problem of identifying corrupted updates as a multidimensional (i.e., matrix-valued) time series anomaly detection task. 
The key idea is that legitimate local updates, resulting from well-calibrated iterative procedures like stochastic gradient descent (SGD) with an appropriate learning rate, show \textit{higher predictability} compared to malicious updates. This hypothesis stems from the fact that the sequence of gradients (thus, model parameters) observed during legitimate training exhibit regular patterns, as validated in Section~\ref{subsec:intuition}. %until convergence. 
%This regularity may be more pronounced for smooth convex loss functions, but it can still be captured within an appropriate time window, even for more complex and convoluted loss surfaces. 
%We provide evidence of this claim in Appendix~B, where we show that the average mutual information (i.e., ``predictability''), calculated over pairs of legitimate model updates sent at different FL rounds, is significantly higher than the corresponding computation for a malicious client.
\\
Inspired by the matrix autoregressive (MAR) framework for multidimensional time series forecasting~\cite{chen2021je}, we propose the FLANDERS ({\em \textbf{F}ederated \textbf{L}earning meets \textbf{AN}omaly \textbf{DE}tection for a \textbf{R}obust and \textbf{S}ecure}) filter.
The main advantages of FLANDERS over existing strategies like FLDetector~\cite{zhao2020multivariate} are its resilience to large-scale attacks, where $50\%$ or more FL participants are hostile, and the capability of working under realistic non-iid scenarios.
We attribute such a capability to two key factors: $(i)$ FLANDERS works without knowing a priori the ratio of corrupted clients, and $(ii)$ it embodies temporal dependencies between intra- and inter-client updates, quickly recognizing local model drifts caused by evil players. Below, we summarize our main contributions:

\begin{itemize}
\item[{\em(i)}]
We provide empirical evidence that the sequence of models sent by legitimate clients is more predictable than those of malicious participants performing untargeted model poisoning attacks.
\\
\item[{\em(ii)}] 
We introduce FLANDERS, the first pre-aggregation filter for FL robust to untargeted model poisoning based on multidimensional time series anomaly detection.
\\
\item[{\em(iii)}] 
We integrate FLANDERS into Flower,\footnote{\scriptsize{\url{https://flower.dev/}}} a popular FL simulation framework for reproducibility.
\\
\item[{\em(iv)}] 
We show that FLANDERS improves the robustness of the existing aggregation methods under multiple settings: different datasets, client's data distribution (non-iid), models, and attack scenarios.
\\
\item[{\em(v)}] 
We publicly release all the implementation code of FLANDERS along with our experiments.\footnote{\scriptsize{\url{https://anonymous.4open.science/r/flanders_exp-7EEB}}}
\end{itemize}

% Paper's structure and organization
The remainder of the paper is structured as follows. %some related work and the current state-of-the-art solutions to security issues that FL entails. 
Section~\ref{sec:background} covers background and preliminaries. 
In Section~\ref{sec:related}, we discuss related work.
Section~\ref{sec:problem} and Section~\ref{sec:method} describe the problem formulation and the method proposed. % to tackle it. 
Section~\ref{sec:experiments} gathers experimental results. %, and Section~\ref{sec:limitations} discusses some limitations of this work.
Finally, we conclude in Section~\ref{sec:conclusion}.
 %discusses the limitations of this work and draws future research directions.
%reports conclusions and draws perspectives for future research directions.

%%%%%%% OLD %%%%%%%
%to overcome the resilience of Byzantine failures in distributed Stochastic Gradient Descent computations. 
% The strength of Krum is its time complexity, which is linear in the gradient dimension. 
% However, the robustness of the approach is guaranteed for gradient-based learning applications only when the majority of the clients are not compromised. 
% Besides, the aggregation mechanism of Krum, as well as that of similar methods, is robust from a coarse-grained perspective and does not provide solutions to errors and perturbations that may occur at inference time.
%A related approach to~\cite{blanchard2017nips} is the work of Su et al.~\cite{su2016dc}. Here, the authors propose an iterated approximate agreement to tackle a multi-layer scenario attacked by Byzantine agents. 
%However, the method works efficiently on the sole discrete context and it is inapplicable to continuous state environments.
%\gabri{Maybe, we should just talk about the main limitations of existing countermeasures without digging into their details (or, we can just mention Krum as this is the most popular one). I will move the description of all these methods to the Related Work section.}


%%%%%%%%%%%%%%%%%%%%%%%%%%%%%%%%%%%%%%%%%%%%%%%%%%%%%%%%%%%%%%%
\section{Background}\label{sec:bacground}

\section{Background on Network Calculus}
\label{sec: background}


\begin{figure*}[tbh]
\centering
\begin{subfigure}[b]{0.3\textwidth}
    \centering
    \includegraphics[width=\linewidth]{images/in-out.png}
    \caption{Arrival and departure data and their relation with delay $d(t)$ and backlog $b(t)$. For a FIFO system, the delay is the horizontal distance between $R(t)$ and $R^*(t)$ but some other multiplexing techniques may shift the data to a later priority, causing a longer delay.}
    \label{fig: data in-out}
\end{subfigure}
\hfill
\begin{subfigure}[b]{0.35\textwidth}
    \centering
    \includegraphics[width=\linewidth]{images/arrival-service.png}
    \caption{Characteristics of an arrival curve and a service curve. From any point of observation, the arriving data never exceeds its arrival curve; the departure data is also never less than the service curve with respect to the data arrival.}
    \label{fig: arrival-service curves}
\end{subfigure}
\hfill
\begin{subfigure}[b]{0.33\textwidth}
    \centering
    \includegraphics[width=\linewidth]{images/bound.png}
    \caption{Delay and backlog bounds of a system. Backlog is the maximum vertical distance between $\alpha(t)$ and $\beta(t)$; FIFO delay is their maximum horizontal distance; but for arbitrary multiplexing, the delay guarantee is when the system clears its buffer, thus it's the intersection of $\alpha(t)$ and $\beta(t)$.}
    \label{fig: system bounds}
\end{subfigure}
\caption{Network calculus framework. We let $R(t)$ and $R^*(t)$ be the arrival and departure data flow of a system; $\alpha(t)$ be the piecewise linear concave arrival curve and $\beta(t)$ be the piecewise linear convex service curve of a system.}
% \hossein{Better to show piece-wise linear concave arrival curve and piece-wise linear convex service curve instead of token-bucket and rate-latency.}}
\end{figure*}

We recall some of the network calculus essentials for a better understanding of the framework used in Saihu. In the following context, we use the following notation: $\mbb{R}^+$ is the set of non-negative real numbers; $[x]_+$ denotes $\max(0, x)$

The data flow is by convention modeled as a left-continuous wide-sense increasing function $R(t): \mbb{R}^+ \mapsto \mbb{R}^+$ with respect to time $t$~\cite{ncbook2001leboudec}. 

A system $\mcal{S}$ receives arrival data described as a cumulative function $R(t)$ and delivers departure data as another cumulative function $R^*(t)$. Figure~\ref{fig: data in-out} illustrates such a system $\mcal{S}$. The benefit of representing a system like this is that we can observe system backlog and delay with such a model. 

\begin{definition}[Backlog and Delay~\cite{ncbook2001leboudec}]
    The backlog of a system at time~$t$ is
    \begin{equation}
        b(t) = R(t) - R^*(t)
    \end{equation}
    
    The virtual delay of a FIFO system at time $t$ is
    \begin{equation}
        d_{FIFO}(t) = \inf \lbp \tau \geq 0 : R(t) \leq R^*(t+\tau) \rbp
    \end{equation}
\end{definition}



The backlog of a system can be viewed as the vertical distance between $R$ and $R^*$. The FIFO (\textit{First-in First-out}) delay is the horizontal distance between $R$ and $R^*$. One may obtain other delay values if the multiplexing technique is not FIFO.

% \begin{figure}
%     \centering
%     \includegraphics[width=0.9\linewidth]{images/in-out.png}
%     \caption{In/out data flow; delay and backlog}
%     \label{fig: data in-out}
% \end{figure}

Since we are interested in the system guarantee instead of a single instance of data flow, we would like to have general bounds to the arrival and departure data flows. Therefore, we define \textit{arrival curve} and \textit{service curve} as the bounds of arrival and departure data flows.

\begin{definition}[Arrival Curve~\cite{ncbook2001leboudec}]
    Given a wide-sense increasing function $\alpha: \mbb{R}^+ \mapsto \mbb{R}^+$, we say that a flow $R(t)$ is $\alpha$-constrained if and only if for all $s \leq t$:
    \begin{equation}
        R(t) - R(s) \leq \alpha(t-s)
    \end{equation}
    We say $R(t)$ has $\alpha$ as an arrival curve.
\end{definition}

\begin{definition}[Service Curve~\cite{ncbook2001leboudec}]
    Given a wide-sense increasing function $\beta: \mbb{R}^+ \mapsto \mbb{R}^+$ and $\beta(0) = 0$. A system $\mcal{S}$ having $R(t)$ and $R^*(t)$ as its arrival and departure flows. We say $\mcal{S}$ offers a service curve $\beta$ if and only if
    \begin{equation}
        R^*(t) \geq (R \otimes \beta)(t) =: \inf_{s \leq t} \lbp R(s) + \beta(t-s) \rbp
    \end{equation}
    where $\otimes$ denotes the min-plus convolution
\end{definition}

Figure~\ref{fig: arrival-service curves} illustrates the arrival and service curves. Any segment of arrival flow $R(t)$ is constrained by arrival curve $\alpha$ and the output curve $R^*(t)$ is always no less than the curve $R\otimes\beta$. As a result, an arrival curve upper bounds the incoming traffic, and a service curve lower bounds the outgoing traffic.

% \begin{figure}
%     \centering
%     \includegraphics[width=\linewidth]{images/arrival-service.png}
%     \caption{Arrival/Service curve}
%     \label{fig: arrival-service curves}
% \end{figure}

We consider 2 special types of curves throughout this paper, \textit{token-bucket} (or sometimes called \textit{leaky-bucket}) curve and \textit{rate-Latency} curve.

\begin{definition}[Token-bucket and Rate-latency~\cite{ncbook2001leboudec}]
    A token-bucket curve $\gamma_{r,b}$ with arrival rate $r$ and burst $b$ is defined as
    \begin{equation}
        \gamma_{r,b}(t) = b + rt
    \end{equation}

    A rate-latency curve $\beta_{R,T}$ with service rate $R$ and latency $T$ is defined as
    \begin{equation}
        \beta_{R,T}(t) = R \lb t - T \rb_+
    \end{equation}
\end{definition}

A token-bucket curve is determined by a burst $b$ and an arrival rate~$r$. Burst represents the maximum possible data volume that can arrive simultaneously, and arrival rate represents the maximum long-term data rate~\cite{bouillard2022tradeoff}.
A rate-latency curve is determined by a latency~$T$ and a service rate~$R$. Latency represents the time a server needs before starting to process the incoming data, and service rate represents the minimum rate to process data after the initial latency.

With the help of arrival and service curves, we can derive delay and backlog bounds for a system $\mcal{S}$ illustrated in Figure~\ref{fig: system bounds}. Suppose a system $\mcal{S}$ has arrival curve $\alpha$ and service curve~$\beta$, its worst-case backlog $b^*$ is the maximum vertical distance between~$\alpha$ and~$\beta$. Similarly, depending on the multiplexing technique applied to the system, its worst-case delay bound $d^*$ is the maximum horizontal distance between $\alpha$ and $\beta$ if $\mcal{S}$ is a FIFO system. If we don't have any information about its multiplexing technique, referred to as arbitrary multiplexing, the best we can say is that when $\alpha$ and $\beta$ intersect each other, where all data has been delivered out of the system. Consequently, the worst-case delay bound for arbitrary multiplexing is the time required for $\mcal{S}$ to clear its buffer.

% \begin{figure}
%     \centering
%     \includegraphics[width=\linewidth]{images/bound.png}
%     \caption{System delay/backlog bounds}
%     \label{fig: system bounds}
% \end{figure}

While a service curve captures the slowest possible output speed of a system, a link's transmission capacity limits the speed as well. Hence, we model this phenomenon using a \textit{greedy shaper} with a sub-additive function $\sigma: \mbb{R}^+ \mapsto \mbb{R}^+$ concatenated with a server. We consider a concatenation as shown in Figure \ref{fig: system}. By convention we assume $\sigma(0) = 0$ and $\beta(t) \leq \sigma(t), \forall t \in \mbb{R}^+$, meaning that the buffer is cleared at the beginning and the service never exceed its physical limitation. With the above definition, such greedy shaper conserves the service provided by the system due to theorem \ref{thm: shaping}.

\begin{figure}[thb]
    \centering
    \includegraphics[width=0.7\linewidth]{images/system.png}
    \caption{Shaping of departure data. A flow that has an arrival curve $\alpha$ feeds into a server with an arrival data flow $R(t)$. The server having service curve $\beta$ takes $R(t)$ and gives a departure data flow $R^*(t)$ to a shaper with shaping function $\sigma$. The shaper takes $R^*(t)$ and shape the data flow as another departure $D(t)$.}
    \label{fig: system}
\end{figure}


\begin{theorem}[Shaping conserves service \cite{ncbook2001leboudec}]
\label{thm: shaping}
Following the system shown in Figure \ref{fig: system}, we have
\begin{equation}
     D = R^* \otimes \sigma \geq \lp R \otimes \beta \rp \otimes \sigma = R \otimes \lp \beta \otimes \sigma \rp = R \otimes \beta
\end{equation}
\end{theorem}

In the following context, we model the shaping function $\sigma$ as a token-bucket curve $\gamma_{C,L}$ with transmission capacity $C$ and the packet size $L$ to capture the link capacity and packetization~\cite{bouillard2022tradeoff}.



%%%%%%%%%%%%%%%%%%%%%%%%%%%%%%%%%%%%%%%%%%%%%%%%%%%%%%%%%%%%%%%
\section{Well-rounded ideal lattices of cyclic cubic fields} \label{sec:cubic}

Let $F$ be a cyclic cubic field with conductor $m$. In this section, we will find WR ideals of $F$ and compute minimal bases of these ideals. 

We denote by $P_i$ the unique prime ideal above the prime $p_i\mid m$ for each $i\ge 0$ and $\alpha$ a root of the defining polynomial $df(x)$ as in \eqref{df-polynomial-cubic}. We will fix these notations for the whole section. 

%%%%%%%%%%%%%%%%%%%%%%%%%%%%
\subsection{The case $9\nmid m$}
Let $m= p_1 \cdots p_r$ here $7 \le p_1 < p_2 \cdots < p_r$, $p_i \equiv 1 \mod 3$ for all $i$ and $r \ge 1$. In this section, we will show that: \\
1) the ideal $(P_1\cdots P_r)^2$ is an orthogonal and WR- this result has not been proven before; and\\
2) if $I\subset \{1,\cdots,r\}$, then $\prod_{i\in I}P_i$ is WR if and only if $\dfrac{m}{4}\le \tron{\prod_{i\in I}p_i}^2\le 4m$.


% We have two following lemmata. 




\begin{lemma}\label{integralbasis-3notdiv9}
	One has $\{\alpha, \sigma(\alpha), \sigma^2(\alpha)\}$ and $\{1, \alpha, \sigma(\alpha)\}$  are two integral bases of $\mathcal{O}_F$.
\end{lemma}
From now on, we will use one of the integral bases as mentioned in Lemma \ref{integralbasis-3notdiv9} depending on which one is convenient for our calculation. 


By \cite[page 166]{narkiewicz1974elementary}, also \cite[page 2]{de2017integral} and by \cite{maki2006determination}, we obtain Lemma \ref{lencoeff}.

\begin{lemma}\label{lencoeff}
	Let $z = z_1 \alpha + z_2  \sigma(\alpha)  + z_3 \sigma^2(\alpha) \in O_F$ where $z_i \in \mathbb{Z}, 1 \le i \le 3$. Then 
	$$\|z\|^2 = \Tr(z^2)= m (z_1^2 + z_2 ^2 + z_3^2) + \frac{(1-m)(z_1+z_2+z_3)^2}{3}.$$
	Moreover, one can rewrite this expression as 
	$$\|z\|^2= \frac{m}{3}\left( (z_1- z_2)^2 + (z_2-z_3)^2 + (z_3- z_1)^2 \right) + \frac{1}{3}(z_1+z_2+z_3)^2.$$
\end{lemma}


Since $\alpha$ is a root of the defining polynomial $df(x)$ (see \eqref{df-polynomial-cubic}), using \cite[Proposition 2.2]{TranPeng1}, one can show that $\Tr(\alpha)=1$ and  $\alpha$ is a shortest vector in $O_F\backslash \mathbb{Z}$ with $\|\alpha\|^2 = \frac{2m+1}{3}$. 

For $\ell \in \mathbb{Z}$ and $\ell>0$, as in \cite{DC19}, we define
$$M_{\ell} = \{z = z_1 \alpha + z_2  \sigma(\alpha)  + z_3 \sigma^2(\alpha) \in O_F:  z_1 + z_2 + z_3 \equiv 0 \mod \ell \}.$$

For all $\ell$, the set $M_{\ell}$ is a $\mathbb{Z}$-module. %Now we prove a condition so that $M_{\ell}$ is an ideal of $O_F$. 
We remark that in \cite{DC19}, it is proved that the sublattice $M_{\ell}$ of $O_F$ has index $\ell$ and it is WR if $\ell\equiv 1\pmod 3$ and $\sqrt{\dfrac{m}{4}}\le \ell \le \sqrt{4m}$. Thus if an ideal of $O_F$ of norm satisfies these conditions,  then that ideal is also WR. We prove the following. 
\begin{lemma}\label{idealcondition}
	The set $M_{\ell}$ is an ideal of $O_F$ if and only if $\ell|m$.
\end{lemma}
\begin{proof}
	See Appendix \ref{proof_of_lemma_11}.
\end{proof}

\begin{lemma}\label{idealPi0}
	Assume that $p_i = 3 n_i +1$. Then $p_i O_F = P_i^3$ where $P_i$ is the unique prime ideal above $p_i$ and $P_i = \langle p_i, \alpha + n_i \rangle$. Moreover, one has $-\alpha + \sigma(\alpha) \in P_i$,  $ \|-\alpha + \sigma(\alpha)\|^2= 2m$ and $\| \alpha + n_i\|^2= \frac{2m+p_i^2}{3}$.
\end{lemma}

\begin{proof}See Appendix \ref{proof_of_lemma_12}
\end{proof}

\begin{lemma}\label{traceP}
	We have $M_{p_i}=P_i$. As a consequence, $p_i| \Tr(z)$ for all $z \in P_i$.
\end{lemma}
\begin{proof}
	When ${p_i}\mid m$, by Lemma \eqref{idealcondition}, $M_{p_i}$ is an ideal. Moreover, it is a prime ideal above $p_i$ as its index is $p_i$. Therefore $M_{p_i} = P_i$. 
\end{proof}  

\begin{lemma}\label{lemcomph1}
	Let $m=p_1\cdots p_r(r\ge 1)$ and $9\nmid m$. Let $\rho=\alpha -\sigma(\alpha)$. Then $\rho\in P_i$ for all $i = 1,\cdots r$ and $\|\rho^2\|^2=\Tr(\rho^4) = 2m^2$. \end{lemma}
\begin{proof}
	By Lemma \ref{traceP}, we have $P_i =\langle p_i, \alpha-n_i\rangle $. The statement $g\in P_i$ is implied from equalities $\alpha-\sigma(\alpha) = (\alpha-n_i)+(\sigma(\alpha)-n_i)$ and $\sigma(P_i)=P_i,\forall i=1,\cdots , r$.
	
	Now, we compute $\|\rho^2\|^2$. First, one has \begin{align}\label{eq:gsquare}
	\|\rho^2\|^2 = (\alpha-\sigma(\alpha))^4+(\sigma(\alpha)-\sigma^2(\alpha))^4+(\sigma^2(\alpha)-\alpha)^4.
	\end{align}
	The right side of \eqref{eq:gsquare} is a symmetric polynomial in $\delta_1= \alpha+\sigma(\alpha)+\sigma^2(\alpha)=1, \delta_2 = \alpha\sigma(\alpha)+\sigma(\alpha)\sigma^2(\alpha)+\sigma^2(\alpha)\alpha=\dfrac{1-m}{3} , \delta_3= \alpha\sigma(\alpha)\sigma^2(\alpha)= \dfrac{m(a-3)+1}{27}$. Expressing it in terms of these symmetric polynomials, one implies $\|\rho^2\|^2 =  2m^2$.
\end{proof}



The following result is new and has not been studied before. We remark that our WR lattice $P$ in Proposition \ref{prop:Psquare_is_WR}  is not one of the sublattices mentioned in \cite[Theorem 4.9]{DM20} since its norm is $m^2$.

\begin{proposition}
	\label{prop:Psquare_is_WR} Let $m =  p_1\cdots p_r $ where $r\ge 1$ and let $P = P_1\cdots P_r$. Then $P^2$ is an orthogonal WR ideal lattice with a minimal basis $\{\kappa,\sigma(\kappa),\sigma^2(\kappa)\}$ where $\kappa= m-(\alpha-\sigma(\alpha))^2$.
\end{proposition}

\begin{proof}
	One has $Tr(\kappa)= m$ and $\|\kappa\|^2 = m^2$. By Lemma \ref{lemindependentcubic}, the set $\{\kappa,\sigma(\kappa),\sigma^2(\kappa)\}$ is $\mathbb{R}-$ linear independent. It is clear $m\in P^2$ and thus $\kappa\in P^2$ by Lemma \ref{idealPi0}
	
	Now, we prove $\kappa$ is a shortest vector in $P^2$. First, consider the sublattice $L=  \ZZ \kappa+\ZZ \sigma(\kappa)+\ZZ \sigma^2(\kappa)\ZZ$ of $P^2$. Remark that $\kappa+\sigma(\kappa) = \Tr(\kappa)-\sigma^2(\kappa)= (\alpha-\sigma^2(\alpha))^2$. It leads to \begin{align}\label{eq:eqintheo}
	\|\kappa\|^2 +\|\sigma(\kappa)\|^2+2\Tr(\kappa\sigma(\kappa)) =  \|(\alpha-\sigma^2(\alpha))^2\|=2m^2
	\end{align} by Lemma \ref{lemcomph1}. Since $\|\kappa\|^2 =\|\sigma(\kappa)\|^2 = m^2$, the equality in \eqref{eq:eqintheo} implies that $\Tr(\kappa\sigma(\kappa)) =0$. It follows that $\kappa,\sigma(\kappa),\sigma^2(\kappa) $ are pairwise orthogonal. As the consequence, $\det(L) = m^3 =\det(P^2)$ and $\kappa$ is a shortest vector of $L$. By Lemma \ref{sublatticeequal}, one has $L= P^2$. Thus $P^2$ is an orthogonal WR ideal lattice with a minimal basis $\{\kappa,\sigma(\kappa),\sigma^2(\kappa)\}$.
\end{proof}

Let $I$ be a non-empty subset of set $\{1,\cdots,r\}$, $p_I = \prod_{i\in I}p_i$ and $P_I=\prod_{i\in I}P_i$. As a consequence of Lemma \ref{traceP}, $P_I = M_{p_I}$. By \cite[Theorem 4.1]{de2017integral}, if $p_I\in \left[\dfrac{\sqrt{m}}{2},2\sqrt{m}\right]$ then $P_I$ is WR. Moreover, by using an independent technique with the one in proof of \cite[Theorem 4.1]{de2017integral}, we can prove a stronger result that  the condition $p_I\in \left[\dfrac{\sqrt{m}}{2},2\sqrt{m}\right]$ is not only necessary but also sufficient for $P_I$ to be WR. 

%\HT{Related to the comment of the reviewer: ``Are there infinitely many non-%equivalent WR ideal
%lattices arising from (different) cubic/quartic fields?"}

\begin{proposition}\label{prop:cubic9ndividem}
	Let $m = p_1 \cdots p_r$ be the conductor of $F$ and let $P_i$ be the prime ideals above $p_i$ for all $i=1,\cdots r$. For each nonempty subset $I$ of $\{1,\cdots ,r\}$, let $P_I = \prod_{i\in I }P_i,p_I=\prod_{i\in I }p_i $ and $n_I = \dfrac{p_I-1}{3}$. Then $P_I$ is WR if and only if $\dfrac{m}{4}\le p_I^2 \le 4m$. In this case, $P_I$ has a minimal basis $\alpha+n_I,\sigma(\alpha)+n_I, \sigma^2(\alpha)+n_I$.
\end{proposition}
\begin{proof}
	By Lemma \ref{idealPi0}, $P_i = \langle p_i, \alpha+n_i\rangle$ where $n_i = \dfrac{p_i-1}{3}$ for all $i\in I$. It implies that $\ZZ\tron{\alpha+n_I}+\ZZ \tron{\sigma(\alpha)+n_I}+\ZZ\tron{\sigma^2(\alpha)+n_I}\subset P_I$. Moreover, these lattices have the same indices in $O_F$ and thus $ P_I=\ZZ\tron{\alpha+n_I}+\ZZ \tron{\sigma(\alpha)+n_I}+\ZZ\tron{\sigma^2(\alpha)+n_I}$.
	
	Let $\delta$ be  a nonzero vector of $P_I$. There exist integers $x_1,x_2,x_3$ such that $\delta = x_1\tron{\alpha+n_I}+x_2\tron{\sigma(\alpha)+n_I}+x_3\tron{\sigma^2(\alpha)+n_I}$. %Then \begin{align*}
	%\delta &= \tron{(n_I+1)x_1+n_Ix_2+n_Ix_3}\alpha + \tron{n_Ix_1+\tron{n_I+1}x_2+n_Ix_3}\sigma(\alpha)+\tron{n_Ix_1+n_Ix_2+\tron{n_I+1}x_3}\sigma^2(\alpha).
	%\end{align*}
 By Lemma \ref{lencoeff}, we have $$\|\delta\|^2 = \dfrac{m}{3}\tron{\tron{x_1-x_2}^2+\tron{x_2-x_3}^2+\tron{x_3-x_1}^2}+\dfrac{\tron{3n_I+1}^2}{3}\tron{x_1+x_2+x_3}^2.$$ 	
	Now, we will find the minimum value of $\|\delta\|^2$ when $\delta\ne 0$. Note that $z_1z_2z_3\ne 0$. We consider all cases as below.
	\begin{enumerate}[(i)]
		\item If $z_1+z_2+z_3 =0$, then $(z_1-z_2)^2 +(z_2-z_3 )^2 +(z_3-z_1)^2\ge 2$. Here $(z_1-z_2)^2 +(z_2-z_3 )^2 +(z_3-z_1)^2$ is an even non-negative integer. If $(z_1-z_2)^2 +(z_2-z_3 )^2 +(z_3-z_1)^2\in \{2,4\}$, then two of the three numbers $z_1,z_2,z_3$ are zero. Without loss of generality, we can assume $z_1=z_2$. It implies that $z_3 =-2z_1$ and thus $(z_1-z_2)^2 +(z_2-z_3 )^2 +(z_3-z_1)^2$ is a multiple of $9$. Hence $(z_1-z_2)^2 +(z_2-z_3 )^2 +(z_3-z_1)^2\ge 6$. Therefore, $\|\delta\|^2\ge 2m$ in this case. The equality occurs if and only if $\delta \in \{\pm \tron{\alpha-\sigma(\alpha)},\pm \tron{\sigma(\alpha)-\sigma^2(\alpha)},\pm \tron{\sigma^2(\alpha)-\alpha}\}$.
		\item If $(z_1-z_2)^2 +(z_2-z_3 )^2 +(z_3-z_1)^2=0$, then $z_1=z_2=z_3=z\in \ZZ$ and thus $\delta =3zp_I$. Hence $\|\delta\|^2 \ge 3p_I^2$. The equality occurs if and only if $\delta \in \{\pm p_I\}$.
		\item If $z_1+z_2+z_3\ne 0 $ and $(z_1-z_2)^2 +(z_2-z_3)^2 +(z_3-z_1)^2\ne 0$, then $(z_1+z_2+z_3)^2 \ge 1 $ and $(z_1-z_2)^2 +(z_2-z_3 )^2 +(z_3-z_1)^2\ge 2$. Thus $\|\delta \|^2\ge \dfrac{p_I^2+2m}{3}$. The equality occurs if and only if $\delta \in \{\pm \tron{\alpha+n_I}, \pm \tron{\sigma(\alpha)+n_I},\pm \tron{\sigma^2(\alpha)+n_I}\}$. 
	\end{enumerate}
	Therefore $P_I$ is WR if and only if $\dfrac{2m+p_I^2}{3}\le \min\{ 2m,3p_I^2\}$ which is equivalent to $ \dfrac{m}{4} \le p_I^2 \le 4m$. 
\end{proof}


%%%%%%%%%%%%%%%%%%%%%%%%%%%%%%%%%
\vspace*{0.5cm}
\subsection{The case $9\mid m$}


Let $m= p_0^2  p_1 \cdots p_r$ here $p_0=3 < p_1 < p_2 \cdots < p_r$ and $r \ge 0$. For each nonempty subset $I$ of $\{1\cdots ,r\}$, we denote $P_I =  \prod_{i\in I}P_i$. In this section, we will show that: 

i) if $m= 9$, then $P_0$ is WR; 

ii) the ideal $P_0(P_1\cdots P_r)^2$ is orthogonal and WR; 

iii) if $I$ is a nonempty subset of $\{1,2\cdots,r\}$, then $P_0P_I$ is WR if and only if $\dfrac{m}{36}\le p_I^2\le \dfrac{4m}{9}$; and 

iv) if $r\ge 2$ and $I, J$ are two nonempty and disjoint subsets of $\{1,2,\cdots,r\}$, then $P_0P_I^2P_J$ is WR if and only if $\dfrac{m}{36}\le p_Ip_J^2\le \dfrac{4m}{9}$. The field $F$ is not tame, hence is  not  studied in \cite{DM20} and \cite{DC19}. Indeed, all of our results in this subsection are  new and have not been investigated before. 

By \cite{maki2006determination}, one has $\{1,\alpha, \sigma(\alpha)\}$ is an integral basis. It can be verify easily that $\|\alpha\|^2 = \frac{2m}{3}$. Thus, $\alpha$ is a shortest vector in $O_F\backslash \mathbb{Z}$  (see \cite{TranPeng1} for more details).

\begin{lemma}\label{idealPi}
	Let $m= 9 p_1 \cdots p_r$ where $r \ge 0$. Then $p_i O_F = P_i^3$ where $P_i$ is the unique prime ideal above $p_i$ and $P_0=\langle 3, \alpha -1 \rangle$ and $P_i = \langle p_i, \alpha \rangle$ for all $1 \le i \le r$.
\end{lemma}

\begin{proof}
	To compute generators for $P_i$ we can apply the decomposition of primes  \cite[Theorem 4.8.13]{cohen1993course}  since Lemma \ref{lem:index-cubic} says that $p_i$ does not divide the index $[O_F: \mathbb{Z}[\alpha]$. In other words, the result is obtained by factoring the defining polynomial $df(x)$ over the finite field $F_{p_i}$ and by using the fact that $p_i |m$, and $a \equiv 6 \mod 9$. 
\end{proof}
In case $m>9$, using Lemma \ref{idealPi} and the fact that $\frac{2m}{3}> 27 =\|3\|^2$ leads to the following.

\begin{corollary}\label{coridealPi1}
	Let $m >9$. Then the vector $\alpha$ is a shortest vector in the set $P_i\backslash \mathbb{Z}$ for all $ 1 \le i \le r$, and $\|\alpha\|^2 = \frac{2m}{3}$. In the ideal $P_0$, the element $p_0$ is shortest and $\|p_0\|^2 = 27$.
	
\end{corollary}


\begin{proposition}
	\label{prop:cubic9dividem2}
	Let $m =9$. Then $P_0$ is orthogonal and WR with a minimal basis $\{\alpha-1, \sigma(\alpha)-1,  \sigma^2(\alpha )-1\}$.
\end{proposition}


\begin{proof}
	Note that $\alpha-1 \in P_0$ and this element has trace $-3$ since $\Tr(\alpha)=0$, thus the three elements $\alpha-1, \sigma(\alpha)-1,  \sigma^2(\alpha)-1 $ are all in $P_0$ and are independent by Lemma \ref{lemindependentcubic}. To show that $P_0$ is WR, it is sufficient to show that $\alpha-1$ is shortest in $P_0$. 
	
	We have that $\|\alpha-1\|^2= \|\alpha\|^2 + \|-1\|^2 = \frac{2m}{3}+ 3 = 9$ because $\alpha$ has trace $0$. One can easily compute all the shortest vectors of  the ideal lattice $P_0$  (see the Fincke--Pohst algorithm- Algorithm 2.12 in \cite{fincke1985improved}) and verify that $\alpha-1$ is indeed shortest in $P_0$.
\end{proof}

Lemma \ref{lencoeff} cannot be applied for the case $9\mid m$. Therefore, we recalculate the length of vectors in $\mathcal{O}_F$ in this case as follows.
\begin{lemma}\label{lem:length-cubic-3divm}
	Let $\delta = m_1 + m_2 \alpha + m_3 \sigma(\alpha) \in O_F$. Then $$\|\delta\|^2= 3m_1^2+
	\dfrac{2m}{3}(m_2^2+m_3^2-m_2m_3).$$
\end{lemma}
\begin{proof}
    See Appendix \ref{proof_of_lemma_17}
\end{proof}

Next, we will claim that the $P_0P^2$ is an orthogonal and WR lattice where $P= P_1\cdots P_r$. To prove that, we need some lemmata as below.
\begin{lemma}
	\label{lem:idealP_3midm}For all $1\le i\le r$, we have $P_i = \ZZ p_i\oplus\ZZ \alpha \oplus\ZZ \sigma(\alpha)$.
\end{lemma}
\begin{proof}
	It is clear that $L_i =  \ZZ p_i\oplus\ZZ \alpha \oplus\ZZ \sigma(\alpha)$ is the sublattice of $P_i$ and $\det(L_i) =\det( P_i)$. Therefore $P_i = L_i$ by Lemm \ref{sublatticeequal}.
\end{proof}
By using the same technique as in the proof of Lemma \ref{lem:idealP_3midm}, one has the following result.
\begin{corollary}
	\label{cor:idealP_I} Let $I$ be a subset of $\{1,\cdots,r\}$. Then $P_I =  \ZZ p_I +\ZZ \alpha +\ZZ \sigma(\alpha)$. In particular, $P_1 \cdots P_r =\ZZ \dfrac{m}{9}\oplus\ZZ \alpha \oplus\ZZ \sigma(\alpha).$
\end{corollary}

\begin{proposition}\label{prop:cubic9dividem1}
	Let $I$ be a subset of $\{1,\cdots,r\}$. Then $P_I$ is not WR.
\end{proposition}
\begin{proof}
	By Corollary \ref{cor:idealP_I}, we have  $P_I =  \ZZ p_I +\ZZ \alpha +\ZZ \sigma(\alpha)$. Let $\delta \in P_I$, then $\delta = z_1p_I + z_2\alpha +z_3 \sigma(\alpha)$ where $z_1,z_2,z_3\in \ZZ$. By applying Lemma \ref{lem:length-cubic-3divm},  one obtains 
	\[\|\delta\|^2 = 3z_1^2p_I^2+\dfrac{2m}{3}(z_2^2+z_3^2-z_2z_3).\]
		Now, we will find the minimum value of $\|\delta\|^2$ when $\delta\ne 0$. We consider all cases as below.
		\begin{enumerate}
			\item If $z_1 = 0 $, then $\|\delta\|^2 \ge \dfrac{2m}{3}$ (since $z_2^2+z_3^2-z_2z_3\ge 1$), here the equality  occurs when $z_2 = 1,z_3=0$ or $z_2=0,z_3=1$, therefore $\delta \in \{\alpha,\sigma(\alpha)\}$.
			\item If $z_1\ne 0$, then $\|\delta\|^2\ge 3z_1^2p_I^2+\dfrac{2m}{3}(z_2^2+z_3^2-z_2z_3)\ge 3z_1^2p_I^2\ge 3p_I^2$, here the equality  occurs when $z_2=z_3=0,z_1=1$ and thus $\delta = p_I$.
		\end{enumerate}
	In conclusion, $\min_{\delta \ne 0}\|\delta\| \in \left\{\|\alpha\|^2, \|p_I\|^2\right\} = \left\{\dfrac{2m}{3},3p_I^2\right\}$. Note that $\dfrac{2m}{3}\ne 3p_I^2$, so in the case $\|p_I\|^2 < \|\alpha\|^2$,  we have $\pm p_I$ are the only two shortest vectors in $P_I$. Therefore $P_I$ is not WR. In another case $\|p_I\|^2 > \|\alpha\|^2$ and hence $\alpha$ is shortest in $P_I$. We will next compute  the set of all shortest vectors $L$ of $P_I$. Let $\delta \in O_F$ such that $\|\delta \|= \|\alpha\|$. Since $O_F = \mathbb{Z} \oplus \mathbb{Z}[\sigma] \cdot \alpha$ (see \cite[Proposition 2.2 and Proposition 2.3]{TranPeng1}), we can show easily that $\delta\in L=\{\pm \alpha, \pm \sigma(\alpha), \pm \sigma^2(\alpha)\}$.  Moreover one has $\Tr(\alpha) = \alpha + \sigma(\alpha) + \sigma^2(\alpha)=0$ and $\{\alpha, \sigma(\alpha),\alpha^2(\alpha)\}$ linearly dependent. Therefore, there does not exist three independent vectors from $L$. In other words, $P_I$ is not WR. 
\end{proof}



\begin{lemma}		\label{lem:exprA_B} 
	There exist integers $A,B$ such that $A^2 -AB+B^2  = \dfrac{m}{9}$ and 
	$\alpha^2 = \dfrac{2m}{9}+A\alpha +B \sigma(\alpha)$.
\end{lemma}
\begin{proof}
See Appendix \ref{proof_of_lemma_19}
\end{proof}
\begin{lemma}
	\label{lem:idealP1Pr} Let $\alpha,A,B$ be in Lemma \ref{lem:exprA_B} and let $\kappa = \dfrac{m}{9}+A\alpha +B\sigma(\alpha)$. Then $P_0 ( P_1\cdots P_r)^2 =  \ZZ \kappa \oplus\ZZ \sigma(\kappa) \oplus\ZZ\sigma^2(\kappa).$  
\end{lemma}
\begin{proof}It is clear that the two lattices $P_0 ( P_1\cdots P_r)$ and $ \ZZ \kappa \oplus\ZZ \sigma(\kappa) \oplus\ZZ\sigma^2(\kappa)$ have the same index in $O_F$ and thus it is sufficient to prove that $ \ZZ \kappa \oplus\ZZ \sigma(\kappa) \oplus\ZZ\sigma^2(\kappa)$ is a sublattice of $P_0 ( P_1\cdots P_r) $. It is obvious that $\dfrac{m}{9} \in (P_1 \cdots P_r)^2$. Since $\kappa = \alpha ^2 -\dfrac{m}{9}$, then $\kappa \in (P_1\cdots P_r)^2$. Moreover, $\kappa =  \alpha^2 -\dfrac{m}{9}= (\alpha-1 )(\alpha +1 )+(p_1 \cdots p_r -1 ) \in  P_0 $ as $P_0 =  \langle 3 ,\alpha -1 \rangle $ and $p_1 \equiv 1 \pmod3$. Hence, $\kappa \in  P_0 (P_1\cdots P_r)^2 $. As a consequence, $\sigma (\kappa), \sigma^2 (\kappa)\in  P_0(P_1\cdots P_r)^2.$   
\end{proof}

Lemma \ref{lem:idealP1Pr} gives us an integral basis of $P_0(P_1\cdots P_r)^2$. Let $\delta =  z_1 \kappa +z_2 \sigma(\kappa)+z_3\sigma^2(\kappa)\in P_0(P_1\cdots P_r)^2.$ One has \begin{align*}
\delta = \dfrac{m}{9}( z_1+z_2+z_3)+(Az_1 -Bz_2+(B-A)z_3 )\alpha + (Bz_1+(A-B)z_2-Az_3)\sigma(\alpha).
\end{align*} 
We then apply Lemma \ref{lem:length-cubic-3divm}, to obtain that 
\begin{align}\label{eq:eq_before_propP0Pr_WR}
\|\delta\|^2 = \dfrac{m^2 }{27}(z_1+z_2+z_3)^2+\dfrac{2m}{3}(A^2-AB+B^2)(z_1^2+z_2^2+z_3^2-z_1z_2-z_1z_3-z_2z_3).
\end{align}


Since $\dfrac{m}{9} = A^2 -AB+B^2$, the following result is followed.
\begin{proposition}\label{prop:P0Psquare_WR}
	The ideal $P_0(P_1\cdots P_r)^2$ is orthogonal and WR with a minimal basis 
	$\{\kappa,\sigma(\kappa),\sigma^2(\kappa)\}$ with $\kappa$ as in Lemma \ref{lem:idealP1Pr}.
\end{proposition}

\begin{proof}
	Let $\delta  \in P_0 (P_1\cdots P_r)^2$. Then there exist integers $z_1,z_2,z_3$ such that $\delta =z_1 \kappa+z_2 \sigma(\kappa)+z_3 \sigma^2 (\kappa)$ by Lemma \ref{lem:idealP1Pr}. 
	Since $\dfrac{m}{9} =A^2 -AB+B^2$, the equality in \eqref{eq:eq_before_propP0Pr_WR} implies that \begin{align*}
	\|\delta\|^2 = \dfrac{m}{9}(z_1^2 +z_2^2 +z_3^2).
	\end{align*}
	When $\delta \ne 0$, it is clear that $\|\delta\|^2 \ge \dfrac{m^2 }{9}$ as at least one of $z_1,z_2 ,z_3$ is a nonzero integer. The equality is occur if and only if $\delta \in \{\pm \kappa,\pm \sigma(\kappa),\pm  \sigma^2(\kappa)\}$. Hence $\{\pm \kappa,\pm \sigma(\kappa),\pm  \sigma^2(\kappa)\}$ is the set of all  shortest vectors of $P_0 (P_1 \cdots P_r)^2.$ Therefore, $P_0(P_1\cdots P_r)^2$ is WR. Moreover, we can verify that $\Tr\tron{\kappa \sigma\tron{\kappa}} = 0$ and thus $P_0(P_1\cdots P_r)^2$ is also orthogonal.
\end{proof}
From now on, for each nonempty subset $I$ of $\{ 1,2,\cdots , r\}$, we denote by $p_I = \prod_{i\in I }p_i$ and $P_I   =  \prod_{i\in I } P_i$.

For each $i\in \{ 1,\cdots, r\}$, let $\rho_i = p_i+\alpha +\sigma(\alpha)$. Since $\rho_i =(p_i-1 )+(\alpha -1)+(\sigma(\alpha)-1) \in P_0$ and it is clear that $p_i \in P_i$, then $\rho_i\in P_0P_i$. Hence $\ZZ \rho_i +\ZZ\sigma(\rho_i) +\ZZ\sigma^2(\rho_i)$ is a sublattice of $P_0P_i$ and this sublattice has the same determinant as the one of $P_0P_i$. Therefore $\ZZ \rho_i +\ZZ\sigma(\rho_i) +\ZZ\sigma^2(\rho_i) = P_0P_i$.


By using the same argument, we can prove the following lemma.

\begin{lemma}
	\label{lem:P0PI_ideal} Let $r\ge 1$ and $I$ be a nonempty subset of $\{ 1,\cdots ,r\}$ and let $ \rho_I = p_I +\alpha +\sigma(\alpha).$ Then $P_I = \ZZ \rho_I  \oplus\ZZ \sigma(\rho_I)\oplus\ZZ\sigma^2 (\rho_I)$. In particular, $P_0P_1\cdots P_r = \ZZ \rho \oplus\ZZ \sigma(\rho)\oplus\ZZ\sigma^2(\rho)$ where  $\rho= \dfrac{m}{9}+\alpha+\sigma(\alpha)$.  
\end{lemma}

The following proposition shows the necessary and sufficient conditions for the ideal $P_0P_I^2$ of a given subset $I$ of $\{1,2\cdots,r\}$ to be a WR lattice. \\



\begin{proposition}
	\label{prop:P0PI_WR} Let $I$ be a nonempty subset of $\{ 1,2 \cdots,r\}$. The ideal $P_0P_I$ is WR if and  only if $\dfrac{m}{36 }\le p_I^2 \le \dfrac{4m}{9}$. In this case, a minimal basis of $P_0P_I$ is $\{\rho_I,\sigma(\rho_I),\sigma^2(\rho_I)\}$ where  $\rho_I = p_I+\alpha+\sigma(\alpha)$.
\end{proposition}
\begin{proof}
	By Lemma \ref{lem:P0PI_ideal}, one has $P_I =  \ZZ \rho_I \oplus\ZZ \sigma(\rho_I)\oplus\ZZ\sigma^2 (\rho_I)$. Let $\delta =  z_1\rho_I +z_2 \sigma(\rho_I)+z_3 \sigma^2(\rho_I)$. Lemma \ref{lem:length-cubic-3divm} says that\begin{align*}
	\|\delta\|^2 =  3p_I^2 (z_1+z_2+z_3)^2+\dfrac{m}{3}\left((z_1-z_2)^2+(z_2-z_3)^2+(z_3-z_1)^2\right).
	\end{align*} 
	Now, we will find the minimum value of $\|\delta\|^2$ when $\delta\ne 0$. We consider all cases as below.
	\begin{enumerate}[(i)]
		\item If $z_1+z_2+z_3 =0$, then $(z_1-z_2)^2 +(z_2-z_3 )^2 +(z_3-z_1)^2\ge 2$. Note that $(z_1-z_2)^2 +(z_2-z_3 )^2 +(z_3-z_1)^2$ is an even negative integer. If $(z_1-z_2)^2 +(z_2-z_3 )^2 +(z_3-z_1)^2\in \{2,4\}$, then two of the three numbers $z_1,z_2,z_3$ are zero. Without loss of generality, we can assume $z_1=z_2$. It implies that $z_3 =-2z_1$ and thus $(z_1-z_2)^2 +(z_2-z_3 )^2 +(z_3-z_1)^2$ is a multiple of $9$. Hence $(z_1-z_2)^2 +(z_2-z_3 )^2 +(z_3-z_1)^2\ge 6$. Therefore, $\|\delta\|\ge 2m$ in this case. The equality occurs if and only if $\delta \in \{\pm (\alpha -\sigma(\alpha)),\pm (\alpha-\sigma^2(\alpha)), \pm (\sigma(\alpha)-\sigma^2(\alpha))\}$.
		\item If $(z_1-z_2)^2 +(z_2-z_3 )^2 +(z_3-z_1)^2=0$, then $z_1=z_2=z_3=z\in \ZZ$ and thus $\delta =3zp_I$. Hence $\|\delta\|^2 \ge 27p_I^2$. The equality occurs if and only if $\delta \in \{\pm 3p_I\}$.
		\item If $z_1+z_2+z_3\ne 0 $ and $(z_1-z_2)^2 +(z_2-z_3 )^2 +(z_3-z_1)^2\ne 0$, then $(z_1+z_2+z_3)^2 \ge 1 $ and $(z_1-z_2)^2 +(z_2-z_3 )^2 +(z_3-z_1)^2\ge 2$. Thus $\|\delta \|^2\ge 3p_I^2 +\dfrac{2m}{3}$. The equality occurs if and only if $\delta \in \{\pm g_I, \pm\sigma(g_I),\pm \sigma^2(g_I)\}$.
	\end{enumerate}
	It implies that $\min_{\delta \ne 0}\|g\|^2 = \min\set{2m,27p_I^2, 3p_I^2+\dfrac{2m}{3}}.$ Since $\Tr(\rho_I)\ne 0$, the ideal $P_0P_I$ is WR if and only if $\min_{\delta \ne 0} \|\delta\|^2 =  3p_I^2 +\dfrac{2m}{3}$. It is equivalent to the statement $ 3p_I^2+\dfrac{2m}{3}\le 2m$ and $ 3p_I^2\le 27p_I^2$. These inequalities occur if and only if $\dfrac{m}{36}\le p_I^2\le \dfrac{4m}{9}$.
 \end{proof}



 
Using Proposition \ref{prop:P0PI_WR} for $I =\{1\cdots,r\}$, we have the following result.
\begin{corollary}
	\label{cor:P0P1Pr_nowWR} Let $r\ge 1$. Then the ideal $P_0P_1\cdots P_r$ is not WR.
\end{corollary}

Let $I,J$ be two disjoint nonempty subsets of $\{1,2,\cdots ,r\}$. Now, we will claim the necessary and sufficient condition for $P_0P_I^2P_J$ to be a WR lattice (Proposition \ref{prop:P0PIsqPj}). 
\newcommand{\PP}{\mathcal{P}}
Let $\xi_3=\dfrac{-1-\sqrt{-3}}{2}$ be a primitive cube root of $1$ and $K'=  \QQ(\xi_3)$. The minimal polynomial of $\xi_3$ is $x^2+x+1$. For each $i\in \{1,\cdots ,r\}$, the polynomial $x^2+x+1\pmod{ p_i}$ has a root. It means $\mathcal{O}_{K'}$ has an ideal $\mathcal{P}_i$ of $\mathcal{O}_{K'}$ of norm $p_i$. For each subset $I$ of $\{1,\cdots,r\}$, let $\PP_I = \prod_{i\in I }\PP_i$. Then $\PP_I$ is an ideal of $\mathcal{O}_{K'}$ norm $p_I$. Moreover, since $\mathcal{O}_{K'}$ is a PID, then there exist integers $x_I,y_I$ such that $\PP_I=\langle x_I+y_I\xi_3 \rangle$ and thus $p_I=\N (x_I+y_I\xi_3)=x_I^2-x_Iy_I+y_I^2$. In other words, the following result is implied.
\begin{lemma}
	\label{lem:exp_pI_xy} For each nonempty subset $I$ of $\{1,\cdots,r\}$, there exist integers $x_I,y_I$ such that $x_I+y_I+1\equiv 0 \pmod 3$ and $		p_I = x_I^2-x_Iy_I+y_I^2.$
\end{lemma}

%One has the lemma as below.
\begin{lemma}\label{lem:xI_y_I}
	Let $r\ge 2$ and $N=p_1\cdots p_r$ where $p_i$ is a prime such that $p_i\equiv 1 \pmod 3$ for each $i\in \{1,\cdots ,r\}$. Assume that $N= A^2 -AB+B^2$ where $A,B$ are integers that $A+B+1\equiv 0 \pmod 3$. For each nonempty subset $I$ of $\{ 1,\cdots ,r\}$, let $p_I=  \prod_{i\in I} p_i$. Then there exist integers $x_I,y_I$ such that $x_I+y_I+1\equiv 0\pmod 3, p_I =x_I^2-x_Iy_I+y_I^2$ and $p_I\mid \tron{Ax_I-By_I-Ay_I},p_I\mid (Bx_I-Ay_I)$.  
\end{lemma}
\begin{proof}
	See Appendix \ref{proof_of_lemma_24}
\end{proof}

\begin{lemma}\label{lem:multipleofpIsq}
	Let $N=\dfrac{m}{9}=p_1\cdots p_r= A^2-AB+B^2$ where $A$ and $B$ as in Lemma \ref{lem:exprA_B}. With the notation in Lemma \ref{lem:xI_y_I}, one has $p_I^2 \mid \N_{K/\QQ}(x_I\alpha+y_I\sigma(\alpha))$. In particular, $x_I\alpha +y_I\sigma(\alpha)\in P_I^2$.
\end{lemma}
\begin{proof}
	See Appendix \ref{proof_of_lemma_25}
\end{proof}


\begin{proposition}
	\label{prop:P0PIsqPj}Let $r\ge 2$ and $I,J$ be two disjoint nonempty subsets of $\{ 1,2,\cdots, r\}$. The ideal $P_0P_I^2P_J$ is WR if and only if $\dfrac{m}{36}\le p_I^2p_J \le \dfrac{4m}{9}$. In this case, $P_0P_IP_J^2$ has a minimal basis $\{\kappa_{IJ},\sigma(\kappa_{IJ}),\sigma^2(\kappa_{IJ}\}$ where $\kappa_{IJ} = p_{IJ}+x_I+y_I$ and $x_I$ and $y_I$ are given in Lemma \ref{lem:xI_y_I}. 
\end{proposition}
\begin{proof}
	With $x_I, y_I$ in Lemma \ref{lem:multipleofpIsq}, one has $x_I\alpha +y_I\sigma(\alpha) \in P_I^2$. By Corollary \ref{cor:idealP_I}, $x_I\alpha +y_I\sigma(\alpha)\in P_J$. Thus $\kappa_{IJ} \in P_I^2P_J$ as $I,J$ are disjoint. Moreover, $\kappa_{IJ}\in P_0$ as $\kappa_{IJ}= (p_Ip_J-1)+(\alpha-1)x_I+(\sigma(\alpha)-1)y_I+(x_I+y_I+1),P_0 = \langle 3,\alpha-1 \rangle, \sigma(P_0)=P_0$ and $3\mid (x_I+y_I+1)$ by Lemma \ref{lem:xI_y_I}. Hence $\kappa_{IJ}\in P_0P_I^2P_J$ and thus $L_{IJ} = \ZZ \kappa_{IJ}\oplus\ZZ \sigma(\kappa_{IJ})\oplus\ZZ\sigma^2(h_{IJ})$ is a sublattice of $P_0P_I^2P_J$. It is easy to verify that $\det(L_{IJ})=\det(P_0P_I^2P_J)$ and thus $L_{IJ}=  P_0P_I^2P_J$ by Lemm \ref{sublatticeequal}.
	
	Let $\delta = z_1 \kappa_{IJ}+z_2\sigma(\kappa_{IJ})+z_3\sigma^2(\kappa_{IJ})$ be a nonzero vector of $P_0P_I^2P_J$. We can write \begin{align*}
	\delta = p_Ip_J\tron{z_1+z_2+z_3}&+\tron{x_Iz_1-y_Iz_2+\tron{y_I-x_I}z_3}\alpha +\tron{y_Iz_1+\tron{x_I-y_I}z_2-x_Iz_3}\sigma(\alpha)	\end{align*}and hence by Lemma \ref{lem:length-cubic-3divm}	
	\begin{align*}
	\|\delta\|^2 =3p_I^2p_J^2\tron{z_1+z_2+z_3}^2+ \dfrac{2m}{3}\prod_{i\in I}p_i\tron{z_1^2+z_2^2+z_3^2-z_1z_2-z_2z_3-z_1z_3}.
	\end{align*}
	By using a similar argument as the one in the proof of Proposition \ref{prop:P0PI_WR}, one has  \begin{align*}
	\min_{\delta\ne 0}\|\delta\|^2 =\min\set{27p_I^2p_J^2,2mp_I,3p_I^2p_J^2+\dfrac{2m}{3}p_I},
	\end{align*} 
and the lattice $P_0P_I^2P_J$ is WR if and only if $\min_{\delta\ne 0} \|\delta\|^2 = 3p_I^2p_J^2 +\dfrac{2m}{3}p_I$. It is equivalent to the statement $3p_I^2p_J^2 +\dfrac{2m}{3}p_I\le 27p_I^2p_J^2$, $ 3p_I^2p_J^2 +\dfrac{2m}{3}p_I\le 2mp_I$. In other words, $P_0P_I^2P_J$ is WR if and only if  $ \dfrac{m}{36} \le p_Ip_J^2\le \dfrac{4m}{9}$.
\end{proof}




% \begin{corollary}\label{cor:density-cubic}
%   There are at least $25\%$ of cyclic cubic fields with the conductor $m$ that have WR ideals of norms at most $2 \sqrt{m}$.
% \end{corollary}

% \begin{proof}
% By Proposition \ref{prop:cubic9ndividem}  (res. Proposition \ref{prop:P0PI_WR}),  $F$ has a WR ideal if $m$ (res. $m/9$) has a divisor $d$ such that $\dfrac{\sqrt{m}}{2}\le d\le 2 \sqrt{m}$ (res. $\dfrac{\sqrt{m/9}}{2}\le d \le \sqrt{m/9}/2$). Here $m$ (res. $m/9$) is squarefree. Applying inequality (46) from \cite[Proof of Theorem 1.1.]{FHLPSW13}, for $\nu = 2$, one can deduce that the number of cyclic cubic fields of which WR ideals of norms bounded by $2\sqrt{m}$ is at least  $\frac{\nu-1}{2\nu}= 25\%$. 
% \end{proof}







%%%%%%%%%%%%%%%%%%%%%%%%%%%%%%%%%%%%%%%%%%%%%%%%%%%%%%%%%%%%%%%
\section{Well-rounded ideal lattices of cyclic quartic fields}\label{sec:quartic}
In this section, we denote by $F$ a cyclic quartic field defined by $(a,b,c,d)$ as in Section \ref{sec:quarticfields}. We fix these notations where $d= b^2+c^2, \gcd(a,d)=1$ and $a,d$ are squarefree. Let $d= p_1 \cdots p_r$ and $a = sign(a)q_1 \cdots q_s$ be the factorizations of $d$ and $a$ where $sign(a)=1$ if $a>0$ and $sign(a)=-1$ otherwise. Note that all $p_i$ and $q_j$ are distinct since $a$ and $d$ are squarefree and $\gcd(a,d)=1$. For each subset $I$ of $\{1,\cdots,r\}$, let $p_I = \prod_{i\in I}$ and $P_I=\prod_{i\in I}P_I$ where $P_i$ is the unique prime ideal of $\mathcal{O}_F$ above $p_i$ by Lemma \ref{lem:quart_int_bas}, \ref{idealPi0quartic} for all $i \in I$. In the case $I = \emptyset$, we define $p_I = 1$ and  $P_I = \mathcal{O}_K$. If $J$ is a subset of $\{1,\cdots ,s\}$, we denote $q_J =  \prod_{j\in J}q_j$. Let $J $ be any subset of $\{1, \hdots, s\}$ such that for each $j \in J$, there is a unique prime ideal $Q_j$ above $q_j$. In that case, we denote by $Q_J = \prod_{j\in J}Q_j$.



%%%%%%%%%%%%%%%%%%%%%%%%%%%%%%%%%%%%%%%

\subsection{Prime decomposition of $p\mathcal{O}_F$ and integral bases of ideals of $F$}\label{sec:int_basis_ideal}


In this subsection, we provide a number of results concerning the prime factorization of the ideal  $p\mathcal{O}_F$ for an arbitrary prime number $p$. Especially, we aim to classify an odd prime $p$ based on the decomposition of $p\mathcal{O}_F$ (see Theorem \ref{theo:class_p}).  In addition, we construct integral bases for certain ideals of $O_F$ which can be used to prove the well-roundness of ideals in Section \ref{sec:wr-ideals-quartic}.


By \cite[Theorem 1.3]{conraddiscriminants}, a prime $p$ ramifies in $F$ if and only if $p\mid \Delta_F$. In this case, by Lemmata \ref{lem:quart_int_bas} and \ref{lem:quartic_int_basis_divisor_index}, the decomposition of $p\mathcal{O}_F$ is given as below.
\begin{itemize}
	\item If $p\mid d$, then $p\mathcal{O}_F=  P^4$ where $P $ is a unique prime ideal of $\mathcal{O}_F$ above $p$.
	\item If $p\mid a$ and $d $  is a quadratic non-residue $\pmod p$, then  $p\mathcal{O}_F=  P^2$ where $P $ is a unique prime ideal of $\mathcal{O}_F$ above $p$. 
	\item If $p\mid a$ and $d $  is a quadratic residue $\pmod p$, then  $p\mathcal{O}_F=  P_1^2P_2^2$ where $P_1,P_2 $ are two distinct prime ideals of $\mathcal{O}_F$ above $p$. 
\end{itemize}

Lemmata  \ref{lem:quart_int_bas} and \ref{lem:quartic_int_basis_divisor_index} show the prime decomposition of $p\mathcal{O}_F$ where $p\mid ac$. Furthermore, if $p\nmid abcd$, then the composition of $p\mathcal{O}_F$ is given as in \ref{lem:quart_int_bas},\ref{lem:quart_int_bas_iii}. Eventually, to classify all odd primes $p$, we consider an odd prime divisor $p$ of $b$ such that $p\nmid a$. Lemma \ref{lem:prime_decompose_divisor_b}  is the key component that completes the classification of all odd prime numbers $p$. 

By Lemmata \ref{lem:quart_int_bas}, \ref{idealPi0quartic} and \ref{lem:quartic_int_basis_divisor_index}, there is a unique prime ideal $P_i$ above a prime $p_i$ for all $p_i\mid d$ and there exists a unique prime ideal $Q_i$ above $q_i$ for all $q_i\mid a$ if $d$ is not a quadratic residue $\pmod{q_i}$. We will identify the necessary and sufficient conditions for a prime $p$ such that $\mathcal{O}_F$ has a unique prime ideal above $p$ (see Theorem \ref{thm:main6}). 




\begin{remark}\label{rem:belong_PIQJ}
 	Let $\delta \in O_F$. Since $P_i$ is the unique prime ideal above $p_i$, to show that $\delta \in P_I$ it is sufficient to show that $\delta \in P_i$ for all $i \in I$. By Lemma \ref{lem:quartic_int_basis_divisor_index}, $Q_j$ is the unique prime ideal above $q_j$ for all $j\in J$. As consequence, to show $\delta\in Q_J$, it is sufficient to show that $\delta\in Q_j$ for all $j\in J$. Moreover, to claim $\delta\in P_IQ_J$, it is sufficient to show that $\delta\in P_i$ and $\delta\in Q_j$ for all $i\in I, j\in J.$
 \end{remark}






 
 When ($p\mid d$ or $p\mid a$) and $d$ is a quadratic non-residue $\pmod p$, an integral basis of the unique prime ideal above $p$ is obtained as a consequence of Lemmata \ref{lem:d_even}, \ref{lem:d_odd_b_odd}, \ref{lem:d_odd_b_even_aplusb3}, \ref{lem:d_odd_b_even_ab1_1} and \ref{lem:d_odd_b_even_ab1_2}.
 \begin{lemma}
 	\label{lem:d_even} Let $d\equiv 2 \pmod 4$. Then $P_IQ_J = \ZZ p_Iq_J\oplus \ZZ q_J\sqrt{d}\oplus \ZZ \beta \oplus \ZZ \sigma(\beta)$.
 \end{lemma}
 \begin{proof}
 	For each $i\in I$ and $j\in J$, since $P_i$ is the unique prime ideal above $p_i$ and $Q_j$ is the unique prime ideal above $q_j$ and by Lemma \ref{lem:norm_some_ele}, $\beta,\sigma(\beta)\in P_i$ and $\beta,\sigma(\beta)\in Q_j$. By Remark \ref{rem:belong_PIQJ}, one has $\beta,\sigma(\beta)\in P_IQ_J$. It is obvious to see that $p_Iq_J\in P_IQ_J$ and $q_J\sqrt{d}\in Q_J$. Since $p_I\mid d^2 =\N(\sqrt{d})$ and $P_i$ is the unique prime ideal above $p_i$, $\sqrt{d}\in P_i$ for all $i\in I$ and thus $q_J\sqrt{d}\in P_I$ by Remark \ref{rem:belong_PIQJ}. It means $q_J\sqrt{d}\in P_IQ_J$. It implies that $L_{IJ} =  \ZZ p_Iq_J\oplus \ZZ q_J\sqrt{d}\oplus \ZZ \beta \oplus \ZZ \sigma(\beta)$ is a sublattice of $P_IQ_J$. However, two lattices $P_IQ_J$ and $L_{IJ}$ have the same indices in $\mathcal{O}_F$. Therefore $P_IQ_J=L_{IJ}$ by Lemma \ref{sublatticeequal}.
 \end{proof}








 
\begin{lemma}\label{lem:d_odd_b_odd}
	If $d\equiv 1\pmod 4$ and $b$ is odd. Then $P_IQ_J = \ZZ p_Iq_J \oplus \ZZ \dfrac{q_I\tron{p_I+\sqrt{d}}}{2}\oplus \ZZ \beta \oplus \ZZ\sigma(\beta)$.
\end{lemma}
\begin{proof}
	Be Remark \ref{rem:integralbasis}, $\dfrac{1+\sqrt{d}}{2}\in \mathcal{O}_F$ and thus $\dfrac{p_I+\sqrt{d}}{2}=\dfrac{p_I-1}{2}+\dfrac{1+\sqrt{d}}{2}\in \mathcal{O}_F$. Since $p_i\mid \tron{\dfrac{p_I^2-d}{4}}^2=\N\tron{\dfrac{p_I+\sqrt{d}}{2}}$ and $P_i$ the unique prime ideal above $p_i$ for all $i\in I$. Hence $\dfrac{q_J\tron{p_I+\sqrt{d}}}{2}\in P_IQ_J$. By Lemma \ref{lem:norm_some_ele}, $\beta,\sigma(\beta)\in P_i, Q_j$ for all $i\in I$ and $j\in J$. By Remark \ref{rem:belong_PIQJ}, we obtain $\beta,\sigma(\beta)\in P_IQ_J$. One can prove the result using a similar argument as in the proof of Lemma \ref{lem:d_even}.
\end{proof}






\begin{lemma}
	\label{lem:d_odd_b_even_aplusb3}Let $d\equiv 1 \pmod 4$, $b$ be even and $a+b\equiv 3\pmod 4$. Then $P_IQ_J = \ZZ p_Iq_J\oplus \ZZ \dfrac{q_J\tron{p_I+\sqrt{d}}}{2}\oplus \ZZ \dfrac{\beta+\sigma(\beta)}{2}\oplus \ZZ\dfrac{-\beta+\sigma(\beta)}{2}.$
\end{lemma}
Next, we consider the case $d\equiv 1 \pmod 4,b \equiv 0\pmod 2, a+b\equiv 1 \pmod 4$ and $a\equiv-c\pmod 4$. Let $\gamma'_1, \gamma'_2, \gamma'_3,\gamma'_4$ be a integral basis of $\mathcal{O}_F$ as in Remark \ref{rem:integralbasis},iv. We define \begin{align}\label{eq:basis_aplusb1_1}
\gamma_1= \gamma_1', \gamma_2= \gamma_2', \gamma_3 = -\gamma_4', \gamma_4 = \gamma_2'-\gamma_3'.
\end{align} It is obvious to see that $\{\gamma_1,\gamma_2,\gamma_3,\gamma_4\}$ is a basis of $\mathcal{O}_F$ by \ref{rem:integralbasis}, iv. One has the following result.
\begin{lemma}
	\label{lem:d_odd_b_even_ab1_1}Let $d\equiv 1 \pmod 4, b$ be even, $a+b\equiv 1\pmod 4$ and $a\equiv -c\pmod 4$. Then $P_IQ_J = \ZZ \rho_{IJ}\oplus \ZZ \sigma\tron{\rho_{IJ}}\oplus \ZZ\sigma^2\tron{\rho_{IJ}}\ZZ\sigma^3\tron{\rho_{IJ}} $ where $\rho_{IJ}= \dfrac{-p_Iq_J+q_J\sqrt{d}-\beta-\sigma(\beta)}{4}.$
\end{lemma}

\begin{proof}
	By Remark \ref{rem:belong_PIQJ}, it is sufficient to prove $\rho_{IJ}\in P_i,Q_J$ for all $i\in I$ and $j\in J$. Let $\gamma_1,\gamma_2,\gamma_3,\gamma_4$ as in \eqref{eq:basis_aplusb1_1}. Then $\rho_{IJ}=  \dfrac{-p_Iq_J-q_J+2}{4}\gamma_1+\dfrac{q_J-1}{2}\gamma_2+\gamma_4$ and thus $\rho_{IJ}\in \mathcal{O}_F$. Moreover, $\N\tron{\rho_{IJ}} = \dfrac{\tron{p_Iq_J^2+q_I^2d-2ad}^2-2d\tron{p_Iq_J^2+|a|c}^2}{256}$ an thus $\rho_{IJ}\in P_i,Q_j$ for all $i\in I$ and $j\in J$. As a result $\rho_{IJ},\sigma\tron{\rho_{IJ}},\sigma\tron{\rho_{IJ}},\sigma^3\tron{\rho_{IJ}}\in P_IQ_J$. Hence $L_{IJ}= \ZZ \rho_{IJ}\oplus \ZZ \sigma\tron{\rho_{IJ}}\oplus \ZZ\sigma^2\tron{\rho_{IJ}}\ZZ\sigma^3\tron{\rho_{IJ}}$ is a sublattice of $P_IQ_J$. Two lattices $P_IQ_J$ and $L_{IJ}$ have the same indices $p_Iq_J^2$ in $\mathcal{O}_F$. Therefore $P_IQ_J=L_{IJ}$.
\end{proof}
In the remaining case $d \equiv 1 \pmod 4,b\equiv 0\pmod 2, a+b\equiv 1 \pmod 4$ and $a\equiv c\pmod 4$, using the similar technique with the proof of Lemma \ref{lem:d_odd_b_even_ab1_1}, one obtains a result as below.
\begin{lemma}
	\label{lem:d_odd_b_even_ab1_2}Let $d \equiv 1 \pmod 4,b\equiv 0\pmod 2, a+b\equiv 1 \pmod 4$ and $a\equiv c\pmod 4$. Then $P_IQ_J = \ZZ \rho_{IJ}\oplus \ZZ \sigma\tron{\rho_{IJ}}\oplus \ZZ\sigma^2\tron{\rho_{IJ}}\ZZ\sigma^3\tron{\rho_{IJ}} $ where $\rho_{IJ} =  \dfrac{p_Iq_J-q_J\sqrt{d}-\beta+\sigma(\beta)}{4}.$
\end{lemma}

Next, we will describe a prime ideal above $q_i$ where $q_i\mid a$ and $d$ are a quadratic residue $\pmod q_i$. By Lemma \ref{lem:quartic_int_basis_divisor_index}, there exist exactly two prime ideals above $q_i$. Let $z_1,z_2$ be two positive integers such that $z_i^2\equiv d\pmod {q_j}$. By the result on the decomposition of primes \cite[Theorem 4.8.13]{cohen1993course}, one has $q_j\mathcal{O}_K =  \mathfrak{q}_{1j}\mathfrak{q}_{2j}.$ 


Before proceeding, we will outline a strategy to prove that a certain lattice is ideal in Lemmata \ref{lem:q_i_d_even}, \ref{lem:q_i_db_odd}, \ref{lem:q_i_d_odd_b_even_ab3}, \ref{lem:qnotquadratic_ab_1_mod_4_1},\ref{lem:qnotquadratic_ab_1_mod_4_2},\ref{lem:prime_decompose_divisor_b}, \ref{lem:ideal_above_2_bodd} and \ref{lem:prideals_above_2_2}. The proofs of these lemmata can be seen in the Appendix \ref{appendix_B}.

\begin{remark}\label{stra:strategy_to_prove_ideal}
Let $\mathcal{O}_F = \ZZ \gamma_1'\oplus \ZZ \gamma_2'\oplus\ZZ \gamma_3'\oplus \ZZ\gamma_4'$ as in Remark \ref{rem:integralbasis} and let $L = \ZZ \delta_1\oplus \ZZ \delta_2 \oplus  \ZZ \delta_3\oplus \ZZ\delta_4$ where $\delta_i \in \mathcal{O}_F$. To prove $L$ is an ideal of $\mathcal{O}_F$, we will show that $\delta_i\gamma_j' \in L$ for all $i, j$. In other words,  we perform the following steps for all $ 1 \le i, j  \le 4$.
	\begin{enumerate}[(1)]
		\item Compute $\delta_i\gamma_j'$.
		\item Express $\delta_i\gamma_j' = z_1\delta_1'+z_2\delta_2' +z_3 \delta_3'+z_4\delta_4'$.
		\item Prove that all numbers  $z_1,z_2,z_3,z_4$ are integers.
	\end{enumerate}
\end{remark}




When $d$ is even, $df_K(x) = x^2-d$ is a defining polynomial of $K$.  Then \begin{align}\label{eq:z_1_z_2}
    df_K(x) = \tron{x-z_i}\tron{x-z_2}\pmod {p_j}.
\end{align} By using the result on the  decomposition of primes  \cite[Theorem 4.8.13]{cohen1993course}, one has $\mathfrak{q}_{kj}=  \ZZ q_j\oplus \ZZ\tron{z_i+\sqrt{d}}$. With  $z_1,z_2$ as in \eqref{eq:z_1_z_2}, one has the result as follows. 

\begin{lemma}
	\label{lem:q_i_d_even} If $d$ is even and $q_i\mid a$ such that $d$ is a quadratic residue $\pmod {q_j}$, then there exist exactly two prime ideals $Q_{1j},Q_{2j}$ above $q_j$ where \begin{align*}
	Q_{kj}&= \ZZ q_j\oplus \ZZ \tron{z_k+\sqrt{d}}\oplus \ZZ \beta\oplus \ZZ \sigma(\beta).
	\end{align*}
\end{lemma}

When $d$ is odd, $df_K(x)=  x^2-x+\dfrac{1-d}{4}$ is a defining polynomial of $K$. One has \begin{align*}
4df_K(x)\pmod {q_j}=  (2x-1)^2-d \pmod {q_j}= \tron{2x-1-z_1}\tron{2x-1-z_2}\pmod {q_j}.
\end{align*} As $q\equiv 1 \pmod 4$, there exist integers $t_1,t_2$ such that $z_k = 4t_k-1\pmod{ q_j}$ for $k=1,2,$  and thus $df_K(x) = \tron{x-t_1}\tron{x-t_2}\pmod {q_j}$.


\begin{lemma}\label{lem:q_i_db_odd} 
If $d\equiv 1\pmod 4, b\equiv 1\pmod 2$ and $q_i\mid a$ such that $d$ is a quadratic residue $\pmod {q_j}$, then there exist exactly two prime ideals $Q_{1i},Q_{2i}$ above $q_i$ where \begin{align*}
	Q_{kj}&= \ZZ q_j\oplus \ZZ \tron{\dfrac{4t_k-1+\sqrt{d}}{2}}\oplus \ZZ \beta\oplus \ZZ \sigma(\beta)
	\end{align*} for $k=1,2$.
\end{lemma}


\begin{lemma}
	\label{lem:q_i_d_odd_b_even_ab3} If $d\equiv 1\pmod 4, b\equiv 0\pmod 2, a+b\equiv 3\pmod 4$ and $q_i\mid a$ such that $d$ is a quadratic residue $\pmod {q_j}$, then there exist exactly two prime ideals $Q_{1i},Q_{2i}$ above $q_i$ where 
	\begin{align*}
Q_{kj}&= \ZZ q_j\oplus \ZZ \tron{\dfrac{4t_k-1+\sqrt{d}}{2}}\oplus \ZZ \dfrac{\beta+\sigma(\beta)}{2}\oplus \ZZ \dfrac{\beta-\sigma(\beta)}{2}
\end{align*} 
for $k=1,2.$
\end{lemma}

\begin{lemma}\label{lem:qnotquadratic_ab_1_mod_4_1}
	If $d\equiv 1\pmod 4, b\equiv   0 \pmod 2, a +b \equiv 1 \pmod 4$ and $a\equiv -c\pmod 4$ and $p_j\mid a$ such that $d$ is a quadratic residue $\pmod {q_j}$, then there are exactly two prime ideals $Q_{1j}, Q_{2j}$ above $q_j$ such that 
	\begin{align*}
	Q_{kj}= \ZZ q_j\oplus \ZZ \dfrac{4t_k-1+\sqrt{d}}{2}\oplus \ZZ \dfrac{4t_k-1 +\sqrt{d}-\beta- \sigma(\beta)}{4}\oplus \ZZ \dfrac{2q_j+4t_k-1+\sqrt{d} +\beta-\sigma(\beta)}{4}.
	\end{align*}
\end{lemma}

\begin{lemma}\label{lem:qnotquadratic_ab_1_mod_4_2}
	If $d\equiv 1\pmod 4, b\equiv   0 \pmod 2, a +b \equiv 1 \pmod 4$ and $a\equiv c\pmod 4$ and $p_j\mid a$ such that $d$ is a quadratic residue $\pmod {q_j}$, then there exist integers $t_1,t_2$ and exactly two prime ideals $Q_{1j}, Q_{2j}$ above $q_j$ such that $q_j\nmid t_1-t_2$, $d = (4t_i-1)^2\pmod {q_j}$ and 
	\begin{align*}
	Q_{ij}= \ZZ q_j\oplus \ZZ \dfrac{4t_i-1+\sqrt{d}}{2}\oplus \ZZ \dfrac{4t_i-1+2q_j +\sqrt{d}-\beta- \sigma(\beta)}{4}\oplus \ZZ \dfrac{4t_i-1+\sqrt{d} +\beta-\sigma(\beta)}{4}.
	\end{align*}
	
\end{lemma}

Now, consider a prime $p$ such that $p\mid b$ and $p\nmid a$, Lemma \ref{lem:quartic_int_basis_divisor_b} does not provide us the exact prime decomposition of $p\mathcal{O}_F$. It is sufficient to show that $\mathcal{O}_F$ has either a prime ideal of norm $p^2$ or a prime ideal of $p.$



\begin{lemma}	\label{lem:prime_decompose_divisor_b}
Let $p\mid b$ and $p\nmid a$. One has the following statement.
	\begin{enumerate}[i)]
		\item Assume $2 \mid d$. If $a$ is a quadratic non-residue $\pmod p$ then $p\mathcal{O}_F= P_1P_2$ where 
		\begin{align*}
		P_1 &= \ZZ  p \oplus \ZZ \tron{c+\sqrt{d}}\oplus \ZZ p\sigma(\beta)\oplus \ZZ\tron{\beta+\sigma(\beta)}, \text{ and} \\
		P_2&= \ZZ  p \oplus \ZZ \tron{-c+\sqrt{d}}\oplus \ZZ p\sigma(\beta)\oplus \ZZ\tron{\beta-\sigma(\beta)}
		\end{align*}
  are all prime ideals of $\mathcal{O}_F$ above $p$. If $a$ is a quadratic residue $\pmod p$ and we write $a = l^2 \pmod p$, then $p\mathcal{O}_F =P_1P_2P_3P_4$ where 
  \begin{align*}
		P_1& =  \ZZ p \oplus \ZZ \tron{c+\sqrt{d}}\oplus \ZZ  \tron{lc-\sigma(\beta)}\oplus \ZZ \tron{lc+\beta}, \\
				P_2& =  \ZZ p \oplus \ZZ \tron{c+\sqrt{d}}\oplus \ZZ  \tron{lc+\sigma(\beta)}\oplus \ZZ \tron{-lc+\beta}, \\
				P_3& = \ZZ p \oplus\ZZ \tron{-c+\sqrt{d}}\oplus \ZZ \tron{lc-\sigma(\beta)}\oplus \ZZ \tron{lc-\beta}, \text{ and}\\
    P_4 & =\ZZ p \oplus\ZZ \tron{-c+\sqrt{d}}\oplus \ZZ \tron{lc+\sigma(\beta)}\oplus \ZZ \tron{lc+\beta} 
		\end{align*} 
		are all prime ideals  of $\mathcal{O}_F$ above $p$.
  
		\item  Assume $d\equiv 1\pmod 4$ and $b\equiv 1 \pmod 2$. If $ a$ is a quadratic non-residue $\pmod p$, then $p\mathcal{O}_F= P_1P_2$ where 
  \begin{align*}
		P_1 = \ZZ p \oplus \ZZ \dfrac{p+c+\sqrt{d}}{2} \oplus \ZZ p\sigma\tron{\beta}\oplus\ZZ \tron{\beta+\sigma(\beta)}, \text{ and}\\
		P_2 = \ZZ p \oplus \ZZ \dfrac{p-c+\sqrt{d}}{2} \oplus \ZZ p\sigma\tron{\beta}\oplus\ZZ \tron{\beta-\sigma(\beta)} 
\end{align*} 
are all prime ideals of $\mathcal{O}_F$ above $p$.  If $a$ is a quadratic residue $\pmod p$ and we write $a = l^2 \pmod p$, then $p\mathcal{O}_F =P_1P_2P_3P_4$ where
\begin{align*}
		P_1& =  \ZZ p \oplus \ZZ \dfrac{p-c+\sqrt{d}}{2}\oplus \ZZ  \tron{lc-\sigma(\beta)}\oplus \ZZ \tron{lc+\beta}, \\
		P_2& =  \ZZ p \oplus \ZZ \dfrac{p-c+\sqrt{d}}{2}\oplus \ZZ  \tron{lc+\sigma(\beta)}\oplus \ZZ \tron{-lc+\beta}, \\
		P_3& = \ZZ p \oplus\ZZ \dfrac{p+c+\sqrt{d}}{2}\oplus \ZZ \tron{lc-\sigma(\beta)}\oplus \ZZ \tron{lc-\beta}, \text{ and}\\
  P_4 & =\ZZ p \oplus\ZZ \dfrac{p+c+\sqrt{d}}{2}\oplus \ZZ \tron{lc+\sigma(\beta)}\oplus \ZZ \tron{lc+\beta} 
		\end{align*}
are all prime ideals of $\mathcal{O}_F$ above $p$.
				
    \item Assume $d \equiv 1 \pmod 4, b\equiv 0\pmod 2$ and $a+b\equiv 3\pmod 4$. If $a$ is a quadratic non-residue $\pmod p$, then $p\mathcal{O}_F = P_1P_2$ where 
    \begin{align*}
	P_1 &=\ZZ p\oplus \ZZ\dfrac{-c+\sqrt{d}}{2}\oplus \ZZ \dfrac{\sigma(\beta)-\beta}{2}\oplus \ZZ p\dfrac{\beta+\sigma(\beta)}{2}, \text{ and}\\
	P_2 &=\ZZ p\oplus \ZZ\dfrac{c+\sqrt{d}}{2}\oplus \ZZ p\dfrac{\sigma(\beta)-\beta}{2}\oplus \ZZ \dfrac{\beta+\sigma(\beta)}{2}
\end{align*} 
are all prime ideals above $p$. If $a$ is a quadratic residue $\pmod p$ and we write $a = l^2 \pmod p$, then $p\mathcal{O}_F =P_1P_2P_3P_4$ where 
\begin{align*}
P_1 & = \ZZ p\oplus\ZZ \dfrac{-c+\sqrt{d}}{2}\oplus \ZZ \dfrac{\sigma(\beta)-\beta}{2}\oplus \ZZ\tron{lc-\dfrac{\beta+\sigma(\beta)}{2}}, \\
P_2 & = \ZZ p\oplus\ZZ \dfrac{-c+\sqrt{d}}{2}\oplus \ZZ \dfrac{\sigma(\beta)-\beta}{2}\oplus \ZZ\tron{lc+\dfrac{\beta+\sigma(\beta)}{2}}, \\
P_3 & = \ZZ p\oplus\ZZ \dfrac{c+\sqrt{d}}{2}\oplus \ZZ \tron{lc+\dfrac{\sigma(\beta)-\beta}{2}}\oplus \ZZ\dfrac{\beta+\sigma(\beta)}{2}, \text{ and}\\
P_4 & = \ZZ p\oplus\ZZ \dfrac{c+\sqrt{d}}{2}\oplus \ZZ \tron{lc-\dfrac{\sigma(\beta)-\beta}{2}}\oplus \ZZ\dfrac{\beta+\sigma(\beta)}{2}				
\end{align*}
are all primes ideals of $\mathcal{O}_F$ above $p$.
				
\item Assume $d\equiv 1 \pmod 4, b \equiv 2\pmod 4, a+b\equiv 1\pmod 4$ and $a\equiv -c \pmod 4$. If $a$ is a quadratic non-residue $\pmod p$, then $p\mathcal{O}_F=P_1P_2$ where 
\begin{align*}
P_1 &=  \ZZ p\oplus \ZZ \dfrac{-c+\sqrt{d}}{2}\oplus \ZZ \dfrac{b-c+\sqrt{d}-\beta+\sigma(\beta)}{4}\oplus \ZZ\dfrac{-p+p\sqrt{d}+p\beta+p\sigma(\beta)}{4}, \text{ and}\\
	P_2 &=  \ZZ p\oplus \ZZ \dfrac{c+\sqrt{d}}{2}\oplus \ZZ \dfrac{p+p\sqrt{d}-p\beta+p\sigma(\beta)}{4}\oplus \ZZ\dfrac{b-c-\sqrt{d}-\beta-\sigma(\beta)}{4}
	\end{align*} 
 are all prime ideals of $\mathcal{O}_F$  above $p$.  If $a$ is a quadratic residue $\pmod p$ and we write $a = l^2\pmod p$, then $p\mathcal{O}_F =  P_1P_2P_3P_4$ where 
 \begin{align*}
P_1 & =  \ZZ p \oplus \ZZ \dfrac{c+\sqrt{d}}{2}\oplus\ZZ \dfrac{\tron{-2l+1}c+\sqrt{d}-\beta+\sigma(\beta)}{4}\oplus \ZZ\dfrac{b-c-\sqrt{d}-\beta-\sigma(\beta)}{4}, \\
	P_2 & = \ZZ p \oplus \ZZ \dfrac{-c+\sqrt{d}}{2}\oplus\ZZ\dfrac{b-c+\sqrt{d}-\beta+\sigma(\beta)}{4}\oplus \ZZ\dfrac{\tron{2l+1}c-\sqrt{d}-\beta-\sigma(\beta)}{4}, \\
	P_3 & = \ZZ p \oplus \ZZ \dfrac{c+\sqrt{d}}{2}\oplus\ZZ\dfrac{\tron{2l+1}c+\sqrt{d}-\beta+\sigma(\beta)}{4}\oplus \ZZ\dfrac{b-c-\sqrt{d}-\beta-\sigma(\beta)}{4}, \text{ and}\\
P_4 & = \ZZ p \oplus \ZZ \dfrac{-c+\sqrt{d}}{2}\oplus\ZZ\dfrac{b-c+\sqrt{d}-\beta+\sigma(\beta)}{4}\oplus \ZZ\dfrac{\tron{-2l+1}c-\sqrt{d}-\beta-\sigma(\beta)}{4}
\end{align*}
 are all prime ideals of $\mathcal{O}_F$ above $p$.
		
  \item Assume $d\equiv 1 \pmod 4, b \equiv 2\pmod 4, a+b\equiv 1\pmod 4$ and $a\equiv c \pmod 4$. If $a$ is a quadratic non-residue $\pmod p$, then $p\mathcal{O}_F=P_1P_2$ where 
  \begin{align*}
	P_1 &=  \ZZ p\oplus \ZZ \dfrac{-c+\sqrt{d}}{2}\oplus \ZZ \dfrac{b-c+\sqrt{d}-\beta+\sigma(\beta)}{4}\oplus \ZZ\dfrac{p+p\sqrt{d}+p\beta+p\sigma(\beta)}{4}, \text{ and}\\
	P_2 &=  \ZZ p\oplus \ZZ \dfrac{c+\sqrt{d}}{2}\oplus \ZZ \dfrac{-p+p\sqrt{d}-p\beta+p\sigma(\beta)}{4}\oplus \ZZ\dfrac{b-c-\sqrt{d}-\beta-\sigma(\beta)}{4}
\end{align*} 
are all prime ideals of $\mathcal{O}_F$  above $p$.  If $a$ is a quadratic residue $\pmod p$ and we write $a = l^2\pmod p$, then $p\mathcal{O}_F =  P_1P_2P_3P_4$ where 
\begin{align*}
P_1 & =  \ZZ p \oplus \ZZ \dfrac{c+\sqrt{d}}{2}\oplus\ZZ \dfrac{\tron{2l+1}c+\sqrt{d}-\beta+\sigma(\beta)}{4}\oplus \ZZ\dfrac{b-c-\sqrt{d}-\beta-\sigma(\beta)}{4}, \\
P_2 & = \ZZ p \oplus \ZZ \dfrac{-c+\sqrt{d}}{2}\oplus\ZZ\dfrac{b-c+\sqrt{d}-\beta+\sigma(\beta)}{4}\oplus \ZZ\dfrac{\tron{2l+1}c-\sqrt{d}-\beta-\sigma(\beta)}{4}, \\
P_3 & = \ZZ p \oplus \ZZ \dfrac{c+\sqrt{d}}{2}\oplus\ZZ\dfrac{\tron{-2l+1}c+\sqrt{d}-\beta+\sigma(\beta)}{4}\oplus \ZZ\dfrac{b-c-\sqrt{d}-\beta-\sigma(\beta)}{4}, \text{ and}\\
	P_4 & = \ZZ p \oplus \ZZ \dfrac{-c+\sqrt{d}}{2}\oplus\ZZ\dfrac{b-c+\sqrt{d}-\beta+\sigma(\beta)}{4}\oplus \ZZ\dfrac{\tron{-2l+1}c-\sqrt{d}-\beta-\sigma(\beta)}{4}
\end{align*}
are all prime ideals of $\mathcal{O}_F$ above $p$.
\end{enumerate}
\end{lemma} 

 \begin{proof}
 	The given lattices are completely distinct, and we can prove that they are ideals by following the steps in Remark \ref{stra:strategy_to_prove_ideal}.
 \end{proof}
Finally, we will consider prime ideals above $2$ when $\Delta_F$ is even. The following result is obtain from Lemma \ref{lem:quart_int_bas},i.

\begin{lemma}\label{lem:ideal_above_2_dodd}
	Let $d$ be even. Then there exists a unique prime ideal $P_0$ above $p_0=2$. Moreover, $P_0 = \langle 2, \beta \rangle$ and $\N\tron{P_0}=2$.
\end{lemma}

\begin{lemma}\label{lem:ideal_above_2_bodd}
	Let $d\equiv 1 \pmod 4$ and $b$ be odd.
	\begin{enumerate}[(i)]
		\item If $d\equiv 5\pmod 8$, then there is a unique prime ideal $P_0$ above $p_0=2$ where $\N(P_0)=4$ and 
		\begin{align*}
		P_0 = \ZZ 2\oplus \ZZ(1+\sqrt{d})\oplus \ZZ\beta \oplus \ZZ \sigma(\beta).
		\end{align*}
  
		\item If $d\equiv 1 \pmod 8$, then there are exactly two distinct prime ideals $P_{01}, P_{02}$ above $p_0=2$ where $\N(P_{01})=  \N\tron{P_{02}}=2$ and \begin{align*}
		P_{01}= \ZZ 2\oplus \ZZ\tron{\dfrac{-1+\sqrt{d}}{2}}\oplus \ZZ \beta \oplus \ZZ\sigma(\beta), \text{ and }\\  
		P_{02}= \ZZ 2\oplus \ZZ\tron{\dfrac{1+\sqrt{d}}{2}}\oplus \ZZ \beta \oplus \ZZ\sigma(\beta).
	\end{align*}
	\end{enumerate}
\end{lemma} 

\begin{lemma}\label{lem:prideals_above_2_2}
	Let $d\equiv 1 \pmod 4$ and $b\equiv 0 \pmod 2$ and $a+b\equiv 3 \pmod 4$.\begin{enumerate}[(i)]
		\item If $d\equiv 5\pmod 8$ , then there is a unique prime ideal $P_0$ above $p_0=2$ where $\N(P_0)=4$ and 
		\begin{align*}
		P_0 = \ZZ 2\oplus \ZZ(1+\sqrt{d})\oplus \ZZ\dfrac{-1+\sqrt{d}-\beta-\sigma(\beta)}{2} \oplus \ZZ \dfrac{1+\sqrt{d}+\beta-\sigma(\beta)}{2}.
		\end{align*}
		\item If $d\equiv 1 \pmod 8$, then there are exactly two prime ideals $P_{01}, P_{02}$ above $p_0=2$ where $\N(P_{01})=  \N\tron{P_{02}}=2$ and 
		\begin{align*}
		P_{01}= \ZZ 2\oplus \ZZ\tron{\dfrac{-1+\sqrt{d}}{2}}\oplus \ZZ \dfrac{2-\beta-\sigma(\beta)}{2} \oplus \ZZ\dfrac{\beta-\sigma(\beta)}{2}, \text{ and }\\  
		P_{02}= \ZZ 2\oplus \ZZ\tron{\dfrac{1+\sqrt{d}}{2}}\oplus \ZZ \dfrac{\beta +\sigma(\beta)}{2} \oplus \ZZ\dfrac{2+\beta -\sigma(\beta)}{2}.
		\end{align*}
	\end{enumerate}
\end{lemma} 
For the case of $p=2$ and $\Delta_F$ odd, we have the following result.

\begin{lemma}	\label{lem:p2_delta_odd}
 Assume that $d\equiv 1 \pmod 4$ and $a+b\equiv 1 \pmod 4$. 
\begin{enumerate}[i)]
	\item If $d \equiv 1\pmod 8$, then $2\mathcal{O}_F$ can be factored as one of the forms $ P_1P_2,P_1P_2P_3P_4$ where $P_1,P_2,P_3,P_4$ are  prime ideals of $\mathcal{O}_F$ above $2$.
	\item If $d\equiv 5 \pmod 8$, then $2\mathcal{O}_F$ is prime.  
\end{enumerate}
\end{lemma}
\begin{proof}
	\begin{enumerate}[i)]
		\item It implies directly from the fact that $p\mathcal{O}_K$ splits totally in $\mathcal{O}_K$ where $\mathcal{O}_K$ as in \eqref{eq:exten_tower}.
		\item See Appendix \ref{proof_lem_p2}.
	\end{enumerate}
\end{proof}
The below theorem follows directly from the combination of Lemmata \ref{lem:quart_int_bas}, \ref{lem:quartic_int_basis_divisor_index},\ref{lem:quartic_int_basis_divisor_b} and \ref{lem:prime_decompose_divisor_b}. 




\begin{theorem}\label{theo:class_p}
	Let $F$ be a cyclic quartic field and $a,b,c,d$ as in \eqref{df-polynomial-cubic} and $p$ be an odd prime. One has the following statements.
	\begin{enumerate}[i)]
		\item The prime $p$ is totally ramified if and only if $p\mid d$.
		\item (The ideal $p\mathcal{O}_F = P^2$ where $P$ is a unique prime ideal of $\mathcal{O}_F$ above $p$) if and only if $p\mid a$ and $d$ is a quadratic non-residue $\pmod p$.
         \item (The ideal $p\mathcal{O}_F = P_1^2P_2^2$ where $P_1,P_2$ are exactly two prime ideals of $\mathcal{O}_F$ above $p$) if and only if $p\mid a$ and $d$ is a quadratic residue $\pmod p$.
		\item The prime $p$ is inert over $F$ if and only if $p\nmid abcd$ and $d$ is a quadratic non-residue $\pmod p$.
		
		\item The prime $p$ totally splits if and only if $p$ satisfies one of the conditions listed below.
		\begin{itemize}
			\item The prime $p\mid b$ and $a$ is a quadratic residue $\pmod p$.
			\item The prime $p\mid c$ and $2a$ is a quadratic residue $\pmod p$.
			\item The prime $p\nmid abcd$, $d$ is a quadratic residue $\pmod p$, and if $d= z^2 \pmod p$ then $ad+abz$ and $ad-abz$ are also quadratic residues $\pmod p$.	\end{itemize}
		
		\item The ideal $p\mathcal{O}_F$ is the product of two distinct prime ideals in all the remaining cases.
	\end{enumerate}
\end{theorem}





From Theorem \ref{theo:class_p} and Lemmata \ref{lem:ideal_above_2_dodd}, \ref{lem:ideal_above_2_bodd}, and \ref{lem:prideals_above_2_2}, we obtain the necessary and sufficient conditions for a prime $p$ for which  $\mathcal{O}_F$ has a unique prime ideal $P$ over $p$. In the next subsection, we will investigate the conditions for the unique prime ideals $P$ mentioned above to be WR.




%%%%%%%%%%%%%%%%%%%%%%%%%%%%%%%%%%%%%%%%%%%%%%%%%%%%%%%%%%%%%%
\vspace*{0.5cm}
\subsection{Well-rounded ideals of cyclic quartic fields}\label{sec:wr-ideals-quartic}


According to the first part of Theorem \ref{thm:main6}, there are 3 cases in which   $\mathcal{O}_F$ has a unique prime ideal $P$ over a prime number $p$. However, in the last case of the theorem, $P=p\mathcal{O}_F$ and it is not primitive. Therefore, we only investigate prime ideals $P$ belonging to the first two cases of the theorem. In general, we will prove the necessary and sufficient conditions for an ideal of the form $P_IQ_J$ to be WR, where $I$ is a subset of $\{1,\cdots,r\}$ and $J$ is a subset of $\{1,\cdots, s\}$ such that $d$ is a non-quadratic residue modulo $q_j$ for all $j\in J$.
\begin{proposition}
	\label{prop:PIQ_J_d_even_notWR} Let $d\equiv 2 \pmod 4$. Then $P_IQ_J$ is not WR.
\end{proposition}
\begin{proof}
	Let $\delta\in P_IQ_J$ be a nonzero vector of $P_IQ_J$. By Lemma \ref{lem:d_even}, there exist integers $x_1,x_2,x_3,x_4$ such that $\delta  = x_1 p_Iq_J+x_2q_J\sqrt{d}+x_3 \beta+x_4\sigma(\beta)$ and by \eqref{length}, one obtains
	\begin{align*}
		\|\delta \|^2 = 4\tron{x_1^2p_I^2q_J^2 +x_2^2q_J^2d+|a|d\tron{x_3^2+x_4^2}}.
	\end{align*}
	It is easy to verify that $\min_{\delta\ne 0}\|\delta\|^2\in \min \mathcal{S}$ where $\mathcal{S} =\set{4p_I^2q_J^2, 4q_J^2d, 4|a|d}.$ Each value in $\mathcal{S}$ is only correspondent to the squared length of at most two independent vectors. Thus $P_IQ_J$ is not WR.
\end{proof}



\begin{proposition}
	\label{prop:bd_odd_notWR} Let $d\equiv 1\pmod 4$ and $b$ be odd. Then $P_IQ_J$ is not WR.
\end{proposition}
\begin{proof}
	Let $\delta\in P_IQ_J$ be a nonzero vector of $P_IQ_J$. By Lemma \ref{lem:d_even}, there exist integers $x_1,x_2,x_3,x_4$ such that $\delta  = x_1 p_Iq_J+x_2q_J\dfrac{p_I+\sqrt{d}}{2}+x_3 \beta+x_4\sigma(\beta)$ and by \eqref{length}, one obtains
	\begin{align*}
		\|\delta \|^2 = \tron{2x_1+x_2}^2p_I^2q_J^2 +x_2^2q_J^2d+4|a|d\tron{x_3^2+x_4^2}.
	\end{align*}
	Since $2x_1+x_2$ and $x_2$ have the same parity, it is easy to verify that $\min_{\delta\ne 0}\|\delta\|^2\in \min \mathcal{S}$ where $\mathcal{S} =\set{p_I^2q_J^2+q_J^2d, 4|a|d}.$ Each value in $\mathcal{S}$ is only correspondent to the squared length of at most two independent vectors. Thus $P_IQ_J$ is not WR.
\end{proof}

 

\begin{proposition}\label{prop:dodd_b_ven_aplusb3_mod4}
	Let $d\equiv 1 \pmod 4$, $b$ be even and $a+b\equiv 3\pmod 4$. Then $P_IQ_J $ is not WR. 
\end{proposition}
\begin{proof}
	Let $\delta\in P_IQ_J$ be a nonzero vector of $P_IQ_J$. By Lemma \ref{lem:d_even}, there exist integers $x_1,x_2,x_3,x_4$ such that $\delta  = x_1 p_Iq_J+x_2q_J\dfrac{p_I+\sqrt{d}}{2}+x_3 \dfrac{\beta+\sigma(\beta)}{2}+x_4\dfrac{-\beta+\sigma(\beta)}{2}$ and by \eqref{length}, one obtains
	\begin{align*}
		\|\delta \|^2 = \tron{2x_1+x_2}^2p_I^2q_J^2 +x_2^2q_J^2d+2|a|d\tron{x_3^2+x_4^2}.
	\end{align*}
The result is then followed using the same argument in the proof of Proposition \ref{prop:bd_odd_notWR}.
\end{proof}


\begin{proposition}\label{lem:WR_aplusb1_1} Let $d\equiv 1 \pmod 4, b\equiv 0\pmod 2, a+b\equiv1\pmod 4$ and $a\equiv -c\pmod 4$. Then $P_IQ_J$ is WR if and only if $p_I^2q_J^2+q_J^2d+2|a|d\le \min \mathcal{M}$ where $$\mathcal{M}=\set{ 16q_J^2d, 8|a|d,  4q_I^2d+4|a|d, 16p_I^2q_J^2,4p_I^2q_J^2+4|a|d,4p_I^2q_J^2+4q_J^2d}.$$	 
\end{proposition}
\begin{proof}
	Let $\rho_{IJ}$ be in Lemma \ref{lem:d_odd_b_even_ab1_1} and $\delta $ be a nonzero vector of $P_IQ_J$. By Lemma \ref{lem:d_odd_b_even_ab1_1}, there exist integers $x_1,x_2,x_3,x_4$ such that $4\delta =  S_1p_Iq_J+S_2q_J\sqrt{d}+S_3\beta +S_4\sigma(\beta)$ where $S_1= -x_1-x_2-x_3-x_4,S_2 = x_1-x_2+x_3-x_4,S_3 = -x_1+x_2+x_3-x_4, S_4 =  -x_1-x_2+x_3+x_4$. By \eqref{length}, one has 
	\begin{align*}
		4\|\delta\|^2 =  S_1 p_I^2q_J^2 +S_2^2q_J^2d+|a|d\tron{S_3^2+S_4^2}.
	\end{align*} 
	It is easy to prove that $\min_{\delta\ne 0}\|4\delta\|^2 =\min \mathcal{S}$ where \\$$\mathcal{S} = \set{p_I^2q_J^2+q_I^2d+2|a|d,  16q_J^2d,  8|a|d, 4q_I^2d+4|a|d,  16p_I^2q_J^2, 4p_I^2q_J^2+4|a|d, 4p_I^2q_J^2+4q_I^2d}.$$
 Among seven numbers in $\mathcal{S}$, the only one that is correspondent to the squared length of four linearly independent vectors in $P_I$ is $ p_I^2q_J^2+q_J^2d+2|a|d$. Therefore, the lattice $P_IQ_J$ is WR if and only if $\min_{\delta \ne 0}4\|\delta\|^2 = p_I^2q_J^2+q_J^2d+2|a|d$.
\end{proof}

\begin{proposition}
	\label{lem:WR_aplusb1_2} Let $d\equiv 1 \pmod 4, b\equiv 0\pmod 2, a+b\equiv1\pmod 4$ and $a\equiv c\pmod 4$. Then $P_IQ_J$ is WR if and only if $p_I^2q_J^2+q_J^2d+2|a|d\le \min \mathcal{M}$ where $$\mathcal{M}=\set{ 16q_J^2d,  8|a|d,  4q_J^2d+4|a|d, 16p_I^2q_J^2, 4p_I^2q_J^2+4|a|d, 4p_I^2q_J^2+4q_J^2d}.$$	 
\end{proposition}
\begin{proof}
	Let $\rho_{IJ}$ be in Lemma \ref{lem:d_odd_b_even_ab1_2} and $\delta $ be a nonzero vector of $P_IQ_J$. By Lemma \ref{lem:d_odd_b_even_ab1_2}, there exist integers $x_1,x_2,x_3,x_4$ such that $4\delta =  S_1p_Iq_J+S_2q_J\sqrt{d}+S_3\beta +S_4\sigma(\beta)$ where $S_1= x_1+x_2+x_3+x_4,S_2 = -x_1+x_2-x_3+x_4,S_3 = -x_1-x_2+x_3+x_4, S_4 =  x_1-x_2-x_3+x_4$. By \eqref{length}, one has 
	\begin{align*}
		4\|\delta\|^2 =  S_1 p_I^2q_J^2 +S_2^2q_J^2d+|a|d\tron{S_3^2+S_4^2}.
	\end{align*} 
	It is not hard to verify that $\min_{\delta\ne 0}\|4\delta\|^2 =\min \mathcal{S}$ where $$\mathcal{S} = \set{p_I^2q_J^2+q_I^2d+2|a|d,  16q_J^2d,  8|a|d,  4q_I^2d+4|a|d, 16p_I^2q_J^2, 4p_I^2q_J^2+4|a|d, 4p_I^2q_J^2+4q_I^2d}.$$
 Among seven numbers in $\mathcal{S}$, the only one that is correspondent to the squared length of four linearly independent vectors in $P_I$ is $ p_I^2q_J^2+q_J^2d+2|a|d$. Therefore, the lattice $P_IQ_J$ is WR if and only if $\min_{\delta \ne 0}4\|\delta\|^2 = p_I^2q_J^2+q_J^2d+2|a|d$.
\end{proof}


 We now prove Theorem \ref{thm:main_5}.

\begin{proof}[ Proof of Theorem \ref{thm:main_5}]
\begin{enumerate}[i)]
    \item 	By Propositions \ref{prop:PIQ_J_d_even_notWR}, \ref{prop:bd_odd_notWR}, \ref{prop:dodd_b_ven_aplusb3_mod4},\ref{lem:WR_aplusb1_1} and \ref{lem:WR_aplusb1_2}, the ideal $P_I$ is WR if and only if  $d\equiv 1\pmod 4,p\equiv 0\pmod 2, a+b\equiv 1\pmod 4$ and $p_I^2+\tron{2|a|+1}d\le \min \mathcal{S}$ where $\mathcal{S} = \set{16d, 8|a|d, 4d+4|a|d, 16p_I^2, 4p_I^2+4|a|d,4p_I^2+4d}.$ The last inequality is equivalent to the statement $p_I^2+\tron{2|a|+1}d\le 16d, p_I^2+\tron{2|a|+1}d\le 4d+4|a|d, p_I^2+\tron{2|a|+1}d\le 16p_I^2,p_I^2+\tron{2|a|+1}d\le 4p_I^2+4d$. It means 
    \begin{align}
		\label{eq:ineq_WR_PI}\max \left\{\dfrac{(2|a|-3)d}{3},\dfrac{(2|a|+1)d}{15}\right\}\le p_I^2\le \min\left\{ \tron{15-2|a|}d,\tron{2|a|+3}d\right\}.
	\end{align} 
	The inequalities in \eqref{eq:ineq_WR_PI} occur only if $2|a|\le 15$ and thus $|a|\in \{1,3,5,7\}$.
	 \begin{itemize}
		\item If $|a|=1$, the inequalities in \eqref{eq:ineq_WR_PI} become $\dfrac{d}{5}\le p_I^2\le 5d$.
		\item If $|a|=3$, the inequalities in \eqref{eq:ineq_WR_PI} become $d\le p_I^2\le 9d$.
		\item If $|a|=5$, the inequalities in \eqref{eq:ineq_WR_PI} become $\dfrac{7d}{3}\le p_I^2\le 5d$.
		\item If $|a|=7$, the inequalities in \eqref{eq:ineq_WR_PI} imply that $\dfrac{11d}{3}\le p_I^2\le d$, which is impossible.
	\end{itemize} 
 \item By Propositions \ref{prop:PIQ_J_d_even_notWR}, \ref{prop:bd_odd_notWR}, \ref{prop:dodd_b_ven_aplusb3_mod4},\ref{lem:WR_aplusb1_1} and \ref{lem:WR_aplusb1_2}, one can show that $Q_J$ is WR if and only if $d\equiv 1\pmod 4,p\equiv 0\pmod 2, a+b\equiv 1\pmod 4$ and $q_J^2(d+1)+2|a|d\le \min \mathcal{S}$ where $$\mathcal{S} = \set{ 16q_J^2d, 8|a|d, 4q_J^2d+4|a|d,16q_J^2,4q_J^2+4|a|d,4q_J^2+4q_J^2d}.$$
 The last inequality is equivalent to $$q_J^2(d+1)+2|a|d\le 16q_J^2, q_J^2(d+1)+2|a|d\le 8|a|d, q_J^2(d+1)+2|a|d\le 16q_J^2,q_J^2(d+1)+2|a|d\le 4q_J^2+4d.$$ 
 It means $d<15$ and 
 \begin{align}
		\label{eq:ineq_WR_QJ} \max\set{\dfrac{2|a|d}{15-d},\dfrac{2|a|d}{3\tron{d+1}}}\le q_J^2\le \min\left\{ \dfrac{6|a|d}{d+1},\dfrac{2|a|d}{d-3}\right\}.
	\end{align} Since $d$ is odd and squarefree, and $d<15$, one must have $d\in\{5,13\}.$ If $d=13$ then \eqref{eq:ineq_WR_QJ} becomes $13|a|\le q_J^2\le\dfrac{13|a|}{5} $, which is impossible. Thus $d$ must be $5$ and the inequalities in \eqref{eq:ineq_WR_QJ} become 
		$|a|\le q_J^2\le 5|a|$.
\end{enumerate}

\end{proof}




Now we consider prime ideals above $2$.
\begin{lemma}\label{lem:pideal_above_2_not_WR}
	Let $\left(\text{$d$ be even }\right)$ or $\left(\text{$d$ be odd and $b\equiv 1\pmod 2$}\right)$. All prime ideals above $2$ are not WR. 
\end{lemma}
\begin{proof}
	When $d$ is even, the result is directly implied from Proposition \ref{prop:PIQ_J_d_even_notWR}. The result in the remaining case can be obtained by using a similar argument to  the proof of Propositions \ref{prop:PIQ_J_d_even_notWR} and \ref{prop:bd_odd_notWR}.
\end{proof}
By employing the same methodology used to prove Propositions \ref{prop:PIQ_J_d_even_notWR} and \ref{prop:bd_odd_notWR}, we can establish the result of Lemma \ref{lem:2_ab3_not_WR}.

\begin{lemma}\label{lem:2_ab3_not_WR}
	Let $d\equiv 1 \pmod 8, b\equiv 0\pmod 2$ and $a+b\equiv 3\pmod 4$. All prime ideals above $2$ are not WR.
\end{lemma}
\begin{lemma}\label{lem:ideal_2_WR}
	Let $d\equiv 5 \pmod 8, b\equiv 0\pmod 2$ and $a+b\equiv 3\pmod 4$. Then $\mathcal{O}_F$ has a unique prime ideal $P_0$ above $2$. Moreover, $P_0$ is WR if and only if $a =1,b=2,c=1,d=5$.
\end{lemma}
\begin{proof}
	By Lemma \ref{lem:prideals_above_2_2}, there is a unique prime ideal $P_0$ above $2$ and an integral basis of $P_0$ is given as in this lemma. Let $0\ne \delta\in P_0$, there are integers $z_1,z_2,z_3,z_4$ such that $\delta =2z_1+z_2\tron{1+\sqrt{d}}+z_3 \dfrac{-1+\sqrt{d}-\beta-\sigma(\beta)}{2}+z_4 \dfrac{1+\sqrt{d}+\beta-\sigma(\beta)}{2}$ and by \eqref{length}, one obtains 
	\begin{align*}
	\|\delta\|^2 = S_1 ^2 +S_2^2d+|a|d\tron{S_3^2+S_4^2},
	\end{align*}where $S_1=  4x_1+2z_2-z_3+z_4,S_2= 2z_2+z_3+z_4,S_3= -z_3+z_4,S_4= -z_3-z_4$. It is easy to prove that $\min_{\delta \ne 0}\|\delta\|^2 =\min\set{16,1+d\tron{2|a|+1}}$ and $P_0$ is WR if and only if $16\ge 1+\tron{2|a|+1}$. It occurs only if $a=1,b=2,c=1$.   
\end{proof}

\begin{remark}\label{rem:el_WR}
	If $P$ is a ideal above $2$, then $2\in P$. Thus, if $P$ is WR, then there exists $\delta \in P\setminus \QQ(\sqrt{d})$ such that $\|\delta\|^2\le 16$.
\end{remark}


\begin{lemma}\label{lem:ideal_2_notWR}
Let	$d\equiv 1 \pmod 4, b\equiv 0\pmod 2$ and $a+b\equiv 1\pmod 4$. All prime ideals above $2$ are not WR.
\end{lemma} 
\begin{proof}
	If $d\equiv 5 \pmod 8$, then $2\mathcal{O}_F$ is prime (see Lemma \ref{lem:p2_delta_odd}). We now consider $d\equiv 1 \pmod 8$. Note that $d\ge 17 $ as $d\equiv 1 \pmod 4$ and $d$ is squarefree. We divide into two cases: $a\equiv -c\pmod 4$ and $a\equiv c\pmod 4$. Since the techniques used in the proofs of two cases are similar, we only  consider the case $a\equiv -c \pmod 4$. In this case, suppose that there exists a prime ideal $P$ above $2$ such that $P$ is WR. Hence, by Remark \ref{rem:el_WR},iv, there exists $\delta \in P\setminus \QQ(\sqrt{d})$ such that $\|\delta\|^2\le 16$. Let $\gamma_1',\gamma_2',\gamma_3',\gamma_4'$ as in Remark \ref{rem:integralbasis}. There exist integers $z_1,z_2,z_3,z_4$ such that $\delta =  z_1\gamma_1'+z_2\gamma_2'+z_3\gamma_3'+z_4\gamma_4'$ and thus 
	\begin{align*}
	\|\delta\|^2=\dfrac{1}{4}\tron{\tron{4z_1+2z_2+z_3+z_4}^2+d\tron{2z_2+z_3-z_4}+2|a|d\tron{z_3^2+z_4^2}}.
	\end{align*}
 
	Since $\delta\notin \QQ(\sqrt{d})$, one has $z_3^2+z_4^2\ge 1$. Hence $|a|d \le \|\delta\|^2 \le 32 $ which occurs only if $|a|=1$ and $d\le 32$. It means $(a,d)\in \set{(1,17),(-1,17)}$ as $d\equiv 1 \pmod 8$ and $d$ is squarefree. In both cases of $(a,d)$, there are two prime ideals above $2$ and we can verify that these prime ideals are not WR by using Pari/GP. Hence, all prime ideals above $2$ is not WR when $d\equiv 1 \pmod 4, b\equiv 0\pmod 2$ and $a+b\equiv 1\pmod 4$.  
\end{proof}
Combining Lemmata \ref{lem:pideal_above_2_not_WR}, \ref{lem:2_ab3_not_WR},\ref{lem:ideal_2_WR} and \ref{lem:ideal_2_notWR}, we imply Proposition \ref{prop:ideal_2_WR}
\begin{proposition}
	\label{prop:ideal_2_WR}Let $F,a,b,c,d$ as in \eqref{eq:defpoly-quartic}. Then a prime ideal above $2$ of $\mathcal{O}_F$ is WR if and only if $a=1,b=2,c=1,d=5$. In this case, $\mathcal{O}_F$ has a unique prime ideal above $2$.
\end{proposition} 






%%%%%%%%%%%%%%%%%%%%%%%%%%%%%%%%%%%%%%%%%%%%%%%%%%%%%%%%%%%%%%%
\section{Conclusion and future research}\label{sec:conclusion}
This paper investigates WR ideals of cyclic and quartic fields. We show that all cyclic cubic fields have WR ideals. Moreover, we present families of cyclic cubic and quartic fields of which WR ideal lattices exist and also  construct explicit 
 minimal bases of these WR ideals.  

We observe that all WR ideals obtained from our experiment have norms dividing the discriminant of the field if the discriminant is odd.  Therefore, we form the following conjecture.

\textbf{Conjecture:} Let $F$ be a  cyclic cubic  or cyclic quartic field with odd discriminant. If a primitive integral ideal $I$ of $F$ is WR, then $N(I)$  divides the discriminant of $F$.  

If this conjecture holds then there are only finitely many WR ideals from these fields.

 

Note that this conjecture agrees with the observation in \cite{FHLPSW13} for real quadratic fields, and it was later proved for these fields \cite{S20}. In addition, for a cyclic quartic field $F$ of odd discriminant, the conjecture holds for the case when the ideal $I$ of $F$ is the unique prime ideal above a prime number as a result of Theorem \ref{thm:main6}.

We also remark that the conjecture does not hold for cyclic quartic fields of even discriminant. That is, there exist cyclic quartic fields with even discriminant which have WR ideals of norms that do not divide the field discriminant. For example, the cyclic quartic field $F$ define by $(a,b,c,d=(1,2,1,5)$ has WR ideals with norms 484, 2420, 3364, and 3844 which do not divide $\Delta_F=2000$. Another remark is that this is the only case  in which a prime ideal above $2$ is WR by Proposition \ref{prop:ideal_2_WR}.

Our future research will investigate the above conjecture and WR ideals of other number fields.










%%%%%%%%%%%%%%%%%%%%%%%%%%%%%%%%%%%%%%%%%%%%%%%%%%%%%%%%%%%%%%%
\bibliographystyle{abbrv}
\bibliography{myrefs}




%%%%%%%%%%%%%%%%%%%%%%%%%%%%%%%%%%%%%%%%%%%%%%%%%%%%%%%%%%%%%%%
\newpage
\appendix

\section{Appendix for Proofs}

\paragraph{Proof of Theorem \ref{thm:main}.}

\begin{proof}
\label{proof:main}
Our proof has two steps. In Step 1, we will show that SimCLR is equivalent to minimizing the cross entropy loss defined in Eqn.~(\ref{eqn:cross-entropy}). 
In Step 2, we will show  that minimizing the cross-entropy loss 
is equivalent to spectral clustering on $\bfpi$. 
Combining the two steps together, we have proved our theorem. 

\textbf{Step 1: } SimCLR is equivalent to minimizing the cross entropy loss.

The cross-entropy loss takes expectation over 
$\bfW_\bfX\sim \mathbb{P}(\cdot ; \bfpi)$, 
which means $\bfW_\bfX$ has exactly one non-zero entry in each row $i$. By Lemma~\ref{lem:multinomial}, we know every row $i$ of $\bfW_\bfX$ is independent of other rows. Moreover, 
$\bfW_{\bfX,i}\sim \mathcal{M}(1, \bfpi_i/\sum_j \bfpi_{i,j})=\mathcal{M}(1, \bfpi_i)$, because $\bfpi_i$ itself is a probability distribution.
Similarly, we know $\bfW_\bfZ$ also has the row-independent property by sampling over $\mathbb{P}(\cdot;\bfK_\bfZ)$.
Therefore, by Lemma~\ref{lem:cross_split}, we know Eqn.~(\ref{eqn:cross-entropy}) is equivalent to:
\[
 -\sum_{i=1}^n \mathbb{E}_{\bfW_{\bfX,i}}[\log \mathbb{P}(\bfW_{\bfZ,i}=\bfW_{\bfX,i};\bfK_\bfZ)],
\]

This expression takes expectation over $\bfW_{\bfX,i}$ for the given row $i$. Notice that 
$\bfW_{\bfX,i}$ has exactly one non-zero entry, which equals $1$ (same for $\bfW_{\bfZ,i}$). 
As a result
we expand the above expression to be:
\begin{equation}
 -\sum_{i=1}^n \sum_{j\neq i} \Pr(\bfW_{\bfX,i,j}=1)\log \Pr(\bfW_{\bfZ,i,j}=1).
\label{eqn:detailed-expansion}    
\end{equation}


By Lemma~\ref{lem:multinomial}, $\Pr(\bfW_{\bfZ,i,j}=1)=\bfK_{\bfZ,i,j}/\|\bfK_{\bfZ,i}\|_1$ for $j\neq i$. Recall that $\bfK_\bfZ=(k(\bfZ_i-\bfZ_j))_{(i,j)\in[n]^2}$, which means 
$\bfK_{\bfZ,i,j}/\|\bfK_{\bfZ,i}\|_1=\frac{\exp(-\|\bfZ_i-\bfZ_j\|^2/{2\tau})}{\sum_{k\neq i}
\exp(-\|\bfZ_i-\bfZ_k\|^2/{2\tau})
}$ for $j\neq i$, when $k$ is the Gaussian kernel with variance $\tau$. 

Notice that $\bfZ_i=f(\bfX_i)$, so we know
\begin{equation}
-\log \Pr(\bfW_{\bfZ,i,j}=1)=
-\log \frac{\exp(-\|f(\bfX_i)-f(\bfX_j)\|^2/{2\tau})}{\sum_{k\neq i}
\exp(-\|f(\bfX_i)-f(\bfX_k)\|^2/{2\tau}),
}
\label{eqn:infonce-equivalence}    
\end{equation}


The right hand side is exactly the InfoNCE loss defined in Eqn.~(\ref{eqn:infonce}).
Inserting Eqn.~(\ref{eqn:infonce-equivalence}) into Eqn.~(\ref{eqn:detailed-expansion}), we get the SimCLR algorithm, which first samples augmentation pairs $(i,j)$ with $\Pr(\bfW_{\bfX,i,j}=1)$ for each row $i$, and then optimize the InfoNCE loss. 

\textbf{Step 2: } minimizing the cross entropy loss 
is equivalent to spectral clustering on $\bfpi$.


By Lemma~\ref{lem:convert_to_spectral}, we may further convert the loss to 
\begin{equation}
\label{eqn:main-theorem-repul-attr}
\min_{\bfZ}
-\sum_{(i,j)\in [n]^2} \mathbf{P}_{i,j}
\log k (\bfZ_i-\bfZ_j)+\log \mathbf{R}(\bfZ).
\end{equation}
Since $k$ is the Gaussian kernel, this reduces to \[
\min_\bfZ \mathrm{tr}(\bfZ^\top \mathbf{L}(\bfpi) \bfZ)
+\log \mathbf{R}(\bfZ),
\]

where we use the fact that $\mathbb{E}_{\bfW_\bfX\sim \mathbb{P}(\cdot; \bfpi)}[\mathbf{L}(\bfW_\bfX)]
=\mathbf{L}(\bfpi)
$, because the Laplacian operator is linear and $
\mathbb{E}_{\bfW_\bfX\sim \mathbb{P}(\cdot; \bfpi)}(\bfW_\bfX)=\bfpi
$.
\end{proof}

\paragraph{Proof of Theorem \ref{thm:clip}.}
\begin{proof}
Since $\bfW_\bfX\sim \mathbb{P}(\cdot;\bfpi_{\mathbf{A}, \mathbf{B}})$, we know 
$\bfW_\bfX$ has exactly one non-zero entry in each row, denoting the pair that got sampled. 
A notable difference compared to the previous proof is we now have $n_\mathcal{A}+n_\mathcal{B}$ objects in our graph. CLIP deals with this by taking a mini-batch of size $2N$, 
such that $n_\mathcal{A}=n_\mathcal{B}=N$, and adding the $2N$ InfoNCE losses together. We label the objects in $\mathcal{A}$ as $[n_\mathcal{A}]$, and the objects in $\mathcal{B}$ as $\{n_\mathcal{A}+1, \cdots, n_\mathcal{A}+n_\mathcal{B}\}$. 

Notice that $\bfpi_{\mathbf{A}, \mathbf{B}}$ is a bipartite graph, so the edges of objects in $\mathcal{A}$ will only connect to object in $\mathcal{B}$ and vice versa. We can define the similarity matrix in $\cZ$ as $\bfK_\bfZ$, 
where $\bfK_\bfZ(i, j+n_\mathcal{A})=\bfK_\bfZ(j+n_\mathcal{A},i)= k(\bfZ_i-\bfZ_j)$ for $i\in [n_\mathcal{A}], j\in [n_\mathcal{B}]$, and otherwise we set $\bfK_\bfZ(i,j)=0$. 
The rest is same as the previous proof. 
\end{proof}

\paragraph{Proof of Theorem \ref{thm:exponential}.}

\begin{proof}
\label{proof:exponential}
Since the objective function consists of a linear term combined with an entropy regularization, which is a strongly concave function, the maximization problem is a convex optimization problem. Owing to the implicit constraints provided by the entropy function, the problem is equivalent to having only the equality constraint. We then introduce the Lagrangian multiplier $\lambda$ and obtain the following relaxed problem:

$$
\widetilde{E}(\boldsymbol{\alpha})=\psi_{1}-\sum_{i=1}^n \alpha_{i} \psi_{i}+\tau \sum_{i=1}^n \alpha_{i}\log \alpha_{i}+\lambda\left(\boldsymbol{\alpha}^{\top} \mathbf{1}_n-1\right).
$$

As the relaxed problem is unconstrained, taking the derivative with respect to $\alpha_{i}$ yields

$$
\frac{\partial \widetilde{E}(\boldsymbol{\alpha})}{\partial \alpha_{i}}=-\psi_{i}+\tau\left(\log \alpha_{i}+\alpha_{i} \frac{1}{\alpha_{i}}\right)+\lambda=0.
$$

Solving the above equation implies that $\alpha_{i}$ takes the form
$
\alpha_{i}=\exp \left(\frac{1}{\tau} \psi_{i}\right) \exp \left(\frac{-\lambda}{\tau}-1\right).
$ Since $\alpha_{i}$ lies on the probability simplex, the optimal $\alpha_{i}$ is explicitly given by
$
\alpha^{*}_{i}=\frac{\exp \left(\frac{1}{\tau} \psi_{i}\right)}{\sum_{i^{\prime}=1}^n \exp \left(\frac{1}{\tau} \psi_{i^{\prime}}\right)} .
$ Substituting the optimal point into the objective function, we obtain
$$
\begin{aligned}
E\left(\boldsymbol{\alpha}^*\right)  &=\psi_1-\sum_{i=1}^n \frac{\exp \left(\frac{1}{\tau} \psi_{i}\right)}{\sum_{i^{\prime}=1}^n \exp \left(\frac{1}{\tau} \psi_{i^{\prime}}\right)} \psi_{i}+\tau \sum_{i=1}^n \frac{\exp \left(\frac{1}{\tau} \psi_{i}\right)}{\sum_{i^{\prime}=1}^n \exp \left(\frac{1}{\tau} \psi_{i^{\prime}}\right)}\log \frac{\exp \left(\frac{1}{\tau} \psi_{i}\right)}{\sum_{i^{\prime}=1}^n \exp \left(\frac{1}{\tau} \psi_{i^{\prime}}\right)} \\
& =\psi_1 - \tau \log \left(\sum_{i=1}^n \exp \left(\frac{1}{\tau} \psi_{i}\right)\right).
\end{aligned}
$$
Thus, the Lagrangian dual function is given by
\begin{equation*}
-E\left(\boldsymbol{\alpha}^*\right)= -\tau \log \frac{\exp \left(\frac{1}{\tau} \psi_{1}\right)}{\sum_{i=1}^n \exp \left(\frac{1}{\tau} \psi_{i}\right)}.\qedhere
\end{equation*}
\end{proof}



\section{More on Experiments} \label{section: experiment_details}

\paragraph{CIFAR-10 and CIFAR-100} CIFAR-10 ~\citep{krizhevsky2009learning} and CIFAR-100 ~\citep{krizhevsky2009learning} are well-known classic image classification datasets. Both CIFAR-10 and CIFAR-100 contain a total of 60k $32 \times 32$ labeled images of different classes, with 50k for training and 10k for testing. CIFAR-10 is similar to CIFAR-100, except there are 10 different classes in CIFAR-10 and 100 classes in CIFAR-100.

\paragraph{TinyImageNet} TinyImageNet ~\citep{le2015tiny} is a subset of ImageNet ~\citep{deng2009imagenet}. There are 200 different object classes in TinyImageNet, with 500 training images, 50 validation images, and 50 test images for each class. All the images in TinyImageNet are colored and labeled with a size of $64 \times 64$.

\textbf{Pseudo-code.} Algorithm \ref{alg:Training Procedure} presents the pseudo-code for our empirical training procedure.

\begin{algorithm}[!htbp]
\caption{Training Procedure}
\label{alg:Training Procedure}
\begin{algorithmic}[1]
\REQUIRE trainable encoder network $f$, batch size $N$, augmentation strategy \textit{aug}, loss function $L$ with hyperparameters \textit{args}
\FOR {sampled minibatch ${x_i}_{i=1}^N$}
\FORALL{$i \in { 1, ..., N }$}
\STATE draw two augmentations $t_i = \textit{aug}\left(x_i\right) $, $t_i' = \textit{aug}\left(x_i\right) $
\STATE $z_i = f\left(t_i\right)$, $z_i' = f\left(t_i'\right)$
\ENDFOR
\STATE compute loss $\mathcal{L} = L(N, z, z', \textit{args})$
\STATE update encoder network $f$ to minimize $\mathcal{L}$
\ENDFOR
\STATE \textbf{Return} encoder network $f$
\end{algorithmic}
\end{algorithm}

We also provide the pseudo-code for our core loss function used in the training procedure in Algorithm \ref{alg:Core loss}. The pseudo-code is almost identical to SimCLR's loss function, with the exception of an extra parameter $\gamma$.

\begin{algorithm}[!htbp]
\caption{Core loss function $\mathcal{C}$}
\label{alg:Core loss}
\begin{algorithmic}[1]
\REQUIRE batch size $N$, two encoded minibatches $z_1, z_2$, $\gamma$, temperature $\tau$
\STATE $z = \textit{concat}\left(z_1, z_2\right)$
\FOR {$i \in {1, ..., 2N }, j \in {1, ..., 2N}$ }
\STATE $s_{i,j} = \Vert z_i - z_j \Vert_2^{\gamma}$
\ENDFOR
\STATE \textbf{define} $l(i, j)$ \textbf{as} $l(i, j) = - \log \frac{exp\left(s_{i,j}/\tau \right)}{\sum_{k=1}^{2N} \mathbf{1}{[k \ne i]} exp\left(s{i, j} / \tau \right)} $
\STATE \textbf{Return} $\frac{1}{2N} \sum_{k=1}^N\left[l(i, i+N) + l(i+N, i)\right]$
\end{algorithmic}
\end{algorithm}

Utilizing the core loss function $\mathcal{C}$, we can define all kernel loss functions used in our experiments in Table \ref{table: loss definition}. For all $z_i \in z$ with even dimensions $n$, we define $z_{L_i} = z_i\left[0:n/2\right]$ and $z_{R_i} = z_i\left[n/2:n\right]$.

\begin{table}[ht]
\centering
\begin{tabular}{{@{}l|l@{}}}
Kernel  &  Loss function \\ \midrule
Laplacian & $\mathcal{C}\left(N, z, z', \gamma=1, \tau\right)$\\ \midrule
Sum       & $\lambda * \mathcal{C}\left(N, z, z', \gamma=1, \tau_1\right) + (1-\lambda) * \mathcal{C}\left(N, z, z', \gamma=2, \tau_2\right)$  \\ \midrule
Concatenation Sum&$\lambda * \mathcal{C}\left(N, z_L, z'_L, \gamma=1, \tau_1\right) + (1-\lambda) * \mathcal{C}\left(N, z_R, z'_R, \gamma=2, \tau_2\right)$\\ \midrule
$\gamma = 0.5$ & $\mathcal{C}\left(N, z, z', \gamma=0.5, \tau\right)$          \\ 

\end{tabular}

\caption{Definition of kernel loss functions in our experiments}
\label {table: loss definition}
\end{table}

\textbf{Baselines.} We reproduce the SimCLR algorithm using PyTorch Lightning~\citep{PytorchLightning}.

\textbf{Encoder details.}
The encoder $f$ consists of a backbone network and a projection network. We employ ResNet50~\citep{ResNet} as the backbone and a 2-layer MLP (connected by a batch normalization~\citep{ioffe2015batch} layer and a ReLU \cite{nair2010rectified} layer) with hidden dimensions 2048 and output dimensions 128 (or 256 in the concatenation kernel case).

\textbf{Encoder hyperparameter tuning.}
For each encoder training case, we randomly sample 500 hyperparameter groups (sample details are shown in Table \ref{table: Hyperparameter sample}) and train these samples simultaneously using Ray Tune ~\citep{RayTune}, with the ASHA scheduler~\citep{li2018massively}. Ultimately, the hyperparameter group that maximizes the online validation accuracy (integrated in PyTorch Lightning) within 5000 validation steps is chosen for the given encoder training case.

\begin{table}[ht]
\centering

\begin{tabular}{@{}l|l|l@{}}
\midrule
Hyperparameter  & Sample Range & Sample Strategy \\ \midrule
start learning rate & $\left[10^{-2}, 10\right]$ & log uniform \\ \midrule
$\lambda$       & $\left[0, 1\right]$ & uniform \\ \midrule
$\tau$, $\tau_1$, $\tau_2$ & $\left[0, 1\right]$ & log uniform \\ \midrule
\end{tabular}

\caption{Hyperparameters sample strategy}
\label {table: Hyperparameter sample}
\end{table}

\textbf{Encoder training.} 
We train each encoder using the LARS optimizer~\citep{LARSOptimizer}, LambdaLR Scheduler in PyTorch, momentum 0.9, weight decay $10^{-6}$, batch size 256, and the aforementioned hyperparameters for 400 epochs on a single A-100 GPU.

\textbf{Image transformation.} The image transformation strategy, including augmentation, is identical to the default transformation strategy provided by PyTorch Lightning.

\textbf{Linear evaluation.}
The linear head is trained using the SGD optimizer with a cosine learning rate scheduler, batch size 64, and weight decay $10^{-6}$ for 100 epochs. The learning rate starts at $0.3$ and ends at $0$.

\textbf{Moco Experiments.} We also tested our method based on MoCo~\citep{he2019moco}. The results are summarized in Table \ref{tab:results-moco}. Here we choose ResNet18~\citep{ResNet} as the backbone and set a temperature of $0.1$ as default. For our simple sum kernel, we set $\lambda=0.8$. The results show that our method outperforms the original MoCo method.

\begin{table}[thb]
\centering
\caption{MoCo Experiment Results on CIFAR-10 and CIFAR-100.}
\label{tab:results-moco}
\resizebox{\textwidth}{!}{%
\begin{tabular}{@{}c|ccc|ccc@{}}
\toprule
\multirow{3}{*}{Method} & \multicolumn{3}{c|}{CIFAR-10} & \multicolumn{3}{c}{CIFAR-100} \\ \cmidrule(lr){2-4} \cmidrule(lr){5-7} 
                        & 200 epochs & 400 epochs    & 1000 epochs   & 200 epochs & 400 epochs & 1000 epochs         \\ \midrule
MoCo (repro.)         & $76.41 \pm 0.12$    & $80.01 \pm 0.15$          & $84.45 \pm 0.08$    & $\mathbf{47.02 \pm 0.11}$ & $52.50 \pm 0.07$ & $57.62 \pm 0.15$            \\
\midrule
Laplacian Kernel        & ${78.09 \pm 0.10}$    & $\mathbf{83.85 \pm 0.09}$          & $\mathbf{88.34 \pm 0.16}$    & $46.12 \pm 0.22$   & $53.44 \pm 0.17$ & $59.10 \pm 0.14$        \\
Simple Sum Kernel & $\mathbf{78.12 \pm 0.15}$   & $83.23 \pm 0.18$ & $87.50 \pm 0.20$ & $46.65 \pm 0.06$ & $\mathbf{53.62 \pm 0.19}$ & $\mathbf{59.83 \pm 0.12}$\\
\bottomrule
\end{tabular}
}
\end{table}



\section{More Experiments on Synthetic Data}


Consider a scenario with $n$ clusters, each containing $k$ vertices. Let the probability of vertices $u$ and $v$ from the same cluster belonging to $\bfpi$ be $p$. Conversely, for vertices $u$ and $v$ from different clusters, let the probability of belonging to $\pi$ be $q$. We generate the graph $\bfpi$ randomly, based on $p$ and $q$. We experiment with values of $k=100$ and $n=6$ for ease of visualization, embedding all points in a two-dimensional space. Each vertex's initial position originates from a normal distribution. In each iteration, we sample a subgraph of $\bfpi$ uniformly, ensuring each vertex has an out-degree of $1$. We then optimize the corresponding vectors using InfoNCE loss with an SGD optimizer and iterate until convergence. Our experimental setup consists of an SGD learning rate of $1$, an InfoNCE loss temperature of $0.5$, and a batch size of $50$. We evaluate two scenarios with different $p$ and $q$ values: $p=1$, $q=0$, and $p=0.75$, $q=0.2$. The results of these experiments are visualized in Figure \ref{fig:vis-spectral-cluster}. The obtained embeddings exhibit the hallmark pattern of spectral clustering of graph $\bfpi$.

\begin{figure}[!tb]
\centering
\subfigure{
\includegraphics[width=1\textwidth]{Figures/cluster_pi.png}
\label{fig:vis-cluster}
}
\subfigure{
\includegraphics[width=1\textwidth]{Figures/noised_cluster_pi.png}
\label{fig:vis-noised-cluster}
}
\caption{Visualizations of the optimization process using InfoNCE Loss on the vectors corresponding to $\bfpi$. Points of identical color belong to the same cluster within $\bfpi$. To showcase the internal structure of $\bfpi$, we randomly select 10 vertices from each cluster to display the edge distribution of $\bfpi$.}
\label{fig:vis-spectral-cluster}
\end{figure}


\end{document}


