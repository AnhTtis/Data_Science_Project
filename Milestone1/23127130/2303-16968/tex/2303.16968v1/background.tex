
In this section,  we will recall some fundamental knowledge about WR ideal latices,  cyclic cubic, and cyclic  quartic fields.



%%%%%%%%%%%%%%%%%%%%%%%%%
\subsection{Well-rounded ideal lattices }

Let $\mathcal{B}=\{v_1,v_2,...,v_m\}$ be a linearly independent set of vectors in $\mathbb{R}^n$, $1\le m \le n$. The set $L=\left\{\sum_{i=1}^{m}a_iv_i|a_i\in \mathbb{Z}\right\}$ is called a \textit{lattice} in $\mathbb{R}^n$ of rank $m$ and the set $\mathcal{B}$ is said to be a \textit{basis} of $L$. In case $m=n$, we say that $L$ is a \textit{full rank lattice}.\\

The value $|L| = \min_{0\ne u \in L}\|u\|^2$ is called the \textit{minimum norm} of the lattice $L\subset \mathbb{R}^n$, where $\|.\|$ denotes the usual Euclidean norm in $\mathbb{R}^n$, and the set of \textit{minimum vectors} of $L$ is defined as 
\[S(L):=\{u\in L:\|u\|^2=|L|\}.\]

\begin{definition} \label{def:WR}
	Let $L$ be a lattice in $\mathbb{R}^n$.
	\begin{enumerate}
		\item The lattice $L$ is \textbf{WR} if $S(L)$ generates $\mathbb{R}^n$, that is, if $S(L)$ contains $n$ linearly independent vectors.
		\item The lattice $L$ is said \textbf{strongly WR} if $S(L)$ consists of  a basis of $L$. In this case, we call this basis a \textit{minimal} basis of $L$. 
		
	\end{enumerate}
\end{definition}
We denote by $B$ is an $n\times m$-matrix whose columns are the vectors of $\mathcal{B}$.
\begin{definition}
	Let $L$ be a lattice of rank $n$ and its matrix basis $B$. The \textbf{determinant} of $L$, denoted by $\det(L)$, is defined $\det(L):=\sqrt{\det(B^TB)}$. In the special case that $L$ is a full rank lattice, $B$ is a square matrix, then we have $\det(L)=|\det(B)|$.
\end{definition}
The determinant of a lattice is well-defined since it is independent of our choice of basis $B$. Indeed,  $B_1$ and $B_2$ are two bases of $L$, if and only if $B_2=B_1U$ for some unimodular matrix $U$ with integer entries. Hence,
\[\sqrt{\det(B_2^TB_2)}=\sqrt{\det(U^TB_1^TB_1U)}=\sqrt{\det(B_1^TB_1)}.\]

We recall the following result.
\begin{lemma} \label{sublatticeequal}
	Let $L$ and $L'$ be two full rank lattices in $\mathbb{R}^n$ ($n \ge 1$). Assume that $L' \subseteq L$ and $\det(L) = \det(L')$. Then $L'=L$.
\end{lemma}
\begin{proof}
	Let $B,B'$ be bases of $L,L'$, respectively. Suppose $B'=BA$, then $[L:L']=|\det(A)|=\dfrac{|\det(B')|}{|\det(B)|}=\dfrac{\det(L')}{\det(L)}=1$. Hence, $L=L'.$ 
\end{proof}

Let $F$ be a number field of degree $n$ and signature $(r_1, r_2)$. Then $F$ has $r_1+r_2$ embeddings up to conjugate: $\sigma_1, \dots,  \sigma_{r_1+r_2}$ where the first $r_1$ of them are real, and the last $r_2$ are complex. We denote by $\Phi: F \hookrightarrow F \otimes \mathbb{R} \cong \mathbb{R}^{r_1} \times \mathbb{C}^{r_2}$ the map defined by $\Phi(f)= (\sigma_1(f), \cdots, \sigma_{r_1+r_2}(f))$. Here $\mathbb{R}^{r_1} \times \mathbb{C}^{r_2}$ is a Euclidean space with the scalar product: $\langle u, v\rangle = \sum_{i=1}^{r_1} u_i v_i + 2\sum_{i= r_1+1}^{r_2} \Re(u_i \overline{v_i}) $ where $\overline{v_i}$ is the complex conjugate of $v_i$. \\
Let  $Q$ be a (fractional) ideal of $F$. Then it is known that $\Phi(Q)$ is a lattice in $\mathbb{R}^{r_1} \times \mathbb{C}^{r_2}$  by \cite{Bayer-Fluckiger99}. By identifying  $Q$ and $\Phi(Q)$, one has that  $Q$ is an ideal of $F$ and also a lattice in $\mathbb{R}^{r_1} \times \mathbb{C}^{r_2}$. Hence we call ideals of $F$ ideal lattices, see  \cite{Bayer-Fluckiger99} and also \cite[Section 4]{Schoof08} for more details. An ideal lattice $Q$ is called WR if the  lattice  $\Phi(Q)$ is WR. 




%%%%%%%%%%%%%%%%%%%%%%%%%%%%%
\subsection{Cyclic cubic fields} \label{sec:cubicfields} 


Let $F$ be a cyclic cubic field with conductor $m$. By \cite[pp.6-10]{maki2006determination}, one has \begin{align}
\label{conductor}m= \frac{a^2+3b^2}{4}
\end{align} where $a$ and $b$ are integers satisfying one of
the following conditions,
\begin{align} \label{eq:a_b_cubic}
a \equiv 2 \mod 3, b \equiv 0 \mod 3&\text{ and } b>0 \text{  for  } 3 \not| m, \text{ and}\\\nonumber
a \equiv 6 \mod 9, b \equiv 3 \text{  or  } 6 \mod 9& \text{ and } b>0 \text{  for  } 3 | m.
\end{align}

We recall that the conductor $m$ of $F$ has the form
$$m = p_1 p_2 \cdots p_r,$$
where $r \in \mathbb{Z}_{>0}$ and $p_1, \cdots , p_r$ are distinct integers from the set
$$ \{9\} \cup  \{q: q\text{ is prime and } q\equiv 1\mod 3\} = \{7, 9, 13, 19, 31, 37, \dots \}.$$
The discriminant of $F$ is $\Delta_F=m^2$.
See Hasse \cite{hasse1930arithmetische} for more details. From \cite{maki2006determination}, the following polynomial,  denoted by $df$, can be used to define $F$,
\begin{align}\label{df-polynomial-cubic}
df(x)=\left\{\begin{matrix}x^3 - x^2 + \frac{1-m}{3}x -\frac{m(a-3)+1}{27},& \text{if} \ 3 \not | \ m\\x^3 -\frac{m}{3}x -\frac{a m}{27},& \text{if}\  3 |m
\end{matrix}\right..
\end{align}
Let $m = p_1 \cdots p_r$ or $m= 9 \cdot p_1 \cdots p_r$ here all $p_i$ are distinct prime numbers and $p_i \equiv 1 \mod 3$ for $i=1,\cdots,r$ and $p_0=3$. We arrange $p_i$ such that $p_1 < p_2 < \cdots < p_r$. We denote by $\alpha$ a root of the defining polynomial $df(x)$ in \eqref{df-polynomial-cubic}.






\begin{lemma}\label{lem:index-cubic}
	Let $id_3=[O_F: \mathbb{Z}[\alpha]]$. Then $p_i$ does not divide the index $id_3$ for all $i\ge 0$.
\end{lemma}
\begin{proof}
	We suppose by contradiction that there exists $i\ge 0$ such that $p_i | id_3$. 
	By \ref{df-polynomial-cubic}, we can calculate the discriminant of $df$ as
	$$\Delta_{df}= \frac{m^2(4m-a^2)}{27}.$$
	Since $F$ has discriminant $m^2$, one must have $id^2$ divides $\frac{4m-a^2}{27}$. Thus $p_i^2$ divides $\frac{4m-a^2}{27}$. Moreover, $p_i|m$. It leads to $p_i^2|m$ which implies that $p_i=3$ since $3$ is the only prime of which square dividesthe conductor $m$ given in \eqref{conductor}. In other words, $3|m$ and hence $9$ divides $\frac{4m-a^2}{27} = \frac{b^2}{9} $ which is a contradiction since $b \equiv 3 \text{  or  } 6 \mod 9$ in \eqref{df-polynomial-cubic}. Thus, $p_i \not| id_3$ for all $i$.
\end{proof}

We can easily prove the following.
\begin{lemma}\label{lemindependentcubic}
	Let $g\in \mathcal{O}_F\backslash \mathbb{Z}$. Then $\Tr(g) \ne 0$ if and only if $\set{g, \sigma(g), \sigma^2(g)}$ is $\mathbb{R}$-linearly independent.
\end{lemma}



%%%%%%%%%%%%%%%%%%%%%%%%%%%%%%%%%
\vspace*{0.5cm}
\subsection{Cyclic quartic fields}\label{sec:quarticfields}
We first recall the facts about cyclic quartic fields and their properties. See \cite{hudson1990integers} for more details. Let
$F=\mathbb{Q}(\beta)$
where
$a, b, c, d$ are integers such that 
$a$ is squarefree and odd,
$d= b^2+c^2$ is squarefree,
$b>0, c>0$,
$\gcd(a, d)=1$ and $\beta=\sqrt{a(d - b \sqrt{d})}$. If $a>0$ then $F$ is a totally real cyclic quartic field. If $a<0$ then $F$ is a totally imaginary cyclic quartic field.
%%%%%%%%%%%%%%%%%%%%%%%%%%%%%%%%%



A defining polynomial of $F$, which is also the minimum polynomial of $\beta$, is 
\begin{equation}\label{eq:defpoly-quartic}
df(x)= x^4 -2ad x^2 + a^2c^2d.
\end{equation}
It is easy to verify that the discriminant $\Delta_{df}$ of $df(x)$ is $256a^6b^4c^2d^3$ and by \cite{hudson1990integers}, the discriminant of $F$ is \begin{align}\label{quarticdisc}
\Delta_F= \left\{
\begin{matrix}
2^8a^2d^3 &\text{ if } d\equiv 0\mod 2,  \\
2^6a^2d^3 & \text{ if } d\equiv 1\mod2,b\equiv 1\mod 2, \\ 
2^4a^2d^3 &\text{ if } d\equiv 1\mod2,b\equiv 0\mod 2, a+b\equiv 3\mod 4,\\ 
a^2d^3 &\text{ if } d\equiv 1\mod2,b\equiv 0\mod 2, a+b\equiv 1\mod 4.\\ 
\end{matrix}\right.
\end{align}
It implies that the index $id_4$ of $\ZZ[\beta]$ in $\mathcal{O}_F$ is given as follow \begin{align}\label{eq:index_quartic_cyclic}
id_4 =  \dfrac{\Delta_{df}}{\Delta_F} =\left\{
\begin{matrix}
a^2b^2c &\text{ if } d\equiv 0\mod 2,  \\
2a^2b^2c & \text{ if } d\equiv 1\mod2,b\equiv 1\mod 2, \\ 
2^2a^2b^2c &\text{ if } d\equiv 1\mod2,b\equiv 0\mod 2, a+b\equiv 3\mod 4,\\ 
2^4a^2b^2c &\text{ if } d\equiv 1\mod2,b\equiv 0\mod 2, a+b\equiv 1\mod 4.\\ 
\end{matrix}\right.
\end{align}
For $K = \mathbb{Q}(\sqrt{d})$, we always have the tower of field extensions \begin{align}
\label{eq:exten_tower} \mathbb{Q} \leq K \leq F.
\end{align}
The field $F$ has four embeddings: $1, \sigma, \sigma^2, \sigma^3$ where 
\begin{equation}\label{embbeddings}
\sigma: \beta \longmapsto \sigma(\beta), \qquad \sigma(\beta) \longmapsto -\beta, \qquad \sqrt{d} \longmapsto -\sqrt{d}.
\end{equation}

In case $a<0$, the field $F$ is totally complex and  the 4 roots of $df(x)$ are $\beta, -\beta, \sigma(\beta)=\sqrt{a(d+b \sqrt{d})},  -\sigma(\beta)$ which are all in $\mathbb{R} i$. Hence when we embed them in $\mathbb{R}^4$, they have the form $ (0, z,0,t)$. Here one has $\overline{1}=  \sigma^2$ and $\overline{\sigma}=  \sigma^3$.  Thus $F$ has  2 embeddings $1$ and $ \sigma$ up to conjugate. For $\delta\in F$, we embed it to $(\delta, \sigma(\delta) ) \in \mathbb{C}^2$ which is then can be viewed as $(\Re(\delta), \Im(\delta), \Re(\sigma(\delta) ), \Im(\sigma(\delta) ) \in \mathbb{R}^4$. The four roots of $df(x)$ are totally imaginary hence they will have the form $(0, z_1, 0, z_2)$ for some $z_1, z_2 \in \mathbb{R}$ when embedded in $\mathbb{R}^4$.

In case  $a>0$, the field $F$ is  totally real   and the 4 roots of $df(x)$ are $\beta, -\beta, \sigma(\beta)=\sqrt{a(d+b \sqrt{d})},  -\sigma(\beta)$ which are all in $\mathbb{R}$. When we embed an element $\delta\in F$ in $\mathbb{R}^4$, we obtain the vector $ (\delta,\sigma(\delta),\sigma^2(\delta),\sigma^3(\delta))$. 

Although the embeddings of imaginary and the totally real fields are different, we can  still verify that if $\delta=s_1+s_2\sqrt{d}+s_3\beta+s_4\sigma(\beta) \in F$ where $s_i \in \mathbb{Q}$ for all $i \in \{1, 2, 3, 4\}$, then 
\begin{align}
\label{length} \|\delta\|^2=4\left(s_1^2+s_2^2d+|a|ds_3^2+|a|ds_4^2\right).
\end{align} In particular,
\[\|\beta\|^2=4|a|d.\]

\begin{remark}\label{rem:integralbasis}
	The following integral basis $\mathcal{B}=\left\{\gamma_1',\gamma_2',\gamma_3',\gamma_4'\right\}$ in this order of $F$ is provided in \cite{hudson1990integers}  which we will use in the later sections.
	\begin{enumerate}[i)]
		\item $\set{1,\sqrt{d},\sigma(\beta),\beta}$, if $d\equiv 0 \pmod 2$;
		\item $\left\{1,\dfrac{1}{2}(1+\sqrt{d}),\sigma(\beta),\beta\right\}$, if $d\equiv b\equiv 1 \pmod 2$;
		\item $\set{1,\dfrac{1}{2}(1+\sqrt{d}),\dfrac{1}{2}(\sigma(\beta)+\beta),\dfrac{1}{2}(\sigma(\beta) -\beta)}$, if $d\equiv 1\pmod 2,b \equiv 0 \pmod 2,a+b\equiv 3\pmod 4$;
		\item $\set{1,\dfrac{1}{2}(1+\sqrt{d}),\dfrac{1}{4}(1+\sqrt{d}+\sigma(\beta)-\beta),\dfrac{1}{4}(1-\sqrt{d}+\sigma(\beta) +\beta)}$, if $d\equiv 1\pmod 2,b\equiv 0 \pmod 2,a+b\equiv 1 \pmod 4,a\equiv -c\pmod 4$;
		\item $\set{1,\dfrac{1}{2}(1+\sqrt{d}),\dfrac{1}{4}(1+\sqrt{d}+\sigma(\beta)+\beta),\dfrac{1}{4}(1-\sqrt{d}+\sigma(\beta) -\beta)}$, if $d\equiv 1\pmod 2,b\equiv 0 \pmod 2,a+b\equiv 1 \pmod 4,a\equiv c\pmod 4$;
		
	\end{enumerate}
	where $\beta=\sqrt{a(d-b\sqrt{d})}, \sigma(\beta)=\sqrt{a(d+b\sqrt{d})}$. 
	
\end{remark}

\begin{lemma}\label{lemindependentquartic}
	Let $\delta\in \mathcal{O}_F\backslash \mathbb{Z}$. Then $\Tr(\delta) \ne 0$ if and only if $\set{\delta, \sigma(\delta), \sigma^2(\delta ), \sigma^3(\delta)}$ is $\mathbb{R}$-linearly independent.
\end{lemma}

\begin{lemma}
	\label{lem:norm_some_ele}One has the following results. 
 \begin{align}\label{eq:norm}
	\N\tron{\sqrt{d}} = d^2,
	\N\tron{\beta }=a^2c^2d, \N\tron{\dfrac{\beta+\sigma(\beta)}{2}} = \dfrac{a^2b^2d}{4},
	\end{align}
	\begin{align}\label{eq:mul_sqrt_d}
	\beta\cdot\sqrt{d} = c \sigma(\beta)-b\beta,
	\sigma(\beta)\cdot\sqrt{d} = c\beta +b\sigma(\beta).
	\end{align}
\end{lemma}
\begin{proof}
	It is easy to verify all equalities in \eqref{eq:norm}. Hence, we only claim two equalities in \eqref{eq:mul_sqrt_d}. It is sufficient to show that $\beta\sqrt{d} = c\sigma(\beta)-b\beta$. Indeed, one has \begin{align*}
	\beta\sqrt{d}& = \dfrac{acd}{\sigma(\beta)} = c \dfrac{\beta^2+\sigma(\beta)^2}{2\sigma(\beta)}= c \dfrac{\tron{\beta+\sigma(\beta)}^2-2\sigma(\beta)\beta}{2\sigma(\beta)}= c\dfrac{\tron{\beta+\sigma(\beta)}^2}{2\sigma(\beta)}-c\beta.
	\end{align*}   Moreover, $c\dfrac{\tron{\beta+\sigma(\beta)}^2 }{2\sigma(\beta)}=c\dfrac{ad+ac\sqrt{d}}{\sqrt{ad+ab\sqrt{d}}}=c\sigma(\beta)+ (c-b)\beta$. Therefore $\beta\sqrt{d}= c\sigma(\beta)-b\beta$. 
\end{proof}
\newcommand{\gene}[1]{\langle #1\rangle }
\begin{lemma}\label{lem:quart_int_bas}
	Let $a,b,c,d,F,\beta$ in \eqref{df-polynomial-cubic} and $p$ be a prime number. Then:
	\begin{enumerate}[i)]
			\item\label{idealPi0quartic} If $p\mid d$ then $p\mathcal{O}_F= P^4 $ where $p =\gene{p,\beta}$ is then unique prime ideal of $\mathcal{O}_F$ above $p$.
			\item \label{lem:quart_int_bas_ii}  Assume that $p$ is odd, $p\nmid abcd$ and $d$ is not a quadratic residue $\pmod p$. Then $p\mathcal{O}_F$ is inert in $F$.
			 \item\label{lem:quart_int_bas_iii} Assume that $p$ is odd, $p\nmid abcd$ and $d$ is a quadratic residue $\pmod p$. Let $d = z^2 \pmod p$. \\If $ad+abz,ad-abz$ are quadratic residues $\pmod p$ and $ad+abz =t_1^2,ad-abz= t_2^2\pmod p$  then $p\mathcal{O}_F$ totally splits in $F$, i.e, $p\mathcal{O}_F=P_1P_2P_3P_4$ where $P_1 = \gene{p,\beta +t_1 }, P_2 =\gene{p,\beta-t_1 }, P_3 = \gene{p, \beta +t_2}, P_3 = \gene{p, \beta -t_2}$ are all prime ideals of $\mathcal{O}_F$ above $p$. Otherwise, $p\mathcal{O}_F=P_1P_2$ where $P_1 =  \gene{p, abz+ab\sqrt{d}}, P_2 = \gene{p, abz-ab\sqrt{d}}$ are all prime ideals above $p$.   
	\end{enumerate}
\end{lemma}
\begin{proof}
	In all the above cases, primes $p$ are not divisors of index $id_4$ (see \eqref{eq:index_quartic_cyclic}). By using the result on the decomposition of primes \cite[Theorem 4.8.13]{cohen1993course}, the prime composition of $p\mathcal{O}_F$ can be obtained by factorizing $df(x)$ over $\ZZ_p$. Note that $df(x)=  \tron{x^2-ad}^2-a^2b^2d$. \begin{enumerate}[i)]
		\item If $p\mid d$, then $df(x)=  x^4 \pmod p$ and thus $p\mathcal{O}_F = P^4$ where $P= \gene{p,\beta}$ and $P$ is a unique prime ideal above $p$.
		\item  To prove \ref{lem:quart_int_bas_ii}, it is sufficient to prove $df(x)$ is irreducible in $\ZZ_p[x]$. By contradiction, suppose that the polynomial $df(x)$ is reducible over the field $F_p$. Since $d$ is not a quadratic residue $\pmod p$, $df(x)$ has no root in $\ZZ_p$. We now claim that $df(x)$ cannot be decomposed into the product of two quadratic polynomials. Indeed, if $df(x)=  \tron{x^2+Ax+B}\tron{x^2+Cx+D}\pmod p$, then $A+C =0, B+D+AC =2ad, AC+BD= 0, BD =  a^2c^2d\pmod p$. It implies that $A=-C,C\tron{B-D}=0,BD=a^2c^2d\pmod p$. The integer $C$ must be nonzero because otherwise $BD=0=a^2c^2d$ and thus $p\mid acd$, which is the assumption that $p\nmid abcd$. From $A=-C, C\tron{B-D}=0\pmod p$ and $C$ is nonzero, one obtains $B=D\pmod p$ and thus $D^2 = BD =  a^2c^2d\pmod p$, which also contradicts the fact that $d$ is a quadratic non-residue $\pmod p$. It means $df(x)$ is irreducible over $\ZZ_p$ and hence $p\mathcal{O}_F$ is prime.
		\item One has \begin{align*}
		df(x) = \tron{x^2-ad}^2-a^2b^2z^2 =  \tron{x^2-\tron{ad-abz}}\tron{x^2-\tron{ad+abz}}\pmod p.
		\end{align*}
  If $x^2-\tron{ad-abz}, x^2-\tron{ad+abz}$ are irreducible  over $
F_p$, then  $p\mathcal{O}_F = P_1P_2$ where $P_1 = \gene{p, abz+ab\sqrt{d}},P_2 =\gene{p,abz-ab\sqrt{d}}$.
		Otherwise, $df(x) = \tron{x-t_1}\tron{x+t_1}\tron{x-t_2}\tron{x+t_2}\pmod p$. Thus $p\mathcal{O}_F = P_1P_2P_3P_4$ where $P_1 = \gene{p,\beta +t_1 }, P_2 =\gene{p,\beta-t_1 }, P_3 = \gene{p, \beta +t_2}, P_3 = \gene{p, \beta -t_2}$ are all prime ideals of $\mathcal{O}_F$ above $p$. 
	\end{enumerate}   
\end{proof}
In Lemmata \ref{lem:quartic_int_basis_divisor_index} and \ref{lem:quartic_int_basis_divisor_b}, we will consider prime divisors of the index of the field. In these cases, we cannot apply the result on the decomposition of primes \cite[Theorem 4.8.13]{cohen1993course}, instead, we can apply \cite[Proposition 6.2.1]{cohen1993course}.



\begin{lemma}	\label{lem:quartic_int_basis_divisor_index}
	Let $a,b,c,d, F, K,\beta$ as in \eqref{df-polynomial-cubic} and $p$ be a prime number. Then:
	\begin{enumerate}[i)]
		\item Assume $p\mid a$. If $d$ is a quadratic non-residue $\pmod p$ then there is a unique prime ideal $P$ above $p$ and $p\mathcal{O}_F= P^2.$ If $d$ is a quadratic residue $\pmod p$ then there are exactly two prime ideals $P_1,P_2$ above $p$ and $p\mathcal{O}_F= P_1^2P_2^2.$
		\item Assume $p\mid c$ and $p\nmid a$. If $2a$ is a quadratic non-residue $\pmod p$, then there are exactly two primes ideals $P_1,P_2$ above $p$ and  $p\mathcal{O}_F=  P_1P_2$. In this case, $P_1 = \gene{p,  ad - ab\sqrt{d}}$ and $ P_2 = \gene{p, ad +ab\sqrt{d}}$. Otherwise, let $2a = l^2\pmod p$. Then $p\mathcal{O}_F=P_1P_2P_3P_4$ where $P_1,P_2,P_3,P_4$ are all prime ideals of $\mathcal{O}_F$ above $p$, $P_1 = \gene{p, \beta - lb},P_2 = \gene{p,\beta+lb}$,$P_3P_4=  \gene{p, \beta^2}$ and each ideal $P_3$ and $P_4$ is coprime with the ideals $P_1$ and $P_2$.  
	\end{enumerate}
\end{lemma}
\begin{proof} 
\begin{enumerate}[i)]
	\item 	 We have $p|a$, hence $p|\Delta_F$ by \ref{quarticdisc}. Thus $p$ is ramified in $F$, i.e., one has that the prime decomposition of  $p\mathcal{O}_F$ is of the forms 
		$P^4,P_1^2Q P_2^2$ or $P^2$ 
	where $P,P_1,P_2$ are prime ideals above $p$ of $\mathcal{O}_F$ since $F$ is Galois. On the other hand we have $\gcd(a,d)=1$ then $p\nmid d$ and $\left(\dfrac{d}{p}\right)\ne 0$ and therefore $p$ is unramified in $K=\mathbb{Q}(\sqrt{d})$. As a result $p\mathcal{O}_F$ does not have the form $P^4$ but $ P_1^2P_2^2$ or $P^2$.
	Now, if $d$ is a quadratic residue $\pmod p$, it implies that $p$ splits in $K$. It follows that $p\mathcal{O}_F=P_1^2P_2^2$.
	In the other cases, $d$ is not a quadratic residue $\pmod p$, which implies $p$ is inert in $K$ and hence $p\mathcal{O}_F=P^2$.
 
	\item If $p\mid c$ then $d = b^2 \pmod p, b\neq 0 \pmod p$ and thus $df(x)= x^2\tron{x^2-2ad} \pmod p$. If $2a$ is a quadratic non-residue $\pmod p$ then $x^2-2ad$ is irreducible $\pmod p$. By \cite[Proposition 6.2.1]{cohen1993course}, we have $p\mathcal{O}_F = P_1P_2$ where $P_1 = \gene{p,\beta^2}, P_2 =  \gene{p, \beta^2-2ad}$ and $P_1,P_2$ are co-prime. Since $x^2-2ad$ is irreducible, $P_2$ is prime, and thus $P_1$ is also prime as $F$ is Galois. Hence there are only two prime ideals of $\mathcal{O}_F$ above $p$, namely $P_1$ and $P_2$. Similarly, considering the remaining case and by  \cite[Proposition 6.2.1]{cohen1993course}, one has that $df(x)= \tron{x-lb}\tron{x+lb}x^2$ and $p\mathcal{O}_F = P_1P_2A$ where $P_1 = \gene{p, \beta - lb},P_2 = \gene{p,\beta+lb}$, $A=  \gene{p, \beta^2}$. Moreover, \cite[Proposition 6.2.1]{cohen1993course} also yields that $P_1,P_2$ are prime and due to the Galois property of $F$, $A = P_3P_4$. 
\end{enumerate}
\end{proof}


\begin{lemma}
\label{lem:quartic_int_basis_divisor_b}
Let $a,b,c,d,F,K,\beta$ as in \eqref{df-polynomial-cubic} and $p$ be an odd prime divisor of $b$ such that $p\nmid a$. Then:
\begin{enumerate}[i)]
	\item If $a$ is a quadratic non-residue $\pmod p$, then there are at most two prime ideals above $p$ in $\mathcal{O}_F$ and $p\mathcal{O}_F$ is equal to the product of these prime ideals. 
	\item If $a$ is a quadratic residue $\pmod p$, then there are at least two prime ideals above $p$ in $\mathcal{O}_F$ and $p\mathcal{O}_F$ is equal to the product of these prime ideals.
\end{enumerate}
\end{lemma}
\begin{proof}
	One has $df(x)= \tron{x^2-ad}^2\pmod p$. We consider the first case in which $x^2-ad$ is irreducible. According to \cite[Proposition 6.2.1]{cohen1993course}, if $P$ is a prime ideal such that $P\mid p\mathcal{O}_F$, then $\N\tron{P} = p^m$ where $m\ge 2$. It implies that $p\mathcal{O}_F$ is a product of at most two prime ideals. In the remaining case, $df(x)$ is the square of the product of two linear polynomials. By using \cite[Proposition 6.2.1]{cohen1993course}, $p\mathcal{O}_F$ is product of two nontrivial coprime ideals. Hence, there are at least two prime ideals in the prime decomposition of $p\mathcal{O}_F$.     
\end{proof}
The following lemma tells us the factorization of $2\mathcal{O}_F$ when $\Delta_F$ is even. In the case where $\Delta_F$ is odd, the factorization of $2\mathcal{O}_F$ will have one of the three forms: $P$, $P_1P_2$, and $P_1P_2P_3P_4$.


\begin{lemma}
	\label{lem:prime_ideal_2} Assume that $2 \mid \Delta_F$. Then: \begin{enumerate}[i)]
		\item If $d$ is even, then there exists a unique prime ideal $P_0$ above $2$ and $\N\tron{P_0}=2$.
		\item If $d \equiv 5 \pmod 8$, then there exists a unique prime ideal $P_0$ above $2$ and $\N\tron{P_0}=4$.
		\item If $d \equiv 1 \pmod 8$, then $2\mathcal{O}_F= P_1^2P_2^2$, where $P_1,P_2$ are two distinct primes and $\N\tron{P_1}= \N\tron{P_2}=2$. 
	\end{enumerate}
	
\end{lemma}
\begin{proof}
	If $d$ is even, then by Lemma \ref{lem:quart_int_bas},i), one has that $P_0 = \langle 2,\beta\rangle $ is a unique prime ideal above 2. If $d\equiv1,5\pmod 8$, then $2$ ramifies in $F$ since $2 \mid  \Delta _F$. Thus the factorization of $2\mathcal{O}_F$ has one of the forms $ R^4, R_1^2R_2^2, R^2$ for some prime ideals $R,R_1,R_2$ above $2$ (since $F$ is Galois). Let $K = \QQ\tron{\sqrt{d}}$ then $df_K(x) = x^2 -x-\dfrac{d-1}{4}$ is a defining polynomial of $K$ and $2$ does not ramify in $K$. Hence $2\mathcal{O}_F\ne R^4$. If $d\equiv 5 \pmod 8$ then $df_K(x)$ is irreducible $\pmod 2$ and thus $2$ is inert in $K$. Hence $2\mathcal{O}_F =R^2$. If $d\equiv 1 \pmod 8$ then $df_K(x)$ is reducible $\pmod 2$ and thus $2$ splits in $K$.  Hence $2\mathcal{O}_F = P_1^2P_2^2.$ 
\end{proof}
