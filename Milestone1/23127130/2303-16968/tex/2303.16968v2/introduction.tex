
A well-rounded (WR) ideal lattice or a WR ideal is an ideal of a number field for which the associated lattice is well-rounded. WR ideal lattices  can be used to investigate various problems such as kissing numbers \cite{martinet2013perfect},  sphere packing problems \cite{JFC12, JDCS15}, and Minkowski’s conjecture  \cite{M05}. They also have a variety of applications to coding theory \cite{WR1,WR2}. 
Previously, Fukshanksy et. al. proved results on WR ideals in real quadratic fields  \cite{FP12, FHLPSW13}, and Araujo and Costa obtained results on WR lattices (but not necessarily for WR ideals) of cyclic fields with degree equal to an odd prime \cite{DC19}. Generalizing this work, Damir and Mantilla-Soler \cite{DM20} construct a parametric  family of WR sub-lattices of a tame lattice with a Lagrangian basis.  Another generalization of WR lattices are WR twists of ideal lattices which are investigated for real quadratic fields  \cite{DK19} and imaginary fields \cite{LTT22}. In \cite{S19}, it is shown that for any lattice $L$ there exists a diagonal real matrix $D$ with determinant equal to one and with positive entries
such that $DL$ is WR. Further, \cite{DGAH18} provides an analysis of some WR lattices used in wiretap channels and \cite{DKAGKH21} shows how to use WR lattices to optimize coset codes for Gaussian and fading wiretap channels.


In this paper, we investigate WR ideals of cyclic cubic and cyclic quartic fields. In the cyclic cubic case, let $F$ be a cyclic cubic field with discriminant $\Delta_F$ and Galois group $\Gal(F) = \langle \sigma \rangle$. If a prime $p$ divides $\Delta_F$, it is ramified in $F$ and $pO_F = P^3$ for a unique prime ideal $P$ and $\sigma^i(P)=P$ for $i\in\{0,1,2\}$. If $x$ is a shortest vector in $P$ and the set $\{\sigma^i(x):0\leq i\leq2\}$ is linearly independent then $P$ is WR (see Definition \ref{def:WR}). This idea is not only for prime ideals, it also works for other ideals which have norms divide $\Delta_F$, for example, ideals of the form $\prod_i P_i^{m_i}$ where $P_i$ are  ramified prime ideals and $0 <  m_i \in \mathbb{Z}$. We can also do similarly for  cyclic quartic fields with some modifications.


%The idea that we propose is to consider a cyclic cubic field $F$ with discriminant $\Delta_F$ and Galois group $\Gal(F) = \langle \sigma \rangle$. If a prime $p$ divides $\Delta_F$, it is ramified in $F$, and $pO_F$ is equal to $P^3$ for a unique prime ideal $P$ above $p$. Moreover, $\sigma^i(P) = P$ for all $i = 0, 1, 2$. By selecting a shortest $x$ in $P$, the three vectors $\{\sigma^i(x), 0 \le i \le 2\}$ are in $P$, and if they are linearly independent, then $P$ is WR (as per Definition \ref{def:WR}). This idea is not only for prime ideals, it also works for other ideals which have norms divide $\Delta_F$, for example, ideals of the form $\prod_i P_i^{m_i}$ where $P_i$ are  ramified prime ideals and $0 \le  m_i \in \mathbb{Z}$. We can also do similarly for  cyclic quartic fields with some modifications.




\textbf{Our experiment:} 
 To implement the idea outlined above, we do the following: First we find the defining polynomials of cyclic cubic and cyclic quartic fields. Using these polynomials together with Pari/GP \cite{PARI2}, we generate a list of all integral ideals of norms bounded by a certain number for each field. We then test which ideals in the list are WR by listing the shortest vectors  of each ideal, using the function \texttt{qfminim}  in Pari/GP. We check if their conjugates form a set of rank $3$ in $\RR^3$ (for the cyclic cubic case) or rank $4$ in $\RR^4$ (for the cyclic quartic case). After identifying the WR ideals we examined their properties such as the geometry of their integral bases, the coordinates of shortest vectors with respect to a given integral basis, etc., and formulated conjectures. Finally, we proved these conjectures.

%To implement the above idea, we do as follows. Firstly, we searched for the defining polynomials of cyclic cubic and cyclic quartic fields. Using these polynomials and  Pari/GP \cite{PARI2}, we generated a list of all integral ideals with norms  bounded by a certain number for each field. Then, we tested which ideals in the list are WR. To accomplish this, we listed all shortest vectors in each ideal and checked if they formed a set of rank 3 in $\RR^3$ or rank 4 in $\RR^4$, depending on whether the field is cubic or quartic. The enumeration of shortest vectors was performed by calling the function \texttt{qfminim} in Pari/GP. After identifying the WR ideals, we examined their properties, such as the geometry of their integral bases, the coordinates of shortest vectors with respect to a given integral basis, etc., and formulated conjectures. Finally, we proved these conjectures.



\textbf{Our contributions:} Our main contribution is establishing the conditions for the existence of WR ideal lattices in cyclic number fields of degrees $3$ and $4$. For cyclic quartic fields, we consider both the real and complex cases. The results can be seen in Theorems \ref{thm:main1}- \ref{thm:main6}. This is the first time such results are obtained for these classes of number fields. Further, we give families of cyclic cubic and cyclic quartic fields that admit WR ideals. We explicitly construct minimal integral bases of these ideals, which has applications in coding theory \cite{WR1, WR2}. Our other major contribution is that we provide the type decomposition of all odd primes in cyclic quartic fields (see Theorem \ref{theo:class_p}) and construct an explicit integral basis for every prime ideal (see Section \ref{sec:int_basis_ideal}).

%Our main contribution is the establishment of the conditions for the existence of WR ideal lattices for cyclic number fields of degrees 3 and 4, for cyclic quartic fields we consider both real and complex cases. The results can be seen in Theorems \ref{thm:main1},\ref{thm:theo_2}, \ref{thm:theo_3}, \ref{theo:WR_condition_PIQJ},\ref{thm:main_5} and \ref{thm:main6}. This is the first time such results have been obtained for these classes of number fields. Furthermore, we provide families of cyclic cubic and cyclic quartic fields that admit WR ideals. In addition, we explicitly construct minimal integral bases of these WR ideals, which is useful for applications in coding theory \cite{WR1, WR2}.  Another contribution of our work is that in cyclic quartic fields, we provide the types decomposition of all odd prime  (see Theorem \ref{theo:class_p}) and construct an explicit integral basis for every prime ideal (see Section \ref{sec:int_basis_ideal})


%%%%%%%%%%%%%%%%%%%%%%%%%
%The results in Theorems \ref{thm:main1}, \ref{thm:theo_3}, \ref{theo:WR_condition_PIQJ}, \ref{thm:main_5}  and the one with $3 \mid m$ in Theorem \ref{thm:theo_2} are new and have not been studied before. The WR ideals in these theorems in general are not tame and hence not mentioned in \cite{DM20}. In \cite{DC19}, WR ideals of quartic fields (in Theorems \ref{theo:WR_condition_PIQJ} and \ref{thm:main_5}) and of cyclic cubic fields with $3 \mid m$ (Theorem \ref{thm:main1}. ii, Theorems \ref{thm:theo_2} and \ref{thm:theo_3}) are not investigated, for the case cyclic cubic fields with  $3 \nmid m$,  it showed that if $\dfrac{m}{4}\le q^2\le 4m$ then $Q$ is WR \cite[Theorem 4.1]{DC19}. For the last case, we use a different technique and prove that this condition is not only sufficient but also necessary (see Theorem \ref{thm:theo_2}). The ideals in Theorem \ref{thm:main1}. i are of larger norms $m^2$ which are not in the interval $[m/4, 4m]$ and therefore are not the ones in \cite[Theorem 4.1]{DC19}.

The results in Theorems \ref{thm:main1}, \ref{thm:theo_3}, \ref{theo:WR_condition_PIQJ}, \ref{thm:main_5}, and the one in Theorem \ref{thm:theo_2} where $3 \mid m$ are new and have not been studied before. The WR ideals presented in these theorems are generally not tame and are hence not mentioned in \cite{DM20}. In \cite{DC19}, WR ideals of quartic fields (found in Theorems \ref{theo:WR_condition_PIQJ} and \ref{thm:main_5}) and of cyclic cubic fields with $3 \mid m$ (found in Theorem \ref{thm:main1}. ii, Theorems \ref{thm:theo_2} and \ref{thm:theo_3}) are not investigated. For the case of cyclic cubic fields where $3 \nmid m$, it has showed that if $\dfrac{m}{4}\le q^2\le 4m$ then $Q$ is WR \cite[Theorem 4.1]{DC19}. For this last case, we used a different technique to prove that this condition is not only sufficient but also necessary (see Theorem \ref{thm:theo_2}). Moreover, the ideals in  Theorems \ref{thm:main1}.i have larger norms,  $m^2$, which fall outside the range of $[m/4, 4m]$, and thus, they are distinct from those discussed in \cite[Theorem 4.1]{DC19}.




We remark that in this paper, all the ideals are integral, and we only consider the well-roundedness of an ideal if it is primitive. 


The following theorem regarding cyclic cubic fields can be obtained from Propositions \ref{prop:Psquare_is_WR},  \ref{prop:cubic9dividem2} and \ref{prop:P0Psquare_WR}.

\begin{theorem}\label{thm:main1}
Every cyclic cubic field $F$ has orthogonal and WR ideal lattices. In particular,  let $m$ be the conductor of $F$. Then we have the following.
\begin{enumerate}[i)]
    \item If $9\nmid m$, then the unique ideal of norm $m^2$ is orthogonal and WR.
    \item If $9\mid m$, then the unique ideal of norm $\dfrac{m^2}{27}$ is orthogonal and WR.
\end{enumerate}	
\end{theorem}
\noindent Moreover, we obtain the following theorem by combining Propositions \ref{prop:cubic9ndividem},\ref{prop:cubic9dividem1} and \ref{prop:P0PI_WR}.

\begin{theorem}
    \label{thm:theo_2}
    Let $q$ be a square-free divisor of the conductor $m$ of a cyclic cubic field $F$. There is a unique ideal $Q$ of $O_F$ such that $\textrm{N}(Q)=q.$ In this case, $Q$ is WR if and only if  $\left(\text{$\dfrac{m}{4}\le q^2\le 4m$ when $3\nmid m$}\right)$ and $\left(\text{$3\mid q,\dfrac{m}{4}\le q^2\le 4m$ when $3\mid m$}\right)$. 
\end{theorem}

\noindent When the conductor of a cyclic cubic field is divisible by $9$, we have the following result (see Proposition \ref{prop:P0PIsqPj}).

%Especially, when the conductor of a cyclic cubic field is of the form $m=9p_1p_2\cdots  p_r(r\ge 2)$, we have the following result (see \ref{prop:P0PIsqPj})
%\HT{Rewrite  ``When the conductor of a cyclic cubic field is of the form $m=9p_1p_2\cdots  p_r(r\ge 2)$, we have the following result". Also, cite the propositions/corollaries, ... that implies this theorem.}

\begin{theorem}
    \label{thm:theo_3} Let $m= 9p_1p_2\cdots p_r(r\ge 2)$ and $q,q'$ be two coprime divisors of $p_1p_2\cdots p_r.$ The unique ideal of norm $3q^2q'$ is WR if and  only if $\dfrac{m}{36}\le qq'^2 \le \dfrac{4m}{9}$.
\end{theorem}


 Combining Theorem \ref{theo:class_p},  Propositions \ref{prop:PIQ_J_d_even_notWR}, \ref{prop:bd_odd_notWR}, \ref{prop:dodd_b_ven_aplusb3_mod4},\ref{lem:WR_aplusb1_1} and \ref{lem:WR_aplusb1_2}. One obtains the below theorem.
\begin{theorem}\label{theo:WR_condition_PIQJ}
	Let $F$ be a cyclic quartic field defined by $a,b,c,d$ as in \eqref{eq:defpoly-quartic} and $p_I\mid d$, $q_J\mid a $ such that $d$ is a quadratic non-residue $\pmod q$ for each prime divisor $q$ of $q_J$. Then there are unique ideals of norm $p_I$ and $q_J$, denoted by $P_I$ and $Q_J$ respectively.  Let 
 $$\mathcal{M}=\set{ 16q_J^2d, 8|a|d,  4q_I^2d+4|a|d,  16p_I^2q_J^2, 4p_I^2q_J^2+4|a|d, 4p_I^2q_J^2+4q_J^2d}.$$
 
\noindent Then the ideal $P_IQ_J$  is WR if and only if $d\equiv 1\pmod 4$, $b\equiv 1 \pmod 2$, $a+b\equiv 1\pmod 4$ and~$p_I^2q_J^2+q_J^2d+2|a|d\le\min\mathcal{M}$.
\end{theorem}


\begin{theorem}
    \label{thm:main_5} With the notation given in Theorem \ref{theo:WR_condition_PIQJ}, the following hold.
    \begin{enumerate}[i)]
        \item  The lattice $P_I$ is WR if and only if  $d\equiv 1\pmod 4, b\equiv 0 \pmod 2, a+b\equiv 1 \pmod 4$ and one of the following conditions is satisfied.
	\begin{itemize}
		\item  $|a|=1$ and $\dfrac{1}{5}d\le p_I^2\le 5d$,
		\item $|a|=3$ and $d\le p_I^2\le 9d$,
		\item $|a|=5$ and $\dfrac{7}{3}d\le p_I^2\le 5d$.   
\end{itemize}	
\item The lattice $Q_J$ is WR if and only if $d=5, b=2 ,c = 1$ and $|a|\le q_J^2\le 5|a|.$ 
    \end{enumerate}
\end{theorem}
Note that the proof of Theorem \ref{thm:main_5} is presented after the proof of Proposition \ref{lem:WR_aplusb1_2}. 


For cyclic quartic fields $F$, considering any odd prime integer $p$, Theorem \ref{theo:class_p} provides a classification of classes of prime $p$ based on the ideal factorization of $p\mathcal{O}_F$. This can be done because its defining polynomial of $F$ (see in \eqref{eq:defpoly-quartic}) has the special form  $\tron{x^2-ad}^2 -a^2b^2d$. However, it has not been done for cyclic cubic fields since we do not know how its defining polynomial (see in \eqref{df-polynomial-cubic}) is factorized modulo an arbitrary prime. 




Let $p$ be any prime number. Based on the result of Theorem \ref{theo:class_p}, we can establish necessary and sufficient conditions on $p$ to have a unique prime ideal above $p$. Given this condition and by  Theorem \ref{thm:main_5}, we obtain the if and only if conditions for the well-roundedness of these prime ideals as below.

\begin{theorem}\label{thm:main6}
	Let $F$ be a cyclic quartic field defined by $a,b,c,d$ as in \eqref{eq:defpoly-quartic} and a prime $p$. There is a unique prime ideal of $\mathcal{O}_F$ above $p$ if and only one of the following conditions is satisfied.
	\begin{enumerate}[i)]
		\item The prime $p\mid d$.
		\item The prime $p\mid a$ and $d$ is a quadratic non-residue $\pmod p$.
		\item The prime $p\nmid abcd$ and $d$ is a quadratic non-residue $\pmod p$.
	\end{enumerate}
	Moreover, let $P$ denote the unique prime ideal of $\mathcal{O}_F$ above $p$. Then $P$ is WR if and only if the conditions in Theorem \ref{thm:main_5} are satisfied.
	% \begin{enumerate}[(i)]
 % \item If $p\mid d$, then $P$ is WR if and only if $d\equiv 1 \pmod 4, b\equiv 0 \mod 2, a+b\equiv 1\pmod 4$ and one of the following conditions is satisfied.
	% 	\begin{itemize}
	% 		\item  $|a|=1$ and $\dfrac{1}{5}d\le p^2\le 5d$,
	% 		\item $|a|=3$ and $d\le p^2\le 9d$,
	% 		\item $|a|=5$ and $\dfrac{7}{3}d\le p^2\le 5d$.   
	% 	\end{itemize}	
	% 	\item If $p\mid a$ and $d$ is a quadratic non-residue $\pmod p$, then $P$ is WR if and only if $d=5,b=2,c=1,a \equiv  3\pmod 4$ and $|a|\le p^2\le 5|a|$.
	% 	\item When $p\nmid abcd$ then $P$ is not WR.
	% \end{enumerate}
\end{theorem}

% \HT{need to add more results on the other ideals in cyclic quartic fields as well, for example, Lemma 45, Proposition 11, Corollary 4, ...)}

An explicit minimal basis of these WR ideals can be seen in above mentioned propositions and lemmata. Additionally, since $\Delta_F$ is gien in \eqref{quarticdisc}, Theorem \ref{thm:main6} also tells us that if $\mathcal{O}_F$ has only one prime ideal $P$ above given prime $p$, then $P$ is WR implies $p\mid \Delta_F$.
% \HT{is it true? should it be ``if $P$ is WR, then $p\mid \Delta_F$"? (the reverse is not true), also, add more comments about this fact. How about the ring of integers? is it ever WR?}.

The structure of this paper is as follows. Section \ref{sec:bacground} serves to provide an initial review of WR ideal lattices and their properties, defining polynomials, integral bases, discriminants, and prime factorizations of ideals in cyclic cubic and cyclic quartic fields. We then investigate WR ideals of cyclic cubic fields in Section \ref{sec:cubic} and of cyclic quartic fields in Section \ref{sec:quartic}. Finally, in Section \ref{sec:conclusion}  we provide some conclusions and a conjecture related to WR ideals of these fields for future research.