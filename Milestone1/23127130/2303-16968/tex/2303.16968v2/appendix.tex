
\section{Some results related to cyclic cubic fields}
\begin{proof} [Proof of Lemma \ref{lem:length-cubic-3divm}]\label{proof_of_lemma_17}
	Recall that $\Tr(\alpha) = \alpha + \sigma(\alpha)+\sigma^2(\alpha) = 0$. 	We have
	\begin{align*}
%	\delta &= m_1+m_2\alpha+m_3\sigma(\alpha);\\
%	\sigma(\delta) &= m_1+m_2\sigma(\alpha) +m_3(-\alpha-\sigma(\alpha)) =m_1-m_3\alpha +(m_2-m_3)\sigma(\alpha) ;\\
	\sigma^2(\delta) &= m_1-m_3\sigma(\alpha)+ (m_2-m_3)(-\alpha-\sigma(\alpha)) =m_1+(m_3-m_2)\alpha - m_2\sigma(\alpha).
	\end{align*}
	Thus 
	\begin{align*}
	\|\delta\|^2 &=\delta^2 +\sigma(\delta)^2+(\sigma^2(\delta))^2= 3m_1^2 + 2(m_2^2+m_3^2-m_2m_3)(\alpha^2+\sigma(\alpha)^2+\alpha\sigma(\alpha))\\&= 3m_1^2 +\dfrac{2m}{3}(m_2^2+m_3^2-m_2m_3).
	\end{align*}
	The last equality occurs because of the fact that \[\alpha^2+\sigma(\alpha)^2+\alpha\sigma(\alpha) 
	= -\alpha\sigma(\alpha) + (\alpha+\sigma(\alpha))^2 
	= -\alpha\sigma(\alpha) - (\alpha+\sigma(\alpha))\sigma^2(\alpha)= \dfrac{m}{3}.\]
\end{proof}
\begin{proof}[Proof of Lemma \ref{lem:exp_pI_xy}]\label{proof_of_lemma_19}
    	Let $P = P_1 \cdots P_r$. From Corollary \ref{cor:idealP_I} and from $\alpha ^2 \in P^2$, there exists integers $k,A,B$ such that $\alpha^2 = k\dfrac{m}{9}+A \alpha +B\sigma(\alpha )$. The value of $k$ is $2$ as $\Tr(\alpha )=\Tr(\sigma(\alpha)) = 0$ and $\Tr(\alpha^2)=\dfrac{2m}{3}$. By using Lemma \ref{lem:length-cubic-3divm}, one implies that $\|\alpha^2\| ^2 =  \dfrac{4m^2}{29}+\dfrac{2m}{3}(A^2-AB+B^2)$. It is easy to compute $\|\alpha\|^2 =  \dfrac{2m^2}{9}$. Therefore $A^2 -AB+B^2 = \dfrac{m}{9}$.
\end{proof}
\begin{proof}[Proof of Lemma \ref{idealcondition}]\label{proof_of_lemma_11}
    By using the coefficients of the defining polynomial of $F$ in \eqref{df-polynomial-cubic}, one has 
	\begin{equation}\label{eqtrace}
	\Tr(\alpha) = \Tr(\sigma(\alpha))=1, \Tr(\alpha^2)= \frac{2m+1}{3}  \text{  and  } \Tr(\alpha \sigma(\alpha)) = \frac{1-m}{3}.
	\end{equation}
	Note that the  set $M_\ell $ can be defined equivalently as $M_\ell=\{\delta\in O_F:\Tr (\delta)\equiv 0\pmod{\ell}\}$. Let $\delta=a_1\alpha+a_2\sigma(\alpha)+a_3\sigma^2(\alpha)\in M_\ell$. Then $a_1+a_2+a_3\equiv 0\pmod\ell$. By computation, we obtain \begin{align*}
	\Tr(\delta\alpha)&=\dfrac{1-m}{3}(a_1+a_2+a_3)+a_1m\\
	\Tr(\delta\sigma(\alpha))&=\dfrac{1-m}{3}(a_1+a_2+a_3)+a_2m\\
	\Tr(\delta\sigma^2(\alpha))&=\dfrac{1-m}{3}(a_1+a_2+a_3)+a_3m.
	\end{align*} 
	If the $\ell \mid m$, then $\Tr(\delta\alpha)=\Tr(\delta\sigma(\alpha))=\Tr(\delta\sigma^2(\alpha))\equiv 0\pmod\ell$. Thus all $\delta\alpha, \delta\sigma(\alpha), \delta\sigma^2(\alpha)$ are in $I$. Since $\{\alpha, \sigma(\alpha), \sigma^2(\alpha)\}$ is a basis of $O_F$ (Lemma \ref{integralbasis-3notdiv9}), one has that  $M_\ell$ is an ideal. 
	
	Conversely, assume that $M_\ell$ is ideal. 
	Then the element  $\alpha-\sigma(\alpha)$ has trace $0$ and hence is in  $ M_{\ell}$.  Thus $\alpha(\alpha- \sigma(\alpha)) \in M_{\ell}$ since $\alpha \in O_F$ and  $M_{\ell}$ is an ideal. Therefore  by \ref{eqtrace}, $\Tr(\alpha(\alpha- \sigma(\alpha)))= \Tr(\alpha^2) - \Tr(\alpha \sigma(\alpha)) = m\equiv 0 \mod \ell$. In other words, $\ell|m$.
\end{proof}
\begin{proof}[Proof of Lemma \ref{idealPi0}] \label{proof_of_lemma_12}
    By  using the fact that $p_i |m$ and $n_i = -3^{-1} \mod p_i$, one can factor $df(x)$ as $df(x) = (\alpha + n_i)^3 \mod p_i$. On the other hand, Lemma \ref{lem:index-cubic} says that $p_i$ does not divide the index $[O_F: \mathbb{Z}[\alpha]]$.  Therefore, one has $P_i = \langle p_i, \alpha + n_i \rangle$ by using the result on the  decomposition of primes  \cite[Theorem 4.8.13]{cohen1993course}.
	
	First $-\alpha + \sigma(\alpha) = -(\alpha+n) + (\sigma(\alpha)+n) \in P_i$ sine we have proved that $P_i = \langle p_i, \alpha + n_i \rangle$ and by the fact that $\sigma(P_i) =P_i $. The length of this element is easily seen by applying Lemma \ref{lencoeff}.
	
	Next, we compute the length of $\alpha + n_i$. By writing $\alpha + n_i = \alpha + n_i(\alpha + \sigma(\alpha) + \sigma^2(\alpha) )= (n_i+1) \alpha + n_i \sigma(\alpha) + n_i\sigma^2(\alpha)$ and applying Lemma \ref{lencoeff}, the result is obtained.
\end{proof}
\begin{proof}
    [Proof of Lemma \ref{lem:xI_y_I}]
\label{proof_of_lemma_24}Let $F=  \Q(\xi_3)$ and $\theta = A+B\xi_3$. Then $\N(\theta)= N$ and there exits ideal $\PP_1,\cdots \PP_i$ such that $\N(\PP_i)=p_i$ and $\theta \mathcal{O}_K = \PP_1\cdots \PP_r = \prod_{i\in I}\PP_i \prod_{j\notin I}\PP_j$. Since $\mathcal{O}$ is PID, then there exit $x_j+y_j\xi_3\in \mathcal{O}_K$ and $x_I+y_I\xi_3 \in\mathcal{O}_K$ such that $x_j+y_j\equiv 1\pmod 3$ for all $j\notin I,x_I+y_I\equiv 1\pmod 3$ and 
	$\PP_i = \langle \delta_i \rangle, \PP_I=\langle \delta_I\rangle$ whereas $\delta_i = x_i+y_i\xi_3$ and $\delta_I = x_I +y_I\xi_3$. It leads to the equality $\theta\mathcal{O}_K = \left(\delta_I\prod_{j\notin I}\delta_j\right)\mathcal{O}_K$ and thus there exists $\varepsilon\in \mathcal{O}_K^*$ such that $\theta\varepsilon =  \delta_I\prod_{j\notin I}\delta_j$. Let $\sigma_F(\delta_I)$ be the conjugate of $\delta_I$ over $F$. One has $\delta_I\sigma_F(\delta_I) = p_I$ and thus $\theta \varepsilon \sigma_F\tron{\delta_I} = \tron{\prod_{j\notin I}\delta_j}\tron{\delta_I\sigma_F\tron{\delta_I}} =  p_I\tron{\prod_{j\ne i}\delta_j}$. It means $\theta\sigma_F(\delta_I)\in p_I\mathcal{O}_K$. Moreover, $\theta \sigma_F(\delta_I)= Ax_I+By_I-Ay_I+(Bx_I-Ay_I)\xi_3$ and thus $Bx_I-Ay_I,Ax_I+By_I-Ay_I$ are multiples of $p_I$.\end{proof}
 \begin{proof}
     [Proof of Lemma \ref{lem:multipleofpIsq}]
\label{proof_of_lemma_25}Let $\gamma = x_I\alpha+y_I\sigma(\alpha).$ Remark that $\dfrac{m}{9}=A^2-AB+B^2$ and $ \alpha^2 = \dfrac{2m}{9}+A\alpha +B\sigma(\alpha)$. Since $\Tr(\alpha \sigma(\alpha ))=-\dfrac{n}{3}$, then we can write $ \alpha \sigma(\alpha ) =\dfrac{-m}{9}+C\alpha +D\sigma\alpha$ for some integers $C,D$. One has $\alpha^3 = \dfrac{m\alpha}{3}+\dfrac{am}{27}$ and $\alpha^3 = \dfrac{2m\alpha}{9}+A\alpha^2+B\alpha\sigma\alpha$. It implies that  \begin{align*}
	\dfrac{m\alpha}{3}+\dfrac{am}{27}= \tron{AB+BD}\sigma(\alpha)+\tron{\dfrac{2m}{9}+A^2+BC}\alpha +\tron{\dfrac{2mA}{9}-\dfrac{mB}{9}}
	\end{align*}
	and thus $AB+BD =0, \dfrac{2m}{9}+A^2+BC =\dfrac{m}{3},\dfrac{ma}{27} = \dfrac{2mA}{9}-\dfrac{mB}{9}$. Since $B$ must be nonzero, $A=-D$ and it is easy to prove $C=B-A$.
	
	It is easy to verify that \begin{align*}
	%\gamma &=  x_I\alpha+y_I\sigma(\alpha)\\
	\gamma \alpha &= \dfrac{m}{9}\tron{2x_I-y_I} +\tron{Ax_I+By_I-Ay_I}\alpha+\tron{Bx_I-Ay_I}\sigma(\alpha),\\
	\gamma\sigma(\alpha)&= \dfrac{m}{9}\tron{-x_I+2y_I}+\tron{Bx_I-Ay_I-By_I}\alpha +\tron{-Ax_I+Ay_I-By_I}\sigma\alpha.	
	\end{align*}
	%sửa lại hệ số ma trận 
	Let $M_\gamma= \begin{pmatrix}
	0&\dfrac{m}{9}\tron{2x_I-y_I}&\dfrac{m}{9}\tron{-x_I+2y_I}\\x_I&Ax_I-By_I-Ay_I&Bx_I-Ay_I-By_I\\y_I&Bx_I-Ay_I&-Ax_I+Ay_I-By_I
	\end{pmatrix}.$  
	Since all the entries in the second and third columns are multiples of $p_I$, one has that  $\det \tron{M_\gamma}$ is a multiple of $p_I^2$. Hence, $p_I^2\mid \N_{K/\Q}(\gamma)$ as $\N_{K/\Q}(\gamma) = \det(M_\gamma)$ by \cite{milne2008algebraic}.  \end{proof}













 %%%%%%%%%%%%%%%%%%%%%%%%%%%%
\section{Some results related to cyclic quartic fields}\label{appendix_B}

\begin{proof}[\it Proof of Lemma \ref{lem:q_i_d_even}]
	First, we prove that that $Q_{1i}, Q_{2i}$ are ideals. By Remark \ref{rem:integralbasis},i, it is sufficient to show $(z_k+\sqrt{d})\beta\in Q_{kj}$. Indeed, one has \begin{align*}
	(z_k+\sqrt{d})\beta = z_k\beta + \sqrt{d}\beta = z_k\beta +c\sigma(\beta)-b\beta = \tron{z_k-b}\beta +c\sigma(\beta)\in Q_{kj}  
	\end{align*}for $k=1,2$. Hence $Q_{1j},Q_{2j}$ are ideals of $\mathcal{O}_F.$ Two ideals have norm $q_j$ and thus  they are prime ideals. Moreover,  one has $\QQ\le K = \QQ(\sqrt{d})\le F$ and $Q_{kj}\cap \mathcal{O}_K = \mathfrak{q}_{kj}$. Hence $Q_{1j}, Q_{2j}$ are distinct. By Lemme \ref{lem:quartic_int_basis_divisor_index}, These ideals are the only prime ideals above $q_j$.
\end{proof}


\begin{proof}[\it Proof of Lemma \ref{lem:q_i_db_odd}]
	To prove $Q_{1j}, Q_{2j}$ is ideals, it is sufficient to prove that $\dfrac{4t_k-1+\sqrt{d}}{2}\beta\in Q_{kj}\dfrac{4t_k-1+\sqrt{d}}{2}\sigma(\beta)$ for $k=1,2$. By using Lemma \ref{lem:norm_some_ele}, we have \begin{align*}
	\dfrac{4t_k-1+\sqrt{d}}{2}\beta& = \tron{2t_k-1}\beta +\dfrac{\beta+\beta\sqrt{d}}{2}= (2t_k-1)\beta +\dfrac{\beta+c\sigma(\beta)-b\beta}{2}=\tron{2t_k-1+\dfrac{1-b}{2}}\beta +\dfrac{c}{2}\sigma(\beta)\in Q_{kj}\\
	\dfrac{4t_k-1+\sqrt{d}}{2}\beta&= (2t_k-1)\sigma(\beta)+\dfrac{\sigma(\beta)+\sigma(\beta)\sqrt{d}}{2}= \tron{2t_k-1+\dfrac{1+b}{2}}\sigma(\beta)+\dfrac{c}{2}\beta\in Q_{kj}
	\end{align*} for $k=1,2$. Hence $Q_{1j},Q_{2j}$ are ideals and thus they are prime as their norms are $q_j$. Moreover, $Q_{kj}\cap \mathcal{O}_K = \mathfrak{q}_{kj}$. Hence these ideals are distinct. By Lemma \ref{lem:quartic_int_basis_divisor_index}, $Q_{1j},Q_{2j}$ are two only prime ideals of $\mathcal{O}_F$ above $q_j$.
\end{proof}
The proof of Lemma \ref{lem:q_i_d_odd_b_even_ab3} is similar to Lemma \ref{lem:q_i_db_odd}.
\begin{proof}[\it Proof of Lemma \ref{lem:qnotquadratic_ab_1_mod_4_1}] Let $\rho_{kj}=  \dfrac{4t_k-1+\sqrt{d}-\beta-\sigma(\beta)}{4}$ and $\psi_{kj}= \dfrac{2q_j+4t_k-1+\sqrt{d}+\beta-\sigma(\beta)}{4}$. Let $\gamma_1, \gamma_2',\gamma_3',\gamma_4$ as in Remark \ref{rem:integralbasis},iv. First, we prove that $Q_{kj}$ are ideals for all $k=1,2$. To do that, it is sufficient to prove that $q_j\gamma_i', \dfrac{4t_k-1+\sqrt{d}}{2}\gamma_i', \rho_{kj}\gamma_i', \psi_{kj}\gamma_i'\in Q_{kj}$ for all $i=1,2,3,4$ and $k=1,2$. It is obvious that $q_j\gamma_i', \dfrac{4t_k-1+\sqrt{d}}{2}\gamma_i',\rho_{kj}, \psi_{kj}\in Q_{kj}, $ for all $k=1,2$ and $i=1,2$. One has \begin{align*}
		q_j\gamma_3' &= \dfrac{q_j+1-2t_k}{2}q_j+q_j\dfrac{4t_k-1+\sqrt{d}}{2}-q_j\psi_{kj}\\
		q_j\gamma_4'&= t_kq_j-q_j\rho_{kj}\\
		\dfrac{4t_k-1+\sqrt{d}}{2}\gamma_3'&= \dfrac{d-\tron{4t_k-1}^2-2q_j\tron{c+1-4t_k}}{8q_j}q_j+\dfrac{b-c-1+8t}{4}\dfrac{4t_k-1+\sqrt{d}}{2}-\dfrac{b}{2}\rho_{kj}+\dfrac{c-1-4t_k}{2}\psi_{kj}\\\dfrac{4t_k-1+\sqrt{d}}{2}\gamma_4'& =\dfrac{2bq_j-d+\tron{4t_k-1}^2}{8q_j}q_j+\dfrac{b+c+1}{4}\dfrac{4t_k-1+\sqrt{d}}{2}+\dfrac{-c+1-4t}{2}\rho_{kj}-\dfrac{b}{2}\psi_{kj}\\\rho_{kj}\gamma_2'&= \dfrac{d-\tron{4t_k-1}^2+2bq_j}{8q_j}q_j+\dfrac{-b-c+4t-1}{4}\dfrac{4t_k-1+\sqrt{d}}{2}+\dfrac{c+1}{2}\rho_{kj}+\dfrac{b}{2}\psi_{kj}\\
		\psi_{kj}\gamma_2'&= \dfrac{d-\tron{4t_k-1}^2+2q_j\tron{c+1-4t_k}}{8q_j}q_j+\dfrac{-b+c-1+2q_j+4t_k}{4}\dfrac{4t_k-1+\sqrt{d}}{2}+\dfrac{b}{2}\rho_{kj}+\dfrac{1-c}{2}\psi_{kj}\\\rho_{kj}\gamma_3'& = \dfrac{d-\tron{4k_1-1}^2-2ab+8abt-2q\tron{b+c+1+8t}}{16q_j}q_j+\dfrac{4t-ab-c-1}{4}\dfrac{4t_k+1+\sqrt{d}}{2}\\&+\dfrac{c-b+1}{4}\rho_{kj}+\dfrac{b+c+1-4t}{4}\psi_{kj}\\
		\rho_{kj}\gamma_4'&=\dfrac{(4t_k-1)^2-d-2a\tron{c+d-4ct}+4bq}{16q_j}q_j+\dfrac{b+c-ac}{4}\dfrac{4t_k+1-\sqrt{d}}{2}+\dfrac{1-c-2t_k}{2}\rho_{kj}-\dfrac{b}{2}\psi_{kj}\\
		\psi_{kj}\gamma_3'&=  \dfrac{2a\tron{c-d-4ct_k}+4q_j^2+d-\tron{4t_k-1}^2}{16q_j}q_j+\dfrac{ac+2q_j+4t_k-1}{4}\dfrac{4t_k-1+\sqrt{d}}{2}+\dfrac{-q_j-2t_k+1}{2}\psi_{kj}\\
		\psi_{kj}\gamma_4'&= \dfrac{(4t_k-1)^2-d-2ab\tron{1-4t_k}+2q_j\tron{b-c-1+4t_k}}{16q_j}q_j -\dfrac{ab-b}{4}\dfrac{4t_k-1+\sqrt{d}}{2}\\&-\dfrac{b+c+2q_j-4t_k+1}{4}\rho_{kj}+\dfrac{-b+c+1}{4}\psi_{kj}.
	\end{align*}
	It is not hard to prove all the coefficients of the above expressions are integers. Thus $Q_{1j}, Q_{2j}$ are ideals. Moreover, $Q_{kj}\cap \mathcal{O}_K = \mathfrak{q}_{kj}$ and thus $Q_{1j}\ne Q_{2j}$ and they are all prime ideals of $\mathcal{O}_F$ above $q_j$.
	\end{proof}
To prove Lemma \ref{lem:p2_delta_odd},ii), we considering the two cases $a \equiv -c \pmod{4}$ and $a \equiv c \pmod{4}$. The proofs of both cases use the same technique, thus we only prove the first case. The notations $\gamma_1',\gamma_2',\gamma_3',\gamma_4'$ are as defined in Remark \ref{rem:integralbasis}. One has 
\begin{align*}
\gamma_1'\cdot \gamma_i' &= \gamma_i', i=1,2,3,4\\\gamma_2'^2 &=\dfrac{d-1}{4}\gamma_1'+\gamma_2'\\
\gamma_2'\cdot\gamma_3' &= \dfrac{-2b+d-1}{8}\gamma_1'+\dfrac{b+c+1}{4}\gamma_2'+\dfrac{1-c}{2}\gamma_3'+\dfrac{b}{2}\gamma_4'\\\gamma_2'\cdot \gamma_4' & =  \dfrac{-d-2c-1}{8}\gamma_1'+\dfrac{-b+c+1}{4}\gamma_2'+\dfrac{b}{2}\gamma_3'+\dfrac{c+1}{2}\gamma_4'\\
\gamma_3'^2&=  \dfrac{-4b+2ac+2ad+d-1}{16}\gamma_1' +  \dfrac{b+c-ac}{4}\gamma_2'+\dfrac{-c+1}{2}\gamma_3'+\dfrac{b}{2}\gamma_4'\\ \gamma_3'\cdot\gamma_4'& = \dfrac{-2ab+2b-2c-d-1}{16}\gamma_1'+\dfrac{ab-b}{4}\gamma_2'+\dfrac{b+c+1}{4}\gamma_3'+\dfrac{-b+c+1}{4}\gamma_4'\\
\gamma_4'^2& =\dfrac{-2ac+4c+2ad+d-1}{16}\gamma_1' + \dfrac{b+ac-c}{4}\gamma_2'-\dfrac{b}{2}\gamma_3'+\dfrac{1-c}{2}\gamma_4'.
\end{align*}
Let $\delta =  z_1 \gamma_1'+z_2\gamma_2+z_3\gamma_3'+z_4\gamma_4'$ and $\psi =  t_1 \gamma_1'+t_2\gamma_2+t_3\gamma_3'+t_4\gamma_4'$ be arbitrary elements of $\mathcal{O}_F$. Then \begin{align*}
\delta\cdot\psi &= S_1 \gamma_1'+S_2\gamma_2'+S_3\gamma_3'+S_4\gamma_3'
\end{align*}where \begin{align*}
S_1 &= z_1t_1 +z_2t_2\dfrac{-2b+d-1}{4}+z_2t_3\dfrac{-2b+d-1}{8}+z_2t_4\dfrac{-d-2c-1}{8}+z_3t_2\dfrac{-b+d-1}{8}+z_3t_3\dfrac{-4b+2ac+2ad+d-1}{16}\\&+z_3t_4\dfrac{-2ab+2b-2c-d-1}{16}+z_4t_2\dfrac{-d-2c-1}{8}+z_4t_3\dfrac{-2ab+2b-2c-d-1}{16}+z_4t_4\dfrac{-2ac+4c+2ad+d-1}{16}\\
S_2&= z_1t_2+z_2t_1+z_2t_2+z_2t_3\dfrac{b+c+1}{4}+z_2t_4\dfrac{-b+c+1}{4}+z_3t_2\dfrac{b+c+1}{4}+z_3t_3\dfrac{b+c-ac}{4}+z_3t_4\dfrac{ab-b}{4}\\&+z_4t_2\dfrac{-b+c+1}{4}+z_4t_3\dfrac{ab-b}{4}+z_4t_4\dfrac{b+ac-c}{4}\\
S_3& = z_1t_3+z_2t_3\dfrac{1-c}{2}+z_2t_4\dfrac{b}{2}+z_3t_1+z_3t_2\dfrac{1-c}{2}+z_3t_3\dfrac{1-c}{2}+z_3t_4\dfrac{b+c+1}{4}+z_4t_2\dfrac{b}{2}+z_4t_3\dfrac{b+c+1}{4}+z_4t_4\dfrac{-b}{2}\\
S_4&= z_1t_4 +z_2t_3\dfrac{b}{2}+z_2t_4\dfrac{c+1}{2}+z_3t_2\dfrac{b}{2}+z_3t_3\dfrac{b}{2}+z_3t_4\dfrac{-b+c+1}{4}+z_4t_1+z_4t_2\dfrac{c+1}{2}+z_4t_3\dfrac{-b+c+1}{4}+z_4t_4\dfrac{1-c}{2}
\end{align*}
\begin{proof}[Proof of Lemma \ref{lem:p2_delta_odd},ii)]\label{proof_lem_p2}
To prove $2\mathcal{O}_F$ is prime, we claim that $\delta\cdot \psi\notin 2\mathcal{O}_F$ wherever $\delta\notin 2\mathcal{O}_F$ and $\psi \notin 2\mathcal{O}_F$. It is sufficient to claim that if the two tuples $\tron{t_1,t_2,t_3,t_4}$ and $\tron{z_1,z_2,z_3,z_4}$ are not simultaneously equal to $\tron{0,0,0,0} \pmod 2$, then $S_1,S_2,S_3$ and $S_4$ are also not simultaneously equal to $0 \pmod 2$. Since the largest denominator of $S_1,S_2,S_3,S_4$ is $16$, one can prove this by considering the integers $a, b, c, d$ modulo $32$ and verify whether $S_1, S_2, S_3, S_4 $ are all zero $\pmod 2$ or not. It is done by using any programming language.
\end{proof}