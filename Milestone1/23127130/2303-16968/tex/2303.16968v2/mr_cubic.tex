
Let $F$ be a cyclic cubic field with conductor $m$. In this section, we will find WR ideals of $F$ and compute minimal bases of these ideals. 

We denote by $P_i$ the unique prime ideal above the prime $p_i\mid m$ for each $i\ge 0$ and $\alpha$ a root of the defining polynomial $df(x)$ as in \eqref{df-polynomial-cubic}. We will fix these notations for the whole section. 

%%%%%%%%%%%%%%%%%%%%%%%%%%%%
\subsection{The case $9\nmid m$}
Let $m= p_1 \cdots p_r$ here $7 \le p_1 < p_2 \cdots < p_r$, $p_i \equiv 1 \mod 3$ for all $i$ and $r \ge 1$. In this section, we will show that: \\
1) the ideal $(P_1\cdots P_r)^2$ is an orthogonal and WR- this result has not been proven before; and\\
2) if $I\subset \{1,\cdots,r\}$, then $\prod_{i\in I}P_i$ is WR if and only if $\dfrac{m}{4}\le \tron{\prod_{i\in I}p_i}^2\le 4m$.


% We have two following lemmata. 




\begin{lemma}\label{integralbasis-3notdiv9}
	One has $\{\alpha, \sigma(\alpha), \sigma^2(\alpha)\}$ and $\{1, \alpha, \sigma(\alpha)\}$  are two integral bases of $\mathcal{O}_F$.
\end{lemma}
From now on, we will use one of the integral bases as mentioned in Lemma \ref{integralbasis-3notdiv9} depending on which one is convenient for our calculation. 


By \cite[page 166]{narkiewicz1974elementary}, also \cite[page 2]{de2017integral} and by \cite{maki2006determination}, we obtain Lemma \ref{lencoeff}.

\begin{lemma}\label{lencoeff}
	Let $z = z_1 \alpha + z_2  \sigma(\alpha)  + z_3 \sigma^2(\alpha) \in O_F$ where $z_i \in \mathbb{Z}, 1 \le i \le 3$. Then 
	$$\|z\|^2 = \Tr(z^2)= m (z_1^2 + z_2 ^2 + z_3^2) + \frac{(1-m)(z_1+z_2+z_3)^2}{3}.$$
	Moreover, one can rewrite this expression as 
	$$\|z\|^2= \frac{m}{3}\left( (z_1- z_2)^2 + (z_2-z_3)^2 + (z_3- z_1)^2 \right) + \frac{1}{3}(z_1+z_2+z_3)^2.$$
\end{lemma}


Since $\alpha$ is a root of the defining polynomial $df(x)$ (see \eqref{df-polynomial-cubic}), using \cite[Proposition 2.2]{TranPeng1}, one can show that $\Tr(\alpha)=1$ and  $\alpha$ is a shortest vector in $O_F\backslash \mathbb{Z}$ with $\|\alpha\|^2 = \frac{2m+1}{3}$. 

For $\ell \in \mathbb{Z}$ and $\ell>0$, as in \cite{DC19}, we define
$$M_{\ell} = \{z = z_1 \alpha + z_2  \sigma(\alpha)  + z_3 \sigma^2(\alpha) \in O_F:  z_1 + z_2 + z_3 \equiv 0 \mod \ell \}.$$

For all $\ell$, the set $M_{\ell}$ is a $\mathbb{Z}$-module. %Now we prove a condition so that $M_{\ell}$ is an ideal of $O_F$. 
We remark that in \cite{DC19}, it is proved that the sublattice $M_{\ell}$ of $O_F$ has index $\ell$ and it is WR if $\ell\equiv 1\pmod 3$ and $\sqrt{\dfrac{m}{4}}\le \ell \le \sqrt{4m}$. Thus if an ideal of $O_F$ of norm satisfies these conditions,  then that ideal is also WR. We prove the following. 
\begin{lemma}\label{idealcondition}
	The set $M_{\ell}$ is an ideal of $O_F$ if and only if $\ell|m$.
\end{lemma}
\begin{proof}
	See Appendix \ref{proof_of_lemma_11}.
\end{proof}

\begin{lemma}\label{idealPi0}
	Assume that $p_i = 3 n_i +1$. Then $p_i O_F = P_i^3$ where $P_i$ is the unique prime ideal above $p_i$ and $P_i = \langle p_i, \alpha + n_i \rangle$. Moreover, one has $-\alpha + \sigma(\alpha) \in P_i$,  $ \|-\alpha + \sigma(\alpha)\|^2= 2m$ and $\| \alpha + n_i\|^2= \frac{2m+p_i^2}{3}$.
\end{lemma}

\begin{proof}See Appendix \ref{proof_of_lemma_12}
\end{proof}

\begin{lemma}\label{traceP}
	We have $M_{p_i}=P_i$. As a consequence, $p_i| \Tr(z)$ for all $z \in P_i$.
\end{lemma}
\begin{proof}
	When ${p_i}\mid m$, by Lemma \eqref{idealcondition}, $M_{p_i}$ is an ideal. Moreover, it is a prime ideal above $p_i$ as its index is $p_i$. Therefore $M_{p_i} = P_i$. 
\end{proof}  

\begin{lemma}\label{lemcomph1}
	Let $m=p_1\cdots p_r(r\ge 1)$ and $9\nmid m$. Let $\rho=\alpha -\sigma(\alpha)$. Then $\rho\in P_i$ for all $i = 1,\cdots r$ and $\|\rho^2\|^2=\Tr(\rho^4) = 2m^2$. \end{lemma}
\begin{proof}
	By Lemma \ref{traceP}, we have $P_i =\langle p_i, \alpha-n_i\rangle $. The statement $g\in P_i$ is implied from equalities $\alpha-\sigma(\alpha) = (\alpha-n_i)+(\sigma(\alpha)-n_i)$ and $\sigma(P_i)=P_i,\forall i=1,\cdots , r$.
	
	Now, we compute $\|\rho^2\|^2$. First, one has \begin{align}\label{eq:gsquare}
	\|\rho^2\|^2 = (\alpha-\sigma(\alpha))^4+(\sigma(\alpha)-\sigma^2(\alpha))^4+(\sigma^2(\alpha)-\alpha)^4.
	\end{align}
	The right side of \eqref{eq:gsquare} is a symmetric polynomial in $\delta_1= \alpha+\sigma(\alpha)+\sigma^2(\alpha)=1, \delta_2 = \alpha\sigma(\alpha)+\sigma(\alpha)\sigma^2(\alpha)+\sigma^2(\alpha)\alpha=\dfrac{1-m}{3} , \delta_3= \alpha\sigma(\alpha)\sigma^2(\alpha)= \dfrac{m(a-3)+1}{27}$. Expressing it in terms of these symmetric polynomials, one implies $\|\rho^2\|^2 =  2m^2$.
\end{proof}



The following result is new and has not been studied before. We remark that our WR lattice $P$ in Proposition \ref{prop:Psquare_is_WR}  is not one of the sublattices mentioned in \cite[Theorem 4.9]{DM20} since its norm is $m^2$.

\begin{proposition}
	\label{prop:Psquare_is_WR} Let $m =  p_1\cdots p_r $ where $r\ge 1$ and let $P = P_1\cdots P_r$. Then $P^2$ is an orthogonal WR ideal lattice with a minimal basis $\{\kappa,\sigma(\kappa),\sigma^2(\kappa)\}$ where $\kappa= m-(\alpha-\sigma(\alpha))^2$.
\end{proposition}

\begin{proof}
	One has $Tr(\kappa)= m$ and $\|\kappa\|^2 = m^2$. By Lemma \ref{lemindependentcubic}, the set $\{\kappa,\sigma(\kappa),\sigma^2(\kappa)\}$ is $\mathbb{R}-$ linear independent. It is clear $m\in P^2$ and thus $\kappa\in P^2$ by Lemma \ref{idealPi0}
	
	Now, we prove $\kappa$ is a shortest vector in $P^2$. First, consider the sublattice $L=  \ZZ \kappa+\ZZ \sigma(\kappa)+\ZZ \sigma^2(\kappa)\ZZ$ of $P^2$. Remark that $\kappa+\sigma(\kappa) = \Tr(\kappa)-\sigma^2(\kappa)= (\alpha-\sigma^2(\alpha))^2$. It leads to \begin{align}\label{eq:eqintheo}
	\|\kappa\|^2 +\|\sigma(\kappa)\|^2+2\Tr(\kappa\sigma(\kappa)) =  \|(\alpha-\sigma^2(\alpha))^2\|=2m^2
	\end{align} by Lemma \ref{lemcomph1}. Since $\|\kappa\|^2 =\|\sigma(\kappa)\|^2 = m^2$, the equality in \eqref{eq:eqintheo} implies that $\Tr(\kappa\sigma(\kappa)) =0$. It follows that $\kappa,\sigma(\kappa),\sigma^2(\kappa) $ are pairwise orthogonal. As the consequence, $\det(L) = m^3 =\det(P^2)$ and $\kappa$ is a shortest vector of $L$. By Lemma \ref{sublatticeequal}, one has $L= P^2$. Thus $P^2$ is an orthogonal WR ideal lattice with a minimal basis $\{\kappa,\sigma(\kappa),\sigma^2(\kappa)\}$.
\end{proof}

Let $I$ be a non-empty subset of set $\{1,\cdots,r\}$, $p_I = \prod_{i\in I}p_i$ and $P_I=\prod_{i\in I}P_i$. As a consequence of Lemma \ref{traceP}, $P_I = M_{p_I}$. By \cite[Theorem 4.1]{de2017integral}, if $p_I\in \left[\dfrac{\sqrt{m}}{2},2\sqrt{m}\right]$ then $P_I$ is WR. Moreover, by using an independent technique with the one in proof of \cite[Theorem 4.1]{de2017integral}, we can prove a stronger result that  the condition $p_I\in \left[\dfrac{\sqrt{m}}{2},2\sqrt{m}\right]$ is not only necessary but also sufficient for $P_I$ to be WR. 

%\HT{Related to the comment of the reviewer: ``Are there infinitely many non-%equivalent WR ideal
%lattices arising from (different) cubic/quartic fields?"}

\begin{proposition}\label{prop:cubic9ndividem}
	Let $m = p_1 \cdots p_r$ be the conductor of $F$ and let $P_i$ be the prime ideals above $p_i$ for all $i=1,\cdots r$. For each nonempty subset $I$ of $\{1,\cdots ,r\}$, let $P_I = \prod_{i\in I }P_i,p_I=\prod_{i\in I }p_i $ and $n_I = \dfrac{p_I-1}{3}$. Then $P_I$ is WR if and only if $\dfrac{m}{4}\le p_I^2 \le 4m$. In this case, $P_I$ has a minimal basis $\alpha+n_I,\sigma(\alpha)+n_I, \sigma^2(\alpha)+n_I$.
\end{proposition}
\begin{proof}
	By Lemma \ref{idealPi0}, $P_i = \langle p_i, \alpha+n_i\rangle$ where $n_i = \dfrac{p_i-1}{3}$ for all $i\in I$. It implies that $\ZZ\tron{\alpha+n_I}+\ZZ \tron{\sigma(\alpha)+n_I}+\ZZ\tron{\sigma^2(\alpha)+n_I}\subset P_I$. Moreover, these lattices have the same indices in $O_F$ and thus $ P_I=\ZZ\tron{\alpha+n_I}+\ZZ \tron{\sigma(\alpha)+n_I}+\ZZ\tron{\sigma^2(\alpha)+n_I}$.
	
	Let $\delta$ be  a nonzero vector of $P_I$. There exist integers $x_1,x_2,x_3$ such that $\delta = x_1\tron{\alpha+n_I}+x_2\tron{\sigma(\alpha)+n_I}+x_3\tron{\sigma^2(\alpha)+n_I}$. %Then \begin{align*}
	%\delta &= \tron{(n_I+1)x_1+n_Ix_2+n_Ix_3}\alpha + \tron{n_Ix_1+\tron{n_I+1}x_2+n_Ix_3}\sigma(\alpha)+\tron{n_Ix_1+n_Ix_2+\tron{n_I+1}x_3}\sigma^2(\alpha).
	%\end{align*}
 By Lemma \ref{lencoeff}, we have $$\|\delta\|^2 = \dfrac{m}{3}\tron{\tron{x_1-x_2}^2+\tron{x_2-x_3}^2+\tron{x_3-x_1}^2}+\dfrac{\tron{3n_I+1}^2}{3}\tron{x_1+x_2+x_3}^2.$$ 	
	Now, we will find the minimum value of $\|\delta\|^2$ when $\delta\ne 0$. Note that $z_1z_2z_3\ne 0$. We consider all cases as below.
	\begin{enumerate}[(i)]
		\item If $z_1+z_2+z_3 =0$, then $(z_1-z_2)^2 +(z_2-z_3 )^2 +(z_3-z_1)^2\ge 2$. Here $(z_1-z_2)^2 +(z_2-z_3 )^2 +(z_3-z_1)^2$ is an even non-negative integer. If $(z_1-z_2)^2 +(z_2-z_3 )^2 +(z_3-z_1)^2\in \{2,4\}$, then two of the three numbers $z_1,z_2,z_3$ are zero. Without loss of generality, we can assume $z_1=z_2$. It implies that $z_3 =-2z_1$ and thus $(z_1-z_2)^2 +(z_2-z_3 )^2 +(z_3-z_1)^2$ is a multiple of $9$. Hence $(z_1-z_2)^2 +(z_2-z_3 )^2 +(z_3-z_1)^2\ge 6$. Therefore, $\|\delta\|^2\ge 2m$ in this case. The equality occurs if and only if $\delta \in \{\pm \tron{\alpha-\sigma(\alpha)},\pm \tron{\sigma(\alpha)-\sigma^2(\alpha)},\pm \tron{\sigma^2(\alpha)-\alpha}\}$.
		\item If $(z_1-z_2)^2 +(z_2-z_3 )^2 +(z_3-z_1)^2=0$, then $z_1=z_2=z_3=z\in \ZZ$ and thus $\delta =3zp_I$. Hence $\|\delta\|^2 \ge 3p_I^2$. The equality occurs if and only if $\delta \in \{\pm p_I\}$.
		\item If $z_1+z_2+z_3\ne 0 $ and $(z_1-z_2)^2 +(z_2-z_3)^2 +(z_3-z_1)^2\ne 0$, then $(z_1+z_2+z_3)^2 \ge 1 $ and $(z_1-z_2)^2 +(z_2-z_3 )^2 +(z_3-z_1)^2\ge 2$. Thus $\|\delta \|^2\ge \dfrac{p_I^2+2m}{3}$. The equality occurs if and only if $\delta \in \{\pm \tron{\alpha+n_I}, \pm \tron{\sigma(\alpha)+n_I},\pm \tron{\sigma^2(\alpha)+n_I}\}$. 
	\end{enumerate}
	Therefore $P_I$ is WR if and only if $\dfrac{2m+p_I^2}{3}\le \min\{ 2m,3p_I^2\}$ which is equivalent to $ \dfrac{m}{4} \le p_I^2 \le 4m$. 
\end{proof}


%%%%%%%%%%%%%%%%%%%%%%%%%%%%%%%%%
\vspace*{0.5cm}
\subsection{The case $9\mid m$}


Let $m= p_0^2  p_1 \cdots p_r$ here $p_0=3 < p_1 < p_2 \cdots < p_r$ and $r \ge 0$. For each nonempty subset $I$ of $\{1\cdots ,r\}$, we denote $P_I =  \prod_{i\in I}P_i$. In this section, we will show that: 

i) if $m= 9$, then $P_0$ is WR; 

ii) the ideal $P_0(P_1\cdots P_r)^2$ is orthogonal and WR; 

iii) if $I$ is a nonempty subset of $\{1,2\cdots,r\}$, then $P_0P_I$ is WR if and only if $\dfrac{m}{36}\le p_I^2\le \dfrac{4m}{9}$; and 

iv) if $r\ge 2$ and $I, J$ are two nonempty and disjoint subsets of $\{1,2,\cdots,r\}$, then $P_0P_I^2P_J$ is WR if and only if $\dfrac{m}{36}\le p_Ip_J^2\le \dfrac{4m}{9}$. The field $F$ is not tame, hence is  not  studied in \cite{DM20} and \cite{DC19}. Indeed, all of our results in this subsection are  new and have not been investigated before. 

By \cite{maki2006determination}, one has $\{1,\alpha, \sigma(\alpha)\}$ is an integral basis. It can be verify easily that $\|\alpha\|^2 = \frac{2m}{3}$. Thus, $\alpha$ is a shortest vector in $O_F\backslash \mathbb{Z}$  (see \cite{TranPeng1} for more details).

\begin{lemma}\label{idealPi}
	Let $m= 9 p_1 \cdots p_r$ where $r \ge 0$. Then $p_i O_F = P_i^3$ where $P_i$ is the unique prime ideal above $p_i$ and $P_0=\langle 3, \alpha -1 \rangle$ and $P_i = \langle p_i, \alpha \rangle$ for all $1 \le i \le r$.
\end{lemma}

\begin{proof}
	To compute generators for $P_i$ we can apply the decomposition of primes  \cite[Theorem 4.8.13]{cohen1993course}  since Lemma \ref{lem:index-cubic} says that $p_i$ does not divide the index $[O_F: \mathbb{Z}[\alpha]$. In other words, the result is obtained by factoring the defining polynomial $df(x)$ over the finite field $F_{p_i}$ and by using the fact that $p_i |m$, and $a \equiv 6 \mod 9$. 
\end{proof}
In case $m>9$, using Lemma \ref{idealPi} and the fact that $\frac{2m}{3}> 27 =\|3\|^2$ leads to the following.

\begin{corollary}\label{coridealPi1}
	Let $m >9$. Then the vector $\alpha$ is a shortest vector in the set $P_i\backslash \mathbb{Z}$ for all $ 1 \le i \le r$, and $\|\alpha\|^2 = \frac{2m}{3}$. In the ideal $P_0$, the element $p_0$ is shortest and $\|p_0\|^2 = 27$.
	
\end{corollary}


\begin{proposition}
	\label{prop:cubic9dividem2}
	Let $m =9$. Then $P_0$ is orthogonal and WR with a minimal basis $\{\alpha-1, \sigma(\alpha)-1,  \sigma^2(\alpha )-1\}$.
\end{proposition}


\begin{proof}
	Note that $\alpha-1 \in P_0$ and this element has trace $-3$ since $\Tr(\alpha)=0$, thus the three elements $\alpha-1, \sigma(\alpha)-1,  \sigma^2(\alpha)-1 $ are all in $P_0$ and are independent by Lemma \ref{lemindependentcubic}. To show that $P_0$ is WR, it is sufficient to show that $\alpha-1$ is shortest in $P_0$. 
	
	We have that $\|\alpha-1\|^2= \|\alpha\|^2 + \|-1\|^2 = \frac{2m}{3}+ 3 = 9$ because $\alpha$ has trace $0$. One can easily compute all the shortest vectors of  the ideal lattice $P_0$  (see the Fincke--Pohst algorithm- Algorithm 2.12 in \cite{fincke1985improved}) and verify that $\alpha-1$ is indeed shortest in $P_0$.
\end{proof}

Lemma \ref{lencoeff} cannot be applied for the case $9\mid m$. Therefore, we recalculate the length of vectors in $\mathcal{O}_F$ in this case as follows.
\begin{lemma}\label{lem:length-cubic-3divm}
	Let $\delta = m_1 + m_2 \alpha + m_3 \sigma(\alpha) \in O_F$. Then $$\|\delta\|^2= 3m_1^2+
	\dfrac{2m}{3}(m_2^2+m_3^2-m_2m_3).$$
\end{lemma}
\begin{proof}
    See Appendix \ref{proof_of_lemma_17}
\end{proof}

Next, we will claim that the $P_0P^2$ is an orthogonal and WR lattice where $P= P_1\cdots P_r$. To prove that, we need some lemmata as below.
\begin{lemma}
	\label{lem:idealP_3midm}For all $1\le i\le r$, we have $P_i = \ZZ p_i\oplus\ZZ \alpha \oplus\ZZ \sigma(\alpha)$.
\end{lemma}
\begin{proof}
	It is clear that $L_i =  \ZZ p_i\oplus\ZZ \alpha \oplus\ZZ \sigma(\alpha)$ is the sublattice of $P_i$ and $\det(L_i) =\det( P_i)$. Therefore $P_i = L_i$ by Lemm \ref{sublatticeequal}.
\end{proof}
By using the same technique as in the proof of Lemma \ref{lem:idealP_3midm}, one has the following result.
\begin{corollary}
	\label{cor:idealP_I} Let $I$ be a subset of $\{1,\cdots,r\}$. Then $P_I =  \ZZ p_I +\ZZ \alpha +\ZZ \sigma(\alpha)$. In particular, $P_1 \cdots P_r =\ZZ \dfrac{m}{9}\oplus\ZZ \alpha \oplus\ZZ \sigma(\alpha).$
\end{corollary}

\begin{proposition}\label{prop:cubic9dividem1}
	Let $I$ be a subset of $\{1,\cdots,r\}$. Then $P_I$ is not WR.
\end{proposition}
\begin{proof}
	By Corollary \ref{cor:idealP_I}, we have  $P_I =  \ZZ p_I +\ZZ \alpha +\ZZ \sigma(\alpha)$. Let $\delta \in P_I$, then $\delta = z_1p_I + z_2\alpha +z_3 \sigma(\alpha)$ where $z_1,z_2,z_3\in \ZZ$. By applying Lemma \ref{lem:length-cubic-3divm},  one obtains 
	\[\|\delta\|^2 = 3z_1^2p_I^2+\dfrac{2m}{3}(z_2^2+z_3^2-z_2z_3).\]
		Now, we will find the minimum value of $\|\delta\|^2$ when $\delta\ne 0$. We consider all cases as below.
		\begin{enumerate}
			\item If $z_1 = 0 $, then $\|\delta\|^2 \ge \dfrac{2m}{3}$ (since $z_2^2+z_3^2-z_2z_3\ge 1$), here the equality  occurs when $z_2 = 1,z_3=0$ or $z_2=0,z_3=1$, therefore $\delta \in \{\alpha,\sigma(\alpha)\}$.
			\item If $z_1\ne 0$, then $\|\delta\|^2\ge 3z_1^2p_I^2+\dfrac{2m}{3}(z_2^2+z_3^2-z_2z_3)\ge 3z_1^2p_I^2\ge 3p_I^2$, here the equality  occurs when $z_2=z_3=0,z_1=1$ and thus $\delta = p_I$.
		\end{enumerate}
	In conclusion, $\min_{\delta \ne 0}\|\delta\| \in \left\{\|\alpha\|^2, \|p_I\|^2\right\} = \left\{\dfrac{2m}{3},3p_I^2\right\}$. Note that $\dfrac{2m}{3}\ne 3p_I^2$, so in the case $\|p_I\|^2 < \|\alpha\|^2$,  we have $\pm p_I$ are the only two shortest vectors in $P_I$. Therefore $P_I$ is not WR. In another case $\|p_I\|^2 > \|\alpha\|^2$ and hence $\alpha$ is shortest in $P_I$. We will next compute  the set of all shortest vectors $L$ of $P_I$. Let $\delta \in O_F$ such that $\|\delta \|= \|\alpha\|$. Since $O_F = \mathbb{Z} \oplus \mathbb{Z}[\sigma] \cdot \alpha$ (see \cite[Proposition 2.2 and Proposition 2.3]{TranPeng1}), we can show easily that $\delta\in L=\{\pm \alpha, \pm \sigma(\alpha), \pm \sigma^2(\alpha)\}$.  Moreover one has $\Tr(\alpha) = \alpha + \sigma(\alpha) + \sigma^2(\alpha)=0$ and $\{\alpha, \sigma(\alpha),\alpha^2(\alpha)\}$ linearly dependent. Therefore, there does not exist three independent vectors from $L$. In other words, $P_I$ is not WR. 
\end{proof}



\begin{lemma}		\label{lem:exprA_B} 
	There exist integers $A,B$ such that $A^2 -AB+B^2  = \dfrac{m}{9}$ and 
	$\alpha^2 = \dfrac{2m}{9}+A\alpha +B \sigma(\alpha)$.
\end{lemma}
\begin{proof}
See Appendix \ref{proof_of_lemma_19}
\end{proof}
\begin{lemma}
	\label{lem:idealP1Pr} Let $\alpha,A,B$ be in Lemma \ref{lem:exprA_B} and let $\kappa = \dfrac{m}{9}+A\alpha +B\sigma(\alpha)$. Then $P_0 ( P_1\cdots P_r)^2 =  \ZZ \kappa \oplus\ZZ \sigma(\kappa) \oplus\ZZ\sigma^2(\kappa).$  
\end{lemma}
\begin{proof}It is clear that the two lattices $P_0 ( P_1\cdots P_r)$ and $ \ZZ \kappa \oplus\ZZ \sigma(\kappa) \oplus\ZZ\sigma^2(\kappa)$ have the same index in $O_F$ and thus it is sufficient to prove that $ \ZZ \kappa \oplus\ZZ \sigma(\kappa) \oplus\ZZ\sigma^2(\kappa)$ is a sublattice of $P_0 ( P_1\cdots P_r) $. It is obvious that $\dfrac{m}{9} \in (P_1 \cdots P_r)^2$. Since $\kappa = \alpha ^2 -\dfrac{m}{9}$, then $\kappa \in (P_1\cdots P_r)^2$. Moreover, $\kappa =  \alpha^2 -\dfrac{m}{9}= (\alpha-1 )(\alpha +1 )+(p_1 \cdots p_r -1 ) \in  P_0 $ as $P_0 =  \langle 3 ,\alpha -1 \rangle $ and $p_1 \equiv 1 \pmod3$. Hence, $\kappa \in  P_0 (P_1\cdots P_r)^2 $. As a consequence, $\sigma (\kappa), \sigma^2 (\kappa)\in  P_0(P_1\cdots P_r)^2.$   
\end{proof}

Lemma \ref{lem:idealP1Pr} gives us an integral basis of $P_0(P_1\cdots P_r)^2$. Let $\delta =  z_1 \kappa +z_2 \sigma(\kappa)+z_3\sigma^2(\kappa)\in P_0(P_1\cdots P_r)^2.$ One has \begin{align*}
\delta = \dfrac{m}{9}( z_1+z_2+z_3)+(Az_1 -Bz_2+(B-A)z_3 )\alpha + (Bz_1+(A-B)z_2-Az_3)\sigma(\alpha).
\end{align*} 
We then apply Lemma \ref{lem:length-cubic-3divm}, to obtain that 
\begin{align}\label{eq:eq_before_propP0Pr_WR}
\|\delta\|^2 = \dfrac{m^2 }{27}(z_1+z_2+z_3)^2+\dfrac{2m}{3}(A^2-AB+B^2)(z_1^2+z_2^2+z_3^2-z_1z_2-z_1z_3-z_2z_3).
\end{align}


Since $\dfrac{m}{9} = A^2 -AB+B^2$, the following result is followed.
\begin{proposition}\label{prop:P0Psquare_WR}
	The ideal $P_0(P_1\cdots P_r)^2$ is orthogonal and WR with a minimal basis 
	$\{\kappa,\sigma(\kappa),\sigma^2(\kappa)\}$ with $\kappa$ as in Lemma \ref{lem:idealP1Pr}.
\end{proposition}

\begin{proof}
	Let $\delta  \in P_0 (P_1\cdots P_r)^2$. Then there exist integers $z_1,z_2,z_3$ such that $\delta =z_1 \kappa+z_2 \sigma(\kappa)+z_3 \sigma^2 (\kappa)$ by Lemma \ref{lem:idealP1Pr}. 
	Since $\dfrac{m}{9} =A^2 -AB+B^2$, the equality in \eqref{eq:eq_before_propP0Pr_WR} implies that \begin{align*}
	\|\delta\|^2 = \dfrac{m}{9}(z_1^2 +z_2^2 +z_3^2).
	\end{align*}
	When $\delta \ne 0$, it is clear that $\|\delta\|^2 \ge \dfrac{m^2 }{9}$ as at least one of $z_1,z_2 ,z_3$ is a nonzero integer. The equality is occur if and only if $\delta \in \{\pm \kappa,\pm \sigma(\kappa),\pm  \sigma^2(\kappa)\}$. Hence $\{\pm \kappa,\pm \sigma(\kappa),\pm  \sigma^2(\kappa)\}$ is the set of all  shortest vectors of $P_0 (P_1 \cdots P_r)^2.$ Therefore, $P_0(P_1\cdots P_r)^2$ is WR. Moreover, we can verify that $\Tr\tron{\kappa \sigma\tron{\kappa}} = 0$ and thus $P_0(P_1\cdots P_r)^2$ is also orthogonal.
\end{proof}
From now on, for each nonempty subset $I$ of $\{ 1,2,\cdots , r\}$, we denote by $p_I = \prod_{i\in I }p_i$ and $P_I   =  \prod_{i\in I } P_i$.

For each $i\in \{ 1,\cdots, r\}$, let $\rho_i = p_i+\alpha +\sigma(\alpha)$. Since $\rho_i =(p_i-1 )+(\alpha -1)+(\sigma(\alpha)-1) \in P_0$ and it is clear that $p_i \in P_i$, then $\rho_i\in P_0P_i$. Hence $\ZZ \rho_i +\ZZ\sigma(\rho_i) +\ZZ\sigma^2(\rho_i)$ is a sublattice of $P_0P_i$ and this sublattice has the same determinant as the one of $P_0P_i$. Therefore $\ZZ \rho_i +\ZZ\sigma(\rho_i) +\ZZ\sigma^2(\rho_i) = P_0P_i$.


By using the same argument, we can prove the following lemma.

\begin{lemma}
	\label{lem:P0PI_ideal} Let $r\ge 1$ and $I$ be a nonempty subset of $\{ 1,\cdots ,r\}$ and let $ \rho_I = p_I +\alpha +\sigma(\alpha).$ Then $P_I = \ZZ \rho_I  \oplus\ZZ \sigma(\rho_I)\oplus\ZZ\sigma^2 (\rho_I)$. In particular, $P_0P_1\cdots P_r = \ZZ \rho \oplus\ZZ \sigma(\rho)\oplus\ZZ\sigma^2(\rho)$ where  $\rho= \dfrac{m}{9}+\alpha+\sigma(\alpha)$.  
\end{lemma}

The following proposition shows the necessary and sufficient conditions for the ideal $P_0P_I^2$ of a given subset $I$ of $\{1,2\cdots,r\}$ to be a WR lattice. \\



\begin{proposition}
	\label{prop:P0PI_WR} Let $I$ be a nonempty subset of $\{ 1,2 \cdots,r\}$. The ideal $P_0P_I$ is WR if and  only if $\dfrac{m}{36 }\le p_I^2 \le \dfrac{4m}{9}$. In this case, a minimal basis of $P_0P_I$ is $\{\rho_I,\sigma(\rho_I),\sigma^2(\rho_I)\}$ where  $\rho_I = p_I+\alpha+\sigma(\alpha)$.
\end{proposition}
\begin{proof}
	By Lemma \ref{lem:P0PI_ideal}, one has $P_I =  \ZZ \rho_I \oplus\ZZ \sigma(\rho_I)\oplus\ZZ\sigma^2 (\rho_I)$. Let $\delta =  z_1\rho_I +z_2 \sigma(\rho_I)+z_3 \sigma^2(\rho_I)$. Lemma \ref{lem:length-cubic-3divm} says that\begin{align*}
	\|\delta\|^2 =  3p_I^2 (z_1+z_2+z_3)^2+\dfrac{m}{3}\left((z_1-z_2)^2+(z_2-z_3)^2+(z_3-z_1)^2\right).
	\end{align*} 
	Now, we will find the minimum value of $\|\delta\|^2$ when $\delta\ne 0$. We consider all cases as below.
	\begin{enumerate}[(i)]
		\item If $z_1+z_2+z_3 =0$, then $(z_1-z_2)^2 +(z_2-z_3 )^2 +(z_3-z_1)^2\ge 2$. Note that $(z_1-z_2)^2 +(z_2-z_3 )^2 +(z_3-z_1)^2$ is an even negative integer. If $(z_1-z_2)^2 +(z_2-z_3 )^2 +(z_3-z_1)^2\in \{2,4\}$, then two of the three numbers $z_1,z_2,z_3$ are zero. Without loss of generality, we can assume $z_1=z_2$. It implies that $z_3 =-2z_1$ and thus $(z_1-z_2)^2 +(z_2-z_3 )^2 +(z_3-z_1)^2$ is a multiple of $9$. Hence $(z_1-z_2)^2 +(z_2-z_3 )^2 +(z_3-z_1)^2\ge 6$. Therefore, $\|\delta\|\ge 2m$ in this case. The equality occurs if and only if $\delta \in \{\pm (\alpha -\sigma(\alpha)),\pm (\alpha-\sigma^2(\alpha)), \pm (\sigma(\alpha)-\sigma^2(\alpha))\}$.
		\item If $(z_1-z_2)^2 +(z_2-z_3 )^2 +(z_3-z_1)^2=0$, then $z_1=z_2=z_3=z\in \ZZ$ and thus $\delta =3zp_I$. Hence $\|\delta\|^2 \ge 27p_I^2$. The equality occurs if and only if $\delta \in \{\pm 3p_I\}$.
		\item If $z_1+z_2+z_3\ne 0 $ and $(z_1-z_2)^2 +(z_2-z_3 )^2 +(z_3-z_1)^2\ne 0$, then $(z_1+z_2+z_3)^2 \ge 1 $ and $(z_1-z_2)^2 +(z_2-z_3 )^2 +(z_3-z_1)^2\ge 2$. Thus $\|\delta \|^2\ge 3p_I^2 +\dfrac{2m}{3}$. The equality occurs if and only if $\delta \in \{\pm g_I, \pm\sigma(g_I),\pm \sigma^2(g_I)\}$.
	\end{enumerate}
	It implies that $\min_{\delta \ne 0}\|g\|^2 = \min\set{2m,27p_I^2, 3p_I^2+\dfrac{2m}{3}}.$ Since $\Tr(\rho_I)\ne 0$, the ideal $P_0P_I$ is WR if and only if $\min_{\delta \ne 0} \|\delta\|^2 =  3p_I^2 +\dfrac{2m}{3}$. It is equivalent to the statement $ 3p_I^2+\dfrac{2m}{3}\le 2m$ and $ 3p_I^2\le 27p_I^2$. These inequalities occur if and only if $\dfrac{m}{36}\le p_I^2\le \dfrac{4m}{9}$.
 \end{proof}



 
Using Proposition \ref{prop:P0PI_WR} for $I =\{1\cdots,r\}$, we have the following result.
\begin{corollary}
	\label{cor:P0P1Pr_nowWR} Let $r\ge 1$. Then the ideal $P_0P_1\cdots P_r$ is not WR.
\end{corollary}

Let $I,J$ be two disjoint nonempty subsets of $\{1,2,\cdots ,r\}$. Now, we will claim the necessary and sufficient condition for $P_0P_I^2P_J$ to be a WR lattice (Proposition \ref{prop:P0PIsqPj}). 
\newcommand{\PP}{\mathcal{P}}
Let $\xi_3=\dfrac{-1-\sqrt{-3}}{2}$ be a primitive cube root of $1$ and $K'=  \QQ(\xi_3)$. The minimal polynomial of $\xi_3$ is $x^2+x+1$. For each $i\in \{1,\cdots ,r\}$, the polynomial $x^2+x+1\pmod{ p_i}$ has a root. It means $\mathcal{O}_{K'}$ has an ideal $\mathcal{P}_i$ of $\mathcal{O}_{K'}$ of norm $p_i$. For each subset $I$ of $\{1,\cdots,r\}$, let $\PP_I = \prod_{i\in I }\PP_i$. Then $\PP_I$ is an ideal of $\mathcal{O}_{K'}$ norm $p_I$. Moreover, since $\mathcal{O}_{K'}$ is a PID, then there exist integers $x_I,y_I$ such that $\PP_I=\langle x_I+y_I\xi_3 \rangle$ and thus $p_I=\N (x_I+y_I\xi_3)=x_I^2-x_Iy_I+y_I^2$. In other words, the following result is implied.
\begin{lemma}
	\label{lem:exp_pI_xy} For each nonempty subset $I$ of $\{1,\cdots,r\}$, there exist integers $x_I,y_I$ such that $x_I+y_I+1\equiv 0 \pmod 3$ and $		p_I = x_I^2-x_Iy_I+y_I^2.$
\end{lemma}

%One has the lemma as below.
\begin{lemma}\label{lem:xI_y_I}
	Let $r\ge 2$ and $N=p_1\cdots p_r$ where $p_i$ is a prime such that $p_i\equiv 1 \pmod 3$ for each $i\in \{1,\cdots ,r\}$. Assume that $N= A^2 -AB+B^2$ where $A,B$ are integers that $A+B+1\equiv 0 \pmod 3$. For each nonempty subset $I$ of $\{ 1,\cdots ,r\}$, let $p_I=  \prod_{i\in I} p_i$. Then there exist integers $x_I,y_I$ such that $x_I+y_I+1\equiv 0\pmod 3, p_I =x_I^2-x_Iy_I+y_I^2$ and $p_I\mid \tron{Ax_I-By_I-Ay_I},p_I\mid (Bx_I-Ay_I)$.  
\end{lemma}
\begin{proof}
	See Appendix \ref{proof_of_lemma_24}
\end{proof}

\begin{lemma}\label{lem:multipleofpIsq}
	Let $N=\dfrac{m}{9}=p_1\cdots p_r= A^2-AB+B^2$ where $A$ and $B$ as in Lemma \ref{lem:exprA_B}. With the notation in Lemma \ref{lem:xI_y_I}, one has $p_I^2 \mid \N_{K/\QQ}(x_I\alpha+y_I\sigma(\alpha))$. In particular, $x_I\alpha +y_I\sigma(\alpha)\in P_I^2$.
\end{lemma}
\begin{proof}
	See Appendix \ref{proof_of_lemma_25}
\end{proof}


\begin{proposition}
	\label{prop:P0PIsqPj}Let $r\ge 2$ and $I,J$ be two disjoint nonempty subsets of $\{ 1,2,\cdots, r\}$. The ideal $P_0P_I^2P_J$ is WR if and only if $\dfrac{m}{36}\le p_I^2p_J \le \dfrac{4m}{9}$. In this case, $P_0P_IP_J^2$ has a minimal basis $\{\kappa_{IJ},\sigma(\kappa_{IJ}),\sigma^2(\kappa_{IJ}\}$ where $\kappa_{IJ} = p_{IJ}+x_I+y_I$ and $x_I$ and $y_I$ are given in Lemma \ref{lem:xI_y_I}. 
\end{proposition}
\begin{proof}
	With $x_I, y_I$ in Lemma \ref{lem:multipleofpIsq}, one has $x_I\alpha +y_I\sigma(\alpha) \in P_I^2$. By Corollary \ref{cor:idealP_I}, $x_I\alpha +y_I\sigma(\alpha)\in P_J$. Thus $\kappa_{IJ} \in P_I^2P_J$ as $I,J$ are disjoint. Moreover, $\kappa_{IJ}\in P_0$ as $\kappa_{IJ}= (p_Ip_J-1)+(\alpha-1)x_I+(\sigma(\alpha)-1)y_I+(x_I+y_I+1),P_0 = \langle 3,\alpha-1 \rangle, \sigma(P_0)=P_0$ and $3\mid (x_I+y_I+1)$ by Lemma \ref{lem:xI_y_I}. Hence $\kappa_{IJ}\in P_0P_I^2P_J$ and thus $L_{IJ} = \ZZ \kappa_{IJ}\oplus\ZZ \sigma(\kappa_{IJ})\oplus\ZZ\sigma^2(h_{IJ})$ is a sublattice of $P_0P_I^2P_J$. It is easy to verify that $\det(L_{IJ})=\det(P_0P_I^2P_J)$ and thus $L_{IJ}=  P_0P_I^2P_J$ by Lemm \ref{sublatticeequal}.
	
	Let $\delta = z_1 \kappa_{IJ}+z_2\sigma(\kappa_{IJ})+z_3\sigma^2(\kappa_{IJ})$ be a nonzero vector of $P_0P_I^2P_J$. We can write \begin{align*}
	\delta = p_Ip_J\tron{z_1+z_2+z_3}&+\tron{x_Iz_1-y_Iz_2+\tron{y_I-x_I}z_3}\alpha +\tron{y_Iz_1+\tron{x_I-y_I}z_2-x_Iz_3}\sigma(\alpha)	\end{align*}and hence by Lemma \ref{lem:length-cubic-3divm}	
	\begin{align*}
	\|\delta\|^2 =3p_I^2p_J^2\tron{z_1+z_2+z_3}^2+ \dfrac{2m}{3}\prod_{i\in I}p_i\tron{z_1^2+z_2^2+z_3^2-z_1z_2-z_2z_3-z_1z_3}.
	\end{align*}
	By using a similar argument as the one in the proof of Proposition \ref{prop:P0PI_WR}, one has  \begin{align*}
	\min_{\delta\ne 0}\|\delta\|^2 =\min\set{27p_I^2p_J^2,2mp_I,3p_I^2p_J^2+\dfrac{2m}{3}p_I},
	\end{align*} 
and the lattice $P_0P_I^2P_J$ is WR if and only if $\min_{\delta\ne 0} \|\delta\|^2 = 3p_I^2p_J^2 +\dfrac{2m}{3}p_I$. It is equivalent to the statement $3p_I^2p_J^2 +\dfrac{2m}{3}p_I\le 27p_I^2p_J^2$, $ 3p_I^2p_J^2 +\dfrac{2m}{3}p_I\le 2mp_I$. In other words, $P_0P_I^2P_J$ is WR if and only if  $ \dfrac{m}{36} \le p_Ip_J^2\le \dfrac{4m}{9}$.
\end{proof}




% \begin{corollary}\label{cor:density-cubic}
%   There are at least $25\%$ of cyclic cubic fields with the conductor $m$ that have WR ideals of norms at most $2 \sqrt{m}$.
% \end{corollary}

% \begin{proof}
% By Proposition \ref{prop:cubic9ndividem}  (res. Proposition \ref{prop:P0PI_WR}),  $F$ has a WR ideal if $m$ (res. $m/9$) has a divisor $d$ such that $\dfrac{\sqrt{m}}{2}\le d\le 2 \sqrt{m}$ (res. $\dfrac{\sqrt{m/9}}{2}\le d \le \sqrt{m/9}/2$). Here $m$ (res. $m/9$) is squarefree. Applying inequality (46) from \cite[Proof of Theorem 1.1.]{FHLPSW13}, for $\nu = 2$, one can deduce that the number of cyclic cubic fields of which WR ideals of norms bounded by $2\sqrt{m}$ is at least  $\frac{\nu-1}{2\nu}= 25\%$. 
% \end{proof}




