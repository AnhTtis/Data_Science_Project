

%What we have done?\\
This paper investigates WR ideals of cyclic and quartic fields. We show that all cyclic cubic fields have WR ideals. 
Moreover, we present families of cyclic cubic and quartic fields of which WR ideal lattices exist and also  construct explicit minimal bases of these WR ideals. 
 





We observe that all WR ideals obtained from our experiment have norms dividing the discriminant of the field if the discriminant is odd.  Therefore, we form the following conjecture.

\textbf{Conjecture:} Let $F$ be a  cyclic cubic  or cyclic quartic field with odd discriminant. If a primitive integral ideal $I$ of $F$ is WR, then $N(I)$  divides the discriminant of $F$.  

If this conjecture holds then there are only finitely many WR ideals from these fields.

%If this conjecture is true, then there are only finitely many WR ideals from those fields. 

Note that this conjecture agrees with the observation in \cite{FHLPSW13} for real quadratic fields, and it was later proved for these fields \cite{S20}. In addition, for a cyclic quartic field $F$ of odd discriminant, the conjecture holds for the case when the ideal $I$ of $F$ is the unique prime ideal above a prime number as a result of Theorem \ref{thm:main6}.

We also remark that the conjecture does not hold for cyclic quartic fields of even discriminant. That is, there exist cyclic quartic fields with even discriminant which have WR ideals of norms that do not divide the field discriminant. 
%We also remark that the conjecture does not hold in case the cyclic quartic field has even discriminant. In other words,  there exist cyclic quartic fields of even discriminants of which WR ideals have norms that do not divide the field discriminant.
For example, the cyclic quartic field $F$ define by $(a,b,c,d=(1,2,1,5)$ has WR ideals with norms 484, 2420, 3364, and 3844 which do not divide $\Delta_F=2000$. Another remark is that this is the only case  in which a prime ideal above $2$ is WR by Proposition \ref{prop:ideal_2_WR}.

Our future research will investigate the above conjecture and WR ideals of other number fields.

%, with the exception of the case $[a,b,c,d] = [1,2,1,5],\Delta_F = 2000$ and a unique prime ideal above 2 is WR \HT{rewrite ``with the exception of the case $[a,b,c,d] = [1,2,1,5],\Delta_F = 2000$ and a unique prime ideal above 2 is WR" since it is wrong}. 

%Our future research will be studying the above conjecture and WR ideals of other number fields. % such as non-Galois cubic fields,  Galois quartic fields of Galois group $C_2 \times C_2$, quintic cyclic fields, and higher degree number fields. 


%What is our future research? 
%(Galois quartic fields of Galois group $C_2 \times C_2$,  quintic cyclic fields, non Galois cubic fields, and higher degree number fields. \\
%WR twists? (cite our quadratic paper and the paper about WR twists of the group in Finland here).\\
%\textbf{Question:} Let $F$ be a cyclic quartic field defined by $a, b, c, d$ and let $K= \mathbb{Q}(\sqrt{d})$ be a subfield of $F$. If $d$ is even, then $K$ does not have any primitive WR ideals by \cite{S20}. Assume that $d$ is odd (thus $d=b^2+c^2\equiv 1 \mod 4$) and an integral primitive ideal $I$ of $F$ is WR. Is $N_{F/K}(I)$ also WR? How about the reserve, i.e., if the ideal $J$ in $K$ is WR, is it true that the ideal $JO_F$ is also WR?\\
%\textbf{Partly answer these questions:} 
% Consider the case $5 \ne d \equiv 1 \mod 4, a+ b \equiv 1 \mod 4$, $b$ is even, $a+c \equiv 0 \mod 4$. \\
%Let $P$ be the unique prime ideal of $F$ of prime norm $p|d$ as in this paper. Assume that the ideal $P$ is WR. Then the ideal $P'= N_{F/K}(P)= p\mathbb{Z} \oplus \frac{p- \sqrt{d}}{2}\mathbb{Z}$ of $K$ is not always WR. It is only WR if $\frac{d}{3} < p^2 < 3d$ by \cite{S20}. Thus there is some WR ideal $P$ of $F$ (see Corollary \ref{cor:WR_whendne5}) that gives the idea $N_{F/K}(P)$ of $K$ is not WR. For example $[a,b,c,d]= [1, 4, 35, 1241]$, $p= 73|d$, the unique prime ideal above $P$ of $F$ is WR but the ideal  $N_{F/K}(P)$ (which is the unique prime ideal above $p$ in $K$) is not WR.\\
%Now assume that a primitive prime ideal $P'$ of $K$ is WR. Then by \cite{S20}, $p= N(P')|d$ and $\frac{d}{3} < p^2 < 3d$, 
%$P'= p \mathbb{Z} \oplus \frac{p- \sqrt{d}}{2}\mathbb{Z}$. By Corollary \ref{cor:WR_whendne5}), the unique prime ideal  $P$ of $F$ of norm $p$   is WR if one of the following conditions holds:\\
%i) $|a|=1$,\\
%ii) $|a|=3$ and $d < p^2 < 3d$,\\
%iii) $|a|=5$ and $\frac{7d}{3} < p^2 < 3d$.\\
%In other cases, $P$ is not WR.\\

%\textbf{New example:} there exists some WR ideal of norm divides $a^2d$, but not divides $a$ or $d$. For example $[a,b,c,d]=[3, 2, 23, 533]$. The field defined by these numbers has a WR ideal of norm $117= 3^2 \cdot 13$ where $a=3, d= 13 \cdot 41$.