\documentclass{article}
\usepackage[utf8]{inputenc}
\usepackage{graphicx, subfigure}	% Including figure files
\usepackage[a4paper, total={7in, 10in}]{geometry}
\usepackage{xcolor}
\usepackage{multirow} % for multicolumns
\usepackage{ccaption}

\newcommand{\teff}{T_{\rm eff}}
\newcommand{\logg}{\log g}
\newcommand{\mh}{\rm{[M/H]}}
\newcommand{\vbroad}{v_{\rm b}}
\newcommand{\kms}{\rm km.s^{-1}}
%%
% \setlength\paperwidth{210mm}%
% \setlength\paperheight{276mm}%
% \setlength\marginparwidth{4pc}
% \setlength\marginparsep{0.5pc}

% VERTICAL SPACING:
%\if@twocolumn
  \setlength\topmargin{-2pc}
%\else
  \setlength\topmargin{-4pc}
%\fi
\setlength{\headheight}{14pt}
\setlength{\headsep}   {15pt}
\setlength{\topskip}   {9pt}
\setlength{\footskip}  {22pt} %was 26 pt

\title{Supplementary material -- Constraining atmospheric parameters and surface magnetic fields with \texttt{ZeeTurbo}: an application to SPIRou spectra}
\author{}
\author{
%	P. I. Cristofari\thanks{E-mail: paul.cristofari@irap.omp.eu (IRAP)},
%	J.-F. Donati,
%	T. Masseron, 
%	P. Fouqué, 
%	C. Moutou,
%	A. Carmona,\\
%	E. Artigau,
%	E. Martioli,
%	G. Hébrard,
%	E. Gaidos,
%	X. Delfosse,
%	and the SLS consortium
	\\
	% List of institutions
%	$^{1}$Univ. de Toulouse, CNRS, IRAP, 14 av. Belin, 31400 Toulouse, France\\
%	$^{2}$Instituto de Astrofísica de Canarias, E-38205 La Laguna, Tenerife, Spain\\
%	$^{3}$Departamento de Astrofísica, Universidad de La Laguna, E-38206 La Laguna, Tenerife, Spain\\
%	$^{4}$Canada-France-Hawaii Telescope, CNRS, Kamuela, HI 96743, USA \\
%	$^{5}$Univ. Grenoble Alpes, CNRS, IPAG, F-38000 Grenoble, France\\
%	$^{6}$Universit\'e de Montréal, D\'epartement de Physique, IREX, Montréal, QC H3C 3J7, Canada  \\
%	$^{7}$Institut d’Astrophysique de Paris, CNRS, UMR 7095, Sorbonne Universit\'e, 75014 Paris, France\\
%	$^{8}$Laborat\'orio National de Astrof\'isica, 37504-364 Itajub\'a, MG, Brazil\\
%	$^{9}$Observatoire de Haute Provence, France\\
%	$^{10}$Department of earth sciences, University of  Hawai'i at M\=anoa, Honolulu HI 96822, USA \\
}

\date{}

\usepackage{subfig}
\usepackage{float}

\begin{document}

\maketitle

\section*{APPENDIX B: BEST FITS}
Figure~B1 presents the best fits obtained for AU Mic, AD Leo, EV Lac, CN Leo, PM~J18482+0741 and DS Leo for all lines used in our analysis.


\begin{figure*}
    \centering
    % \includegraphics[scale=.47]{figures/chapter6/r12.pdf}
    \includegraphics[scale=.47]{figures/plots/r0.pdf}
    \caption*{\textbf{Figure B1.} Best fit obtained for the seven stars included in our study with \texttt{ZeeTurbo}. Black points present the data. The grey solid line shows the best fit, and the blue solid blue line presents the part of the windows used for the fit. The green dotted line shows the model obtained for the same atmospheric parameters but with a zero magnetic field. The name PM~J18482+0741 was replaced by PM~J18 for better readability.}
    \label{fig:example_lines_annex}
\end{figure*}

\begin{figure*}
    \centering
    % \includegraphics[scale=.47]{figures/chapter6/r12.pdf}
    \includegraphics[scale=.47]{figures/plots/r5.pdf}
    % \caption{Same as Fig.~\ref{fig:example_lines} for additional lines.}
    \caption*{\textbf{Figure B1} -- \textit{continued}}
    % \label{fig:example_lines}
\end{figure*}

% \begin{figure*}
%     \centering
%     % \includegraphics[scale=.47]{figures/chapter6/r12.pdf}
%     \includegraphics[scale=.47]{figures/plots/r2.pdf}
%    \caption*{\textbf{Figure B1} -- \textit{continued}}
% \end{figure*}

\begin{figure*}
    \centering
    % \includegraphics[scale=.47]{figures/chapter6/r12.pdf}
    \includegraphics[scale=.47]{figures/plots/r6.pdf}
    \caption*{\textbf{Figure B1} -- \textit{continued}}
\end{figure*}

\begin{figure*}
    \centering
    % \includegraphics[scale=.47]{figures/chapter6/r12.pdf}
    \includegraphics[scale=.47]{figures/plots/r7.pdf}
    \caption*{\textbf{Figure B1} -- \textit{continued}}
\end{figure*}

\begin{figure*}
    \centering
    % \includegraphics[scale=.47]{figures/chapter6/r12.pdf}
    \includegraphics[scale=.47]{figures/plots/r8.pdf}
    \caption*{\textbf{Figure B1} -- \textit{continued}}
\end{figure*}


\begin{figure*}
	\centering
	% \includegraphics[scale=.47]{figures/chapter6/r12.pdf}
	\includegraphics[scale=.47]{figures/plots/r9.pdf}
    \caption*{\textbf{Figure B1} -- \textit{continued}}
\end{figure*}

\begin{figure*}
    \centering
    % \includegraphics[scale=.47]{figures/chapter6/r12.pdf}
    \includegraphics[scale=.47]{figures/plots/r10.pdf}
    \caption*{\textbf{Figure B1} -- \textit{continued}}
\end{figure*}

\begin{figure*}
    \centering
    % \includegraphics[scale=.47]{figures/chapter6/r12.pdf}
    \includegraphics[scale=.47]{figures/plots/r11.pdf}
    \caption*{\textbf{Figure B1} -- \textit{continued}}
\end{figure*}


\begin{figure*}
    \centering
    % \includegraphics[scale=.47]{figures/chapter6/r12.pdf}
    \includegraphics[scale=.47]{figures/plots/r12.pdf}
    \caption*{\textbf{Figure B1} -- \textit{continued}}
\end{figure*}


\begin{figure*}
    \centering
    % \includegraphics[scale=.47]{figures/chapter6/r12.pdf}
    \includegraphics[scale=.47]{figures/plots/r13.pdf}
    \caption*{\textbf{Figure B1} -- \textit{continued}}
\end{figure*}


\begin{figure*}
    \centering
    % \includegraphics[scale=.47]{figures/chapter6/r12.pdf}
    \includegraphics[scale=.47]{figures/plots/r14.pdf}
    \caption*{\textbf{Figure B1} -- \textit{continued}}
\end{figure*}


\begin{figure*}
    \centering
    % \includegraphics[scale=.47]{figures/chapter6/r12.pdf}
    \includegraphics[scale=.47]{figures/plots/r15.pdf}
    \caption*{\textbf{Figure B1} -- \textit{continued}}
\end{figure*}


\begin{figure*}
    \centering
    % \includegraphics[scale=.47]{figures/chapter6/r12.pdf}
    \includegraphics[scale=.47]{figures/plots/r16.pdf}
    \caption*{\textbf{Figure B1} -- \textit{continued}}
\end{figure*}


\begin{figure*}
    \centering
    % \includegraphics[scale=.47]{figures/chapter6/r12.pdf}
    \includegraphics[scale=.47]{figures/plots/r17.pdf}
    \caption*{\textbf{Figure B1} -- \textit{continued}}
\end{figure*}


\begin{figure*}
    \centering
    % \includegraphics[scale=.47]{figures/chapter6/r12.pdf}
    \includegraphics[scale=.47]{figures/plots/r18.pdf}
    \caption*{\textbf{Figure B1} -- \textit{continued}}
\end{figure*}

\begin{figure*}
    \centering
    % \includegraphics[scale=.47]{figures/chapter6/r12.pdf}
    \includegraphics[scale=.47]{figures/plots/r19.pdf}
    \caption*{\textbf{Figure B1} -- \textit{continued}}
\end{figure*}

\begin{figure*}
    \centering
    % \includegraphics[scale=.47]{figures/chapter6/r12.pdf}
    \includegraphics[scale=.47]{figures/plots/r20.pdf}
    \caption*{\textbf{Figure B1} -- \textit{continued}}
\end{figure*}

\begin{figure*}
    \centering
    % \includegraphics[scale=.47]{figures/chapter6/r12.pdf}
    \includegraphics[scale=.47]{figures/plots/r21.pdf}
    \caption*{\textbf{Figure B1} -- \textit{continued}}
\end{figure*}

\begin{figure*}
    \centering
    % \includegraphics[scale=.47]{figures/chapter6/r12.pdf}
    \includegraphics[scale=.47]{figures/plots/r22.pdf}
    \caption*{\textbf{Figure B1} -- \textit{continued}}
\end{figure*}

\begin{figure*}
	\centering
	% \includegraphics[scale=.47]{figures/chapter6/r12.pdf}
	\includegraphics[scale=.47]{figures/plots/r23.pdf}
    \caption*{\textbf{Figure B1} -- \textit{continued}}
\end{figure*}

\begin{figure*}
	\centering
	% \includegraphics[scale=.47]{figures/chapter6/r12.pdf}
	\includegraphics[scale=.47]{figures/plots/r24.pdf}
    \caption*{\textbf{Figure B1} -- \textit{continued}}
\end{figure*}

\begin{figure*}
	\centering
	% \includegraphics[scale=.47]{figures/chapter6/r12.pdf}
	\includegraphics[scale=.47]{figures/plots/r25.pdf}
    \caption*{\textbf{Figure B1} -- \textit{continued}}
\end{figure*}

\section*{APPENDIX C: CORNER PLOTS}
Figure~C1 presents the corner plots obtained for AU Mic, AD Leo, EV Lac, CN Leo, PM~J18482+0741 and DS Leo.

\begin{figure*}
    \centering
    % \includegraphics[scale=.47]{figures/chapter6/r12.pdf}
    \includegraphics[scale=.27]{figures/corners/corner-aumic-rt-new.pdf}
    \caption*{\textbf{Figure C1.} Corner plot presenting the posterior distribution for filling factors and atmospheric parameters obtained for AU~Mic.}
    \label{fig:example_lines_annex}
\end{figure*}

\begin{figure*}
    \centering
    % \includegraphics[scale=.47]{figures/chapter6/r12.pdf}
    \includegraphics[scale=.27]{figures/corners/corner-aumic-g-new.pdf}
    % \caption{Same as Fig.~\ref{fig:example_lines} for additional lines.}
    \caption*{\textbf{Figure C2. } Same as Fig.~C1 but with a Gaussian macroturbulence model.}
    % \label{fig:example_lines}
\end{figure*}

% \begin{figure*}
%     \centering
%     % \includegraphics[scale=.47]{figures/chapter6/r12.pdf}
%     \includegraphics[scale=.27]{figures/corners/corner-aumic-rt-noalpha.pdf}
%     % \caption{Same as Fig.~\ref{fig:example_lines} for additional lines.}
%     \caption*{\textbf{Figure C3. } Same as Fig.~C1 but with $\rm [\alpha/Fe]=0$~dex.}
%     % \label{fig:example_lines}
% \end{figure*}

% \begin{figure*}
%     \centering
%     % \includegraphics[scale=.47]{figures/chapter6/r12.pdf}
%     \includegraphics[scale=.27]{figures/corners/corner-amic-rt-logg40.pdf}
%     % \caption{Same as Fig.~\ref{fig:example_lines} for additional lines.}
%     \caption*{\textbf{Figure C4. } Same as Fig.~C1 but with $\log{g}=4.0$~dex.}
%     % \label{fig:example_lines}
% \end{figure*}

\begin{figure*}
    \centering
    % \includegraphics[scale=.47]{figures/chapter6/r12.pdf}
    \includegraphics[scale=.27]{figures/corners/corner-evlac-rt.pdf}
    % \caption{Same as Fig.~\ref{fig:example_lines} for additional lines.}
    \caption*{\textbf{Figure C5. } Same as Fig.~C1 but for EV~Lac.}
    % \label{fig:example_lines}
\end{figure*}

\begin{figure*}
    \centering
    % \includegraphics[scale=.47]{figures/chapter6/r12.pdf}
    \includegraphics[scale=.27]{figures/corners/corner-evlac-g.pdf}
    % \caption{Same as Fig.~\ref{fig:example_lines} for additional lines.}
    \caption*{\textbf{Figure C6. } Same as Fig.~C5  but with a Gaussian macroturbulence model.}
    % \label{fig:example_lines}
\end{figure*}

%% gl388
\begin{figure*}
    \centering
    % \includegraphics[scale=.47]{figures/chapter6/r12.pdf}
    \includegraphics[scale=.27]{figures/corners/corner-gl388-rt.pdf}
    % \caption{Same as Fig.~\ref{fig:example_lines} for additional lines.}
    \caption*{\textbf{Figure C7. } Same as Fig.~C1 but for AD~Leo.}
    % \label{fig:example_lines}
\end{figure*}

\begin{figure*}
    \centering
    % \includegraphics[scale=.47]{figures/chapter6/r12.pdf}
    \includegraphics[scale=.27]{figures/corners/corner-gl388-g.pdf}
    % \caption{Same as Fig.~\ref{fig:example_lines} for additional lines.}
    \caption*{\textbf{Figure C8. } Same as Fig.~C7  but with a Gaussian macroturbulence model.}
    % \label{fig:example_lines}
\end{figure*}

%% gl410
\begin{figure*}
    \centering
    % \includegraphics[scale=.47]{figures/chapter6/r12.pdf}
    \includegraphics[scale=.27]{figures/corners/corner-gl410-rt.pdf}
    % \caption{Same as Fig.~\ref{fig:example_lines} for additional lines.}
    \caption*{\textbf{Figure C9. } Same as Fig.~C1 but for DS~Leo.}
    % \label{fig:example_lines}
\end{figure*}

\begin{figure*}
    \centering
    % \includegraphics[scale=.47]{figures/chapter6/r12.pdf}
    \includegraphics[scale=.27]{figures/corners/corner-gl410-g.pdf}
    % \caption{Same as Fig.~\ref{fig:example_lines} for additional lines.}
    \caption*{\textbf{Figure C10. } Same as Fig.~C9  but with a Gaussian macroturbulence model.}
    % \label{fig:example_lines}
\end{figure*}

%% gl406
\begin{figure*}
    \centering
    % \includegraphics[scale=.47]{figures/chapter6/r12.pdf}
    \includegraphics[scale=.27]{figures/corners/corner-gl406-rt.pdf}
    % \caption{Same as Fig.~\ref{fig:example_lines} for additional lines.}
    \caption*{\textbf{Figure C11. } Same as Fig.~C1 but for CN~Leo.}
    % \label{fig:example_lines}
\end{figure*}

\begin{figure*}
    \centering
    % \includegraphics[scale=.47]{figures/chapter6/r12.pdf}
    \includegraphics[scale=.27]{figures/corners/corner-gl406-g.pdf}
    % \caption{Same as Fig.~\ref{fig:example_lines} for additional lines.}
    \caption*{\textbf{Figure C12. } Same as Fig.~C11  but with a Gaussian macroturbulence model.}
    % \label{fig:example_lines}
\end{figure*}

%% PM
\begin{figure*}
    \centering
    % \includegraphics[scale=.47]{figures/chapter6/r12.pdf}
    \includegraphics[scale=.27]{figures/corners/corner-pm-rt.pdf}
    % \caption{Same as Fig.~\ref{fig:example_lines} for additional lines.}
    \caption*{\textbf{Figure C13. } Same as Fig.~C1 but for PM~J18482+0741.}
    % \label{fig:example_lines}
\end{figure*}

\begin{figure*}
    \centering
    % \includegraphics[scale=.47]{figures/chapter6/r12.pdf}
    \includegraphics[scale=.27]{figures/corners/corner-pm-g.pdf}
    % \caption{Same as Fig.~\ref{fig:example_lines} for additional lines.}
    \caption*{\textbf{Figure C14. } Same as Fig.~C13  but with a Gaussian macroturbulence model.}
    % \label{fig:example_lines}
\end{figure*}


\end{document}
