\documentclass[letterpaper, 10 pt, conference]{ieeeconf}  % Comment this line out if you need a4paper

%\documentclass[a4paper, 10pt, conference]{ieeeconf}      % Use this line for a4 paper

\IEEEoverridecommandlockouts                              % This command is only needed if 
% you want to use the \thanks command

\overrideIEEEmargins                                      % Needed to meet printer requirements.

%In case you encounter the following error:
%Error 1010 The PDF file may be corrupt (unable to open PDF file) OR
%Error 1000 An error occurred while parsing a contents stream. Unable to analyze the PDF file.
%This is a known problem with pdfLaTeX conversion filter. The file cannot be opened with acrobat reader
%Please use one of the alternatives below to circumvent this error by uncommenting one or the other
%\pdfobjcompresslevel=0
\pdfminorversion=4

% See the \addtolength command later in the file to balance the column lengths
% on the last page of the document

% The following packages can be found on http:\\www.ctan.org
%\usepackage{graphics} % for pdf, bitmapped graphics files
%\usepackage{epsfig} % for postscript graphics files
%\usepackage{mathptmx} % assumes new font selection scheme installed
%\usepackage{times} % assumes new font selection scheme installed
%\usepackage{amsmath} % assumes amsmath package installed
%\usepackage{amssymb}  % assumes amsmath package installed


\usepackage{color,soul}
\usepackage{amsmath}
\usepackage{multirow}
\usepackage{graphicx}
\usepackage{booktabs}
\usepackage{lineno}
\usepackage{hyperref}

%\usepackage[default]{roboto}  
% Amazon Warehouse Dataset: A large-scale, object-centric dataset for Robotic Manipulation in Warehouses
\title{\LARGE \bf
	{ARMBench}: An Object-centric Benchmark Dataset\\ for Robotic Manipulation
}

\author{Chaitanya Mitash$^{1}$, Fan Wang$^{1}$, Shiyang Lu$^{2}$, Vikedo Terhuja$^{1}$,\\ Tyler Garaas$^{1}$, Felipe Polido$^{1}$, Manikantan Nambi$^{1}$% <-this % stops a space
\thanks{$^{1}$Amazon Robotics, MA, USA. \{cmitash, fanwanf, terhuja, tggaraas, polidof, mnambi\}@amazon.com}%
\thanks{$^{2}$Computer Science Department, Rutgers University, NJ, USA. shiyang.lu@rutgers.edu. Work done during a co-op at Amazon Robotics.}%
}
\begin{document}
	\maketitle

	%%%%%%%%% ABSTRACT
	\begin{abstract}
	This paper introduces Amazon Robotic Manipulation Benchmark ({ARMBench}), a large-scale, object-centric benchmark dataset for robotic manipulation in the context of a warehouse. Automation of operations in modern warehouses requires a robotic manipulator to deal with a wide variety of objects, unstructured storage, and dynamically changing inventory. Such settings pose challenges in perceiving the identity, physical characteristics, and state of objects during manipulation. Existing datasets for robotic manipulation consider a limited set of objects or utilize 3D models to generate synthetic scenes with limitation in capturing the variety of object properties, clutter, and interactions. We present a large-scale dataset collected in an Amazon warehouse using a robotic manipulator performing object singulation from containers with heterogeneous contents. ARMBench contains images, videos, and metadata that corresponds to 235K+ pick-and-place activities on 190K+ unique objects. The data is captured at different stages of manipulation, i.e., pre-pick, during transfer, and after placement. Benchmark tasks are proposed by virtue of high-quality annotations and baseline performance evaluation are presented on three visual perception challenges, namely 1) object segmentation in clutter, 2) object identification, and 3) defect detection. ARMBench can be accessed at \href{http://armbench.com/}{http://armbench.com}
	%\vspace{-0.1in}
	% This paper introduces the Amazon Warehouse (AWARE) dataset, a large-scale dataset and benchmarks for robotic manipulation of warehouse objects. Automating operations in modern warehouses will require a robotic manipulator to deal with a clutter of heterogeneous objects with a wide variety of physical properties on a continuously changing inventory. Such interactions pose challenges in sensing the identity, physical characteristics, and state of objects before, during, and after manipulation. Existing datasets consider a limited set of objects or utilize 3D object models to generate synthetic scenes. Accurately modeling the variety of object properties, clutter, and interactions can be challenging. We present a large-scale dataset collected in an Amazon warehouse using a robotic manipulator performing object singulation from a heterogeneous pile. The AWARE dataset contains images, videos and metadata corresponding to 185+ pick activities over 150K+ unique objects. The images are captured at different stages of manipulation, i.e. pre-pick, during transfer, and after placement. Benchmark tasks are proposed by virtue of high-quality annotations and baseline performance evaluation on three visual perception challenges, namely object segmentation in clutter, object identification, and defect detection. The dataset, annotations, and benchmark tasks can be accessed here: link here
	%This paper introduces the Amazon Warehouse (AWARE) dataset, a large-scale dataset that captures the challenges of robotic manipulation within a modern warehouse setting. Many operations in warehouses require a robotic manipulator to deal with a heterogeneous clutter of objects with a wide variety of physical properties on a continuously changing inventory. Such interactions pose complex challenges in terms of sensing the identity, characteristics and state of objects before, during and after manipulation. Existing datasets either consider a limited set of objects and operate under a closed-set assumption or utilize 3d object models to generate synthetic scenes. Accurately modeling the variety of object properties, clutter configurations and interactions can be challenging. To that end, a large-scale dataset is collected within an Amazon warehouse using a robotic manipulator singulating objects from a heterogeneous pile with pick-and-place operations. The AWARE dataset contains images, videos and metadata corresponding to 185+ pick activities over 150K+ unique objects. The images are captured at different stages of manipulation, i.e. pre-pick, transfer and after placement. Benchmark tasks are proposed by virtue of high-quality annotations and baseline performance evaluation on three visual perception challenges, namely object segmentation in clutter, object identification and defect detection. The dataset, annotations and benchmark tasks are anticipated to be continually growing and can be accessed here: link here
	\end{abstract}

	\section{Introduction}
	\label{sec:introduction}
	\section{Introduction}
\label{sec:introduction}
% \begin{itemize}
%     % Diffusion of FL
%     \item {\st{Diffusion of FL}}
%     % Security threats to FL
%     \item {\st{Security threats to FL with particular focus on model poisoning}}
%     % Limitations of existing countermeasures
%     \item {\st{Current countermeasures (e.g., KRUM) and their limitations}}
%     % Proposed method and its advantages
%     \item {\st{Intuitive description of the proposed method and its difference (i.e., advantages) w.r.t. state of the art}}
%     % Main contributions
%     \item {\st{Summary of the main contributions of this work}}
%     % Paper's structure and organization
%     \item {\st{Paper's structure and organization}}
% \end{itemize}

% Diffusion of FL
Recently, {\em federated learning} (FL) has emerged as the leading paradigm for training distributed, large-scale, and privacy-preserving machine learning (ML) systems~\cite{mcmahan2017googleai,mcmahan2017aistats}. 
The core idea of FL is to allow multiple edge clients to collaboratively train a shared, global model without disclosing their local private training data.
%Specifically, an FL system consists of a central server and many edge clients; 
A typical FL round involves the following steps: {\em(i)} the server randomly picks some clients and sends them the current, global model; {\em(ii)} each selected client locally trains its model with its own private data; then, it sends the resulting local model to the server;\footnote{Whenever we refer to global/local model, we mean global/local model {\em parameters}.} {\em(iii)} the server updates the global model by computing an \emph{aggregation function}, usually the average (FedAvg), on the local models received from clients.
% \begin{enumerate}
%     \item[{\em(i)}] the server sends the current, global model to the clients and appoints some of them for training;
%     \item[{\em(ii)}] each selected client locally trains its copy of the global model with its own private data; then, it sends the resulting local model back to the server;\footnote{Whenever we refer to global/local model, we mean global/local model {\em parameters}.}
%     \item[{\em(iii)}] the server updates the global model by computing an \emph{aggregation function} on the local models received from clients (by default, the average, also referred to as FedAvg~\cite{mcmahan2017aistats}).
% \end{enumerate}
This process goes on until the global model converges. %(e.g., after a certain number of rounds or other similar stopping criteria).
%\\
% The advantages of FL over the traditional, centralized learning paradigm are undoubtedly clear in terms of flexibility/scalability (clients can join/disconnect from the FL network dynamically), network communications (only model weights\footnote{We will use \textit{parameters} and \textit{weights} interchangeably.} are exchanged between clients and server), and privacy (each client's private training data is kept local at the client's end and not uploaded to the server).
\\
% Security threats to FL
%However, the growing adoption of FL also raises security concerns~\cite{costa2022covert}, particularly about its confidentiality, integrity, and availability.
Although its advantages over standard ML, FL also raises security concerns~\cite{costa2022covert}. %, particularly about its confidentiality, integrity, and availability~\cite{costa2022covert}.
% OLD, LONG VERSION
% Indeed, some work deals with privacy leakage that may expose the local data of some clients~\cite{melis2019sp}. 
% A large body of work, instead, investigates attacks that usually aim to detriment the predictive accuracy of the learned global model. For instance, \emph{data poisoning} attacks achieve this goal by letting an adversary pollute the training set of some corrupt FL clients with maliciously crafted examples~\cite{jagielski2018sp}.
% Similarly, in \emph{model poisoning} the attacker attempts to tweak the global model weights~\cite{bhagoji2019pmlr} by directly perturbing the local model's weights of some infected FL clients before these are sent to the central server for aggregation, usually via so-called Byzantine attacks. 
% It turns out that Byzantine model poisoning attacks severely impact standard FedAvg; therefore, more robust aggregation functions must be designed to make FL systems secure.
Here, we focus on \emph{untargeted model poisoning} attacks~\cite{bhagoji2019pmlr}, where an adversary attempts to tweak the global model weights %\footnote{We will use the terms \textit{parameters} and \textit{weights} interchangeably.} 
by directly perturbing the local model's parameters of some infected clients before these are sent to the central server for aggregation.
In doing so, the adversary aims to jeopardize the global model \textit{indiscriminately} at inference time.
Such model poisoning attacks severely impact standard FedAvg; therefore, more robust aggregation functions must be designed to secure FL systems.
\\
% In this paper, we focus on designing a novel robust aggregation scheme at the server's end to contrast the effect of Byzantine model poisoning attacks.
%
% Current countermeasures and their limitations
%Several countermeasures have been proposed in the literature to combat model poisoning attacks on FL systems.
% Some methods use simple statistics more robust than plain average to smooth the impact of malicious updates (e.g., Trimmed Mean and FedMedian~\cite{yin2018icml}). 
% Other defenses implement outlier detection techniques to discard malicious updates from the aggregation performed at the server's end. Those are either based on heuristics (e.g., Krum/Multi-Krum~\cite{blanchard2017nips} and Bulyan~\cite{mhamdi2018pmlr}) or data-driven approaches (e.g., K-means clustering~\cite{shen2016acm} or DnC via spectral analysis~\cite{shejwalkar2021ndss}). 
% Finally, some strategies rely on a centralized ``source of trust'' to spot potential malicious updates (e.g., FLTrust~\cite{cao2020fltrust}).
% Several countermeasures have been proposed in the literature to combat model poisoning attacks on FL systems, i.e., to discard possible malicious local updates from the aggregation performed at the server's end. 
% These techniques range from simple statistics more robust than plain average (e.g., Trimmed Mean and FedMedian~\cite{yin2018icml}) to outlier detection heuristics (e.g., Krum/Multi-Krum~\cite{blanchard2017nips} and Bulyan~\cite{mhamdi2018pmlr}) or data-driven approaches (e.g., spectral analysis via K-means clustering~\cite{shen2016acm} or spectral analysis), or methods based on ``source of trust'' (e.g., FLTrust~\cite{cao2020fltrust}).
% OLD, LONG VERSION
%Several countermeasures have been proposed in the literature to combat Byzantine model poisoning attacks on FL systems.
% Descriptive statistics
% For example, Trimmed Mean and FedMedian aggregate local model updates using more robust statistics than standard average~\cite{yin2018icml}.
%
% % Heuristics for outlier detection
% Many existing Byzantine-resilient strategies implement some outlier detection heuristics to discard the model updates sent by potentially malicious clients from the input of the aggregation function.
% One of the most popular heuristics is Krum~\cite{blanchard2017nips}.
% This strategy tries to mitigate the impact of Byzantine attacks by selecting as a global model the local model with the smallest sum of Euclidean distances to {\em all} the other local models.
% Although powerful, Krum requires the server to know (or, at least, estimate) the number of malicious FL clients upfront, which is generally impossible in a realistic attack scenario. %
% Moreover, Krum may become ineffective for complex, high-dimensional model parameter spaces due to the curse of dimensionality.
% Bulyan~\cite{mhamdi2018pmlr} tries to overcome this issue by combining Krum with a variant of Trimmed Mean.
% % Data-driven outlier detection
% Other strategies use data-driven outlier detection techniques -- e.g., via K-means clustering~\cite{shen2016acm} -- to spot potential malicious local model updates. 
% %For instance, Shen et al. propose to cluster local model updates with K-means and thus identify outliers.
%
% % Other techniques
% As far as the server is concerned, any local model received can be from a potential malicious client. 
% FLTrust~\cite{cao2020fltrust} assumes the server acts as a client, i.e., trains a local model on an additional {\em trustworthy} dataset at the server's end and compares it against all the local models from other clients. 
% This way, the server can rely on some ``source of trust'' when discarding potentially malicious clients.
%\\
% Limitations of existing Byzantine-resilient strategies
Unfortunately, existing defense mechanisms either rely on simple heuristics (e.g., Trimmed Mean and FedMedian by~\cite{yin2018icml}) or need strong and unrealistic assumptions to work effectively (e.g., foreknowledge or estimation of the number of malicious clients in the FL system, as for Krum/Multi-Krum~\cite{blanchard2017nips} and Bulyan~\cite{mhamdi2018pmlr}, which, however, cannot exceed a fixed threshold).
Furthermore, outlier detection methods using K-means clustering~\cite{shen2016acm} or spectral analysis like DnC~\cite{shejwalkar2021ndss} do not directly consider the temporal evolution of local model updates received.
Finally, strategies like FLTrust~\cite{cao2020fltrust} require the server to collect its own dataset and act as a proper client, thereby altering the standard FL protocol.
\\
% OLD, LONG VERSION
% Overall, existing Byzantine-resilient strategies are either simple heuristics (e.g., FedMedian) or, if they are more complex, they rely on strong and unrealistic assumptions to work effectively (e.g., knowing the number of malicious clients in the FL system in advance, as for Krum and alike).
% Furthermore, data-driven outlier detection methods do not consider the temporary evolution of local model updates received (e.g., K-means clustering). 
% Finally, strategies like FLTrust requires the server to collect its own dataset and act as a proper client, thereby altering the standard FL protocol.
%
% Description of the proposed method
This work introduces a novel pre-aggregation \textit{filter} robust to untargeted model poisoning attacks. Notably, this filter $(i)$ operates without requiring prior knowledge or constraints on the number of malicious clients and $(ii)$ inherently integrates temporal dependencies. 
The FL server can employ this filter as a preprocessing step before applying \textit{any} aggregation function, be it standard like FedAvg or robust like Krum or Bulyan.
Specifically, we formulate the problem of identifying corrupted updates as a multidimensional (i.e., matrix-valued) time series anomaly detection task. 
The key idea is that legitimate local updates, resulting from well-calibrated iterative procedures like stochastic gradient descent (SGD) with an appropriate learning rate, show \textit{higher predictability} compared to malicious updates. This hypothesis stems from the fact that the sequence of gradients (thus, model parameters) observed during legitimate training exhibit regular patterns, as validated in Section~\ref{subsec:intuition}. %until convergence. 
%This regularity may be more pronounced for smooth convex loss functions, but it can still be captured within an appropriate time window, even for more complex and convoluted loss surfaces. 
%We provide evidence of this claim in Appendix~B, where we show that the average mutual information (i.e., ``predictability''), calculated over pairs of legitimate model updates sent at different FL rounds, is significantly higher than the corresponding computation for a malicious client.
\\
Inspired by the matrix autoregressive (MAR) framework for multidimensional time series forecasting~\cite{chen2021je}, we propose the FLANDERS ({\em \textbf{F}ederated \textbf{L}earning meets \textbf{AN}omaly \textbf{DE}tection for a \textbf{R}obust and \textbf{S}ecure}) filter.
The main advantages of FLANDERS over existing strategies like FLDetector~\cite{zhao2020multivariate} are its resilience to large-scale attacks, where $50\%$ or more FL participants are hostile, and the capability of working under realistic non-iid scenarios.
We attribute such a capability to two key factors: $(i)$ FLANDERS works without knowing a priori the ratio of corrupted clients, and $(ii)$ it embodies temporal dependencies between intra- and inter-client updates, quickly recognizing local model drifts caused by evil players. Below, we summarize our main contributions:

\begin{itemize}
\item[{\em(i)}]
We provide empirical evidence that the sequence of models sent by legitimate clients is more predictable than those of malicious participants performing untargeted model poisoning attacks.
\\
\item[{\em(ii)}] 
We introduce FLANDERS, the first pre-aggregation filter for FL robust to untargeted model poisoning based on multidimensional time series anomaly detection.
\\
\item[{\em(iii)}] 
We integrate FLANDERS into Flower,\footnote{\scriptsize{\url{https://flower.dev/}}} a popular FL simulation framework for reproducibility.
\\
\item[{\em(iv)}] 
We show that FLANDERS improves the robustness of the existing aggregation methods under multiple settings: different datasets, client's data distribution (non-iid), models, and attack scenarios.
\\
\item[{\em(v)}] 
We publicly release all the implementation code of FLANDERS along with our experiments.\footnote{\scriptsize{\url{https://anonymous.4open.science/r/flanders_exp-7EEB}}}
\end{itemize}

% Paper's structure and organization
The remainder of the paper is structured as follows. %some related work and the current state-of-the-art solutions to security issues that FL entails. 
Section~\ref{sec:background} covers background and preliminaries. 
In Section~\ref{sec:related}, we discuss related work.
Section~\ref{sec:problem} and Section~\ref{sec:method} describe the problem formulation and the method proposed. % to tackle it. 
Section~\ref{sec:experiments} gathers experimental results. %, and Section~\ref{sec:limitations} discusses some limitations of this work.
Finally, we conclude in Section~\ref{sec:conclusion}.
 %discusses the limitations of this work and draws future research directions.
%reports conclusions and draws perspectives for future research directions.

%%%%%%% OLD %%%%%%%
%to overcome the resilience of Byzantine failures in distributed Stochastic Gradient Descent computations. 
% The strength of Krum is its time complexity, which is linear in the gradient dimension. 
% However, the robustness of the approach is guaranteed for gradient-based learning applications only when the majority of the clients are not compromised. 
% Besides, the aggregation mechanism of Krum, as well as that of similar methods, is robust from a coarse-grained perspective and does not provide solutions to errors and perturbations that may occur at inference time.
%A related approach to~\cite{blanchard2017nips} is the work of Su et al.~\cite{su2016dc}. Here, the authors propose an iterated approximate agreement to tackle a multi-layer scenario attacked by Byzantine agents. 
%However, the method works efficiently on the sole discrete context and it is inapplicable to continuous state environments.
%\gabri{Maybe, we should just talk about the main limitations of existing countermeasures without digging into their details (or, we can just mention Krum as this is the most popular one). I will move the description of all these methods to the Related Work section.}

	\section{Related Work}
	\label{sec:related_work}
	\section{Related work}
% There is extensive recent work on speeding up analytical queries due to the need for consistent execution times in the face of the explosive growth in the volume of available data.
% In this section, we divide existing work into two categories: maintaining data freshness in MVs (\Cref{sec:server_side}) and optimizations for minimizing ad-hoc query latency (\Cref{sec:client_side}).

% \subsection{Maintaining Data Freshness in MVs}
% \label{sec:server_side}
% There exists a variety of data warehousing applications aimed at supporting low-latency analytical queries on fresh data.
% In particular, these applications require efficiency in the propagation of newly ingested data into downstream MVs.
 
\mypara{Efficient MV Refresh}
Incremental view maintenance (IVM) aims to update MVs to reflect newly ingested data, taking advantage of already computed results to perform the update in a manner more efficient than computing from scratch (full refresh)
~\cite{ahmad2012dbtoaster,mcsherry2013differential,armbrust2013generalized,zeng2016iolap, palpanas2002incremental, griffin1995incremental, agiwal2021napa, braun2015analytics}. 
There is an abundance of work in IVM, including incremental updates on duplicate values~\cite{griffin1995incremental}, non-distributive aggregate functions~\cite{palpanas2002incremental}, higher-order views~\cite{ahmad2012dbtoaster}, and sliding windows~\cite{braun2015analytics}. 
More recent works also investigate the scalability aspect of IVM, proposing scale-independent updates~\cite{armbrust2013generalized} and sampled views~\cite{zeng2016iolap}. Since \system is applicable to arbitrary SQL statements, \system is orthogonal to and is fully compatible with existing IVM techniques.

\mypara{MV Refresh Scheduling}
There exist works on scheduling the refresh of a MV set focusing on resolving cyclic dependencies~\cite{folkert2005optimizing}, minimizing weighted average staleness~\cite{golab2009scheduling}, and prioritizing MVs with the highest speedups on predicted future queries~\cite{ahmed2020automated}.
\system's scheduling to speed up the end-to-end refresh of the MV set is not addressed in existing works.

\mypara{DAG Workflow Scheduling}
The execution of workloads consisting of individual jobs with acyclic dependencies is a well-studied topic~\cite{apacheoozie,sparkdag,marchal2018parallel,bathie2020revisiting,baruah2022ilp}; many of these techniques can be applied to MV refresh runs studied in this paper.
Existing workflow scheduling systems such as Apache Oozie~\cite{apacheoozie}, Apache Airflow~\cite{airflow}, and Spark DAG scheduler~\cite{sparkdag} automate the execution of user-defined workflows following a topological order.
There are recent works aimed at finding more optimal execution orders in terms of peak memory usage~\cite{marchal2018parallel, bathie2020revisiting} and execution time on parallel platforms~\cite{baruah2022ilp}.
While \system is designed for use with MV refresh runs/workloads, our technique on joint scheduling and optimization can be reasonably applied to general workloads as a possible future direction.

% \paragraph{Incremental MV indexing}
% Update-optimized indices such as the log-structured merge-trees (LSM)~\cite{o1996log} are used for indexing MVs due to frequent updates induced by data ingestion~\cite{gupta2016mesa,agiwal2021napa}.
% \system is orthogonal to indexing: \system is capable of efficiently performing MV refresh runs regardless of whether the individual MVs are indexed or not.

% \subsection{Ad-hoc Query Latency Reduction}
% \label{sec:client_side}

% The minimization of ad-hoc analytical query response times is a well-studied topic due to latency being negatively correlated with the productivity of a data analyst during a data analysis session~\cite{liu2014effects}.
% Sessions are commonly conducted within visualization systems that contain a variety of optimization techniques to ensure that query response times fall within a certain latency tolerance.

% \mypara{Data prefetching}
% Data is often loaded into memory on a by-need basis in visualization systems to minimize interference with user-issued query computations~\cite{mani2017effective,xin2021enhancing,galakatos2017revisiting, yan2020auto, battle2016dynamic, crotty2016case, jalaparti2018netco}. 
% Query-time data retrieval can be significantly expedited by anticipating the data usage of the user in future queries and pre-loading the data into memory during the downtime between user queries (`think time'). SMART~\cite{mani2017effective} prefetches data for modified versions of current user-issued queries with different filters and dimensions. A-WARE~\cite{crotty2016case} maintains a sample store constantly refined through ingesting data based on speculations of future plots.
% ForeCache~\cite{battle2016dynamic} uses an SVM to predict the user's current analysis phase and accordingly prefetches data tiles partitioned based on different numerical values. NetCo predicts future queries via log analysis, and solves an ILP formulation to prefetch data to maximize the number of SLO-meeting queries~\cite{jalaparti2018netco}.
% In the case of MV refresh workloads, `think time' is nonexistent as individual MVs are refreshed back-to-back, rendering data prefetching techniques non-applicable.

\mypara{Intermediate Data Caching}
Some existing data visualization systems cache user-defined variables to support the typical incremental construction of data visualizations~\cite{zgraggen2016progressive, eichmann2020idebench} during data analysis sessions~\cite{jupyter, rstudio, colab}. 
Recent work proposes a management scheme for these cached variables under a memory constraint that greedily keeps variables with the highest estimated time savings based on predicted future user behavior via neural networks~\cite{xin2021enhancing}.
While useful for data visualization, a greedy approach to memory management fails to achieve satisfactory results compared to \system.

\mypara{Intermediate Result Reuse}

There exist works on storing intermediate results from computations to speedup future computations~\cite{yang2018intermediate, dursun2017revisiting, nagel2013recycling, michiardi2019memory, galakatos2017revisiting}.
Studied topics include the identification of reuse opportunities by finding overlaps in computation graphs of successive jobs~\cite{yang2018intermediate, michiardi2019memory},
selective storage under a space constraint with heuristics such as reuse probability~\cite{dursun2017revisiting}, expected savings~\cite{yang2018intermediate}, and recompute-storage cost difference~\cite{nagel2013recycling},
and rewriting incoming jobs to take advantage of stored intermediates~\cite{galakatos2017revisiting}.
These works share similarity with \system in their selection of items to store under a memory constraint, however, \system's problem setting requires it to uniquely consider the joint (re)ordering of job executions along with the selection of items.

% work that considers both job execution (re)order as well as intermediate result caching with a bounded amount of memory. but notably lack the joint aspect of \system and cannot be used to achieve immediate speedup on an incoming MV refresh run if no intermediates are stored beforehand. 

\mypara{Incremental Query Processing} Incremental processing (IQP) is useful for cases where not all data required for a query is immediately available. Similar to online aggregation~\cite{hellerstein1997online}, initial results of a query are computed on a subset of required data and progressively refined as the rest of the required data arrives in a predictable pattern~\cite{tang2019intermittent,wangtempura}. Tang et al. propose a dynamic programming formulation to pick intermediate states to store in memory given a limited memory budget~\cite{tang2019intermittent}. Tempura rewrites the query plan for more efficient execution based on predicted data arrival patterns~\cite{wangtempura}. While similarities exist between the problem setting of IQP and \system, such as management of bounded memory, \system notably includes additional joint optimization for the order of MV updates.

% \paragraph{Sampling}
% Sampling has seen wide use in visualization systems for reducing the computation time of ad-hoc queries by computing an approximate result over a subset of data as exact results are not always required by the user~\cite{crotty2016case, mani2017effective, zgraggen2014panoramicdata, kraska2021northstar, galakatos2017revisiting, kandula2016quickr}. 
% Commonly studied topics in sampling for ad-hoc queries include complex query sampling~\cite{kandula2016quickr}, rare event aggregation~\cite{kraska2021northstar, galakatos2017revisiting}, and maintaining consistency between related sampled visualizations~\cite{zgraggen2014panoramicdata}.
% Sampling server-side at the MV level compromises the assumptions of downstream applications and is thus not considered in \system.

% \paragraph{Progressive visualization}
% The latency tolerance for time-consuming queries can be circumvented by presenting a partially-computed visualization to the user within the tolerance, which is then incrementally refined until it is fully accurate~\cite{rahman2017ve, zgraggen2016progressive, crotty2015vizdom, kraska2021northstar, kamat2017infiniviz}.
% Example plots which benefit from progressive visualization include bar charts~\cite{kamat2017infiniviz} and heatmaps~\cite{rahman2017ve}.
% Similar to sampling, study on this topic is orthogonal to \system as pushing out partially-updated MVs compromises downstream assumptions.

	\section{ARMBench Dataset}
	\label{sec:dataset}
	The ARMBench dataset presents: 1) a collection of sensor data acquired by a robotic manipulation workcell performing pick-and-place operation, 2) metadata and reference images for objects in containers, 3) a set of annotations acquired either automatically, by virtue of the system design, or via manual labeling, and 4) tasks and metrics to benchmark perception algorithms for robotic manipulation. Fig.\ \ref{fig:contributions} illustrates the benchmark tasks and variety of objects captured in the dataset. The dataset captures diversity in objects with respect to Amazon product categories as well as physical characteristics such as size, shape, material, deformability, appearance, fragility, etc. 

The data collection platform is a robotic manipulation workcell performing pick-and-place operation in a warehouse \cite{Sparrow2022}. The workcell contains a robotic arm mounted with a vacuum-based end-effector. It is presented with a heterogeneous collection of objects placed in unstructured configurations within a container (storage tote). The robotic arm is tasked with picking one object at a time (singulation) and place it on moving trays until the container is empty. The empty container ejects the workcell and is replaced by a new container. While the operation is completely autonomous, it includes a human-in-the-loop to monitor the status of each pick-and-place activity, annotate, and resolve any defects during manipulation. Multiple imaging sensors are placed in the workcell to facilitate and validate the pick-and-place operation. Following is a list of sensor data (Fig.\ \ref{fig:intro}) associated with each pick activity:
\begin{itemize}
\item Pick-image: A 5\,MP camera is used to capture a top-down image of the container.
% \item Pick-3D: Two Ensenso sensors capture the 3D point cloud of the source container.
\item Transfer-images: Multiple 5\,MP cameras are placed on different sides in the workcell to capture the moving object from different viewpoints.
% \item Transfer-Barcode: Multiple Cognex barcode sensors are used to scan the barcode of the object during transfer.
\item Place-image: A top-down view of the object is captured once it is placed on the tray.
\item Video: A camera is mounted to capture 720p videos of pick-and-place manipulation processes at 30\,FPS
\end{itemize}
Additionally, the following metadata (Fig.\ \ref{fig:contributions} (b)) is available by virtue of a warehouse tracking system:
\begin{itemize}
\item Container-manifest: A list of objects present in the container along with data such as product description, coarse dimensions, and weight.
\item Reference images: One or more images of objects from previous operations within the warehouse.
\end{itemize}
The sensor data and metadata were consumed by perception algorithms required to autonomously operate the robotic workcell. Benchmarking against these algorithms would not only optimize a manipulation task such as the one used for data collection but also enable more complex and intentional manipulation. This work considers a subset of such perception tasks namely object segmentation, object identification, and defect detection. These are critical not only to make informed grasping and motion decisions but also to track the state of the objects and containers within the warehouse. The following sections will describe these tasks and present the challenges using annotations, baseline algorithms, and evaluation metrics.

	\section{Object Segmentation}
	\label{sec:segmentation}
	The object instance segmentation task is to identify and delineate distinct objects stored in containers in a warehouse. In the context of robotic object manipulation, instance segmentation is used to inform downstream robotic processes such as grasp generation, motion planning, and placement. Accuracy of instance segmentation can have an impact on picking success, object identification, and defects introduced in the process. For example, under-segmentation can result in picking multiple objects at a time, while over-segmentation can result in a bad choice of grasp leading to damage or dropping of objects. Fig.\ \ref{fig:segmentation_subsets}(a) shows manually annotated object segments on the pick-image. Presence of deformable and transparent objects in clutter makes the task challenging.

Our object instance segmentation dataset contains 50K+ images of objects stored in containers in a warehouse with 500K+ annotations. The annotations include instance-level segmentation masks and bounding box for two classes (object and container). Technicians with task-specific training generated high-quality annotations for object boundaries and object class which are verified by two additional quality assurance technicians.

%The verified images and annotations are added to the dataset.

% Object instance segmentation 
%Instance segmentation is used in object manipulation to identify and outline distinct objects presented in storage units. The resulting object segments are useful for subsequent item understanding, grasp detection, and manipulation planning.
% Good instance segmentation can increase picking eligibility, improve identification success, as well as reduce pick defects and item damage.

%Accurate segmentation is one of the critical enablers for scaling high-performing robotic manipulation at Amazon.
%Reliable segmentation at the Amazon scale is very challenging. The segmentation algorithm needs to generalize to millions of unique items in Amazon warehouses, many of which are unseen. The algorithm must be robust in heavy clutter and occlusion, as objects are packed tightly in containers. Finally, the algorithm needs to transfer successfully to changing containers, lighting conditions, and ... 

%Meanwhile, academic researchers in robotic manipulation are facing similar challenges. Robots working in an unstructured environment need to handle a wide variety of seen and novel items, presented in totes, drawers, shelves, and on tabletops. To operate reliably with this diversity, the robot needs to learn the object concepts from a wide distribution of object data to generalize to unseen items. 

%We present the large-scale Amazon warehouse instance segmentation dataset. The dataset contains over 50k images of Amazon products placed in containers and has over 500k instance annotations. 
%The dataset aims to inspire research into object segmentation in clutter and to provide the robotic community with a wide distribution of real-world objects for learning and benchmarking. 
%The annotations include instance-level segmentation masks, bounding boxes, and class labels. Amazon's internal ML labeling team hand labeled the pixel-wise instance masks and class labels. The associates undergo task-specific training and auditing to produce high-quality class labels and segmentation with exact boundaries. Meanwhile, all labeled images are examined by two verifiers to detect defects such as incomplete and missing segments, in-precise boundaries, and wrong classifications. Images and labels are only used when both verifiers vote no issue. 

% We divide the object segmentation dataset into three subsets: 1) \textit{mix-object-tote} which comprises images of objects in yellow and blue totes with region of interest cropped to the boundary of the tote. 2) \textit{zoomed-out-tote-transfer-set} which comprises images of objects in a yellow tote with the tote centered in the image and covering 50\% of the image, and 3) \textit{same-object-transfer-set} which comprises multiple instances of the same object stored in packaging containers (Fig.\ \ref{fig:segmentation_subsets}). The three subsets in the dataset enable us to understand the impact of variation in background, container, and object distribution. The mix-object-tote subset comprises 44,253 images and 467,225 annotations. It has the highest degree of clutter with 10.5 object instances per image. The zoomed-out-tote-transfer-set subset comprises 5,837 images and 43,401 annotations with an average of 7.5 instances per image. The same-object-transfer-set subset comprises 3,323 images and 12,664 annotations with an average of 3.8 instances per image. 

We divide the object segmentation dataset into three subsets. The primary set, \textit{mix-object-tote}, comprises 44,253 images and 467,225 annotations of objects in yellow and blue storage totes. The totes contain a heterogeneous clutter of objects with an average of 10.5 object segments (ranging from 1 to 50 segments) in each image. The other two subsets, namely \textit{zoomed-out-tote-transfer-set} and \textit{same-object-transfer-set} (Fig.\ \ref{fig:segmentation_subsets}(b) and (c)) enable us to understand the impact of variation in data distribution. The \textit{zoomed-out-tote-transfer-set} subset with 5,837 images and 43,401 annotations captures images of containers from a different warehouse. It poses a transfer learning challenge due to significant differences in background, scale, and object distribution. The \textit{same-object-transfer-set} subset contains 3,323 images and 12,664 annotations. It captures a common and visually challenging scenario in warehouses where multiple instances of the same object are tightly packed in a container. 

%We divide the dataset into three subsets to highlight the challenge of transferring learned knowledge across tasks. The first and largest subset is called the mix-object-tote. It comprises images of mixed objects in yellow or blue totes. The second subset is named the zoomed-out-tote-transfer-set. It contains mixed objects placed in a yellow tote but captured with sensors placed further away from the tote and under different lighting. The third subset is named the same-object-transfer-set. It contains multiple instances of the same object placed tightly packed in different storage containers (Fig.~\ref{fig:segmentation_items}).

\begin{figure}
	\centering
	\includegraphics[width = 0.5\textwidth]{images/three-itemsets.jpg}
	\caption{(a) Segmentation annotation overlaid on an image from {\it mix-object-tote}. Each identifiable item is segmented regardless of its size and occlusion. Multiple objects in the same package are considered as one object and is delineated by the boundary of the package. In particular, items wrapped in transparent packaging are segmented by the peripheral of the package, although other products may be seen through them. (b-c) Example images from {\it zoomed-out-tote-transfer-set} and {\it same-object-transfer-set} subsets representing variations in background, scale, and clutter.}
	\label{fig:segmentation_subsets}
	\vspace{-0.2in}
\end{figure}



%The mix-object-tote is the largest dataset among the three subsets, containing 44,253 images and 467,225 annotations. On average, it has the highest clutter level of 10.5 instances per tote. The zoomed-out-tote-transfer-set contains 5,837 images and 43,401 annotations, averaging 7.5 objects per tote. The same-object-transfer-set dataset has 3,323 images and 12,664 annotations, averaging 3.8 objects per scene. We adopt a class-agnostic labeling scheme where each pick scene image is labeled with two classes: Tote and Object. Examples of labeled images are shown in Fig.~\ref{fig:segmentation_rules}:

% Add (b) to Fig. 3(a)
%\begin{figure}
%	\centering
%	\includegraphics[width = 0.5\textwidth]{images/segmentation_masks.png}
%	\caption{Each identifiable distinct item is segmented regardless of its size and occlusion. Multiple objects in the same package are considered as one object and is delineated by the boundary of the package. In particular, items wrapped in transparent packaging (e.g., plastic bags) are segmented by the peripheral of the packages, although other products may be seen through them.}
%	\label{fig:segmentation_rules}
%\end{figure}

%Accuracy and speed are both critical metrics used to evaluate segmentation algorithms for warehouse picking. 
To establish a performance baseline, we trained Matterport's implementation of Mask R-CNN~\cite{matterport_maskrcnn_2017, DBLP:journals/corr/HeGDG17} with ResNet-50 backbone~\cite{DBLP:journals/corr/HeZRS15} on the {\it mix-object-tote} dataset. Default training schedule (for MS-COCO) and hyper-parameters were used along with a train-valid-test split of 0.7:0.15:0.15. % and the images are split by timestamps, so similar images from the same order do not exist across splits. . 
Table~\ref{table:segmentation_results_inference} shows the results for our baseline experiment. Mean average precision ($mAP$) for a threshold of 0.5 ($mAP_{50}$) and 0.75 ($mAP_{75}$) are used to evaluate the performance of the baseline model on test set. 
%For computation time we cite the 5 fps as reported in ~\cite{DBLP:journals/corr/HeGDG17}, and we test the segmentation accuracy on the testing split of all three subsets, and the evaluation results ($mAP_{50}$ and $mAP_{75}$) are reported in :


We observe that applying model weights trained on {\it mix-object-tote} to the {\it zoomed-out-tote-transfer-set} ($mAP_{50}=0.25$) and {\it same-object-transfer-set} subsets ($mAP_{50}=0.11$) yields poor results. While techniques like transfer learning can improve performance on a new scenario when a reasonable amount of domain-specific labeled data is available, labeling specifically for each variation is time-consuming, if feasible at all. The ultimate goal is to readily transfer segmentation to new scenarios with minimal additional annotations.

%It should be noted that directly applying weight trained on mix-object-tote to the two untrained subsets yields poor results. While techniques like transfer learning can significantly improve performance on a new task when a reasonable amount of task-specific labeled data is available, labeling specifically for each variation is time-consuming, if feasible at all. The ultimate goal is to readily transfer segmentation to new tasks with minimal additional annotations. 

\setlength{\tabcolsep}{4pt}
\begin{table}
\centering
\caption{Mask R-CNN performance for object segmentation task. The model was trained on \textit{mix-object-tote} dataset}
\label{table:segmentation_results_inference}
\begin{tabular}{@{}rccc@{}}
    \hline
 & mix-object-tote & zoomed-out-tote- & same-object- \\ 
 & & transfer-set & transfer-set \\ \hline
$mAP_{50}$  & 0.72 & 0.25 & 0.11 \\ \hline
$mAP_{75}$  & 0.61 & 0.19 & 0.10\\ \hline
\end{tabular}
\end{table}
\setlength{\tabcolsep}{1.4pt}

We observe that segmentation performance for our baseline model has a strong correlation to the level of clutter. Fig.~\ref{fig:segmentation_clutter} shows that the performance drops significantly as the number of ground-truth object instances increases in the image. The $mAP_{50}$ score drops sharply from 0.95 when the tote has one to five object instances to a low of 0.38 when there are more than 26 object instances in the image. This motivates developing algorithms that are robust against clutter and occlusion to further improve object segmentation performance.

\begin{figure}[h]
	\centering
	\includegraphics[width = 0.45\textwidth]{images/segmentation_performance}
	\caption{Performance on {\it mix-object-tote} with varying degree of clutter.}
	\label{fig:segmentation_clutter}
	\vspace{-0.15in}
\end{figure}

















	\section{Object Identification}
	\label{sec:identification}
	\section{Proof of Theorem \ref{thm:identification}}
\label{subsec:identification_thm_proof}
%
\begin{proof} Below, the symbol $\stackrel{AX}{=}$ and $\stackrel{DX}{=}$ imply that the equality follows from Assumption $X$ and Definition $X$, respectively. 
%
We begin with the proof of Theorem \ref{thm:identification} (a). 
%
\noindent \emph{Proof of Theorem \ref{thm:identification} (a):} For a donor unit $u \in \mathcal{I}$ and $\pi \in \Pi \setminus \Pi_u$, we have
\begin{align*}
    \E[Y^{(\pi)}_{u}~|~\mathcal{A}] & \stackrel{A\ref{ass:observation_model}}{=} \E[\langle \balpha_u, \bchi^{\pi} \rangle + \epsilon_u^{\pi} ~|~\mathcal{A}] \\ 
    & \stackrel{A\ref{ass:observation_model}(c)}{=} \langle \balpha_u, \bchi^\pi \rangle ~| ~\mathcal{A} \\
    & \stackrel{}{=} \langle \balpha_u, \bchi^\pi \rangle ~| ~\mathcal{A}, \ \mathcal{D} \\
    &  \stackrel{A\ref{ass:observation_model}(b)}{=} \langle \balpha_{u}, \tilde{\bchi}_{u}^{\pi} \rangle ~|~ \mathcal{A}, \ \mathcal{D}\\
    & \stackrel{A\ref{ass:donor_set_identification}(a)}{=} \langle \balpha_{u}, \sum_{\pi_u \in  \Pi_u} \beta_{\pi_{u}}^{\pi}\tilde{\bchi}_{u}^{\pi_u} \rangle  ~ | ~  \mathcal{A}, \ \mathcal{D} \\ 
    & = \sum_{\pi_u \in \Pi_u} \beta^{\pi}_{\pi_u} \langle \balpha_{u}, \tilde{\bchi}_{u}^{\pi_u}  \rangle  ~|~ \mathcal{A}, \ \mathcal{D} \\
    & \stackrel{A\ref{ass:observation_model}(c), A\ref{ass:selection_on_fourier}}{=}  \sum_{\pi_u \in \Pi_u} \beta^{\pi}_{\pi_u} \E[\langle \balpha_{u}, \tilde{\bchi}_{u}^{\pi_u}  \rangle + \epsilon_u^{\pi_u} ~ | ~ \mathcal{A}, \ \mathcal{D} ] \\
    & \stackrel{A\ref{ass:observation_model}}{=} \sum_{\pi_u \in \Pi_u} \beta^{\pi}_{\pi_u} \E[Y_{u,\pi_u} ~ | ~ \mathcal{A}, \ \mathcal{D}]
\end{align*}
%
\noindent \emph{Proof of Theorem \ref{thm:identification} (b):} For a donor unit $n \in [N] \setminus I^{D}$ and  $\pi \in \Pi \setminus \Pi_n$, we have
\begin{align*}
    \E[Y^{(\pi)}_{n}~|~\mathcal{A}] & \stackrel{A\ref{ass:observation_model}}{=} \E[\langle \balpha_n, \bchi^{\pi} \rangle + \epsilon_n^{\pi} ~|~\mathcal{A}] \\ 
    & \stackrel{A\ref{ass:observation_model}(c)}{=} \langle \balpha_n, \bchi^\pi \rangle ~| ~\mathcal{A} \\
    & \stackrel{}{=} \langle \balpha_n, \bchi^\pi \rangle ~| ~\mathcal{A}, \  \mathcal{D} \\
    & \stackrel{A\ref{ass:donor_set_identification}(b)}{=} \langle \sum_{u \in \mathcal{I}}w_u^n\balpha_u, \bchi^{\pi} \rangle  ~ | ~  \mathcal{A}, \ \mathcal{D} \\ 
    & = \sum_{u \in \mathcal{I}}w_u^n \langle \balpha_{u}, \bchi^{\pi} \rangle  ~|~ \mathcal{A}, \ \mathcal{D} \\
    & \stackrel{A\ref{ass:observation_model}(c), A\ref{ass:selection_on_fourier}}{=}  \sum_{u \in \mathcal{I}} w_u^n \E[\langle \balpha_{u}, \bchi^{\pi} \rangle + \epsilon_u^{\pi} ~ | ~ \mathcal{A}, \ \mathcal{D} ] \\
    & \stackrel{A\ref{ass:observation_model}}{=} \sum_{u \in \mathcal{I}} w_u^n \E[Y^{(\pi)}_{u} ~ | ~ \mathcal{A}, \ \mathcal{D}] \\
    & = \sum_{u \in \mathcal{I}} \sum_{\pi_u \in \Pi_u}  w_u^n \beta^{\pi}_{\pi_u} \E[Y_{u,\pi_u} ~ | ~ \mathcal{A},\  \mathcal{D}]
\end{align*}
where the last equality follows from Theorem \ref{thm:identification} (a). 
\end{proof}


	\section{Defect Detection}
	\label{sec:defect_detection}
	% Distribution of damage types between multi-pick, open, and deconstruction
% Can we obtain distribution of taxonomy
% Distribution of object categories: Probably GL code based



% Problem statement and setup
% From a business standpoint, we care about multi-view/multi-model classification
% From a computer vision standpoint, we care about single image and single video classification


The defect detection task is to identify if a robotic manipulation activity resulted in a defect. Two types of robot-induced defects are included in the dataset: 1) \textit{multi-pick}, and 2) \textit{package-defect}. \textit{Multi-pick} is used to describe activities where multiple objects were picked and transferred from the source container to the destination container. \textit{Package-defect} is used to describe activities where the object packaging \textit{open}ed and/or the object separated into multiple parts (\textit{deconstruction)}. Two subclasses, \textit{open} and \textit{deconstruction}, are defined for package-defect.  %In warehouse operations, open packages need to be inspected to ensure integrity of the object, whereas in the case of deconstruction, the operation will be interrupted to recover the fallen object parts. 
Fig.\ \ref{fig:defects} shows examples of multi-pick and package-defect in our dataset. Multi-pick are often observed when there is a high degree of clutter, there are multiple instances of the same object, or when objects of significantly different sizes are placed together. Fig.\ \ref{fig:defects} (a-c) shows package-defect on a variety of objects. Defects on deformable objects like plastic bags can be challenging for visual detection. %While the defects presented in this dataset are generated by a vacuum-based end-effector, they generalize to grasping end-effectors as well. Additionally, vacuum end-effectors are ubiquitous in object handling for their relatively low hardware and algorithmic complexity.

Our dataset comprises 19,303 images of objects from multiple viewpoints (Transfer-images) and 4,070 videos of pick-and-place activities that resulted in a defect. % In addition, 100,000 images of objects and 100,000 videos of activities that did not result in a defect are included in the dataset (defined as {\it nominal}). 
Videos are excluded from our dataset for multi-pick defect as such defects are not observable along specific viewpoints. Multi-view Transfer-images are best suited to detect multi-pick defect. On the other hand, \textit{open} and \textit{deconstruction} defects can happen at any time during an activity. As a result, they are best captured using videos. The dataset includes 100,000 images of objects and videos of activities that do not have any defects and are defined as \textit{nominal} activities. 
Tables\ \ref{tab:defect_image_baseline} and \ref{tab:defect_video_baseline} shows the distribution of defect types in our dataset. In addition to Transfer-images, the dataset includes Pick-image and Place-image that provide context for an activity. 

%Our dataset includes images and videos for activites that resulted in a defect. For each activity, one video was recorded from a static viewpoint (Webcam) that captures the whole manipulation process for each activity. Four images (Transfer-RGB) show the object being manipulated from different viewpoints during the activity, one image (Pick-RGB) shows the object in the source container before being picked, and one image (Place-RGB) shows the object in the destination container after placement.

%Fig.\ \ref{fig:defects}(a)--(c) shows package-defects for different types of objects in our dataset. Fig.\ \ref{fig:defects}(a) shows and open box with contents of the box falling out (Fig.\ \ref{fig:defects})(a). However, an open bag is very hard to differentiate from a closed/sealed bag even to an expert eye (Fig.\ \ref{fig:defects}(d). Occlusions and homogeneity of materials makes multi-pick detection challenging. Fig.\ \ref{fig:defects}(g) shows a multi-pick defect where only a small part of one of the objects is visible and in Fig.\ \ref{fig:defects}(h) two instances of the same object was picked which is hard to differentiate. 

%Three types of defects may appear in each activity: 1) \textit{multi-pick}, 2) \textit{open}, and 3) \textit{deconstruction}. \textit{Multi-pick} means that multiple objects are picked and transferred by the end-effector. \textit{Open} means that an object package is opened during an activity. \textit{Deconstruction} means that an object separates into two or more parts. \textit{Deconstruction} can happen either due to an object breaking into multiple parts or the contents of a package falling out. Sometimes, more than one types of defect can appear in a single activity, e.g., a shoe box \textit{open}s and then \textit{deconstruct}s (shoes fall out). % Thus, the defect detection task is a multiclass classification problem on both images and videos. 
%Early detection of defects can enable the system to take corrective action like place the defective object in the source container or in a special container for processing. 

A two-step process was used to annotate data. A technician operating our system labeled each activity as successful/nominal (a single object transferred from the source container to the destination container), \textit{multi-pick}, \textit{open}, or \textit{deconstruction} defect. Expert annotators verified the annotations for each activity and augmented the annotations for Transfer-images as multi-pick, or package-defect if a defect was observable, and as {nominal} if no defect was observable in the image. In addition to the defect type, we also provide segmentation polygons for the objects to enable development of models that can benefit from additional attention cues. For video annotations, expert annotators verified the type of package defect, i.e., \textit{open} and \textit{deconstruction}, observed in the video. In addition to the type of defect observed in each video, the index of the first frame where a defect becomes observable is also provided to enable development of real-time defect detection methods. 



\begin{figure}[t]
	\centering
	\includegraphics[width=0.5\textwidth]{images/defects.jpg} %defects/defects_02
	\caption{ Multi-view images in the defect detection dataset showing (a)--(c) \textit{package-defect} and (d)--(f) \textit{multi-pick} defect for different types of objects. }
	\label{fig:defects}
	\vspace{-0.1in}
\end{figure}

To establish a baseline for defect detection, we performed two experiments. In the first experiment, we train an image classifier with ResNet-50 \cite{DBLP:journals/corr/HeZRS15} backbone, global average pooling, and focal loss for predicting the type of defect observed in the Transfer-images. In the second experiment, we trained a multi-scale vision transformer model (MViT-B) \cite{fan2021MultiscaleVT} for action classification on videos. Since a defect can be introduced at any time during the manipulation process, we uniformly sampled 32 frames ($\sim$5\,FPS) from each video for training. %The sampled frames are cropped given an universal region of interest across all videos and resized to 224x224 resolution. 
The classification head outputs a two-channel vector that predicts binary classification on two categories: \textit{open} and \textit{deconstruction}. %The model is pretrained on  Kinetics-400 dataset \cite{}, and finetuned on our defect detection dataset. 
%For both experiments, we downsampled the nominal category to 1x of the number of samples in the defect category to compensate for class imbalance. 
We used a train-test split of 0.7:0.3 for multi-pick and package-defects. The nominal category in the train set was downsampled to match the size of the defect category to compensate for class imbalance. 10,000 samples from the nominal category were added to our test set. 

% Metrics and evaluation
Table\ \ref{tab:defect_image_baseline} and \ref{tab:defect_video_baseline} show performance of baseline models for defect detection on images and videos. We used recall and false positive rate (fpr) as metrics to evaluate performance over defect classes. A missed defect (lower recall) is more expensive than classifying a nominal activity as defective (fpr). Results for image defect detection shows that multi-picks are a harder to detect than package-defects. On the other hand, results for video defect detection show that open defects are harder to detect than deconstruction. There is significant scope for improvement in defect detection methods to be effective in warehouses operations which typically require high recall ($>$0.95) and low fpr ($<$0.01). 
% For defect detection using multi-view images, we evaluate models for single-view metrics ($P_{si}$, $FNR_{si}$, $FPR_{si}$) and multi-view metrics ($P_{mi}$, $FNR_{mi}$, $FPR_{mi}$).


%\noindent\textbf{Video Classification Baseline}
%Three baselines are performed for the video classification task: 1) R(2+1)D, 2) SlowFast and 3) MViT-B. ResNet-50 is used as the backbone of the first two CNN-based models. Since an object can be damaged at any stage of the manipulation process, we uniformly sample 32 frames ($\sim$5 FPS) from each video for training. The sampled frames are cropped given an universal RoI across all videos and resized to 224x224 resolution. The classfication head outputs a two-channel vector that does binary classification on the two categories: \textit{open} and \textit{deconstruction}. All the baseline models are pretrained on Kinetics-400 dataset, and finetuned on the proposed dataset for 50 epochs. For SlowFast network, we set the default speed ratio $\alpha = 4$ and channel ratio $\beta = 8$. During testing, two strategies are performed. The first one is the same as training, i.e. uniformly sample 32 frames from each video and make one prediction for each video. The second strategy is to make multiple predictions on shorter video clips (views). If defections are found in any clip, then the whole video is predicted as defected. In practice, 32 frames are sampled from each $3s$ clip, where a clip is sampled every $1s$ (stride = 1). This strategy can also give us a rough idea of when the defection happens.





% \setlength{\tabcolsep}{6pt}
% \begin{table}[t]
% 	\centering
% 	\caption{Distribution of defect types in the dataset}
% 	\label{tab:defect_dataset}
% %	\begin{tabular}{c|c|c|c|c}
% %		\hline  
% %		\multirow{3}{*}{} & \multirow{2}{*}{Nominal} &  \multirow{2}{*}{Multipick}  & \multicolumn{2}{c}{Package defect} \\  \cline{4-5}
% %		&                   &  &   Open & Deconstruction \\  \cline{1-5}
% %		Video      &    100,000  &    -      &  XX &    XX  \\ \hline
% %		Image     &   100,000  &  7,813    &     \multicolumn{2}{c}{5,759}     \\ \hline		
% %	\end{tabular}
% 	\begin{tabular}{c|c|c|c}
% 	\hline  
% 	 & {nominal} &  {multi-pick}  & {package-defect} \\  \hline
% %	&                   &  &   Open & Deconstruction \\  \cline{1-5}
% 	Video      &    100,000  &    -    &    4,075  \\ 
% 	Image     &   100,000  &  7,813    &    11,490   \\ \hline		
% \end{tabular}
% \end{table}
% \setlength{\tabcolsep}{1.4pt}

\begin{table}[t]
	\centering
	\caption{Baseline for single-view image defect detection}
	\label{tab:defect_image_baseline}
	%\begin{tabular}{|c|c|c|c|c|}
	%	\hline 
	%	\multirow{2}{*}{Model} & \multirow{2}{*}{Model}  &  \multicolumn{3}{c|}{Metric} \\   \cline{3-5}
	%	 & &  FPR  & FNR  & Precision \\  \hline
	%	 Image &  ResNet-50 &  XX  & XX  & XX \\ \hline
	%	 Video & XX & XX & XX & XX \\ \hline
	%\end{tabular}
	\setlength{\tabcolsep}{3pt}
	\begin{tabular}{c|c||c|c||c}
	    \hline 
	    {model} & {metric}  &  {multi-pick}  & { package-defect} & combined \\
	    \hline  
		\multirow{3}{*}{ResNet-50 \cite{DBLP:journals/corr/HeZRS15}}   & count     &  7,813  & 11,490  & 19,303 \\& recall     &  0.34  & 0.73  & 0.57 \\ %\cline{2-6}
									                	               & fpr        &  0.05  & 0.05  & 0.05  \\  \hline
									                	             %  precision  &  0.54  & 0.81      \\ \hline
	\end{tabular}
\end{table}
\setlength{\tabcolsep}{1.4pt}

\begin{table}[t]
	\centering
	\caption{Baseline for video defect detection}
	\label{tab:defect_video_baseline}
	%\begin{tabular}{|c|c|c|c|c|}
	%	\hline 
	%	\multirow{2}{*}{Model} & \multirow{2}{*}{Model}  &  \multicolumn{3}{c|}{Metric} \\   \cline{3-5}
	%	 & &  FPR  & FNR  & Precision \\  \hline
	%	 Image &  ResNet-50 &  XX  & XX  & XX \\ \hline
	%	 Video & XX & XX & XX & XX \\ \hline
	%\end{tabular}
	\setlength{\tabcolsep}{3pt}
	\begin{tabular}{c|c||c|c||c}
	    \hline 
		model & metric  & open & deconstruction & combined \\ 
		\hline
		\multirow{3}{*}{MViT-B \cite{fan2021MultiscaleVT}}   
		& count     &  2,951  &     2,165  &   4,070 \\
		& recall    &  0.69   &     0.79   &   0.73 \\ %\cline{2-6}
	    & fpr 	    &  0.23   &     0.03   &   0.13 \\ \hline
									            	      %   &            &  0.21  &     0.63    \\ \hline
	\end{tabular}
	%\vspace{-0.15in}
\end{table}
\setlength{\tabcolsep}{1.4pt}


%At $T_1$ we have four viewpoints ($V_i, i \in [0, 4]$) of the objects. Damage defects are annotated for each object as well as for each viewpoint. Viewpoint specific annotations are provided by the expert annotator during post processing. Multiple viewpoints are useful to detecting defects as defects might not be visible from certain viewpoints. We provide instance masks for the object manipulated in images at $T_0$ and at $T_1$ for all viewpoints $V_i$. Fig.\  \ref{fig:damageExamples}a shows an object that is pristine condition at $T_0$ and is damaged at $T_1$. Fig.\  \ref{fig:damageExamples} shows multiple objects being picked by the system. The annotations provided by the technician and verified or corrected by the expert is used to generate annotations for videos provided for defects. 

%Multi-class annotations are provided for videos: 1) multi-pick 2) open and 3) deconstruction. For our video data, we also provide time stamps on when the defect started and ended. For multi-picks the start time is defined by the when mutiple objects are observable. This typically happens when the robot is above the source tote. The end time for multi-pick is typically at the end of the manipulation process when multiple objects are dropped in the destination. For open packages, the start time is defined by when the object packaging opens and the end time is when the object is placed in the destination container. For deconstruction damages, the start time is defined as the time when a part of the object separates from the part of the object held by the end-effector. The end time is defined by when separated part is static in the scene. 




	\section{Discussion and Future Work}
	\label{sec:discussion}
	We provide some comments on the growth conditions which constituted the majority of our analysis in sections \ref{sec:Hmixing} and \ref{sec:Hsigma}. In the simplest cases of Lemma \ref{lemma:unstableGrowth}, growth was established in an analogous fashion to the old one-step expansion condition (\ref{eq:oldOneStepExpansion}), finding the relevant Jacobians $M_j$ and checking that their expansion factors $K(M_j)$ satisfy
\begin{equation}
    \label{eq:discussionOneStep}
    \sum_j \frac{1}{K(M_j)} <1.
\end{equation}
For the more complicated cases, the inductive method used to establish growth near the accumulation points in Lemma \ref{lemma:unstableGrowth} and the weakened one-step expansion condition (\ref{eq:oneStep}) both address the same fundamental issue: the splitting of unstable curves by singularities into an unbounded number of small components. They circumvent this obstacle in rather different ways, however. While (\ref{eq:oneStep}) generalises (\ref{eq:discussionOneStep}) to ensure an growth of unstable curves `on average' (see \cite{chernov_statistical_2009} for a precise statement), our inductive method is a more direct adaptation of (\ref{eq:discussionOneStep}), using it to generate contradictory geometric conditions which a hypothetical non-growing unstable curve must satisfy. It may be possible to prove Theorem \ref{sec:Hmixing} using (\ref{eq:oneStep}) as the basis for growth. Since we required (\ref{eq:oneStep}) anyway for proving Theorem \ref{thm:HsigmaExp}, this could potentially condense our analysis, but only to a minor extent. A convenience of the method used in section \ref{sec:Hmixing} is that, by way of the `simple intersection' property, it naturally gives geometric information on the images of manifolds, useful for proving the property \textbf{(M)} of Theorem \ref{thm:katok-strelcyn}.

We expect that essentially analogous analysis can be applied to establish mixing properties in a wide class of piecewise linear non-uniformly hyperbolic maps, including those (like the OTM) which sit on the boundary of ergodicity and beyond. While we have relied on the precise partition structure of $H_\sigma$, its fundamental feature (self-similar sequences of elements $A^k$, sharing boundaries with its neighbours $A^{k-1},A^{k+1}$ and accumulating onto some point $p$) is quite typical to return map systems. See, for example, those of various stadium billiards \cite{chernov_chaotic_2006,chernov_improved_2008,chernov_statistical_2009} and LTMs \cite{springham_polynomial_2014}. Indeed, the same method can be used to prove the Bernoulli property for non-monotonic LTMs \cite{myers_hill_mixing_2022}, where monotonicity of the manifold images cannot be assumed and the classical argument \cite{sturman_mathematical_2006} fails. The OTM is the pointwise limit of these maps as the boundary shrinks to null measure. It further has utility in proving growth conditions for maps which are uniformly hyperbolic but possess regions $A_j$ where the hyperbolicity is very weak, signified by $K(M_j) \approx 1$, so that (\ref{eq:discussionOneStep}) fails. Typically this leads to suboptimal bounds on mixing windows, see e.g. \cite{wojtkowski_model_1981,przytycki_ergodicity_1983,myers_hill_family_2022}. The map $H_{(\eta,\eta)}$ for $\eta \approx 1/2$ is another example, possessing weak hyperbolicity over $A_2, A_3$. Letting $\varepsilon = |\eta-1/2|>0$, there is an upper bound $N = N(\varepsilon)$ on escape times from the intersections $A_2\cap \sigma, A_3 \cap \sigma$. The growth lemma then follows by applying the inductive step roughly $N$ times and can be established for arbitrarily small $\varepsilon$, opening the door to establishing optimal mixing windows.

The above gives two examples of piecewise linear perturbations to $H$ where mixing with respect to Lebesgue is preserved and our methods can be applied. Nonlinear perturbations to the shear profiles complicate the analysis in several ways. Firstly as the map's Jacobians takes on a broader range of values, cone invariance becomes an increasingly harder condition to establish. Cones must be widened, giving looser bounds on expansion factors, which may already be weak due to new regions of weaker stretching. This, together with the change from polygonal to curvilinear return time partition elements and nonlinear local manifolds, adds some complexity to showing growth conditions. This does not rule out certain (small) nonlinear perturbations however. There is some leeway in the inequalities which govern cone invariance and growth of local manifolds, the latter of which is not too dissimilar from the piecewise linear setting (see Lemmas \ref{lemma:piecewiseApprox}, \ref{lemma:componentLength}). Certain small perturbations would not alter the \emph{topological} structure of the return time partition, i.e. which elements share boundaries, the key information needed for setting up the induction. Finally while the partition elements would no longer be polygonal, only coarse geometric information is required for verifying each inductive step. Following the above, a potential perturbation could be to replace the linear portions of each shear by a cubic, perturbing the tent profile
\[  f(t) = \begin{cases} 2t & 0 \leq t \leq 1/2, \\ 2(1-t) & 1/2 \leq t \leq 1 ,\end{cases} \]
of the OTM shears to
\[  f_a(t) = \begin{cases} \frac{1}{8} t \left(16 - a + 6at - 8at^{2} \right) & 0 \leq t \leq 1/2, \\ \frac{1}{8}\left(1-t\right)\left( 16 - a + 6a\left(1-t\right) - 8a\left(1-t\right)^{2}\right)  & 1/2 \leq t \leq 1, \end{cases}   \]
for $a>0$. For small enough $a$ the gradient range $f'(t)$ is restricted to small neighbourhoods of $\{ 2, -2\}$ and the escape time partition retains a similar structure. We illustrate this in Figure \ref{fig:perturbations}, showing escapes from the square $S_3$ under the map $G \circ F$, equivalent to escapes from the perturbed $A_3$ under the $G \circ F$, but with a cleaner geometry for comparison. When $a$ is too large the analogy to the OTM breaks down. At $a=16$ the map is twice differentiable everywhere and features a new source of slowed mixing, the Jacobian is the identity at the corner points $x,y \in \{  0, 1/2 \}$ giving locally parabolic behaviour (visible in the escape time partition). 

\begin{figure}
    \centering
    \includegraphics[width=0.24 \linewidth]{0.png}
    \includegraphics[width=0.24 \linewidth]{4.png}
    \includegraphics[width=0.24 \linewidth]{8.png}
    \includegraphics[width=0.24 \linewidth]{16.png}
    \caption{Partition of escape times from $S_3$ under the mapping $F \circ G$ for $a= 0,4,8,16$. }
    \label{fig:perturbations}
\end{figure}
	
	\section{Acknowledgements}
	We would like to thank the Sparrow \cite{Sparrow2022} team members for deployment and operation of the robotic workcell, Aalekh (Raj) Ray Chaudhury and the Go-AI team for data annotation support and the Item-matrix team for curating the reference image dataset. We would also like to thank Joey Durham, Andy Marchese, Clay Flannigan, Parris Wellman, Jane Shi and Kapil Katyal for their valuable feedback.
	
	\bibliographystyle{IEEETranS}
	\bibliography{bibtex}	
\end{document}
