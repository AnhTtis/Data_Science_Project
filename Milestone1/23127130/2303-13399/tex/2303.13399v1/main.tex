\documentclass[10pt,twocolumn,letterpaper]{article}

\usepackage{iccv}
\usepackage{times}
\usepackage{epsfig}
\usepackage{graphicx}
\usepackage{amsmath}
\usepackage{amssymb}

\usepackage{color}
\usepackage{multirow}
\usepackage{tabularx}
\usepackage{booktabs}
\usepackage{makecell}
\usepackage[table,dvipsnames]{xcolor}
\usepackage[ruled,linesnumbered]{algorithm2e}
\usepackage{pifont}
\usepackage[accsupp]{axessibility}


\newcommand{\bbox}{\text{bbox}}
\newcommand{\alphapck}{\alpha_\bbox}
\newcommand{\kcycle}{\text{k-CyPCK}}
\newcommand{\cycle}{\text{-CyPCK}}

\newcommand{\I}{\mathbf{I}}
\newcommand{\Ia}{\I^\text{a}}
\newcommand{\Ib}{\I^\text{b}}
\newcommand{\Iatob}{\I^\text{a $\rightarrow$ b}}
\newcommand{\F}{\mathbf{F}}
\newcommand{\Fa}{\F^\text{a}}
\newcommand{\Fb}{\F^\text{b}}
\newcommand{\f}{\mathbf{f}}
\newcommand{\fa}{\f^\text{a}}
\newcommand{\fb}{\f^\text{b}}
\newcommand{\p}{\mathbf{p}}
\newcommand{\pa}{\p^\text{a}}
\newcommand{\pb}{\p^\text{b}}
\newcommand{\A}{\boldsymbol{\Phi}_\text{align}}
\newcommand{\G}{\mathbf{G}}
\newcommand{\C}{\mathbf{C}}
\newcommand{\Ca}{\C^\text{a}}
\newcommand{\Cb}{\C^\text{b}}
\newcommand{\cc}{\mathbf{c}}
\newcommand{\cca}{\cc^\text{a}}
\newcommand{\ccb}{\cc^\text{b}}
\newcommand{\Irec}{\I_\text{Recon}}
\newcommand{\M}{\mathbf{M}}
\newcommand{\Mrec}{\M_\text{Recon}}
\newcommand{\loss}{\mathcal{L}}
\newcommand{\T}{\mathcal{T}}
\newcommand{\W}{\mathcal{W}}
\newcommand{\Id}{\mathcal{I}}



\definecolor{aliceblue}{rgb}{0.94, 0.97, 1.0}
\definecolor{citecolor}{HTML}{0071BC}
\definecolor{linkcolor}{HTML}{ED1C24}
\definecolor{darkgreen}{HTML}{539165}

\makeatletter
\newcommand{\thickhline}{%
 \noalign {\ifnum 0=`}\fi \hrule height 1pt
 \futurelet \reserved@a \@xhline
}
\makeatother

\newcommand{\increase}[1]{{
  \fontsize{6.6pt}{0.5em}\selectfont({\color{darkgreen}{$\uparrow$~\textbf{#1}}})
}}
\newcommand{\decrease}[1]{{
  \fontsize{6.6pt}{0.5em}\selectfont({\color{purple}{$\downarrow$~\textbf{#1}}})
}}


% Include other packages here, before hyperref.

% If you comment hyperref and then uncomment it, you should delete
% egpaper.aux before re-running latex.  (Or just hit 'q' on the first latex
% run, let it finish, and you should be clear).
% \usepackage[pagebackref=true,breaklinks=true,letterpaper=true,colorlinks,bookmarks=false]{hyperref}
\usepackage[pagebackref, breaklinks=true, colorlinks, citecolor=NavyBlue, linkcolor=linkcolor, bookmarks=false]{hyperref}


\usepackage[capitalize]{cleveref}
\crefname{section}{Sec.}{Secs.}
\Crefname{section}{Section}{Sections}
\Crefname{table}{Table}{Tables}
\crefname{table}{Tab.}{Tabs.}


% \iccvfinalcopy % *** Uncomment this line for the final submission

\def\iccvPaperID{****} % *** Enter the ICCV Paper ID here
\def\httilde{\mbox{\tt\raisebox{-.5ex}{\symbol{126}}}}

\iccvfinalcopy

% Pages are numbered in submission mode, and unnumbered in camera-ready
% \ificcvfinal\pagestyle{empty}\fi


\begin{document}

%%%%%%%%% TITLE
\title{Multi-granularity Interaction Simulation for Unsupervised \\ Interactive Segmentation}


\author{
Kehan Li$^{1,3}$ \quad Yian Zhao$^{5}$ \quad Zhennan Wang$^{2}$ \quad Zesen Cheng$^{1,3}$ \quad Peng Jin$^{1,3}$ \\ Xiangyang Ji$^{4}$ \quad Li Yuan$^{1,2,3}$ \quad Chang Liu$^{4}$\thanks{Corresponding author.} \quad Jie Chen$^{1,2,3}$$^*$ \and
$^{1}$ School of Electronic and Computer Engineering, Peking University, Shenzhen, China \\
$^{2}$ Peng Cheng Laboratory, Shenzhen, China \\
$^{3}$ AI for Science (AI4S)-Preferred Program, Peking University, Shenzhen, China \\
$^{4}$ Department of Automation and BNRist, Tsinghua University, Beijing, China \\
$^{5}$ Dalian University of Technology, Dalian, China
}

\maketitle
% Remove page # from the first page of camera-ready.
\ificcvfinal\thispagestyle{empty}\fi


%%%%%%%%% ABSTRACT
\begin{abstract}

Interactive segmentation enables users to segment as needed by providing cues of objects, which introduces human-computer interaction for many fields, such as image editing and medical image analysis.
Typically, massive and expansive pixel-level annotations are spent to train deep models by object-oriented interactions with manually labeled object masks.
In this work, we reveal that informative interactions can be made by simulation with semantic-consistent yet diverse region exploration in an unsupervised paradigm.
Concretely, we introduce a \textbf{M}ulti-granularity \textbf{I}nteraction \textbf{S}imulation (\textbf{MIS}) approach to open up a promising direction for unsupervised interactive segmentation.
Drawing on the high-quality dense features produced by recent self-supervised models, we propose to gradually merge patches or regions with similar features to form more extensive regions and thus, every merged region serves as a semantic-meaningful multi-granularity proposal.
By randomly sampling these proposals and simulating possible interactions based on them, we provide meaningful interaction at multiple granularities to teach the model to understand interactions.
Our MIS significantly outperforms non-deep learning unsupervised methods and is even comparable with some previous deep-supervised methods without any annotation.

\end{abstract}


%%%%%%%%% BODY TEXT
\section{Introduction}

\section{Threat Model and Advantages of Our Hardware-based Adversarial Detector} \label{sec: motivation}
\ry{In this part, I want to highlight the comparison between hardware and software attacks}
%Normally, software-based adversarial detectors are easier to implement, cheaper to develop and more well-studied than those based on hardware computational signals.
% We would like to stress that our goal for investigating hardware-based adversarial detectors is not to achieve better performance in detection than the conventional white-box software based methods.  
\subsection{Threat Model} \label{sec: threat model}
\ry{This section is threat model: attack is `white-box', detector is `black-box'}
The victim is a DNN classifier, which is pre-trained with a public dataset. The testing dataset may be kept private.
We assume the strongest `white-box' attack model, where the attacker has full knowledge of the victim model and training dataset in order to generate adversarial samples with minimum perturbations. 
On the contrary, the detection system assumes the most limited scenario, under a `black-box' view of the victim, without access to the victim's inputs, parameters, and intermediate outputs or execution details. 
The only information available to the detector to distinguish adversarial samples is the EM side-channel measurement and the victim model's prediction class.
For training the adversarial detector with EM traces, a public benign dataset is used. 

\if false 
\ry{In this part, we discuss more settings of the detector especially the data used in two phases.}
In general, the detecting process can be summed up into two phases, training phase and detecting phase.
To begin with, we train an Out-of-Distribution(OOD) detector on a public benign dataset of the same classification task, which should be distinct from the victim's training dataset.
For each query, the detector will obtain the classification result and an EM trace along with the model execution to fit its EM classifiers and anomaly detectors.  
During the detection phase, the victim model is in operation and under attack when the pre-trained detector decides whether the current input is adversarial or not, only based on the victim model output and its EM trace.
\fi 

\subsection{Advantages}
Compared to software-based adversarial detection methods, our hardware-based detector, EMShepherd, has three distinct advantages: privacy-preserving, portability, and robustness.

\begin{itemize}[leftmargin=*]
    \item \ry{Add a new motivation here. The motivation is that using \name can help the user protect their privacy.} 
    \name protects the DNN model user's data privacy as it is agnostic to the model's inputs, which instead are always required by prior reconstruction-based detection methods~\cite{meng2017magnet, yang2022you}. 
    %Most model users are benign whose inputs may be sensitive and should not be shared with \textit{third-party detectors}. 
    The sensitive inputs should not be shared with \textit{third-party detectors}. 
    Our design only requires the output class labels and the EM signals, which are passively leaked to common acquisition equipment. 
    %    Our design is suitable for such cases as it only requires the EM signals and the inference outputs during the model execution. Generally speaking, EM signals and labels have less private information leakage.
    \item \ry{The second motivation is still related to privacy. This time we consider model privacy when the model structure or parameters should be kept private.}
   \name also protects the model confidentiality.  No model information, including %Using hardware-based detectors can prevent the third-party defender from accessing some confidential model information such as  
   hyper-parameters, parameters, and logits, is needed, in stark contrast to the previous software-based detection methods~\cite{ma2019nic,feinman2017detecting}.
    %Our \name only acquires the EM traces during model inference in a passive and noninvasive manner, 
    The EM data processing and the adversarial detector training process are both victim model-agnostic. 
    Therefore, our method has more general usage, applicable to closed-source DNN applications, which are pervasive in edge devices where the user only queries the models for the final prediction output. 
    \item \ry{The third motivation is portability.}  
    Owing to the model-agnostic feature, EMShepherd can be easily ported for wide-range hardware devices with different DNN implementations for diverse applications. It can be used as a `plug and play' (PnP) device, aside from the target system, to work automatically without user intervention or contact with the victim system. 
    \item \ry{The last motivation is about adaptive attacks, we should propose that EM signal is hard to imitate, so it is hard for adaptive attacks to generate sample fraud both detector and victim.} 
    Adaptive attack~\cite{adaptive} is a threat to most software defense methods where the attacker adjusts the adversarial perturbations to mislead both the victim models and defense systems.
   %  The hardware-based detection method can provide a double protection on top of most software defense methods such as adversarial training.
   %  Although the adptive adversarial example fools the robust model, its computation patterns during the DNN model execution are still well kept in the EM traces and our EMShepherd framework still works well for detecting the new type of adversarial examples.  
   %  Meanwhile, due to the high complexity of EM signals and non-explicit dependency of the EM signals on computations, it is extremely hard to have an adaptive attack on our detection method, i.e., adversarial examples whose EM signals are deliberately controlled to evade the EM-based detector.
   However, due to the high complexity and non-explicit dependency of the EM signals on computations and data, 
   it is extremely hard to have an adaptive attack on our detection method, 
   i.e., adversarial examples whose EM signals are deliberately controlled to evade the EM-based detector. 
\end{itemize}







Interactive segmentation aims at obtaining high-quality object masks with limited user interaction, allowing users to segment objects as needed.
During the development of interactive segmentation, various interactive forms like bounding boxes~\cite{lempitsky2009image,rother2004grabcut,wu2014milcut}, scribbles~\cite{bai2014error,grady2006random,li2004lazy}, and clicks~\cite{xu2016deep,jang2019interactive,lin2020interactive,sofiiuk2020f,chen2021conditional,sofiiuk2022reviving,lin2022focuscut,chen2022focalclick} have been explored.
Among them, the click-based interaction becomes the mainstream due to its simplicity and well-established training and evaluation protocols.


\begin{figure*}[t]
    \centering
    \includegraphics[width=\textwidth]{figs/intuition_src.pdf}
    \caption{
        \textbf{Illustration of the multi-granularity region proposal generation.}
        For an input image, We first introduce a self-supervised ViT to produce semantic features for each patch of an image.
        We then gradually merge two similar patches or regions with similar features to perform a new region.
        All newly generated regions in the merging process constitute meaningful proposals at multiple granularities.
    }
    \label{fig:intuition}
\end{figure*}



State-of-the-art methods for click-based interactive segmentation are based on deep learning, following the basic paradigm proposed by Xu \emph{et al.}~\cite{xu2016deep}.
Specifically, these methods encode clicks to a distance map and adapt a semantic segmentation model (\emph{e.g.}, FCN~\cite{long2015fully}) to take the encoded click map as input, and then train the model with interaction-segmentation pairs.
Since it is impractical to collect interaction sequences from real users, previous methods adopt an interaction simulation strategy~\cite{xu2016deep,sofiiuk2022reviving} that randomly samples clicks based on object segmentation annotations.
By training with the simulated clicks and corresponding object masks, the model is able to correctly understand user needs and segment objects with a few clicks, significantly outperforming non-deep learning methods.
However, the requirement for a large number of expensive pixel-level annotations behind the powerful performance has been ignored by previous methods, which is derived from the object-level interaction simulation.


In this work, we aim at exploring an annotation-free alternation for interaction simulation to eliminate the reliance on object annotations in interactive segmentation, namely unsupervised interactive segmentation.
To achieve it, as shown in the top half of \Cref{fig:motivation}, our basic idea is to switch previous object-oriented interaction simulation to some semantic-consistent regions, which can be discovered unsupervisedly.
In this case, the model requires diverse examples to understand changeable interactions due to the lack of precise information about objects.


Therefore, we propose \textbf{M}ulti-granularity \textbf{I}nteraction \textbf{S}imulation (\textbf{MIS}) to achieve diversity.
To ensure the semantic consistency of a region, we first parse the semantic of patches in an image by a Vision Transformer (ViT)~\cite{dosovitskiy2020image} pre-trained unsupervisedly with DINO~\cite{caron2021emerging}.
As shown in \Cref{fig:intuition}, we then gradually merge two patches or regions with similar features to form a more extensive region for an image, until only one region remains.
In this process, the newly generated region in every step makes up semantic-meaningful multi-granularity proposals.
The MIS is finally achieved by randomly sampling these proposals and simulating possible interactions based on them when training the model.
Moreover, in order to help the model fight against the unsmoothness on the boundary of proposals caused by the feature extractor, we design a smoothness constraint that considers the low-level feature of pixels.
With the meaningful interaction simulated at multiple granularities and the smoothness constraint, the model learns to understand clicks by related regions and produce an acceptable segmentation with several user interactions when inference, while completely discarding object annotations, as shown in the bottom half of \Cref{fig:motivation}.


We take the standard protocol~\cite{xu2016deep} to evaluate our method on commonly used datasets including GrabCut~\cite{rother2004grabcut}, Berkeley~\cite{mcguinness2010comparative}, SBD~\cite{hariharan2011semantic}, and DAVIS~\cite{perazzi2016benchmark}.
Benefiting from the rich interaction examples from our MIS, the trained model achieves inspiring performance.
Specifically, it significantly outperforms non-deep learning methods under the premise of unsupervised setting, and even surpasses some early deep supervised methods.
In addition, it can also perform as an unsupervised pre-training to improve the performance when only limited annotations are available.
The surprising results support that the interactive segmentation task can be solved in a more label-efficient way.
Moreover, we believe the proposed method can reduce the labeling costs in segmentation tasks by training an interactive segmentation model unsupervisedly and using it to efficiently make task-specific annotations as needed.


\begin{algorithm}[h]
   \caption{GCRL with planning + \highlight{\ALGname}}
   \label{alg:framework}
\begin{algorithmic}
\State {\bfseries Input}: Number of training episodes $M$, horizon $H$
\State Initialize replay buffer $\mathcal{B} \leftarrow \varnothing$.
\State Initialize the parameters of goal-conditioned policy $\pi_{\theta}$.
\State Initialize the parameters of action-value function $Q_{\phi}$.
\For{$m=1, 2, 3, \ldots M$}
\State Reset the environment.
\State Sample a target goal $g$ and an initial state $s_{0}$.
    \For{$t=1, 2, 3, \ldots H$}
    \State Build a graph $\mathcal{H} = (\mathcal{V}, \mathcal{E}, d)$ using $\mathcal{B}$.
    \State Find the shortest subgoal-path $\tau_{g}$ from $s_{t}$ to $g$.
    \State \highlight{Find a desired subgoal $l^{*}$ via Algorithm~\ref{alg:skip}.}
    \State Collect a transition $(s_{t}, a_{t}, r_{t})$ using $\pi_{\theta} (s_{t}, l^{*})$.
    \State Store the transition and the planned path $\tau_{g}$ in $\mathcal{B}$.
    \EndFor
\State Update $Q_{\phi}$ using $\mathcal{L}_{\mathtt{critic}} (\phi)$ of Equation~\ref{eq:ddpg_critic}
\State Update $\pi_{\theta}$ using $\mathcal{L}_{\mathtt{actor}} (\theta) + \highlight{\lambda \mathcal{L}_{\mathtt{\ALGname}} (\theta)} $ of Equation~\ref{eq:total_loss}
\EndFor
\end{algorithmic}
\end{algorithm}



\section{Related Works}


\noindent \textbf{Interactive Image Segmentation.}
Interactive segmentation has been an active field for over two decades due to its wide applications.
Before the era of deep learning, the solutions are based on low-level features and optimization.
One of the best-known methods is proposed by Boykov and Jolly~\cite{boykov2001interactive}, who define a graph to describe user-provided hard constraints and pixel relationship, then solve interactive segmentation by graph cut with max-flow algorithm~\cite{boykov2004experimental}.
After that, Rother \emph{et al.}~\cite{rother2004grabcut} augment graph cut by more powerful color data modeling and an iterative strategy.
Other algorithms include random walk~\cite{grady2006random}, geodesic distance~\cite{bai2007geodesic}, and star-convexity~\cite{gulshan2010geodesic}.
Due to the dependence on low-level features, these methods stuck when processing complex images.
Xu \emph{el al.}~\cite{xu2016deep} are the first to introduce deep learning on this task by extending FCN~\cite{long2015fully} with click map and click sampling strategies.
Based on this pipeline, the researchers gradually increase the performance ceiling of deep learning methods by emphasizing the first click~\cite{lin2020interactive}, online optimization~\cite{jang2019interactive,sofiiuk2020f}, diffusing prediction~\cite{chen2021conditional}, edge-guided flow~\cite{hao2021edgeflow}, iterative click sampling~\cite{sofiiuk2022reviving}, powerful feature extractor~\cite{liu2022simpleclick}, and Gaussian process posterior~\cite{zhou2023interactive}.
On the other hand, some works~\cite{lin2022focuscut,chen2022focalclick} focus on efficiency and propose to segment on local crops to accelerate inference.
Although the deep learning approaches achieve uplifting performance and efficiency, they require large-scale pixel-level annotations to train, which are expensive and laborious to obtain.
Unlike previous works, we focus on the dependency on annotations and show that a fully unsupervised framework can also handle interactive segmentation task well.


\smallskip


\noindent \textbf{Unsupervised Image Segmentation.}
With the development of self-supervised and unsupervised learning, unsupervised methods for image segmentation task start to emerge.
This work is also related to some unsupervised image segmentation methods based on the idea that extracting segments from a pre-trained high-quality dense feature extractor, (\emph{e.g.}, DenseCL~\cite{wang2021dense}, DINO~\cite{caron2021emerging}).
For instance, Simeoni \emph{et al.}~\cite{simeoni2021localizing} proposed a series of hand-made rules to choose pixels belonging to the same object according to their feature similarity.
Wang \emph{et al.}~\cite{wang2022self} introduce normalized cuts~\cite{shi2000normalized} on the affinity graph constructed by pixel-level representations from DINO to divide the foreground and background of an image.
Hamilton \emph{et al.}~\cite{hamilton2022unsupervised} train a segmentation head by distilling the feature correspondences from DINO.
Melas \emph{et al.}~\cite{melas2022deep} adopt spectral decomposition on the affinity graph to discover meaningful parts in an image.
For instance segmentation, FreeSOLO~\cite{wang2022freesolo} design pseudo instance mask generation based on multi-scale feature correspondences from densely pre-trained models and train an instance segmentation model with these pseudo masks.
Although the masks produced by these methods can serve as pseudo labels to train an interactive segmentation model unsupervisedly, only ordinary results are observed through our experiments in \Cref{sec:results}, which demonstrate that these task-specific designs is not apposite for interactive segmentation task.
In contrast, we propose a novel strategy to discover semantic-consistent yet diverse regions to fit the interactive segmentation task. 



\section{Method}


In this section, we introduce the overall framework to solve unsupervised interactive segmentation in detail.
Macroscopically, we simulate informative interactions from the proposed \textbf{M}ulti-granularity \textbf{I}nteraction \textbf{S}imulation (\textbf{MIS}) to train an interactive segmentation model. 
For implementation, the MIS is composed of two stages, as shown in \Cref{fig:framework}.
In the first pre-processing stage, we first parse the semantic of each patch in the image using a ViT~\cite{dosovitskiy2020image} trained self-supervisedly with DINO~\cite{caron2021emerging}, which has been proven to produce high-quality dense semantic features~\cite{simeoni2021localizing,hamilton2022unsupervised,melas2022deep}.
With these features, we take a gradual merging strategy to produce semantic-consistent region proposals at multiple granularities and save them efficiently with a tree structure.
In the training stage, given an image, we randomly select proposals from the corresponding merging tree in a top-down manner and further simulate possible clicks based on them, thus providing the model informative interactions for learning to understand interactions during the training process.


For training, the model is optimized by the provided interactions, which include meaningful regions and clicks.
Specifically, it takes the image and clicks as input and is optimized to produce similar segmentation as the region.
Moreover, we design a smoothness constraint when optimizing, for the purpose of mitigating the misleading brought by the inaccurate boundary of the region, which is caused by the down-sampling in the ViT.
The details of MIS and the training objective are described as follows.



\subsection{Multi-granularity Interaction Simulation}
\label{sec:proposal}


\noindent \textbf{Bottom-up Merging.}
At the core of our MIS is to discover semantic-consistent yet diverse regions, which is achieved efficiently by gradually merging.
In the beginning, we produce high-quality semantic features from a ViT~\cite{dosovitskiy2020image} trained in a self-supervised manner with DINO~\cite{caron2021emerging} to measure the semantic similarity and ensure the semantic-consistency of generated region proposals.
In the ViT, an image $ \mI \in \sR^{h \times w \times 3} $ is divided to non-overlapping $ n = \frac{h}{s} \times \frac{w}{s} $ patches with stride $ s $, which are then processed by some Transformer~\cite{vaswani2017attention} encoder layers.
The patch features $ \mF \in \sR^{n \times c} $ are obtained from the last layer of the ViT, where $ c $ is the feature dimension.
With the patch features, we implement the hierarchical merging process by Algorithm~\ref{algo:merging} and finally get a tree $ \bm{T} $ which records every merge process.


Initially, $ n $ patches form as $ n $ leaf nodes of the tree, and a set $ \sS $ is maintained for saving unmerged nodes, which initially contains all the leaf nodes.
In each iteration, the two unmerged nodes with the least cost according to their features will be merged to form a new node, until only one unmerged node remains.
Thus, there should be totally $ n - 1 $ iterations, and the index of the newly generated node in $ k $-th iteration is $ n + k $.
Moreover, considering the prior that parts belonging to the same object are usually adjacent, we add connectivity constraints when merging to greatly accelerate the search space for finding the minimum cost.
Specifically, each patch is only connected to its four adjacent patches and two regions are connected only if at least a pair of patches from them respectively are connected.


\begin{figure}[t]
    \centering
    \includegraphics[width=\linewidth]{figs/ward_src.pdf}
    \caption{
        \textbf{Illustration of the cost of merging.}
        The points represent patch features and the circles denote regions that consist of the patches surrounded by it.
        Merging two regions into one leads to a more divergent cluster and the total SSE increases, thus the cost of merging can be measured as this increment.
    }
    \label{fig:ward}
\end{figure}



We follow Ward's method~\cite{ward1963hierarchical} to measure the cost of merging, which minimizes the increment in the total within-cluster sum of squared error (SSE) after merging, as shown in \Cref{fig:ward}.
Regarding a region $ \cA $ as a cluster that contains some patches, the SSE is defined as
\begin{equation}
    e_\cA = \sum_{i \in \cA} ||\mF_i - \vmu_\cA||^2,
\end{equation}
where $ \vmu_\cA \in \sR ^ c $ is the center of it, which can be computed as
\begin{equation}
    \vmu_\cA = \frac{1}{s_\cA} \sum_{i \in \cA} \mF_i,
\end{equation}
where $ s_\cA $ is the number of patches within $ \cA $.
When merging two regions, the new center is
\begin{equation}
    \vmu_{\cA \cup \cB} = \frac{s_\cA \vmu_\cA + s_\cB \vmu_\cB}{s_\cA + s_\cB}.
\label{eq:center}
\end{equation}
Then the cost of merging is defined as the difference of SSE before and after merging.
Since the SSE of all clusters except the clusters being merged are unchanged, it can be formulated as
\begin{equation}
\begin{aligned}
    \mathrm{Cost}(\cA, \cB) &= e_{\cA \cup \cB} - (e_\cA + e_\cB) \\
                &= \frac{s_\cA \cdot s_\cB}{s_\cA + s_\cB} ||\vmu_\cA - \vmu_\cB||^2.
\end{aligned}
\label{eq:cost}
\end{equation}
During the merging process, the center $ \vmu $ and the size $ s $ of each region are recorded.
Initially, the size is set to 1 for each patch and the center is its feature.
The detailed calculation after each merging is as follows:
\noindent \textbf{(1)} Computing the center of the newly generated region according to \Cref{eq:center}.
\noindent \textbf{(2)} Computing the size of the newly generated region as $ s_{\cA \cup \cB} = s_\cA + s_\cB $.
\noindent \textbf{(3)} Computing the cost of merging newly generated regions with other regions according to \Cref{eq:cost}.
Since the other regions are not changed, the cost between them remains the same.


\begin{algorithm}[t]
  \caption{Bottom-up Merging}
  \label{algo:merging}

  \SetKwInOut{Input}{Input}
  \SetKwInOut{Output}{Output}

  \Input{Patch features $ \mF \in \sR^{n \times c} $}
  \Output{Merge tree $ \mT \in \sZ^{(n - 1) \times 2} $}

  \emph{Initialization: $ \sS \leftarrow \{1,2,\dots,n\} $}

  \For{$ k \leftarrow 1 $ \KwTo $ n - 1 $}{
    $ i,j \leftarrow \underset{i,j \in \bm{S}}{\arg\min} \ \mathrm{Cost}(i,j,\mF) \ \mathrm{s.t.}\ \mathrm{Connect}(i,j) $\;
    $ \mT_{k, :} \leftarrow (i, j) $\;
    $  $
    $ \sS \leftarrow \complement_{\sS}{\{i, j\}} \cup \{n + k\} $\;
  }
\end{algorithm}

\algrenewcommand\algorithmicindent{1.0em}%
\begin{algorithm}[H]
  \caption{\textbf{Sampling}} \label{alg:sampling}
  \small
\begin{algorithmic}[1]
     \State Trained diffusion model $\theta$, $\bm{x}_T \sim \mathcal{N}(\bm{0}, \bm{I})$
    \For{$t=T, \dotsc, 1$}
      \State $\hat{\bm{\epsilon}}_\theta = \bm{\epsilon}_\theta(\bm{x}_t, t) + s \cdot (\bm{\epsilon}_{\theta}(\bm{x}_t, \bm{c}, t) - \bm{\epsilon}_\theta(\bm{x}_t, t))$
        \State $\bm{z}_0(t) \sim \mathcal{N}(\bm{0}, \sigma^2_a(t)  \bm{I})$ 
        \For{$i=1, \dotsc, N$}
        \State $\bm{z}_i(t) \sim \mathcal{N}(\bm{z}_0(t) , (1-\sigma^2_a(t)) \bm{I})$ 
    \EndFor
      \State $\bm{z}(t)=\{\bm{z}_1(t),\dots,\bm{z}_N(t)\}$, if $t > 1$, else $\bm{z}(t) = \bm{0}$
      \State $\bm{x_{t-1}} = \frac{1}{\sqrt{\alpha_t}}\left( \bm{x_t} - \frac{1-\alpha_t}{\sqrt{1-\bar\alpha_t}} \hat{\bm{\epsilon}}_\theta \right) + \sigma_t \bm{z}(t)$
    \EndFor
    \State \textbf{return} $\bm{x}_0$
  \end{algorithmic}
\end{algorithm}

\begin{figure*}[t]
    \centering
    \includegraphics[width=\textwidth]{figs/proposal_src.pdf}
    \caption{
        \textbf{Qualitative results of discovered regions for the baselines and our MIS.}
        While the baselines only focus on some salient regions in a fixed granularity, our MIS produces semantic-consistent regions at multiple granularities thus catching more objects and parts.  
    }
    \label{fig:proposal}
\end{figure*}



After $ n - 1 $ iterations, we get a tree structure that is stored by a matrix $ \mT \in \sZ^{(n - 1) \times 2} $, where the $ i $-th row is the indices of the two children of the $ i + n $ node.
In the tree, each node represents a merge and also represents a region proposal which consists of all patches within the subtree rooted at this node.
Hence, with this hierarchical relationship, we obtain $ n - 1 $ region proposals at multiple granularities for each image, which help the interactive segmentation model to learn various interactions.
For storage, there is no need to store all the $ n - 1 $ masks but only the tree structure, that is, the matrix $ \mT $, which makes it more space-efficient.


\smallskip


\noindent \textbf{Top-down Sampling.}
When training, the MIS is worked based on the stored merging tree.
Specifically, when sampling an image for training, we select a mask proposal from its corresponding merging tree for simulating an interaction. 
In the merging tree, shallow nodes correspond to relatively complete parts or their combinations and deeper nodes are more relevant to finely divided parts.
Intuitively, we prefer the complete one while maintaining diversity.
Therefore, we design a sampling strategy in a top-down manner and gradually decrease the probability to go deeper, as shown in Algorithm~\ref{algo:sampling}.
Concretely, we start at the root node (\emph{i.e.}, the node with an index of $ 2n - 1 $) and the probability $ p = 1.0 $.
Then in each iteration, we choose to go deeper or stop with probability $ p $, and decrease $ p $ with a coefficient $ \alpha $.
The loop will terminate when choosing to stop or reach a leaf node (\emph{i.e.}, nodes whose index is less than or equal to $ n $).
If it is decided to go deeper, we randomly move the current node to one of its children.
Finally, we get the index of a node $ k $ corresponding to a region proposal, where the patches in the sub-tree rooted at $ k $ are annotated as foreground.


\smallskip


\noindent \textbf{Click Sampling.}
With the sampled region, the last step is to simulate possible clicks according to the region.
Drawing on the experience of previous methods, we use a combination of random and iterative click simulation strategies to produce a click map following RITM~\cite{sofiiuk2022reviving}.
Specifically, some positive and negative clicks are first generated in parallel according to the foreground and background defined by the region.
Then some additional clicks are appended by checking the erroneous region of the prediction which is produced by previous clicks.
Finally, the MIS generates complete interactions which consist of a segmentation target and corresponding clicks.



\subsection{Training}


\noindent \textbf{Interactive Segmention Model.}
Typical interactive segmentation models are built based on semantic segmentation models (\emph{e.g.}, FCN~\cite{long2015fully}) by taking an additional click map as input, which is transformed from the coordinate of clicks.
Formulaically, given an image $ \mI \in \sR^{h \times w \times 3} $, a click map $ \mD \in \sR^{h \times w \times 2} $ which presents positive and negative clicks respectively, and previous prediction $ \mB \in \{0, 1\}^{h \times w} $, the model $ \mathcal{F}(\cdot) $ outputs the probability of being foreground for each pixel
\begin{equation}
    \mQ = \mathcal{F}(\mathrm{Fusion}(\mI, \mD, \mB)),\ \mQ \in [0,1]^{h \times w},
\end{equation}
where $ \mathrm{Fusion}(\cdot) $ is some kind of fusion operation such as concatenate.
Through training, the model learns to map the image and the user clicks to foreground and background segmentation thus can be used interactively when inference.


\smallskip



%\iffalse 
\begin{table*}[t]
\centering
\begin{tabular}{c|c|c|c|c|c||c|c|c}
\toprule
\multirow{2}{*}{Backbone} &    \multirow{2}{*}{Method} & \multicolumn{4}{c||}{ID: Pascal} & \multicolumn{3}{c}{ID: Cityscapes}  \\
 &           & Comic & Watercolor & Clipart & ID                           & Foggy & BDD  & ID    \\\hline
\multirow{4}{*}{ResNet50 Instagram~\cite{mahajan2018exploring}}&DP                                   & 15.7  & 21.2       & 15.3    & 44.6 &                                    13.9 & 7.7 & 28.3  \\
&FT                                   & 7.5   & 19.4       & 11.4    & 50.4 &                                     12.8  & 5.1  & 33.5 \\
%&FT + Augmix                          & 10.2  & 21.9       & 12.4    & 46.3 &                                    &       &             \\
& \CCG DP-FT                                &\CCG 9.1   & \CCG21.0       & \CCG12.9    &\CCG 52.6 &\CCG14.8  &\CCG 5.5  &\CCG\bf{34.7}  \\
 &\CCG DP-FT + WR           &\CCG \bf{16.8}  &\CCG \bf{26.5}       &\CCG \bf{17.6}    &\CCG \bf{52.9} &\CCG \bf{19.3}  &\CCG \bf{9.6}  &\CCG 34.5   \\\hline
 %&\CCG All                &\CCG \bf{18.9}  &\CCG \bf{27.5}       &\CCG \bf{21.4}    &\CCG 52.2 &\CCG  \\\hline                                     
\multirow{4}{*}{ConvNeXt IN21K~\cite{liu2022convnet}}&DP                                   & 11.7  & 17.3       & 14.0    & 39.7 &                                     14.7 & 7.8 & 31.1 \\
&FT                              &      11.5  & 22.9       & 16.8    & 60.6 &  18.1 & 9.7 & 35.8    \\
%&FT + Augmix                  &         15.5  & 28.6       & 20.7    & 61.4 &                                    &      &             \\
&\CCG  DP-FT                   & \CCG              13.6  &\CCG  24.7       &\CCG  19.1    &\CCG  \bf{62.3} & \CCG                                   20.5&\CCG 11.5&\CCG   37.1\\
%&\CCG  DP-FT + Seblock         &\CCG               15.4  &\CCG  27.5       &\CCG  20.9    &\CCG  61.6 &\CCG   \bf{22.0}  &\CCG  11.3 &\CCG  36.6            \\
&\CCG DP-FT + WR     &\CCG       \bf{14.6}  &\CCG  \bf{27.8}       &\CCG  \bf{19.7}    &\CCG  61.4 & \CCG                          \textbf{21.1}          &\CCG    \textbf{11.7}     &\CCG  \bf{37.2}   \\\hline
%&\CCG  All              &\CCG   \bf{17.8}  &\CCG  \bf{29.3}       &\CCG  \bf{23.8}    &\CCG  \bf{61.0}   &\CCG  21.7&\CCG  \bf{11.8}&\CCG  37.2            \\\hline
%&DP-FT + Reg + Augmix & 21.0  & 27.9       & 22.6    & 50.4 &            \\    \hline
\multirow{4}{*}{Eff-B2 JFT~\cite{xie2020self}}&  DP                                   & 12.6  & 20.4       & 15.1    & 40.2 &                                     11.1 & 6.9 & 25.2 \\
&FT                                   & 17.1  & 27.2       & 18.0    & 53.4 &          10.7                          &  5.1     &     31.5        \\
%&FT + Augmix                          & 20.1  & 31.2       & 19.7    & 52.7 &                                    &       &             \\
&\CCG  DP-FT                                &\CCG  17.4  &\CCG  29.4       &\CCG  20.7    &\CCG \bf{55.3} &\CCG   12.9&\CCG  7.3&\CCG   \bf{32.9}\\
&\CCG  DP-FT + WR        &\CCG  \bf{19.5}  &\CCG  \bf{30.0}         &\CCG  \bf{22.0}      &\CCG  54.2 &\CCG \bf{13.5} &\CCG  \bf{7.6} &\CCG  32.5\\

% \bf{13.1}&\CCG  \bf{7.5}&\CCG  32.7\\
\bottomrule
\end{tabular}
\vspace{-3mm}
\caption{Effect of weight regularization. DP, FT, and WR denote decoder-probing, fine-tuning, and weight regularization.} 
\label{tb:main}
\end{table*}

\iffalse 

\begin{table*}[]
\centering
\begin{tabular}{c|c|c|c|c|c|c}
\toprule
 \multirow{1}{*}{Backbone} &    \multirow{1}{*}{Method}           & Comic & Watercolor & Clipart & OOD Average& ID: Pascal          \\\hline
\multirow{6}{*}{ResNet50 Instagram}&DP                                   & 15.7  & 21.2       & 15.3    & 17.4&44.6   \\
&FT                                   & 7.5   & 19.4       & 11.4    & 12.8&50.4 \\
%&FT + Augmix                          & 10.2  & 21.9       & 12.4    & 46.3 &                                    &       &             \\
& \CCG DP-FT                                &\CCG 9.1   & \CCG21.0       & \CCG12.9  &\CCG 14.3  &\CCG 52.6 \\
&\CCG DP-FT + Seblock                      &\CCG 10.2  & \CCG 22.5       &\CCG 15.1    &\CCG 15.9 & \CCG \bf{53.4} \\
 &\CCG DP-FT + Reg           &\CCG 16.8  &\CCG 26.5       &\CCG 17.6    &\CCG 19.9&\CCG 52.9 \\
 &\CCG All                &\CCG \bf{18.9}  &\CCG \bf{27.5}       &\CCG \bf{21.4} & \CCG \bf{22.6} &\CCG 52.2 \\\hline                                     
\multirow{6}{*}{Convnext IN21K}&DP                                   & 11.7  & 17.3       & 14.0    & 14.3&39.7 \\
&FT                              &      11.5  & 22.9       & 16.8    &17.1& 60.6    \\
%&FT + Augmix                  &         15.5  & 28.6       & 20.7    & 61.4 &                                    &      &             \\
&\CCG  DP-FT                   & \CCG              13.6  &\CCG  24.7       &\CCG  19.1    &\CCG19.1&\CCG  62.3 \\
&\CCG  DP-FT + Seblock         &\CCG               15.4  &\CCG  27.5       &\CCG  20.9    &\CCG 21.3&\CCG  61.6 \\
&\CCG DP-FT + Reg     &\CCG       14.6  &\CCG  27.8       &\CCG  19.7    &\CCG20.7&\CCG  61.4 \\
&\CCG  All              &\CCG   \bf{17.8}  &\CCG  \bf{29.3}       &\CCG23.6&\CCG  \bf{23.8}    &\CCG  \bf{61.0}   \\\hline
%&DP-FT + Reg + Augmix & 21.0  & 27.9       & 22.6    & 50.4 &            \\    \hline
\multirow{4}{*}{Eff-B2 JFT}&  DP                                   & 12.6  & 20.4       & 15.1    &  16.0&40.2 \\
&FT                                   & 17.1  & 27.2       & 18.0    &20.8& 53.4 \\
%&FT + Augmix                          & 20.1  & 31.2       & 19.7    & 52.7 &                                    &       &             \\
&\CCG  DP-FT                                &\CCG  17.4  &\CCG  29.4       &\CCG  20.7    &\CCG 22.5&\CCG  \bf{55.3} \\
&\CCG  DP-FT + WR        &\CCG  \bf{19.5}  &\CCG  \bf{30.0}         &\CCG  \bf{22.0}      &\CCG 23.8&\CCG  54.2 \\
\bottomrule
\end{tabular}
\caption{Improvements by introducing our proposed modules.}
\label{tb:main}
\end{table*}
\fi

%\fi


\iffalse 
\begin{table*}[]
\centering
\begin{tabular}{c|c|c|c|c|c||c|c|c}
\toprule
\multirow{2}{*}{Model} &    \multirow{2}{*}{Method} & \multicolumn{4}{c||}{ID: Pascal} & \multicolumn{3}{c}{ID: Cityscape}  \\\cline{3-9}
 &           & Comic & Watercolor & Clipart & ID                           & Foggy & BDD  & ID    \\\hline
\multirow{6}{*}{ResNet50 Instagram}&DP                                   & 15.7  & 21.2       & 15.3    & 44.6 &                                    13.9 & 7.66 & 28.3  \\
&FT                                   & 7.5   & 19.4       & 11.4    & 50.4 &                                     12.8  & 5.1  & 33.52 \\
%&FT + Augmix                          & 10.2  & 21.9       & 12.4    & 46.3 &                                    &       &             \\
& \CCG DP-FT                                &\CCG 9.1   & \CCG21.0       & \CCG12.9    &\CCG 52.6 &\CCG14.8  &\CCG 5.5  &\CCG34.7  \\
&\CCG DP-FT + SE                      &\CCG 10.2  & \CCG 22.5       &\CCG 15.1    &\CCG \bf{53.4} &\CCG                                    &\CCG       &\CCG             \\
 &\CCG DP-FT + WR           &\CCG 16.8  &\CCG 26.5       &\CCG 17.6    &\CCG 52.9 &\CCG \bf{19.3}  &\CCG \bf{9.6}  &\CCG 34.5   \\
 &\CCG All                &\CCG \bf{18.9}  &\CCG \bf{27.5}       &\CCG \bf{21.4}    &\CCG 52.2 &\CCG  \\\hline                                     
\multirow{6}{*}{Convnext IN21K}&DP                                   & 11.7  & 17.3       & 14.0    & 39.7 &                                     14.7 & 7.8 & 31.1 \\
&FT                              &      11.5  & 22.9       & 16.8    & 60.6 &  18.1 & 9.7 & 35.8    \\
%&FT + Augmix                  &         15.5  & 28.6       & 20.7    & 61.4 &                                    &      &             \\
&\CCG  DP-FT                   & \CCG              13.6  &\CCG  24.7       &\CCG  19.1    &\CCG  62.3 & \CCG                                   20.5&\CCG 11.5&\CCG  37.1\\
&\CCG  DP-FT + Seblock         &\CCG               15.4  &\CCG  27.5       &\CCG  20.9    &\CCG  61.6 &\CCG   \bf{22.0}  &\CCG  11.3 &\CCG  36.6            \\
&\CCG DP-FT + Reg     &\CCG       14.6  &\CCG  27.8       &\CCG  19.7    &\CCG  61.4 & \CCG           20.9                         &\CCG   11.8     &\CCG  \bf{37.5}             \\
&\CCG  All              &\CCG   \bf{17.8}  &\CCG  \bf{29.3}       &\CCG  \bf{23.8}    &\CCG  \bf{61.0}   &\CCG  21.7&\CCG  \bf{11.8}&\CCG  37.2            \\\hline
%&DP-FT + Reg + Augmix & 21.0  & 27.9       & 22.6    & 50.4 &            \\    \hline
\multirow{4}{*}{Eff-B2 JFT}&  DP                                   & 12.6  & 20.4       & 15.1    & 40.2 &                                     11.1 & 6.9 & 25.2 \\
&FT                                   & 17.1  & 27.2       & 18.0    & \bf{53.4} &          10.7                          &  5.1     &     31.5        \\
%&FT + Augmix                          & 20.1  & 31.2       & 19.7    & 52.7 &                                    &       &             \\
&\CCG  DP-FT                                &\CCG  17.4  &\CCG  29.4       &\CCG  20.7    &\CCG  55.3 &\CCG   12.9&\CCG  7.3&\CCG   \bf{32.9}\\
&\CCG  DP-FT + Reg        &\CCG  \bf{19.5}  &\CCG  \bf{30.0}         &\CCG  \bf{22.0}      &\CCG  54.2 &\CCG  \bf{13.1}&\CCG  \bf{7.5}&\CCG  32.7\\
\bottomrule
\end{tabular}
\caption{Improvements by Applying weight regularization.}
\label{tb:main}
\end{table*}
\fi



\smallskip


\noindent \textbf{Training with Proposals.}
Under our unsupervised setting, the model learns this mapping by the simulated interactions from the MIS.
The model is first learned by the supervision from the multi-granularity region proposals $ \mM $ by optimizing the objective
\begin{equation}
    \mathcal{L}_{pseudo} = \mathrm{BCE}(\mM, \mQ),
\end{equation}
where $ \mathrm{BCE}(\cdot) $ denotes the binary cross entropy loss.
Although the region proposals generated by the MIS contains semantics, the possible inaccuracy of the self-supervised feature extractor and the patching process in the ViT will bring some inaccurate part.
Fortunately, as discovered by previous research~\cite{arpit2017closer} that the model tends to learn effect patterns rather than memorize the noise, we find that the model can learn to correctly map clicks to segmentation after training with these simulated interactions.


\smallskip


\noindent \textbf{Smoothness Constraint.}
In order to help the model correct the error caused by patching, we propose a smoothness constraint.
We first introduce the bilateral affinity matrix~\cite{tomasi1998bilateral,barron2016fast} $ \mW $ which presents the affinity of pixels based on their low-level feature and the item $ \mW_{i,j} $ is
\begin{equation}
    exp(-\frac{||\va_i^{xy} - \va_j^{xy}||^2}{2\sigma_{xy}^2} - \frac{(a_i^l - a_j^l)^2}{2\sigma_l^2} - \frac{||\va_i^{uv} - \va_j^{uv}||^2}{2\sigma_{uv}^2}), \nonumber
\end{equation}
where $ \va_i $ is a pixel with a spatial position $ (p_i^x, p_i^y) $ and color $ (p_i^l, p_i^u, p_i^v) $ and $ \sigma_{xy}, \sigma_l, \sigma_{uv} $ control the extent of the spatial, luma, and chroma support of the filter.
Motivated by the bilateral solver~\cite{barron2016fast}, the smooth constraint is implemented by restricting the model to produce consistent predictions on pixels with high bilateral affinity, which can be formulated as
\begin{equation}
    \mathcal{L}_{smooth} = \sum_i \frac{1}{|\mathcal{N}_i|} \sum_{j \in \mathcal{N}_i} \mW_{ij} \cdot (\mQ_i - \mQ_j)^2.
\end{equation}
As we mainly concern with the predictions on the boundary, we only perform the loss locally for fast computation and memory efficiency, \emph{i.e.}, for pixel $ i $ the loss is calculated in its $ 5 \times 5 $ neighborhoods $ \mathcal{N}_i $.
Finally, the overall training objective is
\begin{equation}
\label{eq:loss}
    \mathcal{L} = \mathcal{L}_{pseudo} + \lambda \cdot \mathcal{L}_{smooth},
\end{equation}
where $ \lambda $ controls the strength of the smoothness constraint.


\begin{figure*}[t]
    \centering
    \includegraphics[width=\textwidth]{figs/curve_src.pdf}
    \caption{\textbf{Comparison of IoU-NoC curve.} The two dotted lines indicate the performance of supervised approaches. The proposed method achieves surprising performance that significantly exceeds traditional methods and is close to supervised method when the NoC increases.}
    \label{fig:curve}
\end{figure*}

\begin{figure}
       \centering
        \setlength{\tabcolsep}{1pt}
        {\scriptsize
        \begin{tabular}{c c c c c c c }
            { Original } &
            \multicolumn{2}{c}{  } &
            \multicolumn{4}{c}{$\longleftarrow$ Object level variations $\longrightarrow$} \\
            \includegraphics[width=0.185\linewidth]{images/ablation/chair.jpg} &
            \multicolumn{2}{c}{  } &
            \includegraphics[width=0.185\linewidth]{images/ablation/1_only_prompt_mixing/bench.jpg} &
            \includegraphics[width=0.185\linewidth]{images/ablation/1_only_prompt_mixing/stool.jpg} &
            \includegraphics[width=0.185\linewidth]{images/ablation/1_only_prompt_mixing/armchair.jpg} &
            \includegraphics[width=0.185\linewidth]{images/ablation/1_only_prompt_mixing/saddle.jpg} \\
            \multicolumn{3}{c}{  } &
            \multicolumn{4}{c}{ Only Prompt Mixing } \\
            \multicolumn{3}{c}{ } &
            \includegraphics[width=0.185\linewidth]{images/ablation/2_with_self_attn_injection/bench.jpg} &
            \includegraphics[width=0.185\linewidth]{images/ablation/2_with_self_attn_injection/stool.jpg} &
            \includegraphics[width=0.185\linewidth]{images/ablation/2_with_self_attn_injection/armchair.jpg} &
            \includegraphics[width=0.185\linewidth]{images/ablation/2_with_self_attn_injection/saddle.jpg} \\
            \multicolumn{3}{c}{  } &
            \multicolumn{4}{c}{ + Attention-Based Shape Localization } \\
            \multicolumn{3}{c}{ } &
            \includegraphics[width=0.185\linewidth]{images/ablation/3_background_blending/bench.jpg} &
            \includegraphics[width=0.185\linewidth]{images/ablation/3_background_blending/stool.jpg} &
            \includegraphics[width=0.185\linewidth]{images/ablation/3_background_blending/armchair.jpg} &
            \includegraphics[width=0.185\linewidth]{images/ablation/3_background_blending/saddle.jpg} \\
            \multicolumn{3}{c}{  } &
            \multicolumn{4}{c}{ + Controllable Background Preservation } \\
        \end{tabular}
        }
    \vspace{1mm}
    \captionof{figure}{
    Ablating our full object variations pipeline. Original image was crated using the prompt ``A \emph{chair} with a dog on it''. 
    }
    \vspace{-10pt}
    \label{fig:ablation}
\end{figure}



\section{Experiments}


\noindent \textbf{Implementation Details.}
We follow SimpleClick~\cite{liu2022simpleclick} to build the interactive segmentation model, which consists of two patch embedding modules for image and click map respectively, a ViT~\cite{dosovitskiy2020image} backbone initialized with MAE~\cite{he2022masked}, a simple feature pyramid~\cite{li2022exploring}, and an MLP segmentation head.
For training, we optimize the model for 55 epochs using the Adam~\cite{kingma2014adam} optimizer with a learning rate of 5e-5, which decays by a factor of 10 at 50 epochs.
The decay coefficient $ \alpha $ in Algorithm~\ref{algo:sampling} and the weight $ \lambda $ in \Cref{eq:loss} are 0.9 and 10.0, respectively.
In addition, we set a threshold of 0.05 to filter small proposals.
That is to say, we ignore the proposals whose areas are less than 0.05 times the image area.
For inference, we set the threshold of binarizing the prediction to 0.5 as normal binary segmentation and use the same pre-processing as \cite{liu2022simpleclick}.
To ensure the fully unsupervised setting, we employ a self-supervisedly pre-trained model as the feature extractor.
Specifically, a Vision Transformer (ViT)~\cite{dosovitskiy2020image} trained with DINO~\cite{caron2021emerging} is adopted since the property of representing patch-level semantics~\cite{caron2021emerging,hamilton2022unsupervised,simeoni2021localizing}.
In order to prevent the extracted features from ignoring small objects or parts, we use ViT-Small~\cite{simeoni2021localizing} with a patch size of 8.
Before feeding an image into the ViT, we resize the image to make its height and width divisible by the patch size (\emph{i.e.}, the target height is $ ( h + p - h \mod p ) $ when the original height $ h $ is not divisible by the patch size $ p $), and the position embedding of the ViT is interpolated to fit the image size using bilinear interpolation.
We finally use all output tokens of the last block except the \emph{cls} token as the patch features.


\smallskip


\noindent \textbf{Evaluation.}
The evaluation is done under the standard evaluation protocol of click-based interactive segmentation.
Specifically, we adopt the same click simulator as previous work~\cite{xu2016deep,jang2019interactive,lin2020interactive,sofiiuk2020f,chen2021conditional,sofiiuk2022reviving,lin2022focuscut,chen2022focalclick} to sample clicks when evaluation.
Roughly, the next click will be placed at the center of the largest error region by comparing the ground-truth and prediction.
We adopt the Number of Click (NoC) as the evaluation metric, which counts the average number of clicks needed to achieve a fixed Intersection over Union (IoU), the smaller the better.
We use two commonly used target intersection-over-union (IoU) thresholds 85\% and 90\%, denoted as NoC@85 and NoC@90 respectively.
Additionally, the IoU-NoC curves are also adopted to represent the convergence trend of IoU when the NoC increases, as well as compare the IoU metric under the same NoC.


\smallskip


\noindent \textbf{Dataset.}
We evaluate the performance of our MIS on the following four benchmarks.
(1) \textbf{GradCut}~\cite{rother2004grabcut}: The dataset contains 50 relatively easy images whose background and foreground have a clear difference.
(2) \textbf{Berkeley}~\cite{mcguinness2010comparative}: The dataset contains 96 images with 100 instances and some of them are more challenging than GrabCut.
(3) \textbf{SBD}~\cite{hariharan2011semantic}: The dataset contains 2,857 images with 6,671 challenging instances for evaluation, which does not overlap with our training set.
(4) \textbf{DAVIS}~\cite{perazzi2016benchmark}: The dataset contains 50 high-quality videos. We follow previous works~\cite{lin2022focuscut,chen2022focalclick,liu2022pseudoclick,sofiiuk2022reviving,liu2022simpleclick} and use the same 345 frames for evaluation.


\section{Visualization On Demand} %Visualization Elements
\label{sec:visrisk}
Based on environment data and trajectory evaluation, we now present ways of communicating the situation and risks on a visual display to achieve an ADAS.
In this context, we employ a renderer that visualizes all the information in a joint Cartesian coordinate system (see section \ref{subsec:sim}). 
Once driving risks are detected, design elements are overlayed on the display with section \ref{subsec:active} and section \ref{subsec:warning}. 

\subsection{Simulator Environment}
\label{subsec:sim}
Nodes of the R-LDM have a range of potential attributes, such as the 3D position or geometrical shape of objects. 
% For instance, the road centerline is a polyline with bounderies to the left and right. Crosswalks have a defined width and buildings a polygonal outline description. 
In the renderer, we always visualize static and quasi-static data that lie in the field of view from the ego vehicle. 
For this, a local 3D model is generated by converting geographic points with (lat, lon, alt) into Cartesian coordinates of (x, y, z). 
% and project the positonal relations from a view perspective with a transformation matrix. 
Fig. \ref{fig:3Dsimulator} depicts an exemplary map section having several intersections in bird's-eye view.
% with several intersections, stop lines and crosswalks. 
On the top right, the first person view of a vehicle approaching a crosswalk is shown. 

The dynamic data is then added to this static view. A zoomed-in excerpt from the map is given at the bottom of Fig. \ref{fig:3Dsimulator} that includes a recorded GNSS trace (red).
We project the trace onto the connected lane center, which is pictured in green. 
% Because we project the ego position on the closest lane segment, on the bottom right the measured trace is changed in red and the aligned trace is marked in green.
Consequently, the virtual horizon and its possible paths are retrieved as described in section \ref{subsec:ldm}. 
We can lastly update and move the excerpt with the current position from the GNSS to obtain a live simulation.

\subsection{Proactive Support}
\label{subsec:active}
Communication of spatial as well as spatio-temporal relations is crucial for risk-averse driver support. 
% This has the reason that humans can estimate the time better than positions (especially for risks). 
% Velocity contains implicitly the time as well. 
Further sources of information are cause, likelihood and severity of a potential risks.  
% if a collision happens. 
The next step for RNS is the choice of suitable design elements. 
In this process, we suppose that we know where the ego vehicle is driving (i.e., the ego path) from its navigation route. 
Yet, for surrounding vehicles, all paths are considered.

\subsubsection{Hazard Route Element}
The so-called hazard route in Fig. \ref{fig:charts} is a concept that consists of a scale portraying distances to an upcoming risk element.
Furthermore, the geometrical area or length of risks is considered.
Risk is thus measured with respect to the ego path, ranging from the current position  $\Delta l \hspace{-0.03cm}=\hspace{-0.03cm} \unit[0]{m}$ to the end of the path $\Delta l_{h}$.
Here, the length $\Delta l_{h}$ can be chosen according to own preferences. 

At an upcoming intersection, risk is defined by the section of the path that lies within the junction.
Since risk corresponds to exposition time, we encode the path part from the intersection $I_z$ with a color, ranging from green for short intersections to red for long ones. 
%allgemein risiko entlang des pfades zu intersection zone
%share of junction segment to navigation route + 
%one case with large intersection far and one case with small intersection close
Fig. \ref{fig:charts}~a) gives two examples of the hazard route.
The left bar shows a large intersection (e.g. multi-lane four-way stop) in vicinity and the right bar has a small and consecutive medium junction. 
% In the case of collision risk, the intersection zone $I_z$ can be used.
% Depending on the value of $I_z$ (low, medium and large), the area is marked from green, to yellow until red for conveying the criticality. 
This emphasizes that we may include more than one intersection in our warnings.

\begin{figure}[t]
  \centering
  \includegraphics[width=0.95\linewidth]{./img/simulator.png}
  \caption{Rendered road network from two perspectives with the ego position being projected on the navigation route. \vspace{0.45cm}}
  \label{fig:3Dsimulator}
\end{figure}

\begin{figure}[t]
  \centering
  \resizebox{\linewidth}{!}{
  \import{img/}{velocity_scale_new.pdf_tex}}  
  \caption{Chart elements for proactive support. Hazard route (left) and velocity scale (right).} %\vspace{-0.3cm}}
  \label{fig:charts} 
\end{figure} 

\subsubsection{Velocity Scale Element}
The velocity scale, Fig. \ref{fig:charts}~b), is a second chart element which qualifies the difference between the current velocity of the vehicle $v_0$ and the target velocity $v_{\text{tar}}$ from the trajectory evaluation of section \ref{subsec:trajeval}. 
The scale shows possible velocity values, from standstill $v\hspace{-0.05cm}=\hspace{-0.05cm}\unit[0]{m/s}$ to a maximal velocity $v_{\text{max}}$. Depending on the difference $|v_0 \hspace{0.05cm} - \hspace{0.05cm} v_{\text{tar}}|$, the situation is rated as safe with $v_0 \hspace{-0.042cm} \approx \hspace{-0.042cm} v_{\text{tar}}$ (green, left), as dangerous with e.g. $v_0 \hspace{-0.05cm} < \hspace{-0.05cm} v_{\text{tar}}$ (yellow, middle) to critical with $v_0 \hspace{-0.07cm} \ll \hspace{-0.07cm} v_{\text{tar}}$ (red, right). The same cases hold true for the opposite circumstances, i.e., $v_0 \hspace{-0.032cm} > \hspace{-0.032cm} v_{\text{tar}}$. 
This velocity scale can be employed for curve or regulatory risks. 
Moreover, we may set an enforced speed limit as the target velocity $v_{\text{tar}}$ for proactive behavior, once there is no risk ahead. 
%\noindent -Warning vs behavior support \\
%-Ghost vehicle as in game \\

\subsection{Short-Term Warning Elements}
\label{subsec:warning}
In order to emphasize the criticality of the situation, we propose to add further intuitive warning elements as e.g. pop-up signs and lane colorings. 
The following elements augment the proactive elements.

\subsubsection{Pop-up Signs}
Explicit symbols indicate the risk cause accompanied with the event time for collisions ($s_E$), distances to the risk spot for turns (i.e., right curve with $d_r$ and left curve with $d_l$) or stopping distance for crosswalks ($d_c$). In Fig. \ref{fig:popups}~a), the pop-up signs are pictured. 
% Besides the velocity difference, the risk type is an indication for the severity of the situation.
%Examples for collision risk are car-to-car crash., curve risk can be  as a single-car accident and regulatory risks will be a car-to-object collision. 
We want to stress that this is just a selection and more risk causes can be added. 
The purpose is also to clarify the reason for the warning and give more human-understandable information.

\subsubsection{Colored Events}
Finally, we highlight lane parts or positions according to the corresponding risks.  
% the determined color rating from the hazard route and velocity scale and relate the risks to the simulator environment. 
In the instance of curve and regulatory risk, the lane is colored from the ego position up to the point of maximal risk. 
For collision risk, we mark the point of the closest encounter as a red cube.
An illustration for regulatory risk induced from a stop line is depicted in Fig. \ref{fig:popups}~b). Again, the color is defined by the deviation $|v_0-v_{\text{tar}}|$. It also shows the therein considered navigation route with length $\Delta l_h$ and another unlikely path. 

It should be noted that the visualization of warnings only occurs if the risks are actually present. 
%\textcolor{red}{improve language, repeat intersection zone and navigation route}
%eingrauen unlikely paths and navigation path and describe in text, maybe delete Iz -> put line from unlikely path to green arrow
Altogether, the RNS provides a variety of tools to analyze and circumvent critical situations in intersection scenarios, while not overloading the driver's awareness.

\begin{figure}[t]
  \centering
  \resizebox{\linewidth}{!}{
  \import{img/}{colored_lane_new.pdf_tex}}  
  \vspace{-0.53cm}
  \caption{Short-term warning elements. Selected pop-up warnings (left) and colored lane (right).}
  \label{fig:popups} 
\end{figure} 


\begin{table*}[t]
  \footnotesize
  \centering
  \renewcommand{\arraystretch}{1.25}
  \setlength{\tabcolsep}{3.15pt}
  \begin{tabular*}{\textwidth}{c l | l l | l l | l l | l l}
    \thickhline

    \multirow{2}{*}{\textbf{Data Ratio}} & \multirow{2}{*}{\textbf{Initialization}} & 
    \multicolumn{2}{c|}{\textbf{GrabCut}} & \multicolumn{2}{c|}{\textbf{Berkeley}} & \multicolumn{2}{c|}{\textbf{SBD}} & \multicolumn{2}{c}{\textbf{DAVIS}} \\
    \cline{3-10}
    & & \textbf{NoC@85 $ \downarrow $} & \textbf{NoC@90 $ \downarrow $} & \textbf{NoC@85 $ \downarrow $} & \textbf{NoC@90 $ \downarrow $} & \textbf{NoC@85 $ \downarrow $} & \textbf{NoC@90 $ \downarrow $} & \textbf{NoC@85 $ \downarrow $} & \textbf{NoC@90 $ \downarrow $} \\

    \hline

    \multirow{2}{*}{1\%} & Default (MAE~\cite{he2022masked}) & 2.32 & 2.52 & 2.36 & 4.44 & 8.01 & 10.43 & 5.81 & 8.28 \\
    & MIS (\emph{Ours}) & \textbf{1.54} \decrease{0.78} & \textbf{1.90} \decrease{0.62} & \textbf{1.95} \decrease{0.41} & \textbf{3.73} \decrease{0.71} & \textbf{5.57} \decrease{2.44} & \textbf{8.15} \decrease{2.28} & \textbf{5.36} \decrease{0.45} & \textbf{7.40} \decrease{0.88} \\

    \rowcolor[gray]{0.9}
    & Default (MAE~\cite{he2022masked}) & 1.68 & 2.22 & 1.97 & 3.64 & 5.48 & 8.13 & 5.14 & 7.23 \\
    \rowcolor[gray]{0.9}
    \multirow{-2}{*}{5\%} & (\emph{Ours}) & \textbf{1.62} \decrease{0.06} & \textbf{1.88} \decrease{0.34} & \textbf{1.95} \decrease{0.02} & \textbf{3.32} \decrease{0.32} & \textbf{5.21} \decrease{0.27} & \textbf{7.49} \decrease{0.64} & \textbf{4.54} \decrease{0.60} & \textbf{6.11} \decrease{1.12} \\

    \multirow{2}{*}{10\%} & Default (MAE~\cite{he2022masked}) & 1.64 & 1.84 & 1.88 & 3.21 & 5.33 & 7.89 & 4.88 & 6.77 \\
    & MIS (\emph{Ours}) & \textbf{1.40} \decrease{0.24} & \textbf{1.50} \decrease{0.34} & \textbf{1.64} \decrease{0.24} & \textbf{2.73} \decrease{0.48} & \textbf{4.61} \decrease{0.72} & \textbf{6.95} \decrease{0.94} & \textbf{4.25} \decrease{0.63} & \textbf{5.94} \decrease{0.83} \\

    \thickhline
  \end{tabular*}
  \vspace{1pt}
  \caption{
    \textbf{Performance with limited annotations.} When training the model in a supervised manner with a limited number of annotations (1\%, 5\%, 10\%), the performance can be significantly improved by unsupervised pre-training using the proposed method.}
  \label{tab:few}
\end{table*}



\subsection{Quantitative Results}
\label{sec:results}


\noindent \textbf{Baselines.}
Since the fully unsupervised interactive segmentation has not been explored by previous research, we introduce some baselines which are originally designed for unsupervised salient detection (TokenCut~\cite{wang2022self}), instance segmentation (FreeMask~\cite{wang2022freesolo}), and semantic segmentation (DSM~\cite{melas2022deep}) to demonstrate the effectiveness of our method.
We follow the official implementation of these baselines to extract mask proposals and use them to train the same model as ours.
We denote these baselines by ``$ \star $" in \Cref{tab:main}.


\smallskip


\noindent \textbf{Comparison with Previous Work.}
For comparison, we present experimental results of previous supervised deep learning methods, non-deep learning methods, baselines, and the proposed MIS on four aforementioned datasets.
Under the unsupervised setting, our method significantly outperforms all other methods as shown in the second part of \Cref{tab:main}.
It is also worth mentioning that the proposed MIS achieves inspiring performance that is even comparable with some previous supervised approaches.
This support that the interactive segmentation task can be solved in a more label-efficient way.
Moreover, we show the IoU-NoC curve of previous interactive segmentation methods and MIS in \Cref{fig:curve}, where the dotted lines indicate the upper and lower performance of previous supervised methods.
It can be found that our method reaches obviously higher IoU compared to the non-deep learning methods, especially in the first few clicks, and gradually converge to the supervised counterpart when NoC increases.


\smallskip


\noindent \textbf{Comparison with Baselines.}
MIS also gets advanced results compared to the baselines which only focus on the specific granularity such as the most salient object or fixed number of objects, demonstrating the effectiveness of diverse interactions under the unsupervised setting.


\smallskip


\noindent \textbf{Supervised Fine-tuning.}
Besides the fully unsupervised setting, we further fine-tune our model in a supervised manner with only a limited number of annotations~(1\%, 5\%, 10\%).
As shown in \Cref{tab:few}, the model initialized with MIS pre-training
the MAE~\cite{he2022masked} counterpart especially on the relatively hard dataset (\emph{i.e.}, SBD and DAVIS).
These results demonstrate that MIS can serve as a strong unsupervised pre-training method for interactive segmentation when the available annotations are rare.


\begin{table*}[t]
  \footnotesize
  \centering
  \renewcommand{\arraystretch}{1.2}
  \setlength{\tabcolsep}{3.58pt}
  \begin{tabular*}{\textwidth}{c | l | l l | l l | l l | l l}
    \thickhline

    \multirow{2}{*}{\textbf{Connectivity}} & \multirow{2}{*}{\makecell[c]{\textbf{Speed} \\ \textbf{(image/s) $ \uparrow $}}} & \multicolumn{2}{c|}{\textbf{GrabCut}} & \multicolumn{2}{c|}{\textbf{Berkeley}} & \multicolumn{2}{c|}{\textbf{SBD}} & \multicolumn{2}{c}{\textbf{DAVIS}} \\
    \cline{3-10}
    & & \textbf{NoC@85 $ \downarrow $} & \textbf{NoC@90 $ \downarrow $} & \textbf{NoC@85 $ \downarrow $} & \textbf{NoC@90 $ \downarrow $} & \textbf{NoC@85 $ \downarrow $} & \textbf{NoC@90 $ \downarrow $} & \textbf{NoC@85 $ \downarrow $} & \textbf{NoC@90 $ \downarrow $} \\

    \hline

    \ding{56} & 1.10 & 2.12 & 2.52 & \textbf{3.08} & 4.69 & \textbf{6.88} & \textbf{9.46} & \textbf{5.63} & \textbf{7.56} \\

    \rowcolor[gray]{0.9} \ding{51} & \textbf{3.73} \increase{3.39 $ \times $} & \textbf{1.94} \decrease{0.18} & \textbf{2.32} \decrease{0.20} & 3.09 \increase{0.01} & \textbf{4.58} \decrease{0.11} & 6.91 \increase{0.03} & 9.51 \increase{1.36} & 6.33 \increase{0.70} & 8.44 \increase{0.88} \\

    \thickhline
  \end{tabular*}
  \vspace{1pt}
  \caption{\textbf{Ablation on connectivity constraint.} The performance of using or not using connectivity constraint is not much different, but using connectivity constraints greatly improves the speed (3.39 $ \times $ improvement) and make the pre-processing more efficient.}
  \label{tab:connectivity}
  \vspace{5pt}
\end{table*}

\algrenewcommand\algorithmicindent{1.0em}%
\begin{algorithm}[H]
  \caption{\textbf{Sampling}} \label{alg:sampling}
  \small
\begin{algorithmic}[1]
     \State Trained diffusion model $\theta$, $\bm{x}_T \sim \mathcal{N}(\bm{0}, \bm{I})$
    \For{$t=T, \dotsc, 1$}
      \State $\hat{\bm{\epsilon}}_\theta = \bm{\epsilon}_\theta(\bm{x}_t, t) + s \cdot (\bm{\epsilon}_{\theta}(\bm{x}_t, \bm{c}, t) - \bm{\epsilon}_\theta(\bm{x}_t, t))$
        \State $\bm{z}_0(t) \sim \mathcal{N}(\bm{0}, \sigma^2_a(t)  \bm{I})$ 
        \For{$i=1, \dotsc, N$}
        \State $\bm{z}_i(t) \sim \mathcal{N}(\bm{z}_0(t) , (1-\sigma^2_a(t)) \bm{I})$ 
    \EndFor
      \State $\bm{z}(t)=\{\bm{z}_1(t),\dots,\bm{z}_N(t)\}$, if $t > 1$, else $\bm{z}(t) = \bm{0}$
      \State $\bm{x_{t-1}} = \frac{1}{\sqrt{\alpha_t}}\left( \bm{x_t} - \frac{1-\alpha_t}{\sqrt{1-\bar\alpha_t}} \hat{\bm{\epsilon}}_\theta \right) + \sigma_t \bm{z}(t)$
    \EndFor
    \State \textbf{return} $\bm{x}_0$
  \end{algorithmic}
\end{algorithm}


\subsection{Qualitative Results}

\noindent \textbf{Multi-granularity Region Proposal.}
In \Cref{fig:proposal} we show the qualitative results of the region proposals generated by the baselines and our MIS, where different colors mean different mask proposals and our mask proposals.
Compared with them, our multi-granularity proposals capture more informative regions and are more diverse.
Through the top-down random sampling strategy, our proposals provide the model with rich interactions, which makes the model learn a better mapping from clicks to segmentation under the unsupervised setting.


\smallskip \noindent \textbf{Interactive Segmentation.}
\Cref{fig:visualization} show some interactive segmentation examples of our method.
The trained interactive segmentation model is able to capture the outline of the needed object by the first one or two clicks and achieves high IoU with a few clicks, which demonstrates the feasibility of our newly proposed unsupervised setting and the effectiveness of the MIS.
For some simple objects, our model can produce satisfactory predictions within the first few clicks.
It can also handle complex scenes such as partially occluded objects and adjacent instances.


\subsection{Ablation Study}


\noindent \textbf{Decay Coefficient.}
For the decay coefficient $ \alpha $, we take 0.0 as a baseline.
In this case, we always use one of the two children of the root node in the merging tree, thus multi-granularity is not allowed.
Comparing the results with $ \alpha = 0.8, 0.9, 0.95 $ we can find that the introduction of multi-granularity brings noticeable improvement, especially in the hard dataset.
It is also obvious that this hyper-parameter is relatively robust.


\smallskip


\noindent \textbf{Smoothness Constraint.}
For the smoothness constraint, we switch the weight $ \lambda $ in \Cref{eq:loss} to show its effectiveness.
The experimental results in \Cref{tab:ablation} demonstrate that the smoothness constraint is helpful for improving the performance through the principle that helps the model correct the boundary errors in the proposals.
The effectiveness is robust when $ \lambda $ is within a reason range.


\smallskip


\noindent \textbf{Connectivity.}
In bottom-up merging (Algorithm~\ref{algo:merging}), we introduce the connectivity constraint to restrict merges to only occur locally.
Here we discuss the effectiveness of it by ablation.
As shown in \Cref{tab:connectivity}, the performance of using or not using connectivity constraint is not much different, but using connectivity constraints greatly improves the speed (3.39 $ \times $ improvement) and make the pre-processing more efficient.
This is because we reduce the search space for the minimum cost from each pair of patches or regions to the adjacent patches or regions based on the prior on images.


\smallskip


\noindent \textbf{Top-down Sampling.}
The top-down sampling (Algorithm~\ref{algo:sampling}) is the core part of our MIS to achieve diversity.
We evaluate its effectiveness by comparing with a fully random strategy that randomly sample nodes on the merging tree from a uniform distribution.
The results in \Cref{tab:sampling} show clear advantages of our top-down sampling strategy.
When sampling uniformly, the finely divided object components are more likely to be selected due to the large number, resulting in local perception and thus requiring more clicks to localize objects.
In contrast, our top-down sampling strategy reasonably assigns different weights to fragmented and complete components, so as to sample complete objects as much as possible while maintaining diversity.



\section{Conclusion}

In this work, we open up a promising direction for deep interactive segmentation that completely discards the dependence on manual annotations.
While previous methods learn the interaction through object-oriented simulation with pixel-level annotations, we propose MIS and demonstrate the feasibility to simulate informative interaction with semantic-consistent regions discovered in an unsupervised manner.
Inspiring experimental results show the effectiveness of our method and the potential to reduce the labor of creating annotation without cost via training an interactive segmentation model unsupervisedly. 
Our experiment also supports that the interactive segmentation task can be solved in a more label-efficient way to trade off the performance and annotations.
We hope our exploration can promote the attendance of label efficiency on interactive segmentation.


{\small
\bibliographystyle{ieee_fullname}
\bibliography{main}
}

\end{document}
