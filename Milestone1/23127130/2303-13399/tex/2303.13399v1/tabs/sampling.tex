\begin{table*}[ht]
  \footnotesize
  \centering
  \renewcommand{\arraystretch}{1.2}
  \setlength{\tabcolsep}{7.34pt}
  \begin{tabular*}{\textwidth}{c | l l | l l | l l | l l}
    \thickhline

    \multirow{2}{*}{\textbf{Sampling}} & \multicolumn{2}{c|}{\textbf{GrabCut}} & \multicolumn{2}{c|}{\textbf{Berkeley}} & \multicolumn{2}{c|}{\textbf{SBD}} & \multicolumn{2}{c}{\textbf{DAVIS}} \\
    \cline{2-9}
    & \textbf{NoC@85 $ \downarrow $} & \textbf{NoC@90 $ \downarrow $} & \textbf{NoC@85 $ \downarrow $} & \textbf{NoC@90 $ \downarrow $} & \textbf{NoC@85 $ \downarrow $} & \textbf{NoC@90 $ \downarrow $} & \textbf{NoC@85 $ \downarrow $} & \textbf{NoC@90 $ \downarrow $} \\

    \hline

    Uniform & 3.84 & 4.32 & 4.42 & 6.20 & 7.11 & 9.95 & 7.71 & 10.10 \\

    \rowcolor[gray]{0.9} Top-down & \textbf{1.94} \decrease{1.90} & \textbf{2.32} \decrease{2.00} & \textbf{3.09} \decrease{1.33} & \textbf{4.58} \decrease{1.62} & \textbf{6.91} \decrease{0.20} & \textbf{9.51} \decrease{0.44} & \textbf{6.33} \decrease{1.38} & \textbf{8.44} \decrease{1.66} \\

    \thickhline
  \end{tabular*}
  \vspace{1pt}
  \caption{\textbf{Ablation on sampling strategy.} The proposed top-down sampling strategy is effective because it reasonably assigns different weights to fragmented and complete components, so as to sample complete objects as much as possible while maintaining diversity.}
  \label{tab:sampling}
\end{table*}
