\documentclass[aps,prl,reprint]{revtex4-2}
\pdfoutput=1
\usepackage{blindtext}
\usepackage{graphics}
\usepackage{hyperref}
\usepackage{amsmath}
\usepackage{physics}
\usepackage{graphicx}
\usepackage[capitalise]{cleveref}
\usepackage[font=small]{caption}
\graphicspath{ {./images/} }
\DeclareGraphicsExtensions{.png,.eps}
\begin{document}
\newcommand{\Br}{{\bf r}}
\newcommand{\BR}{{\bf R}}
\newcommand{\BK}{{\bf K}}
\title{Supplemental Material for“Deterministic vortices evolving from partially coherent fields"}
\author{Wenrui Miao$^{1}$, Yongtao Zhang$^{2}$, Greg Gbur$^{1,*}$}

\address{$^{1}$Department of Physics and Optical Science, UNC Charlotte, Charlotte, North Carolina 28223, USA}
\address{$^{2}$College of Physics and Information Engineering, Minnan Normal University, Zhangzhou 363000, China}


\begin{abstract}

\end{abstract}

\maketitle
In this supplemental material, we show a detailed derivation process of the cross-spectral density (CSD) along propagation, which is Eq.~(18) in the original manuscript.

We begin with a Gaussian Schell-model vortex (GSMV) beam

\begin{equation}
\begin{split}
W_{0}(\textbf{r}_1,\textbf{r}_2)=&(x_1-iy_1)(x_2+iy_2)e^{-r_1^2/2\sigma^2}\\
& e^{-r_2^2/2\sigma^2}e^{-\vert\textbf{r}_2-\textbf{r}_1\vert^2/2\delta^2 }.
\end{split}
\label{eq1}
\end{equation}
First, we write the spectral degree of coherence in terms of its Fourier transform, 
\begin{equation}
\mu_0(\BR) = \int \tilde{\mu}_0(\BK)e^{i\BK\cdot\BR}d^2K,
\label{eq2}
\end{equation}
where
\begin{equation}
\begin{split}
\tilde{\mu}_0(\textbf{K})=&\frac{1}{(2\pi)^2}\int e^{-R^2/2\delta^2}e^{-i\textbf{K}\cdot{\textbf{R}}}d^2R, \\
=&\frac{\delta^2}{\pi}e^{-K^2\delta^2/2}. \\
\end{split}
\label{eq3}
\end{equation}
The cross-spectral density may then be expressed in the form,
\begin{equation}
W_{0}(\Br_1,\Br_2) = \int \tilde{\mu}_0(\BK) U_0^\ast(\Br_1,\BK)U_0(\Br_2,\BK)d^2K, 
\label{eq4}
\end{equation}
where $U_0(\Br,\BK)$ represents a monochromatic tilted vortex beam,
\begin{equation}
U_0(\Br,\BK)=\frac{(x+iy)}{\sigma}e^{-r^2/2\sigma^2}e^{i\textbf{K}\cdot{\textbf{r}}}.
\label{eq5}
\end{equation}
The 2-D FracFT for a tiled vortex beam can be defined as an integral transform
\begin{equation}
 U_\alpha(\textbf{r},\BK)= \int_{-\infty}^{\infty} \textbf{F}_\alpha(\textbf{r},\textbf{r}')U_0(\textbf{r}',\BK) \,d^2r', 
 \label{eq6}
\end{equation}
where $\textbf{F}_\alpha(\textbf{r},\textbf{r}')$ represents the 2-D FracFT kernel defined as
\begin{equation}
K_\alpha(\textbf{r},\textbf{r} ')=\frac{ie^{-i\alpha}}{2\pi\tau^2\sin\alpha}e^\frac{-i\cot\alpha r^2}{2\tau^2}e^{\frac{i\textbf{r}\cdot\textbf{r}'}{\tau^2\sin\alpha}}e^{-\frac{i\cot\alpha r'^2}{2\tau^2}}.
\label{eq7}
\end{equation}

Proper $\tau$ value needs to be chosen so that the beam width is invariant regardless of the choice of the FracFT order $\alpha$ in the source plane. To find the beam width, we integrate all the exponential terms of a normally incident Gaussian beam over the source plane,
\begin{equation}
\begin{split}
I=&\int e^{\frac{-i\cot\alpha r^2}{2\tau^2}}e^{\frac{i\Br\Br'}{\tau^2\sin\alpha}}e^{\frac{-i\cot\alpha r'^2}{2\tau^2}}e^{\frac{-r'^2}{2\sigma^2}}d^2r' \\
=&e^{\frac{-i\cot\alpha r^2}{2\tau^2}}\int e^{-\left(\frac{i\cot\alpha}{2\tau^2}+\frac{1}{2\sigma^2}\right)r'^2}e^{\frac{i\Br\Br'}{\tau^2\sin\alpha}}d^2r'.
\end{split}
\label{eq8}
\end{equation}
We use the relation,
\begin{equation}
e^{-A(r-B)^2}=e^{-Ax^2+2ABx'-AB^2},
\label{eq9}
\end{equation}
For the above integral, 
\begin{equation}
A=\frac{i\cot\alpha}{2\tau^2}+\frac{1}{2\sigma^2}, \ B=\frac{ir}{i\cos\alpha+\frac{\tau^2}{\sigma^2}\sin\alpha},
\label{eq10}
\end{equation}
and thus,
\begin{equation}
\begin{split}
I=& e^{AB^2}\int e^{-A(\Br\prime-B)^2}d^2r'= \frac{\pi}{A}\frac{\pi}{\frac{i\cot\alpha}{2\tau^2}+\frac{1}{2\sigma^2}}e^{AB^2},
\end{split}
\label{eq11}
\end{equation}

where
\begin{equation}
e^{AB^2}=e^{-\frac{\sigma^2}{2\tau^2\sin{\alpha}(i\cos\alpha\sigma^2+\tau^2\sin{\alpha})}r^2}.    
\label{eq12}
\end{equation}
The beam width $\omega$ is found as 
\begin{equation}
\begin{split}
 \omega=& \Re  \left[-\frac{\sigma^2}{2\tau^2\sin{\alpha}(i\cos\alpha\sigma^2+\tau^2\sin{\alpha})}\right]^\frac{-1}{2}\\&=\frac{\sqrt{2}(\cos^2\alpha\sigma^4+\tau^4\sin^2\alpha)^\frac{1}{2}}{\sigma}.
\end{split}
\label{eq13}
\end{equation}
Beam width goes from $\sqrt{2}\sigma$ to $\frac{\sqrt{2}\tau^2}{\sigma}$, as the FracFT order $\alpha$ increases from $0$ to $\pi/2$. So in order to keep all fractional beams sharing the same width, $\tau$ needs to be set as $\sigma$.\\
Then, the 2-D FracFT kernel is expressed as
\begin{equation}
\textbf{F}_\alpha(\textbf{r},\textbf{r}')=\frac{ie^{-i\alpha}}{2\pi\sigma^2\sin\alpha}e^\frac{-i\cot\alpha r^2}{2\sigma^2}e^{\frac{i\textbf{r}\cdot\textbf{r}'}{\sigma^2\sin\alpha}}e^{-\frac{i\cot\alpha r'^2}{2\sigma^2}}.
\label{eq14}
\end{equation}

After applying the FracFT to the tilted beams in the source plane, Fresnel diffraction can be used to propagate them to any desired distance. The field distribution along propagation is expressed as
\begin{equation}
U_\alpha(\textbf{r},\BK,z)=\int\textbf{G}(\textbf{r},\textbf{r}^\prime) U_\alpha(\textbf{r}^{\prime},\BK)d^2r',
\label{eq15}
\end{equation}
Where $\textbf{G}(\textbf{r},\textbf{r}^\prime)$ is the Fresnel diffraction kernel, given by
\begin{equation}
\textbf{G}(\textbf{r},\textbf{r}^\prime)=\frac{e^{ikz}}{i\lambda z} e^{\frac{ik\lvert \textbf{r}-\textbf{r}^\prime\rvert^2}{2z}}.
\label{eq16}
\end{equation}
The 2-D FracFT and Fresnel diffraction integrals can be combined to write as
\begin{equation}
U_\alpha(\textbf{r},\BK,z)=\int\int \textbf{G}(\textbf{r},\textbf{r}^\prime) \textbf{F}_\alpha(\textbf{r}^\prime,\textbf{r}^{\prime\prime}) U_0(\textbf{r}^{\prime\prime},\BK)d^2r''d^2r',
\label{eq17}
\end{equation}
The integral over $r'$ can be calculated first to combine the FracFT and Fresnel kernel into the combined kernel

\begin{equation}
\begin{split}
\textbf{H}(\textbf{r},\textbf{r}^{\prime\prime})&=\int \textbf{G}(\textbf{r},\textbf{r}^\prime) \textbf{K}_\alpha(\textbf{r}^\prime,\textbf{r}^{\prime\prime}) d^2r'   \\
&=\frac{ie^{ikz}e^{-i\alpha}}{2\pi \beta^2}e^{\frac{ik}{2z}r^2}e^{\frac{-i\cot{\alpha}\textbf{r}^{\prime\prime2}}{2\sigma^2}} e^{-\frac{i\gamma(\textbf{r}^{\prime\prime}-\textbf{r}/\gamma)^2}{2\beta^2}},
\end{split}
\label{eq18}
\end{equation}
where $\beta^2\equiv\sigma^2\sin{\alpha}-\frac{z}{k}\cos{\alpha}$, 
 $\gamma\equiv\frac{z}{k\sigma^2\sin{\alpha}}$.

Then the field distribution along
propagation is obtained by the following integral
\begin{equation}
U_\alpha(\textbf{r},\BK,z)=\int \textbf{H}(\textbf{r},\textbf{r}^{\prime\prime}) U_0(\textbf{r}^{\prime\prime},\BK)d^2r''.
\label{eq19}
\end{equation}
Substituting Eq.~(\ref{eq5}) and Eq.~(\ref{eq18}) and after lengthy calculations, Eq.~(\ref{eq19}) yields
\begin{equation}
\begin{split}
U_\alpha(\textbf{r},\BK,z)=&\frac{-e^{ikz}e^{-i\alpha}}{4\beta^4A^2\sigma}e^{\frac{-(\sin{\alpha}+i\cos{\alpha})}{4\beta^2A\sigma^2}r^2}e^{\frac{-\textbf{K}\cdot\textbf{r}}{2\beta^2A}}\\
&e^{\frac{-K^2}{4A}}[(x+k_x \beta^2)+i(y+K_y \beta^2)],
\end{split}
\label{eq20}
\end{equation}
where
$A\equiv\frac{i\Tilde{\beta}^2}{2\beta^2\sigma^2}+\frac{1}{2\sigma^2}$, $\Tilde{\beta}^2\equiv\sigma^2\cos{\alpha}+\frac{z}{k}\sin{\alpha}$.

Then, the cross-spectral density along propagation can be obtained using a formula analogous to Eq.~(\ref{eq4}),
\begin{equation}
W_\alpha(\textbf{r}_1,\textbf{r}_2,z)=\int \tilde{\mu}_0(\textbf{K})U^*_\alpha(\textbf{r}_1,\BK,z)U_\alpha(\textbf{r}_2,\BK,z)\, d^2K.
\label{eq21}
\end{equation}
Substituting from Eq. (\ref{eq3}) and Eq. (\ref{eq20}) into the above
integral yields
\begin{equation}
\begin{split}
W_\alpha(\textbf{r}_1,\textbf{r}_2,z)&=\frac{\delta^2}{16\beta^8\lvert A \rvert^4 \sigma^2}e^{-\frac{(\sin{\alpha}+i\cos{\alpha})}{4\beta^2A\sigma^2}r_2^2}\\
&e^{-\frac{(\sin{\alpha}-i\cos{\alpha})}{4\beta^2A^*\sigma^2}r_1^2}e^{\frac{(\textbf{r}_1A+\textbf{r}_2A^*)^2}{16\beta^4\lvert A \rvert^4 \eta}}\\
&\Biggl[\frac{r_1^2A^2+r_2^2A^{*2}+2\lvert A\rvert^2(x_1x_2+y_1y_2)}{16\lvert A\rvert^4\eta^3}\\
-&\frac{r_1^2A+r_2^2A^{*}+(A+A^{*})(x_1-iy_1)(x_2+iy_2)}{4\lvert A\rvert^2\eta^2}\\
&+\frac{(x_1-iy_1)(x_2+iy_2)}{\eta}+\frac{\beta^4}{\eta^2}\Biggr],
\end{split}
\label{eq22}
\end{equation}
where $\eta\equiv\frac{\delta^2}{2}+\frac{1}{4A}+\frac{1}{4A^*}$.

At the special distance $z_0$, the CSD reduces to
\begin{equation}
\begin{split}
W_\alpha(\textbf{r}_1,\textbf{r}_2,z_0)&=\frac{\delta^2}{16\beta^8\lvert A \rvert^4 \sigma^2\eta}
e^{\frac{(\textbf{r}_2-\textbf{r}_1)^2}{16\beta^4A^2 \eta}}e^{-\frac{(\sin{\alpha}+i\cos{\alpha})}{4\beta^2A\sigma^2}r_2^2}\\
&e^{-\frac{(\sin{\alpha}-i\cos{\alpha})}{4\beta^2A^*\sigma^2}r_1^2}
(x_1-iy_1)(x_2+iy_2),
\end{split}
\label{eq23}
\end{equation}
which is in the form of a GSMV beam.
\end{document}
