\documentclass{article}
\usepackage{amsmath, amssymb, amsthm}
\newtheorem{theorem}{Theorem}
\newtheorem{lemma}{Lemma}
\setlength{\parindent}{0em}

\begin{document}

In this part, we prove that $k$-WL can't count all connected substructures with $(k+1)$ nodes. Specifically, we prove the following

\begin{theorem}
	For any $k\geqslant 2$, there exists a pair of graphs $G$ and $H$, such that $G$ contains a $(k+1)$-clique as its subgraph while $H$ does not, and that $k$-WL can't distinguish $G$ from $H$.
\end{theorem}

\begin{proof}
	The counter-example is inspired by the well-known Cai-F\"urer-Immerman (CFI) graphs. We define a sequence of graphs $G_k^{(\ell)}, \ell = 0, 1, \ldots, k+1$ as following,
	\begin{equation}
		\begin{split}
			V_{G_k^{(\ell)}} =& \Big\{u_{a, \vec{v}}\Big| a\in[k+1], \vec{v}\in\{0,1\}^k \text{ and }\\
			&\quad\begin{array}{ll}
				\vec{v} \text{ contains an even number of } 1 \text{'s}, & \text{if }a=1,2,\ldots, k-\ell+1, \\
				\vec{v} \text{ contains an odd number of } 1 \text{'s},  & \text{if }a=k-\ell+2,\ldots, k+1.
			\end{array}\Big\}
		\end{split}
	\end{equation}
	Two nodes $u_{a,\vec{v}}$ and $u_{a',\vec{v}'}$ of $G_k^{(\ell)}$ are connected iff there exists $m\in [k]$ such that $a' = (a+m) \mod k$ and $v_m = v'_{k-m+1}$. We have the following lemma.

	\begin{lemma}
		$(a)$ For each $\ell = 0, 1, \ldots, k+1$, $G_k^{(\ell)}$ is an undirected graph with $(k+1)2^{k-1}$ nodes;

		$(b)$ The set of graphs $G_k^{(\ell)}$ with an odd $\ell$ are mutually isomorphic; similarly, the set of graphs $G_k^{(\ell)}$ with an even $\ell$ are mutually isomorphic.
	\end{lemma}

	It's easy to verify (a). To prove (b), it suffices to prove $G_k^{(\ell)}$ is isomorphic to $G_k^{(\ell+2)}$ for all $\ell = 0,1,\ldots, k-1$. We apply a \emph{renaming} to the nodes of $G_k^{(\ell)}$: we flip the $1^\mathrm{st}$ bit of $\vec{v}$ in every node named $u_{k-\ell, \vec{v}}$, and flip the $k^\mathrm{th}$ bit of $\vec{v}$ in every node named $u_{k-\ell+1, \vec{v}}$. Since this is a mere renaming of nodes, the resulting graph is isomorphic to $G_k^{(\ell)}$. However, it's also easy to see that the resulting graph follows the construction of $G_k^{(\ell+2)}$. Therefore, we assert that $G_k^{(\ell)}$ must be isomorphic to $G_k^{(\ell+2)}$.

	Now, let's ask $G=G_k^{(0)}$ and $H=G_k^{(1)}$. Obviously there is a $(k+1)$-clique in $G$: nodes $u_{j,0^k}, j=1,2, \ldots, k+1$ are mutually adjacent by definition of $G_k^{(0)}$. On the contrary, we have
	\begin{lemma}
		There's no $(k+1)$-clique in $H$.
	\end{lemma}

	The proof is given below. Assume there is a $(k+1)$-clique in $H$. Since there's no edge between nodes $u_{a, \vec{v}}$ with an identical $a$, the $(k+1)$-clique must contain exactly one node from every node set $\{u_{a,\vec{v}}\}$ for each fixed $a\in[k+1]$. We further assume that the $(k+1)$ nodes are $u_{a, b_{a1}b_{a2}\ldots b_{ak}}, a=1,2,\ldots, k+1$. Using the condition for adjacency, we have
	\begin{align}
		       & b_{2k} = b_{11},                                                               \\
		       & b_{3k} = b_{21}, b_{3(k-1)} = b_{12},                                          \\
		       & b_{4k} = b_{31}, b_{4(k-1)} = b_{22}, b_{4(k-2)} = b_{13},                     \\
		\notag & \cdots\cdots\cdots\cdots                                                       \\
		       & b_{(k+1)k} = b_{k1}, b_{(k+1)(k-1)} = b_{(k-1)2}, \ldots, b_{(k+1)1} = b_{1k}.
	\end{align}
	Applying the above identities to the summation
	\begin{align}
		\label{contradiction}
		\sum_{a=1}^{k+1}\sum_{j=1}^k b_{aj} = 2\sum_{j=1}^k\left(b_{1j}+b_{2j}+\cdots+b_{(k-j+1)j}\right),
	\end{align}
	we see that it should be even. However, by definition of $G_k^{(1)}$, there are an even number of $1$'s in $b_{a1}b_{a2}\ldots b_{ak}$ when $a\in [k]$, and an odd number of $1$'s when $a=k+1$. Therefore, the sum in \eqref{contradiction} should be odd. This leads to a contradiction.

	Finally, to prove the $k$-WL equivalence of $G$ and $H$, we have
	\begin{lemma}
		$k$-WL can't distinguish $G$ and $H$.
	\end{lemma}

	By virtue of the classical result of Cai, F\"urer and Immerman, it suffices to prove that Player II will win the $\mathcal{C}_{k}$ bijective pebble game on $G$ and $H$. We state the winning strategy for Player II as following. Since $G$ and $H$ are isomorphic with nodes $\{u_{k+1,*}\}$ deleted, Player II can always choose an isomorphism $f:G-\{u_{k+1,*}\}\rightarrow H-\{u_{k+1,*}\}$ to survive if Player I never places a pebble on nodes $u_{k+1,*}$. Furthermore, since $k$ pebbles can occupy nodes with at most $k$ different values of $a$ (in $u_{a,\vec{v}}$), there's always a set of pebbleless nodes $\{u_{a_0,\vec{v}}\}$ with some $a_0\in[k+1]$. Therefore, Player II only needs to do proper renaming on $H$ between $u_{k+1,*}$ and $u_{a_0,*}$ as stated above, and chooses his isomorphism on $G-\{u_{a_0,*}\}$ and $H^{\mathrm{renamed}}-\{u_{a_0,*}\}$. This way, Player II never loses since there are not enough pebbles for Player I to make use of the oddity at the currently pebbleless set of nodes.

\end{proof}

\end{document}
