
%% bare_conf.tex
%% V1.4b
%% 2015/08/26
%% by Michael Shell
%% See:
%% http://www.michaelshell.org/
%% for current contact information.
%%
%% This is a skeleton file demonstrating the use of IEEEtran.cls
%% (requires IEEEtran.cls version 1.8b or later) with an IEEE
%% conference paper.
%%
%% Support sites:
%% http://www.michaelshell.org/tex/ieeetran/
%% http://www.ctan.org/pkg/ieeetran
%% and
%% http://www.ieee.org/

%%*************************************************************************
%% Legal Notice:
%% This code is offered as-is without any warranty either expressed or
%% implied; without even the implied warranty of MERCHANTABILITY or
%% FITNESS FOR A PARTICULAR PURPOSE! 
%% User assumes all risk.
%% In no event shall the IEEE or any contributor to this code be liable for
%% any damages or losses, including, but not limited to, incidental,
%% consequential, or any other damages, resulting from the use or misuse
%% of any information contained here.
%%
%% All comments are the opinions of their respective authors and are not
%% necessarily endorsed by the IEEE.
%%
%% This work is distributed under the LaTeX Project Public License (LPPL)
%% ( http://www.latex-project.org/ ) version 1.3, and may be freely used,
%% distributed and modified. A copy of the LPPL, version 1.3, is included
%% in the base LaTeX documentation of all distributions of LaTeX released
%% 2003/12/01 or later.
%% Retain all contribution notices and credits.
%% ** Modified files should be clearly indicated as such, including  **
%% ** renaming them and changing author support contact information. **
%%*************************************************************************


% *** Authors should verify (and, if needed, correct) their LaTeX system  ***
% *** with the testflow diagnostic prior to trusting their LaTeX platform ***
% *** with production work. The IEEE's font choices and paper sizes can   ***
% *** trigger bugs that do not appear when using other class files.       ***                          ***
% The testflow support page is at:
% http://www.michaelshell.org/tex/testflow/



\documentclass[conference,a4paper]{IEEEtran}
% Some Computer Society conferences also require the compsoc mode option,
% but others use the standard conference format.
%
% If IEEEtran.cls has not been installed into the LaTeX system files,
% manually specify the path to it like:
% \documentclass[conference]{../sty/IEEEtran}





% Some very useful LaTeX packages include:
% (uncomment the ones you want to load)


% *** MISC UTILITY PACKAGES ***
%
%\usepackage{ifpdf}
% Heiko Oberdiek's ifpdf.sty is very useful if you need conditional
% compilation based on whether the output is pdf or dvi.
% usage:
% \ifpdf
%   % pdf code
% \else
%   % dvi code
% \fi
% The latest version of ifpdf.sty can be obtained from:
% http://www.ctan.org/pkg/ifpdf
% Also, note that IEEEtran.cls V1.7 and later provides a builtin
% \ifCLASSINFOpdf conditional that works the same way.
% When switching from latex to pdflatex and vice-versa, the compiler may
% have to be run twice to clear warning/error messages.
\usepackage{xcolor}
\usepackage{balance}






% *** CITATION PACKAGES ***
%
%\usepackage{cite}
% cite.sty was written by Donald Arseneau
% V1.6 and later of IEEEtran pre-defines the format of the cite.sty package
% \cite{} output to follow that of the IEEE. Loading the cite package will
% result in citation numbers being automatically sorted and properly
% "compressed/ranged". e.g., [1], [9], [2], [7], [5], [6] without using
% cite.sty will become [1], [2], [5]--[7], [9] using cite.sty. cite.sty's
% \cite will automatically add leading space, if needed. Use cite.sty's
% noadjust option (cite.sty V3.8 and later) if you want to turn this off
% such as if a citation ever needs to be enclosed in parenthesis.
% cite.sty is already installed on most LaTeX systems. Be sure and use
% version 5.0 (2009-03-20) and later if using hyperref.sty.
% The latest version can be obtained at:
% http://www.ctan.org/pkg/cite
% The documentation is contained in the cite.sty file itself.






% *** GRAPHICS RELATED PACKAGES ***
%
\ifCLASSINFOpdf
  \usepackage[pdftex]{graphicx}
  % declare the path(s) where your graphic files are
  % \graphicspath{{../pdf/}{../jpeg/}}
  % and their extensions so you won't have to specify these with
  % every instance of \includegraphics
  % \DeclareGraphicsExtensions{.pdf,.jpeg,.png}
\else
  % or other class option (dvipsone, dvipdf, if not using dvips). graphicx
  % will default to the driver specified in the system graphics.cfg if no
  % driver is specified.
  \usepackage[dvips]{graphicx}
  % declare the path(s) where your graphic files are
  % \graphicspath{{../eps/}}
  % and their extensions so you won't have to specify these with
  % every instance of \includegraphics
  % \DeclareGraphicsExtensions{.eps}
\fi
% graphicx was written by David Carlisle and Sebastian Rahtz. It is
% required if you want graphics, photos, etc. graphicx.sty is already
% installed on most LaTeX systems. The latest version and documentation
% can be obtained at: 
% http://www.ctan.org/pkg/graphicx
% Another good source of documentation is "Using Imported Graphics in
% LaTeX2e" by Keith Reckdahl which can be found at:
% http://www.ctan.org/pkg/epslatex
%
% latex, and pdflatex in dvi mode, support graphics in encapsulated
% postscript (.eps) format. pdflatex in pdf mode supports graphics
% in .pdf, .jpeg, .png and .mps (metapost) formats. Users should ensure
% that all non-photo figures use a vector format (.eps, .pdf, .mps) and
% not a bitmapped formats (.jpeg, .png). The IEEE frowns on bitmapped formats
% which can result in "jaggedy"/blurry rendering of lines and letters as
% well as large increases in file sizes.
%
% You can find documentation about the pdfTeX application at:
% http://www.tug.org/applications/pdftex





% *** MATH PACKAGES ***
%
\usepackage{amsmath,amssymb}
% A popular package from the American Mathematical Society that provides
% many useful and powerful commands for dealing with mathematics.
%
% Note that the amsmath package sets \interdisplaylinepenalty to 10000
% thus preventing page breaks from occurring within multiline equations. Use:
%\interdisplaylinepenalty=2500
% after loading amsmath to restore such page breaks as IEEEtran.cls normally
% does. amsmath.sty is already installed on most LaTeX systems. The latest
% version and documentation can be obtained at:
% http://www.ctan.org/pkg/amsmath


% *** SPECIALIZED LIST PACKAGES ***
%
%\usepackage{algorithmic}
% algorithmic.sty was written by Peter Williams and Rogerio Brito.
% This package provides an algorithmic environment fo describing algorithms.
% You can use the algorithmic environment in-text or within a figure
% environment to provide for a floating algorithm. Do NOT use the algorithm
% floating environment provided by algorithm.sty (by the same authors) or
% algorithm2e.sty (by Christophe Fiorio) as the IEEE does not use dedicated
% algorithm float types and packages that provide these will not provide
% correct IEEE style captions. The latest version and documentation of
% algorithmic.sty can be obtained at:
% http://www.ctan.org/pkg/algorithms
% Also of interest may be the (relatively newer and more customizable)
% algorithmicx.sty package by Szasz Janos:
% http://www.ctan.org/pkg/algorithmicx


% *** ALIGNMENT PACKAGES ***
%
%\usepackage{array}
% Frank Mittelbach's and David Carlisle's array.sty patches and improves
% the standard LaTeX2e array and tabular environments to provide better
% appearance and additional user controls. As the default LaTeX2e table
% generation code is lacking to the point of almost being broken with
% respect to the quality of the end results, all users are strongly
% advised to use an enhanced (at the very least that provided by array.sty)
% set of table tools. array.sty is already installed on most systems. The
% latest version and documentation can be obtained at:
% http://www.ctan.org/pkg/array


% IEEEtran contains the IEEEeqnarray family of commands that can be used to
% generate multiline equations as well as matrices, tables, etc., of high
% quality.



% *** SUBFIGURE PACKAGES ***
\ifCLASSOPTIONcompsoc
 \usepackage[caption=false,font=normalsize,labelfont=sf,textfont=sf]{subfig}
\else
 \usepackage[caption=false,font=footnotesize]{subfig}
\fi
% subfig.sty, written by Steven Douglas Cochran, is the modern replacement
% for subfigure.sty, the latter of which is no longer maintained and is
% incompatible with some LaTeX packages including fixltx2e. However,
% subfig.sty requires and automatically loads Axel Sommerfeldt's caption.sty
% which will override IEEEtran.cls' handling of captions and this will result
% in non-IEEE style figure/table captions. To prevent this problem, be sure
% and invoke subfig.sty's "caption=false" package option (available since
% subfig.sty version 1.3, 2005/06/28) as this is will preserve IEEEtran.cls
% handling of captions.
% Note that the Computer Society format requires a larger sans serif font
% than the serif footnote size font used in traditional IEEE formatting
% and thus the need to invoke different subfig.sty package options depending
% on whether compsoc mode has been enabled.
%
% The latest version and documentation of subfig.sty can be obtained at:
% http://www.ctan.org/pkg/subfig




% *** FLOAT PACKAGES ***
%
%\usepackage{fixltx2e}
% fixltx2e, the successor to the earlier fix2col.sty, was written by
% Frank Mittelbach and David Carlisle. This package corrects a few problems
% in the LaTeX2e kernel, the most notable of which is that in current
% LaTeX2e releases, the ordering of single and double column floats is not
% guaranteed to be preserved. Thus, an unpatched LaTeX2e can allow a
% single column figure to be placed prior to an earlier double column
% figure.
% Be aware that LaTeX2e kernels dated 2015 and later have fixltx2e.sty's
% corrections already built into the system in which case a warning will
% be issued if an attempt is made to load fixltx2e.sty as it is no longer
% needed.
% The latest version and documentation can be found at:
% http://www.ctan.org/pkg/fixltx2e


%\usepackage{stfloats}
% stfloats.sty was written by Sigitas Tolusis. This package gives LaTeX2e
% the ability to do double column floats at the bottom of the page as well
% as the top. (e.g., "\begin{figure*}[!b]" is not normally possible in
% LaTeX2e). It also provides a command:
%\fnbelowfloat
% to enable the placement of footnotes below bottom floats (the standard
% LaTeX2e kernel puts them above bottom floats). This is an invasive package
% which rewrites many portions of the LaTeX2e float routines. It may not work
% with other packages that modify the LaTeX2e float routines. The latest
% version and documentation can be obtained at:
% http://www.ctan.org/pkg/stfloats
% Do not use the stfloats baselinefloat ability as the IEEE does not allow
% \baselineskip to stretch. Authors submitting work to the IEEE should note
% that the IEEE rarely uses double column equations and that authors should try
% to avoid such use. Do not be tempted to use the cuted.sty or midfloat.sty
% packages (also by Sigitas Tolusis) as the IEEE does not format its papers in
% such ways.
% Do not attempt to use stfloats with fixltx2e as they are incompatible.
% Instead, use Morten Hogholm'a dblfloatfix which combines the features
% of both fixltx2e and stfloats:
%
% \usepackage{dblfloatfix}
% The latest version can be found at:
% http://www.ctan.org/pkg/dblfloatfix




% *** PDF, URL AND HYPERLINK PACKAGES ***
%
%\usepackage{url}
% url.sty was written by Donald Arseneau. It provides better support for
% handling and breaking URLs. url.sty is already installed on most LaTeX
% systems. The latest version and documentation can be obtained at:
% http://www.ctan.org/pkg/url
% Basically, \url{my_url_here}.



% *** Do not adjust lengths that control margins, column widths, etc. ***
% *** Do not use packages that alter fonts (such as pslatex).         ***
% There should be no need to do such things with IEEEtran.cls V1.6 and later.
% (Unless specifically asked to do so by the journal or conference you plan
% to submit to, of course. )

% my add
\usepackage{cite}

\usepackage[colorlinks]{hyperref}
\usepackage{xcolor}
\hypersetup{
colorlinks,
    linkcolor={red!50!black},
    citecolor={blue!50!black},
    urlcolor={blue!80!black}
}

\usepackage{mathtools}
\DeclarePairedDelimiter\ceil{\lceil}{\rceil}
\DeclarePairedDelimiter\floor{\lfloor}{\rfloor}

\def\Xint#1{\mathchoice
   {\XXint\displaystyle\textstyle{#1}}%
   {\XXint\textstyle\scriptstyle{#1}}%
   {\XXint\scriptstyle\scriptscriptstyle{#1}}%
   {\XXint\scriptscriptstyle\scriptscriptstyle{#1}}%
   \!\int}
\def\XXint#1#2#3{{\setbox0=\hbox{$#1{#2#3}{\int}$}
     \vcenter{\hbox{$#2#3$}}\kern-.5\wd0}}
\def\ddashint{\Xint=}
\def\dashint{\Xint-}



\usepackage{bm}% bold math
\newcommand{\R}{\mathbb{R}}
\newcommand{\C}{\mathbb{C}}
\newcommand{\Z}{\mathbb{Z}}

\newcommand{\ie}{\textit{i.e.}\/, }
\newcommand{\eg}{\textit{e.g.}\/, }
\newcommand{\cf}{\textit{cf.}\/, }

\providecommand*{\mrm}[1]{\mathrm{#1}}
\providecommand*{\unit}[1]{\ensuremath{\mrm{\,#1}}}
\providecommand*{\eu}{\ensuremath{\mrm{e}}}
\providecommand*{\iu}{\ensuremath{\mrm{i}}}
\providecommand*{\ju}{\ensuremath{\mrm{j}}}
\providecommand*{\diff}{\operatorname{d}\!}
%\renewcommand{\Re}{\operatorname{Re}}	% The LaTeX standard is not ISO!
%\renewcommand{\Im}{\operatorname{Im}}	% The LaTeX standard is not ISO!
\renewcommand{\Re}{\ensuremath{\mrm{Re}}}	% The LaTeX standard is not ISO!
\renewcommand{\Im}{\ensuremath{\mrm{Im}}}	% The LaTeX standard is not ISO!
\providecommand*{\ohm}{\ensuremath{\mrm{\Omega}}}
\providecommand*{\micro}{\ensuremath{\mrm{\mu}}}
\providecommand*{\degree}{\ensuremath{^\circ}}

\newcommand{\minimize}{\mrm{minimize}}
\newcommand{\maximize}{\mrm{maximize}}
%\newcommand{\mini}{\mrm{min.}}
%\newcommand{\maxi}{\mrm{max.}}
\newcommand{\argmax}{\mathop{\mrm{argmax}}}
\newcommand{\subto}{\mrm{subject\ to}}
%\newcommand{\sto}{\mrm{s.t.}}

\newtheorem{theorem}{Theorem}[section]

% correct bad hyphenation here
\hyphenation{op-tical net-works semi-conduc-tor}

\pagestyle{plain}

\begin{document}
%
% paper title
% Titles are generally capitalized except for words such as a, an, and, as,
% at, but, by, for, in, nor, of, on, or, the, to and up, which are usually
% not capitalized unless they are the first or last word of the title.
% Linebreaks \\ can be used within to get better formatting as desired.
% Do not put math or special symbols in the title.
\title{A simple approach to line mixing combining narrowband resolution with far wing accuracy}
\title{A modified projection approach to line mixing combining narrowband resolution with far wing accuracy}
\title{A modified projection approach to line mixing}

% author names and affiliations
% use a multiple column layout for up to three different
% affiliations
\author{\IEEEauthorblockN{
Sven~Nordebo\IEEEauthorrefmark{1},   % 1st author, 1st affiliations
%An~Author2\IEEEauthorrefmark{2},   % 2nd author, 2nd affiliations
%An~Author3\IEEEauthorrefmark{3}    % 3rd author, 3rd affiliations
%An~Author4\IEEEauthorrefmark{4}     % 4th author, 4th affiliations
}                                     % ...
%\\
\IEEEauthorblockA{\IEEEauthorrefmark{1}% 1st affiliations
Department of Physics and Electrical Engineering, Linn\ae us University,   351 95 V\"{a}xj\"{o}, Sweden. E-mail: sven.nordebo@lnu.se} 
% \\ E-mail: \{sven.nordebo,yevhen.ivanenko\}@lnu.se
%\IEEEauthorblockA{\IEEEauthorrefmark{2}
%Department of,.... E-mail: an.author2.@inst.se} 
%\IEEEauthorblockA{\IEEEauthorrefmark{3}
%Department of,.... E-mail: an.author3.@inst.se} 
%\IEEEauthorblockA{\IEEEauthorrefmark{4}
%Department of,.... E-mail: an.author4.@inst.se} 
}

% conference papers do not typically use \thanks and this command
% is locked out in conference mode. If really needed, such as for
% the acknowledgment of grants, issue a \IEEEoverridecommandlockouts
% after \documentclass

% use for special paper notices
%\IEEEspecialpapernotice{(Invited Paper)}

% make the title area
\maketitle

% As a general rule, do not put math, special symbols or citations
% in the abstract
\begin{abstract}
This paper presents a simple approach to combine the high-resolution narrowband features of some
desired isolated line models with the projection based strong collision (SC) method to line mixing which was 
introduced by Bulanin, Dokuchaev, Tonkov and Filippov. The method can be viewed in terms of a small diagonal perturbation
of the SC relaxation matrix providing the required narrowband accuracy and resolution close to the line centers, at the same time
as the SC line coupling transfer rates can be fine tuned to accurately match some given far wing absorption data.
The method can conveniently be placed in the framework of the Boltzmann-Liouville transport equation where a rigorous diagonalization of the line mixing problem
requires that molecular phase and velocity changes are assumed to be uncorrelated.
Exact solutions and numerical examples are provided for the case with pure pressure broadening and velocity independent parameters.
A detailed analysis for the general Doppler case is given based on the first order Rosenkranz approximation,
including the possibility to incorporate quadratically speed dependent parameters such as with the Hartmann-Tran (HT) profile
in the case with uncorrelated collisions.
%partially Correlated quadratic-Speed-Dependent Hard-Collision profile (pCqSD-HCP), also known as the Hartmann-Tran Profile (HTP),
\end{abstract}

\vskip0.5\baselineskip
\begin{IEEEkeywords}
Line mixing, wide-band molecular absorption spectra, spectral line shapes, Doppler broadening, collisional broadening, radiative transfer in the atmosphere.
\end{IEEEkeywords}

\section{Introduction}

%\cite{Tennyson+etal2014b} % Recommended isolated-line profile for representing high-resolution spectroscopic transitions (IUPAC Technical Report)
%\cite{Ngo+etal2013} % An isolated line-shape model to go beyond the Voigt profile in spectroscopic databases and radiative transfer codes
%\cite{Ngo+etal2014} % Erratum to ``An isolated line-shape model to go beyond the Voigt profile in spectroscopic databases and radiative transfer codes'' [J. Quant. Spectrosc. Radiat. Transf. 129 (2013) 89-100]
%\cite{Tran+etal2013} % Efficient computation of some speed-dependent isolated line profiles
%\cite{Rautian+Sobelman1967} % The effect of collisions on the {D}oppler broadening of spectral lines
%\cite{Joubert+etal1999} % A partially correlated strong collision model for velocity and state changing collisions: Application to Ar-broadened HF rovibrational line shape
%\cite{Ngo+etal2012} % A pure $\mrm{H}_2\mrm{O}$ isolated line-shape model based on classical molecular dynamics simulations of velocity changes and semi-classical calculations of speed-dependent collisional parameters
%\cite{Tran+Hartmann2009} % An isolated line-shape model based on the Keilson and Storer function for velocity changes. I. Theoretical approaches
%\cite{Hartmann+etal2008} % Collisional effects on molecular spectra. Laboratory experiments and models, consequences for applications
% Hartmann+etal2018 Recent advances in collisional effects on spectra of molecular gases and their practical consequences

% Robert+Galatry1971 Infrared Absorption of Diatomic Polar Molecules in Liquid Solutions

%\cite{Filippov+etal2002} % Line mixing effect on the pure $\mrm{CO}_2$ absorption in the 15 $\mu\mrm{m}$ region
%% Doppler effect eq. (9). Interesting experimental data CO2 at 667 cm-1 (nu2)
%\cite{Filippov+Tonkov1993} % Semiclassical analysis of line mixing in the infrared bands of $\mrm{CO}$ and $\mrm{CO}_2$
%% see p. 113, ref to 13 Dokuchaev et al and 14 Bulanin et al
%\cite{Tonkov+Filippov2003} % Collision Induced Far Wings of $\mrm{CO}_2$ and $\mrm{H}_2\mrm{O}$ Bands in Ir Spectra}
%% see Eq. (16). Interesting experimental data for CO2 at 2400 cm-1 band (nu3)
%\cite{Tonkov+etal1996} % A simple model of the line mixing effect for atmospheric applications: Theoretical background and comparison with experimental profiles
%% See explicit expression eq. (10)-(12).
%\cite{Bulanin+etal1984} % Influence of line interference on the vibration-rotation band shapes
%\cite{Dokuchaev+etal1982} % Line interference in $\nu_3$ rotational-vibrational band of $\mrm{N}_2\mrm{O}$ in the strong interaction approximation
%\cite{Rosenkranz1975} % Shape of the 5 mm oxygen band in the atmosphere

% Strow+etal1994 A compilation of first-order line-mixing coefficients for $\mrm{CO}_2$ Q-branches

% Kolb+Griem1958 Theory of Line Broadening in Multiplet Spectra
% Baranger1958 General impact theory of pressure broadening
% Gordon1967 On the pressure broadening of molecular multiplet spectra
% Gordon1966 Semiclassical theory of spectra and relaxation in molecular gases
% Gordon1966a Theory of the Width and Shift of Molecular Spectral Lines in Gases
% Ciurylo+Pine2000 Speed-dependent line mixing profiles
% Smith+etal1971a An impact theory for Doppler and pressure broadening. I. General theory
% Smith+etal1971b An impact theory for Doppler and pressure broadening. II. Atomic and molecular systems
% Pine+Gabard2000 Speed-dependent broadening and line mixing in $\mrm{CH}_4$ perturbed by $\mrm{Ar}$ and $\mrm{N}_2$ from multispectrum fits



%This paper presents a simple projection based approach to line mixing within the framework of the Boltzmann-Liouville transport equation
%and the Rosenkranz approximation. 
%The method takes as a starting point any desired isolated line model with required (high) narrowband accuracy.
%
%including velocity dependent broadening and shifts such as the uncorrelated HTP, provided only that the
%corresponding line shapes can be computed explicitly in terms of the Faddeeva function.
%The corresponding diagonal elements of the relaxation matrix are then complemented with the line coupling transfer rates 
%which can be derived directly from a projection in line space. 
%The approach can thus be viewed as a perturbation of the relaxation matrix within the so called 
%strong collision (SC) method introduced by Bulanin, Dokuchaev, Tonkov and Filippov.
%It is illustrated how the line coupling transfer rates can be properly chosen and fine tuned to accurately match some given far wing measured data 
%without significantly affecting the narrowband properties of the isolated line model.

%The perturbative approach is thus able to combine the narrowband features of the isolated line model 
%with the broadband features of the line mixing (SC) approach at the same time as it is eliminating the corresponding disadvantages 
%of these two methods in their respective band.


The presence of line mixing has been recognized as the main reason for the deviation of actual far wing line shapes 
from those that are calculated as a simple sum of Lorentzian lines, see \eg \cite[Fig.~1]{Filippov+etal2002} and \cite[Fig.~2]{Tonkov+Filippov2003}. 
The effects of incomplete (soft) collisions does only play a minor role 
which motivates the use of the so called hard collision models within the impact approximation, see \eg \cite{Tonkov+Filippov2003,Hartmann+etal2008}. 
In this paper, we will investigate the possibility of complementing the high-resolution narrowband features of some
desired isolated line models with the line coupling transfer rates obtained from the projection based strong collision (SC) method introduced in
\cite{Dokuchaev+etal1982,Bulanin+etal1984,Filippov+Tonkov1993,Tonkov+etal1996,Filippov+etal2002,Tonkov+Filippov2003}. 
The aim is to derive a simple and flexible approach to line mixing with high accuracy close to the line centers as well as in the far wing.
It is furthermore required that the spectral function should be implemented as a simple sum of individual lines based on a 
rigorous diagonalization of the line mixing problem.

The proposed approach can be motivated by the computationally exhaustive line-by-line calculations of broadband radiative transfer in the atmosphere, see \eg \cite{Liou2002,Berk+Hawes2017,Nordebo2021a}.
In particular, as may be quoted from Liou \cite[p.~126]{Liou2002}: ``The computer time required for line-by-line calculations, even with the availability of a supercomputer, is formidable. 
This is especially true for flux calculations in which an integration over all absorption bands is necessary.''
It is for this reason that simple isolated line shapes are usually employed for this purpose, but it is quite unclear how much the lack of far wing accuracy and consequently 
the overestimated atmospheric absorption may affect the results.
In this paper, we will focus on the derivation of a simple projection based approach to line mixing and which can use as sole input parameters
the molecular transition frequencies, the line strengths and the line widths that are readily available for 
a large variety of species in spectroscopic databases such as \eg HITRAN
\cite{HITRANorg,HITRANdefs,Rothman+etal1998,Simeckova+etal2006,Gordon+etal2017b}.
%However, in some wide-band applications such as the calculation of broadband radiative transfer in the atmosphere,
%it may also be of interest to achieve a very high accuracy when modeling the absorption in the far wing of the spectral lines. 
%An example would be a precise calculation of the ``base-line'' level of atmospheric absorption for given vertical profiles and hence to
%correctly predict the radiative forcing associated with the green-house effect.  

We will take as a starting point the kinetic equation method by 
Rautian and Sobelman \cite{Rautian+Sobelman1967,Hartmann+etal2008,Tran+Hartmann2009,Ngo+etal2012,Ngo+etal2013,Tennyson+etal2014b}
which can readily be adapted to include the line mixing effects as described in \eg \cite{Smith+etal1971a,Smith+etal1971b,Ciurylo+Pine2000}.
There is a vast literature in this field, see \eg \cite{Hartmann+etal2008} with references, and
among the pioneering work is in particular worth mentioning \cite{Kolb+Griem1958,Baranger1958,Gordon1966,Gordon1967,Rosenkranz1975}.
Later developments include the so called ``beyond Voigt'' profiles such as the Hartmann-Tran (HT) profile \cite{Ngo+etal2013} encompassing
partially correlated collisions and speed dependent pressure broadening and shifts, and
which is now becoming standardized in spectroscopic databases such as HITRAN \cite{Tennyson+etal2014b,Gordon+etal2022}.
However, following \cite{Ciurylo+Pine2000}, it turns out that a rigorous diagonalization of the line mixing problem based on the kinetic equation method
can only be achieved with uncorrelated collisions, line independent Dicke narrowing and speed independent pressure broadening and shifts,
see in particular \cite[Appendix~A]{Ciurylo+Pine2000}.
The more general line mixing methods may be computationally huge, \cf \eg \cite[p.~379]{Ciurylo+Pine2000} and \cite[p.~73]{Pine+Gabard2000}.
Nevertheless, it has been reported that the isolated HT line shapes can take line mixing effects into account by simple empirical modifications, 
\cf \eg \cite[Eq.~(9)]{Ngo+etal2013} and \cite[Eq.~(15)]{Pine+Gabard2000}.

A simple projection based strong collision (SC) model for the relaxation matrix has previously been proposed in 
\cite{Dokuchaev+etal1982,Bulanin+etal1984,Filippov+Tonkov1993,Tonkov+etal1996,Filippov+etal2002,Tonkov+Filippov2003}.
It has been reported that this model suffers from the major disadvantages 
that it does not take into account the distinguishing features of various perturbing gases nor of the different ro-vibrational branches 
of the radiator, see \eg \cite[p.~131]{Tonkov+Filippov2003}. In particular, the model actually produces the same line width for all lines.
However, as we will show in this paper,  it will only require a small adjustment of this model
based on a diagonal perturbation to obtain a line-mixing model that is able to take line specific data
of various host gases and radiators into account and thus providing a high accuracy close to the line centers as well as in the far wing. 
The latter is achieved by fine tuning the one remaining model parameter (the collision frequency for line mixing)  %the strength of the line coupling transfer rates, or 
with respect to some given far wing measurement data. To this end, the perturbed matrix will retain the condition of detailed balancing,
but its property as a rigorous projector will be slightly relaxed in favor of the above mentioned new features.

Following \cite[Eq.~(10)-(12)]{Tonkov+etal1996} and \cite[Eq.~(14)-(15)]{Tonkov+Filippov2003}, it is now possible to express explicitly an exact solution to the line mixing problem 
in the case of uncorrelated collisions, pure pressure broadening and velocity independent broadening and shifts parameters.
To include Doppler broadening one can readily apply the first order Rosenkranz approximation. This will even
allow the diagonal elements of the relaxation matrix to depend on speed, provided of course that the corresponding line shapes can be
integrated in terms of the Faddeeva function, \cf the HT profile with uncorrelated collisions \cite{Ngo+etal2013}, etc.

%The rest of the paper is organized as follows...

\section{A standard model for line mixing}\label{sect:Linemix_General}
 
As a standard hard collision line mixing model we consider here the following Boltzmann-Liouville transport equation formulated in velocity space as
\begin{multline}\label{eq:Linemixing_dFdtpartcorr}
\frac{\partial}{\partial t}F_n(\bm{v},t)=-\iu\left(\omega_{0n} + \bm{k}\cdot\bm{v} \right) F_n(\bm{v},t)-\beta_n F_n(\bm{v},t) \\
+\beta_n f(\bm{v})\int_{\bm{v}^\prime}F_n(\bm{v}^\prime,t)\mrm{d}\bm{v}^\prime 
-\sum_{n^\prime=1}^N W_{nn^\prime}F_{n^\prime}(\bm{v},t) \\
-f(\bm{v})\sum_{n^\prime=1}^N C_{nn^\prime} \int_{\bm{v}^\prime} F_{n^\prime}(\bm{v}^\prime,t)\mrm{d}\bm{v}^\prime,
\end{multline}
\cf \cite{Smith+etal1971a,Smith+etal1971b,Ciurylo+Pine2000}, and where the notation has been largely adapted to \cite{Ciurylo+Pine2000}.
Here, $\bm{v}$ is the velocity, $\omega_{0n}$ is the transition frequency of a particular line $n$ and the factor $\bm{k}\cdot\bm{v}$ models the Doppler dephasing 
where $\bm{k}=k\hat{\bm{k}}$ is the wave vector, $k=\omega/\mrm{c}_0$ the wavenumber of the incident radiation and $\mrm{c}_0$ the speed of light in vacuum.
Further, $W_{nn^\prime}$ are the relaxation coefficients for the phase-changing collisions which are uncorrelated with velocity changes and 
$C_{nn^\prime}$ are the relaxation coefficients for the correlated phase-changing collisions which are simultaneously changing the velocity of the molecule.
The parameters $\beta_n$ are the line specific velocity-changing collision rates. 
The Maxwell-Boltzmann distribution is given by $f(\bm{v})=(\sqrt{\pi}\widetilde{v})^{-3}\eu^{-(v/\widetilde{v})^2}$
where $v=|\bm{v}|$, $\widetilde{v}=\sqrt{2k_\mrm{B}T/\mu_\mrm{r}}$ is the most probable speed, $k_\mrm{B}$ the Boltzmanns constant, 
$T$ the temperature and $\mu_\mrm{r}$ the mass of the radiator.
It is furthermore assumed here that the parameters $C_{nn^\prime}$ and $\beta_n$ are independent of velocity $\bm{v}$.
The off-diagonal elements $W_{nn^\prime}$ are the negative of the line coupling transfer rates and
the diagonal elements $W_{nn}=\gamma_{0n}(v)+\iu\delta_{0n}(v)$ consists of the broadening and frequency shift parameters
that generally may depend on speed. The parameters $W_{nn^\prime}$, $C_{nn^\prime}$ and $\beta_n$ are typically taken to be linear with pressure $p$.

The total dipole autocorrelation function is represented here in velocity space as $C(t)=\int_{\bm{v}}C(\bm{v},t)\mrm{d}\bm{v}$ where 
\begin{equation}\label{eq:Linemix_Cvtsumdef}
C(\bm{v},t)=\sum_n \mu_n F_n(\bm{v},t),
\end{equation}
and where $\mu_n$ are the transition dipole moments associated with a particular line. 
Based on the dipole approximation ($\eu^{-\iu \bm{k}\cdot\bm{r}}\approx 1$ where $\bm{r}$ is the position of charges)
the transition dipole moments can be assumed here to be real valued.
We have also
\begin{equation}\label{eq:Linemix_Fntdef}
F_n(t)=\int_{\bm{v}}F_n(\bm{v},t)\mrm{d}\bm{v},
\end{equation}
so that the total dipole autocorrelation function is given by
\begin{equation}\label{eq:Linemix_Ctsumdef}
C(t)=\sum_n \mu_n F_n(t).
\end{equation}
The initial condition in the hard collision model \eqref{eq:Linemixing_dFdtpartcorr} is based on an assumption of thermal equilibrium at time zero,
and is hence given by $F_n(\bm{v},0)=f(\bm{v})\rho_n\mu_n$ 
where $\rho_n$ is the canonical (thermal equilibrium) density associated with the 
lower transition level of the unperturbed absorbing molecule.
We have thus $F_n(0)=\rho_n\mu_n$ and we can now define the band strength as $C(0)=\sum_n S_n$ where $S_n=\rho_n\mu_n^2$ is the line strength.
Finally, the condition of detailed balancing is assumed for the relaxation matrix so that $W_{kn}\rho_n=\rho_kW_{nk}$, \cf \cite{Ciurylo+Pine2000}.

We denote by $\widetilde{C}(\omega)=\int_0^\infty C(t)\eu^{\iu\omega t}\mrm{d}t$ the Fourier-Laplace transform of the dipole autocorrelation function $C(t)$ having 
symmetry $C(-t)=C^*(t)$.
The spectral density is then given by $I(\omega)=\frac{1}{2\pi}\int_{-\infty}^{\infty}C(t)\eu^{\iu\omega t}\mrm{d}t=\frac{1}{\pi}\Re\{\widetilde{C}(\omega)\}$
for $\omega\in\R$ where we are assuming that the real line belongs to the region of convergence of the corresponding Fourier-Laplace transform.
Now, by using quantum mechanical principles, it can be shown that the absorption coefficient of a gaseous media, $\sigma_\mrm{a}$ (in \unit{m^2/molecule}), 
can be expressed quite generally (in SI-units) as\footnote{It may be noticed here that this formula is most oftenly referred to in Gaussian units as
$\sigma_\mrm{a}=\frac{4\pi^2\omega}{3\hbar\mrm{c_0}}I(\omega)\left(1-\eu^{-\beta\hbar\omega}\right)$, \cf \eg \cite[p.~3084]{Gordon1966a}, \cite[p.~2348]{Robert+Galatry1971}, \cite[p.~111]{Filippov+Tonkov1993} and \cite[p.~12]{Hartmann+etal2008}.}
\begin{equation}\label{eq:RefL_sigmaaexpr}
\sigma_\mrm{a}(\omega)=\frac{\pi\eta_0\omega}{3\hbar}(1-\eu^{-\beta\hbar\omega})I(\omega),
\end{equation}
where $I(\omega)$ is the spectral density defined as above, $\omega$ the angular frequency, $\eta_0$ the wave impedance of vacuum,
$\beta=1/k_\mrm{B}T$ and $\hbar=h/2\pi$ where $h$ is Plancks constant \cf \eg \cite[p.~12]{Hartmann+etal2008}.
Here, the factor $1-\eu^{-\beta\hbar\omega}$ above is due to the fluctuation-dissipation theorem stating that 
$I(-\omega)=\eu^{-\beta\hbar\omega}I(\omega)$, \cf \cite[p.~16]{Hartmann+etal2008}.
It is finally noticed here that the frequency dependent parameter $k=\omega/\mrm{c}_0$ relating to the wavenumber of the incident radiation 
is present already in the time-domain in the expression \eqref{eq:Linemixing_dFdtpartcorr} above.

\subsection{Solving the line mixing problem}
We will now formulate the solution to \eqref{eq:Linemixing_dFdtpartcorr} based on various simplifying assumptions.
In essence, this will follow the developments made in \cite{Ciurylo+Pine2000}.
To simplify the notation we introduce a vector-matrix notation where ${\bf F}(\bm{v},t)$ and $\bm{\mu}$ are column vectors with elements $F_n(\bm{v},t)$ and $\mu_n$, 
${\bf W}$ and  ${\bf C}$ are matrices with elements $W_{nn^\prime}$ and $C_{nn^\prime}$, and $\bm{\beta}$, $\bm{\omega}_0$ and $\bm{\rho}$ are the diagonal matrices 
$\bm{\beta}=\mrm{diag}\{\beta_n\}$, $\bm{\omega}_0=\mrm{diag}\{\omega_{0n}\}$ and $\bm{\rho}=\mrm{diag}\{\rho_n\}$, respectively. 
The identity matrix is denoted ${\bf I}$.
The line mixing problem \eqref{eq:Linemixing_dFdtpartcorr} can now be formulated as the following initial value problem for $t\geq 0$
\begin{equation}\label{eq:Linemix_dFdtDelayedLinemix}
\left\{\begin{array}{l}
\displaystyle\frac{\partial}{\partial t}{\bf F}(\bm{v},t)=-\left( {\bf W}+\bm{\beta}+\iu\left(\bm{\omega}_0+\bm{k}\cdot\bm{v}\cdot {\bf I}\right)\right) {\bf F}(\bm{v},t) \vspace{0.2cm} \\
\hspace{3.0cm} +f(\bm{v})\left( \bm{\beta}- {\bf C} \right)\int_{\bm{v}^\prime}{\bf F}(\bm{v}^\prime,t)\mrm{d}\bm{v}^\prime \vspace{0.2cm}  \\
{\bf F}(\bm{v},0) = f(\bm{v})\bm{\rho}\bm{\mu}.
\end{array}\right.
\end{equation}
We define also the vector ${\bf F}(t)$ with elements $F_n(t)$, so that the correlation function can be expressed as
$C(t)=\bm{\mu}^\mrm{T}{\bf F}(t)$ and its Fourier-Laplace transform as $\widetilde{C}(\omega)=\bm{\mu}^\mrm{T}\widetilde{\bf F}(\omega)$
where $\widetilde{\bf F}(\omega)$ is the Fourier-Laplace transform of the vector ${\bf F}(t)$.
From \eqref{eq:Linemix_Fntdef} we have also that $\widetilde{\bf F}(\omega)=\int_{\bm{v}}\widetilde{\bf F}(\bm{v},\omega)\mrm{d}\bm{v}$
where $\widetilde{\bf F}(\bm{v},\omega)$ is the Fourier-Laplace transform of the vector ${\bf F}(\bm{v},t)$.

We proceed now by taking the Fourier-Laplace transform of \eqref{eq:Linemix_dFdtDelayedLinemix} yielding
\begin{multline}
-\iu\omega \widetilde{\bf F}(\bm{v},\omega)-f(\bm{v})\bm{\rho}\bm{\mu} \\
=-\left( {\bf W}+\bm{\beta}+\iu\left(\bm{\omega}_0+\bm{k}\cdot\bm{v}\cdot {\bf I}\right)\right) \widetilde{\bf F}(\bm{v},\omega) \\
+f(\bm{v})\left( \bm{\beta}- {\bf C} \right)\int_{\bm{v}^\prime}\widetilde{\bf F}(\bm{v}^\prime,\omega)\mrm{d}\bm{v}^\prime, 
\end{multline}
or
\begin{multline}\label{eq:Linemix_F2vomegaeqD0}
\left( {\bf W}+\bm{\beta}-\iu\left(\omega\cdot{\bf I}-\bm{\omega}_0-\bm{k}\cdot\bm{v}\cdot {\bf I}\right)\right) \widetilde{\bf F}(\bm{v},\omega) \\
-f(\bm{v})\left( \bm{\beta}- {\bf C} \right)\int_{\bm{v}^\prime}\widetilde{\bf F}(\bm{v}^\prime,\omega)\mrm{d}\bm{v}^\prime 
=f(\bm{v})\bm{\rho}\bm{\mu}. 
\end{multline}

Following the same procedure as in \cite{Ciurylo+Pine2000}, we introduce now
\begin{equation}\label{eq:Linemix_GvomegadefD}
{\bf G}(\bm{v},\omega)=\left( {\bf W}+\bm{\beta}-\iu\left(\omega\cdot{\bf I}-\bm{\omega}_0-\bm{k}\cdot\bm{v}\cdot {\bf I}\right)\right)^{-1},
\end{equation}
so that
\begin{multline}\label{eq:Linemix_F2vomegaeqD}
 \widetilde{\bf F}(\bm{v},\omega) 
-f(\bm{v}){\bf G}(\bm{v},\omega)\left( \bm{\beta}- {\bf C} \right)\int_{\bm{v}^\prime}\widetilde{\bf F}(\bm{v}^\prime,\omega)\mrm{d}\bm{v}^\prime \\
=f(\bm{v}){\bf G}(\bm{v},\omega)\bm{\rho}\bm{\mu}.
\end{multline}
Next, by introducing
\begin{equation}\label{eq:Linemix_GomegadefD}
{\bf G}(\omega)=\int_{\bm{v}}f(\bm{v}){\bf G}(\bm{v},\omega)\mrm{d}\bm{v},
\end{equation}
and integrating \eqref{eq:Linemix_F2vomegaeqD} over velocity space, we obtain
\begin{equation}\label{eq:Linemix_FtildeeqD}
 \widetilde{\bf F}(\omega) 
-{\bf G}(\omega)\left( \bm{\beta}- {\bf C} \right)\widetilde{\bf F}(\omega)
={\bf G}(\omega)\bm{\rho}\bm{\mu}.
\end{equation}
The solution to \eqref{eq:Linemix_FtildeeqD} can thus be expressed as
\begin{equation}
\widetilde{\bf F}(\omega)=\left({\bf I}- {\bf G}(\omega)\left( \bm{\beta}- {\bf C} \right) \right)^{-1}{\bf G}(\omega)\bm{\rho}\bm{\mu}.
\end{equation}
The Fourier-Laplace transform $\widetilde{C}(\omega)$ is now given by
\begin{multline}\label{eq:Linemix_genComegasol2D}
\widetilde{C}(\omega)=\bm{\mu}^\mrm{T}\widetilde{\bf F}(\omega) \\
=\bm{\mu}^\mrm{T}\left({\bf I}- {\bf G}(\omega)\left( \bm{\beta}- {\bf C} \right) \right)^{-1}{\bf G}(\omega)\bm{\rho}\bm{\mu}.
\end{multline}
This is the result given in \cite[Eq.~(2.13)]{Ciurylo+Pine2000}.

In order to effectively exploit a diagonalization of the matrix ${\bf W}+\bm{\beta}+\iu\bm{\omega}_0$ and to write \eqref{eq:Linemix_genComegasol2D}
as a sum over individual lines, we will see below that 
the eigenvectors of ${\bf W}+\iu\bm{\omega}_0$ must be independent of velocity (whereas its eigenvalues may depend on speed) and that 
$\bm{\beta}-{\bf C}$ must be proportional to the identity matrix ${\bf I}$, \cf also \cite[Appendix A]{Ciurylo+Pine2000}.
Hence, in the following we will assume that we have a case of Dicke narrowing with uncorrelated hard collisions where
${\bf C}={\bf 0}$ and $\bm{\beta}=\beta{\bf I}$ where $\beta$ is the line-independent frequency of velocity changing collisions.
We recall that the relaxation matrix ${\bf W}$ satisfies the condition of detailed balancing, \ie ${\bf W}\bm{\rho}=\bm{\rho}{\bf W}^\mrm{T}$.
We can then introduce the symmetric matrix  $\bm{\Gamma}=\bm{\rho}^{-1/2}{\bf W}\bm{\rho}^{1/2}$ and start
by diagonalizing the complex symmetric matrix 
\begin{equation}\label{eq:Linemix_spectdecompGammaD}
\bm{\Gamma}+\iu\bm{\omega}_0={\bf Q}\bm{\Lambda}{\bf Q}^\mrm{T},
\end{equation}
where $\bm{\Lambda}$ is a diagonal matrix of complex eigenvalues $\lambda_n=\gamma_n+\iu\omega_n$ and 
${\bf Q}^{-1}={\bf Q}^\mrm{T}$, \cf \cite[Theorem 4.4.13 on p.~211-212]{Horn+Johnson1985}.
By pre- and post-multiplying \eqref{eq:Linemix_spectdecompGammaD} with $\bm{\rho}^{1/2}$ and $\bm{\rho}^{-1/2}$, respectively, we can readily see that
\begin{equation}\label{eq:Linemix_spectdecompWD}
{\bf W}+\iu\bm{\omega}_0={\bf A}\bm{\Lambda}{\bf A}^\mrm{-1},
\end{equation}
where ${\bf A}=\bm{\rho}^{1/2}{\bf Q}$ and ${\bf A}^{-1}={\bf Q}^\mrm{T}\bm{\rho}^{-1/2}$.
We can also see that
\begin{multline}
 {\bf W}+\beta{\bf I}-\iu\left(\omega\cdot{\bf I}-\bm{\omega}_0-\bm{k}\cdot\bm{v}\cdot {\bf I}\right) \\
 ={\bf A}\left(\bm{\Lambda}+(\beta-\iu\omega+\iu\bm{k}\cdot\bm{v})\cdot{\bf I}\right){\bf A}^\mrm{-1},
\end{multline}
and hence diagonalize the expression \eqref{eq:Linemix_GvomegadefD} as
\begin{multline}\label{eq:Linemix_Gvomegasol}
{\bf G}(\bm{v},\omega)={\bf A}\left(\bm{\Lambda}+(\beta-\iu\omega+\iu\bm{k}\cdot\bm{v})\cdot{\bf I}\right)^{-1}{\bf A}^\mrm{-1}  \\
={\bf A}\cdot\mrm{diag}\left\{ \frac{1}{\gamma_n+\beta-\iu(\omega-\omega_n)+\iu\bm{k}\cdot\bm{v}}\right\}{\bf A}^\mrm{-1}.
\end{multline}

As already mentioned above, we will now assume that ${\bf A}$ is independent of velocity. 
It follows then from \eqref{eq:Linemix_GomegadefD} and \eqref{eq:Linemix_Gvomegasol} that 
\begin{equation}\label{eq:Linemix_Gomegaexpr1D}
{\bf G}(\omega)=\int_{\bm{v}}f(\bm{v}){\bf G}(\bm{v},\omega)\mrm{d}\bm{v}
={\bf A}{\bf D}(\omega){\bf A}^\mrm{-1},
\end{equation}
where 
\begin{multline}\label{eq:Linemix_DomegaD}
{\bf D}(\omega)=\mrm{diag}\left\{ \int_{\bm{v}} \frac{f(\bm{v})\mrm{d}\bm{v}}{\gamma_n+\beta-\iu(\omega-\omega_n)+\iu\bm{k}\cdot\bm{v}}\right\} \\
=\mrm{diag}\left\{ \frac{\sqrt{\pi}}{k\widetilde{v}}w\left(\frac{\omega-\omega_n+\iu(\gamma_n+\beta)}{k\widetilde{v}}\right) \right\}
\end{multline}
and where the last line is valid when the eigenvalues $\lambda_n$ are independent of velocity.
Here, $w(z)$ denotes the Faddeeva function as defined in Appendix \ref{sect:integrals} and \eqref{eq:Linemix_ComegaDoppler6} has been used in the last step,
\cf also \cite[p.~381]{Ciurylo+Pine2000}.

Based on \eqref{eq:Linemix_genComegasol2D} with ${\bf C}={\bf 0}$ and $\bm{\beta}=\beta{\bf I}$ as well as the factorization \eqref{eq:Linemix_Gomegaexpr1D}, 
we can now write the final Fourier-Laplace transform as
\begin{multline}\label{eq:Linemix_finComegasol2D}
\widetilde{C}(\omega)=\bm{\mu}^\mrm{T}\left({\bf I}- \beta{\bf G}(\omega) \right)^{-1}{\bf G}(\omega)\bm{\rho}\bm{\mu} \\
=\bm{\mu}^\mrm{T}\left({\bf A}{\bf I}{\bf A}^\mrm{-1}- \beta{\bf A}{\bf D}(\omega){\bf A}^\mrm{-1} \right)^{-1} 
{\bf A}{\bf D}(\omega){\bf A}^\mrm{-1}\bm{\rho}\bm{\mu} \\
=\bm{\mu}^\mrm{T}{\bf A}\left({\bf I}-\beta{\bf D}(\omega) \right)^{-1}{\bf D}(\omega){\bf A}^\mrm{-1}\bm{\rho}\bm{\mu} \\
=\bm{\mu}^\mrm{T}\bm{\rho}^{1/2}{\bf Q}\left({\bf I}-\beta{\bf D}(\omega) \right)^{-1}{\bf D}(\omega){\bf Q}^\mrm{T}\bm{\rho}^{-1/2}\bm{\rho}\bm{\mu} \\
={\bf M}^\mrm{T}{\bf Q}\left({\bf I}-\beta{\bf D}(\omega) \right)^{-1}{\bf D}(\omega){\bf Q}^\mrm{T}{\bf M},
\end{multline}
where ${\bf A}=\bm{\rho}^{1/2}{\bf Q}$, ${\bf A}^{-1}={\bf Q}^\mrm{T}\bm{\rho}^{-1/2}$ and ${\bf M}=\bm{\rho}^{1/2}\bm{\mu}$.
Let us now just briefly return to the general case \eqref{eq:Linemix_genComegasol2D} and realize that the factorization ${\bf G}(\omega)={\bf A}{\bf D}(\omega){\bf A}^\mrm{-1}$
does not help diagonalize the final expression \eqref{eq:Linemix_finComegasol2D} unless the matrices ${\bf A}^\mrm{-1}$ and $\bm{\beta}- {\bf C}$ commute, \cf \cite[Appendix A]{Ciurylo+Pine2000}.
We can also see in the simplified analysis above why it is essential that ${\bf A}$ is independent of velocity.
Thus, coming back to our special case with uncorrelated Dicke narrowing where $\bm{\beta}- {\bf C}=\beta{\bf I}$ as in \eqref{eq:Linemix_finComegasol2D} above, we can now see that
$\widetilde{C}(\omega)$ can be written as the following sum over individual lines
\begin{equation}\label{eq:Linemix_C2Dickesol}
\widetilde{C}(\omega)=\sum_n \frac{a_n^2D_{n}(\omega)}{1-\beta D_n(\omega)},
\end{equation}
where $D_{n}(\omega)$ are the diagonal elements defined by \eqref{eq:Linemix_DomegaD}, and
\begin{equation}\label{eq:Linemix_anbndefD}
a_n={\bf M}^\mrm{T}{\bf q}_n
\end{equation}
where ${\bf q}_n$ is the $n$th column of ${\bf Q}$.

The effect of Doppler broadening can be ignored by putting the wave number $k=0$ while keeping $\omega$ fixed, which immediately yields
\begin{equation}\label{eq:Linemix_DomegaDk0}
D_{n}(\omega)=\frac{1}{\gamma_n+\beta-\iu(\omega-\omega_n)},
\end{equation}
provided that the eigenvalues $\lambda_n=\gamma_n+\iu\omega_n$ are independent of velocity.
However, in this case we can readily see that the solution becomes independent of $\beta$, and hence
\begin{equation}\label{eq:Linemix_C2Dickesol2}
\widetilde{C}(\omega)=\sum_n \frac{a_n^2}{\gamma_n-\iu(\omega-\omega_n)}.
\end{equation}
In fact, we can readily see already in \eqref{eq:Linemix_F2vomegaeqD0} that the solution will be independent of $\bm{\beta}$ when $\bm{k}=\bm{0}$.
Physically, this means that the frequency of velocity changing collisions are of no significance if there is no Doppler effect ($k=0$)
and that Dicke narrowing is a pure Doppler phenomena \cite{Rautian+Sobelman1967}.

\subsection{Rosenkranz parameters}

We consider now the spectral decomposition \eqref{eq:Linemix_spectdecompGammaD}, or equivalently
\eqref{eq:Linemix_spectdecompWD} where $\bm{\Gamma}=\bm{\rho}^{-1/2}{\bf W}\bm{\rho}^{1/2}$.
We recall that $\left(\bm{\Gamma}+\iu\bm{\omega}_0\right){\bf q}_n=\lambda_n{\bf q}_n$
and $\lambda_n=\gamma_n+\iu\omega_n$.
Assuming that $\bm{\Gamma}=p\widehat{\bm{\Gamma}}$ where $p$ is pressure, 
the Rosenkranz parameters are given by the following first order approximations
\begin{equation}\label{eq:Linemix_RosenkranzlambdanD}
\lambda_n=p\widehat{\Gamma}_{nn}+\iu\omega_{0n},
\end{equation}
and
\begin{equation}\label{eq:Linemix_RosenkranzqnD}
q_{kn}=\left\{\begin{array}{ll}
1 & k=n \vspace{0.2cm} \\
\displaystyle\frac{\iu p\widehat{\Gamma}_{kn}}{\omega_{0k}-\omega_{0n}} & k\neq n,
\end{array}\right.
\end{equation}
where $q_{kn}$ are the elements of ${\bf q}_n$, \cf \cite[(A.4) and (A.5)]{Rosenkranz1975}.
It is noted that this analysis can be performed already in velocity space if the eigenvalues $\lambda_n$ depend on $\bm{v}$.
Based on \eqref{eq:Linemix_anbndefD} and \eqref{eq:Linemix_RosenkranzqnD}, we can now find that
\begin{equation}\label{eq:Linemix_anfirstorder}
a_n=M_n+\iu p\sum_{k\neq n}M_k\frac{\widehat{\Gamma}_{kn}}{\omega_{0k}-\omega_{0n}}
\end{equation}
and to first order in $p$ we can also derive the following expression
\begin{multline}\label{eq:Linemix_standardanbn}
a_n^2=M_n^2+2\iu p M_n\sum_{k\neq n}\frac{M_k\widehat{\Gamma}_{kn}}{\omega_{0k}-\omega_{0n}} \\
=\rho_n\mu_n^2+2\iu p \rho_n\mu_n\sum_{k\neq n}\frac{\mu_k\widehat{W}_{kn}}{\omega_{0k}-\omega_{0n}},
\end{multline}
where we have also employed the definitions $M_n=\rho_n^{1/2}\mu_n$ and $\widehat{\Gamma}_{kn}=\rho_k^{-1/2}\widehat{W}_{kn}\rho_n^{1/2}$, \cf \cite[Eq.~(2) and (3)]{Rosenkranz1975}. 

\section{A modified projection based model for line mixing}\label{sect:Linemix_SimpleStrongD}
A simple projection based strong collision model for the relaxation matrix ${\bf W}$ has previously been proposed in 
\cite{Dokuchaev+etal1982,Bulanin+etal1984,Filippov+Tonkov1993,Tonkov+etal1996,Filippov+etal2002,Tonkov+Filippov2003}.
The determination of this matrix requires no additional spectral parameters except for the (theoretical or experimental)
center frequencies $\omega_{0n}$, the transition dipole moments $\mu_n$, the molecular canonical densities $\rho_n$
and the line widths $\gamma_{0n}$ of the individual lines. 
Here, the line strengths are furthermore given by $S_n=M_n^2$ where $M_n=\rho_n^{1/2}\mu_n$.
It has previously been reported that this model suffers from the major disadvantage 
that it does not take into account the distinguishing features of various perturbing gases nor of the different ro-vibrational branches 
of the radiator, \cf \cite{Tonkov+Filippov2003}. The model also produces the same line width for all lines.
However, as we will show below,  it will only require a small adjustment of this model
based on a diagonal perturbation to obtain a line-mixing model that is able to take into account line specific data
relating to various perturbing gases. The one remaining model parameter $v_\mrm{s}$ (collision frequency for line mixing)
can furthermore be adjusted empirically to the particular radiator and host being used. 
%Comparisons with experimental data show that accurate
%matching and predictions of far wing behavior can be obtained using this simple approach.
A major advantage of this line mixing model is that the computation of
line shapes as in \eqref{eq:Linemix_C2Dickesol} will only require the prior knowledge of individual center frequencies $\omega_{0n}$, line widths $\gamma_{0n}$ and line strengths $S_n$,
in addition to the parameter $v_\mrm{s}$ to be discussed below.
% in order to establish the corresponding first order Rosenkranz parameters, as well as an exact solution in the case of no Doppler broadening.

\subsection{The basic projection based method}
The basic projection based strong collsion (SC) method has been proposed 
in \eg \cite{Dokuchaev+etal1982,Bulanin+etal1984,Filippov+Tonkov1993,Tonkov+etal1996,Filippov+etal2002,Tonkov+Filippov2003}
and is briefly summarized below.
Within this strong collision model it is assumed that the relaxation time $\tau_s$ is equal to the mean duration between successive collisions for 
any rotational state of the absorbing molecule \cite{Filippov+Tonkov1993}.
It is furthermore assumed that $\omega_{0n}\tau_\mrm{s} \ll 1$, and the corresponding collision frequency is defined as $v_\mrm{s}=\tau_\mrm{s}^{-1}$.
The symmetrized relaxation matrix $\bm{\Gamma}$ can now be defined in terms of a projector ${\bf I}-\tau_\mrm{s}\bm{\Gamma}$
that restores thermal equilibrium at the characteristic time $\tau_\mrm{s}$, and hence 
\begin{equation}\label{eq:GammaSC}
\bm{\Gamma}=v_\mrm{s}\left({\bf I}-\frac{{\bf M}{\bf M}^\mrm{T}}{{\bf M}^\mrm{T}{\bf M}}\right),
\end{equation}
where ${\bf M}=\bm{\rho}^{1/2}\bm{\mu}$, \cf \eg \cite[Eq.~(13)]{Tonkov+Filippov2003} and \cite{Filippov+Tonkov1993}.
The corresponding unsymmetric relaxation matrix is then given by 
\begin{equation}
{\bf W}=\bm{\rho}^{1/2}\bm{\Gamma}\bm{\rho}^{-1/2}=v_\mrm{s}\left({\bf I}-\frac{\bm{\rho}\bm{\mu}\bm{\mu}^\mrm{T}}{\bm{\mu}^\mrm{T}\bm{\rho}\bm{\mu}}\right),
\end{equation}
and which thus satisfies the condition of detailed balancing ${\bf W}\bm{\rho}=\bm{\rho}{\bf W}^\mrm{T}$.
It is also noticed here that $C(0)=\bm{\mu}^\mrm{T}\bm{\rho}\bm{\mu}={\bf M}^\mrm{T}{\bf M}$.

The line width parameter $v_\mrm{s}$ can now be chosen so that  the theoretical or experimental line widths $\gamma_{0n}$
are suitably approximated by the diagonal elements $W_{nn}=\Gamma_{nn}=v_\mrm{s}(1-S_n/C(0))$.
To this end, the following formula 
\begin{equation}\label{eq:Linemix_vssol2D}
v_\mrm{s}=\frac{\displaystyle\sum_n \gamma_{0n} S_n}{\displaystyle\sum_n S_n},
\end{equation}
has been suggested in \eg \cite[p.~113]{Filippov+Tonkov1993} and \cite[Eq.~(16)]{Tonkov+Filippov2003}.

The strong collision (SC) model above has been validated against experimental data and compared to 
an improved technique referred to as adjustable branch coupling (ABC) in \eg \cite{Tonkov+etal1996,Tonkov+Filippov2003}.
It has been demonstrated that both the SC and ABC methods are able to provide significantly better predictions of the far wing 
behavior of an absorbing gas in comparison to a simple sum of isolated Lorentzian lines.
The ABC method is however more sophisticated as it requires the subdivision of lines into isolated branches, but it is also able to provide better predictions
 as it employs one additional free parameter (interbranch interaction) to match the model to experimental data.
 In our approach here we wish to retain the simplicity of the SC projection method to line coupling described above,
 while at the same time reinstalling the high-resolution aspects of some standard isolated line models.
 This will be the topic of the sections to follow.
 
\subsection{Perturbation}
We will now aim to improve the simple strong collision (SC) projection method above by introducing the slightly perturbed model
\begin{equation}\label{eq:GammaSCmod}
{\Gamma}_{kn}=\left\{\begin{array}{ll}
\displaystyle \gamma_{0n}(v)+\iu\delta_{0n}(v) & k=n, \vspace{0.2cm} \\
\displaystyle -v_\mrm{s} \frac{M_kM_n}{C(0)} & k\neq n,
\end{array}\right.
\end{equation}
where the diagonal elements in \eqref{eq:GammaSC} have been replaced by any (theoretical or experimental)
presumably more accurate and possibly even speed dependent broadening and shift parameters $\gamma_{0n}(v)+\iu\delta_{0n}(v)$.
The same off-diagonal elements as in \eqref{eq:GammaSC} are retained
as a model of the line coupling transfer rates. However, the parameter $v_\mrm{s}$ will now be treated as
an empirical parameter that can readily be adjusted to match experimental data.
It is noted that the unsymmetric relaxation matrix ${\bf W}=\bm{\rho}^{1/2}\bm{\Gamma}\bm{\rho}^{-1/2}$ is still satisfying the condition of detailed balance, just as before.

The modification introduced in \eqref{eq:GammaSCmod} means that we have now added to \eqref{eq:GammaSC} the diagonal perturbation matrix 
\begin{equation}\label{eq:Pdef}
{\bf P}=\mrm{diag}\{\gamma_{0n}+\iu\delta_{0n}-v_\mrm{s}(1-S_n/C(0))\},
\end{equation}
and consequently the perturbed matrix \eqref{eq:GammaSCmod} does no longer correspond to a rigorous projector. 
However, the perturbed model does provide a useful compromise 
in the sense that the rigorous projector is only slightly relaxed in favor of providing two new features: 
An accurate modeling close to the line centers as well as an accurate modeling in the far wings, the latter being achieved by fine tuning the parameter $v_\mrm{s}$.

Let us now briefly discuss the properties of the two orthogonal projectors ${\bf I}-\tau_\mrm{s}\bm{\Gamma}$ and $\tau_\mrm{s}\bm{\Gamma}$,
the former ideally being a projector onto the space parallel to ${\bf M}$ and the latter orthogonal to ${\bf M}$. Let us now consider a vector
${\bf x}={\bf x}_\parallel+{\bf x}_\perp$ being correspondingly represented in line space, and where 
\begin{equation}\label{eq:Linespacedecomp}
\left({\bf I}-\tau_\mrm{s}\bm{\Gamma}\right){\bf x}={\bf x}_\parallel+\left({\bf I}-\tau_\mrm{s}\bm{\Gamma}\right){\bf x}_\perp-\tau_\mrm{s}\bm{\Gamma}{\bf x}_\parallel.
\end{equation}
The first term on the right-hand side of \eqref{eq:Linespacedecomp} is what we aim for. 
It is now assumed that the state of the system under consideration which is involving molecular collisions 
is always relatively close to thermal equilibrium, and hence that the vector ${\bf x}_\parallel$ is dominant over the vector ${\bf x}_\perp$. 
The perturbed matrix  ${\bf I}-\tau_\mrm{s}\bm{\Gamma}$ is furthermore an approximate projector almost ortogonal to ${\bf x}_\perp$, all of which now makes the
second term negligible. Hence, it is the vanishing of the last term in \eqref{eq:Linespacedecomp} that is the most important for maintaining the required projection property.
To this end, it is noted that the required ortogonality property $\bm{\Gamma}{\bf M}={\bf 0}$ is also referred to as a sum rule in \cite[p.~113]{Filippov+Tonkov1993} and \cite[Eq.~(4)]{Tonkov+Filippov2003}.
In the present context this means that we should now choose the parameter $v_\mrm{s}$ to
minimize the least squares norm of the vector ${\bf P}{\bf M}$, yielding
\begin{equation}\label{eq:Linemix_vssol1D}
v_\mrm{s}^\mrm{ls}=\frac{\displaystyle\sum_n \gamma_{0n} S_n \left(1-\frac{S_n}{C(0)} \right)}{\displaystyle\sum_n S_n\left(1-\frac{S_n}{C(0)} \right)^2 },
\end{equation}
where $C(0)=\sum_n S_n$. It may be noticed that the expression \eqref{eq:Linemix_vssol1D} also corresponds to an $S_n$-weighted least squares solution
to minimize the error $v_\mrm{s}(1-S_n/C(0))-\gamma_{0n}$ related to \eqref{eq:GammaSC}, and that \eqref{eq:Linemix_vssol2D} is obtained if the ratio $S_n/C(0)$ can be neglected.
Notably, if $\gamma_{0n}$ is a speed dependent parameter, we would use in \eqref{eq:Linemix_vssol1D} 
either an average $\bar{\gamma}_{0n}=\int_{\bm{v}}f(\bm{v})\gamma_{0n}\mrm{d}\bm{v}$, or we would evaluate $\gamma_{0n}$ at the
most probable speed $\widetilde{v}$ to obtain a speed independent parameter $v_\mrm{s}$.

In practice, we have found that it is very useful to make a very small fine tuning of \eqref{eq:Linemix_vssol1D} to slightly increase the line coupling transfer rates 
for a better match to measurement data. Hence, we may choose $v_\mrm{s}=cv_\mrm{s}^\mrm{ls}$ where $c$ is a constant very close to 1 ($c=1.005$ in our numerical 
examples for $\mrm{CO}_2$ in the $\nu_3$-band).
This fine tuning of $v_\mrm{s}$ has insignificant effect on the absorption close to the line centers, but it can provide an appropriate correction of the far wing behavior.
Obviously, the parameter $c$ can readily be adjusted to match the experimental data of any specific species and ro-vibrational bands of interest.

\subsection{Rosenkranz parameters}
Since the matrix ${\bf W}$ is assumed to be linear with pressure $p$ and $\bm{\Gamma}=p\widehat{\bm{\Gamma}}$, 
we introduce now also the notation $\gamma_{0n}+\iu\delta_{0n}=p(\widehat{\gamma}_{0n}+\iu\widehat{\delta}_{0n})$ 
and $v_\mrm{s}=p\widehat{v}_\mrm{s}$.
By following \eqref{eq:Linemix_RosenkranzlambdanD} through \eqref{eq:Linemix_standardanbn}
and to the first order in $p$, the corresponding Rosenkranz parameters are now obtained as 
\begin{equation}\label{eq:Linemix_RosenkranzparamsDSC}
\left\{\begin{array}{l}
\gamma_n =p\widehat{\gamma}_{0n}, \vspace{0.2cm} \\ 
\omega_n=\omega_{0n}+p\widehat{\delta}_{0n}, 
\end{array}\right.
\end{equation}
as well as
\begin{equation}\label{eq:Linemix_anfirstorderSC}
a_n=M_n+\frac{\iu p\widehat{v}_\mrm{s}M_n}{C(0)}\sum_{k\neq n}\frac{S_k }{\omega_{0n}-\omega_{0k}}
\end{equation}
and
\begin{equation}\label{eq:Linemix_anbnfirstorderSC}
a_n^2=M_n^2+\frac{2\iu p\widehat{v}_\mrm{s}M_n^2}{C(0)}\sum_{k\neq n}\frac{S_k }{\omega_{0n}-\omega_{0k}}.
\end{equation}
It is finally noted that the only prior knowledge required for the computation of \eqref{eq:Linemix_RosenkranzparamsDSC},
\eqref{eq:Linemix_anfirstorderSC} and \eqref{eq:Linemix_anbnfirstorderSC} (except for the design parameter $c\approx 1$ where $\widehat{v}_\mrm{s}=c\widehat{v}_\mrm{s}^\mrm{ls}$)
are the center frequencies $\omega_{0n}$, the line widths $\gamma_{0n}$ and the line strengths $S_n$. These are parameters that are readily available for 
a large variety of species in spectroscopic databases such as \eg HITRAN
\cite{HITRANorg,HITRANdefs,Rothman+etal1998,Simeckova+etal2006,Gordon+etal2017b}.


\subsection{Exact solution}

%\cite{Tonkov+etal1996} % A simple model of the line mixing effect for atmospheric applications: Theoretical background and comparison with experimental profiles
%% See explicit expression eq. (10)-(12).

Following the ideas presented in \cite[Eq.~(10)-(12)]{Tonkov+etal1996} and \cite[Eq.~(14)-(15)]{Tonkov+Filippov2003},
it is possible to develop a simple closed form solution to \eqref{eq:Linemix_dFdtDelayedLinemix}
for the case with uncorrelated collisions without velocity changes and where $\bm{\Gamma}$
is given by the modified projection model \eqref{eq:GammaSCmod}.
Hence, in this case we have ${\bf C}={\bf 0}$ and $\bm{\beta}=\bm{0}$, and \eqref{eq:Linemix_GvomegadefD} and \eqref{eq:Linemix_F2vomegaeqD} yield
\begin{multline}\label{eq:Linemix_genComegasol2Dexact}
\widetilde{C}(\bm{v},\omega)=\bm{\mu}^\mrm{T}\widetilde{\bf F}(\bm{v},\omega) \\
=\bm{\mu}^\mrm{T}\left( {\bf W}-\iu\left(\omega\cdot{\bf I}-\bm{\omega}_0-\bm{k}\cdot\bm{v}\cdot {\bf I}\right)\right)^{-1} 
f(\bm{v})\bm{\rho}\bm{\mu}.
\end{multline}
By employing ${\bf W}=\bm{\rho}^{1/2}\bm{\Gamma}\bm{\rho}^{-1/2}$ and ${\bf M}=\bm{\rho}^{1/2}\bm{\mu}$ we obtain the symmetrized form
\begin{equation}\label{eq:Linemix_genComegasol2Dexact2}
\widetilde{C}(\bm{v},\omega)
={\bf M}^\mrm{T}\left( \bm{\Gamma}-\iu\left(\omega\cdot{\bf I}-\bm{\omega}_0-\bm{k}\cdot\bm{v}\cdot {\bf I}\right)\right)^{-1} {\bf M}f(\bm{v}),
\end{equation}
and by inserting \eqref{eq:GammaSCmod} we obtain
\begin{multline}
\widetilde{C}(\bm{v},\omega) 
={\bf M}^\mrm{T}\left({\bf P}+v_\mrm{s}\left({\bf I}-\frac{{\bf M}{\bf M}^\mrm{T}}{C(0)}\right) \right. \\
\left. -\iu\left(\omega\cdot{\bf I}-\bm{\omega}_0-\bm{k}\cdot\bm{v}\cdot {\bf I}\right) \right)^{-1} {\bf M}f(\bm{v}),
\end{multline}
where ${\bf P}=\mrm{diag}\{\gamma_{0n}+\iu\delta_{0n}-v_\mrm{s}(1-S_n/C(0))\}$.
Now, we introduce the matrices
\begin{equation}
{\bf D}^{-1}=\mrm{diag}\left\{\gamma_{0n}+\iu\delta_{0n}+v_\mrm{s}\frac{S_n}{C(0)}-\iu\left(\omega-\omega_{0n}- \bm{k}\cdot\bm{v}\right) \right\},
\end{equation}
and 
\begin{equation}
{\bf E}^{-1}=-\frac{v_\mrm{s}}{C(0)}
\end{equation}
and write
\begin{equation}
\widetilde{C}(\bm{v},\omega)={\bf M}^\mrm{T}\left( {\bf D}^{-1} + {\bf M}{\bf E}^{-1}{\bf M}^\mrm{T} \right)^{-1} {\bf M}f(\bm{v}).
\end{equation}
By making use of the so called matrix inversion lemma \cite[p.~30]{Mendel1987}, it can readily be seen that
the exact solution in velocity space is given by
\begin{multline}
\widetilde{C}(\bm{v},\omega) \\ ={\bf M}^\mrm{T}\left({\bf D}-{\bf D}{\bf M}\left({\bf M}^\mrm{T}{\bf D}{\bf M}+{\bf E} \right)^{-1}{\bf M}^\mrm{T}{\bf D}  \right){\bf M}f(\bm{v}).
\end{multline}
After some algebra, it is found that
\begin{equation}\label{eq:extactCIIvomega} 
\widetilde{C}(\bm{v},\omega)=\frac{{\bf M}^\mrm{T}{\bf D}{\bf M}f(\bm{v})}{1+{\bf E}^{-1}{\bf M}^\mrm{T}{\bf D}{\bf M}}=
\frac{\widetilde{C}_1(\bm{v},\omega)f(\bm{v})}{1-\frac{v_\mrm{s}}{C(0)}\widetilde{C}_1(\bm{v},\omega)}
\end{equation}
where
\begin{multline}\label{eq:extactC1vomegasol}
\widetilde{C}_1(\bm{v},\omega)={\bf M}^\mrm{T}{\bf D}{\bf M} \\
=\sum_n \frac{S_n}{\gamma_{0n}+v_\mrm{s}\frac{S_n}{C(0)}-\iu\left(\omega-\omega_{0n}-\delta_{0n} - \bm{k}\cdot\bm{v}\right)},
\end{multline}
and where $S_n=M_n^2$ and $C(0)=\sum_n S_n$.
The total Fourier-Laplace transform $\widetilde{C}(\omega)$ is finally obtained by integrating over velocity space as
\begin{equation}\label{eq:exactCIIomega}
\widetilde{C}(\omega)=\int_{\bm{v}}\widetilde{C}(\bm{v},\omega)\mrm{d}\bm{v}.
\end{equation}

In the case when the Doppler effect can be neglected we can set $k=0$ in \eqref{eq:extactC1vomegasol},
and if the broadening and shift parameters $\gamma_{0n}+\iu\delta_{0n}$ are furthermore independent of velocity then
\eqref{eq:extactCIIvomega} gives after integration
\begin{equation}\label{eq:extactCIIomegasolkeq0}
\widetilde{C}(\omega)=\frac{\widetilde{C}_1(\omega)}{1-\frac{v_\mrm{s}}{C(0)}\widetilde{C}_1(\omega)}
\end{equation}
where
\begin{equation}\label{eq:extactC1omegasolkeq0}
\widetilde{C}_1(\omega)
=\sum_n \frac{S_n}{\gamma_{0n}+v_\mrm{s}\frac{S_n}{C(0)}-\iu\left(\omega-\omega_{0n}-\delta_{0n}\right)}.
\end{equation}

Unfortunately, the function \eqref{eq:extactCIIvomega}  can not readily be integrated over velocity space 
based on \eqref{eq:extactC1vomegasol}  when Doppler broadening is present and $k\neq 0$. 
However, if we can assume that the Boltzmann factor $\eu^{-(v/\widetilde{v})^2}$ is very narrow in comparison to the variations in $\widetilde{C}_1(\bm{v},\omega)$,
 \ie if $k\widetilde{v}/\gamma_{0n}\ll 1$, 
then we may approximate \eqref{eq:exactCIIomega} by applying the $f(\bm{v})$-weighted integration (averaging) separately to the numerator and the denominator 
of \eqref{eq:extactCIIvomega}, respectively.
Assuming once again that $\gamma_{0n}+\iu\delta_{0n}$ are independent of velocity this will then yield a result of the same form as in \eqref{eq:extactCIIomegasolkeq0} where
\begin{multline}\label{eq:approxC1omegasol}
\widetilde{C}_1(\omega)=\int_{\bm{v}}\widetilde{C}_1(\bm{v},\omega)f(\bm{v})\mrm{d}\bm{v} \\
=\frac{\sqrt{\pi}}{k\widetilde{v}}\sum_n S_n w\left(\frac{\omega-\omega_{0n}-\delta_{0n}
+\iu \left(\gamma_{0n}+v_\mrm{s}\frac{S_n}{C(0)}\right)}{k\widetilde{v}} \right)
\end{multline}
and where $w(\cdot)$ is the Faddeeva function based on the integral identity \eqref{eq:Linemix_ComegaDoppler6}.
Eventhough \eqref{eq:approxC1omegasol} does not provide a rigorous calculation of \eqref{eq:exactCIIomega},
it is similar to the results obtained in \cite[Eq.~(9)]{Filippov+etal2002}, 
and it has the correct asymptotics as $k\rightarrow 0$ in accordance with \eqref{eq:extactC1omegasolkeq0}.
However, perhaps a more rigorous alternative for including the Doppler effect is then to employ the diagonalization \eqref{eq:Linemix_DomegaD}
and \eqref{eq:Linemix_C2Dickesol} together with the Rosenkranz parameters 
\eqref{eq:Linemix_RosenkranzparamsDSC} and \eqref{eq:Linemix_anfirstorderSC}, and which is also providing
the option to include Dicke narrowing with parameter $\beta$.


\section{Wavenumber domain}\label{sect:wavenumber}

\subsection{Parameter scalings}
We summarize the results of this section by transforming the expressions to the wavenumber domain via the substitution $\omega=2\pi\mrm{c}_0\nu$ where 
$\nu=\lambda^{-1}$ is the wavenumber and $\lambda$ the wavelength of the radiation. 
We have thus $I(\nu)=2\pi\mrm{c}_0 I(\omega)$ and
\begin{equation}
\int I(\omega)\mrm{d}\omega=\int I(\nu)\mrm{d}\nu=C(0)=\sum_n S_n,
\end{equation}
where the correlation function $C(t)$ as well as the line strengths $S_n$ are invariant to the substitution.
The corresponding Fourier-Laplace transform is similarly defined so that $\widetilde{C}(\nu)=2\pi\mrm{c}_0\widetilde{C}(\omega)$
and $I(\nu)=\frac{1}{\pi}\Re\{\widetilde{C}(\nu)\}$. 

Now, we have seen previously in \eqref{eq:RefL_sigmaaexpr} that the absorption coefficient
can be expressed as $\sigma_\mrm{a}=\frac{\pi\eta_0\omega}{3\hbar}(1-\eu^{-\beta\hbar\omega})I(\omega)$ where 
$I(\omega)=\frac{1}{2\pi}\int_{-\infty}^{\infty}C(t)\eu^{\iu\omega t}\mrm{d}t$ is the Fourier transform of the 
dipole autocorrelation function $C(t)$. 
However, since the factor $\omega(1-\eu^{-\beta\hbar\omega})$ is 
only slowly varying over a typical ro-vibrational band, the absorption coefficient $\sigma_\mrm{a}$ can be approximated as being proportional to $I(\omega)$.
Here, we will assume that $\sigma_\mrm{a}=I(\nu)$ as is customary with spectroscopic data bases such as \eg HITRAN
\cite{HITRANorg,HITRANdefs,Rothman+etal1998,Simeckova+etal2006,Gordon+etal2017b}.
This means that the correlation function $C(t)$ as well as the line strengths $S_n$  are now suitable scaled to have
length dimensions and $I(\nu)$ is given in area units.

We introduce now the following parameter scaling
\begin{equation}
\left[ \gamma_{0n}, \delta_{0n}, \gamma_n, v_\mrm{s}, \beta, \gamma_\mrm{D} \right]
=2\pi\mrm{c}_0\left[ \gamma_{0n}^\prime, \delta_{0n}^\prime, \gamma_n^\prime, v_\mrm{s}^\prime, \beta^\prime, \gamma_\mrm{D}^\prime \right],
\end{equation}
where the unprimed parameters refer to the frequency domain and the primed parameters to the wavenumber domain.
Here, we are furthermore introducing the Doppler half-width parameter $\gamma_\mrm{D}=k\widetilde{v}\sqrt{\ln 2}$ so that
\begin{equation}
\gamma_\mrm{D}^\prime
=\frac{\nu\widetilde{v}\sqrt{\ln 2}}{\mrm{c}_0},
\end{equation}
where $k=2\pi\nu$. For notational convenience it is also natural to write $\omega_{0n}=2\pi\mrm{c}_0\nu_{0n}$ and $\omega_{n}=2\pi\mrm{c}_0\nu_{n}$.
The Fourier-Laplace transform in \eqref{eq:Linemix_C2Dickesol} now becomes
\begin{equation}\label{eq:Linemix_C2Dickesolnu}
\widetilde{C}(\nu)
=\sum_n \frac{a_n^2D_{n}(\nu)}{1-\beta^\prime D_n(\nu)},
\end{equation}
where 
\begin{multline}\label{eq:Linemix_Dnnu}
D_n(\nu)=2\pi\mrm{c}_0 D_n(\omega) \\
=\int_{\bm{v}} \frac{f(\bm{v})\mrm{d}\bm{v}}{\gamma_n^\prime+\beta^\prime-\iu(\nu-\nu_n)+\iu\frac{\gamma_\mrm{D}^\prime}{\widetilde{v}\sqrt{\ln 2}}\widehat{\bm{k}}\cdot\bm{v}} \\
=\frac{\sqrt{\pi}}{\gamma_\mrm{D}^\prime/\sqrt{\ln 2}}w\left(\frac{\nu-\nu_n+\iu(\gamma_n^\prime+\beta^\prime)}{\gamma_\mrm{D}^\prime/\sqrt{\ln 2}}\right),
\end{multline}
and where the last line is valid when the eigenvalues $\lambda_n^\prime=\gamma_n^\prime+\iu\nu_n$ are independent of velocity, \cf \eqref{eq:Linemix_DomegaD}.
Without Doppler broadening for $k=0$, we employ instead \eqref{eq:Linemix_DomegaDk0} to yield
\begin{equation}\label{eq:Linemix_DnuDk0}
D_{n}(\nu)=\frac{1}{\gamma_n^\prime+\beta^\prime-\iu(\nu-\nu_n)},
\end{equation}
where $\beta^\prime$ is again redundant as in \eqref{eq:Linemix_C2Dickesol2}, yielding
\begin{equation}\label{eq:Linemix_C2Dickesol2k0}
\widetilde{C}(\nu)=\sum_n \frac{a_n^2}{\gamma_n^\prime-\iu(\nu-\nu_n)}.
\end{equation}

We can see from the definition made in \eqref{eq:Linemix_anbndefD} 
that the parameter $a_n={\bf M}^\mrm{T}{\bf q}_n$ used in \eqref{eq:Linemix_C2Dickesolnu} and \eqref{eq:Linemix_C2Dickesol2k0}
is invariant under the substitution $\omega=2\pi\mrm{c}_0\nu$.
In particular, the spectral decomposition $\bm{\Gamma}+\iu\bm{\omega}_0={\bf Q}\bm{\Lambda}{\bf Q}^\mrm{T}$ 
which was defined in \eqref{eq:Linemix_spectdecompGammaD} now becomes
$\bm{\Gamma}^\prime+\iu\bm{\nu}_0={\bf Q}\bm{\Lambda}^\prime{\bf Q}^\mrm{T}$ where
$\bm{\Gamma}^\prime=\bm{\Gamma}/2\pi\mrm{c}_0$, $\bm{\nu}_0=\mrm{diag}\{\nu_{0n}\}$ and 
$\lambda_n^\prime=\lambda_n/2\pi\mrm{c}_0=\gamma_n^\prime+\iu\nu_n$.
Thus, the eigenvalues $\lambda_n$ are scaled as the frequency parameters, but the eigenvectors $\bm{q}_n$ are dimensionfree invariants.
However, since $I(\nu)$ is now given in area units, the parameter $a_n^2$ must be given in length dimension just like $S_n$
and ${\bf M}=\bm{\rho}^{1/2}\bm{\mu}$ in square root of length, just like ${\bm\mu}$.

The modified projection method based on the perturbed relaxation matrix \eqref{eq:GammaSCmod} is now given by
\begin{equation}\label{eq:GammaSCmodnu}
{\Gamma}_{kn}^\prime=\left\{\begin{array}{ll}
\displaystyle \gamma_{0n}^\prime(v)+\iu\delta_{0n}^\prime(v) & k=n, \vspace{0.2cm} \\
\displaystyle -v_\mrm{s}^\prime \frac{M_kM_n}{C(0)} & k\neq n,
\end{array}\right.
\end{equation}
and where $\gamma_{0n}^\prime+\iu\delta_{0n}^\prime=p(\widehat{\gamma}_{0n}^\prime+\iu\widehat{\delta}_{0n}^\prime)$ 
and $v_\mrm{s}^\prime=p\widehat{v}_\mrm{s}^\prime$ where $p$ is pressure.
Assuming that the broadening and shift parameters $\gamma_{0n}^\prime+\iu\delta_{0n}^\prime$ are furthermore independent of velocity,
the design parameter $\widehat{v}_\mrm{s}^\prime$ can initially be chosen as in \eqref{eq:Linemix_vssol1D}, which becomes
\begin{equation}\label{eq:Linemix_vssol1Dnu}
\widehat{v}_\mrm{s}^{\mrm{ls}\prime}=\frac{\displaystyle\sum_n \widehat{\gamma}_{0n}^\prime S_n \left(1-\frac{S_n}{C(0)} \right)}{\displaystyle\sum_n S_n\left(1-\frac{S_n}{C(0)} \right)^2 },
\end{equation}
where $C(0)=\sum_n S_n$ and $M_n=\sqrt{S_n}$.
In order to fine tune the far wing accuracy of model the parameter $\widehat{v}_\mrm{s}^\prime$ can now be chosen as $\widehat{v}_\mrm{s}^\prime=c\widehat{v}_\mrm{s}^{\mrm{ls}\prime}$
where $c$ is a constant very close to one.

\subsection{Rosenkranz parameters}
The general first order Rosenkranz parameters are now obtained from \eqref{eq:Linemix_RosenkranzlambdanD} through \eqref{eq:Linemix_anfirstorder}  as
\begin{equation}\label{eq:Linemix_RosenkranzlambdanDnu}
\lambda_n^\prime=p\widehat{\Gamma}_{nn}^\prime+\iu\nu_{0n},
\end{equation}
as well as
\begin{equation}\label{eq:Linemix_anfirstordernu}
a_n=M_n+\iu p\sum_{k\neq n}M_k\frac{\widehat{\Gamma}_{kn}^\prime}{\nu_{0k}-\nu_{0n}}
\end{equation}
where $\widehat{\Gamma}_{kn}^\prime=\widehat{\Gamma}_{kn}/2\pi\mrm{c}_0$.
The corresponding Rozenkranz parameters based on the modified projection method \eqref{eq:GammaSCmod} are 
given by \eqref{eq:Linemix_RosenkranzparamsDSC} and \eqref{eq:Linemix_anfirstorderSC} 
and which transforms to the wavenumber domain as
\begin{equation}\label{eq:Linemix_RosenkranzparamsDSCnu}
\left\{\begin{array}{l}
\gamma_n^\prime =p\widehat{\gamma}_{0n}^\prime, \vspace{0.2cm} \\ 
\nu_n=\nu_{0n}+p\widehat{\delta}_{0n}^\prime, 
\end{array}\right.
\end{equation}
and
\begin{equation}\label{eq:Linemix_anfirstorderSCnu}
a_n=M_n+\frac{\iu p\widehat{v}_\mrm{s}^\prime M_n}{C(0)}\sum_{k\neq n}\frac{S_k }{\nu_{0n}-\nu_{0k}}.
\end{equation}

\subsection{Exact solution}
In the case when there is no Doppler broadening ($k=0$)
and if the broadening and shift parameters $\gamma_{0n}^\prime+\iu\delta_{0n}^\prime$ are furthermore independent of velocity then
the exact solution in \eqref{eq:extactCIIomegasolkeq0} and \eqref{eq:extactC1omegasolkeq0} becomes
\begin{equation}\label{eq:extactCIInusolkeq0}
\widetilde{C}(\nu)=\frac{\widetilde{C}_1(\nu)}{1-\frac{v_\mrm{s}^\prime}{C(0)}\widetilde{C}_1(\nu)}
\end{equation}
where
\begin{equation}\label{eq:extactC1nusolkeq0}
\widetilde{C}_1(\nu)
=\sum_n \frac{S_n}{\gamma_{0n}^\prime+v_\mrm{s}^\prime\frac{S_n}{C(0)}-\iu\left(\nu-\nu_{0n}-\delta_{0n}^\prime\right)}.
\end{equation}
Similar expressions can be derived with regard to \eqref{eq:approxC1omegasol}.
%C0=sum(S);
%vsp=sum(gamma.*S.*(1-S/C0))/sum((1-S/C0).*(1-S/C0).*S);
%vsp=vsp*1.005;
%C1_numat(:,ind)=S(ind)./(gamma(ind)+vsp*S(ind)/C0-1i*(nu_calc-nu(ind)-delta_air(ind)*p));
%C1=sum(C1_numat,2);
%Ctilde=C1./(1-(vsp/C0)*C1);
%k_nu=(1/pi)*real(Ctilde);

\subsection{Basic projection method}
As for a comparison, based on the original unperturbed strong collision model \eqref{eq:GammaSC}, we will use instead
\begin{equation}\label{eq:Linemix_vssol2Dnu}
\widehat{v}_\mrm{s}^\prime=\frac{\displaystyle\sum_n \widehat{\gamma}_{0n}^\prime S_n}{\displaystyle\sum_n S_n },
\end{equation}
and in the case with no Doppler broadening ($k=0$) and an exact solution, we will employ \eqref{eq:extactCIInusolkeq0}
together with
\begin{equation}\label{eq:extactC1nusolkeq0SC}
\widetilde{C}_1(\nu)
=\sum_n \frac{S_n}{v_\mrm{s}^\prime-\iu\left(\nu-\nu_{0n}\right)},
\end{equation} 
as suggested in \cite[Eq.~(10)-(12)]{Tonkov+etal1996} and \cite[Eq.~(14)-(15)]{Tonkov+Filippov2003}.
%C0=sum(S);
%vsp=sum(gamma.*S)/C0;
%%vsp=vsp*0.6;
%C1_numat(:,ind)=S(ind)./(vsp-1i*(nu_calc-nu(ind)));
%C1=sum(C1_numat,2);
%Ctilde=C1./(1-(vsp/C0)*C1);
%k_nu=(1/pi)*real(Ctilde);
As for the Rosenkranz approximation based on the unperturbed model \eqref{eq:GammaSC}, the only difference is with the computation of eigenvalues where 
\begin{equation}\label{eq:Linemix_RosenkranzparamsDSCnuSC}
\left\{\begin{array}{l}
\gamma_n^\prime =p\widehat{v}_\mrm{s}^\prime(1-S_n/C(0)), \vspace{0.2cm} \\ 
\nu_n=\nu_{0n},
\end{array}\right.
\end{equation}
instead of \eqref{eq:Linemix_RosenkranzparamsDSCnu}.
The other relations \eqref{eq:Linemix_C2Dickesolnu} through \eqref{eq:Linemix_C2Dickesol2k0}
and \eqref{eq:Linemix_anfirstorderSCnu} are obtained as above.


%Extension to velocity dependent parameters...


\section{Numerical examples}

%nu_calc=(2250:0.01:2500)'; % Wavenumber domain for the computation
%d=10; % layer thickness in centimeter
%p=1;     % total pressure (atm)
%ps=415*10^(-6)*p; % partial pressure CO2 415 ppm

% A perturbative projection approach to line mixing 

%Typically,  $c$ is a constant very close to 1 ($c=1.005$ in our numerical examples for $\mrm{CO}_2$ in the $\nu_3$-band).

In Figs. \ref{fig:Projection_fig4} through \ref{fig:Projection_fig6} are shown the relative absorption coefficients for the $\nu_3$ $\mrm{CO}_2$-band in dry air 
at $T=20$\unit{\degree C} and total pressure $p=1$\unit{atm}, calculated using HITRAN2016 parameters \cite{Gordon+etal2017b}. 
The input parameters $\gamma_{0n}$ and $\delta_{0n}$ are calculated for $415$\unit{ppm} $\mrm{CO}_2$  and the absorption coefficient
is then scaled for path length in\unit{cm} and partial pressure in\unit{atm}.
In Figs. \ref{fig:Projection_fig4} and \ref{fig:Projection_fig5} are also included a comparison to measurement data which have been visually interpreted from \cite[Fig.~2]{Tonkov+Filippov2003}
as indicated here with the blue rings.
The blue dashdotted lines indicate the sum of isolated Lorentzian lines, the red solid lines the basic projection method \eqref{eq:extactCIInusolkeq0} together with 
\eqref{eq:Linemix_vssol2Dnu} and \eqref{eq:extactC1nusolkeq0SC} and the dashed black lines the modified projection method \eqref{eq:extactCIInusolkeq0}
together with  \eqref{eq:Linemix_vssol1Dnu} and \eqref{eq:extactC1nusolkeq0}. 
Here, $v_\mrm{s}^\prime=p\widehat{v}_\mrm{s}^\prime$ and 
the parameter $\widehat{v}_\mrm{s}^\prime$ is furthermore chosen as
$\widehat{v}_\mrm{s}^\prime=c\widehat{v}_\mrm{s}^{\mrm{ls}\prime}$ with $c=1.005$ 
in order to fine tune the absorption for a better match in the far wing.
The Figs. \ref{fig:Projection_fig5} and \ref{fig:Projection_fig6} illustrate that this can be done without affecting the accuracy
close to the line centers.

\begin{figure}[htb]
\begin{center}
\includegraphics[width=0.48\textwidth]{Projection_fig4}
\end{center}
\vspace{-5mm}
\caption{Relative absorption coefficient for the $\nu_3$ $\mrm{CO}_2$-band in dry air at $T=20$\unit{\degree C} and total pressure $p=1$\unit{atm} calculated using HITRAN2016 parameters.
The blue dashdotted line (Lorentzian) indicates the sum of isolated Lorentzian lines, the red solid line (proj.) the basic projection method 
and the dashed black line (mod. proj.) the modified projection method.
The blue rings (meas) indicate the corresponding measurement data which have been visually interpreted from \cite[Fig.~2]{Tonkov+Filippov2003}. }
\label{fig:Projection_fig4}
\end{figure}

\begin{figure}[htb]
\begin{center}
\includegraphics[width=0.48\textwidth]{Projection_fig5}
\end{center}
\vspace{-5mm}
\caption{Same plot as in Fig.~\ref{fig:Projection_fig4} focusing on the wavenumber range 2385-2500\unit{cm^{-1}}.
Notice how the modified projection (black dashed line) is able to improve the prediction of the basic projection method (red solid line) in the far wing region.}
\label{fig:Projection_fig5}
\end{figure}

\begin{figure}[htb]
\begin{center}
\includegraphics[width=0.48\textwidth]{Projection_fig6}
\end{center}
\vspace{-5mm}
\caption{Same plot as in Fig.~\ref{fig:Projection_fig4} focusing on the wavenumber range 2361.35-2361.60\unit{cm^{-1}}, 
close to one of the ro-vibrational line centers.
Notice how the modified projection (black dashed line) is able to improve the prediction of the basic projection method (red solid line) close to resonance.
The modified projection is coincident with the isolated Lorentzian in this plot.}
\label{fig:Projection_fig6}
\end{figure}

\section{Summary and future research}

To summarize the present setting, we assume that we wish to express the spectral function $\widetilde{C}(\omega)$ 
based on a diagonalization of the line mixing problem implemented as the following sum over individual lines
\begin{equation}\label{eq:Linemix_C2Dickesolcopy}
\widetilde{C}(\omega)=\sum_n \frac{a_n^2D_{n}(\omega)}{1-\beta D_n(\omega)},
\end{equation}
where $\beta$ is a Dicke narrowing parameter and $a_n$ a complex valued constant 
as was derived in \eqref{eq:Linemix_C2Dickesol} and \eqref{eq:Linemix_anbndefD}, \cf also \cite[p.~381]{Ciurylo+Pine2000}.
We have also from \eqref{eq:Linemix_DomegaD}
\begin{multline}\label{eq:Linemix_DomegaDcopy}
D_n(\omega)= \int_{\bm{v}} \frac{f(\bm{v})\mrm{d}\bm{v}}{\gamma_n+\beta-\iu(\omega-\omega_n)+\iu\bm{k}\cdot\bm{v}}\\
= \frac{\sqrt{\pi}}{k\widetilde{v}}w\left(\frac{\omega-\omega_n+\iu(\gamma_n+\beta)}{k\widetilde{v}}\right) 
\sim \frac{1}{\gamma_n+\beta-\iu(\omega-\omega_n)}
\end{multline}
where the second equality is valid when the eigenvalues of the relaxation matrix $\lambda_n=\gamma_n+\iu\omega_n$
are independent of velocity and the last expression is the asymptotic limit when the wavenumber $k$ approaches zero (in which case $\beta$ becomes redundant).
Here, $w(\cdot)$ denotes the Faddeeva function.

The modified projection method is defined by \eqref{eq:GammaSCmod} and \eqref{eq:Linemix_vssol1D} where 
$\gamma_{0n}(v)$ and $\delta_{0n}(v)$ are any high resolution isolated line broadening and shift parameters of interest that may or may not depend on speed $v$.
The parameter $v_\mrm{s}=cv_\mrm{s}^\mrm{ls}$  quantifies the strength of the line coupling transfer rates where $v_\mrm{s}^\mrm{ls}$ is given by \eqref{eq:Linemix_vssol1D}.
The parameter $c$ is very close to one and can be used to fine tune the far wing absorption to match any given measurement data.
An exact solution for the static case (setting $k=0$ means pure pressure broadening) and with velocity independent parameters is given by \eqref{eq:extactCIIomegasolkeq0} and \eqref{eq:extactC1omegasolkeq0}.

The first order Rosenkranz parameters for the modified projection method are given by \eqref{eq:Linemix_RosenkranzparamsDSC} 
and \eqref{eq:Linemix_anfirstorderSC} where $p$ is pressure and $v_\mrm{s}=p\widehat{v}_\mrm{s}$, etc.
The eigenvalue parameters are $\gamma_n(v)=p\widehat{\gamma}_{0n}(v)$ and $\omega_n(v)=\omega_{0n}+p\widehat{\delta}_{0n}(v)$ where $\omega_{0n}$ are the molecular transition frequencies. 
The parameter $a_n$ given by \eqref{eq:Linemix_anfirstorderSC} is determined solely by the design parameter $c$ together with
the center frequencies $\omega_{0n}$, the line widths $\gamma_{0n}$ and the line strengths $S_n$
which are readily available in spectroscopic databases such as \eg HITRAN
\cite{HITRANorg,HITRANdefs,Rothman+etal1998,Simeckova+etal2006,Gordon+etal2017b}.

The following research topics of interest can now be foreseen.
\subsection{Validation of the modified projection method using measured spectroscopic data}
The exact solution \eqref{eq:extactCIIomegasolkeq0} and \eqref{eq:extactC1omegasolkeq0} 
for the static case with pressure broadening will be further validated against measured
spectroscopic data for other species and frequency bands, similar as in Figs. \ref{fig:Projection_fig4} through \ref{fig:Projection_fig6}.
We will also validate the accuracy of the modified projection method in the case when Doppler broadening is present
and the Rosenkranz approximation is incorporated in the implementation of the line mixing parameters as explained above.
In the case with velocity independent broadening and shift parameters $\gamma_{0n}$ and $\delta_{0n}$, 
the Doppler effect can be incorporated by considering the second equality in \eqref{eq:Linemix_DomegaDcopy}
where $w(\cdot)$ is the Faddeeva function. The investigation is aiming to explore our hypothesis that a high model accuracy
can be maintained close to the line centers at the same time as the absorption in the far wing can be fine tuned to reach the required accuracy
for various species and frequency bands.

\subsection{Efficient computation for Doppler broadening}
There are many efficient numerical codes developed for the implementation
of the Faddeeva function, see \eg \cite{Armstrong1967,Schreier1992,Schreier2011,Abrarov+Quine2011,Tennyson+etal2014b}.
However, for computationally exhaustive broadband line by line analysis of radiative transfer in the atmosphere, the numerical evaluation of the Faddeeva function
is still a bottleneck. It is much more efficient just to compute its asymptotic limit when this is appropriate, \ie at higher pressure when $\gamma_{0n}$ is relatively large
and in the far wing regions where $\omega-\omega_{0n}$ is relatively large. 
It has been proven in \cite{Nordebo2021a} that the Voigt profile converges uniformly to the Lorentzian on the whole of its domain as $\gamma_{0n}/k\widetilde{v}\rightarrow \infty$.
It has furthermore been proven that the Voigt profile converges uniformly to the Lorentzian as $(\omega-\omega_{0n})/k\widetilde{v}\rightarrow \infty$.
Based on these theoretical assertions, it is safe to employ numerical means to determine the corresponding criteria for achieving rigorous error bounds, \cf \cite{Nordebo2021a}.
However, the corresponding bounds have not yet been established in the case of line mixing as in \eqref{eq:Linemix_C2Dickesolcopy}.
%Ideally, we would like to establish uniform bounds for relative errors such as
%\begin{equation}\label{eq:idealuniformbounds}
%\frac{ \Re\left\{a_n^2\frac{\sqrt{\pi}}{k\widetilde{v}}w\left(\frac{\omega-\omega_n+\iu(\gamma_n+\beta)}{k\widetilde{v}}\right) -  \frac{a_n^2}{\gamma_n+\beta-\iu(\omega-\omega_n)}\right\}}
%{  \Re\left\{ a_n^2\frac{\sqrt{\pi}}{k\widetilde{v}}w\left(\frac{\omega-\omega_n+\iu(\gamma_n+\beta)}{k\widetilde{v}}\right) \right\}},
%\end{equation}
%where $a_n$ is a complex valed parameter.
This investigation is therefore aiming to develop rigorous and readily implementable criteria for 
applying the last line in \eqref{eq:Linemix_DomegaDcopy} in terms of the given ratios $(\omega-\omega_{0n})/k\widetilde{v}$, $(\gamma_{0n}+\beta)/k\widetilde{v}$ 
as well as the line mixing parameter $a_n$.

\subsection{Extension to velocity dependent parameters}
It has been reported that the highly accurate Hartmann-Tran (HT) line shapes can take line mixing effects into account by simple empirical modifications, 
\cf \eg \cite[Eq.~(9)]{Ngo+etal2013} and \cite[Eq.~(15)]{Pine+Gabard2000}. However, it is not clear how accurate these modifications are 
with respect to the far wing absorption.
The Hartmann-Tran profile is based on the following quadratic model for the speed-dependent pressure broadening and shifts
\begin{equation}\label{eq:RefL_quadraticGammaDelta1}
\gamma_{0n}(v)+\iu\delta_{0n}(v)=C_{0n}+C_{2n}\left(v^2/\widetilde{v}^2-3/2\right)
\end{equation}
where
\begin{equation}\label{eq:RefL_quadraticGammaDelta2}
\left\{\begin{array}{l}
C_{0n=}\gamma_{0n}+\iu\delta_{0n}, \vspace{0.2cm} \\
C_{2n}=\gamma_{2n}+\iu\delta_{2n},
\end{array}\right.
\end{equation}
and where it is noted that $\langle v^2/\widetilde{v}^2 \rangle=\int_{\bm{v}}f(\bm{v})\left(v^2/\widetilde{v}^2\right)\mrm{d}\bm{v}=3/2$.
This quadratic model is now amenable for an analytic solution in the framework of the proposed modified projection method under the Rosenkranz approximation as described above.
In particular, the defining integral in \eqref{eq:Linemix_DomegaDcopy} becomes
\begin{multline}\label{eq:Linemix_DomegaHTPex}
D_n(\omega)= \int_{\bm{v}} \frac{f(\bm{v})\mrm{d}\bm{v}}{\gamma_n+\beta-\iu(\omega-\omega_n)+\iu\bm{k}\cdot\bm{v}}\\
= \int_{\bm{v}} \frac{f(\bm{v})\mrm{d}\bm{v}}{C_{0n}+C_{2n}\left(v^2/\widetilde{v}^2-3/2\right)+\beta-\iu(\omega-\omega_{0n})+\iu\bm{k}\cdot\bm{v}},
\end{multline}
and which can be expressed explicitly in terms of two Faddeeva function evaluations as explained in \cite[Appendix A and B]{Ngo+etal2013}.
The distinguishing feature of the proposed modified projection method is then the way in which the Rosenkranz parameter $a_n$ is calculated with an option
to fine tune the parameter $v_\mrm{s}=cv_\mrm{s}^\mrm{ls}$ for a better far wing accuracy as described above.

It should be noted that the isolated HT profile in general is able to handle partially correlated collisions.
However, following \cite{Ciurylo+Pine2000}, it turns out that a rigorous diagonalization of the line mixing problem as in \eqref{eq:Linemix_C2Dickesolcopy}
can only be achieved with uncorrelated collisions, line independent Dicke narrowing and speed independent pressure broadening and shifts,
see in particular \cite[Appendix~A]{Ciurylo+Pine2000}.
On the other hand, by relaxing the statement above we can readily see that within the first order Rosenkranz approximation it is in fact possible
to achieve a rigorous diagonalization including speed dependent parameters and with an option to efficiently fine tune the absorption in the far wing. 
This investigation is aiming to explore and validate the HT line mixing method as well as the proposed projection based method in these regards,
based on theory as well as on experimental spectroscopic data.


\subsection{Positivity of the spectral function}
%The spectral density is given $I(\omega)=\frac{1}{\pi}\Re\{\widetilde{C}(\omega)\}$ where $\widetilde{C}(\omega)$ is the Fourier-Laplace transform of the dipole autocorrelation function $C(t)$.
From a quantum mechanical point of view, it can readily be seen that dipole autocorrelation function
$C(t)$ is positive definite and hence that it follows from Bochner’s theorem that its Fourier transform $I(\omega)$ must be a positive measure, \cf \cite[p.~13]{Reed+Simon1975}.
However, from the point of view of the general Boltzmann-Liouville transport equation \eqref{eq:Linemixing_dFdtpartcorr},
it is not at all clear what the conditions are for producing a positive spectral density function. There are of course related criteria available
in the literature concerning state space equations such as the positive real lemma \cite{Boyd+etal1994}.
However, these are not readily applicable in the present context where we have an integro-differential equation involving velocity space
and complex valued parameters.

Virtually all spectroscopic models of today involve the Faddeeva function $w(z)$, \cf \eg \cite{Abrarov+Quine2011,Ngo+etal2013,Tennyson+etal2014b}.
The Faddeeva function itself is indeed a positive real function as $\Re\{w(\iu s)\}>0$ for $\Re\{s\}>0$, see \eg \cite[Eq.~7.1.4]{Abramowitz+Stegun1970}. 
%, or if viewed as a Herglotz function where $\Im\{\iu w(z)\}>0$ for $\Im\{z\}>0$,
%see \eg \cite{Nedic+etal2019} and \cite[Eq.~7.1.4]{Abramowitz+Stegun1970}. 
However, it is not at all clear what the conditions are for spectroscopic parameters to produce a positive real function
$\widetilde{C}(\omega)$ in complicated scenarios such as with the partially Correlated quadratic-Speed-Dependent Hard-Collision profile (pCqSD-HCP), 
also known as the Hartmann-Tran Profile (HTP) \cite{Tennyson+etal2014b}. For all practical purposes, both the HTP as well as the 
basic projection based strong collsion (SC) method \cite{Tonkov+Filippov2003} 
will produce positive spectra as long as all the related parameters have sound physical meaning and are appropriately chosen to match experimental data of high quality.
However, we can readily see that the proposed modified projection method \eqref{eq:GammaSCmod} which is based on a perturbation may easily produce an indefinite 
spectral density $I(\omega)$ if the strength of the line coupling transfer rates $v_\mrm{s}$ is chosen too large in relation to the given diagonal elements. 
In fact, based on our experience, it seems that optimal values for $v_\mrm{s}$ may be very close to the point where $I(\omega)$ becomes indefinite. 
This is a very interesting and important observation.
This investigation is therefore aiming to explore this topic in the framework of analytic function theory, passive systems and Herglotz (or positive real) functions \cite{Nedic+etal2019}.
What exactly are the conditions on the line mixing parameters in \eqref{eq:Linemixing_dFdtpartcorr} to produce positive definite spectra?
How can we understand this from a physical/mathematical point of view?

\appendix
\subsection{Integrals involving the Faddeeva function}\label{sect:integrals}

Some of the important integral relationships involving the Maxwell-Boltzmann distribution as well as the Faddeeva function used in this paper are summarized below.
The Faddeeva function is an entire function defined by $w(z)=\eu^{-z^2}\mrm{erfc}(-\iu z)$ where $\mrm{erfc}(z)$ is the complementary error function
defined by $\mrm{erfc}(z)=\frac{2}{\sqrt{\pi}}\int_z^\infty\eu^{-t^2}\mrm{d}t$, see \eg \cite[Eq.~(7.2.1) -- (7.2.3)]{Olver+etal2010}. 
The Faddeeva function is of fundamental importance and very convenient to use in spectroscopic modeling due to the vast literature 
that is available on the theory, algorithms and computer codes for its efficient numerical evaluation, see \eg \cite{Armstrong1967,Schreier1992,Schreier2011,Abrarov+Quine2011,Tennyson+etal2014b}.
The Maxwell-Boltzmann velocity distribution is given by $f(\bm{v})=(\sqrt{\pi}\widetilde{v})^{-3}\eu^{-(v/\widetilde{v})^2}$
where $v=|\bm{v}|$, $\widetilde{v}=\sqrt{2k_\mrm{B}T/\mu_\mrm{r}}$ is the most probable speed, $k_\mrm{B}$ the Boltzmanns constant, 
$T$ the temperature and $\mu_\mrm{r}$ the mass of the radiator, \cf \cite[Chapt.~5]{Blundell+Blundell2010}.

The first integral of interest here is the standard result
\begin{equation}\label{eq:Linemix_CtDopplerresult}
\int_{\bm{v}}f(\bm{v})\eu^{-\iu\bm{k}\cdot\bm{v} t}\mrm{d}\bm{v}=\eu^{-(k\widetilde{v}t/2)^2},
\end{equation}
where $\bm{k}=k\hat{\bm{k}}$ is the wave vector associated with the incident radiation, see \eg \cite{Rautian+Sobelman1967}. 
The integral identity \eqref{eq:Linemix_CtDopplerresult} can readily be obtained by a substitution to spherical coordinates and is valid for all $t\in\R$.

By completing the squares in the exponent and employing the definition of the complementary error function, 
one can also derive the following Fourier-Laplace transform
\begin{equation}\label{eq:Linemix_ComegaDoppler5}
\int_0^\infty\eu^{-(k\widetilde{v}t/2)^2}\eu^{\iu\omega t}\mrm{d}t
=\frac{\sqrt{\pi}}{k\widetilde{v}}w\left(\frac{\omega}{k\widetilde{v}}\right),
\end{equation}
which is valid for all $\omega\in\C$. Another useful integral identity can also be derived by inserting the result \eqref{eq:Linemix_CtDopplerresult}
into \eqref{eq:Linemix_ComegaDoppler5} and changing the order of integration to yield
\begin{equation}\label{eq:Linemix_ComegaDoppler6}
\int_{\bm{v}} \frac{f(\bm{v})\mrm{d}\bm{v}}{\iu\bm{k}\cdot\bm{v}-\iu\omega}
=\frac{\sqrt{\pi}}{k\widetilde{v}}w\left(\frac{\omega}{k\widetilde{v}}\right),
\end{equation}
but which is now valid only for $\Im\{\omega\}>0$. The latter restriction is usually not a problem since the resulting right-hand side can be analytically extended to the whole complex plane. 
An important example is the relation $\frac{1}{\pi}\Re\left\{\frac{\sqrt{\pi}}{k\widetilde{v}} w\left(\frac{\omega}{k\widetilde{v}}\right)\right\}
=\frac{1}{\sqrt{\pi}k\widetilde{v}}\eu^{-\omega^2/k^2\widetilde{v}^2}$ where $\omega\in\R$, and which is a well known formula for Doppler broadening, \cf \eg \cite{Rautian+Sobelman1967,Hartmann+etal2008,Tennyson+etal2014b}.
The integral identities \eqref{eq:Linemix_CtDopplerresult}, \eqref{eq:Linemix_ComegaDoppler5} and \eqref{eq:Linemix_ComegaDoppler6}
are standard integrals which are employed in many papers, such as \eg{\cite{Ngo+etal2013,Ciurylo+Pine2000}}.

%\bibliographystyle{IEEEtran}
%\bibliography{total}

\begin{thebibliography}{10}
\providecommand{\url}[1]{#1}
\csname url@samestyle\endcsname
\providecommand{\newblock}{\relax}
\providecommand{\bibinfo}[2]{#2}
\providecommand{\BIBentrySTDinterwordspacing}{\spaceskip=0pt\relax}
\providecommand{\BIBentryALTinterwordstretchfactor}{4}
\providecommand{\BIBentryALTinterwordspacing}{\spaceskip=\fontdimen2\font plus
\BIBentryALTinterwordstretchfactor\fontdimen3\font minus
  \fontdimen4\font\relax}
\providecommand{\BIBforeignlanguage}[2]{{%
\expandafter\ifx\csname l@#1\endcsname\relax
\typeout{** WARNING: IEEEtran.bst: No hyphenation pattern has been}%
\typeout{** loaded for the language `#1'. Using the pattern for}%
\typeout{** the default language instead.}%
\else
\language=\csname l@#1\endcsname
\fi
#2}}
\providecommand{\BIBdecl}{\relax}
\BIBdecl

\bibitem{Filippov+etal2002}
N.~N. Filippov, V.~P. Ogibalov, and M.~V. Tonkov, ``Line mixing effect on the
  pure $\mrm{CO}_2$ absorption in the 15 $\mu\mrm{m}$ region,'' \emph{Journal
  of Quantitative Spectroscopy \& Radiative Transfer}, vol.~72, pp. 315--325,
  2002.

\bibitem{Tonkov+Filippov2003}
M.~V. Tonkov and N.~N. Filippov, ``{Collision Induced Far Wings of $\mrm{CO}_2$
  and $\mrm{H}_2\mrm{O}$ Bands in IR Spectra},'' \emph{Camy-Peyret C., Vigasin
  A.A. (eds) Weakly Interacting Molecular Pairs: Unconventional Absorbers of
  Radiation in the Atmosphere. NATO Science Series (Series IV: Earth and
  Environmental Sciences)}, vol.~27, pp. 125--136, 2003.

\bibitem{Hartmann+etal2008}
J.-M. Hartmann, C.~Boulet, and D.~Robert, \emph{Collisional effects on
  molecular spectra. Laboratory experiments and models, consequences for
  applications}.\hskip 1em plus 0.5em minus 0.4em\relax Amsterdam: Elsevier,
  2008.

\bibitem{Dokuchaev+etal1982}
A.~B. Dokuchaev, N.~N. Filippov, and M.~V. Tonkov, ``{Line interference in
  $\nu_3$ rotational-vibrational band of $\mrm{N}_2\mrm{O}$ in the strong
  interaction approximation},'' \emph{Physica Scripta}, vol.~25, pp. 378--380,
  1982.

\bibitem{Bulanin+etal1984}
M.~O. Bulanin, A.~B. Dokuchaev, M.~V. Tonkov, and N.~N. Filippov, ``{Influence
  of line interference on the vibration-rotation band shapes},'' \emph{Journal
  of Quantitative Spectroscopy \& Radiative Transfer}, vol.~31, no.~6, pp.
  521--543, 1984.

\bibitem{Filippov+Tonkov1993}
N.~N. Filippov and M.~V. Tonkov, ``Semiclassical analysis of line mixing in the
  infrared bands of $\mrm{CO}$ and $\mrm{CO}_2$,'' \emph{Journal of
  Quantitative Spectroscopy \& Radiative Transfer}, vol.~50, no.~1, pp.
  111--125, 1993.

\bibitem{Tonkov+etal1996}
M.~V. Tonkov, N.~N. Filippov, Y.~M. Timofeyev, and A.~V. Polyakov, ``{A simple
  model of the line mixing effect for atmospheric applications: Theoretical
  background and comparison with experimental profiles},'' \emph{Journal of
  Quantitative Spectroscopy \& Radiative Transfer}, vol.~56, no.~5, pp.
  783--795, 1996.

\bibitem{Liou2002}
K.~N. Liou, \emph{An introduction to atmospheric radiation}.\hskip 1em plus
  0.5em minus 0.4em\relax London, UK: Academic Press, 2002.

\bibitem{Berk+Hawes2017}
A.~Berk and F.~Hawes, ``{Validation of MODTRAN 6 and its line-by-line
  algorithm},'' \emph{Journal of Quantitative Spectroscopy \& Radiative
  Transfer}, vol. 203, pp. 542--556, 2017.

\bibitem{Nordebo2021a}
S.~Nordebo, ``Uniform error bounds for fast calculation of approximate {V}oigt
  profiles,'' \emph{Journal of Quantitative Spectroscopy \& Radiative
  Transfer}, vol. 270, p. 107715, 2021.

\bibitem{HITRANorg}
``{HITRANonline. The HITRAN Database}. https://hitran.org.''

\bibitem{HITRANdefs}
``{HITRANonline. Definitions and units: Line-by-line parameters}.
  https://hitran.org/docs/definitions-and-units/.''

\bibitem{Rothman+etal1998}
L.~S. Rothman and et~al, ``The {HITRAN} molecular spectroscopic database and
  {HAWKS} ({HITRAN} atmospheric workstation): 1996 edition,'' \emph{Journal of
  Quantitative Spectroscopy \& Radiative Transfer}, vol.~60, no.~5, pp.
  665--710, 1998.

\bibitem{Simeckova+etal2006}
M.~\v{S}ime\v{c}kov\'{a}, D.~Jacquemart, L.~S. Rothman, R.~R. Gamache, and
  A.~Goldman, ``Einstein ${A}$-coefficients and statistical weights for
  molecular absorption transitions in the {HITRAN} database,'' \emph{Journal of
  Quantitative Spectroscopy \& Radiative Transfer}, vol.~98, pp. 130--155,
  2006.

\bibitem{Gordon+etal2017b}
I.~E. Gordon, L.~S. Rothman, C.~Hill, R.~V. Kochanov, Y.~Tan, P.~F. Bernath,
  M.~Birk, V.~Boudon, A.~Campargue, K.~V. Chance, B.~J. Drouin, J.-M. Flaud,
  R.~R. Gamache, J.~T. Hodges, D.~Jacquemart, V.~I. Perevalov, A.~Perrin, K.~P.
  Shine, M.-A.~H. Smith, J.~Tennyson, G.~C. Toon, H.~Tran, V.~G. Tyuterev,
  A.~Barbe, A.~G. Cs\'{a}sz\'{a}r, V.~M. Devi, J.~J.~H. T.~Furtenbacher, J.-M.
  Hartmann, A.~Jolly, T.~J. Johnson, T.~Karman, I.~Kleiner, A.~A. Kyuberis,
  J.~Loos, O.~M. Lyulin, S.~T. Massie, S.~N. Mikhailenko, N.~Moazzen-Ahmadi,
  H.~S.~P. M\"{u}ller, O.~V. Naumenko, A.~V. Nikitin, O.~L. Polyansky, M.~Rey,
  M.~Rotger, S.~W. Sharpe, K.~Sung, E.~Starikova, S.~A. Tashkun, J.~V. Auwera,
  G.~Wagner, J.~Wilzewski, P.~Wcis\l{}o, S.~Yu, and E.~J. Zak, ``The
  {HITRAN2016} molecular spectroscopic database,'' \emph{Journal of
  Quantitative Spectroscopy \& Radiative Transfer}, vol. 203, pp. 3--69, 2017.

\bibitem{Rautian+Sobelman1967}
S.~G. Rautian and I.~I. Sobel'man, ``The effect of collisions on the {D}oppler
  broadening of spectral lines,'' \emph{Soviet Physics Uspekhi}, vol.~9, no.~5,
  pp. 701--716, 1967.

\bibitem{Tran+Hartmann2009}
H.~Tran and J.-M. Hartmann, ``{An isolated line-shape model based on the
  Keilson and Storer function for velocity changes. I. Theoretical
  approaches},'' \emph{J. Chem. Phys.}, vol. 130, p. 094301, 2009.

\bibitem{Ngo+etal2012}
N.~H. Ngo, H.~Tran, and R.~R. Gamache, ``{A pure $\mrm{H}_2\mrm{O}$ isolated
  line-shape model based on classical molecular dynamics simulations of
  velocity changes and semi-classical calculations of speed-dependent
  collisional parameters},'' \emph{J. Chem. Phys.}, vol. 136, p. 154310, 2012.

\bibitem{Ngo+etal2013}
N.~Ngo, D.~Lisak, H.~Tran, and J.-M. Hartmann, ``{An isolated line-shape model
  to go beyond the Voigt profile in spectroscopic databases and radiative
  transfer codes},'' \emph{Journal of Quantitative Spectroscopy \& Radiative
  Transfer}, vol. 129, pp. 89--100, 2013.

\bibitem{Tennyson+etal2014b}
J.~Tennyson, P.~F. Bernath, A.~Campargue, A.~G. Cs\'{a}sz\'{a}r, L.~Daumont,
  R.~R. Gamache, J.~T. Hodges, D.~Lisak, O.~V. Naumenko, L.~S. Rothman,
  H.~Tran, N.~F. Zobov, J.~Buldyreva, C.~D. Boone, M.~D.~D. Vizia,
  L.~Gianfrani, J.-M. Hartmann, R.~McPheat, D.~Weidmann, J.~Murray, N.~H. Ngo,
  and O.~L. Polyansky, ``{Recommended isolated-line profile for representing
  high-resolution spectroscopic transitions (IUPAC Technical Report)},''
  \emph{Pure Appl. Chem.}, vol.~86, pp. 1931--1943, 2014.

\bibitem{Smith+etal1971a}
E.~W. Smith, J.~Cooper, W.~R. Chappell, and T.~Dillon, ``{An impact theory for
  Doppler and pressure broadening. I. General theory},'' \emph{Journal of
  Quantitative Spectroscopy \& Radiative Transfer}, vol.~11, pp. 1547--1565,
  1971.

\bibitem{Smith+etal1971b}
------, ``{An impact theory for Doppler and pressure broadening. II. Atomic and
  molecular systems},'' \emph{Journal of Quantitative Spectroscopy \& Radiative
  Transfer}, vol.~11, pp. 1567--1576, 1971.

\bibitem{Ciurylo+Pine2000}
R.~Ciury{\l}o and A.~S. Pine, ``Speed-dependent line mixing profiles,''
  \emph{Journal of Quantitative Spectroscopy \& Radiative Transfer}, vol.~67,
  pp. 375--393, 2000.

\bibitem{Kolb+Griem1958}
A.~C. Kolb and H.~Griem, ``Theory of line broadening in multiplet spectra,''
  \emph{Physical Review}, vol. 111, no.~2, pp. 514--521, 1958.

\bibitem{Baranger1958}
M.~Baranger, ``General impact theory of pressure broadening,'' \emph{Physical
  Review}, vol. 112, no.~3, pp. 855--865, 1958.

\bibitem{Gordon1966}
R.~G. Gordon, ``Semiclassical theory of spectra and relaxation in molecular
  gases,'' \emph{J. Chem. Phys.}, vol.~45, no.~5, pp. 1649--1655, 1966.

\bibitem{Gordon1967}
------, ``On the pressure broadening of molecular multiplet spectra,'' \emph{J.
  Chem. Phys.}, vol.~46, no.~2, pp. 448--455, 1967.

\bibitem{Rosenkranz1975}
P.~W. Rosenkranz, ``Shape of the 5 mm oxygen band in the atmosphere,''
  \emph{IEEE Trans. Antennas Propagat.}, vol.~23, no.~4, pp. 498--506, 1975.

\bibitem{Gordon+etal2022}
I.~E. Gordon and et~al, ``The {HITRAN2020} molecular spectroscopic database,''
  \emph{Journal of Quantitative Spectroscopy \& Radiative Transfer}, vol. 277,
  pp. 1--82, 2022.

\bibitem{Pine+Gabard2000}
A.~S. Pine and T.~Gabard, ``{Speed-dependent broadening and line mixing in
  $\mrm{CH}_4$ perturbed by $\mrm{Ar}$ and $\mrm{N}_2$ from multispectrum
  fits},'' \emph{Journal of Quantitative Spectroscopy \& Radiative Transfer},
  vol.~66, pp. 69--92, 2000.

\bibitem{Gordon1966a}
R.~G. Gordon, ``Theory of the width and shift of molecular spectral lines in
  gases,'' \emph{J. Chem. Phys.}, vol.~44, no.~8, pp. 3083--3089, 1966.

\bibitem{Robert+Galatry1971}
D.~Robert and L.~Galatry, ``Infrared absorption of diatomic polar molecules in
  liquid solutions,'' \emph{J. Chem. Phys.}, vol.~55, no.~5, pp. 2347--2359,
  1971.

\bibitem{Horn+Johnson1985}
R.~A. Horn and C.~R. Johnson, \emph{Matrix Analysis}.\hskip 1em plus 0.5em
  minus 0.4em\relax Cambridge University Press, 1985.

\bibitem{Mendel1987}
J.~M. Mendel, \emph{Lessons in digital estimation theory}.\hskip 1em plus 0.5em
  minus 0.4em\relax Englewood Cliffs, New Jersey: Prentice-Hall, Inc., 1987.

\bibitem{Armstrong1967}
B.~H. Armstrong, ``{Spectrum line profiles: The Voigt function},''
  \emph{Journal of Quantitative Spectroscopy \& Radiative Transfer}, vol.~7,
  pp. 61--88, 1967.

\bibitem{Schreier1992}
F.~Schreier, ``{The Voigt and complex error function: A comparison of
  computational methods},'' \emph{Journal of Quantitative Spectroscopy \&
  Radiative Transfer}, vol.~48, no. 5/6, pp. 743--762, 1992.

\bibitem{Schreier2011}
------, ``{Optimized implementations of rational approximations for the Voigt
  and complex error function},'' \emph{Journal of Quantitative Spectroscopy \&
  Radiative Transfer}, vol. 112, pp. 1010--1025, 2011.

\bibitem{Abrarov+Quine2011}
S.~M. Abrarov and B.~M. Quine, ``{Efficient algorithmic implementation of the
  Voigt/complex error function based on exponential series approximation},''
  \emph{Appl. Math. Comput.}, vol. 218, pp. 1894--1902, 2011.

\bibitem{Reed+Simon1975}
M.~Reed and B.~Simon, \emph{Methods of modern mathematical physics}.\hskip 1em
  plus 0.5em minus 0.4em\relax New York: Academic Press, 1975, vol. II: Fourier
  analysis, Self-adjointness.

\bibitem{Boyd+etal1994}
S.~Boyd, L.~Ghaoui, E.~Feron, and V.~Balakrishnan, \emph{Linear Matrix
  Inequalities in System and Control Theory}.\hskip 1em plus 0.5em minus
  0.4em\relax Philadelphia, PA: Society for Industrial and Applied Mathematics,
  1994.

\bibitem{Abramowitz+Stegun1970}
M.~Abramowitz and I.~A. Stegun, Eds., \emph{Handbook of Mathematical
  Functions}, ser. Applied Mathematics Series No. 55.\hskip 1em plus 0.5em
  minus 0.4em\relax Washington D.C.: National Bureau of Standards, 1970.

\bibitem{Nedic+etal2019}
M.~Nedic, C.~Ehrenborg, Y.~Ivanenko, A.~Ludvig-Osipov, S.~Nordebo, A.~Luger,
  B.~L.~G. Jonsson, D.~Sj\"{o}berg, and M.~Gustafsson, \emph{Herglotz functions
  and applications in electromagnetics}.\hskip 1em plus 0.5em minus 0.4em\relax
  IET, 2019, editor: K. Kobayashi and P. Smith.

\bibitem{Olver+etal2010}
F.~W.~J. Olver, D.~W. Lozier, R.~F. Boisvert, and C.~W. Clark, \emph{{NIST}
  {H}andbook of mathematical functions}.\hskip 1em plus 0.5em minus 0.4em\relax
  New York: Cambridge University Press, 2010.

\bibitem{Blundell+Blundell2010}
S.~J. Blundell and K.~M. Blundell, \emph{Concepts in thermal physics}.\hskip
  1em plus 0.5em minus 0.4em\relax New York: Oxford University Press Inc.,
  2010.

\end{thebibliography}

\end{document}


