% !TEX root = thesis.tex

\chapter{Dagger linearly distributive categories}
\label{Chap: dagger-LDC}

In this chapter we define dagger linearly distributive categories, dagger linear functors and 
transformations, and provide examples. We also explore the relationship among 
dual, dagger and conjugation functors for LDCs. 

%%%%%%%%%%%%%%%%%%%%%%%%%%%%%%%%%%%%%%%%%%%%%%%%%%
\section{Dagger for LDCs}
\label{Section: dagger LDC}

Conventionally, in categorical quantum mechanics a dagger is defined as a contravariant endofunctor which is 
stationary on objects $(A^\dagger = A)$ and an involution ($f^{\dag \dag} = f$). However, 
in an LDC, the dagger must minimally flip the tensor products to maintain the directionality of the distributor maps. 
Recall that, $\partial^\dagger: (A \ox B) \oa C \to A \ox (B \oa C)$ is not a valid map in an LDC.  Hence, 
for LDCs we cannot expect the dagger to be stationary on objects, 
however, it is still possible for it to be an involution. This section deals with the coherences 
of $\dagger$-functor for LDC and its variants, that is, mix and isomix categories. 

\subsection{Dagger linearly distributive categories}

Before proceeding to define the dagger functor for LDCs, the notion of the opposite 
LDC and the notion of a contravariant linear functors have to be developed.  

If $(\X, \ox, \top, \oa, \bot)$ is a linearly distributive category, the {\bf opposite linear distributive category} 
is $(\X, \ox, \top, \oa, \bot)^{\op} := (\X^{\op}, \oa, \bot, \ox, \top)$ where $\X^{\op}$ is the usual opposite 
category with the monoidal structures flipped as follows: 
\[\ox^{\op} := \oa ~~~~~~~ \top^{\op} := \bot ~~~~~~~ \oa^{\op} := \ox ~~~~~~~ \bot^{\op} := \top\]
$(\_)^\op$ is an endo functor for the category of LDCs and linear functors. It is also an involution:  

$(\X, \ox, \top, \oa, \bot)^{{\op}~{\op}} = (\X, \ox, \top, \oa, \bot) $. 

Let $(F_\ox, F_\oa): (\X, \ox, \top, \oa, \bot)^{\op} \to (\X, \ox, \top, \oa, \bot)$ be a linear functor. The opposite linear functor  $(F_\ox, F_\oa)^{\op}: (\X, \ox, \top, \oa, \bot) \to  (\X, \ox, \top, \oa, \bot)^{\op}$ given by the pair of opposite functors $(F_\oa^{\op}, F_\ox^{\op})$. Observe that $F^{\op}$ is  a mix Frobenius linear functor if and only if $F$ is.

\medskip

\begin{definition}
\label{Definition: daggerLDC succinct}
A {\bf dagger linearly distributive category} ($\dagger$-LDC), is an LDC, $\X$, with a contravariant Frobenius linear functor $(\_)^\dagger: \X^{\op} \to \X$ which is a linear involutive equivalence   $(\_)^\dagger ~\dashvv ~ (\_)^{\dagger^{\op}}: \X^{\op} \to \X$.
\end{definition}

We unfold this definition in Proposition \ref{Definition: daggerLDC elaborate}. However, first note that saying the dagger is an {\bf involutive} equivalence asserts that the unit and 
counit of the equivalence are the same (although one is in the opposite category).  
Thus, the adjunction expands to take the form $(\imath,\imath): (\_)^\dagger ~\dashvv ~ (\_)^{\dagger^{\op}}: \X^{\op} \to \X$.  
However, the unit and counit are linear natural transformations so $\imath$ expands to $\imath = (\imath_\ox,\imath_\oa)$.  
As the dagger functor is a left adjoint, it is strong and, thus, is normal.  
Furthermore, as the unit of an equivalence, $\imath$ is a linear natural isomorphism,   
this means  $\imath = (\imath_\ox,\imath_\oa)$ satisfies the requirements of 
Lemma \ref{Lemma: Frobenius linear transformation}, implying that $\imath_\ox^{-1}  = \imath_\oa$.  
Simplifying notation we shall set $\iota:= \imath_\oa$  so the unit linear transformation is 
$\imath := (\iota^{-1},\iota)$. We then can simplify the requirements of $\imath$ to the map  
$\iota: A \to (A^\dagger)^\dagger$ which we refer to as the {\bf involutor}.

A {\bf symmetric  $\dagger$-LDC} is a $\dagger$-LDC which is a symmetric LDC 
for which the dagger is a symmetric linear functor. A {\bf cyclic $\dagger$-$*$-autonomous category} is a 
$\dagger$-LDC with chosen left and right duals, and a cyclor which is preserved 
by the dagger.  A $\dagger$-{\bf mix} category is a $\dagger$-LDC for which $(\_)^\dagger: \X^{\op} \to \X$ 
is a mix functor.  As the dagger functor is strong (and so normal) if the category is an isomix category then being {\bf $\dagger$-mix} already implies that 
the dagger is an isomix functor.  Thus, a {\bf $\dagger$-\bf isomix} category is a $\dagger$-mix category which happens to be an isomix category.

In the remainder of the section, we unfold the definition of a $\dagger$-isomix category and give the coherence requirements explicitly.  


\begin{proposition}
\label{Definition: daggerLDC elaborate}
A dagger linearly distributive category is an LDC with a functor $(\_)^\dag:\X^\op\to \X$ and 
natural isomorphisms 
\begin{align*}
\text{ \bf laxors: }  A^\dag \ox B^\dag &\xrightarrow{ \lambda_\ox} (A\oa B)^\dag ~~~~~ A^\dag \oa B^\dag \xrightarrow{ \lambda_\oa} (A\ox B)^\dag \\
\top &\xrightarrow{\lambda_\top} \bot^\dag ~~~~~~~~~~~~~~~~~~~~~ \bot \xrightarrow{\lambda_\bot} \top^\dag \\
\text{ \bf involutor: }  A &\xrightarrow{\iota} (A^\dag)^\dag 
\end{align*}
such that the following coherences hold:
\begin{enumerate}[{\bf [$\dagger$-ldc.1]}]
\item Interaction of $\lambda_\ox, \lambda_\oa$  with associators:
\[
\begin{tabular}{cc}
\xymatrix{
A^\dag \ox (B^\dag \ox C^\dag)                \ar@{->}[r]^{{a_\ox}^{-1}}    \ar@{->}[d]_{1 \ox  \lambda_\ox}   
  & (A^\dag \ox B^\dag) \ox C^\dag              \ar@{->}[d]^{\lambda_\ox \ox 1}  \\
A^\dag \ox ( B \oa C)^\dag                     \ar@{->}[d]_{\lambda_\ox}  
  & (A \oa B)^\dag \ox C^\dag                   \ar@{->}[d]^{\lambda_\ox}    \\
(A \oa (B \oa C))^\dag                          \ar@{->}[r]_{a_\oa^\dag}
  & ( (A\oa B) \oa C)^\dag
} &\xymatrix{
A^\dag \oa (B^\dag \oa C^\dag)                \ar@{->}[r]^{a_\oa^{-1}}    \ar@{->}[d]_{1 \oa \lambda_\oa }  
  & (A^\dag \oa B^\dag) \oa C^\dag              \ar@{->}[d]^{\lambda_\oa \oa 1}  \\
   A^\dag \oa (B \ox C)^\dag                 \ar@{->}[d]_{\lambda_\oa}  
  & (A \ox B)^\dag \oa C^\dag                    \ar@{->}[d]^{\lambda_\oa}    \\
(A \ox (B\ox C))^\dag                          \ar@{->}[r]_{a_\ox^\dag}
  & ((A\ox B) \ox C)^\dag
}
\end{tabular}
\]

\item Interaction of $\lambda_\top, \lambda_\bot$ with unitors: 
\begin{center}
\begin{tabular}{cc}
\xymatrix{
\top \ox A^\dag                            \ar@{->}[rr]^{\lambda_\top\ox 1} \ar@{->}[d]_{u_\ox^R}  
&  & \bot^\dag\ox A^\dag           \ar@{->}[d]^{\lambda_\ox}\\
A^\dag                                     
&  & (\bot \oa A)^\dag                      \ar@{<-}[ll]^{(u_\oa^R)^\dag}\\
} &
\xymatrix{
\bot \oa A^\dag                            \ar@{->}[rr]^{\lambda_\bot\oa 1} \ar@{->}[d]_{u_\oa^R} 
& & \top^\dag\oa A^\dag                    \ar@{->}[d]^{\lambda_\oa}\\
A^\dag                                     
&  & (\top \ox A)^\dag                      \ar@{<-}[ll]^{(u_\ox^R)^\dag}\\
} 
\end{tabular}
\end{center}
and two symmetric diagrams for $u_\ox^L$ and $u_\oa^L$ must also be satisfied.

\item Interaction of $\lambda_\ox, \lambda_\oa$ with linear distributors:
\begin{center}
\begin{tabular}{cc}
\xymatrix{
A^\dag \ox(B^\dag\oa C^\dag)              \ar@{->}[r]^{\partial^L} \ar@{->}[d]_{1\ox\lambda_\oa}
  & (A^\dag \ox B^\dag)\oa C^\dag         \ar@{->}[d]_{\lambda_\ox\oa 1}\\
A^\dag \ox (B\ox C)^\dag                  \ar@{->}[d]_{\lambda_\ox}
  & (A\oa B)^\dag \oa C^\dag              \ar@{->}[d]^{\lambda_\oa}\\
(A\oa (B\ox C))^\dag                      \ar@{->}[r]_{(\partial^R)^\dag}
  & ((A\oa B)\ox C)^\dag
} &
\xymatrix{
(A^\dag \oa B^\dag) \ox C^\dag     \ar@{->}[r]^{\partial^R} \ar@{->}[d]_{\lambda_\oa\ox 1} 
  & A^\dag \oa (B^\dag \ox C^\dag) \ar@{->}[d]_{1\oa \lambda_\ox}\\
(A\ox B)^\dag \ox C^\dag           \ar@{->}[d]_{\lambda_\ox}
  & A^\dag \oa (B\oa C)^\dag       \ar@{->}[d]^{\lambda_\oa}\\
((A\ox B)\oa C)^\dag               \ar@{->}[r]_{(\partial^L)^\dag}
  & (A\ox (B\oa C))^\dag
} 
\end{tabular}
\end{center}

\item Interaction of $\iota: A \rightarrow A^{\dagger\dagger}$ with $\lambda_\ox$, $\lambda_\oa$:
\begin{center}
\begin{tabular}{cc}
\xymatrix{
A\oa B                          \ar@{->}[r]^{\iota} \ar@{->}[d]_{\iota \oa \iota} 
  & ((A\oa B)^\dag)^\dag        \ar@{->}[d]^{\lambda_\ox^\dag}\\
(A^\dag)^\dag   \oa (B^\dag)^\dag                \ar@{->}[r]_{\lambda_\oa}
  & (A^\dag\ox B^\dag)^\dag
} &  \xymatrix{
A\ox B                          \ar@{->}[r]^{\iota} \ar@{->}[d]_{\iota \ox \iota} 
  & ((A\ox B)^\dag)^\dag        \ar@{->}[d]^{\lambda_\oa^\dag}\\ 
(A^\dag)^\dag  \ox (B^\dag)^\dag                \ar@{->}[r]_{\lambda_\ox}
  & (A^\dag\oa B^\dag)^\dag
}
\end{tabular}
\end{center}
\item Interaction of $\iota: A \rightarrow A^{\dagger\dagger}$ with $\lambda_\top$, $\lambda_\bot$:
\begin{center}
\begin{tabular}{cc}
$\begin{matrix} \xymatrix{
&\bot                   \ar@{->}[r]^{\iota} \ar@{->}[dr]_{\lambda_\bot}  
  & (\bot^\dag)^\dag   \ar@{->}[d]^{\lambda_\top^\dag}\\
&{}
  & \top^\dag 
} \end{matrix}$ & 
 $\begin{matrix} \xymatrix{
&\top                   \ar@{->}[r]^{\iota} \ar@{->}[dr]_{\lambda_\top} 
  & (\top^\dag)^\dag   \ar@{->}[d]^{\lambda_\bot^\dag} \\
&{}
  & \bot^\dag 
} \end{matrix}$
\end{tabular}
\end{center}
\item $\iota_{A^\dagger} = (\iota_A^{-1})^\dagger: A^\dagger \to A^{\dagger\dagger\dagger}$  
\end{enumerate}
\end{proposition}

The dagger structure is obtained from the previous proposition using strong monoidal laxors:  to form a linear functor the laxor $\lambda_\oa$ 
needs to be reversed by taking its inverse. Then, we have $\nu_\ox^l = \nu_\ox^r := \lambda_\oa^{-1}$ and $\nu_\oa^l = \nu_\oa^r := \lambda_\ox$.  
Once this adjustment is made all the required coherences for $\dagger$ to be a linear functor are present.
Note that {\bf [$\dagger$-ldc.6]} equivalently expresses the triangle identities of the 
adjunction $(\iota, \iota):: \dagger^{\op} \dashv \dagger : \X^{\op} \to \X$.   
The coherences for the involutor asserts that it is a monoidal transformation for both the tensor and par: 
by Lemma \ref{simplifying-coherences} (ii) this suffices to show that it is a linear transformation.

A {\bf symmetric $\dagger$-LDC} is a $\dagger$-LDC which is a symmetric LDC and for which the following additional diagrams commute:
\begin{enumerate}[{\bf [$\dagger$-ldc.7]}]
\item Interaction of $\lambda_\ox , \lambda_\oa$ with symmetry maps: 
\[
\begin{tabular}{cc}
\xymatrix{
A^\dag \ox B^\dag                          \ar@{->}[r]^{\lambda_\ox} \ar@{->}[d]_{c_\ox}         
  & (A\oa B)^\dag                          \ar@{->}[d]^{c_\oa^\dag}\\
B^\dag \ox A^\dag                          \ar@{->}[r]_{\lambda_\ox}
  & (B\oa A)^\dag\\
} & \xymatrix{
A^\dag \oa  B^\dag                          \ar@{->}[r]^{\lambda_\oa} \ar@{->}[d]_{c_\oa}  
  & (A\ox B)^\dag                          \ar@{->}[d]^{c_\ox^\dag}\\
B^\dag \oa A^\dag                          \ar@{->}[r]_{\lambda_\oa}
  & (B\ox A)^\dag\\
}
\end{tabular}
\]
\end{enumerate}

A {\bf $\dagger$-mix category} is a $\dagger$-LDC which has a mix map and satisfies the following additional coherence:

\[ \mbox{\bf [$\dagger$-\text{mix}]}  ~~~~\begin{array}[c]{c} 
\xymatrix{
\bot                 \ar@{->}[r]^{{\sf m}} \ar@{->}[d]_{\lambda_\bot}  
  & \top             \ar@{->}[d]^{\lambda_\top}\\
\top^\dag            \ar@{->}[r]_{{\sf m}^\dag}
  & \bot^\dag
} 
\end{array} \]
If ${\sf m}$ is an isomorphism, then $\X$ is a {\bf $\dagger$-isomix category} and, 
since $(\_)^\dagger$ is normal, $(\_)^\dagger$ is an isomix Frobenius functor.

\begin{lemma}
\label{lemma: mixdagger}
Suppose $\X$ is a $\dagger$-mix category then the following diagram commutes:
\[
\xymatrix{ A^\dag \ox B^\dag  \ar[r]^{\mx} \ar[d]_{\lambda_\ox}&  A^\dag \oa B^\dag \ar[d]^{\lambda_\oa} \\
                (A \oa B)^\dag \ar[r]_{\mx^\dag}  & (A \ox B)^\dag }
\]
\end{lemma}
\begin{proof}
The proof follows directly from Lemma \ref{Lemma: Mix Frobenius linear functor}.
\end{proof}

 With respect to its applications to quantum theory, this thesis primarily focuses on $\dagger$-isomix categories. 
 As we will see in Section \ref{Sec: unitary}, the notion of unitary objects and unitary isomorphisms 
 is supported only within a $\dagger$-isomix category.

It is useful to observe that the core of a mix category is closed under taking the dagger and duals.

\begin{lemma}
\label{Lemma: mixdagger}
Suppose $\X$ is a $\dagger$-mix category and $A \in \Core(\X)$ then $A^\dagger \in$ $\Core(\X)$.
\end{lemma}
\begin{proof}
The natural transformation $A^\dagger \ox X \xrightarrow{\mx} A^\dagger \oa X$ is an isomorphism as follows:
\[
\xymatrix{
A^\dagger \ox X \ar[r]^{1 \ox \iota} \ar[d]_{\mx} \ar@{}[dr]|{\scalebox{0.95}{\tiny\bf (nat. {\sf mx})}}
& A^\dagger \ox X^{\dagger\dagger} \ar[r]^{\lambda_\ox} \ar[d]_{\mx} \ar@{}[dr]|{\scalebox{.95}{\tiny\bf {lem. \ref{lemma: mixdagger}}}}
& (A \oa X^\dagger)^\dagger \ar[d]^{\mx^\dagger} \\
A^\dagger \oa X \ar[r]_{1 \oa \iota} 
& A^\dagger \oa X^{\dagger \dagger} \ar[r]_{\lambda_\oa}
&(A \ox X^\dagger)^\dagger
}
\]

\end{proof}


\begin{lemma}
Let $\X$ be $\dagger$-LDC. If $A \dashvv B$ then $B^\dagger \dashvv A^\dagger$.
\end{lemma}
\begin{proof}
The statement  follows from Lemma \ref{Lemma: linear adjoints}: Frobenius functors preserve linear adjoints. 
Explicitly, if $(\eta,\epsilon): A \dashvv B$ then $(\lambda_\top\epsilon^\dag\lambda_\oa^{-1},\lambda_\ox\eta^\dagger \lambda_\bot^{-1}): B^\dagger \dashvv A^\dagger$. 
\end{proof}

Suppose $\X$ is a $\dagger$-$*$-autonomous category and $(\eta*, \epsilon*): A^* \dashvv A$, then $((\epsilon*)^\dagger, (\eta*)^\dagger): A^\dagger \dashvv (A^*)^\dagger$, where $((\epsilon*)^\dagger, (\eta*)^\dagger) :=   (\lambda_\top\epsilon*^\dag\lambda_\oa^{-1},\lambda_\ox\eta*^\dagger \lambda_\bot^{-1})$. We draw $(\epsilon*)^\dagger$ and $(*\epsilon)^\dagger$ as dagger cups, and $(\eta*)^\dagger$ and $(*\eta)^\dagger$ as dagger caps which are pictorially represented as follows:
 \[
 \begin{tikzpicture}
	\begin{pgfonlayer}{nodelayer}
		\node [style=none] (0) at (-1, 2) {};
		\node [style=none] (1) at (1, 2) {};
		\node [style=none] (2) at (-1, 0.5) {};
		\node [style=none] (3) at (1, 0.5) {};
		\node [style=none] (4) at (-1.5, 1.5) {$X^\dagger$};
		\node [style=none] (5) at (1.5, 1.5) {$(^*X)^\dagger$};
		\node [style=none] (6) at (0, -1) {$(*\eta)^\dagger$};
	\end{pgfonlayer}
	\begin{pgfonlayer}{edgelayer}
		\draw (2.center) to (0.center);
		\draw [bend right=90, looseness=2.00] (2.center) to (3.center);
		\draw (3.center) to (1.center);
	\end{pgfonlayer}
\end{tikzpicture} ~~~~~~~~~ \begin{tikzpicture}
	\begin{pgfonlayer}{nodelayer}
		\node [style=none] (0) at (-1, 2) {};
		\node [style=none] (1) at (1, 2) {};
		\node [style=none] (2) at (-1, 0.5) {};
		\node [style=none] (3) at (1, 0.5) {};
		\node [style=none] (4) at (-1.5, 1.5) {$X^{*\dagger}$};
		\node [style=none] (5) at (1.5, 1.5) {$X^\dagger$};
		\node [style=none] (6) at (0, -1) {$(\eta*)^\dagger$};
	\end{pgfonlayer}
	\begin{pgfonlayer}{edgelayer}
		\draw (2.center) to (0.center);
		\draw [bend right=90, looseness=2.00] (2.center) to (3.center);
		\draw (3.center) to (1.center);
	\end{pgfonlayer}
\end{tikzpicture} ~~~~~~~~~ \begin{tikzpicture}
	\begin{pgfonlayer}{nodelayer}
		\node [style=none] (0) at (-1, -1) {};
		\node [style=none] (1) at (1, -1) {};
		\node [style=none] (2) at (-1, 0.5) {};
		\node [style=none] (3) at (1, 0.5) {};
		\node [style=none] (4) at (-1.5, -0.5) {$X^\dagger$};
		\node [style=none] (5) at (1.5, -0.5) {$(^*X)^\dagger$};
		\node [style=none] (6) at (0, 2) {$(*\epsilon)^\dagger$ };
	\end{pgfonlayer}
	\begin{pgfonlayer}{edgelayer}
		\draw (2.center) to (0.center);
		\draw [bend left=90, looseness=2.00] (2.center) to (3.center);
		\draw (3.center) to (1.center);
	\end{pgfonlayer}
\end{tikzpicture} ~~~~~~~~~ \begin{tikzpicture}
	\begin{pgfonlayer}{nodelayer}
		\node [style=none] (0) at (-1, -1) {};
		\node [style=none] (1) at (1, -1) {};
		\node [style=none] (2) at (-1, 0.5) {};
		\node [style=none] (3) at (1, 0.5) {};
		\node [style=none] (4) at (-1.5, -0.5) {$X^{*^\dagger}$};
		\node [style=none] (5) at (1.5, -0.5) {$X^\dagger$};
		\node [style=none] (6) at (0, 2) {$(\epsilon*)^\dagger$};
	\end{pgfonlayer}
	\begin{pgfonlayer}{edgelayer}
		\draw (2.center) to (0.center);
		\draw [bend left=90, looseness=2.00] (2.center) to (3.center);
		\draw (3.center) to (1.center);
	\end{pgfonlayer}
\end{tikzpicture} 
 \]

A $\dagger$-$*$-autonomous category is a {\bf cyclic} $\dagger$-$*$-autonomous category when the dagger preserves the cyclor in the following sense. 
\[ %last1
\begin{tikzpicture}
	\begin{pgfonlayer}{nodelayer}
		\node [style=none] (0) at (5.25, 2) {};
		\node [style=none] (1) at (5.25, -1) {};
		\node [style=none] (2) at (4, -0.9999999) {};
		\node [style=none] (3) at (4, 0.9999999) {};
		\node [style=none] (4) at (2.75, 0.9999999) {};
		\node [style=circle] (5) at (2.75, -0.5) {$\psi^\dagger$};
		\node [style=circle, scale=2.5] (6) at (2.75, -1.75) {};
		\node [style=none] (13) at (2.75, -1.75) {$\psi^{-1 \dagger}$};
		\node [style=none] (7) at (2.75, -2.75) {};
		\node [style=none] (8) at (5.75, 1.5) {$(A^\dagger)^*$};
		\node [style=none] (9) at (4.75, -1.75) {$\epsilon*$};
		\node [style=none] (10) at (3.25, 1.75) {$(*\epsilon)^\dagger$};
		\node [style=none] (11) at (2.25, -2.5) {$(^*A)^\dagger$};
		\node [style=none] (12) at (4.25, -0) {$A^\dagger$};
	\end{pgfonlayer}
	\begin{pgfonlayer}{edgelayer}
		\draw (0.center) to (1.center);
		\draw [bend left=90, looseness=1.25] (1.center) to (2.center);
		\draw (2.center) to (3.center);
		\draw [bend right=90, looseness=1.25] (3.center) to (4.center);
		\draw (4.center) to (5);
		\draw (5) to (6);
		\draw (6) to (7.center);
	\end{pgfonlayer}
\end{tikzpicture} = %last2
\begin{tikzpicture}
	\begin{pgfonlayer}{nodelayer}
		\node [style=none] (0) at (5.5, -0.9999999) {};
		\node [style=none] (1) at (4, -1) {};
		\node [style=none] (2) at (4, -3) {};
		\node [style=none] (3) at (2.5, -3) {};
		\node [style=circle, scale=1.3] (4) at (2.5, -0.9999999) {$\psi$};
		\node [style=circle, scale=2.5] (5) at (5.5, -4) {};
		\node [style=none] (13) at (5.5, -4) {$\psi^{-1 \dagger}$};
		\node [style=none] (6) at (2.5, 0.9999999) {};
		\node [style=none] (7) at (1.25, -0.25) {$(A^\dagger)^*$};
		\node [style=none] (8) at (4.75, -0.25) {$(\epsilon*)^\dagger$};
		\node [style=none] (9) at (3.25, -3.75) {$*\epsilon$};
		\node [style=none] (10) at (6.999999, -5) {$(^*A)^\dagger$};
		\node [style=none] (11) at (4.5, -2) {$A^\dagger$};
		\node [style=none] (12) at (5.5, -5.25) {};
	\end{pgfonlayer}
	\begin{pgfonlayer}{edgelayer}
		\draw [bend right=90, looseness=1.25] (0.center) to (1.center);
		\draw (1.center) to (2.center);
		\draw [bend left=90, looseness=1.25] (2.center) to (3.center);
		\draw (3.center) to (4);
		\draw (0.center) to (5);
		\draw (5) to (12.center);
		\draw (4) to (6.center);
	\end{pgfonlayer}
\end{tikzpicture} \]

\begin{lemma} 
\label{Lemma: cyclic dagger}
In a cyclic, $\dagger$-$*$-autonomous category,
\[\begin{tikzpicture}
	\begin{pgfonlayer}{nodelayer}
		\node [style=none] (0) at (6, 1) {};
		\node [style=none] (1) at (4, 1) {};
		\node [style=none] (2) at (4, -1.25) {};
		\node [style=none] (3) at (2, -1.25) {};
		\node [style=circle, scale=1.25] (4) at (2, -0.5) {$\psi$};
		\node [style=none] (5) at (2, 2.25) {};
		\node [style=none] (6) at (5, 2) {$(*\epsilon)^\dagger$};
		\node [style=none] (7) at (3, -2.25) {$*\epsilon$};
		\node [style=none] (8) at (6, -2.5) {};
		\node [style=circle] (9) at (4, -0.25) {$\psi^\dagger$};
		\node [style=none] (10) at (1.25, 2) {$A^{*\dagger*}$};
		\node [style=none] (11) at (6.5, -2.25) {$A^\dagger$};
		\node [style=none] (12) at (1.25, -1.25) {$~^*(A^{\dagger*})$};
		\node [style=none] (13) at (4.5, -1) {$A^{\dagger*}$};
		\node [style=none] (14) at (4.5, 0.5) {$~^*(A^\dagger)$};
	\end{pgfonlayer}
	\begin{pgfonlayer}{edgelayer}
		\draw [bend right=90, looseness=1.25] (0.center) to (1.center);
		\draw [bend left=90, looseness=1.25] (2.center) to (3.center);
		\draw (3.center) to (4);
		\draw (4) to (5.center);
		\draw (0.center) to (8.center);
		\draw (1.center) to (9);
		\draw (9) to (2.center);
	\end{pgfonlayer}
\end{tikzpicture} =\begin{tikzpicture}
	\begin{pgfonlayer}{nodelayer}
		\node [style=none] (15) at (13, -0.75) {};
		\node [style=none] (16) at (11, -0.75) {};
		\node [style=none] (17) at (11, 0.25) {};
		\node [style=none] (18) at (9, 0.25) {};
		\node [style=none] (19) at (10, 1.25) {$(\epsilon*)^\dagger$};
		\node [style=none] (20) at (12, -1.75) {$\epsilon*$};
		\node [style=none] (21) at (13, 2) {};
		\node [style=none] (22) at (13.75, 1.5) {$A^{*\dagger*}$};
		\node [style=none] (23) at (8.25, -1) {$A^\dagger$};
		\node [style=none] (24) at (9, -2.5) {};
		\node [style=none] (25) at (11.5, -0.5) {$A^{*\dagger}$};
	\end{pgfonlayer}
	\begin{pgfonlayer}{edgelayer}
		\draw [bend left=90, looseness=1.25] (15.center) to (16.center);
		\draw [bend right=90, looseness=1.25] (17.center) to (18.center);
		\draw (15.center) to (21.center);
		\draw (17.center) to (16.center);
		\draw (18.center) to (24.center);
	\end{pgfonlayer}
\end{tikzpicture} \]
\end{lemma}
\begin{proof}
Proved by direct application of Lemma \ref{Lemma: cyclic Frob}.
\end{proof}

\subsection{Sequent calculus for $\dagger$-linear logic}
\label{Sec: dagger sequent rules}

$\dagger$-LDCs provide a categorical semantics for the proof theory for multiplicative linear logic with the dagger ($\dagger$-MLL).
Along with the sequent rules of MLL, $\dagger$-MLL includes the additional rules $(\dagger)$, and ($\iota$), as shown in 
Figure \ref{Fig: dagger sequent}.
\begin{figure}[h]
	\centering
	\AxiomC{$\Gamma \vdash \Delta$}
	\LeftLabel{($\dagger$)}
	\UnaryInfC{$\Delta^\dag \vdash \Gamma^\dag$}
	\DisplayProof
	\hspace{1.5em}
	\AxiomC{$\Gamma_1,A, \Gamma_2 \vdash \Delta$}
	\LeftLabel{($\iota$L)}
	\UnaryInfC{$\Gamma_1, A^{\dag \dag}, \Gamma_2 \vdash \Delta$}
	\DisplayProof 
	\hspace{1.5em}
	\AxiomC{$\Gamma \vdash \Delta_1, A, \Delta_2$}
	\LeftLabel{($\iota$R)}
	\UnaryInfC{$\Gamma \vdash \Delta_1, A^{\dag \dag}, \Delta_2$}
	\DisplayProof	

  	\vspace{1em}

	where, if $\Gamma = A_1, A_2, A_3, \cdots, A_n$, then $\Gamma^\dagger = A_1^\dag, A_2^\dag, A_3^\dag, \cdots, A_n^\dag$
	\caption{Sequent rules for $\dagger$}
	\label{Fig: dagger sequent}
\end{figure}

The rule $(\dagger)$ corresponds to the contravariance of the $\dagger$-functor, and the rules ($\iota$L) and ($\iota$R) correspond 
to the involutor natural isomorphism, $\iota: A \to A^{\dagger \dagger}$. The derivation of $(\iota L)^{(-1)}$ and $(\iota R)^{(-1)}$ are 
shown in Figure \ref{Fig: iota inv}.

\begin{figure}[h]
    \centering
	\AxiomC{$A \vdash A$}
	\RightLabel{$\iota R$}
	\UnaryInfC{$A \vdash A^{\dag \dag}$} 
    \DisplayProof
	\hspace{1.5em}
	\AxiomC{$A \vdash A$}
	\RightLabel{$\iota L$}
	\UnaryInfC{$A^{\dag \dag} \vdash A$} 
    \DisplayProof
	\caption{Sequent proof:- $\iota$ is an isomorphism}
	\label{Fig: iota inv}
\end{figure}

One can derive the unit and the tensor laxors using the sequent rules, ($\dagger$), ($\iota$L), and ($\iota$R).
For example, Figure \ref{Fig: tensor laxor rule} shows the derivation of the $\oa$-laxor, and Figure \ref{Fig: unit laxor rule} 
shows the derivation of the $\bot$-laxor. The steps labelled $(*)$ in Figure \ref{Fig: tensor laxor rule} involve a cut. This 
leads to the question of whether to system satisfies cut elimination: with the current presentation, the system 
not appear to satisfy cut elimination. 

\begin{figure}[h]
    \centering
	\AxiomC{$ $}
	\RightLabel{id}
	\UnaryInfC{$A \vdash A$}
	\AxiomC{$ $}
	\RightLabel{id}
	\UnaryInfC{$B \vdash B$}
	\RightLabel{$\ox$R}
	\BinaryInfC{$A, B \vdash A \ox B$}
	\UnaryInfC{$(A \ox B)^\dag \vdash A^\dag, B^\dag$}
	\RightLabel{$\oa$R}
    \UnaryInfC{$(A \ox B)^\dag \vdash A^\dag \oa B^\dag$}
    \DisplayProof
	\hspace{1.5em}
	\AxiomC{$ $}
	\RightLabel{id}
	\UnaryInfC{$A^\dag \vdash A^\dag$}
	\AxiomC{$ $}
	\RightLabel{id}
	\UnaryInfC{$B^\dag \vdash B^\dag$}
	\RightLabel{$\oa$L}
	\BinaryInfC{$A^\dag \oa B^\dag \vdash A^\dag, B^\dag$}
	\RightLabel{$\dagger$}
	\UnaryInfC{$A^{\dag \dag}, B^{\dag \dag} \vdash (A^\dag \oa B^\dag)^\dag$}
	\RightLabel{$(*)~ 2 \times \iota L ^{(-1)}$}
	\UnaryInfC{$A, B \vdash (A^\dag \oa B^\dag)^\dag$}
	\RightLabel{$\ox$L}
	\UnaryInfC{$A \ox B \vdash (A^\dag \oa B^\dag)^\dag$}
	\RightLabel{$\iota$L}
	\UnaryInfC{$ (A^\dag \oa B^\dag)^{\dag \dag} \vdash (A \ox B)^\dag$}
	\RightLabel{$(*)~ \iota L ^{(-1)}$}
	\UnaryInfC{$(A^\dag \oa B^\dag) \vdash (A \ox B)^\dag$}
	\DisplayProof
	\caption{Derivation of the $\oa$-laxor}
	\label{Fig: tensor laxor rule}
\end{figure}

\begin{figure}
	\centering
    \AxiomC{$ $ } 
	\RightLabel{id}
	\UnaryInfC{$ \vdash \top$} 
    \RightLabel{$\dagger$}
    \UnaryInfC{$\top^\dag \vdash $}
	\RightLabel{$\bot$R}
	\UnaryInfC{$\top^{\dag} \vdash \bot$}
	\DisplayProof
	\hspace{1.5em}    
		\AxiomC{$ $}
		\RightLabel{id}
		\UnaryInfC{$\bot \vdash \bot$} 
		\RightLabel{$\iota$R}
		\UnaryInfC{$\bot \vdash \bot^{\dag \dag}$}
						\AxiomC{$ $ } 
						\RightLabel{$\bot$L}
						\UnaryInfC{$ \bot \vdash $} 
						\RightLabel{$\dagger$}
						\UnaryInfC{$ \vdash \bot^\dagger $} 
						\RightLabel{$\top$L}
						\UnaryInfC{$ \top \vdash \bot^\dagger $} 
						\RightLabel{$\dagger$}
						\UnaryInfC{$ \bot^{\dagger \dagger} \vdash \top^\dagger $} 
	\RightLabel{Cut}  
	\BinaryInfC{$\bot \vdash \top^\dagger$}
	\DisplayProof
	\caption{Derivation of $\bot$-laxor} 
    \label{Fig: unit laxor rule}
\end{figure}

\iffalse 
\RightLabel{$\oa$R}
	\UnaryInfC{$\top^\dag \vdash \top^\dag \oa \bot$}
	\RightLabel{$\iota$R}
	\UnaryInfC{$\top^\dag \vdash (\top^\dag \oa \bot)^{\dag \dag}$}
	\RightLabel{$\dagger$}
	\UnaryInfC{$(\top^\dag \oa \bot)^{\dag \dag \dag} \vdash \top^{\dag \dag}$}
	\RightLabel{$\iota$L, $\iota$R}
	\UnaryInfC{$(\top^\dag \oa \bot)^{\dag} \vdash \top$}
\fi

\FloatBarrier

\subsection{Dagger functor box} 


Suppose $\X$ is a $\dagger$-LDC and  $f: A \rightarrow B \in \X$. Then, the map $f^\dagger: B^\dagger \rightarrow A^\dagger$ is graphically depicted as follows:
\[ \begin{tikzpicture}
	\begin{pgfonlayer}{nodelayer}
		\node [style=circle, scale=1.5] (0) at (0, 1) {};
		\node [style=none] (15) at (0, 1) {$f$};
		\node [style=none] (1) at (0.75, 2) {};
		\node [style=none] (2) at (-0.75, 0) {};
		\node [style=none] (3) at (-1, 2) {};
		\node [style=none] (4) at (-1, 0) {};
		\node [style=none] (5) at (1, 0) {};
		\node [style=none] (6) at (1, 2) {};
		\node [style=none] (7) at (0, 2) {};
		\node [style=none] (8) at (0, 2.75) {};
		\node [style=none] (9) at (0, 0) {};
		\node [style=none] (10) at (0, -0.75) {};
		\node [style=none] (11) at (0.5, 1.5) {$A$};
		\node [style=none] (12) at (-0.5, 0.5) {$B$};
		\node [style=none] (13) at (0.5, -0.75) {$A^\dagger$};
		\node [style=none] (14) at (0.5, 2.75) {$B^\dagger$};
	\end{pgfonlayer}
	\begin{pgfonlayer}{edgelayer}
		\draw [style=none, in=-90, out=90, looseness=1.00] (0) to (1.center);
		\draw [style=none, in=90, out=-90, looseness=1.00] (0) to (2.center);
		\draw [style=none] (3.center) to (4.center);
		\draw [style=none] (4.center) to (5.center);
		\draw [style=none] (5.center) to (6.center);
		\draw [style=none] (6.center) to (3.center);
		\draw [style=none] (8.center) to (7.center);
		\draw [style=none] (9.center) to (10.center);
	\end{pgfonlayer}
\end{tikzpicture} \]
The rectangle is a functor box for the $\dag$-functor.  Notice how we use vertical mirroring to express the contravariance of the $\dag$-functor. By the functoriality of $(\_)^\dagger$, we have:
\begin{tikzpicture}
	\begin{pgfonlayer}{nodelayer}
		\node [style=none] (0) at (-2, 1.25) {};
		\node [style=none] (1) at (-1.5, 1.25) {};
		\node [style=none] (2) at (-2, 0.7499999) {};
		\node [style=none] (3) at (-1.5, 0.7499999) {};
		\node [style=none] (4) at (-1.75, 1.5) {};
		\node [style=none] (5) at (-1.75, 0.5000001) {};
	\end{pgfonlayer}
	\begin{pgfonlayer}{edgelayer}
		\draw (0) to (1);
		\draw (3) to (1);
		\draw (0) to (2);
		\draw (2) to (3);
		\draw (5) to (4);
	\end{pgfonlayer}
\end{tikzpicture}
=
\begin{tikzpicture}
	\begin{pgfonlayer}{nodelayer}
		\node [style=none] (0) at (-1.75, 1.5) {};
		\node [style=none] (1) at (-1.75, 0.5) {};
	\end{pgfonlayer}
	\begin{pgfonlayer}{edgelayer}
		\draw (1) to (0);
	\end{pgfonlayer}
\end{tikzpicture}.

These contravariant functor boxes compose contravariantly. Given maps $f: A \to B$ and $g: B \to C$:
\[
\begin{tikzpicture} %daggercomposition
	\begin{pgfonlayer}{nodelayer}
		\node [style=none] (0) at (-1, 2) {};
		\node [style=none] (1) at (0.5, 2) {};
		\node [style=none] (2) at (-1, 0.5) {};
		\node [style=none] (3) at (0.5, 0.5) {};
		\node [style=none] (4) at (0.5, -1.5) {};
		\node [style=none] (5) at (0.5, -0) {};
		\node [style=none] (6) at (-1, -1.5) {};
		\node [style=none] (7) at (-1, -0) {};
		\node [style=circle, scale=1.5] (8) at (-0.25, 1.25) {};
		\node [style=none] (9) at (-0.5, 2) {};
		\node [style=none] (10) at (0, 0.5) {};
		\node [style=none] (11) at (-0.5, -0) {};
		\node [style=circle, scale=1.5] (12) at (-0.25, -0.75) {};
		\node [style=none] (13) at (0, -1.5) {};
		\node [style=none] (14) at (-0.25, 2.75) {};
		\node [style=none] (15) at (-0.25, -2.25) {};
		\node [style=none] (16) at (-0.25, 0.5) {};
		\node [style=none] (17) at (-0.25, -0) {};
		\node [style=none] (18) at (-0.25, 2) {};
		\node [style=none] (19) at (-0.25, -1.5) {};
		\node [style=none] (20) at (-0.25, -0.75) {$f$};
		\node [style=none] (21) at (-0.25, 1.25) {$g$};
		\node [style=none] (22) at (0, 2.5) {$C^\dagger$};
		\node [style=none] (23) at (0, 0.25) {$B^\dagger$};
		\node [style=none] (24) at (0, -2) {$A^\dagger$};
	\end{pgfonlayer}
	\begin{pgfonlayer}{edgelayer}
		\draw (0.center) to (1.center);
		\draw (1.center) to (3.center);
		\draw (3.center) to (2.center);
		\draw (2.center) to (0.center);
		\draw (7.center) to (5.center);
		\draw (5.center) to (4.center);
		\draw (4.center) to (6.center);
		\draw (6.center) to (7.center);
		\draw [in=90, out=-90, looseness=1.25] (9.center) to (8);
		\draw [in=90, out=-90, looseness=1.25] (8) to (10.center);
		\draw [in=90, out=-90, looseness=1.25] (11.center) to (12);
		\draw [in=90, out=-90, looseness=1.25] (12) to (13.center);
		\draw (16.center) to (17.center);
		\draw (19.center) to (15.center);
		\draw (14.center) to (18.center);
	\end{pgfonlayer}
\end{tikzpicture} = \begin{tikzpicture}
	\begin{pgfonlayer}{nodelayer}
		\node [style=none] (0) at (-1, 2) {};
		\node [style=none] (1) at (0.5, 2) {};
		\node [style=none] (2) at (0.5, -1.5) {};
		\node [style=none] (3) at (-1, -1.5) {};
		\node [style=circle, scale=2] (4) at (-0.25, 1.25) {};
		\node [style=none] (5) at (-0.5, 2) {};
		\node [style=circle, scale=2] (6) at (-0.25, -0.75) {};
		\node [style=none] (7) at (0, -1.5) {};
		\node [style=none] (8) at (-0.25, 2.75) {};
		\node [style=none] (9) at (-0.25, -2.25) {};
		\node [style=none] (10) at (-0.25, 2) {};
		\node [style=none] (11) at (-0.25, -1.5) {};
		\node [style=none] (12) at (-0.25, -0.75) {$g$};
		\node [style=none] (13) at (-0.25, 1.25) {$f$};
		\node [style=none] (14) at (0, 2.5) {$C^\dagger$};
		\node [style=none] (15) at (0, -2) {$A^\dagger$};
	\end{pgfonlayer}
	\begin{pgfonlayer}{edgelayer}
		\draw (0.center) to (1.center);
		\draw (2.center) to (3.center);
		\draw [in=90, out=-90, looseness=1.25] (5.center) to (4);
		\draw [in=90, out=-90, looseness=1.25] (6) to (7.center);
		\draw (11.center) to (9.center);
		\draw (8.center) to (10.center);
		\draw (4) to (6);
		\draw (0.center) to (3.center);
		\draw (1.center) to (2.center);
	\end{pgfonlayer}
\end{tikzpicture}
\]

The following are the representations of the basic natural isomorphisms of a $\dagger$-LDC:
\[ \begin{array}{l l}
\lambda_\top: \top \rightarrow \bot^\dagger = 
\begin{tikzpicture}
	\begin{pgfonlayer}{nodelayer}
		\node [style=none] (0) at (-2, 0) {};
		\node [style=none] (1) at (-2, -1) {};
		\node [style=none] (2) at (-1, -1) {};
		\node [style=none] (3) at (-1, 0) {};
		\node [style=circle] (4) at (-1.5, -0.5) {$\bot$};
		\node [style=none] (5) at (-1.25, 0) {};
		\node [style=none] (6) at (-1.75, -1) {};
		\node [style=none] (7) at (-1.5, -2) {};
		\node [style=none] (8) at (-1.5, -1) {};
		\node [style=circle, scale=0.4] (9) at (-1.5, -1.5) {};
		\node [style=circle] (10) at (-2.5, -0.5) {$\top$};
		\node [style=none] (11) at (-2.5, 0.5) {};
	\end{pgfonlayer}
	\begin{pgfonlayer}{edgelayer}
		\draw [style=none] (0.center) to (1.center);
		\draw [style=none] (1.center) to (2.center);
		\draw [style=none] (2.center) to (3.center);
		\draw [style=none] (3.center) to (0.center);
		\draw [style=none] (4) to (5.center);
		\draw [style=none] (7.center) to (8.center);
		\draw [style=none] (11.center) to (10);
		\draw [densely dotted, bend right=45, looseness=1.25] (10) to (9);
	\end{pgfonlayer}
\end{tikzpicture}  & 
\lambda_\top^{-1}: \bot^\dagger \rightarrow \top =
\begin{tikzpicture}
	\begin{pgfonlayer}{nodelayer}
		\node [style=none] (0) at (-2, 0) {};
		\node [style=none] (1) at (-2, -1) {};
		\node [style=none] (2) at (-1, -1) {};
		\node [style=none] (3) at (-1, 0) {};
		\node [style=circle] (4) at (-1.5, -0.5) {$\bot$};
		\node [style=none] (5) at (-1.25, 0) {};
		\node [style=none] (6) at (-1.75, -1) {};
		\node [style=none] (7) at (-1.5, -1) {};
		\node [style=circle] (8) at (-0.5, -0.25) {$\top$};
		\node [style=none] (9) at (-1.5, 0) {};
		\node [style=none] (10) at (-1.5, 0.5) {};
		\node [style=none] (11) at (-0.5, -2) {};
		\node [style=circle, scale=0.4] (12) at (-0.5, -1.25) {};
	\end{pgfonlayer}
	\begin{pgfonlayer}{edgelayer}
		\draw [style=none] (0.center) to (1.center);
		\draw [style=none] (1.center) to (2.center);
		\draw [style=none] (2.center) to (3.center);
		\draw [style=none] (3.center) to (0.center);
		\draw [style=none] (4) to (6.center);
		\draw [style=none] (10.center) to (9.center);
		\draw [style=none] (8) to (11.center);
		\draw [densely dotted, bend right=45, looseness=1.00] (7.center) to (12);
		\draw [densely dotted] (4) to (5.center);
	\end{pgfonlayer}
\end{tikzpicture}  \\
\lambda_\bot: \bot \rightarrow \top^\dagger = \begin{tikzpicture}
	\begin{pgfonlayer}{nodelayer}
		\node [style=none] (0) at (-2, -1.5) {};
		\node [style=none] (1) at (-2, -0.5) {};
		\node [style=none] (2) at (-1, -0.5) {};
		\node [style=none] (3) at (-1, -1.5) {};
		\node [style=circle] (4) at (-1.5, -1) {$\top$};
		\node [style=none] (5) at (-1.75, -1.5) {};
		\node [style=none] (6) at (-1.25, -0.5) {};
		\node [style=none] (7) at (-1.5, -0.5) {};
		\node [style=circle] (8) at (-0.5, -1) {$\bot$};
		\node [style=none] (9) at (-1.5, -1.5) {};
		\node [style=none] (10) at (-1.5, -2) {};
		\node [style=none] (11) at (-0.5, 0.5) {};
		\node [style=circle, scale=0.4] (12) at (-0.5, 0) {};
	\end{pgfonlayer}
	\begin{pgfonlayer}{edgelayer}
		\draw [style=none] (0.center) to (1.center);
		\draw [style=none] (1.center) to (2.center);
		\draw [style=none] (2.center) to (3.center);
		\draw [style=none] (3.center) to (0.center);
		\draw [style=none] (10.center) to (9.center);
		\draw [style=none] (8) to (11.center);
		\draw [densely dotted] (4) to (5.center);
		\draw [style=none] (4) to (6.center);
		\draw [densely dotted, bend left, looseness=1.25] (7.center) to (12);
	\end{pgfonlayer}
\end{tikzpicture} &
\lambda_\bot^{-1}: \top^\dagger \rightarrow \bot = \begin{tikzpicture}
	\begin{pgfonlayer}{nodelayer}
		\node [style=none] (0) at (-2, 0) {};
		\node [style=none] (1) at (-2, -1) {};
		\node [style=none] (2) at (-1, -1) {};
		\node [style=none] (3) at (-1, 0) {};
		\node [style=circle] (4) at (-1.5, -0.5) {$\top$};
		\node [style=none] (5) at (-1.75, 0) {};
		\node [style=none] (6) at (-1.25, -1) {};
		\node [style=none] (7) at (-1.5, -1) {};
		\node [style=circle] (8) at (-0.5, -0.5) {$\bot$};
		\node [style=none] (9) at (-1.5, 0) {};
		\node [style=none] (10) at (-1.5, 0.5) {};
		\node [style=none] (11) at (-0.5, -2) {};
		\node [style=circle,scale=0.4] (12) at (-0.5, -1.25) {};
	\end{pgfonlayer}
	\begin{pgfonlayer}{edgelayer}
		\draw [style=none] (0.center) to (1.center);
		\draw [style=none] (1.center) to (2.center);
		\draw [style=none] (2.center) to (3.center);
		\draw [style=none] (3.center) to (0.center);
		\draw [style=none] (10.center) to (9.center);
		\draw [style=none] (8) to (11.center);
		\draw [densely dotted] (4) to (5.center);
		\draw [style=none] (4) to (6.center);
		\draw [densely dotted, bend right=45, looseness=1.25] (7.center) to (12);
	\end{pgfonlayer}
\end{tikzpicture}\\
\lambda_\ox: A^\dagger \ox B^\dagger \rightarrow (A \oa B)^\dagger = \begin{tikzpicture}[rotate=180]
	\begin{pgfonlayer}{nodelayer}
		\node [style=oa] (0) at (-3, 0) {};
		\node [style=none] (1) at (-3.5, 0.75) {};
		\node [style=none] (2) at (-2.5, 0.75) {};
		\node [style=none] (3) at (-3.5, -0.75) {};
		\node [style=none] (4) at (-2.5, -0.75) {};
		\node [style=ox] (5) at (-3, -1.5) {};
		\node [style=none] (6) at (-3, -2.25) {};
		\node [style=none] (7) at (-3, 1.5) {};
		\node [style=none] (8) at (-3, -0.75) {};
		\node [style=none] (9) at (-3, 0.75) {};
		\node [style=none] (10) at (-4, 0.75) {};
		\node [style=none] (11) at (-2, 0.75) {};
		\node [style=none] (12) at (-2, -0.75) {};
		\node [style=none] (13) at (-4, -0.75) {};
	\end{pgfonlayer}
	\begin{pgfonlayer}{edgelayer}
		\draw [style=none, bend right, looseness=1.00] (1.center) to (0);
		\draw [style=none, bend right, looseness=1.00] (0) to (2.center);
		\draw [style=none] (0) to (8.center);
		\draw [style=none, bend right, looseness=1.00] (3.center) to (5);
		\draw [style=none, bend right, looseness=1.00] (5) to (4.center);
		\draw [style=none] (5) to (6.center);
		\draw [style=none] (7.center) to (9.center);
		\draw [style=none] (10.center) to (13.center);
		\draw [style=none] (13.center) to (12.center);
		\draw [style=none] (12.center) to (11.center);
		\draw [style=none] (11.center) to (10.center);
	\end{pgfonlayer}
	\end{tikzpicture}  &
\lambda_\oa: A^\dagger \ox B^\dagger \rightarrow (A \oa B)^\dagger = \begin{tikzpicture}[rotate=180]
	\begin{pgfonlayer}{nodelayer}
		\node [style=ox] (0) at (-3, 0) {};
		\node [style=none] (1) at (-3.5, 0.75) {};
		\node [style=none] (2) at (-2.5, 0.75) {};
		\node [style=none] (3) at (-3.5, -0.75) {};
		\node [style=none] (4) at (-2.5, -0.75) {};
		\node [style=oa] (5) at (-3, -1.5) {};
		\node [style=none] (6) at (-3, -2.25) {};
		\node [style=none] (7) at (-3, 1.5) {};
		\node [style=none] (8) at (-3, -0.75) {};
		\node [style=none] (9) at (-3, 0.75) {};
		\node [style=none] (10) at (-4, 0.75) {};
		\node [style=none] (11) at (-2, 0.75) {};
		\node [style=none] (12) at (-2, -0.75) {};
		\node [style=none] (13) at (-4, -0.75) {};
	\end{pgfonlayer}
	\begin{pgfonlayer}{edgelayer}
		\draw [style=none, bend right, looseness=1.00] (1.center) to (0);
		\draw [style=none, bend right, looseness=1.00] (0) to (2.center);
		\draw [style=none] (0) to (8.center);
		\draw [style=none, bend right, looseness=1.00] (3.center) to (5);
		\draw [style=none, bend right, looseness=1.00] (5) to (4.center);
		\draw [style=none] (5) to (6.center);
		\draw [style=none] (7.center) to (9.center);
		\draw [style=none] (10.center) to (13.center);
		\draw [style=none] (13.center) to (12.center);
		\draw [style=none] (12.center) to (11.center);
		\draw [style=none] (11.center) to (10.center);
	\end{pgfonlayer}
\end{tikzpicture} \\
\lambda_\otimes^{-1}: (A \oa B)^\dagger \to A^\dagger \ox B^\dagger =
\begin{tikzpicture}
	\begin{pgfonlayer}{nodelayer}
		\node [style=oa] (0) at (-3, 0) {};
		\node [style=none] (1) at (-3.5, 0.75) {};
		\node [style=none] (2) at (-2.5, 0.75) {};
		\node [style=none] (3) at (-3.5, -0.75) {};
		\node [style=none] (4) at (-2.5, -0.75) {};
		\node [style=ox] (5) at (-3, -1.5) {};
		\node [style=none] (6) at (-3, -2.25) {};
		\node [style=none] (7) at (-3, 1.5) {};
		\node [style=none] (8) at (-3, -0.75) {};
		\node [style=none] (9) at (-3, 0.75) {};
		\node [style=none] (10) at (-4, 0.75) {};
		\node [style=none] (11) at (-2, 0.75) {};
		\node [style=none] (12) at (-2, -0.75) {};
		\node [style=none] (13) at (-4, -0.75) {};
	\end{pgfonlayer}
	\begin{pgfonlayer}{edgelayer}
		\draw [style=none, bend right, looseness=1.00] (1.center) to (0);
		\draw [style=none, bend right, looseness=1.00] (0) to (2.center);
		\draw [style=none] (0) to (8.center);
		\draw [style=none, bend right, looseness=1.00] (3.center) to (5);
		\draw [style=none, bend right, looseness=1.00] (5) to (4.center);
		\draw [style=none] (5) to (6.center);
		\draw [style=none] (7.center) to (9.center);
		\draw [style=none] (10.center) to (13.center);
		\draw [style=none] (13.center) to (12.center);
		\draw [style=none] (12.center) to (11.center);
		\draw [style=none] (11.center) to (10.center);
	\end{pgfonlayer}
\end{tikzpicture} 
&
\lambda_\oa^{-1}:  (A \ox B)^\dagger \to A^\dagger \oa B^\dagger =
\begin{tikzpicture}
	\begin{pgfonlayer}{nodelayer}
		\node [style=ox] (0) at (-3, 0) {};
		\node [style=none] (1) at (-3.5, 0.75) {};
		\node [style=none] (2) at (-2.5, 0.75) {};
		\node [style=none] (3) at (-3.5, -0.75) {};
		\node [style=none] (4) at (-2.5, -0.75) {};
		\node [style=oa] (5) at (-3, -1.5) {};
		\node [style=none] (6) at (-3, -2.25) {};
		\node [style=none] (7) at (-3, 1.5) {};
		\node [style=none] (8) at (-3, -0.75) {};
		\node [style=none] (9) at (-3, 0.75) {};
		\node [style=none] (10) at (-4, 0.75) {};
		\node [style=none] (11) at (-2, 0.75) {};
		\node [style=none] (12) at (-2, -0.75) {};
		\node [style=none] (13) at (-4, -0.75) {};
	\end{pgfonlayer}
	\begin{pgfonlayer}{edgelayer}
		\draw [style=none, bend right, looseness=1.00] (1.center) to (0);
		\draw [style=none, bend right, looseness=1.00] (0) to (2.center);
		\draw [style=none] (0) to (8.center);
		\draw [style=none, bend right, looseness=1.00] (3.center) to (5);
		\draw [style=none, bend right, looseness=1.00] (5) to (4.center);
		\draw [style=none] (5) to (6.center);
		\draw [style=none] (7.center) to (9.center);
		\draw [style=none] (10.center) to (13.center);
		\draw [style=none] (13.center) to (12.center);
		\draw [style=none] (12.center) to (11.center);
		\draw [style=none] (11.center) to (10.center);
	\end{pgfonlayer}
\end{tikzpicture} 
\end{array}
\]


Dagger boxes interact with involutor $A \xrightarrow{\iota} A^{\dagger\dagger}$ as follows:
$$
\begin{tikzpicture}
\begin{pgfonlayer}{nodelayer}
\node [style=circle] (0) at (-3.5, 0) {$f$};
\node [style=none] (1) at (-4.5, 1) {};
\node [style=none] (2) at (-2.5, 1) {};
\node [style=none] (3) at (-4.5, -0.5) {};
\node [style=none] (4) at (-2.5, -0.5) {};
\node [style=none] (5) at (-4.75, -1) {};
\node [style=none] (6) at (-2.25, -1) {};
\node [style=none] (7) at (-2.25, 1.25) {};
\node [style=none] (8) at (-4.75, 1.25) {};
\node [style=circle] (9) at (-4.25, 1.75) {$\iota$};
\node [style=none] (10) at (-4.25, 2.25) {};
\node [style=none] (11) at (-4.25, 1.25) {};
\node [style=none] (12) at (-3.5, -1) {};
\node [style=none] (13) at (-3.5, -1.5) {};
\node [style=none] (14) at (-2.75, 1) {};
\node [style=none] (15) at (-3.5, -0.5) {};
\node [style=none] (16) at (-3.5, 1.25) {};
\node [style=none] (17) at (-4.25, -1) {};
\node [style=none] (18) at (-4.25, -0.5) {};
\node [style=none] (19) at (-3.5, 1) {};
\node [style=circle] (20) at (-2.75, 1.75) {$\iota$};
\node [style=none] (21) at (-2.75, 1.25) {};
\node [style=none] (22) at (-2.75, 2.25) {};
\node [style=none] (23) at (-4.25, 1) {};
\node [style=none] (24) at (-2.75, -0.5) {};
\node [style=none] (25) at (-2.75, -1) {};
\node [style=none] (26) at (-3.5, 0.75) {$\cdots$};
\node [style=none] (27) at (-3.5, 1.75) {$\cdots$};
\node [style=none] (28) at (-3.5, -0.75) {$\cdots$};
\end{pgfonlayer}
\begin{pgfonlayer}{edgelayer}
\draw [style=none] (5.center) to (6.center);
\draw [style=none] (6.center) to (7.center);
\draw [style=none] (7.center) to (8.center);
\draw [style=none] (8.center) to (5.center);
\draw [style=none] (1.center) to (3.center);
\draw [style=none] (3.center) to (4.center);
\draw [style=none] (4.center) to (2.center);
\draw [style=none] (2.center) to (1.center);
\draw [style=none, in=-90, out=60, looseness=1.00] (0) to (14.center);
\draw [style=none] (0) to (15.center);
\draw [style=none] (18.center) to (17.center);
\draw [style=none] (17.center) to (5.center);
\draw [style=none] (19.center) to (16.center);
\draw [style=none] (12.center) to (13.center);
\draw [style=none] (11.center) to (9);
\draw [style=none] (10.center) to (9);
\draw [style=none] (21.center) to (20);
\draw [style=none] (22.center) to (20);
\draw [style=none, in=120, out=-90, looseness=1.00] (23.center) to (0);
\draw [style=none] (25.center) to (24.center);
\end{pgfonlayer}
\end{tikzpicture}
=
\begin{tikzpicture}
\begin{pgfonlayer}{nodelayer}
\node [style=circle] (0) at (-2.5, 1) {$f$};
\node [style=none] (1) at (-2, 2) {};
\node [style=none] (2) at (-2.5, -1) {};
\node [style=circle] (3) at (-2.5, 0) {$i$};
\node [style=none] (4) at (-3, 2) {};
\node [style=none] (5) at (-2.5, 1.75) {$\cdots$};
\end{pgfonlayer}
\begin{pgfonlayer}{edgelayer}
\draw [style=none, in=45, out=-90, looseness=1.00] (1.center) to (0);
\draw [style=none] (0) to (3);
\draw [style=none] (3) to (2.center);
\draw [style=none, in=135, out=-90, looseness=1.00] (4.center) to (0);
\end{pgfonlayer}
\end{tikzpicture}
$$

%the issue is more nuanced than this, i think
It is worth noting that one need not have a legal proof net inside a $\dagger$-box. 
This complicates the correctness criterion. However, the required 
correctness criterion is discussed in \cite{MP05}.

\subsection{Functors for $\dagger$-linearly distributive categories}
\label{Sec: dagger linear}

Clearly the functors and transformations between $\dagger$-LDCs must ``preserve'' the dagger in some sense.  Precisely we have:

\begin{definition}
	$F: \X \to \Y$ is a {\bf $\dagger$-linear functor} between $\dagger$-LDCs when $F$ is a linear functor equipped with a linear natural isomorphism 
	$\rho^F= (\rho_\ox^F: F_\ox(A^\dagger) \to F_\oa(A)^\dagger ,\rho_\oa^F:  F_\ox(A)^\dagger \to F_\oa(A^\dagger))$ called the {\bf preservator}, 
	such that  the following diagrams commute:
	\[ 
	\xymatrix{
		F_\ox(X) \ar[r]^{\iota} \ar[d]_{F_\ox(\iota)} \ar@{}[dr]|{\mbox{\tiny {\bf [$\dagger$-LF.1]}}} & 
		F_\ox(X)^{\dagger \dagger} \ar@{<-}[d]^{(\rho^F_\oa)^\dagger} \\
		F_\ox(X^{\dagger \dagger}) \ar[r]_{\rho^F_\ox} & F_\oa(X^\dagger)^\dagger
	} ~~~~~~~~~ \xymatrix{
		F_\oa(X) \ar[r]^{\iota} \ar[d]_{F_\oa(\iota)}  \ar@{}[dr]|{\mbox{\tiny {\bf [$\dagger$-LF.2]}}} & F_\oa(X)^{\dagger \dagger} \ar[d]^{(\rho^F_\ox)^\dagger} \\
		F_\oa(X^{\dagger \dagger}) \ar@{<-}[r]_{\rho^F_\oa} & F_\ox(X^\dagger)^\dagger
	}
	\]
\end{definition}

In case that $F$ is a normal mix functor between $\dagger$-isomix categories, then by Lemma \ref{Lemma: isomix functor}, $F$ is an isomix functor and, therefore by Corollary \ref{Corollary: normal-nat-iso},  the preservators become inverses, $\rho^F_\ox = (\rho^F_\oa)^{-1}$.  
This means the squares {\bf [$\dagger$-LF.1]} and {\bf [$\dagger$-LF.2]} coincide to give a single condition for the tensor preservator:
\[ \xymatrix{
	F(X) \ar[r]^{\iota} \ar[d]_{F(\iota)} \ar@{}[dr]|{\mbox{\tiny {\bf [$\dagger$-isomix]}}} & 
	F(X)^{\dagger \dagger} \ar@{->}[d]^{(\rho^F_\ox)^\dagger} \\
	F(X^{\dagger \dagger}) \ar[r]_{\rho^F_\ox} & F(X^\dagger)^\dagger
} \]

%%%%%%%%%%%%%%%%

In case when $F$ is an isomix functor, by Lemma \ref{Lemma: Frobenius linear transformation}, $\rho := \rho_\ox$ is monoidal on $\ox$ and comonoidal on $\oa$:

\[
\bf{[P.1]} ~~~~~~~
\xymatrix{
	F(A^\dagger) \ox F(B^\dagger) \ar[r]^{\rho \ox \rho} \ar[d]_{m_\ox} \ar@{}[ddr]|{(a)} & F(A)^\dagger \ox F(B)^\dagger \ar[d]^{\lambda_\ox} \\
	F(A^\dagger \ox B^\dagger) \ar[d]_{F(\lambda_\ox)} & (F(A) \ox F(B))^\dagger \ar[d]^{n_\oa^\dagger} \\
	F((A \oa B)^\dagger) \ar[r]_{\rho} & (F(A\oa B))^\dagger
}~~~~~~~~~ \xymatrix{
	\top \ar[d]_{m_\top} \ar[dr]^{\lambda_\top} \ar@{}[ddr]|{(b)} & \\
	F(\top) \ar[d]_{F(\lambda_\top)} & \bot^\dagger \ar[dr]^{n_\bot^\dagger} \\
	F(\bot^\dagger) \ar[rr]_{\rho} & & (F(\bot))^\dagger
}
\]

\[
\bf{[P.2]} ~~~~~~~
\xymatrix{
	F((A \ox B)^\dagger) \ar[r]^{\rho} \ar[d]_{F(\lambda_\oa^{-1})} \ar@{}[ddr]|{(a)} & F(A \ox B)^\dagger \ar[d]^{m_\ox^\dagger} \\
	F(A^\dagger \oa B^\dagger) \ar[d]_{n_\oa^F} & (F(A) \ox F(B))^\dagger \ar[d]^{\lambda_\oa^{-1}} \\
	F(A^\dagger) \oa F(B^\dagger) \ar[r]_{\rho \oa \rho} & F(A)^\dagger \oa F(B)^\dagger
} ~~~~~~~~~ \xymatrix{
	F(\top^\dagger) \ar[dd]_{\rho} \ar[dr]^{F(\lambda_\bot^{-1})} \ar@{}[ddr]|{(b)} & \\
	& F(\bot) \ar[dr]^{n_\bot} & \\
	F(\top)^\dagger \ar[r]_{m_\top^\dagger} & \top^\dagger \ar[r]_{\lambda_\bot^{-1}} & \bot
}
\]

Pictorial representation of {\bf[P.2]-(a)} is as follows:
\[
\begin{tikzpicture} %rho-comon-b
\begin{pgfonlayer}{nodelayer}
\node [style=none] (0) at (1.75, 3.25) {};
\node [style=none] (1) at (2.25, -0) {};
\node [style=none] (2) at (1.25, -0) {};
\node [style=none] (3) at (0.25, -0) {};
\node [style=none] (4) at (0.75, 2.25) {};
\node [style=ox] (5) at (1.75, 1.5) {};
\node [style=none] (6) at (2.75, 2.25) {};
\node [style=none] (7) at (2.75, 1) {};
\node [style=none] (8) at (3.25, 2.75) {};
\node [style=none] (9) at (1.25, 2.25) {};
\node [style=circle] (10) at (2.25, -3) {$\rho$};
\node [style=none] (11) at (1.75, 2.25) {};
\node [style=none] (12) at (1.75, 1) {};
\node [style=none] (13) at (2.25, -3.5) {};
\node [style=none] (14) at (0.75, 1) {};
\node [style=none] (15) at (3, 0.25) {$M$};
\node [style=none] (16) at (3.25, -0) {};
\node [style=circle] (17) at (1.25, -3) {$\rho$};
\node [style=none] (18) at (1.25, -3.5) {};
\node [style=none] (19) at (2.25, 2.25) {};
\node [style=none] (20) at (0.25, 2.75) {};
\node [style=none] (21) at (1.25, -0.75) {};
\node [style=ox] (22) at (1.75, -1.25) {};
\node [style=none] (23) at (2.25, -0.75) {};
\node [style=none] (24) at (2.75, -2) {};
\node [style=none] (25) at (0.75, -0.75) {};
\node [style=none] (26) at (0.75, -2) {};
\node [style=none] (27) at (1.25, -2) {};
\node [style=none] (28) at (1.75, -0.75) {};
\node [style=none] (29) at (2.75, -0.75) {};
\node [style=none] (30) at (2.25, -2) {};
\node [style=none] (31) at (2.25, 1) {};
\node [style=none] (32) at (1.25, 1) {};
\node [style=none] (33) at (1.75, -0) {};
\node [style=ox] (34) at (1.75, 0.5) {};
\node [style=none] (35) at (3, 3) {$F((A \ox B)^\dagger)$};
\node [style=none] (36) at (3, -0.25) {$F(A^\dagger \ox B^\dagger)$};
\node [style=none] (37) at (0.25, -2.5) {$F(A^\dagger)$};
\node [style=none] (38) at (3, -2.5) {$F(B^\dagger)$};
\node [style=none] (39) at (0, -3.5) {$F(A)^\dagger$};
\node [style=none] (40) at (3, -3.5) {$F(B)^\dagger$};
\end{pgfonlayer}
\begin{pgfonlayer}{edgelayer}
\draw [bend right, looseness=1.50] (9.center) to (5);
\draw [bend right, looseness=1.50] (5) to (19.center);
\draw (5) to (12.center);
\draw (4.center) to (6.center);
\draw (6.center) to (7.center);
\draw (14.center) to (7.center);
\draw (14.center) to (4.center);
\draw (20.center) to (3.center);
\draw (3.center) to (16.center);
\draw (16.center) to (8.center);
\draw (8.center) to (20.center);
\draw (17) to (18.center);
\draw (10) to (13.center);
\draw (11.center) to (0.center);
\draw [bend left, looseness=1.50] (27.center) to (22);
\draw [bend left, looseness=1.50] (22) to (30.center);
\draw (22) to (28.center);
\draw (26.center) to (24.center);
\draw (24.center) to (29.center);
\draw (25.center) to (29.center);
\draw (25.center) to (26.center);
\draw [bend right, looseness=1.50] (32.center) to (34);
\draw [bend right, looseness=1.50] (34) to (31.center);
\draw (34) to (33.center);
\draw (33.center) to (28.center);
\draw (27.center) to (17);
\draw (30.center) to (10);
\end{pgfonlayer}
\end{tikzpicture} = \begin{tikzpicture} %rho-comon-a
\begin{pgfonlayer}{nodelayer}
\node [style=none] (0) at (1.5, 0.5) {};
\node [style=none] (1) at (-0.25, -0.75) {};
\node [style=none] (2) at (-0.25, 1.75) {};
\node [style=none] (3) at (1.25, -0.75) {};
\node [style=ox] (4) at (1, -0) {};
\node [style=none] (5) at (0.5, 0.5) {};
\node [style=none] (6) at (2.25, 1.75) {};
\node [style=none] (7) at (1.5, 0.75) {};
\node [style=none] (8) at (1.25, -1.25) {};
\node [style=none] (9) at (1.25, 1.75) {};
\node [style=ox] (10) at (1, 1.25) {};
\node [style=none] (11) at (2, 0.75) {};
\node [style=none] (12) at (0.5, 1.75) {};
\node [style=none] (13) at (1, -0.75) {};
\node [style=none] (14) at (2, -0.5) {};
\node [style=none] (15) at (0, -0.5) {};
\node [style=circle] (16) at (1.25, 2.5) {$\rho$};
\node [style=none] (17) at (0, 0.75) {};
\node [style=none] (18) at (2.25, -0.75) {};
\node [style=none] (19) at (1, 1.75) {};
\node [style=none] (20) at (1.75, -0.25) {$M$};
\node [style=none] (21) at (0.5, 0.75) {};
\node [style=none] (22) at (1.25, 3.25) {};
\node [style=none] (23) at (0, -1.25) {};
\node [style=none] (24) at (0.5, -1.25) {};
\node [style=ox] (25) at (1, -2) {};
\node [style=none] (26) at (1, -2.5) {};
\node [style=none] (27) at (2, -1.25) {};
\node [style=none] (28) at (0, -2.5) {};
\node [style=none] (29) at (1.5, -1.25) {};
\node [style=none] (30) at (2, -2.5) {};
\node [style=none] (31) at (0.5, -2.5) {};
\node [style=none] (32) at (0.5, -3) {};
\node [style=none] (33) at (1.5, -3) {};
\node [style=none] (34) at (1.5, -2.5) {};
\node [style=none] (35) at (2.25, 3) {$F((A \ox B)^\dagger)$};
\node [style=none] (36) at (3.25, -1) {$F((A) \ox F(B))^\dagger$};
\node [style=none] (37) at (-0.5, -2.75) {$F(A)^\dagger$};
\node [style=none] (38) at (2.75, -2.75) {$F(B)^\dagger$};
\node [style=none] (39) at (2.25, 2.25) {};
\end{pgfonlayer}
\begin{pgfonlayer}{edgelayer}
\draw [bend left, looseness=1.50] (5.center) to (10);
\draw [bend left, looseness=1.50] (10) to (0.center);
\draw (1.center) to (18.center);
\draw (18.center) to (6.center);
\draw (2.center) to (6.center);
\draw (2.center) to (1.center);
\draw [bend right, looseness=1.50] (21.center) to (4);
\draw [bend right, looseness=1.50] (4) to (7.center);
\draw (4) to (13.center);
\draw (17.center) to (11.center);
\draw (11.center) to (14.center);
\draw (15.center) to (14.center);
\draw (15.center) to (17.center);
\draw (16) to (9.center);
\draw (19.center) to (10);
\draw (3.center) to (8.center);
\draw (22.center) to (16);
\draw [bend right, looseness=1.50] (24.center) to (25);
\draw [bend right, looseness=1.50] (25) to (29.center);
\draw (25) to (26.center);
\draw (23.center) to (27.center);
\draw (27.center) to (30.center);
\draw (28.center) to (30.center);
\draw (28.center) to (23.center);
\draw (31.center) to (32.center);
\draw (34.center) to (33.center);
\end{pgfonlayer}
\end{tikzpicture}
\]

%%%%%%%%%%%%%%%%

%%%%%%%%%%%%%%%%
For linear natural transformations $(\beta_\ox, \beta_\oa): F \to G$ between $\dagger$-linear functors, we demand that $\beta_\ox$ and $\beta_\oa$ are related by:

\[ \xymatrix{F_\ox(A^\dagger) \ar[d]_{\rho^F_\ox} \ar[rr]^{\beta_\ox} && G_\ox(A^\dagger)  \ar[d]^{\rho^G_\ox} \\
	(F_\oa(X))^\dagger \ar[rr]_{\beta_\oa^\dagger} & &  (G_\oa(X))^\dagger}
~~~~~~
\xymatrix{(G_\ox(X))^\dagger \ar[d]_{\rho^G_\oa} \ar[rr]^{\beta_\ox^\dagger} &&  (F_\ox(X))^\dagger \ar[d]^{\rho^F_\oa} \\
	G_\oa(A^\dagger) \ar[rr]_{\beta_\oa} & &  F_\oa(A^\dagger)} \]
Notice that this means that $\beta_\ox$ is completely determined by $\beta_\oa$ in the following sense:
\[ \xymatrix{ F_\ox(A) \ar[d]_{F_\ox(\iota)}  \ar[rr]^{\beta_\ox} && G_\ox(A) \ar[d]^{G_\ox(\iota)} \\
	F_\ox(A^{\dagger\dagger})  \ar[d]_{\rho^F_\ox} \ar[rr]^{\beta_\ox} & & G_\ox(A^{\dagger\dagger}) \ar[d]^{\rho^G_\ox}  \\
	F_\oa(A^\dagger)^\dagger \ar[rr]_{\beta_\oa^\dagger} && G_\oa(A^\dagger)^\dagger } \]
Because the vertical maps are isomorphisms, this diagram can be used to express $\beta_\ox$ in terms of $\beta_\oa$.  Similarly $\beta_\oa$ can be expressed in terms 
of $\beta_\ox$.  Thus, it is possible to express the coherences in terms of just one of these transformations.

%%%%%%%%%%%%%%%%%%%%%%%%%%%%%%%%%%%%%%%%%%%%%%%%%%%%%%%%%%%%%%%%%%%

\section{Examples: $\dagger$-LDCs}
\label{subsection: Dagger LDC examples}

In this section, we discuss a few basic examples of $\dagger$-isomix categories. 
The first two are compact LDCs. All these examples have a non-stationary dagger functor.  
More examples of $\dagger$-isomix categories can be found in the next chapter when we discuss the dagger 
and the conjugation functor.

%%%%%%%%%%%%%%%%%%%%%%%%%%%%%%%%%%%%%%%%%%%%%%%%%%%%%%%%%%%%%%%%%%%

\subsection{Every $\dagger$-monoidal category is a $\dagger$-LDC}

A $\dagger$-monoidal category \cite{Sel07} is defined as a symmetric monoidal category, $\X$, with a  
contravariant functor $\dagger: \X^\op \to \X$ which is stationary on objects ($A = A^\dag$) such that:
\begin{enumerate}[(i)]
	\item for all $f$, $f^{\dag \dag} = f$
	\item for all $f$ and $g$, $(f \ox g)^\dag = f^\dag \ox g^\dag$
	\item $a_\ox^\dag = a_\ox^{-1}$
	\item $(u_\ox^l)^\dag = (u_\ox^l)^{-1}$
	\item $(u_\ox^r)^\dag = (u_\ox^r)^{-1}$
\end{enumerate}

Note that every $\dagger$-monoidal category is a compact $\dagger$-LDC in which the 
the laxors and the involutor are identity transformations. 

\subsection{Finite dimensional framed vector spaces}
\label{subsection:fdfv}

%Our first example is significant as it provides an example of a unitary category which is non-trivial (in the sense that the unitary structure is not given by identity maps).  
In this section we describe the category of ``framed'' finite dimensional vector spaces, where a frame in this context is just a 
choice of basis.  Thus, the objects in this category are vector spaces with a chosen basis while the maps, ignoring the basis, are simply homomorphisms of 
the vector spaces.

The category of finite dimensional framed vectors spaces, ${\sf FFVec}_K$, is a monoidal category defined as follows:
\begin{description}
\item[Objects:]  The objects are pairs $(V,{\cal V})$ where $V$ is a finite dimensional $K$-vector space and ${\cal V} = \{ v_1,...,v_n \}$ is a basis. 
\item[Maps:]  A map $(V,{\cal V}) \xrightarrow{f} (W,{\cal W})$ is a linear map  $V \xrightarrow{f} W$ in ${\sf FdVec}_K$.
\item[Tensor:] $(V,{\cal V}) \ox (W,{\cal W}) = (V \ox W,\{ v \ox w | v \in {\cal V}, w \in {\cal W} \})$ where $V \ox W$ is the usual tensor product.  The unit is 
$(K,\{ e \})$ where $e$ is the unit of the field $K$.
\end{description}

To define the ``dagger'' we  must first choose a conjugation $\overline{(\_)}: K \to K$ (see more details in Section \ref{Sec: conjugation}), that is a field homomorphism with $k = \overline{(\overline{k})}$. The 
canonical example of conjugation is conjugation of the complex numbers, however, the conjugation can be arbitrarily chosen -- so could also, for example, be the identity.  
This conjugation then can be extended to a (covariant) functor:
\[ \overline{(\_)}: {\sf FFVec}_K \to {\sf FFVec}_K; \begin{array}[c]{c} \xymatrix{(V,{\cal V}) \ar[d]^{f} \\ (W,{\cal W})} \end{array} 
                 \mapsto \begin{array}[c]{c} \xymatrix{\overline{(V,{\cal V}) } \ar[d]^{\overline{f}} \\ \overline{(W,{\cal W}) }} \end{array} \]
where $\overline{(V,{\cal V})}$ is the vector space with the same basis but with the conjugate action $c~\overline{\cdot}~v = \overline{c} \cdot  v$.  The conjugate 
homomorphism, $\overline{f}$, is then the same underlying map which is homomorphism between the conjugate spaces.

${\sf FFVec}_K$ is also a compact closed category with $(V,{\cal B})^{*} = (V^{*}, \{ \widetilde{b_i} | b_i \in {\cal B} \})$ where 
\[ V^{*} = V \multimap K~~~~\mbox{and}~~~~\widetilde{b_i}: V \to K; \left(\sum_j \beta_j \cdot b_j \right) \mapsto  \beta_i \]
This makes $(\_)^{*}: {\sf FFVec}_K^{\rm op} \to {\sf FFVec}_K$ a contravariant functor whose action is determined by precomposition.   Finally, we 
define the ``dagger'' to be the composite $(V,{\cal B})^\dagger = \overline{(V,{\cal B})^{*}}$.

This is a monoidal category with tensor and par being identified (so the linear distribution is the associator) and is isomix.  We must show that it is a $\dagger$-LDC.  
Towards this aim we define the required natural transformations on the basis:

\[ \lambda_\ox = \lambda_\oa: (V,{\cal V})^\dag \ox (W,{\cal W})^\dag \to ((V,{\cal V}) \ox (W,{\cal W}))^\dag;  \widetilde{v_i} \ox \widetilde{w_j} \mapsto \widetilde{v_i \ox w_j} \]

\[ \lambda_\top = \lambda_\bot: (K,\{ e\}) \to (K,\{ e\})^\dag; k \mapsto \overline{k} \]
\[ \iota: (V,{\cal V}) \to ((V,{\cal V})^\dag)^\dag; v \mapsto \lambda f. f(v) \]
Note that the last transformation is given in a basis independent manner. Importantly, it may also be given in a basis dependent manner as
$\iota(v_i) = \widetilde{\widetilde{v_i}}$ as the behaviour of these two maps is the  same when applied to the basis of $(V,{\cal V})^\dag$ namely 
the elements $\widetilde{v_j}$:
\[ \iota(v_i)(\widetilde{v_j}) = (\lambda f. f(v_i)) \widetilde{v_j} = \widetilde{v_j} v_i = \partial_{i,j} = \widetilde{\widetilde{v_i}}(\widetilde{v_j})  \]
Also note that $\widetilde{v_i \ox w_j} = (\widetilde{v_i} \ox \widetilde{w_j}) u_\ox$, where $u_\ox: K \ox K \to K$ is the multiplication of the field.
With these definitions in hand it is straightforward to check that this gives a $\dagger$-LDC by checking the required coherences on basis elements.
To demonstrate the technique consider the coherence {\bf [$\dagger$-ldc.4]}:
\[ \xymatrix{A \oa B \ar[d]_{\iota \oa \iota} \ar[rr]^\iota & & ((A \oa B)^\dag)^\dag \ar[d]^{\lambda_\ox^\dag} \\
                   (A^\dag)^\dag \oa (B^\dag)^\dag  \ar[rr]_{\lambda_\oa} & & (A^\dag \ox B^\dag)^\dag} \]
 We must show (identifying tensor and par) that $\lambda_\ox^\dag (\iota(a_i \ox b_j)) = \lambda_\ox(\iota \ox \iota(a_i \ox b_j))$.  Now the result  is a 
 higher-order term so it suffices to show the evaluations on basis elements are the same.  This means we need to show:
 $\lambda_\ox^\dag (\iota(a_i \ox b_j))(\widetilde{a_p} \ox \widetilde{b_q}) = \lambda_\ox(\iota \ox \iota(a_i \ox b_j))(\widetilde{a_p} \ox \widetilde{b_q})$
 \begin{eqnarray*}
(\lambda_\ox(\iota \ox \iota(a_i \ox b_j)))(\widetilde{a_p} \ox \widetilde{b_q}) & = & (\lambda_\ox(\widetilde{\widetilde{a_i}} \ox \widetilde{\widetilde{b_j}}))(\widetilde{a_p} \ox \widetilde{b_q}) \\
&  = & (\widetilde{\widetilde{a_i} \ox \widetilde{b_j}}) (\widetilde{a_p} \ox \widetilde{b_q}) \\
& = & (\widetilde{a_p} \ox \widetilde{b_q})(\widetilde{\widetilde{a_i}} \ox \widetilde{\widetilde{b_j}}) u_\ox ~~~\mbox{(diagrammatic order)}\\
& = & \partial_{p,i} \partial_{q,j} \\
(\lambda_\ox^\dag (\iota(a_i \ox b_j)))(\widetilde{a_p} \ox \widetilde{b_q}) 
& = & (\lambda_\ox^\dag (\widetilde{\widetilde{a_i \ox b_j}}))(\widetilde{a_p} \ox \widetilde{b_q})  \\ 
& = & (\widetilde{a_p} \ox \widetilde{b_q}) \lambda_\ox \widetilde{\widetilde{a_i \ox b_j}} ~~~\mbox{(diagrammatic order)}\\
& = & \widetilde{a_p \ox b_q} \widetilde{\widetilde{a_i \ox b_j}} \\
& = & \partial_{p,i} \partial_{q,j} 
\end{eqnarray*}

Thus, $ {\sf FFVec}_K$ is a compact $\dagger$-isomix category where the $\dagger$ functor shifts objects i.e., $A \neq A^\dagger$.

%%%%%%%%%%%%%%%%%%%%%%%%%%%%%%%%%%%%%%%%%%%%%%%%%%%%%%%%%%%%%%%%%%%%%%%%

\subsection{Category of abstract state spaces}
\label{Sec: Asp}

This source of examples for $\dagger$-isomix categories is inspired by the category of convex operational models \cite{BaW11}. 
The following is a way to construct a new $\dagger$-isomix category, the category of abstract state spaces, from an existing one.

\begin{definition}
	Let $\X$ be a $\dagger$-isomix category. Define $\Asp(\X)$ as follows:
	\begin{description}
		\item[Objects:] $(A, e_A:A \to \bot, u_A: \top \to A)$
		\item[Arrows:]  $f: A \to B \in \X$ such that the following diagram commutes:
		
		$
		\xymatrix{
			& \top \ar[dl]_{u_A}  \ar[dr]^{u_B} & \\
			A \ar[rr]^{f} \ar[dr]_{e_A}  & & B \ar[ld]^{e_B} \\
			& \bot &
		}
		$
	\end{description}
	Identity arrow and composition are inherited directly from $\X$. $\Asp(\X)$ is a LDC:
	\begin{description}
		\item[$\ox$ on objects:] $(A, e_A, u_A) \ox (B, e_B, u_B) := (A \ox B, e', u')$ where, $e' := \mx (e_A \oa e_B) u_\oa$ and $u' := u_\ox^{-1} (u_A \ox u_B)$. The unit of $\ox$ is given by $(\top, \m^{-1}: \top \to \bot, 1_\top)$.
		\item[$\oa$ on objects:] $(A, e_A, u_A) \oa (B, e_B, u_B) := (A \oa B, e', u')$ where, $e' := (e_A \oa e_B) u_\oa$ and $u' := u_\ox^{-1} (u_A \ox u_B) \mx$. The unit of $\oa$ is $(\bot, 1_\bot, \m^{-1}: \top \to \bot)$
	\end{description}
\end{definition}

$\Asp(\X)$ is also $\dagger$-isomix category with \[ (A, e, u)^\dagger := 
(A^\dagger, u^\dagger \lambda_\bot^{-1}, \lambda_\top e^\dagger)\] 
All the basic natural isomorphisms are inherited from $\X$. Hence, $\Asp(\X)$ is a $\dagger$-isomix category.
