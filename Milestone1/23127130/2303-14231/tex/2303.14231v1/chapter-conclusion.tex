% !TEX root = thesis.tex

\chapter{Conclusion and future work}
\label{Chap: conclusion}
\section{Conclusion}

This thesis is a product of our journey towards developing a suitable categorical semantics for quantum 
processes without the constraint of dimensionality. Since its inception, CQM has piqued the interest 
of researchers by its versatile and elegant approach towards studying quantum foundations and 
quantum processes. While CQM was built on compact closed categories - 
categorical semantics of linear logic with the pair of multiplicatives combined into one and the pair of 
additives combined into one - which forces the Hilbert space model to be finite dimensional, there have 
been efforts to address the dimensionality constraint of CQM. 

While one approach was to construct a category which is still compact closed but can suitably describe quantum processes of 
arbitrary dimensions \cite{GG17, HeR18}, the other approach was to construct suitable algebraic structures 
in $\dagger$-SMCs \cite{CoH16, AbH12}.  Attempts have also been made by adding properties 
to the underlying monoidal category so that it accommodates objects of arbitrary dimensional \cite{Heu08,Vic10}. 
In our journey, we stepped out of the well-trodden path of $\dagger$-monoidal categories in CQM, went back to the 
fundamentals, and as category theorists asked, {\em ``Can we consider the semantics of $\dagger$-linear logic that indeed is not degenerate?"} 
This turned our attention towards linearly distributive categories and $*$-autonomous categories (LDCs without negation).

Proceeding to define a $\dagger$ functor for LDCs, and a unitary structure for $\dagger$-isomix categories, we arrived at 
the notion of mixed unitary categories (MUCs) which are $\dagger$-isomix categories with a chosen 
unitary core. As revealed by the schematic diagram of MUC in Figure \ref{Fig: MUC}, the $\dagger$-isomix 
category is a larger space in which the {(smaller)} unitary category resides. For any MUC, its unitary 
core is equivalent to a $\dagger$-monoidal category, and in the presence of unitary duals, it is equivalent to a 
$\dagger$-compact closed category. Thus, one gets a neat description of the traditional CQM framework 
in terms of a larger and a more general framework. We have discussed quite a few examples of MUCs 
as we developed the structure. Significant among these examples are, the category of finiteness relations $\FRel$ 
which has a faithful functor into $\Rel$, and the category of finiteness matrices over of commutative rig $R$, $\FMat(R)$. 
In fact, the category of finite complex matrices $\Mat(\C)$ is isomorphic to the core of $\FMat(\C)$. 

Further in our journey, we tested our framework in its applicability to quantum mechanics by generalizing 
the algebraic structures fundamental to CQM - completely positive maps modeling quantum processes, 
special commutative $\dagger$-Frobenius algebras modeling quantum observables, 
and bialgebras and Hopf algebras modeling complementary observables - from $\dagger$-SMCs to $\dagger$-isomix categories. 
By generalizing the $\CP^\infty$ construction from $\dagger$-SMCs to MUCs, we noticed that, in MUCs, 
interestingly, a completely positive map always factors through the unitary core in its Kraus decomposition form.  
Moreover, in order to characterize the $\CP^\infty$ construction, it suffices that one can discard information 
for unitary objects. Hence, the notion of environment structure, in other words, discarding information, 
is a requirement only within the unitary core. 

Moving on, we turned our focus on {\em linear monoids} \cite{BCST96} in LDCs which are a general version of 
Frobenius algebras: in linear monoids, the monoid and the comonoid pair generally occur on different objects, 
the monoid is on the $\ox$-product, the comonoid is on the $\oa$-product, and the monoid and the comonoid 
objects are dual to one another. We introduced $\dagger$-linear monoids, which in a unitary category are 
equivalent to $\dagger$-Frobenius algebras when the unitary 
structure map satisfies a certain condition. Linear comonoids which are same as linear monoids except with  
the monoid on the $\oa$-product and the comonoid on the $\ox$-product, are a surprising consequence 
of our attempt to understand bialgebraic interaction of linear monoids. A bialgebraic interaction 
between two linear monoids implies a bialgebraic interaction between a $\ox$-monoid and 
a $\oa$-comonoid, which is not supported in an LDC. This led to the idea of linear comonoids. Indeed, 
linear comonoids are not mere algebraic constructs defined for convenience, but they are significant, since 
exponential modalities of linear logic provide a source of examples for linear comonoids. 

A linear monoid and linear comonoid may interact beautifully to produce a $\ox$-bialgebra and a $\oa$-bialgebra, 
we refer to this interaction as a linear bialgebra. At this stage, one can perceive the significance of 
keeping the tensor products (the multiplicatives of linear logic) distinct 
in the framework.  Complementary observables are intimately connected to the notion of measurement. 
In a MUC, we showed that a measurement takes place in two steps: first an arbitrary (non-unitary) type is 
compacted into the unitary core, while inside the unitary core, one can use the traditional machinery of CQM 
to perform a measurement. Interestingly, compaction produces a binary idempotent with certain properties. 
We called this a coring $\dagger$-binary idempotent. Conversely, a coring $\dagger$-binary idempotent 
induces a compaction. 

Finally, turning our attention to complementary systems, we considered such a system in 
$\dagger$-isomix categories as a self-linear bialgebra satisfying certain equations.
These equations implied that the pair of bialgebras are indeed Hopf with a particular anitpode.
We noted that when the $\dagger$-isomix category has free exponential modalities (the exponential operator of  
linear logic which provides for non-linear types with infinite duplication and discarding), every complementary system 
induces a linear bialgebra on the free exponential modalities (in the larger space). Indeed, in such a setting, 
every complementary system arises by splitting a coring $\dag$-binary idempotent 
on the induced $\dag$-linear bialgebra. This is perhaps, the most interesting result in this thesis in relation to 
quantum mechanics. 

Bohr's principle of complementarity \cite{Gri18} states that, due to the wave and particle nature of matter, 
physical properties occur in complementary pairs.  Our result connecting complementarity with exponential modalities  
displays a complementary system as a compaction of a $\dagger$-linear bialgebra in which the 
two dual structures (one pertaining to linear monoid and the other one of the linear comonoid) 
occur separately providing an interesting perspective on Bohr's principle.   

\section{Future work}

The current journey has many interesting further directions, a few of which are discussed below:

\subsection{Models of physical systems}

In this thesis, we have discussed multiple examples of MUCs such as 
 $\FRel$, $\FMat(R)$ and Chu spaces from a mathematical viewpoint.  
The role of these categories in the study of quantum foundations and the
other areas of quantum theory is yet to be explored. In particular, $\FMat(R)$ seems 
to be an interesting candidate to study quantum mechanics since its canonical unitary core is isomorphic to $\FHilb$ 
which is a well-studied model in CQM. Moreover, $\FMat(R)$ is a model of full linear logic and 
comes with the free exponential modalities.  

In \cite{Vic08}, Vicary uses the $!$ exponential modality and a differential map \cite{BCS06} 
to categorify the creation and annihilation operators for a Fock space in $\dagger$-monoidal categories 
with $\dagger$-biproducts. He assumes that the $\dagger$ commutes with the $!$ modality. 
Recall that, in a linear setting, we saw that applying $\dagger$ to $!$ modality gives the $?$ modality. 
Vicary applies the machinery to a category ${\sf Inner}$ with complex inner product spaces of countable 
dimensions and well-defined linear maps to study the state space of Quantum Harmonic Oscillators (QHO). 
However,  the claim that the adjunction between the cofree and the forgetful functor producing the 
$!$-modality is well-defined is left as a conjecture. It has been mentioned that the difficulty lies in proving 
that $\eta$ map of the adjunction is well-defined. It would be interesting to revisit Vicary's ideas 
on categorifying QHO \cite{Vic08}  in a linearly distributive setting, in particular, in $\FMat(R)$ with distinct 
$!$ and $?$  modalities. 

In \cite{Vic08}, Vicary points out that with exponential modality one can model the state space of a QHO, 
however, in order to achieve a categorical description of the dynamics of the system, one needs the 
ability to express differential equations categorically. Differential categories \cite{BCS06, BCL19} which 
are additive symmetric monoidal categories with a coalgebra comodality and a differential combinator 
seems to be a natural candidate satisfying the requirement. \cite{Lem20} emphasizes the relevance of differential 
categories to quantum foundations. In fact, $\FRel$ and $\FMat(R)$ are indeed differential 
categories due to the presence of the free exponential modalities.  This arises a question if can one obtain a complete 
categorical description of quantum harmonic oscillator in differential categories?

Our thesis proved that in a $\dagger$-isomix category, in the presence of free exponential modalities, 
every complementary system inside the canonical unitary core arises from the linear bialgebra induced on the 
free exponential modalities. Free exponential modalities imply the presence of a differential combinator \cite{Fio07, BCL19}
 and provide a categorical description for the state space of a QHO \cite{BPS94, Vic08}. This leads one to wonder 
if there is any interesting connection between quantum harmonic oscillators and complementary observables in physics. 

\subsection{Joyal's Bicompletion construction}

An interesting source of examples for MUC is Joyal's \cite{Joy95, Joy95b} bicompletion 
procedure on monoidal categories. Starting with a $\dagger$-monoidal category, or unitary category, $\C$, 
one can form a MUC $i : \C \to \Lambda(\C)$ by simply bicompleting 
(by adding arbitrary limits and colimits) to the $\dagger$-monoidal category. 
Furthermore, the bicompletion,  $\Lambda(\C)$, is a (non-compact) $\dagger$-isomix category which, 
when the starting point, $\C$, is $\dagger$-compact 
closed, is a $\dagger$-isomix $*$-autonomous category. Bicompleting a monoidal category, $\C$, 
causes its tensor to split into two linearly related tensors products $\ox$ and $\oa$ and 
induces a cofree functor $i: \C \to \Lambda(\C)$ on the category of bicomplete categories and 
bicontinuous functors. The free bicomplete category generated by a single object is a $*$-autonomous 
category \cite[Corollary - Theorem 3]{Joy95b}.

\subsection{Clifford algebras in linear settings}

Clifford algebras\footnote{Note that the term Clifford algebras and geometric algebra are used 
interchangeably in the physics literature. However, Clifford algebras are free geometric algebras 
which satisfy a universal property.}  \cite{Hes12, LuS09, DoL03} are regarded as a universal 
language for physics due to their intimate connection to geometry.  They neatly geometrify 
algebra and algebraize geometry.  Clifford algebras have been applied in many fields of physics including 
quantum gravity \cite{FrK04, DoL07}, quantum field theory \cite{VaR19} and quantum electrodynamics \cite{Bay04}. 

The Clifford algebra of space time captures the geometry of special relativity \cite{Hes12}. 
 In his discussion on categorification of a quantum harmonic oscillator \cite[Sec. Discussion]{Vic08}, Vicary  
notes that within the current CQM formalism,

{\em``\ldots an elegant categorical description of the other ingredients 
of the Schr\"odinger equation, such as Planck’s constant $h$ and the imaginary unit $i$, is far from apparent."}

Clifford algebra addresses the above concern by providing a geometric interpretation for the 
imaginary constant $i$ as rotations in space time. The gamma matrices which are $4 \times 4$ anti-commuting 
unitary matrices solving the Dirac equation  in quantum field theory gives a matrix representation for 
a Clifford algebra, $Cl_{1,3}(\R)$ \cite{VaR19}. The Pauli $I, X, Y, Z$ matrices which are quite significant for 
quantum mechanics and quantum computing provides a representation for the Clifford algebra 
$Cl_{0,3}(\R)$ \cite{DoL03}.  Schr\"odinger's equation can be represented as an element in the Clifford algebra $Cl_{0,1}(\R)$ \cite{HiC10}. 

Clifford algebras enable an interesting possibility of understanding and formalizing quantum computation 
via geometry rather than the other standard non-intuitive methods such as unitary matrices formalism or the 
circuit language method borrowed from classical computing.  This approach is quite different from the 
existing approaches and can provide a fresh  perspective on quantum computation. 
Efforts \cite{SCH98,  Vla99, Vla01, AeC07, Cza07, Lin2021, Lin2021b} to apply 
Clifford algebras to quantum computing in a non-categorical setting 
emphasize the conceptual clarity and computational 
advantages provided by Clifford algebras. In \cite{SCH98} the authors show that the Clifford  
algebra description of quantum computation operations has a direct correlation to NMR spectroscopy, hence
can be implemented in NMR quantum computing without further translation. 
The more recent works \cite{Lin2021, Lin2021b} interesting uses string diagrams of monoidal categories to describe Clifford operations. 

In the future, we would like to formulate abstract Clifford algebras in linear settings. It would also be interesting 
to adapt the string diagrams of CQM to the Clifford algebras.  The elements of a Clifford algebra 
always anti commute. This suggests a connection between these algebras to complementary 
measurements. Can Clifford algebras be used to provide a fresh perspective on quantum computation 
by taking an approach quite different from classical computation? Drawing inspiration from the ZX calculus \cite{CoD11}, can one build a
 universal Clifford calculus that would bridge multiple areas of quantum research?  
Can such a Clifford calculus provide a neat generalization to the ZX calculus to arbitrary dimensional systems?
It would be quite  interesting and highly useful to devise a diagrammatic calculus for the
complicated equations  areas in quantum chemistry and other branches of quantum mechanics. In ZX calculus, the bialgebraic interaction between 
the complementary $Z$ and the $X$ observables is exploited to construct gates for quantum computation. 
The Pauli $X, Y$ and $Z$ matrices along with the identity matrix is a representation for a Clifford algebra. 
This arises a question if there is any interesting connection between bialgebras and Clifford algebras?  
These questions are yet to be explored.  
 

