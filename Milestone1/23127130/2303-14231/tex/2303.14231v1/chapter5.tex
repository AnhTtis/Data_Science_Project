% !TEX root = thesis.tex

\chapter{Mixed unitary categories (MUCs)}
\label{Chap: MUCs}

%%%%%%%%%%%%%%%%%%%%%%%%%%%%%%%%%%%%%%%%%%%%%%%%

The notion of unitary isomorphism is important in categorical quantum mechanics since these 
isomorphisms model the unitary evolution of a quantum system. An isomorphism 
in a $\dagger$-monoidal category is said to be unitary when the inverse of the map 
coincides with its dagger. This idea cannot be directly applied to define 
unitary isomorphisms in $\dagger$-LDCs due to the non-stationary dagger functor ($A \neq A^\dagger$). 
This arises the following question: {\em what are unitary isomorphisms in $\dagger$-LDCs?}
The objective of this chapter is resolve this question and to introduce mixed unitary 
categories (MUCs).   
%To this end, we first introduce unitary categories which are compact $\dagger$-LDCs in which every object 
%isomorphic to its dagger via a {\em unitary structure} map. Any unitary category is $\dagger$-linearly 
%equivalent to a $\dagger$-monoidal category. 

%A mixed unitary category consists of a unitary category, $\U$, with a $\dagger$-isomix Frobenius functor 
%$M: \U \to \C$ into the core of a ``large'' $\dagger$ isomix category $\C$.   We refer to $\U$ as the {\em unitary core\/} of 
%the MUC. The unitary core provides the analogue of scalars for the larger category much as a field provides scalars 
%for an algebra over that field.  

%%%%%%%%%%%%%%%%%%%%%%%%%%%%%%%%%%%%%%%%%%%%%%%%%

\section{Unitary categories}
\label{Sec: unitary}

%In a $\dagger$-isomix category, an object and its dagger are not necessarily equal. However, within the 
%core, the tensor product is isomorphic to the par product, that is, the core 
%is linearly equivalent to a monoidal category. Similarly, in order to accommodate 
%$\dagger$-monoidal categories as a subtheory of $\dagger$-isomix categories, 
%we introduce the notion of unitary structure. 

\subsection{Unitary structure}

 Categorically, within a $\dagger$-monoidal category, 
 a unitary map is an isomorphism $f: A \to B$ such that $f^{-1} =  f^\dagger$. 
This definition of unitary isomorphism cannot be used directly within the framework of $\dagger$-LDCs 
since the types of $f^{-1}: B \to A$ and  $f^\dagger: B^\dagger \to A^\dagger$ are different. 
However, one can define such a unitary isomorphism if, minimally, 
$A \simeq A^\dagger$ and $B \simeq B^\dagger$, and the isomorphisms behave 
coherently with the $\dagger$-linear structure. We call such isomorphisms 
{\em unitary structure} maps and the objects equipped with such isomorphisms 
as {\em unitary objects}:

\begin{definition}
	\label{defn: unitary structure}
A  $\dagger$-isomix category, $\X$ has {\bf unitary structure} in case there is an essentially small class of objects $\mathcal{U}$, called the {\bf unitary objects} of $\X$ such that
\begin{enumerate}[{\bf [U.1]}]
\item for all $A \in \mathcal{U}$, $A \in  \Core(\X)$, and $A$ is equipped with an isomorphism, $\varphi_A: A \to A^\dag$, called the {\bf unitary structure map} of $A$
\item $\mathcal{U}$ is closed under $(\_)^\dag$ so that for all $A \in \mathcal{U}$, $\varphi_{A^\dag} = ((\varphi_A)^{-1})^\dag$ 
\item for all $A \in \mathcal{U}$, the following diagram commutes:
 \[   \xymatrix{  A   \ar[d]_{\varphi_A} \ar[drrr]^{\iota}  & \\ A^\dag \ar[rrr]_{\varphi_{A^\dag}}  & & & (A^\dag)^\dag  } \]
\item $\bot, \top \in \mathcal{U}$ satisfy:
\[ \xymatrixcolsep{2pc}
\xymatrix{
\bot \ar[r]^{\varphi_\bot} \ar[d]_{\lambda_\bot} \ar[dr]^{\m} & \bot^\dagger \ar[d]^{\lambda_\top^{-1}}  \\
\top^\dagger \ar[r]_{\varphi_\top^{-1}} & \top
}
\]
\item If $A , B \in \mathcal{U}$, then $A \ox B$ and $A \oa B \in \mathcal{U}$ satisfy:
\[ (a) ~~~~~ \xymatrixcolsep{3pc}
\xymatrix{
A \ox B \ar[r]^{\varphi_A \ox \varphi_B}_{\simeq} \ar@/_2pc/[rrr]_{\mx}&
 A^\dagger \ox B^\dagger \ar[r]^{\lambda_\oa}_{\simeq} & 
 (A \oa B) ^\dagger \ar[r]^{\varphi_{A \oa B}^{-1}} _{\simeq} &
A \oa B
}
\]
\[ (b) ~~~~~ \xymatrixcolsep{3pc}
\xymatrix{
A \ox B \ar[r]^{\varphi_{A \ox B}}_{\simeq} \ar@/_2pc/[rrr]_{\mx}&
 (A \ox B)^\dagger \ar[r]^{\lambda_\ox^{-1}}_{\simeq} & 
 A^\dagger \oa B^\dagger \ar[r]^{\varphi_A^{-1} \oa \varphi_B^{-1}} _{\simeq} &
A \oa B
}
\]
 \end{enumerate}
\end{definition}

%%%%%%%%%%%%%%%%%%%%%%%%%%%%

\begin{lemma}
\label{Lemma: square root tensor unitary}
When $A$ and $B$ are unitary objects in a $\dagger$-isomix category then, $\varphi_{A^{\dagger\dagger}} = (\varphi_A)^{\dagger \dagger}: A^{\dagger\dagger} \to A^{\dagger \dagger \dagger}$.
\end{lemma}
\begin{proof}~
\[ \varphi_{(A^\dagger)^{\dagger}} = ((\varphi_{A^\dagger})^{-1})^{\dagger} = ((((\varphi_A)^{-1})^\dagger)^{-1})^\dagger = ((((\varphi_A)^{-1})^{-1})^\dagger)^\dagger = ((\varphi_A)^\dagger)^\dagger \]
\end{proof}

%%%%%%%%%%%%%%%%%%

Often we shall want the unitary objects to have linear adjoints (or duals) but we shall need the analogue of $\dagger$-duals $(\eta^\dagger = c_\ox \epsilon$ and $\epsilon^\dagger = \eta c_\ox)$ from categorical quantum mechanics:

\begin{definition} \label{defn: unitary dual}
A {\bf unitary linear duality} $(\eta, \epsilon): A \dashvv_{~u} B$ between unitary objects  $A$ and $B$ is a linear duality satisfying in addition:
\[
\begin{matrix}
\xymatrix{ \\
{\bf [Udual.]} \\
}~~~
\xymatrix{
\top \ar@{}[ddrr]|{(a)} \ar[rr]^{\eta} \ar[d]_{\lambda_\top}  & & A \oa B \ar[d]^{\varphi_A \oa \varphi_B} \\
\bot^\dagger \ar[d]_{\epsilon^\dag} & & A^\dagger \oa B^\dagger \ar[d]^{c_\oa} \\ 
(B \ox A)^\dag \ar[rr]_{\lambda_\oa^{-1}} & & B^\dagger \oa A^\dagger} 
~~~~~ & \text{(or)} & ~~~~~~
\xymatrix{
A \ox B \ar@{}[ddrr]|{(b)} \ar[rr]^{\varphi_A \ox \varphi_B} \ar[d]_{c_\ox} & & A^\dag \ox B^\dag \ar[d]^{\lambda_\ox} \\
B \ox A \ar[d]_{\epsilon} & & (A \oa B)^\dagger \ar[d]^{\eta^\dagger} \\
\bot \ar[rr]_{\lambda_\bot} & & \top^\dagger } 
\end{matrix}
\]
\end{definition}

Observe that ${\bf [Udual.]} (a) \Leftrightarrow (b)$. In a compact $\dagger$-LDC, $\top \dashvv_{~u} \bot$. {\bf [Udual] (a)} is shown diagrammatically as follows:
\[\begin{tikzpicture}
	\begin{pgfonlayer}{nodelayer}
		\node [style=none] (0) at (-2, 4) {};
		\node [style=none] (1) at (1, 4) {};
		\node [style=none] (2) at (-2, 2) {};
		\node [style=none] (3) at (1, 2) {};
		\node [style=circle] (4) at (-0.5, 2.5) {$\epsilon$};
		\node [style=none] (5) at (-1.25, 4) {};
		\node [style=none] (6) at (0.25, 4) {};
		\node [style=none] (7) at (-1.25, 2) {};
		\node [style=none] (8) at (0.25, 2) {};
		\node [style=none] (9) at (-1.25, 1) {};
		\node [style=none] (10) at (0.25, 1) {};
	\end{pgfonlayer}
	\begin{pgfonlayer}{edgelayer}
		\draw [in=150, out=-90, looseness=1.25] (5.center) to (4);
		\draw [in=30, out=-90, looseness=1.25] (6.center) to (4);
		\draw (0.center) to (1.center);
		\draw (1.center) to (3.center);
		\draw (3.center) to (2.center);
		\draw (2.center) to (0.center);
		\draw (7.center) to (9.center);
		\draw (8.center) to (10.center);
	\end{pgfonlayer}
\end{tikzpicture}  = \begin{tikzpicture}
	\begin{pgfonlayer}{nodelayer}
		\node [style=circle] (0) at (0.5, 5.75) {$\eta$};
		\node [style=none] (1) at (-0.5, 4.75) {};
		\node [style=none] (2) at (0, 4.75) {};
		\node [style=none] (3) at (-0.25, 4.5) {};
		\node [style=none] (4) at (1, 4.75) {};
		\node [style=none] (5) at (1.5, 4.75) {};
		\node [style=none] (6) at (1.25, 4.5) {};
		\node [style=none] (7) at (-0.25, 3) {};
		\node [style=none] (8) at (1.25, 3) {};
		\node [style=none] (9) at (-0.25, 4.75) {};
		\node [style=none] (10) at (1.25, 4.75) {};
	\end{pgfonlayer}
	\begin{pgfonlayer}{edgelayer}
		\draw (1.center) to (2.center);
		\draw (2.center) to (3.center);
		\draw (3.center) to (1.center);
		\draw (4.center) to (5.center);
		\draw (5.center) to (6.center);
		\draw (6.center) to (4.center);
		\draw [in=90, out=-150, looseness=1.00] (0) to (9.center);
		\draw [in=90, out=-30, looseness=1.00] (0) to (10.center);
		\draw [in=90, out=-75, looseness=0.75] (6.center) to (7.center);
		\draw [in=90, out=-90, looseness=1.00] (3.center) to (8.center);
	\end{pgfonlayer}
\end{tikzpicture} \]

\begin{lemma}
Suppose $(\eta_1, \epsilon_1): V_1 \dashvv_{~u} U_1$ and $(\eta_2, \epsilon_2): V_2 \dashvv_{~u} U_2$. Then, $(V_1 \otimes V_2) \dashvv_{~u} (U_1 \oa U_2)$.
\end{lemma}
\begin{proof}
Define $(\eta', \epsilon'): (V_1 \otimes V_2) \dashvv_{~u} (U_1 \oa U_2)$ so that  
$\eta' = \begin{tikzpicture} %opluseta
	\begin{pgfonlayer}{nodelayer}
		\node [style=circle] (0) at (-4, 3) {$\eta_1$};
		\node [style=circle] (1) at (-2, 3) {$\eta_2$};
		\node [style=ox] (2) at (-4, 1.75) {};
		\node [style=oa] (3) at (-2, 1.75) {};
		\node [style=none] (4) at (-4, 1) {};
		\node [style=none] (5) at (-2, 1) {};
	\end{pgfonlayer}
	\begin{pgfonlayer}{edgelayer}
		\draw [style=none, in=15, out=-165, looseness=1.00] (1) to (2);
		\draw [style=none, bend left, looseness=1.25] (1) to (3);
		\draw [style=none, in=180, out=-15, looseness=1.00] (0) to (3);
		\draw [style=none, bend left=45, looseness=1.25] (2) to (0);
		\draw [style=none] (2) to (4.center);
		\draw [style=none] (3) to (5.center);
	\end{pgfonlayer}
\end{tikzpicture} ~~~~~~~ 
\epsilon' = \begin{tikzpicture} %oplusepsi
	\begin{pgfonlayer}{nodelayer}
		\node [style=circle] (0) at (-4, 1) {$\epsilon_1$};
		\node [style=circle] (1) at (-2, 1) {$\epsilon_2$};
		\node [style=oa] (2) at (-4, 2.25) {};
		\node [style=ox] (3) at (-2, 2.25) {};
		\node [style=none] (4) at (-4, 3) {};
		\node [style=none] (5) at (-2, 3) {};
	\end{pgfonlayer}
	\begin{pgfonlayer}{edgelayer}
		\draw [style=none, in=-15, out=165, looseness=1.00] (1) to (2);
		\draw [style=none, bend right, looseness=1.25] (1) to (3);
		\draw [style=none, in=180, out=15, looseness=1.00] (0) to (3);
		\draw [style=none, bend right=45, looseness=1.25] (2) to (0);
		\draw [style=none] (2) to (4.center);
		\draw [style=none] (3) to (5.center);
	\end{pgfonlayer}
\end{tikzpicture}
$. This is easily checked to be a unitary linear adjoint.
\end{proof}

We can now define what it means for an isomorphism to be unitary:

\begin{definition}
Suppose $A$ and $B$ are unitary objects. An isomorphism $A\xrightarrow{f} B$ is said to be a {\bf unitary isomorphism} if the following diagram commutes:
\[  \xymatrix{A   \ar[r]^{\varphi_A}    \ar[d]_{f} \ar[r]^{\varphi_A} & A^\dag \\ B  \ar[r]_{\varphi_B} & B^\dag  \ar[u]_{f^\dag}  }  \]
\end{definition}

Observe that $\varphi$ is ``twisted'' natural for all unitary isomorphisms, thus, unitary isomorphisms compose and contain the identity maps. In a category in which the unitary structure maps are identity morphisms, one recovers the usual notion of unitary isomorphisms.

Our next objective is to show that all the coherence isomorphisms between unitary objects are unitary maps. First a warmup: %too poetic ... poetry is good!

\begin{lemma}
\label{lemma:MUCProperties}
In a $\dagger$-isomix category with unitary structure:
\begin{enumerate}[(i)]
\item If $f$ is a unitary isomorphism, then so is $f^\dagger$;
\item If $f$ and $g$ are unitary, then so are $f \ox g$ and $f \oa g$;
\item Unitary isomorphisms are closed under composition.
\end{enumerate}
\end{lemma}

\begin{proof}~
\begin{enumerate}[{\em (i)}]
\item Recall that $\varphi_{A^\dag} = (\varphi_A^{-1})^\dag$,  then $f^\dagger$ is unitary because 
\[ \xymatrix{B^\dag \ar[d]_{(\varphi_B^{-1})^\dag = \varphi_{B^\dag}} \ar[rr]^{f^\dag} & & A^\dag \ar[d]^{(\varphi_A^{-1})^\dag = \varphi_{A^\dag}} \\
   B^{\dag\dag}  & & A^{\dag\dag} \ar[ll]^{f^{\dag\dag}}} \]
is just the dagger functor applied to the unitary diagram of $f$.
\item Suppose $f$ and $g$ are unitary morphisms, then:
\[
\xymatrix{
A \ox B \ar@{->}[rrr]^{\varphi_{A \ox B}} \ar[ddd]_{f \ox g} \ar[dr]_{\mx}  \ar@{}[dddr]|{\mbox{\tiny \bf (nat. $\mx$)}~~~}
&   \ar@{}[dr]|{\mbox{\tiny {\bf [U.5(b)]}}} &  &  (A \ox B)^\dagger \ar@{}[lddd]|{~~~~~\mbox{\tiny \bf (nat. $\lambda_\oa)$}} \\
& A \oa B \ar[r]^{\varphi_A \oa \varphi_B} \ar[d]_{f \oa g} 
& A^\dagger \oa B^\dagger \ar[ur]_{\lambda_\oa}  & 
\\ & A' \oa B' \ar[r]_{\varphi_{A'} \ox \varphi_{B'}} \ar@{}[dr]|{\mbox{\tiny { \bf [U.5(b)]}}}
& A'^\dagger \oa B'^\dagger \ar[dr]^{\lambda_\oa} \ar[u]_{f^\dagger \oa g^\dagger}
& \\ A' \ox B' \ar[rrr]_{\varphi_{A' \ox B'}} \ar[ur]_{\mx}
& & & (A' \ox B')^\dagger \ar[uuu]_{(f \ox g)^\dagger}
}
\]
The inner square commutes because $f$ and $g$ are unitary maps.
Similarly, using {\bf [U.5(b)]}, one can show that if $f$ and $g$ are unitary, then $f \oa g$ is unitary. %unclear

\item
The proof is trivial.
\end{enumerate}
\end{proof}

The following lemma will be used to prove that the natural isomorphisms in a $\dagger$-isomix category are 
unitary for unitary objects.
\begin{lemma}
\label{lemma: auxiliary}
The following diagram commutes:
\[ \xymatrix{
(A \ox B) \ox C \ar[r]^{\mx} \ar[d]_{a_\ox}  & (A \ox B) \oa C  \ar[r]^{\mx \oa 1}  & (A \oa B) \oa C \ar[d]^{a_\oa} \\
A \ox (B \ox C) \ar[r]_{\mx} & A \oa ( B \ox C)  \ar[r]_{1 \oa \mx} & A \oa (B \oa C) }
\]
\end{lemma}
\begin{proof} The given diagram commutes due to the naturality of the mixor, and due to the
	rules governing the interaction of mixor, associator and distributor, see Section \ref{Sec: mix, isomix, compact LDC}. 
\[ \xymatrix{
(A \ox B) \ox C \ar[r]^{\mx} \ar[d]_{a_\ox} \ar@{}[dr]|{{\sf \bf mix}~(b)} & (A \ox B) \oa C \ar@{<-}[d]^{\partial^L}  \ar[r]^{\mx \oa 1} \ar@{}[dr]|{{\sf \bf mix}~(a)}  & (A \oa B) \oa C \ar[d]^{a_\oa} \\
A \ox (B \ox C) \ar[r]_{1 \ox \mx}  \ar@/{_1pc}/[dr]_{\mx}  & A \ox (B \oa C) \ar[r]_{\mx} \ar@{}[d]|{nat. \mx} & A \oa (B \oa C) \\
& A \oa ( B \ox C)  \ar@/{_1pc}/[ur]_{1 \oa \mx}  & } \]
\end{proof}

\begin{lemma}
\label{lemma:cohUnitary}
	
Suppose $\X$ is a $\dagger$-isomix category with unitary structure and 
$A$, $B$, and $C$ are unitary objects. Then the following are unitary isomorphisms:

\begin{multicols}{2}
\begin{enumerate}[(i)]
\item $\lambda_\ox: A^\dagger \ox B^\dagger \rightarrow (A \oa B)^\dagger$
\item $\lambda_\oa:  A^\dagger \oa B^\dagger \rightarrow (A \ox B)^\dagger$
\item $\lambda_\top: \top \rightarrow \bot^\dagger$
\item $\lambda_\bot: \bot \to \top^\dagger$
\item $\varphi_A: A \rightarrow A^\dagger$ 
\item $m: \top \rightarrow \bot$
\item $\mx_{A,B}: A \ox B \rightarrow A \oa B$
\item $\iota : A \rightarrow (A^{\dagger})^\dagger$
\item $a_\ox: (A \ox B) \ox C \rightarrow A \ox (B \ox C)$
\item $a_\oa: (A \oa  B) \oa C \rightarrow A \oa (B \oa C)$
\item $c_\ox: A \ox B \rightarrow B \ox A$
\item $c_\oa: A \oa B \rightarrow B \oa A$
\item $\partial_L: A \ox (B \oa C) \rightarrow (A \ox B) \oa C$
\item $\partial_R: (A \oa B) \ox C \rightarrow A \oa (B \ox C)$
\end{enumerate}
\end{multicols}
\end{lemma}

\begin{proof}~
\begin{enumerate}[(i)]
\item $\lambda_\ox: A^\dagger \ox B^\dagger \rightarrow (A \oa B)^\dagger$ is a unitary map because:


\[\xymatrixcolsep{4pc}\xymatrix{
	{}&&&&\\
	A^\dag\ox B^\dag \ar[r]^{\phi_A^{-1}\ox\phi_B^{-1}} \ar[d]^{\lambda_\ox} \ar@{=}@/^3pc/[rr]     \ar@{}[dr]|{\mbox{\tiny {\bf [U.5(a)] }}}
	 &  A\ox B \ar[r]^{\phi_A\ox \phi_B}  \ar[d]^{\mx} 	                     \ar@{}[dr]|{\mbox{\tiny {\bf nat.}}}
	 &  A^\dag \ox B^\dag \ar[r]^{\phi_{A^\dag \ox B^\dag}} \ar[d]^{\mx}									 \ar@{}[dr]|{\mbox{\tiny {\bf [U.5(a)]}}}
	 &  (A^\dag \ox B^\dag)^\dag \ar[d]_{\lambda_\pr^{-1}} \ar@{=}@/^4pc/[ddd]\\
	(A\pr B)^\dag \ar[r]^{\phi_{A\pr B}^{-1}}  \ar[ddr]_{\phi_{(A\pr B)^\dag}} 								  \ar@{}[dr]|{\mbox{\tiny { \bf [U.3]}}}
	 & A\pr B  \ar[r]^{\phi_A\pr\phi_B} \ar@{=}[d]
	 & A^\dag \pr B^\dag \ar[r]^{\phi_{A^\dag}\pr\phi_{B^\dag}}										 \ar@{}[d]|{\mbox{\tiny { \bf [U.3]}}\ \pr\mbox{\tiny { \bf [U.3]}}}
	 & (A^\dag)^\dag \pr (B^\dag)^\dag  \ar@{=}[d]\\
   {}
     & A \pr B  \ar[rr]^{\iota \pr \iota} \ar[d]^{\iota}
     & {}																					\ar@{}[d]|{\mbox{\tiny {\bf [$\dagger$-ldc.5(a)]}}}
     & (A^\dag)^\dag \pr (B^\dag)^\dag  \ar[d]_{\lambda_\pr}\\
   {}
     & ((A \pr B)^\dag)^\dag  \ar[rr]_{\lambda_\ox^\dag}
     & {}
     & (A^\dag \ox B^\dag)^\dag
}\]

\item $\lambda_\oa$ is unitary because:



\[
\xymatrix{
A^\dag\pr B^\dag                            \ar[rrr]^{\phi_{A^\dag\pr B^\dag}} \ar[dr]^{\mx^{-1}}   \ar[ddd]_{\lambda_\pr} \ar@{}[dddr]|{\mbox{\tiny {\bf Lem. \ref{lemma: mixdagger}}}} 
  &
  &
  &
  (A^\dag\pr B^\dag)^\dag               \ar@{}[dddl]|{\mbox{\tiny {\bf (Lem. \ref{lemma: mixdagger})}}^\dag} \\
{}
  & A^\dag\ox B^\dag                       \ar[r]^{\phi_{A^\dag\ox B^\dag}} \ar[d]_{\lambda_\ox}   \ar@{}[ur]|{\mbox{\tiny {\bf Lem. \ref{lemma:cohUnitary} (vi)}}} \ar@{}[dr]|{\mbox{\tiny {\bf Lem. \ref{lemma:cohUnitary} (i)}}}
  & (A^\dag\ox B^\dag)^\dag           \ar[ur]^{(\mx^{-1})^\dag}
  &\\
{}
  & (A\pr B)^\dag                       \ar[r]^{\phi_{(A\pr B)^\dag}}  \ar@{}[dr]|{\mbox{\tiny {\bf Lems. \ref{lemma:cohUnitary} (vi), \ref{lemma:MUCProperties} (i)}}}
  & ((A\pr B)^\dag)^\dag                \ar[u]_{\lambda_\ox^\dag} \ar[dr]^{((\mx^{-1})^\dag)^\dag}
  &\\
(A\ox B)^\dag                            \ar[rrr]^{\phi_{(A\pr B)^\dag}} \ar[ur]^{(\mx^{-1})^\dag} 
  &
  &
  &
  ((A\ox B)^\dag)^\dag              \ar[uuu]_{\lambda_\pr^\dag}
}
\]

\item $\lambda_\bot: \bot \rightarrow \top^\dagger$ is unitary because:
\[ 
\xymatrix{
\bot \ar[d]_{\lambda_\bot} \ar[rr]^{\varphi_\bot} & & \bot^\dagger \ar[d]^{(\lambda_\bot^{-1})^{\dagger}} \\
\top^\dagger \ar[urr]_{m^\dagger} \ar[rr]_{\varphi_{\top^\dagger} = (\varphi_{\top}^{-1})^\dagger} &  &\top^{\dagger \dagger}
}
\]
The left triangle commutes by {\bf [U.4]} and  {\bf [$\dagger$-mix]}.  The right triangle commutes by {\bf [U.4]} and the functoriality of $\dag$.

\item $\lambda_\top: \top \rightarrow \bot^\dagger$ is unitary because:

\[ 
\xymatrix{
\top \ar[d]_{\lambda_\top} \ar[rr]^{\varphi_\top} & & \top^\dagger \ar[d]^{(\lambda_\top^{-1})^{\dagger}} \\
\bot^\dagger \ar[urr]_{(m^{-1})^\dagger} \ar[rr]_{\varphi_{\bot^\dagger} = (\varphi_{\bot}^{-1})^\dagger} &  &\bot^{\dagger \dagger}
}
\]

The left triangle commutes by {\bf [U.4]} and  {\bf [$\dagger$-mix]}.  The right triangle commutes by {\bf [U.4]} and the functoriality of $\dag$.

\item $\varphi_A$ is unitary because the following square commutes by {\bf [U.3]} and {\bf [U.4]}.
\[
\xymatrix{
A \ar[r]^{\varphi_A} \ar[d]_{\varphi_A} & A^\dagger \ar[d]^{(\varphi^{-1})^\dagger} \\
A^\dagger \ar[r]^{\varphi_{A^\dagger}} & A^{\dagger \dagger}
}
\]

\item $m: \bot \rightarrow \top$ is unitary because:
\[
\xymatrix{
\bot \ar[r]^{\varphi_\bot} \ar[d]_{\m} & \bot^\dagger \ar[d]^{(\m^{-1})^\dagger} \\
\top \ar[r]_{\varphi_\top} \ar[ur]^{\lambda_\top} & \top^\dagger
}
\]
The left and right triangles commute by {\bf [U.4]} and {\bf [$\dagger$-mix]} respectively. Hence, the outer squares commutes.

\item $\mx_{A,B}: A \ox B \rightarrow A \oa B$ is unitary as:
\[ \xymatrix{
{}
  &
  &
  &\\
A \ox B \ar[dd]_\mx\ar@/^2.5pc/[rrr]^{\varphi_{A \ox B}} \ar[r]^\mx \ar@/_1.5pc/[drr]_{\varphi_A \ox \varphi_B}  \ar@{}[drr]|{\mbox{\tiny {\bf nat.}}}   \ar@{}[urrr]|{\mbox{\tiny {\bf [U.5(a)]}}}  
  & A \oa B \ar[r]^{\varphi_A \oa \varphi_B} 
  & A^\dag \oa B^\dag \ar[r]^{\lambda_\oa}  
  & (A \ox B)^\dag \\
{}
  &
  & A^\dag \ox B^\dag \ar[u]^\mx \ar[dr]^{\lambda_\ox}   \ar@{}[ur]|{\mbox{\tiny {\bf Lem. \ref{lemma: mixdagger}}}} \\
A \oa B \ar[rrr] _{\varphi_{A\oa B}}    \ar@{}[urr]|{\mbox{\tiny {\bf [U.3]}}} 
  &
  &
  & (A \oa B)^\dag \ar[uu]_{\mx^\dag}
}
\]
                    
\item $\iota: A \rightarrow A^{\dagger \dagger}$ is unitary as in
\[
\xymatrix{
A \ar[d]_{\iota} \ar[r]^{\varphi_A} & A^\dagger \ar[d]^{(\iota^{-1})^\dagger} \ar[ld]_{\varphi_{A^\dagger}} \\ 
A^{\dagger \dagger} \ar[r]_{\varphi_{A^{\dagger \dagger}}} & A^{\dagger \dagger \dagger}
}
\] 
the left triangle commutes by {\bf [U.3]} and the right triangle commutes by: 
\begin{eqnarray*}
(\iota^{-1})^\dagger &= & ((\varphi_{A^\dagger})^{-1} \varphi_A^{-1})^\dagger =  (((\varphi_{A}^{-1})^\dagger)^{-1} \varphi_A^{-1})^\dagger \\
                         & =  & ((\varphi_A^\dagger)(\varphi_A^{-1}) )^\dagger = ( \varphi_A^{-1} )^{\dagger} ( \varphi_A )^{\dagger \dagger} \\
                         & = &  \varphi_{A^\dagger} (\varphi_A)^{\dagger \dagger} = \varphi_{A^\dagger} (\varphi_{A^{\dagger \dagger}})
\end{eqnarray*}

\item $a_\ox$ is unitary as:
%\[
%\xymatrix{
%A \ox (B \ox C)                                           \ar[rrr]^{\varphi_{A \ox (B \ox C)}} \ar[ddddd]_{a_\ox} \ar[dr]_{\mx}   \ar@{}[drrr]|{\mbox{\tiny {\bf [U.3]}}} 
% & {} 
% & {}
% & ( A \ox (B \ox C) )^\dagger                      \ar[ddddd]^{(a_\ox^{-1})^\dagger} \\ %1
% & {}
%A \oa (B \ox C)                                              \ar[r]^{\varphi_A \pr \varphi_{B\ox C}} \ar[d]_{1 \oa \mx}   \ar@{}[dr]|{\mbox{\tiny {\bf (id)$\pr$ [U.6]}}} 
% & A^\dagger \oa (B \ox C)^\dagger              \ar[d]^{1 \oa \lambda_\ox}   \ar[ur]_{\lambda_\oa}   \ar@{}[dddr]|{\mbox{\tiny {\bf [\dag-ldc.i]}}} 
% & {}  \\ %2
%{}
% & A \oa (B \oa C)                                            \ar[r]_{\varphi_A \oa (\varphi_B \oa \varphi_C)} \ar[d]_{a_\oa}  \ar@{}[dr]|{\mbox{\tiny {nat.}}} 
% & A^\dagger \oa (B^\dagger \oa C^\dagger)  \ar[d]^{a_\oa} 
% & {} \\ %3
%{}
% & (A \oa B) \oa C                                            \ar[r]_{(\varphi_A \oa \varphi_B) \oa \varphi_C}   \ar@{}[dr]|{\mbox{\tiny {\bf [U.6]$\pr$ (id)}}} 
% &  (A^\dagger \oa B^\dagger) \oa C^\dagger     \ar[d]^{\lambda_\oa \oa 1}
% & {} \\ %4
%{}
% & (A \ox B) \oa C                                          \ar[r]_{\varphi_{A\ox B} \oa \varphi_C} \ar[u]^{\mx \oa 1}
% & (A \ox B)^\dagger \oa C^\dagger               \ar[dr]^{\lambda_\oa} 
% & {} \\ %5
%(A \ox B) \ox C                                          \ar[rrr]^{\varphi_{(A \ox B) \ox C}} \ar[ur]_{\mx}   \ar@{}[urrr]|{\mbox{\tiny {\bf [U.3]}}} 
% & {}
% & {}
% & ( (A \ox B) \ox C)^\dagger %6
%}
%\]
%
%

\[
\xymatrix{
(A \ox B) \ox C                                           \ar[rrr]^{\varphi_{(A \ox B) \ox C}} \ar[ddddd]_{a_\ox} \ar[dr]_{\mx}   \ar@{}[drrr]|{\mbox{\tiny {\bf [U.5(b)]}}} \ar@{}[dddddr]|{\mbox{\tiny \bf Lem.~ \ref{lemma: auxiliary}}}
 & {} 
 & {}
 & ( (A \ox B) \ox C )^\dagger                     \\ %1
 & {}
(A \ox B) \pr C                                             \ar[r]^{\varphi_{A\ox B} \pr \varphi_{C}} \ar[d]_{\mx\pr 1}   \ar@{}[dr]|{\mbox{\tiny {\bf  [U.5(b)]$\pr$(id)}}} 
 & (A \ox B)^\dag \pr C^\dag                               \ar[d]^{ \lambda_\oa^{-1}\pr 1}   \ar[ur]_{\lambda_\oa}   \ar@{}[dddr]|{\mbox{\tiny {\bf [\dag-ldc.1]}}} 
 & {}  \\ %2
{}
 & (A \pr B) \pr C                                            \ar[r]_{(\varphi_A \oa \varphi_B) \oa \varphi_C} \ar[d]_{a_\oa}  \ar@{}[dr]|{\mbox{\tiny {\bf nat.}}} 
 & (A^\dag \pr B^\dag) \pr C^\dag                         \ar[d]^{a_\oa} 
 & {} \\ %3
{}
 & A \pr (B \pr C)                                           \ar[r]_{\phi_A \pr (\phi_B \pr \phi_C)}   \ar@{}[dr]|{\mbox{\tiny {\bf (id)$\pr$[U.5(b)] }}} 
 & A^\dag \pr (B^\dag \pr C^\dag)       \ar[d]^{1\oa\lambda_\oa}
 & {} \\ %4
{}
 & A \pr (B \ox C)                                          \ar[r]_{\varphi_{A} \pr \varphi_{B\ox C}} \ar[u]^{1\pr \mx}
 & A^\dag \pr (B \ox C)^\dag               \ar[dr]^{\lambda_\oa} 
 & {} \\ %5
A \ox (B \ox C)                                          \ar[rrr]^{\varphi_{A \ox (B \ox C)}} \ar[ur]^{\mx}   \ar@{}[urrr]|{\mbox{\tiny {\bf [U.5(b)]}}} 
 & {}
 & {}
 & ( A \ox (B \ox C)  )^\dagger \ar[uuuuu]_{a_\ox^\dag} %6    
}
\]

\item $a_\oa$ is unitary because:

\[
\xymatrix{
(A \pr B) \pr C                                           \ar[rrr]^{\varphi_{(A \pr B) \pr C}} \ar[ddddd]_{a_\pr} \ar[dr]_{\mx^{-1}}   \ar@{}[drrr]|{\mbox{\tiny {\bf [U.5(a)]}}}  \ar@{}[dddddr]|{\mbox{\tiny \bf Lem.~ \ref{lemma: auxiliary}}} 
 & {} 
 & {}
 & ( (A \pr B) \pr C )^\dagger                     \\ %1
 & {}
(A \pr B) \ox C                                             \ar[r]^{\varphi_{A\pr B} \ox \varphi_{C}} \ar[d]_{\mx^{-1}\ox 1}   \ar@{}[dr]|{\mbox{\tiny {\bf  [U.5(a)]$\ox$(id)}}} 
 & (A \pr B)^\dag \ox C^\dag                               \ar[d]^{ \lambda_\ox^{-1}\ox 1}   \ar[ur]_{\lambda_\pr}   \ar@{}[dddr]|{\mbox{\tiny {\bf [\dag-ldc.1]}}} 
 & {}  \\ %2
{}
 & (A \ox B) \ox C                                            \ar[r]_{(\varphi_A \pr \varphi_B) \pr \varphi_C} \ar[d]_{a_\ox}  \ar@{}[dr]|{\mbox{\tiny {\bf nat.}}} 
 & (A^\dag \ox B^\dag) \ox C^\dag                         \ar[d]^{a_\ox} 
 & {} \\ %3
{}
 & A \ox (B \ox C)                                           \ar[r]_{\phi_A \ox (\phi_B \ox \phi_C)}   \ar@{}[dr]|{\mbox{\tiny {\bf (id)$\ox$[U.5(a)] }}} 
 & A^\dag \ox (B^\dag \ox C^\dag)       \ar[d]^{1\ox\lambda_\ox}
 & {} \\ %4
{}
 & A \ox (B \pr C)                                          \ar[r]_{\varphi_{A} \ox \varphi_{B\pr C}} \ar[u]^{1\ox \mx^{-1}}
 & A^\dag \ox (B \oa C)^\dag               \ar[dr]^{\lambda_\ox} 
 & {} \\ %5
A \oa (B \oa C)                                          \ar[rrr]^{\varphi_{A \oa (B \oa C)}} \ar[ur]^{\mx^{-1}}   \ar@{}[urrr]|{\mbox{\tiny {\bf [U.5(a)]}}} 
 & {}
 & {}
 & ( A \oa (B \oa C)  )^\dagger \ar[uuuuu]_{a_\oa^\dag} %6    
}
\]


\item $c_\ox$ is unitary because:

\[
\xymatrix{
A\ox B                            \ar[rrr]^{\phi_{A\ox B}} \ar[dr]^{\mx} \ar[ddd]_{c_\ox}
 & {}                                  \ar@{}[dr]|{\mbox{\tiny {\bf [U.5(b)]}}}
 & {}
 & (A\ox B)^\dag              \\
{}                                    
 & A \pr B                        \ar[r]^{\phi_A \pr \phi_B} \ar[d]_{c_\pr}   \ar@{}[dr]|{\mbox{\tiny {\bf nat.}}}
 & A^\dag \pr B^\dag       \ar[d]^{c_\pr} \ar[ur]^{\lambda_\pr}           \ar@{}[dr]|{\mbox{\tiny {\bf [$\dagger$-ldc.2(b)]}}}
 & {} \\
{}
 & B \pr A                       \ar[r]^{\phi_B \pr \phi_A}
 & B^\dag \pr A^\dag      \ar[dr]^{\lambda_\pr}
 & {} \\
B\ox A                           \ar[ur]^{\mx} \ar[rrr]^{\phi_{B\ox A}}
& {}                                 \ar@{}[ur]|{\mbox{\tiny {\bf [U.5(b)]}}}
& {}
& (B\ox A)^\dag                 \ar[uuu]_{(c_\ox^{-1})^\dag = c_\ox^\dag}\\
}
\]

where the left square commutes because

$$
\begin{tikzpicture}
	\begin{pgfonlayer}{nodelayer}
		\node [style=circ] (0) at (0.5, -0.25) {};
		\node [style=circ] (1) at (0, -1) {$\top$};
		\node [style=map] (2) at (0, -1.75) {};
		\node [style=circ] (3) at (0, -2.5) {$\bot$};
		\node [style=circ] (4) at (-0.5, -3.25) {};
		\node [style=none] (5) at (0.5, -3.25) {};
		\node [style=none] (6) at (-0.5, -4.25) {};
		\node [style=none] (7) at (0.5, -4.25) {};
		\node [style=none] (8) at (-0.5, 0.5) {};
		\node [style=none] (9) at (0.5, 0.5) {};
	\end{pgfonlayer}
	\begin{pgfonlayer}{edgelayer}
		\draw [densely dotted, in=-90, out=45, looseness=1.00] (4) to (3);
		\draw (3) to (2);
		\draw (2) to (1);
		\draw [densely dotted, in=-135, out=90, looseness=1.00] (1) to (0);
		\draw [style=none] (0) to (9);
		\draw [style=none] (8) to (4);
		\draw [style=none] (0) to (5);
		\draw [style=none, in=90, out=-90, looseness=1.00] (5) to (6);
		\draw [style=none, in=90, out=-90, looseness=1.00] (4) to (7);
	\end{pgfonlayer}
\end{tikzpicture}
=
\begin{tikzpicture}
	\begin{pgfonlayer}{nodelayer}
		\node [style=circ] (0) at (0, -1.25) {$\top$};
		\node [style=map] (1) at (0, -2) {};
		\node [style=circ] (2) at (0.5, -3.5) {};
		\node [style=circ] (3) at (0, -2.75) {$\bot$};
		\node [style=circ] (4) at (-0.5, -0.5) {};
		\node [style=none] (5) at (0.5, -0.5) {};
		\node [style=none] (6) at (-0.5, 0.5) {};
		\node [style=none] (7) at (0.5, 0.5) {};
		\node [style=none] (8) at (0.5, -4.25) {};
		\node [style=none] (9) at (-0.5, -4.25) {};
	\end{pgfonlayer}
	\begin{pgfonlayer}{edgelayer}
		\draw [densely dotted, in=-90, out=150, looseness=1.25] (2) to (3);
		\draw (3) to (1);
		\draw (1) to (0);
		\draw [densely dotted, in=-45, out=90, looseness=1.00] (0) to (4);
		\draw [style=none] (8) to (2);
		\draw [style=none] (9) to (4);
		\draw [style=none, in=-90, out=90, looseness=1.00] (4) to (7);
		\draw [style=none, in=-90, out=90, looseness=1.00] (5) to (6);
		\draw [style=none] (5) to (2);
	\end{pgfonlayer}
\end{tikzpicture}
=
\begin{tikzpicture}
	\begin{pgfonlayer}{nodelayer}
		\node [style=circ] (0) at (0.5, -0.25) {};
		\node [style=circ] (1) at (0, -1) {$\top$};
		\node [style=map] (2) at (0, -1.75) {};
		\node [style=circ] (3) at (0, -2.5) {$\bot$};
		\node [style=circ] (4) at (-0.5, -3.25) {};
		\node [style=none] (5) at (-0.5, -4) {};
		\node [style=none] (6) at (0.5, -4) {};
		\node [style=none] (7) at (-0.5, 0.75) {};
		\node [style=none] (8) at (0.5, 0.75) {};
		\node [style=none] (9) at (-0.5, -0.25) {};
	\end{pgfonlayer}
	\begin{pgfonlayer}{edgelayer}
		\draw [densely dotted, in=-90, out=45, looseness=1.00] (4) to (3);
		\draw (3) to (2);
		\draw (2) to (1);
		\draw [densely dotted, in=-135, out=90, looseness=1.00] (1) to (0);
		\draw [style=none] (6) to (0);
		\draw [style=none] (5) to (4);
		\draw [style=none] (4) to (9);
		\draw [style=none, in=-90, out=90, looseness=1.00] (9) to (8);
		\draw [style=none, in=-90, out=90, looseness=1.00] (0) to (7);
	\end{pgfonlayer}
\end{tikzpicture}
$$

\item $c_\pr$ is unitary because:


\[
\xymatrix{
A\pr B                            \ar[rrr]^{\phi_{A\pr B}}  \ar[dr]^{\mx^{-1}}   \ar[ddd]_{c_\pr}  \ar@{}[drrr]|{\mbox{\tiny {\bf Lem. \ref{lemma:cohUnitary} (vii)}}}
  &
  &
  & (A\pr B)^\dag \\
{}
  & A \ox B                      \ar[r]^{\phi_{A\ox B}} \ar[d]_{c_\ox}    \ar@{}[dr]|{\mbox{\tiny {\bf Lem. \ref{lemma:cohUnitary} (xi)}}}
  & (A\ox B)^\dag     \ar[ur]^{(\mx^{-1})^\dag} &\\
{}
  & B \ox A                      \ar[r]^{\phi_{B\ox A}} 
  & (B \ox A)^\dag    \ar[dr]^{(\mx^{-1})^\dag}  \ar[u]_{c_\ox^\dag} &\\
B\pr A                            \ar[rrr]^{\phi_{B\pr A}}   \ar[ur]^{\mx^{-1}}   \ar@{}[urrr]|{\mbox{\tiny {\bf Lem. \ref{lemma:cohUnitary} (vii)}}}
  &
  &
  & (B\pr A)^\dag              \ar[uuu]_{c_\pr^\dag} 
  }
\]

where the left square commutes for the same reason and the right square is the dagger of the left square.

\item $\partial_L$ is unitary see Figure \ref{Fig: linear dist. unitary}.

\begin{figure}
\newpage


\begin{sideways}
\scalebox{.84}{
$
\xymatrix{
{}
& {}
& {}
& {}
& {}
& {}
& {}
& {}
& {}\\
{}
& {}
& {}
& {}
& {}
& {}
& {}
& {}
& {}\\
{}
%%%%%%%%%%%%%%%%%%%%%%%%%
%                              0                                     %
%%%%%%%%%%%%%%%%%%%%%%%%%
 & {}
 & A\ox (B \pr C)                      \ar[rr]^{\phi_{A}\ox \phi_{B\pr C}}    \ar[d]_{\mx}       \ar@/^4pc/[rrrrrr]^{\phi_{A\ox (B\pr C)}} \ar@/_6pc/[dddddd]_{\partial_L}
 & {}                                                                                                                                          \ar@{}[d]|{\mbox{\tiny {\bf [U.6(b)]}}}
 & A^\dag \ox (B \pr C)^\dag            \ar@=[r]   \ar@=[d]                                                                \ar@{}[dr]|{\mbox{\tiny {\bf id}}}
 & A^\dag \ox (B \pr C)^\dag            \ar[r]^{\mx}    \ar[d]^{1 \ox \lambda_\ox^{-1} }     \ar@{}[dr]|{\mbox{\tiny {\bf nat.}}}    \ar@{}[u]|{\mbox{\tiny {\bf [U6.(b)]}}}  
 & A^\dag \pr (B \pr C)^\dag            \ar[rr]^{\lambda_\pr}    \ar[d]^{1 \pr \lambda_\ox^{-1}}                                              \ar@{}[ddrr]|{\mbox{\tiny {\bf [$\dagger$-ldc.4(b)]}}}  
 & {}
 & (A\ox (B\pr C))^\dag                         \\
%%%%%%%%%%%%%%%%%%%%%%%%%
%                              1                                     %
%%%%%%%%%%%%%%%%%%%%%%%%%
{}
 & {}
 & A\pr (B \pr C)                      \ar[r]^{\phi_{A\pr (B \pr C)}}     \ar[d]_{a_\pr^{-1}}                                \ar@{}[dr]|{\mbox{\tiny {\bf Lem. \ref{lemma:cohUnitary} (ix)}}}
 & (A\pr (B\pr C))^\dag               \ar[r]^{\lambda_\ox^{-1}}      \ar[d]^{(a_\pr)^\dag}
 & A^\dag \ox (B\pr C)^\dag           \ar[r]^{1\ox \lambda_\ox^{-1}}                                                 \ar@{}[d]|{\mbox{\tiny {\bf [$\dagger$-ldc.1(b)]}}}
 & A^\dag \ox (B^\dag \ox C^\dag)     \ar[r]^{\mx }    \ar[d]^{a_\ox^{-1}}                                                 \ar@{}[dr]|{\mbox{\tiny {\bf mx.~cat.}}}
 & A^\dag \pr (B^\dag \ox C^\dag)             %   \ar[d]^{\partial_R}
 & {}
 & {}\\
%%%%%%%%%%%%%%%%%%%%%%%%%
%                              2                                     %
%%%%%%%%%%%%%%%%%%%%%%%%%
{}
 & {}
 & (A\pr B)\pr C                       \ar[r]^{\phi_{(A\pr B)\pr C}}     \ar@=[d]
 & ((A\pr B)\pr C)^\dag                \ar[r]^{\lambda_\ox^{-1}}                                                            \ar@{}[d]|{\mbox{\tiny {\bf [U.6(a)]}}}
 & (A\pr B)^\dag \ox C^\dag           \ar[r]^{\lambda_\ox^{-1}\ox 1}    \ar[d]_{\mx}                             \ar@{}[dr]|{\mbox{\tiny {\bf nat.}}}
 & (A^\dag \ox B^\dag) \ox C^\dag      \ar[r]^{ \mx\ox 1}     \ar[d]^{\mx}                                            \ar@{}[dr]|{\mbox{\tiny {\bf nat.}}}
 & (A^\dag \pr B^\dag) \ox C^\dag                 \ar[d]^{\mx}\ar@=[r]          \ar[u]_{\partial_R}                                          \ar@{}[ddddr]|{\mbox{\tiny {\bf nat.}}} 
 & (A^\dag \pr B^\dag) \ox C^\dag                 \ar[dddd]^{\lambda_\pr\ox 1}
 & {}\\
%%%%%%%%%%%%%%%%%%%%%%%%%
%                              3                                     %
%%%%%%%%%%%%%%%%%%%%%%%%%
{}
 & {}
 & (A\pr B)\pr C                      \ar[rr]^{\phi_{A\pr B}\pr \phi_{C}}     \ar[d]_{\mx^{-1} \oa 1}                     \ar@{}[ll]|{\mbox{\tiny {\bf mx ~ cat.}}}     \ar@{}[drr]|{\mbox{\tiny{\bf (id)$\ox$(Lem. \ref{lemma:cohUnitary}   (vii))} } }
 & {}                                                                                                                                              
 & (A\pr B)^\dag \pr C^\dag           \ar[r]^{\lambda_\ox^{-1}\pr 1}     \ar[d]_{\mx^\dag\pr 1}         \ar@{}[dr]|{\mbox{\tiny {\bf 1$\ox$(Lem. \ref{lemma: mixdagger})  }}}
 & (A^\dag \ox B^\dag) \pr C^\dag                \ar[d]^{\mx\pr 1}  \ar[r]^{\mx\pr 1}                          \ar@{}[ddr]|{\mbox{\tiny {\bf id}}}
 & (A^\dag \pr B^\dag) \pr C^\dag                \ar@=[dd]
 & {}
 & {}\\
%%%%%%%%%%%%%%%%%%%%%%%%%
%                              4                                     %
%%%%%%%%%%%%%%%%%%%%%%%%%
{}
 & {}
 & (A\ox B)\oa C                       \ar[rr]^{\phi_{A\pr B}\oa \phi_{C}}    \ar@=[d]                                    
 & {}                                                                                                                                              \ar@{}[d]|{\mbox{\tiny {\bf [U.6(a)] }}}
 & (A\ox B)^\dag \oa C^\dag           \ar[r]^{\lambda_\ox^{-1}\pr 1}     \ar[d]_{\lambda_\ox}            \ar@{}[dr]|{\mbox{\tiny {\bf id}}}
 & (A^\dag\oa B^\dag) \oa C^\dag                \ar[d]^{\lambda_\oa \oa 1}
 & {}
 & {}
 & {}\\
%%%%%%%%%%%%%%%%%%%%%%%%%
%                              5                                     %
%%%%%%%%%%%%%%%%%%%%%%%%%
{}
 & {}
 & (A\ox B)\oa C                       \ar[r]^{\mx^{-1}}     \ar@=[d]
 & (A\pr B)\ox C                       \ar[r]^{\phi_{(A\pr B)\pr C}}                                                            \ar@{}[dr]|{\mbox{\tiny {\bf [U.6(a)]}}}
 & ((A\ox B) \ox C)^\dag                \ar[r]^{\lambda_\ox^{-1}}
 & (A\ox B)^\dag \oa C^\dag           \ar[r]^{ \lambda_\oa^{-1} \ox 1}     \ar@=[d]                               \ar@{}[dr]|{\mbox{\tiny {\bf id}}}
 & (A^\dag \oa B^\dag) \oa C^\dag       \ar[d]^{\lambda_\oa \ox 1}
 & {}
 & {}\\
%%%%%%%%%%%%%%%%%%%%%%%%%
%                              6                                     %
%%%%%%%%%%%%%%%%%%%%%%%%%
{}
 & {}
 & (A\ox B) \oa C                      \ar[rrr]^{\phi_{A\ox B} \oa \phi_{C}}                           \ar@/_4pc/[rrrrrr]_{\phi_{(A\ox B) \oa C}}
 & {}
 & {}
 & (A \ox B)^\dag \oa C^\dag          \ar@=[r]                                                                                   \ar@{}[d]|{\mbox{\tiny {\bf [U.6(a)]}}}
 & (A \ox B)^\dag \oa C^\dag          \ar[r]^{\mx^{-1}}
 & (A \ox B)^\dag \ox C^\dag          \ar[r]^{\lambda_\ox}
 &  ((A \ox B) \oa C)^\dag               \ar[uuuuuu]^{\partial_L^\dag}\\
{}
 & {}
 & {}
 & {}
 & {}
 & {}
 & {}
 & {}
 & {}\\
}
$
}

\end{sideways}
\caption{$\partial_L$ is a unitary isomorphism}
\label{Fig: linear dist. unitary}
\newpage
\end{figure}

\item $\partial_R$ is unitary because:

\[ \hspace{-1.25cm} 
\xymatrixrowsep{4pc}
\xymatrix{
{}
  & {}
  & {}
  & {}
  & {}
  & {}\\
(\!A\!\pr\! B\!) \!\ox\! C                           \ar[r]^{\mx} \ar[d]_{\partial_R}    \ar@/^4pc/[rrrrr]^{\phi_{(A\pr B) \ox C }}
  & (\!A\!\pr\! B\!) \!\pr\! C                      \ar[r]^{\mx^{-1}\pr 1}                                                        \ar@{}[d]|{\mbox{\tiny {\bf }}}
  & (\!A\!\ox\! B\!) \!\pr\! C                      \ar[r]^{\phi_{(A\ox B) \pr C}}                                              \ar@{}[dr]|{\mbox{\tiny {\bf Lem. \ref{lemma:cohUnitary}  (xiii)}}}  \ar@{}[ur]|{\mbox{\tiny {\bf Lem. \ref{lemma:cohUnitary} (vii), \ref{lemma:MUCProperties}}}}
  & (\!(\!A\!\ox\! B\!) \!\pr\! C \!)^\dag         \ar[r]^{(\mx^{-1}\pr 1 )^\dag} \ar[d]_{\partial_L^\dag}   
  & (\!(\!A\!\pr\! B\!) \!\pr\! C \!)^\dag         \ar[r]^{\mx^\dag}                                                               \ar@{}[d]|{\mbox{\tiny {\bf }}}
  & (\!(\!A\!\pr\! B\!) \!\ox\! C \!)^\dag \\
%%%%%%%%%
A\!\pr\! (\!B \!\ox\! C\!)                           \ar[r]^{\mx^{-1}}   \ar@/_4pc/[rrrrr]_{\phi_{A\pr (B \ox C) }}
  & A\!\pr\! (\!B \!\pr\! C\!)                      \ar[r]^{1 \pr \mx} 
  & A\!\ox\! (\!B \!\pr\! C\!)                      \ar[r]^{\phi_{A\ox (B \pr C)}} \ar[u]_{\partial_L}  \ar@{}[dr]|{\mbox{\tiny {\bf Lem. \ref{lemma:cohUnitary}  (vii), \ref{lemma:MUCProperties}}}}
  & (\!A\!\ox\! (\!B\!\pr\! C\!) \!)^\dag         \ar[r]^{(\mx \pr 1 )^\dag}
  & (\!A\!\pr\! (\!B \!\pr\! C\!) \!)^\dag         \ar[r]^{(\mx^{-1})^\dag}
  & (\!A\!\pr\! (\!B \!\ox\! C\!) \!)^\dag        \ar[u]_{\partial_R^\dag}\\
{}
  & {}
  & {}
  & {}
  & {}
  & {}
}
\]
\end{enumerate}
\end{proof}


%%%%%%%%%%%%%%%%%%%%%%%%%%%%%%%%%%%%%%%%%%%%%%

\subsection{Unitary categories}
\label{subsection: unitary categories}

With the notion of unitary objects in place, one can consider $\dagger$-isomix categories in which all the objects are unitary: 
these are called {\em unitary categories\/}. This section develops the theory of unitary categories.

\begin{definition}
A {\bf (symmetric) unitary category} is a (symmetric) $\dagger$-isomix category with a unitary structure for which every object is unitary.
\end{definition}

Clearly, a unitary category must be a compact $\dagger$-LDC, since the mixor is a unitary isomorphism, see Lemma \ref{lemma:cohUnitary}-(iii).

 A $\dagger$-monoidal category is a strict unitary category in which the unitary structure map and the mix map are identity morphisms. Similarily, a $\dagger$-compact closed category is a strict unitary category in which all objects have unitary duals.

 In the rest of this subsection, we show that any unitary category is $\dagger$-linearly equivalent to a conventional dagger monoidal category. 
 A unitary category being a compact LDC is linearly equivalent, using ${\sf Mx}^*_\uparrow: (\X, \ox,\oa) \to 
 (\X,\oa,\oa)$ (see Corollary \ref{compact-mix-functor}) to the underlying monoidal category based on the 
 par (and the tensor). We now show that for a unitary category one can induce a stationary on objects dagger 
 on $(\X,\oa,\oa)$. We denote this dagger by $(\_)^\ddagger$ and define it by $f^\ddagger := 
 \varphi_Bf^\dagger\varphi_A^{-1}$ as illustrated by the left diagram below:
 
 \[ \xymatrix{ B \ar[d]_{\varphi_B}\ar[rr]^{f^\ddagger} \ar@{}[rrd]|{:=}& & A \ar[d]^{\varphi_A} \\
 	B^\dagger \ar[rr]_{f^\dagger} && A^\dagger} ~~~~~~~~~~~~~
 \xymatrix{ A \ar@/_1pc/[dd]_{\iota} \ar[d]^{\varphi_A}\ar[rr]^{f^{\ddagger\ddagger}} & & B \ar[d]_{\varphi_B} \ar@/^1pc/[dd]^{\iota} \\	
 	A^\dagger \ar[d]^{(\varphi_A^{-1})^\dagger} \ar[rr]_{(f^\ddagger)^\dagger} && B^\dagger \ar[d]_{(\varphi_B^{-1})^\dagger} \\ 	
 	A^{\dagger\dagger} \ar[rr]_{f^{\dagger\dagger}} & & B^{\dagger\dagger} } \]
 
 This new dagger clearly preserves composition and is also a stationary on objects involution as proven by the second diagram:
 the lower square of this diagram is the dagger of the inverted definition and the resulting outer square is the naturality of $\iota$ forcing $f^{\ddagger\ddagger} = f$.
 
 Next, we observe that $u: X \to Y$ is a unitary isomorphism in $\X$ if and only if $u^{-1}= u^\ddagger$.  This makes unitary isomorphisms in the traditional sense of categorical quantum mechanics coincide 
 with the notion introduced here.   Thus, $u$ is unitary in the sense here if and only if the diagram below commutes
 \[ \xymatrix{ B \ar[d]_{\varphi_B} \ar[rr]^{u^{-1}} && A  \ar[d]^{\varphi_A} \\ B^\dagger \ar[rr]_{u^\dagger} && A} \]
 but this diagram commutes if and only if $u^{-1} = u^\ddagger$. 
 
 \begin{definition} \label{preserving-unitary-structure}
 	A $\dagger$-Frobenius mix functor, $F: \X \to \Y$, between compact $\dagger$-isomix categories with unitary structure {\bf preserves unitary structure} if
 	\begin{enumerate}[(i)]
 		\item for all unitary objects $A \in \X$, $F(A)$ is a unitary object such that $\varphi_{F(A)} = F(\varphi_A) \rho^F$ 
 		\item Either $n_\bot^F$ or $m_\top^F$ are unitary isomorphisms i.e.,
 		\[
 		\xymatrix{
 			F(\bot) \ar[r]^{F(\varphi_\bot)} \ar[d]_{n_\bot} & F(\bot^\dagger) \ar[r]^{\rho} & F(\bot)^\dagger \\
 			\bot \ar[rr]_{\varphi_\bot} & & \bot^\dagger \ar[u]_{n_\bot^\dagger}
 		} (or)  \xymatrix{
 			\top \ar[rr]^{\varphi_\top} \ar[d]_{m_\top} & & \top^\dagger \\
 			F(\top) \ar[r]_{F(\varphi_\top)} & F(\top^\dagger) \ar[r]_{\rho} & F(\top)^\dagger \ar[u]_{m_\top^\dagger}
 		}
 		\]
 	\end{enumerate}
 \end{definition}
 
Notice that if $F$ preserves unitary structure, it must be an isomix functor by Lemma \ref{Lemma: isomix functor}. Also, when $A \in \X$ is a unitary object,  then $F(A)$ must be a unitary object, and so $F(A)$ is in the core.
 
 We now show that ${\sf Mx}_\uparrow: (\X,\oa,\oa) \to (\X,\ox,\oa)$ provides a unitary structure preserving equivalence of a dagger monoidal category into a unitary category:
 
 \begin{proposition}  \label{unitary-2-dagger}
 	Unitary categories are $\dagger$-linearly equivalent via the mix functor ${\sf Mx}_\uparrow: (\X,\oa,\oa) \to (\X,\ox,\oa)$ to the underlying dagger monoidal category on the par.  
 	Furthermore, closed unitary categories under this equivalence become dagger compact closed categories.
 \end{proposition}
 
 \begin{proof}
 	We must exhibit a preservator, that is a natural transformation showing that the involution is preserved:
 	\[ \infer={A \to_{\varphi_A} A^\dagger}{{\sf Mx}_\uparrow(A^\ddagger) \to^{\varphi_A} {\sf Mx}_\uparrow(A)^\dagger} \]
 	Note that $\varphi$ is a natural transformation by the definition of $(\_)^\ddagger$ and its coherence requirements 
 	make it a linear natural equivalence.  Making this the preservator immediately means that unitary structure is preserved.
 	
 	Finally, we must show that unitary linear duals under ${\sf Mx}^{*}_\uparrow$ become $\ddagger$-duals.  Given $(\eta,\epsilon): A \dashvv_u B$  we must 
 	show that under ${\sf Mx}^{*}_\uparrow$ this produces a dagger dual.  ${\sf Mx}^{*}_\uparrow(\eta) = {\sf m} ~\eta: \bot \to A \oa B$ and 
 	${\sf Mx}^{*}_\uparrow(\epsilon) = {\sf mx}^{-1} \epsilon: B \oa A \to \bot$
 	We then require that $c_\oa {\sf Mx}^{*}_\uparrow(\epsilon) = {\sf Mx}^{*}_\uparrow(\eta)^\ddagger$. This is provided by:
 	\[ \xymatrix{A \oa B  \ar[d]^{{\sf mx}^{-1}}  \ar@/_2pc/[ddd]_{\varphi_{A \oa B}} \ar[r]^{c_\oa} & B \oa A \ar@/^1pc/[rr]^{{\sf Mx}^{*}_\uparrow(\epsilon)} \ar[r]_{{\sf mx}^{-1}} 
 		& B \ox A \ar[r]_{\epsilon} & \bot \ar[dd]_{{\sf m}} \ar@/^2pc/[ddd]^{\varphi_\bot} \ar@/_/[dddl]_{\lambda_\bot}\\
 		A \ox B \ar[d]^{\varphi_A \ox \varphi_B} \ar@/_/[rru]^{c_\ox} \\
 		A^\dagger \ox B^\dagger \ar[d]^{\lambda_\ox} & & & \top \ar[d]_{\lambda_\top} \\
 		(A \oa B)^\dagger \ar@{}[rrruuu]|{{\rm Defn.} ~\ref{defn: unitary dual}~(b)} \ar@/_2pc/[rrr]_{{\sf Mx}^{*}_\uparrow(\eta)^\dagger} \ar[rr]_{\eta^\dagger} & & \top^\dagger \ar[r]_{{\sf m}^\dagger} & \bot^\dagger } \]
 \end{proof}

%%%%%%%%%%%%%%%%%%%%%%%%%%%%%%%%%%%%%%%%%%%
\subsection{The unitary construction}
\label{Sec: unitary construction}
%%%%%%%%%%%%%%%%%%%%%%%%%%%%%%%%%%%%%%%%%%%
A $\dagger$-isomix category can have many different unitary structures, as 
we shall describe in this section, 
thus it is {\em structure\/}, and not a property.   The requirements, however, 
do mean that for a $\dagger$-isomix category, $\X$, there is always a 
smallest unitary structure, referred to as the ``trivial'' unitary structure, 
that produces a full unitary subcategory in $\X$.  In this subsection, we 
provide a construction called the unitary construction which produces this unitary category from any 
$\dagger$-isomix category.   The construction is based on identifying 
objects with pre-unitary structure: the tensor units always have a 
canonical ``pre-unitary" structure so the construction always produces 
a non-empty category.   However, to ensure that an application of 
the construction yields a unitary category in which there are objects 
which are not isomorphic to the units, one must exhibit concretely 
such objects.  Fortunately this is often not difficult to do, making 
the construction quite applicable.

\begin{definition} ~
\label{defn: pre-unitary}
\begin{enumerate}[(i)] 
\item In a $\dagger$-isomix category, a {\bf pre-unitary object} is an object $U \in \Core(\X)$, together with an isomorphism $\alpha: U \to U^\dagger$ such that  $\alpha (\alpha^{-1})^\dagger = \iota$. 

\item Suppose $\X$ is a $\dagger$-isomix category, then define ${\sf Unitary}(\X)$, the {\bf canonical unitary core} of $\X$, as follows:
\begin{description}
\item[Objects:] Pre-unitary objects $(U, \alpha)$,
\item[Maps:] $(U, \alpha) \to^f (V, \beta)$ where $U \to^f V $ is any map of $\X$.
\end{description}
\end{enumerate}
\end{definition}

We note that any object which is isomorphic to a preunitary object is also pre-unitary:
\begin{lemma}
In a $\dagger$-isomix category, if $U$ is a pre-unitary object 
and there exists an isomorphism $f: U \to U'$, then $U'$ is pre-unitary. 
\end{lemma}

Our objective is to show that Unitary($\X$) is endowed with all the structure of a unitary category.

\begin{lemma}
For any $\dagger$-isomix category, its canonical unitary core is a compact $\dagger$-LDC with tensor and par defined by
\[ (\top,{\sf m}^{-1}\lambda_\bot: \top \to \top^\dagger) ~~~~~(A, \alpha) \ox (B, \beta) := (A \ox B, \mx(\alpha \oa \beta) \lambda_\oa: A \ox B \to (A \ox B)^\dagger)\]
\[ (\bot, {\sf m} ~\lambda_\top: \bot \to \bot^\dagger)  ~~~~~(A, \alpha) \oa (B, \beta) := (A \oa B, \mx^{-1}(\alpha \ox \beta) \lambda_\ox: A \oa B \to (A \oa B)^\dagger) \]
and $(U,\alpha)^\dagger := (U^\dagger, (\alpha^{-1})^\dagger)$.
\end{lemma}

\begin{proof}
The proof uses the techniques of Lemma \ref{Lemma: square root tensor unitary}. 

 Note that, as the map and tensor structure is inherited from $\X$, it suffices to show that these objects are all pre-unitary objects.  Starting with $(U \alpha)^\dagger$ 
we have:
\[ (\alpha^{-1})^\dagger (((\alpha^{-1})^\dagger)^{-1})^\dagger = (\alpha^{-1})^\dagger (\alpha^\dagger)^\dagger = (\alpha^\dagger \alpha^{-1})^\dagger = (\iota^{-1})^\dagger = \iota \]
For the tensor and par we have:
\begin{eqnarray*}
{\sf m}^{-1}\lambda_\bot (({\sf m}^{-1}\lambda_\bot)^{-1})^\dagger & = & {\sf m}^{-1}\lambda_\bot {\sf m}^\dagger \lambda_\bot^\dagger \\
& \stackrel{\text{\tiny {\bf [$\dagger$-mix]}}}{=} &  {\sf m}^{-1} {\sf m} \lambda_\top  \lambda_\bot^\dagger  = \iota \\
\mx^{-1}(\alpha \oa \beta) \lambda_\oa ((\mx^{-1}(\alpha \oa \beta) \lambda_\oa)^{-1})^\dagger 
& = & \mx^{-1}(\alpha \oa \beta) \lambda_\oa (\mx^\dagger) (\alpha^{-1} \oa \beta^{-1})^\dagger (\lambda_\oa^{-1})^\dagger \\
& = & \mx^{-1}(\alpha \oa \beta) \mx \lambda_\ox (\alpha^{-1} \oa \beta^{-1})^\dagger (\lambda_\oa^{-1})^\dagger \\
& = & (\alpha \ox \beta) \lambda_\ox (\alpha^{-1} \oa \beta^{-1})^\dagger (\lambda_\oa^{-1})^\dagger \\
& = & (\alpha \ox \beta) ((\alpha^{-1})^\dagger \ox (\beta^{-1})^\dagger) \lambda_\ox (\lambda_\oa^{-1})^\dagger \\
&\stackrel{\text{\tiny {\bf Defn \ref{defn: pre-unitary}-(i)}}}{=}& (\iota \ox \iota) \lambda_\ox (\lambda_\oa^{-1})^\dagger \\
&\stackrel{\text{\tiny {\bf [$\dagger$-ldc.4]}}}{=}& \iota 
\end{eqnarray*}
\begin{eqnarray*}
{\sf m} \lambda_\top (({\sf m} \lambda_\top)^{-1})^\dagger & = & {\sf m} \lambda_\top ({\sf m}^{-1})^\dagger  (\lambda_\top^{-1})^\dagger  \\
& = & {\sf m} ~{\sf m}^{-1} \lambda_\bot (\lambda_\top^{-1})^\dagger = \iota \\
\mx(\alpha \ox \beta) \lambda_\ox ((\mx(\alpha \ox \beta) \lambda_\ox)^{-1})^\dagger 
& = & \mx(\alpha \ox \beta) \lambda_\ox (\mx^{-1})^\dagger (\alpha^{-1} \ox \beta^{-1})^\dagger (\lambda_\ox^{-1})^\dagger \\
& = & (\alpha \oa \beta) \mx~ \mx^{-1} \lambda_\oa (\alpha^{-1} \ox \beta^{-1})^\dagger (\lambda_\ox^{-1})^\dagger \\
&= & (\alpha \oa \beta) ((\alpha^{-1})^\dagger \oa (\beta^{-1})^\dagger) \lambda_\oa (\lambda_\ox^{-1})^\dagger \\
& = & (\iota \oa \iota) \lambda_\oa (\lambda_\ox^{-1})^\dagger  = \iota.
\end{eqnarray*}
\end{proof}

This makes $\Unitary(\X)$ into a compact $\dagger$-LDC with all the structure inherited directly from $\X$. 
However, more is true: each object now has an obvious unitary structure.  This gives:

\begin{proposition}
For any $\dagger$-isomix category, $\X$, $\Unitary(\X)$ is a unitary category with a full and faithful underlying $\dagger$-isomix functor $U: {\sf Unitary}(\X) \to \X$.
\end{proposition}

\begin{proof}
The laxors are all identity maps so that the underlying functors is immediately a $\dagger$-mix functor.

It remains to show that every object is unitary:  we set the unitary structure of an object to be $\alpha: (X,\alpha) \to (X,\alpha)^\dagger$.   However, {\bf [U.1]} -- {\bf [U.5]} are immediately satisfied by construction implying this provides unitary structure for every object.
\end{proof}

%%%%%%%%%%%%%%%%%%%%%%%%%%%%%%%%%%%%%%%%%%%%%%%%%%%%
% Couniversality of universal construction %
Next, we prove the couniversal property of the unitary construction. Define ${\sf UCat}$ to be the category of unitary categories and $\dagger$-isomix functors that preserve   unitary structure in the sense of Definition \ref{preserving-unitary-structure}, thus, whenever ${\varphi_A}$ is the unitary structure  then $F'(\varphi_A) \rho^{F'}$ is unitary structure. Define {\sf Kompact} to be the category of compact $\dagger$-LDCs and $\dagger$-isomix functors.

We now show that the unitary construction produces a right adjoint to the underlying functor $U: {\sf UCat} \to {\sf Kompact}$ which is the identity functor. Preliminary to this result we prove that Frobenius functors preserve preunitary objects:

\begin{lemma}
\label{Lemma: Frobenius preunitary}
If $F: \X \to \Y$ is a $\dagger$-isomix functor between compact $\dagger$-LDCs and $(A,\varphi)$ is a preunitary object of $\X$, then $(F(A),F(\varphi)\rho)$ is a preunitary object of $\Y$.
\end{lemma}
\begin{proof}
To prove that $(F(A),F(\varphi)\rho)$ is a preunitary object, one has the following computation:
\begin{eqnarray*}
F(\varphi)\rho ((F(\varphi) \rho)^{-1})^\dagger 
& = & F(\varphi)\rho F(\varphi^{-1})^\dagger (\rho^{-1})^\dagger \\
& = & F(\varphi (\varphi^{-1})^\dagger) \rho (\rho^{-1})^\dagger \\
& = & F(\iota) \rho (\rho^{-1})^\dagger \stackrel{{\bf [\dagger-isomix]}}{=} \iota.
\end{eqnarray*}
\end{proof}

\begin{proposition}
\label{Prop: Couniversal}
$U: {\sf UCat} \to {\sf Kompact}$ has a right adjoint ${\sf Unitary}: {\sf Kompact} \to {\sf UCat}; \C \mapsto {\sf Unitary}(\C)$.
\end{proposition}
\begin{proof}
The couniversal diagram is as follows:
\[ \xymatrix{ U(\U) \ar[rr]^{F} \ar@{.>}[d]_{U(F^\flat)} && \C \\
                    U({\sf Unitary}(\C)) \ar[urr]_{\epsilon}} \]

Since $F$ is a $\dagger$-isomix functor it preserves preunitary structure (see Lemma \ref{Lemma: Frobenius preunitary}).  This means that each $(U,\varphi_U)$ in $\U$ is carried by $F$ onto a preunitary object in $\C$, $(F(U),F(\varphi)\rho^F)$.  But a preunitary object in $\C$ is an object of ${\sf Unitary}(\C)$ and this determines $F^\flat$.   The functor $F^\flat$ is uniquely determined as it must preserve the unitary structure.
\end{proof}
%%%%%%%%%%%%%%%%%%%%%%%%%%%%%%%%%%%%%%%%%%%%%%%%%%%%

\section{Examples:  The unitary construction}
\label{Sec: Examples The unitary construction}

In Section \ref{daggers-duals-conjugation}, we discussed examples of $\dagger$-isomix categories in which the $\dagger$ 
is given by composing the conjugation functor and the dualizing functor. In the rest of the section, we apply the unitary 
construction to each of those examples to construct a unitary category:

%%%%%%%%%%%%%%%%%%%%%%%%%%%%%%%%%%%%%%%%%%%

\subsection{Category of abstract state spaces}
In Section \ref{Sec: Asp}, we discussed a construction on a $\dagger$-isomix category, $\X$, that produces a 
category of abstract state spaces, $\Asp(\X)$, which is a $\dagger$-isomix category. In this section, we examine 
the preunitary objects of $\Asp(\X)$. Since all the basic natural isomorphisms are inherited from $\X$, $\Core(\X)$ 
determines $\Core(\Asp(\X))$. If $(A ,\alpha)$ is a preunitary object for $\X$, and $(A, e_A, u_A) \in \Asp(\X)$ then, 
$((A, e_A, u_A), \alpha)$ is a preunitary object for $\Asp(\X)$ if $u_A \alpha = \lambda_\top e_A^\dagger$.

%%%%%%%%%%%%%%%%%%%%%%%%%%%%%%%%%%%%%%%%%%%

\subsection{Category of a group with involution}
We discussed a source of examples of compact $\dagger$-LDCs which are given by groups with conjugation. Applying 
unitary construction to each of the example categories results in the following unitary categories. It could be noticed 
that the preunitary objects in each of these categories includes those group elements such that $\overline{g^{-1}} = g$. 
More explicitly, the preunitary objects are $(g,1)$ such that $\overline{g^{-1}} = g$.

\begin{itemize}
    \item In the discrete category of complex numbers, $\D(\C, +, 0)$, \[(a + ib)^\dagger := \overline{(a+ib)^*} = \overline{(-a-ib)} = 
    -a + ib\] The preunitary objects in this category are given by all complex numbers, i.e., $(ib, 1)$. 
    
    \item In the discrete category of non-zero complex numbers, $\D(\C, ., 1)$, the preunitary objects are given by complex 
    numbers on a unit circle.
    
    \item In the discrete category, $\D(P(x), +, 0)$, where $P(x)$ is a polynomial ring, $P(x)^\dagger = -P(-x)$ and the preunitary 
    objects are polynomials $ P(x) = \sum_n a_n x^n$ such that n is odd. 
    
    \item In $\D(\mathbb{M}_2, \cdot, I_2)$ where $\mathbb{M}_2$ is the group of $2 \times 2$ invertible matrices over 
    $\mathbb{C}$. The $\dagger$ structure is as follows:
     \[\left(
      \begin{matrix}
     a+ib & m+in \\
     c+id & p+iq 
     \end{matrix}
     \right)^\dagger := \overline{\left(
     \begin{matrix}
     a+ib & m+in \\
     c+id & p+iq 
     \end{matrix}
     \right)^*} = \left(
     \begin{matrix}
    a-ib & c-id \\
    m-in & p-iq
     \end{matrix}
     \right)^{-1} \]
     The preunitary objects in this category are the unitary matrices.
    \end{itemize}
%%%%%%%%%%%%%%%%%%%%%%%%%%%%%%%%%%%%%%%%%%%%%%%%%%%%
\subsection{Category of Hopf modules in a $*$-automonous category}
In Section \ref{Sec: HModx}, we described a construction of $\dagger$-isomix categories  from any  symmetric isomix 
$*$-autonomous category, $\X$, by choosing the Hopf Modules over a  cocommutative $\ox$-Hopf Algebra. 
We referred to the resulting category as ${\mbox{\bf H-Mod}}_\X$. Now we shall look at the preunitary objects in 
${\mbox{\bf H-Mod}}_\X$ in order to apply the unitary construction to this category. We begin by identifying the 
objects in the core of ${\mbox{\bf H-Mod}}_\X$:

%We know that in any isomix category, $\X$, the core of the category, $\Core({\mbox{\bf H-Mod}}_\X)$ is a compact $\dagger$-isomix category. If the core of ${\mbox{\bf H-Mod}}_\X$ is non-trivial, i.e., the core includes objects other than tensor units too, then, one can apply unitary construction to the core to get a MUC. ${\mbox{\bf H-Mod}}_\X$ being a $\dagger$-isomix category, this technique can be applied to the category to get a MUC. In order to do so, we identify the preunitary objects in $\Core({\mbox{\bf H-Mod}}_\X)$. We begin by identifying the objects in the core of ${\mbox{\bf H-Mod}}_\X$:

\begin{lemma}
Suppose $\X$ is a mix $*$-autonomous category and $H$ is a cocommutative Hopf Algebra in $\X$. If $(A, \leftaction{0.4}{white})$ is a H-Module and $A \in \Core(\X)$, then $ (A, \leftaction{0.4}{white}) \in \Core({\mbox{\bf H-Mod}}_\X)$.
\end{lemma}
\begin{proof}
The mixor $\mx: A \ox B \to A \oa B$ is inherited directly from $\X$. Henc,e  $ (A, \leftaction{0.4}{white}) \in \Core(${\bf H-Mod$_\X)$}.
\end{proof}

Now that we identified the objects in the core, we prove a lemma that will be used later to identify the preunitary objects from the core:

\begin{lemma}
\label{Lemma: aux} The following equality holds for a Frobenius algebra: 
\[ \begin{tikzpicture}
	\begin{pgfonlayer}{nodelayer}
		\node [style=circle] (0) at (-2.5, 0.5) {};
		\node [style=none] (1) at (-2, 1) {};
		\node [style=none] (3) at (-3, 1) {};
		\node [style=none] (4) at (-2.5, 0) {};
		\node [style=none] (5) at (-1.5, 0) {};
		\node [style=circle] (6) at (-2, -1) {};
		\node [style=circle] (7) at (-2, -0.5) {};
		\node [style=none] (8) at (-3, 1) {};
		\node [style=none] (9) at (-4, 1) {};
		\node [style=circle] (10) at (-3.5, 2.25) {};
		\node [style=circle] (11) at (-3.5, 1.75) {};
		\node [style=circle] (12) at (-3.5, 2.75) {};
		\node [style=none] (13) at (-2, 1) {};
		\node [style=none] (14) at (-4.75, 1) {};
		\node [style=circle] (15) at (-3.5, 3.25) {};
		\node [style=none] (16) at (-1.5, 3.5) {};
		\node [style=none] (17) at (-4, -1) {};
		\node [style=none] (18) at (-4.75, -1) {};
	\end{pgfonlayer}
	\begin{pgfonlayer}{edgelayer}
		\draw [in=-90, out=30, looseness=1.25] (0) to (1.center);
		\draw [in=150, out=-90] (3.center) to (0);
		\draw [in=-90, out=45] (7) to (5.center);
		\draw (7) to (6);
		\draw [in=127, out=-90, looseness=0.75] (4.center) to (7);
		\draw [in=90, out=-30] (11) to (8.center);
		\draw (11) to (10);
		\draw [in=-150, out=90] (9.center) to (11);
		\draw [in=90, out=-30, looseness=0.75] (12) to (13.center);
		\draw (12) to (15);
		\draw [in=-150, out=90, looseness=0.75] (14.center) to (12);
		\draw (0) to (4.center);
		\draw (5.center) to (16.center);
		\draw (17.center) to (9.center);
		\draw (18.center) to (14.center);
	\end{pgfonlayer}
\end{tikzpicture}
 = \begin{tikzpicture}
	\begin{pgfonlayer}{nodelayer}
		\node [style=none] (19) at (1.75, -1) {};
		\node [style=circle] (20) at (1, 1.25) {};
		\node [style=none] (21) at (0.25, -1) {};
		\node [style=none] (22) at (1, 3.5) {};
	\end{pgfonlayer}
	\begin{pgfonlayer}{edgelayer}
		\draw (20) to (22.center);
		\draw [in=-150, out=90] (21.center) to (20);
		\draw [in=90, out=-30] (20) to (19.center);
	\end{pgfonlayer}
\end{tikzpicture} \]
\end{lemma}
\begin{proof}
\[\begin{tikzpicture}
	\begin{pgfonlayer}{nodelayer}
		\node [style=circle] (0) at (-2.5, 0.75) {};
		\node [style=none] (1) at (-2, 1.25) {};
		\node [style=none] (2) at (-2.5, 0.25) {};
		\node [style=none] (3) at (-3, 1.25) {};
		\node [style=none] (4) at (-2.5, 0.25) {};
		\node [style=none] (5) at (-1.5, 0.25) {};
		\node [style=circle] (6) at (-2, -1) {};
		\node [style=circle] (7) at (-2, -0.25) {};
		\node [style=none] (8) at (-3, 1.25) {};
		\node [style=none] (9) at (-4, 1.25) {};
		\node [style=circle] (10) at (-3.5, 2.5) {};
		\node [style=circle] (11) at (-3.5, 1.75) {};
		\node [style=circle] (12) at (-3.5, 3.25) {};
		\node [style=none] (13) at (-2, 1.25) {};
		\node [style=none] (14) at (-4.75, 1.25) {};
		\node [style=circle] (15) at (-3.5, 4) {};
		\node [style=none] (16) at (-1.5, 4) {};
		\node [style=none] (17) at (-4, -1) {};
		\node [style=none] (18) at (-4.75, -1) {};
	\end{pgfonlayer}
	\begin{pgfonlayer}{edgelayer}
		\draw [in=-90, out=30] (0) to (1.center);
		\draw (0) to (2.center);
		\draw [in=150, out=-90] (3.center) to (0);
		\draw [in=-90, out=60, looseness=0.75] (7) to (5.center);
		\draw (7) to (6);
		\draw [in=127, out=-90, looseness=0.75] (4.center) to (7);
		\draw [in=90, out=-15] (11) to (8.center);
		\draw (11) to (10);
		\draw [in=-165, out=90] (9.center) to (11);
		\draw [in=90, out=-30, looseness=0.75] (12) to (13.center);
		\draw (12) to (15);
		\draw [in=-150, out=90, looseness=0.75] (14.center) to (12);
		\draw (16.center) to (5.center);
		\draw (9.center) to (17.center);
		\draw (14.center) to (18.center);
	\end{pgfonlayer}
\end{tikzpicture}  = \begin{tikzpicture}
	\begin{pgfonlayer}{nodelayer}
		\node [style=none] (19) at (3.25, 1) {};
		\node [style=circle] (20) at (2.75, -1) {};
		\node [style=circle] (21) at (2.75, -0.25) {};
		\node [style=circle] (22) at (1, 3.25) {};
		\node [style=none] (23) at (0, -1) {};
		\node [style=circle] (24) at (1, 4) {};
		\node [style=circle] (25) at (1, 2.5) {};
		\node [style=none] (26) at (2, 2.5) {};
		\node [style=circle] (27) at (1.5, 1.75) {};
		\node [style=none] (28) at (2, 2.5) {};
		\node [style=none] (30) at (1, -1) {};
		\node [style=circle] (32) at (1.5, 1) {};
		\node [style=none] (33) at (3.25, 4) {};
		\node [style=none] (34) at (3.25, 4) {};
	\end{pgfonlayer}
	\begin{pgfonlayer}{edgelayer}
		\draw [in=-90, out=45] (21) to (19.center);
		\draw (21) to (20);
		\draw (22) to (24);
		\draw [in=-150, out=90, looseness=0.75] (23.center) to (22);
		\draw [in=-90, out=30] (27) to (28.center);
		\draw [in=150, out=-90, looseness=1.25] (25) to (27);
		\draw [in=-127, out=90, looseness=0.75] (30.center) to (32);
		\draw [in=90, out=0, looseness=1.25] (22) to (26.center);
		\draw (27) to (32);
		\draw (32) to (21);
		\draw (34.center) to (19.center);
	\end{pgfonlayer}
\end{tikzpicture}  =
\begin{tikzpicture}
	\begin{pgfonlayer}{nodelayer}
		\node [style=none] (35) at (7.5, 2) {};
		\node [style=circle] (36) at (7, -1) {};
		\node [style=circle] (37) at (7, 1) {};
		\node [style=circle] (38) at (5.75, 3.25) {};
		\node [style=none] (39) at (5, 2) {};
		\node [style=circle] (40) at (5.75, 4) {};
		\node [style=none] (41) at (5.75, 1) {};
		\node [style=circle] (43) at (6.25, 2) {};
		\node [style=none] (44) at (5, -1) {};
		\node [style=none] (45) at (5.75, -1) {};
		\node [style=none] (46) at (7.5, 4) {};
	\end{pgfonlayer}
	\begin{pgfonlayer}{edgelayer}
		\draw [in=-90, out=30] (37) to (35.center);
		\draw (37) to (36);
		\draw (38) to (40);
		\draw [in=-165, out=90] (39.center) to (38);
		\draw [in=-127, out=90] (41.center) to (43);
		\draw [in=90, out=-30, looseness=1.25] (38) to (43);
		\draw (43) to (37);
		\draw (44.center) to (39.center);
		\draw (45.center) to (41.center);
		\draw (35.center) to (46.center);
	\end{pgfonlayer}
\end{tikzpicture} = \begin{tikzpicture}
	\begin{pgfonlayer}{nodelayer}
		\node [style=none] (47) at (11.5, 4) {};
		\node [style=none] (48) at (10.75, -1) {};
		\node [style=circle] (49) at (10.75, 1) {};
		\node [style=circle] (50) at (10, 2) {};
		\node [style=none] (51) at (9.25, -1) {};
		\node [style=circle] (52) at (10, 4) {};
	\end{pgfonlayer}
	\begin{pgfonlayer}{edgelayer}
		\draw [in=-90, out=30, looseness=0.75] (49) to (47.center);
		\draw (49) to (48.center);
		\draw (50) to (52);
		\draw [in=-150, out=90, looseness=0.75] (51.center) to (50);
		\draw (50) to (49);
	\end{pgfonlayer}
\end{tikzpicture} = \begin{tikzpicture}
	\begin{pgfonlayer}{nodelayer}
		\node [style=none] (0) at (3, -1) {};
		\node [style=circle] (1) at (2, 2) {};
		\node [style=none] (2) at (1, -1) {};
		\node [style=none] (3) at (2, 4) {};
	\end{pgfonlayer}
	\begin{pgfonlayer}{edgelayer}
		\draw (1) to (3.center);
		\draw [in=-150, out=90, looseness=0.75] (2.center) to (1);
		\draw [in=90, out=-30, looseness=0.75] (1) to (0.center);
	\end{pgfonlayer}
\end{tikzpicture} \]
\end{proof}

In the following Proposition we identify the preunitary objects in the core:

\begin{proposition} 
Suppose $\X$ is a symmetric mix $*$-autonomous category and $H$ is a cocommutative Hopf Algebra in $\X$. If $A \in 
\Core(\X)$ and $(A, \mulmap{1.2}{white}, \unitmap{1.2}{white}, \comulmap{1.2}{white}, \counitmap{1.2}{white})$ 
is a cocommutative Frobenius Algebra with an algebra homomorphism $H \to^{h} A$ then, 
\begin{enumerate}[(a)]
\item $(A, \leftaction{0.4}{white})$ is a H-Module where, $\leftaction{0.4}{white}: H \ox A \to A := \begin{tikzpicture}
	\begin{pgfonlayer}{nodelayer}
		\node [style=circle] (0) at (0, -0) {};
		\node [style=none] (1) at (-0.5, 1.75) {};
		\node [style=none] (2) at (0.75, 1.75) {};
		\node [style=none] (3) at (0, -0.75) {};
		\node [style=circle, scale=2] (4) at (-0.5, 1) {};
		\node [style=none] (5) at (-0.5, 1) {$h$};
	\end{pgfonlayer}
	\begin{pgfonlayer}{edgelayer}
		\draw (1.center) to (4);
		\draw [bend left, looseness=1.00] (0) to (4);
		\draw [in=-90, out=15, looseness=1.00] (0) to (2.center);
		\draw (0) to (3.center);
	\end{pgfonlayer}
\end{tikzpicture}$ 
 \item $\overline{(A, \leftaction{0.4}{white})^*} = (A, \leftaction{0.4}{white})$ where $A^*$ is the self-dual Frobenius 
 Algebra with cups and caps defined as
$
\begin{tikzpicture}
	\begin{pgfonlayer}{nodelayer}
		\node [style=circle] (0) at (0, -0) {};
		\node [style=none] (1) at (-0.5, 1) {};
		\node [style=none] (2) at (0.75, 1) {};
		\node [style=circle] (3) at (0, -1) {};
		\node [style=circle] (4) at (0, -1) {};
	\end{pgfonlayer}
	\begin{pgfonlayer}{edgelayer}
		\draw [in=-90, out=15, looseness=1.00] (0) to (2.center);
		\draw [in=150, out=-90, looseness=1.00] (1.center) to (0);
		\draw (0) to (3);
	\end{pgfonlayer}
\end{tikzpicture}  and  \begin{tikzpicture}
	\begin{pgfonlayer}{nodelayer}
		\node [style=circle] (0) at (0, 0) {};
		\node [style=none] (1) at (-0.5, -1) {};
		\node [style=none] (2) at (0.75, -1) {};
		\node [style=circle] (3) at (0, 1) {};
		\node [style=circle] (4) at (0, 1) {};
	\end{pgfonlayer}
	\begin{pgfonlayer}{edgelayer}
		\draw [in=90, out=-15, looseness=1.00] (0) to (2.center);
		\draw [in=-150, out=90, looseness=1.00] (1.center) to (0);
		\draw (0) to (3);
	\end{pgfonlayer}
\end{tikzpicture}
$ respectively. Hence, $A^* = A$ and $(A, \leftaction{0.4}{white})^\dagger =  (A, \leftaction{0.4}{white})$.
\end{enumerate}
\end{proposition}
\begin{proof}~
\begin{enumerate}[(a)]
\item $\begin{tikzpicture} %act7
	\begin{pgfonlayer}{nodelayer}
		\node [style=circle] (0) at (0, -0) {};
		\node [style=none] (1) at (0.75, 1.25) {};
		\node [style=none] (2) at (0, -0.75) {};
		\node [style=circle, scale=2] (3) at (-0.5, 0.5) {};
		\node [style=none] (4) at (-0.5, 0.5) {$h$};
		\node [style=none] (5) at (-0.5, 1.25) {};
	\end{pgfonlayer}
	\begin{pgfonlayer}{edgelayer}
		\draw [bend left, looseness=1.00] (0) to (3);
		\draw [in=-90, out=15, looseness=1.00] (0) to (1.center);
		\draw (0) to (2.center);
		\draw (5.center) to (3);
	\end{pgfonlayer}
\end{tikzpicture}: H \ox A \to A$ is a left action because $h: H \to A$ is an algebra homomorphism.
\item $
\begin{tikzpicture} %Frob0
	\begin{pgfonlayer}{nodelayer}
		\node [style=none] (0) at (1.25, -1) {};
		\node [style=none] (1) at (2.25, -1) {};
		\node [style=none] (2) at (1.5, 0.5) {};
		\node [style=none] (3) at (1.5, 1) {};
		\node [style=none] (4) at (0.25, 1) {};
		\node [style=none] (5) at (2.25, 3) {};
		\node [style=none] (6) at (0.25, -1.25) {};
		\node [style=none] (7) at (1, 2.75) {$H$};
		\node [style=none] (8) at (2.5, 2.75) {$A^*$};
		\node [style=none] (9) at (0.25, -3) {};
		\node [style=none] (10) at (0.5, -2.75) {$A^*$};
		\node [style=circle, scale=1.5] (11) at (0.75, 0.25) {};
		\node [style=none] (12) at (0.75, 3) {};
		\node [style=circle] (13) at (1.25, -0.5) {};
		\node [style=none] (14) at (0.75, 0.25) {$h$};
	\end{pgfonlayer}
	\begin{pgfonlayer}{edgelayer}
		\draw [in=-90, out=90, looseness=1.00] (2.center) to (3.center);
		\draw [bend left=90, looseness=2.75] (4.center) to (3.center);
		\draw [bend right=90, looseness=2.00] (0.center) to (1.center);
		\draw (5.center) to (1.center);
		\draw (4.center) to (6.center);
		\draw [bend right, looseness=1.00] (11) to (13);
		\draw [bend right=15, looseness=1.00] (13) to (2.center);
		\draw (13) to (0.center);
		\draw (12.center) to (11);
		\draw (6.center) to (9.center);
	\end{pgfonlayer}
\end{tikzpicture} = \iffalse
\begin{tikzpicture} %tensorFrob
	\begin{pgfonlayer}{nodelayer}
		\node [style=none] (0) at (1.25, -1) {};
		\node [style=none] (1) at (2.25, -1) {};
		\node [style=none] (2) at (1.5, 0.5) {};
		\node [style=none] (3) at (0.25, 1.5) {};
		\node [style=none] (4) at (1.5, 1) {};
		\node [style=none] (5) at (-0.25, 1) {};
		\node [style=none] (6) at (2.25, 3) {};
		\node [style=none] (7) at (0.25, -1.25) {};
		\node [style=none] (8) at (-0.25, -1.25) {};
		\node [style=none] (9) at (-0.5, -2) {};
		\node [style=none] (10) at (0.25, -2) {};
		\node [style=none] (11) at (-1.5, -2) {};
		\node [style=none] (12) at (-1.5, 3) {};
		\node [style=none] (13) at (-1.25, 2.75) {$H$};
		\node [style=none] (14) at (2.5, 2.75) {$A^*$};
		\node [style=none] (15) at (0.25, -3) {};
		\node [style=none] (16) at (0.5, -2.75) {$A^*$};
		\node [style=circle, scale=1.5] (17) at (0.75, 0.25) {};
		\node [style=none] (18) at (0.75, 1.5) {};
		\node [style=circle] (19) at (1.25, -0.5) {};
		\node [style=none] (20) at (0.75, 0.25) {$h$};
	\end{pgfonlayer}
	\begin{pgfonlayer}{edgelayer}
		\draw [in=-90, out=90, looseness=1.00] (2.center) to (4.center);
		\draw [bend left=90, looseness=2.75] (5.center) to (4.center);
		\draw [bend right=90, looseness=2.00] (0.center) to (1.center);
		\draw (6.center) to (1.center);
		\draw (3.center) to (7.center);
		\draw (5.center) to (8.center);
		\draw [in=105, out=-90, looseness=1.25] (8.center) to (10.center);
		\draw [in=90, out=-75, looseness=0.75] (7.center) to (9.center);
		\draw [bend right=90, looseness=1.25] (11.center) to (9.center);
		\draw (15.center) to (10.center);
		\draw [bend right, looseness=1.00] (17) to (19);
		\draw [bend right=15, looseness=1.00] (19) to (2.center);
		\draw (19) to (0.center);
		\draw [bend left=90, looseness=3.00] (3.center) to (18.center);
		\draw (12.center) to (11.center);
		\draw (18.center) to (17);
	\end{pgfonlayer}
\end{tikzpicture} = \fi
\begin{tikzpicture} %Frob2
	\begin{pgfonlayer}{nodelayer}
		\node [style=none] (0) at (1.25, -1) {};
		\node [style=none] (1) at (2.25, -1) {};
		\node [style=none] (2) at (1.5, 0.5) {};
		\node [style=none] (3) at (0.25, 0.25) {};
		\node [style=none] (4) at (1.5, 1) {};
		\node [style=none] (5) at (-0.25, 1) {};
		\node [style=none] (6) at (2.25, 3) {};
		\node [style=none] (7) at (0.25, -1.25) {};
		\node [style=none] (8) at (-0.25, -1.25) {};
		\node [style=none] (9) at (-0.5, -2) {};
		\node [style=none] (10) at (0.25, -2) {};
		\node [style=none] (11) at (-1.5, -2) {};
		\node [style=none] (12) at (-1.5, 3) {};
		\node [style=none] (13) at (0.25, -3) {};
		\node [style=none] (14) at (0.75, 0.25) {};
		\node [style=circle] (15) at (1.25, -0.5) {};
		\node [style=circle, scale=1.5] (16) at (-1.5, -0.75) {};
		\node [style=none] (17) at (-0.25, 1) {};
		\node [style=none] (18) at (1.5, 1) {};
		\node [style=circle] (19) at (0.5, 2.75) {};
		\node [style=circle] (20) at (0.5, 2) {};
		\node [style=none] (21) at (0.25, 0.25) {};
		\node [style=circle] (22) at (0.5, 1.5) {};
		\node [style=none] (23) at (0.75, 0.25) {};
		\node [style=circle] (24) at (0.5, 0.75) {};
		\node [style=none] (25) at (1.25, -1) {};
		\node [style=circle] (26) at (1.75, -2.75) {};
		\node [style=none] (27) at (2.25, -1) {};
		\node [style=circle] (28) at (1.75, -2) {};
		\node [style=none] (29) at (-1.5, -0.75) {$h$};
	\end{pgfonlayer}
	\begin{pgfonlayer}{edgelayer}
		\draw [in=-90, out=90, looseness=1.00] (2.center) to (4.center);
		\draw (6.center) to (1.center);
		\draw (3.center) to (7.center);
		\draw (5.center) to (8.center);
		\draw [in=105, out=-90, looseness=1.25] (8.center) to (10.center);
		\draw [in=90, out=-75, looseness=0.75] (7.center) to (9.center);
		\draw [bend right=90, looseness=1.25] (11.center) to (9.center);
		\draw (13.center) to (10.center);
		\draw [bend right=15, looseness=1.00] (15) to (2.center);
		\draw (15) to (0.center);
		\draw (16) to (11.center);
		\draw [in=-90, out=135, looseness=1.00] (15) to (14.center);
		\draw [bend right, looseness=1.00] (20) to (17.center);
		\draw [bend left, looseness=1.00] (20) to (18.center);
		\draw (19) to (20);
		\draw [bend right, looseness=1.00] (24) to (21.center);
		\draw [bend left, looseness=1.00] (24) to (23.center);
		\draw (22) to (24);
		\draw [bend left, looseness=1.00] (28) to (25.center);
		\draw [bend right, looseness=1.00] (28) to (27.center);
		\draw (26) to (28);
		\draw (12.center) to (16);
	\end{pgfonlayer}
\end{tikzpicture} \stackrel{Lemma ~ \ref{Lemma: aux}}{=} 
\begin{tikzpicture} %Frob3
	\begin{pgfonlayer}{nodelayer}
		\node [style=none] (0) at (0.75, 0.25) {};
		\node [style=none] (1) at (-0.25, 0.25) {};
		\node [style=none] (2) at (0.25, -2) {};
		\node [style=none] (3) at (-1.5, -2) {};
		\node [style=none] (4) at (-1.5, 1.75) {};
		\node [style=none] (5) at (0.25, -3.5) {};
		\node [style=circle, scale=1.5] (6) at (-1.5, 0.25) {};
		\node [style=none] (7) at (-1.5, 0.25) {$h$};
		\node [style=none] (8) at (0.75, 0.25) {};
		\node [style=none] (9) at (0.25, 1.75) {};
		\node [style=circle] (10) at (0.25, 1) {};
		\node [style=none] (11) at (-0.25, 0.25) {};
		\node [style=none] (12) at (-1.5, -2) {};
		\node [style=none] (13) at (-0.5, -2) {};
		\node [style=circle] (14) at (-1, -3.25) {};
		\node [style=circle] (15) at (-1, -2.5) {};
		\node [style=none] (16) at (-1.5, -2) {};
	\end{pgfonlayer}
	\begin{pgfonlayer}{edgelayer}
		\draw (5.center) to (2.center);
		\draw (6) to (3.center);
		\draw (4.center) to (6);
		\draw (7.center) to (6);
		\draw [bend right, looseness=1.00] (10) to (11.center);
		\draw [bend left, looseness=1.00] (10) to (8.center);
		\draw (9.center) to (10);
		\draw [bend left, looseness=1.00] (15) to (16.center);
		\draw [bend right, looseness=1.00] (15) to (13.center);
		\draw (14) to (15);
		\draw [in=90, out=-90, looseness=1.00] (1.center) to (2.center);
		\draw [in=90, out=-90, looseness=1.00] (0.center) to (13.center);
	\end{pgfonlayer}
\end{tikzpicture} \stackrel{\text{cocomm.}}{=} 
\begin{tikzpicture} %Frob4
	\begin{pgfonlayer}{nodelayer}
		\node [style=none] (0) at (0.75, 0.25) {};
		\node [style=none] (1) at (-0.25, 0.25) {};
		\node [style=none] (2) at (0.75, -2) {};
		\node [style=none] (3) at (-1.5, -2) {};
		\node [style=none] (4) at (-1.5, 1.75) {};
		\node [style=none] (5) at (0.75, -3.5) {};
		\node [style=circle, scale=1.5] (6) at (-1.5, 0.25) {};
		\node [style=none] (7) at (-1.5, 0.25) {$h$};
		\node [style=none] (8) at (0.75, 0.25) {};
		\node [style=none] (9) at (0.25, 1.75) {};
		\node [style=circle] (10) at (0.25, 1) {};
		\node [style=none] (11) at (-0.25, 0.25) {};
		\node [style=none] (12) at (-1.5, -2) {};
		\node [style=none] (13) at (-0.5, -2) {};
		\node [style=circle] (14) at (-1, -3.25) {};
		\node [style=circle] (15) at (-1, -2.5) {};
		\node [style=none] (16) at (-1.5, -2) {};
	\end{pgfonlayer}
	\begin{pgfonlayer}{edgelayer}
		\draw (5.center) to (2.center);
		\draw (6) to (3.center);
		\draw (4.center) to (6);
		\draw (7.center) to (6);
		\draw [bend right, looseness=1.00] (10) to (11.center);
		\draw [bend left, looseness=1.00] (10) to (8.center);
		\draw (9.center) to (10);
		\draw [bend left, looseness=1.00] (15) to (16.center);
		\draw [bend right, looseness=1.00] (15) to (13.center);
		\draw (14) to (15);
		\draw (1.center) to (13.center);
		\draw (0.center) to (2.center);
	\end{pgfonlayer}
\end{tikzpicture} =
\begin{tikzpicture}
	\begin{pgfonlayer}{nodelayer}
		\node [style=circle] (17) at (3.75, -1.5) {};
		\node [style=none] (18) at (4.25, 0.5) {};
		\node [style=none] (19) at (3.75, -2.25) {};
		\node [style=circle, scale=2] (20) at (3.25, -0.5) {};
		\node [style=none] (21) at (3.25, -0.5) {$h$};
		\node [style=none] (22) at (3.25, 0.5) {};
		\node [style=none] (23) at (3.25, 1.75) {};
		\node [style=none] (24) at (4.25, 1.75) {};
		\node [style=none] (25) at (3.75, -3.5) {};
	\end{pgfonlayer}
	\begin{pgfonlayer}{edgelayer}
		\draw [bend left] (17) to (20);
		\draw [in=-90, out=30, looseness=0.75] (17) to (18.center);
		\draw (17) to (19.center);
		\draw (20) to (22.center);
		\draw (23.center) to (22.center);
		\draw (24.center) to (18.center);
		\draw (19.center) to (25.center);
	\end{pgfonlayer}
\end{tikzpicture}$
\end{enumerate}
\end{proof}

\begin{corollary}
$(((A, \mulmap{1.2}{white}, \unitmap{1.2}{white}, \comulmap{1.2}{white}, \counitmap{1.2}{white}), \leftaction{0.4}{white}), 1)$ is a preunitary object.
\end{corollary}

Thus, we have a source of non-trivial preunitary objects so that we can form a non-trivial unitary category.

%%%%%%%%%%%%%%%%%%%%%%%%%%%%%%%%%%%%%%%%%%%%%%%%%%%%%%%%%%%%%

\section{Mixed unitary categories}
\label{Sec: MUCs}

A mixed unitary category has a unitary core which is a model of classical categorical quantum mechanics 
extended by a larger setting in which possibly infinite dimensional objects can 
be modelled. We are now ready for the definition of mixed unitary categories, which is the key structure developed in 
the first part of this thesis.

\subsection{Mixed unitary category}

\begin{definition}
A {\bf mixed unitary category} (MUC) is a $\dagger$-isomix category, $\C$, equipped with a strong $\dagger$-isomix functor 
$M: \U \to \C$ from a unitary category $\U$ to $\C$ such that there exists the following natural transformations:
\[ \mx': M(U) \oa X \to M(U) \ox X  \text{ with } \mx  ~\mx' = 1 \text{ and }\mx' ~ \mx = 1 \]
\[ \mx'': X \oa M(U) \to X \ox M(U) \text{ with } \mx  ~\mx'' = 1 \text{ and }\mx'' ~ \mx = 1 \]
A mixed unitary category, $M: \U \to \C$ is {\bf symmetric} if the functor $M$, the 
unitary category $\U$, and the $\dagger$-isomix category $\C$ are symmetric.
\end{definition}

In the definition of a MUC, the requirement of a transformation $\mx'$ which is inverse to $\mx$ ensures that the functor 
$M: \U \to \C$ factors through $\Core(\C)$. Figure \ref{Fig: MUC} is a schematic diagram of a MUC.

 \begin{figure}[h]
	\centering
  \begin{tikzpicture} [scale=1.5]
	\begin{pgfonlayer}{nodelayer}
		\node [style=circle, scale=14] (0) at (4.5, 0) {};
		\node [style=none] (1) at (-3, 2) {};
		\node [style=none] (2) at (-5, 0) {};
		\node [style=none] (3) at (-3, -2) {};
		\node [style=none] (4) at (-1, 0) {};
		\node [style=none] (5) at (-3, -0.75) {$A \to^{\varphi_A}_{\simeq} A^\dagger$};
		\node [style=none] (6) at (-3, 0.5) {Unitary};
		\node [style=none] (7) at (-3, 0) {category};
		\node [style=none] (8) at (-0.75, 0) {};
		\node [style=none] (9) at (3, 0) {};
		\node [style=none] (10) at (0.25, 0.25) {$\dagger$-isomix};
		\node [style=none] (11) at (0.25, -0.25) {functor};
		\node [style=none] (12) at (4.5, 2) {$\dagger$-isomix};
		\node [style=none] (13) at (4.5, 1.5) {category};
		\node [style=none] (14) at (4, -2) {$B$};
		\node [style=none] (15) at (5.5, -2) {$B^\dagger$};
		\node [style=none] (16) at (4, 0.75) {};
		\node [style=none] (17) at (3.25, 0) {};
		\node [style=none] (18) at (3.5, -1) {};
		\node [style=none] (19) at (4.25, -0.25) {};
		\node [style=none] (20) at (2.75, 1.25) {};
		\node [style=none] (21) at (4.75, 1.25) {};
		\node [style=none] (22) at (4.75, -1.25) {};
		\node [style=none] (23) at (2.75, -1.25) {};
		\node [style=none] (24) at (3.25, 1) {$\Core$};
	\end{pgfonlayer}
	\begin{pgfonlayer}{edgelayer}
		\draw (1.center) to (2.center);
		\draw (2.center) to (3.center);
		\draw (3.center) to (4.center);
		\draw (4.center) to (1.center);
		\draw [->] (8.center) to (9.center);
		\draw [dotted] (16.center) to (17.center);
		\draw [dotted] (17.center) to (18.center);
		\draw [dotted] (18.center) to (19.center);
		\draw [dotted] (19.center) to (16.center);
		\draw (20.center) to (21.center);
		\draw (21.center) to (22.center);
		\draw (22.center) to (23.center);
		\draw (23.center) to (20.center);
	\end{pgfonlayer}
\end{tikzpicture}
\caption{Schematic diagram for MUC}
\label{Fig: MUC}
\end{figure} 

The $\dagger$-isomix category of a MUC is to be thought of as a larger space inside which a (small) unitary category embeds. 
Within the unitary category, $A \simeq A^\dagger$ by the means of the unitary structure map, 
however, outside the unitary core,  an object is not in general isomorphic to its dagger. 
	
In the rest of the thesis, MUCs are exclusively assumed to be symmetric unless stated otherwise.
%Next, we show that the unitary construction on a $\dagger$-isomix category produces a mixed unitary category (MUC) which is couniversal.

Mix unitary categories organize themselves into a 2-category ${\sf MUC}$ (although we shall not discuss the 2-cell structure):
\begin{description}
\item[0-cells:]  Are mix unitary categories $M: \U \to \X$;
\item[1-cells:]  Are MUC morphisms: these are squares of $\dagger$-isomix functors $(F',F,\gamma): M \to N$ commuting up to a
 $\dagger$-linear natural isomorphism $\gamma: MF \Rightarrow F'N$:
 \[ \xymatrix{ \U \ar[d]_{F'} \ar@{}[drr]|{\Downarrow~\gamma} \ar[rr]^M & & \X \ar[d]^F \\ \V \ar[rr]_{N} & & \Y} \]
 The functor $F': \U \to \V$ is between unitary categories, and we demand of it that it preserves unitary structure in the 
 sense of Definition \ref{preserving-unitary-structure}, thus, whenever ${\varphi_A}$ is the unitary structure  then $F'(\varphi_A) 
 \rho^F$ is unitary structure.
 \item[2-cells:] These are ``pillows''  of natural transformations. $(\beta, \beta') : (F, F', \gamma_F) \Rightarrow (G, G', \gamma_G)$ 
is a 2-cell if and only if it satisfies the following equality:
\[ \xymatrix{ \U \ar@/_1pc/[dd]_{G'} \ar@/^1pc/[dd]^{F'}   \ar@{}[ddrr]|
{\Downarrow~\gamma_F} \ar[rr]^M & & \X \ar@/^1pc/[dd]^F \\ 
{\xLeftarrow{\beta'}} & & \\ 
\V \ar[rr]_{N} & & \Y} ~ \xymatrix{ \\ = \\ } ~ \xymatrix{ \U \ar@/_1pc/[dd]_{G'} \ar@{}[ddrr]|
{\Downarrow~\gamma_G} \ar[rr]^M & & \X \ar@/_1pc/[dd]_G \ar@/^1pc/[dd]^F \\ 
 & & {\xLeftarrow{\beta}} \\ 
\V \ar[rr]_{N} & & \Y} \]
 \end{description}
 
 We remark that we have observed that any MUC can be ``simplified'' to a dagger monoidal category with a strong $\dagger$-mix 
 Frobenius functor into a $\dagger$--isomix category: this is achieved by precomposing with ${\sf Mx}_\downarrow$.   This may 
 seem a worthwhile simplification, but it should be recognized that it simply transfers complexity from the unitary category itself 
 onto the preservator which must now ``create'' unitary structure:
 \[ \xymatrix{\U  \ar[d]_{{\sf Mx}^{*}_{\downarrow}} \ar[rr]^M & & \C \ar@{=}[d] \\
                     \U_\downarrow \ar[rr]_{{\sf Mx}_\downarrow;M} & & \C} \]
 Here $\U_\downarrow = (\U,\oa,\oa)$ is viewed as a dagger monoidal category and ${\sf Mx}_\downarrow^{*}$ is the inverse 
 of ${\sf Mx}_{\downarrow}$.  
 The point is that the preservator of the lower arrow ${\sf Mx}_\downarrow;M$ is non-trivial as it must encode the unitary 
 structure of $\U$.
 
\subsection{Canonical mixed unitary categories}

Our objective is now to show that the unitary construction of the previous section gives rise to a right adjoint to the underlying 
2-functor $U: {\sf MUC} \to {\sf MCC}$  where 
the 2-category ${\sf MCC}$ is defined as:

\begin{description}
\item[0-cells:] Its objects are  {\bf mixed $\dagger$-compact categories} (MCC), that is strong $\dagger$-Frobenius functors 
$V: \C \to \Y$ where $\C$ is a compact $\dagger$-LDC, $\Y$ is a $\dagger$-isomix category, 
and $V$ factors through the core of $\Y$ i.e, for all $\forall$ objects $C \in \C$, $Y \in \Y$,   $\exists$ $\mx': V(C) \oa Y \to V(C) \ox Y$ 
such that $\mx~\mx' = 1$ and $\mx' ~ \mx = 1$.
\item[1-cells:]   The 1-cells are squares of mix Frobenius functors which commute up to a linear natural isomorphism;
\item[2-cells:]    Are pillows of natural transformations (which we shall ignore).
\end{description}

An example of a mix $\dagger$-compact category is, of course, the inclusion of the core into a $\dagger$-isomix category 
$C:{\sf Core}(\X) \hookrightarrow \X$;

\begin{proposition}
$U: {\sf MUC} \to {\sf MCC}$ has a right adjoint ${\sf Unitary}: {\sf MCC} \to {\sf MUC}; (\C \to^V \X) \mapsto ({\sf Unitary}(\C) 
\to^{U;V} \X)$.
\end{proposition}

\begin{proof}

%%%%%% TODO: It is $\gamma$ rather than $\gamma^\f_\flat$
The couniversal diagram is as follows:
\[ \xymatrix{ {\U \to^M \X} \ar[rr]^{(F,G,\gamma)} \ar[d]_{(F^\flat,G,\gamma^\flat)} && {\C \to^V \Y} \\
                    {{\sf Unitary}(\C) \to_{U;V} \Y} \ar[urr]_{\epsilon} } \]
where $\epsilon$ is the square on the left and $(F^\flat,G,\gamma^\flat)$ is the square on the right:
 \[ \xymatrix{{\sf Unitary}(\C) \ar[d]_U \ar[r]^{~~~~U} & \C \ar[r]^V & \Y \ar@{=}[d] \\
                         \C \ar[rr]_V && \Y} 
     ~~~~~~~~
     \xymatrix{\U \ar[dr]_F \ar[d]^{F^\flat} \ar@{}[drr]|{~~~~~~~~~\uparrow~\gamma} \ar[rr]^M & & \X \ar[d]^{G} \\
                      {\sf Unitary}(\C) \ar[r]_{~~~~U} & \C \ar[r]_V & \Y} \]
                      
It follows from Proposition \ref{Prop: Couniversal} that the couniversal diagram commute.

\end{proof}

This proposition means that in building a non-trivial MUC from a mixed $\dagger$-compact categories it suffices to show that the 
compact $\dagger$-LDC contains non-trivial pre-unitary objects. 

%%%%%%%%%%%%%%%%%%%%%%%%%%%%%%%%%%%%%%%%%%%%%%%%%%%%%%%

\section{Examples: Mixed unitary categories}
\label{Sec: MUC examples}
%%%%%%%%%%%%%%%%%%%%%%%%%%%%%%%%%%%%%%%%%%%%%%%%%%%%%%%

We have already noted that dagger monoidal categories are automatically 
unitary categories in which the unitary structure is given by identity maps.  
The identity functors then give a rather trivial MUC. 
One can construct a MUC from any $\dagger$-isomix category using the unitary construction: 
for any $\dagger$-isomix category, $\X$, ${\sf Unitary}(\Core(\X)) \to^{U} \Core(\X) \hookrightarrow \X$ is a MUC. 
More excitingly one can take the bicompletion \cite{Joy95} of the $\dagger$-monoidal category: this is a non-trivial $\dagger$-isomix $*$-autonomous category 
extension of the original $\dagger$-monoidal category and provides, thus, an interesting example of how MUCs arise.  

Our purpose in this section is to exhibit some non-trivial manifestations of the various structural components of a MUC.  
To this end we discuss in some detail three basic examples.

%%%%%%%%%%%%%%%%%%%%%%%%%%%%%%%%%%%%%%%%%%%%%%%%%%%%%%%%

\subsection{Finite dimensional framed vector spaces}
\label{Sec: FFVec}
In this section we show that the example ${\sf FFVec}_K$, the category of 
finite dimensional framed vector spaces defined in 
Section \ref{subsection:fdfv} is a unitary category (hence is immediately a mixed unitary category). The unitary structure map on each object $(V, {\cal V})$ is defined as follows:
\[ \varphi_{(V,{\cal  V})}: (V,{\cal  V}) \to (V,{\cal  V})^\dag; v_i \mapsto \widetilde{v_i} \]
and it remains to check the coherences {\bf [U.3]}--{\bf [U.6]}.  First note that {\bf [U.4]} holds immediately by the observation above that 
$\iota(v_i) = \widetilde{\widetilde{v_i}}$.  For {\bf [U.3]} we require that $\varphi_{A^\dag}(\widetilde{a_i}) = (\varphi_A^{-1})^\dag (\widetilde{a_i})$ 
the result is a higher-order term so, we may check that the evaluations are the same on basis elements:
\begin{eqnarray*}
	(\varphi_{A^\dag}(\widetilde{a_i}) ) (\widetilde{a_j}) & = & \widetilde{\widetilde{a_i}}(\widetilde{a_j}) = \partial_{i,j} \\
	((\varphi_{A}^{-1})^\dag(\widetilde{a_i}))(\widetilde{a_j}) & = & \widetilde{a_i} ((\varphi_{A}^{-1})^\dag(\widetilde{a_j})) = \widetilde{a_i}(a_j) =  \partial_{i,j}
\end{eqnarray*}
Note that {\bf [U.5]}(a) and {\bf [U.5]}(b), in this example, require $\lambda_\top = \varphi_\top$ which can easily be verified as each reduces to conjugation.
{\bf [U.6]}(a) and {\bf [U.6]}(b), in this example, are the same requirement which is verified by:
\[ \lambda_\ox(\varphi_A \ox \varphi_B(a_i \ox b_j) ) = \lambda_\ox (\widetilde{a_i} \ox \widetilde{b_j}) = \widetilde{a_i \ox b_j} = \varphi_{A \ox B} (a_i \ox b_j) \]

This gives:

\begin{proposition}
	${\sf FFVec}_K$ with the unitary structure above is a MUC.
\end{proposition}

This raises the question of what precisely the unitary maps of this example are.  To elucidate this we note that  a functor can easily be constructed
$U:{\sf FFVec}_K \to {\sf Mat}(K)$ where, for each object in ${\sf FFVec}_K$ we choose a total order on the elements of the basis and note that 
any map is then given by a matrix acting on the bases: thus a matrix in ${\sf Mat}(K)$ with the appropriate dimensions.  We now observe:


%typed wrong. A!=B.  Do you mean that they have the same underlying vector space associated to them.  <-- No as they have the same dimension in Mat(\X) they are the same objects!
\begin{lemma} 
	An isomorphism $u: (A,{\cal A}) \to (B,{\cal B})$ in ${\sf FFVec}_K$ is unitary 
	if and only if $U(f)$ is unitary in ${\sf Mat}(K)$.
\end{lemma}

\proof While $U$ does not preserve $(\_)^\dag$ on the nose it does so up to the natural equivalence determined by $U(\varphi_A)$ which being a basis 
permutation is a unitary equivalence.   Thus, it is not hard to see that the following diagram commutes:
\[ \xymatrix{ U(B,{\cal B})  \ar[d]_{U(f)^\dag} \ar[rr]^{U(\varphi_B)} & &U((B,{\cal B})^\dag) \ar[d]^{U(f^\dag)} \\
	U(A,{\cal A}) \ar[rr]_{U(\varphi_A)} & & U((A,{\cal A})^\dag) } \]
Recall that in the category of matrices, the dagger is stationary on objects so $U(B,{\cal B}) = U(B,{\cal B})^\dag$.  

Now suppose $u$ is unitary in ${\sf FFVec}_K$  then $u^{-1} = \varphi_B u^\dagger \varphi_A^{-1}$ so that 
\[ U(u)^{-1} = U(u^{-1}) = U(\varphi_B u^\dagger \varphi_A^{-1}) = U(\varphi_B) U(u^\dagger) U(\varphi_A^{-1}) = U(u)^\dagger \]
so that its underlying map is unitary. Conversely, if $U(u)$ is unitary then 
\[ U(u^{-1}) = U(u)^{-1} = U(u)^\dag = U(\varphi_B u^\dagger \varphi_A^{-1}) \]
which immediately implies, as $U$ is faithful, that $u$ is unitary in ${\sf FFVec}_K$.
\endproof

One might reasonably regard this as a rather roundabout way to describe the standard notion of a unitary map.  However, two things of importance have been 
achieved.  First an example of a unitary category with a non-stationary dagger and, thus, a non-identity unitary structure, has been exhibited.  Second we have 
shown how the standard unitary structure may be re-expressed in this formalism using non-stationary constructs. 

%%%%%%%%%%%%%%%%%%%%%%%%%%%%%%%%%%%%%%%%%%%%%%%%%%%%%%%%

\subsection{Finiteness matrices}
In Section \ref{Sec: Finiteness matrices}, we described the category of finiteness matrices, ${\sf FMat}(\C)$. 
The core of ${\sf FMat}(\C)$ is the subcategory determined by objects whose webs are finite sets, that is the objects are $(X, P(X))$ 
where $X$ is a finite set. 
Clearly, $\Core({\sf FMat}(\C))$ is then equivalent to the category of finite dimensional matrices, ${\sf Mat}(\C)$.  
This is a well-known $\dagger$-compact closed category, which is a unitary category with unitary structure given by 
identity maps (as $(\_)^\dagger$ is stationary on objects).  The inclusion ${\cal I}: {\sf Mat}(\C) \to {\sf FMat}(\C)$ provides an important example of a MUC.   

%%%%%%%%%%%%%%%%%%%%%%%%%%%%%%%%%%%%%%%%%%%%%%%%%%%%%%%%

\subsection{The embedding of finite dimensional Hilbert Spaces into Chu Spaces}
\label{Sec: CHU MUC}
In Section \ref{Section: Chu}, we showed that the Chu construction applied to a symmetric conjugative closed monoidal 
category, $\X$, with pullbacks gives a $\dagger$-isomix category. Recall that the dagger in the resulting category of Chu 
spaces is given by composing the conjugation with the dualizing functor.  In this section, we start by discussing, in  general, 
the construction of a mixed unitary category from a Chu category ${\sf Chus}_\X(I)$.  A crucial step in this is to identify objects 
which are in the core of this category.

Recall that a  symmetric monoidal closed category, $\X$, is (degenerately) a compact linearly distributive category and, thus, 
there may be objects which have linear adjoints: these are called {\bf nuclear} objects \cite{HiR89}.  Explicitly a nuclear 
object $A$ in a symmetric monoidal closed  category is an object with $A \multimap B \cong A^{*} \ox B$, where 
$A^* := A \multimap I$.   The nuclear objects form a compact closed subcategory of $\X$ which is conjugative when $\X$ is 
conjugative.  In ${\sf Vec}_\mathbb{C}$ the nucleus consist precisely of the finite dimensional vector spaces.  
If $(\eta, \epsilon): A \dashv\!\!\!\dashv B$ is witness that $A$ (and $B$) are nuclear in $\X$ then the object 
$(A,B,\epsilon,c_\otimes\epsilon)$ is in the core of ${\sf Chus}_\X(I)$ because in the second component of the tensor product 
with any other object $(X,Y,\nu,c_\ox \nu)$ one has the degenerate pullback:
\[ \xymatrix{
& Y \ox B \ar[rd]^{\simeq} \ar[ld] & \\ %Ask Robin how to add a pullback corner
X \multimap B \ar[rd]^{\simeq}  & & Y \ox A^{*} \ar[ld]  \\
& X \multimap A^* \to^{\simeq} (X \multimap I) \ox A &
} \]
where we use the isomorphism $B \to^{\simeq} A^{*}$.

In this manner the nuclear objects of ${\sf Nuclear}(\X)$, which form a compact closed category with a dagger, 
may be embedded into the core of ${\sf Chus}_\X(I)$.  To obtain a unitary category it suffices then to use the unitary 
construction for which, to obtain a non-trivial result, we need to show that there are non-trivial examples of pre-unitary objects. 
To achieve this we consider an object $H$ for which $(e, n): H \dashv\!\!\!\dashv \overline{H}$ and such that $e$ satisfies:
\[ \xymatrix{\overline{\overline{H}} \otimes \overline{H} \ar[d]_{\varepsilon \otimes 1} \ar[r]^{\chi} & \overline{\overline{H} \otimes 
\overline{H}} \ar[dd]^{\overline{e}}\\
                   H \otimes \overline{H} \ar[d]_{e} \\
                   I \ar[r]_{\chi^{\!\!\!\circ}} & \overline{I} } \]
               
 For such an object we note:
 \begin{align*}
{ \overline{(H,\overline{H},e,c_\otimes e)}}^* & =  (\overline{H},\overline{\overline{H}},\chi\overline{c_\otimes e} 
(\chi^{\!\!\!\circ})^{-1},\chi\overline{e} (\chi^{\!\!\!\circ})^{-1})^* \\
 & =  (\overline{\overline{H}},\overline{H},\chi\overline{e} (\chi^{\!\!\!\circ})^{-1},\chi\overline{c_\otimes e} (\chi^{\!\!\!\circ})^{-1})
 \end{align*}
 
This makes
 $$(\varepsilon^{-1},1) : (H,\overline{H},e,c_\otimes e) \to (\overline{\overline{H}},\overline{H},\chi\overline{e} 
 (\chi^{\!\!\!\circ})^{-1},\chi\overline{c_\otimes e} (\chi^{\!\!\!\circ})^{-1})$$
a preunitay map.  Note that it is a Chu map by the commuting diagram above and as $\overline{\varepsilon} =\varepsilon$ we have 
$$(\varepsilon^{-1},1) (1,\overline{\varepsilon}) = (\varepsilon^{-1},1) (1,\varepsilon) = (\varepsilon^{-1},\varepsilon)$$
where $(\varepsilon^{-1},\varepsilon)$ is the involutor.

In ${\sf Vec}_{\mathbb{C}}$ a map $e: H \otimes \overline{H} \to \mathbb{C}$ is a ``sesquilinear form'' and the diagram above 
asserts that it is in addition a symmetric form.  Any  Hilbert space with its inner product, thus, satisfies the above conditions.   
Thus,  it is clear that the embedding of the category of finite dimensional Hilbert Spaces into Chu spaces, ${\sf FHilb} 
\hookrightarrow {\sf Chus}_{{\sf Vec}_\C} (\C)$ is a mixed unitary category.  The embedding is in fact a full and faithful 
embedding which extends to {\em all\/} Hilbert spaces (although only the finite dimensional ones land in the core).  

Explicitly the embedding is defined as follows: suppose $H$ is a (finite dimensional) Hilbert Space, then the corresponding 
Chu Space is given by $(H, \overline{H}, \langle - | - \rangle_H)$, where $\langle - | - \rangle_H: H \ox \overline{H} \to \C$ is 
the inner product. For any linear map $H \to^{f} K$ between Hilbert Spaces, the corresponding Chu map is given by $(f, f^\dagger): 
(H, \overline{H}, \langle - | - \rangle_H) \to (K, \overline{K}, \langle - | - \rangle_K)$, where $f^\dagger$ is the Hermitian adjoint of 
$f$ so, $\langle f(a) | b \rangle = \langle a | f^\dagger(b) \rangle$.

Furthermore, observe that $(H, \overline{H}, \langle - | - \rangle_H)^\dagger :=  \overline{(H, \overline{H}, \langle - | - \rangle_H)^*} = 
(H, \overline{H}, \langle - | - \rangle_H)$. Hence, this embedding preserves the (stationary) dagger for all Hilbert spaces.  However, 
the par of two infinite dimensional Hillbert spaces in this Chu category is not a Hilbert space so that the duality cannot be seen within 
the category of Hilbert spaces.

%%%%%%%%%%%%%%%%%%%%%%%%%%%%%%%%%%%%%%%%%%%%%%%%%%%%%%%%

%\subsection{Constructing MUCs using the unitary construction}
%One can construct a MUC from any $\dagger$-isomix category using the unitary construction: 
%for any $\dagger$-isomix category, $\X$, ${\sf Unitary}(\Core(\X)) \to^{U} \Core(\X) \hookrightarrow \X$ is a MUC. 
%In this manner we have already many examples of MUCs:

%\begin{itemize}
%	\item The inclusion $\C \hookrightarrow \D( \C, +, 0)$
%	\item ${\sf Unitary}(\Core({\mbox{\bf H-Mod}}_\X)) \to^{U} \Core(\mbox{\bf H-Mod}) \hookrightarrow {\mbox{\bf H-Mod}}_\X$
%	\item ${\sf Unitary}(\Core({\sf Chus_\X}(I)) \to^{U} \Core({\sf Chus_\X}(I)) \hookrightarrow {\sf Chus_\X}(I)$
%\end{itemize}

%\medskip

%\noindent
%Another instructive source of examples of MUCs left for future work, uses Joyal's bicompletion procedure \cite{Joy95}:  
%here, starting with a $\dagger$-monoidal category, or a compact $\dagger$-isomix category, $\C$, one can form a MUC $\iota: \C \to \Lambda(\C)$ by 
%simply bicompleting.  Furthermore, the bicompletion is a (non-compact) $\dagger$-isomix category which, when the starting point, $\C$, is $\dagger$-compact 
%closed, is a $\dagger$-isomix $*$-autonomous category.


\chapter{Summary}
\label{chap: part 1 summary}

Chapters \ref{Chap: dagger-LDC}, and \ref{Chap: MUCs} extend the theory of $\dagger$-monoidal categories and $\dagger$-compact closed categories to linearly distributive 
and $*$-autonomous settings to obtain the categorical semantics of (multiplicative) $\dagger$-linear logic.  In these linear settings, the 
two different tensor products (tensor and par) must be flipped by the dagger.  Thus, one cannot have a stationary (identity on objects) 
dagger, and hence one is forced to replace the conventional dagger by a contravariant structure-preserving involution.  This has coherence  
consequences: section \ref{Section: dagger LDC} is dedicated to understanding the details of these coherences.

If multiplicative  $\dagger$-linear logic is to provide a semantics for a generalized categorical quantum mechanics (CQM), then notions 
such as isometry and unitary isomorphism, which are central to CQM, should have an expression in this logic.  In section \ref{Sec: unitary} 
we showed that with additional ``unitary structure'' one can recapture classical CQM as a ``unitary core''  of multiplicative  $\dagger$-linear logic.
Furthermore, we showed how, from any $\dagger$-isomix category, it is always possible to extract  
a ``unitary core'' which is, up to equivalence, a $\dagger$-monoidal category (i.e a classical semantic setting for CQM).   

This led to the notion of a mixed unitary category (MUC) given by a $\dagger$-isomix category with a chosen unitary 
core as our proposal for an extension of CQM.   A MUC can be viewed as an extension much as a $K$-algebra 
extends a field $K$ and permits the expression of properties which are difficult to express within $K$ itself.  
In the extended setting of a MUC -- finiteness matrices with its core for example -- provides an  
extension of the classical CQM setting in which infinite dimensional types, such as those given by the exponential modalities,  
 are present.   Furthermore, in the extended setting one can bend, and yank wires without the category being compact.

 \iffalse
The fact that a unitary category is a component of a MUC allows one to mimic the construction of completely positive 
maps, ${\sf CP}^\infty$, see \cite{CoH16}, in a way which displays 
some interesting features.  To start with the ancillary objects (which are to be traced out) must now, necessarily, be chosen from the 
unitary core and these
 it can be supposed are an essentially small class even though the overall category may be large.  This keeps the number completely 
 positive maps between any two objects small.  The resulting category is under reasonable assumptions a MUC (see \cite{CS19}) 
 which has an appropriate analogue of environment structure \cite{Coecke10}.  Furthermore, in the presence of duals the 
 whole construction is functorial.  
 
An important observation of CQM is that an orthogonal basis, for a Hilbert space, correspond to a special commutative Frobenius algebra 
\cite{CPV12}.  This allows one to replace the notion of a basis by algebraic structure.  A significant consequence of this has been the 
algebraic expression of ``uncertainty" using complementary Frobenius algebras \cite{CoD11} , which, in turn, led to the 
formulation of the ZX-calculus.   As was mentioned in the introduction, linear settings allow the expression of structures which parallel Frobenius 
algebras and this can be exploited to allow an expression complementarity in MUCs and, furthermore, to link this to the exponential 
modalities in $\dagger$-linear logic \cite{CS21}.  
\fi

This concludes the first part of this thesis. The second part discusses the application of MUCs to CQM. 