% !TEX root = thesis.tex

%% Include extra packages here

\usepackage[pdftex,dvipsnames]{xcolor}  % Coloured text etc.
\usepackage[utf8]{inputenc}
\usepackage[colorlinks, allcolors=blue]{hyperref} 
\usepackage{enumerate}
\usepackage{graphicx}
 
\usepackage{pdflscape}

\usepackage{bussproofs}
\usepackage{placeins}

 %% extra packages included by Priyaa
\usepackage{amsmath,amsthm,amsfonts,graphicx}
%\let\circledS\undefined % here - PS
\usepackage{amssymb}
\usepackage{stmaryrd}
\usepackage{tikz}
\usepackage{proof}
\usepackage{enumerate}
\usepackage{mathtools}
\usepackage{xypic}
\usepackage{lscape}
\usepackage{multicol}
\usepackage{leftidx}
\usepackage{bbm}
\usepackage{rotating}
\usepackage{extarrows}
\usepackage{cmll}

\usepackage{multirow}

%%% -- Use biblatex instead of bibtex
\usepackage[english]{babel}
\usepackage[style=numeric, sorting=nty]{biblatex}
\addbibresource{thesis.bib}

%%-------------------------------- tabular in math mode
\usepackage{float}

%%-------------------------------- tabular in math mode
\usepackage{array}   % for \newcolumntype macro
\newcolumntype{L}{>{$}l<{$}} % math-mode version of "l" column type

\addtolength{\parskip}{-0.45mm}

%%  Include other user-defined commands here

  \theoremstyle{plain}
  \newtheorem{theorem}{Theorem}[chapter]
  \newtheorem{lemma}[theorem]{Lemma}
  \newtheorem{corollary}[theorem]{Corollary}
  \newtheorem{proposition}[theorem]{Proposition}
  \newtheorem{remark}[theorem]{Remark}
  \newtheorem{example}[theorem]{Example}

  %\theoremstyle{definition}
  \newtheorem{definition}[theorem]{Definition}
   
  \newcommand{\invamalg}{\mathbin{\rotatebox[origin=c]{180}{$\amalg$}}}

  \newcommand{\s}{{\sf s}}
  \renewcommand{\t}{{\sf t}}
  \renewcommand{\u}{{\sf{u}}}
  \renewcommand{\v}{{\sf{v}}}
  \newcommand{\<}{\langle}
  \renewcommand{\>}{\rangle}
  \newcommand{\X}{\mathbb{X}}
  \newcommand{\A}{\mathbb{A}}
  \newcommand{\B}{\mathbb{B}}
  \newcommand{\C}{\mathbb{C}}
  \newcommand{\D}{\mathbb{D}}
  \newcommand{\I}{\mathbb{I}}
  \newcommand{\J}{\mathbb{J}}
  \newcommand{\N}{\mathbb{N}}
  \newcommand{\U}{\mathbb{U}}
  \newcommand{\V}{\mathbb{V}}
  \newcommand{\R}{\mathbb{R}}
  \newcommand{\Z}{\mathbb{Z}}
  \newcommand{\Y}{\mathbb{Y}}
  \newcommand{\dsa}{$\dag$-$*$-autonomous}  
  \newcommand{\dldc}{$\dag$-LDC}  
  \newcommand{\m}{{\sf m}}
   
  \newcommand{\nat}{\text{nat. }} 
  \newcommand{\id}{\text{id}} 
  \newcommand{\CP}{\mathsf{CP}}
  \newcommand{\ox}{\otimes}
  \newcommand{\pr}{\oplus}
  \newcommand{\oa}{\oplus}
  \newcommand{\op}{\mathsf{op}}
  \newcommand{\rev}{\mathsf{rev}}
  \newcommand{\mx}{\mathsf{mx}}
  \newcommand{\Mx}{\mathsf{Mx}}
  \newcommand{\Chu}{\mathsf{Chu}}
  \newcommand{\Chus}{\mathsf{Chus}}
  \newcommand{\FRel}{\mathsf{FRel}}
  \newcommand{\FMat}{\mathsf{FMat}}
  \newcommand{\Rel}{\mathsf{Rel}}
  \newcommand{\Mat}{\mathsf{Mat}}
  \newcommand{\Hilb}{\mathsf{Hilb}}
  \newcommand{\FHilb}{\mathsf{FHilb}}
  \newcommand{\Core}{\mathsf{Core}}
  \newcommand{\Unitary}{\mathsf{Unitary}}
  \newcommand{\dual}{\text{\reflectbox{$\Vdash$}}}
  \newcommand{\fin}{\mathsf{FinSp}}
  \newcommand{\lollipop}{\ensuremath{\!-\!\!\circ}}
  \renewcommand{\bar}[1]{\overline{#1}}
  \newcommand{\x}{\times}
  \newcommand{\poppilol} {\reflectbox{$\multimap$}}
  
  \newcommand{\dashvv}{\dashv \!\!\!\!\! \dashv}  
  \newcommand{\lindual}{\dashvv}
  
  %\renewcommand{\phi}{\varphi}
  \newcommand{\Asp}{\mathsf{Asp}}

%%%%%%%%%%%%%%%%%%%%%%%%%%%%%%%%%%%%%%%%%%%%%%%

\DeclarePairedDelimiter\ceil{\lceil}{\rceil}
\DeclarePairedDelimiter\floor{\lfloor}{\rfloor}

%%%%%%%%%%%%%%%%%%%%%%%%%%%%%%%%%%%%%%%%%%%%%%%

%%%%%%%%%%%%%%%%%%%%%%%%%%%%%%%%%%%%%%%%%%%%%%%%%%%%%%%%%%%%%%%%%%%%%%%%%
% M. Barr uses the following:  "It gives a \to that can be used as
% $A\to B$ or $A\to^f B$ or $A\to^{f\o g\o h}B$ or even $A\to^f_gB$.  The
% arrow will grow to fit the label(s).  There are similar definitions for
% \two and \tofro, for which you really might want labels both above and
% below.  Actually, by reading your definition of \kto, I was able to
% simplify this.  But it is still nice to have the optional arguments.
% There is only caveat: although you can have one or the other or both
% labels, if you have both the upper must precede the lower.  These defs
% must either be placed in a style file xor surrounded by \makeatletter
% and \makeatother (but NOT both)."  (Modifications by rags)
% The definitions below look more elaborate than they need to be.
% The reason is that an empty asscript will still cause extra vertical
% spacing and the only way to avoid ugly extra space seems to be using
% some such method as this.

\makeatletter

% In-text size:

\newdimen\w@dth

\def\setw@dth#1#2{\setbox\z@\hbox{\scriptsize $#1$}\w@dth=\wd\z@
\setbox\@ne\hbox{\scriptsize $#2$}\ifnum\w@dth<\wd\@ne \w@dth=\wd\@ne \fi
\advance\w@dth by 1.2em}

\def\t@^#1_#2{\allowbreak\def\n@one{#1}\def\n@two{#2}\mathrel
{\setw@dth{#1}{#2}
\mathop{\hbox to \w@dth{\rightarrowfill}}\limits
\ifx\n@one\empty\else ^{\box\z@}\fi
\ifx\n@two\empty\else _{\box\@ne}\fi}}
\def\t@@^#1{\@ifnextchar_ {\t@^{#1}}{\t@^{#1}_{}}}


\def\t@left^#1_#2{\def\n@one{#1}\def\n@two{#2}\mathrel{\setw@dth{#1}{#2}
\mathop{\hbox to \w@dth{\leftarrowfill}}\limits
\ifx\n@one\empty\else ^{\box\z@}\fi
\ifx\n@two\empty\else _{\box\@ne}\fi}}
\def\t@@left^#1{\@ifnextchar_ {\t@left^{#1}}{\t@left^{#1}_{}}}


\def\two@^#1_#2{\def\n@one{#1}\def\n@two{#2}\mathrel{\setw@dth{#1}{#2}
\mathop{\vcenter{\hbox to \w@dth{\rightarrowfill}\kern-1.7ex
                 \hbox to \w@dth{\rightarrowfill}}%
       }\limits
\ifx\n@one\empty\else ^{\box\z@}\fi
\ifx\n@two\empty\else _{\box\@ne}\fi}}
\def\tw@@^#1{\@ifnextchar_ {\two@^{#1}}{\two@^{#1}_{}}}


\def\tofr@^#1_#2{\def\n@one{#1}\def\n@two{#2}\mathrel{\setw@dth{#1}{#2}
\mathop{\vcenter{\hbox to \w@dth{\rightarrowfill}\kern-1.7ex
                 \hbox to \w@dth{\leftarrowfill}}%
       }\limits
\ifx\n@one\empty\else ^{\box\z@}\fi
\ifx\n@two\empty\else _{\box\@ne}\fi}}
\def\t@fr@^#1{\@ifnextchar_ {\tofr@^{#1}}{\tofr@^{#1}_{}}}

% Displaysize:

\newdimen\W@dth
\def\setW@dth#1#2{\setbox\z@\hbox{$#1$}\W@dth=\wd\z@
\setbox\@ne\hbox{$#2$}\ifnum\W@dth<\wd\@ne \W@dth=\wd\@ne \fi
\advance\W@dth by 1.2em}

\def\T@^#1_#2{\allowbreak\def\N@one{#1}\def\N@two{#2}\mathrel
{\setW@dth{#1}{#2}
\mathop{\hbox to \W@dth{\rightarrowfill}}\limits
\ifx\N@one\empty\else ^{\box\z@}\fi
\ifx\N@two\empty\else _{\box\@ne}\fi}}
\def\T@@^#1{\@ifnextchar_ {\T@^{#1}}{\T@^{#1}_{}}}


\def\T@left^#1_#2{\def\N@one{#1}\def\N@two{#2}\mathrel{\setW@dth{#1}{#2}
\mathop{\hbox to \W@dth{\leftarrowfill}}\limits
\ifx\N@one\empty\else ^{\box\z@}\fi
\ifx\N@two\empty\else _{\box\@ne}\fi}}
\def\T@@left^#1{\@ifnextchar_ {\T@left^{#1}}{\T@left^{#1}_{}}}


\def\Tofr@^#1_#2{\def\N@one{#1}\def\N@two{#2}\mathrel{\setW@dth{#1}{#2}
\mathop{\vcenter{\hbox to \W@dth{\rightarrowfill}\kern-1.7ex
                 \hbox to \W@dth{\leftarrowfill}}%
       }\limits
\ifx\N@one\empty\else ^{\box\z@}\fi
\ifx\N@two\empty\else _{\box\@ne}\fi}}
\def\T@fr@^#1{\@ifnextchar_ {\Tofr@^{#1}}{\Tofr@^{#1}_{}}}


\def\Two@^#1_#2{\def\N@one{#1}\def\N@two{#2}\mathrel{\setW@dth{#1}{#2}
\mathop{\vcenter{\hbox to \W@dth{\rightarrowfill}\kern-1.7ex
                 \hbox to \W@dth{\rightarrowfill}}%
       }\limits
\ifx\N@one\empty\else ^{\box\z@}\fi
\ifx\N@two\empty\else _{\box\@ne}\fi}}
\def\Tw@@^#1{\@ifnextchar_ {\Two@^{#1}}{\Two@^{#1}_{}}}


\def\to{\@ifnextchar^ {\t@@}{\t@@^{}}}
\def\from{\@ifnextchar^ {\t@@left}{\t@@left^{}}}
\def\tofro{\@ifnextchar^ {\t@fr@}{\t@fr@^{}}}
\def\To{\@ifnextchar^ {\T@@}{\T@@^{}}}
\def\From{\@ifnextchar^ {\T@@left}{\T@@left^{}}}
\def\Two{\@ifnextchar^ {\Tw@@}{\Tw@@^{}}}
\def\Tofro{\@ifnextchar^ {\T@fr@}{\T@fr@^{}}}

\makeatother

\newcommand{\pullbackcorner}[1][ul]{\save*!/#1+1.2pc/#1:(1,-1)@^{|-}\restore}
\newcommand{\pushoutcorner}[1][dr]{\save*!/#1-1.2pc/#1:(-1,1)@^{|-}\restore}

%%%%%%%%%%%%%%%%%% TikZ %%%%%%%%%%%%%%%%%%%%%%
%\newcolumntype{L}{>{$}l<{$}} % math-mode version of "l" column type

\tikzstyle{strings}=[baseline={([yshift=-.5ex]current bounding box.center)}]

%Global tikz scaling

\tikzset{every picture/.append style={scale=.5}, transform shape, strings}

\tikzset{%
symbol/.style={%
draw=none,
every to/.append style={%
edge node={node [sloped, allow upside down, auto=false]{$#1$}}}
}
}

\usetikzlibrary{shapes.geometric}
\usetikzlibrary{patterns}
\usetikzlibrary{fit}
\usetikzlibrary{positioning}
\usetikzlibrary{calc}
\usetikzlibrary{arrows}
\usetikzlibrary{decorations.markings}
\usetikzlibrary{decorations.pathreplacing}
\usetikzlibrary{shapes}

%% -------------------------------------- Declare the layers
\pgfdeclarelayer{nodelayer}
\pgfdeclarelayer{edgelayer}
\pgfsetlayers{edgelayer,nodelayer,main}


%% -------------------------------------- Declare the styles
\tikzset{simple/.style={}}
\tikzset{nothing/.style={outer sep=-3.4pt}}
\tikzset{map/.style={draw,fill=white, rectangle}}
% Edge styles
\tikzstyle{filled}=[-, fill=black]

\tikzset{dot/.style={thick, fill=black, circle, scale=1, inner sep = .05cm}}

\tikzset{oa/.style={draw, scale=0.9,minimum height=.1cm,circle,append after command={
[shorten >=\pgflinewidth, shorten <=\pgflinewidth,]
(\tikzlastnode.north) edge (\tikzlastnode.south)
(\tikzlastnode.east) edge (\tikzlastnode.west)
} } }

\tikzset{ox/.style={draw, scale=0.9,minimum height=.1cm,circle,append after command={
[shorten >=\pgflinewidth, shorten <=\pgflinewidth,]
(\tikzlastnode.north west) edge (\tikzlastnode.south east)
(\tikzlastnode.north east) edge (\tikzlastnode.south west) } } }

\tikzset{circ/.style={
shape=circle, inner sep=1pt, draw}}

% Styles added by Priyaa
\tikzstyle{none}=[inner sep=-1pt]
\tikzstyle{circle}=[shape=circle,draw]

\tikzstyle{onehalfcircle}=[shape=circle, scale=1.5, draw]
\tikzstyle{twocircle}=[shape=circle, scale=2, draw]
\tikzstyle{black}=[shape=circle, fill=black, draw]


\newcommand*{\StrikeThruDistance}{0.15cm}%
\newcommand*{\StrikeThru}{\StrikeThruDistance,\StrikeThruDistance}%

\tikzset{wires/.style={}}

\tikzset{box/.style={inner sep=0pt, thick, draw=black, text height=1.5ex, text depth=.25ex, 
text centered, minimum height=3em, anchor=center}}

%%%%%%%%%%%%%%%%%%%%%%%%%%%%%%%%%%%%%%%%%%%%%%%%%%%%%%%%%%%%%%%%%%%%%%%%
%                           TikZ definition                            %


\newcommand{\envmap}{
\begin{tikzpicture}[scale=1.5]
\draw (0,0.25) -- (0,-0.05);
\draw (-0.15,-0.05) -- (0.15,-0.05);
\draw (-0.10,-0.1) -- (0.10,-0.1);
\draw (-0.05,-0.15) -- (0.05,-0.15);
\end{tikzpicture}
}

\newcommand{\anotherenvmap}[1]{
\begin{tikzpicture}[scale=#1]
	\begin{pgfonlayer}{nodelayer}
		\node [style=none] (0) at (-2, 0.7) {};
		\node [style=none] (1) at (-2, 0.5) {};
		\node [style=none] (2) at (-2.12, 0.5) {};
		\node [style=none] (3) at (-1.88, 0.5) {};
	\end{pgfonlayer}
	\begin{pgfonlayer}{edgelayer}
		\draw [style=none] (0.center) to (1.center);
		\draw [style=none, bend right=90, looseness=1.75] (2.center) to (3.center);
		\draw [style=none] (2.center) to (3.center);
	\end{pgfonlayer}
\end{tikzpicture}
}

% for multiplication and comultiplication maps
% arguments - scale and color of dot
\newcommand{\mulmap}[2]{
	\begin{tikzpicture}[scale={#1}]
		\begin{pgfonlayer}{nodelayer}
			\node [style=circle, scale=0.4, fill={#2}] (5) at (0.32, 0.25) {};
			\node [style=none] (6) at (0.07, 0.5) {};
			\node [style=none] (7) at (0.57, 0.5) {};
			\node [style=none] (8) at (0.32, 0) {};
			\node [style=none] (9) at (0.64, 0.5) {};
		\end{pgfonlayer}
		\begin{pgfonlayer}{edgelayer}
			\draw [style=none] (8.center) to (5);
			\draw [style=none, bend left, looseness=1.25] (5) to (6.center);
			\draw [style=none, bend right, looseness=1.25] (5) to (7.center);
		\end{pgfonlayer}
	\end{tikzpicture}	
}

%Argument 1: scale
%Argument 2: color
\newcommand{\leftaction}[2]{
\begin{tikzpicture}[scale=#1]
	\begin{pgfonlayer}{nodelayer}
		\node [style=none] (0) at (0.5, 0.25) {};
		\node [style=none] (1) at (0.5, 0.75) {};
		\node [style=none] (2) at (0, 0.75) {};
		\node [style=none] (3) at (0.5, -0.75) {};
		\node [style=none] (4) at (0.5, 2) {};
		\node [style=none] (5) at (-0.75, 2) {};
	\end{pgfonlayer}
	\begin{pgfonlayer}{edgelayer}
		\draw[fill=#2] (0.center) -- (1.center) -- (2.center) -- (0.center);
		\draw (3.center) to (0.center);
		\draw (1.center) to (4.center);
		\draw [bend left, looseness=1.00] (2.center) to (5.center);
	\end{pgfonlayer}
\end{tikzpicture}
}

%Argument 1: scale
%Argument 2: color
\newcommand{\rightaction}[2]{
	\begin{tikzpicture}[scale=#1, xscale=-1]
	\begin{pgfonlayer}{nodelayer}
	\node [style=none] (0) at (0.5, 0.25) {};
	\node [style=none] (1) at (0.5, 0.75) {};
	\node [style=none] (2) at (0, 0.75) {};
	\node [style=none] (3) at (0.5, -0.75) {};
	\node [style=none] (4) at (0.5, 2) {};
	\node [style=none] (5) at (-0.75, 2) {};
	\end{pgfonlayer}
	\begin{pgfonlayer}{edgelayer}
	\draw[fill=#2] (0.center) -- (1.center) -- (2.center) -- (0.center);
	\draw (3.center) to (0.center);
	\draw (1.center) to (4.center);
	\draw [bend left, looseness=1.00] (2.center) to (5.center);
	\end{pgfonlayer}
	\end{tikzpicture} }


%Argument 1: scale
%Argument 2: color
\newcommand{\leftcoaction}[2]{
	\begin{tikzpicture}[scale=#1, yscale=-1]
	\begin{pgfonlayer}{nodelayer}
	\node [style=none] (0) at (0.5, 0.25) {};
	\node [style=none] (1) at (0.5, 0.75) {};
	\node [style=none] (2) at (0, 0.75) {};
	\node [style=none] (3) at (0.5, -0.75) {};
	\node [style=none] (4) at (0.5, 2) {};
	\node [style=none] (5) at (-0.75, 2) {};
	\end{pgfonlayer}
	\begin{pgfonlayer}{edgelayer}
	\draw[fill=#2] (0.center) -- (1.center) -- (2.center) -- (0.center);
	\draw (3.center) to (0.center);
	\draw (1.center) to (4.center);
	\draw [bend left, looseness=1.00] (2.center) to (5.center);
	\end{pgfonlayer}
	\end{tikzpicture} }

%Argument 1: scale
%Argument 2: color
\newcommand{\rightcoaction}[2]{
	\begin{tikzpicture}[scale=#1, yscale=-1, xscale=-1]
	\begin{pgfonlayer}{nodelayer}
	\node [style=none] (0) at (0.5, 0.25) {};
	\node [style=none] (1) at (0.5, 0.75) {};
	\node [style=none] (2) at (0, 0.75) {};
	\node [style=none] (3) at (0.5, -0.75) {};
	\node [style=none] (4) at (0.5, 2) {};
	\node [style=none] (5) at (-0.75, 2) {};
	\end{pgfonlayer}
	\begin{pgfonlayer}{edgelayer}
	\draw[fill=#2] (0.center) -- (1.center) -- (2.center) -- (0.center);
	\draw (3.center) to (0.center);
	\draw (1.center) to (4.center);
	\draw [bend left, looseness=1.00] (2.center) to (5.center);
	\end{pgfonlayer}
	\end{tikzpicture}
}

% need to specify scale and color of dot
\newcommand{\unitmap}[2]{
\begin{tikzpicture}[scale=#1]
	\begin{pgfonlayer}{nodelayer}
		\node [style=circle, scale=0.4, fill=#2] (0) at (0, 0) {};
		\node [style=none] (1) at (0, -0.4) {};
		\node [style=none] (4) at (0.13, 0) {};
	\end{pgfonlayer}
	\begin{pgfonlayer}{edgelayer}
		\draw [style=none] (0) to (1.center);
	\end{pgfonlayer}
\end{tikzpicture} }

% need to specify scale and color of dot
\newcommand{\counitmap}[2]{
\begin{tikzpicture}[scale=#1, rotate=180]
	\begin{pgfonlayer}{nodelayer}
		\node [style=circle, scale=0.4, fill=#2] (0) at (0, 0) {};
		\node [style=none] (1) at (0, -0.4) {};
		\node [style=none] (4) at (0.13, 0) {};
	\end{pgfonlayer}
	\begin{pgfonlayer}{edgelayer}
		\draw [style=none] (0) to (1.center);
	\end{pgfonlayer}
\end{tikzpicture}
}

% need to specify scale and color of dot
\newcommand{\bialgunitmap}[1]{
\begin{tikzpicture}[scale=#1]
	\begin{pgfonlayer}{nodelayer}
		\node [style=none] (0) at (-1, 2) {};
		\node [style=none] (1) at (-0.5, 2) {};
		\node [style=none] (2) at (-0.75, 1.75) {};
		\node [style=none] (3) at (-0.75, 1.25) {};
	\end{pgfonlayer}
	\begin{pgfonlayer}{edgelayer}
		\draw (0.center) to (1.center);
		\draw (1.center) to (2.center);
		\draw (2.center) to (0.center);
		\draw (2.center) to (3.center);
		\draw[fill=white] (0) -- (1) -- (2) -- (0);
	\end{pgfonlayer}
\end{tikzpicture}
}

% need to specify scale and color of dot
\newcommand{\bialgcounitmap}[1]{
\begin{tikzpicture}[scale=#1]
	\begin{pgfonlayer}{nodelayer}
		\node [style=none] (0) at (-1, 1) {};
		\node [style=none] (1) at (-0.5, 1) {};
		\node [style=none] (2) at (-0.75, 1.25) {};
		\node [style=none] (3) at (-0.75, 1.75) {};
	\end{pgfonlayer}
	\begin{pgfonlayer}{edgelayer}
		\draw (0.center) to (1.center);
		\draw (1.center) to (2.center);
		\draw (2.center) to (0.center);
		\draw (3.center) to (2.center);
		\draw[fill=white] (0) -- (1) -- (2) -- (0);
	\end{pgfonlayer}
\end{tikzpicture}
}

\newcommand{\twistedmulmap}[2]{
\begin{tikzpicture}[scale=#1]
	\begin{pgfonlayer}{nodelayer}
		\node [style=none] (0) at (1.75, -0.5) {};
		\node [style=circle, scale=0.4, fill=#2] (1) at (1.5, -1.25) {};
		\node [style=none] (2) at (1.25, -0.5) {};
		\node [style=none] (3) at (1.5, -1.5) {};
	\end{pgfonlayer}
	\begin{pgfonlayer}{edgelayer}
		\draw [style=none, in=150, out=-90, looseness=2.00] (0.center) to (1);
		\draw [style=none, in=-90, out=30, looseness=1.75] (1) to (2.center);
		\draw [style=none] (1) to (3.center);
	\end{pgfonlayer}
\end{tikzpicture}
}

\newcommand{\twincomul}[2] {
\begin{tikzpicture}[scale={#1}]
	\begin{pgfonlayer}{nodelayer}
		\node [style=circle, scale=0.6, fill=#2] (0) at (-2, 1) {};
		\node [style=circle, scale=0.6, fill=#2] (1) at (-1.25, 1) {};
		\node [style=none] (2) at (-2, 1.5) {};
		\node [style=none] (3) at (-1.25, 1.5) {};
		\node [style=none] (4) at (-1.75, 0.5) {};
		\node [style=none] (5) at (-0.75, 0.5) {};
		\node [style=none] (6) at (-2.5, 0.5) {};
		\node [style=none] (7) at (-1.5, 0.5) {};
	\end{pgfonlayer}
	\begin{pgfonlayer}{edgelayer}
		\draw [bend right=45] (1) to (4.center);
		\draw [bend left=45] (0) to (7.center);
		\draw [bend right=45] (0) to (6.center);
		\draw (0) to (2.center);
		\draw (3.center) to (1);
		\draw [bend left=45] (1) to (5.center);
	\end{pgfonlayer}
\end{tikzpicture}}

\newcommand{\twinmul}[2] {
\begin{tikzpicture}[scale={#1}, yscale=-1]
	\begin{pgfonlayer}{nodelayer}
		\node [style=circle, scale=0.6, fill=#2] (0) at (-2, 1) {};
		\node [style=circle, scale=0.6, fill=#2] (1) at (-1.25, 1) {};
		\node [style=none] (2) at (-2, 1.5) {};
		\node [style=none] (3) at (-1.25, 1.5) {};
		\node [style=none] (4) at (-1.75, 0.5) {};
		\node [style=none] (5) at (-0.75, 0.5) {};
		\node [style=none] (6) at (-2.5, 0.5) {};
		\node [style=none] (7) at (-1.5, 0.5) {};
	\end{pgfonlayer}
	\begin{pgfonlayer}{edgelayer}
		\draw [bend right=45] (1) to (4.center);
		\draw [bend left=45] (0) to (7.center);
		\draw [bend right=45] (0) to (6.center);
		\draw (0) to (2.center);
		\draw (3.center) to (1);
		\draw [bend left=45] (1) to (5.center);
	\end{pgfonlayer}
\end{tikzpicture}}

\newcommand{\twincounit}[2] {
\begin{tikzpicture}[scale={#1}]
	\begin{pgfonlayer}{nodelayer}
		\node [style=circle, scale=0.6, fill=#2] (0) at (-2, 0.5) {};
		\node [style=circle, scale=0.6, fill=#2] (1) at (-1.5, 0.5) {};
		\node [style=none] (2) at (-2, 1.25) {};
		\node [style=none] (3) at (-1.5, 1.25) {};
	\end{pgfonlayer}
	\begin{pgfonlayer}{edgelayer}
		\draw (0) to (2.center);
		\draw (3.center) to (1);
	\end{pgfonlayer}
\end{tikzpicture} }

\newcommand{\twinunit}[2] {
\begin{tikzpicture}[scale={#1}, yscale=-1]
	\begin{pgfonlayer}{nodelayer}
		\node [style=circle, scale=0.6, fill=#2] (0) at (-2, 0.5) {};
		\node [style=circle, scale=0.6, fill=#2] (1) at (-1.5, 0.5) {};
		\node [style=none] (2) at (-2, 1.25) {};
		\node [style=none] (3) at (-1.5, 1.25) {};
	\end{pgfonlayer}
	\begin{pgfonlayer}{edgelayer}
		\draw (0) to (2.center);
		\draw (3.center) to (1);
	\end{pgfonlayer}
\end{tikzpicture} }

\newcommand{\productunitmap}[3]{
\begin{tikzpicture}[scale=#1]
	\begin{pgfonlayer}{nodelayer}
		\node [style=circle, scale=0.4, fill=#2] (0) at (0, 0) {};
		\node [style=none] (1) at (0, -0.5) {};
		\node [style=none] (2) at (0.25, -0.5) {};
		\node [style=circle, scale=0.4, fill=#3] (3) at (0.25, 0) {};
	\end{pgfonlayer}
	\begin{pgfonlayer}{edgelayer}
		\draw [style=none] (0) to (1.center);
		\draw [style=none] (3) to (2.center);
	\end{pgfonlayer}
\end{tikzpicture}
}

\newcommand{\productmulmap}[3]{
\begin{tikzpicture}[scale=#1]
	\begin{pgfonlayer}{nodelayer}
		\node [style=circle, scale=0.4, fill=#2] (0) at (0, -0.5) {};
		\node [style=circle, scale=0.4, fill=#3] (1) at (0.5, -0.5) {};
		\node [style=none] (2) at (-0.5, 0) {};
		\node [style=none] (3) at (0, 0) {};
		\node [style=none] (4) at (0.5, 0) {};
		\node [style=none] (5) at (1, 0) {};
		\node [style=none] (6) at (0, -1) {};
		\node [style=none] (7) at (0.5, -1) {};
	\end{pgfonlayer}
	\begin{pgfonlayer}{edgelayer}
		\draw [style=none, bend right, looseness=1.00] (2.center) to (0);
		\draw [style=none, bend right, looseness=1.00] (0) to (4.center);
		\draw [style=none, bend right=15, looseness=1.25] (3.center) to (1);
		\draw [style=none, bend right, looseness=1.00] (1) to (5.center);
		\draw [style=none] (0) to (6.center);
		\draw [style=none] (1) to (7.center);
	\end{pgfonlayer}
\end{tikzpicture}
}

\newcommand{\comulmap}[2]{
	\begin{tikzpicture}[scale={#1}]
		\begin{pgfonlayer}{nodelayer}
			\node [style=circle, scale=0.4, fill={#2}] (5) at (0.32, 0.25) {};
			\node [style=none] (6) at (0.07, 0) {};
			\node [style=none] (7) at (0.57, 0) {};
			\node [style=none] (8) at (0.32, 0.5) {};
			\node [style=none] (9) at (0.64, 0) {};
		\end{pgfonlayer}
		\begin{pgfonlayer}{edgelayer}
			\draw [style=none] (8.center) to (5);
			\draw [style=none, bend right, looseness=1.25] (5) to (6.center);
			\draw [style=none, bend left, looseness=1.25] (5) to (7.center);
		\end{pgfonlayer}
	\end{tikzpicture}
}

%%argument 1: scale
%%argument 2: color
\newcommand{\conjugateaction}[2]{ 
\begin{tikzpicture}[scale=#1]
	\begin{pgfonlayer}{nodelayer}
		\node [style=none] (0) at (0, 1) {};
		\node [style=none] (1) at (0, 0.75) {};
		\node [style=none] (2) at (-0.25, 1) {};
		\node [style=circle] (3) at (0, 1.5) {};
		\node [style=none] (4) at (0, 2) {};
		\node [style=none] (5) at (-0.75, 2) {};
		\node [style=none] (6) at (0, 0.5) {};
		\node [style=none] (7) at (0, 1.5) {$s$};
	\end{pgfonlayer}
	\begin{pgfonlayer}{edgelayer}
		\draw (4.center) to (3);
		\draw (3) to (0.center);
		\draw (2.center) to (0.center);
		\draw (0.center) to (1.center);
		\draw (1.center) to (2.center);
		\draw [bend right, looseness=1.00] (5.center) to (2.center);
		\draw (1.center) to (6.center);
	\end{pgfonlayer}
	\draw[fill=#2] (0.center) -- (1.center) -- (2.center) -- (0.center);
\end{tikzpicture}
}

\newcommand{\trianglemult}[1]{
\begin{tikzpicture}[scale=#1]
	\begin{pgfonlayer}{nodelayer}
		\node [style=none] (0) at (-0.25, 3.5) {};
		\node [style=none] (1) at (-0.5, 3.75) {};
		\node [style=none] (2) at (0, 3.75) {};
		\node [style=none] (3) at (-0.25, 3) {};
		\node [style=none] (4) at (0.25, 4.25) {};
		\node [style=none] (5) at (-0.75, 4.25) {};
	\end{pgfonlayer}
	\begin{pgfonlayer}{edgelayer}
		\draw [bend right, looseness=1.00] (2.center) to (4.center);
		\draw (0.center) to (1.center);
		\draw (0.center) to (2.center);
		\draw (2.center) to (1.center);
		\draw [in=-90, out=165, looseness=0.75] (1.center) to (5.center);
		\draw (0.center) to (3.center);
	\end{pgfonlayer}
\end{tikzpicture} }

\newcommand{\trianglemultblack}[1]{
\begin{tikzpicture}[scale=#1]
	\begin{pgfonlayer}{nodelayer}
		\node [style=none] (0) at (-0.25, 3.5) {};
		\node [style=none] (1) at (-0.5, 3.75) {};
		\node [style=none] (2) at (0, 3.75) {};
		\node [style=none] (3) at (-0.25, 3) {};
		\node [style=none] (4) at (0.25, 4.25) {};
		\node [style=none] (5) at (-0.75, 4.25) {};
	\end{pgfonlayer}
	\begin{pgfonlayer}{edgelayer}
		\draw [bend right, looseness=1.00] (2.center) to (4.center);
		\draw [fill=black] (0.center) -- (1.center) --  (2.center) -- (0.center);
		\draw [in=-90, out=165, looseness=0.75] (1.center) to (5.center);
		\draw (0.center) to (3.center);
	\end{pgfonlayer}
\end{tikzpicture} }

\newcommand{\trianglecomult}[1]{
\begin{tikzpicture}[scale=#1]
	\begin{pgfonlayer}{nodelayer}
		\node [style=none] (0) at (-0.25, 3.75) {};
		\node [style=none] (1) at (-0.5, 3.5) {};
		\node [style=none] (2) at (0, 3.5) {};
		\node [style=none] (3) at (-0.25, 4.25) {};
		\node [style=none] (4) at (0.25, 3) {};
		\node [style=none] (5) at (-0.75, 3) {};
	\end{pgfonlayer}
	\begin{pgfonlayer}{edgelayer}
		\draw [bend left, looseness=1.00] (2.center) to (4.center);
		\draw (0.center) to (1.center);
		\draw (0.center) to (2.center);
		\draw (2.center) to (1.center);
		\draw [in=90, out=-165, looseness=0.75] (1.center) to (5.center);
		\draw (0.center) to (3.center);
	\end{pgfonlayer}
\end{tikzpicture}
}

\newcommand{\trianglecomultblack}[1]{
\begin{tikzpicture}[scale=#1]
	\begin{pgfonlayer}{nodelayer}
		\node [style=none] (0) at (-0.25, 3.75) {};
		\node [style=none] (1) at (-0.5, 3.5) {};
		\node [style=none] (2) at (0, 3.5) {};
		\node [style=none] (3) at (-0.25, 4.25) {};
		\node [style=none] (4) at (0.25, 3) {};
		\node [style=none] (5) at (-0.75, 3) {};
	\end{pgfonlayer}
	\begin{pgfonlayer}{edgelayer}
		\draw [bend left, looseness=1.00] (2.center) to (4.center);
		\draw [fill=black] (0.center) -- (1.center) --  (2.center) -- (0.center);
		\draw [in=90, out=-165, looseness=0.75] (1.center) to (5.center);
		\draw (0.center) to (3.center);
	\end{pgfonlayer}
\end{tikzpicture}
}

\newcommand{\trianglecounit}[1]{
\begin{tikzpicture}[scale=#1]
	\begin{pgfonlayer}{nodelayer}
		\node [style=none] (0) at (-0.25, 3.5) {};
		\node [style=none] (1) at (-0.5, 3.25) {};
		\node [style=none] (2) at (0, 3.25) {};
		\node [style=none] (3) at (-0.25, 4.25) {};
		\node [style=none] (4) at (-0.25, 2.8) {};
	\end{pgfonlayer}
	\begin{pgfonlayer}{edgelayer}
		\draw (0.center) to (1.center);
		\draw (0.center) to (2.center);
		\draw (2.center) to (1.center);
		\draw (0.center) to (3.center);
	\end{pgfonlayer}
\end{tikzpicture}~\!\!}

\newcommand{\trianglecounitblack}[1]{
\begin{tikzpicture}[scale=#1]
	\begin{pgfonlayer}{nodelayer}
		\node [style=none] (0) at (-0.25, 3.5) {};
		\node [style=none] (1) at (-0.5, 3.25) {};
		\node [style=none] (2) at (0, 3.25) {};
		\node [style=none] (3) at (-0.25, 4.25) {};
		\node [style=none] (4) at (-0.25, 2.8) {};
	\end{pgfonlayer}
	\begin{pgfonlayer}{edgelayer}
		\draw [fill=black] (0.center) -- (1.center) --  (2.center) -- (0.center);
		\draw (0.center) to (3.center);
	\end{pgfonlayer}
\end{tikzpicture}~\!\!}

\newcommand{\triangleunit}[1]{
\begin{tikzpicture}[scale=#1]
	\begin{pgfonlayer}{nodelayer}
		\node [style=none] (0) at (-0.25, 4) {};
		\node [style=none] (1) at (-0.5, 4.25) {};
		\node [style=none] (2) at (0, 4.25) {};
		\node [style=none] (3) at (-0.25, 3.25) {};
		\node [style=none] (4) at (-0.25, 3) {};
	\end{pgfonlayer}
	\begin{pgfonlayer}{edgelayer}
		\draw (0.center) to (1.center);
		\draw (0.center) to (2.center);
		\draw (2.center) to (1.center);
		\draw (0.center) to (3.center);
	\end{pgfonlayer}
\end{tikzpicture}~\!\!}

\newcommand{\triangleunitblack}[1]{
\begin{tikzpicture}[scale=#1]
	\begin{pgfonlayer}{nodelayer}
		\node [style=none] (0) at (-0.25, 4) {};
		\node [style=none] (1) at (-0.5, 4.25) {};
		\node [style=none] (2) at (0, 4.25) {};
		\node [style=none] (3) at (-0.25, 3.25) {};
		\node [style=none] (4) at (-0.25, 3) {};
	\end{pgfonlayer}
	\begin{pgfonlayer}{edgelayer}
		\draw [fill=black] (0.center) -- (1.center) --  (2.center) -- (0.center);
		\draw (0.center) to (3.center);
	\end{pgfonlayer}
\end{tikzpicture}~\!\!}

% Linear monoids and comonoids and bialgebras
\newcommand{\linmonwtik} {\begin{tikzpicture}
	\begin{pgfonlayer}{nodelayer}
		\node [style=none] (0) at (-2.7, 1.17) {};
		\node [style=none] (1) at (-1.85, 1.17) {};
		\node [style=none] (2) at (-2, 1.35) {};
		\node [style=none] (3) at (-2, 1) {};
		\node [style=none] (4) at (-1.85, 1) {};
		\node [style=none] (5) at (-1.85, 1.35) {};
		\node [style=circle, scale=0.6] (6) at (-2.35, 1.45) {};
		\node [style=none] (7) at (-1.6, 1.17) {};
		\node [style=none] (8) at (-2.95, 1.17) {};
	\end{pgfonlayer}
	\begin{pgfonlayer}{edgelayer}
		\draw (2.center) to (3.center);
		\draw (5.center) to (4.center);
		\draw (0.center) to (1.center);
	\end{pgfonlayer}
\end{tikzpicture}}

\newcommand{\expmonwtik} {\begin{tikzpicture}
	\begin{pgfonlayer}{nodelayer}
		\node [style=none] (0) at (-2.8, 1.17) {};
		\node [style=none] (1) at (-1.85, 1.17) {};
		\node [style=none] (2) at (-2, 1.35) {};
		\node [style=none] (3) at (-2, 1) {};
		\node [style=none] (4) at (-1.85, 1) {};
		\node [style=none] (5) at (-1.85, 1.35) {};
		\node [style=circle, scale=0.6] (6) at (-2.35, 1.55) {};
		\node [style=map, scale=1.7, fill opacity=0] (9) at (-2.35, 1.55) {};
		\node [style=none] (7) at (-1.6, 1.17) {};
		\node [style=none] (8) at (-3.05, 1.17) {};
	\end{pgfonlayer}
	\begin{pgfonlayer}{edgelayer}
		\draw (2.center) to (3.center);
		\draw (5.center) to (4.center);
		\draw (0.center) to (1.center);
	\end{pgfonlayer}
\end{tikzpicture}}

\newcommand{\dagmonwtik} {\begin{tikzpicture}
	\begin{pgfonlayer}{nodelayer}
		\node [style=none] (0) at (-2.7, 1.17) {};
		\node [style=none] (1) at (-1.85, 1.17) {};
		\node [style=none] (2) at (-2, 1.35) {};
		\node [style=none] (3) at (-2, 1) {};
		\node [style=none] (4) at (-1.85, 1) {};
		\node [style=none] (5) at (-1.85, 1.35) {};
		\node [style=circle, scale=0.6] (6) at (-2.25, 1.45) {};
		\node [style=none] (7) at (-1.6, 1.17) {};
		\node [style=none] (8) at (-2.95, 1.17) {};
		\node [style=none, scale=1.5] (9) at (-2.55, 1.45) {$\dag$};
	\end{pgfonlayer}
	\begin{pgfonlayer}{edgelayer}
		\draw (2.center) to (3.center);
		\draw (5.center) to (4.center);
		\draw (0.center) to (1.center);
	\end{pgfonlayer}
\end{tikzpicture}}

\newcommand{\linmonbtik} {\begin{tikzpicture}
	\begin{pgfonlayer}{nodelayer}
		\node [style=none] (0) at (-2.7, 1.17) {};
		\node [style=none] (1) at (-1.85, 1.17) {};
		\node [style=none] (2) at (-2, 1.35) {};
		\node [style=none] (3) at (-2, 1) {};
		\node [style=none] (4) at (-1.85, 1) {};
		\node [style=none] (5) at (-1.85, 1.35) {};
		\node [style=circle, scale=0.6, fill=black] (6) at (-2.35, 1.45) {};
		\node [style=none] (7) at (-1.6, 1.17) {};
		\node [style=none] (8) at (-2.95, 1.17) {};
	\end{pgfonlayer}
	\begin{pgfonlayer}{edgelayer}
		\draw (2.center) to (3.center);
		\draw (5.center) to (4.center);
		\draw (0.center) to (1.center);
	\end{pgfonlayer}
\end{tikzpicture}}

\newcommand{\lincomonwtik} {\begin{tikzpicture} %linear comonoid with white circle
	\begin{pgfonlayer}{nodelayer}
		\node [style=none] (0) at (-2.7, 1.17) {};
		\node [style=none] (1) at (-1.85, 1.17) {};
		\node [style=none] (2) at (-2, 1.35) {};
		\node [style=none] (3) at (-2, 1) {};
		\node [style=none] (4) at (-1.85, 1) {};
		\node [style=none] (5) at (-1.85, 1.35) {};
		\node [style=circle, scale=0.6] (6) at (-2.35, 0.9) {};
		\node [style=none] (7) at (-1.6, 1.17) {};
		\node [style=none] (8) at (-2.95, 1.17) {};
	\end{pgfonlayer}
	\begin{pgfonlayer}{edgelayer}
		\draw (2.center) to (3.center);
		\draw (5.center) to (4.center);
		\draw (0.center) to (1.center);
	\end{pgfonlayer}
\end{tikzpicture}}

\newcommand{\dagcomonwtik} {\begin{tikzpicture} %linear comonoid with white circle
	\begin{pgfonlayer}{nodelayer}
		\node [style=none] (0) at (-2.7, 1.17) {};
		\node [style=none] (1) at (-1.85, 1.17) {};
		\node [style=none] (2) at (-2, 1.35) {};
		\node [style=none] (3) at (-2, 1) {};
		\node [style=none] (4) at (-1.85, 1) {};
		\node [style=none] (5) at (-1.85, 1.35) {};
		\node [style=circle, scale=0.5] (6) at (-2.2, 0.9) {};
		\node [style=none, scale=1.4] (9) at (-2.5, 0.8) {$\dag$};
		\node [style=none] (7) at (-1.6, 1.17) {};
		\node [style=none] (8) at (-2.95, 1.17) {};
	\end{pgfonlayer}
	\begin{pgfonlayer}{edgelayer}
		\draw (2.center) to (3.center);
		\draw (5.center) to (4.center);
		\draw (0.center) to (1.center);
	\end{pgfonlayer}
\end{tikzpicture}}

\newcommand{\lincomonbtik} {\begin{tikzpicture} % linear comonoid with black circle
	\begin{pgfonlayer}{nodelayer}
		\node [style=none] (0) at (-2.7, 1.17) {};
		\node [style=none] (1) at (-1.85, 1.17) {};
		\node [style=none] (2) at (-2, 1.35) {};
		\node [style=none] (3) at (-2, 1) {};
		\node [style=none] (4) at (-1.85, 1) {};
		\node [style=none] (5) at (-1.85, 1.35) {};
		\node [style=circle, scale=0.6, fill=black] (6) at (-2.35, 0.9) {};
		\node [style=none] (7) at (-1.6, 1.17) {};
		\node [style=none] (8) at (-2.95, 1.17) {};
	\end{pgfonlayer}
	\begin{pgfonlayer}{edgelayer}
		\draw (2.center) to (3.center);
		\draw (5.center) to (4.center);
		\draw (0.center) to (1.center);
	\end{pgfonlayer}
\end{tikzpicture}}

\newcommand{\lincomonwtritik} {\begin{tikzpicture} %linear comonoid  with white triangle
	\begin{pgfonlayer}{nodelayer}
		\node [style=none] (0) at (-2.7, 1.17) {};
		\node [style=none] (1) at (-1.85, 1.17) {};
		\node [style=none] (2) at (-2, 1.35) {};
		\node [style=none] (3) at (-2, 1) {};
		\node [style=none] (4) at (-1.85, 1) {};
		\node [style=none] (5) at (-1.85, 1.35) {};
		\node [style=none] (6) at (-2.2, 1) {};
		\node [style=none] (7) at (-2.5, 1) {};
		\node [style=none] (8) at (-2.35, 0.82) {};
		\node [style=none] (9) at (-1.6, 1.17) {};
		\node [style=none] (10) at (-2.95, 1.17) {};
	\end{pgfonlayer}
	\begin{pgfonlayer}{edgelayer}
		\draw (2.center) to (3.center);
		\draw (5.center) to (4.center);
		\draw (0.center) to (1.center);
		\draw (6.center) -- (7.center) -- (8.center) -- (6.center);
	\end{pgfonlayer}
\end{tikzpicture}}

\newcommand{\dagcomonwtritik} {\begin{tikzpicture} %dagger linear comonoid  with white triangle
	\begin{pgfonlayer}{nodelayer}
		\node [style=none] (0) at (-2.7, 1.17) {};
		\node [style=none] (1) at (-1.85, 1.17) {};
		\node [style=none] (2) at (-2, 1.35) {};
		\node [style=none] (3) at (-2, 1) {};
		\node [style=none] (4) at (-1.85, 1) {};
		\node [style=none] (5) at (-1.85, 1.35) {};
		\node [style=none] (6) at (-2.1, 0.85) {};
		\node [style=none] (7) at (-2.4, 0.85) {};
		\node [style=none] (8) at (-2.25, 1) {};
		\node [style=none] (9) at (-1.6, 1.17) {};
		\node [style=none] (10) at (-2.95, 1.17) {};
		\node [style=none, scale=1.5] (11) at (-2.6, 0.85) {$\dag$};
	\end{pgfonlayer}
	\begin{pgfonlayer}{edgelayer}
		\draw (2.center) to (3.center);
		\draw (5.center) to (4.center);
		\draw (0.center) to (1.center);
		\draw (6.center) -- (7.center) -- (8.center) -- (6.center);
	\end{pgfonlayer}
\end{tikzpicture}}

\newcommand{\lincomonbtritik} {\begin{tikzpicture} %linear comonoid  with black triangle
	\begin{pgfonlayer}{nodelayer}
		\node [style=none] (0) at (-2.7, 1.17) {};
		\node [style=none] (1) at (-1.85, 1.17) {};
		\node [style=none] (2) at (-2, 1.35) {};
		\node [style=none] (3) at (-2, 1) {};
		\node [style=none] (4) at (-1.85, 1) {};
		\node [style=none] (5) at (-1.85, 1.35) {};
		\node [style=none] (6) at (-2.2, 1) {};
		\node [style=none] (7) at (-2.5, 1) {};
		\node [style=none] (8) at (-2.35, 0.82) {};
		\node [style=none] (9) at (-1.6, 1.17) {};
		\node [style=none] (10) at (-2.95, 1.17) {};
	\end{pgfonlayer}
	\begin{pgfonlayer}{edgelayer}
		\draw (2.center) to (3.center);
		\draw (5.center) to (4.center);
		\draw (0.center) to (1.center);
		\draw[fill=black] (6.center) -- (7.center) -- (8.center) -- (6.center);
	\end{pgfonlayer}
\end{tikzpicture}}

\newcommand{\linbialgwtik} {\begin{tikzpicture}
	\begin{pgfonlayer}{nodelayer}
		\node [style=none] (0) at (-2.8, 1.17) {};
		\node [style=none] (1) at (-1.85, 1.17) {};
		\node [style=none] (2) at (-2, 1.35) {};
		\node [style=none] (3) at (-2, 1) {};
		\node [style=none] (4) at (-1.85, 1) {};
		\node [style=none] (5) at (-1.85, 1.35) {};
		\node [style=none] (6) at (-2.2, 1) {};
		\node [style=none] (7) at (-2.5, 1) {};
		\node [style=none] (8) at (-2.35, 0.82) {};
		\node [style=none] (9) at (-1.6, 1.17) {};
		\node [style=none] (10) at (-3.05, 1.17) {};
		\node [style=circle, scale=0.6] (11) at (-2.35, 1.45) {};
		%\node [style=none] (12) at (-2.25, 0.5) {}; %extra node for spacing
	\end{pgfonlayer}
	\begin{pgfonlayer}{edgelayer}
		\draw (2.center) to (3.center);
		\draw (5.center) to (4.center);
		\draw (0.center) to (1.center);
		\draw (6.center) -- (7.center) -- (8.center) -- (6.center);
	\end{pgfonlayer}
\end{tikzpicture}}

\newcommand{\expbialgwtik} {\begin{tikzpicture}
	\begin{pgfonlayer}{nodelayer}
		\node [style=none] (0) at (-2.7, 1.17) {};
		\node [style=none] (1) at (-1.85, 1.17) {};
		\node [style=none] (2) at (-2, 1.35) {};
		\node [style=none] (3) at (-2, 1) {};
		\node [style=none] (4) at (-1.85, 1) {};
		\node [style=none] (5) at (-1.85, 1.35) {};
		\node [style=none] (6) at (-2.2, 1) {};
		\node [style=none] (7) at (-2.5, 1) {};
		\node [style=none] (8) at (-2.35, 0.82) {};
		\node [style=none] (9) at (-1.6, 1.17) {};
		\node [style=none] (10) at (-2.95, 1.17) {};
		\node [style=circle, scale=0.6 ] (12) at (-2.35, 1.55) {};
		\node [style=map, scale=1.7, fill opacity=0] (11) at (-2.35, 1.55) {};
		%\node [style=none] (12) at (-2.25, 0.5) {}; %extra node for spacing
	\end{pgfonlayer}
	\begin{pgfonlayer}{edgelayer}
		\draw (2.center) to (3.center);
		\draw (5.center) to (4.center);
		\draw (0.center) to (1.center);
		\draw (6.center) -- (7.center) -- (8.center) -- (6.center);
	\end{pgfonlayer}
\end{tikzpicture}}

\newcommand{\dagbialgwtik} {\begin{tikzpicture}
	\begin{pgfonlayer}{nodelayer}
		\node [style=none] (0) at (-2.7, 1.17) {};
		\node [style=none] (1) at (-1.85, 1.17) {};
		\node [style=none] (2) at (-2, 1.35) {};
		\node [style=none] (3) at (-2, 1) {};
		\node [style=none] (4) at (-1.85, 1) {};
		\node [style=none] (5) at (-1.85, 1.35) {};
		\node [style=none] (6) at (-2.2, 1) {};
		\node [style=none] (7) at (-2.5, 1) {};
		\node [style=none] (8) at (-2.35, 0.82) {};
		\node [style=none] (9) at (-1.6, 1.17) {};
		\node [style=none] (10) at (-2.95, 1.17) {};
		\node [style=circle, scale=0.5] (11) at (-2.25, 1.45) {};		
		\node [style=none, scale=1.5] (12) at (-2.55, 1.45) {$\dagger$};
		\node [style=none] (13) at (-2.25, 0.5) {}; %extra node for spacing
	\end{pgfonlayer}
	\begin{pgfonlayer}{edgelayer}
		\draw (2.center) to (3.center);
		\draw (5.center) to (4.center);
		\draw (0.center) to (1.center);
		\draw (6.center) -- (7.center) -- (8.center) -- (6.center);
	\end{pgfonlayer}
\end{tikzpicture}}

\newcommand{\linbialgbtik} {\begin{tikzpicture}
	\begin{pgfonlayer}{nodelayer}
		\node [style=none] (0) at (-2.7, 1.17) {};
		\node [style=none] (1) at (-1.85, 1.17) {};
		\node [style=none] (2) at (-2, 1.35) {};
		\node [style=none] (3) at (-2, 1) {};
		\node [style=none] (4) at (-1.85, 1) {};
		\node [style=none] (5) at (-1.85, 1.35) {};
		\node [style=none] (6) at (-2.2, 1) {};
		\node [style=none] (7) at (-2.5, 1) {};
		\node [style=none] (8) at (-2.35, 0.82) {};
		\node [style=none] (9) at (-1.6, 1.17) {};
		\node [style=none] (10) at (-2.95, 1.17) {};
		\node [style=circle, scale=0.6, fill=black] (11) at (-2.35, 1.45) {};
		%\node [style=none] (12) at (-2.25, 0.5) {}; %extra node for spacing
	\end{pgfonlayer}
	\begin{pgfonlayer}{edgelayer}
		\draw (2.center) to (3.center);
		\draw (5.center) to (4.center);
		\draw (0.center) to (1.center);
		\draw[fill=black] (6.center) -- (7.center) -- (8.center) -- (6.center);
	\end{pgfonlayer}
\end{tikzpicture}}


\newcommand{\expbialgbtik} {\begin{tikzpicture}
	\begin{pgfonlayer}{nodelayer}
		\node [style=none] (0) at (-2.7, 1.17) {};
		\node [style=none] (1) at (-1.85, 1.17) {};
		\node [style=none] (2) at (-2, 1.35) {};
		\node [style=none] (3) at (-2, 1) {};
		\node [style=none] (4) at (-1.85, 1) {};
		\node [style=none] (5) at (-1.85, 1.35) {};
		\node [style=none] (6) at (-2.2, 1) {};
		\node [style=none] (7) at (-2.5, 1) {};
		\node [style=none] (8) at (-2.35, 0.82) {};
		\node [style=none] (9) at (-1.6, 1.17) {};
		\node [style=none] (10) at (-2.95, 1.17) {};
		\node [style=circle, scale=0.6, fill=black ] (12) at (-2.35, 1.55) {};
		\node [style=map, scale=1.7, fill opacity=0] (11) at (-2.35, 1.55) {};
		%\node [style=none] (12) at (-2.25, 0.5) {}; %extra node for spacing
	\end{pgfonlayer}
	\begin{pgfonlayer}{edgelayer}
		\draw (2.center) to (3.center);
		\draw (5.center) to (4.center);
		\draw (0.center) to (1.center);
		\draw (6.center) -- (7.center) -- (8.center) -- (6.center);
	\end{pgfonlayer}
\end{tikzpicture}}

\newcommand{\tricomul}[1]{\trianglecomult{#1}}
\newcommand{\trimul}[1]{\trianglemul{#1}}
\newcommand{\tricounit}[1]{\trianglecounit{#1}}
\newcommand{\triunit}[1]{\triangleunit{#1}}

\newcommand{\tricomulb}[1]{\trianglecomultblack{#1}}
\newcommand{\trimulb}[1]{\trianglemultblack{#1}}
\newcommand{\tricounitb}[1]{\trianglecounitblack{#1}}
\newcommand{\triunitb}[1]{\triangleunitblack{#1}}

\newcommand{\tinycomulmap}{
{\tiny \comulmap{0.5}{white}}
}
%%%% writing above and below %%%%%%%%
%\usepackage{stackengine}
%\stackMath
%\newcommand{\daglin}[2]{\stackrel{#1}{\underset{#2}{\dashvv}}}
\newcommand{\xyalign}{\raisebox{-4.5\height}}
\newcommand{\dagmon}[1]{\stackrel{#1}{\dashvv}}
% argument 1: (p,q) linear monoid hom
\newcommand{\dagdual}{\xymatrixcolsep{4mm} \xymatrix{ \ar@{-||}[r]^{\dagger} & }}

%white linear monoid and black linear monoid
\newcommand{\whitelin}{\xymatrixcolsep{3mm} \xymatrix{  \ar@{-||}[r]^{\circ} &  }}
\newcommand{\blacklin}{\xymatrixcolsep{3mm} \xymatrix{ \ar@{-||}[r]^{\bullet} & }}

%white linear monoid and black linear monoid
% argument 1: (p,q) linear monoid hom
\newcommand{\whitedag}[1]\dagmonw %{\xymatrixcolsep{6mm} \xymatrix{ \ar@{-||}[r]^{\circ}_{#1} & }}
\newcommand{\blackdag}[1]\dagmonb %{\xymatrixcolsep{6mm} \xymatrix{  \ar@{-||}[r]^{\bullet}_{#1} & }}

\newcommand{\linmonw} {\xymatrixcolsep{4mm} \xymatrix{ \ar@{-||}[r]^{\circ} & }}
\newcommand{\linmonb} {\xymatrixcolsep{4mm} \xymatrix{  \ar@{-||}[r]^{\bullet} & }}
\newcommand{\dagmonw} {\xymatrixcolsep{4mm} \xymatrix{ \ar@{-||}[r]_{}^{\dagger\circ} & }}
\newcommand{\dagmonb} {\xymatrixcolsep{4mm} \xymatrix{  \ar@{-||}[r]_{}^{\dagger\bullet} & }}

\newcommand{\lincomonw} {\xymatrixcolsep{4mm} \xymatrix{ \ar@{-||}[r]_{\circ} & }}
\newcommand{\lincomonb} {\xymatrixcolsep{4mm} \xymatrix{  \ar@{-||}[r]_{\bullet} & }}
\newcommand{\dagcomonw} {\xymatrixcolsep{4mm} \xymatrix{ \ar@{-||}[r]_{\dagger\circ} & }}
\newcommand{\dagcomonb} {\xymatrixcolsep{4mm} \xymatrix{  \ar@{-||}[r]_{\dagger\bullet} & }}

\newcommand{\monoid}[1]{(#1, \mulmap{1.5}{white}, \unitmap{1.5}{white})}
\newcommand{\comonoid}[1]{(#1, \comulmap{1.5}{white}, \counitmap{1.5}{white})}
\newcommand{\comonoidb}[1]{(#1, \comulmap{1.5}{black}, \counitmap{1.5}{black})}
\newcommand{\Frob}[1]{(#1, \mulmap{1.5}{white}, \unitmap{1.5}{white}, \comulmap{1.5}{white}, \counitmap{1.5}{white})}
\newcommand{\bFrob}[1]{(#1, \mulmap{1.5}{black}, \unitmap{1.5}{black}, \comulmap{1.5}{black}, \counitmap{1.5}{black})}
\newcommand{\bialg}[1]{(#1, \mulmap{1.5}{white}, \unitmap{1.5}{white}, \comulmap{1.5}{black}, \counitmap{1.5}{black})}
\newcommand{\bialgb}[1]{(#1, \mulmap{1.5}{black}, \unitmap{1.5}{black}, \comulmap{1.5}{white}, \counitmap{1.5}{white})}


\newcommand{\binidem}[1]{ \stackrel{(#1)}{\xrightleftharpoons{\hspace{0.4cm}}}} %binary idempotent
\newcommand{\dagidem}[1]{ \stackrel{\dagger(#1)}{\xrightleftharpoons{\hspace{0.6cm}}}} % dagger binary idempotent


\newcommand{\tr}{\triangleright}
\newcommand{\tl}{\triangleleft}


\xymatrixrowsep{5mm}
\newdir{(>}{{}*!/-5pt/\dir{>}}  %for non-colliding monic arrows

\newcommand{\Lim}[1]{\underset{#1}{\sf Lim~}}
\newcommand{\Colim}[1]{\underset{#1}{\sf Colim~}}

\newcommand{\tri}[1]{
\begin{tikzpicture}[scale=#1]
	\begin{pgfonlayer}{nodelayer}
		\node [style=none] (0) at (2, 3.5) {};
		\node [style=none] (1) at (1.75, 3.25) {};
		\node [style=none] (2) at (2.25, 3.25) {};
	\end{pgfonlayer}
	\begin{pgfonlayer}{edgelayer}
		\draw (0.center) to (1.center);
		\draw (0.center) to (2.center);
		\draw (2.center) to (1.center);
	\end{pgfonlayer}
\end{tikzpicture} } 

\newcommand{\btri}[1]{
\begin{tikzpicture}[scale=#1]
	\begin{pgfonlayer}{nodelayer}
		\node [style=none] (0) at (2, 3.5) {};
		\node [style=none] (1) at (1.75, 3.25) {};
		\node [style=none] (2) at (2.25, 3.25) {};
	\end{pgfonlayer}
	\begin{pgfonlayer}{edgelayer}
		\draw [fill=black] (0.center) -- (1.center) -- (2.center) -- (0.center);
	\end{pgfonlayer}
\end{tikzpicture} } 

\newcommand{\linbialg}{\xymatrixcolsep{5mm} \xymatrix{ \ar@{-||}[r]^{\tri{0.75}} & }}
\newcommand{\linbialgblack}{\xymatrixcolsep{5mm} \xymatrix{ \ar@{-||}[r]^{\btri{0.75}} & }}
\newcommand{\dagbialg}{\xymatrixcolsep{6mm} \xymatrix{ \ar@{-||}[r]^{\dagger \tri{0.75}} & }}
\newcommand{\dagbialgb}{\xymatrixcolsep{6mm} \xymatrix{ \ar@{-||}[r]^{\dagger \btri{0.75}} & }}
\newcommand{\linbialgb}{\linbialgblack} 
%%%%%%%%%%%%%%%%%%%%%%%%%%%%%%%%%%%%%%%%%%%%%%%%%%%%%%%%%%%%%%%%%%%%%%%%

% Robin's Macros
% Jeff Egger's tensor and par
\newlength{\llcfoo}
\def\superimpose#1#2{
  \settowidth{\llcfoo}{#2}
  \makebox[\llcfoo]{\makebox[0pt]{#1}\makebox[0pt]{#2}}}
%% \superimpose assumes that the first argument is narrower (or
%% equal-in-width) to the second.
\def\mathsuperimpose#1#2{\mathchoice{
  \superimpose{\ensuremath{\displaystyle#1}}{\ensuremath{\displaystyle#2}}}{
  \superimpose{\ensuremath{\textstyle#1}}{\ensuremath{\textstyle#2}}}{
  \superimpose{\ensuremath{\scriptstyle#1}}{\ensuremath{\scriptstyle#2}}}{
  \superimpose{\ensuremath{\scriptscriptstyle#1}}{\ensuremath{\scriptscriptstyle#2}}}}

      \def\quasipt{1pt}     	% if 12pt font
%     \def\quasipt{0.75pt}  	% if 11pt font
%     \def\quasipt{0.67pt}  	% if 10pt font
      \def\minipt{0.6pt}	% seems to work reasonably well in all fonts
      \def\tinypt{0.4pt}	% seems to work reasonably well in all fonts
    \def\smalltimes{\raisebox{\quasipt}{$\scriptstyle\times$}}
    \def\tinytimes{\raisebox{\minipt}{$\scriptscriptstyle\times$}}
    \def\teenytimes{\cdot} % or {\scriptscriptstyle\ast}} or similar
    \def\fixnormalcup{\raisebox{-\quasipt}{$\cup$}}
    \def\fixsmallcup{\raisebox{-\minipt}{$\scriptstyle\cup$}}
    \def\fixtinycup{\raisebox{-\tinypt}{$\scriptscriptstyle\cup$}}
  \def\fixtimes{
    \mathchoice{\smalltimes}{\smalltimes}{\tinytimes}{\teenytimes}}
  \def\fixcup{
    \mathchoice{\fixnormalcup}{\fixnormalcup}{\fixsmallcup}{\fixtinycup}}
\def\ocap{\mathrel{\mathsuperimpose{\fixtimes}{\cap}}}
\def\ocup{\mathrel{\mathsuperimpose{\fixtimes}{\fixcup}}}
\def\bigocap{\mathop{\mathsuperimpose{\times}{\bigcap}}\limits}
\def\bigocup{\mathop{\mathsuperimpose{\times}{\bigcup}}\limits}
\def\bip{\mathop{\mathsuperimpose{\times}{+}}\limits}

%%%%%%%%%%%%%%%%%%%%%%%%%%%%%%%%%%%%%%%%%%%%%%%%%%%%%%%%%%%%%%
% sequent proof trees on same line
\newenvironment{bprooftree}
  {\leavevmode\hbox\bgroup}
  {\DisplayProof\egroup}
  