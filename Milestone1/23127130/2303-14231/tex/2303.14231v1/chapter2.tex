% !TEX root = thesis.tex

\chapter{Categorical Quantum Mechanics}
\label{Chap: CQM}

The program of Categorical Quantum Mechanics (CQM) was started by Coecke and Abramsky \cite{AC04} 
in 2004 with the aim of developing a high-level formal language for quantum mechanics while moving 
away from the standard formalism based on Hilbert spaces. CQM derived ideas from logic and 
computer science, and used the diagrammatic language of compact closed categories for an intuitive 
but mathematically rigorous presentation of the fundamental axioms of quantum theory. 
%Since its inception, CQM research has been ever widening and has been applied to 
%various areas of quantum research: to name a few applications, in quantum foundations to study quantum-like theories \cite{CoE11, Bac15b, HeT15, Gog17, GoS18}, 
%in quantum computation for circuit description and optimization \cite{CoD11, DKP20, KiW20, Bea20, CDS20}, and in quantum information (QI) theory 
%to construct structures \cite{MuV16,ReV19} crucial to many QI protocols, and even in linguistics \cite{Coe16, KSS12,CGS18,CoK20}. 
%The list of applications in each of these areas is, of course, not exhaustive. 

The purpose this chapter is to review the fundamentals of Categorical Quantum Mechanics 
and to describe its essential features used to study quantum mechanics. 

\section{Dagger monoidal categories}
\label{Sec: dag mon cat}

Categorical quantum mechanics (CQM) models physical systems as objects within a monoidal category and 
the physical processes as the maps in the category. The identity maps equate to the do-nothing processes.
 Sequential composition of processes is given by the composition of maps while parallel composition 
 is modelled using the tensor product.  Categorifying the notion of inner product  in traditional quantum 
 theory produces  a $\dagger$-functor for monoidal categories giving rise to the theory of 
 $\dagger$-monoidal categories. 

 \subsection{Graphical calculus for monoidal categories}
 Monoidal categories come equipped with a graphical calculus \cite{Sel10},
 which allows for diagrammatic representation and manipulation of systems and processes. The 
 availability of this graphical calculus is perhaps the most attractive aspect of using monoidal 
 categories to study quantum mechanics.

 In the graphical calculus of monoidal categories, the objects are represented by wires, and arrows by circles
\footnote{In the CQM community, maps are often drawn as boxes. A circle and a box are topologically same.}. 
 An identity  arrow is given by a wire without by circle.
 \[ \begin{tikzpicture}
	\begin{pgfonlayer}{nodelayer}
		\node [style=none] (0) at (-2, 3) {};
		\node [style=none] (1) at (-2, -0.25) {};
		\node [style=none] (2) at (-2.5, 1.5) {$A$};
		\node [style=none, scale=2] (3) at (-2, -1) {An object $A$};
	\end{pgfonlayer}
	\begin{pgfonlayer}{edgelayer}
		\draw (0.center) to (1.center);
	\end{pgfonlayer}
\end{tikzpicture} ~~~~~~~~
\begin{tikzpicture}
	\begin{pgfonlayer}{nodelayer}
		\node [style=none] (0) at (-2, 3) {};
		\node [style=none] (1) at (-2, -0.25) {};
		\node [style=none] (2) at (-2.5, 2.5) {$A$};
		\node [style=none, scale=2] (3) at (-2, -1) {$f: A \to B$};
		\node [style=none] (4) at (-2.5, 0.5) {$B$};
		\node [style=circle, scale=1.5] (5) at (-2, 1.5) {};
		\node [style=none] (6) at (-2, 1.5) {$f$};
	\end{pgfonlayer}
	\begin{pgfonlayer}{edgelayer}
		\draw (0.center) to (5);
		\draw (1.center) to (5);
	\end{pgfonlayer}
\end{tikzpicture} ~~~~~~~~
\begin{tikzpicture}
	\begin{pgfonlayer}{nodelayer}
		\node [style=none] (0) at (-2, 3) {};
		\node [style=none] (1) at (-2, -0.25) {};
		\node [style=none] (2) at (-2.25, 1.5) {$A$};
		\node [style=none, scale=2] (3) at (-2, -1) {Identity arrow $1_A$};
	\end{pgfonlayer}
	\begin{pgfonlayer}{edgelayer}
		\draw (1.center) to (0.center);
	\end{pgfonlayer}
\end{tikzpicture}\]

Note that the diagram for the object $A$ is same as the identity arrow of $A$ which is to be interpreted as 
follows: a physical system which does nothing is same as the system itself. 

Composition of maps is given by connecting the wires sequentially. 
Tensor product of object is given by parallel juxtaposition of wires. The unit object is given by an 
empty circle representing the empty system. Recall that in LDCs, when the tensor and the par units are different, we used 
labelled wires to represent units. 
  \[ \begin{tikzpicture}
	 \begin{pgfonlayer}{nodelayer}
		 \node [style=none] (0) at (-2, 3) {};
		 \node [style=none] (1) at (-2, -0.25) {};
		 \node [style=none] (2) at (-2.5, 2.75) {$A$};
		 \node [style=none, scale=2] (3) at (-2, -1) {Composing maps};
		 \node [style=none] (4) at (-2.5, 1.5) {$B$};
		 \node [style=circle, scale=1.5] (5) at (-2, 2) {};
		 \node [style=none] (6) at (-2, 2) {$f$};
		 \node [style=circle, scale=1.5] (7) at (-2, 0.75) {};
		 \node [style=none] (8) at (-2, 0.75) {$g$};
		 \node [style=none] (9) at (-2.5, -0) {$C$};
	 \end{pgfonlayer}
	 \begin{pgfonlayer}{edgelayer}
		 \draw (0.center) to (5);
		 \draw (1.center) to (7);
		 \draw (7) to (5);
	 \end{pgfonlayer}
 \end{tikzpicture} ~~~~~~~~
  \begin{tikzpicture}
	 \begin{pgfonlayer}{nodelayer}
		 \node [style=none] (0) at (-2.25, 3) {};
		 \node [style=none] (1) at (-2.25, -0.25) {};
		 \node [style=none] (2) at (-2.5, 2.5) {$A$};
		 \node [style=none, scale=2] (3) at (-1.75, -1) {$f : A \ox C \to B \ox D$};
		 \node [style=none] (4) at (-2.5, 0.25) {$B$};
		 \node [style=none] (5) at (-1, 2.5) {$C$};
		 \node [style=none] (6) at (-1.25, 3) {};
		 \node [style=none] (7) at (-1, 0.25) {$D$};
		 \node [style=none] (8) at (-1.25, -0.25) {};
		 \node [style=circle, scale=1.5] (9) at (-1.75, 1.5) {};
		 \node [style=none] (10) at (-1.75, 1.5) {$f$};
	 \end{pgfonlayer}
	 \begin{pgfonlayer}{edgelayer}
		 \draw [in=-90, out=120, looseness=1.00] (9) to (0.center);
		 \draw [in=60, out=-90, looseness=1.00] (6.center) to (9);
		 \draw [in=-120, out=90, looseness=1.00] (1.center) to (9);
		 \draw [in=-60, out=90, looseness=1.00] (8.center) to (9);
	 \end{pgfonlayer}
 \end{tikzpicture}  ~~~~~~~~
 \begin{tikzpicture}
	 \begin{pgfonlayer}{nodelayer}
		 \node [style=none] (0) at (-2, 3) {};
		 \node [style=none] (1) at (-2, -0.25) {};
		 \node [style=none] (2) at (-2.5, 2.5) {$A$};
		 \node [style=none, scale=2] (3) at (-1.75, -1) {$f \ox g : A \ox C \to B \ox D$};
		 \node [style=circle, scale=1.5] (4) at (-2, 1.25) {};
		 \node [style=none] (5) at (-2.5, 0.25) {$B$};
		 \node [style=none] (6) at (-2, 1.25) {$f$};
		 \node [style=none] (7) at (-0.8, 2.5) {$C$};
		 \node [style=circle, scale=1.5] (8) at (-1.25, 1.25) {};
		 \node [style=none] (9) at (-1.25, 3) {};
		 \node [style=none] (10) at (-0.8, 0.25) {$D$};
		 \node [style=none] (11) at (-1.25, 1.25) {$g$};
		 \node [style=none] (12) at (-1.25, -0.25) {};
	 \end{pgfonlayer}
	 \begin{pgfonlayer}{edgelayer}
		 \draw (1.center) to (4);
		 \draw (4) to (0.center);
		 \draw (12.center) to (8);
		 \draw (8) to (9.center);
	 \end{pgfonlayer}
 \end{tikzpicture} ~~~~~~~~ I := 
 \begin{tikzpicture}
	 \begin{pgfonlayer}{nodelayer}
		 \node [style=none] (0) at (0, 2) {};
	 \end{pgfonlayer}
 \end{tikzpicture} \]

  
 In CQM, maps from the tensor unit to any other object are referred to as {\bf states}, maps 
 into the tensor unit from any other object are referred to as {\bf effects}, and maps that 
 start and end in the tensor unit are referred to as {\bf scalars}. States and effects are represented 
 using triangles: 
 \[ \begin{tikzpicture}
	\begin{pgfonlayer}{nodelayer}
		\node [style=none] (0) at (0, 2.75) {$\phi$};
		\node [style=none] (1) at (0, 3.25) {};
		\node [style=none] (2) at (-0.5, 2.5) {};
		\node [style=none] (3) at (0.5, 2.5) {};
		\node [style=none] (4) at (0, 2.5) {};
		\node [style=none] (5) at (0, 1) {};
		\node [style=none, scale=2] (7) at (0, 0.5) {State};
	\end{pgfonlayer}
	\begin{pgfonlayer}{edgelayer}
		\draw (1.center) to (2.center);
		\draw (2.center) to (3.center);
		\draw (3.center) to (1.center);
		\draw (4.center) to (5.center);
	\end{pgfonlayer}
\end{tikzpicture} ~~~~~~~~~~~~ \begin{tikzpicture}
	\begin{pgfonlayer}{nodelayer}
		\node [style=none] (0) at (0, 1.5) {$\phi$};
		\node [style=none] (1) at (0, 1) {};
		\node [style=none] (2) at (-0.5, 1.75) {};
		\node [style=none] (3) at (0.5, 1.75) {};
		\node [style=none] (4) at (0, 1.75) {};
		\node [style=none] (5) at (0, 3.25) {};
		\node [style=none, scale=2] (7) at (0, 0.5) {Effect};
	\end{pgfonlayer}
	\begin{pgfonlayer}{edgelayer}
		\draw (1.center) to (2.center);
		\draw (2.center) to (3.center);
		\draw (3.center) to (1.center);
		\draw (4.center) to (5.center);
	\end{pgfonlayer}
\end{tikzpicture} \]

 The assosciativity, left unitor, right unitor and the symmetry natural isomorphisms are given
  as follows:
 \[ \begin{tikzpicture}
	\begin{pgfonlayer}{nodelayer}
		\node [style=none] (0) at (-2.5, 3) {};
		\node [style=none] (1) at (-2.5, -0.25) {};
		\node [style=none] (2) at (-2.5, 3.25) {$A$};
		\node [style=none, scale=2] (3) at (-1.75, -1) {$a_\ox: (A \ox B) \ox C \to A \ox (B \ox C)$};
		\node [style=none] (4) at (-1.75, 3.25) {$B$};
		\node [style=none] (5) at (-1, 3) {};
		\node [style=none] (6) at (-1, -0.25) {};
		\node [style=none] (7) at (-1.75, 3) {};
		\node [style=none] (8) at (-1.75, -0.25) {};
		\node [style=none] (9) at (-1, 3.25) {$C$};
	\end{pgfonlayer}
	\begin{pgfonlayer}{edgelayer}
		\draw (0.center) to (1.center);
		\draw (7.center) to (8.center);
		\draw (5.center) to (6.center);
	\end{pgfonlayer}
\end{tikzpicture}
~~~~~~~~
\begin{tikzpicture}
	\begin{pgfonlayer}{nodelayer}
		\node [style=none] (0) at (-2.25, 3) {};
		\node [style=none] (1) at (-2.25, -0.25) {};
		\node [style=none] (2) at (-2.25, 3.25) {$A$};
		\node [style=none, scale=2] (3) at (-2.25, -1) {$u_\ox^r: A \ox I \to A$};
	\end{pgfonlayer}
	\begin{pgfonlayer}{edgelayer}
		\draw (0.center) to (1.center);
	\end{pgfonlayer}
\end{tikzpicture} ~~~~~~~~~~~~
\begin{tikzpicture}
	\begin{pgfonlayer}{nodelayer}
		\node [style=none] (0) at (-2.25, 3) {};
		\node [style=none] (1) at (-2.25, -0.25) {};
		\node [style=none] (2) at (-2.25, 3.25) {$A$};
		\node [style=none, scale=2] (3) at (-2.25, -1) {$u_\ox^l: I \ox A \to  A$};
	\end{pgfonlayer}
	\begin{pgfonlayer}{edgelayer}
		\draw (0.center) to (1.center);
	\end{pgfonlayer}
\end{tikzpicture} 
 \]
 
Note that the associativity isomorphisms, the left and the right unitors are identity maps. In fact, this 
is the graphical caluclus for a strict monoidal category. Since every monoidal category is equivalent to a strict 
monoidal category \cite{Mac13} allows one to use this calculus on any monoidal category. 

The symmetry map is represented using crossed wires as shown in the left. The inverse law for a 
symmetric monoidal category is given by the diagrammatic equation in the right: 
\[  \begin{tikzpicture}
	\begin{pgfonlayer}{nodelayer}
		\node [style=none] (0) at (-1.5, 3.25) {};
		\node [style=none] (1) at (0, -0.5) {};
		\node [style=none] (2) at (-1.75, 2.5) {$A$};
		\node [style=none, scale=2] (3) at (-0.75, -1.25) {$c_\ox: A \ox B \to B \ox A$};
		\node [style=none] (4) at (0, 3.25) {};
		\node [style=none] (5) at (-1.5, -0.5) {};
		\node [style=none] (6) at (0.25, 2.5) {$B$};
		\node [style=none] (7) at (-1.75, 0) {$B$};
		\node [style=none] (8) at (0.25, 0) {$A$};
	\end{pgfonlayer}
	\begin{pgfonlayer}{edgelayer}
		\draw [in=90, out=-90, looseness=1.25] (0.center) to (1.center);
		\draw [in=-90, out=90, looseness=1.25] (5.center) to (4.center);
	\end{pgfonlayer}
\end{tikzpicture}
~~~~~~~~
\begin{tikzpicture}
	\begin{pgfonlayer}{nodelayer}
		\node [style=none] (0) at (-1.75, 3) {};
		\node [style=none] (1) at (-0.75, 3) {};
		\node [style=none] (2) at (-1.75, -0.5) {};
		\node [style=none] (3) at (-0.75, -0.5) {};
		\node [style=none] (4) at (0.75, 3) {};
		\node [style=none] (5) at (1.75, 3) {};
		\node [style=none] (6) at (1.75, -0.5) {};
		\node [style=none] (7) at (0.75, -0.5) {};
		\node [style=none] (8) at (0, 1.25) {$=$};
		\node [style=none, scale=2] (9) at (0, -1.25) {$(c_\ox)_{B,A} = (c_\ox)_{A,B}^{-1}$};
		\node [style=none] (10) at (-1.75, 1.25) {};
		\node [style=none] (11) at (-0.75, 1.25) {};
	\end{pgfonlayer}
	\begin{pgfonlayer}{edgelayer}
		\draw [in=90, out=-90, looseness=1.25] (1.center) to (10.center);
		\draw [in=90, out=-90, looseness=1.25] (0.center) to (11.center);
		\draw [in=90, out=-90, looseness=1.25] (11.center) to (2.center);
		\draw [in=90, out=-90, looseness=1.25] (10.center) to (3.center);
		\draw (4.center) to (7.center);
		\draw (6.center) to (5.center);
	\end{pgfonlayer}
\end{tikzpicture} \]

An equation holds in a SMC if and only if it holds in the graphical calculus. 
Proving equations using graphical calculus is simpler than proving equations by 
reasoning with mathematical symbols because human brain excels at processing 
visual information. 

Two diagrams are the same in the graphical calculus if one can be transformed to another 
up to planar isotopy or by using an axiom. For details on graphical calculus for monoidal 
categories see \cite{Sel10}. 
 
\subsection{Dagger monoidal categories}

Complex Hilbert spaces are used as the de facto framework for describing quantum 
processes. The inner product structure on these spaces allows the notion of 
adjoint which is in turn needed to define quantum observables: 
self-adjoint operators on the space.  
In quantum computing, every quantum logic gate is represented by a unitary (in other words a self-adjoint) matrix.
CQM abstracts the notion of inner products as dagger \cite{Sel07, CoK17} functors for $\dagger$-monoidal categories. 

\begin{definition}
A {\bf dagger} category is a category, $\X$, equipped with a involutive 
$(f^{\dag \dag} = f)$ contravariant functor 
$\dagger: \X^\op \to \X$ which is the identity on objects $(A = A^\dag)$. 
\end{definition}

Note that, a given category can have more than one $\dagger$-functor, for example for 
the category of complex matrices (see Section \ref{Sec: Mat(R)}) conjugate transpose and transpose 
operations on matrices gives two different dagger functors. Hence, $\dagger$ is a 
structure rather than a property for a category. Because $\dagger$ is a contravariant functor, $(fg)^\dag = g^\dag f^\dag$.

Given any map $f: A \to B$, the map $f^\dagger: B \to A$ is referred to as its {\bf adjoint}. 
A {\bf unitary} isomorphism in a $\dagger$-category is an isomorphism $f$ such that $f^\dag = f^{-1}$. 
A map $f: A \to B$ is an isometry if $f f^\dag = 1_A$. 

\begin{definition}
A {\bf $\dagger$-symmetric monoidal} category is a symmetric monoidal category 
which is also a $\dagger$-category such that the
$\dagger$ behaves coherently with the monoidal structure:
\begin{enumerate}
	\item for all maps $f$, $g$, $(f \ox g)^\dag = f^\dag \ox g^\dag$
	\item the assosciator, the unitors and the symmetry map are unitary natural isomorphisms
\end{enumerate}
\end{definition}

Next, we discuss a key correspondence often used in 
 quantum information theory called the operator-state duality, also referred to as the Choi-Jamiolkowski isomorphism: 
every linear map between (finite dimensional) Hilbert spaces $H$ and $K$ corresponds precisely to a state 
in the tensor product space $H \ox K$. This correspondence is abstracted as follows using the compact structure \cite{Kel97}
for monoidal categories. 

\begin{definition}
	\label{defn: right dual}
In a monoidal category, an object $B$ is {\bf right dual} to an object $A$  if there exists maps:
\[ \eta: I \to A \ox B ~~~~~~~~~~	\epsilon: B \ox A \to I \]
such that:
\[ (1 \ox \eta) (\epsilon \ox 1) = 1_B ~~~~~~~~~~~~
(\eta \ox 1)(1 \ox \epsilon) = 1_A \]
\end{definition}

The maps $\eta$ and $\epsilon$ are represented in graphical calculus by a cap and a cup respectively:
\[ \eta: I \to A \ox B = \begin{tikzpicture}
		\begin{pgfonlayer}{nodelayer}
			\node [style=none] (0) at (0, 2) {};
			\node [style=none] (1) at (0, 1) {};
			\node [style=none] (2) at (1, 1) {};
			\node [style=none] (3) at (1, 2) {};
		\end{pgfonlayer}
		\begin{pgfonlayer}{edgelayer}
			\draw (1.center) to (0.center);
			\draw [bend left=90, looseness=2.25] (0.center) to (3.center);
			\draw (3.center) to (2.center);
		\end{pgfonlayer}
	\end{tikzpicture} ~~~~~~~~~~~ 
	\epsilon: B \ox A \to I = \begin{tikzpicture}
		\begin{pgfonlayer}{nodelayer}
			\node [style=none] (0) at (0, 1.75) {};
			\node [style=none] (1) at (0, 2.75) {};
			\node [style=none] (2) at (1, 2.75) {};
			\node [style=none] (3) at (1, 1.75) {};
		\end{pgfonlayer}
		\begin{pgfonlayer}{edgelayer}
			\draw (1.center) to (0.center);
			\draw [bend right=90, looseness=2.25] (0.center) to (3.center);
			\draw (3.center) to (2.center);
		\end{pgfonlayer}
	\end{tikzpicture}	
\]
The equations satisfied by a right dual 
are also referred to as the snake equations owing to their shape 
in the graphical calculus:
	\[ 	\begin{tikzpicture}
		\begin{pgfonlayer}{nodelayer}
			\node [style=none] (0) at (0, 1.75) {};
			\node [style=none] (1) at (0, 3.5) {};
			\node [style=none] (2) at (1, 2.75) {};
			\node [style=none] (3) at (1, 1.75) {};
			\node [style=none] (4) at (2, 2.75) {};
			\node [style=none] (5) at (2, 0.75) {};
			\node [style=none] (6) at (-0.25, 3.25) {$B$};
			\node [style=none] (7) at (0.65, 2.25) {$A$};
			\node [style=none] (8) at (2.25, 1) {$B$};
		\end{pgfonlayer}
		\begin{pgfonlayer}{edgelayer}
			\draw (1.center) to (0.center);
			\draw [bend right=90, looseness=2.25] (0.center) to (3.center);
			\draw (3.center) to (2.center);
			\draw (5.center) to (4.center);
			\draw [bend right=90, looseness=1.75] (4.center) to (2.center);
		\end{pgfonlayer}
	\end{tikzpicture} = \begin{tikzpicture}
		\begin{pgfonlayer}{nodelayer}
			\node [style=none] (0) at (3.25, 2.25) {$B$};
			\node [style=none] (1) at (3, 3.5) {};
			\node [style=none] (2) at (3, 0.75) {};
		\end{pgfonlayer}
		\begin{pgfonlayer}{edgelayer}
			\draw (2.center) to (1.center);
		\end{pgfonlayer}
	\end{tikzpicture}
	~~~~~~~~~~~~
	\begin{tikzpicture}
		\begin{pgfonlayer}{nodelayer}
			\node [style=none] (0) at (2, 1.75) {};
			\node [style=none] (1) at (2, 3.5) {};
			\node [style=none] (2) at (1, 2.75) {};
			\node [style=none] (3) at (1, 1.75) {};
			\node [style=none] (4) at (0, 2.75) {};
			\node [style=none] (5) at (0, 0.75) {};
			\node [style=none] (6) at (2.35, 3.25) {$A$};
			\node [style=none] (7) at (1.25, 2.25) {$B$};
			\node [style=none] (8) at (-0.35, 1) {$A$};
		\end{pgfonlayer}
		\begin{pgfonlayer}{edgelayer}
			\draw (1.center) to (0.center);
			\draw [bend left=90, looseness=2.25] (0.center) to (3.center);
			\draw (3.center) to (2.center);
			\draw (5.center) to (4.center);
			\draw [bend left=90, looseness=1.75] (4.center) to (2.center);
		\end{pgfonlayer}
	\end{tikzpicture} = \begin{tikzpicture}
		\begin{pgfonlayer}{nodelayer}
			\node [style=none] (0) at (3.15, 2.25) {$A$};
			\node [style=none] (1) at (3, 3.5) {};
			\node [style=none] (2) at (3, 0.75) {};
		\end{pgfonlayer}
		\begin{pgfonlayer}{edgelayer}
			\draw (2.center) to (1.center);
		\end{pgfonlayer}
	\end{tikzpicture} \]

Similarly, an object $B$ is {\bf left dual} to an object $A$ if there exists maps 
$\eta: I \to B \ox A$, and $\epsilon: A \ox B \to I$ such that the corresponding snake equations hold.
	
\begin{definition}
	A {\bf compact closed category} (KCC) is a monoidal category in which each 
	object $A$ is equipped with chosen right and left duals.
\end{definition}
For a KCC which is also a SMC, a right dual of an object is also its left dual, and vice versa.  
Such a dual of an object $A$ is written as $A^*$. 
	
\begin{definition} \cite{AC04}
A {\bf $\dagger$-compact closed} category ($\dagger$-KCC) is a $\dagger$-symmetric monoidal category which is also a 
compact closed category such that for each object $\eta^\dag = c_\ox \epsilon$ 
(equivalently $\epsilon^\dag = \eta c_\ox$).
\end{definition}

Dagger compact closed categories were introduced by Coecke and Abramsky \cite{AC04} 
as an axiomatic framework for quantum information theory under the title strongly compact closed categories. 
The operator-state correspondence is straightforward in a $\dagger$-compact closed category. 
Every map $f: A \to B$ precisely corresponds to a state $\ceil{f}: I \to B \ox A^*$  
and an effect $\floor{f}: B^* \ox A \to I$ defined as follows:
\[  \ceil{f} := \begin{tikzpicture}
	\begin{pgfonlayer}{nodelayer}
		\node [style=circle, scale=1.5] (0) at (0, 2) {};
		\node [style=none] (10) at (0, 2) {$f$};
		\node [style=none] (1) at (0, 2.75) {};
		\node [style=none] (2) at (0, 1) {};
		\node [style=none] (3) at (0.75, 2.75) {};
		\node [style=none] (4) at (0.75, 1) {};
		\node [style=none] (5) at (-0.5, 2.5) {$A$};
		\node [style=none] (6) at (-0.5, 1.25) {$B$};
		\node [style=none] (7) at (1.25, 1.25) {$A^*$};
	\end{pgfonlayer}
	\begin{pgfonlayer}{edgelayer}
		\draw (0) to (2.center);
		\draw (0) to (1.center);
		\draw (3.center) to (4.center);
		\draw [bend left=90, looseness=1.75] (1.center) to (3.center);
	\end{pgfonlayer}
\end{tikzpicture} 
~~~~~~~~~~~~
\floor{f} :=  \begin{tikzpicture}
	\begin{pgfonlayer}{nodelayer}
		\node [style=circle] (0) at (0.75, 1.75) {$f$};
		\node [style=none] (1) at (0.75, 1) {};
		\node [style=none] (2) at (0.75, 2.75) {};
		\node [style=none] (3) at (0, 1) {};
		\node [style=none] (4) at (0, 2.75) {};
		\node [style=none] (5) at (1.25, 1.25) {$B$};
		\node [style=none] (6) at (1.25, 2.5) {$A$};
		\node [style=none] (7) at (-0.5, 2.5) {$B^*$};
	\end{pgfonlayer}
	\begin{pgfonlayer}{edgelayer}
		\draw (0) to (2.center);
		\draw (0) to (1.center);
		\draw (3.center) to (4.center);
		\draw [bend left=90, looseness=1.75] (1.center) to (3.center);
	\end{pgfonlayer}
\end{tikzpicture} \]

In CQM, $\dagger$-KCCs are the fundamental framework for quantum information theory and 
quantum computing.  We will discuss the presentation of quantum channels, quantum observables, 
and strong complementarity within these categories in the later sections. 



\section{Examples}
In this section, we recall a few standard examples of $\dagger$-KCCs which are used in CQM. 

\subsection{Categories of sets and relations, $\Rel$}
\label{Sec: relations}
In CQM, the category of sets and relations is used as a non-standard model of 
quantum  information theory. This category is often studied in comparison with the category 
of Hilbert spaces and bounded linear maps to distinguish quantum versus non-quantum 
features \cite{HeV19, HeT15}. 
The category $\Rel$ is 
defined as follows:
\begin{description}
	\item[Objects:] Sets
	\item[Arrows:] $R:X \rightarrow Y$ where $R \subseteq X \times Y$ 
	\item[Identity maps:] $I_X:= \{ (x,x) | ~ \ x \in X \}$
	\item[Composition:] Suppose $A \xrightarrow{R} B \xrightarrow{S} C$, 
	then \[ RS: A \rightarrow C := \{ (x,z) | ~ \exists y \in Y, (x,y) \in R \text{ and } (y,z) \in S \} \]
	\item[Tensor product:] $A \otimes B := A \times B$  is the cartesian product on both sets 
	(and arrows), the associativity map is $a_\ox \subseteq ((A \times B) \times C) \times (A \times (B \times C ))
	:= \{ (((a,b),c),(a,(b,c))) | a \in A,b \in B, c \in C\}$, and the unit is the one element set $I = \{ \star \}$.
    \item[Symmetry map:] $c_\otimes \subseteq (A \times B) \times (B \otimes A) := \{ ((a,b),(b,a)) | (a,b) \in A \times B \} $
	\item[Dagger:] Given $R: A \to B$, $B \xrightarrow{R^\dagger} A := \{ (b,a) | (a,b) \in R \}$ is the converse relation of $f$.
\end{description}
Thus, $\Rel$ is a symmetric $\dagger$-monoidal category. In ${\sf Rel}$, every object is self-dual: $(\eta, \epsilon): A \dashvv A$ with 
$\eta : I \to A \times A := \{ ( \star, (x,x)) | ~ x \in A \}$, and $\epsilon$ is given by the converse relation. 
So, $\epsilon: A \times A \to I :=  \{ ( (x,x), \star) | ~ x \in A \}$.  This makes $\Rel$ a 
$\dagger$-compact closed category.

\subsection{Categories of Hilbert spaces and bounded linear maps, $\Hilb$}

The category of Hilbert Spaces and bounded linear maps, $\Hilb$, is a $\dagger$-SMC. 
The subcategory $\FHilb$ of $\Hilb$ consisting of only finite-dimensional 
Hilbert spaces is a $\dagger$-KCC. $\FHilb$ is used in CQM as a standard setting 
for studying processes in quantum information theory and quantum computing.

The category $\Hilb$ is defined as follows:
\begin{description}
	\item[Objects:] Hilbert spaces
	\item[Maps:] Bounded linear maps
	\item[Composition:] Usual composition of linear maps 
	\item[Tensor product:] Standard tensor product of of the underlying vector spaces. The tensor 
	product on the maps is given by the Kronecker product. The tensor unit is $\C$.  
	\item[Dagger:] Given any map $f: A \to B$, its adjoint $f^\dag: B \to A$ is defined as 
	the map satisfying the following equation for all $a: I \to A$, and $b: I \to B$.
   \[ \left< b f^\dag, a \right> = \left< b, af \right> \]
\end{description}
$\Hilb$ is a $\dagger$-symmetric monoidal category. In finite-dimensions, Hilbert Spaces are equipped 
with a {\bf compact structure}. Given any finite-dimensional 
Hilbert Space, $H^*$ refers to the dual space of all functionals $H \rightarrow H$. 
Recall that every finite-dimensional Hilbert space has an orthonormal basis. 
Suppose $\{ e_i \}_{i = 1}^{dimH}$ is an orthonormal basis for $H$, then: 
\[ \eta: 1 \rightarrow H^* \otimes H ; 1 \mapsto \sum_i e_i^* \otimes e_i 
\text{ and }
\epsilon: H \otimes H^* \rightarrow I ; e_i \otimes e_j^* \mapsto \delta_{ij} \] 
Note that one cannot define a counit $\epsilon$ as above for a Hilbert space since it 
could possibly result in an infinite sum for an infinite vector. 

The $\eta$ and $\epsilon$ satisfy snake equations as follows. Suppose $a \in H$ and $a = \sum_i a_i e_i$.
	\begin{align*}
	(1 \otimes \eta)(\epsilon \otimes 1) (\sum_i a_i e_i) &= (\epsilon \otimes 1) (\sum_i a_i e_i \otimes \sum_j e_j^* \otimes e_j) \\
	&= \sum_i \sum_j a_i ((\epsilon \otimes 1) e_i \otimes e_j^* \otimes e_j))  \text{ (By linearlity) }\\
	&= \sum_{ij} a_i (\delta_{ij} \otimes e_j) =  \sum_j a_j e_j = a
\end{align*}
The subcategory finite-dimensional Hilbert Spaces, $\FHilb$, is $\dagger$-compact closed.

\subsection{Categories of finite matrices over a commutative rig $R$, $\Mat(R)$}
\label{Sec: Mat(R)}
Consider the category $\Mat(R)$ defined as follows:
\begin{description}
	\item[Objects]: $n \in \N$;
	\item[Arrows:] $n \to^{M} m$, where $M$ is a $n \times m$ matrix over a commutative 
	rig\footnote{Ring without negatives, hence a semiring} $R$;
	\item[Composition:] matrix multiplication;
	\item[Tensor product:] $n \ox m := n \cdot m$; $M \ox N$ is the outer product of matrices. 
	\end{description} 
	$\Mat(R)$ is a strict symmetric monoidal category. %https://ncatlab.org/nlab/show/FinRel#as_a_category_of_matrices
	Every object in ${\sf Mat}(R)$ is self-dual: 
	$(\eta_n, \epsilon_n): n \dashvv n$, where the $\eta_n$ and $\epsilon_n$ are defined as follows:
	\[
	  \eta_n: 1 \to n \ox n := \sum_{i=1}^{n} (e_i \ox e_i) ~~~~ \epsilon_n: n \ox n \to 1 := \sum_{i=1}^{n} (e_i^T \ox e_i^T)
	\]
	Here $e_i$ is a standard  basis vector for ${\sf Mat}(R)$, and $e_i^T$ is the transpose 
	of $e_i$. For example, for $n=2$, $e_1 := [1~0]$, $e_2 := [0~1]$ and, thus,  
	$\eta_2:=[1~0~0~1]$. If $M: n \to m$ then $M^*: m \to n$, the dual of $M$, 
	$M^{*}$, is just the transpose, $M^T$. 
	
	When the commutative rig, $R$, has a conjugation, $\overline{(\_)}: R \to R$ such that 
	$\overline{r}+\overline{s} = \overline{r+s}$, $\overline{0} = 0$, $\overline{r}~\overline{s} = 
	\overline{rs}$ and $\overline{1} = 1$, then ${\sf Mat}(R)$ has a dagger given by 
	conjugate transpose.   In particular, this means ${\sf Mat}(\C)$, finite-dimensional matrices over the 
	complex numbers, in addition, has a dagger given by conjugate transpose.  
	In fact, the category $\Mat(\C)$ is equivalent to $\FHilb$ \cite[Example 1.34]{HeV19}.
	
	Note that the two element ordered set, $\mathbbm{2}$, with join as addition and meet as multiplication is a 
	rig and ${\sf Mat}(\mathbbm{2})$ is then equivalent to the category of finite sets and relations.  
	This equivalence can be turned into an isomorphism if one inflates ${\sf Mat}(R)$ 
	so that it has objects finite sets, $I,J \in {\sf Set}_f$, and maps matrices given by maps $M: I \x J \to R$.  

\section{Complete positivity}
In this section we discuss quantum channels in $\dagger$-SMCs and $\dagger$-KCCs. 
%https://quantumcomputing.stackexchange.com/questions/1365/whats-the-difference-between-a-pure-and-mixed-quantum-state
%https://ncatlab.org/nlab/show/pure+state 
In quantum mechanics, the representation of a physical state can be either {\em pure} or {\em mixed}. 
A representation is mixed when it is a statistical ensemble of possible states of the system. 
If the representation is pure, then one knows the exact quantum state of the system.  

%Mixed states contains information about the system and its environment, hence 
%contains only partial information about the system. 
%TODO: WRITE EXAMPLE
 
While a pure state is represented as a vector in a Hilbert Space, a mixed state is represented as a 
positive self-adjoint operator on the Hilbert Space. The mixed state formalism of 
quantum mechanics is very useful in practice since in an experimental setting our knowledge about the state of 
a quantum system is often limited. In the mixed state formalism, a quantum process is a 
completely positive map (sending positive self-adjoint operators to positive self-adjoint operators)  
which preserves trace. In this section, we discuss how completely positive maps and traces are  
abstracted in a categorical setting.

\subsection{The CPM construction}

We begin our discussion with linear operators on finite-dimensional Hilbert Spaces. 
A {\bf linear operator} is a linear map from a Hilbert space to itself. 
The space of all linear operators, $\mathcal{L}(H)$, on a finite-dimensional Hilbert Space 
$H$ is a $\C^\star$-algebra with the $\star$ on $\mathcal{L}(H)$ defined to be the adjoint operation. 
Considering the linear map in matrix form, the adjoint of the matrix is its conjugate transpose.
An element $a$ of a $\C^\star$ algebra is {\bf positive} if there exists an element $b$ in 
the algebra such that $a = b^\star b$. Note that a positive element is 
self-adjoint, that is $a^\star = a$. The mixed state of quantum system is given a by positive element of 
$\mathcal{L}(H)$ of norm 1. 

A linear map between two $\C^\star$ algebras is {\bf positive} 
if  it preserves the positive elements, that is, $f(x^\star x) = y^\star y$. A linear map is 
said to be {\bf completely positive} if it is positive and for all $n > 0$, the map $1_n \ox f : \C^{n \times n} 
\ox A \to \C^{n \times n} \ox B$ is positive. If the representation of a quantum state is mixed, 
then completely positive maps that preserve trace are used to represent quantum processes. 

The following theorem by Choi characterizes the form of completely positive maps:
\begin{theorem} \cite[Theorem 1]{Cho75} Let $H$ and $K$ be finite-dimensional Hilbert spaces.  
	A linear map $\phi: \mathcal{L}(H) \to \mathcal{L}(K)$ is completely 
	positive if and only if there exists a collection of  linear maps, $\{M_i | M_i \in \mathcal{L}(H,K)$ 
	such that for all $A \in \mathcal{L}(H)$, 
	\[ \phi(A) = \sum_i M_i^\dag A M_i \] 
	where $M_i^\dag$ is a adjoint of $M_i$.
\end{theorem}

The equation in the above theorem is referred to as the {\bf Kraus decomposition} of $\phi$, and the 
collection of maps, $\{M_i | M_i \in \mathcal{L}(H,K)$, are referred to as the {\bf Kraus operators}.  

Selinger \cite{Sel07} abstracted the notion of completely positive maps to 
$\dagger$-compact closed categories using the characterization discussed above.  
In a $\dagger$-compact closed category, a map $f: A \to A$ is {\bf positive} if there exists a $g: A \to B$ 
such that $f = g g^\dag$. A map $f: A^* \ox A \to B^* \ox B$ is {\bf completely positive} if $f$ is of the 
following form: 
\begin{equation}
	\label{eqn: CPM map}
 f = \begin{tikzpicture}
	\begin{pgfonlayer}{nodelayer}
		\node [style=twocircle] (0) at (-1, 3) {};
		\node [style=twocircle] (1) at (1, 3) {};
		\node [style=none, scale=1.5] (2) at (1, 3) {$g$};
		\node [style=none, scale=1.5] (3) at (-1, 3) {$\overline{g}$};
		\node [style=none] (4) at (-1.75, 2) {};
		\node [style=none] (5) at (-0.5, 2) {};
		\node [style=none] (6) at (0.5, 2) {};
		\node [style=none] (7) at (1.75, 2) {};
		\node [style=none] (8) at (-1, 4) {};
		\node [style=none] (9) at (1, 4) {};
		\node [style=none] (10) at (-1.5, 3.75) {$A^*$};
		\node [style=none] (11) at (1.5, 3.75) {$A$};
		\node [style=none] (12) at (-2.25, 1.75) {$B^*$};
		\node [style=none] (13) at (0.25, 2.75) {$E$};
		\node [style=none] (14) at (1.75, 1.25) {};
		\node [style=none] (15) at (2.25, 1.75) {$B$};
		\node [style=none] (16) at (-1.75, 1.25) {};
	\end{pgfonlayer}
	\begin{pgfonlayer}{edgelayer}
		\draw [in=90, out=-150] (1) to (6.center);
		\draw [bend left] (1) to (7.center);
		\draw [in=90, out=-30] (0) to (5.center);
		\draw [bend right] (0) to (4.center);
		\draw (8.center) to (0);
		\draw (9.center) to (1);
		\draw [bend right=90, looseness=1.75] (5.center) to (6.center);
		\draw (7.center) to (14.center);
		\draw (16.center) to (4.center);
	\end{pgfonlayer}
\end{tikzpicture} 
\end{equation} 
where $\overline{g} := g^{\dag *} = g^{*\dag}$. The object $E$ in the above diagram 
is interpreted as the {\em environment} which is {\em discarded} after the process $f$ is complete. More on 
environment and discarding will be discussed later in Section \ref{Sec: environment}. The map 
$g: A \to E \ox B$ can be interpreted as a Kraus operator and the diagram itself as a description of the 
Kraus decomposition. 

Selinger also introduced the completely positive maps (CPM) construction which reconciles
 the pure states and mixed states formalisms of quantum theory within the theory 
of $\dagger$-compact closed categories.  
Given a $\dagger$-compact closed category $\X$, the category CPM$(\X)$ consists of objects of the form 
$A^* \ox A$ for all $A \in \C$ and the maps are chosen to be the completely positive maps. CPM$(\X)$ is 
again a $\dagger$-compact closed category \cite[Theorem 4.20]{Sel07} with the same $\dagger$ functor 
as $\C$. The CPM construction applied to the category of finite dimensional Hilbert spaces produces 
a category containing mixed states and quantum processes.  

\subsection{The $\CP^\infty$ construction}

Note that the CPM construction uses the compact structure of $\dagger$-KCCs, thus is applicable only 
to the category of finite dimensional Hilbert spaces. 
Coecke and Heunen \cite{CoH16} generalized the CPM construction to $\dagger$-monoidal categories 
thereby eliminating the restriction on dimensions. Accordingly, their 
generalized construction is referred to as the $\CP^\infty$ construction.

In $\dagger$-SMCs, one cannot bend wires, hence the representation of a 
completely positive map as shown in equation \ref{eqn: CPM map} cannot be used. 
In order to define a completely positive map within a $\dagger$-SMC, Coecke and Heunen 
straightened the wire for the environment in diagram \ref{eqn: CPM map}, 
obtaining the following form for {\bf completely positive} maps in this setting:
\begin{equation}
	\label{eqn: CP map}
	 \begin{tikzpicture}
		\begin{pgfonlayer}{nodelayer}
			\node [style=twocircle] (0) at (1, -0.25) {};
			\node [style=twocircle] (1) at (1, 3.25) {};
			\node [style=none] (2) at (1, 3.25) {$g$};
			\node [style=none] (3) at (1, -0.25) {$g^\dag$};
			\node [style=none] (4) at (1.75, 0.75) {};
			\node [style=none] (5) at (0.25, 0.75) {};
			\node [style=none] (6) at (0.25, 2.25) {};
			\node [style=none] (7) at (1.75, 2.25) {};
			\node [style=none] (8) at (1, -1.25) {};
			\node [style=none] (9) at (1, 4.25) {};
			\node [style=none] (10) at (0.75, -1) {$A$};
			\node [style=none] (11) at (0.75, 4) {$A$};
			\node [style=none] (13) at (0, 1.5) {$E$};
			\node [style=none] (15) at (2, 0.25) {$B$};
			\node [style=none] (16) at (2, 2.75) {$B$};
			\node [style=circle, scale=3, dashed] (17) at (1.75, 1.5) {};
		\end{pgfonlayer}
		\begin{pgfonlayer}{edgelayer}
			\draw [in=90, out=-150] (1) to (6.center);
			\draw [in=90, out=-30, looseness=1.25] (1) to (7.center);
			\draw [in=-90, out=150] (0) to (5.center);
			\draw [in=270, out=30, looseness=1.25] (0) to (4.center);
			\draw (8.center) to (0);
			\draw (9.center) to (1);
			\draw (6.center) to (5.center);
		\end{pgfonlayer}
	\end{tikzpicture}	
\end{equation}
Within the dashed circle, `test maps' are plugged in as shown in the equation \ref{eqn: CP equivalence}. 
The map $g: A \to E \ox B$ is called a {\bf Kraus} map and $E$ is referred to as the 
environment or the ancillary system. 

Any two Kraus maps, $f: A \to E \ox B$, and $g: A \to F \ox B$ are said to be {\bf equivalent}, 
that is $f \sim g$, if for all $h: B \ox C \to D$, they satisfy the following equation:
\begin{equation}
\label{eqn: CP equivalence}
\begin{tikzpicture}
	\begin{pgfonlayer}{nodelayer}
		\node [style=twocircle] (0) at (1, -0.25) {};
		\node [style=twocircle] (1) at (1, 3.25) {};
		\node [style=none] (2) at (1, 3.25) {$f$};
		\node [style=none] (3) at (1, -0.25) {$f^\dag$};
		\node [style=none] (5) at (0.25, 0.75) {};
		\node [style=none] (6) at (0.25, 2.25) {};
		\node [style=none] (8) at (1, -1.25) {};
		\node [style=none] (9) at (1, 4.25) {};
		\node [style=none] (10) at (0.75, -1) {$A$};
		\node [style=none] (11) at (0.75, 4) {$A$};
		\node [style=none] (13) at (0, 1.5) {$E$};
		\node [style=none] (15) at (2.75, -1) {$B$};
		\node [style=none] (16) at (2.75, 4) {$B$};
		\node [style=twocircle] (18) at (1.75, 0.75) {};
		\node [style=twocircle] (19) at (1.75, 2.25) {};
		\node [style=none] (20) at (2.5, 4.25) {};
		\node [style=none] (21) at (2.5, -1.25) {};
		\node [style=none] (22) at (2, 1.5) {$D$};
		\node [style=none] (23) at (1.75, 2.25) {$h$};
		\node [style=none] (24) at (1.75, 0.75) {$h^\dag$};
	\end{pgfonlayer}
	\begin{pgfonlayer}{edgelayer}
		\draw [in=90, out=-150] (1) to (6.center);
		\draw [in=-90, out=150] (0) to (5.center);
		\draw (8.center) to (0);
		\draw (9.center) to (1);
		\draw (6.center) to (5.center);
		\draw [in=-90, out=15, looseness=1.25] (0) to (18);
		\draw [in=-15, out=90] (19) to (1);
		\draw (19) to (18);
		\draw [in=-90, out=45] (19) to (20.center);
		\draw [in=90, out=-30] (18) to (21.center);
	\end{pgfonlayer}
\end{tikzpicture} = \begin{tikzpicture}
	\begin{pgfonlayer}{nodelayer}
		\node [style=twocircle] (0) at (1, -0.25) {};
		\node [style=twocircle] (1) at (1, 3.25) {};
		\node [style=none] (2) at (1, 3.25) {$g$};
		\node [style=none] (3) at (1, -0.25) {$g^\dag$};
		\node [style=none] (5) at (0.25, 0.75) {};
		\node [style=none] (6) at (0.25, 2.25) {};
		\node [style=none] (8) at (1, -1.25) {};
		\node [style=none] (9) at (1, 4.25) {};
		\node [style=none] (10) at (0.75, -1) {$A$};
		\node [style=none] (11) at (0.75, 4) {$A$};
		\node [style=none] (13) at (0, 1.5) {$F$};
		\node [style=none] (15) at (2.75, -1) {$B$};
		\node [style=none] (16) at (2.75, 4) {$B$};
		\node [style=twocircle] (18) at (1.75, 0.75) {};
		\node [style=twocircle] (19) at (1.75, 2.25) {};
		\node [style=none] (20) at (2.5, 4.25) {};
		\node [style=none] (21) at (2.5, -1.25) {};
		\node [style=none] (22) at (2, 1.5) {$D$};
		\node [style=none] (23) at (1.75, 2.25) {$h$};
		\node [style=none] (24) at (1.75, 0.75) {$h^\dag$};
	\end{pgfonlayer}
	\begin{pgfonlayer}{edgelayer}
		\draw [in=90, out=-150] (1) to (6.center);
		\draw [in=-90, out=150] (0) to (5.center);
		\draw (8.center) to (0);
		\draw (9.center) to (1);
		\draw (6.center) to (5.center);
		\draw [in=-90, out=15, looseness=1.25] (0) to (18);
		\draw [in=-15, out=90] (19) to (1);
		\draw (19) to (18);
		\draw [in=-90, out=45] (19) to (20.center);
		\draw [in=90, out=-30] (18) to (21.center);
	\end{pgfonlayer}
\end{tikzpicture}
\end{equation}
For any Kraus map, $f: A \to E \ox B$, we will write its equivalence class as $[f]$.


\begin{definition} \cite[Definition 3]{CoH16} The {\bf $\CP^\infty$ construction} is defined as follows. 
Let $\C$ be any $\dagger$-monoidal category.
\begin{description}
	\item[Objects:] Same as objects as $\C$
	\item[Maps:] Equivalence classes of Kraus maps in $\C$
	\item[Identity maps:] Equivalence class of Kraus maps given by the left 
unitor, $[(u_\ox^l)^{-1} : A \to I \ox A]$. 
	\item[Composition:] The composition of two maps $f: A \to B$, and $g: B \to C \in \CP^\infty(\C)$  
	is given by composing their respective Kraus maps upto equivalence:
	\[ fg := \left[ \begin{tikzpicture}
		\begin{pgfonlayer}{nodelayer}
			\node [style=twocircle] (1) at (1, 3.25) {};
			\node [style=none] (2) at (1, 3.25) {$f$};
			\node [style=none] (5) at (0.25, 0.75) {};
			\node [style=none] (6) at (0.25, 2.25) {};
			\node [style=none] (9) at (1, 4.25) {};
			\node [style=none] (10) at (0.65, 1) {$E'$};
			\node [style=none] (11) at (0.75, 4) {$A$};
			\node [style=none] (13) at (0, 1) {$E$};
			\node [style=none] (15) at (1.85, 3) {$B$};
			\node [style=twocircle] (19) at (1.75, 2.25) {};
			\node [style=none] (22) at (1, 0.75) {};
			\node [style=none] (23) at (2.5, 0.75) {};
			\node [style=none] (24) at (1.75, 2.25) {$g$};
			\node [style=none] (25) at (2.75, 1) {$C$};
		\end{pgfonlayer}
		\begin{pgfonlayer}{edgelayer}
			\draw [in=90, out=-150] (1) to (6.center);
			\draw (9.center) to (1);
			\draw (6.center) to (5.center);
			\draw [in=-15, out=90] (19) to (1);
			\draw [in=90, out=-150, looseness=1.25] (19) to (22.center);
			\draw [in=90, out=-30, looseness=1.25] (19) to (23.center);
		\end{pgfonlayer}
	\end{tikzpicture} \right] \]   
\end{description} 
The composition is well-defined because if $f \sim f'$ and $g \sim g'$, then $fg \sim f'g'$.
\end{definition}

If $\C$ is a $\dagger$-SMC, then $\CP^\infty(\C)$ is a SMC \cite[Proposition 5]{CoH16}. 
Note that, in this case, $\CP^\infty(\C)$ does not have a $\dagger$-functor.
Moreover, if $\C$ is a $\dagger$-KCC, then $\CP^\infty(\C)$ isomorphic to CPM$(\C)$ 
\cite[Proposition 6]{CoH16}. 

The set of all bounded linear maps from a Hilbert space to itself is a Von Neumann algebra. 
The category of Von Neumann algebras and normal completely positive maps is isomorphic to $\CP^\infty(\Hilb)$ 
\cite[Theorem 13]{CoH16}.

\subsection{Environment structures}
\label{Sec: environment}

In the previous sections, we discussed the constructions for transforming a category of pure states 
and processes into a category of mixed states and completely positive maps. In this section we 
discuss characterizations of these constructions using the notion of {\em environment} \cite{CoH16,Coecke10,Coe13}. 
While pure states contain information {\em purely} about the system, 
mixed states contain information about the system and its environment together.
One can thus characterize the categories of mixed states by the presence of environment 
structure and discarding maps. 

\begin{definition}
An {\bf environment structure} for a $\dagger$-SMC, $\C_{pure}$, consists of a strict monoidal functor 
$F: \C_{pure} \to \C$ where $\C$ is a SMC  with designated map, $\gamma_A: A \to I$, called the 
{\bf discarding} map for all objects $A \in \C$ such that the following equations hold:
\begin{description}
\item[Env 1:] For all objects $A, B \in \C_{pure}$,  $\gamma_{F(A \ox B)} = \gamma_{F(A)} \ox \gamma_{F(B)} (u_\ox)_I $
\item[Env 2:] $\gamma_I = id_I$ (F(I) = I)
\item[Env 3:] For all Kraus maps $f: A \to X \ox B, g: A  \to Y \ox B \in \C_{pure}$, 

$f \sim g$ if and only if $F(f) (\gamma_{F(X)} \ox 1_B) = F(g) (\gamma_{F(Y)} \ox 1_B)$.
\end{description}
\end{definition}
The axioms for an environment structure are drawn as follows:
\[ {\bf [Env~1]:}~~~ \begin{tikzpicture}
	\begin{pgfonlayer}{nodelayer}
		\node [style=none] (0) at (-1, 0) {};
		\node [style=none] (1) at (0, 0) {};
		\node [style=none] (4) at (-0.8, -0.25) {};
		\node [style=none] (5) at (-0.2, -0.25) {};
		\node [style=none] (6) at (-0.6, -0.5) {};
		\node [style=none] (7) at (-0.4, -0.5) {};
		\node [style=none] (2) at (-0.5, 0) {};
		\node [style=none] (3) at (-0.5, 1) {};
		\node [style=none] (12) at (0.5, 0.75) {$F(A \ox B)$};
	\end{pgfonlayer}
	\begin{pgfonlayer}{edgelayer}
		\draw (3.center) to (2.center);
		\draw[thick] (0.center) to (1.center);
		\draw[thick] (4.center) to (5.center);
		\draw[thick] (6.center) to (7.center);
	\end{pgfonlayer}
\end{tikzpicture} = \begin{tikzpicture}
	\begin{pgfonlayer}{nodelayer}
		\node [style=none] (0) at (-1, 0) {};
		\node [style=none] (1) at (0, 0) {};
		\node [style=none] (14) at (-0.8, -0.25) {};
		\node [style=none] (15) at (-0.2, -0.25) {};
		\node [style=none] (16) at (-0.6, -0.5) {};
		\node [style=none] (17) at (-0.4, -0.5) {};
		\node [style=none] (2) at (-0.5, 0) {};
		\node [style=none] (3) at (-0.5, 1) {};
		\node [style=none] (4) at (0.5, 0) {};
		\node [style=none] (5) at (1.5, 0) {};
		\node [style=none] (6) at (0.7, -0.25) {};
		\node [style=none] (7) at (1.3, -0.25) {};
		\node [style=none] (8) at (0.9, -0.5) {};
		\node [style=none] (9) at (1.1, -0.5) {};
		\node [style=none] (10) at (1, 0) {};
		\node [style=none] (11) at (1, 1) {};
		\node [style=none] (12) at (-1.2, 0.75) {$F(A)$};
		\node [style=none] (13) at (1.5, 0.75) {$F(B)$};
	\end{pgfonlayer}
	\begin{pgfonlayer}{edgelayer}
		\draw (3.center) to (2.center);
		\draw[thick] (0.center) to (1.center);
		\draw[thick] (14.center) to (15.center);
		\draw[thick] (16.center) to (17.center);
		\draw (11.center) to (10.center);
		\draw[thick] (8.center) to (9.center);
		\draw[thick] (4.center) to (5.center);
		\draw[thick] (6.center) to (7.center);
	\end{pgfonlayer}
\end{tikzpicture}
~~~~~~~~
{\bf [Env~2]:} ~~~\begin{tikzpicture}
	\begin{pgfonlayer}{nodelayer}
		\node [style=none] (0) at (-1, 0) {};
		\node [style=none] (1) at (0, 0) {};
		\node [style=none] (4) at (-0.8, -0.25) {};
		\node [style=none] (5) at (-0.2, -0.25) {};
		\node [style=none] (6) at (-0.6, -0.5) {};
		\node [style=none] (7) at (-0.4, -0.5) {};
		\node [style=none] (2) at (-0.5, 0) {};
		\node [style=none] (3) at (-0.5, 1) {};
		\node [style=none] (12) at (-0.25, 0.75) {$I$};
	\end{pgfonlayer}
	\begin{pgfonlayer}{edgelayer}
		\draw (3.center) to (2.center);
		\draw[thick] (0.center) to (1.center);
		\draw[thick] (4.center) to (5.center);
		\draw[thick] (6.center) to (7.center);
	\end{pgfonlayer}
\end{tikzpicture} = id_I \] \[ {\bf [Env~3]:}~~~ f \sim g ~~~\Leftrightarrow~~~
\begin{tikzpicture}
	\begin{pgfonlayer}{nodelayer}
		\node [style=none] (0) at (-1, 0) {};
		\node [style=none] (1) at (0, 0) {};
		\node [style=none] (11) at (-0.8, -0.25) {};
		\node [style=none] (12) at (-0.2, -0.25) {};
		\node [style=none] (13) at (-0.6, -0.5) {};
		\node [style=none] (14) at (-0.4, -0.5) {};
		\node [style=none] (2) at (-0.5, 0) {};
		\node [style=none] (3) at (-0.5, 0.5) {};
		\node [style=circle, scale=2.5] (4) at (0.25, 1.25) {};
		\node [style=none] (5) at (1, 0.5) {};
		\node [style=none] (6) at (0.25, 2.25) {};
		\node [style=none] (7) at (0.25, 1.25) {$F(f)$};
		\node [style=none] (8) at (1, 2) {$F(A)$};
		\node [style=none] (9) at (-1.25, 0.5) {$F(X)$};
		\node [style=none] (10) at (1.75, -0.5) {$F(B)$};
		\node [style=none] (15) at (1, -0.75) {};
	\end{pgfonlayer}
	\begin{pgfonlayer}{edgelayer}
		\draw (3.center) to (2.center);
		\draw [thick] (0.center) to (1.center);
		\draw [thick] (11.center) to (12.center);
		\draw [thick] (13.center) to (14.center);
		\draw [in=90, out=-15, looseness=1.25] (4) to (5.center);
		\draw [in=90, out=-165, looseness=1.25] (4) to (3.center);
		\draw (6.center) to (4);
		\draw (5.center) to (15.center);
	\end{pgfonlayer}
\end{tikzpicture} = \begin{tikzpicture}
	\begin{pgfonlayer}{nodelayer}
		\node [style=none] (0) at (-1, 0) {};
		\node [style=none] (1) at (0, 0) {};
		\node [style=none] (11) at (-0.8, -0.25) {};
		\node [style=none] (12) at (-0.2, -0.25) {};
		\node [style=none] (13) at (-0.6, -0.5) {};
		\node [style=none] (14) at (-0.4, -0.5) {};
		\node [style=none] (2) at (-0.5, 0) {};
		\node [style=none] (3) at (-0.5, 0.5) {};
		\node [style=circle, scale=2.5] (4) at (0.25, 1.25) {};
		\node [style=none] (5) at (1, 0.5) {};
		\node [style=none] (6) at (0.25, 2.25) {};
		\node [style=none] (7) at (0.25, 1.25) {$F(g)$};
		\node [style=none] (8) at (1, 2) {$F(A)$};
		\node [style=none] (9) at (-1.25, 0.5) {$F(Y)$};
		\node [style=none] (10) at (1.75, -0.5) {$F(B)$};
		\node [style=none] (15) at (1, -0.75) {};
	\end{pgfonlayer}
	\begin{pgfonlayer}{edgelayer}
		\draw (3.center) to (2.center);
		\draw [thick] (0.center) to (1.center);
		\draw [thick] (11.center) to (12.center);
		\draw [thick] (13.center) to (14.center);
		\draw [in=90, out=-15, looseness=1.25] (4) to (5.center);
		\draw [in=90, out=-165, looseness=1.25] (4) to (3.center);
		\draw (6.center) to (4);
		\draw (5.center) to (15.center);
	\end{pgfonlayer}
\end{tikzpicture} \]

A process $f: A \to B$ in a $\dagger$-SMC, $\C_{pure}$, with an environment structure 
$(F: \C_{pure} \to \C, \envmap)$ is said to be {\bf normalised} if $F(f) \gamma_{F(B)} = \gamma_{F(A)}$. 
The normalised processes of $\C_{pure}$ form a sub-$\dagger$-SMC of $\C$ with the same 
environment structure. In that case, for the sub-$\dagger$-SMC, the discarding map $\gamma$ is a 
monoidal transformation for from the functor $F$ to the constant endofunctor that sends all objects to $I$ 
and all maps to $id_I$. 

\begin{definition}
A $\dagger$-SMC, $\C_{pure}$, with an environment structure $(F: \C_{pure} \to \C, \envmap)$ is said to allow 
{\bf purification} if  every map in $\C$ is of the following form for some Kraus map $f: A \to X \ox B \in \C_{pure}$
\[ {\bf[Env~4]}~~~ \begin{tikzpicture}
	\begin{pgfonlayer}{nodelayer}
		\node [style=none] (0) at (-1, 0) {};
		\node [style=none] (1) at (0, 0) {};
		\node [style=none] (11) at (-0.8, -0.25) {};
		\node [style=none] (12) at (-0.2, -0.25) {};
		\node [style=none] (13) at (-0.6, -0.5) {};
		\node [style=none] (14) at (-0.4, -0.5) {};
		\node [style=none] (2) at (-0.5, 0) {};
		\node [style=none] (3) at (-0.5, 0.5) {};
		\node [style=circle, scale=2.5] (4) at (0.25, 1.25) {};
		\node [style=none] (5) at (1, 0.5) {};
		\node [style=none] (6) at (0.25, 2.25) {};
		\node [style=none] (7) at (0.25, 1.25) {$F(f)$};
		\node [style=none] (8) at (1, 2) {$F(A)$};
		\node [style=none] (9) at (-1.25, 0.5) {$F(X)$};
		\node [style=none] (10) at (1.75, -0.5) {$F(B)$};
		\node [style=none] (15) at (1, -0.75) {};
	\end{pgfonlayer}
	\begin{pgfonlayer}{edgelayer}
		\draw (3.center) to (2.center);
		\draw [thick] (0.center) to (1.center);
		\draw [thick] (11.center) to (12.center);
		\draw [thick] (13.center) to (14.center);
		\draw [in=90, out=-15, looseness=1.25] (4) to (5.center);
		\draw [in=90, out=-165, looseness=1.25] (4) to (3.center);
		\draw (6.center) to (4);
		\draw (5.center) to (15.center);
	\end{pgfonlayer}
\end{tikzpicture}  \]
\end{definition}

The idea of {\bf [Env 3]} is that every process in $\C$ is equal to a pure process followed by discarding information. 

Every $\dagger$-SMC, $\C_{pure}$,  comes with a canonical environment structure which allows purification \cite[Theorem 16]{CoH16}:
\[ F: \C_{pure} \to \CP^\infty(\C_{pure}); A \to^f B \mapsto  A \to^{[f (u_\ox^l)^{-1}]} B \]
\[ \gamma_A : A \to I \in \CP^\infty(\C_{pure}) := [(u_\ox^r)^{-1}] \] 
The environment structure also allows for purification:  for each $[f]: A \to B$ in $\CP^\infty(\C_{pure})$, 
$F(f) (\gamma_I \ox 1_B) = [f]$. 

\begin{theorem}\cite[Theorem 15]{CoH16}
	For any $\dagger$-SMC, $\C_{pure}$, $\CP^\infty(\C_{pure}) \simeq \C$ 
	is an isomorphism of monoidal categories if $\C_{pure}$ is equipped 
	with an environment structure and purification. 
\end{theorem}

The above result can be extended to $\dagger$-compact closed categories  when 
the environment structure behaves coherently with the compact structure in the 
following sense:
\begin{definition}
	An {\bf environment structure} for a $\dagger$-KCC, $\C_{pure}$,  consists of a 
	strict $\dagger$-monoidal functor $F: \C_{pure} \to \C$, where $\C$ is also a $\dagger$-KCC
	and the following conditions hold:
	\begin{enumerate}[(i)]
	\item For all $A \in \C_{pure}$, $F(A^*) = F(A)^*$ 
	\item For all $A \in \C_{pure}$, $F(A)^* \simeq F(A)$
	\item For each object $X \in \C$, there exists a designated map $\gamma_X: X \to I$ such that 
	{\bf [Env 1]}-{\bf[Env 3]} holds and:
	\[ \begin{tikzpicture}
		\begin{pgfonlayer}{nodelayer}
			\node [style=none] (0) at (-1, 0) {};
			\node [style=none] (1) at (0, 0) {};
			\node [style=none] (4) at (-0.8, -0.25) {};
			\node [style=none] (5) at (-0.2, -0.25) {};
			\node [style=none] (6) at (-0.6, -0.5) {};
			\node [style=none] (7) at (-0.4, -0.5) {};
			\node [style=none] (2) at (-0.5, 0) {};
			\node [style=none] (3) at (-0.5, 1) {};
			\node [style=none] (12) at (1.25, -0.25) {$F(A)^*$};
			\node [style=none] (13) at (0.5, 1) {};
			\node [style=none] (14) at (0.5, -0.75) {};
			\node [style=none] (15) at (-1, 0.5) {};
			\node [style=none] (16) at (1, 0.5) {};
			\node [style=none] (17) at (1, 1.75) {};
			\node [style=none] (18) at (-1, 1.75) {};
			\node [style=none] (19) at (0.75, 0.75) {$F$};
		\end{pgfonlayer}
		\begin{pgfonlayer}{edgelayer}
			\draw (3.center) to (2.center);
			\draw [thick] (0.center) to (1.center);
			\draw [thick] (4.center) to (5.center);
			\draw [thick] (6.center) to (7.center);
			\draw [bend left=90, looseness=1.50] (3.center) to (13.center);
			\draw (14.center) to (13.center);
			\draw (15.center) to (16.center);
			\draw (16.center) to (17.center);
			\draw (17.center) to (18.center);
			\draw (18.center) to (15.center);
		\end{pgfonlayer}
	\end{tikzpicture} = \left( \begin{tikzpicture}
		\begin{pgfonlayer}{nodelayer}
			\node [style=none] (0) at (-1, 0) {};
			\node [style=none] (1) at (0, 0) {};
			\node [style=none] (4) at (-0.8, -0.25) {};
			\node [style=none] (5) at (-0.2, -0.25) {};
			\node [style=none] (6) at (-0.6, -0.5) {};
			\node [style=none] (7) at (-0.4, -0.5) {};
			\node [style=none] (2) at (-0.5, 0) {};
			\node [style=none] (3) at (-0.5, 1.25) {};
			\node [style=none] (8) at (0, 1.2) {$F(A)$};
		\end{pgfonlayer}
		\begin{pgfonlayer}{edgelayer}
			\draw (3.center) to (2.center);
			\draw [thick] (0.center) to (1.center);
			\draw [thick] (4.center) to (5.center);
			\draw [thick] (6.center) to (7.center);
		\end{pgfonlayer}
	\end{tikzpicture} \right)^\dag	 \] 
\end{enumerate}
The environment structure allows purification if {\bf[Env 5]} holds.
\end{definition}

Every $\dagger$-compact closed category comes with a canonical environment structure given 
as follows:
\[ F: \C_{pure} \to \text{CPM}(\C_{pure}); A \to^f B \mapsto (A^* \ox A) \to^{f^* \ox f}  B^* \ox B\]
\[ \gamma_A :=  \begin{tikzpicture}
	\begin{pgfonlayer}{nodelayer}
		\node [style=none] (0) at (-5.25, 6.75) {};
		\node [style=none] (1) at (-3.75, 6.75) {};
		\node [style=none] (2) at (-5.65, 6.5) {$A^*$};
		\node [style=none] (3) at (-3.5, 6.5) {$A$};
	\end{pgfonlayer}
	\begin{pgfonlayer}{edgelayer}
		\draw [bend right=90, looseness=2.25] (0.center) to (1.center);
	\end{pgfonlayer}
\end{tikzpicture} \]

The discarding map in the category CPM$(\FHilb)$ is referred to as the {\bf trace}. 
A map $g: A \to B$ in CPM$(\FHilb)$ as shown below is precisely a {\bf quantum channel} (completely positive and trace 
preserving) when $g \in \FHilb$ is normalized i.e, $F(g) \gamma_{F(E_\ox B)} = \gamma_{F(A)}$. 
\[ \begin{tikzpicture}
	\begin{pgfonlayer}{nodelayer}
		\node [style=twocircle] (0) at (-1, 3) {};
		\node [style=twocircle] (1) at (1, 3) {};
		\node [style=none, scale=1.5] (2) at (1, 3) {$g$};
		\node [style=none, scale=1.5] (3) at (-1, 3) {$\overline{g}$};
		\node [style=none] (4) at (-1.75, 2) {};
		\node [style=none] (5) at (-0.5, 2) {};
		\node [style=none] (6) at (0.5, 2) {};
		\node [style=none] (7) at (1.75, 2) {};
		\node [style=none] (8) at (-1, 4) {};
		\node [style=none] (9) at (1, 4) {};
		\node [style=none] (10) at (-1.35, 3.75) {$A^*$};
		\node [style=none] (11) at (1.35, 3.75) {$A$};
		\node [style=none] (12) at (-2.15, 1.75) {$B^*$};
		\node [style=none] (13) at (0.25, 2.75) {$E$};
		\node [style=none] (15) at (2, 1.75) {$B$};
		\node [style=none] (16) at (-1.75, 1) {};
		\node [style=none] (17) at (1.75, 1) {};
	\end{pgfonlayer}
	\begin{pgfonlayer}{edgelayer}
		\draw [in=90, out=-150] (1) to (6.center);
		\draw [bend left] (1) to (7.center);
		\draw [in=90, out=-30] (0) to (5.center);
		\draw [bend right] (0) to (4.center);
		\draw (8.center) to (0);
		\draw (9.center) to (1);
		\draw [bend right=90, looseness=1.75] (5.center) to (6.center);
		\draw (4.center) to (16.center);
		\draw (7.center) to (17.center);
	\end{pgfonlayer}
\end{tikzpicture}\]

The environment structure in the above example also allows for purification because every map in CPM$(\C_{pure})$ is of the 
following form for some $g: A \to E \ox B \in \C_{pure}$:
\[ \begin{tikzpicture}
	\begin{pgfonlayer}{nodelayer}
		\node [style=twocircle] (0) at (-1, 3) {};
		\node [style=twocircle] (1) at (1, 3) {};
		\node [style=none, scale=1.5] (2) at (1, 3) {$g$};
		\node [style=none, scale=1.5] (3) at (-1, 3) {$\overline{g}$};
		\node [style=none] (4) at (-1.75, 2) {};
		\node [style=none] (5) at (-0.5, 2) {};
		\node [style=none] (6) at (0.5, 2) {};
		\node [style=none] (7) at (1.75, 2) {};
		\node [style=none] (8) at (-1, 4) {};
		\node [style=none] (9) at (1, 4) {};
		\node [style=none] (10) at (-1.5, 3.75) {$A^*$};
		\node [style=none] (11) at (1.5, 3.75) {$A$};
		\node [style=none] (12) at (-2.25, 1.75) {$B^*$};
		\node [style=none] (13) at (0.25, 2.75) {$E$};
		\node [style=none] (14) at (1.75, 1.25) {};
		\node [style=none] (15) at (2.25, 1.75) {$B$};
		\node [style=none] (16) at (-1.75, 1.25) {};
	\end{pgfonlayer}
	\begin{pgfonlayer}{edgelayer}
		\draw [in=90, out=-150] (1) to (6.center);
		\draw [bend left] (1) to (7.center);
		\draw [in=90, out=-30] (0) to (5.center);
		\draw [bend right] (0) to (4.center);
		\draw (8.center) to (0);
		\draw (9.center) to (1);
		\draw [bend right=90, looseness=1.75] (5.center) to (6.center);
		\draw (7.center) to (14.center);
		\draw (16.center) to (4.center);
	\end{pgfonlayer}
\end{tikzpicture} = \begin{tikzpicture}
	\begin{pgfonlayer}{nodelayer}
		\node [style=none] (0) at (-1, 0) {};
		\node [style=none] (1) at (0, 0) {};
		\node [style=none] (11) at (-0.8, -0.25) {};
		\node [style=none] (12) at (-0.2, -0.25) {};
		\node [style=none] (13) at (-0.6, -0.5) {};
		\node [style=none] (14) at (-0.4, -0.5) {};
		\node [style=none] (2) at (-0.5, 0) {};
		\node [style=none] (3) at (-0.5, 0.5) {};
		\node [style=circle, scale=2.5] (4) at (0.25, 1.25) {};
		\node [style=none] (5) at (1, 0.5) {};
		\node [style=none] (6) at (0.25, 2.25) {};
		\node [style=none] (7) at (0.25, 1.25) {$F(g)$};
		\node [style=none] (8) at (1, 2) {$F(A)$};
		\node [style=none] (9) at (-1.25, 0.5) {$F(E)$};
		\node [style=none] (10) at (1.75, -0.5) {$F(B)$};
		\node [style=none] (15) at (1, -0.75) {};
	\end{pgfonlayer}
	\begin{pgfonlayer}{edgelayer}
		\draw (3.center) to (2.center);
		\draw [thick] (0.center) to (1.center);
		\draw [thick] (11.center) to (12.center);
		\draw [thick] (13.center) to (14.center);
		\draw [in=90, out=-15, looseness=1.25] (4) to (5.center);
		\draw [in=90, out=-165, looseness=1.25] (4) to (3.center);
		\draw (6.center) to (4);
		\draw (5.center) to (15.center);
	\end{pgfonlayer}
\end{tikzpicture}  \]

\section{Measurement and complementarity}
We covered quantum channels in the previous section. In this section we review the 
notions of quantum observables, measurement and complementarity categorically. 

\subsection{Quantum observables}
\label{Sec: observables}

A quantum observable is a physical property which can be measured. 
Frobenius algebras are one of the pillars of CQM because they are used to 
abstract the notion of quantum observables. 
In traditional quantum mechanics, an observable is 
represented by a self-adjoint operator a.k.a Hermitian operator 
on a Hilbert space. The set of all eigenvectors 
for an observable gives an orthogonal basis for the state space of the quantum system.
After a measurement, the state of the quantum system will be one of these basis states.
The eigenvalue corresponding to an eigenvector represents its probability amplitude 
i.e., the probablity that the quantum system will end up in the particular basis state after 
measurement. Frobenius algebras provide a neat abstraction of these ideas in 
$\dagger$-monoidal categories. In this section, we review Frobenius algebras and 
their correspondence to quantum observables. 

In a SMC, a monoid $\monoid{A}$ consists of an object $A$ with a multiplication map, 
$\mulmap{1.5}{white}: A \ox A \to A$, and a unit map, $\unitmap{1.5}{white}: I \to A$, 
such that the multiplication is assosciative (see diagram (a)) and the unit law holds 
(see diagram (b)).
\[ (a)~~~	\begin{tikzpicture}
	\begin{pgfonlayer}{nodelayer}
		\node [style=circle] (1) at (-1, 2.5) {};
		\node [style=none] (7) at (-1.75, 3.5) {};
		\node [style=none] (22) at (-0.25, 3.5) {};
		\node [style=circle] (23) at (-0.25, 1.75) {};
		\node [style=none] (25) at (0.5, 2.75) {};
		\node [style=none] (27) at (-0.25, 1) {};
		\node [style=none] (28) at (0.5, 3.5) {};
	\end{pgfonlayer}
	\begin{pgfonlayer}{edgelayer}
		\draw [bend left] (1) to (7.center);
		\draw [bend right] (1) to (22.center);
		\draw [bend right] (23) to (25.center);
		\draw [bend right] (1) to (23);
		\draw (23) to (27.center);
		\draw (28.center) to (25.center);
	\end{pgfonlayer}
\end{tikzpicture} = \begin{tikzpicture}
	\begin{pgfonlayer}{nodelayer}
		\node [style=circle] (1) at (-0.25, 2.5) {};
		\node [style=none] (7) at (0.5, 3.5) {};
		\node [style=none] (22) at (-1, 3.5) {};
		\node [style=circle] (23) at (-1, 1.75) {};
		\node [style=none] (25) at (-1.75, 2.75) {};
		\node [style=none] (27) at (-1, 1) {};
		\node [style=none] (28) at (-1.75, 3.5) {};
	\end{pgfonlayer}
	\begin{pgfonlayer}{edgelayer}
		\draw [bend right] (1) to (7.center);
		\draw [bend left] (1) to (22.center);
		\draw [bend left] (23) to (25.center);
		\draw [bend left] (1) to (23);
		\draw (23) to (27.center);
		\draw (28.center) to (25.center);
	\end{pgfonlayer}
\end{tikzpicture}  ~~~~~~~~ (b) ~~~
 \begin{tikzpicture}
	\begin{pgfonlayer}{nodelayer}
		\node [style=circle] (1) at (-0.25, 2.5) {};
		\node [style=none] (7) at (0.5, 3.5) {};
		\node [style=none] (27) at (-0.25, 1.5) {};
		\node [style=circle] (28) at (-1, 3.5) {};
	\end{pgfonlayer}
	\begin{pgfonlayer}{edgelayer}
		\draw [bend right] (1) to (7.center);
		\draw (1) to (27.center);
		\draw [bend right] (28) to (1);
	\end{pgfonlayer}
\end{tikzpicture} = \begin{tikzpicture}
	\begin{pgfonlayer}{nodelayer}
		\node [style=none] (7) at (-0.25, 3.5) {};
		\node [style=none] (27) at (-0.25, 1.5) {};
	\end{pgfonlayer}
	\begin{pgfonlayer}{edgelayer}
		\draw (7.center) to (27.center);
	\end{pgfonlayer}
\end{tikzpicture} = \begin{tikzpicture}
	\begin{pgfonlayer}{nodelayer}
		\node [style=circle] (1) at (-0.25, 2.5) {};
		\node [style=none] (7) at (-1, 3.5) {};
		\node [style=none] (27) at (-0.25, 1.5) {};
		\node [style=circle] (28) at (0.5, 3.5) {};
	\end{pgfonlayer}
	\begin{pgfonlayer}{edgelayer}
		\draw [bend left] (1) to (7.center);
		\draw (1) to (27.center);
		\draw [bend left] (28) to (1);
	\end{pgfonlayer}
\end{tikzpicture} \]
Similarly, a comonoid $\comonoid{A}$ consists of an object $A$ with a comultiplication 
$\comulmap{1.5}{white}: A \to A \ox A$, and a counit map $e: A \to I$ such that the 
coumutiplication is coassociative (vertical reflection of equation for associativity), 
and that the counit law holds (vertical reflection of equation for unit law).

\begin{definition}
In a SMC, a {\bf Frobenius algebra}, $\Frob{A}$, consists of a monoid $\monoid{A}$ and 
a comonoid $\comonoid{A}$ such that the multiplication and the comultiplication 
interacts as follows:
\[  \begin{tikzpicture}
	\begin{pgfonlayer}{nodelayer}
		\node [style=circle] (0) at (-1, 3) {};
		\node [style=circle] (1) at (0, 2) {};
		\node [style=none] (4) at (-1.75, 2) {};
		\node [style=none] (7) at (0.75, 2.75) {};
		\node [style=none] (8) at (-1, 4) {};
		\node [style=none] (9) at (0, 1) {};
		\node [style=none] (11) at (0.25, 1.25) {$A$};
		\node [style=none] (16) at (-1.75, 1) {};
		\node [style=none] (17) at (0.75, 4) {};
		\node [style=none] (18) at (-2, 1.25) {$A$};
		\node [style=none] (19) at (1, 3.75) {$A$};
		\node [style=none] (20) at (-1.25, 3.75) {$A$};
	\end{pgfonlayer}
	\begin{pgfonlayer}{edgelayer}
		\draw [bend right] (1) to (7.center);
		\draw [bend right] (0) to (4.center);
		\draw (8.center) to (0);
		\draw (9.center) to (1);
		\draw (4.center) to (16.center);
		\draw (7.center) to (17.center);
		\draw (0) to (1);
	\end{pgfonlayer}
\end{tikzpicture} = \begin{tikzpicture}
	\begin{pgfonlayer}{nodelayer}
		\node [style=circle] (0) at (0, 3) {};
		\node [style=circle] (1) at (-1, 2) {};
		\node [style=none] (4) at (0.75, 2) {};
		\node [style=none] (7) at (-1.75, 2.75) {};
		\node [style=none] (8) at (0, 4) {};
		\node [style=none] (9) at (-1, 1) {};
		\node [style=none] (11) at (-1.25, 1.25) {$A$};
		\node [style=none] (16) at (0.75, 1) {};
		\node [style=none] (17) at (-1.75, 4) {};
		\node [style=none] (18) at (1, 1.25) {$A$};
		\node [style=none] (19) at (-2, 3.75) {$A$};
		\node [style=none] (20) at (0.25, 3.75) {$A$};
	\end{pgfonlayer}
	\begin{pgfonlayer}{edgelayer}
		\draw [bend left] (1) to (7.center);
		\draw [bend left] (0) to (4.center);
		\draw (8.center) to (0);
		\draw (9.center) to (1);
		\draw (4.center) to (16.center);
		\draw (7.center) to (17.center);
		\draw (0) to (1);
	\end{pgfonlayer}
\end{tikzpicture} \]
\end{definition}
The above equation is referred to as the Frobenius law. It can be proven that the Frobenius 
law holds for a monoid and a comonoid on $A$ if and only if the following equation holds:
\[ \begin{tikzpicture}
	\begin{pgfonlayer}{nodelayer}
		\node [style=circle] (0) at (-1, 3) {};
		\node [style=circle] (1) at (0, 2) {};
		\node [style=none] (4) at (-1.75, 2) {};
		\node [style=none] (7) at (0.75, 2.75) {};
		\node [style=none] (8) at (-1, 4) {};
		\node [style=none] (9) at (0, 1) {};
		\node [style=none] (11) at (0.25, 1.25) {$A$};
		\node [style=none] (16) at (-1.75, 1) {};
		\node [style=none] (17) at (0.75, 4) {};
		\node [style=none] (18) at (-2, 1.25) {$A$};
		\node [style=none] (19) at (1, 3.75) {$A$};
		\node [style=none] (20) at (-1.25, 3.75) {$A$};
	\end{pgfonlayer}
	\begin{pgfonlayer}{edgelayer}
		\draw [bend right] (1) to (7.center);
		\draw [bend right] (0) to (4.center);
		\draw (8.center) to (0);
		\draw (9.center) to (1);
		\draw (4.center) to (16.center);
		\draw (7.center) to (17.center);
		\draw (0) to (1);
	\end{pgfonlayer}
\end{tikzpicture}  = \begin{tikzpicture}
	\begin{pgfonlayer}{nodelayer}
		\node [style=circle] (0) at (-1, 2) {};
		\node [style=circle] (1) at (-1, 3) {};
		\node [style=none] (4) at (-0.25, 1) {};
		\node [style=none] (7) at (-1.75, 4) {};
		\node [style=none] (20) at (-0.75, 2.5) {$A$};
		\node [style=none] (21) at (-1.75, 1) {};
		\node [style=none] (22) at (-0.25, 4) {};
	\end{pgfonlayer}
	\begin{pgfonlayer}{edgelayer}
		\draw [bend left] (1) to (7.center);
		\draw [bend left] (0) to (4.center);
		\draw (0) to (1);
		\draw [bend left] (21.center) to (0);
		\draw [bend right] (1) to (22.center);
	\end{pgfonlayer}
\end{tikzpicture} \]
The diagram on the right of the equation is fondly referred to as a `spider' in the CQM community. 

In an SMC, if an object $A$ is Frobenius algbera, then $A$ is a self-dual object with the following 
cup and cap: 
\[ \eta: I \to A \ox A := \begin{tikzpicture}
	\begin{pgfonlayer}{nodelayer}
		\node [style=circle] (0) at (0, 5) {};
		\node [style=circle] (1) at (0, 4) {};
		\node [style=none] (2) at (-0.75, 3) {};
		\node [style=none] (3) at (0.75, 3) {};
	\end{pgfonlayer}
	\begin{pgfonlayer}{edgelayer}
		\draw [bend left] (2.center) to (1);
		\draw [bend left] (1) to (3.center);
		\draw (1) to (0);
	\end{pgfonlayer}
\end{tikzpicture} ~~~~~~~~~~~~ \epsilon : A \ox A \to I :=  \begin{tikzpicture}
	\begin{pgfonlayer}{nodelayer}
		\node [style=circle] (0) at (0, 3) {};
		\node [style=circle] (1) at (0, 4) {};
		\node [style=none] (2) at (-0.75, 5) {};
		\node [style=none] (3) at (0.75, 5) {};
	\end{pgfonlayer}
	\begin{pgfonlayer}{edgelayer}
		\draw [bend right] (2.center) to (1);
		\draw [bend right] (1) to (3.center);
		\draw (1) to (0);
	\end{pgfonlayer}
\end{tikzpicture} \]

For a Frobenius algebra, $\Frob{A}$,  the monoid $\monoid{A}$, and the comonoid $\comonoid{A}$ are 
dual to one another by the means of the self-dual cup and cap: 
\[ \begin{tikzpicture}
	\begin{pgfonlayer}{nodelayer}
		\node [style=circle] (0) at (0, 4.75) {};
		\node [style=circle] (1) at (0, 4) {};
		\node [style=none] (2) at (-0.5, 3) {};
		\node [style=none] (3) at (0.5, 3) {};
		\node [style=circle] (4) at (0, 6.25) {};
		\node [style=circle] (5) at (0, 5.5) {};
		\node [style=none] (6) at (-1.25, 3) {};
		\node [style=none] (7) at (1.5, 3) {};
		\node [style=circle] (8) at (1, 2.25) {};
		\node [style=circle] (9) at (1, 1.5) {};
		\node [style=none] (10) at (-0.5, 1.5) {};
		\node [style=none] (11) at (-1.25, 1.5) {};
	\end{pgfonlayer}
	\begin{pgfonlayer}{edgelayer}
		\draw [bend left] (2.center) to (1);
		\draw [bend left] (1) to (3.center);
		\draw (1) to (0);
		\draw [in=210, out=90] (6.center) to (5);
		\draw [in=90, out=-30] (5) to (7.center);
		\draw (5) to (4);
		\draw [bend right] (3.center) to (8);
		\draw [bend left] (7.center) to (8);
		\draw (8) to (9);
		\draw (10.center) to (2.center);
		\draw (6.center) to (11.center);
	\end{pgfonlayer}
\end{tikzpicture} = \begin{tikzpicture}
	\begin{pgfonlayer}{nodelayer}
		\node [style=circle] (1) at (0, 4) {};
		\node [style=none] (2) at (-0.5, 3) {};
		\node [style=none] (3) at (0.5, 3) {};
		\node [style=none] (10) at (-0.5, 1.5) {};
		\node [style=none] (12) at (0, 6.25) {};
		\node [style=none] (13) at (0.5, 1.5) {};
	\end{pgfonlayer}
	\begin{pgfonlayer}{edgelayer}
		\draw [bend left] (2.center) to (1);
		\draw [bend left] (1) to (3.center);
		\draw (10.center) to (2.center);
		\draw (13.center) to (3.center);
		\draw (1) to (12.center);
	\end{pgfonlayer}
\end{tikzpicture} = \begin{tikzpicture}
	\begin{pgfonlayer}{nodelayer}
		\node [style=circle] (0) at (0.25, 4.75) {};
		\node [style=circle] (1) at (0.25, 4) {};
		\node [style=none] (2) at (0.75, 3) {};
		\node [style=none] (3) at (-0.25, 3) {};
		\node [style=circle] (4) at (0.25, 6.25) {};
		\node [style=circle] (5) at (0.25, 5.5) {};
		\node [style=none] (6) at (1.5, 3) {};
		\node [style=none] (7) at (-1.25, 3) {};
		\node [style=circle] (8) at (-0.75, 2.25) {};
		\node [style=circle] (9) at (-0.75, 1.5) {};
		\node [style=none] (10) at (0.75, 1.5) {};
		\node [style=none] (11) at (1.5, 1.5) {};
	\end{pgfonlayer}
	\begin{pgfonlayer}{edgelayer}
		\draw [bend right] (2.center) to (1);
		\draw [bend right] (1) to (3.center);
		\draw (1) to (0);
		\draw [in=-30, out=90] (6.center) to (5);
		\draw [in=90, out=-150] (5) to (7.center);
		\draw (5) to (4);
		\draw [bend left] (3.center) to (8);
		\draw [bend right] (7.center) to (8);
		\draw (8) to (9);
		\draw (10.center) to (2.center);
		\draw (6.center) to (11.center);
	\end{pgfonlayer}
\end{tikzpicture} ~~~~~~~~~~~~ \begin{tikzpicture}
	\begin{pgfonlayer}{nodelayer}
		\node [style=circle] (1) at (0, 4) {};
		\node [style=none] (2) at (-0.5, 3) {};
		\node [style=none] (3) at (0.5, 3) {};
		\node [style=none] (10) at (-0.5, 1.5) {};
		\node [style=circle] (14) at (0, 6.25) {};
		\node [style=circle] (15) at (0.5, 1.5) {};
	\end{pgfonlayer}
	\begin{pgfonlayer}{edgelayer}
		\draw [bend left] (2.center) to (1);
		\draw [bend left] (1) to (3.center);
		\draw (10.center) to (2.center);
		\draw (14) to (1);
		\draw (15) to (3.center);
	\end{pgfonlayer}
\end{tikzpicture}  = \begin{tikzpicture}
	\begin{pgfonlayer}{nodelayer}
		\node [style=none] (2) at (0.5, 3) {};
		\node [style=none] (10) at (0.5, 1.5) {};
		\node [style=circle] (14) at (0.5, 6.25) {};
	\end{pgfonlayer}
	\begin{pgfonlayer}{edgelayer}
		\draw (10.center) to (2.center);
		\draw (14) to (2.center);
	\end{pgfonlayer}
\end{tikzpicture}  = \begin{tikzpicture}
	\begin{pgfonlayer}{nodelayer}
		\node [style=circle] (1) at (0, 4) {};
		\node [style=none] (2) at (0.5, 3) {};
		\node [style=none] (3) at (-0.5, 3) {};
		\node [style=none] (10) at (0.5, 1.5) {};
		\node [style=circle] (14) at (0, 6.25) {};
		\node [style=circle] (15) at (-0.5, 1.5) {};
	\end{pgfonlayer}
	\begin{pgfonlayer}{edgelayer}
		\draw [bend right] (2.center) to (1);
		\draw [bend right] (1) to (3.center);
		\draw (10.center) to (2.center);
		\draw (14) to (1);
		\draw (15) to (3.center);
	\end{pgfonlayer}
\end{tikzpicture} \] 
Conversely, if $A$ is a self-dual object $A \dashvv A$ and a monoid $\monoid{A}$, and there exists a 
map $\counitmap{1.5}{white}: A \to I$ such that the self-dual cup is given by $
\begin{tikzpicture}[scale=0.5]
	\begin{pgfonlayer}{nodelayer}
		\node [style=none] (3) at (-0.25, 3) {};
		\node [style=none] (7) at (-1.25, 3) {};
		\node [style=circle] (8) at (-0.75, 2.25) {};
		\node [style=circle] (9) at (-0.75, 1.5) {};
	\end{pgfonlayer}
	\begin{pgfonlayer}{edgelayer}
		\draw [bend left] (3.center) to (8);
		\draw [bend right] (7.center) to (8);
		\draw (8) to (9);
	\end{pgfonlayer}
\end{tikzpicture} $, then $A$ has a Frobenius structure, see \cite[Proposition 5.16]{HeV19}. The map 
$\counitmap{1.5}{white}: A \to I$ is referred to the non-degenerate form of the Frobenius structure. 


\begin{definition}
	In a $\dagger$-SMC, a {\bf $\dagger$-Frobenius algebra} is a Frobenius algebra $\Frob{A}$ such that 
	\[ \left( ~ \begin{tikzpicture}
		\begin{pgfonlayer}{nodelayer}
			\node [style=none] (0) at (-1, 2) {};
			\node [style=circle] (1) at (-1, 3) {};
			\node [style=none] (7) at (-1.75, 4) {};
			\node [style=none] (20) at (-0.75, 2.5) {$A$};
			\node [style=none] (22) at (-0.25, 4) {};
		\end{pgfonlayer}
		\begin{pgfonlayer}{edgelayer}
			\draw [bend left] (1) to (7.center);
			\draw (0.center) to (1);
			\draw [bend right] (1) to (22.center);
		\end{pgfonlayer}
	\end{tikzpicture}
	 ~ \right)^\dag = \begin{tikzpicture}
		\begin{pgfonlayer}{nodelayer}
			\node [style=none] (0) at (-1, 4) {};
			\node [style=circle] (1) at (-1, 3) {};
			\node [style=none] (7) at (-1.75, 2) {};
			\node [style=none] (20) at (-0.75, 3.5) {$A$};
			\node [style=none] (22) at (-0.25, 2) {};
		\end{pgfonlayer}
		\begin{pgfonlayer}{edgelayer}
			\draw [bend right] (1) to (7.center);
			\draw (0.center) to (1);
			\draw [bend left] (1) to (22.center);
		\end{pgfonlayer}
	\end{tikzpicture}	 \]
\end{definition}



In $\dagger$-KCCs, {\em pants algebra} provide an important class of examples for $\dagger$-FAs. 
To define pants algebra, note that in a KCC, for each object $A$, there exists a monoid on 
$A \ox A^*$ with the following multiplication and unit respectively:
\[ \begin{tikzpicture}
	\begin{pgfonlayer}{nodelayer}
		\node [style=none] (0) at (0.25, 5) {};
		\node [style=none] (1) at (1.25, 5) {};
		\node [style=none] (2) at (0.25, 4.5) {};
		\node [style=none] (3) at (1.25, 4.5) {};
		\node [style=none] (4) at (-0.5, 5) {};
		\node [style=none] (5) at (2, 5) {};
		\node [style=none] (6) at (-0.5, 4.25) {};
		\node [style=none] (7) at (2, 4.25) {};
		\node [style=none] (8) at (2, 5.25) {$A^*$};
		\node [style=none] (9) at (0.25, 3.5) {};
		\node [style=none] (10) at (1.25, 3.5) {};
		\node [style=none] (11) at (1.25, 3) {};
		\node [style=none] (12) at (0.25, 3) {};
		\node [style=none] (13) at (-0.5, 5.25) {$A$};
		\node [style=none] (14) at (1.25, 5.25) {$A$};
		\node [style=none] (15) at (0.25, 5.25) {$A^*$};
	\end{pgfonlayer}
	\begin{pgfonlayer}{edgelayer}
		\draw [bend right=75, looseness=1.75] (2.center) to (3.center);
		\draw (5.center) to (7.center);
		\draw [in=270, out=90] (6.center) to (4.center);
		\draw [in=90, out=-90, looseness=1.25] (7.center) to (10.center);
		\draw (10.center) to (11.center);
		\draw [in=90, out=-90, looseness=1.25] (6.center) to (9.center);
		\draw (9.center) to (12.center);
		\draw (1.center) to (3.center);
		\draw (0.center) to (2.center);
	\end{pgfonlayer}
\end{tikzpicture} ~~~~~~~~ \begin{tikzpicture}
	\begin{pgfonlayer}{nodelayer}
		\node [style=none] (0) at (0, 5.25) {};
		\node [style=none] (1) at (1.5, 5.25) {};
		\node [style=none] (14) at (0, 4) {$A$};
		\node [style=none] (15) at (1.5, 4) {$A^*$};
		\node [style=none] (16) at (1.5, 4.25) {};
		\node [style=none] (17) at (0, 4.25) {};
	\end{pgfonlayer}
	\begin{pgfonlayer}{edgelayer}
		\draw (0.center) to (17.center);
		\draw (1.center) to (16.center);
		\draw [bend left=90, looseness=2.00] (0.center) to (1.center);
	\end{pgfonlayer}
\end{tikzpicture} \]
The above algebra is referred to as a {\bf pants algebra} due to the shape of its multiplication map. 

Every monoid in a $KCC$ embeds into the pants monoid:
\begin{lemma}
\label{Lemma: monoidal pants embedding}
In a KCC, every monoid, $(A, \mulmap{1.5}{white}, \unitmap{1.5}{white})$, embeds into its 
pants monoid on $A^* \ox A$ by means of the following monoid morphism:
\[ \begin{tikzpicture}
	\begin{pgfonlayer}{nodelayer}
		\node [style=none] (0) at (0, 3) {};
		\node [style=circle] (1) at (0, 4) {};
		\node [style=none] (2) at (-0.75, 5) {};
		\node [style=none] (3) at (0.75, 5) {};
		\node [style=none] (4) at (1.5, 5) {};
		\node [style=none] (5) at (1.5, 3) {};
		\node [style=none] (6) at (2, 3.25) {$A^*$};
		\node [style=none] (7) at (-0.25, 3.25) {$A$};
		\node [style=none] (8) at (-0.75, 5.75) {};
	\end{pgfonlayer}
	\begin{pgfonlayer}{edgelayer}
		\draw [bend right] (2.center) to (1);
		\draw [bend right] (1) to (3.center);
		\draw (1) to (0.center);
		\draw [bend left=90, looseness=1.75] (3.center) to (4.center);
		\draw (4.center) to (5.center);
		\draw (8.center) to (2.center);
	\end{pgfonlayer}
\end{tikzpicture}  \]  
\end{lemma}

\begin{lemma}
In a $\dagger$-KCC, pants algebra are $\dagger$-Frobenius.
\end{lemma}

Applying the operator-state duality to pants $\dagger$-algebra we note the following. Let $f, g: A \to A$ 
be any two processes in a $\dagger$-KCC. Multliplying the states corresponding to these processes 
using the pants monoid gives the state corresponding to composition $fg$, see equation $(a)$. Moreover, 
applying the counit to the state of $f$ gives the trace of $f$, see diagram $(b)$.
\[ (a) ~~~ \begin{tikzpicture}
	\begin{pgfonlayer}{nodelayer}
		\node [style=twocircle] (0) at (-1, 2) {};
		\node [style=none] (2) at (-1, 2.5) {};
		\node [style=none] (3) at (-1, 1.5) {};
		\node [style=none] (4) at (0, 1.5) {};
		\node [style=none] (5) at (0, 2.5) {};
		\node [style=twocircle] (6) at (1, 2) {};
		\node [style=none] (7) at (1, 2.5) {};
		\node [style=none] (8) at (1, 1.5) {};
		\node [style=none] (9) at (2, 1.5) {};
		\node [style=none] (10) at (2, 2.5) {};
		\node [style=none] (11) at (-1, 2) {$f$};
		\node [style=none] (12) at (1, 2) {$g$};
		\node [style=none] (13) at (0, -0.25) {};
		\node [style=none] (14) at (1, -0.25) {};
		\node [style=none] (17) at (-1.5, 2.5) {$A$};
		\node [style=none] (18) at (-0.5, 0) {$A$};
		\node [style=none] (19) at (2.5, 2.5) {$A^*$};
		\node [style=none] (20) at (1.5, 0) {$A^*$};
	\end{pgfonlayer}
	\begin{pgfonlayer}{edgelayer}
		\draw [bend left=75, looseness=1.25] (2.center) to (5.center);
		\draw (5.center) to (4.center);
		\draw (3.center) to (0);
		\draw (0) to (2.center);
		\draw [bend left=75, looseness=1.25] (7.center) to (10.center);
		\draw (10.center) to (9.center);
		\draw (8.center) to (6);
		\draw (6) to (7.center);
		\draw [in=90, out=-90] (3.center) to (13.center);
		\draw [in=-90, out=90] (14.center) to (9.center);
		\draw [bend right=90, looseness=1.50] (4.center) to (8.center);
	\end{pgfonlayer}
\end{tikzpicture} = \begin{tikzpicture}
	\begin{pgfonlayer}{nodelayer}
		\node [style=twocircle] (0) at (-1, 2) {};
		\node [style=none] (2) at (-1, 2.5) {};
		\node [style=none] (4) at (0, 1.5) {};
		\node [style=none] (5) at (0, 2.5) {};
		\node [style=twocircle] (6) at (-1, 1) {};
		\node [style=none] (11) at (-1, 2) {$f$};
		\node [style=none] (12) at (-1, 1) {$g$};
		\node [style=none] (13) at (-1, -0.25) {};
		\node [style=none] (17) at (-1.5, 2.5) {$A$};
		\node [style=none] (18) at (-1.25, 0) {$A$};
		\node [style=none] (19) at (0, -0.25) {};
		\node [style=none] (20) at (0.5, 0) {$A^*$};
	\end{pgfonlayer}
	\begin{pgfonlayer}{edgelayer}
		\draw [bend left=75, looseness=1.25] (2.center) to (5.center);
		\draw (5.center) to (4.center);
		\draw (0) to (2.center);
		\draw (0) to (6);
		\draw (6) to (13.center);
		\draw (4.center) to (19.center);
	\end{pgfonlayer}
\end{tikzpicture} ~~~~~~~~ (b)~~~ \begin{tikzpicture}
	\begin{pgfonlayer}{nodelayer}
		\node [style=twocircle] (0) at (-1, 2) {};
		\node [style=none] (2) at (-1, 2.5) {};
		\node [style=none] (4) at (0, 1.25) {};
		\node [style=none] (5) at (0, 2.5) {};
		\node [style=none] (11) at (-1, 2) {$f$};
		\node [style=none] (13) at (-1, 1.25) {};
		\node [style=none] (17) at (-1.5, 2.5) {$A$};
		\node [style=none] (20) at (0.5, 0.25) {$A^*$};
		\node [style=none] (21) at (-0.25, 0.25) {};
		\node [style=none] (22) at (-0.75, 0.25) {};
	\end{pgfonlayer}
	\begin{pgfonlayer}{edgelayer}
		\draw [bend left=75, looseness=1.25] (2.center) to (5.center);
		\draw (5.center) to (4.center);
		\draw (0) to (2.center);
		\draw (0) to (13.center);
		\draw [in=-90, out=90] (22.center) to (4.center);
		\draw [in=90, out=-90] (13.center) to (21.center);
		\draw [bend right=75, looseness=1.50] (22.center) to (21.center);
	\end{pgfonlayer}
\end{tikzpicture} \] 

Let us review a concrete example of pants algebra. Consider
the space of $n \times n$ complex matrices written as $M_n$ 
which gives the pants algebra over the space $\C^n$ in $\FHilb$
 \cite[Example 4.12]{HeV19}. We will show that $M_n$ is the pants algebra for $\C^n$.
The space of $n \times n$ complex matrices is a Hilbert space 
with the inner product given by $\langle A | B \rangle := Tr(A^\dagger B)$ ($A^\dagger$ is 
conjugate transpose of $A$) and comes with a canonical basis $\{ e_{ij} | i,j=1,\cdots,n \}$. 
The basis $e_{ij}$ is a $n \times n$ matrix with zero for all entries except the entry $(i,j)$. 

The algebra of $n \times n$ complex matrices has a $\dagger$ Frobenius structure in $\FHilb$:

{\bf Monoid structure:} The multiplication, $m$, is given by matrix multiplication 
and the unit, $u$, is the $n \times n$ identity matrix.

{\bf Comonoid structure:} Define the counit map $e := u^\dagger$. Then, by definition of $\dagger$ for 
in Hilbert spaces, $ \langle e(e_{ij}) | 1 \rangle = \langle (e_{ij}) | e^\dagger(1) \rangle = \langle e_{ij} 
| u (1) \rangle = \langle e_{ij} | I_A \rangle = \delta_{ij} = Tr(e_{ij})$. By extension of linearity, for 
all $A \in M_n$, $e(A) = Tr(A)$.

Similarly, define the comultiplication to be, $d := m^\dagger$. 
By the definition of $\dagger$ for $\mathsf{FHilb}$, $\langle m(e_{ij} \otimes e_{kl}) | 
e_{pq} \rangle = \langle e_{ij} \otimes e_{kl} | m^\dagger(e_{pq}) \rangle$. 
As before, in order to derive the definition of $m^\dagger$ we expand the equation on both sides.

\begin{align*}
 \langle m^\dagger(e_{ij}) | e_{kl} \otimes e_{pq} \rangle &= \langle e_{ij} | m(e_{kl} \otimes e_{pq}) \rangle \\
&= \langle e_{ij} | \delta_{lp} e_{kq} \rangle \\
&= \delta_{lp} Tr(e_{ij}^* e_{kq}) \\
&= \delta_{lp} Tr(e_{ij}^* e_kq) \\
&=\delta_{lp} \delta_{ik} \delta_{jq}. 
\end{align*}
By defining $m^\dagger(e_{ij})  := \sum_l e_{il} \otimes e_{lj}$, $\langle m^\dagger(e_{ij}) | e_{kl} \otimes e_{pq} \rangle$ evaluates to $\delta_{lp} \delta_{ik} \delta_{jq}$.

A $\dagger$-Frobenius algebra in a symmetric $\dagger$-SMC is said to be {\bf special} if $(a)$ holds, 
{\bf commutative} if $(b)$ holds, and {\bf symmetric} if $(c)$ holds:
\[ (a)~~~ \begin{tikzpicture}
	\begin{pgfonlayer}{nodelayer}
		\node [style=circle] (0) at (0, 4) {};
		\node [style=circle] (1) at (0, 3) {};
		\node [style=none] (2) at (0, 2.25) {};
		\node [style=none] (3) at (0, 4.75) {};
	\end{pgfonlayer}
	\begin{pgfonlayer}{edgelayer}
		\draw [bend left=60, looseness=1.25] (0) to (1);
		\draw [bend right=60, looseness=1.25] (0) to (1);
		\draw (1) to (2.center);
		\draw (0) to (3.center);
	\end{pgfonlayer}
\end{tikzpicture} = \begin{tikzpicture}
	\begin{pgfonlayer}{nodelayer}
		\node [style=none] (4) at (1, 4.75) {};
		\node [style=none] (5) at (1, 2.25) {};
	\end{pgfonlayer}
	\begin{pgfonlayer}{edgelayer}
		\draw (4.center) to (5.center);
	\end{pgfonlayer}
\end{tikzpicture}  ~~~~~~~~ (b) ~~~ \begin{tikzpicture}
	\begin{pgfonlayer}{nodelayer}
		\node [style=circle] (0) at (0.25, 4) {};
		\node [style=none] (1) at (0.25, 3.25) {};
		\node [style=none] (2) at (0.75, 5.5) {};
		\node [style=none] (3) at (-0.25, 5.5) {};
	\end{pgfonlayer}
	\begin{pgfonlayer}{edgelayer}
		\draw [in=45, out=-90, looseness=1.75] (3.center) to (0);
		\draw [in=135, out=-90, looseness=1.75] (2.center) to (0);
		\draw (0) to (1.center);
	\end{pgfonlayer}
\end{tikzpicture} = \begin{tikzpicture}
	\begin{pgfonlayer}{nodelayer}
		\node [style=circle] (0) at (0.25, 4.25) {};
		\node [style=none] (1) at (0.25, 3.25) {};
		\node [style=none] (2) at (-0.5, 5.5) {};
		\node [style=none] (3) at (1, 5.5) {};
	\end{pgfonlayer}
	\begin{pgfonlayer}{edgelayer}
		\draw [bend left] (3.center) to (0);
		\draw [bend right] (2.center) to (0);
		\draw (0) to (1.center);
	\end{pgfonlayer}
\end{tikzpicture}  ~~~~~~~~ (c) ~~~ \begin{tikzpicture}
	\begin{pgfonlayer}{nodelayer}
		\node [style=circle] (0) at (0.25, 4) {};
		\node [style=circle] (1) at (0.25, 3.25) {};
		\node [style=none] (2) at (0.75, 5.5) {};
		\node [style=none] (3) at (-0.25, 5.5) {};
	\end{pgfonlayer}
	\begin{pgfonlayer}{edgelayer}
		\draw [in=45, out=-90, looseness=1.75] (3.center) to (0);
		\draw [in=135, out=-90, looseness=1.75] (2.center) to (0);
		\draw (0) to (1);
	\end{pgfonlayer}
\end{tikzpicture} = \begin{tikzpicture}
	\begin{pgfonlayer}{nodelayer}
		\node [style=circle] (0) at (0.25, 4.25) {};
		\node [style=circle] (1) at (0.25, 3.25) {};
		\node [style=none] (2) at (-0.5, 5.5) {};
		\node [style=none] (3) at (1, 5.5) {};
	\end{pgfonlayer}
	\begin{pgfonlayer}{edgelayer}
		\draw [bend left] (3.center) to (0);
		\draw [bend right] (2.center) to (0);
		\draw (0) to (1);
	\end{pgfonlayer}
\end{tikzpicture} \] 

Commutativity is  a stronger condition than the symmetry. For example, the pants algebra 
is non-commutative but symmetric. 

The connection between Frobenius algebras and quantum observables given by the 
fact that in the $\FHilb$ category, every special commutative $\dagger$-FA ($\dagger$-SCFA) precisely 
corresponds to an orthonormal basis. This correspondence arises from the 
notion of {\bf classical states} for a $\dagger$-FA: the states, $a: I \to A$, which can be copied, and 
deleted as shown below. 
\[ \text{copy:}~~~\begin{tikzpicture}
	\begin{pgfonlayer}{nodelayer}
		\node [style=none] (0) at (0, 5.25) {};
		\node [style=none] (1) at (-0.5, 4.5) {};
		\node [style=none] (2) at (0.5, 4.5) {};
		\node [style=circle] (3) at (0, 3.75) {};
		\node [style=none] (4) at (-0.5, 2.75) {};
		\node [style=none] (5) at (0.5, 2.75) {};
		\node [style=none] (6) at (0, 4.5) {};
		\node [style=none] (7) at (0, 4.75) {$a$};
	\end{pgfonlayer}
	\begin{pgfonlayer}{edgelayer}
		\draw (1.center) to (2.center);
		\draw (2.center) to (0.center);
		\draw (0.center) to (1.center);
		\draw [bend right] (3) to (4.center);
		\draw [bend left] (3) to (5.center);
		\draw (6.center) to (3);
	\end{pgfonlayer}
\end{tikzpicture} = \begin{tikzpicture}
	\begin{pgfonlayer}{nodelayer}
		\node [style=none] (0) at (-0.5, 5.25) {};
		\node [style=none] (1) at (-1, 4.5) {};
		\node [style=none] (2) at (0, 4.5) {};
		\node [style=none] (4) at (-0.5, 2.75) {};
		\node [style=none] (6) at (-0.5, 4.5) {};
		\node [style=none] (7) at (-0.5, 4.75) {$a$};
		\node [style=none] (8) at (0.75, 5.25) {};
		\node [style=none] (9) at (0.25, 4.5) {};
		\node [style=none] (10) at (1.25, 4.5) {};
		\node [style=none] (11) at (0.75, 2.75) {};
		\node [style=none] (12) at (0.75, 4.5) {};
		\node [style=none] (13) at (0.75, 4.75) {$a$};
	\end{pgfonlayer}
	\begin{pgfonlayer}{edgelayer}
		\draw (1.center) to (2.center);
		\draw (2.center) to (0.center);
		\draw (0.center) to (1.center);
		\draw (6.center) to (4.center);
		\draw (9.center) to (10.center);
		\draw (10.center) to (8.center);
		\draw (8.center) to (9.center);
		\draw (12.center) to (11.center);
	\end{pgfonlayer}
\end{tikzpicture}
~~~~~~~~~~~ \text{delete:}~~~\begin{tikzpicture}
	\begin{pgfonlayer}{nodelayer}
		\node [style=none] (0) at (0, 5.25) {};
		\node [style=none] (1) at (-0.5, 4.5) {};
		\node [style=none] (2) at (0.5, 4.5) {};
		\node [style=circle] (3) at (0, 3) {};
		\node [style=none] (6) at (0, 4.5) {};
		\node [style=none] (7) at (0, 4.75) {$a$};
	\end{pgfonlayer}
	\begin{pgfonlayer}{edgelayer}
		\draw (1.center) to (2.center);
		\draw (2.center) to (0.center);
		\draw (0.center) to (1.center);
		\draw (6.center) to (3);
	\end{pgfonlayer}
\end{tikzpicture} = id_I\] 

\begin{theorem}
	\label{Theorem: scfa}
	\cite[Theorem 5.1]{CPV12}
In $\FHilb$, the set of classical states for a $\dagger$-SCFA on an Hilbert space $H$ 
precisely corresponds to an orthonormal basis for $H$.
\end{theorem}
Hence, there exists a bijective correspondence between $\dagger$-SCFAs and 
orthonormal bases. The basis states are the only 
states of the quantum system that can be copied and deleted, in other words, 
a classical state. Hence, $\dagger$-SCFA are also referred to as {\em classical structures}. 
In the previous theorem if we drop the keyword special, then corresponding basis is orthogonal. 

A quantum measurement is the process of extracting classical data 
(can be copied and deleted) from a quantum state. 
Categorically, Coecke and Pavlovic \cite{CoP07} described a ``demolition'' measurement in a 
$\dagger$-monoidal category as a process, $m: A \to X$, with $m^\dagger m = 1_X$, to a 
special commutative $\dagger$-Frobenius algebra, $X$. The object $A$ refers to the 
state space of a quantum system. Demolition means that we ignore the resulting state of the 
quantum system after measurement and preserve only the classical data. 

\iffalse
Selinger classified classical and quantum types in $\dagger$-monoidal categories using  
$\dagger$-idempotents \cite{Sel08}. In $\dagger$-monoidal categories, a $\dagger$-idempotent is 
an idempotent $e: A \to A$ such that $e = e^\dagger$. A $\dagger$-idempotent, 
$e$, $\dagger$-splits if the idempotent splits as follows: $A \to^{f} E \to^{f^\dag} A$. 
An object $A$ with a $\dagger$-idempotent is  said to be of {\em quantum type}. 
Classical types are given by  $\dagger$-splitting objects of quantum type. 
Note that Selinger's $\dagger$-idempotents are precisely demolition measurements 
when the split object $E$ is a $\dagger$-SCFA. 
%https://arxiv.org/pdf/1308.4557.pdf

First he notices that, given any category $\C$, the category CPM$(\C)$ contains only
quantum types: any simple type $A$ and tensor product of simple types. The category does 
contain classical types, that is, objects which are direct sums of quantum types $A \oa A$ (copies of $A$). 
In other words, the CPM construction does not preserve biproducts. In order to include 
classical types in CPM$(\C)$ by taking its biproduct completion: in the resulting category 
objects are sequences of types $(A_1, A_2, \cdots, A_n)$ which refers to the biproduct of 
$A_1 \cdots A_n \in$ CPM$(\C)$. 

Another way to include classical types is by splitting the $\dagger$-idempotents in CPM$(\C)$.
\fi

In Theorem \ref{Theorem: scfa}, we saw that $\dagger$-SCFA model classical data. 
However, Frobenius algebras which are non-commutative but symmetric model quantum information:  
\begin{theorem}\cite{Vic10}
In $\FHilb$, every special symmetric $\dagger$-FA precisely corresponds to a 
$\C^*$-algebra.
\end{theorem}
The proof of the above theorem relies on Lemma \ref{Lemma: monoidal pants embedding}.
Due to their correspondence to $\C^*$-algebras, 
special symmetric $\dagger$-FAs are also referred to as {\em quantum algebras}. Note that 
quantum algebras are non-commutative. 

\subsection{Strong complementarity}
\label{Sec: strong comp}

Bohr's complementarity \cite{Gri18} is a key feature that distinguishes quantum mechanics from classical mechanics. 
Two quantum observables are said to be complementary if measuring one observable leads to 
maximum uncertainty regarding the value of the other. An example of complementary observables is 
position and momentum of an electron. All physical properties occur in complementary pairs due to the 
wave and the particle nature of matter. 

In CQM, complementary interaction of two observables  are axiomatized using Hopf algebras \cite{Abe04}, 
which are bialgebras with an antipode: 

\begin{definition}\cite{Han08}
In a SMC, a {\bf bialgebra} $\bialg{A}$ consists of a monoid $\monoid{A}$ and a comonoid $\comonoidb{A}$ 
satisfying the following equations: 
\[ (a)~~~  \begin{tikzpicture}[yscale=-1]
	\begin{pgfonlayer}{nodelayer}
		\node [style=circle] (0) at (0, 4) {};
		\node [style=black] (1) at (0, 5) {};
		\node [style=none] (2) at (-0.5, 3) {};
		\node [style=none] (3) at (0.5, 3) {};
	\end{pgfonlayer}
	\begin{pgfonlayer}{edgelayer}
		\draw [bend right, looseness=1.25] (0) to (2.center);
		\draw [bend left, looseness=1.25] (0) to (3.center);
		\draw (1) to (0);
	\end{pgfonlayer}
\end{tikzpicture} = \begin{tikzpicture}[yscale=-1]
	\begin{pgfonlayer}{nodelayer}
		\node [style=black] (1) at (0, 5) {};
		\node [style=none] (3) at (0, 3) {};
		\node [style=black] (4) at (0.75, 5) {};
		\node [style=none] (5) at (0.75, 3) {};
	\end{pgfonlayer}
	\begin{pgfonlayer}{edgelayer}
		\draw (3.center) to (1);
		\draw (5.center) to (4);
	\end{pgfonlayer}
\end{tikzpicture} ~~~~~~~~~~~~ (b)~~~ \begin{tikzpicture}
	\begin{pgfonlayer}{nodelayer}
		\node [style=black] (0) at (0, 4) {};
		\node [style=circle] (1) at (0, 5) {};
		\node [style=none] (2) at (-0.5, 3) {};
		\node [style=none] (3) at (0.5, 3) {};
	\end{pgfonlayer}
	\begin{pgfonlayer}{edgelayer}
		\draw [bend right, looseness=1.25] (0) to (2.center);
		\draw [bend left, looseness=1.25] (0) to (3.center);
		\draw (1) to (0);
	\end{pgfonlayer}
\end{tikzpicture} = \begin{tikzpicture}
	\begin{pgfonlayer}{nodelayer}
		\node [style=circle] (1) at (0, 5) {};
		\node [style=none] (3) at (0, 3) {};
		\node [style=circle] (4) at (0.75, 5) {};
		\node [style=none] (5) at (0.75, 3) {};
	\end{pgfonlayer}
	\begin{pgfonlayer}{edgelayer}
		\draw (3.center) to (1);
		\draw (5.center) to (4);
	\end{pgfonlayer}
\end{tikzpicture} ~~~~~~~~~~~ (c)~~~ \begin{tikzpicture}
	\begin{pgfonlayer}{nodelayer}
		\node [style=circle] (1) at (0, 5) {};
		\node [style=black] (2) at (0, 3) {};
	\end{pgfonlayer}
	\begin{pgfonlayer}{edgelayer}
		\draw (1) to (2);
	\end{pgfonlayer}
\end{tikzpicture} = id_I  ~~~~~~~~~~~~ (d)~~~ \begin{tikzpicture}
	\begin{pgfonlayer}{nodelayer}
		\node [style=black] (0) at (0, 5.25) {};
		\node [style=black] (1) at (1.25, 5.25) {};
		\node [style=circle] (2) at (0, 3.75) {};
		\node [style=circle] (3) at (1.25, 3.75) {};
		\node [style=none] (4) at (0, 3) {};
		\node [style=none] (5) at (1.25, 3) {};
		\node [style=none] (6) at (0, 6) {};
		\node [style=none] (7) at (1.25, 6) {};
		\node [style=none] (8) at (1.25, 3) {};
	\end{pgfonlayer}
	\begin{pgfonlayer}{edgelayer}
		\draw (1) to (2);
		\draw (0) to (3);
		\draw [bend right=45] (0) to (2);
		\draw [bend left=45] (1) to (3);
		\draw (3) to (8.center);
		\draw (2) to (4.center);
		\draw (6.center) to (0);
		\draw (7.center) to (1);
	\end{pgfonlayer}
\end{tikzpicture} = \begin{tikzpicture}
	\begin{pgfonlayer}{nodelayer}
		\node [style=none] (4) at (0, 3) {};
		\node [style=none] (6) at (0, 6) {};
		\node [style=none] (7) at (1.5, 6) {};
		\node [style=none] (8) at (1.5, 3) {};
		\node [style=black] (9) at (0.75, 4) {};
		\node [style=circle] (10) at (0.75, 5.25) {};
	\end{pgfonlayer}
	\begin{pgfonlayer}{edgelayer}
		\draw [in=90, out=-165] (9) to (4.center);
		\draw [in=90, out=-15] (9) to (8.center);
		\draw (9) to (10);
		\draw [in=255, out=15] (10) to (7.center);
		\draw [in=285, out=165] (10) to (6.center);
	\end{pgfonlayer}
\end{tikzpicture} \] 
\end{definition}


The final equation is often referred to as the {\em bialgebra rule}. 
Observing equations $(a)$-$(d)$, we note that the multiplication map acts as a comonoid morphism for the 
tensor comonoid $A \ox A$. Equivalently, the comultiplication acts as a monoid morphism for the 
tensor monoid $A \ox A$. The following lemma gives alternate descriptions for a bialgebra:
\begin{lemma} In an SMC, the following are equivalent:
\begin{enumerate}[(i)]
\item  $\bialg{A}$ is a bialgebra.
\item $\monoid{A}$ is a monoid and $\comonoidb{A}$ is a comonoid such that $\comulmap{1.5}{black}$ is a monoid morphism 
for the tensor monoid $(A\ox A, \twinmul{1}{white}, \twinunit{1}{white})$.
\item $\monoid{A}$ is a monoid and $\comonoidb{A}$ is a comonoid such that $\mulmap{1.5}{white}$ 
is a comonoid morphism for the tensor comonoid $(A\ox A, \twincomul{1}{black}, \twincounit{1}{black})$.
\end{enumerate}
\end{lemma}

The following are a few examples of bialgebras:
\begin{itemize}
\item In a category with biproducts, every object has a bialgebra structure given by the product and the coproduct maps:
\[ A + A \to^{\begin{bmatrix} id_A \\ id_A \end{bmatrix}} A ~~~~~~~~ 0 \to^{0_{0,A}} A 
~~~~~~~~ A \to^{\left< id_A, id_A \right>} A + A~~~~~~~~ A \to^{0_A,0} 0 \]
The category of vector spaces and linear maps has biproducts: `+' is given by direct sum and $0$ is the trivial vector space.
\item Consider the category of sets and functions, ${\sf Set}$. 
The category ${\sf Set}$ does not have biproducts, however, 
it is an SMC with the tensor product given by the cartesian product and the 
unit object being a chosen singleton set. In ${\sf Set}$, every monoid $(A, \circ , u)$ (that is, 
$A$ is a set with a binary operation $\circ$ and a unit $u$),  
is a bialgebra with the comonoid structure given 
as follows:
\[ \text{ for all } a \in A, ~ \comulmap{1.5}{black}: A \to A \times A;  a \mapsto (a,a) 
~~~~~~~~  \counitmap{1.5}{black}: A \to \{*\} ; a \mapsto * \]
\end{itemize}

We are now ready to define Hopf algebras:
\begin{definition} \cite{Blu96, BlS04}
In a SMC, a {\bf Hopf algebra} is a bialgbera $\bialg{A}$ with a map $s: A \to A$ referred to as an {\bf antipode} such that:
\[ \begin{tikzpicture}
	\begin{pgfonlayer}{nodelayer}
		\node [style=black] (0) at (0, 3.75) {};
		\node [style=circle] (1) at (0, 1.75) {};
		\node [style=none] (2) at (0, 4.5) {};
		\node [style=none] (3) at (0, 1) {};
		\node [style=onehalfcircle] (4) at (-0.5, 2.75) {};
		\node [style=none] (5) at (-0.5, 2.75) {$s$};
	\end{pgfonlayer}
	\begin{pgfonlayer}{edgelayer}
		\draw (1) to (3.center);
		\draw (2.center) to (0);
		\draw [bend right] (0) to (4);
		\draw [bend right] (4) to (1);
		\draw [bend right=60] (1) to (0);
	\end{pgfonlayer}
\end{tikzpicture} = \begin{tikzpicture}
	\begin{pgfonlayer}{nodelayer}
		\node [style=black] (0) at (-0.5, 3.25) {};
		\node [style=circle] (1) at (-0.5, 2.25) {};
		\node [style=none] (2) at (-0.5, 4.5) {};
		\node [style=none] (3) at (-0.5, 1) {};
	\end{pgfonlayer}
	\begin{pgfonlayer}{edgelayer}
		\draw (1) to (3.center);
		\draw (2.center) to (0);
	\end{pgfonlayer}
\end{tikzpicture} = \begin{tikzpicture}
	\begin{pgfonlayer}{nodelayer}
		\node [style=black] (0) at (-0.5, 3.75) {};
		\node [style=circle] (1) at (-0.5, 1.75) {};
		\node [style=none] (2) at (-0.5, 4.5) {};
		\node [style=none] (3) at (-0.5, 1) {};
		\node [style=onehalfcircle] (4) at (0, 2.75) {};
		\node [style=none] (5) at (0, 2.75) {$s$};
	\end{pgfonlayer}
	\begin{pgfonlayer}{edgelayer}
		\draw (1) to (3.center);
		\draw (2.center) to (0);
		\draw [bend left] (0) to (4);
		\draw [bend left] (4) to (1);
		\draw [bend left=60] (1) to (0);
	\end{pgfonlayer}
\end{tikzpicture} \]
\end{definition}
The above equation is referred to as the {\em Hopf law}. 
The following are a few examples of Hopf algberas:

\begin{itemize}
\item  In the examples of bialgebras, we saw that every monoid $(A, \circ, e)$ in the category ${\sf Set}$ 
is a bialgbera. If this monoid is also a group, that is, each element of $A$ is equipped with an inverse, 
then the bialgebra is Hopf with anitpode $s: A \to A$ defined to be, for all $a \in A$, $s(a) := a^{-1}$.  

\item Another example of a Hopf algbera is a group $K$-algebra \cite{Maj00}. Given a  
finite group $(G, \circ, u)$,  (that is, $G$ is a set with a binary operation $\circ$, 
a unit $u$ and an inverse for each element) and a field $K$, it is 
the free  $K$-vector space, $K[G]$, over $G$ with basis $\{e_g\}_{g \in G}$ .  
$K[G]$ is a Hopf algebra with the following $K$-linear maps:  for all $g,h \in G$, 
\[ m: K[G] \ox K[G] \to K[G];  e_g \ox e_h \mapsto e_{g \circ h}
~~~~~~~~ u: K \to K[G]; 1 \mapsto e_u \] 
\[ d: K[G] \to K[G] \ox K[G]; e_g \mapsto e_g \ox e_g 
~~~~~~~~ e: K[G] \to K ; e_g \mapsto 1 \]
Finally, the antipode is:
\[ s: K[G] \to K[G]; e_g \mapsto e_{g^{-1}} \]

\end{itemize}

\iffalse
\[ d: V \to  V \ox V ; g \mapsto \sum_{h \in G} (h \ox  (h^{-1} \circ g))
~~~~~~~~ e: H \to K; g \mapsto 
  \begin{cases} 
	1, &  \text{ for } g = u_G \\ 
	0, & \text{ otherwise }
\end{cases}  \]
\fi

The bialgebras and the Frobenius algberas capture interactions between 
a monoid and comonoid on a single underlying object. However, the 
axioms for a bialgbera are characterized by disconnected diagrams while the 
axioms for a Frobenius algebra are characterized by connected diagrams. 
The Hopf law depicts the maximum disconnect and can be interpreted as 
as a process in which there is no information flow from the 
multiplication to the comultiplication. This makes Hopf algberas 
an appealing algebraic structure to model complementarity of 
two quantum observables \cite{CoD11}. 

It is worthwhile to note that an object, $A$, which is both a Frobenius algbera and 
a bialgebra is trivial i.e., $A \simeq I$ :
\begin{lemma} \cite[Theorem 6.23]{HeV19} In a $\dagger$-SMC, suppose 
a monoid $\monoid{A}$ and a comonoid $\comonoidb{A}$ is both a 
bialgbera and Frobenius algebra, then:  
\[ \begin{tikzpicture}
	\begin{pgfonlayer}{nodelayer}
		\node [style=black] (1) at (0.25, 4) {};
		\node [style=none] (2) at (0.25, 5) {};
		\node [style=circle] (3) at (0.25, 3) {};
		\node [style=none] (4) at (0.25, 2) {};
	\end{pgfonlayer}
	\begin{pgfonlayer}{edgelayer}
		\draw (2.center) to (1);
		\draw (4.center) to (3);
	\end{pgfonlayer}
\end{tikzpicture} = \begin{tikzpicture}
	\begin{pgfonlayer}{nodelayer}
		\node [style=none] (2) at (0.25, 2) {};
		\node [style=none] (4) at (0.25, 5) {};
	\end{pgfonlayer}
	\begin{pgfonlayer}{edgelayer}
		\draw (2.center) to (4.center);
	\end{pgfonlayer}
\end{tikzpicture}  \] 
\end{lemma}
\begin{proof}
\[ \begin{tikzpicture}
	\begin{pgfonlayer}{nodelayer}
		\node [style=black] (1) at (0.25, 4) {};
		\node [style=none] (2) at (0.25, 5) {};
		\node [style=circle] (3) at (0.25, 3) {};
		\node [style=none] (4) at (0.25, 2) {};
	\end{pgfonlayer}
	\begin{pgfonlayer}{edgelayer}
		\draw (2.center) to (1);
		\draw (4.center) to (3);
	\end{pgfonlayer}
\end{tikzpicture} \stackrel{(1)}{=}  \begin{tikzpicture}
	\begin{pgfonlayer}{nodelayer}
		\node [style=circle] (0) at (-0.75, 5) {};
		\node [style=none] (1) at (-0.75, 3) {};
		\node [style=circle] (2) at (0, 5) {};
		\node [style=black] (3) at (0.5, 3) {};
		\node [style=circle] (4) at (0.5, 4) {};
		\node [style=none] (5) at (1, 5) {};
	\end{pgfonlayer}
	\begin{pgfonlayer}{edgelayer}
		\draw (0) to (1.center);
		\draw [bend left] (5.center) to (4);
		\draw [bend right] (2) to (4);
		\draw (4) to (3);
	\end{pgfonlayer}
\end{tikzpicture} \stackrel{(2)}{=}\begin{tikzpicture}
	\begin{pgfonlayer}{nodelayer}
		\node [style=circle] (0) at (-0.5, 5.25) {};
		\node [style=none] (1) at (-1, 3) {};
		\node [style=black] (3) at (0.25, 3) {};
		\node [style=circle] (4) at (0.25, 4) {};
		\node [style=none] (5) at (0.75, 5.25) {};
		\node [style=black] (6) at (-0.5, 4.5) {};
	\end{pgfonlayer}
	\begin{pgfonlayer}{edgelayer}
		\draw [in=45, out=-90] (5.center) to (4);
		\draw (4) to (3);
		\draw (6) to (4);
		\draw [in=90, out=-135, looseness=1.25] (6) to (1.center);
		\draw (0) to (6);
	\end{pgfonlayer}
\end{tikzpicture} \stackrel{(3)}{=} \begin{tikzpicture}
	\begin{pgfonlayer}{nodelayer}
		\node [style=circle] (0) at (-0.25, 5.25) {};
		\node [style=none] (1) at (-0.25, 3) {};
		\node [style=black] (3) at (0.75, 3) {};
		\node [style=circle] (4) at (0.25, 4.5) {};
		\node [style=none] (5) at (0.75, 5.25) {};
		\node [style=black] (6) at (0.25, 3.75) {};
	\end{pgfonlayer}
	\begin{pgfonlayer}{edgelayer}
		\draw [in=30, out=-90] (5.center) to (4);
		\draw [bend right, looseness=1.25] (6) to (1.center);
		\draw [bend left, looseness=1.25] (6) to (3);
		\draw [in=150, out=-90] (0) to (4);
		\draw (4) to (6);
	\end{pgfonlayer}
\end{tikzpicture} = \begin{tikzpicture}
	\begin{pgfonlayer}{nodelayer}
		\node [style=none] (1) at (0.75, 3) {};
		\node [style=none] (5) at (0.75, 5.25) {};
	\end{pgfonlayer}
	\begin{pgfonlayer}{edgelayer}
		\draw (5.center) to (1.center);
	\end{pgfonlayer}
\end{tikzpicture} \]
\end{proof}
Step $(1)$ of the proof uses the unit law for monoids, step $(2)$ uses equation $(b)$ of bialgebras, 
and step $(3)$ uses the Frobenius law. Thus, for a Frobenius algebra which is also a bialgebra, 
the unit map is inverse of the counit map. 

In the previous section, we saw that quantum observables are precisely 
$\dagger$-SCFAs in $\FHilb$. We now present the conditions for any two  $\dagger$-SCFA 
to be complementary:

\begin{definition} \cite{HeV19}
In a $\dagger$-SMC, any two $\dagger$-SCFAs, say $\Frob{A}$ and $\bFrob{A}$ are 
{\bf complementary} if $\bialg{A}$ ( equivalently $\bialgb{A}$) is a Hopf algebra 
with antipode:
\[ \begin{tikzpicture}
	\begin{pgfonlayer}{nodelayer}
		\node [style=black] (0) at (-0.5, 3.5) {};
		\node [style=black] (1) at (-0.5, 4.5) {};
		\node [style=circle] (2) at (0.25, 2.25) {};
		\node [style=circle] (3) at (0.25, 1.25) {};
		\node [style=none] (4) at (-1, 1.25) {};
		\node [style=none] (5) at (0.75, 4.5) {};
	\end{pgfonlayer}
	\begin{pgfonlayer}{edgelayer}
		\draw (1) to (0);
		\draw (2) to (3);
		\draw [in=90, out=-135] (0) to (4.center);
		\draw (0) to (2);
		\draw [in=-90, out=45] (2) to (5.center);
	\end{pgfonlayer}
\end{tikzpicture} = \begin{tikzpicture}
	\begin{pgfonlayer}{nodelayer}
		\node [style=black] (0) at (0.25, 3.5) {};
		\node [style=black] (1) at (0.25, 4.5) {};
		\node [style=circle] (2) at (-0.5, 2.25) {};
		\node [style=circle] (3) at (-0.5, 1.25) {};
		\node [style=none] (4) at (0.75, 1.25) {};
		\node [style=none] (5) at (-1, 4.5) {};
	\end{pgfonlayer}
	\begin{pgfonlayer}{edgelayer}
		\draw (1) to (0);
		\draw (2) to (3);
		\draw [in=90, out=-45] (0) to (4.center);
		\draw (0) to (2);
		\draw [in=-90, out=135] (2) to (5.center);
	\end{pgfonlayer}
\end{tikzpicture} \]
\end{definition}
Note that in the CQM literature, the above definition is referred to as {\em strong complementarity}. 

%TODO: Ask Amolak about normalization
Let us look at an example of complementary observables in $\FHilb$.
A simple yet significant example is given by the spin of an electron 
along the $X$, $Y$ and $Z$ axes. The spin of an electron is either 
{\em up} or {\em down} or a superposition these two states. This system 
is referred to as a qubit in quantum computation, and its state space is $\C^2$. 

Each of the $X$, $Y$, and $Z$ observables are complementary to one another. 
These observables are given by {\em Pauli matrices}  $X$, $Y$ and $Z$ defined as follows:
 \[ X =  \begin{pmatrix}  0 & 1 \\ 1 & 0  \end{pmatrix} 
 ~~~~~~~~ Y = \begin{pmatrix}  0 & -i \\ i & 0  \end{pmatrix} 
~~~~~~~~ Z = \begin{pmatrix}  1 & 0 \\ 0 & -1  \end{pmatrix} \]
The eigenbasis of $X$, $Y$, and $Z$ are $\left \{(|0 \rangle + | 1 \rangle),  (|0 \rangle - | 1 \rangle) \right \}$, 
$Y$ is $\left \{ (|0 \rangle + i | 1 \rangle), (|0 \rangle - i | 1 \rangle) \right \}$, 
and $Z$ is $\{ |0 \rangle, | 1 \rangle \}$ respectively, where:
\[ | 0 \rangle = \begin{pmatrix}  1  \\  0   \end{pmatrix} ~~~~~~~  | 1 \rangle = \begin{pmatrix}  0  \\  1   \end{pmatrix} \]
These eigenvectors provide an orthonormal basis for $\C^2$, hence they are $\dagger$-SCFAs in $\FHilb$, see \cite[Example 2.11]{CoD11}. Note that, 
when used in calculations, the $X$ and $Y$-eigenvectors are normalized so that for any of these vectors $v$, $\langle v | v \rangle  = v v^\dag = 1$. 
In CQM, normalization is handled by introducing scalars $(a: I \to I)$ in the bialgebra equations \cite{CoD11, Bac15}. 

For a pair of complementary observables, the antipode is the identity map if and only if the self-dual 
cups and caps of the Frobenius algebras coincide. This is true in the case of $Z$ and $X$ observables.
(and not for $Z$-$Y$ and $X$-$Y$ pairs). Coecke and Duncan developed a diagrammatic 
calculus for $Z$ and $X$ observables, called the ZX-calculus \cite{CoD11}. 
The bialgebraic interaction between the $Z$ and the $X$ observables is used in the calculus 
 to construct logic gates in quantum computation. The ZX calculus is considered one of the most significant 
outcomes of the CQM program and is used extensively for quantum circuits optimization.

We end this section by providing an alternate and a useful characterization of complementary systems: 

\begin{theorem}\cite[Theorem 6.4]{DuK16} 
In a $\dagger$-SMC, two $\dagger$-SCFA $\Frob{A}$, and $\bFrob{A}$ are complementary if and only if 
the following equations hold: 
\[ \begin{tikzpicture}
	\begin{pgfonlayer}{nodelayer}
		\node [style=black] (0) at (0.25, 3.5) {};
		\node [style=black] (1) at (0.25, 4.5) {};
		\node [style=circle] (2) at (-0.25, 2.25) {};
		\node [style=none] (4) at (0.75, 2.25) {};
	\end{pgfonlayer}
	\begin{pgfonlayer}{edgelayer}
		\draw (1) to (0);
		\draw [in=90, out=-30] (0) to (4.center);
		\draw [in=90, out=-150] (0) to (2);
	\end{pgfonlayer}
\end{tikzpicture} = \begin{tikzpicture}
	\begin{pgfonlayer}{nodelayer}
		\node [style=circle] (1) at (0.25, 4.5) {};
		\node [style=none] (2) at (0.25, 2.25) {};
	\end{pgfonlayer}
	\begin{pgfonlayer}{edgelayer}
		\draw (2.center) to (1);
	\end{pgfonlayer}
\end{tikzpicture}
~~~~~~~~ \begin{tikzpicture}
	\begin{pgfonlayer}{nodelayer}
		\node [style=circle] (0) at (0.25, 3.25) {};
		\node [style=circle] (1) at (0.25, 2.25) {};
		\node [style=black] (2) at (-0.25, 4.5) {};
		\node [style=none] (4) at (0.75, 4.5) {};
	\end{pgfonlayer}
	\begin{pgfonlayer}{edgelayer}
		\draw (1) to (0);
		\draw [in=-90, out=30] (0) to (4.center);
		\draw [in=-90, out=150] (0) to (2);
	\end{pgfonlayer}
\end{tikzpicture} = \begin{tikzpicture}
	\begin{pgfonlayer}{nodelayer}
		\node [style=black] (1) at (0.25, 2.25) {};
		\node [style=none] (2) at (0.25, 4.5) {};
	\end{pgfonlayer}
	\begin{pgfonlayer}{edgelayer}
		\draw (2.center) to (1);
	\end{pgfonlayer}
\end{tikzpicture}\] 
\end{theorem}

In the previous section, for every Frobenius algebra, the multiplication is dual the comultiplication, and the 
unit is to dual to the counit via the self-dual cup and cap of the Frobenius algebra. 
The above theorem implies two $\dagger$-SCFAs are complementary if and only if 
the unit and the counit of each of these algebras are 
dual to one another using the self-dual cup and the cap of its complementary algebra. 
