% !TEX root = thesis.tex

\section{Dagger, duals, and conjugation}
\label{daggers-duals-conjugation}

%%%%%%%%%%%%%%%%%%%%%%%%%%%%%%%%%%%%%%%%%%%%%%%%

The goal of this section is to review the interaction of the dualizing, conjugation and dagger functors. 
In dagger compact closed categories, the dagger functor $(\_)^\dagger$, and the dualizing functor $(\_)^*$ 
commute with each other and their composite gives the conjugate functor $(\_)_*$. Similary, $(\_)_*$ and $(\_)^*$ 
when composed gives the dagger functor.  Our aim is to  generalize these interactions to $\dagger$-LDCs and to achieve 
this at a reasonable level of abstraction.   To achieve this we shall need the notion which we here refer to as ``conjugation'' 
but was investigated by Egger in \cite{Egg11} under the moniker of ``involution'' (which clashes with our usage).  

%%%%%%%%%%%%%%%%%%%%%%%%%%%%%%%%%%%%%%%%%%%%%%%%%%%%%%%%%%%%%%%%%%%%%%

\subsection{Duals}
\label{Sec: dduals}
The reverse of an LDC, $\X$, written $\X^\rev := (\X, \ox, \top, \oa, \bot)^{\sf rev} = (\X, \ox^{\sf rev}, 
\top, \oa^{\sf rev}, \bot)$ where,
\[ A \ox^{\sf rev} B := B \ox A ~~~~~~~~~~~ A \oa^{\sf rev} B := B \oa A \]
and the associators and distributors are adjusted accordingly.  Similar to the opposite of an LDC, we have 
$(\X^{\sf rev})^{\sf rev} = \X$.  

In a $*$-autonomous category, taking the left (or right) linear dual of an object extends to a Frobenius linear 
functor as follows:
\begin{align*}
(\_)^*: (\X^{\sf op})^{\sf rev} &\to \X  ; ~~ A \mapsto A^* ; ~~
\begin{tikzpicture}
	\begin{pgfonlayer}{nodelayer}
		\node [style=none] (0) at (0, 2) {};
		\node [style=none] (1) at (-0.5, 1) {};
		\node [style=none] (2) at (0.5, 1) {};
		\node [style=none] (3) at (-0.5, -0) {};
		\node [style=none] (4) at (0.5, -0) {};
		\node [style=none] (5) at (0, 1) {};
		\node [style=none] (6) at (0, -1) {};
		\node [style=none] (7) at (0, -0) {};
		\node [style=none] (8) at (0, 0.5) {$f$};
		\node [style=none] (9) at (0.25, 1.75) {$A$};
		\node [style=none] (10) at (0.25, -0.75) {$B$};
	\end{pgfonlayer}
	\begin{pgfonlayer}{edgelayer}
		\draw (1.center) to (3.center);
		\draw (3.center) to (4.center);
		\draw (4.center) to (2.center);
		\draw (2.center) to (1.center);
		\draw (0.center) to (5.center);
		\draw (7.center) to (6.center);
	\end{pgfonlayer}
\end{tikzpicture} \mapsto  \begin{tikzpicture}
	\begin{pgfonlayer}{nodelayer}
		\node [style=none] (0) at (0, 1.5) {};
		\node [style=none] (1) at (-0.5, 1) {};
		\node [style=none] (2) at (0.5, 1) {};
		\node [style=none] (3) at (-0.5, -0) {};
		\node [style=none] (4) at (0.5, -0) {};
		\node [style=none] (5) at (0, 1) {};
		\node [style=none] (6) at (0, -0.5) {};
		\node [style=none] (7) at (0, -0) {};
		\node [style=none] (8) at (-1, 1.5) {};
		\node [style=none] (9) at (-1, -1) {};
		\node [style=none] (10) at (1, -0.5) {};
		\node [style=none] (11) at (1, 2) {};
		\node [style=none] (12) at (0, 0.5) {$f$};
		\node [style=none] (13) at (1.25, 1.75) {$B^*$};
		\node [style=none] (14) at (-1.25, -0.75) {$A^*$};
	\end{pgfonlayer}
	\begin{pgfonlayer}{edgelayer}
		\draw (1.center) to (3.center);
		\draw (3.center) to (4.center);
		\draw (4.center) to (2.center);
		\draw (2.center) to (1.center);
		\draw (0.center) to (5.center);
		\draw (7.center) to (6.center);
		\draw [bend right=90, looseness=1.25] (0.center) to (8.center);
		\draw (8.center) to (9.center);
		\draw [bend right=90, looseness=1.25] (6.center) to (10.center);
		\draw (10.center) to (11.center);
	\end{pgfonlayer}
\end{tikzpicture}
\end{align*}

The $(\_)^*$ functor is both contravariant and, op-monoidal and op-comonoidal:


\[ m_{\ox}:  A^* \ox B^* \to (B \oa A)^* :=  \begin{tikzpicture}
\begin{pgfonlayer}{nodelayer}
\node [style=none] (0) at (-0.25, 2) {};
\node [style=ox] (1) at (-0.25, 1) {};
\node [style=none] (2) at (-0.5, 0.25) {};
\node [style=none] (3) at (0.25, -0) {};
\node [style=none] (4) at (-1.25, 0.25) {};
\node [style=none] (5) at (-2, -0) {};
\node [style=oa] (6) at (-1.5, 1) {};
\node [style=none] (7) at (-1.5, 1.5) {};
\node [style=none] (8) at (-2.25, 1.5) {};
\node [style=none] (9) at (-2.25, -1) {};
\node [style=none] (10) at (0.5, 1.75) {$B^* \ox A^*$};
\node [style=none] (11) at (-0.8, 0.5) {$B^*$};
\node [style=none] (12) at (0.5, 0.5) {$A^*$};
\node [style=none] (13) at (-3.2, -0.75) {$(A \oa B)^* $};
\end{pgfonlayer}
\begin{pgfonlayer}{edgelayer}
\draw [in=90, out=-45, looseness=1.00] (1) to (3.center);
\draw [in=90, out=-135, looseness=1.00] (1) to (2.center);
\draw [bend left=90, looseness=1.25] (2.center) to (4.center);
\draw [in=-60, out=90, looseness=1.00] (4.center) to (6);
\draw (6) to (7.center);
\draw [bend right=90, looseness=1.25] (7.center) to (8.center);
\draw (8.center) to (9.center);
\draw [in=90, out=-135, looseness=0.75] (6) to (5.center);
\draw [in=-90, out=-90, looseness=1.00] (5.center) to (3.center);
\draw (1) to (0.center);
\end{pgfonlayer}
\end{tikzpicture} 
~~~~~~~~~~~~~~~ m_\top: \top \to \bot^* :=  \begin{tikzpicture} %pic2
\begin{pgfonlayer}{nodelayer}
\node [style=circle] (0) at (0, 1.75) {$\top$};
\node [style=none] (1) at (0, 3) {};
\node [style=circle, scale=0.5] (2) at (-1, 1) {};
\node [style=circle] (3) at (-1, -0.25) {$\bot$};
\node [style=none] (4) at (-2, -1) {};
\node [style=none] (5) at (-2, 1.5) {};
\node [style=none] (6) at (-1, 1.5) {};
\node [style=none] (7) at (0.25, 2.75) {$\top$};
\node [style=none] (8) at (-2.5, -0.75) {$\bot^*$};
\end{pgfonlayer}
\begin{pgfonlayer}{edgelayer}
\draw (1.center) to (0);
\draw (3) to (6.center);
\draw [bend left=90, looseness=2.00] (5.center) to (6.center);
\draw (5.center) to (4.center);
\draw [in=0, out=-90, looseness=1.50, dotted] (0) to (2);
\end{pgfonlayer}
\end{tikzpicture} \]
\[ n_{\oa}:  (A \ox B)^* \to B^* \oa A^* :=  \begin{tikzpicture} %pic3
\begin{pgfonlayer}{nodelayer}
\node [style=none] (0) at (2, 2) {};
\node [style=ox] (1) at (1, 0.25) {};
\node [style=none] (2) at (2, -0.5) {};
\node [style=none] (3) at (1, -0.5) {};
\node [style=none] (4) at (0.5, 0.75) {};
\node [style=none] (5) at (1.5, 0.75) {};
\node [style=oa] (6) at (-0.75, 0.25) {};
\node [style=none] (7) at (-0.75, -1.75) {};
\node [style=none] (8) at (-0.25, 0.75) {};
\node [style=none] (9) at (-1.25, 0.75) {};
\node [style=none] (10) at (2, 2.25) {$(A \ox B)^*$};
\node [style=none] (11) at (-0.75, -2) {$B^* \oa A^*$};
\end{pgfonlayer}
\begin{pgfonlayer}{edgelayer}
\draw (0.center) to (2.center);
\draw [bend left=90, looseness=1.50] (2.center) to (3.center);
\draw (3.center) to (1);
\draw (6) to (7.center);
\draw [in=150, out=-90, looseness=1.00] (9.center) to (6);
\draw [in=-90, out=11, looseness=1.00] (6) to (8.center);
\draw [in=165, out=-90, looseness=1.25] (4.center) to (1);
\draw [in=-90, out=15, looseness=1.25] (1) to (5.center);
\draw [bend left=90, looseness=2.00] (8.center) to (4.center);
\draw [bend left=90, looseness=1.25] (9.center) to (5.center);
\end{pgfonlayer}
\end{tikzpicture} ~~~~~~~~~~~~~~~ n_\bot: \top^* \to \bot :=  \begin{tikzpicture}
\begin{pgfonlayer}{nodelayer}
\node [style=circle] (0) at (-2, 1) {$\top$};
\node [style=none] (1) at (-1, 2) {};
\node [style=none] (2) at (-2, -0) {};
\node [style=none] (3) at (-1, -0) {};
\node [style=circle] (4) at (0, -1) {$\bot$};
\node [style=circle, scale=0.5] (5) at (-1, 1) {};
\node [style=none] (6) at (0, -2) {};
\node [style=none] (7) at (-0.6, 1.75) {$\top^*$};
\end{pgfonlayer}
\begin{pgfonlayer}{edgelayer}
\draw (0) to (2.center);
\draw [bend right=90, looseness=1.75] (2.center) to (3.center);
\draw (3.center) to (1.center);
\draw [in=90, out=-15, looseness=1.25, dotted] (5) to (4);
\draw (6.center) to (4);
\end{pgfonlayer}
\end{tikzpicture} \]

These maps are op-monoidal and op-comonoidal laxors, hence are isomorphisms, which satisfy the obvious coherences. 
Thus, $(\_)^*$ is a strong Frobenius linear functor. 

In the rest of the section, we will write $(\X^\op)^\rev$ as $\X^{\op\rev}$.

\begin{lemma} If $\X$ is an isomix category, then $(\_)^*: \X^{\op\rev} \to \X$ is an isomix functor. \end{lemma}
\begin{proof}
Because  $(\_)^*$ is a strong Frobenius functor, by Lemma \ref{Lemma: isomix functor}, it suffices  to prove that 
$(\_)^*$ preserves mix, i.e.,  $(\_)^*$  is a mix functor i.e., we need  to show that $n_\bot \!~\m~ m_\top : \top^* \to \bot^* =  
\m^* $. The proof is as follows:
\[ n_\bot \m m_\top = \begin{tikzpicture}
	\begin{pgfonlayer}{nodelayer}
		\node [style=circle] (0) at (-3, 2.75) {$\top$};
		\node [style=none] (1) at (-2, 2) {};
		\node [style=none] (2) at (-2, 3) {};
		\node [style=circle] (3) at (-1, 1.25) {$\bot$};
		\node [style=map] (4) at (-1, 0.5) {};
		\node [style=circle] (5) at (-1, -0.25) {$\top$};
		\node [style=circle] (6) at (-2, -2) {$\bot$};
		\node [style=none] (7) at (-3, -1.25) {};
		\node [style=none] (8) at (-3, -2.25) {};
		\node [style=circle, scale=0.5] (9) at (-2, 1.75) {};
		\node [style=circle, scale=0.5] (10) at (-2, -0.75) {};
		\node [style=none] (11) at (-3, 2) {};
		\node [style=none] (12) at (-2, -1.25) {};
	\end{pgfonlayer}
	\begin{pgfonlayer}{edgelayer}
		\draw (1.center) to (2.center);
		\draw (7.center) to (8.center);
		\draw [bend left, looseness=1.00, dotted] (9) to (3);
		\draw [bend left=45, looseness=1.00, dotted] (5) to (10);
		\draw (3) to (4);
		\draw (4) to (5);
		\draw [bend right=90, looseness=3.25] (11.center) to (1.center);
		\draw (0) to (11.center);
		\draw [bend left=90, looseness=3.75] (7.center) to (12.center);
		\draw (12.center) to (6);
	\end{pgfonlayer}
\end{tikzpicture} =\begin{tikzpicture}
	\begin{pgfonlayer}{nodelayer}
		\node [style=circle] (0) at (1.25, 3) {$\top$};
		\node [style=none] (1) at (0.25, 3) {};
		\node [style=circle] (2) at (-1, 1.25) {$\bot$};
		\node [style=map] (3) at (-1, 0.5) {};
		\node [style=circle] (4) at (-1, -0.25) {$\top$};
		\node [style=circle] (5) at (-2, -2.25) {$\bot$};
		\node [style=none] (6) at (-3, -2.25) {};
		\node [style=circle, scale=0.5] (7) at (0.25, 2) {};
		\node [style=circle, scale=0.5] (8) at (-2, -0.75) {};
		\node [style=none] (9) at (0.25, 1.5) {};
		\node [style=none] (10) at (1.25, 1.5) {};
		\node [style=none] (11) at (-2, -0.25) {};
		\node [style=none] (12) at (-3, -0.25) {};
	\end{pgfonlayer}
	\begin{pgfonlayer}{edgelayer}
		\draw [dotted, bend right, looseness=1.25] (7) to (2);
		\draw [dotted, bend left=45, looseness=1.00] (4) to (8);
		\draw (2) to (3);
		\draw (3) to (4);
		\draw (1.center) to (9.center);
		\draw [bend right=90, looseness=2.00] (9.center) to (10.center);
		\draw (10.center) to (0);
		\draw (5) to (11.center);
		\draw (12.center) to (6.center);
		\draw [bend left=90, looseness=1.75] (12.center) to (11.center);
	\end{pgfonlayer}
\end{tikzpicture} = \begin{tikzpicture} %m3
	\begin{pgfonlayer}{nodelayer}
		\node [style=circle] (0) at (1.25, 3) {$\top$};
		\node [style=none] (1) at (0.25, 2.75) {};
		\node [style=circle] (2) at (-1, 1.25) {$\bot$};
		\node [style=map] (3) at (-1, 0.5) {};
		\node [style=circle] (4) at (-1, -0.25) {$\top$};
		\node [style=circle] (5) at (-2, -2.25) {$\bot$};
		\node [style=none] (6) at (-3, -2) {};
		\node [style=circle, scale=0.5] (7) at (0.25, 2) {};
		\node [style=circle, scale=0.5] (8) at (-2, -0.75) {};
		\node [style=none] (9) at (0.25, -1.75) {};
		\node [style=none] (10) at (1.25, -1.75) {};
		\node [style=none] (11) at (-2, 2.5) {};
		\node [style=none] (12) at (-3, 2.5) {};
	\end{pgfonlayer}
	\begin{pgfonlayer}{edgelayer}
		\draw [dotted, bend right, looseness=1.25] (7) to (2);
		\draw [dotted, bend left=45, looseness=1.00] (4) to (8);
		\draw (2) to (3);
		\draw (3) to (4);
		\draw (1.center) to (9.center);
		\draw [bend right=90, looseness=2.00] (9.center) to (10.center);
		\draw (10.center) to (0);
		\draw (5) to (11.center);
		\draw (12.center) to (6.center);
		\draw [bend left=90, looseness=1.75] (12.center) to (11.center);
	\end{pgfonlayer}
\end{tikzpicture} \stackrel{\mx}{=} \begin{tikzpicture} %m3
	\begin{pgfonlayer}{nodelayer}
		\node [style=circle] (0) at (1.25, 3) {$\top$};
		\node [style=none] (1) at (0.25, 2.75) {};
		\node [style=circle] (2) at (-0.75, 1.25) {$\bot$};
		\node [style=map] (3) at (-0.75, 0.5) {};
		\node [style=circle] (4) at (-0.75, -0.25) {$\top$};
		\node [style=circle] (5) at (-2, -2.25) {$\bot$};
		\node [style=none] (6) at (-3, -2) {};
		\node [style=circle, scale=0.5] (7) at (-2, 2) {};
		\node [style=circle, scale=0.5] (8) at (0.25, -0.75) {};
		\node [style=none] (9) at (0.25, -1.75) {};
		\node [style=none] (10) at (1.25, -1.75) {};
		\node [style=none] (11) at (-2, 2.5) {};
		\node [style=none] (12) at (-3, 2.5) {};
	\end{pgfonlayer}
	\begin{pgfonlayer}{edgelayer}
		\draw [dotted, bend left, looseness=1.25] (7) to (2);
		\draw [dotted, bend right=45, looseness=1.00] (4) to (8);
		\draw (2) to (3);
		\draw (3) to (4);
		\draw (1.center) to (9.center);
		\draw [bend right=90, looseness=2.00] (9.center) to (10.center);
		\draw (10.center) to (0);
		\draw (5) to (11.center);
		\draw (12.center) to (6.center);
		\draw [bend left=90, looseness=1.75] (12.center) to (11.center);
	\end{pgfonlayer}
\end{tikzpicture} = \begin{tikzpicture}
	\begin{pgfonlayer}{nodelayer}
		\node [style=map] (0) at (-0.75, 1) {};
		\node [style=none] (1) at (-1.75, -1.75) {};
		\node [style=none] (2) at (-0.75, -0.25) {};
		\node [style=none] (3) at (0.25, -0.25) {};
		\node [style=none] (4) at (-0.75, 2.25) {};
		\node [style=none] (5) at (-1.75, 2.25) {};
		\node [style=none] (6) at (0.25, 3.25) {};
	\end{pgfonlayer}
	\begin{pgfonlayer}{edgelayer}
		\draw [bend right=90, looseness=2.00] (2.center) to (3.center);
		\draw (5.center) to (1.center);
		\draw [bend left=90, looseness=1.75] (5.center) to (4.center);
		\draw (4.center) to (0);
		\draw (0) to (2.center);
		\draw (6.center) to (3.center);
	\end{pgfonlayer}
\end{tikzpicture} = \m^*
\]
\end{proof}

\begin{lemma}
$(\eta, \epsilon) :: (\_)^* \dashvv ~ {^*(\_)^{\op\rev}} : \X^{\op \rev} \to \X$  
\[ 
\eta_\ox: X \to ~^*(X^*) := \begin{tikzpicture} %snakea
	\begin{pgfonlayer}{nodelayer}
		\node [style=none] (0) at (1, -2) {};
		\node [style=none] (1) at (1, 0.75) {};
		\node [style=none] (2) at (0, 0.75) {};
		\node [style=none] (12) at (0.5, 1.75) {$*\eta$};		
		\node [style=none] (3) at (0, -0.25) {};
		\node [style=none] (4) at (-1, -0.25) {};
		\node [style=none] (12) at (-0.5, -1.5) {$\epsilon*$};		
		\node [style=none] (5) at (-1, 2.5) {};
		\node [style=none] (6) at (-1.25, 2) {$X$};
		\node [style=none] (7) at (-0.35, 0.5) {$X^*$};
		\node [style=none] (8) at (1.6, -1.75) {$^*(X^*)$};
	\end{pgfonlayer}
	\begin{pgfonlayer}{edgelayer}
		\draw (4.center) to (5.center);
		\draw [bend right=90, looseness=3.25] (4.center) to (3.center);
		\draw (3.center) to (2.center);
		\draw [bend left=90, looseness=2.50] (2.center) to (1.center);
		\draw (1.center) to (0.center);
	\end{pgfonlayer}
\end{tikzpicture} \in \X
~~~~~~~~~~~~ \eta_\oa := \eta_\ox^{-1} \]
\[
\epsilon_\oa: X \to  (^*X)^*:= \begin{tikzpicture}
	\begin{pgfonlayer}{nodelayer}
		\node [style=none] (0) at (-0.75, -2) {};
		\node [style=none] (1) at (-0.75, 0.75) {};
		\node [style=none] (2) at (0.25, 0.75) {};
		\node [style=none] (12) at (-0.5, 1.75) {$\eta*$};
		\node [style=none] (3) at (0.25, -0.25) {};
		\node [style=none] (4) at (1.25, -0.25) {};
		\node [style=none] (34) at (0.65, -1.5) {$*\epsilon$};
		\node [style=none] (5) at (1.25, 2.5) {};
		\node [style=none] (6) at (1.65, 2) {$X$};
		\node [style=none] (7) at (0.6, 0.5) {$^*X$};
		\node [style=none] (8) at (-1.25, -1.75) {$(^*X)^*$};
	\end{pgfonlayer}
	\begin{pgfonlayer}{edgelayer}
		\draw (4.center) to (5.center);
		\draw [bend left=90, looseness=3.25] (4.center) to (3.center);
		\draw (3.center) to (2.center);
		\draw [bend right=90, looseness=2.50] (2.center) to (1.center);
		\draw (1.center) to (0.center);
	\end{pgfonlayer}
\end{tikzpicture} \in \X ~~~~~~~~~~~ \epsilon_\ox := \epsilon_\oa^{-1}
\]
is a linear equivalence of Frobenius linear functors.
\end{lemma}
\begin{proof}
The proof is straightforward in the graphical calculus.
\end{proof}


For a cyclic $*$-autonomous category, we can straighten out this equivalence to be a dualizing involutive equivalence 
(i.e. so that the unit and counit are equal):

\begin{lemma}
$(\eta', \epsilon') :: (\_)^* \dashvv  ((\_)^{*})^{\op\rev}: \X^{\op\rev} \to \X$  where $\eta'_\ox = {\eta'_\oa}^{-1} := 
\eta_\ox \psi^{-1}$, $\epsilon'_\ox = \epsilon'_\oa:= \epsilon \psi^*$ and $\eta' = \epsilon'$.
\end{lemma}
\begin{proof}

The unit and counit are drawn as follows:

\[
\eta_\ox' =  \begin{tikzpicture}
	\begin{pgfonlayer}{nodelayer}
		\node [style=none] (0) at (-3, 3) {};
		\node [style=none] (1) at (-3, 1) {};
		\node [style=none] (2) at (-2, 1) {};
		\node [style=none] (3) at (-2, 2) {};
		\node [style=none] (4) at (-1, 2) {};
		\node [style=circle] (5) at (-1, -0) {$\psi^{-1}$};
		\node [style=none] (6) at (-1, -1) {};
		\node [style=none] (7) at (-2.5, 0.25) {$\epsilon*$};
		\node [style=none] (8) at (-1.5, 2.75) {$~^*\eta$};
		\node [style=none] (9) at (-3.25, 2.75) {$X$};
		\node [style=none] (10) at (-0.5, 1.75) {$~^*(X^*)$};
		\node [style=none] (11) at (-0.5, -0.7) {$X^{**}$};
		\node [style=none] (12) at (-2.35, 1.5) {$X^*$};
	\end{pgfonlayer}
	\begin{pgfonlayer}{edgelayer}
		\draw (0.center) to (1.center);
		\draw [bend right=90, looseness=1.75] (1.center) to (2.center);
		\draw (2.center) to (3.center);
		\draw [bend left=90, looseness=2.00] (3.center) to (4.center);
		\draw (4.center) to (5);
		\draw (5) to (6.center);
	\end{pgfonlayer}
\end{tikzpicture} \in \X ~~~~~~~~~~ \epsilon_\ox' = \begin{tikzpicture}
	\begin{pgfonlayer}{nodelayer}
		\node [style=none] (0) at (-0.75, -0.5) {};
		\node [style=none] (1) at (-0.75, 2.25) {};
		\node [style=none] (2) at (0.25, 2.25) {};
		\node [style=none] (3) at (0.25, 1) {};
		\node [style=none] (4) at (1.25, 1) {};
		\node [style=none] (5) at (1.25, 3) {};
		\node [style=none] (6) at (0.75, 0.25) {$*\epsilon$};
		\node [style=none] (7) at (-0.25, 3) {$\eta*$};
		\node [style=none] (8) at (1.5, 2.5) {$X^*$};
		\node [style=none] (9) at (-1.75, -0.5) {};
		\node [style=circle, scale=2] (10) at (-1.75, 0.5) {};
		\node [style=none] (11) at (-1.75, 1.25) {};
		\node [style=none] (12) at (-2.75, 1.25) {};
		\node [style=none] (13) at (-2.75, -1.25) {};
		\node [style=none] (14) at (-1.25, -1.25) {$\epsilon*$};
		\node [style=none] (15) at (-2.25, 2) {$\eta*$};
		\node [style=none] (16) at (-1.75, 0.5) {$\psi$};
		\node [style=none] (17) at (-3.25, -0.75) {$X^{**}$};
		\node [style=none] (18) at (-0.25, -0) {$~^*(X^*)$};
	\end{pgfonlayer}
	\begin{pgfonlayer}{edgelayer}
		\draw [bend left=90, looseness=2.00] (1.center) to (2.center);
		\draw (2.center) to (3.center);
		\draw [bend right=90, looseness=2.00] (3.center) to (4.center);
		\draw (4.center) to (5.center);
		\draw (1.center) to (0.center);
		\draw [bend right=90, looseness=1.75] (9.center) to (0.center);
		\draw (11.center) to (10);
		\draw (10) to (9.center);
		\draw [bend right=90, looseness=1.75] (11.center) to (12.center);
		\draw (12.center) to (13.center);
	\end{pgfonlayer}
\end{tikzpicture} = %dualizor1
 \begin{tikzpicture}
	\begin{pgfonlayer}{nodelayer}
		\node [style=none] (0) at (-0.75, 3) {};
		\node [style=none] (1) at (-0.75, -0) {};
		\node [style=none] (2) at (-1.75, -0) {};
		\node [style=none] (3) at (-1.75, 2) {};
		\node [style=none] (4) at (-2.75, 2) {};
		\node [style=none] (5) at (-2.75, -1) {};
		\node [style=none] (6) at (-1.25, -0.75) {$*\epsilon$};
		\node [style=none] (7) at (-2.25, 2.75) {$\eta^*$};
		\node [style=none] (8) at (-0.5, 2.75) {$X$};
		\node [style=none] (9) at (-3.25, -0.5) {$X^{**}$};
		\node [style=circle] (10) at (-1.75, 1) {$\psi$};
		\node [style=none] (11) at (-2.25, 0.25) {$~^*X$};
		\node [style=none] (12) at (-2.1, 1.7) {$X^*$};
	\end{pgfonlayer}
	\begin{pgfonlayer}{edgelayer}
		\draw (0.center) to (1.center);
		\draw [bend left=90, looseness=1.75] (1.center) to (2.center);
		\draw [bend right=90, looseness=2.00] (3.center) to (4.center);
		\draw (4.center) to (5.center);
		\draw (3.center) to (10);
		\draw (10) to (2.center);
	\end{pgfonlayer}
\end{tikzpicture} \stackrel{{\bf \tiny [C.2]}}{=} %dualizor1
 \begin{tikzpicture}
	\begin{pgfonlayer}{nodelayer}
		\node [style=none] (0) at (-3, 3) {};
		\node [style=none] (1) at (-3, 1) {};
		\node [style=none] (2) at (-2, 1) {};
		\node [style=none] (3) at (-2, 2) {};
		\node [style=none] (4) at (-1, 2) {};
		\node [style=circle, scale=2.3] (5) at (-1, -0) {};
		\node [style=none] (13) at (-1, -0) {$\psi^{-1}$};
		\node [style=none] (6) at (-1, -1) {};
		\node [style=none] (7) at (-2.5, 0.25) {$\epsilon*$};
		\node [style=none] (8) at (-1.5, 2.75) {$~^*\eta$};
		\node [style=none] (9) at (-3.25, 2.75) {$X$};
		\node [style=none] (10) at (-0.5, 1.75) {$~^*(X^*)$};
		\node [style=none] (11) at (-0.5, -0.7) {$X^{**}$};
		\node [style=none] (12) at (-2.35, 1.5) {$X^*$};
	\end{pgfonlayer}
	\begin{pgfonlayer}{edgelayer}
		\draw (0.center) to (1.center);
		\draw [bend right=90, looseness=1.75] (1.center) to (2.center);
		\draw (2.center) to (3.center);
		\draw [bend left=90, looseness=2.00] (3.center) to (4.center);
		\draw (4.center) to (5);
		\draw (5) to (6.center);
	\end{pgfonlayer}
\end{tikzpicture} \in \X
\]

The cyclor is a linear transformation which is an isomorphism as it is monoidal with respect to both tensor 
and par and adjoints are determined only upto isomorphism. It remains to check that the triangle identities hold:
\[
\eta_{\ox'_{X^*}}(\epsilon_\ox')^* = 
\begin{tikzpicture} %triangle1
	\begin{pgfonlayer}{nodelayer}
		\node [style=circle] (0) at (0.75, 0.25) {$\psi^{-1}$};
		\node [style=none] (1) at (0.75, -2.5) {};
		\node [style=none] (2) at (0.75, 2.25) {};
		\node [style=none] (3) at (-0.25, 2.25) {};
		\node [style=none] (4) at (-0.25, 1) {};
		\node [style=none] (5) at (-1.25, 1) {};
		\node [style=none] (6) at (-1.25, 3) {};
		\node [style=none] (7) at (1.5, -0.5) {$X^{***}$};
		\node [style=none] (8) at (-0.75, 0.25) {$\epsilon*$};
		\node [style=none] (9) at (0.25, 3) {$*\eta$};
		\node [style=none] (10) at (-1.5, 2.5) {$X^*$};
		\node [style=none] (11) at (-1, -2.5) {};
		\node [style=circle] (12) at (-1, -1.5) {$\psi^{-1}$};
		\node [style=none] (13) at (-1, -0.75) {};
		\node [style=none] (14) at (-2, -0.75) {};
		\node [style=none] (15) at (-2, -3.25) {};
		\node [style=none] (16) at (-1.5, -0.1) {$*\eta$};
		\node [style=none] (17) at (0, -3.35) {$\epsilon*$};
		\node [style=none] (18) at (-2.5, -3) {};
	\end{pgfonlayer}
	\begin{pgfonlayer}{edgelayer}
		\draw (2.center) to (0);
		\draw (0) to (1.center);
		\draw [bend right=90, looseness=2.00] (2.center) to (3.center);
		\draw (3.center) to (4.center);
		\draw [bend left=90, looseness=2.00] (4.center) to (5.center);
		\draw (5.center) to (6.center);
		\draw [bend right=75, looseness=1.25] (11.center) to (1.center);
		\draw (11.center) to (12);
		\draw (12) to (13.center);
		\draw [bend right=90, looseness=1.50] (13.center) to (14.center);
		\draw (14.center) to (15.center);
	\end{pgfonlayer}
\end{tikzpicture} = \begin{tikzpicture} %triangle2
	\begin{pgfonlayer}{nodelayer}
		\node [style=none] (0) at (0, -2.5) {};
		\node [style=none] (1) at (0, 2.5) {};
		\node [style=none] (2) at (1, 2.5) {};
		\node [style=none] (3) at (1, 1.25) {};
		\node [style=none] (4) at (2, 1.25) {};
		\node [style=none] (5) at (2, 3.25) {};
		\node [style=none] (6) at (-0.5, 1.25) {$X^{***}$};
		\node [style=none] (7) at (1.5, 0.5) {$*\epsilon$};
		\node [style=none] (8) at (0.5, 3.25) {$\eta*$};
		\node [style=none] (9) at (2.25, 2.75) {$X^*$};
		\node [style=none] (10) at (-1, -2.5) {};
		\node [style=circle] (11) at (-1, -1.5) {$\psi^{-1}$};
		\node [style=none] (12) at (-1, -0.75) {};
		\node [style=none] (13) at (-2, -0.75) {};
		\node [style=none] (14) at (-2, -3.25) {};
		\node [style=none] (15) at (-1.5, -0) {$*\eta$};
		\node [style=none] (16) at (-0.5, -3) {$\epsilon*$};
		\node [style=circle] (17) at (1, 2) {$\psi$};
		\node [style=none] (18) at (-2.25, -2.75) {$X^*$};
	\end{pgfonlayer}
	\begin{pgfonlayer}{edgelayer}
		\draw [bend left=90, looseness=2.00] (1.center) to (2.center);
		\draw [bend right=90, looseness=2.00] (3.center) to (4.center);
		\draw (4.center) to (5.center);
		\draw [bend right=75, looseness=1.25] (10.center) to (0.center);
		\draw (10.center) to (11);
		\draw (11) to (12.center);
		\draw [bend right=90, looseness=1.50] (12.center) to (13.center);
		\draw (13.center) to (14.center);
		\draw (1.center) to (0.center);
		\draw (2.center) to (17);
		\draw (17) to (3.center);
	\end{pgfonlayer}
\end{tikzpicture} = \begin{tikzpicture}
	\begin{pgfonlayer}{nodelayer}
		\node [style=none] (0) at (-1, -1) {};
		\node [style=none] (1) at (0, -1) {};
		\node [style=none] (2) at (0, 3.25) {};
		\node [style=none] (3) at (-0.5, -1.75) {$*\epsilon$};
		\node [style=none] (4) at (0.25, 2.75) {$X^*$};
		\node [style=circle] (5) at (-1, 1.5) {$\psi^{-1}$};
		\node [style=none] (6) at (-1, 2.25) {};
		\node [style=none] (7) at (-2, 2.25) {};
		\node [style=none] (8) at (-2, -3.25) {};
		\node [style=none] (9) at (-1.5, 3) {$*\eta$};
		\node [style=circle] (10) at (-1, -0.25) {$\psi$};
		\node [style=none] (11) at (-2.25, -2.75) {$X^*$};
	\end{pgfonlayer}
	\begin{pgfonlayer}{edgelayer}
		\draw [bend right=90, looseness=2.00] (0.center) to (1.center);
		\draw (1.center) to (2.center);
		\draw (5) to (6.center);
		\draw [bend right=90, looseness=1.50] (6.center) to (7.center);
		\draw (7.center) to (8.center);
		\draw (10) to (0.center);
		\draw (5) to (10);
	\end{pgfonlayer}
\end{tikzpicture} = ~ \begin{tikzpicture} \draw (2, 3.25) -- (2,-3.5); \end{tikzpicture} ~ = ~ 1
\] The other triangle identity holds similarly.
\end{proof}

The equality of $\eta'$ and $\epsilon'$ is immediate from {\bf [C.2]} for cyclors with the 
map $\eta'=\epsilon'$ being the {\bf dualizor}.   In the symmetric case, the dualizor of this 
equivalence may be drawn as:
 \[\begin{tikzpicture}
	\begin{pgfonlayer}{nodelayer}
		\node [style=none] (0) at (-3, 3) {};
		\node [style=none] (1) at (-3, 1) {};
		\node [style=none] (2) at (-2, 1) {};
		\node [style=none] (3) at (-2, 2) {};
		\node [style=none] (4) at (-1, 2) {};
		\node [style=circle, scale=2] (5) at (-1, -0) {};
		\node [style=none] (5) at (-1, -0) {$\psi^{-1}$};
		\node [style=none] (6) at (-1, -1) {};
		\node [style=none] (7) at (-2.5, 0.25) {$\epsilon*$};
		\node [style=none] (8) at (-1.5, 2.75) {$*\eta$};
		\node [style=none] (9) at (-3.25, 2.75) {$A$};
		\node [style=none] (10) at (-0.5, 1.75) {$~^*(A^*)$};
		\node [style=none] (11) at (-0.5, -0.7) {$A^{**}$};
		\node [style=none] (12) at (-2.25, 1.5) {$A^*$};
	\end{pgfonlayer}
	\begin{pgfonlayer}{edgelayer}
		\draw (0.center) to (1.center);
		\draw [bend right=90, looseness=1.75] (1.center) to (2.center);
		\draw (2.center) to (3.center);
		\draw [bend left=90, looseness=2.00] (3.center) to (4.center);
		\draw (4.center) to (5);
		\draw (5) to (6.center);
	\end{pgfonlayer}
\end{tikzpicture} =
\begin{tikzpicture}
\begin{pgfonlayer}{nodelayer}
\node [style=none] (0) at (-1.75, 2) {};
\node [style=none] (1) at (-1.75, -0) {};
\node [style=none] (2) at (0, -0) {};
\node [style=none] (12) at (-0.75, -1) {$\epsilon*$};
\node [style=none] (3) at (0, 1) {};
\node [style=none] (4) at (-1, 1) {};
\node [style=none] (34) at (-0.5, 1.75) {$\eta*$};
\node [style=none] (5) at (-1, -2.75) {};
\node [style=none] (6) at (-2, 1.75) {$A$};
\node [style=none] (8) at (-1.5, -2.25) {$A^{**}$};
\end{pgfonlayer}
\begin{pgfonlayer}{edgelayer}
\draw (0.center) to (1.center);
\draw [bend right=90, looseness=1.50] (1.center) to (2.center);
\draw (2.center) to (3.center);
\draw [bend right=90, looseness=1.75] (3.center) to (4.center);
\draw (4.center) to (5.center);
\end{pgfonlayer}
\end{tikzpicture} =  \begin{tikzpicture}
	\begin{pgfonlayer}{nodelayer}
		\node [style=none] (0) at (-0.75, 3) {};
		\node [style=none] (1) at (-0.75, -0) {};
		\node [style=none] (2) at (-1.75, -0) {};
		\node [style=none] (3) at (-1.75, 2) {};
		\node [style=none] (4) at (-2.75, 2) {};
		\node [style=none] (5) at (-2.75, -1) {};
		\node [style=none] (6) at (-1.25, -0.75) {$*\epsilon$};
		\node [style=none] (7) at (-2.25, 2.75) {$\eta*$};
		\node [style=none] (8) at (-0.5, 2.75) {$A$};
		\node [style=none] (9) at (-3.25, -0.5) {$A^{**}$};
		\node [style=circle] (10) at (-1.75, 1) {$\psi$};
		\node [style=none] (11) at (-2.25, 0.25) {$~^*A$};
		\node [style=none] (12) at (-2, 1.7) {$A^*$};
	\end{pgfonlayer}
	\begin{pgfonlayer}{edgelayer}
		\draw (0.center) to (1.center);
		\draw [bend left=90, looseness=1.75] (1.center) to (2.center);
		\draw [bend right=90, looseness=2.00] (3.center) to (4.center);
		\draw (4.center) to (5.center);
		\draw (3.center) to (10);
		\draw (10) to (2.center);
	\end{pgfonlayer}
\end{tikzpicture} \]
%%%%%%%%%%%%%%%%%%%%%%%%%%%%%%%%%%%%%%%%%%%%%%%%%%%%%%%%%%%%%%%%%%%%%%

\subsection{Conjugation}
\label{Sec: conjugation}

Recall the following structure from Egger \cite{Egg11}:

\begin{definition}
A {\bf conjugation} for a monoidal category $(X, \otimes, I)$ consists of a functor $\overline{(\_)}: \X^\rev \to \X$ 
with natural isomorphisms:
\[ \bar{A} \otimes \bar{B} \to^{\chi} \bar{B \otimes A} ~~~~~~~~~~~~~ \bar{\bar{A}} \to^{\varepsilon} A  \]
called respectively the (tensor reversing) {\bf conjugating laxor} and the {\bf conjugator}
such that 
\[\bar{\bar{\bar{A}}} \to^{\bar{\varepsilon_A} = \varepsilon_{\bar{A}} } \bar{A} \] 
and  
\[
\xymatrix{
(\bar{A} \ox \bar{B}) \ox \bar{C} \ar[rr]^{a_\ox} \ar[d]_{\chi \ox 1}  \ar@{}[ddrr]|{\bf [CF.1]_\ox}  
& & \bar{A} \ox (\bar{B} \ox \bar{C}) \ar[d]^{1 \ox \chi} \\
\bar{(B \ox A)} \ox \bar{C} \ar[d]_{\chi} & & \bar{A} \ox \bar{(C \ox B)} \ar[d]^{\chi} \\
\bar{C \ox (B \ox A)} \ar[rr]_{\bar{a_\otimes^{-1}}} & & \bar{(C \ox B) \ox A}
} ~~~~~~~~ \xymatrix{
\bar{\bar{A}} \ox \bar{\bar{B}} \ar[rr]^{\chi} \ar[dd]_{\varepsilon \ox \varepsilon}  \ar@{}[ddrr]|
{\bf [CF.2]_\ox}  & & \bar{\bar{B} \ox \bar{A}} \ar[dd]^{\bar{\chi}} \\ 
& & \\
A \ox B & & \bar{\bar{A \ox B}} \ar[ll]^{\varepsilon}
}
\]
\end{definition}

A monoidal category is {\bf conjugative} when it has a conjugation functor.

A symmetric monoidal category, which is conjugative, is {\bf symmetric conjugative} in case it 
satisfies the additional coherence:
\[ 
\xymatrix{
\bar{A} \otimes \bar{B} \ar[d]_{c_\otimes} \ar[rr]^{\chi}  \ar@{}[drr]|{\bf [CF.3]_\ox}  &  & 
\bar{B \otimes A} \ar[d]^{\bar{c_\otimes}} \\
\bar{B} \otimes \bar{A} \ar[rr]_{\chi} &  & \bar{A \otimes B} 
}
\]

No coherences have been specified for the unit $I$ because the expected coherences are automatic:

\begin{lemma}
\label{Lemma: unit conjugate} \cite[Lemma 2.3]{Egg11}
For every conjugative monoidal category, there exists a unique isomorphism 
$I \xrightarrow{\chi^{\!\!\!\circ}} \bar{I}$ such that 
\[
\xymatrix{
I \otimes \bar{A} \ar[rr]^{\chi^{\!\!\!\circ} \otimes 1} \ar[d]_{u_\otimes}  \ar@{}[drr]|{\bf [CF.4]_\top}  
& & \bar{I} \otimes \bar{A} \ar[d]^{\chi} \\
\bar{A} \ar[rr]_{\bar{u_\otimes^{-1}}} & & \bar{A \otimes I } } ~~~~~ 
\xymatrix{ \bar{A} \ox I \ar[rr]^{1 \otimes \chi^{\!\!\!\circ}} \ar[d]_{u_\otimes}  \ar@{}[drr]|{\bf [CF.5]_\top}  & & \bar{A} 
\ox \bar{I} \ar[d]^{\chi} \\
\bar{A} \ar[rr]_{\bar{u_\otimes^{-1}}} & & \bar{I \otimes A} } ~~~~~
\xymatrix{
I \ar[rr]^{\chi^{\!\!\!\circ}} \ar@{=}[d]  \ar@{}[drr]|{\bf [CF.6]_\top}  & & \bar{I} \ar[d]^{\bar{\chi^{\!\!\!\circ}}} \\
I \ar[rr]_{\varepsilon^{-1}} & & \bar{\bar{I}}
}
\]
\end{lemma}

\begin{definition} \cite{Egg11}
A {\bf conjugative LDC} is a linearly distributive category $(\X, \ox, \top, \oa, \bot)$ together with a 
conjugating functor $\bar{(\_)}: \X \to \X$ and natural isomorphisms:
\[ \bar{A} \ox \bar{B} \xrightarrow{\chi_\ox} \bar{B \ox A} ~~~~~~~~~~~~ \bar{A \oa B} 
\xrightarrow{\chi_\oa} \bar{B} \oa \bar{A} ~~~~~~~~~~~~ \bar{\bar{A}} \xrightarrow{\varepsilon} A \] 
\end{definition}

such that $(\X, \ox, \top, \chi_\ox, \varepsilon)$ and $(\X, \oa, \bot, \chi_\oa^{-1}, \varepsilon)$ are 
conjugative (symmetric) monoidal categories with respect to the conjugating functor and  the following diagrams commute:

\[ 
\xymatrix{
\bar{B \oa C} \ox \bar{A} \ar[rr]^{\chi_\oa \ox 1} \ar[d]_{\chi_\ox}  \ar@{}[ddrr]|{\bf [CF.7]}  & & 
(\bar{C} \oa \bar{B}) \ox \bar{A} \ar[d]^{\partial} \\
\bar{(A \ox (B \oa C))} \ar[d]_{\bar{\partial}} & & \bar{C} \oa ( \bar{B} \ox \bar{A}) \ar[d]^{1 \oa \chi_\ox} \\
\bar{((A \ox B) \oa C)} \ar[rr]_{\chi_\oa} & & \bar{C} \oa \bar{A \ox B}
} ~~~~~~~~~~ 
\xymatrix{
\bar{A} \ox \bar{C \oa B} \ar[rr]^{\chi_\ox} \ar[d]_{1 \ox \chi_\oa} \ar@{}[ddrr]|{\bf [CF.8]}   & & 
\bar{(C \oa B) \ox A} \ar[d]^{\bar{\partial}} \\
\bar{A} \ox (\bar{B} \oa \bar{C}) \ar[d]_{\partial} & & \bar{C \oa (B \ox A)} \ar[d]^{\chi_\oa} \\
(\bar{A} \ox \bar{B}) \oa \bar{C} \ar[rr]_{\chi_\ox \oa 1} & & \bar{(B \ox A)} \oa \bar{C}
} 
\]

Note, by Lemma \ref{Lemma: unit conjugate}, there exists canonical isomorphisms 
$\top \xrightarrow{\chi^{\!\!\!\circ}_\top} \bar{\top}$ and $\bot \xrightarrow{\chi^{\!\!\!\circ}_\bot} \bar{\bot}$, 
hence conjugation is a normal functor.  However, the conjugation is not necessarily a mix functor when $\X$ is a mix category.  
 For conjugation to be a mix functor, the following extra condition must be satisfied:
\[ {\bf \small [CF.9]}~~~~~ \xymatrix{\overline{\bot} \ar[rd]_{(\chi^{\!\!\!\circ}_\bot)^{-1}} \ar@/^/[rrr]^{\overline{{\sf m}}} 
& & & \overline{\top} \\ & \bot \ar[r]_{{\sf m}} & \top \ar[ur]_{\chi^{\!\!\!\circ}_\top} } \]


\begin{proposition}
A conjugative LDC is precisely a LDC, $\X$, with a Frobenius adjoint $(\epsilon^{-1}, \epsilon): 
\overline{(\_)} \dashv \overline{(\_)}^\rev: \X^\rev \to \X$ where $\epsilon := (\varepsilon,\varepsilon^{-1})$.  
Furthermore, if $\X$ is an isomix category and conjugation is a mix functor then conjugation is an isomix equivalence.
\end{proposition}

\begin{proof}
It is clear that $\overline{(\_)}$ is a strong Frobenius functor so being mix implies isomix.   
Also, $\varepsilon$ is clearly monoidal for tensor and par.  The triangle equalities give $ \overline{\varepsilon^{-1}} 
\varepsilon = 1: \overline{A} \to \overline{A}$ thus $\varepsilon = \overline{\varepsilon}$.
\end{proof}


Clearly conjugation should flip left duals into right duals:

\begin{lemma}
\label{Lemma: involutive linear adjoint}
If $B \dashvv A$ is a linear dual then $\bar{A} \dashvv \bar{B}$ is a linear dual.
\end{lemma}

\begin{proof}
Suppose $(\eta, \varepsilon): B \dashvv A$. Then, $(\chi^{\!\!\!\circ}_\top \bar{\eta} \chi_\oa,\chi_\ox \bar{\varepsilon} 
\chi^{\!\!\!\circ}_\bot): \bar{A} \dashvv \bar{B}$.
\end{proof}

When a $*$-autonomous category is cyclic one expects that conjugation should interact with the cyclor in a coherent fashion:

\begin{definition}\cite{EggMcCurd12}
\label{Defn: conjugative cyclic}
A {\bf conjugative cyclic $*$-autonomous category} is a conjugative $*$-autonomous category together 
with a cyclor $A^* \to^{\psi} \!\!~^{*}\!A$ such that 
\[
\xymatrix{
(\bar{A})^* \ar[rr]^{\psi} \ar[d]_{\simeq}  & & ^{*}(\bar{A}) \ar[d]^{\simeq} \\
\bar{(^{*}A)} \ar[rr]_{\bar{\psi^{-1}}} & & \bar{(A^*)}
}
\]
which gives a map $\sigma: (\overline{A})^{*} \to \overline{(A^{*})}$.
\end{definition}

The above condition is drawn as follows:
\[ \sigma =
\begin{tikzpicture} %cca1
	\begin{pgfonlayer}{nodelayer}
		\node [style=none] (0) at (-1, 0.75) {$\overline{\psi^{-1}}$};
		\node [style=circle, scale=2.5] (0) at (-1, 0.75) {};
		\node [style=none] (1) at (-1, -0.5) {};
		\node [style=none] (2) at (-1, 2.25) {};
		\node [style=none] (3) at (0, 2.25) {};
		\node [style=none] (4) at (0, 1) {};
		\node [style=none] (5) at (1, 1) {};
		\node [style=none] (6) at (1, 3) {};
		\node [style=none] (7) at (1.5, 2.75) {$(\overline{A})^*$};
		\node [style=none] (8) at (0.5, 0.2) {$\epsilon^*$};
		\node [style=none] (9) at (-0.5, 3.1) {$\overline{~^*\eta}$};
		\node [style=none] (10) at (-1.5, -0.25) {$\overline{A^*}$};
		\node [style=none] (12) at (-0.25, 1.5) {$\overline{A}$};
		\node [style=none] (11) at (-1.5, 1.75) {$\overline{~^*A}$};
	\end{pgfonlayer}
	\begin{pgfonlayer}{edgelayer}
		\draw (2.center) to (0);
		\draw (0) to (1.center);
		\draw [bend left=90, looseness=2.00] (2.center) to (3.center);
		\draw (3.center) to (4.center);
		\draw [bend right=90, looseness=2.00] (4.center) to (5.center);
		\draw (5.center) to (6.center);
	\end{pgfonlayer}
\end{tikzpicture} = \begin{tikzpicture} %cca1
	\begin{pgfonlayer}{nodelayer}
		\node [style=circle] (0) at (-1, 2.5) {$\psi$};
		\node [style=none] (1) at (-1, 3.25) {};
		\node [style=none] (2) at (-1, 0.5) {};
		\node [style=none] (3) at (0, 0.5) {};
		\node [style=none] (4) at (0, 1.75) {};
		\node [style=none] (5) at (1, 1.75) {};
		\node [style=none] (6) at (1, -0.25) {};
		\node [style=none] (7) at (-1.75, 3.25) {$(\overline{A})^*$};
		\node [style=none] (8) at (0.5, 2.6) {$\overline{\eta^*}$};
		\node [style=none] (9) at (-0.5, -0.35) {$~^*\epsilon$};
		\node [style=none] (10) at (1.5, 0.25) {$\overline{A^*}$};
		\node [style=none] (11) at (-1.75, 1.5) {$~^*(\overline{A})$};
		\node [style=none] (12) at (-0.25, 1.25) {$\overline{A}$};
	\end{pgfonlayer}
	\begin{pgfonlayer}{edgelayer}
		\draw (2.center) to (0);
		\draw (0) to (1.center);
		\draw [bend right=90, looseness=2.00] (2.center) to (3.center);
		\draw (3.center) to (4.center);
		\draw [bend left=90, looseness=2.00] (4.center) to (5.center);
		\draw (5.center) to (6.center);
	\end{pgfonlayer}
\end{tikzpicture}
\]

When the $*$-autonomous category is symmetric, conjugation automatically preserves the canonical cyclor.

\begin{lemma}
\label{Lemma: varepsi monoidal}
In a conjugative $*$-autonomous category, 
\[ %epsi1 
\begin{tikzpicture} 
	\begin{pgfonlayer}{nodelayer}
		\node [style=circle] (0) at (0, 1) {$\varepsilon$};
		\node [style=none] (1) at (0, -0) {};
		\node [style=none] (2) at (-1.75, -0) {};
		\node [style=none] (3) at (-1.75, 2) {};
		\node [style=none] (4) at (0, 2) {};
		\node [style=none] (5) at (-1, 3) {$\overline{\overline{\eta*}}$};
		\node [style=none] (6) at (-2.25, 0.3) {$\overline{\overline{X^*}}$};
		\node [style=none] (7) at (0.5, 1.7) {$\overline{\overline{X}}$};
		\node [style=none] (8) at (0.5, 0.25) {$X$};
	\end{pgfonlayer}
	\begin{pgfonlayer}{edgelayer}
		\draw (4.center) to (0);
		\draw (0) to (1.center);
		\draw [bend left=90, looseness=1.50] (3.center) to (4.center);
		\draw (3.center) to (2.center);
	\end{pgfonlayer}
\end{tikzpicture} = %epsi2
\begin{tikzpicture}
	\begin{pgfonlayer}{nodelayer}
		\node [style=circle] (0) at (-1.75, 1) {$\varepsilon^{-1}$};
		\node [style=none] (1) at (-1.75, -0) {};
		\node [style=none] (2) at (0, -0) {};
		\node [style=none] (3) at (0, 2) {};
		\node [style=none] (4) at (-1.75, 2) {};
		\node [style=none] (5) at (-1, 3) {$\eta*$};
		\node [style=none] (6) at (-2.5, 0.3) {$\overline{\overline{X^*}}$};
		\node [style=none] (7) at (0.5, 0.5) {$X$};
		\node [style=none] (8) at (-2.25, 1.7) {$X^*$};
	\end{pgfonlayer}
	\begin{pgfonlayer}{edgelayer}
		\draw (4.center) to (0);
		\draw (0) to (1.center);
		\draw [bend right=90, looseness=1.50] (3.center) to (4.center);
		\draw (3.center) to (2.center);
	\end{pgfonlayer}
\end{tikzpicture}
\]
\[
\chi_\top^{\!\!\!\circ}~ \overline{\chi_\top^{\!\!\!\circ}} ~ \overline{\overline{\eta*}} ~ \overline{\chi_\oa}^{-1} \chi_\oa^{-1} (1 \oa \varepsilon) = \eta* (\varepsilon^{-1} \oa 1)  : \top \to \overline{\overline{X^*}} \oa X 
\]
\end{lemma}
\begin{proof}
\begin{align*}
\chi_\top^{\!\!\!\circ} ~ \overline{\chi_\top^{\!\!\!\circ}} ~ \overline{\overline{\eta*}} ~ \overline{\chi}^{-1} \chi^{-1} (1 \oa \varepsilon) &=  \chi_\top^{\!\!\!\circ} ~ \overline{\chi_\top^{\!\!\!\circ}} ~ \overline{\overline{\eta*}} ~ \overline{\chi}^{-1} \chi^{-1} (\varepsilon \varepsilon^{-1} \oa \varepsilon) \\
&=\chi_\top^{\!\!\!\circ} ~ \overline{\chi_\top^{\!\!\!\circ}}  ~ \overline{\overline{\eta*}} ~ \overline{\chi}^{-1} \chi^{-1} (\varepsilon  \oa \varepsilon) (\varepsilon^{-1} \oa 1) \\
&\stackrel{{\bf \small [CF.2]_\oa}}{=} \chi^{\!\!\!\circ} ~ \overline{\chi_\top^{\!\!\!\circ}} ~ \overline{\overline{\eta*}} \varepsilon (\varepsilon^{-1} \oa 1) \\
&\stackrel{{\bf \small nat.}}{=} \chi_\top^{\!\!\!\circ} ~ \overline{\chi_\top^{\!\!\!\circ}} ~  \varepsilon \eta\!* (\varepsilon^{-1} \oa 1) \\
&\stackrel{{\bf \small [CF.6]_\top}}{=} \eta\!* (\varepsilon^{-1} \oa 1)
\end{align*}
\end{proof}

%%%%%%%%%%%%%%%%%%%%%%%%%%%%%%%%%%%%%%%%%%%%%%%%%%%%%%%%%%%%%%%%%%%%%%
\subsection{Dagger and conjugation}
The interaction of the dagger and conjugation for cyclic $*$-autonomous categories in the presence of the dualizing functor is 
illustrated by the following diagram:
\[
\xymatrixcolsep{5pc}
\xymatrixrowsep{5pc}
\xymatrix{
\X^{\op} \ar@/^1pc/[rr]^{(\_)^\dagger} \ar@/^1pc/[dr]|{((\_)^*)^\rev} \ar@{}[rr]|{\bot} &~ \ar@{}[d]|{\cong} & 
\X  \ar@/^1pc/[ll]|{((\_)^{\dagger})^{\op}} \ar@/_1pc/[dl]|{\overline{(\_)}^\rev} \ar@{}[dl]|{\dashv} \\
 & \X^\rev \ar@/^1pc/[ul]^{(\_)^{*^{\op}}} \ar@/_1pc/[ur]_{\overline{(\_)}} \ar@{}[ul]|{\dashv} &
 }
\]

Specifically we have: 

\begin{proposition}
	\label{Prop: dagger+dualizing}
Every cyclic $\dagger$-$*$-autonomous category is a conjugative $*$-autonomous category.
\end{proposition}
\begin{proof}
Let $\X$ be a cyclic, $\dagger$-$*$-autonomous category. Then composing adjoints gives the equivalence  $(\_)^{\dagger^*} 
\dashv (\_)^{*^\dagger}$. To build a conjugation, however,
we need an equivalence between the same functors: to obtain such an equivalence we use the natural equivalence $\omega: 
(\_)^{\dagger*} \to (\_)^{*\dagger} $ from the cyclor preserving condition for Frobenius linear functors.  
A conjugative equivalence, in addition, requires that the unit and counit of the equivalence be inverses of each of other.  
The unit and counit of the equivalence are given by {\em (a)} and {\em (b)} respectively;
\[ \mbox{\em (a)}~~~~~~~~~  \xymatrix{
\X^{\sf rev} \ar@{=}[rrrr] \ar[dr]_{(\_)^{*^\op}} & 
&  
\ar@{}[d]|{\Downarrow ~ \eta_\ox'}&  
& 
\X^{\sf rev} \ar@{<-}[ld]^{(\_)^{*^\op}}  
& \\
&
 \X^{\op} \ar@{=}[rr] \ar[dr]_{\dagger} &  
 \ar@{}[d]|{\Downarrow ~ \iota^{-1}} & 
 \X^{\sf op} \ar@{<-}[ld]^{\dagger^\op} \ar@{}[rr]_{\omega}  \ar@{}[rr]^{\Longrightarrow}  & 
 & 
 \X^{\sf oprev} \ar@{<-}@/^1pc/[llld]^{(\_)^{*^{\sf oprev}}} \ar@/_1pc/[ul]_{\dagger^\rev}  \\
& & \X & &
}
\]

\[
\mbox{\em (b)}~~~~~~~~~~~~~~ \xymatrix{
& & &
\X^{\rev}  \ar@{<-}@/_1pc/[llld]_{\dagger^\rev} \ar@{<-}[dl]_{(\_)^{*^{\rev}}} \ar[dr]^{(\_)^{*^{\sf op}}} \ar@{}[d]|{\Downarrow ~ \epsilon'_\ox} & &\\
\X^{\sf op rev} \ar@{<-}[rd]_{(\_)^{*^{\sf oprev}}} \ar@{}[rr]_{\omega^{-1}}  \ar@{}[rr]^{\Longrightarrow}  &  & 
\X^{\sf op} \ar@{=}[rr] \ar@{<-}[dl]^{\dagger^\op} & \ar@{}[d]|{\Downarrow ~ \iota^{-1}} &
\X^{\sf op} \ar[dr]^{\dagger} \\
& \X \ar@{=}[rrrr]&  & & & \X }
\] where the isomorphism $\omega: (\_)^{\dagger*} \to (\_)^{*\dagger}$ is from the cyclor preserving condition, {\bf [CFF]}, for Frobenius linear functors:
\[
\omega := \begin{tikzpicture}
	\begin{pgfonlayer}{nodelayer}
		\node [style=none] (0) at (0, 5) {};
		\node [style=none] (1) at (0, 3) {};
		\node [style=none] (2) at (-1, 3) {};
		\node [style=none] (3) at (-1, 4) {};
		\node [style=none] (4) at (-2, 4) {};
		\node [style=circle, scale=2.5] (5) at (-2, 2.25) {};
		\node [style=none] (6) at (-2, 1) {};
		\node [style=none] (7) at (-2, 2.25) {$\psi^\dagger$};
		\node [style=none] (8) at (0.5, 4.5) {$X^{\dagger*}$};
		\node [style=none] (9) at (-2.5, 1.5) {$X^{*\dagger}$};
		\node [style=none] (10) at (-0.5, 2.25) {$\epsilon*$};
		\node [style=none] (11) at (-1.5, 4.75) {$(*\epsilon)^\dagger$};
	\end{pgfonlayer}
	\begin{pgfonlayer}{edgelayer}
		\draw (0.center) to (1.center);
		\draw [bend left=90, looseness=2.00] (1.center) to (2.center);
		\draw (2.center) to (3.center);
		\draw [bend right=90, looseness=1.75] (3.center) to (4.center);
		\draw (4.center) to (5);
		\draw (5) to (6.center);
	\end{pgfonlayer}
\end{tikzpicture} ~~~~~~~~~~~~~ \omega^{-1} := \begin{tikzpicture}
	\begin{pgfonlayer}{nodelayer}
		\node [style=none] (0) at (-2, 1) {};
		\node [style=none] (1) at (-2, 3) {};
		\node [style=none] (2) at (-1, 3) {};
		\node [style=none] (3) at (-1, 2) {};
		\node [style=none] (4) at (0, 2) {};
		\node [style=circle, scale=2.5] (5) at (0, 3.75) {};
		\node [style=none] (6) at (0, 5) {};
		\node [style=none] (7) at (0, 3.75) {$\psi^{-1 \dagger}$};
		\node [style=none] (8) at (-2.5, 1.5) {$X^{\dagger*}$};
		\node [style=none] (9) at (0.5, 4.5) {$X^{*\dagger}$};
		\node [style=none] (10) at (-1.5, 3.75) {$\eta*$};
		\node [style=none] (11) at (-0.5, 1.25) {$(*\eta^{-1})^\dagger$};
	\end{pgfonlayer}
	\begin{pgfonlayer}{edgelayer}
		\draw (0.center) to (1.center);
		\draw [bend left=90, looseness=2.00] (1.center) to (2.center);
		\draw (2.center) to (3.center);
		\draw [bend right=90, looseness=1.75] (3.center) to (4.center);
		\draw (4.center) to (5);
		\draw (5) to (6.center);
	\end{pgfonlayer}
\end{tikzpicture}
\]


It remains to show that the unit and the counit maps are inverses of each other in $\X$:

\[ (a) ~~~~~ %dbig2
 \begin{tikzpicture}
	\begin{pgfonlayer}{nodelayer}
		\node [style=none] (0) at (-2, 5) {};
		\node [style=none] (1) at (-2, 3) {};
		\node [style=none] (2) at (-1, 3) {};
		\node [style=none] (3) at (-1, 4) {};
		\node [style=none] (4) at (0, 4) {};
		\node [style=circle, scale=2.5] (5) at (0, 3) {};
		\node [style=none] (6) at (0, 2) {};
		\node [style=none] (7) at (0, 3) {$\psi^{-1}$};
		\node [style=none] (8) at (-1.5, 2.25) {$\epsilon*$};
		\node [style=none] (9) at (-0.5, 4.75) {$*\eta$};
		\node [style=none] (10) at (0, 2) {};
		\node [style=none] (11) at (-1.75, 1.75) {$\eta*$};
		\node [style=none] (12) at (-1, -1) {};
		\node [style=none] (13) at (-2.5, 0.75) {};
		\node [style=none] (14) at (-2.5, -1.25) {};
		\node [style=none] (15) at (-1, 0.75) {};
		\node [style=none] (16) at (-0.5, -1.75) {$\epsilon*$};
		\node [style=none] (17) at (0, -1) {};
		\node [style=none] (18) at (-3.5, -1.25) {};
		\node [style=none] (19) at (-3.5, -0.25) {};
		\node [style=none] (20) at (-4.5, -0.25) {};
		\node [style=circle, scale=2.5] (21) at (-4.5, -1.25) {};
		\node [style=none] (22) at (-4.5, -2.25) {};
		\node [style=none] (23) at (-4.5, -1.25) {$\psi^\dagger$};
		\node [style=circle, scale=2.5] (24) at (-1, -0) {};
		\node [style=none] (25) at (-1, -0) {$\iota^{-1}$};
		\node [style=none] (26) at (-5.5, -2) {$X^{*\dagger*\dagger}$};
		\node [style=none] (27) at (-4, 0.5) {$(*\epsilon)^\dagger$};
		\node [style=none] (28) at (-3, -2) {$\epsilon*$};
		\node [style=none] (29) at (-1.75, 0.5) {$(X^*)^{\dagger \dagger}$};
		\node [style=none] (30) at (-1.5, -0.75) {$X^*$};
		\node [style=none] (31) at (0.5, 0.5) {$X^{**}$};
		\node [style=none] (32) at (-2.5, 4.5) {$X$};
	\end{pgfonlayer}
	\begin{pgfonlayer}{edgelayer}
		\draw (0.center) to (1.center);
		\draw [bend right=90, looseness=2.00] (1.center) to (2.center);
		\draw (2.center) to (3.center);
		\draw [bend left=90, looseness=1.75] (3.center) to (4.center);
		\draw (4.center) to (5);
		\draw (5) to (6.center);
		\draw (10.center) to (17.center);
		\draw [bend left=90, looseness=2.00] (17.center) to (12.center);
		\draw [bend right=90, looseness=1.75] (15.center) to (13.center);
		\draw (14.center) to (13.center);
		\draw [bend right=90, looseness=1.50] (18.center) to (14.center);
		\draw (18.center) to (19.center);
		\draw [bend right=90, looseness=1.50] (19.center) to (20.center);
		\draw (20.center) to (21);
		\draw (21) to (22.center);
		\draw (15.center) to (24);
		\draw (24) to (12.center);
	\end{pgfonlayer}
\end{tikzpicture} = %dbig2
\begin{tikzpicture}
	\begin{pgfonlayer}{nodelayer}
		\node [style=none] (0) at (-2, 5) {};
		\node [style=none] (1) at (-2, 3) {};
		\node [style=none] (2) at (-1, 3) {};
		\node [style=none] (3) at (-1, 4) {};
		\node [style=none] (4) at (0, 4) {};
		\node [style=circle, scale=2.5] (5) at (0, 3) {};
		\node [style=none] (6) at (0, 2) {};
		\node [style=none] (7) at (0, 3) {$\psi^{-1}$};
		\node [style=none] (8) at (-1.5, 2.25) {$\epsilon*$};
		\node [style=none] (9) at (-0.5, 4.75) {$*\eta$};
		\node [style=none] (10) at (0, 2) {};
		\node [style=none] (11) at (-1, -1) {};
		\node [style=none] (12) at (-2.5, 0.75) {};
		\node [style=none] (13) at (-1, 0.75) {};
		\node [style=none] (14) at (-0.5, -1.75) {$\epsilon*$};
		\node [style=none] (15) at (0, -1) {};
		\node [style=circle, scale=2.5] (16) at (-2.5, -1) {};
		\node [style=none] (17) at (-2.5, -2) {};
		\node [style=none] (18) at (-2.5, -1) {$\psi^\dagger$};
		\node [style=circle, scale=2.5] (19) at (-1, -0) {};
		\node [style=none] (20) at (-1, -0) {$\iota^{-1}$};
		\node [style=none] (21) at (-3.5, -1.75) {$X^{*\dagger*\dagger}$};
		\node [style=none] (22) at (-2, 1.75) {$(*\epsilon)^\dagger$};
		\node [style=none] (23) at (-1.75, 0.5) {$(X^*)^{\dagger \dagger}$};
		\node [style=none] (24) at (-1.5, -0.75) {$X^*$};
		\node [style=none] (25) at (0.5, 0.5) {$X^{**}$};
		\node [style=none] (26) at (-2.5, 4.5) {$X$};
	\end{pgfonlayer}
	\begin{pgfonlayer}{edgelayer}
		\draw (0.center) to (1.center);
		\draw [bend right=90, looseness=2.00] (1.center) to (2.center);
		\draw (2.center) to (3.center);
		\draw [bend left=90, looseness=1.75] (3.center) to (4.center);
		\draw (4.center) to (5);
		\draw (5) to (6.center);
		\draw (10.center) to (15.center);
		\draw [bend left=90, looseness=2.00] (15.center) to (11.center);
		\draw [bend right=90, looseness=1.75] (13.center) to (12.center);
		\draw (16) to (17.center);
		\draw (13.center) to (19);
		\draw (19) to (11.center);
		\draw (12.center) to (16);
	\end{pgfonlayer}
\end{tikzpicture} = %dbig3
\begin{tikzpicture}
	\begin{pgfonlayer}{nodelayer}
		\node [style=none] (0) at (2, 4.75) {};
		\node [style=none] (1) at (2, 2.75) {};
		\node [style=none] (2) at (1, 2.75) {};
		\node [style=none] (3) at (1, 3.75) {};
		\node [style=none] (4) at (0, 3.75) {};
		\node [style=none] (5) at (0, 2) {};
		\node [style=none] (6) at (1.5, 2) {$*\epsilon$};
		\node [style=none] (7) at (0.5, 4.5) {$\eta*$};
		\node [style=none] (8) at (0, 2) {};
		\node [style=none] (9) at (-1, -1) {};
		\node [style=none] (10) at (-2.5, 0.75) {};
		\node [style=none] (11) at (-1, 0.75) {};
		\node [style=none] (12) at (-0.5, -1.75) {$\epsilon*$};
		\node [style=none] (13) at (0, -1) {};
		\node [style=circle, scale=2.5] (14) at (-2.5, -1) {};
		\node [style=none] (15) at (-2.5, -2) {};
		\node [style=none] (16) at (-2.5, -1) {$\psi^{\dagger}$};
		\node [style=circle, scale=2.5] (17) at (-1, -0) {};
		\node [style=none] (18) at (-1, -0) {$\iota^{-1}$};
		\node [style=none] (19) at (-3.5, -1.75) {$X^{*\dagger*\dagger}$};
		\node [style=none] (20) at (-2, 1.75) {$(*\epsilon)^\dagger$};
		\node [style=none] (21) at (-1.75, 0.5) {$(X^*)^{\dagger \dagger}$};
		\node [style=none] (22) at (-1.5, -0.75) {$X^*$};
		\node [style=none] (23) at (0.5, 0.5) {$X^{**}$};
		\node [style=none] (24) at (2.5, 4.25) {$X$};
		\node [style=circle, scale=2.5] (25) at (1, 3.25) {};
		\node [style=none] (26) at (1, 3.25) {$\psi$};
	\end{pgfonlayer}
	\begin{pgfonlayer}{edgelayer}
		\draw (0.center) to (1.center);
		\draw [bend left=90, looseness=2.00] (1.center) to (2.center);
		\draw [bend right=90, looseness=1.75] (3.center) to (4.center);
		\draw (8.center) to (13.center);
		\draw [bend left=90, looseness=2.00] (13.center) to (9.center);
		\draw [bend right=90, looseness=1.75] (11.center) to (10.center);
		\draw (14) to (15.center);
		\draw (11.center) to (17);
		\draw (17) to (9.center);
		\draw (10.center) to (14);
		\draw (4.center) to (5.center);
		\draw (3.center) to (25);
		\draw (25) to (2.center);
	\end{pgfonlayer}
\end{tikzpicture} = %dbig4
\begin{tikzpicture}
	\begin{pgfonlayer}{nodelayer}
		\node [style=none] (0) at (0, 4.5) {};
		\node [style=none] (1) at (0, -0.5) {};
		\node [style=none] (2) at (-1, -0.5) {};
		\node [style=none] (3) at (-1, 1.5) {};
		\node [style=none] (4) at (-0.5, -1.25) {$*\epsilon$};
		\node [style=none] (5) at (-1, 1.5) {};
		\node [style=none] (6) at (-2.5, 3.25) {};
		\node [style=none] (7) at (-1, 3.25) {};
		\node [style=circle, scale=2.5] (8) at (-2.5, 0.75) {};
		\node [style=none] (9) at (-2.5, -2) {};
		\node [style=none] (10) at (-2.5, 0.75) {$\psi^{\dagger}$};
		\node [style=circle, scale=2.5] (11) at (-1, 2.5) {};
		\node [style=none] (12) at (-1, 2.5) {$\iota^{-1}$};
		\node [style=none] (13) at (-3.25, -1.75) {$X^{*\dagger*\dagger}$};
		\node [style=none] (14) at (-2, 4.25) {$(*\epsilon)^\dagger$};
		\node [style=none] (15) at (-1.75, 3) {$X^{*^{\dagger \dagger}}$};
		\node [style=none] (16) at (-1.5, 1.75) {$X^*$};
		\node [style=none] (17) at (0.5, 4) {$X$};
		\node [style=circle, scale=2.5] (18) at (-1, 0.75) {};
		\node [style=none] (19) at (-1, 0.75) {$\psi$};
		\node [style=none] (20) at (-1.5, -0) {$~^*X$};
	\end{pgfonlayer}
	\begin{pgfonlayer}{edgelayer}
		\draw (0.center) to (1.center);
		\draw [bend left=90, looseness=2.00] (1.center) to (2.center);
		\draw [bend right=90, looseness=1.75] (7.center) to (6.center);
		\draw (8) to (9.center);
		\draw (7.center) to (11);
		\draw (11) to (5.center);
		\draw (6.center) to (8);
		\draw (3.center) to (18);
		\draw (18) to (2.center);
	\end{pgfonlayer}
\end{tikzpicture} = %dbig5
\begin{tikzpicture}
	\begin{pgfonlayer}{nodelayer}
		\node [style=none] (0) at (0, 4.5) {};
		\node [style=none] (1) at (0, -0.5) {};
		\node [style=none] (2) at (-1, -0.5) {};
		\node [style=none] (3) at (-1, 1.5) {};
		\node [style=none] (4) at (-0.5, -1.25) {$*\epsilon$};
		\node [style=none] (5) at (-1, 1.5) {};
		\node [style=none] (6) at (-2.5, 3.25) {};
		\node [style=none] (7) at (-1, 3.25) {};
		\node [style=circle, scale=2.5] (8) at (-2.5, 0.75) {};
		\node [style=none] (9) at (-2.5, -2) {};
		\node [style=none] (10) at (-2.5, 0.75) {$\psi^{\dagger}$};
		\node [style=circle, scale=2.5] (11) at (-1, 2.5) {};
		\node [style=none] (12) at (-1, 2.5) {$\psi^{\dagger \dagger}$};
		\node [style=none] (13) at (-3, -1.75) {$X^{*\dagger*\dagger}$};
		\node [style=none] (14) at (-2, 4.25) {$(*\epsilon)^\dagger$};
		\node [style=none] (15) at (-1.75, 3) {$(X^*)^{\dagger \dagger}$};
		\node [style=none] (16) at (-1.75, 1.75) {$(~^*X)^{\dagger \dagger}$};
		\node [style=none] (17) at (0.25, 4) {$X$};
		\node [style=circle, scale=2.5] (18) at (-1, 0.75) {};
		\node [style=none] (19) at (-1, 0.75) {$\iota^{-1}$};
		\node [style=none] (20) at (-1.5, -0) {$~^*X$};
	\end{pgfonlayer}
	\begin{pgfonlayer}{edgelayer}
		\draw (0.center) to (1.center);
		\draw [bend left=90, looseness=2.00] (1.center) to (2.center);
		\draw [bend right=90, looseness=1.75] (7.center) to (6.center);
		\draw (8) to (9.center);
		\draw (7.center) to (11);
		\draw (11) to (5.center);
		\draw (6.center) to (8);
		\draw (3.center) to (18);
		\draw (18) to (2.center);
	\end{pgfonlayer}
\end{tikzpicture} =
\]
\[
= %dbig6
\begin{tikzpicture}
	\begin{pgfonlayer}{nodelayer}
		\node [style=none] (0) at (0.5, 4.5) {};
		\node [style=none] (1) at (0.5, -0.5) {};
		\node [style=none] (2) at (-1, -0.5) {};
		\node [style=none] (3) at (-2.5, 3.25) {};
		\node [style=none] (4) at (-1, 3.25) {};
		\node [style=circle, scale=2.5] (5) at (-2.5, 0.75) {};
		\node [style=none] (6) at (-2.5, -2) {};
		\node [style=none] (7) at (-2.5, 0.75) {$\psi^{\dagger}$};
		\node [style=circle, scale=2.5] (8) at (-1, 1.75) {};
		\node [style=none] (9) at (-1, 1.75) {$\psi^{\dagger \dagger}$};
		\node [style=none] (10) at (-3, -1.75) {$X^{*\dagger*\dagger}$};
		\node [style=none] (11) at (-2, 4.25) {$(*\epsilon)^\dagger$};
		\node [style=none] (12) at (-1.75, 3) {$(X^*)^{\dagger \dagger}$};
		\node [style=none] (13) at (-1.5, -0) {$(~^*X)^{\dagger \dagger}$};
		\node [style=none] (14) at (0.75, 4) {$X$};
		\node [style=circle, scale=2] (15) at (0.5, 1.5) {};
		\node [style=none] (16) at (0.5, 1.5) {$\iota$};
		\node [style=none] (17) at (-0.5, -1.75) {$*\epsilon^{\dagger \dagger}$};
	\end{pgfonlayer}
	\begin{pgfonlayer}{edgelayer}
		\draw [bend left=90, looseness=2.00] (1.center) to (2.center);
		\draw [bend right=90, looseness=1.75] (4.center) to (3.center);
		\draw (5) to (6.center);
		\draw (4.center) to (8);
		\draw (3.center) to (5);
		\draw (0.center) to (15);
		\draw (15) to (1.center);
		\draw (8) to (2.center);
	\end{pgfonlayer}
\end{tikzpicture} \stackrel{*}{=} \begin{tikzpicture}
	\begin{pgfonlayer}{nodelayer}
		\node [style=none] (0) at (0.5, 6.25) {};
		\node [style=none] (1) at (0.5, 2) {};
		\node [style=none] (2) at (-1, 2) {};
		\node [style=none] (3) at (-2.5, 0.25) {};
		\node [style=none] (4) at (-1, 0.25) {};
		\node [style=circle, scale=2.5] (5) at (-2.5, 1) {};
		\node [style=none] (6) at (-2.5, -2.75) {};
		\node [style=none] (7) at (-2.5, 1) {$\psi$};
		\node [style=circle, scale=2.5] (8) at (-1, 1) {};
		\node [style=none] (9) at (-1, 1) {$\psi^\dagger$};
		\node [style=none] (10) at (-3, -2.5) {$X^{*\dagger*\dagger}$};
		\node [style=none] (11) at (-2, -0.75) {$*\epsilon$};
		\node [style=none] (12) at (-1.75, 0.25) {$(X^*)^{\dagger}$};
		\node [style=none] (13) at (-1.75, 1.75) {$(~^*X)^{\dagger}$};
		\node [style=none] (14) at (0.75, 5.75) {$X$};
		\node [style=circle, scale=2.5] (15) at (0.5, 5) {};
		\node [style=none] (16) at (0.5, 5) {$\iota$};
		\node [style=none] (17) at (-0.5, 3.25) {$(*\epsilon)^\dagger$};
		\node [style=none] (18) at (-3.75, 4) {};
		\node [style=none] (19) at (1.75, 4) {};
		\node [style=none] (20) at (1.75, -1.5) {};
		\node [style=none] (21) at (-3.75, -1.5) {};
		\node [style=none] (22) at (-2.5, 4) {};
		\node [style=none] (23) at (0.5, -1.5) {};
		\node [style=none] (24) at (1.5, -1.25) {$\dagger$};
		\node [style=none] (25) at (0.5, 4) {};
		\node [style=none] (26) at (-2.5, -1.5) {};
		\node [style=none] (27) at (-3.25, 0.25) {$~^*(X^{*\dagger})$};
		\node [style=none] (28) at (-3, 2.5) {$X^{*\dagger*}$};
		\node [style=none] (29) at (0, -1) {$X^{\dagger}$};
	\end{pgfonlayer}
	\begin{pgfonlayer}{edgelayer}
		\draw [bend right=90, looseness=2.00] (1.center) to (2.center);
		\draw [bend left=90, looseness=1.75] (4.center) to (3.center);
		\draw (4.center) to (8);
		\draw (3.center) to (5);
		\draw (0.center) to (15);
		\draw (8) to (2.center);
		\draw (1.center) to (23.center);
		\draw (5) to (22.center);
		\draw (18.center) to (19.center);
		\draw (19.center) to (20.center);
		\draw (20.center) to (21.center);
		\draw (21.center) to (18.center);
		\draw (15) to (25.center);
		\draw (26.center) to (6.center);
	\end{pgfonlayer}
\end{tikzpicture} \stackrel{\ref{Lemma:  cyclic dagger}}{=} %big8
\begin{tikzpicture}
	\begin{pgfonlayer}{nodelayer}
		\node [style=none] (0) at (-0.75, 6.25) {};
		\node [style=none] (1) at (-2.5, 2) {};
		\node [style=none] (2) at (-1, 2) {};
		\node [style=none] (3) at (0.5, 0.25) {};
		\node [style=none] (4) at (-1, 0.25) {};
		\node [style=none] (5) at (-1, -3.25) {};
		\node [style=none] (6) at (-1.75, -2.5) {$X^{*\dagger*\dagger}$};
		\node [style=none] (7) at (-0.5, 5.75) {$X$};
		\node [style=circle, scale=2.5] (8) at (-0.75, 5) {};
		\node [style=none] (9) at (-0.75, 5) {$\iota$};
		\node [style=none] (10) at (-1.5, 3.25) {$(*\epsilon)^\dagger$};
		\node [style=none] (11) at (-3.75, 4) {};
		\node [style=none] (12) at (1.75, 4) {};
		\node [style=none] (13) at (1.75, -1.5) {};
		\node [style=none] (14) at (-3.75, -1.5) {};
		\node [style=none] (15) at (0.5, 4) {};
		\node [style=none] (16) at (-2.5, -1.5) {};
		\node [style=none] (17) at (1.5, -1.25) {$\dagger$};
		\node [style=none] (18) at (-0.75, 4) {};
		\node [style=none] (19) at (-1, -1.5) {};
		\node [style=none] (20) at (-0.25, 4.5) {$X^{\dagger}$};
		\node [style=none] (21) at (-3, -1) {$X^\dagger$};
		\node [style=none] (22) at (1, 3.5) {$X^{*\dagger*}$};
		\node [style=none] (23) at (-1.5, 1) {$(X^*)^\dagger$};
		\node [style=none] (24) at (-0.25, -0.75) {$\epsilon*$};
	\end{pgfonlayer}
	\begin{pgfonlayer}{edgelayer}
		\draw [bend left=90, looseness=2.00] (1.center) to (2.center);
		\draw [bend right=90, looseness=1.75] (4.center) to (3.center);
		\draw (0.center) to (8);
		\draw (11.center) to (12.center);
		\draw (12.center) to (13.center);
		\draw (13.center) to (14.center);
		\draw (14.center) to (11.center);
		\draw (8) to (18.center);
		\draw (19.center) to (5.center);
		\draw (15.center) to (3.center);
		\draw (2.center) to (4.center);
		\draw (1.center) to (16.center);
	\end{pgfonlayer}
\end{tikzpicture}
\]

\[
(b) ~~~~~~ %bbig1
 \begin{tikzpicture}
	\begin{pgfonlayer}{nodelayer}
		\node [style=none] (0) at (-3, 3.5) {};
		\node [style=none] (1) at (-3, -2.5) {};
		\node [style=none] (2) at (4.5, 3.5) {};
		\node [style=none] (3) at (4.5, -2.5) {};
		\node [style=none] (4) at (3, 3.5) {};
		\node [style=none] (5) at (3, -1) {};
		\node [style=none] (6) at (-0.25, -1) {};
		\node [style=none] (7) at (-0.25, 1) {};
		\node [style=none] (8) at (0.75, 1) {};
		\node [style=none] (9) at (0.75, -0.25) {};
		\node [style=none] (10) at (2, -0.25) {};
		\node [style=circle, scale=2.5] (11) at (2, 0.75) {};
		\node [style=none] (12) at (2, 1.75) {};
		\node [style=none] (13) at (-2, 1.75) {};
		\node [style=none] (14) at (-2, -2.5) {};
		\node [style=none] (15) at (0, 3.25) {$\eta*$};
		\node [style=none] (16) at (0.25, 1.75) {$\eta*$};
		\node [style=none] (17) at (1.5, -1.25) {$(*\eta)^\dagger$};
		\node [style=none] (18) at (1.5, -2.25) {$\epsilon*$};
		\node [style=none] (19) at (3.5, 3) {$X^{\dagger**}$};
		\node [style=none] (20) at (2.5, 1.5) {$X^{*\dagger}$};
		\node [style=none] (21) at (2.5, -0) {$(^*X)^\dagger$};
		\node [style=none] (22) at (0.25, 0.5) {$X^\dagger$};
		\node [style=none] (23) at (-0.75, -0.5) {$X^{\dagger*}$};
		\node [style=none] (24) at (-2.5, -2) {$X^{*\dagger*}$};
		\node [style=none] (25) at (1, 3.5) {};
		\node [style=none] (26) at (1, 5.5) {};
		\node [style=none] (27) at (1.5, 5) {$X^{*\dagger*\dagger}$};
		\node [style=none] (28) at (2, 0.75) {$\psi^{-1\dagger}$};
		\node [style=none] (29) at (-2, -4) {};
		\node [style=none] (30) at (-2, -6) {};
		\node [style=none] (31) at (-0.75, -6) {};
		\node [style=none] (32) at (-0.75, -5.5) {};
		\node [style=none] (33) at (1, -5.5) {};
		\node [style=none] (34) at (1, -8.5) {};
		\node [style=circle, scale=2.5] (35) at (1, -7) {};
		\node [style=none] (99) at (1, -7) {$\psi^{-1}$};
		\node [style=none] (36) at (0, -4.5) {$*\eta$};
		\node [style=none] (37) at (-1.5, -6.75) {$\epsilon*$};
		\node [style=none] (38) at (1.5, -6.25) {$~^*(X^{\dagger*})$};
		\node [style=none] (39) at (1.5, -8) {$X^{\dagger * *}$};
		\node [style=none] (40) at (-0.25, -5.75) {$X^{\dagger*}$};
		\node [style=none] (41) at (-2.5, -4.5) {$X^\dagger$};
		\node [style=none] (42) at (-3, -4) {};
		\node [style=none] (43) at (2.5, -4) {};
		\node [style=none] (44) at (2.5, -8.5) {};
		\node [style=none] (45) at (-3, -8.5) {};
		\node [style=none] (46) at (0, -2.5) {};
		\node [style=none] (47) at (0, -4) {};
		\node [style=none] (48) at (0.5, -3.25) {$X^{\dagger**\dagger}$};
		\node [style=circle, scale=2.5] (49) at (0, -9.5) {};
		\node [style=none] (50) at (0, -10.5) {};
		\node [style=none] (51) at (0, -8.5) {};
		\node [style=none] (52) at (0, -9.5) {$\iota^{-1}$};
		\node [style=none] (53) at (-0.5, -9) {$X^{\dagger \dagger}$};
		\node [style=none] (54) at (-0.5, -10.25) {$X$};
		\node [style=none] (55) at (4, -2) {$\dagger$};
		\node [style=none] (56) at (2, -8.25) {$\dagger$};
	\end{pgfonlayer}
	\begin{pgfonlayer}{edgelayer}
		\draw (0.center) to (1.center);
		\draw (1.center) to (3.center);
		\draw (3.center) to (2.center);
		\draw (2.center) to (0.center);
		\draw (4.center) to (5.center);
		\draw [bend left=90, looseness=1.00] (5.center) to (6.center);
		\draw (6.center) to (7.center);
		\draw [bend left=90, looseness=1.50] (7.center) to (8.center);
		\draw (8.center) to (9.center);
		\draw [bend right=90, looseness=1.50] (9.center) to (10.center);
		\draw (10.center) to (11);
		\draw (11) to (12.center);
		\draw [bend right=90, looseness=1.00] (12.center) to (13.center);
		\draw (13.center) to (14.center);
		\draw (26.center) to (25.center);
		\draw (29.center) to (30.center);
		\draw [bend right=90, looseness=1.50] (30.center) to (31.center);
		\draw (31.center) to (32.center);
		\draw [bend left=90, looseness=1.25] (32.center) to (33.center);
		\draw (33.center) to (35);
		\draw (35) to (34.center);
		\draw (43.center) to (44.center);
		\draw (44.center) to (45.center);
		\draw (45.center) to (42.center);
		\draw (42.center) to (43.center);
		\draw (46.center) to (47.center);
		\draw (51.center) to (49);
		\draw (49) to (50.center);
	\end{pgfonlayer}
\end{tikzpicture} = %bbig2
\begin{tikzpicture}
	\begin{pgfonlayer}{nodelayer}
		\node [style=none] (0) at (-3, 2.5) {};
		\node [style=none] (1) at (-3, -8.5) {};
		\node [style=none] (2) at (4.5, 2.5) {};
		\node [style=none] (3) at (4.5, -8.5) {};
		\node [style=none] (4) at (3, -2) {};
		\node [style=none] (5) at (3, -7) {};
		\node [style=none] (6) at (-0.25, -7) {};
		\node [style=none] (7) at (-0.25, -5) {};
		\node [style=none] (8) at (0.75, -5) {};
		\node [style=none] (9) at (0.75, -6.25) {};
		\node [style=none] (10) at (2, -6.25) {};
		\node [style=circle, scale=2.5] (11) at (2, -5.25) {};
		\node [style=none] (12) at (2, -4.25) {};
		\node [style=none] (13) at (-2, -4.25) {};
		\node [style=none] (14) at (-2, -8.5) {};
		\node [style=none] (15) at (0, -2.75) {$\eta*$};
		\node [style=none] (16) at (0.25, -4.25) {$\eta*$};
		\node [style=none] (17) at (1.5, -7.25) {$(*\eta)^\dagger$};
		\node [style=none] (18) at (1.5, -8.25) {$\epsilon*$};
		\node [style=none] (19) at (2.5, -4.5) {$X^{*\dagger}$};
		\node [style=none] (20) at (2.5, -6) {$(^*X)^\dagger$};
		\node [style=none] (21) at (0.25, -5.5) {$X^\dagger$};
		\node [style=none] (22) at (-0.75, -6.5) {$X^{\dagger*}$};
		\node [style=none] (23) at (-2.5, -8) {$X^{*\dagger*}$};
		\node [style=none] (24) at (0.9999999, 2.5) {};
		\node [style=none] (25) at (0.9999999, 4.5) {};
		\node [style=none] (26) at (1.5, 4) {$X^{*\dagger*\dagger}$};
		\node [style=none] (27) at (2, -5.25) {$\psi^{-1\dagger}$};
		\node [style=none] (28) at (0.5, 1.75) {$X^\dagger$};
		\node [style=none] (29) at (0, -8.5) {};
		\node [style=circle, scale=2.5] (30) at (0, -11) {};
		\node [style=none] (31) at (0, -12.25) {};
		\node [style=none] (32) at (0, -11) {$\iota^{-1}$};
		\node [style=none] (33) at (-0.9999999, -9.75) {$X^{\dagger \dagger}$};
		\node [style=none] (34) at (-0.4999999, -12) {$X$};
		\node [style=none] (35) at (4, -8) {$\dagger$};
		\node [style=none] (36) at (1.25, 1) {};
		\node [style=none] (37) at (1.25, 0.4999999) {};
		\node [style=none] (38) at (0, 2.5) {};
		\node [style=circle, scale=2.5] (39) at (3, -0.7499999) {};
		\node [style=none] (39) at (3, -0.7499999) {$\psi^{-1}$};
		\node [style=none] (40) at (0, 0.4999999) {};
		\node [style=none] (41) at (1.75, 0.7500001) {$X^{\dagger*}$};
		\node [style=none] (42) at (2, 2) {$*\eta$};
		\node [style=none] (43) at (3, 1) {};
		\node [style=none] (44) at (3.5, 0.25) {$~^*(X^{\dagger*})$};
		\node [style=none] (45) at (3, -2) {};
		\node [style=none] (46) at (0.4999999, -0.25) {$\epsilon*$};
		\node [style=none] (47) at (3.5, -2.75) {$X^{\dagger * *}$};
		\node [style=none] (48) at (2, 2.5) {};
	\end{pgfonlayer}
	\begin{pgfonlayer}{edgelayer}
		\draw (0.center) to (1.center);
		\draw (1.center) to (3.center);
		\draw (3.center) to (2.center);
		\draw (2.center) to (0.center);
		\draw (4.center) to (5.center);
		\draw [bend left=90, looseness=1.00] (5.center) to (6.center);
		\draw (6.center) to (7.center);
		\draw [bend left=90, looseness=1.50] (7.center) to (8.center);
		\draw (8.center) to (9.center);
		\draw [bend right=90, looseness=1.50] (9.center) to (10.center);
		\draw (10.center) to (11);
		\draw (11) to (12.center);
		\draw [bend right=90, looseness=1.00] (12.center) to (13.center);
		\draw (13.center) to (14.center);
		\draw (25.center) to (24.center);
		\draw (30) to (31.center);
		\draw (38.center) to (40.center);
		\draw [bend right=90, looseness=1.50] (40.center) to (37.center);
		\draw (37.center) to (36.center);
		\draw [bend left=90, looseness=1.25] (36.center) to (43.center);
		\draw (43.center) to (39);
		\draw (39) to (45.center);
		\draw (29.center) to (30);
	\end{pgfonlayer}
\end{tikzpicture} = %bbig3
\begin{tikzpicture}
	\begin{pgfonlayer}{nodelayer}
		\node [style=none] (0) at (-3, 2.5) {};
		\node [style=none] (1) at (-3, -8.5) {};
		\node [style=none] (2) at (6.75, 2.5) {};
		\node [style=none] (3) at (6.75, -8.5) {};
		\node [style=none] (4) at (-0.25, -7) {};
		\node [style=none] (5) at (-0.25, -5) {};
		\node [style=none] (6) at (0.75, -5) {};
		\node [style=none] (7) at (0.75, -6.25) {};
		\node [style=none] (8) at (2, -6.25) {};
		\node [style=circle, scale=2.5] (9) at (2, -5.25) {};
		\node [style=none] (10) at (2, -4.25) {};
		\node [style=none] (11) at (-2, -4.25) {};
		\node [style=none] (12) at (-2, -8.5) {};
		\node [style=none] (13) at (0, -2.75) {$\eta*$};
		\node [style=none] (14) at (0.25, -4.25) {$\eta*$};
		\node [style=none] (15) at (1.5, -7.25) {$(*\eta)^\dagger$};
		\node [style=none] (16) at (1.5, -8.25) {$\epsilon*$};
		\node [style=none] (17) at (0.25, -5.5) {$X^\dagger$};
		\node [style=none] (18) at (-0.75, -6.5) {$X^{\dagger*}$};
		\node [style=none] (19) at (-2.5, -8) {$X^{*\dagger*}$};
		\node [style=none] (20) at (0.9999999, 2.5) {};
		\node [style=none] (21) at (0.9999999, 4.5) {};
		\node [style=none] (22) at (1.5, 4) {$X^{*\dagger*\dagger}$};
		\node [style=none] (23) at (2, -5.25) {$\psi^{-1\dagger}$};
		\node [style=none] (24) at (0, -8.5) {};
		\node [style=circle, scale=2.5] (25) at (0, -11) {};
		\node [style=none] (26) at (0, -12.25) {};
		\node [style=none] (27) at (0, -11) {$\iota^{-1}$};
		\node [style=none] (28) at (-0.9999999, -9.75) {$X^{\dagger \dagger}$};
		\node [style=none] (29) at (-0.4999999, -12) {$X$};
		\node [style=none] (30) at (6.25, -8) {$\dagger$};
		\node [style=none] (31) at (2, 2.5) {};
		\node [style=none] (32) at (4.5, 1.25) {};
		\node [style=circle, scale=2.5] (33) at (4.5, -0) {};
		\node [style=none] (34) at (4.5, -0.9999999) {};
		\node [style=none] (35) at (5.75, -0.9999999) {};
		\node [style=none] (36) at (5.75, 2.5) {};
		\node [style=none] (37) at (4.5, -0) {$\psi$};
		\node [style=none] (38) at (5, -2) {$*\epsilon$};
		\node [style=none] (39) at (5.25, 2) {$X^\dagger$};
		\node [style=none] (40) at (4, 0.7499999) {$(X^{\dagger*}$};
		\node [style=none] (41) at (4, -0.9999999) {$~^*(X^\dagger)$};
		\node [style=none] (42) at (3, 1.25) {};
		\node [style=none] (43) at (2.5, -6) {$(^*X)^\dagger$};
		\node [style=none] (44) at (3, -2) {};
		\node [style=none] (45) at (2.5, -4.5) {$X^{*\dagger}$};
		\node [style=none] (46) at (3.5, -2.75) {$X^{\dagger * *}$};
		\node [style=none] (47) at (3.75, 2) {$\eta*$};
		\node [style=none] (48) at (3, -7) {};
	\end{pgfonlayer}
	\begin{pgfonlayer}{edgelayer}
		\draw (0.center) to (1.center);
		\draw (1.center) to (3.center);
		\draw (3.center) to (2.center);
		\draw (2.center) to (0.center);
		\draw (4.center) to (5.center);
		\draw [bend left=90, looseness=1.50] (5.center) to (6.center);
		\draw (6.center) to (7.center);
		\draw [bend right=90, looseness=1.50] (7.center) to (8.center);
		\draw (8.center) to (9);
		\draw (9) to (10.center);
		\draw [bend right=90, looseness=1.00] (10.center) to (11.center);
		\draw (11.center) to (12.center);
		\draw (21.center) to (20.center);
		\draw (25) to (26.center);
		\draw (24.center) to (25);
		\draw (32.center) to (33);
		\draw (33) to (34.center);
		\draw [bend right=90, looseness=2.00] (34.center) to (35.center);
		\draw (35.center) to (36.center);
		\draw [bend left=90, looseness=1.00] (48.center) to (4.center);
		\draw (42.center) to (48.center);
		\draw [bend left=90, looseness=1.25] (42.center) to (32.center);
	\end{pgfonlayer}
\end{tikzpicture} =  \]
\[ \begin{tikzpicture}
	\begin{pgfonlayer}{nodelayer}
		\node [style=none] (0) at (-3, 2.5) {};
		\node [style=none] (1) at (-3, -8.5) {};
		\node [style=none] (2) at (4.25, 2.5) {};
		\node [style=none] (3) at (4.25, -8.5) {};
		\node [style=none] (4) at (-0.25, -3.25) {};
		\node [style=none] (5) at (-0.25, -0.7499999) {};
		\node [style=none] (6) at (0.7499999, -0.7499999) {};
		\node [style=none] (7) at (0.7499999, -2) {};
		\node [style=none] (8) at (2, -2) {};
		\node [style=circle, scale=2.5] (9) at (2, -0.9999999) {};
		\node [style=none] (10) at (2, -0) {};
		\node [style=none] (11) at (-2, -0) {};
		\node [style=none] (12) at (-2, -8.5) {};
		\node [style=none] (13) at (0, 1.5) {$\eta*$};
		\node [style=none] (14) at (0.25, -0) {$\eta*$};
		\node [style=none] (15) at (1.5, -2.75) {$(*\eta)^\dagger$};
		\node [style=none] (16) at (0.25, -1.25) {$X^\dagger$};
		\node [style=none] (17) at (-0.7499999, -2.25) {$X^{\dagger*}$};
		\node [style=none] (18) at (-2.5, -8) {$X^{*\dagger*}$};
		\node [style=none] (19) at (0.9999999, 2.5) {};
		\node [style=none] (20) at (0.9999999, 4.5) {};
		\node [style=none] (21) at (1.5, 4) {$X^{*\dagger*\dagger}$};
		\node [style=none] (22) at (2, -0.9999999) {$\psi^{-1\dagger}$};
		\node [style=none] (23) at (0, -8.5) {};
		\node [style=circle, scale=2.5] (24) at (0, -11) {};
		\node [style=none] (25) at (0, -12.25) {};
		\node [style=none] (26) at (0, -11) {$\iota^{-1}$};
		\node [style=none] (27) at (-0.9999999, -9.75) {$X^{\dagger \dagger}$};
		\node [style=none] (28) at (-0.4999999, -12) {$X$};
		\node [style=none] (29) at (3.75, -8) {$\dagger$};
		\node [style=none] (30) at (2, 2.5) {};
		\node [style=circle, scale=2.5] (31) at (2, -5.75) {};
		\node [style=none] (32) at (2, -6.25) {};
		\node [style=none] (33) at (3.25, -6.25) {};
		\node [style=none] (34) at (3.25, 2.5) {};
		\node [style=none] (35) at (2, -5.75) {$\psi$};
		\node [style=none] (36) at (2.5, -7.25) {$*\epsilon$};
		\node [style=none] (37) at (2.75, 2) {$X^\dagger$};
	\end{pgfonlayer}
	\begin{pgfonlayer}{edgelayer}
		\draw (0.center) to (1.center);
		\draw (1.center) to (3.center);
		\draw (3.center) to (2.center);
		\draw (2.center) to (0.center);
		\draw (4.center) to (5.center);
		\draw [bend left=90, looseness=1.50] (5.center) to (6.center);
		\draw (6.center) to (7.center);
		\draw [bend right=90, looseness=1.50] (7.center) to (8.center);
		\draw (8.center) to (9);
		\draw (9) to (10.center);
		\draw [bend right=90, looseness=1.00] (10.center) to (11.center);
		\draw (11.center) to (12.center);
		\draw (20.center) to (19.center);
		\draw (24) to (25.center);
		\draw (23.center) to (24);
		\draw (31) to (32.center);
		\draw [bend right=90, looseness=2.00] (32.center) to (33.center);
		\draw (33.center) to (34.center);
		\draw [in=90, out=-90, looseness=0.75] (4.center) to (31);
	\end{pgfonlayer}
\end{tikzpicture} \stackrel{*}{=}%bbig5
\begin{tikzpicture}
	\begin{pgfonlayer}{nodelayer}
		\node [style=none] (0) at (-3, 2.5) {};
		\node [style=none] (1) at (-3, -8.5) {};
		\node [style=none] (2) at (4.5, 2.5) {};
		\node [style=none] (3) at (4.5, -8.5) {};
		\node [style=none] (4) at (2, -1.5) {};
		\node [style=none] (5) at (2, -0.7499999) {};
		\node [style=none] (6) at (1, -0.7499999) {};
		\node [style=none] (7) at (1, -2) {};
		\node [style=none] (8) at (-0.25, -2) {};
		\node [style=none] (9) at (-0.25, 0.7500001) {};
		\node [style=none] (10) at (-2, 0.7500001) {};
		\node [style=none] (11) at (-2, -8.5) {};
		\node [style=none] (12) at (-0.9999999, 1.5) {$\eta*$};
		\node [style=none] (13) at (-2.5, -8) {$X^{*\dagger*}$};
		\node [style=none] (14) at (0.9999999, 2.5) {};
		\node [style=none] (15) at (0.9999999, 4.5) {};
		\node [style=none] (16) at (1.5, 4) {$X^{*\dagger*\dagger}$};
		\node [style=none] (17) at (0, -8.5) {};
		\node [style=circle, scale=2.5] (18) at (0, -11) {};
		\node [style=none] (19) at (0, -12.25) {};
		\node [style=none] (20) at (0, -11) {$\iota^{-1}$};
		\node [style=none] (21) at (-0.9999999, -9.75) {$X^{\dagger \dagger}$};
		\node [style=none] (22) at (-0.4999999, -12) {$X$};
		\node [style=none] (23) at (4, -8) {$\dagger$};
		\node [style=none] (24) at (2, 2.5) {};
		\node [style=circle, scale=2.5] (25) at (2, -5.75) {};
		\node [style=none] (26) at (2, -6.25) {};
		\node [style=none] (27) at (3.5, -6.25) {};
		\node [style=none] (28) at (3.5, 2.5) {};
		\node [style=none] (29) at (2, -5.75) {$\psi$};
		\node [style=none] (30) at (2.75, -7) {$*\epsilon$};
		\node [style=none] (31) at (3, 2) {$X^\dagger$};
		\node [style=circle, scale=2.5] (32) at (2, -3.25) {};
		\node [style=none] (33) at (2, -5) {};
		\node [style=none] (34) at (2, -3.25) {$\psi^{-1}$};
		\node [style=none] (35) at (0.25, -2.75) {$(\eta*)^\dagger$};
		\node [style=none] (36) at (1.5, -0) {$*\eta$};
	\end{pgfonlayer}
	\begin{pgfonlayer}{edgelayer}
		\draw (0.center) to (1.center);
		\draw (1.center) to (3.center);
		\draw (3.center) to (2.center);
		\draw (2.center) to (0.center);
		\draw (4.center) to (5.center);
		\draw [bend right=90, looseness=1.50] (5.center) to (6.center);
		\draw (6.center) to (7.center);
		\draw [bend left=90, looseness=1.50] (7.center) to (8.center);
		\draw [bend right=90, looseness=1.00] (9.center) to (10.center);
		\draw (10.center) to (11.center);
		\draw (15.center) to (14.center);
		\draw (18) to (19.center);
		\draw (17.center) to (18);
		\draw (25) to (26.center);
		\draw [bend right=90, looseness=0.75] (26.center) to (27.center);
		\draw (27.center) to (28.center);
		\draw (4.center) to (32);
		\draw (32) to (33.center);
		\draw (9.center) to (8.center);
		\draw (33.center) to (25);
	\end{pgfonlayer}
\end{tikzpicture} = %bbig6
\begin{tikzpicture}
	\begin{pgfonlayer}{nodelayer}
		\node [style=none] (0) at (-2.5, 2.5) {};
		\node [style=none] (1) at (-2.5, -8.5) {};
		\node [style=none] (2) at (4.75, 2.5) {};
		\node [style=none] (3) at (4.75, -8.5) {};
		\node [style=none] (4) at (2.5, -3.75) {};
		\node [style=none] (5) at (2.5, -5) {};
		\node [style=none] (6) at (0.7499999, -5) {};
		\node [style=none] (7) at (0.7499999, 0.4999999) {};
		\node [style=none] (8) at (-1.5, 0.7499999) {};
		\node [style=none] (9) at (-1.5, -8.5) {};
		\node [style=none] (10) at (-0.4999999, 1.5) {$\eta*$};
		\node [style=none] (11) at (-2, -8) {$X^{*\dagger*}$};
		\node [style=none] (12) at (0.9999999, 2.5) {};
		\node [style=none] (13) at (0.9999999, 4.5) {};
		\node [style=none] (14) at (1.5, 4) {$X^{*\dagger*\dagger}$};
		\node [style=none] (15) at (0, -8.5) {};
		\node [style=circle, scale=2.5] (16) at (0, -11) {};
		\node [style=none] (17) at (0, -12.25) {};
		\node [style=none] (18) at (0, -11) {$\iota^{-1}$};
		\node [style=none] (19) at (-0.9999999, -9.75) {$X^{\dagger \dagger}$};
		\node [style=none] (20) at (-0.4999999, -12) {$X$};
		\node [style=none] (21) at (4.25, -8) {$\dagger$};
		\node [style=none] (22) at (2.5, -3.75) {};
		\node [style=none] (23) at (2.5, 2.5) {};
		\node [style=none] (24) at (2, 2) {$X^\dagger$};
		\node [style=none] (25) at (1.75, -6.25) {$(\eta*)^\dagger$};
		\node [style=none] (26) at (1.25, -2.5) {$X^{*\dagger}$};
	\end{pgfonlayer}
	\begin{pgfonlayer}{edgelayer}
		\draw (0.center) to (1.center);
		\draw (1.center) to (3.center);
		\draw (3.center) to (2.center);
		\draw (2.center) to (0.center);
		\draw (4.center) to (5.center);
		\draw [bend left=90, looseness=1.75] (5.center) to (6.center);
		\draw [bend right=90, looseness=1.00] (7.center) to (8.center);
		\draw (8.center) to (9.center);
		\draw (13.center) to (12.center);
		\draw (16) to (17.center);
		\draw (15.center) to (16);
		\draw (22.center) to (23.center);
		\draw (7.center) to (6.center);
	\end{pgfonlayer}
\end{tikzpicture}
\]
 $(*)$ holds because $\dagger$ preserves the cyclor. Thus, $(a)$ and $(b)$ are inverses of each other.
\end{proof}

Next, we show that a conjugation functor together with a dualizing functors gives a $\dagger$:

\begin{proposition}
\label{Theorem: conjugation+dualizing}
Every cyclic, conjugative $*$-autonomous category  is also a $\dagger$-$*$-autonomous category.
\end{proposition}
\begin{proof}
Let $\X$ be a cyclic, conjugative $*$-autonomous category then  $\bar{(\_)^*} \dashv \bar{(\_)}^*$ 
is an equivalence. To build a dagger we need an equivalence on the same 
functor: we obtain this by using the natural equivalence $\sigma: \bar{(\_)^*} \to \bar{(\_)}^*$ from Definition
 \ref{Defn: conjugative cyclic}.  An involutive equivalence, in addition, 
requires the unit and counit of the (contravariant) equivalence to be the same map (which we called the involutor, $\iota$). 
We show that this is the case: 

The unit and counit of the equivalence is given by {\em (a)} and {\em (b)} respectively;
\[ \mbox{\em (a)}~~~~~~~~~  \xymatrix{
\X^{\op} \ar@{=}[rrrr] \ar[dr]_{(\_)^{*^{\sf rev}}} & 
&  
\ar@{}[d]|{\Downarrow ~ \eta_\ox'}&  
& 
\X^{\op} \ar@{<-}[ld]^{(\_)^{*^{\sf op}}}  
& \\
&
 \X^{\sf rev} \ar@{=}[rr] \ar[dr]_{\bar{(\_)}} &  
 \ar@{}[d]|{\Downarrow ~ \varepsilon^{-1}} & 
 \X^{\sf rev} \ar@{<-}[ld]^{\bar{(\_)}^{\sf rev}} \ar@{}[rr]_{\sigma}  \ar@{}[rr]^{\Longrightarrow}  & 
 & 
 \X^{\sf oprev} \ar@{<-}@/^1pc/[llld]^{(\_)^{*^{\sf oprev}}} \ar@/_1pc/[ul]_{\bar{(\_)}^\op}  \\
& & \X & &
}
\]

\[
\mbox{\em (b)}~~~~~~~~~~~~~~ \xymatrix{
& & &
\X^{\op}  \ar@{<-}@/_1pc/[llld]_{\bar{(\_)}^\op} \ar@{<-}[dl]_{(\_)^{*^{\op}}} \ar[dr]^{(\_)^{*^{\sf rev}}} 
\ar@{}[d]|{\Downarrow ~ \epsilon'_\ox} & &\\
\X^{\sf op rev} \ar@{<-}[rd]_{(\_)^{*^{\sf oprev}}} \ar@{}[rr]_{\sigma^{-1}}  \ar@{}[rr]^{\Longrightarrow}  &  & 
\X^{\sf rev} \ar@{=}[rr] \ar@{<-}[dl]^{\bar{(\_)}^{\sf rev}} & \ar@{}[d]|{\Downarrow ~ \varepsilon} &
\X^{\sf rev} \ar[dr]^{\bar{(\_)}} \\
& \X \ar@{=}[rrrr]&  & & & \X }
\] where $\sigma: \bar{A}^* \to \bar{A^*}$ is given in Definition \ref{Defn: conjugative cyclic}.  Below we 
show that the unit and counit coincide in $\X$.

\[
(a)~~~~~ %big1 
\begin{tikzpicture}
	\begin{pgfonlayer}{nodelayer}
		\node [style=none] (0) at (-2, 3.5) {};
		\node [style=circle, scale=2.5] (1) at (-1, 2.75) {};
		\node [style=none] (2) at (-2, 2) {};
		\node [style=none] (3) at (-1, 2) {};
		\node [style=none] (4) at (-1, 3.5) {};
		\node [style=none] (5) at (-1, -1.25) {};
		\node [style=none] (6) at (-2, -1.25) {};
		\node [style=circle, scale=2.5] (7) at (-2, -0.5) {};
		\node [style=none] (8) at (-2, 0.25) {};
		\node [style=none] (9) at (-3, 0.25) {};
		\node [style=circle, scale=2.5] (10) at (-3, -2.5) {};
		\node [style=none] (11) at (-3, -3.5) {};
		\node [style=none] (12) at (-2, -3.5) {};
		\node [style=none] (13) at (-2, -3) {};
		\node [style=none] (14) at (-1, -3) {};
		\node [style=none] (15) at (-1, -5) {};
		\node [style=none] (16) at (-2.75, 2) {};
		\node [style=none] (17) at (-2.75, 4.5) {};
		\node [style=none] (18) at (-1, 2.75) {$\psi^{-1}$};
		\node [style=none] (19) at (-2, -0.5) {$\varepsilon^{-1}$};
		\node [style=none] (20) at (-3, -2.5) {$\psi$};
		\node [style=none] (21) at (-3.25, 4) {$\bar{\left(\bar{X^*}\right)^*}$};
		\node [style=none] (22) at (-2.5, 1.5) {$\bar{\epsilon*}$};
		\node [style=none] (23) at (-2.25, 2.75) {$\bar{\bar{X^*}}$};
		\node [style=none] (24) at (-1.5, 4.25) {$*\eta$};
		\node [style=none] (25) at (-0.4, 3.5) {$~^*\!\!\left(\bar{\bar{X^*}}\right)$};
		\node [style=none] (26) at (-0.4, 2) {$\left(\bar{\bar{X^*}}\right)^*$};
		\node [style=none] (27) at (-1.75, -2) {$\epsilon*$};
		\node [style=none] (28) at (-1, -2.5) {$\eta*$};
		\node [style=none] (29) at (-2.5, 1) {$\eta*$};
		\node [style=none] (30) at (-3.5, -1) {$X^{**}$};
		\node [style=none] (31) at (-1.5, -0) {$X^*$};
		\node [style=none] (32) at (-1.5, -1.25) {$\bar{\bar{X^*}}$};
		\node [style=none] (33) at (-2.5, -4.25) {$*\epsilon$};
		\node [style=none] (34) at (-3.75, -3.25) {$~^*(X^*)$};
		\node [style=none] (35) at (-2.5, -3.25) {$X^*$};
		\node [style=none] (36) at (-0.5, -4.75) {$X$};
	\end{pgfonlayer}
	\begin{pgfonlayer}{edgelayer}
		\draw (14.center) to (15.center);
		\draw [bend right=90, looseness=1.50] (14.center) to (13.center);
		\draw (13.center) to (12.center);
		\draw [bend left=90, looseness=1.50] (12.center) to (11.center);
		\draw (10) to (11.center);
		\draw (10) to (9.center);
		\draw [bend left=90, looseness=1.75] (9.center) to (8.center);
		\draw (8.center) to (7);
		\draw (7) to (6.center);
		\draw [bend right=90, looseness=1.75] (6.center) to (5.center);
		\draw (3.center) to (5.center);
		\draw (3.center) to (1);
		\draw (1) to (4.center);
		\draw [bend right=90, looseness=1.75] (4.center) to (0.center);
		\draw (0.center) to (2.center);
		\draw [bend left=75, looseness=1.50] (2.center) to (16.center);
		\draw (16.center) to (17.center);
	\end{pgfonlayer}
\end{tikzpicture}\stackrel{{\bf [C.2]}}{=} %big2
 \begin{tikzpicture}
	\begin{pgfonlayer}{nodelayer}
		\node [style=none] (0) at (-2, 3.5) {};
		\node [style=circle, scale=2.5] (1) at (-1, 2.75) {};
		\node [style=none] (2) at (-2, 2) {};
		\node [style=none] (3) at (-1, 2) {};
		\node [style=none] (4) at (-1, 3.5) {};
		\node [style=none] (5) at (-1, -1.25) {};
		\node [style=none] (6) at (-2, -1.25) {};
		\node [style=circle, scale=2.5] (7) at (-2, -0.5) {};
		\node [style=none] (8) at (-2, 0.25) {};
		\node [style=none] (9) at (-3, 0.25) {};
		\node [style=none] (10) at (-3, -4.5) {};
		\node [style=none] (11) at (-4, -4.5) {};
		\node [style=none] (12) at (-4, -3) {};
		\node [style=none] (13) at (-5, -3) {};
		\node [style=none] (14) at (-5, -5.25) {};
		\node [style=none] (15) at (-2.75, 2) {};
		\node [style=none] (16) at (-2.75, 4.5) {};
		\node [style=none] (17) at (-1, 2.75) {$\psi^{-1}$};
		\node [style=none] (18) at (-2, -0.5) {$\varepsilon^{-1}$};
		\node [style=none] (19) at (-3.25, 4) {$\bar{\left(\bar{X^*}\right)^*}$};
		\node [style=none] (20) at (-2.5, 1.5) {$\bar{\epsilon*}$};
		\node [style=none] (21) at (-2.25, 2.75) {$\bar{\bar{X^*}}$};
		\node [style=none] (22) at (-1.5, 4.25) {$*\eta$};
		\node [style=none] (23) at (-0.4, 3.5) {$~^*\!\!\left(\bar{\bar{X^*}}\right)$};
		\node [style=none] (24) at (-0.4, 2) {$\left(\bar{\bar{X^*}} \right)^*$};
		\node [style=none] (25) at (-1.75, -2) {$\epsilon*$};
		\node [style=none] (26) at (-2.5, 1) {$\eta*$};
		\node [style=none] (27) at (-2.5, -3.5) {$X^{**}$};
		\node [style=none] (28) at (-1.5, -0) {$X^*$};
		\node [style=none] (29) at (-1.5, -1.25) {$\bar{\bar{X^*}}$};
		\node [style=none] (30) at (-3.5, -5.25) {$\epsilon*$};
		\node [style=none] (31) at (-4.5, -4.35) {$X^*$};
		\node [style=none] (32) at (-5.25, -4.75) {$X$};
		\node [style=circle, scale=2.5] (33) at (-4, -3.75) {};
		\node [style=none] (34) at (-4, -3.75) {$\psi^{-1}$};
		\node [style=none] (35) at (-4.5, -2.3) {$*\eta$};
		\node [style=none] (36) at (-3.75, -3) {$~^*X$};
	\end{pgfonlayer}
	\begin{pgfonlayer}{edgelayer}
		\draw (13.center) to (14.center);
		\draw [bend left=90, looseness=1.50] (13.center) to (12.center);
		\draw [bend right=90, looseness=1.50] (11.center) to (10.center);
		\draw [bend left=90, looseness=1.75] (9.center) to (8.center);
		\draw (8.center) to (7);
		\draw (7) to (6.center);
		\draw [bend right=90, looseness=1.75] (6.center) to (5.center);
		\draw (3.center) to (5.center);
		\draw (3.center) to (1);
		\draw (1) to (4.center);
		\draw [bend right=90, looseness=1.75] (4.center) to (0.center);
		\draw (0.center) to (2.center);
		\draw [bend left=75, looseness=1.50] (2.center) to (15.center);
		\draw (15.center) to (16.center);
		\draw (9.center) to (10.center);
		\draw (12.center) to (33);
		\draw (33) to (11.center);
	\end{pgfonlayer}
\end{tikzpicture} = %big3
\begin{tikzpicture}
	\begin{pgfonlayer}{nodelayer}
		\node [style=none] (0) at (-2, 3.5) {};
		\node [style=circle, scale=2.5] (1) at (-1, 2.75) {};
		\node [style=none] (2) at (-2, 2) {};
		\node [style=none] (3) at (-1, 2) {};
		\node [style=none] (4) at (-1, 3.5) {};
		\node [style=none] (5) at (-1, -3.5) {};
		\node [style=none] (6) at (-2, -3.5) {};
		\node [style=circle, scale=2.5] (7) at (-2, -2.75) {};
		\node [style=none] (8) at (-2, -2) {};
		\node [style=none] (9) at (-2, -2) {};
		\node [style=none] (10) at (-2, -0.5) {};
		\node [style=none] (11) at (-3, -0.5) {};
		\node [style=none] (12) at (-3, -5.25) {};
		\node [style=none] (13) at (-2.75, 2) {};
		\node [style=none] (14) at (-2.75, 4.5) {};
		\node [style=none] (15) at (-1, 2.75) {$\psi^{-1}$};
		\node [style=none] (16) at (-2, -2.75) {$\varepsilon^{-1}$};
		\node [style=none] (17) at (-3.25, 4) {$\bar{\left(\bar{X^*}\right)^*}$};
		\node [style=none] (18) at (-2.5, 1.5) {$\bar{\epsilon*}$};
		\node [style=none] (19) at (-2.25, 2.75) {$\bar{\bar{X^*}}$};
		\node [style=none] (20) at (-1.5, 4.25) {$*\eta$};
		\node [style=none] (21) at (-0.5, 3.5) {$~^*(\bar{\bar{X^*}})$};
		\node [style=none] (22) at (-0.5, 2) {$(\bar{\bar{X^*}})^*$};
		\node [style=none] (23) at (-1.75, -4.25) {$\epsilon*$};
		\node [style=none] (24) at (-1.5, -2.25) {$X^*$};
		\node [style=none] (25) at (-1.5, -3.5) {$\bar{\bar{X^*}}$};
		\node [style=none] (26) at (-2.5, -1.75) {$X^*$};
		\node [style=none] (27) at (-3.25, -4.75) {$X$};
		\node [style=circle, scale=2.5] (28) at (-2, -1.25) {};
		\node [style=none] (29) at (-2, -1.25) {$\varphi^{-1}$};
		\node [style=none] (30) at (-2.5, 0.25) {$*\eta$};
		\node [style=none] (31) at (-1.75, -0.5) {$~^*X$};
	\end{pgfonlayer}
	\begin{pgfonlayer}{edgelayer}
		\draw (11.center) to (12.center);
		\draw [bend left=90, looseness=1.50] (11.center) to (10.center);
		\draw (8.center) to (7);
		\draw (7) to (6.center);
		\draw [bend right=90, looseness=1.75] (6.center) to (5.center);
		\draw (3.center) to (5.center);
		\draw (3.center) to (1);
		\draw (1) to (4.center);
		\draw [bend right=90, looseness=1.75] (4.center) to (0.center);
		\draw (0.center) to (2.center);
		\draw [bend left=75, looseness=1.50] (2.center) to (13.center);
		\draw (13.center) to (14.center);
		\draw (10.center) to (28);
		\draw (28) to (9.center);
	\end{pgfonlayer}
\end{tikzpicture} = %big3
 \begin{tikzpicture}
	\begin{pgfonlayer}{nodelayer}
		\node [style=none] (0) at (-2, 3.5) {};
		\node [style=circle, scale=2.5] (1) at (-1, 2.75) {};
		\node [style=none] (2) at (-2, 2) {};
		\node [style=none] (3) at (-1, 2) {};
		\node [style=none] (4) at (-1, 3.5) {};
		\node [style=none] (5) at (-1, -3.5) {};
		\node [style=none] (6) at (-3, -3.5) {};
		\node [style=none] (7) at (-3, -2) {};
		\node [style=none] (8) at (-3, -2) {};
		\node [style=none] (9) at (-3, -0.5) {};
		\node [style=none] (10) at (-4, -0.5) {};
		\node [style=none] (11) at (-4, -5.25) {};
		\node [style=none] (12) at (-2.75, 2) {};
		\node [style=none] (13) at (-2.75, 4.5) {};
		\node [style=none] (14) at (-1, 2.75) {$\psi^{-1}$};
		\node [style=none] (15) at (-1, -1) {$\varepsilon^{-1^*}$};
		\node [style=none] (16) at (-3.25, 4) {$\bar{\left(\bar{X^*}\right)^*}$};
		\node [style=none] (17) at (-2.5, 1.5) {$\bar{\epsilon*}$};
		\node [style=none] (18) at (-2.25, 2.75) {$\bar{\bar{X^*}}$};
		\node [style=none] (19) at (-1.5, 4.25) {$*\eta$};
		\node [style=none] (20) at (-0.5, 3.5) {$~^*(\bar{\bar{X^*}})$};
		\node [style=none] (21) at (-0.5, -0) {$(\bar{\bar{X^*}})^*$};
		\node [style=none] (22) at (-2, -4.75) {$\epsilon*$};
		\node [style=none] (23) at (-2.5, -2) {$X^*$};
		\node [style=none] (24) at (-4.25, -4.75) {$X$};
		\node [style=circle, scale=2.5] (25) at (-3, -1.25) {};
		\node [style=none] (26) at (-3, -1.25) {$\varphi^{-1}$};
		\node [style=none] (27) at (-3.5, 0.25) {$*\eta$};
		\node [style=none] (28) at (-2.75, -0.5) {$~^*X$};
		\node [style=circle, scale=2.5] (29) at (-1, -1) {};
		\node [style=none] (30) at (-0.5, -2) {$X^{**}$};
	\end{pgfonlayer}
	\begin{pgfonlayer}{edgelayer}
		\draw (10.center) to (11.center);
		\draw [bend left=90, looseness=1.50] (10.center) to (9.center);
		\draw [bend right=90, looseness=1.75] (6.center) to (5.center);
		\draw (3.center) to (1);
		\draw (1) to (4.center);
		\draw [bend right=90, looseness=1.75] (4.center) to (0.center);
		\draw (0.center) to (2.center);
		\draw [bend left=75, looseness=1.50] (2.center) to (12.center);
		\draw (12.center) to (13.center);
		\draw (9.center) to (25);
		\draw (25) to (8.center);
		\draw (7.center) to (6.center);
		\draw (3.center) to (29);
		\draw (29) to (5.center);
	\end{pgfonlayer}
\end{tikzpicture} = %big4
\begin{tikzpicture}
	\begin{pgfonlayer}{nodelayer}
		\node [style=none] (0) at (-2, 3.5) {};
		\node [style=circle, scale=2.5] (1) at (-1, 2.5) {};
		\node [style=none] (2) at (-2, 2) {};
		\node [style=none] (3) at (-1, 2) {};
		\node [style=none] (4) at (-1, 3.5) {};
		\node [style=none] (5) at (-1, -3.5) {};
		\node [style=none] (6) at (-3, -3.5) {};
		\node [style=none] (7) at (-3, -2) {};
		\node [style=none] (8) at (-3, -2) {};
		\node [style=none] (9) at (-3, -0.5) {};
		\node [style=none] (10) at (-4, -0.5) {};
		\node [style=none] (11) at (-4, -5.25) {};
		\node [style=none] (12) at (-2.75, 2) {};
		\node [style=none] (13) at (-2.75, 4.5) {};
		\node [style=none] (14) at (-3.25, 4) {$\bar{\left(\bar{X^*}\right)^*}$};
		\node [style=none] (15) at (-2.5, 1.5) {$\bar{\epsilon*}$};
		\node [style=none] (16) at (-2.25, 2.75) {$\bar{\bar{X^*}}$};
		\node [style=none] (17) at (-1.5, 4.25) {$*\eta$};
		\node [style=none] (18) at (-0.5, 3.5) {$~^*(\bar{\bar{X^*}})$};
		\node [style=none] (19) at (-2, -4.75) {$\epsilon*$};
		\node [style=none] (20) at (-2.5, -2) {$X^*$};
		\node [style=none] (21) at (-4.25, -4.75) {$X$};
		\node [style=circle, scale=2.5] (22) at (-3, -1.25) {};
		\node [style=none] (23) at (-3, -1.25) {$\psi^{-1}$};
		\node [style=none] (24) at (-3.5, 0.25) {$*\eta$};
		\node [style=none] (25) at (-2.75, -0.5) {$~^*X$};
		\node [style=circle, scale=2.5] (26) at (-1, -1) {};
		\node [style=none] (27) at (-1, -1) {$\psi^{-1}$};
		\node [style=none] (28) at (-1, 2.5) {$~^*\varepsilon^{-1}$};
		\node [style=none] (29) at (-0.5, 0.75) {$~^*(X^*)$};
		\node [style=none] (30) at (-0.5, -2.5) {$X^{**}$};
	\end{pgfonlayer}
	\begin{pgfonlayer}{edgelayer}
		\draw (10.center) to (11.center);
		\draw [bend left=90, looseness=1.50] (10.center) to (9.center);
		\draw [bend right=90, looseness=1.75] (6.center) to (5.center);
		\draw (3.center) to (1);
		\draw (1) to (4.center);
		\draw [bend right=90, looseness=1.75] (4.center) to (0.center);
		\draw (0.center) to (2.center);
		\draw [bend left=75, looseness=1.50] (2.center) to (12.center);
		\draw (12.center) to (13.center);
		\draw (9.center) to (22);
		\draw (22) to (8.center);
		\draw (7.center) to (6.center);
		\draw (3.center) to (26);
		\draw (26) to (5.center);
	\end{pgfonlayer}
\end{tikzpicture} =
\]
\[
 %big5
\begin{tikzpicture}
	\begin{pgfonlayer}{nodelayer}
		\node [style=none] (0) at (-3, 3.5) {};
		\node [style=none] (1) at (-3, 2) {};
		\node [style=none] (2) at (0.5, 0.25) {};
		\node [style=none] (3) at (-1, 3.5) {};
		\node [style=none] (4) at (0.5, -1.75) {};
		\node [style=none] (5) at (-3, -1.75) {};
		\node [style=none] (6) at (-3, -2) {};
		\node [style=none] (7) at (-3, -2) {};
		\node [style=none] (8) at (-3, -0.5) {};
		\node [style=none] (9) at (-4, -0.5) {};
		\node [style=none] (10) at (-4, -5.25) {};
		\node [style=none] (11) at (-3.75, 2) {};
		\node [style=none] (12) at (-3.75, 4.5) {};
		\node [style=none] (13) at (-4.25, 4) {$\bar{\left(\bar{X^*}\right)^*}$};
		\node [style=none] (14) at (-3.5, 1.5) {$\bar{\epsilon*}$};
		\node [style=none] (15) at (-3.25, 2.75) {$\bar{\bar{X^*}}$};
		\node [style=none] (16) at (-2, 4.75) {$*\eta$};
		\node [style=none] (17) at (-0.5, 3.5) {$~^*(\bar{\bar{X^*}})$};
		\node [style=none] (18) at (-1, -4) {$\epsilon*$};
		\node [style=none] (19) at (-2.5, -2) {$X^*$};
		\node [style=none] (20) at (-4.25, -4.75) {$X$};
		\node [style=circle, scale=2.5] (21) at (-3, -1.25) {};
		\node [style=none] (22) at (-3, -1.25) {$\psi^{-1}$};
		\node [style=none] (23) at (-3.5, 0.25) {$*\eta$};
		\node [style=none] (24) at (-2.75, -0.5) {$~^*X$};
		\node [style=circle, scale=2.5] (25) at (0.5, -1) {};
		\node [style=none] (26) at (0.5, -1) {$\psi^{-1}$};
		\node [style=none] (27) at (0.75, -0.25) {$~^*(X^*)$};
		\node [style=none] (28) at (1, -2) {$X^{**}$};
		\node [style=circle, scale=2.5] (29) at (-0.25, 1.75) {};
		\node [style=none] (30) at (-0.25, 2.5) {};
		\node [style=none] (31) at (-0.25, 1) {};
		\node [style=none] (32) at (-1, 1) {};
		\node [style=none] (33) at (0.5, 2.5) {};
		\node [style=none] (34) at (0.5, 0.25) {};
		\node [style=none] (35) at (-1, 3.5) {};
		\node [style=none] (36) at (-0.25, 1.75) {$\varepsilon^{-1}$};
		\node [style=none] (37) at (0.25, 3) {$~^*\eta$};
		\node [style=none] (38) at (-0.75, 0.5) {$~^*\epsilon$};
	\end{pgfonlayer}
	\begin{pgfonlayer}{edgelayer}
		\draw (9.center) to (10.center);
		\draw [bend left=90, looseness=1.50] (9.center) to (8.center);
		\draw [bend right=90, looseness=1.75] (5.center) to (4.center);
		\draw [bend right=90, looseness=1.75] (3.center) to (0.center);
		\draw (0.center) to (1.center);
		\draw [bend left=75, looseness=1.50] (1.center) to (11.center);
		\draw (11.center) to (12.center);
		\draw (8.center) to (21);
		\draw (21) to (7.center);
		\draw (6.center) to (5.center);
		\draw (2.center) to (25);
		\draw (25) to (4.center);
		\draw (29) to (31.center);
		\draw [bend left=90, looseness=1.25] (31.center) to (32.center);
		\draw (32.center) to (35.center);
		\draw (30.center) to (29);
		\draw [bend left=90, looseness=1.50] (30.center) to (33.center);
		\draw (33.center) to (34.center);
	\end{pgfonlayer}
\end{tikzpicture}
= %big6
\begin{tikzpicture}
	\begin{pgfonlayer}{nodelayer}
		\node [style=none] (0) at (0.5, 0.25) {};
		\node [style=none] (1) at (0.5, -1.75) {};
		\node [style=none] (2) at (-3, -1.75) {};
		\node [style=none] (3) at (-3, -2) {};
		\node [style=none] (4) at (-3, -2) {};
		\node [style=none] (5) at (-3, -0.5) {};
		\node [style=none] (6) at (-4, -0.5) {};
		\node [style=none] (7) at (-4, -5.25) {};
		\node [style=none] (8) at (-1.5, 2.75) {$\bar{\left(\bar{X^*}\right)^*}$};
		\node [style=none] (9) at (-1, -4) {$\epsilon*$};
		\node [style=none] (10) at (-2.5, -2) {$X^*$};
		\node [style=none] (11) at (-4.25, -4.75) {$X$};
		\node [style=circle, scale=2.5] (12) at (-3, -1.25) {};
		\node [style=none] (13) at (-3, -1.25) {$\psi^{-1}$};
		\node [style=none] (14) at (-3.5, 0.25) {$*\eta$};
		\node [style=none] (15) at (-2.75, -0.5) {$~^*X$};
		\node [style=circle, scale=2.5] (16) at (0.5, -1) {};
		\node [style=none] (17) at (0.5, -1) {$\psi^{-1}$};
		\node [style=none] (18) at (1, -0.25) {$~^*(X^*)$};
		\node [style=none] (19) at (1, -2) {$X^{**}$};
		\node [style=circle, scale=2.5] (20) at (-0.25, 1.75) {};
		\node [style=none] (21) at (-0.25, 2.5) {};
		\node [style=none] (22) at (-0.25, 1) {};
		\node [style=none] (23) at (-1, 1) {};
		\node [style=none] (24) at (0.5, 2.5) {};
		\node [style=none] (25) at (0.5, 0.25) {};
		\node [style=none] (26) at (-1, 3.5) {};
		\node [style=none] (27) at (-0.25, 1.75) {$\varepsilon^{-1}$};
		\node [style=none] (28) at (0.25, 3) {$~^*\eta$};
		\node [style=none] (29) at (-0.75, 0.5) {$\bar{\epsilon*}$};
	\end{pgfonlayer}
	\begin{pgfonlayer}{edgelayer}
		\draw (6.center) to (7.center);
		\draw [bend left=90, looseness=1.50] (6.center) to (5.center);
		\draw [bend right=90, looseness=1.75] (2.center) to (1.center);
		\draw (5.center) to (12);
		\draw (12) to (4.center);
		\draw (3.center) to (2.center);
		\draw (0.center) to (16);
		\draw (16) to (1.center);
		\draw (20) to (22.center);
		\draw [bend left=90, looseness=1.25] (22.center) to (23.center);
		\draw (23.center) to (26.center);
		\draw (21.center) to (20);
		\draw [bend left=90, looseness=1.50] (21.center) to (24.center);
		\draw (24.center) to (25.center);
	\end{pgfonlayer}
\end{tikzpicture}  \stackrel{{\bf \small [C.2](inv)}}{=} %big7
\begin{tikzpicture}
	\begin{pgfonlayer}{nodelayer}
		\node [style=none] (0) at (0.5, -0) {};
		\node [style=none] (1) at (0.5, -2) {};
		\node [style=none] (2) at (1.5, -2) {};
		\node [style=none] (3) at (1.5, -0.5) {};
		\node [style=none] (4) at (2.5, -0.5) {};
		\node [style=none] (5) at (2.5, -5.25) {};
		\node [style=none] (6) at (-1.5, 2.75) {$\bar{\left(\bar{X^*}\right)^*}$};
		\node [style=none] (7) at (2.75, -4.75) {$X$};
		\node [style=circle, scale=2.5] (8) at (-0.25, 1.75) {};
		\node [style=none] (9) at (-0.25, 2.5) {};
		\node [style=none] (10) at (-0.25, 1) {};
		\node [style=none] (11) at (-1, 1) {};
		\node [style=none] (12) at (0.5, 2.5) {};
		\node [style=none] (13) at (0.5, -0) {};
		\node [style=none] (14) at (-1, 3.5) {};
		\node [style=none] (15) at (-0.25, 1.75) {$\varepsilon^{-1}$};
		\node [style=none] (16) at (0.25, 3) {$*\eta$};
		\node [style=none] (17) at (-0.75, 0.5) {$\bar{\epsilon*}$};
		\node [style=none] (18) at (-0.2, -1) {$~^*(X^*)$};
		\node [style=none] (19) at (-0.5, 2.25) {};
		\node [style=none] (20) at (1, -2.75) {$*\epsilon$};
		\node [style=none] (21) at (1.85, -1) {$X^*$};
		\node [style=none] (22) at (2, 0.25) {$\eta*$};
	\end{pgfonlayer}
	\begin{pgfonlayer}{edgelayer}
		\draw (4.center) to (5.center);
		\draw [bend right=90, looseness=1.50] (4.center) to (3.center);
		\draw [bend left=90, looseness=1.75] (2.center) to (1.center);
		\draw (8) to (10.center);
		\draw [bend left=90, looseness=1.25] (10.center) to (11.center);
		\draw (11.center) to (14.center);
		\draw (9.center) to (8);
		\draw [bend left=90, looseness=1.50] (9.center) to (12.center);
		\draw (12.center) to (13.center);
		\draw (0.center) to (1.center);
		\draw (2.center) to (3.center);
	\end{pgfonlayer}
\end{tikzpicture} = %big8
\begin{tikzpicture}
	\begin{pgfonlayer}{nodelayer}
		\node [style=none] (0) at (1.25, 0.5) {};
		\node [style=none] (1) at (1.25, 2) {};
		\node [style=none] (2) at (2.5, 2) {};
		\node [style=none] (3) at (2.5, -5.25) {};
		\node [style=none] (4) at (-0.75, 2.75) {$\bar{\left(\bar{X^*}\right)^*}$};
		\node [style=none] (5) at (2.75, -4.75) {$X$};
		\node [style=circle, scale=2.5] (6) at (1.25, -0.25) {};
		\node [style=none] (7) at (1.25, 0.5) {};
		\node [style=none] (8) at (1.25, -1) {};
		\node [style=none] (9) at (0, -1) {};
		\node [style=none] (10) at (0, 3.5) {};
		\node [style=none] (11) at (1.25, -0.25) {$\varepsilon^{-1}$};
		\node [style=none] (12) at (0.75, -1.75) {$\bar{\epsilon*}$};
		\node [style=none] (13) at (1, 0.25) {};
		\node [style=none] (14) at (1.5, 1.5) {$X^*$};
		\node [style=none] (15) at (1.75, 2.75) {$\eta*$};
	\end{pgfonlayer}
	\begin{pgfonlayer}{edgelayer}
		\draw (2.center) to (3.center);
		\draw [bend right=90, looseness=1.50] (2.center) to (1.center);
		\draw (6) to (8.center);
		\draw [bend left=90, looseness=1.25] (8.center) to (9.center);
		\draw (9.center) to (10.center);
		\draw (7.center) to (6);
		\draw (0.center) to (1.center);
	\end{pgfonlayer}
\end{tikzpicture} =: \iota^{-1}
\]
\[
(b)~~~~ \begin{tikzpicture}
	\begin{pgfonlayer}{nodelayer}
		\node [style=none] (0) at (0, 4.75) {};
		\node [style=none] (1) at (0, -0.5) {};
		\node [style=none] (2) at (-0.75, -0.5) {};
		\node [style=none] (3) at (-0.75, 2) {};
		\node [style=none] (4) at (-1.5, 2) {};
		\node [style=none] (5) at (-1.5, -0) {};
		\node [style=none] (6) at (-2.5, -0) {};
		\node [style=circle, scale=2.5] (7) at (-2.5, 1) {};
		\node [style=none] (8) at (-2.5, 2.5) {};
		\node [style=none] (9) at (-3.75, 2.5) {};
		\node [style=none] (10) at (-3.75, -2.75) {};
		\node [style=none] (11) at (-2.5, 1) {$\psi$};
		\node [style=none] (12) at (-0.5, -1.25) {$\epsilon*$};
		\node [style=none] (13) at (-1.25, 2.75) {$\bar{\eta*}$};
		\node [style=none] (14) at (-2, -0.75) {$*\epsilon$};
		\node [style=none] (15) at (-3, 3.5) {$\eta*$};
		\node [style=circle, scale=2.5] (16) at (-3.75, -3.75) {};
		\node [style=none] (17) at (-3.75, -4.75) {};
		\node [style=none] (18) at (-2.5, -4.75) {};
		\node [style=none] (19) at (-2.5, -3.5) {};
		\node [style=none] (20) at (-1.25, -3.5) {};
		\node [style=circle, scale=2.5] (21) at (-1.25, -6.75) {};
		\node [style=none] (22) at (-1.25, -8) {};
		\node [style=none] (23) at (-3.75, -3.75) {$\psi$};
		\node [style=none] (24) at (-1.25, -6.75) {$\varepsilon$};
		\node [style=none] (25) at (-2, -2.75) {$\eta*$};
		\node [style=none] (26) at (-3, -5.5) {$*\epsilon$};
		\node [style=none] (27) at (-4.75, 3.75) {};
		\node [style=none] (28) at (1.25, 3.75) {};
		\node [style=none] (29) at (-4.75, -1.5) {};
		\node [style=none] (30) at (1.25, -1.5) {};
		\node [style=none] (31) at (-5, -2.25) {};
		\node [style=none] (32) at (-0.25, -2.25) {};
		\node [style=none] (33) at (-0.25, -6) {};
		\node [style=none] (34) at (-5, -6) {};
		\node [style=none] (35) at (0.5, 4.5) {$\bar{\left( \bar{X^*} \right)^*}$};
		\node [style=none] (36) at (-2.75, -2) {$\bar{\left(\bar{X} \right)^{**}}$};
		\node [style=none] (37) at (-0.75, -7.5) {$X$};
		\node [style=none] (38) at (-0.75, -5.25) {$\bar{X}$};
		\node [style=none] (39) at (-2, -4) {$\bar{X}^*$};
		\node [style=none] (40) at (-4.25, -4.5) {$~^*(\bar{X}^*)$};
		\node [style=none] (41) at (-4.25, -3) {$\bar{X}^{**}$};
		\node [style=none] (42) at (-4, 2) {$\bar{X}^{**}$};
		\node [style=none] (43) at (0.5, 2.5) {$\bar{X^*}^*$};
		\node [style=none] (44) at (-1.25, 1) {$\bar{X^*}$};
		\node [style=none] (45) at (-3, 0.25) {$~^*\bar{\left(X^*\right)}$};
		\node [style=none] (46) at (-3, 2) {$\bar{ \left( X^* \right)}^*$};
	\end{pgfonlayer}
	\begin{pgfonlayer}{edgelayer}
		\draw (0.center) to (1.center);
		\draw [bend left=90, looseness=2.00] (1.center) to (2.center);
		\draw (2.center) to (3.center);
		\draw [bend right=90, looseness=1.50] (3.center) to (4.center);
		\draw (4.center) to (5.center);
		\draw [bend left=90, looseness=1.25] (5.center) to (6.center);
		\draw (6.center) to (7);
		\draw (7) to (8.center);
		\draw [bend right=90, looseness=1.75] (8.center) to (9.center);
		\draw (9.center) to (10.center);
		\draw (10.center) to (16);
		\draw (16) to (17.center);
		\draw [bend right=90, looseness=1.50] (17.center) to (18.center);
		\draw (18.center) to (19.center);
		\draw [bend left=90, looseness=1.50] (19.center) to (20.center);
		\draw (20.center) to (21);
		\draw (21) to (22.center);
		\draw (27.center) to (28.center);
		\draw (28.center) to (30.center);
		\draw (30.center) to (29.center);
		\draw (29.center) to (27.center);
		\draw (31.center) to (34.center);
		\draw (34.center) to (33.center);
		\draw (32.center) to (33.center);
		\draw (32.center) to (31.center);
	\end{pgfonlayer}
\end{tikzpicture} = %big10
\begin{tikzpicture}
	\begin{pgfonlayer}{nodelayer}
		\node [style=none] (0) at (0, 4.75) {};
		\node [style=none] (1) at (0, -0.5) {};
		\node [style=none] (2) at (-0.75, -0.5) {};
		\node [style=none] (3) at (-0.75, 2) {};
		\node [style=none] (4) at (-1.5, 2) {};
		\node [style=none] (5) at (-1.5, -0) {};
		\node [style=none] (6) at (-2.5, -0) {};
		\node [style=circle, scale=2.5] (7) at (-2.5, 1) {};
		\node [style=none] (8) at (-2.5, 2.5) {};
		\node [style=none] (9) at (-3.75, 2.5) {};
		\node [style=none] (10) at (-3.75, -2.75) {};
		\node [style=none] (11) at (-2.5, 1) {$\psi$};
		\node [style=none] (12) at (-0.5, -1.25) {$\epsilon*$};
		\node [style=none] (13) at (-1.25, 2.75) {$\bar{\eta*}$};
		\node [style=none] (14) at (-2, -0.75) {$*\epsilon$};
		\node [style=none] (15) at (-3, 3.5) {$\eta*$};
		\node [style=circle, scale=2.5] (16) at (-3.75, -3.75) {};
		\node [style=none] (17) at (-3.75, -4.75) {};
		\node [style=none] (18) at (-2.5, -4.75) {};
		\node [style=none] (19) at (-2.5, -3.5) {};
		\node [style=none] (20) at (-1.25, -3.5) {};
		\node [style=circle, scale=2.5] (21) at (-1.25, -6.75) {};
		\node [style=none] (22) at (-1.25, -8) {};
		\node [style=none] (23) at (-3.75, -3.75) {$\psi$};
		\node [style=none] (24) at (-1.25, -6.75) {$\varepsilon$};
		\node [style=none] (25) at (-2, -2.75) {$\eta*$};
		\node [style=none] (26) at (-3, -5.5) {$*\epsilon$};
		\node [style=none] (27) at (-5.5, 3.75) {};
		\node [style=none] (28) at (1.25, 3.75) {};
		\node [style=none] (29) at (1.25, 3.75) {};
		\node [style=none] (30) at (1.25, -6) {};
		\node [style=none] (31) at (-5.5, -6) {};
		\node [style=none] (32) at (0.5, 4.5) {$\bar{\left( \bar{X^*} \right)^*}$};
		\node [style=none] (33) at (-2.75, -2) {$\bar{\left(\bar{X} \right)^{**}}$};
		\node [style=none] (34) at (-0.75, -7.5) {$X$};
		\node [style=none] (35) at (-0.75, -5.25) {$\bar{X}$};
		\node [style=none] (36) at (-2, -4) {$\left(\bar{X}\right)^*$};
		\node [style=none] (37) at (-4.5, -4.5) {$~^*(\left(\bar{X}\right)^*)$};
		\node [style=none] (38) at (-4.5, -3) {$\bar{X}^{**}$};
		\node [style=none] (39) at (-4, 2) {$\bar{X}^{**}$};
		\node [style=none] (40) at (0.5, 2.5) {$\bar{X^*}^*$};
		\node [style=none] (41) at (-1.25, 1) {$\bar{X^*}$};
		\node [style=none] (42) at (-3, 0.25) {$~^*\left(\bar{X^*} \right)$};
		\node [style=none] (43) at (-3, 2) {$\left(\bar{X^*} \right)^*$};
		\node [style=none] (44) at (0.75, -5.5) {$\bar{(\_)}$};
	\end{pgfonlayer}
	\begin{pgfonlayer}{edgelayer}
		\draw (0.center) to (1.center);
		\draw [bend left=90, looseness=2.00] (1.center) to (2.center);
		\draw (2.center) to (3.center);
		\draw [bend right=90, looseness=1.50] (3.center) to (4.center);
		\draw (4.center) to (5.center);
		\draw [bend left=90, looseness=1.25] (5.center) to (6.center);
		\draw (6.center) to (7);
		\draw (7) to (8.center);
		\draw [bend right=90, looseness=1.75] (8.center) to (9.center);
		\draw (9.center) to (10.center);
		\draw (10.center) to (16);
		\draw (16) to (17.center);
		\draw [bend right=90, looseness=1.50] (17.center) to (18.center);
		\draw (18.center) to (19.center);
		\draw [bend left=90, looseness=1.50] (19.center) to (20.center);
		\draw (20.center) to (21);
		\draw (21) to (22.center);
		\draw (27.center) to (28.center);
		\draw (31.center) to (30.center);
		\draw (29.center) to (30.center);
		\draw (27.center) to (31.center);
	\end{pgfonlayer}
\end{tikzpicture} \stackrel{{\bf [C.2]}}{=} %big11
\begin{tikzpicture}
	\begin{pgfonlayer}{nodelayer}
		\node [style=none] (0) at (0, 4.75) {};
		\node [style=none] (1) at (0, -0.5) {};
		\node [style=none] (2) at (-0.75, -0.5) {};
		\node [style=none] (3) at (-0.75, 1.25) {};
		\node [style=none] (4) at (-4.5, 1.25) {};
		\node [style=none] (5) at (-4.5, -0.25) {};
		\node [style=none] (6) at (-1.75, -4.75) {};
		\node [style=none] (7) at (-1.75, -3.5) {};
		\node [style=none] (8) at (-0.5, -3.5) {};
		\node [style=circle, scale=2.5] (9) at (-0.5, -6.75) {};
		\node [style=none] (10) at (-0.5, -8) {};
		\node [style=none] (11) at (-0.5, -6.75) {$\varepsilon$};
		\node [style=none] (12) at (-1.25, -2.75) {$\eta*$};
		\node [style=none] (13) at (-5, 3.75) {};
		\node [style=none] (14) at (1.25, 3.75) {};
		\node [style=none] (15) at (1.25, 3.75) {};
		\node [style=none] (16) at (1.25, -6) {};
		\node [style=none] (17) at (-5, -6) {};
		\node [style=none] (18) at (0.5, 4.5) {$\bar{\left( \bar{X^*} \right)^*}$};
		\node [style=none] (19) at (0, -7.5) {$X$};
		\node [style=none] (20) at (0, -5.25) {$\bar{X}$};
		\node [style=none] (21) at (-1.25, -4) {$\left(\bar{X}\right)^*$};
		\node [style=none] (22) at (0.5, 2.5) {$\bar{X^*}^*$};
		\node [style=none] (23) at (0.75, -5.5) {$\bar{(\_)}$};
		\node [style=none] (24) at (-2.25, -4.75) {};
		\node [style=none] (25) at (-2.25, -1.25) {};
		\node [style=none] (26) at (-3.25, -1.25) {};
		\node [style=none] (27) at (-3.25, -2.5) {};
		\node [style=none] (28) at (-4.5, -2.5) {};
		\node [style=none] (29) at (-0.5, -1.25) {$\epsilon*$};
		\node [style=none] (30) at (-2.5, 3.25) {$\bar{\eta*}$};
		\node [style=none] (31) at (-4, -3.75) {$\epsilon*$};
		\node [style=none] (32) at (-2.75, -0.75) {$*\eta$};
		\node [style=none] (33) at (-2, -5.25) {$*\epsilon$};
		\node [style=none] (34) at (-1.25, 0.75) {$\bar{X^*}$};
		\node [style=none] (35) at (-4, 1) {$\bar{X}$};
		\node [style=none] (36) at (-3.5, -2) {$\bar{X}^*$};
		\node [style=none] (37) at (-1.75, -2.25) {$~^*(\bar{X}^*)$};
	\end{pgfonlayer}
	\begin{pgfonlayer}{edgelayer}
		\draw (0.center) to (1.center);
		\draw [bend left=90, looseness=2.00] (1.center) to (2.center);
		\draw (2.center) to (3.center);
		\draw [bend right=90, looseness=1.50] (3.center) to (4.center);
		\draw (4.center) to (5.center);
		\draw (6.center) to (7.center);
		\draw [bend left=90, looseness=1.50] (7.center) to (8.center);
		\draw (8.center) to (9);
		\draw (9) to (10.center);
		\draw (13.center) to (14.center);
		\draw (17.center) to (16.center);
		\draw (15.center) to (16.center);
		\draw (13.center) to (17.center);
		\draw [bend left=90, looseness=1.75] (6.center) to (24.center);
		\draw (24.center) to (25.center);
		\draw (27.center) to (26.center);
		\draw [bend left=90, looseness=1.25] (26.center) to (25.center);
		\draw [bend right=90, looseness=2.50] (28.center) to (27.center);
		\draw (5.center) to (28.center);
	\end{pgfonlayer}
\end{tikzpicture} = %big12
\begin{tikzpicture}
	\begin{pgfonlayer}{nodelayer}
		\node [style=none] (0) at (0, 4.75) {};
		\node [style=none] (1) at (0, -0.5) {};
		\node [style=none] (2) at (-0.75, -0.5) {};
		\node [style=none] (3) at (-0.75, 1.25) {};
		\node [style=none] (4) at (-4.5, 1.25) {};
		\node [style=none] (5) at (-4.5, -0.25) {};
		\node [style=circle, scale=2.5] (6) at (-2.25, -6.75) {};
		\node [style=none] (7) at (-2.25, -8) {};
		\node [style=none] (8) at (-2.25, -6.75) {$\varepsilon$};
		\node [style=none] (9) at (-5, 3.75) {};
		\node [style=none] (10) at (1.25, 3.75) {};
		\node [style=none] (11) at (1.25, 3.75) {};
		\node [style=none] (12) at (1.25, -6) {};
		\node [style=none] (13) at (-5, -6) {};
		\node [style=none] (14) at (0.5, 4.5) {$\bar{\left( \bar{X^*} \right)^*}$};
		\node [style=none] (15) at (-1.75, -7.5) {$X$};
		\node [style=none] (16) at (0.5, 2.5) {$\bar{X^*}^*$};
		\node [style=none] (17) at (0.75, -5.5) {$\bar{(\_)}$};
		\node [style=none] (18) at (-2.25, -1.25) {};
		\node [style=none] (19) at (-3.25, -1.25) {};
		\node [style=none] (20) at (-3.25, -2.5) {};
		\node [style=none] (21) at (-4.5, -2.5) {};
		\node [style=none] (22) at (-0.5, -1.25) {$\epsilon*$};
		\node [style=none] (23) at (-2.5, 3.25) {$\bar{\eta*}$};
		\node [style=none] (24) at (-4, -3.75) {$\epsilon*$};
		\node [style=none] (25) at (-2.75, -0.75) {$\eta*$};
		\node [style=none] (26) at (-1.25, 0.75) {$\bar{X^*}$};
		\node [style=none] (27) at (-4, 1) {$\bar{X}$};
		\node [style=none] (28) at (-3.5, -2) {$\bar{X}^*$};
	\end{pgfonlayer}
	\begin{pgfonlayer}{edgelayer}
		\draw (0.center) to (1.center);
		\draw [bend left=90, looseness=2.00] (1.center) to (2.center);
		\draw (2.center) to (3.center);
		\draw [bend right=90, looseness=1.50] (3.center) to (4.center);
		\draw (4.center) to (5.center);
		\draw (6) to (7.center);
		\draw (9.center) to (10.center);
		\draw (13.center) to (12.center);
		\draw (11.center) to (12.center);
		\draw (9.center) to (13.center);
		\draw (20.center) to (19.center);
		\draw [bend left=90, looseness=1.25] (19.center) to (18.center);
		\draw [bend right=90, looseness=2.50] (21.center) to (20.center);
		\draw (5.center) to (21.center);
		\draw (18.center) to (6);
	\end{pgfonlayer}
\end{tikzpicture}= \]
\[%big 14
\begin{tikzpicture}
	\begin{pgfonlayer}{nodelayer}
		\node [style=none] (0) at (0, 4.75) {};
		\node [style=none] (1) at (0, -0.5) {};
		\node [style=none] (2) at (-0.75, -0.5) {};
		\node [style=none] (3) at (-0.75, 2) {};
		\node [style=none] (4) at (-2.25, 2) {};
		\node [style=circle, scale=2.5] (5) at (-2.25, -2.75) {};
		\node [style=none] (6) at (-2.25, -4) {};
		\node [style=none] (7) at (-2.25, -2.75) {$\varepsilon$};
		\node [style=none] (8) at (-5, 3.75) {};
		\node [style=none] (9) at (1.25, 3.75) {};
		\node [style=none] (10) at (1.25, 3.75) {};
		\node [style=none] (11) at (1.25, -2) {};
		\node [style=none] (12) at (-5, -2) {};
		\node [style=none] (13) at (0.75, 4.5) {$\bar{\left( \bar{X^*} \right)^*}$};
		\node [style=none] (14) at (-1.75, -3.5) {$X$};
		\node [style=none] (15) at (0.5, 2.5) {$\bar{X^*}^*$};
		\node [style=none] (16) at (0.7499999, -1.5) {$\bar{(\_)}$};
		\node [style=none] (17) at (-2.25, -1.25) {};
		\node [style=none] (18) at (-0.5, -1.25) {$\epsilon*$};
		\node [style=none] (19) at (-1.5, 3) {$\bar{\eta*}$};
		\node [style=none] (20) at (-1.25, 0.75) {$\bar{X^*}$};
		\node [style=none] (21) at (-1.75, 1.75) {$\bar{X}$};
	\end{pgfonlayer}
	\begin{pgfonlayer}{edgelayer}
		\draw (0.center) to (1.center);
		\draw [bend left=90, looseness=2.00] (1.center) to (2.center);
		\draw (2.center) to (3.center);
		\draw [bend right=90, looseness=1.50] (3.center) to (4.center);
		\draw (5) to (6.center);
		\draw (8.center) to (9.center);
		\draw (12.center) to (11.center);
		\draw (10.center) to (11.center);
		\draw (8.center) to (12.center);
		\draw (17.center) to (5);
		\draw (4.center) to (17.center);
	\end{pgfonlayer}
\end{tikzpicture} = %big15
\begin{tikzpicture}
	\begin{pgfonlayer}{nodelayer}
		\node [style=none] (0) at (-2.5, 4.75) {};
		\node [style=none] (1) at (-2.5, -0.5) {};
		\node [style=none] (2) at (-1, -0.5) {};
		\node [style=none] (3) at (-1, 2) {};
		\node [style=none] (4) at (0.5, 2) {};
		\node [style=circle, scale=2.5] (5) at (0.4999999, -1.25) {};
		\node [style=none] (6) at (0.4999999, -2.5) {};
		\node [style=none] (7) at (0.4999999, -1.25) {$\varepsilon$};
		\node [style=none] (8) at (-3, 4.5) {$\bar{\left( \bar{X^*} \right)^*}$};
		\node [style=none] (9) at (0.9999999, -2) {$X$};
		\node [style=none] (10) at (0.5, -1.25) {};
		\node [style=none] (11) at (-1.75, -1.75) {$\bar{\epsilon*}$};
		\node [style=none] (12) at (-0.25, 3) {$\bar{\bar{\eta*}}$};
		\node [style=none] (13) at (-1.5, 1) {$\bar{\bar{X^*}}$};
		\node [style=none] (14) at (1, 1) {$\bar{\bar{X}}$};
	\end{pgfonlayer}
	\begin{pgfonlayer}{edgelayer}
		\draw (0.center) to (1.center);
		\draw [bend right=90, looseness=2.00] (1.center) to (2.center);
		\draw (2.center) to (3.center);
		\draw [bend left=90, looseness=1.50] (3.center) to (4.center);
		\draw (5) to (6.center);
		\draw (10.center) to (5);
		\draw (4.center) to (5);
	\end{pgfonlayer}
\end{tikzpicture} \stackrel{\ref{Lemma: varepsi monoidal}}{=}
%big8
\begin{tikzpicture}
	\begin{pgfonlayer}{nodelayer}
		\node [style=none] (0) at (1.25, 0.5) {};
		\node [style=none] (1) at (1.25, 2) {};
		\node [style=none] (2) at (2.5, 2) {};
		\node [style=none] (3) at (2.5, -2.5) {};
		\node [style=none] (4) at (-0.75, 2.75) {$\bar{\bar{X^*}^*}$};
		\node [style=none] (5) at (2.75, -2) {$X$};
		\node [style=circle, scale=2.5] (6) at (1.25, -0.25) {};
		\node [style=none] (7) at (1.25, 0.5) {};
		\node [style=none] (8) at (1.25, -1) {};
		\node [style=none] (9) at (0, -1) {};
		\node [style=none] (10) at (0, 3.5) {};
		\node [style=none] (11) at (1.25, -0.25) {$\varepsilon^{-1}$};
		\node [style=none] (12) at (0.75, -1.75) {$\bar{\epsilon*}$};
		\node [style=none] (13) at (1, 0.25) {};
		\node [style=none] (14) at (1.5, 1.5) {$X^*$};
		\node [style=none] (15) at (1.75, 2.75) {$\eta*$};
	\end{pgfonlayer}
	\begin{pgfonlayer}{edgelayer}
		\draw (2.center) to (3.center);
		\draw [bend right=90, looseness=1.50] (2.center) to (1.center);
		\draw (6) to (8.center);
		\draw [bend left=90, looseness=1.25] (8.center) to (9.center);
		\draw (9.center) to (10.center);
		\draw (7.center) to (6);
		\draw (0.center) to (1.center);
	\end{pgfonlayer}
\end{tikzpicture} =: \iota^{-1}
\] 
\end{proof}

Observe that for composition of the dualizing functor and the conjugation functor to yield a dagger, and 
vice versa, a $*$-autonomous category is required to be cyclic with the cyclor being preserved by the 
conjugation (see Definition \ref{Defn: conjugative cyclic}) and the dagger (see just before Lemma \ref{Lemma: cyclic dagger}). 

Combining Propositions \ref{Prop: dagger+dualizing} and \ref{Theorem: conjugation+dualizing}, we get:

\begin{theorem} 
	Every cyclic $*$-autonomous category is conjugative $*$-autonomous if and only if it is 
	$\dagger$-$*$-autonomous.
\end{theorem}
	

%%%%%%%%%%%%%%%%%%%%%%%%%%%%%%%%%%%%%%%%%%%%%%%%%%%%%%%%%%%%%%%%%%%%%%
\section{Examples: Dagger and conjugation}
\label{Sec: Examples Dagger and conjugation}

Let us now look at some examples of $\dagger$-isomix categories in which the dagger is given 
by dualizing and conjugation functors. 

\subsection{A group with conjugation considered as a category}

\begin{definition}
A {\bf group with conjugation} is a group $(G, ., e)$  together with a function $\overline{(\_)}: G \to G$ such that, 
for all $g \in G$, $\overline{\overline{g}} = g$, and for all $g, h \in G$, $\overline{g . h} = \overline{h} \overline{g}$, and 
$\overline e = e$.
\end{definition}

Let $(G,., e)$ be a group with conjugation. The discrete category $\D{ (G,., e})$ whose objects are the elements 
of the group is a monoidal category with the tensor product given by $g \ox h := g.h$, and the monoidal unit $e$. 
Moreover, $\D{(G,.,e)}$ is a compact closed category where $g^* := g^{-1}$, and it has a trivial conjugative cyclor 
(see Definition \ref{Defn: conjugative cyclic}). Thus, $\D{(G,.,e})$ which is a compact $\dagger$-isomix-$*$-autonomous 
category with $g^\dagger := \overline{g^*}$ is an example of how the conjugation gives rise to a dagger.

Here are some examples of groups with conjugation and the discrete categories given by them:
\begin{itemize}
\item Suppose we fix the group to be $(\C, +, 0)$ where the objects are complex numbers and the tensor product is addition. 
The dual and conjugation of complex numbers are given as follows: $(a+ib)^* = -a - ib$ and $\overline{a + ib} := a - ib$. 
Hence, \[(a + ib)^\dagger := \overline{(a+ib)^*} = \overline{(-a-ib)} = -a + ib\] 
\item Consider the multiplicative group $(\C^*, ., 1)$ where the objects are non-zero complex numbers and the 
tensor product is given by multiplication. The dualizing and the conjugation functors are given as follows: 
\[ (a+ib)^* = c+id, \text{ where } ac-bd=1 \text { and }  ad+bc=0 \]  \[ \overline{a + ib} := a-ib \] $(a+ib)^\dagger$ 
is given by $\overline{(a+ib)^*}$.
\item Suppose the group is fixed to be $\D{(P(x), +, 0})$ where $P(x)$ is a polynomial ring. $\D{(P(x), +, 0})$ is a 
conjugative compact closed category: $P(x)^* = -P(x)$ and $\overline{P(x)} = P(-x)$. Then, $P(x)^\dagger = -P(-x)$.
\item Consider the general linear group of degree 2, $(\mathbb{M}_2, . ,I_2)$ over complex numbers. Then, 
the discrete category $\D(\mathbb{M}_2, ., I_2)$ has a dualizing functor given by matrix inverse and conjugation 
is given by conjugate transpose: $\overline{\left(
\begin{matrix}
a+ib & m+in \\
c+id & p+iq
\end{matrix}
\right)} :=  \left(
\begin{matrix}
a-ib & c-id \\
m-in & p-iq
\end{matrix}
\right)
$. Then,  $\D(\mathbb{M}_2, .,I_2)$ is a $\dagger$-isomix $*$-autonomous category with:
 \[ \left(
  \begin{matrix}
 a+ib & m+in \\
 c+id & p+iq 
 \end{matrix}
 \right)^\dagger := \overline{\left(
 \begin{matrix}
a+ib & m+in \\
c+id & p+iq 
 \end{matrix}
 \right)^*} =  \left(
 \begin{matrix}
a-ib & c-id \\
m-in & p-iq 
\end{matrix}
 \right)^{-1} \] 
\end{itemize}

\subsection{Finiteness matrices and finiteness relations}
\label{Sec: Finiteness matrices}

We now describe the conjugation, $\overline{(\_)}$ and, thus, the dagger functor, $(\_)^\dagger$, for ${\sf FRel}$, and ${\sf FMat}(R)$.  Recall that the dagger functor is obtained by composing the conjugacy dualizing functors:  $X^\dagger = \overline{X}^*$.

\begin{lemma}
	\label{Lemma: conjugative cat}
$\FRel$, $\FMat(R)$, where $R$ is a conjugative rig, are conjugative isomix $*$-autonomous categories.
\end{lemma}

A conjugative rig is a rig with a conjugation $\overline{(\_)}: R \to R$ such that $r = \overline{\overline{r}}$ and the conjugation preserves addition, $\overline{0} = 0$ and $\overline{r + s} =\overline{r}+\overline{s}$, and preserves the multiplication $\overline{1} = 1$ and $\overline{r_1 \cdot r_2} = \overline{r_1} \cdot \overline{r_2}$.  

\begin{proof}
For $\FRel$ the conjugation functor is identity on objects and arrows and thus the dagger is the duality, $X^\dagger = X^{*}$.  $\FRel$ is a conjugative $*$-autonomous category with the required natural isomorphisms defined as follows:
\begin{itemize}
\item $\overline{X} \ox \overline{Y} \to^{\chi_\ox} \overline{Y \ox X}$ and $\chi_\oa$ are the symmetry maps. 
\item $\overline{ \overline{ A}} \to^{\varepsilon} A$ is the identity map.
\end{itemize} 

For $\FMat(R)$ the conjugation functor for is determined by the conjugation on the rig. The functor is still identity on the objects and the symmetry map is used to provide $\chi_\ox$ and $\chi_\oa$. For a finiteness matrix, $M$, $\overline{M}$ is given by conjugating all the elements of $M$. $\chi_\ox$, $\chi_\oa$, $\varepsilon$ are the characteristic function of their corresponding finiteness relations.

Both are a conjugative isomix category because $\top = \bot$.
\end{proof}

\subsection{Chu spaces}
\label{Section: Chu}

Applications of Chu Spaces to represent quantum systems have been studied in \cite{Abr12}, \cite{Abr13}. 
In this section we show that the Chu construction over a closed conjugative monoidal category,
 which  has pullbacks, produces a $\dagger$-isomix LDC, ${\sf Chu}_\X(I)$.  To get 
 the $*$-autonomous category and $\dagger$-structure on ${\sf Chu}_\X(I)$ we shall start 
 by explaining how one can produce  conjugative structure on the ${\sf Chu}$ category.  
 To achieve this we develop the structure of this category, starting with a 
 conjugative closed monoidal category, $\X$, which is not necessarily  symmetric.  
 Note that the fact that it is conjugative means that it is both left and right closed which 
 allows us to consider the non-commutative ${\sf Chu}$ construction: in this regard  
 we shall follow J\"urgen Koslowski's construction \cite{Jur06} using simplified ``Chu-cells'' on the 
 same dualizing object to obtain not a ${*}$-linear bicategory but a cyclic  $*$-autonomous category.  
 Furthermore, we shall choose a dualizing object which is conjugative in order  to obtain a conjugative 
 cyclic $*$-autonomous  category.

A conjugative object is an object $D$ of $\X$ with an isomorphism $d: \overline{D} \to D$ such that 
$\overline{d}d = \varepsilon: D \to \overline{\overline{D}}$.  
We can then define ${\sf Chu}_\X(D)$ as follows:

\begin{description}
\item[Objects:] $(A, B, \psi_0, \psi_1)$ where $\psi_0: A \ox B \to D$ and $\psi_1: B \ox A \to D$  in $\X$ 
(these are the simplified Chu cells). %What are the simplified chu cells?
\item[Arrows:] $(f,g): (A, B, \psi_0,\psi_1) \to (A', B', \psi_0',\psi_1')$ where $f: A \to A'$ and $g: B' \to B$ 
and the following diagrams commutes:
\[ \xymatrix{
& A \ox B' \ar[dl]_{1 \ox g} \ar[dr]^{f \ox 1} & \\
A \ox B \ar[dr]_{\psi_0} & & A' \ox B' \ar[ld]^{\psi'_0} \\
& D &}
~~~~~\xymatrix{
& B' \ox A \ar[dl]_{g \ox 1} \ar[dr]^{1 \ox f} & \\
B \ox A \ar[dr]_{\psi_1} & & B' \ox A' \ar[ld]^{\psi'_1} \\
& D &}
\]
\item[Compositon:] $(f,g)(f',g') := (ff', g'g)$. Composition is well-defined as:
\[
\xymatrix{
&& A \ox B'' \ar[dl]_{1 \ox g'} \ar[dr]^{f \ox 1}   && \\
& A \ox B' \ar[dr]_{f \ox 1}  \ar[dl]_{1 \ox g}  & & A' \ox B'' \ar[dl]^{1 \ox g'}  \ar[dr]^{f' \ox 1} \\
A \ox B \ar[drr]^{\psi_0} & & A' \ox B' \ar[d]^{\psi'_0} && A'' \ox B'' \ar[lld]_{\psi_0''} \\
& & D & & 
}
\]
and similarly for the reverse Chu-maps: $\psi_1$, $\psi'_1$ and $\psi_1''$.
The {\bf identity maps} are $(1_A,1_B): (A, B, \psi_0,\psi_1) \to (A, B, \psi_0,\psi_1)$ as expected.

\item[Tensor product $\ox$:] $(A, B, \psi_0, \psi_1) \ox (A', B', \psi_0', \psi_1') := (A \ox A', E, \gamma_0, \gamma_1)$, 
where $E$ is the pullback in the following diagram:
\[
\xymatrix{
& E \ar[rd]^{\pi_1} \ar[ld]_{\pi_0} & \\ %Ask Robin how to add a pullback corner
A' \multimap B \ar[rd]_-{1 \multimap \tilde{\psi_1}} & & B' \poppilol A \ar[ld]^-{\tilde{\psi_1'} \poppilol A}  \\
& A' \multimap (D \poppilol A) \to^{\simeq} (A' \multimap D) \poppilol A &
}
\]
with
\[
\infer{B \ox A \to^{\psi_1} B}{B \to^{\tilde{\psi_1}} D \poppilol A} ~~~~~~~~~~ 
\infer{A' \ox B' \to D}{B' \to^{\tilde{\psi_1'}} A' \multimap D}
\]
and, 
\[
\gamma_0 := (A \ox A') \ox E \to^{1 \ox \pi_0} (A \ox A') \ox (A' \multimap B) \to^{a_\ox} A \ox ( A' \ox A' \multimap B) 
\to^{1 \ox eval_{\multimap}} A \ox B \to^{\psi_0'} D
\]
\[
\gamma_1 := E \ox (A \ox A')  \to^{\pi_1 \ox 1} (B' \poppilol A) \ox (A \ox A')  \to^{a_\ox^{-1}} 
(B' \poppilol A \ox A) \ox A' \to^{eval_{\poppilol} \ox 1} B' \ox A' \to^{\psi_1'} D
\]
The tensor unit is $(I, D, u_\ox^l, u_\ox^r)$.
\end{description}

It is standard that ${\sf Chu}_\X(D)$  is a (non-commutative) $*$-autonomous category. Furthermore, it is cyclic because 
$$~^{*}(A,B,\psi_0,\psi_1) = (A,B,\psi_0,\psi_1)^{*} = (B,A,\psi_1,\psi_0).$$ 
In addition, ${\sf Chu}_\X(D)$  is conjugative with 
$$\overline{(A,B,\psi_0,\psi_1)} := (\overline{A},\overline{B},\chi \overline{\psi_1}d,\chi \overline{\psi_0}d)$$
and $\overline{(f,g)} = (\overline{f},\overline{g})$.
Finally being conjugative cyclic $*$-autonomous implies that one has a dagger!

In the case that $\X$ is a symmetric monoidal closed category we may recapture the usual Chu construction \cite{Bar06}, 
which we denote ${\sf Chus}_\X(D)$.  Consider the full subcategory 
of Chu-objects with special Chu-cells of the form $(A,B, \psi,c_\otimes \psi)$ in which the symmetry map is used to 
obtain the second cell, this gives an inclusion 
${\sf Chus}_\X(D) \to {\sf Chu}_\X(D)$.  

We observe that $\X$ is symmetric conjugative when this subcategory is closed under the conjugation:

\begin{lemma}
	\label{Lemma: Chus lemma}
If $\X$  is an conjugative symmetric monoidal closed category  and $d: \overline{D} \to D$ is an involutive object, 
then ${\sf Chus}_\X(D)$ is a conjugative symmetric $*$-autonomous  category.
\end{lemma}
\begin{proof}
It suffices to observe that the Chu-cells of $\overline{(A,B,\psi,c_\otimes\psi)}$ have the right form.  Using the coherence of the 
involution with symmetry, the first Chu-cell of this object has  $\chi \overline{c_\otimes \psi}d = c_\ox \chi \overline{\psi}d$
which is exactly the symmetry map applied to the second Chu-cell of the object as desired.
\end{proof}

To obtain an isomix category one can choose D = I. ${\sf Chus}_\X(I)$ is an isomix category because the unit for tensor 
and par are the same (namely $\top = \bot = (I,I, u_\ox^l = u_\ox^r)$).  The tensor unit is always a conjugative object 
since $(\chi^{\!\!\!\circ})^{-1}: \overline{I} \to I$; therefore, this is immediately a conjugative symmetric 
$*$-autonomous category.  Composing the conjugation with the dualizing functor gives us a dagger.   
%%%%%%%%%%%%%%%%%%%%%%%%%%%%%%%%%%%%%%%%%%%%%%%%%%%%%%%%%%%%%%%%%%

\subsection{Category of Hopf modules in a $*$-autonomous category}

\label{Sec: HModx}
In this example\footnote{We thank J-S. P. Lemay for bringing our attention to this example.}, 
we start with any symmetric $*$-autonomous category, $\X$, and build the category of modules over a 
Hopf Algebra which is in turn a $\dagger$-$*$-autonomous category.  

First of all, it has been already proven in \cite{PaS09}, that the category of Hopf modules over a $\ox$-Hopf 
algebra in any symmetric $*$-autonomous category is also a $*$-autonomous category. Then we note that, whenever the Hopf Algebra is cocommutative, the resulting $*$-autonomous category has a conjugation functor. One can construct the dagger functor by composing the conjugation functor and dualizing functor as in Theorem \ref{Theorem: conjugation+dualizing}. We establish some basic definitions before describing the category of modules over a Hopf Algerba, ${\mbox{\bf H-Mod}}_\X$.

\begin{definition}
A {\bf bialgebra}  in a symmetric monoidal category is a 4-tuple 
$$(\nabla: B \ox B \to B, e:  I \to B, \Delta : B \to B \ox B, u: B \to I)$$
such that $(A, \nabla,e)$ is a monoid and $(A, \Delta, u)$  is a comonoid and $\nabla$ and $e$ are coalgebra 
homomorphisms with respect to the comultiplication and the counit.
\end{definition}

Note that instead of requiring that  $\nabla$ and $e$ are coalgebra homomorphisms, one could equivalently 
require $\Delta$ and $u$ are algebra homomorphims with respect to the multiplication and the unit.

 The components of a bialgebra are graphically depicted as follows:
 
 \[
 \begin{tikzpicture}
	\begin{pgfonlayer}{nodelayer}
		\node [style=none] (0) at (-3, 1) {};
		\node [style=none] (1) at (-2.5, 1) {};
		\node [style=none] (2) at (-2.75, 0.75) {};
		\node [style=none] (3) at (-3.25, 1.5) {};
		\node [style=none] (4) at (-2.25, 1.5) {};
		\node [style=none] (5) at (-2.75, 0.25) {};
	\end{pgfonlayer}
	\begin{pgfonlayer}{edgelayer}
		\draw (0.center) to (1.center);
		\draw (1.center) to (2.center);
		\draw (0.center) to (2.center);
		\draw [in=-90, out=30, looseness=1.00] (1.center) to (4.center);
		\draw [in=-90, out=150, looseness=1.00] (0.center) to (3.center);
		\draw (2.center) to (5.center);
	\end{pgfonlayer}
\end{tikzpicture} : A \ox A \to A ~~~~~~~ \begin{tikzpicture}
	\begin{pgfonlayer}{nodelayer}
		\node [style=none] (0) at (-3, 0.75) {};
		\node [style=none] (1) at (-2.5, 0.75) {};
		\node [style=none] (2) at (-2.75, 1) {};
		\node [style=none] (3) at (-3.25, 0.25) {};
		\node [style=none] (4) at (-2.25, 0.25) {};
		\node [style=none] (5) at (-2.75, 1.5) {};
	\end{pgfonlayer}
	\begin{pgfonlayer}{edgelayer}
		\draw (0.center) to (1.center);
		\draw (1.center) to (2.center);
		\draw (0.center) to (2.center);
		\draw [in=90, out=-30, looseness=1.00] (1.center) to (4.center);
		\draw [in=90, out=-150, looseness=1.00] (0.center) to (3.center);
		\draw (2.center) to (5.center);
	\end{pgfonlayer}
\end{tikzpicture} : A \to A \ox A ~~~~~~~~ \begin{tikzpicture}
	\begin{pgfonlayer}{nodelayer}
		\node [style=none] (0) at (-3, -0) {};
		\node [style=none] (1) at (-2.5, -0) {};
		\node [style=none] (2) at (-2.75, 0.25) {};
		\node [style=none] (3) at (-2.75, 1.5) {};
	\end{pgfonlayer}
	\begin{pgfonlayer}{edgelayer}
		\draw (0.center) to (1.center);
		\draw (1.center) to (2.center);
		\draw (0.center) to (2.center);
		\draw (2.center) to (3.center);
	\end{pgfonlayer}
\end{tikzpicture}: A \to I ~~~~~~~~~~ \begin{tikzpicture}
	\begin{pgfonlayer}{nodelayer}
		\node [style=none] (0) at (-3, 1.5) {};
		\node [style=none] (1) at (-2.5, 1.5) {};
		\node [style=none] (2) at (-2.75, 1.25) {};
		\node [style=none] (3) at (-2.75, 0) {};
	\end{pgfonlayer}
	\begin{pgfonlayer}{edgelayer}
		\draw (0.center) to (1.center);
		\draw (1.center) to (2.center);
		\draw (0.center) to (2.center);
		\draw (2.center) to (3.center);
	\end{pgfonlayer}
\end{tikzpicture} : I \to A
 \]
This gives a succinct graphical depiction of the coalgebra homomorphism laws; namely:
\[
\begin{tikzpicture}
	\begin{pgfonlayer}{nodelayer}
		\node [style=none] (0) at (-2, 2) {};
		\node [style=none] (1) at (-2.25, 1.75) {};
		\node [style=none] (2) at (-1.75, 1.75) {};
		\node [style=none] (3) at (-2, 2.75) {};
		\node [style=none] (4) at (-0.5, 2.75) {};
		\node [style=none] (5) at (-0.75, 1.75) {};
		\node [style=none] (6) at (-0.5, 2) {};
		\node [style=none] (7) at (-0.25, 1.75) {};
		\node [style=none] (8) at (-1.75, 0.25) {};
		\node [style=none] (9) at (-0.5, -0.75) {};
		\node [style=none] (10) at (-2.25, 0.25) {};
		\node [style=none] (11) at (-2, -0) {};
		\node [style=none] (12) at (-2, -0.75) {};
		\node [style=none] (13) at (-0.5, -0) {};
		\node [style=none] (14) at (-0.25, 0.25) {};
		\node [style=none] (15) at (-0.75, 0.25) {};
	\end{pgfonlayer}
	\begin{pgfonlayer}{edgelayer}
		\draw (0.center) to (1.center);
		\draw (1.center) to (2.center);
		\draw (2.center) to (0.center);
		\draw (3.center) to (0.center);
		\draw (6.center) to (5.center);
		\draw (5.center) to (7.center);
		\draw (7.center) to (6.center);
		\draw (4.center) to (6.center);
		\draw (11.center) to (10.center);
		\draw (10.center) to (8.center);
		\draw (8.center) to (11.center);
		\draw (12.center) to (11.center);
		\draw (13.center) to (15.center);
		\draw (15.center) to (14.center);
		\draw (14.center) to (13.center);
		\draw (9.center) to (13.center);
		\draw [in=15, out=-165, looseness=1.00] (5.center) to (8.center);
		\draw [bend right, looseness=1.25] (1.center) to (10.center);
		\draw [in=150, out=-30, looseness=1.00] (2.center) to (15.center);
		\draw [bend left, looseness=1.00] (7.center) to (14.center);
	\end{pgfonlayer}
\end{tikzpicture} = \begin{tikzpicture}
	\begin{pgfonlayer}{nodelayer}
		\node [style=none] (0) at (-2, -1.5) {};
		\node [style=none] (1) at (-2.25, -1.75) {};
		\node [style=none] (2) at (-1.75, -1.75) {};
		\node [style=none] (3) at (-2, -0.75) {};
		\node [style=none] (4) at (-1.75, 0.25) {};
		\node [style=none] (5) at (-2.25, 0.25) {};
		\node [style=none] (6) at (-2, -0) {};
		\node [style=none] (7) at (-2, -0.75) {};
		\node [style=none] (8) at (-2.75, 1) {};
		\node [style=none] (9) at (-1.25, 1) {};
		\node [style=none] (10) at (-2.75, -2.5) {};
		\node [style=none] (11) at (-1.25, -2.5) {};
	\end{pgfonlayer}
	\begin{pgfonlayer}{edgelayer}
		\draw (0.center) to (1.center);
		\draw (1.center) to (2.center);
		\draw (2.center) to (0.center);
		\draw (3.center) to (0.center);
		\draw (6.center) to (5.center);
		\draw (5.center) to (4.center);
		\draw (4.center) to (6.center);
		\draw (7.center) to (6.center);
		\draw [bend right=45, looseness=0.75] (4.center) to (9.center);
		\draw [bend right, looseness=1.25] (8.center) to (5.center);
		\draw [bend right, looseness=0.75] (1.center) to (10.center);
		\draw [bend left, looseness=1.00] (2.center) to (11.center);
	\end{pgfonlayer}
\end{tikzpicture} ~~~~~~~ \begin{tikzpicture} %bialg-2a
	\begin{pgfonlayer}{nodelayer}
		\node [style=none] (0) at (-1, 0.75) {};
		\node [style=none] (1) at (-1.25, 1) {};
		\node [style=none] (2) at (-0.75, 1) {};
		\node [style=none] (3) at (-1, -0) {};
		\node [style=none] (4) at (-1.25, -0.25) {};
		\node [style=none] (5) at (-0.75, -0.25) {};
		\node [style=none] (6) at (-1.75, -1) {};
		\node [style=none] (7) at (-0.25, -1) {};
	\end{pgfonlayer}
	\begin{pgfonlayer}{edgelayer}
		\draw (1.center) to (2.center);
		\draw (2.center) to (0.center);
		\draw (0.center) to (1.center);
		\draw (4.center) to (5.center);
		\draw (5.center) to (3.center);
		\draw (3.center) to (4.center);
		\draw (0.center) to (3.center);
		\draw [in=71, out=-135, looseness=1.00] (4.center) to (6.center);
		\draw [bend left, looseness=0.75] (5.center) to (7.center);
	\end{pgfonlayer}
\end{tikzpicture} =   \begin{tikzpicture}
	\begin{pgfonlayer}{nodelayer}
		\node [style=none] (0) at (-0.75, 0.75) {};
		\node [style=none] (1) at (-1, 1) {};
		\node [style=none] (2) at (-0.5, 1) {};
		\node [style=none] (3) at (-0.75, -1) {};
		\node [style=none] (4) at (0, 0.75) {};
		\node [style=none] (5) at (0.25, 1) {};
		\node [style=none] (6) at (0, -1) {};
		\node [style=none] (7) at (-0.25, 1) {};
	\end{pgfonlayer}
	\begin{pgfonlayer}{edgelayer}
		\draw (1.center) to (2.center);
		\draw (2.center) to (0.center);
		\draw (0.center) to (1.center);
		\draw (0.center) to (3.center);
		\draw (7.center) to (5.center);
		\draw (5.center) to (4.center);
		\draw (4.center) to (7.center);
		\draw (4.center) to (6.center);
	\end{pgfonlayer}
\end{tikzpicture} ~~~~~~~~ \begin{tikzpicture} %bialg2a
	\begin{pgfonlayer}{nodelayer}
		\node [style=none] (0) at (-1, -0.75) {};
		\node [style=none] (1) at (-1.25, -1) {};
		\node [style=none] (2) at (-0.75, -1) {};
		\node [style=none] (3) at (-1, 0) {};
		\node [style=none] (4) at (-1.25, 0.25) {};
		\node [style=none] (5) at (-0.75, 0.25) {};
		\node [style=none] (6) at (-1.75, 1) {};
		\node [style=none] (7) at (-0.25, 1) {};
	\end{pgfonlayer}
	\begin{pgfonlayer}{edgelayer}
		\draw (1.center) to (2.center);
		\draw (2.center) to (0.center);
		\draw (0.center) to (1.center);
		\draw (4.center) to (5.center);
		\draw (5.center) to (3.center);
		\draw (3.center) to (4.center);
		\draw (0.center) to (3.center);
		\draw [in=-71, out=135, looseness=1.00] (4.center) to (6.center);
		\draw [bend right, looseness=0.75] (5.center) to (7.center);
	\end{pgfonlayer}
\end{tikzpicture} = \begin{tikzpicture}
	\begin{pgfonlayer}{nodelayer}
		\node [style=none] (0) at (-1, -0.75) {};
		\node [style=none] (1) at (-1.25, -1) {};
		\node [style=none] (2) at (-0.75, -1) {};
		\node [style=none] (3) at (-1, 1) {};
		\node [style=none] (4) at (0, -0.75) {};
		\node [style=none] (5) at (0.25, -1) {};
		\node [style=none] (6) at (0, 1) {};
		\node [style=none] (7) at (-0.25, -1) {};
	\end{pgfonlayer}
	\begin{pgfonlayer}{edgelayer}
		\draw (1.center) to (2.center);
		\draw (2.center) to (0.center);
		\draw (0.center) to (1.center);
		\draw (0.center) to (3.center);
		\draw (7.center) to (5.center);
		\draw (5.center) to (4.center);
		\draw (4.center) to (7.center);
		\draw (4.center) to (6.center);
	\end{pgfonlayer}
\end{tikzpicture} ~~~~~~~~ \begin{tikzpicture}
	\begin{pgfonlayer}{nodelayer}
		\node [style=none] (0) at (-1, -0.75) {};
		\node [style=none] (1) at (-1.25, -1) {};
		\node [style=none] (2) at (-0.75, -1) {};
		\node [style=none] (3) at (-1, 0.75) {};
		\node [style=none] (4) at (-1.25, 1) {};
		\node [style=none] (5) at (-0.75, 1) {};
	\end{pgfonlayer}
	\begin{pgfonlayer}{edgelayer}
		\draw (1.center) to (2.center);
		\draw (2.center) to (0.center);
		\draw (0.center) to (1.center);
		\draw (0.center) to (3.center);
		\draw (4.center) to (3.center);
		\draw (3.center) to (5.center);
		\draw (5.center) to (4.center);
	\end{pgfonlayer}
\end{tikzpicture} = I
\]

\begin{definition} An {\bf antipode} for a bialgebra $(B, \nabla, \bialgunitmap{0.8}, \Delta, 
    \bialgcounitmap{0.8})$ is an endomorphism $s: B \to B$ such that 
\[
\begin{tikzpicture} %hopf-1
	\begin{pgfonlayer}{nodelayer}
		\node [style=none] (0) at (-2, 2.25) {};
		\node [style=none] (1) at (-2.25, 2) {};
		\node [style=none] (2) at (-1.75, 2) {};
		\node [style=none] (3) at (-2, 2.75) {};
		\node [style=none] (4) at (-1.75, 0.5) {};
		\node [style=none] (5) at (-2.25, 0.5) {};
		\node [style=none] (6) at (-2, 0.25) {};
		\node [style=none] (7) at (-2, -0.5) {};
		\node [style=circle, scale=1.5] (8) at (-1.5, 1.25) {};
		\node [style=none] (9) at (-1.5, 1.25) {$s$};
	\end{pgfonlayer}
	\begin{pgfonlayer}{edgelayer}
		\draw (0.center) to (1.center);
		\draw (1.center) to (2.center);
		\draw (2.center) to (0.center);
		\draw (3.center) to (0.center);
		\draw (6.center) to (5.center);
		\draw (5.center) to (4.center);
		\draw (4.center) to (6.center);
		\draw (7.center) to (6.center);
		\draw [bend left, looseness=1.00] (2.center) to (8);
		\draw [bend left, looseness=0.75] (8) to (4.center);
		\draw [bend right=45, looseness=1.00] (1.center) to (5.center);
	\end{pgfonlayer}
\end{tikzpicture} = \begin{tikzpicture}
	\begin{pgfonlayer}{nodelayer}
		\node [style=none] (0) at (-1.75, 2.25) {};
		\node [style=none] (1) at (-1.5, 2) {};
		\node [style=none] (2) at (-2, 2) {};
		\node [style=none] (3) at (-1.75, 2.75) {};
		\node [style=none] (4) at (-2, 0.5) {};
		\node [style=none] (5) at (-1.5, 0.5) {};
		\node [style=none] (6) at (-1.75, 0.25) {};
		\node [style=none] (7) at (-1.75, -0.5) {};
		\node [style=circle, scale=1.5] (8) at (-2.25, 1.25) {};
		\node [style=none] (9) at (-2.25, 1.25) {$s$};
	\end{pgfonlayer}
	\begin{pgfonlayer}{edgelayer}
		\draw (0.center) to (1.center);
		\draw (1.center) to (2.center);
		\draw (2.center) to (0.center);
		\draw (3.center) to (0.center);
		\draw (6.center) to (5.center);
		\draw (5.center) to (4.center);
		\draw (4.center) to (6.center);
		\draw (7.center) to (6.center);
		\draw [bend right, looseness=1.00] (2.center) to (8);
		\draw [bend right, looseness=0.75] (8) to (4.center);
		\draw [bend left=45, looseness=1.00] (1.center) to (5.center);
	\end{pgfonlayer}
\end{tikzpicture} = \begin{tikzpicture}
	\begin{pgfonlayer}{nodelayer}
		\node [style=none] (0) at (1, 1.5) {};
		\node [style=none] (1) at (1, 0.5) {};
		\node [style=none] (2) at (1, -0.5) {};
		\node [style=none] (3) at (1, 2.75) {};
		\node [style=none] (4) at (0.75, 0.75) {};
		\node [style=none] (5) at (1.25, 0.75) {};
		\node [style=none] (6) at (1, 0.5) {};
		\node [style=none] (7) at (1.25, 1.25) {};
		\node [style=none] (8) at (0.75, 1.25) {};
		\node [style=none] (9) at (1, 1.5) {};
	\end{pgfonlayer}
	\begin{pgfonlayer}{edgelayer}
		\draw (3.center) to (0.center);
		\draw (1.center) to (2.center);
		\draw (4.center) to (5.center);
		\draw (5.center) to (6.center);
		\draw (6.center) to (4.center);
		\draw (8.center) to (7.center);
		\draw (7.center) to (9.center);
		\draw (9.center) to (8.center);
	\end{pgfonlayer}
\end{tikzpicture}
\]

A {\bf Hopf algebra} is a bialgebra with an antipode. An {\bf involutive Hopf algebra} is a hopf algebra where 
the antipode is self-inverse.
\end{definition}

A standard example of a Hopf algebra is a group algebra over a field:  for all group elements $g$, $\nabla : 
g \mapsto g \ox g$, $\bialgcounitmap{0.8}: g \mapsto 1$, $\Delta: g \ox h \mapsto gh$ and $s: g \mapsto g^{-1}$.

\begin{lemma}
\label{Lemma: involutive s}
Suppose $\X$ is a symmetric monoidal category, then:

\begin{enumerate}[(i)]

\item \cite[Theroem 3.5]{Blu96} If $H$ is a commutative or a cocommutative Hopf Algebra in $\X$, then $s^2 = 1$ where $s$ is the antipode: 
so it is an involutive Hopf algebra.

\item \cite[Lemma 2.11]{Lem19} If $H$ is a commutative Hopf Algebra, then $s$ is a monoid homomorphism. If $H$ is a cocommutative 
Hopf Algebra, then $s$ is  a comonoid homomorphism.

\end{enumerate}

\end{lemma}


\begin{definition}
A {\bf left module} for a bialgebra $(B, \nabla, u, \Delta, e)$ is a tuple $(M, a_M^l:B \ox M \to M)$ 
such that $a_M^l$ is a $B$-action i.e., the following diagram commutes:

\[
\xymatrix{
M \ar[r]^{u_\ox^l} \ar@{=}[dr] & \top \ox M \ar[d]^{\bialgunitmap{0.8}} \\ 
& M
}
\]
\end{definition}

We graphically depict $a_m^l$ as follows:
\[
\begin{tikzpicture}[xscale=-1]
	\begin{pgfonlayer}{nodelayer}
		\node [style=none] (0) at (-2.75, 1.25) {};
		\node [style=none] (1) at (-2.5, 1.25) {};
		\node [style=none] (2) at (-2.75, 1) {};
		\node [style=none] (3) at (-2.75, 0.25) {};
		\node [style=none] (4) at (-2, 2) {};
		\node [style=none] (5) at (-2.75, 2) {};
	\end{pgfonlayer}
	\begin{pgfonlayer}{edgelayer}
		\draw (0.center) to (1.center);
		\draw (1.center) to (2.center);
		\draw (0.center) to (2.center);
		\draw (2.center) to (3.center);
		\draw (5.center) to (0.center);
		\draw [in=-105, out=30, looseness=1.25] (1.center) to (4.center);
	\end{pgfonlayer}
\end{tikzpicture}: B \ox M \to M
\]

giving the graphical presentation of the module laws:

\[
\begin{tikzpicture}%leftaction-rule1
	\begin{pgfonlayer}{nodelayer}
		\node [style=none] (0) at (-1.5, 2.25) {};
		\node [style=none] (1) at (-0.5, 2.25) {};
		\node [style=none] (2) at (0.25, 2.25) {};
		\node [style=none] (3) at (-1, 0.5) {};
		\node [style=none] (4) at (0, -0.25) {};
		\node [style=none] (5) at (0.25, -0.25) {};
		\node [style=none] (6) at (0.25, -0.5) {};
		\node [style=none] (7) at (0.25, -1) {};
		\node [style=none] (8) at (-1.25, 0.75) {};
		\node [style=none] (9) at (-0.75, 0.75) {};
		\node [style=none] (10) at (-1, 0.5) {};
	\end{pgfonlayer}
	\begin{pgfonlayer}{edgelayer}
		\draw (2.center) to (5.center);
		\draw (6.center) to (5.center);
		\draw (5.center) to (4.center);
		\draw (4.center) to (6.center);
		\draw [in=-90, out=150, looseness=1.25] (4.center) to (3.center);
		\draw (6.center) to (7.center);
		\draw (8.center) to (9.center);
		\draw (9.center) to (10.center);
		\draw (10.center) to (8.center);
		\draw [bend right=15, looseness=1.00] (0.center) to (8.center);
		\draw [bend right=15, looseness=1.00] (9.center) to (1.center);
	\end{pgfonlayer}
\end{tikzpicture} = \begin{tikzpicture}
	\begin{pgfonlayer}{nodelayer}
		\node [style=none] (0) at (-1.75, 2) {};
		\node [style=none] (1) at (-1.5, 2) {};
		\node [style=none] (2) at (-1.5, 1.75) {};
		\node [style=none] (3) at (-1.5, 3) {};
		\node [style=none] (4) at (-2.25, 3) {};
		\node [style=none] (5) at (-0.75, -0) {};
		\node [style=none] (6) at (-0.75, 3) {};
		\node [style=none] (7) at (-0.75, 0.25) {};
		\node [style=none] (8) at (-1.5, 1.25) {};
		\node [style=none] (9) at (-1, 0.25) {};
		\node [style=none] (10) at (-0.75, -0.5) {};
	\end{pgfonlayer}
	\begin{pgfonlayer}{edgelayer}
		\draw (3.center) to (1.center);
		\draw (2.center) to (1.center);
		\draw (1.center) to (0.center);
		\draw (0.center) to (2.center);
		\draw [in=-90, out=150, looseness=1.25] (0.center) to (4.center);
		\draw (6.center) to (7.center);
		\draw (5.center) to (7.center);
		\draw (7.center) to (9.center);
		\draw (9.center) to (5.center);
		\draw [in=-90, out=150, looseness=1.25] (9.center) to (8.center);
		\draw (2.center) to (8.center);
		\draw (5.center) to (10.center);
	\end{pgfonlayer}
\end{tikzpicture} ~~~~\text{ and }~~ \begin{tikzpicture} %leftactionrule-1b
	\begin{pgfonlayer}{nodelayer}
		\node [style=none] (0) at (-0.75, 1) {};
		\node [style=none] (1) at (-0.75, 3) {};
		\node [style=none] (2) at (-0.75, 1.25) {};
		\node [style=none] (3) at (-1, 1.25) {};
		\node [style=none] (4) at (-0.75, -0) {};
		\node [style=none] (5) at (-1.75, 3) {};
		\node [style=none] (6) at (-1.25, 3) {};
		\node [style=none] (7) at (-1.5, 2.75) {};
	\end{pgfonlayer}
	\begin{pgfonlayer}{edgelayer}
		\draw (1.center) to (2.center);
		\draw (0.center) to (2.center);
		\draw (2.center) to (3.center);
		\draw (3.center) to (0.center);
		\draw (0.center) to (4.center);
		\draw (5.center) to (6.center);
		\draw (6.center) to (7.center);
		\draw (7.center) to (5.center);
		\draw [bend right, looseness=1.00] (7.center) to (3.center);
	\end{pgfonlayer}
\end{tikzpicture} = \begin{tikzpicture}
	\begin{pgfonlayer}{nodelayer}
		\node [style=none] (0) at (-0.75, 1.75) {};
		\node [style=none] (1) at (-0.75, 3.25) {};
		\node [style=none] (2) at (-0.75, 1.75) {};
		\node [style=none] (3) at (-0.75, 0.25) {};
	\end{pgfonlayer}
	\begin{pgfonlayer}{edgelayer}
		\draw (1.center) to (2.center);
		\draw (0.center) to (2.center);
		\draw (0.center) to (3.center);
	\end{pgfonlayer}
\end{tikzpicture}
\] 

\begin{definition}
Let $\X$ be a $*$-autonomous category and $H$ be a Hopf $\ox$-algebra in $\X$. The category of left H-modules in $\X$, ${\mbox{\bf H-Mod}}_\X$ has:
\begin{description}
\item[Objects:]  Left $H$-modules $(A, a_A^l:H \ox A\to A)$:
\item[Arrows:] A module homomorphism $(A, a_A^L:H \ox A\to A) \to^{f}(B, a_B^L:H \ox B\to B)$ is a map $A \to^{f} B$ such that the following diagram commutes:
\[
\xymatrix{
H \ox A \ar[r]^{ a_A^L} \ar[d]_{1 \ox f} & A \ar[d]^{f} \\
H \ox B \ar[r]_{ a_B^L} & B
}
\]

This is graphically depicted as follows:
\[
 \begin{tikzpicture} %act2
	\begin{pgfonlayer}{nodelayer}
		\node [style=none] (0) at (0, 1) {};
		\node [style=none] (1) at (0, 0.75) {};
		\node [style=none] (2) at (-0.25, 1) {};
		\node [style=none] (3) at (0, 2) {};
		\node [style=none] (4) at (0, -0.5) {};
		\node [style=none] (5) at (-0.75, 2) {};
		\node [style=none] (6) at (-0.75, 1.5) {};
		\node [style=circle, scale=1.5] (7) at (0, 0.25) {};
		\node [style=none] (8) at (0, 0.25) {$f$};
	\end{pgfonlayer}
	\begin{pgfonlayer}{edgelayer}
		\draw (2.center) to (0.center);
		\draw (0.center) to (1.center);
		\draw (1.center) to (2.center);
		\draw [bend left, looseness=1.00] (2.center) to (6.center);
		\draw (6.center) to (5.center);
		\draw (3.center) to (0.center);
		\draw (1.center) to (7);
		\draw (7) to (4.center);
	\end{pgfonlayer}
\end{tikzpicture} =
\begin{tikzpicture} %act1
	\begin{pgfonlayer}{nodelayer}
		\node [style=none] (0) at (0, 0.5) {};
		\node [style=none] (1) at (0, 0.25) {};
		\node [style=none] (2) at (-0.25, 0.5) {};
		\node [style=none] (3) at (0, 2) {};
		\node [style=none] (4) at (0, -0.25) {};
		\node [style=none] (5) at (-0.75, 2) {};
		\node [style=none] (6) at (-0.75, 1.5) {};
		\node [style=circle, scale=1.5] (7) at (0, 1.25) {};
		\node [style=none] (8) at (0, 1.25) {$f$};
	\end{pgfonlayer}
	\begin{pgfonlayer}{edgelayer}
		\draw (2.center) to (0.center);
		\draw (0.center) to (1.center);
		\draw (1.center) to (2.center);
		\draw (1.center) to (4.center);
		\draw (3.center) to (7);
		\draw (7) to (0.center);
		\draw [bend left, looseness=1.00] (2.center) to (6.center);
		\draw (6.center) to (5.center);
		\draw[fill=black] (0.center) -- (1.center) -- (2.center) -- (0.center);
	\end{pgfonlayer}
\end{tikzpicture} \]

\end{description}
\end{definition}

Observe that any left action is indeed a module homomorphism.

\begin{theorem} \cite{PaS09}
Let $\X$ be symmetric $*$-autonomous category and $H$ be a $\ox$-Hopf Algebra in $\X$ with bijective antipode ($s^2 = 1$). Then, ${\mbox{\bf H-Mod}}_\X$ is a $*$-autonomous category. If the Hopf Algebra, $H$, is cocommutative, then ${\mbox{\bf H-Mod}}_\X$ is a symmetric $*$-autonomous category.
\end{theorem}
\begin{proof} (Sketch)
The monoidal product $\ox$ for ${\mbox{\bf H-Mod}}_\X$ is defined as follows:
\[ (A, \leftaction{0.4}{white}) \ox (B, \leftaction{0.4}{black}) := (A \ox B, \leftaction{0.4}{gray} ) 
\text{ where, } \leftaction{0.8}{gray} := \begin{tikzpicture}
	\begin{pgfonlayer}{nodelayer}
		\node [style=none] (0) at (-0.25, -0.5) {};
		\node [style=none] (1) at (-0.25, -0.75) {};
		\node [style=none] (2) at (-0.5, -0.5) {};
		\node [style=none] (3) at (-0.25, 0.25) {};
		\node [style=none] (4) at (-1, 0.75) {};
		\node [style=none] (5) at (-1, -0) {};
		\node [style=none] (6) at (-0.25, -1.25) {};
		\node [style=none] (7) at (1, -0.75) {};
		\node [style=none] (8) at (1, 0.25) {};
		\node [style=none] (9) at (1, -1.25) {};
		\node [style=none] (10) at (0.25, -0) {};
		\node [style=none] (11) at (1, -0.5) {};
		\node [style=none] (12) at (-0.5, 0.75) {};
		\node [style=none] (13) at (0.75, -0.5) {};
		\node [style=none] (14) at (0.25, 1.75) {};
		\node [style=none] (15) at (1, 1.75) {};
		\node [style=none] (16) at (-0.75, 1.75) {};
		\node [style=none] (17) at (-0.75, 1) {};
		\node [style=none] (18) at (-1, 0.75) {};
		\node [style=none] (19) at (-0.5, 0.75) {};
	\end{pgfonlayer}
	\begin{pgfonlayer}{edgelayer}
		\draw (2.center) to (0.center);
		\draw (0.center) to (1.center);
		\draw (1.center) to (2.center);
		\draw [bend left, looseness=1.00] (2.center) to (5.center);
		\draw (5.center) to (4.center);
		\draw (3.center) to (0.center);
		\draw (1.center) to (6.center);
		\draw (13.center) to (11.center);
		\draw (11.center) to (7.center);
		\draw (7.center) to (13.center);
		\draw [bend left, looseness=1.00] (13.center) to (10.center);
		\draw [in=-45, out=90, looseness=1.00] (10.center) to (12.center);
		\draw (8.center) to (11.center);
		\draw (7.center) to (9.center);
		\draw [in=-90, out=90, looseness=1.00] (3.center) to (14.center);
		\draw (15.center) to (8.center);
		\draw (18.center) to (19.center);
		\draw (19.center) to (17.center);
		\draw (17.center) to (18.center);
		\draw (16.center) to (17.center);
		\draw[fill=black] (0.center) --  (1.center) --  (2.center) --  (0.center);
	\end{pgfonlayer}
\end{tikzpicture} \]

\iffalse
$\leftaction{0.6}{blue}$ is a module morphism:
\[
\begin{tikzpicture} %act-tensor2
	\begin{pgfonlayer}{nodelayer}
		\node [style=none] (0) at (-0.25, -1.25) {};
		\node [style=none] (1) at (-0.25, -1.5) {};
		\node [style=none] (2) at (-0.5, -1.25) {};
		\node [style=none] (3) at (-0.25, -0.5) {};
		\node [style=none] (4) at (-1, -0.5) {};
		\node [style=none] (5) at (-1, -0.75) {};
		\node [style=none] (6) at (-0.25, -2) {};
		\node [style=none] (7) at (1.5, -1.5) {};
		\node [style=none] (8) at (1.5, -0.5) {};
		\node [style=none] (9) at (1.5, -2) {};
		\node [style=none] (10) at (0.25, -0.75) {};
		\node [style=none] (11) at (1.5, -1.25) {};
		\node [style=none] (12) at (0.25, -0.5) {};
		\node [style=none] (13) at (1.25, -1.25) {};
		\node [style=circle, scale=0.5] (14) at (-0.75, 0.25) {};
		\node [style=none] (15) at (0.25, -0) {};
		\node [style=none] (16) at (1.5, -0) {};
		\node [style=none] (17) at (-0.75, 3) {};
		\node [style=none] (18) at (1.25, 0.75) {};
		\node [style=none] (19) at (1.5, 0.75) {};
		\node [style=none] (20) at (1.5, 0.5) {};
		\node [style=none] (21) at (0.25, 0.5) {};
		\node [style=none] (22) at (-0.25, 3) {};
		\node [style=none] (23) at (0.25, 1.5) {};
		\node [style=none] (24) at (1.5, 3) {};
		\node [style=none] (25) at (0.75, 1.5) {};
		\node [style=none] (26) at (1.5, -0) {};
		\node [style=circle, scale=0.5] (27) at (-0.25, 2.25) {};
		\node [style=none] (28) at (-0.5, 1.5) {};
		\node [style=none] (29) at (0.75, 1.25) {};
		\node [style=none] (30) at (0.25, -0) {};
		\node [style=none] (31) at (-0.5, 1.25) {};
		\node [style=none] (32) at (0.75, 3) {};
		\node [style=none] (33) at (1.5, 1.5) {};
		\node [style=none] (34) at (0.25, 0.75) {};
		\node [style=none] (35) at (0, 0.75) {};
		\node [style=none] (36) at (-0.75, 3.25) {$H$};
		\node [style=none] (37) at (-0.25, 3.25) {$H$};
		\node [style=none] (38) at (0.75, 3.25) {$A$};
		\node [style=none] (39) at (1.5, 3.25) {$B$};
		\node [style=none] (40) at (-0.25, -2.25) {$A$};
		\node [style=none] (41) at (1.5, -2.25) {$B$};
	\end{pgfonlayer}
	\begin{pgfonlayer}{edgelayer}
		\draw (2.center) to (0.center);
		\draw (0.center) to (1.center);
		\draw (1.center) to (2.center);
		\draw [bend left, looseness=1.00] (2.center) to (5.center);
		\draw (5.center) to (4.center);
		\draw (3.center) to (0.center);
		\draw (1.center) to (6.center);
		\draw (13.center) to (11.center);
		\draw (11.center) to (7.center);
		\draw (7.center) to (13.center);
		\draw [bend left, looseness=1.00] (13.center) to (10.center);
		\draw (10.center) to (12.center);
		\draw (8.center) to (11.center);
		\draw (7.center) to (9.center);
		\draw [in=90, out=-30, looseness=1.00] (14) to (12.center);
		\draw [bend right=15, looseness=1.00] (14) to (4.center);
		\draw [in=-90, out=90, looseness=1.00] (3.center) to (15.center);
		\draw (16.center) to (8.center);
		\draw (17.center) to (14);
		\draw (35.center) to (34.center);
		\draw (34.center) to (21.center);
		\draw (21.center) to (35.center);
		\draw [bend left, looseness=1.00] (35.center) to (31.center);
		\draw (31.center) to (28.center);
		\draw (23.center) to (34.center);
		\draw (21.center) to (30.center);
		\draw (18.center) to (19.center);
		\draw (19.center) to (20.center);
		\draw (20.center) to (18.center);
		\draw [bend left, looseness=1.00] (18.center) to (29.center);
		\draw (29.center) to (25.center);
		\draw (33.center) to (19.center);
		\draw (20.center) to (26.center);
		\draw [in=90, out=-30, looseness=1.00] (27) to (25.center);
		\draw [bend right=15, looseness=1.00] (27) to (28.center);
		\draw [in=-90, out=90, looseness=1.00] (23.center) to (32.center);
		\draw (24.center) to (33.center);
		\draw (22.center) to (27);
		\draw[fill=black] (18.center) -- (19.center) -- (20.center) -- (18.center);
		\draw[fill=black] (13.center) -- (11.center) -- (7.center) -- (13.center);
	%	\draw[fill=black] (35.center) -- (34.center) -- (21.center) -- (35.center);
	\end{pgfonlayer}
\end{tikzpicture} = \begin{tikzpicture}
	\begin{pgfonlayer}{nodelayer}
		\node [style=none] (0) at (-1, 0.25) {};
		\node [style=none] (1) at (-0.25, -0.5) {};
		\node [style=none] (2) at (-0.25, 0.25) {};
		\node [style=none] (3) at (-1, 0.5) {};
		\node [style=none] (4) at (-0.25, -1.25) {};
		\node [style=none] (5) at (-0.25, -0.25) {};
		\node [style=none] (6) at (-0.25, -2) {};
		\node [style=none] (7) at (-0.5, -0.25) {};
		\node [style=none] (8) at (-1.5, 1.75) {};
		\node [style=none] (9) at (-1, 3.5) {};
		\node [style=circle, scale=0.5] (10) at (-1, 2.5) {};
		\node [style=none] (11) at (1.5, -0.25) {};
		\node [style=none] (12) at (1.5, -2) {};
		\node [style=none] (13) at (1.5, -0.5) {};
		\node [style=none] (14) at (1.5, -1.25) {};
		\node [style=none] (15) at (0.75, 0.5) {};
		\node [style=none] (16) at (0.75, 0.25) {};
		\node [style=none] (17) at (1.25, -0.25) {};
		\node [style=none] (18) at (1.5, 0.5) {};
		\node [style=circle, scale=0.5] (19) at (0.75, 2.5) {};
		\node [style=none] (20) at (1.25, 1.75) {};
		\node [style=none] (21) at (0.75, 3.5) {};
		\node [style=circle, scale=0.5] (22) at (0.75, 1) {};
		\node [style=none] (23) at (1.25, 1.75) {};
		\node [style=circle, scale=0.5] (24) at (-1, 1) {};
		\node [style=none] (25) at (-1, 0.5) {};
		\node [style=none] (26) at (-1.5, 1.75) {};
		\node [style=none] (27) at (0.75, 0.5) {};
		\node [style=none] (28) at (3.25, 3.5) {};
		\node [style=none] (29) at (2, 3.5) {};
	\end{pgfonlayer}
	\begin{pgfonlayer}{edgelayer}
		\draw (7.center) to (5.center);
		\draw (5.center) to (1.center);
		\draw (1.center) to (7.center);
		\draw [bend left, looseness=1.00] (7.center) to (0.center);
		\draw (0.center) to (3.center);
		\draw (2.center) to (5.center);
		\draw (1.center) to (6.center);
		\draw [bend right, looseness=1.00] (10) to (8.center);
		\draw (9.center) to (10);
		\draw (17.center) to (11.center);
		\draw (11.center) to (13.center);
		\draw (13.center) to (17.center);
		\draw [bend left, looseness=1.00] (17.center) to (16.center);
		\draw (16.center) to (15.center);
		\draw (18.center) to (11.center);
		\draw (13.center) to (12.center);
		\draw [bend left, looseness=1.00] (19) to (20.center);
		\draw (21.center) to (19);
		\draw [bend left, looseness=1.00] (24) to (26.center);
		\draw (25.center) to (24);
		\draw [bend right, looseness=1.00] (22) to (23.center);
		\draw (27.center) to (22);
		\draw [in=180, out=0, looseness=1.00] (10) to (22);
		\draw [in=-165, out=15, looseness=1.25] (24) to (19);
		\draw [in=-90, out=90, looseness=0.75] (2.center) to (29.center);
		\draw [in=-90, out=90, looseness=1.00] (18.center) to (28.center);
		\draw[fill=black] (11.center) -- (13.center) -- (17.center) -- (11.center);
	\end{pgfonlayer}
\end{tikzpicture} = \begin{tikzpicture}
	\begin{pgfonlayer}{nodelayer}
		\node [style=none] (0) at (-1, 0.25) {};
		\node [style=none] (1) at (-0.25, -0.5) {};
		\node [style=none] (2) at (-0.25, 0.25) {};
		\node [style=none] (3) at (-1, 0.5) {};
		\node [style=none] (4) at (-0.25, -1.25) {};
		\node [style=none] (5) at (-0.25, -0.25) {};
		\node [style=none] (6) at (-0.25, -2) {};
		\node [style=none] (7) at (-0.5, -0.25) {};
		\node [style=none] (8) at (1.5, -0.25) {};
		\node [style=none] (9) at (1.5, -2) {};
		\node [style=none] (10) at (1.5, -0.5) {};
		\node [style=none] (11) at (1.5, -1.25) {};
		\node [style=none] (12) at (0.75, 0.5) {};
		\node [style=none] (13) at (0.75, 0.25) {};
		\node [style=none] (14) at (1.25, -0.25) {};
		\node [style=none] (15) at (1.5, 0.5) {};
		\node [style=circle, scale=0.5] (16) at (0, 1.75) {};
		\node [style=none] (17) at (-1, 0.5) {};
		\node [style=none] (18) at (0.75, 0.5) {};
		\node [style=none] (19) at (3.25, 3.5) {};
		\node [style=none] (20) at (2, 3.5) {};
		\node [style=none] (21) at (0.75, 3.5) {};
		\node [style=none] (22) at (-0.75, 3.5) {};
		\node [style=circle, scale=0.5] (23) at (0, 2.5) {};
	\end{pgfonlayer}
	\begin{pgfonlayer}{edgelayer}
		\draw (7.center) to (5.center);
		\draw (5.center) to (1.center);
		\draw (1.center) to (7.center);
		\draw [bend left, looseness=1.00] (7.center) to (0.center);
		\draw (0.center) to (3.center);
		\draw (2.center) to (5.center);
		\draw (1.center) to (6.center);
		\draw (14.center) to (8.center);
		\draw (8.center) to (10.center);
		\draw (10.center) to (14.center);
		\draw [bend left, looseness=1.00] (14.center) to (13.center);
		\draw (13.center) to (12.center);
		\draw (15.center) to (8.center);
		\draw (10.center) to (9.center);
		\draw [in=-150, out=90, looseness=1.00] (17.center) to (16);
		\draw [in=-90, out=90, looseness=0.75] (2.center) to (20.center);
		\draw [in=-90, out=90, looseness=1.00] (15.center) to (19.center);
		\draw [in=90, out=-15, looseness=1.00] (16) to (12.center);
		\draw [bend right, looseness=1.25] (22.center) to (23);
		\draw [bend right, looseness=1.25] (23) to (21.center);
		\draw (23) to (16);
		\draw[fill=black] (8.center) -- (10.center) -- (14.center) -- (8.center);
	\end{pgfonlayer}
\end{tikzpicture}
\]
\fi
 The unit of $\ox$ is given by $(\top, H \ox \top \to^{u_\ox^R} H \to^{e} \top)$, the left action is drawn as 
 $\begin{tikzpicture} %act6
	\begin{pgfonlayer}{nodelayer}
		\node [style=circle, scale=0.5] (0) at (-1, 0.25) {};
		\node [style=circle, scale=1.5] (1) at (0.25, 1) {};
		\node [style=none] (2) at (0.25, 1) {$\top$};
		\node [style=none] (3) at (-1, 2) {};
		\node [style=none] (4) at (0.25, 2) {};
		\node [style=none] (5) at (-0.25, -0.75) {$\top$};
		\node [style=none] (6) at (-1.5, 1.75) {$H$};
		\node [style=none] (7) at (0.5, 1.75) {$\top$};
		\node [style=none] (8) at (-1.25, -1) {};
		\node [style=none] (9) at (-0.75, -1) {};
		\node [style=none] (10) at (-1, -0.75) {};
		\node [style=none] (11) at (-1, -1) {};
	\end{pgfonlayer}
	\begin{pgfonlayer}{edgelayer}
		\draw (3.center) to (0);
		\draw [bend left, looseness=1.25, dotted] (1) to (0);
		\draw (4.center) to (1);
		\draw (8.center) to (9.center);
		\draw (9.center) to (10.center);
		\draw (8.center) to (10.center);
		\draw (0) to (10.center);
	\end{pgfonlayer}
\end{tikzpicture}$


 The par product is defined as: $(A, \leftaction{0.4}{white}) \oa (B, \leftaction{0.4}{black}) := (A \oa B, \leftaction{0.4}{gray} ) $
where,
\begin{align*}
 \leftaction{0.8}{gray} &:= H \ox (A \oa B) \to^{\Delta \ox 1} (H \ox H) \ox (A \oa B) \to^{a_\ox} H \ox (H \ox (A \oa B)) \\
  &\to^{c_\ox} (H \ox (A \oa B)) \ox H \to^{\partial^L \ox 1} ((H \ox A) \oa B) \ox H \to^{\partial^R} (H \ox A) \oa (B \ox H) \\
  &\to^{1 \oa c_\ox} (H \ox A) \oa (H \ox B) \to^{\leftaction{0.4}{white} \oa \leftaction{0.4}{black}} A \oa B
\end{align*}

and  the unit of $\oa$ is
\[
\bot := ( \bot, H \ox \bot \to^{u \ox \bot} \top \ox \bot \to^{u_\ox} \bot)
\]

All the basic natural isomorphisms are inherited directly from $\X$, and they are module homomorphisms. Thus, {\bf HMod}$_\X$ is a LDC.

The dualizing functor $(\_)^*$ is given as follows:$(A, \leftaction{0.4}{white}: H \ox A \to A)^* := (A^*, \leftaction{0.4}{white}*: H \ox A^* \to A^*) \ \text{ where, }$ \[ \leftaction{0.4}{white}* := 
\begin{tikzpicture} %act-tensor5
	\begin{pgfonlayer}{nodelayer}
		\node [style=none] (0) at (1.5, -0.25) {};
		\node [style=none] (1) at (1.5, -1) {};
		\node [style=none] (2) at (1.5, -0.5) {};
		\node [style=none] (3) at (2.25, -1) {};
		\node [style=none] (4) at (0.75, 0.5) {};
		\node [style=none] (5) at (0.75, 0.25) {};
		\node [style=none] (6) at (1.25, -0.25) {};
		\node [style=none] (7) at (1.5, 0.25) {};
		\node [style=none] (8) at (-0.25, 0.25) {};
		\node [style=none] (9) at (2.25, 3) {};
		\node [style=none] (10) at (-0.25, -1.25) {};
		\node [style=none] (11) at (0.75, 3) {};
		\node [style=none] (12) at (1, 2.75) {$H$};
		\node [style=none] (13) at (2.5, 2.75) {$A^*$};
		\node [style=none] (14) at (-0.25, -3) {};
		\node [style=none] (15) at (0, -2.75) {$A^*$};
		\node [style=circle, scale=1.5] (16) at (0.75, 2) {};
		\node [style=none] (17) at (0.75, 2) {$s$};
	\end{pgfonlayer}
	\begin{pgfonlayer}{edgelayer}
		\draw (6.center) to (0.center);
		\draw (0.center) to (2.center);
		\draw (2.center) to (6.center);
		\draw [bend left, looseness=1.00] (6.center) to (5.center);
		\draw (5.center) to (4.center);
		\draw (2.center) to (1.center);
		\draw [bend left=90, looseness=2.00] (8.center) to (7.center);
		\draw [bend right=90, looseness=2.00] (1.center) to (3.center);
		\draw (9.center) to (3.center);
		\draw (8.center) to (10.center);
		\draw (11.center) to (16);
		\draw (16) to (4.center);
		\draw (14.center) to (10.center);
		\draw (7.center) to (0.center);
	\end{pgfonlayer}
\end{tikzpicture}\]
 Equationally, 

\begin{align*}
\leftaction{0.4}{white}* &:= H \ox A^* \to^{s \ox 1} H \ox A^* \to^{u_\ox^{-1} \ox 1} (H \ox \top) \ox A^* \to^{1 \ox \eta \ox 1} (H \ox (A^* \oa A)) \ox A^* \\
&\to^{c_\ox \ox 1}  ((A^* \oa A) \ox H) \ox A^* \to^{\partial \ox 1} (A^* \oa (A \ox H)) \ox A^* \to^{1 \ox c_\ox \ox 1} (A^* \oa (H \ox A)) \ox A^* \\
& \to^{(1 \oa \leftaction{0.3}{white}) \ox 1} (A^* \oa A) \ox A^* \to^{\partial} A^* \oa (A \ox A^*) \to^{1 \oa \epsilon} A \oa \bot \to^{u_\oa^R} A^*
\end{align*}

The cups and caps are inherited directly from $\X$, hence the snake diagrams hold. 
The antipode in the definition of $\leftaction{0.4}{white}*: H \ox A^* \to A^*$ makes the cup and cap module morphisms.

Suppose $(A, \leftaction{0.4}{white}) \to^{f} (B, \leftaction{0.4}{black})$ is as a module morphism, 
then $f^* := B^* \to^{f^*} A^* \in \X$ which is also a module morphism. Thus, {\bf H-Mod}$_\X$ is a 
monoidal category with a dualizing functor, hence a $*$-autonomous category.

If $H$ is cocommutative, then $(A, \leftaction{0.4}{white}) \ox (B, \leftaction{0.4}{black}) \to^{c_\otimes} 
(B, \leftaction{0.4}{black}) \ox (A, \leftaction{0.4}{white})$ is a module homomorphism.

\iffalse
\[
\begin{tikzpicture}%act3
	\begin{pgfonlayer}{nodelayer}
		\node [style=none] (0) at (-1, -1) {};
		\node [style=none] (1) at (-1.25, -1) {};
		\node [style=none] (2) at (-1, -1.25) {};
		\node [style=none] (3) at (0, -1) {};
		\node [style=none] (4) at (0.25, -1) {};
		\node [style=none] (5) at (0.25, -1.25) {};
		\node [style=circle, scale=1.5] (6) at (-1.75, 1.25) {};
		\node [style=none] (7) at (-1.75, 2) {};
		\node [style=none] (8) at (0, 2) {};
		\node [style=none] (9) at (1, 2) {};
		\node [style=none] (10) at (0.25, -2.25) {};
		\node [style=none] (11) at (-1, -2.25) {};
		\node [style=none] (12) at (-1.75, 1.25) {$s$};
		\node [style=none] (13) at (-1.75, 0.5) {};
		\node [style=none] (14) at (-2, 0.25) {};
		\node [style=none] (15) at (-1.5, 0.25) {};
	\end{pgfonlayer}
	\begin{pgfonlayer}{edgelayer}
		\draw (1.center) to (0.center);
		\draw (2.center) to (1.center);
		\draw (0.center) to (2.center);
		\draw (3.center) to (4.center);
		\draw (5.center) to (3.center);
		\draw (4.center) to (5.center);
		\draw [in=-90, out=90, looseness=1.25] (0.center) to (8.center);
		\draw [in=-90, out=90, looseness=1.00] (4.center) to (9.center);
		\draw [in=90, out=-90, looseness=1.00] (5.center) to (11.center);
		\draw [in=90, out=-90, looseness=1.00] (2.center) to (10.center);
		\draw (7.center) to (6);
		\draw (14.center) to (15.center);
		\draw (14.center) to (13.center);
		\draw (13.center) to (15.center);
		\draw (6) to (13.center);
		\draw [in=165, out=-90, looseness=1.00] (14.center) to (1.center);
		\draw [in=165, out=-15, looseness=1.00] (15.center) to (3.center);
		\draw[fill=black] (3.center) to (4.center) to  (5.center) to (3.center); 
	\end{pgfonlayer}
\end{tikzpicture} = 
\begin{tikzpicture} %act4
	\begin{pgfonlayer}{nodelayer}
		\node [style=none] (0) at (-0.5, -1) {};
		\node [style=none] (1) at (-0.75, -1) {};
		\node [style=none] (2) at (-0.5, -1.25) {};
		\node [style=none] (3) at (-2, -1) {};
		\node [style=none] (4) at (-1.75, -1) {};
		\node [style=none] (5) at (-1.75, -1.25) {};
		\node [style=circle, scale=1.5] (6) at (-1.75, 1.25) {};
		\node [style=none] (7) at (-1.75, 2) {};
		\node [style=none] (8) at (-0.5, 2) {};
		\node [style=none] (9) at (0.25, 2) {};
		\node [style=none] (10) at (-0.5, -2) {};
		\node [style=none] (11) at (-1.75, -2) {};
		\node [style=none] (12) at (-1.75, 1.25) {$s$};
		\node [style=none] (13) at (-1.5, 0.5) {};
		\node [style=none] (14) at (-2, 0.5) {};
		\node [style=none] (15) at (-1.75, 0.75) {};
	\end{pgfonlayer}
	\begin{pgfonlayer}{edgelayer}
		\draw (1.center) to (0.center);
		\draw (2.center) to (1.center);
		\draw (0.center) to (2.center);
		\draw (3.center) to (4.center);
		\draw (5.center) to (3.center);
		\draw (4.center) to (5.center);
		\draw [in=-90, out=90, looseness=1.25] (0.center) to (8.center);
		\draw [in=-105, out=90, looseness=1.00] (4.center) to (9.center);
		\draw [in=90, out=-90, looseness=1.00] (5.center) to (11.center);
		\draw [in=90, out=-90, looseness=1.00] (2.center) to (10.center);
		\draw (7.center) to (6);
		\draw (14.center) to (13.center);
		\draw (14.center) to (15.center);
		\draw (15.center) to (13.center);
		\draw [in=-90, out=105, looseness=1.00] (3.center) to (13.center);
		\draw [in=165, out=-105, looseness=1.00] (14.center) to (1.center);
		\draw (6) to (15.center);
		\draw[fill=black] (3.center) to (4.center) to  (5.center) to (3.center); 
	\end{pgfonlayer}
\end{tikzpicture} = 
\begin{tikzpicture} %act5
	\begin{pgfonlayer}{nodelayer}
		\node [style=none] (0) at (-0.5, -1) {};
		\node [style=none] (1) at (-0.75, -1) {};
		\node [style=none] (2) at (-0.5, -1.25) {};
		\node [style=none] (3) at (-2, -1) {};
		\node [style=none] (4) at (-1.75, -1) {};
		\node [style=none] (5) at (-1.75, -1.25) {};
		\node [style=none] (6) at (-2, 2) {};
		\node [style=none] (7) at (-0.5, 2) {};
		\node [style=none] (8) at (0.25, 2) {};
		\node [style=none] (9) at (-0.5, -2) {};
		\node [style=none] (10) at (-1.75, -2) {};
		\node [style=circle, scale=1.5] (11) at (-2.5, 0.5) {};
		\node [style=circle, scale=1.5] (12) at (-1.5, 0.5) {};
		\node [style=none] (13) at (-2.5, 0.5) {$s$};
		\node [style=none] (14) at (-1.5, 0.5) {$s$};
		\node [style=none] (15) at (-2, 1.5) {};
		\node [style=none] (16) at (-2.25, 1.25) {};
		\node [style=none] (17) at (-1.75, 1.25) {};
	\end{pgfonlayer}
	\begin{pgfonlayer}{edgelayer}
		\draw (1.center) to (0.center);
		\draw (2.center) to (1.center);
		\draw (0.center) to (2.center);
		\draw (3.center) to (4.center);
		\draw (5.center) to (3.center);
		\draw (4.center) to (5.center);
		\draw [in=-90, out=90, looseness=1.25] (0.center) to (7.center);
		\draw [in=-105, out=90, looseness=0.75] (4.center) to (8.center);
		\draw [in=90, out=-90, looseness=1.00] (5.center) to (10.center);
		\draw [in=90, out=-90, looseness=1.00] (2.center) to (9.center);
		\draw [in=150, out=-63, looseness=1.25] (12) to (1.center);
		\draw [in=150, out=-72, looseness=1.00] (11) to (3.center);
		\draw (16.center) to (17.center);
		\draw (17.center) to (15.center);
		\draw (15.center) to (16.center);
		\draw (6.center) to (15.center);
		\draw [bend right=15, looseness=1.00] (16.center) to (11);
		\draw [bend left, looseness=1.00] (17.center) to (12);
		\draw[fill=black] (3.center) to (4.center) to  (5.center) to (3.center); 
	\end{pgfonlayer}
\end{tikzpicture}
\]
\fi

In that case, {\bf H-Mod}$_\X$ is a symmetric $*$-autonomous category.
\end{proof}

Futhermore, we can show that the category of Hopf modules is conjugative.

\begin{lemma}
Let $\X$ be a symmetric $*$-autonomous category. ${\mbox{\bf H-Mod}}_\X$, the category of modules over a 
cocommutative Hopf Algebra H is a conjugative symmetric $*$-autonomous category .
\end{lemma}
\begin{proof}
We already know that ${\mbox{\bf H-Mod}}_\X$ is a symmetric $*$-autonomous category. 
We define the conjugation functor $\bar{(\_)}: {\mbox{\bf H-Mod}}_\X \to {\mbox{\bf H-Mod}}_\X$ as follows:

\begin{itemize}
\item 
$\overline{(A, \leftaction{0.4}{white})} := (A, \overline{\leftaction{0.4}{white}})$ where,
$ \overline{\leftaction{0.5}{white}} := 
\begin{tikzpicture}[scale=1]
	\begin{pgfonlayer}{nodelayer}
		\node [style=none] (0) at (0, 1) {};
		\node [style=none] (1) at (0, 0.75) {};
		\node [style=none] (2) at (-0.25, 1) {};
		\node [style=none] (3) at (0, 2) {};
		\node [style=none] (4) at (0, 0.5) {};
		\node [style=circle] (5) at (-0.75, 1.5) {};
		\node [style=none] (6) at (-0.75, 2) {};
		\node [style=none] (7) at (-0.75, 1.5) {$s$};
	\end{pgfonlayer}
	\begin{pgfonlayer}{edgelayer}
		\draw (2.center) to (0.center);
		\draw (0.center) to (1.center);
		\draw (1.center) to (2.center);
		\draw (1.center) to (4.center);
		\draw (3.center) to (0.center);
		\draw [bend right, looseness=1.00] (5) to (2.center);
		\draw (5) to (6.center);
	\end{pgfonlayer}
\end{tikzpicture}
$
\item Suppose $f: (A, \leftaction{0.4}{white}) \to (B, \leftaction{0.4}{black})$, then $\overline{f} := f$
\end{itemize}

The  basic natural isomorphisms are given by: \[ \overline{(B, \leftaction{0.4}{black})} \ox 
\overline{(A, \leftaction{0.4}{white})}  \to^{\chi}  \overline{(A, \leftaction{0.4}{white}) \ox (B, \leftaction{0.4}{black})} := 
B \ox A \to^{(c_\ox)_{B,A}} A \ox B\]  \[(A, \overline{\overline{\leftaction{0.4}{white}}}) \to^{\varepsilon} 
(A, \leftaction{0.4}{white}) := 1\]

 The natural isormorphisms satisfy all the coherences of conjugative symmetric $*$-autonomous category.
\end{proof}

\begin{lemma}
\label{Lemma conjugative}
Suppose $\X$ is a symmetric (iso)mix $*$-autonomous category, then {\bf H-Mod}$_\X$, the category of Hopf 
modules over a cocommutative Hopf Algebra H is a (iso)mix conjugative symmetric $*$-autonomous category.
\end{lemma}
\begin{proof}~
The mix map $\m: \bot \to \top$ is inherited directly from $\X$.
\end{proof}

\begin{corollary}
Suppose $\X$ is a symmetric (iso)mix $*$-autonomous, then {\bf H-Mod}$_\X$, the category of modules over a 
cocommutative Hopf Algebra H is a symmetric $\dagger$ (iso)mix $*$-autonomous category.
\end{corollary}
\begin{proof}
From Lemma \ref{Lemma conjugative},  ${\mbox{\bf H-Mod}}_\X$ is an  (iso)mix conjugative symmetric 
$*$-autonomous category. Then, by Theorem \ref{Theorem: conjugation+dualizing} one can construct a 
dagger functor by composing the conjugation and the dualizing functor as follows:  $(\_)^\dagger := \overline{(\_)^*}:
{\mbox{\bf H-Mod}}_\X^{\op} \to {\mbox{\bf H-Mod}}_\X$. Therefore,

$(A, \leftaction{0.4}{white})^\dagger := (A^*, \overline{ \leftaction{0.4}{white}^*})$ where,
\[
(A, \overline{\leftaction{0.4}{white}}^*) := 
\begin{tikzpicture} %dagger1
	\begin{pgfonlayer}{nodelayer}
		\node [style=none] (0) at (1.5, -1) {};
		\node [style=none] (1) at (2.25, -1) {};
		\node [style=none] (2) at (1.5, 0.5) {};
		\node [style=none] (3) at (1.5, 0.75) {};
		\node [style=none] (4) at (0.25, 0.75) {};
		\node [style=none] (5) at (2.25, 3.5) {};
		\node [style=none] (6) at (0.25, -1.25) {};
		\node [style=none] (7) at (0.75, 3.5) {};
		\node [style=none] (8) at (1, 3.25) {$H$};
		\node [style=none] (9) at (2.5, 3.25) {$A^*$};
		\node [style=none] (10) at (0.25, -1.75) {};
		\node [style=none] (11) at (0.5, -1.5) {$A^*$};
		\node [style=circle, scale=1.5] (12) at (0.75, 2) {};
		\node [style=none] (13) at (0.75, 2.25) {};
		\node [style=none] (14) at (0.75, 2) {$s$};
		\node [style=circle, scale=1.5] (15) at (0.75, 2.75) {};
		\node [style=none] (16) at (0.75, 2.75) {$s$};
		\node [style=none] (17) at (1.5, 0.5) {};
		\node [style=none] (18) at (1.5, 0.25) {};
		\node [style=none] (19) at (1.25, 0.5) {};
	\end{pgfonlayer}
	\begin{pgfonlayer}{edgelayer}
		\draw [in=-90, out=90, looseness=1.00] (2.center) to (3.center);
		\draw [bend left=90, looseness=1.75] (4.center) to (3.center);
		\draw [bend right=90, looseness=2.00] (0.center) to (1.center);
		\draw (5.center) to (1.center);
		\draw (4.center) to (6.center);
		\draw (13.center) to (12);
		\draw (7.center) to (15);
		\draw (15) to (13.center);
		\draw (6.center) to (10.center);
		\draw (19.center) to (17.center);
		\draw (17.center) to (18.center);
		\draw (18.center) to (19.center);
		\draw [bend right=15, looseness=1.25] (12) to (19.center);
		\draw (18.center) to (0.center);
	\end{pgfonlayer}
\end{tikzpicture} =
\begin{tikzpicture} %dagger2
	\begin{pgfonlayer}{nodelayer}
		\node [style=none] (0) at (1.5, -1) {};
		\node [style=none] (1) at (2.25, -1) {};
		\node [style=none] (2) at (1.5, 0.5) {};
		\node [style=none] (3) at (1.5, 1) {};
		\node [style=none] (4) at (-0.25, 1) {};
		\node [style=none] (5) at (2.25, 3) {};
		\node [style=none] (6) at (0.5, 2.75) {$H$};
		\node [style=none] (7) at (2.5, 2.75) {$A^*$};
		\node [style=none] (8) at (-0.25, -1.75) {};
		\node [style=none] (9) at (0, -1.5) {$A^*$};
		\node [style=none] (10) at (0.75, 3) {};
		\node [style=none] (11) at (1.25, 0.5) {};
		\node [style=none] (12) at (1.5, 0.5) {};
		\node [style=none] (13) at (1.5, 0.25) {};
		\node [style=none] (14) at (2.75, 3.75) {};
	\end{pgfonlayer}
	\begin{pgfonlayer}{edgelayer}
		\draw [in=-90, out=90, looseness=1.00] (2.center) to (3.center);
		\draw [bend left=90, looseness=2.00] (4.center) to (3.center);
		\draw [bend right=90, looseness=2.00] (0.center) to (1.center);
		\draw (5.center) to (1.center);
		\draw (11.center) to (12.center);
		\draw (11.center) to (13.center);
		\draw (13.center) to (12.center);
		\draw (0.center) to (13.center);
		\draw [in=-90, out=135, looseness=1.00] (11.center) to (10.center);
		\draw (4.center) to (8.center);
	\end{pgfonlayer}
\end{tikzpicture}
\]
\end{proof}

Thus, one can generate a $\dagger$-isomix category from a symmetric isomix $*$-autonomous 
category by choosing the Hopf modules over any cocommutative $\ox$- Hopf Algebra. 
