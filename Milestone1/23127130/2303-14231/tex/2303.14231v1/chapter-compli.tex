% !TEX root = thesis.tex

\chapter{Measurement and complementarity}
\label{Chapter: complementarity}

In this chapter, we describe measurement in MUCs and characterize measurements using `compaction'. 
We also develop the notion of $(\dagger)$-complementary systems in $(\dagger)$-isomix categories 
and show that in an isomix category with free exponential modalities, every complementary 
system arises as a canonical compaction of a linear bialgebra on the free exponential modalities.  

Coecke and Pavlovic \cite{CoP07} described a ``demolition''
measurement in a $\dagger$-monoidal category as a map, $m: A \to X$, with 
$m^\dagger m = 1_X$, to a special commutative $\dagger$-Frobenius algebra, $X$.   
In this chapter, we generalize this idea to MUCs. In the $\dagger$-isomix 
category of a MUC, generally, $A \neq A^\dagger$, hence the equation 
$m^\dag m = 1$ does not make sense.  
However, the notion of a demolition measurement is available in the 
unitary core of the MUC. Thus, in a MUC, a measurement can be viewed as a two-step process in 
which one first ``compacts'', by a retraction, into the unitary core 
and then one does a conventional demolition measurement.  The compaction 
process is discussed in Section \ref{Sec: compaction} and gives rise to the notion 
of $\dagger$-binary idempotent.  On the other hand, 
a $\dagger$-binary idempotent which splits and satisfies the technical condition 
of being `coring', gives rise to a compaction to the ``canonical'' unitary core.  

\section{Compaction}
\label{Sec: compaction}

In order to perform a measurement on an object $A$ of the $\dagger$-isomix category of a MUC, we must
first compact $A$ into an object in the unitary core: 

\begin{definition}
Let $M: \U \to \C$ be a MUC. A {\bf compaction } to $U$ of an object $A \in \C$ is a retraction, 
$r: A \to M(U)$.  This means that  there is a section $s: M(U) \to A$ such that $sr = 1_{M(U)}$.
A compaction is said to be {\bf canonical} when $\U$ is the canonical unitary core, in other words,
$U$ is a pre-unitary object.
\end{definition}

An object, $M(U)$ for $U \in \U$, has a unitary structure map which is an isomorphism between $M(U)$ 
and $M(U)^\dagger$ given by composing the unitary structure map of $U$ with the preservator: 
\[ \psi := M(U) \to^{M(\varphi)} M(U^\dagger) \to^{\rho} M(U)^\dagger \] 
Once one has reached $M(U)$, one can follow with a conventional demolition measurement $U \to^w X$ in $\U$ to obtain an overall 
compaction $A \to^{r M(w)} M(X)$.  This composite may be viewed as being the analogue of a demolition measurement in a MUC. 

We start by showing how a compaction gives rise to a binary idempotent:

\begin{definition}
A {\bf binary idempotent} in any category is a pair of maps $(\u,\v)$ with $\u: A \to B$, and $\v: B \to A$ such that $\u\v\u = \u$, and $\v \u \v = \v$.
\end{definition}

A binary idempotent, $(\u, \v): A \to B$ gives a pair of idempotents $e_A := \u \v : A \to A$, and $e_B := \v \u : B \to B$. 
We say a binary idempotent {\bf splits} in case the idempotents $e_A$ and $e_B$ split.

\begin{lemma}
	\label{Lemma: binary idempotent equivalence}
	In any category the following are equivalent: 
\begin{enumerate}[(i)]
\item $(\u, \v) : A \to B$ is a binary idempotent which splits.
\item There exists a pair of idempotents $e: A \to A$, and $d:B \to B$ which split through isomorphic objects.
\end{enumerate}
\end{lemma}
\begin{proof}~
	\begin{description}
	\item[$(i) \Rightarrow (ii)$:]
	Suppose that $\u\v$ splits as $A \to^r A' \to^s A$ (so $e := rs = \u\v$ and $sr=1_{A'}$)and 
	$d := \v\u$ splits as $B \to^p B' \to^q B$ (so $pq = \v\u$ and $qp =1_{B'}$) then 
	we obtain two maps $\alpha:=s \u p: A' \to B'$ and $\beta:=q \v r:  B' \to A'$ which are inverse to each other:
	\begin{align*}
	\alpha \beta & := (s \u p)(q \v r) = s \u (pq) \v r = s \u \v \u \v r = s \u \v r = srsr = 1_{A'} \\
	\beta \alpha & := (q \v r)(s \u p) = q \v (rs) \u p= q \v \u \v \u p = q \v \u p = q p q p = 1_{B'}.
	\end{align*}
	\item[$(ii) \Rightarrow (i)$:]
	Suppose $e_A: A \to A$ and $e_B: B \to B$ are idempotents which split, respectively,  as $A \to^r A' \to^s A$ (so $rs = e_A$ and $sr=1_{A'}$) and $B \to^p B' \to^q B$ 
	(so $pq = e_B$ and $qp =1_{B'}$), so that $\gamma: A' \to B'$ is an isomorphism then we obtain two maps  $\u :=r \gamma q: A \to B$  and $\v:= p \gamma^{-1} s: B \to A$.  We observe:
	\begin{align*}
	\u\v\u &:= r \gamma qp \gamma^{-1} sr \gamma q = r \gamma \gamma^{-1} sr \gamma q = r sr \gamma q = r\gamma q=\u \\
	\v\u\v & := p \gamma^{-1} sr \gamma q p \gamma^{-1} s = p \gamma^{-1} \gamma q p \gamma^{-1} s = p  q p \gamma^{-1} s = p \gamma^{-1} s = \v.
	\end{align*}
	\end{description}
\end{proof}

Observe that a compaction of an object, $A$, in any MUC, gives the following system of maps:

{ \centering 
$\xymatrixcolsep{12mm}
 \xymatrix{
    A  \ar@<3pt>[r]^r &  
    M(U) \ar@<3pt>[l]^s \ar@<3pt>[r]^{\psi := M(\varphi)\rho} & 
    M(U)^\dagger \ar@<3pt>[r]^{r^\dagger} \ar@<3pt>[l]^{\psi^{-1}} &
    A^\dagger \ar@<3pt>[l]^{s^\dagger}    } $
\par }

Thus the compaction gives rise to a binary idempotent  $(\u,\v): A \to A^\dagger$ 
where $\u:= r\psi r^\dagger$ and $\v:= s^\dagger \psi^{-1} s$ 

Because $U$ is a unitary object, we have that $\varphi (\varphi^{-1 \dagger}) = \iota$.   
The preservator, on the other hand, satisfies $\iota \rho^\dagger = M(\iota) \rho$  
(see after Definition 3.17 in \cite{CCS18}).  Thus $\iota \rho^\dagger = M(\iota) \rho = 
M(\varphi\varphi^{-1 \dagger})\rho = M(\varphi)\rho M(\varphi^{-1})^\dagger$ and whence 
$\psi = M(\varphi) \rho= \iota \rho^\dagger M(\varphi)^\dagger = 
\iota(M(\varphi)\rho)^\dagger = \iota \psi^\dagger$.  
This allows us to observe:
\begin{align*} 
\iota\u^\dagger & = \iota (r \psi r^\dagger)^\dagger = \iota r^{\dagger\dagger} \psi^\dagger r^\dagger = r \iota \psi^\dagger r^\dagger = r \psi r^\dagger = \u \\
\v^\dagger & = (s^\dagger \psi^{-1} s)^\dagger = s^\dagger (\psi^{\dagger})^{-1} s^{\dagger\dagger} 
= s^\dagger (\iota^{-1}\psi)^{-1} s^{\dagger\dagger} 
 = s^\dagger \psi^{-1} \iota s^{\dagger\dagger} = s^\dagger \psi^{-1} s \iota  = \v \iota
\end{align*}

This leads to  the following definition:

\begin{definition}
A binary idempotent, $(\u, \v): A \to A^\dagger$ in a $\dagger$-LDC, is a {\bf $\dagger$-binary idempotent}, 
written $\dagger(\u, \v)$, if $\u  = \iota \u^\dagger$,
and $\v^\dagger = \v \iota$.
\end{definition}

In a $\dagger$-monoidal category, where $A=A^\dagger$ and $\iota=1_A$ 
this makes $\u=\u^\dagger$ and $\v=\v^\dagger$, thus  $\u\v= (\v\u)^\dagger$.  This means 
that if we require $\u\v = \v\u$ we obtain a dagger idempotent in the sense of \cite{Sel08}.  

Splitting a dagger binary idempotent almost produces a pre-unitary object.    In a $\dagger$-LDC, we shall call an object $A$
 with an isomorphism $\varphi: A \to A^\dagger$ such that $\varphi \varphi^{\dagger -1} = \iota$ a {\bf weak pre-unitary object}.  
 In a $\dagger$-isomix category, a weak pre-unitary object $(A,\varphi)$ is a pre-unitary object when, in addition, 
 $A$ is in the core.  We next observe that dagger binary idempotent always split through weak pre-unitary objects:


\begin{lemma} In a $\dagger$-LDC with a $\dagger$-binary idempotent $\dagger(\u,v): A \to A^\dagger$:
\label{Lemma: dagger splitting}
\begin{enumerate}[(i)]
\item $e_{A^\dagger} := \v \u = (\u \v)^\dagger =: (e_A)^\dagger$;
\item if $\dagger(\u,\v)$ splits with $e_A = A \to^r E \to^s A$ 
then $(E, s \u s^\dag)$ with  is a weak pre-unitary object.
\end{enumerate}
\end{lemma}
\begin{proof}~
	\begin{enumerate}[{\em (i)}]
	\item  $e_{A^\dagger} = \v \u = \v \iota \u^\dagger = \v^\dagger \u^\dagger = (\u\v)^\dagger = e_A^\dagger$.
	\item Suppose $(\u, \v)$ is a $\dagger$-binary idempotent, and $\u\v$ splits as $\u\v = A \to^{r} U \to^{s} A$. This means $\v\u$ splits as 
		$\v\u = A^\dagger \to^{s^\dagger} U^\dagger \to^{r^\dagger} A^\dagger$. This yields an isomorphism  $\alpha= s \u s^\dagger: E \to E^\dagger$ 
		satisfying:
		 \[ \alpha (\alpha^{-1})^\dagger = s \u s^\dagger (r^\dagger \v r)^\dagger 
			= s \u s^\dagger r^\dagger \v^\dagger r^{\dagger \dagger} 
			= s \u \v \u \v^\dagger r^{\dagger \dagger} 
			= s \u \v^\dagger r^{\dagger \dagger} 
			= s \u \v \iota r^{\dagger \dagger} 
			= s \u \v r \iota 
			= s r s r \iota  = \iota.\]
		\vspace{-2em}
	\end{enumerate}
\end{proof}

Thus, in a $\dagger$-isomix category, an object which splits a $\dagger$-binary idempotent is always weakly pre-unitary.  
In order to ensure that the splitting of a dagger binary idempotent is a pre-unitary object -- and so a canonical compaction -- it 
remains to ensure that the splitting is in the core. This leads to the following definition:

\begin{definition}
	An idempotent $A \to^{e} A$ in an isomix category, $\X$, is a {\bf coring idempotent} if it is equipped with natural
	$\kappa^L_X: X \oa A \to X \ox A$ and $\kappa^R_X: A \oa X \to A \ox X$ such that the following diagrams 
	commute:
    \[ \xymatrix{
    X \ox A \ar[r]^{1 \ox e}  \ar[d]_{1 \ox e} \ar@{}[dr]|{\bf [KL.1]}  & X \ox A \ar[d]^{\mx}\\
    X \ox A  & X \oa A \ar[l]^{\kappa^L_X}
    } ~~~~~~~~ \xymatrix{
    X \oa A \ar[r]^{1 \oa e}  \ar[d]_{1 \oa e} \ar@{}[dr]|{\bf [KL.2]}  & X \oa A \ar[d]^{\kappa^L_X}\\
    X \oa A  & X \ox A \ar[l]^{\mx}
    }
    ~~~~~~~~ \xymatrix{
    A \ox X \ar[r]^{e \ox 1}  \ar[d]_{e \ox 1} \ar@{}[dr]|{\bf [KR.1]}  & A \ox X \ar[d]^{\mx}\\
    A \ox X  & A \oa X \ar[l]^{\kappa^R_X}
    } ~~~~~~~~ \xymatrix{
    A \oa X \ar[r]^{e \oa 1}  \ar[d]_{e \oa 1} \ar@{}[dr]|{\bf [KR.2]}  & A \oa X \ar[d]^{\kappa^R_X}\\
    A \oa X  & A \ox X \ar[l]^{\mx}
    } \]
\end{definition}

For a coring idempotent $A \to^{e} A$, the transformations $\kappa^{\_}_X$ act on a splitting as the inverse of the mixor, $\mx$.  Thus, a coring idempotent  
always splits through the core:

\begin{lemma} In a mix category: 
\label{Lemma: pseudocore}
\begin{enumerate}[(i)]
\item An idempotent splits through the core if and only if it is coring;
\item If $(u,v)$ is a binary idempotent then $\u\v$ is coring if and only if $\v\u$ is coring.
\end{enumerate}
\end{lemma}
\begin{proof} ~
	\begin{enumerate}[{\em (i)}]
	\item  Let $A \to^e A$ be a coring idempotent which splits as $A \to^{r} U \to^{s} A$. Define $\mx'_{U,X} := U \oa X \to^{s \oa 1} 
	A \oa X \to^{\kappa_R^X} A \ox X \to^{r \ox 1} U \ox 1$, then $U$ is in the core because $\mx'_{U,X} = \mx_{U,X}^{-1}$. 
	Conversely, the inverse of the mixor for $U$ and an object $X$ defines the $\kappa^{\_}_X$.
	\item The splitting of $u\v$ is isomorphic to the splitting of $\v\u$ and the core includes isomorphisms.
	\end{enumerate}
\end{proof}
This allows:
\begin{definition}
A {\bf coring binary idempotent} in a mix category is a binary idempotent, $(\u,\v)$, for which either $\u\v$ or $\v \u$ 
is a coring idempotent.
\end{definition}

These observations can be summarized by the following:

\begin{theorem}
In the MUC $M : \Unitary(\X) \to \X$, with $\dagger$-isomix category $\X$, an object $U$ is a compaction of $A$ if and only if $U$ is the
splitting of a coring $\dagger$-binary idempotent $\dagger(\u,\v): A \to A^\dagger$.
\end{theorem}

Using this characterization of canonical compaction, we will show that, in the presence of free $\dagger$-exponential 
modalities, complementarity always arises as a canonical compaction of a $\dagger$-linear bialgebra on 
the exponential modalities. 

%%%%%%%%%%%%%%%%%%%%%%%%%%%%%%%%%%%%%%%%%%%%%%%%%
\section{Binary idempotents}

In this section, we investigate binary idempotents for linear monoids, comonoids 
and bialgebras in order to answer the following question: {\em When does a binary idempotent 
on these structures split through an object with the same structure?} 
For this, we introduce the notion of sectional and retractional 
binary idempotents for duals, linear monoids, comonoids, and bialgebras. 
These idempotents will be used in Section \ref{Sec: dagger exp modalities} 
to obtain a connection between exponential modalities and complementary 
systems.

Splitting a sectional or a retractional 
binary idempotent on a linear monoid induces a self-linear monoid on the splitting. 
The result applies to linear comonoids and bialgebras as well. 

\subsection{For linear monoids}

A binary idempotent can implicitly express a morphism of duals, which becomes explicit when the idempotent splits.

\begin{definition}
In an LDC, a pair of idempotents, $(e_A : A \to A,  e_B: B \to B)$ is {\bf retractional} on a dual $(\eta, \epsilon): A \dashvv B$ 
if equations $(a)$ and $(b)$, below, hold. On the other hand, $(e_A, e_B)$ is {\bf sectional}, if equations $(c)$ and $(d)$ hold: 
\[ (a)~
	\begin{tikzpicture}
		\begin{pgfonlayer}{nodelayer}
			\node [style=none] (0) at (-5, 0.25) {};
			\node [style=none] (1) at (-5, 1.5) {};
			\node [style=none] (2) at (-3.5, 1.5) {};
			\node [style=none] (3) at (-3.5, 0.25) {};
			\node [style=circle, scale=1.8] (4) at (-5, 1) {};
			\node [style=none] (5) at (-5, 1) {$e_A$};
			\node [style=none] (6) at (-5.25, 0.5) {$A$};
			\node [style=none] (7) at (-3.25, 0.5) {$B$};
		\end{pgfonlayer}
		\begin{pgfonlayer}{edgelayer}
			\draw (0.center) to (4);
			\draw (1.center) to (4);
			\draw [bend left=90, looseness=1.75] (1.center) to (2.center);
			\draw (3.center) to (2.center);
		\end{pgfonlayer}
	\end{tikzpicture} =\begin{tikzpicture}
		\begin{pgfonlayer}{nodelayer}
			\node [style=none] (0) at (-5, 0.25) {};
			\node [style=none] (1) at (-5, 1.5) {};
			\node [style=none] (2) at (-3.5, 1.5) {};
			\node [style=none] (3) at (-3.5, 0.25) {};
			\node [style=circle, scale=1.8] (4) at (-5, 1) {};
			\node [style=none] (5) at (-5, 1) {$e_A$};
			\node [style=none] (6) at (-5.25, 0.5) {$A$};
			\node [style=none] (7) at (-3.25, 0.5) {$B$};
			\node [style=circle, scale=1.8] (8) at (-3.5, 1) {};
			\node [style=none] (9) at (-3.5, 1) {$e_B$};
		\end{pgfonlayer}
		\begin{pgfonlayer}{edgelayer}
			\draw (0.center) to (4);
			\draw (1.center) to (4);
			\draw [bend left=90, looseness=1.75] (1.center) to (2.center);
			\draw (3.center) to (8);
			\draw (8) to (2.center);
		\end{pgfonlayer}
	\end{tikzpicture}
~~~~
(b) ~ \begin{tikzpicture}
	\begin{pgfonlayer}{nodelayer}
		\node [style=none] (0) at (0.75, 1.5) {};
		\node [style=none] (1) at (0.75, 0.25) {};
		\node [style=none] (2) at (2.25, 0.25) {};
		\node [style=none] (3) at (2.25, 1.5) {};
		\node [style=circle, scale=1.8] (4) at (0.75, 0.75) {};
		\node [style=none] (5) at (0.75, 0.75) {$e_B$};
		\node [style=none] (6) at (2.5, 1.25) {$A$};
		\node [style=none] (7) at (0.5, 1.25) {$B$};
		\node [style=none] (8) at (1.5, 2) {};
	\end{pgfonlayer}
	\begin{pgfonlayer}{edgelayer}
		\draw (0.center) to (4);
		\draw (1.center) to (4);
		\draw [bend right=90, looseness=1.75] (1.center) to (2.center);
		\draw (3.center) to (2.center);
	\end{pgfonlayer}
\end{tikzpicture}
 =\begin{tikzpicture}
	\begin{pgfonlayer}{nodelayer}
		\node [style=none] (0) at (-3.25, 1.5) {};
		\node [style=none] (1) at (-3.25, 0.25) {};
		\node [style=none] (2) at (-4.75, 0.25) {};
		\node [style=none] (3) at (-4.75, 1.5) {};
		\node [style=circle, scale=1.8] (4) at (-3.25, 0.75) {};
		\node [style=none] (5) at (-3.25, 0.75) {$e_A$};
		\node [style=none] (6) at (-3, 1.25) {$A$};
		\node [style=none] (7) at (-5, 1.25) {$B$};
		\node [style=circle, scale=1.8] (8) at (-4.75, 0.75) {};
		\node [style=none] (9) at (-4.75, 0.75) {$e_B$};
		\node [style=none] (10) at (-4, 2) {};
	\end{pgfonlayer}
	\begin{pgfonlayer}{edgelayer}
		\draw (0.center) to (4);
		\draw (1.center) to (4);
		\draw [bend left=90, looseness=1.75] (1.center) to (2.center);
		\draw (3.center) to (8);
		\draw (8) to (2.center);
	\end{pgfonlayer}
\end{tikzpicture}
~~~~~~
(c) ~\begin{tikzpicture}
	\begin{pgfonlayer}{nodelayer}
		\node [style=none] (0) at (-3.25, 0.25) {};
		\node [style=none] (1) at (-3.25, 1.5) {};
		\node [style=none] (2) at (-4.75, 1.5) {};
		\node [style=none] (3) at (-4.75, 0.25) {};
		\node [style=circle, scale=1.8] (4) at (-3.25, 1) {};
		\node [style=none] (5) at (-3.25, 1) {$e_B$};
		\node [style=none] (6) at (-3, 0.5) {$B$};
		\node [style=none] (7) at (-5, 0.5) {$A$};
	\end{pgfonlayer}
	\begin{pgfonlayer}{edgelayer}
		\draw (0.center) to (4);
		\draw (1.center) to (4);
		\draw [bend right=90, looseness=1.75] (1.center) to (2.center);
		\draw (3.center) to (2.center);
	\end{pgfonlayer}
\end{tikzpicture}  =\begin{tikzpicture}
	\begin{pgfonlayer}{nodelayer}
		\node [style=none] (0) at (-5, 0.25) {};
		\node [style=none] (1) at (-5, 1.5) {};
		\node [style=none] (2) at (-3.5, 1.5) {};
		\node [style=none] (3) at (-3.5, 0.25) {};
		\node [style=circle, scale=1.8] (4) at (-5, 1) {};
		\node [style=none] (5) at (-5, 1) {$e_A$};
		\node [style=none] (6) at (-5.25, 0.5) {$A$};
		\node [style=none] (7) at (-3.25, 0.5) {$B$};
		\node [style=circle, scale=1.8] (8) at (-3.5, 1) {};
		\node [style=none] (9) at (-3.5, 1) {$e_B$};
	\end{pgfonlayer}
	\begin{pgfonlayer}{edgelayer}
		\draw (0.center) to (4);
		\draw (1.center) to (4);
		\draw [bend left=90, looseness=1.75] (1.center) to (2.center);
		\draw (3.center) to (8);
		\draw (8) to (2.center);
	\end{pgfonlayer}
\end{tikzpicture}
~~~
(d) ~ \begin{tikzpicture}
	\begin{pgfonlayer}{nodelayer}
		\node [style=none] (0) at (-3.5, 1.5) {};
		\node [style=none] (1) at (-3.5, 0.25) {};
		\node [style=none] (2) at (-5, 0.25) {};
		\node [style=none] (3) at (-5, 1.5) {};
		\node [style=circle, scale=1.8] (4) at (-3.5, 0.75) {};
		\node [style=none] (5) at (-3.5, 0.75) {$e_A$};
		\node [style=none] (6) at (-5.25, 1.25) {$B$};
		\node [style=none] (7) at (-3.25, 1.25) {$A$};
		\node [style=none] (8) at (-4.25, 2) {};
	\end{pgfonlayer}
	\begin{pgfonlayer}{edgelayer}
		\draw (0.center) to (4);
		\draw (1.center) to (4);
		\draw [bend left=90, looseness=1.75] (1.center) to (2.center);
		\draw (3.center) to (2.center);
	\end{pgfonlayer}
\end{tikzpicture}
 =\begin{tikzpicture}
	\begin{pgfonlayer}{nodelayer}
		\node [style=none] (0) at (-3.25, 1.5) {};
		\node [style=none] (1) at (-3.25, 0.25) {};
		\node [style=none] (2) at (-4.75, 0.25) {};
		\node [style=none] (3) at (-4.75, 1.5) {};
		\node [style=circle, scale=1.8] (4) at (-3.25, 0.75) {};
		\node [style=none] (5) at (-3.25, 0.75) {$e_A$};
		\node [style=none] (6) at (-3, 1.25) {$A$};
		\node [style=none] (7) at (-5, 1.25) {$B$};
		\node [style=circle, scale=1.8] (8) at (-4.75, 0.75) {};
		\node [style=none] (9) at (-4.75, 0.75) {$e_B$};
		\node [style=none] (10) at (-4, 2) {};
	\end{pgfonlayer}
	\begin{pgfonlayer}{edgelayer}
		\draw (0.center) to (4);
		\draw (1.center) to (4);
		\draw [bend left=90, looseness=1.75] (1.center) to (2.center);
		\draw (3.center) to (8);
		\draw (8) to (2.center);
	\end{pgfonlayer}
\end{tikzpicture}\] 
A binary idempotent $(\u, \v)$ is {\bf sectional} (respectively {\bf retractional}) 
if (\u\v, \v\u) is sectional (respectively retractional).
\end{definition}
The idempotent pair $(e_A, e_B)$ is a morphism of duals if and only if it is both sectional and retractional. 
Splitting binary idempotents which are either sectional or retractional on a dual produces a self-duality. 

\begin{lemma} 
	\label{Lemma: sectional dual morphism}
	In an LDC, a pair of idempotents $(e_A, e_B)$, on a dual $(\eta, \epsilon): A \dashvv B$, 
with splitting $A \to^r E \to^s$ A, and $B \to^{r'} E' \to^{s'} B$ is sectional (respectively retractional) if and only if the 
section $(s, r')$ (respectively the retraction $(r,s')$) is a morphism of duals for $(\eta', \epsilon'): E \dashvv E'$ where:

{\centering $\eta' := \begin{tikzpicture}
	\begin{pgfonlayer}{nodelayer}
		\node [style=none] (0) at (-23.25, -22) {};
		\node [style=none] (1) at (-22, -22) {};
		\node [style=onehalfcircle] (2) at (-23.25, -21) {};
		\node [style=onehalfcircle] (3) at (-22, -21) {};
		\node [style=none] (4) at (-22.5, -19.75) {$\eta$};
		\node [style=none] (5) at (-23.25, -21) {$r$};
		\node [style=none] (6) at (-22, -21) {$r'$};
		\node [style=none] (7) at (-23.5, -21.75) {$E$};
		\node [style=none] (8) at (-21.75, -21.75) {$E'$};
	\end{pgfonlayer}
	\begin{pgfonlayer}{edgelayer}
		\draw [bend left=90, looseness=1.75] (2) to (3);
		\draw (1.center) to (3);
		\draw (0.center) to (2);
	\end{pgfonlayer}
\end{tikzpicture} ~~~~~~~~ \epsilon' := \begin{tikzpicture}
	\begin{pgfonlayer}{nodelayer}
		\node [style=none] (0) at (-22, -20) {};
		\node [style=none] (1) at (-23.25, -20) {};
		\node [style=onehalfcircle] (2) at (-22, -21) {};
		\node [style=onehalfcircle] (3) at (-23.25, -21) {};
		\node [style=none, scale=1.5] (4) at (-22.75, -22.25) {$\epsilon$};
		\node [style=none] (5) at (-22, -21) {$s$};
		\node [style=none] (6) at (-23.25, -21) {$s'$};
		\node [style=none] (7) at (-21.75, -20.25) {$E$};
		\node [style=none] (8) at (-23.5, -20.25) {$E'$};
	\end{pgfonlayer}
	\begin{pgfonlayer}{edgelayer}
		\draw [bend left=90, looseness=1.75] (2) to (3);
		\draw (1.center) to (3);
		\draw (0.center) to (2);
	\end{pgfonlayer}
\end{tikzpicture}$ \par}
\end{lemma}
\begin{proof} Suppose $(\u, \v)$ is sectional binary idempotent of a dual 
	$(\eta, \epsilon): A \dashvv B$ with 	the splitting $A \to^r E \to^s$ A, 
	and $B \to^{r'} E' \to^{s'} B$. Let $e_A = \u \v$, and $e_B = \v \u$. We must show that 
	$(\eta(r \oa r'), (s' \ox s) \epsilon): E \dashvv E'$ is dual and $(s,r')$ is a dual homomorphism. 

	We first prove that $(\eta', \epsilon'): E \dashvv E'$ satisfies the snake equations, where 
	$\eta' = \eta(r \oa r')$, $\epsilon' = (s' \ox s) \epsilon$.
	\[ \begin{tikzpicture}
		\begin{pgfonlayer}{nodelayer}
			\node [style=none] (0) at (-2, -0.25) {};
			\node [style=none] (1) at (-1, -0.25) {};
			\node [style=none] (4) at (-2, -2.75) {};
			\node [style=none] (10) at (-2.25, -2.25) {$E$};
			\node [style=none] (12) at (0, -1.75) {};
			\node [style=none] (13) at (-1, -1.75) {};
			\node [style=none] (16) at (0, 0.5) {};
			\node [style=none] (22) at (0.25, 0.25) {$E$};
			\node [style=none] (24) at (-1.5, 0.5) {$\eta'$};
			\node [style=none] (25) at (-0.5, -2.5) {$\epsilon'$};
		\end{pgfonlayer}
		\begin{pgfonlayer}{edgelayer}
			\draw [bend left=90, looseness=1.50] (0.center) to (1.center);
			\draw [bend left=90, looseness=1.50] (12.center) to (13.center);
			\draw (0.center) to (4.center);
			\draw (13.center) to (1.center);
			\draw (12.center) to (16.center);
		\end{pgfonlayer}
	\end{tikzpicture}
	 =\begin{tikzpicture}
		\begin{pgfonlayer}{nodelayer}
			\node [style=none] (0) at (-2, -0.25) {};
			\node [style=none] (1) at (-1, -0.25) {};
			\node [style=circle, scale=1.5] (2) at (-2, -1.25) {};
			\node [style=circle, scale=1.5] (3) at (-1, -0.5) {};
			\node [style=none] (4) at (-2, -2.75) {};
			\node [style=none] (6) at (-2, -1.25) {$r$};
			\node [style=none] (7) at (-1, -0.5) {$r'$};
			\node [style=none] (8) at (-2.25, -0.25) {$A$};
			\node [style=none] (9) at (-0.75, 0) {$B$};
			\node [style=none] (10) at (-2.25, -2.25) {$E$};
			\node [style=none] (12) at (0, -1.75) {};
			\node [style=none] (13) at (-1, -1.75) {};
			\node [style=circle, scale=1.5] (14) at (0, -1) {};
			\node [style=circle, scale=1.5] (15) at (-1, -1.5) {};
			\node [style=none] (16) at (0, 0.5) {};
			\node [style=none] (18) at (0, -1) {$s$};
			\node [style=none] (19) at (-1, -1.5) {$s'$};
			\node [style=none] (20) at (0.25, -1.75) {$A$};
			\node [style=none] (21) at (-1.25, -2) {$B$};
			\node [style=none] (22) at (0.25, 0) {$E$};
			\node [style=none] (23) at (-1.25, -1) {$E'$};
			\node [style=none] (24) at (-1.5, 0.5) {$\eta$};
			\node [style=none] (25) at (-0.5, -2.5) {$\epsilon$};
		\end{pgfonlayer}
		\begin{pgfonlayer}{edgelayer}
			\draw (4.center) to (2);
			\draw (2) to (0.center);
			\draw [bend left=90, looseness=1.50] (0.center) to (1.center);
			\draw (1.center) to (3);
			\draw (16.center) to (14);
			\draw (14) to (12.center);
			\draw [bend left=90, looseness=1.50] (12.center) to (13.center);
			\draw (13.center) to (15);
			\draw (3) to (15);
		\end{pgfonlayer}
	\end{tikzpicture}
		 = \begin{tikzpicture}
		\begin{pgfonlayer}{nodelayer}
			\node [style=none] (0) at (-2, -0.25) {};
			\node [style=none] (1) at (-1, -0.25) {};
			\node [style=circle, scale=1.5] (2) at (-2, -1.25) {};
			\node [style=circle, scale=1.5] (3) at (-1, -1) {};
			\node [style=none] (4) at (-2, -2.75) {};
			\node [style=none] (6) at (-2, -1.25) {$r$};
			\node [style=none] (7) at (-1, -1) {$e_B$};
			\node [style=none] (8) at (-2.25, -0.25) {$A$};
			\node [style=none] (9) at (-0.75, -0.25) {$B$};
			\node [style=none] (10) at (-2.25, -2.25) {$E$};
			\node [style=none] (12) at (0, -1.75) {};
			\node [style=none] (13) at (-1, -1.75) {};
			\node [style=circle, scale=1.5] (14) at (0, -1) {};
			\node [style=none] (16) at (0, 0.5) {};
			\node [style=none] (18) at (0, -1) {$s$};
			\node [style=none] (20) at (0.25, -1.75) {$A$};
			\node [style=none] (21) at (-1.25, -1.75) {$B$};
			\node [style=none] (22) at (0.25, 0) {$E$};
			\node [style=none] (24) at (-1.5, 0.5) {$\eta$};
			\node [style=none] (25) at (-0.5, -2.5) {$\epsilon$};
		\end{pgfonlayer}
		\begin{pgfonlayer}{edgelayer}
			\draw (4.center) to (2);
			\draw (2) to (0.center);
			\draw [bend left=90, looseness=1.50] (0.center) to (1.center);
			\draw (1.center) to (3);
			\draw (16.center) to (14);
			\draw (14) to (12.center);
			\draw [bend left=90, looseness=1.50] (12.center) to (13.center);
			\draw (3) to (13.center);
		\end{pgfonlayer}
	\end{tikzpicture} =\begin{tikzpicture}
		\begin{pgfonlayer}{nodelayer}
			\node [style=none] (0) at (-2, -0.25) {};
			\node [style=none] (1) at (-1, -0.25) {};
			\node [style=circle, scale=1.5] (2) at (-2, -1.25) {};
			\node [style=circle, scale=1.5] (3) at (-1, -1) {};
			\node [style=none] (4) at (-2, -2.75) {};
			\node [style=none] (6) at (-2, -1.25) {$r$};
			\node [style=none] (7) at (-1, -1) {$e_B$};
			\node [style=none] (8) at (-2.25, -0.25) {$A$};
			\node [style=none] (9) at (-0.75, -0.25) {$B$};
			\node [style=none] (10) at (-2.25, -2.25) {$E$};
			\node [style=none] (12) at (0, -1.75) {};
			\node [style=none] (13) at (-1, -1.75) {};
			\node [style=circle, scale=1.5] (14) at (0, -1.5) {};
			\node [style=none] (16) at (0, 0.5) {};
			\node [style=none] (18) at (0, -1.5) {$s$};
			\node [style=none] (20) at (0.5, -1.75) {$A$};
			\node [style=none] (21) at (-1.25, -1.75) {$B$};
			\node [style=none] (22) at (0.5, 0.25) {$E$};
			\node [style=none] (24) at (-1.5, 0.5) {$\eta$};
			\node [style=none] (25) at (-0.5, -2.5) {$\epsilon$};
			\node [style=circle, scale=1.5] (26) at (0, 0) {};
			\node [style=none] (27) at (0, 0) {$s$};
			\node [style=circle, scale=1.5] (29) at (0, -0.75) {};
			\node [style=none] (28) at (0, -0.75) {$r$};
		\end{pgfonlayer}
		\begin{pgfonlayer}{edgelayer}
			\draw (4.center) to (2);
			\draw (2) to (0.center);
			\draw [bend left=90, looseness=1.50] (0.center) to (1.center);
			\draw (1.center) to (3);
			\draw (14) to (12.center);
			\draw [bend left=90, looseness=1.50] (12.center) to (13.center);
			\draw (3) to (13.center);
			\draw (16.center) to (26);
			\draw (26) to (29);
			\draw (29) to (14);
		\end{pgfonlayer}
	\end{tikzpicture}	
	= \begin{tikzpicture}
		\begin{pgfonlayer}{nodelayer}
			\node [style=none] (0) at (-2, -0.25) {};
			\node [style=none] (1) at (-1, -0.25) {};
			\node [style=circle, scale=1.5] (2) at (-2, -1.25) {};
			\node [style=circle, scale=1.5] (3) at (-1, -1) {};
			\node [style=none] (4) at (-2, -2.75) {};
			\node [style=none] (6) at (-2, -1.25) {$r$};
			\node [style=none] (7) at (-1, -1) {$e_B$};
			\node [style=none] (8) at (-2.25, -0.25) {$A$};
			\node [style=none] (9) at (-0.75, -0.25) {$B$};
			\node [style=none] (10) at (-2.25, -2.25) {$E$};
			\node [style=none] (12) at (0, -1.75) {};
			\node [style=none] (13) at (-1, -1.75) {};
			\node [style=none] (16) at (0, 0.5) {};
			\node [style=none] (20) at (0.25, -1.75) {$A$};
			\node [style=none] (21) at (-1.25, -1.75) {$B$};
			\node [style=none] (22) at (0.25, 0.25) {$E$};
			\node [style=none] (24) at (-1.5, 0.5) {$\eta$};
			\node [style=none] (25) at (-0.5, -2.5) {$\epsilon$};
			\node [style=circle, scale=1.5] (26) at (0, -0.25) {};
			\node [style=none] (27) at (0, -0.25) {$s$};
			\node [style=circle, scale=1.5] (29) at (0, -1) {};
			\node [style=none] (28) at (0, -1) {$e_A$};
		\end{pgfonlayer}
		\begin{pgfonlayer}{edgelayer}
			\draw (4.center) to (2);
			\draw (2) to (0.center);
			\draw [bend left=90, looseness=1.50] (0.center) to (1.center);
			\draw (1.center) to (3);
			\draw [bend left=90, looseness=1.50] (12.center) to (13.center);
			\draw (3) to (13.center);
			\draw (16.center) to (26);
			\draw (12.center) to (29);
			\draw (29) to (26);
		\end{pgfonlayer}
	\end{tikzpicture} \stackrel{\text{sectional}}{=}\begin{tikzpicture}
		\begin{pgfonlayer}{nodelayer}
			\node [style=none] (0) at (-2, -0.25) {};
			\node [style=none] (1) at (-1, -0.25) {};
			\node [style=circle, scale=1.5] (2) at (-2, -1.25) {};
			\node [style=none] (4) at (-2, -2.75) {};
			\node [style=none] (6) at (-2, -1.25) {$r$};
			\node [style=none] (8) at (-2.25, -0.25) {$A$};
			\node [style=none] (9) at (-0.75, -0.25) {$B$};
			\node [style=none] (10) at (-2.25, -2.25) {$E$};
			\node [style=none] (12) at (0, -1.75) {};
			\node [style=none] (13) at (-1, -1.75) {};
			\node [style=none] (16) at (0, 0.5) {};
			\node [style=none] (20) at (0.25, -1.75) {$A$};
			\node [style=none] (21) at (-1.25, -1.75) {$B$};
			\node [style=none] (22) at (0.25, 0.25) {$E$};
			\node [style=none] (24) at (-1.5, 0.5) {$\eta$};
			\node [style=none] (25) at (-0.5, -2.5) {$\epsilon$};
			\node [style=circle, scale=1.5] (26) at (0, -0.25) {};
			\node [style=none] (27) at (0, -0.25) {$s$};
			\node [style=circle, scale=1.5] (29) at (0, -1) {};
			\node [style=none] (28) at (0, -1) {$e_A$};
		\end{pgfonlayer}
		\begin{pgfonlayer}{edgelayer}
			\draw (4.center) to (2);
			\draw (2) to (0.center);
			\draw [bend left=90, looseness=1.50] (0.center) to (1.center);
			\draw [bend left=90, looseness=1.50] (12.center) to (13.center);
			\draw (16.center) to (26);
			\draw (12.center) to (29);
			\draw (29) to (26);
			\draw (1.center) to (13.center);
		\end{pgfonlayer}
	\end{tikzpicture}	 = \begin{tikzpicture}
		\begin{pgfonlayer}{nodelayer}
			\node [style=none] (22) at (1.25, -1.25) {$E$};
			\node [style=none] (30) at (1, 0.5) {};
			\node [style=none] (31) at (1, -2.75) {};
		\end{pgfonlayer}
		\begin{pgfonlayer}{edgelayer}
			\draw (30.center) to (31.center);
		\end{pgfonlayer}
	\end{tikzpicture} \]
Similary, the other snake equation can be proven.

Next, we show that $(s, r'): ((\eta', \epsilon'): E \dashvv E') \to ((\eta, \epsilon): A \dashvv B)$ is a 
dual homomorphism:
\[ \begin{tikzpicture}
	\begin{pgfonlayer}{nodelayer}
		\node [style=none] (0) at (-2, 0) {};
		\node [style=none] (1) at (-0.75, 0) {};
		\node [style=none] (4) at (-2, -2.25) {};
		\node [style=none] (5) at (-0.75, -2.25) {};
		\node [style=none] (8) at (-2.25, -0.75) {$E$};
		\node [style=none] (10) at (-2.25, -2) {$A$};
		\node [style=none] (11) at (-0.5, -2) {$E'$};
		\node [style=circle, scale=1.5] (12) at (-2, -1.25) {};
		\node [style=none] (13) at (-2, -1.25) {$s$};
		\node [style=none] (20) at (-1.5, 0.75) {$\eta'$};
	\end{pgfonlayer}
	\begin{pgfonlayer}{edgelayer}
		\draw [bend left=90, looseness=1.50] (0.center) to (1.center);
		\draw (4.center) to (12);
		\draw (5.center) to (1.center);
		\draw (12) to (0.center);
	\end{pgfonlayer}
\end{tikzpicture} = \begin{tikzpicture}
	\begin{pgfonlayer}{nodelayer}
		\node [style=none] (0) at (-2, 0) {};
		\node [style=none] (1) at (-0.75, 0) {};
		\node [style=circle, scale=1.5] (2) at (-2, -0.25) {};
		\node [style=circle, scale=1.5] (3) at (-0.75, -0.25) {};
		\node [style=none] (4) at (-2, -2.25) {};
		\node [style=none] (5) at (-0.75, -2.25) {};
		\node [style=none] (6) at (-2, -0.25) {$r$};
		\node [style=none] (7) at (-0.75, -0.25) {$r'$};
		\node [style=none] (8) at (-2.25, -0.75) {$E$};
		\node [style=none] (9) at (-0.5, 0.2) {$B$};
		\node [style=none] (10) at (-2.25, -2) {$A$};
		\node [style=none] (11) at (-0.5, -2) {$E'$};
		\node [style=circle, scale=1.5] (12) at (-2, -1.25) {};
		\node [style=none] (13) at (-2, -1.25) {$s$};
		\node [style=none] (20) at (-1.5, 0.75) {$\eta$};
	\end{pgfonlayer}
	\begin{pgfonlayer}{edgelayer}
		\draw (2) to (0.center);
		\draw [bend left=90, looseness=1.50] (0.center) to (1.center);
		\draw (1.center) to (3);
		\draw (4.center) to (12);
		\draw (2) to (13.center);
		\draw (3) to (5.center);
	\end{pgfonlayer}
\end{tikzpicture}
 = \begin{tikzpicture}
	\begin{pgfonlayer}{nodelayer}
		\node [style=none] (0) at (-2, 0) {};
		\node [style=none] (1) at (-0.75, 0) {};
		\node [style=circle, scale=1.5] (2) at (-2, -0.25) {};
		\node [style=circle, scale=1.5] (3) at (-0.75, -0.25) {};
		\node [style=none] (4) at (-2, -2.25) {};
		\node [style=none] (5) at (-0.75, -2.25) {};
		\node [style=none] (6) at (-2, -0.25) {$r$};
		\node [style=none] (7) at (-0.75, -0.25) {$r'$};
		\node [style=none] (8) at (-2.25, -0.75) {$E$};
		\node [style=none] (9) at (-0.5, 0.2) {$B$};
		\node [style=none] (10) at (-2.25, -2) {$A$};
		\node [style=none] (11) at (-0.5, -2) {$E'$};
		\node [style=circle, scale=1.5] (12) at (-2, -1.25) {};
		\node [style=none] (13) at (-2, -1.25) {$s$};
		\node [style=circle, scale=1.5] (15) at (-0.75, -1) {};
		\node [style=circle, scale=1.5] (17) at (-0.75, -1.75) {};
		\node [style=none] (18) at (-0.75, -1) {$s'$};
		\node [style=none] (16) at (-0.75, -1.75) {$r'$};
		\node [style=none] (20) at (-1.5, 0.75) {$\eta$};
	\end{pgfonlayer}
	\begin{pgfonlayer}{edgelayer}
		\draw (2) to (0.center);
		\draw [bend left=90, looseness=1.50] (0.center) to (1.center);
		\draw (1.center) to (3);
		\draw (4.center) to (12);
		\draw (3) to (15);
		\draw (15) to (17);
		\draw (2) to (13.center);
		\draw (5.center) to (17);
	\end{pgfonlayer}
\end{tikzpicture} = \begin{tikzpicture}
	\begin{pgfonlayer}{nodelayer}
		\node [style=none] (0) at (-2, 0) {};
		\node [style=none] (1) at (-0.75, 0) {};
		\node [style=circle, scale=1.5] (2) at (-2, -0.25) {};
		\node [style=none] (4) at (-2, -2.25) {};
		\node [style=none] (5) at (-0.75, -2.25) {};
		\node [style=none] (6) at (-2, -0.25) {$r$};
		\node [style=none] (8) at (-2.25, -0.75) {$E$};
		\node [style=none] (9) at (-0.5, 0) {$B$};
		\node [style=none] (10) at (-2.25, -2) {$A$};
		\node [style=none] (11) at (-0.5, -2.2) {$E'$};
		\node [style=circle, scale=1.5] (12) at (-2, -1.25) {};
		\node [style=none] (13) at (-2, -1.25) {$s$};
		\node [style=circle, scale=1.5] (15) at (-0.75, -1) {};
		\node [style=circle, scale=1.5] (17) at (-0.75, -1.75) {};
		\node [style=none] (18) at (-0.75, -1) {$e_B$};
		\node [style=none] (16) at (-0.75, -1.75) {$r'$};
		\node [style=none] (19) at (-0.75, 0) {};
		\node [style=none] (20) at (-1.5, 0.75) {$\eta$};
	\end{pgfonlayer}
	\begin{pgfonlayer}{edgelayer}
		\draw (2) to (0.center);
		\draw [bend left=90, looseness=1.50] (0.center) to (1.center);
		\draw (4.center) to (12);
		\draw (15) to (17);
		\draw (2) to (13.center);
		\draw (5.center) to (17);
		\draw (19.center) to (15);
	\end{pgfonlayer}
\end{tikzpicture} = \begin{tikzpicture}
	\begin{pgfonlayer}{nodelayer}
		\node [style=none] (0) at (-2, 0) {};
		\node [style=none] (1) at (-0.75, 0) {};
		\node [style=none] (4) at (-2, -2.25) {};
		\node [style=none] (5) at (-0.75, -2.25) {};
		\node [style=none] (9) at (-0.5, 0) {$B$};
		\node [style=none] (10) at (-2.25, -2) {$A$};
		\node [style=none] (11) at (-0.5, -2) {$E'$};
		\node [style=circle, scale=1.5] (12) at (-2, -1) {};
		\node [style=none] (13) at (-2, -1) {$e_A$};
		\node [style=circle, scale=1.5] (15) at (-0.75, -0.75) {};
		\node [style=circle, scale=1.5] (17) at (-0.75, -1.5) {};
		\node [style=none] (18) at (-0.75, -0.75) {$e_B$};
		\node [style=none] (16) at (-0.75, -1.5) {$r'$};
		\node [style=none] (19) at (-0.75, 0) {};
		\node [style=none] (20) at (-1.5, 0.75) {$\eta$};
	\end{pgfonlayer}
	\begin{pgfonlayer}{edgelayer}
		\draw [bend left=90, looseness=1.50] (0.center) to (1.center);
		\draw (4.center) to (12);
		\draw (15) to (17);
		\draw (5.center) to (17);
		\draw (19.center) to (15);
		\draw (12) to (0.center);
	\end{pgfonlayer}
\end{tikzpicture} = \begin{tikzpicture}
	\begin{pgfonlayer}{nodelayer}
		\node [style=none] (0) at (-2, 0) {};
		\node [style=none] (1) at (-0.75, 0) {};
		\node [style=none] (4) at (-2, -2.25) {};
		\node [style=none] (5) at (-0.75, -2.25) {};
		\node [style=none] (9) at (-0.5, 0) {$B$};
		\node [style=none] (10) at (-2.25, -2) {$A$};
		\node [style=none] (11) at (-0.5, -2) {$E'$};
		\node [style=circle, scale=1.5] (15) at (-0.75, -0.75) {};
		\node [style=circle, scale=1.5] (17) at (-0.75, -1.5) {};
		\node [style=none] (18) at (-0.75, -0.75) {$e_B$};
		\node [style=none] (16) at (-0.75, -1.5) {$r'$};
		\node [style=none] (19) at (-0.75, 0) {};
		\node [style=none] (20) at (-1.5, 0.75) {$\eta$};
	\end{pgfonlayer}
	\begin{pgfonlayer}{edgelayer}
		\draw [bend left=90, looseness=1.50] (0.center) to (1.center);
		\draw (15) to (17);
		\draw (5.center) to (17);
		\draw (19.center) to (15);
		\draw (4.center) to (0.center);
	\end{pgfonlayer}
\end{tikzpicture}
 = \begin{tikzpicture}
	\begin{pgfonlayer}{nodelayer}
		\node [style=none] (0) at (-2, 0) {};
		\node [style=none] (1) at (-0.75, 0) {};
		\node [style=none] (4) at (-2, -2.25) {};
		\node [style=none] (5) at (-0.75, -2.25) {};
		\node [style=none] (9) at (-0.5, -0.5) {$B$};
		\node [style=none] (10) at (-2.25, -2) {$A$};
		\node [style=none] (11) at (-0.5, -2) {$E'$};
		\node [style=circle, scale=1.5] (17) at (-0.75, -1) {};
		\node [style=none] (16) at (-0.75, -1) {$r'$};
		\node [style=none] (18) at (-1.25, 0.75) {$\eta$};
	\end{pgfonlayer}
	\begin{pgfonlayer}{edgelayer}
		\draw [bend left=90, looseness=1.50] (0.center) to (1.center);
		\draw (5.center) to (17);
		\draw (4.center) to (0.center);
		\draw (1.center) to (17);
	\end{pgfonlayer}
\end{tikzpicture} \]

For the converse assume that $(\eta' := \eta(r \oa r'), \epsilon':= (s' \ox s) \epsilon) : E \dashvv E'$ is a dual 
and $(s,r')$ is a dual homomorphism. We prove that $(\u, \v)$ is sectional on $(\eta, \epsilon): A \dashvv B$:
\[ \begin{tikzpicture}
	\begin{pgfonlayer}{nodelayer}
		\node [style=none] (0) at (-0.75, 0) {};
		\node [style=none] (1) at (-2, 0) {};
		\node [style=none] (4) at (-0.75, -2.25) {};
		\node [style=none] (5) at (-2, -2.25) {};
		\node [style=none] (8) at (-0.5, -0.5) {$B$};
		\node [style=none] (10) at (-0.5, -2) {$B$};
		\node [style=none] (11) at (-2.25, -2) {$A$};
		\node [style=circle, scale=1.5] (12) at (-0.75, -1) {};
		\node [style=none] (13) at (-0.75, -1) {$e_B$};
		\node [style=none] (20) at (-1.25, 0.75) {$\eta$};
	\end{pgfonlayer}
	\begin{pgfonlayer}{edgelayer}
		\draw [bend right=90, looseness=1.50] (0.center) to (1.center);
		\draw (4.center) to (12);
		\draw (5.center) to (1.center);
		\draw (12) to (0.center);
	\end{pgfonlayer}
\end{tikzpicture} = \begin{tikzpicture}
	\begin{pgfonlayer}{nodelayer}
		\node [style=none] (0) at (-0.75, 0) {};
		\node [style=none] (1) at (-2, 0) {};
		\node [style=none] (4) at (-0.75, -2.25) {};
		\node [style=none] (5) at (-2, -2.25) {};
		\node [style=none] (8) at (-0.5, -0.25) {$B$};
		\node [style=none] (10) at (-0.5, -2) {$B$};
		\node [style=none] (11) at (-2.25, -2) {$A$};
		\node [style=circle, scale=1.5] (12) at (-0.75, -0.75) {};
		\node [style=none] (13) at (-0.75, -0.75) {$r'$};
		\node [style=none] (20) at (-1.25, 0.75) {$\eta$};
		\node [style=circle, scale=1.5] (21) at (-0.75, -1.5) {};
		\node [style=none] (21) at (-0.75, -1.5) {$s'$};
	\end{pgfonlayer}
	\begin{pgfonlayer}{edgelayer}
		\draw [bend right=90, looseness=1.50] (0.center) to (1.center);
		\draw (5.center) to (1.center);
		\draw (12) to (0.center);
		\draw (4.center) to (21);
		\draw (12) to (21);
	\end{pgfonlayer}
\end{tikzpicture}
 = \begin{tikzpicture}
	\begin{pgfonlayer}{nodelayer}
		\node [style=none] (0) at (-0.75, 0) {};
		\node [style=none] (1) at (-2, 0) {};
		\node [style=none] (4) at (-0.75, -2.25) {};
		\node [style=none] (5) at (-2, -2.25) {};
		\node [style=none] (10) at (-2.25, -1.75) {$A$};
		\node [style=none] (11) at (-0.5, -1.75) {$B$};
		\node [style=circle, scale=1.5] (12) at (-0.75, -1) {};
		\node [style=none] (20) at (-1.25, 0.75) {$\eta'$};
		\node [style=circle, scale=1.5] (25) at (-2, -1) {};
		\node [style=none] (26) at (-0.75, -1) {$s'$};
		\node [style=none] (27) at (-2, -1) {$s$};
		\node [style=none] (28) at (-2.25, -0.25) {$E$};
		\node [style=none] (29) at (-0.5, -0.25) {$E'$};
	\end{pgfonlayer}
	\begin{pgfonlayer}{edgelayer}
		\draw [bend right=90, looseness=1.50] (0.center) to (1.center);
		\draw (12) to (0.center);
		\draw (1.center) to (25);
		\draw (25) to (5.center);
		\draw (4.center) to (12);
	\end{pgfonlayer}
\end{tikzpicture} =\begin{tikzpicture}
	\begin{pgfonlayer}{nodelayer}
		\node [style=none] (0) at (-0.75, 0) {};
		\node [style=none] (1) at (-2, 0) {};
		\node [style=none] (4) at (-0.75, -2.25) {};
		\node [style=none] (5) at (-2, -2.25) {};
		\node [style=none] (10) at (-2.25, -2) {$A$};
		\node [style=none] (11) at (-0.5, -2) {$B$};
		\node [style=circle, scale=1.5] (12) at (-0.75, -1.5) {};
		\node [style=none] (20) at (-1.25, 0.75) {$\eta$};
		\node [style=circle, scale=1.5] (25) at (-2, -1.5) {};
		\node [style=none] (26) at (-0.75, -1.5) {$s'$};
		\node [style=none] (27) at (-2, -1.5) {$s$};
		\node [style=none] (28) at (-2.25, -1) {$E$};
		\node [style=none] (29) at (-0.5, -1) {$E'$};
		\node [style=circle, scale=1.5] (30) at (-0.75, -0.5) {};
		\node [style=circle, scale=1.5] (31) at (-2, -0.5) {};
		\node [style=none] (32) at (-0.75, -0.5) {$r'$};
		\node [style=none] (33) at (-2, -0.5) {$r$};
		\node [style=none] (34) at (-2.25, 0) {$A$};
		\node [style=none] (35) at (-0.5, 0) {$B$};
	\end{pgfonlayer}
	\begin{pgfonlayer}{edgelayer}
		\draw [bend right=90, looseness=1.50] (0.center) to (1.center);
		\draw (25) to (5.center);
		\draw (4.center) to (12);
		\draw (1.center) to (31);
		\draw (0.center) to (30);
		\draw (25) to (31);
		\draw (12) to (30);
	\end{pgfonlayer}
\end{tikzpicture} = \begin{tikzpicture}
	\begin{pgfonlayer}{nodelayer}
		\node [style=none] (0) at (-0.75, 0) {};
		\node [style=none] (1) at (-2, 0) {};
		\node [style=none] (4) at (-0.75, -2.25) {};
		\node [style=none] (5) at (-2, -2.25) {};
		\node [style=none] (10) at (-2.25, -2) {$A$};
		\node [style=none] (11) at (-0.5, -2) {$B$};
		\node [style=circle, scale=1.5] (12) at (-0.75, -1) {};
		\node [style=none] (20) at (-1.25, 0.75) {$\eta$};
		\node [style=circle, scale=1.5] (25) at (-2, -1) {};
		\node [style=none] (26) at (-0.75, -1) {$e_B$};
		\node [style=none] (27) at (-2, -1) {$e_A$};
		\node [style=none] (34) at (-2.25, 0) {$A$};
		\node [style=none] (35) at (-0.5, 0) {$B$};
	\end{pgfonlayer}
	\begin{pgfonlayer}{edgelayer}
		\draw [bend right=90, looseness=1.50] (0.center) to (1.center);
		\draw (25) to (5.center);
		\draw (4.center) to (12);
		\draw (25) to (1.center);
		\draw (12) to (0.center);
	\end{pgfonlayer}
\end{tikzpicture} \]
Similary, one can prove the statement for retractional binary idempotents. 
\end{proof}
\begin{corollary}
	In an LDC, a binary idempotent $(\u, \v)$ on a dual $(\eta, \epsilon): A \dashvv B$, 
	with splitting $A \to^r E \to^s$ A, and $B \to^{r'} E' \to^{s'} B$ is sectional (respectively retractional) if and only if the 
	section $(s, r')$ (respectively the retraction $(r,s')$) is a morphism of duals for the 
	self-linear dual $(\eta', \epsilon'): E \dashvv E'$ where $\eta' := \eta(r \oa r')$ and $\epsilon' := (s' \ox s) \epsilon$.
\end{corollary}

A consequence of Lemma \ref{Lemma: sectional dual morphism} is that 
in a $\dagger$-LDC, splitting a $\dagger$-binary idempotent on a 
$\dagger$-dual gives a $\dagger$-self-duality if the binary idempotent 
is either sectional or retractional:

\begin{lemma}
	\label{Lemma: split weak dual hom}
	In a $\dagger$-LDC, a pair of idempotents $(e, e^\dag)$, on a $\dag$-dual 
	$(\eta, \epsilon): A \dashvv B$, with splitting $A \to^r E \to^s$ A is sectional (respectively retractional) if and only if the 
	section $(s, s^\dag)$ (respectively the retraction $(r, r^\dag)$) is a morphism for $(\eta(r \oa s^\dag), 
	(r^\dag \ox s) \epsilon): E \dashvv E^\dag$.
\end{lemma}
\begin{proof} 
	Let $(e, e^\dag)$ be sectional on a $\dagger$-dual $(\eta, \epsilon): A \dagdual A^\dagger$. 
	Let the idempotent split: 
	\[ e = A \to^r E \to^s A ~~~~~~~~\text{Hence, } e^\dag = A^\dagger \to^{s^\dagger} E^\dagger \to^{r^\dagger} A^\dagger \] 
	It follows from Lemma \ref{Lemma: sectional dual morphism} that the splitting 
	is self-dual. 
	The self-duality is given by $(\eta', \epsilon'): E \dashvv E^\dagger$, where $\eta' := \eta (r \oa s^\dagger)$, 
	and $\epsilon' := (r^\dagger \ox s) \epsilon $.  
	We must prove that the self-dual is also a $\dagger$-dual 
	i.e, equation \ref{eqn: right dagger dual}-(a) holds for $(\eta', \epsilon'): E \dashvv E^\dagger$:
	\begin {align*}
	\eta' (\iota \oa 1)  &=  \eta (r \oa s^\dagger) (\iota \oa 1) = \eta (r \iota \oa s^\dagger) \\
	&\stackrel{nat.~\iota}{=} \eta (\iota r^{\dagger \dagger} \oa s^\dagger) = 
	\eta (\iota \oa 1) (r^{\dagger \dagger} \oa s^\dagger) \\
	&\stackrel{\dagger-dual}{=} (\epsilon)^\dagger \lambda_\oa^{-1} ((r^{\dagger \dagger} \oa s^\dagger)) 
	\stackrel{nat.~\lambda_\oa}{=} (\epsilon)^\dagger  ((r^{\dagger } \ox s)^\dagger) \lambda_\oa^{-1} \\
	&= ((r^\dagger \ox s) \epsilon)^\dagger  \lambda_\oa^{-1} = (\epsilon')^\dagger \lambda_\oa^{-1}
	\end{align*}
	The statement is proven similarly when the idempotents are retractional.
	The converse is straightforward.
\end{proof}
\begin{corollary}
	In a $\dagger$-LDC, a $\dag$-binary idempotent  $\dag(\u, \v)$, on a $\dag$-dual 
	$(\eta, \epsilon): A \dashvv B$, with splitting $A \to^r E \to^s$ A is sectional (respectively retractional) if and only if the 
	section $(s, s^\dag)$ (respectively the retraction $(r, r^\dag)$) is a morphism for the self $\dagger$-dual $(\eta(r \oa s^\dag), 
	(r^\dag \ox s) \epsilon): E \dashvv E^\dag$.
\end{corollary}

%%%%%%%%%%%%%%%%%%%%%%%%%%%%%%%%%%%%%%%%%%%%%%%%%
%\subsection{Binary idempotents for linear monoids}

We move on to the idempotents for linear monoids. 

Given an idempotent, $e_A: A \to A$, and a monoid, $(A, m, u)$ in a monoidal category, $e_A$ is
{\bf retractional} on the  monoid if $e_A m = e_A m (e_A \ox e_A)$. $e_A$ is 
{\bf sectional} on the monoid if $m (e_A \ox e_A) =  e_A m (e_A \ox e_A)$ and $u e_A = u$.

\begin{lemma} 
\label{Lemma: sectional monoid morphism}
	In a monoidal category, a split idempotent $e: A \to A$ on a monoid $(A,m,u)$, 
with splitting $A \to^r E \to^s$ A, is sectional (respectively retractional) if and only if the 
section $s$ (respectively the retraction $r$) is a monoid morphism for $(E, (s \ox s)m r, u r)$.
\end{lemma}
\begin{proof} Suppose $e: A \to A$ is an idempotent with the splitting $A \to^r E \to^s$  and $(A, m, u)$ 
is a monoid. Suppose $e$ is sectional on A i.e, $(e \ox e) m = (e \ox e )m e$, and $ue =u$. 

We must prove that $(E, m', u')$ is a monoid where $m' := (s \ox s)m r$ and  $u' := u r$.
The unit law and associativity law are proven as follows. 
\[ (u' \ox 1) m' = (ur \ox 1)(s \ox s) m r = (urs \ox s) m r = (ue \ox s) m r 
\stackrel{sectional}{=} (u \ox s) m r =   sr = 1 \]
In the following string diagrams, we use black circle for $(E, m', u')$, 
and white circle $(A,m,u)$.
\[ \begin{tikzpicture}
	\begin{pgfonlayer}{nodelayer}
		\node [style=none] (0) at (0.25, 4) {};
		\node [style=none] (1) at (-0.5, 4) {};
		\node [style=none] (2) at (1.25, 4) {};
		\node [style=none] (3) at (0.25, -1) {};
		\node [style=none] (4) at (1.5, 3.75) {$E$};
		\node [style=none] (5) at (0.5, 3.75) {$E$};
		\node [style=none] (6) at (-0.75, 3.75) {$E$};
		\node [style=none] (7) at (0, -0.75) {$E$};
		\node [style=circle, fill=black] (8) at (0.75, 2.75) {};
		\node [style=circle, fill=black] (9) at (0.25, 1.75) {};
	\end{pgfonlayer}
	\begin{pgfonlayer}{edgelayer}
		\draw [in=45, out=-90] (2.center) to (8);
		\draw [in=-90, out=135] (8) to (0.center);
		\draw [in=30, out=-90, looseness=0.75] (8) to (9);
		\draw [in=-90, out=150, looseness=0.75] (9) to (1.center);
		\draw (9) to (3.center);
	\end{pgfonlayer}
\end{tikzpicture}
 := \begin{tikzpicture}
	\begin{pgfonlayer}{nodelayer}
		\node [style=circle, scale=1.5] (0) at (-1.5, 1) {};
		\node [style=circle, scale=1.5] (1) at (-0.75, 3.25) {};
		\node [style=circle, scale=1.5] (2) at (0.75, 3.25) {};
		\node [style=circle] (3) at (0, 2.25) {};
		\node [style=none] (4) at (-0.75, 4) {};
		\node [style=none] (5) at (0.75, 4) {};
		\node [style=none] (6) at (-1.5, 4) {};
		\node [style=circle, scale=1.5] (7) at (0, 1.5) {};
		\node [style=circle, scale=1.5] (8) at (0, 0.75) {};
		\node [style=circle] (9) at (-0.75, 0) {};
		\node [style=circle, scale=1.5] (10) at (-0.75, -0.75) {};
		\node [style=none] (11) at (-0.75, -1.25) {};
		\node [style=none] (12) at (-1.5, 1) {$s$};
		\node [style=none] (13) at (-0.75, 3.25) {$s$};
		\node [style=none] (14) at (0.75, 3.25) {$s$};
		\node [style=none] (15) at (0, 1.5) {$r$};
		\node [style=none] (16) at (0, 0.75) {$s$};
		\node [style=none] (17) at (-0.75, -0.75) {$r$};
		\node [style=none] (18) at (-1.75, 3.75) {$E$};
		\node [style=none] (19) at (-1, 3.75) {$E$};
		\node [style=none] (20) at (0.5, 3.75) {$E$};
		\node [style=none] (21) at (-1.25, -0.25) {$A$};
		\node [style=none] (22) at (-1.25, -1) {$E$};
	\end{pgfonlayer}
	\begin{pgfonlayer}{edgelayer}
		\draw (4.center) to (1);
		\draw (5.center) to (2);
		\draw [in=165, out=-90, looseness=1.25] (1) to (3);
		\draw [in=-90, out=15, looseness=1.25] (3) to (2);
		\draw (6.center) to (0);
		\draw [in=165, out=-90, looseness=1.25] (0) to (9);
		\draw (3) to (7);
		\draw (7) to (8);
		\draw [in=15, out=-105, looseness=1.25] (8) to (9);
		\draw (9) to (10);
		\draw (10) to (11.center);
	\end{pgfonlayer}
\end{tikzpicture}
= \begin{tikzpicture}
	\begin{pgfonlayer}{nodelayer}
		\node [style=circle, scale=1.5] (0) at (-1.5, 1.5) {};
		\node [style=circle, scale=1.5] (1) at (-0.75, 3.25) {};
		\node [style=circle, scale=1.5] (2) at (0.75, 3.25) {};
		\node [style=circle] (3) at (0, 2.25) {};
		\node [style=none] (4) at (-0.75, 4) {};
		\node [style=none] (5) at (0.75, 4) {};
		\node [style=none] (6) at (-1.5, 4) {};
		\node [style=circle, scale=1.5] (7) at (0, 1.5) {};
		\node [style=circle] (9) at (-0.75, 0) {};
		\node [style=circle, scale=1.5] (10) at (-0.75, -0.75) {};
		\node [style=none] (11) at (-0.75, -1.25) {};
		\node [style=none] (12) at (-1.5, 1.5) {$s$};
		\node [style=none] (13) at (-0.75, 3.25) {$s$};
		\node [style=none] (14) at (0.75, 3.25) {$s$};
		\node [style=none] (15) at (0, 1.5) {$e$};
		\node [style=none] (17) at (-0.75, -0.75) {$r$};
		\node [style=none] (18) at (-1.75, 3.75) {$E$};
		\node [style=none] (19) at (-1, 3.75) {$E$};
		\node [style=none] (20) at (0.5, 3.75) {$E$};
		\node [style=none] (21) at (-1.25, -0.25) {$A$};
		\node [style=none] (22) at (-1.25, -1) {$E$};
	\end{pgfonlayer}
	\begin{pgfonlayer}{edgelayer}
		\draw (4.center) to (1);
		\draw (5.center) to (2);
		\draw [in=165, out=-90, looseness=1.25] (1) to (3);
		\draw [in=-90, out=15, looseness=1.25] (3) to (2);
		\draw (6.center) to (0);
		\draw [in=165, out=-90, looseness=1.25] (0) to (9);
		\draw (3) to (7);
		\draw (9) to (10);
		\draw (10) to (11.center);
		\draw [in=-90, out=15] (9) to (7);
	\end{pgfonlayer}
\end{tikzpicture}
 = \begin{tikzpicture}
	\begin{pgfonlayer}{nodelayer}
		\node [style=circle, scale=1.5] (0) at (-1.5, 1.5) {};
		\node [style=circle, scale=1.5] (1) at (-0.75, 3.5) {};
		\node [style=circle, scale=1.5] (2) at (0.25, 3.5) {};
		\node [style=circle] (3) at (-0.25, 1.25) {};
		\node [style=none] (4) at (-0.75, 4) {};
		\node [style=none] (5) at (0.25, 4) {};
		\node [style=none] (6) at (-1.5, 4) {};
		\node [style=circle, scale=1.5] (7) at (-0.25, 0.75) {};
		\node [style=circle] (9) at (-0.75, 0) {};
		\node [style=circle, scale=1.5] (10) at (-0.75, -0.75) {};
		\node [style=none] (11) at (-0.75, -1.25) {};
		\node [style=none] (12) at (-1.5, 1.5) {$s$};
		\node [style=none] (13) at (-0.75, 3.5) {$s$};
		\node [style=none] (14) at (0.25, 3.5) {$s$};
		\node [style=none] (15) at (-0.25, 0.75) {$e$};
		\node [style=none] (17) at (-0.75, -0.75) {$r$};
		\node [style=none] (18) at (-1.75, 4) {$E$};
		\node [style=none] (19) at (-1, 4) {$E$};
		\node [style=none] (20) at (0, 4) {$E$};
		\node [style=none] (21) at (-1.25, -0.25) {$A$};
		\node [style=none] (22) at (-1.25, -1) {$E$};
		\node [style=circle, scale=1.5] (23) at (-0.75, 2.75) {};
		\node [style=circle, scale=1.5] (25) at (0.25, 2.75) {};
		\node [style=circle, scale=1.5] (26) at (-0.75, 2) {};
		\node [style=circle, scale=1.5] (27) at (0.25, 2) {};
		\node [style=none] (28) at (-0.75, 2.75) {$r$};
		\node [style=none] (29) at (0.25, 2.75) {$r$};
		\node [style=none] (30) at (0.25, 2) {$s$};
		\node [style=none] (31) at (-0.75, 2) {$s$};
	\end{pgfonlayer}
	\begin{pgfonlayer}{edgelayer}
		\draw (4.center) to (1);
		\draw (5.center) to (2);
		\draw (6.center) to (0);
		\draw [in=165, out=-90, looseness=1.25] (0) to (9);
		\draw (3) to (7);
		\draw (9) to (10);
		\draw (10) to (11.center);
		\draw [in=-90, out=15] (9) to (7);
		\draw [in=-90, out=30] (3) to (27);
		\draw (27) to (25);
		\draw [in=-90, out=150] (3) to (26);
		\draw (26) to (23);
		\draw (25) to (2);
		\draw (23) to (1);
	\end{pgfonlayer}
\end{tikzpicture}
\stackrel{(*)}{=} \begin{tikzpicture}
	\begin{pgfonlayer}{nodelayer}
		\node [style=circle, scale=1.5] (0) at (-1.5, 1.5) {};
		\node [style=circle, scale=1.5] (1) at (-0.5, 3.25) {};
		\node [style=circle, scale=1.5] (2) at (0.5, 3.25) {};
		\node [style=circle] (3) at (0, 1.25) {};
		\node [style=none] (4) at (-0.5, 4) {};
		\node [style=none] (5) at (0.5, 4) {};
		\node [style=none] (6) at (-1.5, 4) {};
		\node [style=circle] (9) at (-0.75, 0) {};
		\node [style=circle, scale=1.5] (10) at (-0.75, -0.75) {};
		\node [style=none] (11) at (-0.75, -1.25) {};
		\node [style=none] (12) at (-1.5, 1.5) {$s$};
		\node [style=none] (13) at (-0.5, 3.25) {$s$};
		\node [style=none] (14) at (0.5, 3.25) {$s$};
		\node [style=none] (17) at (-0.75, -0.75) {$r$};
		\node [style=none] (18) at (-1.75, 4) {$E$};
		\node [style=none] (19) at (-0.75, 4) {$E$};
		\node [style=none] (21) at (-1.25, -0.25) {$A$};
		\node [style=none] (22) at (-1.25, -1) {$E$};
		\node [style=circle, scale=1.5] (26) at (-0.5, 2) {};
		\node [style=circle, scale=1.5] (27) at (0.5, 2) {};
		\node [style=none] (30) at (0.5, 2) {$e$};
		\node [style=none] (31) at (-0.5, 2) {$e$};
	\end{pgfonlayer}
	\begin{pgfonlayer}{edgelayer}
		\draw (4.center) to (1);
		\draw (5.center) to (2);
		\draw (6.center) to (0);
		\draw [in=165, out=-90, looseness=1.25] (0) to (9);
		\draw (9) to (10);
		\draw (10) to (11.center);
		\draw [in=-90, out=30] (3) to (27);
		\draw [in=-90, out=150] (3) to (26);
		\draw [in=-90, out=15] (9) to (3);
		\draw (26) to (1);
		\draw (27) to (2);
	\end{pgfonlayer}
\end{tikzpicture} = \begin{tikzpicture}
	\begin{pgfonlayer}{nodelayer}
		\node [style=circle, scale=1.5] (0) at (0.25, 2) {};
		\node [style=circle, scale=1.5] (1) at (-0.75, 3.25) {};
		\node [style=circle] (3) at (-1.25, 1.25) {};
		\node [style=none] (4) at (-0.75, 4) {};
		\node [style=none] (5) at (-1.75, 4) {};
		\node [style=none] (6) at (0.25, 4) {};
		\node [style=circle] (9) at (-0.5, 0) {};
		\node [style=circle, scale=1.5] (10) at (-0.5, -0.75) {};
		\node [style=none] (11) at (-0.5, -1.25) {};
		\node [style=none] (12) at (0.25, 2) {$e$};
		\node [style=none] (13) at (-0.75, 3.25) {$s$};
		\node [style=none] (17) at (-0.5, -0.75) {$r$};
		\node [style=none] (18) at (0.5, 4) {$E$};
		\node [style=none] (19) at (-0.5, 4) {$E$};
		\node [style=none] (20) at (-1.5, 4) {$E$};
		\node [style=none] (21) at (-1.25, -0.25) {$A$};
		\node [style=none] (22) at (-1.25, -1) {$E$};
		\node [style=circle, scale=1.5] (26) at (-0.75, 2) {};
		\node [style=circle, scale=1.5] (27) at (-1.75, 2) {};
		\node [style=none] (30) at (-1.75, 2) {$s$};
		\node [style=none] (31) at (-0.75, 2) {$e$};
		\node [style=circle, scale=1.5] (32) at (0.25, 2.75) {};
		\node [style=none] (33) at (0.25, 2.75) {$s$};
	\end{pgfonlayer}
	\begin{pgfonlayer}{edgelayer}
		\draw (4.center) to (1);
		\draw [in=30, out=-90] (0) to (9);
		\draw (9) to (10);
		\draw (10) to (11.center);
		\draw [in=-90, out=150] (3) to (27);
		\draw [in=-90, out=30] (3) to (26);
		\draw [in=-90, out=165] (9) to (3);
		\draw (26) to (1);
		\draw (6.center) to (32);
		\draw (32) to (0);
		\draw (27) to (5.center);
	\end{pgfonlayer}
\end{tikzpicture} = \begin{tikzpicture}
	\begin{pgfonlayer}{nodelayer}
		\node [style=circle, scale=1.5] (0) at (0.25, 2) {};
		\node [style=circle, scale=1.5] (1) at (-0.75, 3.25) {};
		\node [style=circle] (3) at (-1.25, 1.25) {};
		\node [style=none] (4) at (-0.75, 4) {};
		\node [style=none] (5) at (-1.75, 4) {};
		\node [style=none] (6) at (0.25, 4) {};
		\node [style=circle] (9) at (-0.5, 0) {};
		\node [style=circle, scale=1.5] (10) at (-0.5, -0.75) {};
		\node [style=none] (11) at (-0.5, -1.25) {};
		\node [style=none] (12) at (0.25, 2) {$s$};
		\node [style=none] (13) at (-0.75, 3.25) {$s$};
		\node [style=none] (17) at (-0.5, -0.75) {$r$};
		\node [style=none] (18) at (0.5, 4) {$E$};
		\node [style=none] (19) at (-0.5, 4) {$E$};
		\node [style=none] (20) at (-1.5, 4) {$E$};
		\node [style=none] (21) at (-1.25, -0.25) {$A$};
		\node [style=none] (22) at (-1.25, -1) {$E$};
		\node [style=circle, scale=1.5] (26) at (-0.75, 2) {};
		\node [style=circle, scale=1.5] (27) at (-1.75, 2) {};
		\node [style=none] (31) at (-0.75, 2) {$e$};
		\node [style=circle, scale=1.5] (34) at (-1.75, 3.25) {};
		\node [style=none] (35) at (-1.75, 2) {$e$};
		\node [style=none] (36) at (-1.75, 3.25) {$s$};
	\end{pgfonlayer}
	\begin{pgfonlayer}{edgelayer}
		\draw (4.center) to (1);
		\draw [in=30, out=-90] (0) to (9);
		\draw (9) to (10);
		\draw (10) to (11.center);
		\draw [in=-90, out=150] (3) to (27);
		\draw [in=-90, out=30] (3) to (26);
		\draw [in=-90, out=165] (9) to (3);
		\draw (26) to (1);
		\draw (5.center) to (34);
		\draw (34) to (27);
		\draw (0) to (6.center);
	\end{pgfonlayer}
\end{tikzpicture} \stackrel{(*)}{=} \begin{tikzpicture}
	\begin{pgfonlayer}{nodelayer}
		\node [style=circle, scale=1.5] (0) at (0.25, 2) {};
		\node [style=circle, scale=1.5] (1) at (-0.75, 3.25) {};
		\node [style=circle] (3) at (-1.25, 1.75) {};
		\node [style=none] (4) at (-0.75, 4) {};
		\node [style=none] (5) at (-1.75, 4) {};
		\node [style=none] (6) at (0.25, 4) {};
		\node [style=circle] (9) at (-0.5, 0) {};
		\node [style=circle, scale=1.5] (10) at (-0.5, -0.75) {};
		\node [style=none] (11) at (-0.5, -1.25) {};
		\node [style=none] (12) at (0.25, 2) {$s$};
		\node [style=none] (13) at (-0.75, 3.25) {$s$};
		\node [style=none] (17) at (-0.5, -0.75) {$r$};
		\node [style=none] (18) at (0.5, 4) {$E$};
		\node [style=none] (19) at (-0.5, 4) {$E$};
		\node [style=none] (20) at (-1.5, 4) {$E$};
		\node [style=none] (21) at (0, -0.25) {$A$};
		\node [style=circle, scale=1.5] (26) at (-0.75, 2.5) {};
		\node [style=circle, scale=1.5] (27) at (-1.75, 2.5) {};
		\node [style=none] (31) at (-0.75, 2.5) {$e$};
		\node [style=circle, scale=1.5] (34) at (-1.75, 3.25) {};
		\node [style=none] (35) at (-1.75, 2.5) {$e$};
		\node [style=none] (36) at (-1.75, 3.25) {$s$};
		\node [style=circle, scale=1.5] (37) at (-1.25, 1) {};
		\node [style=none] (38) at (-1.25, 1) {$e$};
		\node [style=none] (39) at (0, -1) {$E$};
	\end{pgfonlayer}
	\begin{pgfonlayer}{edgelayer}
		\draw (4.center) to (1);
		\draw [in=30, out=-90] (0) to (9);
		\draw (9) to (10);
		\draw (10) to (11.center);
		\draw [in=-90, out=150] (3) to (27);
		\draw [in=-90, out=30] (3) to (26);
		\draw (26) to (1);
		\draw (5.center) to (34);
		\draw (34) to (27);
		\draw (0) to (6.center);
		\draw (3) to (37);
		\draw [in=165, out=-90] (37) to (9);
	\end{pgfonlayer}
\end{tikzpicture} = \begin{tikzpicture}
	\begin{pgfonlayer}{nodelayer}
		\node [style=none] (0) at (0.5, 4) {};
		\node [style=none] (1) at (1.25, 4) {};
		\node [style=none] (2) at (-0.5, 4) {};
		\node [style=none] (3) at (0.5, -1) {};
		\node [style=none] (4) at (-0.75, 3.75) {$E$};
		\node [style=none] (5) at (0.25, 3.75) {$E$};
		\node [style=none] (6) at (1.5, 3.75) {$E$};
		\node [style=none] (7) at (0.75, -0.75) {$E$};
		\node [style=circle, fill=black] (8) at (0, 2.75) {};
		\node [style=circle, fill=black] (9) at (0.5, 1.75) {};
	\end{pgfonlayer}
	\begin{pgfonlayer}{edgelayer}
		\draw [in=135, out=-90] (2.center) to (8);
		\draw [in=-90, out=45] (8) to (0.center);
		\draw [in=150, out=-90, looseness=0.75] (8) to (9);
		\draw [in=-90, out=30, looseness=0.75] (9) to (1.center);
		\draw (9) to (3.center);
	\end{pgfonlayer}
\end{tikzpicture}\]
The steps labeled $(*)$ are valid because $e$ is sectional on $(A,m,u)$.

Finally, $s: E \to A $ is a monoid homomorphism because: 
\[ m' s = (s \ox s) m r s = (srs \ox srs ) m e = (s \ox s)(e \ox e) m e = (s \ox s )(e \ox e) m = (s \ox s) m \]
For { the converse} assume that $(E, m', u')$ is a monoid where $m' = (s \ox s) m r$, and $u' = u r$, and 
$s$ is a monoid homomorphism. Then, $e: A \to A$ is sectional on $(A,m,u)$ because:
\begin{align*}
&u e = u r s = u' s = u \\
&(e \ox e) m = (r \ox r)(s \ox s) m = (r \ox r) m' s = (r \ox r)(s \ox s) m r s = (e \ox e) m e 
\end{align*}

The statement is proven similarly when the idempotent is retractional on the monoid.
\end{proof}

A pair of idempotents, $(e_A, e_B)$ is {\bf sectional} (respectively {\bf retractional}) 
on a linear monoid $A \linmonw B$ 
if the idempotent pair satisfies the conditions in the following table.
\[  \begin{tabular}{l|| l} 
		\hline
		 $(e_A, e_B)$ \textbf{sectional} on $A \linmonw B$  & $(e_A, e_B)$ \textbf{retractional} on $A \linmonw B$\\
		\hline
		 $e_A$ preserves $(A,m,u)$ sectionally &  $e_A$ preserves $(A,m,u)$ retractionally \\
		\hline
		 $(e_A, e_B)$ preserves $(\eta_L, \epsilon_L)\!\!:\! A \dashvv B$ sectionally &$(e_A, e_B)$ preserves $(\eta_L, \epsilon_L)\!\!:\! A \dashvv B$ retractionally\\
		\hline
		 $(e_B, e_A)$ preserves $(\eta_R, \epsilon_R)\!\!:\! B \dashvv A$ retractionally & $(e_B, e_A)$ preserves $(\eta_R, \epsilon_R)\!\!:\! B \dashvv A$ sectionally\\
		\hline
	  \end{tabular} \]
A binary idempotent $(\u, \v)$ on a linear monoid is {\bf sectional} (respectively {\bf retractional}) if 
$(\u\v, \v\u)$ is sectional (respectively retractional). 

Splitting a sectional or retractional idempotent on a linear monoid produces a linear monoid 
on the splitting:
\begin{lemma}
	\label{Lemma: split weak linear mon hom}
In an LDC, let $(e_A, e_B)$ be a pair of idempotents on a linear monoid  $A \linmonw B$ 
 and the idempotents  split as follows: $A \to^r E \to^s A$, and $B \to^{r'} E' \to^{s'} B$. 
The idempotent pair $(e_A, e_B)$ is sectional (respectively retractional) if and only if the 
section $(s, r') :(E \linmonb E') \to  (A \linmonw B)$ (respectively the retraction 
$(r, s'):(A \linmonw B) \to  (E \linmonb E')$) is a morphism of linear monoids.
\end{lemma}
\begin{proof}
	Suppose $(e_A, e_B)$ is sectional on $A \linmonw B$. Let,  
	\[ e_A = A \to^{r} E \to^{s} A ~~~~~~~~~~ e_B = B \to^{r'} E' \to^{s'} B \] 
	be a splitting of $(\u, \v)$. From Lemma \ref{Lemma: sectional monoid morphism} 
	we know that $(A, m', u')$ is a monoid where $m' = (s \ox s) m r$ and $u' = ur$, and  from
	Lemma \ref{Lemma: sectional dual morphism} we know that  $(\eta_L (r \ox r'), (s' \ox s) \epsilon_L): E \dashvv E'$, 
	and $(\eta_R (r' \ox r), (s \ox s') \epsilon_R): E' \dashvv E$ are the left and the right duals respectively. 
	
	To prove that the above data is a linear monoid we must  show that equation 
	\ref{eqn: linear monoid} holds:
	
	$\begin{tikzpicture}
		\begin{pgfonlayer}{nodelayer}
			\node [style=black] (0) at (3, 3) {};
			\node [style=none] (4) at (3, 2.25) {};
			\node [style=none] (5) at (4.25, 2.25) {};
			\node [style=none] (6) at (2.5, 4.5) {};
			\node [style=none] (7) at (3.5, 4.5) {};
			\node [style=none] (8) at (1.75, 4.5) {};
			\node [style=none] (9) at (1, 4.5) {};
			\node [style=none] (10) at (1.75, 1.5) {};
			\node [style=none] (11) at (1, 1.5) {};
			\node [style=none] (13) at (4.25, 5.5) {};
			\node [style=none] (23) at (2.25, 5) {$\eta_R'$};
			\node [style=none] (24) at (2.25, 5.5) {$\eta_R'$};
			\node [style=none] (25) at (3.5, 1.75) {$\epsilon_R'$};
			\node [style=none] (26) at (4.5, 5) {$E'$};
			\node [style=none] (27) at (0.75, 2) {$E'$};
			\node [style=none] (28) at (2, 2) {$E'$};
		\end{pgfonlayer}
		\begin{pgfonlayer}{edgelayer}
			\draw [bend left=270] (6.center) to (8.center);
			\draw [bend left=270] (7.center) to (9.center);
			\draw [bend right=90, looseness=0.75] (4.center) to (5.center);
			\draw (5.center) to (13.center);
			\draw [in=-90, out=45] (0) to (7.center);
			\draw [in=-90, out=135] (0) to (6.center);
			\draw (10.center) to (8.center);
			\draw (11.center) to (9.center);
			\draw (4.center) to (0);
		\end{pgfonlayer}
	\end{tikzpicture}  = 	\begin{tikzpicture}
			\begin{pgfonlayer}{nodelayer}
				\node [style=circle] (0) at (3, 3) {};
				\node [style=onehalfcircle] (1) at (2.5, 3.75) {};
				\node [style=onehalfcircle] (2) at (3.5, 3.75) {};
				\node [style=onehalfcircle] (3) at (3, 2.25) {};
				\node [style=none] (4) at (3, 1.75) {};
				\node [style=none] (5) at (4.25, 1.75) {};
				\node [style=none] (6) at (2.5, 4.5) {};
				\node [style=none] (7) at (3.5, 4.5) {};
				\node [style=none] (8) at (1.75, 4.5) {};
				\node [style=none] (9) at (1, 4.5) {};
				\node [style=none] (10) at (1.75, 1.5) {};
				\node [style=none] (11) at (1, 1.5) {};
				\node [style=none] (13) at (4.25, 5.5) {};
				\node [style=onehalfcircle] (14) at (1, 3.25) {};
				\node [style=onehalfcircle] (15) at (1.75, 3.25) {};
				\node [style=onehalfcircle] (16) at (4.25, 3.25) {};
				\node [style=none] (17) at (4.25, 3.25) {$s'$};
				\node [style=none] (18) at (1, 3.25) {$r$};
				\node [style=none] (19) at (1.75, 3.25) {$r$};
				\node [style=none] (20) at (2.5, 3.75) {$e_A$};
				\node [style=none] (21) at (3, 2.25) {$e_A$};
				\node [style=none] (22) at (3.5, 3.75) {$e_A$};
				\node [style=none] (23) at (2.25, 5) {$\eta_R$};
				\node [style=none] (24) at (2.25, 5.5) {$\eta_R$};
				\node [style=none] (25) at (3.5, 1.25) {$\epsilon_R$};
				\node [style=none] (26) at (4.5, 5) {$E'$};
				\node [style=none] (27) at (0.75, 2) {$E'$};
				\node [style=none] (28) at (2, 2) {$E'$};
				\node [style=none] (29) at (0.75, 4) {$B$};
				\node [style=none] (30) at (4.5, 2) {$B$};
			\end{pgfonlayer}
			\begin{pgfonlayer}{edgelayer}
				\draw [bend right] (0) to (2);
				\draw [bend left, looseness=1.25] (0) to (1);
				\draw (1) to (6.center);
				\draw [bend left=270] (6.center) to (8.center);
				\draw (2) to (7.center);
				\draw [bend left=270] (7.center) to (9.center);
				\draw [bend right=90, looseness=0.75] (4.center) to (5.center);
				\draw (5.center) to (16);
				\draw (16) to (13.center);
				\draw (10.center) to (15);
				\draw (15) to (8.center);
				\draw (11.center) to (14);
				\draw (14) to (9.center);
				\draw (4.center) to (3);
				\draw (3) to (0);
			\end{pgfonlayer}
		\end{tikzpicture} = \begin{tikzpicture}
			\begin{pgfonlayer}{nodelayer}
				\node [style=circle] (0) at (3, 3) {};
				\node [style=onehalfcircle] (1) at (2.5, 3.75) {};
				\node [style=onehalfcircle] (2) at (3.5, 3.75) {};
				\node [style=none] (4) at (3, 1.75) {};
				\node [style=none] (5) at (4.25, 1.75) {};
				\node [style=none] (6) at (2.5, 4.5) {};
				\node [style=none] (7) at (3.5, 4.5) {};
				\node [style=none] (8) at (1.75, 4.5) {};
				\node [style=none] (9) at (1, 4.5) {};
				\node [style=none] (10) at (1.75, 1.5) {};
				\node [style=none] (11) at (1, 1.5) {};
				\node [style=none] (13) at (4.25, 5.5) {};
				\node [style=onehalfcircle] (14) at (1, 3.25) {};
				\node [style=onehalfcircle] (15) at (1.75, 3.25) {};
				\node [style=onehalfcircle] (16) at (4.25, 3.25) {};
				\node [style=none] (17) at (4.25, 3.25) {$s'$};
				\node [style=none] (18) at (1, 3.25) {$r$};
				\node [style=none] (19) at (1.75, 3.25) {$r$};
				\node [style=none] (20) at (2.5, 3.75) {$e_A$};
				\node [style=none] (22) at (3.5, 3.75) {$e_A$};
				\node [style=none] (23) at (2.25, 5) {$\eta_R$};
				\node [style=none] (24) at (2.25, 5.5) {$\eta_R$};
				\node [style=none] (25) at (3.5, 1.25) {$\epsilon_R$};
				\node [style=none] (26) at (4.5, 5) {$E'$};
				\node [style=none] (27) at (0.75, 2) {$E'$};
				\node [style=none] (28) at (2, 2) {$E'$};
				\node [style=none] (29) at (0.75, 4) {$B$};
				\node [style=none] (30) at (4.5, 2) {$B$};
			\end{pgfonlayer}
			\begin{pgfonlayer}{edgelayer}
				\draw [bend right] (0) to (2);
				\draw [bend left, looseness=1.25] (0) to (1);
				\draw (1) to (6.center);
				\draw [bend left=270] (6.center) to (8.center);
				\draw (2) to (7.center);
				\draw [bend left=270] (7.center) to (9.center);
				\draw [bend right=90, looseness=0.75] (4.center) to (5.center);
				\draw (5.center) to (16);
				\draw (16) to (13.center);
				\draw (10.center) to (15);
				\draw (15) to (8.center);
				\draw (11.center) to (14);
				\draw (14) to (9.center);
				\draw (4.center) to (0);
			\end{pgfonlayer}
		\end{tikzpicture} \stackrel{(1)}{=} \begin{tikzpicture}
			\begin{pgfonlayer}{nodelayer}
				\node [style=circle] (0) at (3, 3) {};
				\node [style=onehalfcircle] (1) at (2.5, 3.75) {};
				\node [style=onehalfcircle] (2) at (3.5, 3.75) {};
				\node [style=none] (4) at (3, 1.75) {};
				\node [style=none] (5) at (4.25, 1.75) {};
				\node [style=none] (6) at (2.5, 4.5) {};
				\node [style=none] (7) at (3.5, 4.5) {};
				\node [style=none] (8) at (1.75, 4.5) {};
				\node [style=none] (9) at (1, 4.5) {};
				\node [style=none] (10) at (1.75, 1.5) {};
				\node [style=none] (11) at (1, 1.5) {};
				\node [style=none] (13) at (4.25, 5.5) {};
				\node [style=onehalfcircle] (14) at (1, 2.5) {};
				\node [style=onehalfcircle] (15) at (1.75, 2.5) {};
				\node [style=onehalfcircle] (16) at (4.25, 3.25) {};
				\node [style=none] (17) at (4.25, 3.25) {$s'$};
				\node [style=none] (18) at (1, 2.5) {$r'$};
				\node [style=none] (19) at (1.75, 2.5) {$r'$};
				\node [style=none] (20) at (2.5, 3.75) {$e_A$};
				\node [style=none] (22) at (3.5, 3.75) {$e_A$};
				\node [style=none] (23) at (2.25, 5) {$\eta_R$};
				\node [style=none] (24) at (2.25, 5.5) {$\eta_R$};
				\node [style=none] (25) at (3.5, 1.25) {$\epsilon_R$};
				\node [style=none] (26) at (4.5, 5) {$E'$};
				\node [style=none] (27) at (0.75, 2) {$E'$};
				\node [style=none] (28) at (2, 2) {$E'$};
				\node [style=none] (29) at (0.5, 4.25) {$B$};
				\node [style=none] (30) at (4.5, 2) {$B$};
				\node [style=onehalfcircle] (31) at (1, 3.25) {};
				\node [style=onehalfcircle] (32) at (1.75, 3.25) {};
				\node [style=onehalfcircle] (33) at (1, 4) {};
				\node [style=onehalfcircle] (34) at (1.75, 4) {};
				\node [style=none] (35) at (1, 3.25) {$s'$};
				\node [style=none] (36) at (1.75, 3.25) {$s'$};
				\node [style=none] (37) at (1, 4) {$r'$};
				\node [style=none] (38) at (1.75, 4) {$r'$};
			\end{pgfonlayer}
			\begin{pgfonlayer}{edgelayer}
				\draw [bend right] (0) to (2);
				\draw [bend left, looseness=1.25] (0) to (1);
				\draw (1) to (6.center);
				\draw [bend left=270] (6.center) to (8.center);
				\draw (2) to (7.center);
				\draw [bend left=270] (7.center) to (9.center);
				\draw [bend right=90, looseness=0.75] (4.center) to (5.center);
				\draw (5.center) to (16);
				\draw (16) to (13.center);
				\draw (10.center) to (15);
				\draw (11.center) to (14);
				\draw (4.center) to (0);
				\draw (14) to (31);
				\draw (15) to (32);
				\draw (31) to (33);
				\draw (33) to (9.center);
				\draw (32) to (34);
				\draw (34) to (8.center);
			\end{pgfonlayer}
		\end{tikzpicture} = \begin{tikzpicture}
			\begin{pgfonlayer}{nodelayer}
				\node [style=circle] (0) at (3, 3) {};
				\node [style=onehalfcircle] (1) at (2.5, 3.75) {};
				\node [style=onehalfcircle] (2) at (3.5, 3.75) {};
				\node [style=none] (4) at (3, 1.75) {};
				\node [style=none] (5) at (4.25, 1.75) {};
				\node [style=none] (6) at (2.5, 4.5) {};
				\node [style=none] (7) at (3.5, 4.5) {};
				\node [style=none] (8) at (1.75, 4.5) {};
				\node [style=none] (9) at (1, 4.5) {};
				\node [style=none] (10) at (1.75, 1.5) {};
				\node [style=none] (11) at (1, 1.5) {};
				\node [style=none] (13) at (4.25, 5.5) {};
				\node [style=onehalfcircle] (14) at (1, 2.5) {};
				\node [style=onehalfcircle] (15) at (1.75, 2.5) {};
				\node [style=onehalfcircle] (16) at (4.25, 3.25) {};
				\node [style=none] (17) at (4.25, 3.25) {$s'$};
				\node [style=none] (18) at (1, 2.5) {$r'$};
				\node [style=none] (19) at (1.75, 2.5) {$r'$};
				\node [style=none] (20) at (2.5, 3.75) {$e_A$};
				\node [style=none] (22) at (3.5, 3.75) {$e_A$};
				\node [style=none] (23) at (2.25, 5) {$\eta_R$};
				\node [style=none] (24) at (2.25, 5.5) {$\eta_R$};
				\node [style=none] (25) at (3.5, 1.25) {$\epsilon_R$};
				\node [style=none] (26) at (4.5, 5) {$E'$};
				\node [style=none] (27) at (0.75, 2) {$E'$};
				\node [style=none] (28) at (2, 2) {$E'$};
				\node [style=none] (29) at (0.5, 4.25) {$B$};
				\node [style=none] (30) at (4.5, 2) {$B$};
				\node [style=onehalfcircle] (31) at (1, 3.25) {};
				\node [style=onehalfcircle] (32) at (1.75, 3.25) {};
				\node [style=none] (35) at (1, 3.25) {$e_B$};
				\node [style=none] (36) at (1.75, 3.25) {$e_B$};
			\end{pgfonlayer}
			\begin{pgfonlayer}{edgelayer}
				\draw [bend right] (0) to (2);
				\draw [bend left, looseness=1.25] (0) to (1);
				\draw (1) to (6.center);
				\draw [bend left=270] (6.center) to (8.center);
				\draw (2) to (7.center);
				\draw [bend left=270] (7.center) to (9.center);
				\draw [bend right=90, looseness=0.75] (4.center) to (5.center);
				\draw (5.center) to (16);
				\draw (16) to (13.center);
				\draw (10.center) to (15);
				\draw (11.center) to (14);
				\draw (4.center) to (0);
				\draw (14) to (31);
				\draw (15) to (32);
				\draw (32) to (8.center);
				\draw (31) to (9.center);
			\end{pgfonlayer}
		\end{tikzpicture} \stackrel{(2)}{=}\begin{tikzpicture}
			\begin{pgfonlayer}{nodelayer}
				\node [style=circle] (0) at (3, 3) {};
				\node [style=none] (4) at (3, 1.75) {};
				\node [style=none] (5) at (4.25, 1.75) {};
				\node [style=none] (6) at (2.5, 4.5) {};
				\node [style=none] (7) at (3.5, 4.5) {};
				\node [style=none] (8) at (1.75, 4.5) {};
				\node [style=none] (9) at (1, 4.5) {};
				\node [style=none] (10) at (1.75, 1.5) {};
				\node [style=none] (11) at (1, 1.5) {};
				\node [style=none] (13) at (4.25, 5.5) {};
				\node [style=onehalfcircle] (15) at (1.75, 2.5) {};
				\node [style=onehalfcircle] (16) at (4.25, 3.25) {};
				\node [style=none] (17) at (4.25, 3.25) {$s'$};
				\node [style=none] (19) at (1.75, 2.5) {$r'$};
				\node [style=none] (23) at (2.25, 5) {$\eta_R$};
				\node [style=none] (24) at (2.25, 5.5) {$\eta_R$};
				\node [style=none] (25) at (3.5, 1.25) {$\epsilon_R$};
				\node [style=none] (26) at (4.5, 5) {$E'$};
				\node [style=none] (27) at (0.75, 2) {$E'$};
				\node [style=none] (28) at (2, 2) {$E'$};
				\node [style=none] (29) at (0.5, 4.25) {$B$};
				\node [style=none] (30) at (4.5, 2) {$B$};
				\node [style=onehalfcircle] (31) at (1, 3.25) {};
				\node [style=onehalfcircle] (32) at (1.75, 3.25) {};
				\node [style=none] (35) at (1, 3.25) {$e_B$};
				\node [style=none] (36) at (1.75, 3.25) {$e_B$};
				\node [style=onehalfcircle] (37) at (1, 2.5) {};
				\node [style=none] (18) at (1, 2.5) {$r'$};
			\end{pgfonlayer}
			\begin{pgfonlayer}{edgelayer}
				\draw [bend left=270] (6.center) to (8.center);
				\draw [bend left=270] (7.center) to (9.center);
				\draw [bend right=90, looseness=0.75] (4.center) to (5.center);
				\draw (5.center) to (16);
				\draw (16) to (13.center);
				\draw (10.center) to (15);
				\draw (4.center) to (0);
				\draw (15) to (32);
				\draw (32) to (8.center);
				\draw (31) to (9.center);
				\draw [in=-90, out=45, looseness=1.25] (0) to (7.center);
				\draw [in=-90, out=135] (0) to (6.center);
				\draw (11.center) to (37);
				\draw (37) to (31);
			\end{pgfonlayer}
		\end{tikzpicture}
		 \stackrel{(3)}{=} \begin{tikzpicture}
			\begin{pgfonlayer}{nodelayer}
				\node [style=circle] (0) at (2, 3) {};
				\node [style=none] (4) at (2, 1.75) {};
				\node [style=none] (5) at (0.75, 1.75) {};
				\node [style=none] (6) at (2.5, 4.5) {};
				\node [style=none] (7) at (1.5, 4.5) {};
				\node [style=none] (8) at (3.25, 4.5) {};
				\node [style=none] (9) at (4, 4.5) {};
				\node [style=none] (10) at (3.25, 1.5) {};
				\node [style=none] (11) at (4, 1.5) {};
				\node [style=none] (13) at (0.75, 5.5) {};
				\node [style=onehalfcircle] (14) at (4, 2.5) {};
				\node [style=onehalfcircle] (15) at (3.25, 2.5) {};
				\node [style=onehalfcircle] (16) at (0.75, 3.25) {};
				\node [style=none] (17) at (0.75, 3.25) {$s'$};
				\node [style=none] (18) at (4, 2.5) {$r'$};
				\node [style=none] (19) at (3.25, 2.5) {$r'$};
				\node [style=none] (23) at (2.75, 5) {$\eta_L$};
				\node [style=none] (24) at (2.75, 5.5) {$\eta_L$};
				\node [style=none] (25) at (1.5, 1.25) {$\epsilon_L$};
				\node [style=none] (26) at (0.5, 5) {$E'$};
				\node [style=none] (27) at (4.25, 2) {$E'$};
				\node [style=none] (28) at (3, 2) {$E'$};
				\node [style=none] (29) at (4.5, 4.25) {$B$};
				\node [style=none] (30) at (0.5, 2) {$B$};
				\node [style=onehalfcircle] (31) at (4, 3.25) {};
				\node [style=onehalfcircle] (32) at (3.25, 3.25) {};
				\node [style=none] (35) at (4, 3.25) {$e_B$};
				\node [style=none] (36) at (3.25, 3.25) {$e_B$};
			\end{pgfonlayer}
			\begin{pgfonlayer}{edgelayer}
				\draw [bend right=270] (6.center) to (8.center);
				\draw [bend right=270] (7.center) to (9.center);
				\draw [bend left=90, looseness=0.75] (4.center) to (5.center);
				\draw (5.center) to (16);
				\draw (16) to (13.center);
				\draw (10.center) to (15);
				\draw (11.center) to (14.center);
				\draw (4.center) to (0);
				\draw (14.center) to (31);
				\draw (15) to (32);
				\draw (32) to (8.center);
				\draw (31) to (9.center);
				\draw [in=-90, out=135, looseness=1.25] (0) to (7.center);
				\draw [in=-90, out=45] (0) to (6.center);
			\end{pgfonlayer}
		\end{tikzpicture} 				
	 \stackrel{(4)}{=} \begin{tikzpicture}
		\begin{pgfonlayer}{nodelayer}
			\node [style=circle] (0) at (2, 3) {};
			\node [style=none] (4) at (2, 1.75) {};
			\node [style=none] (5) at (0.75, 1.75) {};
			\node [style=none] (6) at (2.5, 4.5) {};
			\node [style=none] (7) at (1.5, 4.5) {};
			\node [style=none] (8) at (3.25, 4.5) {};
			\node [style=none] (9) at (4, 4.5) {};
			\node [style=none] (10) at (3.25, 1.5) {};
			\node [style=none] (11) at (4, 1.5) {};
			\node [style=none] (13) at (0.75, 5.5) {};
			\node [style=onehalfcircle] (14) at (4, 2.5) {};
			\node [style=onehalfcircle] (15) at (3.25, 2.5) {};
			\node [style=onehalfcircle] (16) at (0.75, 3.25) {};
			\node [style=none] (17) at (0.75, 3.25) {$s'$};
			\node [style=none] (18) at (4, 2.5) {$r'$};
			\node [style=none] (19) at (3.25, 2.5) {$r'$};
			\node [style=none] (23) at (2.75, 5) {$\eta_L$};
			\node [style=none] (24) at (2.75, 5.5) {$\eta_L$};
			\node [style=none] (25) at (1.5, 1.25) {$\epsilon_L$};
			\node [style=none] (26) at (0.5, 5) {$E'$};
			\node [style=none] (27) at (4.25, 2) {$E'$};
			\node [style=none] (28) at (3, 2) {$E'$};
			\node [style=none] (29) at (4.5, 4.25) {$B$};
			\node [style=none] (30) at (0.5, 2) {$B$};
			\node [style=onehalfcircle] (31) at (4, 3.25) {};
			\node [style=onehalfcircle] (32) at (3.25, 3.25) {};
			\node [style=none] (35) at (4, 3.25) {$e_B$};
			\node [style=none] (36) at (3.25, 3.25) {$e_B$};
			\node [style=onehalfcircle] (37) at (2.5, 3.75) {};
			\node [style=onehalfcircle] (38) at (1.5, 3.75) {};
			\node [style=none] (39) at (2.5, 3.75) {$e_A$};
			\node [style=none] (40) at (1.5, 3.75) {$e_A$};
		\end{pgfonlayer}
		\begin{pgfonlayer}{edgelayer}
			\draw [bend right=270] (6.center) to (8.center);
			\draw [bend right=270] (7.center) to (9.center);
			\draw [bend left=90, looseness=0.75] (4.center) to (5.center);
			\draw (5.center) to (16);
			\draw (16) to (13.center);
			\draw (10.center) to (15);
			\draw (11.center) to (14);
			\draw (4.center) to (0);
			\draw (14) to (31);
			\draw (15) to (32);
			\draw (32) to (8.center);
			\draw (31) to (9.center);
			\draw [in=-90, out=30] (0) to (37);
			\draw [in=-90, out=150] (0) to (38);
			\draw (37) to (6.center);
			\draw (38) to (7.center);
		\end{pgfonlayer}
	\end{tikzpicture} = \begin{tikzpicture}
		\begin{pgfonlayer}{nodelayer}
			\node [style=circle] (0) at (2.25, 3) {};
			\node [style=onehalfcircle] (1) at (2.75, 3.75) {};
			\node [style=onehalfcircle] (2) at (1.75, 3.75) {};
			\node [style=onehalfcircle] (3) at (2.25, 2.25) {};
			\node [style=none] (4) at (2.25, 1.75) {};
			\node [style=none] (5) at (1, 1.75) {};
			\node [style=none] (6) at (2.75, 4.5) {};
			\node [style=none] (7) at (1.75, 4.5) {};
			\node [style=none] (8) at (3.5, 4.5) {};
			\node [style=none] (9) at (4.25, 4.5) {};
			\node [style=none] (10) at (3.5, 1.5) {};
			\node [style=none] (11) at (4.25, 1.5) {};
			\node [style=none] (13) at (1, 5.5) {};
			\node [style=onehalfcircle] (14) at (4.25, 3.25) {};
			\node [style=onehalfcircle] (15) at (3.5, 3.25) {};
			\node [style=onehalfcircle] (16) at (1, 3.25) {};
			\node [style=none] (17) at (1, 3.25) {$s'$};
			\node [style=none] (18) at (4.25, 3.25) {$r$};
			\node [style=none] (19) at (3.5, 3.25) {$r$};
			\node [style=none] (20) at (2.75, 3.75) {$e_A$};
			\node [style=none] (21) at (2.25, 2.25) {$e_A$};
			\node [style=none] (22) at (1.75, 3.75) {$e_A$};
			\node [style=none] (23) at (3, 5) {$\eta_L$};
			\node [style=none] (24) at (3, 5.5) {$\eta_L$};
			\node [style=none] (25) at (1.75, 1.25) {$\epsilon_L$};
			\node [style=none] (26) at (0.75, 5) {$E'$};
			\node [style=none] (27) at (4.5, 2) {$E'$};
			\node [style=none] (28) at (3.25, 2) {$E'$};
			\node [style=none] (29) at (4.5, 4) {$B$};
			\node [style=none] (30) at (0.75, 2) {$B$};
		\end{pgfonlayer}
		\begin{pgfonlayer}{edgelayer}
			\draw [bend left] (0) to (2);
			\draw [bend right, looseness=1.25] (0) to (1);
			\draw (1) to (6.center);
			\draw [bend right=270] (6.center) to (8.center);
			\draw (2) to (7.center);
			\draw [bend right=270] (7.center) to (9.center);
			\draw [bend left=90, looseness=0.75] (4.center) to (5.center);
			\draw (5.center) to (16);
			\draw (16) to (13.center);
			\draw (10.center) to (15);
			\draw (15) to (8.center);
			\draw (11.center) to (14);
			\draw (14) to (9.center);
			\draw (4.center) to (3);
			\draw (3) to (0);
		\end{pgfonlayer}
	\end{tikzpicture} = \begin{tikzpicture}
		\begin{pgfonlayer}{nodelayer}
			\node [style=black] (0) at (2.25, 3) {};
			\node [style=none] (4) at (2.25, 2.25) {};
			\node [style=none] (5) at (1, 2.25) {};
			\node [style=none] (6) at (2.75, 4.5) {};
			\node [style=none] (7) at (1.75, 4.5) {};
			\node [style=none] (8) at (3.5, 4.5) {};
			\node [style=none] (9) at (4.25, 4.5) {};
			\node [style=none] (10) at (3.5, 1.5) {};
			\node [style=none] (11) at (4.25, 1.5) {};
			\node [style=none] (13) at (1, 5.5) {};
			\node [style=none] (23) at (3, 5) {$\eta_L'$};
			\node [style=none] (24) at (3, 5.5) {$\eta_L'$};
			\node [style=none] (25) at (1.75, 1.75) {$\epsilon_L'$};
			\node [style=none] (26) at (0.75, 5) {$E'$};
			\node [style=none] (27) at (4.5, 2) {$E'$};
			\node [style=none] (28) at (3.25, 2) {$E'$};
		\end{pgfonlayer}
		\begin{pgfonlayer}{edgelayer}
			\draw [bend right=270] (6.center) to (8.center);
			\draw [bend right=270] (7.center) to (9.center);
			\draw [bend left=90, looseness=0.75] (4.center) to (5.center);
			\draw (5.center) to (13.center);
			\draw [in=-90, out=135] (0) to (7.center);
			\draw [in=-90, out=45] (0) to (6.center);
			\draw (10.center) to (8.center);
			\draw (11.center) to (9.center);
			\draw (4.center) to (0);
		\end{pgfonlayer}
	\end{tikzpicture} $
	
	\medskip

	where $(1)$ is true because $e_A$ is section on $(A, m, u)$, $(2)$ because $(e_B, e_A)$ is 
	retractional on $(\eta_R, \epsilon_R): B \dashvv A$, $(3)$ holds because $A$ is a linear monoid, and  
	$(4)$ holds because $(e_A, e_B)$ is sectional on $(\eta_L, \epsilon_L): A \dashvv B$. 
	
	The converse of the statement is straightforward from Lemma \ref{Lemma: sectional dual morphism}, and 
	Lemma \ref{Lemma: sectional monoid morphism}.
\end{proof}
As a consequence of the previous Lemma, a split binary idempotent is sectional (respectively retractional) 
if and only its section (respectively retraction) is a morphism from (respectively to) the self-linear monoid  
given on the splitting.

The following Lemma extends Lemma \ref{Lemma: split weak linear mon hom} to 
sectional/retractional $\dagger$-binary idempotents on $\dagger$-linear monoids:
\begin{lemma}
	\label{Lemma: weak dag lin mon hom}
	In a $\dag$-LDC, let $(e, e^\dag)$ be a pair of idempotents 
	on a $\dag$-linear monoid  $A \linmonw A^\dag$ with 
	splitting $A \to^r E \to^s A$. 
	The idempotent pair $(e, e^\dag)$ is sectional (respectively retractional) if and only if the 
	section $(s, s^\dag)$ (respectively the retraction $(r, r^\dag)$) is a morphism of $\dag$-linear monoids. 
\end{lemma}
\begin{proof} 
Suppose an idempotent pair $(e,e^\dagger)$ is a sectional on a $\dagger$-linear monoid 
$A \dagmonw A^\dagger$. Moreover, the idempotent $e$ splits as follows: 
\[ e = A \to^{r} E \to^{s} A \] 
We must show that $E$ is a $\dagger$-linear monoid.
From Lemma \ref{Lemma: split weak linear mon hom}, we know that $E \linmonw E^\dag$ is a linear monoid. It is also a 
$\dagger$-linear monoid because: 
\[  \begin{tikzpicture}
	\begin{pgfonlayer}{nodelayer}
		\node [style=black] (0) at (3, 3) {};
		\node [style=none] (4) at (3, 2.25) {};
		\node [style=none] (5) at (4.25, 2.25) {};
		\node [style=none] (6) at (2.5, 4.5) {};
		\node [style=none] (7) at (3.5, 4.5) {};
		\node [style=none] (8) at (1.75, 4.5) {};
		\node [style=none] (9) at (1, 4.5) {};
		\node [style=none] (10) at (1.75, 1.5) {};
		\node [style=none] (11) at (1, 1.5) {};
		\node [style=none] (13) at (4.25, 5.5) {};
		\node [style=none] (23) at (2.25, 5) {$\eta_R'$};
		\node [style=none] (24) at (2.25, 5.5) {$\eta_R'$};
		\node [style=none] (25) at (3.5, 1.75) {$\epsilon_R'$};
		\node [style=none] (26) at (4.5, 5) {$E'$};
		\node [style=none] (27) at (0.75, 2) {$E'$};
		\node [style=none] (28) at (2, 2) {$E'$};
	\end{pgfonlayer}
	\begin{pgfonlayer}{edgelayer}
		\draw [bend left=270] (6.center) to (8.center);
		\draw [bend left=270] (7.center) to (9.center);
		\draw [bend right=90, looseness=0.75] (4.center) to (5.center);
		\draw (5.center) to (13.center);
		\draw [in=-90, out=45] (0) to (7.center);
		\draw [in=-90, out=135] (0) to (6.center);
		\draw (10.center) to (8.center);
		\draw (11.center) to (9.center);
		\draw (4.center) to (0);
	\end{pgfonlayer}
\end{tikzpicture} = \begin{tikzpicture}
	\begin{pgfonlayer}{nodelayer}
		\node [style=circle] (0) at (3, 3) {};
		\node [style=none] (4) at (3, 1.75) {};
		\node [style=none] (5) at (4.25, 1.75) {};
		\node [style=none] (6) at (2.5, 4.5) {};
		\node [style=none] (7) at (3.5, 4.5) {};
		\node [style=none] (8) at (1.75, 4.5) {};
		\node [style=none] (9) at (1, 4.5) {};
		\node [style=none] (10) at (1.75, 1.5) {};
		\node [style=none] (11) at (1, 1.5) {};
		\node [style=none] (13) at (4.25, 5.5) {};
		\node [style=onehalfcircle] (15) at (1.75, 2.5) {};
		\node [style=onehalfcircle] (16) at (4.25, 3.25) {};
		\node [style=none] (17) at (4.25, 3.25) {$r^\dagger$};
		\node [style=none] (19) at (1.75, 2.5) {$s^\dagger$};
		\node [style=none] (23) at (2.25, 5) {$\eta_R$};
		\node [style=none] (24) at (2.25, 5.5) {$\eta_R$};
		\node [style=none] (25) at (3.5, 1.25) {$\epsilon_R$};
		\node [style=none] (26) at (4.5, 5) {$E^\dag$};
		\node [style=none] (27) at (0.75, 2) {$E^\dagger$};
		\node [style=none] (28) at (2, 2) {$E'$};
		\node [style=none] (29) at (0.5, 4.25) {$A^\dag$};
		\node [style=none] (30) at (4.5, 2) {$A^\dag$};
		\node [style=circle, scale=2] (31) at (1, 3.25) {};
		\node [style=circle, scale=2] (32) at (1.75, 3.25) {};
		\node [style=none] (35) at (1, 3.25) {$e_{A^\dag}$};
		\node [style=none] (36) at (1.75, 3.25) {$e_{A^\dag}$};
		\node [style=onehalfcircle] (37) at (1, 2.5) {};
		\node [style=none] (18) at (1, 2.5) {$s^\dagger$};
	\end{pgfonlayer}
	\begin{pgfonlayer}{edgelayer}
		\draw [bend left=270] (6.center) to (8.center);
		\draw [bend left=270] (7.center) to (9.center);
		\draw [bend right=90, looseness=0.75] (4.center) to (5.center);
		\draw (5.center) to (16);
		\draw (16) to (13.center);
		\draw (10.center) to (15);
		\draw (4.center) to (0);
		\draw (15) to (32);
		\draw (32) to (8.center);
		\draw (31) to (9.center);
		\draw [in=-90, out=45, looseness=1.25] (0) to (7.center);
		\draw [in=-90, out=135] (0) to (6.center);
		\draw (11.center) to (37);
		\draw (37) to (31);
	\end{pgfonlayer}
\end{tikzpicture} = \begin{tikzpicture}
	\begin{pgfonlayer}{nodelayer}
		\node [style=circle] (0) at (3, 3) {};
		\node [style=none] (4) at (3, 1.75) {};
		\node [style=none] (5) at (4.25, 1.75) {};
		\node [style=none] (6) at (2.5, 4.5) {};
		\node [style=none] (7) at (3.5, 4.5) {};
		\node [style=none] (8) at (1.75, 4.5) {};
		\node [style=none] (9) at (1, 4.5) {};
		\node [style=none] (10) at (1.75, 1.5) {};
		\node [style=none] (11) at (1, 1.5) {};
		\node [style=none] (13) at (4.25, 5.5) {};
		\node [style=onehalfcircle] (15) at (1.75, 2.5) {};
		\node [style=onehalfcircle] (16) at (4.25, 3.25) {};
		\node [style=none] (17) at (4.25, 3.25) {$r^\dagger$};
		\node [style=none] (19) at (1.75, 2.5) {$s^\dagger$};
		\node [style=none] (23) at (2.25, 5) {$\eta_R$};
		\node [style=none] (24) at (2.25, 5.5) {$\eta_R$};
		\node [style=none] (25) at (3.5, 1.25) {$\epsilon_R$};
		\node [style=none] (26) at (4.5, 5) {$E^\dag$};
		\node [style=none] (27) at (0.75, 2) {$E^\dagger$};
		\node [style=none] (28) at (2, 2) {$E'$};
		\node [style=none] (29) at (0.5, 4.25) {$A^\dag$};
		\node [style=none] (30) at (4.5, 2) {$A^\dag$};
		\node [style=onehalfcircle] (37) at (1, 2.5) {};
		\node [style=none] (18) at (1, 2.5) {$s^\dagger$};
	\end{pgfonlayer}
	\begin{pgfonlayer}{edgelayer}
		\draw [bend left=270] (6.center) to (8.center);
		\draw [bend left=270] (7.center) to (9.center);
		\draw [bend right=90, looseness=0.75] (4.center) to (5.center);
		\draw (5.center) to (16);
		\draw (16) to (13.center);
		\draw (10.center) to (15);
		\draw (4.center) to (0);
		\draw [in=-90, out=45, looseness=1.25] (0) to (7.center);
		\draw [in=-90, out=135] (0) to (6.center);
		\draw (11.center) to (37);
		\draw (37) to (9.center);
		\draw (15) to (8.center);
	\end{pgfonlayer}
\end{tikzpicture} = \begin{tikzpicture}
	\begin{pgfonlayer}{nodelayer}
		\node [style=circle] (0) at (1.5, 3.75) {};
		\node [style=none] (4) at (1.5, 3) {};
		\node [style=none] (5) at (1.5, 4.5) {};
		\node [style=none] (6) at (1, 4.5) {};
		\node [style=none] (7) at (2, 4.5) {};
		\node [style=none] (8) at (2, 3) {};
		\node [style=none] (9) at (1, 3) {};
		\node [style=none] (10) at (2, 2) {};
		\node [style=none] (11) at (1, 2) {};
		\node [style=none] (13) at (1.5, 6) {};
		\node [style=onehalfcircle] (15) at (2, 2.5) {};
		\node [style=onehalfcircle] (16) at (1.5, 5.25) {};
		\node [style=none] (17) at (1.5, 5.25) {$r^\dagger$};
		\node [style=none] (19) at (2, 2.5) {$s^\dagger$};
		\node [style=none] (26) at (2, 5.75) {$E^\dag$};
		\node [style=none] (27) at (0.5, 2) {$E^\dagger$};
		\node [style=none] (29) at (0.5, 2.75) {$A^\dag$};
		\node [style=none] (30) at (2, 4.75) {$A^\dag$};
		\node [style=onehalfcircle] (37) at (1, 2.5) {};
		\node [style=none] (18) at (1, 2.5) {$s^\dagger$};
		\node [style=none] (38) at (0.75, 4.5) {};
		\node [style=none] (39) at (2.25, 4.5) {};
		\node [style=none] (40) at (2.25, 3) {};
		\node [style=none] (41) at (0.75, 3) {};
	\end{pgfonlayer}
	\begin{pgfonlayer}{edgelayer}
		\draw (5.center) to (16);
		\draw (16) to (13.center);
		\draw (10.center) to (15);
		\draw (4.center) to (0);
		\draw [in=-90, out=30] (0) to (7.center);
		\draw [in=-90, out=150] (0) to (6.center);
		\draw (11.center) to (37);
		\draw (37) to (9.center);
		\draw (15) to (8.center);
		\draw (41.center) to (40.center);
		\draw (40.center) to (39.center);
		\draw (39.center) to (38.center);
		\draw (38.center) to (41.center);
	\end{pgfonlayer}
\end{tikzpicture} = \begin{tikzpicture}
	\begin{pgfonlayer}{nodelayer}
		\node [style=circle] (0) at (1.5, 4) {};
		\node [style=none] (4) at (1.5, 2.75) {};
		\node [style=none] (5) at (1.5, 5.25) {};
		\node [style=none] (6) at (1, 5.25) {};
		\node [style=none] (7) at (2, 5.25) {};
		\node [style=none] (8) at (2, 2.75) {};
		\node [style=none] (9) at (1, 2.75) {};
		\node [style=none] (10) at (2, 2) {};
		\node [style=none] (11) at (1, 2) {};
		\node [style=none] (13) at (1.5, 6) {};
		\node [style=none] (26) at (2, 5.75) {$E^\dag$};
		\node [style=none] (27) at (0.5, 2) {$E^\dagger$};
		\node [style=none] (38) at (0.5, 5.25) {};
		\node [style=none] (39) at (2.5, 5.25) {};
		\node [style=none] (40) at (2.5, 2.75) {};
		\node [style=none] (41) at (0.5, 2.75) {};
		\node [style=onehalfcircle] (42) at (2, 4.75) {};
		\node [style=onehalfcircle] (43) at (1, 4.75) {};
		\node [style=onehalfcircle] (44) at (1.5, 3.25) {};
		\node [style=none] (45) at (1.5, 3.25) {$r$};
		\node [style=none] (46) at (2, 4.75) {$s$};
		\node [style=none] (47) at (1, 4.75) {$s$};
	\end{pgfonlayer}
	\begin{pgfonlayer}{edgelayer}
		\draw (41.center) to (40.center);
		\draw (40.center) to (39.center);
		\draw (39.center) to (38.center);
		\draw (38.center) to (41.center);
		\draw (10.center) to (8.center);
		\draw (11.center) to (9.center);
		\draw (4.center) to (44);
		\draw (44) to (0);
		\draw (5.center) to (13.center);
		\draw [in=-90, out=30] (0) to (42);
		\draw (42) to (7.center);
		\draw (6.center) to (43);
		\draw [in=150, out=-90] (43) to (0);
	\end{pgfonlayer}
\end{tikzpicture} = \begin{tikzpicture}
	\begin{pgfonlayer}{nodelayer}
		\node [style=black] (0) at (1.5, 4) {};
		\node [style=none] (4) at (1.5, 2.75) {};
		\node [style=none] (5) at (1.5, 5.25) {};
		\node [style=none] (6) at (1, 5.25) {};
		\node [style=none] (7) at (2, 5.25) {};
		\node [style=none] (8) at (2, 2.75) {};
		\node [style=none] (9) at (1, 2.75) {};
		\node [style=none] (10) at (2, 2) {};
		\node [style=none] (11) at (1, 2) {};
		\node [style=none] (13) at (1.5, 6) {};
		\node [style=none] (26) at (2, 5.75) {$E^\dag$};
		\node [style=none] (27) at (0.5, 2) {$E^\dagger$};
		\node [style=none] (38) at (0.5, 5.25) {};
		\node [style=none] (39) at (2.5, 5.25) {};
		\node [style=none] (40) at (2.5, 2.75) {};
		\node [style=none] (41) at (0.5, 2.75) {};
	\end{pgfonlayer}
	\begin{pgfonlayer}{edgelayer}
		\draw (41.center) to (40.center);
		\draw (40.center) to (39.center);
		\draw (39.center) to (38.center);
		\draw (38.center) to (41.center);
		\draw (10.center) to (8.center);
		\draw (11.center) to (9.center);
		\draw (5.center) to (13.center);
		\draw (4.center) to (0);
		\draw [in=-90, out=30] (0) to (7.center);
		\draw [in=-90, out=150] (0) to (6.center);
	\end{pgfonlayer}
\end{tikzpicture} \]
The statements are proven similarly for retractional binary idempotent.  The converse follows directly from 
Lemma \ref{Lemma: split weak linear mon hom}.
\end{proof}

As a consequence of the above Lemma, a split $\dagger$-binary idempotent is sectional 
(respectively retractional) on a $\dagger$-linear monoid if and only if its section (respectively 
retraction) is a morphism from (respectiely to) the $\dagger$-self-linear monoid given 
on the splitting.


From Lemma \ref{Lemma: split weak linear mon hom}, it follows that splitting a sectional or retractional binary idempotent 
produces a self-linear monoid. If the binary idempotent is also coring and satisfies equation \ref{eqn: dag Frob split}, 
then it produces a Frobenius Algebra on the splitting:
\begin{lemma}
	\label{Lemma: Frobenius splitting}
	In an isomix category $\X$, let  $E \linmonb E'$ be a self-linear monoid in $\Core(\X)$ given by splitting a coring sectional/retractional 
	binary idempotent $(\u, \v)$ on linear monoid $A \linmonw B$. Let $\alpha: E \to E'$ be 
	the isomorphism. Then, $E$ is a Frobenius Algebra under the linear equivalence ${\sf Mx}_\downarrow$
	 if and only if the binary idempotent satisfies the following equation where $e_A = \u \v$ and $e_B = \v \u$: 
	\begin{equation}
		\label{eqn: dag Frob split}
		\begin{tikzpicture}
			\begin{pgfonlayer}{nodelayer}
				\node [style=onehalfcircle] (44) at (6.25, 3) {};
				\node [style=none] (45) at (6.25, 0.75) {};
				\node [style=none] (46) at (6.25, 3) {$\u$};
				\node [style=none] (47) at (6.25, 4.75) {};
				\node [style=none] (48) at (6.5, 4.5) {$A$};
				\node [style=none] (49) at (6.5, 1) {$B$};
			\end{pgfonlayer}
			\begin{pgfonlayer}{edgelayer}
				\draw (45.center) to (44);
				\draw (47.center) to (44);
			\end{pgfonlayer}
		\end{tikzpicture} = \begin{tikzpicture}
			\begin{pgfonlayer}{nodelayer}
				\node [style=onehalfcircle] (37) at (3, 3.5) {};
				\node [style=none] (20) at (4, 4) {};
				\node [style=none] (22) at (5, 4) {};
				\node [style=none] (23) at (5, 0.75) {};
				\node [style=onehalfcircle] (24) at (3.5, 1.75) {};
				\node [style=none] (25) at (3.5, 1.75) {$\u$};
				\node [style=none] (26) at (3.25, 1.25) {$B$};
				\node [style=none] (27) at (2.75, 4.25) {$E$};
				\node [style=none] (28) at (5.25, 1.25) {$B$};
				\node [style=circle] (29) at (3.5, 2.5) {};
				\node [style=circle] (30) at (3.5, 0.75) {};
				\node [style=none] (31) at (4.5, 4.75) {$\eta_L$};
				\node [style=none] (36) at (3, 4.5) {};
				\node [style=none] (33) at (3, 3.5) {$e_A$};
				\node [style=onehalfcircle] (38) at (5, 3) {};
				\node [style=none] (39) at (5, 3) {$e_B$};
				\node [style=onehalfcircle] (40) at (4, 3.5) {};
				\node [style=none] (41) at (4, 3.5) {$e_A$};
				\node [style=none] (42) at (4, 3) {};
				\node [style=none] (43) at (3, 3) {};
			\end{pgfonlayer}
			\begin{pgfonlayer}{edgelayer}
				\draw [bend right=90, looseness=1.50] (22.center) to (20.center);
				\draw (37) to (36.center);
				\draw (38) to (22.center);
				\draw (20.center) to (40);
				\draw [in=-90, out=15, looseness=1.25] (29) to (42.center);
				\draw [in=165, out=-90, looseness=1.25] (43.center) to (29);
				\draw (38) to (23.center);
				\draw (40) to (42.center);
				\draw (37) to (43.center);
				\draw (30) to (24);
				\draw (24) to (29);
			\end{pgfonlayer}
		\end{tikzpicture} = \begin{tikzpicture}
		\begin{pgfonlayer}{nodelayer}
			\node [style=onehalfcircle] (37) at (5, 3.5) {};
			\node [style=none] (20) at (4, 4) {};
			\node [style=none] (22) at (3, 4) {};
			\node [style=none] (23) at (3, 0.75) {};
			\node [style=onehalfcircle] (24) at (4.5, 1.75) {};
			\node [style=none] (25) at (4.5, 1.75) {$\u$};
			\node [style=none] (26) at (4.75, 1.25) {$B$};
			\node [style=none] (27) at (5.25, 4.25) {$E$};
			\node [style=none] (28) at (2.75, 1.25) {$B$};
			\node [style=circle] (29) at (4.5, 2.5) {};
			\node [style=circle] (30) at (4.5, 0.75) {};
			\node [style=none] (31) at (3.5, 4.75) {$\eta_R$};
			\node [style=none] (36) at (5, 4.5) {};
			\node [style=none] (33) at (5, 3.5) {$e_A$};
			\node [style=onehalfcircle] (38) at (3, 3) {};
			\node [style=none] (39) at (3, 3) {$e_B$};
			\node [style=onehalfcircle] (40) at (4, 3.5) {};
			\node [style=none] (41) at (4, 3.5) {$e_A$};
			\node [style=none] (42) at (4, 3) {};
			\node [style=none] (43) at (5, 3) {};
		\end{pgfonlayer}
		\begin{pgfonlayer}{edgelayer}
			\draw [bend left=90, looseness=1.50] (22.center) to (20.center);
			\draw (37) to (36.center);
			\draw (38) to (22.center);
			\draw (20.center) to (40);
			\draw [in=-90, out=165, looseness=1.25] (29) to (42.center);
			\draw [in=15, out=-90, looseness=1.25] (43.center) to (29);
			\draw (38) to (23.center);
			\draw (40) to (42.center);
			\draw (37) to (43.center);
			\draw (30) to (24);
			\draw (24) to (29);
		\end{pgfonlayer}
	\end{tikzpicture}		
	\end{equation}
	\end{lemma}
	\begin{proof} 
		Let $E \linmonb E'$ be a self-linear monoid given by 
		splitting a sectional or retractional binary idempotent on $A \linmonw B$.
		 Let the splitting of the binary idempotent be $e_A = \u \v = A \to^r E \to^s A$, and 
		$e_B = \v \u = B \to^{r'} E' \to^{s'} B$. Suppose $\u$ satisfies  the given equation. If we show that the following 
		equation holds, then by Lemma \ref{Lemma: dagmondagFrob} $E$ is a Frobenius algebra:
		\[ \begin{tikzpicture}
			\begin{pgfonlayer}{nodelayer}
				\node [style=onehalfcircle] (0) at (-0.75, 1.5) {};
				\node [style=none] (1) at (-0.75, 0) {};
				\node [style=none] (2) at (-0.75, 1.5) {$\alpha$};
				\node [style=none] (3) at (-0.75, 2.75) {};
				\node [style=none] (4) at (-0.5, 2.5) {$E$};
				\node [style=none] (5) at (-0.5, 0.75) {$E'$};
			\end{pgfonlayer}
			\begin{pgfonlayer}{edgelayer}
				\draw (1.center) to (0);
				\draw (3.center) to (0);
			\end{pgfonlayer}
		\end{tikzpicture} =  \begin{tikzpicture}
			\begin{pgfonlayer}{nodelayer}
				\node [style=none] (0) at (0, 0.5) {};
				\node [style=none] (1) at (-1.5, 1) {};
				\node [style=none] (2) at (1, 0.5) {};
				\node [style=none] (3) at (1, -1.75) {};
				\node [style=onehalfcircle] (4) at (-0.75, -1) {};
				\node [style=none] (5) at (-0.75, -1) {$\alpha$};
				\node [style=none] (6) at (-1.4, -1.5) {$E'$};
				\node [style=none] (7) at (-1.8, 0.75) {$E$};
				\node [style=none] (8) at (1.4, -1.35) {$E'$};
				\node [style=black] (9) at (-0.75, -0.25) {};
				\node [style=black] (10) at (-0.75, -1.75) {};
				\node [style=none] (11) at (0.5, 1.25) {$\eta_L'$};
			\end{pgfonlayer}
			\begin{pgfonlayer}{edgelayer}
				\draw [bend right=90, looseness=1.50] (2.center) to (0.center);
				\draw (3.center) to (2.center);
				\draw [in=15, out=-90, looseness=1.25] (0.center) to (9);
				\draw [in=-90, out=165, looseness=1.25] (9) to (1.center);
				\draw (10) to (4);
				\draw (9) to (4);
			\end{pgfonlayer}
		\end{tikzpicture}  = \begin{tikzpicture}
			\begin{pgfonlayer}{nodelayer}
				\node [style=none] (0) at (-0.5, 0.25) {};
				\node [style=none] (1) at (1, 1) {};
				\node [style=none] (2) at (-1.5, 0.25) {};
				\node [style=none] (3) at (-1.5, -1.75) {};
				\node [style=onehalfcircle] (4) at (0.25, -1) {};
				\node [style=none] (5) at (0.25, -1) {$\alpha$};
				\node [style=none] (6) at (0.8, -1.5) {$E'$};
				\node [style=none] (7) at (1.25, 0.75) {$E$};
				\node [style=none] (8) at (-1.85, -1.25) {$E'$};
				\node [style=black] (9) at (0.25, -0.25) {};
				\node [style=black] (10) at (0.25, -1.75) {};
				\node [style=none] (11) at (-1, 1) {$\eta_R'$};
			\end{pgfonlayer}
			\begin{pgfonlayer}{edgelayer}
				\draw (3.center) to (2.center);
				\draw [in=180, out=-90, looseness=1.00] (0.center) to (9);
				\draw [in=-90, out=15, looseness=1.25] (9) to (1.center);
				\draw (10) to (4);
				\draw (9) to (4);
				\draw [bend left=90, looseness=1.75] (2.center) to (0.center);
			\end{pgfonlayer}
		\end{tikzpicture} \]
	The proof is as follows:
	\[ 	\begin{tikzpicture}
		\begin{pgfonlayer}{nodelayer}
			\node [style=none] (31) at (3.8, 0.5) {};
			\node [style=none] (32) at (2.3, 1) {};
			\node [style=none] (33) at (4.8, 0.5) {};
			\node [style=none] (34) at (4.8, -3.75) {};
			\node [style=onehalfcircle] (35) at (3.05, -2.5) {};
			\node [style=none] (36) at (3.05, -2.5) {$\alpha$};
			\node [style=none] (37) at (2.65, -3.25) {$E'$};
			\node [style=none] (38) at (2, 0.75) {$E$};
			\node [style=none] (39) at (5.2, -1.35) {$E'$};
			\node [style=black] (40) at (3.05, -1.25) {};
			\node [style=black] (41) at (3.05, -3.75) {};
			\node [style=none] (42) at (4.3, 1.25) {$\eta_L'$};
		\end{pgfonlayer}
		\begin{pgfonlayer}{edgelayer}
			\draw [bend right=90, looseness=1.50] (33.center) to (31.center);
			\draw (34.center) to (33.center);
			\draw [in=30, out=-90] (31.center) to (40);
			\draw [in=-90, out=150] (40) to (32.center);
			\draw (41) to (35);
			\draw (40) to (35);
		\end{pgfonlayer}
	\end{tikzpicture}   = \begin{tikzpicture}
			\begin{pgfonlayer}{nodelayer}
				\node [style=none] (0) at (-1, 1) {};
				\node [style=onehalfcircle] (1) at (-1, -0.5) {};
				\node [style=onehalfcircle] (2) at (0, 0.25) {};
				\node [style=onehalfcircle] (3) at (0, -0.5) {};
				\node [style=circle] (4) at (-0.5, -1.25) {};
				\node [style=none] (5) at (0, 0.75) {};
				\node [style=none] (6) at (0.75, 0.75) {};
				\node [style=onehalfcircle] (7) at (0.75, -0.5) {};
				\node [style=none] (8) at (0.75, -4) {};
				\node [style=onehalfcircle] (9) at (-0.5, -1.75) {};
				\node [style=onehalfcircle] (10) at (-0.5, -2.5) {};
				\node [style=onehalfcircle] (11) at (-0.5, -3.25) {};
				\node [style=circle] (12) at (-0.5, -4) {};
				\node [style=none] (13) at (-1, -0.5) {$s$};
				\node [style=none] (14) at (0, -0.5) {$s$};
				\node [style=none] (15) at (-0.5, -3.25) {$s'$};
				\node [style=none] (16) at (0, 0.25) {$r$};
				\node [style=none] (17) at (-0.5, -1.75) {$r$};
				\node [style=none] (18) at (0.75, -0.5) {$r'$};
				\node [style=none] (19) at (-1.25, 0.75) {$E$};
				\node [style=none] (20) at (1, -3.75) {$E'$};
				\node [style=none] (21) at (-0.5, -2.5) {$\alpha$};
				\node [style=none] (22) at (-1.25, -1) {$A$};
				\node [style=none] (23) at (-1, -2) {$E$};
				\node [style=none] (24) at (-1, -3) {$E'$};
				\node [style=none] (25) at (-1, -3.75) {$B$};
				\node [style=none] (26) at (1, 0.25) {$B$};
				\node [style=none] (27) at (0.5, 1.25) {$\eta$};
			\end{pgfonlayer}
			\begin{pgfonlayer}{edgelayer}
				\draw [bend right=15] (4) to (3);
				\draw [bend left=15, looseness=1.25] (4) to (1);
				\draw (3) to (2);
				\draw (1) to (0.center);
				\draw (2) to (5.center);
				\draw [bend left=75, looseness=1.25] (5.center) to (6.center);
				\draw (8.center) to (7);
				\draw (7) to (6.center);
				\draw (12) to (11);
				\draw (11) to (10);
				\draw (10) to (9);
				\draw (9) to (4);
			\end{pgfonlayer}
		\end{tikzpicture} =  \begin{tikzpicture}
			\begin{pgfonlayer}{nodelayer}
				\node [style=onehalfcircle] (10) at (-0.5, -2.5) {};
				\node [style=none] (21) at (-0.5, -2.5) {$\u$};
				\node [style=onehalfcircle] (28) at (0.75, -1.25) {};
				\node [style=onehalfcircle] (7) at (0.75, -0.5) {};
				\node [style=none] (18) at (0.75, -0.5) {$r'$};
				\node [style=none] (31) at (0.75, -1.25) {$s'$};
				\node [style=onehalfcircle] (29) at (0.75, -2) {};
				\node [style=none] (30) at (0.75, -2) {$r'$};
				\node [style=none] (0) at (-1, 1) {};
				\node [style=onehalfcircle] (1) at (-1, -0.5) {};
				\node [style=onehalfcircle] (2) at (0, 0.25) {};
				\node [style=onehalfcircle] (3) at (0, -0.5) {};
				\node [style=circle] (4) at (-0.5, -1.25) {};
				\node [style=none] (5) at (0, 0.75) {};
				\node [style=none] (6) at (0.75, 0.75) {};
				\node [style=none] (8) at (0.75, -4) {};
				\node [style=circle] (12) at (-0.5, -4) {};
				\node [style=none] (13) at (-1, -0.5) {$s$};
				\node [style=none] (14) at (0, -0.5) {$s$};
				\node [style=none] (16) at (0, 0.25) {$r$};
				\node [style=none] (19) at (-1.25, 0.75) {$E$};
				\node [style=none] (20) at (1, -3.75) {$E'$};
				\node [style=none] (22) at (-0.75, -1.75) {$A$};
				\node [style=none] (25) at (-0.75, -3.5) {$B$};
				\node [style=none] (26) at (1, 0.25) {$B$};
				\node [style=none] (27) at (0.5, 1.25) {$\eta$};
			\end{pgfonlayer}
			\begin{pgfonlayer}{edgelayer}
				\draw [bend right=15] (4) to (3);
				\draw [bend left=15, looseness=1.25] (4) to (1);
				\draw (3) to (2);
				\draw (1) to (0.center);
				\draw (2) to (5.center);
				\draw [bend left=75, looseness=1.25] (5.center) to (6.center);
				\draw (7) to (6.center);
				\draw (29) to (28);
				\draw (8.center) to (29);
				\draw (28) to (7);
				\draw (12) to (10);
				\draw (10) to (4);
			\end{pgfonlayer}
		\end{tikzpicture}
			 =  \begin{tikzpicture}
				\begin{pgfonlayer}{nodelayer}
					\node [style=onehalfcircle] (10) at (-0.5, -2.5) {};
					\node [style=none] (21) at (-0.5, -2.5) {$\u$};
					\node [style=onehalfcircle] (3) at (0, -0.5) {};
					\node [style=none] (14) at (0, -0.5) {$e_A$};
					\node [style=onehalfcircle] (29) at (0.75, -2) {};
					\node [style=onehalfcircle] (7) at (0.75, -0.5) {};
					\node [style=none] (18) at (0.75, -0.5) {$e_B$};
					\node [style=none] (30) at (0.75, -2) {$r'$};
					\node [style=none] (0) at (-1, 1) {};
					\node [style=onehalfcircle] (1) at (-1, -0.5) {};
					\node [style=circle] (4) at (-0.5, -1.25) {};
					\node [style=none] (5) at (0, 0.75) {};
					\node [style=none] (6) at (0.75, 0.75) {};
					\node [style=none] (8) at (0.75, -4) {};
					\node [style=circle] (12) at (-0.5, -4) {};
					\node [style=none] (13) at (-1, -0.5) {$s$};
					\node [style=none] (19) at (-1.25, 0.75) {$E$};
					\node [style=none] (20) at (1, -3.75) {$E'$};
					\node [style=none] (22) at (-0.75, -2) {$A$};
					\node [style=none] (25) at (-0.75, -3.5) {$B$};
					\node [style=none] (26) at (1, 0.25) {$B$};
					\node [style=none] (27) at (0.5, 1.25) {$\eta$};
				\end{pgfonlayer}
				\begin{pgfonlayer}{edgelayer}
					\draw [bend right=15] (4) to (3);
					\draw [bend left=15, looseness=1.25] (4) to (1);
					\draw (1) to (0.center);
					\draw [bend left=75, looseness=1.25] (5.center) to (6.center);
					\draw (7) to (6.center);
					\draw (8.center) to (29);
					\draw (29) to (7);
					\draw (3) to (5.center);
					\draw (10) to (4);
					\draw (10) to (12);
				\end{pgfonlayer}
			\end{tikzpicture}
			 = \begin{tikzpicture}
				\begin{pgfonlayer}{nodelayer}
					\node [style=onehalfcircle] (10) at (-0.5, -2.5) {};
					\node [style=none] (21) at (-0.5, -2.5) {$u$};
					\node [style=onehalfcircle] (3) at (0, -0.5) {};
					\node [style=none] (14) at (0, -0.5) {$e_A$};
					\node [style=onehalfcircle] (29) at (0.75, -2) {};
					\node [style=onehalfcircle] (7) at (0.75, -0.5) {};
					\node [style=none] (18) at (0.75, -0.5) {$e_B$};
					\node [style=none] (30) at (0.75, -2) {$r'$};
					\node [style=none] (0) at (-1, 1) {};
					\node [style=onehalfcircle] (1) at (-1, -0.5) {};
					\node [style=circle] (4) at (-0.5, -1.25) {};
					\node [style=none] (5) at (0, 0.75) {};
					\node [style=none] (6) at (0.75, 0.75) {};
					\node [style=none] (8) at (0.75, -4) {};
					\node [style=circle] (12) at (-0.5, -4) {};
					\node [style=none] (19) at (-1.25, 0.75) {$E$};
					\node [style=none] (20) at (1, -3.75) {$E'$};
					\node [style=none] (22) at (-1.25, -1) {$A$};
					\node [style=none] (25) at (-1, -3.75) {$B$};
					\node [style=none] (26) at (1, 0.25) {$B$};
					\node [style=none] (27) at (0.5, 1.25) {$\eta$};
					\node [style=onehalfcircle] (31) at (-1, 0.25) {};
					\node [style=none] (32) at (-1, -0.5) {$e_A$};
					\node [style=none] (13) at (-1, 0.25) {$s$};
				\end{pgfonlayer}
				\begin{pgfonlayer}{edgelayer}
					\draw [bend right=15] (4) to (3);
					\draw [bend left=15, looseness=1.25] (4) to (1);
					\draw [bend left=75, looseness=1.25] (5.center) to (6.center);
					\draw (7) to (6.center);
					\draw (8.center) to (29);
					\draw (29) to (7);
					\draw (3) to (5.center);
					\draw (4) to (10);
					\draw (10) to (12);
					\draw (0.center) to (31);
					\draw (31) to (1);
				\end{pgfonlayer}
			\end{tikzpicture} =  \begin{tikzpicture}
			\begin{pgfonlayer}{nodelayer}
				\node [style=onehalfcircle] (29) at (0.75, -2.25) {};
				\node [style=onehalfcircle] (10) at (0.75, -1.25) {};
				\node [style=onehalfcircle] (1) at (0.75, -0.25) {};
				\node [style=none] (30) at (0.75, -2.25) {$r'$};
				\node [style=none] (13) at (0.75, -0.25) {$s$};
				\node [style=none] (21) at (0.75, -1.25) {$\u$};
				\node [style=none] (0) at (0.75, 1) {};
				\node [style=none] (8) at (0.75, -4) {};
				\node [style=none] (19) at (0.5, 0.75) {$E$};
				\node [style=none] (20) at (0.5, -3.5) {$E'$};
				\node [style=none] (22) at (0.5, -0.75) {$A$};
				\node [style=none] (25) at (0.5, -1.75) {$B$};
			\end{pgfonlayer}
			\begin{pgfonlayer}{edgelayer}
				\draw (1) to (0.center);
				\draw (8.center) to (29);
				\draw (29) to (10);
				\draw (10) to (1);
			\end{pgfonlayer}
		\end{tikzpicture} = \begin{tikzpicture}
			\begin{pgfonlayer}{nodelayer}
				\node [style=onehalfcircle] (10) at (0.75, -1.25) {};
				\node [style=none] (21) at (0.75, -1.25) {$\alpha$};
				\node [style=none] (0) at (0.75, 1) {};
				\node [style=none] (8) at (0.75, -4) {};
				\node [style=none] (19) at (0.5, 0.75) {$E$};
				\node [style=none] (20) at (0.5, -3.5) {$E'$};
			\end{pgfonlayer}
			\begin{pgfonlayer}{edgelayer}
				\draw (8.center) to (10);
				\draw (10) to (0.center);
			\end{pgfonlayer}
		\end{tikzpicture} \]
	
	For the converse assume that $E \linmonb E'$ is a self-linear monoid given by splitting  a coring 
	sectional/retractional binary idempotent $(\u, \v)$ on linear monoid $A \linmonw B$.  
	The linear monoid $E \linmonb E'$ correspond precisely to a Frobenius Algebra $(E, \mulmap{1.5}{white},
	\unitmap{1.5}{white}, \comulmap{1.5}{black}, \counitmap{1.5}{black}) $
	under the linear equivalence ${\sf Mx}_\downarrow$. We must prove that the binary idempotent 
	satisfies the given equation. Let the binary idempotent be either sectional or retractional and split as follows:
	
	{ \centering $ \xymatrix{  A \ar@<0.5ex>[r]^{r}  
		 & E \ar[r]^{\alpha}_{\simeq} \ar@<0.5ex>[l]^{s} & E' \ar@<0.5ex>[r]^{s'}  
		 & B \ar@<0.5ex>[l]^{r'} } $ \par }
	
	Then,
	\[ \begin{tikzpicture}
		\begin{pgfonlayer}{nodelayer}
			\node [style=none] (70) at (10.95, 1) {};
			\node [style=onehalfcircle] (71) at (10.95, -1.5) {};
			\node [style=none] (72) at (10.95, -1.5) {$\u$};
			\node [style=none] (73) at (10.75, 0.75) {$A$};
			\node [style=none] (74) at (11, -3.75) {};
			\node [style=none] (75) at (10.75, -3.5) {$B$};
		\end{pgfonlayer}
		\begin{pgfonlayer}{edgelayer}
			\draw (71) to (70.center);
			\draw (71) to (74.center);
		\end{pgfonlayer}
	\end{tikzpicture} = \begin{tikzpicture}
		\begin{pgfonlayer}{nodelayer}
			\node [style=none] (70) at (11.2, 1) {};
			\node [style=onehalfcircle] (71) at (11.2, -1.25) {};
			\node [style=none] (72) at (11.2, -1.25) {$\alpha$};
			\node [style=none] (73) at (11, 0.75) {$A$};
			\node [style=none] (74) at (11.25, -4) {};
			\node [style=none] (75) at (11, -3.75) {$B$};
			\node [style=onehalfcircle] (76) at (11.25, 0) {};
			\node [style=none] (77) at (11.25, 0) {$r$};
			\node [style=onehalfcircle] (78) at (11.25, -2.75) {};
			\node [style=none] (79) at (11.25, -2.75) {$s'$};
		\end{pgfonlayer}
		\begin{pgfonlayer}{edgelayer}
			\draw (74.center) to (78);
			\draw (78) to (71);
			\draw (71) to (76);
			\draw (76) to (70.center);
		\end{pgfonlayer}
	\end{tikzpicture} = \begin{tikzpicture}
		\begin{pgfonlayer}{nodelayer}
			\node [style=onehalfcircle] (54) at (7.5, 0) {};
			\node [style=onehalfcircle] (55) at (9.5, -1) {};
			\node [style=none] (56) at (9.5, -1) {$s'$};
			\node [style=none] (57) at (8.5, 0) {};
			\node [style=none] (58) at (9.5, 0) {};
			\node [style=none] (59) at (9.5, -3.75) {};
			\node [style=onehalfcircle] (60) at (8.05, -2.25) {};
			\node [style=none] (61) at (8, -2.25) {$\alpha$};
			\node [style=none] (62) at (7.5, -3.25) {$E'$};
			\node [style=none] (63) at (7.25, 0.75) {$E$};
			\node [style=none] (64) at (9.75, -3.35) {$B$};
			\node [style=black] (65) at (8, -1) {};
			\node [style=black] (66) at (8, -3.75) {};
			\node [style=none] (67) at (9, 1) {$\eta_L'$};
			\node [style=none] (68) at (7.5, 1) {};
			\node [style=none] (69) at (7.5, 0) {$r$};
		\end{pgfonlayer}
		\begin{pgfonlayer}{edgelayer}
			\draw [bend left=270, looseness=2.25] (58.center) to (57.center);
			\draw [in=30, out=-90] (57.center) to (65);
			\draw (66) to (60);
			\draw (65) to (60);
			\draw [in=-90, out=150, looseness=1.25] (65) to (54);
			\draw (54) to (68.center);
			\draw (58.center) to (55);
			\draw (55) to (59.center);
		\end{pgfonlayer}
	\end{tikzpicture} = \begin{tikzpicture}
		\begin{pgfonlayer}{nodelayer}
			\node [style=onehalfcircle] (33) at (3.5, 0) {};
			\node [style=onehalfcircle] (34) at (5.5, -1.25) {};
			\node [style=none] (35) at (5.5, -1.25) {$s'$};
			\node [style=none] (36) at (4.5, 0.5) {};
			\node [style=none] (37) at (5.5, 0.5) {};
			\node [style=none] (38) at (5.5, -4) {};
			\node [style=onehalfcircle] (39) at (4, -1.75) {};
			\node [style=none] (40) at (4, -1.75) {$\alpha$};
			\node [style=none] (41) at (3.75, -3.5) {$B$};
			\node [style=none] (42) at (3.25, 0.75) {$E$};
			\node [style=none] (43) at (5.75, -3.75) {$B$};
			\node [style=black] (44) at (4, -1) {};
			\node [style=circle, fill=black] (45) at (4, -4) {};
			\node [style=none] (46) at (5, 1.25) {$\eta_L$};
			\node [style=none] (47) at (3.5, 1) {};
			\node [style=none] (48) at (3.5, 0) {$r$};
			\node [style=onehalfcircle] (49) at (5.5, -0.25) {};
			\node [style=none] (50) at (5.5, -0.25) {$r'$};
			\node [style=onehalfcircle] (51) at (4.5, 0) {};
			\node [style=none] (52) at (4.5, 0) {$r$};
			\node [style=none] (53) at (4.5, -0.5) {};
		\end{pgfonlayer}
		\begin{pgfonlayer}{edgelayer}
			\draw [bend right=90, looseness=1.50] (37.center) to (36.center);
			\draw (45) to (39);
			\draw (44) to (39);
			\draw [in=-90, out=165, looseness=1.25] (44) to (33);
			\draw (33) to (47.center);
			\draw (34) to (38.center);
			\draw (34) to (49);
			\draw (49) to (37.center);
			\draw (36.center) to (51);
			\draw [in=-90, out=15, looseness=1.25] (44) to (53.center);
			\draw (53.center) to (51);
		\end{pgfonlayer}
	\end{tikzpicture} = \begin{tikzpicture}
		\begin{pgfonlayer}{nodelayer}
			\node [style=onehalfcircle] (37) at (3, 4) {};
			\node [style=onehalfcircle] (18) at (5, 2.25) {};
			\node [style=none] (19) at (5, 2.25) {$s'$};
			\node [style=none] (20) at (4, 4.25) {};
			\node [style=none] (22) at (5, 4.25) {};
			\node [style=none] (23) at (5, -0.5) {};
			\node [style=onehalfcircle] (24) at (3.5, 1) {};
			\node [style=none] (25) at (3.5, 1) {$\alpha$};
			\node [style=none] (26) at (3, 0) {$B$};
			\node [style=none] (27) at (2.75, 4.5) {$A$};
			\node [style=none] (28) at (5.25, 0) {$B$};
			\node [style=black] (29) at (3.5, 2.5) {};
			\node [style=black] (30) at (3.5, -0.5) {};
			\node [style=none] (31) at (4.5, 5) {$\eta_L$};
			\node [style=none] (36) at (3, 4.75) {};
			\node [style=none] (33) at (3, 4) {$r$};
			\node [style=onehalfcircle] (38) at (5, 3.25) {};
			\node [style=none] (39) at (5, 3.25) {$r'$};
			\node [style=onehalfcircle] (40) at (4, 4) {};
			\node [style=none] (41) at (4, 4) {$r$};
			\node [style=none] (42) at (4, 3) {};
			\node [style=none] (43) at (3, 3) {};
			\node [style=onehalfcircle] (44) at (3, 3.25) {};
			\node [style=onehalfcircle] (46) at (4, 3.25) {};
			\node [style=none] (47) at (3, 3.25) {$s$};
			\node [style=none] (45) at (4, 3.25) {$s$};
			\node [style=onehalfcircle] (48) at (3.5, 1.75) {};
			\node [style=none] (49) at (3.5, 1.75) {$r$};
			\node [style=onehalfcircle] (50) at (3.5, 0.25) {};
			\node [style=none] (51) at (3.5, 0.25) {$s'$};
		\end{pgfonlayer}
		\begin{pgfonlayer}{edgelayer}
			\draw [bend right=90, looseness=1.50] (22.center) to (20.center);
			\draw (37) to (36.center);
			\draw (18) to (23.center);
			\draw (18) to (38);
			\draw (38) to (22.center);
			\draw (20.center) to (40);
			\draw [in=-90, out=15, looseness=1.25] (29) to (42.center);
			\draw [in=165, out=-90, looseness=1.25] (43.center) to (29);
			\draw (40) to (46);
			\draw (37) to (44);
			\draw (44) to (43.center);
			\draw (46) to (42.center);
			\draw (30) to (50);
			\draw (50) to (24);
			\draw (24) to (48);
			\draw (48) to (29);
		\end{pgfonlayer}
	\end{tikzpicture} = \begin{tikzpicture}
		\begin{pgfonlayer}{nodelayer}
			\node [style=onehalfcircle] (37) at (3, 3.75) {};
			\node [style=none] (20) at (4, 4.25) {};
			\node [style=none] (22) at (5, 4.25) {};
			\node [style=none] (23) at (5, -0.5) {};
			\node [style=onehalfcircle] (24) at (3.5, 1) {};
			\node [style=none] (25) at (3.5, 1) {$\u$};
			\node [style=none] (26) at (3, 0) {$B$};
			\node [style=none] (27) at (2.75, 4.5) {$A$};
			\node [style=none] (28) at (5.25, 0) {$B$};
			\node [style=circle] (29) at (3.5, 2.5) {};
			\node [style=circle] (30) at (3.5, -0.5) {};
			\node [style=none] (31) at (4.5, 5) {$\eta_L$};
			\node [style=none] (36) at (3, 4.75) {};
			\node [style=none] (33) at (3, 3.75) {$e_A$};
			\node [style=onehalfcircle] (38) at (5, 3.25) {};
			\node [style=none] (39) at (5, 3.25) {$e_B$};
			\node [style=onehalfcircle] (40) at (4, 3.75) {};
			\node [style=none] (41) at (4, 3.75) {$e_A$};
			\node [style=none] (42) at (4, 3) {};
			\node [style=none] (43) at (3, 3) {};
		\end{pgfonlayer}
		\begin{pgfonlayer}{edgelayer}
			\draw [bend right=90, looseness=1.50] (22.center) to (20.center);
			\draw (37) to (36.center);
			\draw (38) to (22.center);
			\draw (20.center) to (40);
			\draw [in=-90, out=15, looseness=1.25] (29) to (42.center);
			\draw [in=165, out=-90, looseness=1.25] (43.center) to (29);
			\draw (38) to (23.center);
			\draw (40) to (42.center);
			\draw (37) to (43.center);
			\draw (30) to (24);
			\draw (24) to (29);
		\end{pgfonlayer}
	\end{tikzpicture} \]
	\end{proof}

 In a $\dagger$-isomix category splitting a sectional or retractional $\dagger$-binary idempotent 
 on a $\dagger$-linear monoid  gives a $\dagger$-self-linear monoid on a pre-unitary object. 
 If the binary idempotent satisfies equation \ref{eqn: dag Frob split}(a) in addition, then, by using 
 Lemma \ref{Lemma: dagmondagFrob} and  \ref{Lemma: Frobenius splitting}, one 
 gets a $\dagger$-Frobenius Algebra on the splitting. 

%%%%%%%%%%%%%%%%%%%%%%%%%%%%%%%%%%%%%%%%%%%%%%%%%
\subsection{For linear comonoids}

In a monoidal category, an idempotent $e: A \to A$ 
is {\bf sectional} (respectively {\bf retractional}) on a comonoid $(A, d, k)$  
if $e d = e d (e \ox e)$ (respectively if $d (e \ox e) = e d (e \ox e)$, and $e k = k$). 

\begin{lemma} 
	\label{Lemma: retract is comon hom}
	In a monoidal category, a split idempotent $e: A \to A$ on a comonoid $(A,d,k)$, 
	with splitting $A \to^r E \to^s$ A, is sectional (respectively retractional) if and only if the 
	section $s$ (respectively the retraction $r$) is a comonoid morphism for $(E, s d (r \ox r), s k)$.
\end{lemma}
\begin{proof} Suppose $e: A \to A$ is an idempotent with the splitting $A \to^r E \to^s$  and $(A, d, k)$ 
is a monoid. Suppose $e$ is sectional on A i.e, $d (e \ox e) = e d (e \ox e)$. 

We must prove that $(E, d', k')$ is a monoid where $d' := s d (r \ox r)$ and  $k' := s k$.
Unit law and associativity law are proven as follows. 
\begin{align*} 
	d' (k' \ox 1) &= s d (r \ox r) (s k \ox 1) = s d (rsk \ox r) = s d (ek \ox r)  \\ 
	&= srs d (ek \ox rsr)  = s e d (ek \ox er) \stackrel{sectional}  =  se d (k \ox r) = s e r = srsr = 1 
\end{align*}
In the following string diagrams, we use black circle for $(E, d', k')$, 
and white circle for $(A, d, k)$:
\[ \begin{tikzpicture}
	\begin{pgfonlayer}{nodelayer}
		\node [style=none] (0) at (0.25, -1) {};
		\node [style=none] (1) at (-0.5, -1) {};
		\node [style=none] (2) at (1.25, -1) {};
		\node [style=none] (3) at (0.25, 4) {};
		\node [style=none] (4) at (1.5, -0.75) {$E$};
		\node [style=none] (5) at (0.5, -0.75) {$E$};
		\node [style=none] (6) at (-0.75, -0.75) {$E$};
		\node [style=none] (7) at (0, 3.75) {$E$};
		\node [style=circle, fill=black] (8) at (0.75, 0.25) {};
		\node [style=circle, fill=black] (9) at (0.25, 1.25) {};
	\end{pgfonlayer}
	\begin{pgfonlayer}{edgelayer}
		\draw [in=-45, out=90] (2.center) to (8);
		\draw [in=90, out=-135] (8) to (0.center);
		\draw [in=-30, out=90, looseness=0.75] (8) to (9);
		\draw [in=90, out=-150, looseness=0.75] (9) to (1.center);
		\draw (9) to (3.center);
	\end{pgfonlayer}
\end{tikzpicture}
 := \begin{tikzpicture}
	\begin{pgfonlayer}{nodelayer}
		\node [style=circle, scale=1.5] (0) at (-1.5, 1.75) {};
		\node [style=circle, scale=1.5] (1) at (-0.75, -0.5) {};
		\node [style=circle, scale=1.5] (2) at (0.75, -0.5) {};
		\node [style=circle] (3) at (0, 0.5) {};
		\node [style=none] (4) at (-0.75, -1.25) {};
		\node [style=none] (5) at (0.75, -1.25) {};
		\node [style=none] (6) at (-1.5, -1.25) {};
		\node [style=circle, scale=1.5] (7) at (0, 1.25) {};
		\node [style=circle, scale=1.5] (8) at (0, 2) {};
		\node [style=circle] (9) at (-0.75, 2.75) {};
		\node [style=circle, scale=1.5] (10) at (-0.75, 3.5) {};
		\node [style=none] (11) at (-0.75, 4) {};
		\node [style=none] (12) at (-1.5, 1.75) {$r$};
		\node [style=none] (13) at (-0.75, -0.5) {$r$};
		\node [style=none] (14) at (0.75, -0.5) {$r$};
		\node [style=none] (15) at (0, 1.25) {$s$};
		\node [style=none] (16) at (0, 2) {$r$};
		\node [style=none] (17) at (-0.75, 3.5) {$s$};
		\node [style=none] (18) at (-1.75, -1) {$E$};
		\node [style=none] (19) at (-1, -1) {$E$};
		\node [style=none] (20) at (0.5, -1) {$E$};
		\node [style=none] (21) at (-1.25, 3) {$A$};
		\node [style=none] (22) at (-1.25, 3.75) {$E$};
	\end{pgfonlayer}
	\begin{pgfonlayer}{edgelayer}
		\draw (4.center) to (1);
		\draw (5.center) to (2);
		\draw [in=-165, out=90, looseness=1.25] (1) to (3);
		\draw [in=90, out=-15, looseness=1.25] (3) to (2);
		\draw (6.center) to (0);
		\draw [in=-165, out=90, looseness=1.25] (0) to (9);
		\draw (3) to (7);
		\draw (7) to (8);
		\draw [in=-15, out=105, looseness=1.25] (8) to (9);
		\draw (9) to (10);
		\draw (10) to (11.center);
	\end{pgfonlayer}
\end{tikzpicture}
= \begin{tikzpicture}
	\begin{pgfonlayer}{nodelayer}
		\node [style=circle, scale=1.5] (0) at (-1.5, 1.25) {};
		\node [style=circle, scale=1.5] (1) at (-0.75, -0.5) {};
		\node [style=circle, scale=1.5] (2) at (0.75, -0.5) {};
		\node [style=circle] (3) at (0, 0.5) {};
		\node [style=none] (4) at (-0.75, -1.25) {};
		\node [style=none] (5) at (0.75, -1.25) {};
		\node [style=none] (6) at (-1.5, -1.25) {};
		\node [style=circle, scale=1.5] (7) at (0, 1.25) {};
		\node [style=circle] (9) at (-0.75, 2.75) {};
		\node [style=circle, scale=1.5] (10) at (-0.75, 3.5) {};
		\node [style=none] (11) at (-0.75, 4) {};
		\node [style=none] (12) at (-1.5, 1.25) {$r$};
		\node [style=none] (13) at (-0.75, -0.5) {$r$};
		\node [style=none] (14) at (0.75, -0.5) {$r$};
		\node [style=none] (15) at (0, 1.25) {$e$};
		\node [style=none] (17) at (-0.75, 3.5) {$s$};
		\node [style=none] (18) at (-1.75, -1) {$E$};
		\node [style=none] (19) at (-1, -1) {$E$};
		\node [style=none] (20) at (0.5, -1) {$E$};
		\node [style=none] (21) at (-1.25, 3) {$A$};
		\node [style=none] (22) at (-1.25, 3.75) {$E$};
	\end{pgfonlayer}
	\begin{pgfonlayer}{edgelayer}
		\draw (4.center) to (1);
		\draw (5.center) to (2);
		\draw [in=-165, out=90, looseness=1.25] (1) to (3);
		\draw [in=90, out=-15, looseness=1.25] (3) to (2);
		\draw (6.center) to (0);
		\draw [in=-165, out=90, looseness=1.25] (0) to (9);
		\draw (3) to (7);
		\draw (9) to (10);
		\draw (10) to (11.center);
		\draw [in=90, out=-15] (9) to (7);
	\end{pgfonlayer}
\end{tikzpicture}
 =\begin{tikzpicture}
	\begin{pgfonlayer}{nodelayer}
		\node [style=circle, scale=1.5] (0) at (-1.75, -0.5) {};
		\node [style=circle] (3) at (-0.25, 0.25) {};
		\node [style=none] (4) at (-0.75, -1.25) {};
		\node [style=none] (5) at (0.25, -1.25) {};
		\node [style=none] (6) at (-1.75, -1.25) {};
		\node [style=circle, scale=1.5] (7) at (-0.25, 1) {};
		\node [style=circle] (9) at (-1, 1.75) {};
		\node [style=circle, scale=1.5] (10) at (-1, 3.25) {};
		\node [style=none] (11) at (-1, 4) {};
		\node [style=none] (12) at (-1.75, -0.5) {$r$};
		\node [style=none] (15) at (-0.25, 1) {$e$};
		\node [style=none] (17) at (-1, 3.25) {$s$};
		\node [style=none] (18) at (-2, -1.25) {$E$};
		\node [style=none] (19) at (-1, -1.25) {$E$};
		\node [style=none] (20) at (0, -1.25) {$E$};
		\node [style=none] (21) at (-1.5, 3) {$A$};
		\node [style=none] (22) at (-1.5, 3.75) {$E$};
		\node [style=circle, scale=1.5] (26) at (-0.75, -0.5) {};
		\node [style=circle, scale=1.5] (27) at (0.25, -0.5) {};
		\node [style=none] (30) at (0.25, -0.5) {$r$};
		\node [style=none] (31) at (-0.75, -0.5) {$r$};
		\node [style=circle, scale=1.5] (33) at (-1.75, 1) {};
		\node [style=none] (34) at (-1.75, 1) {$e$};
		\node [style=circle, scale=1.5] (37) at (-1, 2.5) {};
		\node [style=none] (38) at (-1, 2.5) {$e$};
	\end{pgfonlayer}
	\begin{pgfonlayer}{edgelayer}
		\draw (6.center) to (0);
		\draw (3) to (7);
		\draw (10) to (11.center);
		\draw [in=90, out=-15] (9) to (7);
		\draw [in=90, out=-30] (3) to (27);
		\draw [in=90, out=-150] (3) to (26);
		\draw [in=-165, out=90] (33) to (9);
		\draw (5.center) to (27);
		\draw (4.center) to (26);
		\draw (37) to (10);
		\draw (9) to (38.center);
		\draw (33) to (0);
	\end{pgfonlayer}
\end{tikzpicture}
\stackrel{(*)}{=} \begin{tikzpicture}
	\begin{pgfonlayer}{nodelayer}
		\node [style=circle] (3) at (-0.25, 0.25) {};
		\node [style=none] (4) at (-0.75, -1.25) {};
		\node [style=none] (5) at (0.25, -1.25) {};
		\node [style=none] (6) at (-1.75, -1.25) {};
		\node [style=circle] (9) at (-1, 1.75) {};
		\node [style=circle, scale=1.5] (10) at (-1, 3.25) {};
		\node [style=none] (11) at (-1, 4) {};
		\node [style=none] (17) at (-1, 3.25) {$s$};
		\node [style=none] (18) at (-2, -1.25) {$E$};
		\node [style=none] (19) at (-1, -1.25) {$E$};
		\node [style=none] (20) at (0, -1.25) {$E$};
		\node [style=none] (21) at (-1.5, 3) {$A$};
		\node [style=none] (22) at (-1.5, 3.75) {$E$};
		\node [style=circle, scale=1.5] (26) at (-0.75, -0.5) {};
		\node [style=circle, scale=1.5] (27) at (0.25, -0.5) {};
		\node [style=none] (30) at (0.25, -0.5) {$r$};
		\node [style=none] (31) at (-0.75, -0.5) {$r$};
		\node [style=circle, scale=1.5] (33) at (-1.75, 1) {};
		\node [style=none] (34) at (-1.75, 1) {$r$};
		\node [style=circle, scale=1.5] (37) at (-1, 2.5) {};
		\node [style=none] (38) at (-1, 2.5) {$e$};
	\end{pgfonlayer}
	\begin{pgfonlayer}{edgelayer}
		\draw (10) to (11.center);
		\draw [in=90, out=-30] (3) to (27);
		\draw [in=90, out=-150] (3) to (26);
		\draw [in=-165, out=90] (33) to (9);
		\draw (5.center) to (27);
		\draw (4.center) to (26);
		\draw (37) to (10);
		\draw (6.center) to (33);
		\draw [in=-30, out=90, looseness=1.25] (3) to (9);
		\draw (9) to (37);
	\end{pgfonlayer}
\end{tikzpicture}
 = \begin{tikzpicture}
	\begin{pgfonlayer}{nodelayer}
		\node [style=circle] (3) at (-1.5, 0.25) {};
		\node [style=none] (4) at (-1, -1.25) {};
		\node [style=none] (5) at (-2, -1.25) {};
		\node [style=none] (6) at (0, -1.25) {};
		\node [style=circle] (9) at (-0.75, 1.75) {};
		\node [style=circle, scale=1.5] (10) at (-0.75, 3.25) {};
		\node [style=none] (11) at (-0.75, 4) {};
		\node [style=none] (17) at (-0.75, 3.25) {$s$};
		\node [style=none] (18) at (0.25, -1.25) {$E$};
		\node [style=none] (19) at (-0.75, -1.25) {$E$};
		\node [style=none] (20) at (-1.75, -1.25) {$E$};
		\node [style=none] (21) at (-0.25, 3) {$A$};
		\node [style=none] (22) at (-0.25, 3.75) {$E$};
		\node [style=circle, scale=1.5] (26) at (-1, -0.5) {};
		\node [style=circle, scale=1.5] (27) at (-2, -0.5) {};
		\node [style=none] (30) at (-2, -0.5) {$r$};
		\node [style=none] (31) at (-1, -0.5) {$r$};
		\node [style=circle, scale=1.5] (33) at (0, 1) {};
		\node [style=none] (34) at (0, 1) {$r$};
		\node [style=circle, scale=1.5] (37) at (-0.75, 2.5) {};
		\node [style=none] (38) at (-0.75, 2.5) {$e$};
	\end{pgfonlayer}
	\begin{pgfonlayer}{edgelayer}
		\draw (10) to (11.center);
		\draw [in=90, out=-150] (3) to (27);
		\draw [in=90, out=-30] (3) to (26);
		\draw [in=-15, out=90] (33) to (9);
		\draw (5.center) to (27);
		\draw (4.center) to (26);
		\draw (37) to (10);
		\draw (6.center) to (33);
		\draw [in=-150, out=90, looseness=1.25] (3) to (9);
		\draw (9) to (37);
	\end{pgfonlayer}
\end{tikzpicture}
\stackrel{(*)}{=} \begin{tikzpicture}
	\begin{pgfonlayer}{nodelayer}
		\node [style=circle, scale=1.5] (37) at (-0.75, 2.5) {};
		\node [style=none] (38) at (-0.75, 2.5) {$e$};
		\node [style=circle, scale=1.5] (0) at (0, -0.5) {};
		\node [style=circle] (3) at (-1.5, 0.25) {};
		\node [style=none] (4) at (-1, -1.25) {};
		\node [style=none] (5) at (-2, -1.25) {};
		\node [style=none] (6) at (0, -1.25) {};
		\node [style=circle, scale=1.5] (7) at (-1.5, 1) {};
		\node [style=circle] (9) at (-0.75, 1.75) {};
		\node [style=circle, scale=1.5] (10) at (-0.75, 3.25) {};
		\node [style=none] (11) at (-0.75, 4) {};
		\node [style=none] (12) at (0, -0.5) {$r$};
		\node [style=none] (15) at (-1.5, 1) {$e$};
		\node [style=none] (17) at (-0.75, 3.25) {$s$};
		\node [style=none] (18) at (0.25, -1.25) {$E$};
		\node [style=none] (19) at (-0.75, -1.25) {$E$};
		\node [style=none] (20) at (-1.75, -1.25) {$E$};
		\node [style=none] (21) at (-0.25, 3) {$A$};
		\node [style=none] (22) at (-0.25, 3.75) {$E$};
		\node [style=circle, scale=1.5] (26) at (-1, -0.5) {};
		\node [style=circle, scale=1.5] (27) at (-2, -0.5) {};
		\node [style=none] (30) at (-2, -0.5) {$r$};
		\node [style=none] (31) at (-1, -0.5) {$r$};
		\node [style=circle, scale=1.5] (33) at (0, 1) {};
		\node [style=none] (34) at (0, 1) {$e$};
	\end{pgfonlayer}
	\begin{pgfonlayer}{edgelayer}
		\draw (6.center) to (0);
		\draw (3) to (7);
		\draw (10) to (11.center);
		\draw [in=90, out=-165] (9) to (7);
		\draw [in=90, out=-150] (3) to (27);
		\draw [in=90, out=-30] (3) to (26);
		\draw [in=-15, out=90] (33) to (9);
		\draw (5.center) to (27);
		\draw (4.center) to (26);
		\draw (37) to (10);
		\draw (33) to (0);
		\draw (9) to (37);
	\end{pgfonlayer}
\end{tikzpicture}
 =  \begin{tikzpicture}
	\begin{pgfonlayer}{nodelayer}
		\node [style=none] (0) at (0.5, -1) {};
		\node [style=none] (1) at (1.25, -1) {};
		\node [style=none] (2) at (-0.5, -1) {};
		\node [style=none] (3) at (0.5, 4) {};
		\node [style=none] (4) at (-0.75, -0.75) {$E$};
		\node [style=none] (5) at (0.25, -0.75) {$E$};
		\node [style=none] (6) at (1.5, -0.75) {$E$};
		\node [style=none] (7) at (0.75, 3.75) {$E$};
		\node [style=circle, fill=black] (8) at (0, 0.25) {};
		\node [style=circle, fill=black] (9) at (0.5, 1.25) {};
	\end{pgfonlayer}
	\begin{pgfonlayer}{edgelayer}
		\draw [in=-135, out=90] (2.center) to (8);
		\draw [in=90, out=-45] (8) to (0.center);
		\draw [in=-150, out=90, looseness=0.75] (8) to (9);
		\draw [in=90, out=-30, looseness=0.75] (9) to (1.center);
		\draw (9) to (3.center);
	\end{pgfonlayer}
\end{tikzpicture}\]
The steps labelled by $(*)$ are because $e$ is sectional on $(A,d,k)$.

Finally, $s: E \to A $ is a comonoid homomorphism because: 
\[d' (s \ox s) = s d (r \ox r) (s \ox s) = s d (rs \ox rs )  = s d (e \ox e)  
= srs d (e \ox e)   =  s e d (e \ox e) = s e d = s d\]
For the converse, assume that $(E, d', k')$ is a comonoid where $d' = s d (r \ox r)$, and $k' = s k$, and 
$s$ is a monoid homomorphism, then, $e: A \to A$ is sectional on $(A,d,k)$ because:
\[e d = r s d = r d' (s \ox s) = r s d (r \ox r) (s \ox s) = e d (e \ox e)\]
The statement is proven similarly when the idempotent is retractional on the comonoid.
\end{proof}

In an LDC,  a pair of idempotents $(e_A, e_B)$ is {\bf sectional} (respectively {\bf retractional}) on a linear comonoid when 
$e_A$ and $e_B$ satisfies the conditions in table given below. 
\[ 	\begin{tabular}{l|| l} 
\hline
$(e_A, e_B)$	\textbf{sectional} on $A \lincomonw B$  & $(e_A, e_B)$ \textbf{retractional} on $A \lincomonw B$\\
\hline
	$e_A$ preserves $(A,d,k)$ sectionally &  $e_A$ preserves $(A,d,k)$ retractionally \\
\hline
	$(e_A, e_B)$ preserves $(\eta_L, \epsilon_L)\!\!:\! A \dashvv B$ sectionally &$(e_A, e_B)$ preserves $(\eta_L, \epsilon_L)\!\!:\! A \dashvv B$ retractionally\\
\hline
	$(e_B, e_A)$ preserves $(\eta_R, \epsilon_R)\!\!:\! B \dashvv A$ retractionally & $(e_B, e_A)$ preserves $(\eta_R, \epsilon_R)\!\!:\! B \dashvv A$ sectionally\\
\hline
\end{tabular} \]
A binary idempotent $(\u,\v)$ is {\bf sectional} (respectively {\bf retractional}) on a linear monoid when $(\u\v, \v\u)$ 
is sectional (respectively retractional).

\begin{lemma}
	\label{Lemma: split weak linear comon hom}
	In an LDC, let $(e_A, e_B)$ be a pair of idempotents 
	on a linear comonoid  $A \lincomonw B$ with $\ox$-comonoid $(A,d,e)$ and 
	splitting $A \to^r E \to^s A$, and $B \to^{r'} E' \to^{s'} B$. 
	The idempotent pair $(e_A, e_B)$ is sectional (respectively retractional) if and only if the 
	section $(s, r')\!\!:\!(E \lincomonb E') \to (A \lincomonw B)$ (respectively the retraction 
	$(r, s')\!\!:\!(A \lincomonw B) \to (E \lincomonb E')$) is a morphism of linear comonoid.
\end{lemma}
The proof is similar to that of Lemma \ref{Lemma: split weak linear mon hom} for 
sectional/retractional idempotents on linear monoids. 

As a consequence of the above Lemma, a split binary idempotent is sectional 
(respectively retractional) on a linear comonoid if and only if its section (respectively 
retraction) is a morphism from (respectiely to) the self-linear comonoid given 
on the splitting.

When a sectional or retractional $\dagger$-binary idempotent on $\dagger$-linear comonoid splits, 
it gives a $\dagger$-self-linear comonoid on the splitting:

\begin{lemma}
	In a $\dag$-LDC, let $(e, e^\dag)$ be a pair of idempotents 
	on a $\dag$-linear monoid  $A \lincomonw A^\dag$ with 
	splitting $A \to^r E \to^s A$. 
	The idempotent pair $(e, e^\dag)$ is sectional (respectively retractional) if and only if the 
	section $(s, s^\dag)$ (respectively the retraction $(r, r^\dag)$) is a morphism of $\dag$-linear monoids. 
\end{lemma}
The proof is similar to Lemma \ref{Lemma: weak dag lin mon hom} for sectional/retractional idempotents 
on $\dagger$-linear monoids. 

As a consequence of the above Lemma, a split $\dagger$-binary idempotent is sectional 
(respectively retractional) on a $\dagger$-linear monoid if and only if its section (respectively 
retraction) is a morphism from (respectively to) the $\dagger$-self-linear monoid given 
on the splitting.

%%%%%%%%%%%%%%%%%%%%%%%%%%%%%%%%%%%%%%%%%%%%%%%%%
\subsection{For linear bialgebras}

We turn our attention to binary idempotents for linear bialgebras:

\begin{definition}
	A binary idempotent on a linear bialgebra is {\bf sectional (respectively retractional)} if its sectional (respectively retractional) 
	on the linear monoid, and the linear comonoid.  
\end{definition}

Lemma \ref{Lemma: split weak linear mon hom} states that a pair of split idempotents is sectional (respectively retractional) 
on a linear monoid if and only if the section (repectively retraction) a morphism of linear monoid. Lemma 
\ref{Lemma: split weak linear comon hom} states that a pair of split idempotents is sectional (respectively retractional) 
on a linear comonoid if and only if the section (repectively retraction) a morphism of linear comonoid.  
Applying these results to a sectional / retractional binary idempotent on a linear bialgebra we get:  

\begin{lemma} 
	\label{Lemma: split weak linear bialg hom}
    In an LDC, let $(\u, \v)$ be a split binary idempotent 
	on a linear bialgebra  $A \linbialgwtik B$  with
	splitting $A \to^r E \to^s A$, and $B \to^{r'} E' \to^{s'} B$ 
	 is sectional (respectively retractional) if and only if the 
	section $(s, r')$ (respectively the retraction $(r, s')$) is a morphism of linear bialgebra for 
	$E \linbialgbtik E'$ with its linear monoid and linear comonoid given as in Statement $(i)$ of 
	Lemma \ref{Lemma: split weak linear mon hom} and 
	 Lemma \ref{Lemma: split weak linear comon hom} respectively.
\end{lemma}
	
\begin{proof}~  
 Let $(\u, \v)$ be a sectional binary idempotent on a linear bialgebra $A \linbialgwtik B$. Let 
$(\u, \v)$ split as follows: $e_A = \u\v = A \to^r E \to^{s} A$, and  $e_B = \v\u = B \to^{r'} E' \to^{s'} B$.
Using Statement $(i)$ of Lemma \ref{Lemma: split weak linear mon hom}, 
and Statement $(i)$ of  Lemma \ref{Lemma: split weak linear comon hom}, 
we know that $E$ is both a linear monoid and a linear comonoid. 

It remains to show that $E \linmonb E'$, and $E \lincomonb E'$ give a $\ox$-bialgebra on $E$, 
and a $\oa$-bialgebra on $E'$. The proof that $E$ is a $\ox$-bialagebra is as follows:

\bigskip

$\begin{tikzpicture}
	\begin{pgfonlayer}{nodelayer}
		\node [style=circle, fill=black] (0) at (0, 5.75) {};
		\node [style=none] (1) at (-0.75, 1.75) {};
		\node [style=none] (2) at (0.75, 1.75) {};
		\node [style=none] (3) at (-1, 2) {$E$};
		\node [style=none] (4) at (1, 2) {$E$};
		\node [style=none] (5) at (0, 4.25) {};
		\node [style=none] (6) at (-0.25, 4) {};
		\node [style=none] (7) at (0.25, 4) {};
	\end{pgfonlayer}
	\begin{pgfonlayer}{edgelayer}
		\draw [in=90, out=-135, looseness=1.00] (6.center) to (1.center);
		\draw [in=90, out=-45, looseness=1.00] (7.center) to (2.center);
		\draw (5.center) to (0);
		\draw [fill=black] (5.center) -- (6.center) -- (7.center) -- (5.center);
	\end{pgfonlayer}
\end{tikzpicture} := \begin{tikzpicture}[yscale=1]
	\begin{pgfonlayer}{nodelayer}
		\node [style=circle] (0) at (0, 6) {};
		\node [style=circle, scale=1.5] (1) at (0, 5) {};
		\node [style=circle, scale=1.5] (2) at (0, 4.25) {};
		\node [style=none] (3) at (0, 3.5) {};
		\node [style=none] (4) at (-0.25, 3.25) {};
		\node [style=none] (5) at (0.25, 3.25) {};
		\node [style=circle, scale=1.5] (6) at (0.5, 2.5) {};
		\node [style=circle, scale=1.5] (7) at (-0.5, 2.5) {};
		\node [style=none] (8) at (-0.5, 1.75) {};
		\node [style=none] (9) at (0.5, 1.75) {};
		\node [style=none] (10) at (0, 5) {$r$};
		\node [style=none] (11) at (0, 4.25) {$s$};
		\node [style=none] (12) at (-0.5, 2.5) {$r$};
		\node [style=none] (13) at (0.5, 2.5) {$r$};
		\node [style=none] (14) at (0.25, 5.5) {$A$};
		\node [style=none] (15) at (0.25, 3.75) {$A$};
		\node [style=none] (16) at (-0.75, 2) {$E$};
		\node [style=none] (17) at (0.75, 2) {$E$};
	\end{pgfonlayer}
	\begin{pgfonlayer}{edgelayer}
		\draw (0) to (1);
		\draw (1) to (2);
		\draw (2) to (3.center);
		\draw (3.center) to (4.center);
		\draw (4.center) to (5.center);
		\draw (5.center) to (3.center);
		\draw [in=90, out=-150, looseness=1.00] (4.center) to (7);
		\draw (7) to (8.center);
		\draw [in=90, out=-30, looseness=1.00] (5.center) to (6);
		\draw (6) to (9.center);
	\end{pgfonlayer}
\end{tikzpicture} = \begin{tikzpicture}[yscale=1]
	\begin{pgfonlayer}{nodelayer}
		\node [style=circle] (0) at (0, 6) {};
		\node [style=none] (1) at (0, 3.5) {};
		\node [style=none] (2) at (-0.25, 3.25) {};
		\node [style=none] (3) at (0.25, 3.25) {};
		\node [style=circle, scale=1.5] (4) at (0.5, 2.5) {};
		\node [style=circle, scale=1.5] (5) at (-0.5, 2.5) {};
		\node [style=none] (6) at (-0.5, 1.75) {};
		\node [style=none] (7) at (0.5, 1.75) {};
		\node [style=none] (8) at (-0.5, 2.5) {$r$};
		\node [style=none] (9) at (0.5, 2.5) {$r$};
		\node [style=none] (10) at (0.25, 5.5) {$A$};
		\node [style=none] (11) at (-0.75, 2) {$E$};
		\node [style=none] (12) at (0.75, 2) {$E$};
		\node [style=circle, scale=1.5] (13) at (0, 4.75) {};
		\node [style=none] (14) at (0.25, 3.75) {$A$};
		\node [style=none] (15) at (0, 4.75) {$e_A$};
	\end{pgfonlayer}
	\begin{pgfonlayer}{edgelayer}
		\draw (1.center) to (2.center);
		\draw (2.center) to (3.center);
		\draw (3.center) to (1.center);
		\draw [in=90, out=-150, looseness=1.00] (2.center) to (5);
		\draw (5) to (6.center);
		\draw [in=90, out=-30, looseness=1.00] (3.center) to (4);
		\draw (4) to (7.center);
		\draw (13) to (1.center);
		\draw (13) to (0);
	\end{pgfonlayer}
\end{tikzpicture} \stackrel{(*)}{=}  \begin{tikzpicture}
	\begin{pgfonlayer}{nodelayer}
		\node [style=circle] (0) at (0, 6) {};
		\node [style=none] (1) at (0, 3.5) {};
		\node [style=none] (2) at (-0.25, 3.25) {};
		\node [style=none] (3) at (0.25, 3.25) {};
		\node [style=circle, scale=1.5] (4) at (0.5, 2.5) {};
		\node [style=circle, scale=1.5] (5) at (-0.5, 2.5) {};
		\node [style=none] (6) at (-0.5, 1.75) {};
		\node [style=none] (7) at (0.5, 1.75) {};
		\node [style=none] (8) at (-0.5, 2.5) {$r$};
		\node [style=none] (9) at (0.5, 2.5) {$r$};
		\node [style=none] (10) at (0.25, 5.5) {$A$};
		\node [style=none] (11) at (-0.75, 2) {$E$};
		\node [style=none] (12) at (0.75, 2) {$E$};
	\end{pgfonlayer}
	\begin{pgfonlayer}{edgelayer}
		\draw (1.center) to (2.center);
		\draw (2.center) to (3.center);
		\draw (3.center) to (1.center);
		\draw [in=90, out=-150, looseness=1.00] (2.center) to (5);
		\draw (5) to (6.center);
		\draw [in=90, out=-30, looseness=1.00] (3.center) to (4);
		\draw (4) to (7.center);
		\draw (0) to (1.center);
	\end{pgfonlayer}
\end{tikzpicture} = \begin{tikzpicture}
	\begin{pgfonlayer}{nodelayer}
		\node [style=circle] (0) at (-0.5, 5.75) {};
		\node [style=circle, scale=1.5] (1) at (0.5, 3.5) {};
		\node [style=circle, scale=1.5] (2) at (-0.5, 3.5) {};
		\node [style=none] (3) at (-0.5, 1.75) {};
		\node [style=none] (4) at (0.5, 1.75) {};
		\node [style=none] (5) at (-0.5, 3.5) {$r$};
		\node [style=none] (6) at (0.5, 3.5) {$r$};
		\node [style=none] (7) at (-0.75, 2) {$E$};
		\node [style=none] (8) at (0.75, 2) {$E$};
		\node [style=circle] (9) at (0.5, 5.75) {};
	\end{pgfonlayer}
	\begin{pgfonlayer}{edgelayer}
		\draw (2) to (3.center);
		\draw (1) to (4.center);
		\draw (0) to (2);
		\draw (9) to (1);
	\end{pgfonlayer}
\end{tikzpicture} = \begin{tikzpicture}
	\begin{pgfonlayer}{nodelayer}
		\node [style=circle, fill=black] (0) at (-0.5, 5.75) {};
		\node [style=none] (1) at (-0.5, 1.75) {};
		\node [style=none] (2) at (0.5, 1.75) {};
		\node [style=none] (3) at (-0.75, 2) {$E$};
		\node [style=none] (4) at (0.75, 2) {$E$};
		\node [style=circle, fill=black] (5) at (0.5, 5.75) {};
	\end{pgfonlayer}
	\begin{pgfonlayer}{edgelayer}
		\draw (0) to (1.center);
		\draw (5) to (2.center);
	\end{pgfonlayer}
\end{tikzpicture} $

\medskip

\medskip

$ \begin{tikzpicture}[yscale=1]
	\begin{pgfonlayer}{nodelayer}
		\node [style=none] (0) at (-0.75, 6) {};
		\node [style=none] (1) at (0.75, 6) {};
		\node [style=none] (2) at (-1, 5.75) {$E$};
		\node [style=none] (3) at (1, 5.75) {$E$};
		\node [style=none] (4) at (-0.25, 2) {};
		\node [style=none] (5) at (0.25, 2) {};
		\node [style=none] (6) at (0, 2.25) {};
		\node [style=circle, fill=black] (7) at (0, 4) {};
	\end{pgfonlayer}
	\begin{pgfonlayer}{edgelayer}
		\draw [fill=black] (6.center) -- (4.center) -- (5.center) -- (6.center);
		\draw [in=135, out=-90, looseness=1.00] (0.center) to (7);
		\draw [in=-90, out=45, looseness=1.00] (7) to (1.center);
		\draw (7) to (6.center);
	\end{pgfonlayer}
\end{tikzpicture} := \begin{tikzpicture}[yscale=1]
	\begin{pgfonlayer}{nodelayer}
		\node [style=circle] (0) at (0, 4.5) {};
		\node [style=circle, scale=1.5] (1) at (0, 2.75) {};
		\node [style=circle, scale=1.5] (2) at (0, 3.5) {};
		\node [style=circle, scale=1.5] (3) at (0.5, 5.25) {};
		\node [style=circle, scale=1.5] (4) at (-0.5, 5.25) {};
		\node [style=none] (5) at (-0.5, 6) {};
		\node [style=none] (6) at (0.5, 6) {};
		\node [style=none] (7) at (0, 2.75) {$s$};
		\node [style=none] (8) at (0, 3.5) {$r$};
		\node [style=none] (9) at (-0.5, 5.25) {$s$};
		\node [style=none] (10) at (0.5, 5.25) {$s$};
		\node [style=none] (11) at (0.25, 2.25) {$A$};
		\node [style=none] (12) at (0.25, 4) {$A$};
		\node [style=none] (13) at (-0.75, 5.75) {$E$};
		\node [style=none] (14) at (0.75, 5.75) {$E$};
		\node [style=none] (15) at (-0.25, 1.75) {};
		\node [style=none] (16) at (0.25, 1.75) {};
		\node [style=none] (17) at (0, 2) {};
	\end{pgfonlayer}
	\begin{pgfonlayer}{edgelayer}
		\draw (1) to (2);
		\draw (4) to (5.center);
		\draw (3) to (6.center);
		\draw (17.center) to (15.center);
		\draw (15.center) to (16.center);
		\draw (16.center) to (17.center);
		\draw (17.center) to (1);
		\draw (0) to (2);
		\draw [in=150, out=-90, looseness=1.00] (4) to (0);
		\draw [in=-90, out=30, looseness=1.00] (0) to (3);
	\end{pgfonlayer}
\end{tikzpicture} = \begin{tikzpicture}[yscale=1]
	\begin{pgfonlayer}{nodelayer}
		\node [style=circle] (0) at (0, 4.25) {};
		\node [style=circle, scale=1.5] (1) at (0, 3) {};
		\node [style=circle, scale=1.5] (2) at (0.5, 5.25) {};
		\node [style=circle, scale=1.5] (3) at (-0.5, 5.25) {};
		\node [style=none] (4) at (-0.5, 6) {};
		\node [style=none] (5) at (0.5, 6) {};
		\node [style=none] (6) at (0, 3) {$e_A$};
		\node [style=none] (7) at (-0.5, 5.25) {$s$};
		\node [style=none] (8) at (0.5, 5.25) {$s$};
		\node [style=none] (9) at (0.25, 2.25) {$A$};
		\node [style=none] (10) at (0.25, 3.5) {$A$};
		\node [style=none] (11) at (-0.75, 5.75) {$E$};
		\node [style=none] (12) at (0.75, 5.75) {$E$};
		\node [style=none] (13) at (-0.25, 1.75) {};
		\node [style=none] (14) at (0.25, 1.75) {};
		\node [style=none] (15) at (0, 2) {};
	\end{pgfonlayer}
	\begin{pgfonlayer}{edgelayer}
		\draw (3) to (4.center);
		\draw (2) to (5.center);
		\draw (15.center) to (13.center);
		\draw (13.center) to (14.center);
		\draw (14.center) to (15.center);
		\draw (0) to (1);
		\draw [in=150, out=-90, looseness=1.00] (3) to (0);
		\draw [in=-90, out=30, looseness=1.00] (0) to (2);
		\draw (15.center) to (1);
	\end{pgfonlayer}
\end{tikzpicture} \stackrel{(*)}{=} \begin{tikzpicture}[yscale=1]
	\begin{pgfonlayer}{nodelayer}
		\node [style=circle] (0) at (0, 3) {};
		\node [style=circle, scale=1.5] (1) at (0.5, 5.25) {};
		\node [style=circle, scale=1.5] (2) at (-0.5, 5.25) {};
		\node [style=none] (3) at (-0.5, 6) {};
		\node [style=none] (4) at (0.5, 6) {};
		\node [style=none] (5) at (-0.5, 5.25) {$s$};
		\node [style=none] (6) at (0.5, 5.25) {$s$};
		\node [style=none] (7) at (0.25, 2.25) {$A$};
		\node [style=none] (8) at (-0.75, 5.75) {$E$};
		\node [style=none] (9) at (0.75, 5.75) {$E$};
		\node [style=none] (10) at (-0.25, 1.75) {};
		\node [style=none] (11) at (0.25, 1.75) {};
		\node [style=none] (12) at (0, 2) {};
		\node [style=circle, scale=1.5] (13) at (-0.5, 4.25) {};
		\node [style=none] (14) at (-0.5, 4.25) {$e_A$};
		\node [style=circle, scale=1.5] (15) at (0.5, 4.25) {};
		\node [style=none] (16) at (0.5, 4.25) {$e_A$};
	\end{pgfonlayer}
	\begin{pgfonlayer}{edgelayer}
		\draw (2) to (3.center);
		\draw (1) to (4.center);
		\draw (12.center) to (10.center);
		\draw (10.center) to (11.center);
		\draw (11.center) to (12.center);
		\draw (0) to (12.center);
		\draw [in=-90, out=45, looseness=1.25] (0) to (15);
		\draw [in=-90, out=135, looseness=1.25] (0) to (13);
		\draw (2) to (13);
		\draw (1) to (15);
	\end{pgfonlayer}
\end{tikzpicture}  =\begin{tikzpicture}[yscale=1]
	\begin{pgfonlayer}{nodelayer}
		\node [style=circle] (0) at (0, 3) {};
		\node [style=circle, scale=1.5] (1) at (0.5, 5.25) {};
		\node [style=circle, scale=1.5] (2) at (-0.5, 5.25) {};
		\node [style=none] (3) at (-0.5, 6) {};
		\node [style=none] (4) at (0.5, 6) {};
		\node [style=none] (5) at (-0.5, 5.25) {$s$};
		\node [style=none] (6) at (0.5, 5.25) {$s$};
		\node [style=none] (7) at (0.25, 2.25) {$A$};
		\node [style=none] (8) at (-0.75, 5.75) {$E$};
		\node [style=none] (9) at (0.75, 5.75) {$E$};
		\node [style=none] (10) at (-0.25, 1.75) {};
		\node [style=none] (11) at (0.25, 1.75) {};
		\node [style=none] (12) at (0, 2) {};
	\end{pgfonlayer}
	\begin{pgfonlayer}{edgelayer}
		\draw (2) to (3.center);
		\draw (1) to (4.center);
		\draw (12.center) to (10.center);
		\draw (10.center) to (11.center);
		\draw (11.center) to (12.center);
		\draw (0) to (12.center);
		\draw [in=-90, out=135, looseness=1.00] (0) to (2);
		\draw [in=-90, out=45, looseness=1.00] (0) to (1);
	\end{pgfonlayer}
\end{tikzpicture} = \begin{tikzpicture}[yscale=1]
	\begin{pgfonlayer}{nodelayer}
		\node [style=circle, scale=1.5] (0) at (0.5, 4.5) {};
		\node [style=circle, scale=1.5] (1) at (-0.5, 4.5) {};
		\node [style=none] (2) at (-0.5, 6) {};
		\node [style=none] (3) at (0.5, 6) {};
		\node [style=none] (4) at (-0.5, 4.5) {$s$};
		\node [style=none] (5) at (0.5, 4.5) {$s$};
		\node [style=none] (6) at (-0.25, 2.5) {$A$};
		\node [style=none] (7) at (-0.75, 5.75) {$E$};
		\node [style=none] (8) at (0.75, 5.75) {$E$};
		\node [style=none] (9) at (-0.75, 2) {};
		\node [style=none] (10) at (-0.25, 2) {};
		\node [style=none] (11) at (-0.5, 2.25) {};
		\node [style=none] (12) at (0.75, 2.5) {$A$};
		\node [style=none] (13) at (0.5, 2.25) {};
		\node [style=none] (14) at (0.75, 2) {};
		\node [style=none] (15) at (0.25, 2) {};
	\end{pgfonlayer}
	\begin{pgfonlayer}{edgelayer}
		\draw (1) to (2.center);
		\draw (0) to (3.center);
		\draw (11.center) to (9.center);
		\draw (9.center) to (10.center);
		\draw (10.center) to (11.center);
		\draw (13.center) to (15.center);
		\draw (15.center) to (14.center);
		\draw (14.center) to (13.center);
		\draw (1) to (11.center);
		\draw (0) to (13.center);
	\end{pgfonlayer}
\end{tikzpicture} =: \begin{tikzpicture}[yscale=1]
	\begin{pgfonlayer}{nodelayer}
		\node [style=none] (0) at (-0.5, 6) {};
		\node [style=none] (1) at (0.5, 6) {};
		\node [style=none] (2) at (-0.75, 5.75) {$E$};
		\node [style=none] (3) at (0.75, 5.75) {$E$};
		\node [style=none] (4) at (-0.75, 2) {};
		\node [style=none] (5) at (-0.25, 2) {};
		\node [style=none] (6) at (-0.5, 2.25) {};
		\node [style=none] (7) at (0.5, 2.25) {};
		\node [style=none] (8) at (0.75, 2) {};
		\node [style=none] (9) at (0.25, 2) {};
	\end{pgfonlayer}
	\begin{pgfonlayer}{edgelayer}
		\draw [fill=black] (6.center) -- (4.center) -- (5.center) -- (6.center);
		\draw [fill=black] (7.center) -- (9.center) -- (8.center) -- (7.center);
		\draw (0.center) to (6.center);
		\draw (1.center) to (7.center);
	\end{pgfonlayer}
\end{tikzpicture}$
 
\bigskip

Where the steps labelled $(*)$ are because $e_A$ is sectional on the linear monoid. 

\bigskip

$ \begin{tikzpicture}
	\begin{pgfonlayer}{nodelayer}
		\node [style=circle, fill=black] (0) at (-2, -0.25) {};
		\node [style=none] (1) at (-2, -3) {};
		\node [style=circle, fill=black] (2) at (-0.5, -0.25) {};
		\node [style=none] (3) at (-0.5, -3) {};
		\node [style=none] (4) at (-2, 3) {};
		\node [style=none] (5) at (-2, 1.75) {};
		\node [style=none] (6) at (-2.25, 1.5) {};
		\node [style=none] (7) at (-1.75, 1.5) {};
		\node [style=none] (8) at (-0.5, 3) {};
		\node [style=none] (9) at (-0.5, 1.75) {};
		\node [style=none] (10) at (-0.75, 1.5) {};
		\node [style=none] (11) at (-0.25, 1.5) {};
	\end{pgfonlayer}
	\begin{pgfonlayer}{edgelayer}
		\draw [fill=black] (6.center) -- (7.center) -- (5.center) -- (6.center);
		\draw [fill=black] (10.center) -- (11.center) -- (9.center) -- (10.center);
		\draw [in=30, out=-150, looseness=1.00] (10.center) to (0);
		\draw [bend left=60, looseness=1.00] (0) to (6.center);
		\draw [in=150, out=-54, looseness=1.00] (7.center) to (2);
		\draw [bend left=45, looseness=1.00] (11.center) to (2);
		\draw (0) to (1.center);
		\draw (2) to (3.center);
		\draw (4.center) to (5.center);
		\draw (8.center) to (9.center);
	\end{pgfonlayer}
\end{tikzpicture}  := \begin{tikzpicture}
	\begin{pgfonlayer}{nodelayer}
		\node [style=circle, scale=1.5] (0) at (-2.75, -0.5) {};
		\node [style=circle, scale=1.5] (1) at (-1.5, -0.75) {};
		\node [style=circle] (2) at (-2, -1.5) {};
		\node [style=none] (3) at (-2.75, -0.5) {$s$};
		\node [style=none] (4) at (-1.5, -0.75) {$s$};
		\node [style=circle, scale=1.5] (5) at (-2, -2.25) {};
		\node [style=none] (6) at (-2, -2.25) {$r$};
		\node [style=none] (7) at (-2, -3) {};
		\node [style=circle, scale=1.5] (8) at (0, -0.75) {};
		\node [style=none] (9) at (0, -0.75) {$s$};
		\node [style=circle] (10) at (0.5, -1.5) {};
		\node [style=circle, scale=1.5] (11) at (0.5, -2.25) {};
		\node [style=circle, scale=1.5] (12) at (1.25, -0.5) {};
		\node [style=none] (13) at (1.25, -0.5) {$s$};
		\node [style=none] (14) at (0.5, -3) {};
		\node [style=circle, scale=1.5] (15) at (-2.75, 0.5) {};
		\node [style=none] (16) at (-2.75, 0.5) {$r$};
		\node [style=circle, scale=1.5] (17) at (-2, 2.5) {};
		\node [style=circle, scale=1.5] (18) at (-1.25, 0.75) {};
		\node [style=none] (19) at (-2, 3) {};
		\node [style=none] (20) at (-1.25, 0.75) {$r$};
		\node [style=none] (21) at (-2, 2.5) {$s$};
		\node [style=none] (22) at (-2, 1.75) {};
		\node [style=none] (23) at (-2.25, 1.5) {};
		\node [style=none] (24) at (-1.75, 1.5) {};
		\node [style=circle, scale=1.5] (25) at (-0.25, 0.75) {};
		\node [style=circle, scale=1.5] (26) at (0.5, 2.5) {};
		\node [style=none] (27) at (0.5, 3) {};
		\node [style=none] (28) at (0.5, 1.75) {};
		\node [style=none] (29) at (0.25, 1.5) {};
		\node [style=circle, scale=1.5] (30) at (1.25, 0.5) {};
		\node [style=none] (31) at (1.25, 0.5) {$r$};
		\node [style=none] (32) at (-0.25, 0.75) {$r$};
		\node [style=none] (33) at (0.5, 2.5) {$s$};
		\node [style=none] (34) at (0.75, 1.5) {};
		\node [style=none] (35) at (0.5, -2.25) {$r$};
	\end{pgfonlayer}
	\begin{pgfonlayer}{edgelayer}
		\draw [in=-90, out=15, looseness=1.25] (2) to (1);
		\draw [in=-90, out=165, looseness=1.25] (2) to (0);
		\draw (7.center) to (5);
		\draw (5) to (2);
		\draw [in=-90, out=15, looseness=1.25] (10) to (12);
		\draw [in=-90, out=165, looseness=1.25] (10) to (8);
		\draw (14.center) to (11);
		\draw (11) to (10);
		\draw (19.center) to (17);
		\draw (23.center) to (24.center);
		\draw (24.center) to (22.center);
		\draw (22.center) to (23.center);
		\draw [in=90, out=-165, looseness=1.25] (23.center) to (15);
		\draw [in=90, out=-15, looseness=1.00] (24.center) to (18);
		\draw (22.center) to (17);
		\draw (27.center) to (26);
		\draw (29.center) to (34.center);
		\draw (34.center) to (28.center);
		\draw (28.center) to (29.center);
		\draw [in=90, out=-165, looseness=1.25] (29.center) to (25);
		\draw [in=90, out=-15, looseness=1.25] (34.center) to (30);
		\draw (28.center) to (26);
		\draw (0) to (15);
		\draw (12) to (30);
		\draw [in=-105, out=90, looseness=1.00] (1) to (25);
		\draw [in=-105, out=90, looseness=1.00] (8) to (18);
	\end{pgfonlayer}
\end{tikzpicture} = \begin{tikzpicture}
	\begin{pgfonlayer}{nodelayer}
		\node [style=circle, scale=1.5] (0) at (-2.75, -0.5) {};
		\node [style=circle, scale=1.5] (1) at (-1.5, -0.5) {};
		\node [style=circle] (2) at (-2, -1.5) {};
		\node [style=none] (3) at (-2.75, -0.5) {$e_A$};
		\node [style=none] (4) at (-1.5, -0.5) {$e_A$};
		\node [style=circle, scale=1.5] (5) at (-2, -2.25) {};
		\node [style=none] (6) at (-2, -2.25) {$r$};
		\node [style=none] (7) at (-2, -3) {};
		\node [style=circle, scale=1.5] (8) at (-1, -0.75) {};
		\node [style=none] (9) at (-1, -0.75) {$e_A$};
		\node [style=circle] (10) at (-0.5, -1.5) {};
		\node [style=circle, scale=1.5] (11) at (-0.5, -2.25) {};
		\node [style=circle, scale=1.5] (12) at (0.25, -0.5) {};
		\node [style=none] (13) at (0.25, -0.5) {$e_A$};
		\node [style=none] (14) at (-0.5, -3) {};
		\node [style=circle, scale=1.5] (15) at (-2, 2.5) {};
		\node [style=none] (16) at (-2, 3) {};
		\node [style=none] (17) at (-2, 2.5) {$s$};
		\node [style=none] (18) at (-2, 1.75) {};
		\node [style=none] (19) at (-2.25, 1.5) {};
		\node [style=none] (20) at (-1.75, 1.5) {};
		\node [style=circle, scale=1.5] (21) at (-0.5, 2.5) {};
		\node [style=none] (22) at (-0.5, 3) {};
		\node [style=none] (23) at (-0.5, 1.75) {};
		\node [style=none] (24) at (-0.75, 1.5) {};
		\node [style=none] (25) at (-0.5, 2.5) {$s$};
		\node [style=none] (26) at (-0.25, 1.5) {};
		\node [style=none] (27) at (-0.5, -2.25) {$r$};
	\end{pgfonlayer}
	\begin{pgfonlayer}{edgelayer}
		\draw [in=-90, out=15, looseness=1.25] (2) to (1);
		\draw [in=-90, out=165, looseness=1.25] (2) to (0);
		\draw (7.center) to (5);
		\draw (5) to (2);
		\draw [in=-90, out=15, looseness=1.25] (10) to (12);
		\draw [in=-90, out=165, looseness=1.25] (10) to (8);
		\draw (14.center) to (11);
		\draw (11) to (10);
		\draw (16.center) to (15);
		\draw (19.center) to (20.center);
		\draw (20.center) to (18.center);
		\draw (18.center) to (19.center);
		\draw (18.center) to (15);
		\draw (22.center) to (21);
		\draw (24.center) to (26.center);
		\draw (26.center) to (23.center);
		\draw (23.center) to (24.center);
		\draw (23.center) to (21);
		\draw [in=90, out=-105, looseness=0.75] (24.center) to (1);
		\draw [in=90, out=-75, looseness=1.25] (20.center) to (8);
		\draw [in=90, out=-45] (26.center) to (12);
		\draw [in=90, out=-135] (19.center) to (0);
	\end{pgfonlayer}
\end{tikzpicture} =\begin{tikzpicture}
	\begin{pgfonlayer}{nodelayer}
		\node [style=circle, scale=1.5] (0) at (-2.5, -0.75) {};
		\node [style=circle, scale=1.5] (1) at (-1.5, -0.75) {};
		\node [style=circle] (2) at (-2, -1.5) {};
		\node [style=none] (3) at (-2.5, -0.75) {$e_A$};
		\node [style=none] (4) at (-1.5, -0.75) {$e_A$};
		\node [style=circle, scale=1.5] (5) at (-2, -2.25) {};
		\node [style=none] (6) at (-2, -2.25) {$r$};
		\node [style=none] (7) at (-2, -3) {};
		\node [style=circle, scale=1.5] (8) at (-0.75, -0.75) {};
		\node [style=none] (9) at (-0.75, -0.75) {$e_A$};
		\node [style=circle] (10) at (-0.25, -1.5) {};
		\node [style=circle, scale=1.5] (11) at (-0.25, -2.25) {};
		\node [style=circle, scale=1.5] (12) at (0.25, -0.75) {};
		\node [style=none] (13) at (0.25, -0.75) {$e_A$};
		\node [style=none] (14) at (-0.25, -3) {};
		\node [style=circle, scale=1.5] (15) at (-2, 2.25) {};
		\node [style=none] (16) at (-2, 3) {};
		\node [style=none] (17) at (-2, 2.25) {$s$};
		\node [style=none] (18) at (-2, 0.5) {};
		\node [style=none] (19) at (-2.25, 0.25) {};
		\node [style=none] (20) at (-1.75, 0.25) {};
		\node [style=circle, scale=1.5] (21) at (-0.25, 2.25) {};
		\node [style=none] (22) at (-0.25, 3) {};
		\node [style=none] (23) at (-0.25, 0.5) {};
		\node [style=none] (24) at (-0.5, 0.25) {};
		\node [style=none] (25) at (-0.25, 2.25) {$s$};
		\node [style=none] (26) at (0, 0.25) {};
		\node [style=none] (27) at (-0.25, -2.25) {$r$};
		\node [style=circle, scale=1.5] (28) at (-2, 1.25) {};
		\node [style=circle, scale=1.5] (29) at (-0.25, 1.25) {};
		\node [style=none] (30) at (-2, 1.25) {$e_A$};
		\node [style=none] (31) at (-0.25, 1.25) {$e_A$};
	\end{pgfonlayer}
	\begin{pgfonlayer}{edgelayer}
		\draw [in=-90, out=15, looseness=1.25] (2) to (1);
		\draw [in=-90, out=165, looseness=1.25] (2) to (0);
		\draw (7.center) to (5);
		\draw (5) to (2);
		\draw [in=-90, out=15, looseness=1.25] (10) to (12);
		\draw [in=-90, out=165, looseness=1.25] (10) to (8);
		\draw (14.center) to (11);
		\draw (11) to (10);
		\draw (16.center) to (15);
		\draw (19.center) to (20.center);
		\draw (20.center) to (18.center);
		\draw (18.center) to (19.center);
		\draw (22.center) to (21);
		\draw (24.center) to (26.center);
		\draw (26.center) to (23.center);
		\draw (23.center) to (24.center);
		\draw [in=90, out=-135] (24.center) to (1);
		\draw [in=90, out=-45] (20.center) to (8);
		\draw [in=90, out=-60] (26.center) to (12);
		\draw [in=90, out=-120, looseness=1.25] (19.center) to (0);
		\draw (18.center) to (28);
		\draw (28) to (15);
		\draw (23.center) to (29);
		\draw (29) to (21);
	\end{pgfonlayer}
\end{tikzpicture}
\stackrel{(1)}{=}  \begin{tikzpicture}
	\begin{pgfonlayer}{nodelayer}
		\node [style=circle] (2) at (-2, -1.5) {};
		\node [style=circle, scale=1.5] (5) at (-2, -2.25) {};
		\node [style=none] (6) at (-2, -2.25) {$r$};
		\node [style=none] (7) at (-2, -3) {};
		\node [style=circle] (10) at (-0.5, -1.5) {};
		\node [style=circle, scale=1.5] (11) at (-0.5, -2.25) {};
		\node [style=none] (14) at (-0.5, -3) {};
		\node [style=circle, scale=1.5] (15) at (-2, 2.25) {};
		\node [style=none] (16) at (-2, 3) {};
		\node [style=none] (17) at (-2, 2.25) {$s$};
		\node [style=none] (18) at (-2, 0.5) {};
		\node [style=none] (19) at (-2.25, 0.25) {};
		\node [style=none] (20) at (-1.75, 0.25) {};
		\node [style=circle, scale=1.5] (21) at (-0.5, 2.25) {};
		\node [style=none] (22) at (-0.5, 3) {};
		\node [style=none] (23) at (-0.5, 0.5) {};
		\node [style=none] (24) at (-0.75, 0.25) {};
		\node [style=none] (25) at (-0.5, 2.25) {$s$};
		\node [style=none] (26) at (-0.25, 0.25) {};
		\node [style=none] (27) at (-0.5, -2.25) {$r$};
		\node [style=circle, scale=1.5] (28) at (-2, 1.25) {};
		\node [style=circle, scale=1.5] (29) at (-0.5, 1.25) {};
		\node [style=none] (30) at (-2, 1.25) {$e_A$};
		\node [style=none] (31) at (-0.5, 1.25) {$e_A$};
	\end{pgfonlayer}
	\begin{pgfonlayer}{edgelayer}
		\draw (7.center) to (5);
		\draw (5) to (2);
		\draw (14.center) to (11);
		\draw (11) to (10);
		\draw (16.center) to (15);
		\draw (19.center) to (20.center);
		\draw (20.center) to (18.center);
		\draw (18.center) to (19.center);
		\draw (22.center) to (21);
		\draw (24.center) to (26.center);
		\draw (26.center) to (23.center);
		\draw (23.center) to (24.center);
		\draw (18.center) to (28);
		\draw (28) to (15);
		\draw (23.center) to (29);
		\draw (29) to (21);
		\draw [bend right] (10) to (26.center);
		\draw [bend right] (19.center) to (2);
		\draw (2) to (24.center);
		\draw (20.center) to (10);
	\end{pgfonlayer}
\end{tikzpicture} 
 = \begin{tikzpicture}
	\begin{pgfonlayer}{nodelayer}
		\node [style=circle, scale=1.5] (6) at (-2, 1.25) {};
		\node [style=none] (8) at (-2, 1.25) {$e_A$};
		\node [style=circle, scale=1.5] (9) at (-0.5, 1.25) {};
		\node [style=none] (11) at (-0.5, 1.25) {$e_A$};
		\node [style=circle] (0) at (-1.25, 0.25) {};
		\node [style=none] (3) at (-2, -3) {};
		\node [style=none] (5) at (-0.5, -3) {};
		\node [style=none] (7) at (-2, 3) {};
		\node [style=none] (10) at (-0.5, 3) {};
		\node [style=circle, scale=1.5] (13) at (-0.5, -2.25) {};
		\node [style=none] (14) at (-0.5, -2.25) {$r$};
		\node [style=circle, scale=1.5] (15) at (-2, -2.25) {};
		\node [style=none] (16) at (-2, -2.25) {$r$};
		\node [style=none] (17) at (-1.25, -0.75) {};
		\node [style=none] (18) at (-1, -1) {};
		\node [style=none] (19) at (-1.5, -1) {};
		\node [style=circle, scale=1.5] (20) at (-0.5, 2.25) {};
		\node [style=none] (21) at (-0.5, 2.25) {$s$};
		\node [style=circle, scale=1.5] (22) at (-2, 2.25) {};
		\node [style=none] (23) at (-2, 2.25) {$s$};
	\end{pgfonlayer}
	\begin{pgfonlayer}{edgelayer}
		\draw (19.center) to (18.center);
		\draw (18.center) to (17.center);
		\draw (17.center) to (19.center);
		\draw [in=90, out=-150] (19.center) to (15);
		\draw [in=90, out=-30] (18.center) to (13);
		\draw (17.center) to (0);
		\draw [in=-90, out=30] (0) to (9);
		\draw [in=-90, out=150] (0) to (6);
		\draw (3.center) to (15);
		\draw (5.center) to (13);
		\draw (9) to (20);
		\draw (6) to (22);
		\draw (20) to (10.center);
		\draw (22) to (7.center);
	\end{pgfonlayer}
\end{tikzpicture} \stackrel{(2)}{=}  \begin{tikzpicture}
	\begin{pgfonlayer}{nodelayer}
		\node [style=circle, scale=1.5] (6) at (-2, 1.25) {};
		\node [style=none] (8) at (-2, 1.25) {$e_A$};
		\node [style=circle, scale=1.5] (9) at (-0.5, 1.25) {};
		\node [style=none] (11) at (-0.5, 1.25) {$e_A$};
		\node [style=circle] (0) at (-1.25, 0.25) {};
		\node [style=none] (3) at (-2, -3) {};
		\node [style=none] (5) at (-0.5, -3) {};
		\node [style=none] (7) at (-2, 3) {};
		\node [style=none] (10) at (-0.5, 3) {};
		\node [style=circle, scale=1.5] (13) at (-0.5, -2.25) {};
		\node [style=none] (14) at (-0.5, -2.25) {$r$};
		\node [style=circle, scale=1.5] (15) at (-2, -2.25) {};
		\node [style=none] (16) at (-2, -2.25) {$r$};
		\node [style=none] (17) at (-1.25, -1) {};
		\node [style=none] (18) at (-1, -1.25) {};
		\node [style=none] (19) at (-1.5, -1.25) {};
		\node [style=circle, scale=1.5] (20) at (-0.5, 2.25) {};
		\node [style=none] (21) at (-0.5, 2.25) {$s$};
		\node [style=circle, scale=1.5] (22) at (-2, 2.25) {};
		\node [style=none] (23) at (-2, 2.25) {$s$};
		\node [style=circle, scale=1.5] (25) at (-1.25, -0.5) {};
		\node [style=none] (24) at (-1.25, -0.5) {$e_A$};
	\end{pgfonlayer}
	\begin{pgfonlayer}{edgelayer}
		\draw (19.center) to (18.center);
		\draw (18.center) to (17.center);
		\draw (17.center) to (19.center);
		\draw [in=90, out=-150] (19.center) to (15);
		\draw [in=90, out=-30] (18.center) to (13);
		\draw [in=-90, out=30] (0) to (9);
		\draw [in=-90, out=150] (0) to (6);
		\draw (3.center) to (15);
		\draw (5.center) to (13);
		\draw (9) to (20);
		\draw (6) to (22);
		\draw (20) to (10.center);
		\draw (22) to (7.center);
		\draw (17.center) to (25);
		\draw (25) to (0);
	\end{pgfonlayer}
\end{tikzpicture} = \begin{tikzpicture}
	\begin{pgfonlayer}{nodelayer}
		\node [style=circle, fill=black] (0) at (-1.25, 0.75) {};
		\node [style=none] (1) at (-2, -3) {};
		\node [style=none] (2) at (-0.5, -3) {};
		\node [style=none] (3) at (-2, 3) {};
		\node [style=none] (4) at (-0.5, 3) {};
		\node [style=none] (5) at (-1.25, -0.5) {};
		\node [style=none] (6) at (-1, -0.75) {};
		\node [style=none] (7) at (-1.5, -0.75) {};
	\end{pgfonlayer}
	\begin{pgfonlayer}{edgelayer}
		\draw [fill=black] (7.center) -- (6.center) -- (5.center) -- (7.center);
		\draw (0) to (5.center);
		\draw [in=-90, out=30, looseness=1.00] (0) to (4.center);
		\draw [in=-90, out=150, looseness=1.00] (0) to (3.center);
		\draw [in=-120, out=90, looseness=1.00] (1.center) to (7.center);
		\draw [in=90, out=-45, looseness=0.75] (6.center) to (2.center);
	\end{pgfonlayer}
\end{tikzpicture} $

\bigskip

Where step $(1)$  is because $e_A$ is sectional on the linear comonoid, and step $(2)$ is true because $e_A$ 
is sectional on the linear monoid respectively.
Similarly, $E'$ is a $\oa$-bialagebra. 
The converse is follows from Statement $(i)$ of Lemma \ref{Lemma: split weak linear mon hom} and 
\ref{Lemma: split weak linear comon hom}.
\end{proof}

The above Lemma extends directly to $\dagger$-binary idempotents on $\dag$-linear bialgebras. 

%%%%%%%%%%%%%%%%%%%%%%%%%%%%%%%%%%%%%%%%%%%%%%%%%%%%%%%%

%\section{complementarity}

%We first introduce linear bialgebras followed by the conditions under which a linear bialgebra 
%behaves as a complementary system. 

\section{Complementary linear bialgebras} 
\label{Sec: complementary systems}

In categorical quantum mechanics, Bohr's complementary principle is described using interacting 
commutative $\dagger$-Frobenius algebras in $\dagger$-monoidal categories. 
The aim of this section is to describe complementarity in isomix categories, for which, minimally, 
we require a self-linear bialgebra satisfying the equations described below.  

\begin{definition}
A {\bf complementary system} in an isomix category, $\X$, is a commutative and cocommutative 
self-linear bialgebra, $\frac{(a,b)}{(a',b')}\!\!:\! A \linbialgwtik A$ such that the following equations 
(with their `op' symmetries) hold:
\[ \mbox{ \small \bf [comp.1]} ~~~ \begin{tikzpicture}
	\begin{pgfonlayer}{nodelayer}
		\node [style=none] (20) at (6, 4.25) {};
		\node [style=none] (21) at (6.25, 4.5) {};
		\node [style=none] (22) at (5.75, 4.5) {};
		\node [style=none] (23) at (6, 3.5) {};
		\node [style=none] (24) at (6, 3.5) {};
		\node [style=circle] (26) at (6, 2.5) {};
		\node [style=none] (27) at (5.75, 3.5) {};
		\node [style=none] (28) at (5, 4.5) {};
		\node [style=none] (30) at (6, 3.75) {};
		\node [style=none] (31) at (6, 3.25) {};
	\end{pgfonlayer}
	\begin{pgfonlayer}{edgelayer}
		\draw (20.center) to (21.center);
		\draw (20.center) to (22.center);
		\draw (22.center) to (21.center);
		\draw (20.center) to (23.center);
		\draw (24.center) to (26);
		\draw [in=-90, out=-180, looseness=1.25] (27.center) to (28.center);
		\draw [bend right=90, looseness=1.75] (30.center) to (31.center);
	\end{pgfonlayer}
\end{tikzpicture}  = \begin{tikzpicture}
	\begin{pgfonlayer}{nodelayer}
		\node [style=none] (0) at (0.75, 3) {};
		\node [style=none] (1) at (1, 2.75) {};
		\node [style=none] (2) at (0.5, 2.75) {};
		\node [style=none] (3) at (0.75, 4.5) {};
	\end{pgfonlayer}
	\begin{pgfonlayer}{edgelayer}
		\draw (0.center) to (1.center);
		\draw (0.center) to (2.center);
		\draw (2.center) to (1.center);
		\draw (0.center) to (3.center);
	\end{pgfonlayer}
\end{tikzpicture} 
~~~~~~~~	
\mbox{ \small \bf [comp.2]} ~~~ \begin{tikzpicture}
		\begin{pgfonlayer}{nodelayer}
			\node [style=none] (0) at (-1.75, 4.5) {};
			\node [style=none] (1) at (-1.25, 4.5) {};
			\node [style=none] (2) at (-1.5, 4.25) {};
			\node [style=none] (3) at (-1.5, 3.75) {};
			\node [style=none] (4) at (-1.5, 3.75) {};
			\node [style=none] (5) at (-1.75, 3.5) {};
			\node [style=none] (6) at (-1.25, 3.5) {};
			\node [style=circle] (7) at (-0.75, 2.5) {};
			\node [style=none] (8) at (-2.25, 2.5) {};
		\end{pgfonlayer}
		\begin{pgfonlayer}{edgelayer}
			\draw (2.center) to (0.center);
			\draw (2.center) to (1.center);
			\draw (1.center) to (0.center);
			\draw (3.center) to (2.center);
			\draw (4.center) to (5.center);
			\draw (4.center) to (6.center);
			\draw (6.center) to (5.center);
			\draw [in=180, out=90, looseness=1.00] (8.center) to (5.center);
			\draw [in=90, out=-15, looseness=1.25] (6.center) to (7);
		\end{pgfonlayer}
	\end{tikzpicture} =  \begin{tikzpicture}
		\begin{pgfonlayer}{nodelayer}
			\node [style=none] (0) at (-0.75, 2.5) {};
			\node [style=circle] (1) at (-0.75, 4.5) {};
		\end{pgfonlayer}
		\begin{pgfonlayer}{edgelayer}
			\draw (0.center) to (1);
		\end{pgfonlayer}
	\end{tikzpicture} 
~~~~~~~~ 	\mbox{ \small \bf [comp.3]} ~~~ 
\begin{tikzpicture}
	\begin{pgfonlayer}{nodelayer}
		\node [style=none] (0) at (2.75, 0.5) {};
		\node [style=none] (1) at (3, 0.75) {};
		\node [style=none] (2) at (2.5, 0.75) {};
		\node [style=none] (3) at (2.75, -0.25) {};
		\node [style=none] (4) at (2.75, 0) {};
		\node [style=none] (5) at (2.5, -0.25) {};
		\node [style=none] (6) at (2.75, -1.25) {};
		\node [style=none] (7) at (1.75, -1.25) {};
		\node [style=none] (8) at (2.75, 0) {};
		\node [style=none] (9) at (2.75, -0.5) {};
	\end{pgfonlayer}
	\begin{pgfonlayer}{edgelayer}
		\draw (0.center) to (1.center);
		\draw (0.center) to (2.center);
		\draw (2.center) to (1.center);
		\draw (0.center) to (3.center);
		\draw (6.center) to (3.center);
		\draw [in=90, out=-180, looseness=1.50] (5.center) to (7.center);
		\draw [bend right=90, looseness=1.75] (8.center) to (9.center);
	\end{pgfonlayer}
\end{tikzpicture}
 = \begin{tikzpicture}
		\begin{pgfonlayer}{nodelayer}
			\node [style=none] (0) at (1.75, 2.5) {};
			\node [style=none] (1) at (2.25, 2.5) {};
			\node [style=none] (2) at (2, 2.25) {};
			\node [style=none] (3) at (2, 0.5) {};
			\node [style=none] (4) at (2.75, 2.25) {};
			\node [style=none] (5) at (2.75, 0.5) {};
			\node [style=none] (6) at (2.5, 2.5) {};
			\node [style=none] (7) at (3, 2.5) {};
		\end{pgfonlayer}
		\begin{pgfonlayer}{edgelayer}
			\draw[fill=white] (1.center) -- (0.center) -- (2.center) -- (1.center);
			\draw (3.center) to (2.center);
			\draw (4.center) to (6.center);
			\draw (4.center) to (7.center);
			\draw (7.center) to (6.center);
			\draw (5.center) to (4.center);
		\end{pgfonlayer}
	\end{tikzpicture} \]
	A {\bf $\dagger$-complementary system} in a $\dagger$-isomix category is a $\dagger$-self-linear bialgebra which 
	is also a complementary system. 
\end{definition}

By `op' symmetry, we refer to the vertical reflection of the diagrams. The `co' symmetry 
(horizontal reflection) of the equations are immediate 
from the commutativity and cocommutativity of the linear bialgebra. 

Notice that we are using the alternate presentation of linear monoids by actions and coactions (see 
Proposition \ref{Lemma: alternate presentation of linear monoids}).  Thus, {\bf [comp.1]} requires that the counit of the  linear comonoid is
 dual to the counit via the linear monoid dual, while {\bf [comp.2]} requires that the unit of the linear monoid is dual to the 
counit via dual of the linear comonoid.  Finally, {\bf [comp.3] } requires that the coaction map of the linear monoid 
 duplicates the unit of $\dagger$-linear comonoid.  The `op' symmetry of equations 
 {\bf [comp.1]}-{\bf [comp.3]} hold automatically for a $\dagger$-complementary system. 

 In a monoidal category, a  complementary system is given by a pair of commutative 
 and cocommutative Frobenius algebras, $\Frob{A}$ and $\bFrob{A}$, interacting to 
 produce pair of Hopf Algebras with the following antipodes \cite[Definition 6.3]{HeV19}. 
 \[ \begin{tikzpicture}
	\begin{pgfonlayer}{nodelayer}
		\node [style=circle] (0) at (17, 0.25) {};
		\node [style=circle] (1) at (17, 1) {};
		\node [style=black] (2) at (17, -1.25) {};
		\node [style=black] (3) at (17, -2) {};
		\node [style=none] (4) at (16, -2) {};
		\node [style=none] (5) at (16, 1) {};
	\end{pgfonlayer}
	\begin{pgfonlayer}{edgelayer}
		\draw (1) to (0);
		\draw (2) to (3);
		\draw [bend left=60, looseness=1.25] (0) to (2);
		\draw [in=90, out=-135] (0) to (4.center);
		\draw [in=-90, out=135] (2) to (5.center);
	\end{pgfonlayer}
\end{tikzpicture} ~~~~~~~~~~~~ \begin{tikzpicture}
	\begin{pgfonlayer}{nodelayer}
		\node [style=circle] (0) at (17, -1.25) {};
		\node [style=circle] (1) at (17, -2) {};
		\node [style=black] (2) at (17, 0.25) {};
		\node [style=black] (3) at (17, 1) {};
		\node [style=none] (4) at (16, 1) {};
		\node [style=none] (5) at (16, -2) {};
	\end{pgfonlayer}
	\begin{pgfonlayer}{edgelayer}
		\draw (1) to (0);
		\draw (2) to (3);
		\draw [bend right=60, looseness=1.25] (0) to (2);
		\draw [in=-90, out=135] (0) to (4.center);
		\draw [in=90, out=-135] (2) to (5.center);
	\end{pgfonlayer}
\end{tikzpicture} \]
 In \cite[Theorem 6.4 ]{DuK16}, Duncan and Dunne proved that 
 a pair of commutative $\dagger$-Frobenius Algebras 
 is a complementary system if and only if the following conditions 
 hold: 
 \[ \begin{tikzpicture}
	\begin{pgfonlayer}{nodelayer}
		\node [style=black] (0) at (17.5, -1) {};
		\node [style=circle] (2) at (17, 0.25) {};
		\node [style=circle] (3) at (17, 1) {};
		\node [style=none] (5) at (16.5, -1) {};
	\end{pgfonlayer}
	\begin{pgfonlayer}{edgelayer}
		\draw (2) to (3);
		\draw [in=315, out=90] (0) to (2);
		\draw [in=90, out=-135] (2) to (5.center);
	\end{pgfonlayer}
\end{tikzpicture} = \begin{tikzpicture}
	\begin{pgfonlayer}{nodelayer}
		\node [style=black] (3) at (17, 1) {};
		\node [style=none] (5) at (17, -1) {};
	\end{pgfonlayer}
	\begin{pgfonlayer}{edgelayer}
		\draw (3) to (5.center);
	\end{pgfonlayer}
\end{tikzpicture} 
~~~~~~~~~~~~
\begin{tikzpicture}
	\begin{pgfonlayer}{nodelayer}
		\node [style=circle] (0) at (17.5, -1) {};
		\node [style=black] (2) at (17, 0.25) {};
		\node [style=black] (3) at (17, 1) {};
		\node [style=none] (5) at (16.5, -1) {};
	\end{pgfonlayer}
	\begin{pgfonlayer}{edgelayer}
		\draw (2) to (3);
		\draw [in=315, out=90] (0) to (2);
		\draw [in=90, out=-135] (2) to (5.center);
	\end{pgfonlayer}
\end{tikzpicture} = \begin{tikzpicture}
	\begin{pgfonlayer}{nodelayer}
		\node [style=circle] (3) at (17, 1) {};
		\node [style=none] (5) at (17, -1) {};
	\end{pgfonlayer}
	\begin{pgfonlayer}{edgelayer}
		\draw (3) to (5.center);
	\end{pgfonlayer}
\end{tikzpicture}   \]
Note that the `op' symmetry of {\bf [comp.1]} is similar to the equation on the left 
and {\bf [comp.2]} is similar to the equation on the right.
In the following lemma, we prove that in an isomix category, the 
$\ox$ and the $\oa$-bialgebras of a complementary system are Hopf: 

\begin{lemma}
	If $A \linbialgwtik A$ is a complementary system in an isomix category,  then
	A  is $\ox$-bialgebra with antipode given by $(a)$ and $\oa$-bialgebra with antipode 
	given by  $(b)$: 
	\[ 	(a) ~~~~ \begin{tikzpicture} [scale=0.75]
		\begin{pgfonlayer}{nodelayer}
			\node [style=none] (0) at (1, 4.75) {};
			\node [style=none] (1) at (1.5, 4.75) {};
			\node [style=circle] (2) at (1.25, 2.75) {};
			\node [style=none] (3) at (1.25, 5) {};
			\node [style=none] (4) at (1.5, 5.75) {};
			\node [style=none] (5) at (1, 5.75) {};
			\node [style=none] (6) at (1.25, 5.5) {};
			\node [style=circle] (7) at (1.25, 1.75) {};
			\node [style=none] (8) at (0.25, 1.75) {};
			\node [style=none] (9) at (0.25, 5.75) {};
		\end{pgfonlayer}
		\begin{pgfonlayer}{edgelayer}
			\draw (1.center) to (0.center);
			\draw (0.center) to (3.center);
			\draw (3.center) to (1.center);
			\draw [bend left=45, looseness=1.00] (1.center) to (2);
			\draw (4.center) to (5.center);
			\draw (5.center) to (6.center);
			\draw (6.center) to (4.center);
			\draw (3.center) to (6.center);
			\draw (7) to (2);
			\draw [in=90, out=-135, looseness=1.00] (0.center) to (8.center);
			\draw [in=-90, out=135, looseness=1.00] (2) to (9.center);
		\end{pgfonlayer}
	\end{tikzpicture}
	~~~~~~~~
	(b)~~~ \begin{tikzpicture}  [scale=0.75]
		\begin{pgfonlayer}{nodelayer}
			\node [style=none] (0) at (1, 2.75) {};
			\node [style=none] (1) at (1.5, 2.75) {};
			\node [style=circle] (2) at (1.25, 4.75) {};
			\node [style=none] (3) at (1.25, 2.5) {};
			\node [style=none] (4) at (1.5, 1.75) {};
			\node [style=none] (5) at (1, 1.75) {};
			\node [style=none] (6) at (1.25, 2) {};
			\node [style=circle] (7) at (1.25, 5.75) {};
			\node [style=none] (8) at (0.25, 5.75) {};
			\node [style=none] (9) at (0.25, 1.75) {};
		\end{pgfonlayer}
		\begin{pgfonlayer}{edgelayer}
			\draw (1.center) to (0.center);
			\draw (0.center) to (3.center);
			\draw (3.center) to (1.center);
			\draw [bend right=45, looseness=1.00] (1.center) to (2);
			\draw (4.center) to (5.center);
			\draw (5.center) to (6.center);
			\draw (6.center) to (4.center);
			\draw (3.center) to (6.center);
			\draw (7) to (2);
			\draw [in=-90, out=135, looseness=0.75] (0.center) to (8.center);
			\draw [in=90, out=-150, looseness=1.00] (2) to (9.center);
		\end{pgfonlayer}
	\end{tikzpicture}  \]
	\end{lemma}
	\begin{proof}
	Given a complementary system we show the $\ox$-bialgebra has an antipode so is a $\ox$-Hopf algebra:
		
		\bigskip

	\begin{tikzpicture}
		\begin{pgfonlayer}{nodelayer}
			\node [style=none] (0) at (0, 5) {};
			\node [style=none] (1) at (-0.25, 4.75) {};
			\node [style=none] (2) at (0.25, 4.75) {};
			\node [style=circle] (3) at (0, 2.75) {};
			\node [style=circle, scale=1.5] (4) at (0.5, 3.75) {};
			\node [style=none] (5) at (0.5, 3.75) {$s$};
			\node [style=none] (6) at (0, 5.75) {};
			\node [style=none] (7) at (0, 5.75) {};
			\node [style=none] (8) at (0, 1.75) {};
		\end{pgfonlayer}
		\begin{pgfonlayer}{edgelayer}
			\draw (1.center) to (2.center);
			\draw (2.center) to (0.center);
			\draw (0.center) to (1.center);
			\draw [bend right=45, looseness=1.00] (1.center) to (3);
			\draw [in=-90, out=45, looseness=1.00] (3) to (4);
			\draw [bend left=15, looseness=1.00] (2.center) to (4);
			\draw (6.center) to (0.center);
			\draw (3) to (8.center);
		\end{pgfonlayer}
	\end{tikzpicture} $:=$ \begin{tikzpicture}
		\begin{pgfonlayer}{nodelayer}
			\node [style=none] (0) at (0, 5) {};
			\node [style=none] (1) at (-0.25, 4.75) {};
			\node [style=none] (2) at (0.25, 4.75) {};
			\node [style=circle] (3) at (0, 2.75) {};
			\node [style=none] (4) at (0, 5.75) {};
			\node [style=none] (5) at (0, 5.75) {};
			\node [style=none] (6) at (0, 1.75) {};
			\node [style=none] (7) at (1, 4.75) {};
			\node [style=none] (8) at (1.5, 4.75) {};
			\node [style=circle] (9) at (1.25, 2.75) {};
			\node [style=none] (10) at (1.25, 5) {};
			\node [style=none] (11) at (1.5, 5.75) {};
			\node [style=none] (12) at (1, 5.75) {};
			\node [style=none] (13) at (1.25, 5.5) {};
			\node [style=circle] (14) at (1.25, 1.75) {};
		\end{pgfonlayer}
		\begin{pgfonlayer}{edgelayer}
			\draw (1.center) to (2.center);
			\draw (2.center) to (0.center);
			\draw (0.center) to (1.center);
			\draw [bend right=45, looseness=1.00] (1.center) to (3);
			\draw (4.center) to (0.center);
			\draw (3) to (6.center);
			\draw (8.center) to (7.center);
			\draw (7.center) to (10.center);
			\draw (10.center) to (8.center);
			\draw [bend left=45, looseness=1.00] (8.center) to (9);
			\draw (11.center) to (12.center);
			\draw (12.center) to (13.center);
			\draw (13.center) to (11.center);
			\draw (10.center) to (13.center);
			\draw (14) to (9);
			\draw (7.center) to (3);
			\draw (2.center) to (9);
		\end{pgfonlayer}
	\end{tikzpicture} $=$ \begin{tikzpicture}
		\begin{pgfonlayer}{nodelayer}
			\node [style=none] (0) at (0, 3.25) {};
			\node [style=none] (1) at (-0.25, 3) {};
			\node [style=none] (2) at (0.25, 3) {};
			\node [style=circle] (3) at (0, 4) {};
			\node [style=none] (4) at (0, 3.25) {};
			\node [style=none] (5) at (-0.75, 5) {};
			\node [style=circle] (6) at (0.75, 2) {};
			\node [style=none] (7) at (-0.75, 2) {};
			\node [style=none] (8) at (0.5, 5) {};
			\node [style=none] (9) at (1, 5) {};
			\node [style=none] (10) at (0.75, 4.75) {};
		\end{pgfonlayer}
		\begin{pgfonlayer}{edgelayer}
			\draw (1.center) to (2.center);
			\draw (2.center) to (0.center);
			\draw (0.center) to (1.center);
			\draw (3) to (4.center);
			\draw (10.center) to (8.center);
			\draw (8.center) to (9.center);
			\draw (9.center) to (10.center);
			\draw [in=150, out=-90, looseness=1.00] (5.center) to (3);
			\draw [in=-90, out=30, looseness=1.00] (3) to (10.center);
			\draw [in=90, out=-150, looseness=0.75] (1.center) to (7.center);
			\draw [in=90, out=-30, looseness=1.25] (2.center) to (6);
		\end{pgfonlayer}
	\end{tikzpicture} $=$\begin{tikzpicture}
		\begin{pgfonlayer}{nodelayer}
			\node [style=none] (0) at (0, 3.25) {};
			\node [style=none] (1) at (-0.25, 3) {};
			\node [style=none] (2) at (0.25, 3) {};
			\node [style=circle] (3) at (0, 4.5) {};
			\node [style=none] (4) at (0, 3.25) {};
			\node [style=none] (5) at (-0.75, 5.5) {};
			\node [style=circle] (6) at (0.75, 2) {};
			\node [style=none] (7) at (-0.75, 2) {};
			\node [style=none] (8) at (0.5, 5.5) {};
			\node [style=none] (9) at (1, 5.5) {};
			\node [style=none] (10) at (0.75, 5.25) {};
			\node [style=none] (11) at (0, 4) {};
			\node [style=none] (12) at (0, 3.75) {};
			\node [style=none] (13) at (-0.25, 3.75) {};
			\node [style=circle] (14) at (-1.5, 2) {};
			\node [style=none] (15) at (0, 4) {};
			\node [style=none] (16) at (0, 3.5) {};
		\end{pgfonlayer}
		\begin{pgfonlayer}{edgelayer}
			\draw (1.center) to (2.center);
			\draw (2.center) to (0.center);
			\draw (0.center) to (1.center);
			\draw (3) to (4.center);
			\draw (10.center) to (8.center);
			\draw (8.center) to (9.center);
			\draw (9.center) to (10.center);
			\draw [in=150, out=-90, looseness=1.25] (5.center) to (3);
			\draw [in=-90, out=30] (3) to (10.center);
			\draw [in=90, out=-150, looseness=0.75] (1.center) to (7.center);
			\draw [in=90, out=-30, looseness=1.25] (2.center) to (6);
			\draw [in=90, out=-165] (13.center) to (14);
			\draw [bend right=90, looseness=1.75] (15.center) to (16.center);
		\end{pgfonlayer}
	\end{tikzpicture}	
	 $=$ \begin{tikzpicture}
		\begin{pgfonlayer}{nodelayer}
			\node [style=circle] (0) at (3.25, 3) {};
			\node [style=none] (1) at (4.25, 4.75) {};
			\node [style=none] (2) at (4.25, 5.25) {};
			\node [style=none] (3) at (4.25, 4.5) {};
			\node [style=none] (4) at (4, 4.5) {};
			\node [style=none] (5) at (4.25, 3.25) {};
			\node [style=none] (6) at (4, 5.5) {};
			\node [style=none] (7) at (4.5, 5.5) {};
			\node [style=none] (10) at (3, 3.75) {};
			\node [style=none] (11) at (2.5, 5.5) {};
			\node [style=none] (12) at (4.25, 3.25) {};
			\node [style=none] (13) at (4.25, 3.25) {};
			\node [style=circle] (14) at (5, 2) {};
			\node [style=none] (15) at (4, 3) {};
			\node [style=none] (16) at (3.5, 2) {};
			\node [style=none] (17) at (4.5, 3) {};
			\node [style=none] (18) at (4.25, 4.75) {};
			\node [style=none] (19) at (4.25, 4.25) {};
			\node [style=none] (20) at (3.25, 4) {};
			\node [style=none] (21) at (3.25, 3.5) {};
		\end{pgfonlayer}
		\begin{pgfonlayer}{edgelayer}
			\draw [in=90, out=-180, looseness=1.50] (4.center) to (0);
			\draw (2.center) to (5.center);
			\draw (6.center) to (7.center);
			\draw (7.center) to (2.center);
			\draw (2.center) to (6.center);
			\draw [in=-90, out=165] (10.center) to (11.center);
			\draw (15.center) to (17.center);
			\draw (17.center) to (12.center);
			\draw (12.center) to (15.center);
			\draw [in=90, out=-150, looseness=0.75] (15.center) to (16.center);
			\draw [in=90, out=-30, looseness=1.25] (17.center) to (14);
			\draw [bend right=90, looseness=1.75] (18.center) to (19.center);
			\draw [bend right=90, looseness=1.75] (20.center) to (21.center);
		\end{pgfonlayer}
	\end{tikzpicture} $ \stackrel{\bf [comp.3]}{=}$
	\begin{tikzpicture}
		\begin{pgfonlayer}{nodelayer}
			\node [style=circle] (0) at (3, 3) {};
			\node [style=none] (1) at (4, 4) {};
			\node [style=none] (2) at (4, 3.5) {};
			\node [style=none] (3) at (3.75, 4.25) {};
			\node [style=none] (4) at (4.25, 4.25) {};
			\node [style=none] (5) at (3, 3.75) {};
			\node [style=none] (6) at (3, 3.5) {};
			\node [style=none] (7) at (2.75, 3.75) {};
			\node [style=none] (8) at (2, 5.25) {};
			\node [style=none] (9) at (4, 3.5) {};
			\node [style=none] (10) at (4, 3.5) {};
			\node [style=circle] (11) at (4.5, 2.25) {};
			\node [style=none] (12) at (3.75, 3.25) {};
			\node [style=none] (13) at (3.5, 2) {};
			\node [style=none] (14) at (4.25, 3.25) {};
			\node [style=none] (15) at (3, 4.75) {};
			\node [style=none] (16) at (3.25, 5) {};
			\node [style=none] (17) at (2.75, 5) {};
			\node [style=none] (18) at (3, 4) {};
			\node [style=none] (19) at (3, 3.5) {};
		\end{pgfonlayer}
		\begin{pgfonlayer}{edgelayer}
			\draw (1.center) to (2.center);
			\draw (3.center) to (4.center);
			\draw (4.center) to (1.center);
			\draw (1.center) to (3.center);
			\draw [in=-90, out=-180, looseness=1.25] (7.center) to (8.center);
			\draw (12.center) to (14.center);
			\draw (14.center) to (9.center);
			\draw (9.center) to (12.center);
			\draw [in=90, out=-150, looseness=0.75] (12.center) to (13.center);
			\draw [in=90, out=-45] (14.center) to (11);
			\draw (17.center) to (16.center);
			\draw (16.center) to (15.center);
			\draw (15.center) to (17.center);
			\draw (15.center) to (0);
			\draw [bend right=90, looseness=1.75] (18.center) to (19.center);
		\end{pgfonlayer}
	\end{tikzpicture}  $\stackrel{\bf [comp.2]}{=}$
	\begin{tikzpicture}
		\begin{pgfonlayer}{nodelayer}
			\node [style=circle] (0) at (3, 3) {};
			\node [style=none] (5) at (3, 3.75) {};
			\node [style=none] (6) at (3, 3.5) {};
			\node [style=none] (7) at (2.75, 3.75) {};
			\node [style=none] (8) at (2, 5.25) {};
			\node [style=circle] (11) at (3.5, 4.25) {};
			\node [style=none] (13) at (3.5, 2) {};
			\node [style=none] (15) at (3, 4.75) {};
			\node [style=none] (16) at (3.25, 5) {};
			\node [style=none] (17) at (2.75, 5) {};
			\node [style=none] (18) at (3, 4) {};
			\node [style=none] (19) at (3, 3.5) {};
		\end{pgfonlayer}
		\begin{pgfonlayer}{edgelayer}
			\draw [in=-90, out=-180, looseness=1.25] (7.center) to (8.center);
			\draw (17.center) to (16.center);
			\draw (16.center) to (15.center);
			\draw (15.center) to (17.center);
			\draw (15.center) to (0);
			\draw [bend right=90, looseness=1.75] (18.center) to (19.center);
			\draw (11) to (13.center);
		\end{pgfonlayer}
	\end{tikzpicture}	
	$\stackrel{ \bf [comp.1]}{=} $
	\begin{tikzpicture}
		\begin{pgfonlayer}{nodelayer}
			\node [style=circle] (0) at (3, 3) {};
			\node [style=none] (1) at (3, 4) {};
			\node [style=none] (2) at (3, 5) {};
			\node [style=none] (3) at (2.75, 3.75) {};
			\node [style=none] (4) at (3.25, 3.75) {};
			\node [style=none] (5) at (3, 2) {};
		\end{pgfonlayer}
		\begin{pgfonlayer}{edgelayer}
			\draw (1.center) to (2.center);
			\draw (3.center) to (4.center);
			\draw (4.center) to (1.center);
			\draw (1.center) to (3.center);
			\draw (5.center) to (0);
		\end{pgfonlayer}
	\end{tikzpicture}

	\bigskip
	
	Similarly, the $\oa$-bialgebra has an antipode using the `op' versions of {\bf[comp.1]}-{\bf[comp.3]}. 
	\end{proof}

	A $\dagger$-complementary system in a unitary category corresponds to the usual notion 
	of interacting commutative $\dagger$-FAs \cite{CoD11} when its unitary structure map satisfies 
	\ref{eqnn: unitary coincidence-b}  for the linear monoid and the linear comonoid. 
	Splitting binary idempotents on a linear bialgebra produces a complementary system under 
	the following conditions:

	\begin{lemma} 
	\label{Lemma: complementary idempotent}
	In an isomix category, a self-linear bialgebra given by  splitting a coring binary idempotent $(\u, \v)$ on a commutative 
and a cocommutative linear bialgebra $A \linbialgwtik B$ is a complementary system if and only if the binary idempotent 
satisfies the following conditions (and their `op' symmetries):
\begin{equation}
	\label{eqn: idemcomp}
(a) ~ \begin{tikzpicture}
	\begin{pgfonlayer}{nodelayer}
		\node [style=none] (0) at (2, 5) {};
		\node [style=none] (1) at (2.25, 5.25) {};
		\node [style=none] (2) at (1.75, 5.25) {};
		\node [style=none] (3) at (2, 3.25) {};
		\node [style=none] (4) at (2, 3) {};
		\node [style=none] (5) at (2, 3.25) {};
		\node [style=circle] (6) at (2, 1.5) {};
		\node [style=none] (7) at (1.75, 3.25) {};
		\node [style=none] (9) at (0.75, 5) {$A$};
		\node [style=none] (10) at (2.5, 3.25) {$B$};
		\node [style=none] (11) at (1, 5.25) {};
		\node [style=circle, scale=1.8] (12) at (2, 4.25) {};
		\node [style=none] (13) at (2, 4.25) {$e_B$};
		\node [style=none] (14) at (2.5, 4.75) {$B$};
		\node [style=circle, scale=1.8] (15) at (2, 2.25) {};
		\node [style=none] (16) at (2, 2.25) {$e_B$};
		\node [style=circle, scale=1.8] (17) at (1, 4.25) {};
		\node [style=none] (18) at (1, 4.25) {$e_A$};
		\node [style=none] (19) at (2, 3.5) {};
		\node [style=none] (20) at (2, 3) {};
	\end{pgfonlayer}
	\begin{pgfonlayer}{edgelayer}
		\draw (0.center) to (1.center);
		\draw (0.center) to (2.center);
		\draw (2.center) to (1.center);
		\draw (4.center) to (5.center);
		\draw (0.center) to (12);
		\draw (4.center) to (15);
		\draw (6) to (15);
		\draw (12) to (5.center);
		\draw [in=180, out=-90, looseness=1.25] (17) to (7.center);
		\draw (11.center) to (17);
		\draw [bend right=90, looseness=1.75] (19.center) to (20.center);
	\end{pgfonlayer}
\end{tikzpicture} = \begin{tikzpicture}
	\begin{pgfonlayer}{nodelayer}
		\node [style=none] (19) at (4, 2) {};
		\node [style=none] (20) at (4.25, 1.75) {};
		\node [style=none] (21) at (3.75, 1.75) {};
		\node [style=none] (22) at (4, 2) {};
		\node [style=none] (23) at (3.5, 5) {$A$};
		\node [style=none] (24) at (4, 5.5) {};
		\node [style=circle, scale=1.8] (25) at (4, 3.75) {};
		\node [style=none] (26) at (4, 3.75) {$e_A$};
	\end{pgfonlayer}
	\begin{pgfonlayer}{edgelayer}
		\draw (19.center) to (20.center);
		\draw (19.center) to (21.center);
		\draw (21.center) to (20.center);
		\draw (24.center) to (25);
		\draw (25) to (22.center);
	\end{pgfonlayer}
\end{tikzpicture}
~~~~~~~~ (b) ~
\begin{tikzpicture}
	\begin{pgfonlayer}{nodelayer}
		\node [style=none] (33) at (5.5, 3.75) {};
		\node [style=none] (34) at (6, 3.75) {};
		\node [style=none] (36) at (5.75, 4) {};
		\node [style=none] (37) at (5.75, 5.25) {};
		\node [style=circle] (38) at (5, 1.75) {};
		\node [style=circle, scale=1.5] (39) at (5.75, 4.75) {};
		\node [style=none] (40) at (5.75, 4.75) {$\v$};
		\node [style=none] (41) at (6.25, 5) {$B$};
		\node [style=circle, scale=1.5] (42) at (5, 2.75) {};
		\node [style=none] (43) at (5, 2.75) {$\u$};
		\node [style=none] (44) at (5.5, 5.5) {};
		\node [style=none] (45) at (6, 5.5) {};
		\node [style=none] (47) at (6.25, 4.25) {$A$};
		\node [style=none] (48) at (6.5, 1. 5) {};
		\node [style=circle, scale=1.5] (50) at (6.5, 2.75) {};
		\node [style=none] (51) at (6.5, 2.75) {$e_A$};
		\node [style=none] (52) at (7, 2.25) {$A$};
	\end{pgfonlayer}
	\begin{pgfonlayer}{edgelayer}
		\draw (34.center) to (33.center);
		\draw (33.center) to (36.center);
		\draw (36.center) to (34.center);
		\draw (37.center) to (39);
		\draw (36.center) to (39);
		\draw (38) to (42);
		\draw (44.center) to (45.center);
		\draw (45.center) to (37.center);
		\draw (37.center) to (44.center);
		\draw (48.center) to (50);
		\draw [in=90, out=-150, looseness=1.25] (33.center) to (42);
		\draw [in=90, out=-30, looseness=1.25] (34.center) to (50);
	\end{pgfonlayer}
\end{tikzpicture} = \begin{tikzpicture}
	\begin{pgfonlayer}{nodelayer}
		\node [style=none] (39) at (5.5, 2.5) {$A$};
		\node [style=none] (45) at (5, 1.75) {};
		\node [style=circle] (46) at (5, 5.5) {};
		\node [style=circle, scale=1.5] (47) at (5, 3.75) {};
		\node [style=none] (48) at (5, 3.75) {$e_A$};
	\end{pgfonlayer}
	\begin{pgfonlayer}{edgelayer}
		\draw (45.center) to (47);
		\draw (47) to (46);
	\end{pgfonlayer}
\end{tikzpicture}
~~~~~~~~ 	(c) ~
\begin{tikzpicture}
	\begin{pgfonlayer}{nodelayer}
		\node [style=none] (49) at (7.25, 3.5) {};
		\node [style=none] (50) at (7, 3.25) {};
		\node [style=none] (51) at (7, 1.75) {};
		\node [style=none] (52) at (7, 3.75) {};
		\node [style=none] (53) at (7, 5.25) {};
		\node [style=none] (54) at (8, 1.75) {};
		\node [style=circle, scale=1.5] (55) at (7, 4.5) {};
		\node [style=none] (56) at (7, 4.5) {$\v$};
		\node [style=none] (57) at (8.5, 2) {$B$};
		\node [style=none] (58) at (7.25, 5.5) {};
		\node [style=none] (59) at (6.75, 5.5) {};
		\node [style=none] (60) at (6.5, 2) {$B$};
		\node [style=none] (61) at (6.5, 5) {$B$};
		\node [style=circle, scale=1.5] (62) at (7, 2.5) {};
		\node [style=none] (63) at (7, 2.5) {$\u$};
		\node [style=none] (64) at (8, 3) {};
		\node [style=none] (65) at (6.5, 3.5) {$A$};
		\node [style=circle, scale=1.5] (66) at (8, 2.5) {};
		\node [style=none] (67) at (8, 2.5) {$e_B$};
		\node [style=none] (68) at (7, 3.75) {};
		\node [style=none] (69) at (7, 3.25) {};
	\end{pgfonlayer}
	\begin{pgfonlayer}{edgelayer}
		\draw (52.center) to (50.center);
		\draw (53.center) to (55);
		\draw (52.center) to (55);
		\draw (58.center) to (59.center);
		\draw (59.center) to (53.center);
		\draw (53.center) to (58.center);
		\draw (51.center) to (62);
		\draw (50.center) to (62);
		\draw [in=0, out=105] (64.center) to (49.center);
		\draw (54.center) to (66);
		\draw (64.center) to (66);
		\draw [bend left=90, looseness=1.75] (68.center) to (69.center);
	\end{pgfonlayer}
\end{tikzpicture}  =  \begin{tikzpicture}
	\begin{pgfonlayer}{nodelayer}
		\node [style=none] (68) at (9.75, 2) {};
		\node [style=none] (69) at (9.75, 3.75) {};
		\node [style=none] (70) at (9.75, 5.25) {};
		\node [style=circle, scale=1.5] (71) at (9.75, 4.5) {};
		\node [style=none] (72) at (9.75, 4.5) {$\v$};
		\node [style=circle, scale=1.5] (73) at (9.75, 3.25) {};
		\node [style=none] (74) at (9.75, 3.25) {$\u$};
		\node [style=none] (75) at (9.5, 5.5) {};
		\node [style=none] (76) at (10, 5.5) {};
		\node [style=circle, scale=1.5] (77) at (10.75, 4.5) {};
		\node [style=none] (78) at (10.75, 4.5) {$\v$};
		\node [style=none] (79) at (11.25, 2.25) {$B$};
		\node [style=circle, scale=1.5] (80) at (10.75, 3.25) {};
		\node [style=none] (81) at (11, 5.5) {};
		\node [style=none] (82) at (10.5, 5.5) {};
		\node [style=none] (83) at (10.75, 5.25) {};
		\node [style=none] (84) at (10.75, 2) {};
		\node [style=none] (85) at (11.25, 5) {$B$};
		\node [style=none] (86) at (10.75, 3.75) {};
		\node [style=none] (87) at (11.25, 4) {$A$};
		\node [style=none] (88) at (10.75, 3.25) {$\u$};
	\end{pgfonlayer}
	\begin{pgfonlayer}{edgelayer}
		\draw (70.center) to (71);
		\draw (69.center) to (71);
		\draw (75.center) to (76.center);
		\draw (76.center) to (70.center);
		\draw (70.center) to (75.center);
		\draw (68.center) to (73);
		\draw (69.center) to (73);
		\draw (83.center) to (77);
		\draw (86.center) to (77);
		\draw (82.center) to (81.center);
		\draw (81.center) to (83.center);
		\draw (83.center) to (82.center);
		\draw (84.center) to (80);
		\draw (86.center) to (80);
	\end{pgfonlayer}
\end{tikzpicture}
\end{equation}
where $e_A = \u \v$, and $e_B = \v \u$
\end{lemma}
	
\begin{proof}
Suppose $E \linbialgbtik E$ is a self-linear bialgebra given by splitting a binary idempotent 
$(\u, \v)$ on a commutative and a cocommutative linear bialgebra $A \linbialgwtik B$. 
Let the splitting of the bialgebra be given as follows:
$\u \v = A \to^{r} E \to^{s} A$, and $\v \u = B \to^{r'} E' \to^{s'} B$. 
If $(\u, \v)$ satisfy the given equations, then $E \linbialgbtik E$ is a complementary system
 because $\mbox{\small \bf [comp.1]}$-$\mbox{\small \bf [comp.3}$ holds: 
\begin{description}
\item[[comp.1\!\!]]:
$ \begin{tikzpicture}
	\begin{pgfonlayer}{nodelayer}
		\node [style=none] (28) at (6.25, 4.5) {};
		\node [style=none] (29) at (6, 4.25) {};
		\node [style=none] (30) at (5.75, 4.5) {};
		\node [style=none] (31) at (6, 2.25) {};
		\node [style=none] (32) at (6, 2) {};
		\node [style=none] (33) at (5.75, 2) {};
		\node [style=black] (34) at (6, 0) {};
		\node [style=none] (35) at (4.75, 4) {};
		\node [style=none] (36) at (6.5, 4) {$E$};
		\node [style=none] (37) at (6.5, 0.5) {$E$};
		\node [style=none] (38) at (4.25, 4.25) {$E$};
		\node [style=none] (39) at (4.75, 4.5) {};
		\node [style=none] (40) at (6, 2.25) {};
		\node [style=none] (41) at (6, 1.75) {};
	\end{pgfonlayer}
	\begin{pgfonlayer}{edgelayer}
		\draw [style=filled] (29.center)
				to (28.center)
				to (30.center)
				to cycle;
		\draw [style=filled] (32.center) to (31.center);
		\draw [style=filled] (29.center) to (31.center);
		\draw [style=filled] (32.center) to (34);
		\draw [in=-90, out=165, looseness=1.25] (33.center) to (35.center);
		\draw (39.center) to (35.center);
		\draw [style=filled] (41.center)
				to (40.center)
				to [bend right=90, looseness=1.75] cycle;
	\end{pgfonlayer}
\end{tikzpicture} =\begin{tikzpicture}
	\begin{pgfonlayer}{nodelayer}
		\node [style=none] (0) at (-2.25, 6) {};
		\node [style=none] (1) at (-2.75, 6) {};
		\node [style=none] (2) at (-2.5, 5.75) {};
		\node [style=circle, scale=1.5] (3) at (-2.5, 5.25) {};
		\node [style=circle, scale=1.5] (4) at (-2.5, 4.5) {};
		\node [style=none] (5) at (-2.5, 4) {};
		\node [style=none] (6) at (-2.5, 3.5) {};
		\node [style=none] (7) at (-2.75, 3.75) {};
		\node [style=circle, scale=1.5] (8) at (-2.5, 3) {};
		\node [style=circle, scale=1.5] (9) at (-2.5, 2.25) {};
		\node [style=circle] (10) at (-2.5, 1.5) {};
		\node [style=none] (11) at (-3.75, 4.75) {};
		\node [style=none] (12) at (-3.75, 6) {};
		\node [style=none] (13) at (-2.5, 5.25) {$r'$};
		\node [style=none] (14) at (-2.5, 4.5) {$s'$};
		\node [style=none] (15) at (-2.5, 3) {$r'$};
		\node [style=none] (16) at (-2.5, 2.25) {$s'$};
		\node [style=circle, scale=1.5] (17) at (-3.75, 5.25) {};
		\node [style=none] (18) at (-3.75, 5.25) {$s$};
		\node [style=none] (20) at (-2, 4.75) {$E$};
		\node [style=none] (21) at (-2, 5.5) {$B$};
		\node [style=none] (24) at (-4, 4.25) {$A$};
		\node [style=none] (25) at (-4, 5.75) {$E$};
		\node [style=none] (26) at (-2, 3.75) {$B$};
		\node [style=none] (27) at (-2, 2) {$B$};
		\node [style=none] (28) at (-2.5, 4) {};
		\node [style=none] (29) at (-2.5, 3.5) {};
	\end{pgfonlayer}
	\begin{pgfonlayer}{edgelayer}
		\draw (6.center) to (5.center);
		\draw [in=-90, out=180, looseness=1.25] (7.center) to (11.center);
		\draw (5.center) to (4);
		\draw (4) to (3);
		\draw (3) to (2.center);
		\draw (2.center) to (0.center);
		\draw (0.center) to (1.center);
		\draw (1.center) to (2.center);
		\draw (6.center) to (8);
		\draw (8) to (9);
		\draw (9) to (10);
		\draw (11.center) to (17);
		\draw (12.center) to (17);
		\draw [bend right=90, looseness=1.75] (28.center) to (29.center);
	\end{pgfonlayer}
\end{tikzpicture} = \begin{tikzpicture}
	\begin{pgfonlayer}{nodelayer}
		\node [style=none] (0) at (-2.25, 6) {};
		\node [style=none] (1) at (-2.75, 6) {};
		\node [style=none] (2) at (-2.5, 5.75) {};
		\node [style=circle, scale=1.5] (3) at (-2.5, 5.25) {};
		\node [style=circle, scale=1.5] (4) at (-2.5, 4.5) {};
		\node [style=none] (5) at (-2.5, 3.75) {};
		\node [style=none] (6) at (-2.5, 3.5) {};
		\node [style=none] (7) at (-2.75, 3.75) {};
		\node [style=circle, scale=1.5] (8) at (-2.5, 3) {};
		\node [style=circle, scale=1.5] (9) at (-2.5, 2.25) {};
		\node [style=circle] (10) at (-2.5, 1.5) {};
		\node [style=none] (12) at (-3.5, 6) {};
		\node [style=none] (13) at (-2.5, 5.25) {$r'$};
		\node [style=none] (14) at (-2.5, 4.5) {$s'$};
		\node [style=none] (15) at (-2.5, 3) {$r'$};
		\node [style=none] (16) at (-2.5, 2.25) {$s'$};
		\node [style=circle, scale=1.5] (17) at (-3.5, 5.25) {};
		\node [style=none] (18) at (-3.5, 5.25) {$s$};
		\node [style=none] (20) at (-2, 4.75) {$E$};
		\node [style=none] (21) at (-2, 5.5) {$B$};
		\node [style=none] (24) at (-3.75, 4) {$A$};
		\node [style=none] (25) at (-3.75, 5.75) {$E$};
		\node [style=none] (26) at (-2, 3.75) {$B$};
		\node [style=none] (27) at (-2, 2) {$B$};
		\node [style=onehalfcircle] (29) at (-3.5, 4.5) {};
		\node [style=none] (28) at (-3.5, 4.5) {$e_A$};
		\node [style=none] (30) at (-2.5, 4) {};
		\node [style=none] (31) at (-2.5, 3.5) {};
	\end{pgfonlayer}
	\begin{pgfonlayer}{edgelayer}
		\draw (6.center) to (5.center);
		\draw (5.center) to (4);
		\draw (4) to (3);
		\draw (3) to (2.center);
		\draw (2.center) to (0.center);
		\draw (0.center) to (1.center);
		\draw (1.center) to (2.center);
		\draw (6.center) to (8);
		\draw (8) to (9);
		\draw (9) to (10);
		\draw (12.center) to (17);
		\draw (17) to (29);
		\draw [in=165, out=-75, looseness=1.25] (29) to (7.center);
		\draw [bend right=90, looseness=1.50] (30.center) to (31.center);
	\end{pgfonlayer}
\end{tikzpicture} = \begin{tikzpicture}
	\begin{pgfonlayer}{nodelayer}
		\node [style=circle, scale=1.5] (30) at (-0.75, 3.25) {};
		\node [style=none] (31) at (-0.75, 3.25) {$e_{A}$};
		\node [style=none] (32) at (-0.75, 6) {};
		\node [style=circle, scale=1.5] (33) at (-0.75, 4.5) {};
		\node [style=none] (34) at (-0.75, 4.5) {$s$};
		\node [style=none] (35) at (-0.25, 5.5) {$E$};
		\node [style=none] (36) at (-0.75, 1.75) {};
		\node [style=none] (37) at (-1, 1.5) {};
		\node [style=none] (38) at (-0.5, 1.5) {};
	\end{pgfonlayer}
	\begin{pgfonlayer}{edgelayer}
		\draw (32.center) to (33);
		\draw (36.center) to (37.center);
		\draw (37.center) to (38.center);
		\draw (38.center) to (36.center);
		\draw (36.center) to (30);
		\draw (30) to (33);
	\end{pgfonlayer}
\end{tikzpicture} = \begin{tikzpicture}
	\begin{pgfonlayer}{nodelayer}
		\node [style=none] (39) at (1, 6) {};
		\node [style=circle, scale=1.5] (40) at (1, 3.75) {};
		\node [style=none] (41) at (1, 3.75) {$s$};
		\node [style=none] (42) at (1.5, 5.5) {$E$};
		\node [style=none] (43) at (1, 1.75) {};
		\node [style=none] (44) at (0.75, 1.5) {};
		\node [style=none] (45) at (1.25, 1.5) {};
	\end{pgfonlayer}
	\begin{pgfonlayer}{edgelayer}
		\draw (39.center) to (40);
		\draw (43.center) to (44.center);
		\draw (44.center) to (45.center);
		\draw (45.center) to (43.center);
		\draw (43.center) to (40);
	\end{pgfonlayer}
\end{tikzpicture} = \begin{tikzpicture}
	\begin{pgfonlayer}{nodelayer}
		\node [style=none] (23) at (3, 4.25) {};
		\node [style=none] (24) at (3.25, 4) {$E$};
		\node [style=none] (25) at (3, 0.25) {};
		\node [style=none] (26) at (2.75, 0) {};
		\node [style=none] (27) at (3.25, 0) {};
	\end{pgfonlayer}
	\begin{pgfonlayer}{edgelayer}
		\draw (25.center) to (23.center);
		\draw [style=filled] (26.center)
				to (27.center)
				to (25.center)
				to cycle;
	\end{pgfonlayer}
\end{tikzpicture}  $

\item[[comp.2\!\!]]:
$ \begin{tikzpicture}
	\begin{pgfonlayer}{nodelayer}
		\node [style=none] (0) at (-1.25, -0.5) {};
		\node [style=none] (1) at (-1, -0.25) {};
		\node [style=none] (2) at (-0.75, -0.5) {};
		\node [style=none] (3) at (-1, 1.5) {};
		\node [style=none] (4) at (-1.25, 1.75) {};
		\node [style=none] (5) at (-0.75, 1.75) {};
		\node [style=none] (6) at (-0.25, -1.75) {};
		\node [style=circle, fill=black] (7) at (-1.75, -3) {};
		\node [style=none] (8) at (-0.25, -3) {};
	\end{pgfonlayer}
	\begin{pgfonlayer}{edgelayer}
		\draw (1.center) to (3.center);
		\draw [in=-135, out=90] (7) to (0.center);
		\draw [in=90, out=-45] (2.center) to (6.center);
		\draw [style=filled] (8.center) to (6.center);
		\draw [style=filled] (2.center)
				to (1.center)
				to (0.center)
				to cycle;
		\draw [style=filled] (3.center)
				to (4.center)
				to (5.center)
				to cycle;
	\end{pgfonlayer}
\end{tikzpicture} = \begin{tikzpicture}
	\begin{pgfonlayer}{nodelayer}
		\node [style=none] (0) at (-1.25, -0.75) {};
		\node [style=none] (1) at (-1, -0.5) {};
		\node [style=none] (2) at (-0.75, -0.75) {};
		\node [style=none] (3) at (-1, 1.25) {};
		\node [style=none] (4) at (-1.25, 1.5) {};
		\node [style=none] (5) at (-0.75, 1.5) {};
		\node [style=none] (6) at (-0.25, -3) {};
		\node [style=circle] (7) at (-1.75, -3) {};
		\node [style=circle, scale=1.5] (8) at (-1, 0) {};
		\node [style=circle, scale=1.5] (9) at (-1, 0.75) {};
		\node [style=circle, scale=1.5] (10) at (-1.75, -2.25) {};
		\node [style=circle, scale=1.5] (11) at (-0.25, -2.25) {};
		\node [style=circle, scale=1.5] (12) at (-1.75, -1.5) {};
		\node [style=none] (13) at (-1.75, -2.25) {$s'$};
		\node [style=none] (14) at (-1.75, -1.5) {$r$};
		\node [style=none] (15) at (-1, 0) {$s$};
		\node [style=none] (16) at (-1, 0.75) {$r'$};
		\node [style=none] (17) at (-0.25, -2.25) {$r$};
		\node [style=none] (18) at (-2.25, -2.75) {$B$};
		\node [style=none] (19) at (-2.25, -1.75) {$E$};
		\node [style=none] (20) at (-1.75, -0.65) {$A$};
		\node [style=none] (21) at (-0.25, -0.5) {$A$};
		\node [style=none] (22) at (0, -2.75) {$E$};
		\node [style=none] (23) at (-0.5, 0.25) {$E$};
		\node [style=none] (24) at (-0.5, 1) {$B$};
		\node [style=none] (25) at (-0.25, -1.5) {};
	\end{pgfonlayer}
	\begin{pgfonlayer}{edgelayer}
		\draw (0.center) to (2.center);
		\draw (2.center) to (1.center);
		\draw (1.center) to (0.center);
		\draw (3.center) to (4.center);
		\draw (4.center) to (5.center);
		\draw (5.center) to (3.center);
		\draw (7) to (10);
		\draw (10) to (12);
		\draw [in=-165, out=90] (12) to (0.center);
		\draw (6.center) to (11);
		\draw (1.center) to (8);
		\draw (8) to (9);
		\draw (9) to (3.center);
		\draw [in=90, out=-30] (2.center) to (25.center);
		\draw (25.center) to (11);
	\end{pgfonlayer}
\end{tikzpicture} = \begin{tikzpicture}
	\begin{pgfonlayer}{nodelayer}
		\node [style=none] (26) at (2, -1) {};
		\node [style=none] (27) at (2.25, -0.75) {};
		\node [style=none] (28) at (2.5, -1) {};
		\node [style=none] (29) at (2.25, 1.25) {};
		\node [style=none] (30) at (2, 1.5) {};
		\node [style=none] (31) at (2.5, 1.5) {};
		\node [style=none] (32) at (3, -3) {};
		\node [style=circle] (33) at (1.5, -3) {};
		\node [style=circle, scale=1.5] (34) at (2.25, 0.25) {};
		\node [style=circle, scale=1.5] (35) at (1.5, -2.25) {};
		\node [style=circle, scale=1.5] (36) at (3, -2.25) {};
		\node [style=none] (37) at (1.5, -2.25) {$\v$};
		\node [style=none] (38) at (2.25, 0.25) {$\u$};
		\node [style=none] (39) at (3, -2.25) {$r$};
		\node [style=none] (41) at (3.5, -2.5) {$E$};
		\node [style=none] (42) at (2.75, 1) {$B$};
		\node [style=none] (43) at (2.75, -0.25) {$A$};
		\node [style=none] (44) at (1.5, -1.75) {};
		\node [style=none] (45) at (3, -1.75) {};
		\node [style=none] (46) at (1, -2.5) {$B$};
	\end{pgfonlayer}
	\begin{pgfonlayer}{edgelayer}
		\draw (26.center) to (28.center);
		\draw (28.center) to (27.center);
		\draw (27.center) to (26.center);
		\draw (29.center) to (30.center);
		\draw (30.center) to (31.center);
		\draw (31.center) to (29.center);
		\draw (33) to (35);
		\draw (32.center) to (36);
		\draw (34) to (29.center);
		\draw (27.center) to (34);
		\draw (44.center) to (35);
		\draw (45.center) to (36);
		\draw [in=90, out=-150] (26.center) to (44.center);
		\draw [in=90, out=-30, looseness=1.25] (28.center) to (45.center);
	\end{pgfonlayer}
\end{tikzpicture} = \begin{tikzpicture}
	\begin{pgfonlayer}{nodelayer}
		\node [style=none] (26) at (2, -0.25) {};
		\node [style=none] (27) at (2.25, 0) {};
		\node [style=none] (28) at (2.5, -0.25) {};
		\node [style=none] (29) at (2.25, 1.25) {};
		\node [style=none] (30) at (2, 1.5) {};
		\node [style=none] (31) at (2.5, 1.5) {};
		\node [style=none] (32) at (3, -3) {};
		\node [style=circle] (33) at (1.5, -3) {};
		\node [style=circle, scale=1.5] (34) at (2.25, 0.5) {};
		\node [style=circle, scale=1.5] (35) at (1.5, -1.5) {};
		\node [style=circle, scale=1.5] (36) at (3, -2.25) {};
		\node [style=none] (37) at (1.5, -1.5) {$\v$};
		\node [style=none] (38) at (2.25, 0.5) {$\u$};
		\node [style=none] (39) at (3, -2.25) {$r$};
		\node [style=none] (41) at (3.5, -2.75) {$E$};
		\node [style=none] (42) at (2.75, 1) {$B$};
		\node [style=none] (43) at (2.75, 0) {$A$};
		\node [style=none] (44) at (1.5, -1) {};
		\node [style=none] (45) at (3, -1) {};
		\node [style=none] (46) at (1, -2.75) {$B$};
		\node [style=circle, scale=1.5] (47) at (3, -1.5) {};
		\node [style=none] (48) at (3, -1.5) {$e_A$};
	\end{pgfonlayer}
	\begin{pgfonlayer}{edgelayer}
		\draw (26.center) to (28.center);
		\draw (28.center) to (27.center);
		\draw (27.center) to (26.center);
		\draw (29.center) to (30.center);
		\draw (30.center) to (31.center);
		\draw (31.center) to (29.center);
		\draw (33) to (35);
		\draw (32.center) to (36);
		\draw (34) to (29.center);
		\draw (27.center) to (34);
		\draw (44.center) to (35);
		\draw [in=90, out=-150] (26.center) to (44.center);
		\draw [in=90, out=-30] (28.center) to (45.center);
		\draw [style=filled] (36) to (47);
		\draw [style=filled] (45.center) to (47);
	\end{pgfonlayer}
\end{tikzpicture} = \begin{tikzpicture}
	\begin{pgfonlayer}{nodelayer}
		\node [style=circle] (49) at (4.75, 1.5) {};
		\node [style=circle, scale=1.5] (50) at (4.75, 0.25) {};
		\node [style=circle, scale=1.5] (52) at (4.75, -1) {};
		\node [style=none] (54) at (4.75, 0.25) {$r$};
		\node [style=none] (55) at (4.75, -1) {$e_A$};
		\node [style=none] (57) at (5.25, 0.75) {$A$};
		\node [style=none] (58) at (4.75, -3) {};
		\node [style=none] (59) at (5.25, -2.5) {$E$};
	\end{pgfonlayer}
	\begin{pgfonlayer}{edgelayer}
		\draw (50) to (49);
		\draw (50) to (52);
		\draw [style=filled] (52) to (58.center);
	\end{pgfonlayer}
\end{tikzpicture}  =\begin{tikzpicture}
	\begin{pgfonlayer}{nodelayer}
		\node [style=circle] (60) at (6.25, 1.5) {};
		\node [style=circle, scale=1.5] (61) at (6.25, -0.75) {};
		\node [style=none] (62) at (6.25, -0.75) {$r$};
		\node [style=none] (63) at (6.75, 1) {$A$};
		\node [style=none] (64) at (6.25, -3) {};
		\node [style=none] (65) at (6.75, -1.5) {$E$};
	\end{pgfonlayer}
	\begin{pgfonlayer}{edgelayer}
		\draw (61) to (60);
		\draw (64.center) to (61);
	\end{pgfonlayer}
\end{tikzpicture} = \begin{tikzpicture}
	\begin{pgfonlayer}{nodelayer}
		\node [style=circle, fill=black] (66) at (8, 1.5) {};
		\node [style=none] (67) at (8, -3) {};
		\node [style=none] (68) at (8.5, -2.5) {$E$};
	\end{pgfonlayer}
	\begin{pgfonlayer}{edgelayer}
		\draw (67.center) to (66);
	\end{pgfonlayer}
\end{tikzpicture} $

\item[[comp.3\!\!]]:
$\begin{tikzpicture}
	\begin{pgfonlayer}{nodelayer}
		\node [style=none] (15) at (2.75, -1.25) {};
		\node [style=none] (16) at (2.75, -1.5) {};
		\node [style=none] (17) at (3, -1.25) {};
		\node [style=none] (18) at (2.75, 1) {};
		\node [style=none] (29) at (2.75, -3) {};
		\node [style=none] (30) at (3.75, -3) {};
		\node [style=none] (32) at (4, -2.75) {$B$};
		\node [style=none] (33) at (2.5, -2.75) {$A$};
		\node [style=none] (34) at (2.75, -1.25) {};
		\node [style=none] (35) at (2.5, 1.25) {};
		\node [style=none] (36) at (3, 1.25) {};
		\node [style=none] (37) at (2.75, -1) {};
		\node [style=none] (38) at (2.75, -1.5) {};
	\end{pgfonlayer}
	\begin{pgfonlayer}{edgelayer}
		\draw [style=filled] (15.center) to (29.center);
		\draw [style=filled] (34.center) to (18.center);
		\draw [style=filled] (35.center)
				to (36.center)
				to (18.center)
				to cycle;
		\draw [style=filled] (15.center) to (16.center);
		\draw [in=90, out=0, looseness=1.25] (17.center) to (30.center);
		\draw [style=filled, bend left=90, looseness=1.50] (37.center) to (38.center);
	\end{pgfonlayer}
\end{tikzpicture}  =\begin{tikzpicture}
	\begin{pgfonlayer}{nodelayer}
		\node [style=none] (15) at (2.75, -1.5) {};
		\node [style=none] (16) at (2.75, -1.25) {};
		\node [style=none] (17) at (3, -1.25) {};
		\node [style=none] (18) at (2.75, 1) {};
		\node [style=none] (19) at (2.5, 1.25) {};
		\node [style=none] (20) at (3, 1.25) {};
		\node [style=circle, scale=1.5] (21) at (2.75, -0.5) {};
		\node [style=circle, scale=1.5] (22) at (2.75, 0.25) {};
		\node [style=none] (23) at (2.75, -0.5) {$s$};
		\node [style=none] (24) at (2.75, 0.25) {$r'$};
		\node [style=circle, scale=1.5] (25) at (2.75, -2.25) {};
		\node [style=none] (26) at (2.75, -2.25) {$r$};
		\node [style=circle, scale=1.5] (27) at (3.75, -2.25) {};
		\node [style=none] (28) at (3.75, -2.25) {$r'$};
		\node [style=none] (29) at (2.75, -3) {};
		\node [style=none] (30) at (3.75, -3) {};
		\node [style=none] (31) at (2.25, -2.75) {$E$};
		\node [style=none] (32) at (4, -1.75) {$B$};
		\node [style=none] (33) at (2.25, -1.75) {$A$};
		\node [style=none] (34) at (4, -2.75) {$E$};
		\node [style=none] (35) at (2.25, 0) {$E$};
		\node [style=none] (36) at (2.25, 0.75) {$B$};
		\node [style=none] (37) at (2.75, -1) {};
		\node [style=none] (38) at (2.75, -1.5) {};
	\end{pgfonlayer}
	\begin{pgfonlayer}{edgelayer}
		\draw (29.center) to (25);
		\draw (30.center) to (27);
		\draw (25) to (15.center);
		\draw [in=0, out=90, looseness=1.25] (27) to (17.center);
		\draw (16.center) to (15.center);
		\draw (16.center) to (21);
		\draw (21) to (22);
		\draw (22) to (18.center);
		\draw (18.center) to (19.center);
		\draw (19.center) to (20.center);
		\draw (20.center) to (18.center);
		\draw [style=none, bend left=90, looseness=1.75] (37.center) to (38.center);
	\end{pgfonlayer}
\end{tikzpicture}  = \begin{tikzpicture}
	\begin{pgfonlayer}{nodelayer}
		\node [style=none] (15) at (2.75, -1.5) {};
		\node [style=none] (16) at (2.75, -1.25) {};
		\node [style=none] (17) at (3, -1.25) {};
		\node [style=none] (18) at (2.75, 1) {};
		\node [style=none] (19) at (2.5, 1.25) {};
		\node [style=none] (20) at (3, 1.25) {};
		\node [style=circle, scale=1.5] (21) at (2.75, 0) {};
		\node [style=none] (23) at (2.75, 0) {$\v$};
		\node [style=circle, scale=1.5] (25) at (2.75, -2.25) {};
		\node [style=none] (26) at (2.75, -2.25) {$r$};
		\node [style=circle, scale=1.5] (27) at (3.75, -2.25) {};
		\node [style=none] (28) at (3.75, -2.25) {$r'$};
		\node [style=none] (29) at (2.75, -3) {};
		\node [style=none] (30) at (3.75, -3) {};
		\node [style=none] (31) at (2.25, -2.75) {$E$};
		\node [style=none] (32) at (4, -1.75) {$B$};
		\node [style=none] (33) at (2.25, -1.75) {$A$};
		\node [style=none] (34) at (4, -2.75) {$E$};
		\node [style=none] (36) at (2.25, 0.75) {$B$};
		\node [style=none] (37) at (2.75, -1) {};
		\node [style=none] (38) at (2.75, -1.5) {};
	\end{pgfonlayer}
	\begin{pgfonlayer}{edgelayer}
		\draw (29.center) to (25);
		\draw (30.center) to (27);
		\draw (25) to (15.center);
		\draw [in=0, out=90, looseness=1.50] (27) to (17.center);
		\draw (16.center) to (15.center);
		\draw (16.center) to (21);
		\draw (18.center) to (19.center);
		\draw (19.center) to (20.center);
		\draw (20.center) to (18.center);
		\draw [style=filled] (18.center) to (21);
		\draw [bend left=90, looseness=1.75] (37.center) to (38.center);
	\end{pgfonlayer}
\end{tikzpicture} =\begin{tikzpicture}
	\begin{pgfonlayer}{nodelayer}
		\node [style=none] (42) at (4.5, 3.75) {};
		\node [style=none] (43) at (4.5, 4) {};
		\node [style=none] (44) at (4.75, 4) {};
		\node [style=none] (45) at (4.5, 5.25) {};
		\node [style=none] (46) at (4.25, 5.5) {};
		\node [style=none] (47) at (4.75, 5.5) {};
		\node [style=circle, scale=1.5] (48) at (4.5, 4.75) {};
		\node [style=none] (49) at (4.5, 4.75) {$\v$};
		\node [style=circle, scale=1.5] (50) at (4.5, 3.25) {};
		\node [style=none] (51) at (4.5, 3.25) {$r$};
		\node [style=circle, scale=1.5] (52) at (5.75, 2.5) {};
		\node [style=none] (53) at (4.5, 1) {};
		\node [style=none] (54) at (5.75, 1) {};
		\node [style=none] (55) at (4.25, 1.25) {$E$};
		\node [style=none] (56) at (4, 3.75) {$A$};
		\node [style=none] (57) at (6, 3) {$B$};
		\node [style=none] (58) at (6, 1.5) {$E$};
		\node [style=none] (59) at (4, 5) {$B$};
		\node [style=circle, scale=1.5] (60) at (4.5, 1.75) {};
		\node [style=circle, scale=1.5] (61) at (4.5, 2.5) {};
		\node [style=none] (62) at (4.5, 2.5) {$s'$};
		\node [style=none] (63) at (4, 2.75) {$E$};
		\node [style=none] (64) at (5.75, 2.5) {$r'$};
		\node [style=none] (65) at (4.5, 1.75) {$r'$};
		\node [style=none] (66) at (4.5, 4.25) {};
		\node [style=none] (67) at (4.5, 3.75) {};
	\end{pgfonlayer}
	\begin{pgfonlayer}{edgelayer}
		\draw (50) to (42.center);
		\draw [in=0, out=90, looseness=1.25] (52) to (44.center);
		\draw (43.center) to (42.center);
		\draw (43.center) to (48);
		\draw (45.center) to (46.center);
		\draw (46.center) to (47.center);
		\draw (47.center) to (45.center);
		\draw (45.center) to (48);
		\draw (54.center) to (52);
		\draw (53.center) to (60);
		\draw (60) to (61);
		\draw (61) to (50);
		\draw [bend left=90, looseness=1.75] (66.center) to (67.center);
	\end{pgfonlayer}
\end{tikzpicture}  = \begin{tikzpicture}
	\begin{pgfonlayer}{nodelayer}
		\node [style=none] (15) at (2.75, 0.5) {};
		\node [style=none] (16) at (2.75, 0.75) {};
		\node [style=none] (17) at (3, 0.5) {};
		\node [style=none] (18) at (2.75, 2.5) {};
		\node [style=none] (19) at (2.5, 2.75) {};
		\node [style=none] (20) at (3, 2.75) {};
		\node [style=circle, scale=1.5] (21) at (2.75, 2) {};
		\node [style=none] (23) at (2.75, 2) {$r'$};
		\node [style=circle, scale=1.5] (25) at (2.75, -0.25) {};
		\node [style=none] (26) at (2.75, -0.25) {$r$};
		\node [style=circle, scale=1.5] (27) at (3.75, -0.25) {};
		\node [style=none] (28) at (3.75, -0.25) {$r'$};
		\node [style=none] (29) at (2.75, -1.75) {};
		\node [style=none] (30) at (3.75, -1.75) {};
		\node [style=none] (31) at (2.5, -1.5) {$E$};
		\node [style=none] (32) at (4, 0.25) {$B$};
		\node [style=none] (33) at (2.5, 0.25) {$A$};
		\node [style=none] (34) at (4, -1.5) {$E$};
		\node [style=none] (36) at (2.25, 2.25) {$B$};
		\node [style=circle, scale=1.5] (37) at (2.75, -1) {};
		\node [style=none] (38) at (2.75, -1) {$s'$};
		\node [style=circle, scale=1.5] (41) at (3.75, -1) {};
		\node [style=none] (42) at (3.75, -1) {$s'$};
		\node [style=circle, scale=1.5] (43) at (2.75, 1.25) {};
		\node [style=none] (44) at (2.75, 1.25) {$s$};
		\node [style=none] (45) at (2.75, 0.75) {};
		\node [style=none] (46) at (2.75, 0.25) {};
	\end{pgfonlayer}
	\begin{pgfonlayer}{edgelayer}
		\draw (25) to (15.center);
		\draw [in=0, out=105] (27) to (17.center);
		\draw (16.center) to (15.center);
		\draw (18.center) to (19.center);
		\draw (19.center) to (20.center);
		\draw (20.center) to (18.center);
		\draw [style=filled] (18.center) to (21);
		\draw [style=filled] (25) to (37);
		\draw [style=filled] (27) to (41);
		\draw (30.center) to (41);
		\draw (29.center) to (37);
		\draw (43) to (21);
		\draw (43) to (16.center);
		\draw [bend left=90, looseness=2.00] (45.center) to (46.center);
	\end{pgfonlayer}
\end{tikzpicture} = \begin{tikzpicture}
	\begin{pgfonlayer}{nodelayer}
		\node [style=none] (15) at (2.75, -0.25) {};
		\node [style=none] (16) at (2.75, 0) {};
		\node [style=none] (17) at (3, -0.25) {};
		\node [style=none] (18) at (2.75, 1) {};
		\node [style=none] (19) at (2.5, 1.25) {};
		\node [style=none] (20) at (3, 1.25) {};
		\node [style=circle, scale=1.5] (21) at (2.75, 0.5) {};
		\node [style=none] (23) at (2.75, 0.5) {$\v$};
		\node [style=circle, scale=1.5] (25) at (2.75, -1.25) {};
		\node [style=none] (26) at (2.75, -1.25) {$\u$};
		\node [style=circle, scale=1.5] (27) at (3.75, -1.25) {};
		\node [style=none] (28) at (3.75, -1.25) {$e_B$};
		\node [style=none] (29) at (2.75, -3.25) {};
		\node [style=none] (30) at (3.75, -3.25) {};
		\node [style=none] (31) at (2.5, -3) {$E$};
		\node [style=none] (32) at (3.75, -0.25) {$B$};
		\node [style=none] (33) at (2.5, -0.5) {$A$};
		\node [style=none] (34) at (4, -3) {$E$};
		\node [style=none] (36) at (2.5, 1) {$B$};
		\node [style=circle, scale=1.5] (39) at (2.75, -2.5) {};
		\node [style=none] (40) at (2.75, -2.5) {$r'$};
		\node [style=circle, scale=1.5] (43) at (3.75, -2.5) {};
		\node [style=none] (44) at (3.75, -2.5) {$r'$};
		\node [style=none] (45) at (2.75, 0) {};
		\node [style=none] (46) at (2.75, -0.5) {};
	\end{pgfonlayer}
	\begin{pgfonlayer}{edgelayer}
		\draw (25) to (15.center);
		\draw [in=0, out=90, looseness=1.25] (27) to (17.center);
		\draw (16.center) to (15.center);
		\draw (16.center) to (21);
		\draw (18.center) to (19.center);
		\draw (19.center) to (20.center);
		\draw (20.center) to (18.center);
		\draw [style=filled] (18.center) to (21);
		\draw [style=filled] (39) to (29.center);
		\draw [style=filled] (43) to (30.center);
		\draw [style=filled] (25) to (39);
		\draw [style=filled] (27) to (43);
		\draw [bend left=90, looseness=1.75] (45.center) to (46.center);
	\end{pgfonlayer}
\end{tikzpicture} = \begin{tikzpicture}
	\begin{pgfonlayer}{nodelayer}
		\node [style=none] (86) at (9.5, 5) {};
		\node [style=none] (87) at (9.25, 5.25) {};
		\node [style=none] (88) at (9.75, 5.25) {};
		\node [style=circle, scale=1.5] (89) at (10.25, 3.5) {};
		\node [style=none] (90) at (9.5, 1) {};
		\node [style=none] (91) at (10.25, 1) {};
		\node [style=none] (92) at (9.25, 1.25) {$E$};
		\node [style=none] (93) at (10.5, 1.25) {$E$};
		\node [style=circle, scale=1.5] (94) at (9.5, 3.5) {};
		\node [style=none] (95) at (10.5, 5.25) {};
		\node [style=none] (96) at (10, 5.25) {};
		\node [style=none] (97) at (10.25, 5) {};
		\node [style=none] (98) at (9.5, 3.5) {$r'$};
		\node [style=none] (99) at (10.25, 3.5) {$r'$};
	\end{pgfonlayer}
	\begin{pgfonlayer}{edgelayer}
		\draw (86.center) to (87.center);
		\draw (87.center) to (88.center);
		\draw (88.center) to (86.center);
		\draw (91.center) to (89);
		\draw (90.center) to (94);
		\draw (97.center) to (96.center);
		\draw (96.center) to (95.center);
		\draw (95.center) to (97.center);
		\draw (97.center) to (89);
		\draw (86.center) to (94);
	\end{pgfonlayer}
\end{tikzpicture} = \begin{tikzpicture}
	\begin{pgfonlayer}{nodelayer}
		\node [style=none] (4) at (-0.25, -3) {};
		\node [style=none] (6) at (0, -2.75) {$E$};
		\node [style=none] (7) at (0, 1.25) {};
		\node [style=none] (8) at (-0.5, 1.25) {};
		\node [style=none] (9) at (-0.25, 1) {};
		\node [style=none] (10) at (0.75, -3) {};
		\node [style=none] (11) at (1, -2.75) {$E$};
		\node [style=none] (12) at (1, 1.25) {};
		\node [style=none] (13) at (0.5, 1.25) {};
		\node [style=none] (14) at (0.75, 1) {};
	\end{pgfonlayer}
	\begin{pgfonlayer}{edgelayer}
		\draw (4.center) to (9.center);
		\draw [style=filled] (9.center)
				to (8.center)
				to (7.center)
				to cycle;
		\draw (10.center) to (14.center);
		\draw [style=filled] (14.center)
				to (13.center)
				to (12.center)
				to cycle;
	\end{pgfonlayer}
\end{tikzpicture} $
\end{description}
For the converse assume that, $E \linbialgbtik E$ is a complementary system 
given by splitting a sectional or retractional binary idempotent $(\u,\v)$ on a linear bialgebra $A \linbialgwtik B$.
We show that $A \linbialgwtik B$ satisfies the equations, $(a)$-$(c)$ given in the statement of the Lemma.

\begin{enumerate}[(a):]
\item  $\begin{tikzpicture}
	\begin{pgfonlayer}{nodelayer}
		\node [style=none] (28) at (6.5, 0.75) {};
		\node [style=none] (29) at (6.75, 1) {};
		\node [style=none] (30) at (6.25, 1) {};
		\node [style=none] (31) at (6.5, -1) {};
		\node [style=none] (32) at (6.5, -1.25) {};
		\node [style=none] (33) at (6.5, -1) {};
		\node [style=circle] (34) at (6.5, -3) {};
		\node [style=none] (35) at (6.25, -1) {};
		\node [style=none] (36) at (5.25, 0.75) {$A$};
		\node [style=none] (37) at (7, -1) {$B$};
		\node [style=none] (38) at (5.5, 1) {};
		\node [style=circle, scale=1.8] (39) at (6.5, 0) {};
		\node [style=none] (40) at (6.5, 0) {$e_B$};
		\node [style=none] (41) at (7, 0.5) {$B$};
		\node [style=circle, scale=1.8] (42) at (6.5, -2) {};
		\node [style=none] (43) at (6.5, -2) {$e_B$};
		\node [style=circle, scale=1.8] (44) at (5.5, 0) {};
		\node [style=none] (45) at (5.5, 0) {$e_A$};
		\node [style=none] (46) at (6.5, -0.75) {};
		\node [style=none] (47) at (6.5, -1.25) {};
	\end{pgfonlayer}
	\begin{pgfonlayer}{edgelayer}
		\draw (28.center) to (29.center);
		\draw (28.center) to (30.center);
		\draw (30.center) to (29.center);
		\draw (32.center) to (33.center);
		\draw (28.center) to (39);
		\draw (32.center) to (42);
		\draw (34) to (42);
		\draw (39) to (33.center);
		\draw [in=165, out=-90, looseness=1.25] (44) to (35.center);
		\draw (38.center) to (44);
		\draw [bend right=90, looseness=1.75] (46.center) to (47.center);
	\end{pgfonlayer}
\end{tikzpicture} = \begin{tikzpicture}
	\begin{pgfonlayer}{nodelayer}
		\node [style=none] (0) at (-2.25, 6) {};
		\node [style=none] (1) at (-2.75, 6) {};
		\node [style=none] (2) at (-2.5, 5.75) {};
		\node [style=circle, scale=1.5] (3) at (-2.5, 5.25) {};
		\node [style=circle, scale=1.5] (4) at (-2.5, 4.5) {};
		\node [style=none] (5) at (-2.5, 3.75) {};
		\node [style=none] (6) at (-2.5, 3.5) {};
		\node [style=none] (7) at (-2.75, 3.75) {};
		\node [style=circle, scale=1.5] (8) at (-2.5, 3) {};
		\node [style=circle, scale=1.5] (9) at (-2.5, 2.25) {};
		\node [style=circle] (10) at (-2.5, 1.5) {};
		\node [style=none] (11) at (-3.5, 4.5) {};
		\node [style=none] (12) at (-3.5, 6) {};
		\node [style=none] (13) at (-2.5, 5.25) {$r'$};
		\node [style=none] (14) at (-2.5, 4.5) {$s'$};
		\node [style=none] (15) at (-2.5, 3) {$r'$};
		\node [style=none] (16) at (-2.5, 2.25) {$s'$};
		\node [style=circle, scale=1.5] (17) at (-3.5, 5.5) {};
		\node [style=none] (18) at (-3.5, 5.5) {$r$};
		\node [style=none] (20) at (-2, 4.75) {$E$};
		\node [style=none] (21) at (-2, 5.5) {$B$};
		\node [style=none] (24) at (-4, 5.75) {$A$};
		\node [style=none] (25) at (-4, 5) {$E$};
		\node [style=none] (26) at (-2, 3.75) {$B$};
		\node [style=none] (27) at (-2, 2) {$B$};
		\node [style=circle, scale=1.5] (28) at (-3.5, 4.75) {};
		\node [style=none] (29) at (-3.5, 4.75) {$s$};
		\node [style=none] (30) at (-2.5, 4) {};
		\node [style=none] (31) at (-2.5, 3.5) {};
	\end{pgfonlayer}
	\begin{pgfonlayer}{edgelayer}
		\draw (6.center) to (5.center);
		\draw [in=-90, out=-180, looseness=1.25] (7.center) to (11.center);
		\draw (5.center) to (4);
		\draw (4) to (3);
		\draw (3) to (2.center);
		\draw (2.center) to (0.center);
		\draw (0.center) to (1.center);
		\draw (1.center) to (2.center);
		\draw (6.center) to (8);
		\draw (8) to (9);
		\draw (9) to (10);
		\draw (12.center) to (17);
		\draw (11.center) to (28);
		\draw (28) to (17);
		\draw [bend right=90, looseness=1.75] (30.center) to (31.center);
	\end{pgfonlayer}
\end{tikzpicture} =  \begin{tikzpicture}
	\begin{pgfonlayer}{nodelayer}
		\node [style=none] (28) at (6.25, 4.5) {};
		\node [style=none] (29) at (6, 4.25) {};
		\node [style=none] (30) at (5.75, 4.5) {};
		\node [style=none] (31) at (6, 2.25) {};
		\node [style=none] (32) at (6, 2) {};
		\node [style=none] (33) at (5.75, 2) {};
		\node [style=black] (34) at (6, 0) {};
		\node [style=none] (35) at (4.75, 3.25) {};
		\node [style=none] (36) at (6.5, 4) {$E$};
		\node [style=none] (37) at (6.5, 0.5) {$E$};
		\node [style=none] (38) at (4.25, 3.25) {$E$};
		\node [style=none] (39) at (4.75, 4.5) {};
		\node [style=none] (40) at (6, 2.25) {};
		\node [style=none] (41) at (6, 1.75) {};
		\node [style=onehalfcircle] (42) at (4.75, 3.75) {};
		\node [style=none] (43) at (4.75, 3.75) {$r$};
		\node [style=none] (44) at (4.25, 4.25) {$A$};
	\end{pgfonlayer}
	\begin{pgfonlayer}{edgelayer}
		\draw [style=filled] (29.center)
				to (28.center)
				to (30.center)
				to cycle;
		\draw [style=filled] (32.center) to (31.center);
		\draw [style=filled] (29.center) to (31.center);
		\draw [style=filled] (32.center) to (34);
		\draw [in=-90, out=165, looseness=1.25] (33.center) to (35.center);
		\draw [style=filled] (41.center)
				to (40.center)
				to [bend right=90, looseness=1.75] cycle;
		\draw (39.center) to (42);
		\draw (42) to (35.center);
	\end{pgfonlayer}
\end{tikzpicture} = \begin{tikzpicture}
	\begin{pgfonlayer}{nodelayer}
		\node [style=none] (28) at (5, 0) {};
		\node [style=none] (29) at (4.75, 0.25) {};
		\node [style=none] (30) at (4.5, 0) {};
		\node [style=none] (35) at (4.75, 2.25) {};
		\node [style=none] (36) at (4.25, 0.75) {$E$};
		\node [style=none] (39) at (4.75, 4.5) {};
		\node [style=onehalfcircle] (42) at (4.75, 2.75) {};
		\node [style=none] (43) at (4.75, 2.75) {$r$};
		\node [style=none] (44) at (4.25, 4.25) {$A$};
	\end{pgfonlayer}
	\begin{pgfonlayer}{edgelayer}
		\draw [style=filled] (29.center)
				to (28.center)
				to (30.center)
				to cycle;
		\draw (39.center) to (42);
		\draw (42) to (35.center);
		\draw (35.center) to (29.center);
	\end{pgfonlayer}
\end{tikzpicture} = \begin{tikzpicture}
	\begin{pgfonlayer}{nodelayer}
		\node [style=none] (29) at (4.75, 0.25) {};
		\node [style=none] (30) at (4.5, 0) {};
		\node [style=none] (36) at (4.25, 0.75) {$E$};
		\node [style=none] (39) at (4.75, 4.5) {};
		\node [style=onehalfcircle] (42) at (4.75, 3.5) {};
		\node [style=none] (43) at (4.75, 3.5) {$r$};
		\node [style=none] (44) at (4.25, 4.25) {$A$};
		\node [style=onehalfcircle] (45) at (4.75, 1.5) {};
		\node [style=none] (46) at (4.75, 1.5) {$s$};
		\node [style=none] (47) at (5, 0) {};
	\end{pgfonlayer}
	\begin{pgfonlayer}{edgelayer}
		\draw [style=filled] (30.center) to (29.center);
		\draw (39.center) to (42);
		\draw (42) to (45);
		\draw (45) to (29.center);
		\draw (29.center) to (47.center);
		\draw (47.center) to (30.center);
	\end{pgfonlayer}
\end{tikzpicture} = \begin{tikzpicture}
	\begin{pgfonlayer}{nodelayer}
		\node [style=none] (29) at (4.75, 0.25) {};
		\node [style=none] (30) at (4.5, 0) {};
		\node [style=none] (36) at (4.25, 0.75) {$E$};
		\node [style=none] (39) at (4.75, 4.5) {};
		\node [style=none] (44) at (4.25, 4.25) {$A$};
		\node [style=onehalfcircle] (45) at (4.75, 2.25) {};
		\node [style=none] (46) at (4.75, 2.25) {$e_A$};
		\node [style=none] (47) at (5, 0) {};
	\end{pgfonlayer}
	\begin{pgfonlayer}{edgelayer}
		\draw [style=filled] (30.center) to (29.center);
		\draw (45) to (29.center);
		\draw (29.center) to (47.center);
		\draw (47.center) to (30.center);
		\draw (39.center) to (45);
	\end{pgfonlayer}
\end{tikzpicture}$
 \item  $\begin{tikzpicture}
	\begin{pgfonlayer}{nodelayer}
		\node [style=none] (33) at (5.5, 4) {};
		\node [style=none] (34) at (6, 4) {};
		\node [style=none] (36) at (5.75, 4.25) {};
		\node [style=none] (37) at (5.75, 6) {};
		\node [style=circle] (38) at (5, 2) {};
		\node [style=circle, scale=1.5] (39) at (5.75, 5.25) {};
		\node [style=none] (40) at (5.75, 5.25) {$\v$};
		\node [style=none] (41) at (6.25, 5.75) {$B$};
		\node [style=circle, scale=1.5] (42) at (5, 2.75) {};
		\node [style=none] (43) at (5, 2.75) {$\u$};
		\node [style=none] (44) at (5.5, 6.25) {};
		\node [style=none] (45) at (6, 6.25) {};
		\node [style=none] (47) at (6.25, 4.5) {$A$};
		\node [style=none] (48) at (6.5, 2) {};
		\node [style=circle, scale=1.5] (50) at (6.5, 2.75) {};
		\node [style=none] (51) at (6.5, 2.75) {$e_A$};
		\node [style=none] (52) at (7, 2.25) {$A$};
	\end{pgfonlayer}
	\begin{pgfonlayer}{edgelayer}
		\draw (34.center) to (33.center);
		\draw (33.center) to (36.center);
		\draw (36.center) to (34.center);
		\draw (37.center) to (39);
		\draw (36.center) to (39);
		\draw (38) to (42);
		\draw (44.center) to (45.center);
		\draw (45.center) to (37.center);
		\draw (37.center) to (44.center);
		\draw (48.center) to (50);
		\draw [in=90, out=-150, looseness=1.25] (33.center) to (42);
		\draw [in=90, out=-30, looseness=1.25] (34.center) to (50);
	\end{pgfonlayer}
\end{tikzpicture} = \begin{tikzpicture}
	\begin{pgfonlayer}{nodelayer}
		\node [style=none] (0) at (-1.25, -0.75) {};
		\node [style=none] (1) at (-1, -0.5) {};
		\node [style=none] (2) at (-0.75, -0.75) {};
		\node [style=none] (3) at (-1, 1.25) {};
		\node [style=none] (4) at (-1.25, 1.5) {};
		\node [style=none] (5) at (-0.75, 1.5) {};
		\node [style=none] (6) at (-0.25, -3) {};
		\node [style=circle] (7) at (-1.75, -3) {};
		\node [style=circle, scale=1.5] (8) at (-1, 0) {};
		\node [style=circle, scale=1.5] (9) at (-1, 0.75) {};
		\node [style=circle, scale=1.5] (10) at (-1.75, -2.25) {};
		\node [style=circle, scale=1.5] (12) at (-1.75, -1.5) {};
		\node [style=none] (13) at (-1.75, -2.25) {$s'$};
		\node [style=none] (14) at (-1.75, -1.5) {$r$};
		\node [style=none] (15) at (-1, 0) {$s$};
		\node [style=none] (16) at (-1, 0.75) {$r'$};
		\node [style=none] (18) at (-2.25, -2.75) {$A$};
		\node [style=none] (19) at (-2.25, -1.75) {$E$};
		\node [style=none] (20) at (-1.75, -0.65) {$A$};
		\node [style=none] (21) at (-0.25, -0.5) {$A$};
		\node [style=none] (22) at (0, -2.75) {$E$};
		\node [style=none] (23) at (-0.5, 0.25) {$E$};
		\node [style=none] (24) at (-0.5, 1) {$B$};
		\node [style=circle, scale=1.5] (25) at (-0.25, -2.25) {};
		\node [style=circle, scale=1.5] (26) at (-0.25, -1.5) {};
		\node [style=none] (27) at (-0.25, -2.25) {$s$};
		\node [style=none] (28) at (-0.25, -1.5) {$r$};
	\end{pgfonlayer}
	\begin{pgfonlayer}{edgelayer}
		\draw (0.center) to (2.center);
		\draw (2.center) to (1.center);
		\draw (1.center) to (0.center);
		\draw (3.center) to (4.center);
		\draw (4.center) to (5.center);
		\draw (5.center) to (3.center);
		\draw (7) to (10);
		\draw (10) to (12);
		\draw [in=-165, out=90] (12) to (0.center);
		\draw (1.center) to (8);
		\draw (8) to (9);
		\draw (9) to (3.center);
		\draw (25) to (26);
		\draw [style=none, in=90, out=-15] (2.center) to (26);
		\draw [style=none] (25) to (6.center);
	\end{pgfonlayer}
\end{tikzpicture} = \begin{tikzpicture}
	\begin{pgfonlayer}{nodelayer}
		\node [style=none] (0) at (-1.25, -0.5) {};
		\node [style=none] (1) at (-1, -0.25) {};
		\node [style=none] (2) at (-0.75, -0.5) {};
		\node [style=none] (3) at (-1, 1.25) {};
		\node [style=none] (4) at (-1.25, 1.5) {};
		\node [style=none] (5) at (-0.75, 1.5) {};
		\node [style=none] (6) at (-0.25, -1.5) {};
		\node [style=circle, fill=black] (7) at (-1.75, -2.5) {};
		\node [style=none] (8) at (-0.25, -2.75) {};
		\node [style=none] (9) at (-1.75, -1.5) {};
		\node [style=onehalfcircle] (10) at (-0.25, -2) {};
		\node [style=none] (11) at (-0.25, -2) {$s$};
	\end{pgfonlayer}
	\begin{pgfonlayer}{edgelayer}
		\draw (1.center) to (3.center);
		\draw [in=90, out=-45] (2.center) to (6.center);
		\draw [style=filled] (2.center)
				to (1.center)
				to (0.center)
				to cycle;
		\draw [style=filled] (3.center)
				to (4.center)
				to (5.center)
				to cycle;
		\draw [in=90, out=-135] (0.center) to (9.center);
		\draw (9.center) to (7);
		\draw (6.center) to (10);
		\draw (10) to (8.center);
	\end{pgfonlayer}
\end{tikzpicture} = \begin{tikzpicture}
	\begin{pgfonlayer}{nodelayer}
		\node [style=none] (6) at (-0.25, 0.5) {};
		\node [style=none] (8) at (-0.25, -3) {};
		\node [style=onehalfcircle] (10) at (-0.25, -0.75) {};
		\node [style=none] (11) at (-0.25, -0.75) {$s$};
		\node [style=black] (12) at (-0.25, 1.25) {};
		\node [style=none] (13) at (0.25, -2.75) {$B$};
	\end{pgfonlayer}
	\begin{pgfonlayer}{edgelayer}
		\draw (6.center) to (10);
		\draw (10) to (8.center);
		\draw (12) to (6.center);
	\end{pgfonlayer}
\end{tikzpicture} =\begin{tikzpicture}
	\begin{pgfonlayer}{nodelayer}
		\node [style=none] (8) at (-0.25, -3) {};
		\node [style=onehalfcircle] (10) at (-0.25, -0.75) {};
		\node [style=none] (11) at (-0.25, -0.75) {$e_A$};
		\node [style=none] (13) at (0.25, -2.75) {$B$};
		\node [style=circle] (14) at (-0.25, 1.25) {};
	\end{pgfonlayer}
	\begin{pgfonlayer}{edgelayer}
		\draw (10) to (8.center);
		\draw (10) to (14);
	\end{pgfonlayer}
\end{tikzpicture} $
\item $\begin{tikzpicture}
	\begin{pgfonlayer}{nodelayer}
		\node [style=none] (45) at (0.25, 0.25) {};
		\node [style=none] (46) at (0, 0) {};
		\node [style=none] (47) at (0, -1.75) {};
		\node [style=none] (48) at (0, 0.75) {};
		\node [style=none] (49) at (0, 2.5) {};
		\node [style=none] (50) at (1, -1.75) {};
		\node [style=circle, scale=1.5] (51) at (0, 1.75) {};
		\node [style=none] (52) at (0, 1.75) {$\v$};
		\node [style=none] (53) at (1.5, -0.5) {$B$};
		\node [style=none] (54) at (0.25, 2.75) {};
		\node [style=none] (55) at (-0.25, 2.75) {};
		\node [style=none] (56) at (-0.5, -1.25) {$B$};
		\node [style=none] (57) at (-0.5, 2.25) {$B$};
		\node [style=circle, scale=1.5] (58) at (0, -1) {};
		\node [style=none] (59) at (0, -1) {$\u$};
		\node [style=none] (60) at (1, -0.5) {};
		\node [style=none] (61) at (-0.5, 1) {$A$};
		\node [style=circle, scale=1.5] (62) at (1, -1) {};
		\node [style=none] (63) at (1, -1) {$e_B$};
		\node [style=none] (64) at (0, 0.5) {};
		\node [style=none] (65) at (0, 0) {};
	\end{pgfonlayer}
	\begin{pgfonlayer}{edgelayer}
		\draw (49.center) to (51);
		\draw (48.center) to (51);
		\draw (54.center) to (55.center);
		\draw (55.center) to (49.center);
		\draw (49.center) to (54.center);
		\draw (47.center) to (58);
		\draw (46.center) to (58);
		\draw [in=0, out=105, looseness=1.25] (60.center) to (45.center);
		\draw (50.center) to (62);
		\draw (60.center) to (62);
		\draw [bend left=90, looseness=1.75] (64.center) to (65.center);
		\draw (48.center) to (46.center);
	\end{pgfonlayer}
\end{tikzpicture} = \begin{tikzpicture}
	\begin{pgfonlayer}{nodelayer}
		\node [style=none] (15) at (2.75, 0.5) {};
		\node [style=none] (16) at (2.75, 0.75) {};
		\node [style=none] (17) at (3, 0.5) {};
		\node [style=none] (18) at (2.75, 2.5) {};
		\node [style=none] (19) at (2.5, 2.75) {};
		\node [style=none] (20) at (3, 2.75) {};
		\node [style=circle, scale=1.5] (21) at (2.75, 2) {};
		\node [style=none] (23) at (2.75, 2) {$r'$};
		\node [style=circle, scale=1.5] (25) at (2.75, -0.25) {};
		\node [style=none] (26) at (2.75, -0.25) {$r$};
		\node [style=circle, scale=1.5] (27) at (3.75, -0.25) {};
		\node [style=none] (28) at (3.75, -0.25) {$r'$};
		\node [style=none] (29) at (2.75, -1.75) {};
		\node [style=none] (30) at (3.75, -1.75) {};
		\node [style=none] (31) at (2.25, -1.25) {$B$};
		\node [style=none] (32) at (4, 0.25) {$B$};
		\node [style=none] (33) at (2.25, 0.25) {$A$};
		\node [style=none] (34) at (4.25, -1.25) {$B$};
		\node [style=none] (36) at (2.25, 2.25) {$B$};
		\node [style=circle, scale=1.5] (37) at (2.75, -1) {};
		\node [style=none] (38) at (2.75, -1) {$s'$};
		\node [style=circle, scale=1.5] (41) at (3.75, -1) {};
		\node [style=none] (42) at (3.75, -1) {$s'$};
		\node [style=circle, scale=1.5] (43) at (2.75, 1.25) {};
		\node [style=none] (44) at (2.75, 1.25) {$s$};
		\node [style=none] (45) at (2.75, 0.75) {};
		\node [style=none] (46) at (2.75, 0.25) {};
	\end{pgfonlayer}
	\begin{pgfonlayer}{edgelayer}
		\draw (25) to (15.center);
		\draw [in=0, out=90, looseness=1.25] (27) to (17.center);
		\draw (16.center) to (15.center);
		\draw (18.center) to (19.center);
		\draw (19.center) to (20.center);
		\draw (20.center) to (18.center);
		\draw [style=filled] (18.center) to (21);
		\draw [style=filled] (25) to (37);
		\draw [style=filled] (27) to (41);
		\draw (30.center) to (41);
		\draw (29.center) to (37);
		\draw (43) to (21);
		\draw (43) to (16.center);
		\draw [bend left=90, looseness=2.00] (45.center) to (46.center);
	\end{pgfonlayer}
\end{tikzpicture} = \begin{tikzpicture}
	\begin{pgfonlayer}{nodelayer}
		\node [style=none] (3) at (2.75, 1) {};
		\node [style=none] (4) at (2.75, -1.5) {};
		\node [style=none] (5) at (3.75, -1.5) {};
		\node [style=none] (6) at (4, -2.75) {$B$};
		\node [style=none] (7) at (2.5, -2.75) {$B$};
		\node [style=none] (8) at (2.75, -0.5) {};
		\node [style=none] (9) at (2.5, 1.25) {};
		\node [style=none] (10) at (3, 1.25) {};
		\node [style=onehalfcircle] (11) at (2.75, -2) {};
		\node [style=onehalfcircle] (12) at (3.75, -2) {};
		\node [style=none] (13) at (2.75, -3) {};
		\node [style=none] (14) at (3.75, -3) {};
		\node [style=none] (15) at (2.75, -2) {$s'$};
		\node [style=none] (16) at (3.75, -2) {$s'$};
		\node [style=none] (18) at (3, -0.75) {};
		\node [style=none] (19) at (2.75, -0.5) {};
		\node [style=none] (20) at (2.75, -1) {};
		\node [style=none] (21) at (3.75, -1.5) {};
	\end{pgfonlayer}
	\begin{pgfonlayer}{edgelayer}
		\draw [style=filled] (8.center) to (3.center);
		\draw [style=filled] (9.center)
				to (10.center)
				to (3.center)
				to cycle;
		\draw (14.center) to (12);
		\draw (12) to (5.center);
		\draw (13.center) to (11);
		\draw (11) to (4.center);
		\draw [style=filled, bend left=90, looseness=1.75] (19.center) to (20.center);
		\draw [style=filled] (19.center) to (4.center);
		\draw [in=0, out=90] (21.center) to (18.center);
	\end{pgfonlayer}
\end{tikzpicture} = \begin{tikzpicture}
	\begin{pgfonlayer}{nodelayer}
		\node [style=none] (3) at (2.75, 1) {};
		\node [style=none] (6) at (4, -2.75) {$B$};
		\node [style=none] (7) at (2.5, -2.75) {$B$};
		\node [style=none] (8) at (2.75, 0) {};
		\node [style=none] (9) at (2.5, 1.25) {};
		\node [style=none] (10) at (3, 1.25) {};
		\node [style=onehalfcircle] (11) at (2.75, -1.25) {};
		\node [style=onehalfcircle] (12) at (3.75, -1.25) {};
		\node [style=none] (13) at (2.75, -3) {};
		\node [style=none] (14) at (3.75, -3) {};
		\node [style=none] (15) at (2.75, -1.25) {$s'$};
		\node [style=none] (16) at (3.75, -1.25) {$s'$};
		\node [style=none] (18) at (3.75, 1) {};
		\node [style=none] (19) at (3.75, 0) {};
		\node [style=none] (20) at (3.5, 1.25) {};
		\node [style=none] (21) at (4, 1.25) {};
	\end{pgfonlayer}
	\begin{pgfonlayer}{edgelayer}
		\draw [style=filled] (8.center) to (3.center);
		\draw [style=filled] (9.center)
				to (10.center)
				to (3.center)
				to cycle;
		\draw (14.center) to (12);
		\draw (13.center) to (11);
		\draw [style=filled] (19.center) to (18.center);
		\draw [style=filled] (20.center)
				to (21.center)
				to (18.center)
				to cycle;
		\draw [style=filled] (11) to (8.center);
		\draw [style=filled] (12) to (19.center);
	\end{pgfonlayer}
\end{tikzpicture} = \begin{tikzpicture}
	\begin{pgfonlayer}{nodelayer}
		\node [style=none] (3) at (2.75, 1) {};
		\node [style=none] (6) at (4, -2.75) {$B$};
		\node [style=none] (7) at (2.5, -2.75) {$B$};
		\node [style=none] (9) at (2.5, 1.25) {};
		\node [style=onehalfcircle] (11) at (2.75, -1.25) {};
		\node [style=onehalfcircle] (12) at (3.75, -1.25) {};
		\node [style=none] (13) at (2.75, -3) {};
		\node [style=none] (14) at (3.75, -3) {};
		\node [style=none] (15) at (2.75, -1.25) {$s'$};
		\node [style=none] (16) at (3.75, -1.25) {$s'$};
		\node [style=none] (18) at (3.75, 1) {};
		\node [style=none] (20) at (3.5, 1.25) {};
		\node [style=onehalfcircle] (22) at (2.75, 0) {};
		\node [style=onehalfcircle] (23) at (3.75, 0) {};
		\node [style=none] (24) at (2.75, 0) {$r'$};
		\node [style=none] (25) at (3.75, 0) {$r'$};
		\node [style=none] (26) at (3, 1.25) {};
		\node [style=none] (27) at (4, 1.25) {};
	\end{pgfonlayer}
	\begin{pgfonlayer}{edgelayer}
		\draw [style=filled] (3.center) to (9.center);
		\draw (14.center) to (12);
		\draw (13.center) to (11);
		\draw [style=filled] (18.center) to (20.center);
		\draw (26.center) to (9.center);
		\draw (26.center) to (3.center);
		\draw (27.center) to (20.center);
		\draw (27.center) to (18.center);
		\draw (18.center) to (23);
		\draw (22) to (3.center);
		\draw (12) to (23);
		\draw (11) to (22);
	\end{pgfonlayer}
\end{tikzpicture} = \begin{tikzpicture}
	\begin{pgfonlayer}{nodelayer}
		\node [style=none] (3) at (2.75, 1) {};
		\node [style=none] (6) at (4, -2.75) {$B$};
		\node [style=none] (7) at (2.5, -2.75) {$B$};
		\node [style=none] (9) at (2.5, 1.25) {};
		\node [style=none] (13) at (2.75, -3) {};
		\node [style=none] (14) at (3.75, -3) {};
		\node [style=none] (18) at (3.75, 1) {};
		\node [style=none] (20) at (3.5, 1.25) {};
		\node [style=onehalfcircle] (22) at (2.75, -0.5) {};
		\node [style=onehalfcircle] (23) at (3.75, -0.5) {};
		\node [style=none] (24) at (2.75, -0.5) {$e_B$};
		\node [style=none] (25) at (3.75, -0.5) {$e_B$};
		\node [style=none] (26) at (3, 1.25) {};
		\node [style=none] (27) at (4, 1.25) {};
	\end{pgfonlayer}
	\begin{pgfonlayer}{edgelayer}
		\draw [style=filled] (3.center) to (9.center);
		\draw [style=filled] (18.center) to (20.center);
		\draw (26.center) to (9.center);
		\draw (26.center) to (3.center);
		\draw (27.center) to (20.center);
		\draw (27.center) to (18.center);
		\draw (18.center) to (23);
		\draw (22) to (3.center);
		\draw (13.center) to (22);
		\draw (14.center) to (23);
	\end{pgfonlayer}
\end{tikzpicture}  $
\end{enumerate}
\end{proof}
In the next section, we provide an example of Lemma \ref{Lemma: complementary idempotent} using 
exponential modalities.

%%%%%%%%%%%%%%%%%%%%%%%%%%%%%%%%%%%%%%%%%%%%%%%


\section{Exponential modalities}
\label{Sec: exp modalities}

Linear logic \cite{Gir87} treats formulae/types as depletable resources, hence the name `linear' logic. 
However, it also allows for non-linear types, that is, resources that can be renewed indefinitely and erased,
by means of  the exponential operator, $!$ (read as the `bang').  
 The aim of this section is to show that there is a relationship between the $!$ operator of 
 linear logic and complementarity, furthermore, a relationship which suggests a different 
 perspective on measurement and complementarity in quantum systems.

\subsection{Exponential modalities for monoidal categories}

We first explore the semantics of the $!$ operator in symmetric monoidal categories (SMCs).  
In a SMC, the $!$ operator is a functor which is a comonad. The objects to which 
the $!$ functor has been applied can be duplicated and erased, thus, for all objects $A$, $!A$ is a 
cocommutative comonoid. Moreover, the comonad structure behaves coherently with the 
comonoid structure. Such a functor is called a coalgebra comodality. Finally, the 
$!$-coalgebra modality behaves coherently with the tensor product, 
hence is monoidal.

\begin{definition}\cite{BCS06}
A {\bf colagebra modality} for a symmetric monoidal category consists of a comonad $(!, \delta, \epsilon)$ 
and natural transformations $\Delta_A: !A \to !A \ox !A$ and $\tricounit{0.65}_A: !A \to I$ such that $(!A, \Delta_A, e_A)$ 
is a cocommutative comonoid and $\delta_A: !A \to !!A$ preserves the comultiplication, that is, 
\[ \xymatrixcolsep{4pc} 
\xymatrix{
	!A  \ar[r]^{\delta} \ar[d]_{\Delta_A} & !!A  \ar[d]^{\Delta_{!A}} \\ 
	!A \ox !A \ar[r]_{\delta \ox \delta} & !!A \ox !!A} \] 
\end{definition}

For a coalgebra modality, $(!, \delta, \epsilon)$, and for any object $A$, 
$\delta_A: !A \to !!A$ is a comonoid morphism since it can be shown that $\delta$ preserves 
the counit.

%Additive symmetric monoidal categories with a coalgebra modality and a 
%deriving transformation are sufficient to axiomatize the notion of differentiation \cite{BCS06, BCL19}. 
%Such categories are referred to as differential categories. 

\begin{definition}\cite{BCS15}
A coalgebra modality, $(!, \delta, \epsilon, \Delta, \tricounit{0.65})$, is {\bf monoidal} if 
$(!, \delta, \epsilon)$ is a monoidal comonad, (that is, $(!, m_\ox, m_I)$ is a monoidal functor, and 
$\delta$ and $\epsilon$ are monoidal transformations ), such that 
 $\Delta$ and $e$ are monoidal transformations:
\[ \xymatrixcolsep{4pc} 
\xymatrix{ !A \ox !B \ar[r]^{m_\ox} \ar[d]_{\Delta \ox \Delta} & ! (A \ox B) \ar[dd]^{\Delta} \\
    !A \ox !A \ox !B \ox !B \ar[d]_{1 \ox c_\ox \ox 1} &  \\ 
	!A \ox !B \ox !A \ox !B \ar[r]_{m_\ox \ox m_\ox} & !(A \ox B) \ox !(A \ox B) }
	~~~~~~~~~~ 
	\xymatrix{ 
		!A \ox !B \ar[r]^{m_\ox} \ar[d]_{m_I \ox m_I} & !(A \ox B) \ar[d]^{m_I} \\ 
	I \ox I \ar@{=}[r] & I } \]
\[   \xymatrixcolsep{4pc}
	\xymatrix{
I \ar[r]^{m_I} \ar@{=}[d] & !I \ar[d]^{\Delta} \\
I \ox I \ar[r]_{m_I \ox m_I} & !I \ox !I } ~~~~~~~~~~~~
\xymatrix{
	I \ar[r]^{m_I} \ar@{=}[dr] & !I \ar[d]^{\tricounit{0.65}_I} \\
	& I } \]
	and $\Delta$ and $e$ are $!$-colagebra morphisms:
\[  \xymatrixcolsep{4pc}
	 \xymatrix{
	!A \ar[rr]^{\delta} \ar[d]_{\Delta} & & !!A \ar[d]^{!\Delta} \\
	!A \ox !A \ar[r]_{\delta \ox \delta} & !!A \ox !!A \ar[r]_{m_\ox} & !(!A \ox !A)}
	~~~~~~~~~ \xymatrix{ 
	!A \ar[d]_{e} \ar[r]^{\delta} & !!A \ar[d]^{!\tricounit{0.65}} \\
	I \ar[r]_{m_I} & !I	} \]
\end{definition}

Monoidal coalgebra modalities, also referred to as {\bf exponential modalities} \cite{Sch04}, provide a
 semantics for the exponential operator of linear logic in SMCs. Symmetric monoidal closed categories 
with exponential modalities are referred to as {\bf linear categories} \cite{Bie95}.

\begin{definition} \cite{Laf88, Lem19}
	A coalgebra modality is said to be a {\bf free} if, for any object $A$, 
	$(!A, \Delta_A, \trianglecounit{0.55}_A)$ is a cofree cocommutative comonoid, that is,  
	if  $(B, d, e)$ is a cocommutative comonoid, then for all maps $f: B \to A$, there exists a unique 
	comonoid morphism from $f^\flat: (B, d, e) \to (!A, \Delta_A, \trianglecounit{0.65})$ such that the 
	following diagram commutes.
	\[ \xymatrixcolsep{4pc} 
	\xymatrix{ (B,d,e) \ar[dr]^{f} \ar@{.>}[d]_{\exists!~ f^\flat} & \\
	(!A, \Delta_A, \trianglecounit{0.65}) \ar[r]_{\epsilon_A} & A} \]
\end{definition}

A free colagebra modality is also a monoidal colagebra modality. Such a monoidal coalgebra modality 
is called a {\bf free exponential modality}. Lafont, in his PhD thesis \cite{Laf88}, provided an important source 
of examples for free exponential modalities by proving that:
\begin{proposition}
In a symmetric monoidal category, $\X$, if every object $A$ generates a 
cofree cocommutative comonoid $!A$, then $\X$ has a free exponential modality. 
\end{proposition}

The proof can be sketched as follows. If $\X$ is a symmetric monoidal 
category in which every object $A$ 
freely generates a cocommutative comonoid, then the underlying functor $U : {\sf Comon}[\X] \to \X$, where 
${\sf Comon}[\X]$ is the SMC of comonoids in $X$ and comonoid morphisms between them, 
has a right adjoint, say $F: \X \to {\sf Comon}[\X]$, which maps each object $A$ 
to its cofree commutative comonoid, $!A$, and each $f: A \to B$  in $\X$  to 
the unique comonoid morphism $(\epsilon_A f)^\flat : (!A, \Delta_A, \tricounit{0.65}_A) \to 
(!B, \Delta_B, \tricounit{0.65}_B)$ given by the couniversal property of 
$(!B, \Delta_B, \tricounit{0.65}_B)$. The comonad given by composing the forgetful functor, $U$, 
with the cofree functor, $F$, is a exponential modality. We discuss examples of 
categories with free exponential modalities later in section \ref{Section: exp mod examples}.

The free exponential modality has been used as a de facto structure for modelling infinite dimensional systems: 
\cite{Vic08} used the exponential modality to model the quantum harmonic oscillator, 
\cite{BPS94} used it to model the bosonic Fock space.   


\subsection{Exponential modalities for $\dagger$-LDCs}
\label{Sec: dagger exp modalities}

An LDC is said to have exponential modalities if it is equipped with a linear comonad 
$((!,?),(\epsilon,\eta),(\delta,\mu))$. This means that:

\begin{definition}
	\label{defn: !-?-LDC}
	 \cite[Definition 2.1]{BCS96} A {\bf $!$-$?$-LDC (an exponential LDC)} is an LDC, $\X$, with a 
	 comonad $(!, \delta, \epsilon)$, and a monad $(?, \mu, \eta)$ such that 
	\begin{enumerate}[(i)]
	\item $(!,?): \X \to \X$ is a linear functor;
	\item The pairs $(\delta, \mu) $, $(\epsilon, \eta)$, $(\Delta, \nabla)$, and 
	$(\trianglecounit{0.65}, \triangleunit{0.65})$ are linear transformations. 	
	\item For each object $A \in \X$, $(!A, \Delta_A, \trianglecounit{0.65}_A)$ is a 
	commutative $\ox$-comonoid, and $(?A, \nabla_A, \triangleunit{0.65}_A)$ is  a commutative $\oa$-monoid.
	\end{enumerate}
\end{definition}
	

The linearity of the functors in a $(!,?)$-LDC means that $(!,\delta,\epsilon)$ is monoidal 
comonad while $(?,\mu,\eta)$ 
is a comonoidal monad, and $(!(A),\Delta_A,\trianglecounit{0.65}_A)$ is a natural 
cocommutative comonoid while $(?(A),\nabla_A,\triangleunit{0.65}_A)$ is a natural commutative monoid. 

A $\dagger$-$(!,?)$-LDC is a $(!,?)$-LDC in which all the functors and natural transformations are 
$\dagger$-linear: 
\begin{definition}
    \label{AppDefn: !?dagLDC}
    \label{defn: !-?-dagger-LDC}
    A {\bf $\dagger$-$(!,?)$-linearly distributive category } is a $(!,?)$-LDC and a $\dagger$-LDC such that the following holds:
    \begin{enumerate}[(i)]
    \item $(!,?)$ is a $\dagger$-linear functor i.e., we have that:
    \[  !(A^\dagger) \to^{\s}_{\simeq} (?A)^\dagger ~~~~~~~~~ (!A)^\dagger  \to^{\t}_{\simeq} ?(A^\dagger) \]
    is a linear natural isomorphism such that 
    \begin{equation}
    \label{eqn: !?iota}
    \xymatrix{
    !X \ar[r]^{\iota} \ar[d]_{!\iota} \ar@{}[dr]|{(a)} & (!X)^{\dagger \dagger} \ar[d]^{(\t^{-1})^\dagger} \\
    !(X^{\dagger \dagger}) \ar[r]_{\s} & (?(X^\dagger))^\dagger }
    ~~~~~~~~~~~~
    \xymatrix{
    ?X \ar[r]^{\iota} \ar[d]_{?\iota}  \ar@{}[dr]|{(b)} & (?X)^{\dagger \dagger} \ar[d]^{\s^\dagger} \\
    ?(X^{\dagger \dagger}) \ar[r]_{\t^{-1}} & (!(X^\dagger))^\dagger }
    \end{equation}
    
    \item The pair $\Delta_A: !A \to !A \ox !A$, and $\nabla_A: ?A \oa ?A \to ?A$ is a $\dagger$-linear natural transformation. i.e., the following diagrams commute:
    \begin{equation}
    \label{!?mul}
    \xymatrixcolsep{1.5cm} \xymatrix{
    !(A^\dagger) \ar[r]^{\Delta} \ar[dd]_{\s}  \ar@{}[ddr]|{(a)} & !(A^\dagger) \ox !(A^\dagger) \ar[d]^{\s \ox \s} \\
    & (?A)^\dagger \ox (?A)^\dagger \ar[d]^{\lambda_\ox} \\
    (?A)^\dagger \ar[r]_{\nabla^\dagger} & (?A \oa ?A)^\dagger }
    ~~~~~~~~~~~
    \xymatrixcolsep{1.5cm} \xymatrix{
    (!A \ox !A)^\dagger \ar[r]^{\Delta^\dagger} \ar[d]_{\lambda_\oa^{-1}}  \ar@{}[ddr]|{(b)} & (!A)^\dagger \ar[dd]^{\t} \\
     (!A)^\dagger \oa (!A)^\dagger \ar[d]_{\t \oa \t} & \\
    ?(A^\dagger) \oa ?(A^\dagger) \ar[r]_{\nabla} & ?(A^\dagger) }
    \end{equation}
    $\Delta$ is completely determined by $\nabla$ and vice versa.
    
    \item The pair, $\triangleunit{0.65}: !A \to \top$ and $\trianglecounit{0.65}: \bot \to ?A$ is a $\dagger$-linear natural transformation i.e., the following diagrams commute:
    \begin{equation}
    \label{eqn: !?unit}
    \xymatrix{
    !(A^\dagger) \ar[r]^{\trianglecounit{0.65}} \ar[d]_{\s}  \ar@{}[dr]|{(a)} & \top \ar[d]^{\lambda_\top} \\
    (?A)^\dagger \ar[r]_{\triangleunit{0.65}^\dagger} & \bot^\dagger }
    ~~~~~~~~~~~~
    \xymatrix{
    \top^\dagger \ar[r]^{\trianglecounit{0.65}^\dagger} \ar[d]_{\lambda_\bot^{-1}}  \ar@{}[dr]|{(b)} & (!A)^{\dagger} \ar[d]^{\t} \\
    \bot \ar[r]_{\triangleunit{0.65}} & (?A)^\dagger }
    \end{equation}
    
    \item The pair $\delta: !A \to !!A $, and $\mu: ??A \to ?A$ is a $\dagger$-linear natural transformation:
    \begin{equation}
    \label{eqn: !?delta}
    \xymatrixcolsep{1.5cm} \xymatrix{
    !(A^\dagger) \ar[r]^{\delta} \ar[dd]_{\s}  \ar@{}[ddr]|{(a)} & !!(A^\dagger)  \ar[d]^{!\s} \\
    & !((?A)^\dagger) \ar[d]^{\s} \\
    (?A)^\dagger \ar[r]_{\mu^\dagger} & (??A)^\dagger }
    ~~~~~~~~~~~
    \xymatrixcolsep{1.5cm} \xymatrix{
    (!!A)^\dagger \ar[r]^{\delta^\dagger} \ar[d]_{\t}  \ar@{}[ddr]|{(b)} & (!A)^(\dagger) \ar[dd]^{\t} \\
     ?((!A)^\dagger) \ar[d]_{?\t} & \\
    ??(A^\dagger) \ar[r]_{\mu} & ?(A^\dagger) }
    \end{equation}
    \end{enumerate}
    
    \item The pair $\epsilon_A: !A \to A$, and $\eta_A: A \to ?A$ is a $\dagger$-linear natural transformation i.e., the following diagrams commute:
    \begin{equation}
    \label{eqn: !?eta}
     \xymatrix{ !(A^\dagger) \ar[r]^{\epsilon} \ar[d]_{\s}  \ar@{}[dr]|{(a)} & A^\dagger \ar@{=}[d] \\
    (?A)^\dagger \ar[r]_{\eta^\dagger} & A^\dagger
    } ~~~~~~~~~~~~ \xymatrix{
    A^\dagger \ar[r]^{\epsilon^\dagger} \ar@{=}[d]  \ar@{}[dr]|{(b)} & (!A)^\dagger \ar[d]^{\t} \\
    A^\dagger \ar[r]_{\eta} & ?(A^\dagger) } 
    \end{equation}
    \end{definition}

In a $(!,?)$-LDC, any dual, $(a, b): A \dashvv ~B$, induces a dual, $(a_!, b_?): !A \dashvv ~?B$ 
(see the diagrams below), on the exponential modalities using the 
linearity of $(!,?)$. This means that, any dual induces a linear comonoid, $(a_!, b_?): !A \lincomonwtritik ?B$,
where the comonoid structure is given by the modality.
\[ a_! := \begin{tikzpicture}
	\begin{pgfonlayer}{nodelayer}
		\node [style=none] (23) at (3, 2) {};
		\node [style=none] (24) at (5, 2) {};
		\node [style=none] (25) at (3, 1) {};
		\node [style=none] (26) at (5, 1) {};
		\node [style=none] (27) at (3.25, 1) {};
		\node [style=none] (28) at (3.75, 1) {};
		\node [style=none] (29) at (4.5, 1) {};
		\node [style=none] (30) at (3.5, 1) {};
		\node [style=none] (31) at (3.5, 0.5) {};
		\node [style=none] (32) at (4.5, 0.5) {};
		\node [style=none] (33) at (4, 1.75) {$\alpha$};
		\node [style=none] (34) at (4.75, 1.25) {$!$};
		\node [style=none] (35) at (3, 0.5) {$!A$};
		\node [style=none] (36) at (5, 0.5) {$?B$};
	\end{pgfonlayer}
	\begin{pgfonlayer}{edgelayer}
		\draw [bend left=75, looseness=1.25] (27.center) to (28.center);
		\draw [bend left=90, looseness=2.00] (30.center) to (29.center);
		\draw (23.center) to (24.center);
		\draw (24.center) to (26.center);
		\draw (26.center) to (25.center);
		\draw (25.center) to (23.center);
		\draw (31.center) to (30.center);
		\draw (32.center) to (29.center);
	\end{pgfonlayer}
\end{tikzpicture} = m_\top (!\alpha)\nu_\ox  ~~~~~~~~ b_? := \begin{tikzpicture}
	\begin{pgfonlayer}{nodelayer}
		\node [style=none] (23) at (5, 0.5) {};
		\node [style=none] (24) at (3, 0.5) {};
		\node [style=none] (25) at (5, 1.5) {};
		\node [style=none] (26) at (3, 1.5) {};
		\node [style=none] (27) at (3.75, 1.5) {};
		\node [style=none] (28) at (3.25, 1.5) {};
		\node [style=none] (29) at (3.5, 1.5) {};
		\node [style=none] (30) at (4.5, 1.5) {};
		\node [style=none] (31) at (4.5, 2) {};
		\node [style=none] (32) at (3.5, 2) {};
		\node [style=none] (33) at (4, 0.75) {$\beta$};
		\node [style=none] (34) at (4.75, 0.75) {$?$};
		\node [style=none] (35) at (5, 2) {$!A$};
		\node [style=none] (36) at (3, 2) {$?B$};
	\end{pgfonlayer}
	\begin{pgfonlayer}{edgelayer}
		\draw [bend left=75, looseness=1.25] (27.center) to (28.center);
		\draw [bend left=90, looseness=1.75] (30.center) to (29.center);
		\draw (23.center) to (24.center);
		\draw (24.center) to (26.center);
		\draw (26.center) to (25.center);
		\draw (25.center) to (23.center);
		\draw (31.center) to (30.center);
		\draw (32.center) to (29.center);
	\end{pgfonlayer}
\end{tikzpicture} =  \nu_\oa (?\epsilon) n_\bot ~~~~~~~~ m_{F} := \begin{tikzpicture}
	\begin{pgfonlayer}{nodelayer}
		\node [style=none] (23) at (3, 2) {};
		\node [style=none] (24) at (5, 2) {};
		\node [style=none] (25) at (3, 1) {};
		\node [style=none] (26) at (5, 1) {};
		\node [style=none] (29) at (4.5, 2.5) {};
		\node [style=none] (30) at (3.5, 2.5) {};
		\node [style=none] (31) at (4.75, 1.25) {$F_\ox$};
		\node [style=circle] (32) at (4, 1.5) {};
		\node [style=none] (33) at (4, 0.5) {};
		\node [style=none] (34) at (3.75, 1) {};
		\node [style=none] (35) at (4.25, 1) {};
		\node [style=none] (36) at (2.8, 2.25) {$F_\ox(A)$};
		\node [style=none] (37) at (5.2, 2.25) {$F_\ox(A)$};
		\node [style=none] (38) at (3.3, 0.5) {$F_\ox(A)$};
	\end{pgfonlayer}
	\begin{pgfonlayer}{edgelayer}
		\draw (23.center) to (24.center);
		\draw (24.center) to (26.center);
		\draw (26.center) to (25.center);
		\draw (25.center) to (23.center);
		\draw [bend right, looseness=1.25] (30.center) to (32);
		\draw [bend right, looseness=1.25] (32) to (29.center);
		\draw (33.center) to (32);
		\draw [bend left=60, looseness=1.50] (34.center) to (35.center);
	\end{pgfonlayer}
\end{tikzpicture}  = m_\ox F_\ox(m) \]
Any linear functor $(F_\ox,F_\oa)$ applied to a linear monoid $(\alpha,\beta):A \linmonw B$ always produces a linear monoid 
$(\alpha_F,\beta_F): F_\ox(A) \expmonwtik F_\oa(B)$ with multplication $m_F$ as shown in the above diagram.  This simple observation when applied to the 
exponential modalities has a striking effect: 
\begin{lemma} 
	\label{Lemma: !? linear monoid}
	In any $(!,?)$-LDC any linear monoid $(a, b)\!\!:\!\! A \linmonw \!B$ 
	and an arbitrary dual $(a', b')\! : \! A \dashvv B$ gives a linear bialgebra
	$\frac{(a_!, b_?)}{(a'_!, b'_?)}: !A \expbialgwtik ?B$ using the natural cocommutative comonoid 
	$(!A, \Delta_A, \tricounit{0.65}_A)$. 
\end{lemma}
\begin{proof}
	Given that $A \linmonw B$ is a linear monoid, and $(a, b): A \dashvv B$ is a linear dual in a $!$-$?$-LDC. 
	
	Because, $(!,?)$ are linear functors, and linear functors preserve linear monoids, 
	we know that $!A \expmonwtik ?B$ is a linear monoid.
	
	Similarly, by linearity of $(!,?)$, we have the dual $(a'_!, b'_?): !A \dashvv ~?B$. 
	The cocommutative $\ox$-comonoid $(!A, \Delta_A, e_A)$ given by the 
	modality $!$ together with the dual $(a'_!, b'_?): !A \dashvv ~?B$ gives a linear comonoid $!A \lincomonwtritik ?B$
	Note that, $(!A, \Delta_A, e_A)$ is dual to the commutative comonoid $(?B, \nabla_B, u_B)$ 
	 given by the $?$ modality, using $(a', b'): !A \dashvv ~?B$:
	\[ \begin{tikzpicture}
		\begin{pgfonlayer}{nodelayer}
			\node [style=none] (0) at (3.5, -1.75) {};
			\node [style=none] (1) at (3.75, -2) {};
			\node [style=none] (2) at (3.25, -2) {};
			\node [style=none] (3) at (4, -2.5) {};
			\node [style=none] (4) at (3, -2.5) {};
			\node [style=none] (5) at (3.5, -3.5) {};
			\node [style=none] (6) at (1.5, -3.5) {};
			\node [style=none] (7) at (3.5, -2.5) {};
			\node [style=none] (8) at (1.5, -2.5) {};
			\node [style=none] (9) at (3.25, -2.5) {};
			\node [style=none] (10) at (2.75, -2.5) {};
			\node [style=none] (11) at (2, -2.75) {};
			\node [style=none] (12) at (3, -2.75) {};
			\node [style=none] (13) at (2, -0.5) {};
			\node [style=none] (14) at (1.75, -3.25) {$?$};
			\node [style=none] (15) at (4.5, -5) {};
			\node [style=none] (16) at (0.75, -5) {};
			\node [style=none] (17) at (4.5, -3.75) {};
			\node [style=none] (18) at (0.75, -3.75) {};
			\node [style=none] (19) at (4.25, -3.75) {};
			\node [style=none] (20) at (3.75, -3.75) {};
			\node [style=none] (21) at (1, -3.75) {};
			\node [style=none] (22) at (4, -3.75) {};
			\node [style=none] (23) at (4, -3.75) {};
			\node [style=none] (24) at (1, -0.5) {};
			\node [style=none] (25) at (4.25, -4.75) {$?$};
			\node [style=none] (26) at (3, -2.5) {};
			\node [style=none] (27) at (5.75, -0.5) {};
			\node [style=none] (28) at (3, -0.5) {};
			\node [style=none] (29) at (5.75, -1.5) {};
			\node [style=none] (30) at (3, -1.5) {};
			\node [style=none] (31) at (5.25, -1.5) {};
			\node [style=none] (32) at (4.75, -1.5) {};
			\node [style=none] (33) at (3.5, -1.5) {};
			\node [style=none] (34) at (5, -1.5) {};
			\node [style=none] (35) at (5.5, -1.25) {$!$};
			\node [style=none] (36) at (5, -5) {};
			\node [style=none] (37) at (2.75, -2) {$!A$};
			\node [style=none] (38) at (3.25, -3) {$A$};
			\node [style=none] (39) at (1.75, -0.75) {$?B$};
			\node [style=none] (40) at (0.75, -0.75) {$?B$};
			\node [style=none] (41) at (3.5, -4) {$A$};
			\node [style=none] (42) at (5, -1) {$B$};
			\node [style=none] (43) at (3.5, -1) {$A$};
			\node [style=none] (44) at (5.25, -4.75) {$?B$};
			\node [style=none] (45) at (2.5, -4.75) {$b'$};
			\node [style=none] (46) at (2.5, -3.25) {$b'$};
			\node [style=none] (47) at (4.25, -0.75) {$a'$};
			\node [style=none] (48) at (1.5, -4) {$B$};
		\end{pgfonlayer}
		\begin{pgfonlayer}{edgelayer}
			\draw (0.center) to (1.center);
			\draw (1.center) to (2.center);
			\draw (2.center) to (0.center);
			\draw [in=-30, out=90] (3.center) to (1.center);
			\draw [in=90, out=-150] (2.center) to (4.center);
			\draw [bend left=75, looseness=1.25] (9.center) to (10.center);
			\draw [bend left=90, looseness=0.75] (12.center) to (11.center);
			\draw (5.center) to (6.center);
			\draw (6.center) to (8.center);
			\draw (8.center) to (7.center);
			\draw (7.center) to (5.center);
			\draw (13.center) to (11.center);
			\draw [bend left=75, looseness=1.25] (19.center) to (20.center);
			\draw [bend left=90, looseness=0.75] (22.center) to (21.center);
			\draw (15.center) to (16.center);
			\draw (16.center) to (18.center);
			\draw (18.center) to (17.center);
			\draw (17.center) to (15.center);
			\draw (23.center) to (22.center);
			\draw (24.center) to (21.center);
			\draw (26.center) to (12.center);
			\draw (3.center) to (23.center);
			\draw [bend right=75, looseness=1.25] (31.center) to (32.center);
			\draw [bend left=270] (34.center) to (33.center);
			\draw (27.center) to (28.center);
			\draw (28.center) to (30.center);
			\draw (30.center) to (29.center);
			\draw (29.center) to (27.center);
			\draw (33.center) to (0.center);
			\draw (36.center) to (34.center);
		\end{pgfonlayer}
	\end{tikzpicture}  = \begin{tikzpicture}
		\begin{pgfonlayer}{nodelayer}
			\node [style=none] (26) at (2.5, -3.5) {};
			\node [style=none] (27) at (0.5, -3.5) {};
			\node [style=none] (28) at (2.5, -2.75) {};
			\node [style=none] (29) at (0.5, -2.75) {};
			\node [style=none] (30) at (2.25, -2.75) {};
			\node [style=none] (31) at (1.75, -2.75) {};
			\node [style=none] (32) at (1, -2.75) {};
			\node [style=none] (33) at (2, -2.75) {};
			\node [style=none] (35) at (1, -0.75) {};
			\node [style=none] (37) at (0.75, -3.25) {$?$};
			\node [style=none] (40) at (3.5, -5) {};
			\node [style=none] (41) at (-0.5, -5) {};
			\node [style=none] (42) at (3.5, -3.75) {};
			\node [style=none] (43) at (-0.5, -3.75) {};
			\node [style=none] (44) at (3.25, -3.75) {};
			\node [style=none] (45) at (2.75, -3.75) {};
			\node [style=none] (46) at (0, -3.75) {};
			\node [style=none] (47) at (3, -3.75) {};
			\node [style=none] (48) at (3, -3.75) {};
			\node [style=none] (49) at (0, -0.75) {};
			\node [style=none] (51) at (-0.25, -4.5) {$?$};
			\node [style=none] (54) at (4.5, -0.5) {};
			\node [style=none] (55) at (1.5, -0.5) {};
			\node [style=none] (56) at (4.5, -1.5) {};
			\node [style=none] (57) at (1.5, -1.5) {};
			\node [style=none] (58) at (4.25, -1.5) {};
			\node [style=none] (59) at (3.75, -1.5) {};
			\node [style=none] (60) at (2, -1.5) {};
			\node [style=none] (61) at (4, -1.5) {};
			\node [style=none] (62) at (1.75, -1.25) {$?$};
			\node [style=none] (63) at (4, -5) {};
			\node [style=none] (65) at (2.25, -3.25) {$A$};
			\node [style=none] (66) at (0.75, -1) {$?B$};
			\node [style=none] (67) at (-0.25, -1) {$?B$};
			\node [style=none] (68) at (2.75, -4.25) {$A$};
			\node [style=none] (69) at (4, -1) {$B$};
			\node [style=none] (70) at (2, -1) {$A$};
			\node [style=none] (71) at (4.25, -4.75) {$?B$};
			\node [style=none] (72) at (7.5, -0.5) {};
			\node [style=none] (73) at (5, -0.5) {};
			\node [style=none] (74) at (7.5, -1.5) {};
			\node [style=none] (75) at (5, -1.5) {};
			\node [style=none] (76) at (7.25, -1.5) {};
			\node [style=none] (77) at (6.75, -1.5) {};
			\node [style=none] (78) at (5.5, -1.5) {};
			\node [style=none] (79) at (7, -1.5) {};
			\node [style=none] (80) at (5.25, -1.25) {$?$};
			\node [style=none] (81) at (7, -1) {$B$};
			\node [style=none] (82) at (5.5, -1) {$A$};
			\node [style=none] (83) at (6.5, -3.25) {};
			\node [style=none] (84) at (6.75, -3) {};
			\node [style=none] (85) at (6.25, -3) {};
			\node [style=none] (86) at (7, -2.5) {};
			\node [style=none] (87) at (6, -2.5) {};
			\node [style=none] (88) at (6, -2.5) {};
			\node [style=none] (90) at (6.5, -4.75) {};
			\node [style=none] (91) at (6.75, -4.5) {$?B$};
			\node [style=none] (92) at (3, -0.75) {$a'$};
			\node [style=none] (93) at (6.25, -0.75) {$a'$};
			\node [style=none] (94) at (1.5, -4.75) {$b'$};
		\end{pgfonlayer}
		\begin{pgfonlayer}{edgelayer}
			\draw [bend left=75, looseness=1.25] (30.center) to (31.center);
			\draw [bend left=90, looseness=2.00] (33.center) to (32.center);
			\draw (26.center) to (27.center);
			\draw (27.center) to (29.center);
			\draw (29.center) to (28.center);
			\draw (28.center) to (26.center);
			\draw (35.center) to (32.center);
			\draw [bend left=75, looseness=1.25] (44.center) to (45.center);
			\draw [bend left=90, looseness=0.75] (47.center) to (46.center);
			\draw (40.center) to (41.center);
			\draw (41.center) to (43.center);
			\draw (43.center) to (42.center);
			\draw (42.center) to (40.center);
			\draw (48.center) to (47.center);
			\draw (49.center) to (46.center);
			\draw [bend right=75, looseness=1.25] (58.center) to (59.center);
			\draw [bend left=270, looseness=0.75] (61.center) to (60.center);
			\draw (54.center) to (55.center);
			\draw (55.center) to (57.center);
			\draw (57.center) to (56.center);
			\draw (56.center) to (54.center);
			\draw [bend right=75, looseness=1.25] (76.center) to (77.center);
			\draw [bend left=270] (79.center) to (78.center);
			\draw (72.center) to (73.center);
			\draw (73.center) to (75.center);
			\draw (75.center) to (74.center);
			\draw (74.center) to (72.center);
			\draw (83.center) to (84.center);
			\draw (84.center) to (85.center);
			\draw (85.center) to (83.center);
			\draw [in=30, out=-90] (86.center) to (84.center);
			\draw [in=-90, out=150] (85.center) to (87.center);
			\draw (60.center) to (33.center);
			\draw [in=-90, out=90] (48.center) to (78.center);
			\draw (79.center) to (86.center);
			\draw [in=-90, out=90, looseness=0.75] (88.center) to (61.center);
			\draw (83.center) to (90.center);
		\end{pgfonlayer}
	\end{tikzpicture} =  \begin{tikzpicture}
		\begin{pgfonlayer}{nodelayer}
			\node [style=none] (83) at (6.5, -3.25) {};
			\node [style=none] (84) at (6.75, -3) {};
			\node [style=none] (85) at (6.25, -3) {};
			\node [style=none] (86) at (7, -2.5) {};
			\node [style=none] (87) at (6, -2.5) {};
			\node [style=none] (88) at (6, -2.5) {};
			\node [style=none] (90) at (6.5, -4.75) {};
			\node [style=none] (91) at (6.75, -4.5) {$?B$};
			\node [style=none] (92) at (7, -0.75) {};
			\node [style=none] (93) at (6, -0.75) {};
			\node [style=none] (94) at (7.25, -1) {$?B$};
			\node [style=none] (95) at (5.75, -1) {$?B$};
		\end{pgfonlayer}
		\begin{pgfonlayer}{edgelayer}
			\draw (83.center) to (84.center);
			\draw (84.center) to (85.center);
			\draw (85.center) to (83.center);
			\draw [in=30, out=-90] (86.center) to (84.center);
			\draw [in=-90, out=150] (85.center) to (87.center);
			\draw (83.center) to (90.center);
			\draw [in=90, out=-90] (93.center) to (86.center);
			\draw [in=90, out=-90] (92.center) to (88.center);
		\end{pgfonlayer}
	\end{tikzpicture} = \begin{tikzpicture}
		\begin{pgfonlayer}{nodelayer}
			\node [style=none] (83) at (6.5, -3.25) {};
			\node [style=none] (84) at (6.75, -3) {};
			\node [style=none] (85) at (6.25, -3) {};
			\node [style=none] (86) at (7, -2.5) {};
			\node [style=none] (87) at (6, -2.5) {};
			\node [style=none] (88) at (6, -2.5) {};
			\node [style=none] (90) at (6.5, -4.75) {};
			\node [style=none] (91) at (6.75, -4.5) {$?B$};
			\node [style=none] (92) at (7, -0.75) {};
			\node [style=none] (93) at (6, -0.75) {};
			\node [style=none] (94) at (7.25, -1) {$?B$};
			\node [style=none] (95) at (5.75, -1) {$?B$};
		\end{pgfonlayer}
		\begin{pgfonlayer}{edgelayer}
			\draw (83.center) to (84.center);
			\draw (84.center) to (85.center);
			\draw (85.center) to (83.center);
			\draw [in=30, out=-90] (86.center) to (84.center);
			\draw [in=-90, out=150] (85.center) to (87.center);
			\draw (83.center) to (90.center);
			\draw (88.center) to (93.center);
			\draw (92.center) to (86.center);
		\end{pgfonlayer}
	\end{tikzpicture} \]
	The linear monoid $!A \expmonwtik ?B$ and the linear comonoid $!A \lincomonwtritik ?B$ gives a $\ox$-bialgebra on $!A$
	 and a $\oa$-bialgbera on $?B$. The bialgebra rules are immediate  by naturality of $\Delta$, $\nabla$, 
	 $\triunit{0.65}$, and $\tricounit{0.65}$:
	\[ \begin{tikzpicture}
		\begin{pgfonlayer}{nodelayer}
			\node [style=none] (0) at (1.25, 2) {};
			\node [style=none] (1) at (1.25, 0.75) {};
			\node [style=none] (2) at (1, 0.5) {};
			\node [style=none] (3) at (1.5, 0.5) {};
			\node [style=none] (4) at (0.5, 3.25) {};
			\node [style=none] (5) at (2, 3.25) {};
			\node [style=none] (6) at (0.5, 2.25) {};
			\node [style=none] (7) at (2, 2.25) {};
			\node [style=circle] (8) at (1.25, 2.75) {};
			\node [style=none] (9) at (1.25, 2) {};
			\node [style=none] (10) at (0.75, 3.75) {};
			\node [style=none] (11) at (1.75, 3.75) {};
			\node [style=none] (12) at (1.75, 2.5) {$!$};
			\node [style=none] (13) at (1, 2.25) {};
			\node [style=none] (14) at (1.5, 2.25) {};
			\node [style=none] (15) at (0.75, -0.25) {};
			\node [style=none] (16) at (1.75, -0.25) {};
			\node [style=none] (17) at (1.75, 1.5) {$!A$};
			\node [style=none] (18) at (0.5, 3.75) {$!A$};
			\node [style=none] (19) at (0.5, 0) {$!A$};
			\node [style=none] (20) at (2, 0) {$!A$};
			\node [style=none] (21) at (2, 3.75) {$!A$};
		\end{pgfonlayer}
		\begin{pgfonlayer}{edgelayer}
			\draw (0.center) to (1.center);
			\draw (1.center) to (2.center);
			\draw (2.center) to (3.center);
			\draw (3.center) to (1.center);
			\draw [bend right] (10.center) to (8);
			\draw [in=-90, out=30] (8) to (11.center);
			\draw (8) to (9.center);
			\draw (6.center) to (7.center);
			\draw (7.center) to (5.center);
			\draw [in=360, out=180] (5.center) to (4.center);
			\draw (4.center) to (6.center);
			\draw [bend left=75, looseness=1.25] (13.center) to (14.center);
			\draw [in=210, out=90] (15.center) to (2.center);
			\draw [in=90, out=-30] (3.center) to (16.center);
		\end{pgfonlayer}
	\end{tikzpicture} = \begin{tikzpicture}
		\begin{pgfonlayer}{nodelayer}
			\node [style=none] (50) at (-1, 3.75) {};
			\node [style=none] (51) at (-1, 3.25) {};
			\node [style=none] (52) at (-1.25, 3) {};
			\node [style=none] (53) at (-0.75, 3) {};
			\node [style=none] (54) at (-1, 3.75) {};
			\node [style=none] (55) at (-1.5, 2.25) {};
			\node [style=none] (56) at (-0.5, 2.5) {};
			\node [style=none] (57) at (1, 3.75) {};
			\node [style=none] (58) at (1, 3.25) {};
			\node [style=none] (59) at (0.75, 3) {};
			\node [style=none] (60) at (1.25, 3) {};
			\node [style=none] (61) at (1, 3.75) {};
			\node [style=none] (62) at (0.5, 2.5) {};
			\node [style=none] (63) at (1.5, 2.25) {};
			\node [style=none] (64) at (-1, -0.25) {};
			\node [style=none] (65) at (-1.75, 1.25) {};
			\node [style=none] (66) at (-0.25, 1.25) {};
			\node [style=none] (67) at (-1.75, 0.25) {};
			\node [style=none] (68) at (-0.25, 0.25) {};
			\node [style=circle] (69) at (-1, 0.75) {};
			\node [style=none] (70) at (-1, -0.25) {};
			\node [style=none] (71) at (-1.5, 1.75) {};
			\node [style=none] (72) at (-0.5, 1.5) {};
			\node [style=none] (73) at (-0.5, 0.5) {$!$};
			\node [style=none] (74) at (-1.25, 0.25) {};
			\node [style=none] (75) at (-0.75, 0.25) {};
			\node [style=none] (76) at (1, -0.25) {};
			\node [style=none] (77) at (0.25, 1.25) {};
			\node [style=none] (78) at (1.75, 1.25) {};
			\node [style=none] (79) at (0.25, 0.25) {};
			\node [style=none] (80) at (1.75, 0.25) {};
			\node [style=circle] (81) at (1, 0.75) {};
			\node [style=none] (82) at (1, -0.25) {};
			\node [style=none] (83) at (0.5, 1.5) {};
			\node [style=none] (84) at (1.5, 1.75) {};
			\node [style=none] (85) at (1.5, 0.5) {$!$};
			\node [style=none] (86) at (0.75, 0.25) {};
			\node [style=none] (87) at (1.25, 0.25) {};
			\node [style=none] (88) at (1.25, 3.5) {$A$};
			\node [style=none] (89) at (-1.25, 3.5) {$A$};
			\node [style=none] (90) at (-1.5, -0.25) {$A$};
			\node [style=none] (91) at (1.5, -0.25) {$A$};
		\end{pgfonlayer}
		\begin{pgfonlayer}{edgelayer}
			\draw (50.center) to (51.center);
			\draw (51.center) to (52.center);
			\draw (52.center) to (53.center);
			\draw (53.center) to (51.center);
			\draw [in=210, out=90] (55.center) to (52.center);
			\draw [in=90, out=-30] (53.center) to (56.center);
			\draw (57.center) to (58.center);
			\draw (58.center) to (59.center);
			\draw (59.center) to (60.center);
			\draw (60.center) to (58.center);
			\draw [in=210, out=90] (62.center) to (59.center);
			\draw [in=90, out=-30] (60.center) to (63.center);
			\draw [bend right] (71.center) to (69);
			\draw [in=-90, out=30] (69) to (72.center);
			\draw (69) to (70.center);
			\draw (67.center) to (68.center);
			\draw (68.center) to (66.center);
			\draw [in=360, out=180] (66.center) to (65.center);
			\draw (65.center) to (67.center);
			\draw [bend left=75, looseness=1.25] (74.center) to (75.center);
			\draw [bend right] (83.center) to (81);
			\draw [in=-90, out=30] (81) to (84.center);
			\draw (81) to (82.center);
			\draw (79.center) to (80.center);
			\draw (80.center) to (78.center);
			\draw [in=360, out=180] (78.center) to (77.center);
			\draw (77.center) to (79.center);
			\draw [bend left=75, looseness=1.25] (86.center) to (87.center);
			\draw [in=90, out=-105, looseness=0.75] (62.center) to (72.center);
			\draw [in=90, out=-90, looseness=0.75] (56.center) to (83.center);
			\draw (55.center) to (71.center);
			\draw (63.center) to (84.center);
		\end{pgfonlayer}
	\end{tikzpicture}
	~~~~~~~~
	\begin{tikzpicture}
		\begin{pgfonlayer}{nodelayer}
			\node [style=none] (0) at (1.25, 0) {};
			\node [style=none] (1) at (1.25, -0.25) {};
			\node [style=none] (2) at (1, -0.5) {};
			\node [style=none] (3) at (1.5, -0.5) {};
			\node [style=none] (4) at (1.25, -0.25) {};
			\node [style=none] (5) at (0.5, 1.25) {};
			\node [style=none] (6) at (2, 1.25) {};
			\node [style=none] (7) at (0.5, 0.25) {};
			\node [style=none] (8) at (2, 0.25) {};
			\node [style=circle] (9) at (1.25, 0.75) {};
			\node [style=none] (10) at (1.25, -0.25) {};
			\node [style=none] (11) at (0.75, 2) {};
			\node [style=none] (12) at (1.75, 2) {};
			\node [style=none] (13) at (1.75, 0.5) {$!$};
			\node [style=none] (14) at (1, 0.25) {};
			\node [style=none] (15) at (1.5, 0.25) {};
			\node [style=none] (16) at (1.75, -0.25) {$!A$};
			\node [style=none] (17) at (0.5, 1.75) {$!A$};
			\node [style=none] (18) at (2, 1.75) {$!A$};
		\end{pgfonlayer}
		\begin{pgfonlayer}{edgelayer}
			\draw (0.center) to (1.center);
			\draw (1.center) to (2.center);
			\draw (2.center) to (3.center);
			\draw (3.center) to (1.center);
			\draw [bend right=15] (11.center) to (9);
			\draw [bend right=15] (9) to (12.center);
			\draw (9) to (10.center);
			\draw (7.center) to (8.center);
			\draw (8.center) to (6.center);
			\draw [in=360, out=180] (6.center) to (5.center);
			\draw (5.center) to (7.center);
			\draw [bend left=75, looseness=1.25] (14.center) to (15.center);
		\end{pgfonlayer}
	\end{tikzpicture} = \begin{tikzpicture}
		\begin{pgfonlayer}{nodelayer}
			\node [style=none] (1) at (0.75, -0.25) {};
			\node [style=none] (2) at (0.5, -0.5) {};
			\node [style=none] (3) at (1, -0.5) {};
			\node [style=none] (4) at (0.75, -0.25) {};
			\node [style=none] (10) at (0.75, -0.25) {};
			\node [style=none] (11) at (0.75, 2) {};
			\node [style=none] (12) at (1.75, 2) {};
			\node [style=none] (17) at (0.5, 1.75) {$!A$};
			\node [style=none] (18) at (2, 1.75) {$!A$};
			\node [style=none] (19) at (1.75, -0.25) {};
			\node [style=none] (20) at (1.5, -0.5) {};
			\node [style=none] (21) at (2, -0.5) {};
			\node [style=none] (22) at (1.75, -0.25) {};
			\node [style=none] (23) at (1.75, -0.25) {};
		\end{pgfonlayer}
		\begin{pgfonlayer}{edgelayer}
			\draw (1.center) to (2.center);
			\draw (2.center) to (3.center);
			\draw (3.center) to (1.center);
			\draw (19.center) to (20.center);
			\draw (20.center) to (21.center);
			\draw (21.center) to (19.center);
			\draw (12.center) to (23.center);
			\draw (11.center) to (10.center);
		\end{pgfonlayer}
	\end{tikzpicture}
	~~~~~~~~
	\begin{tikzpicture}
		\begin{pgfonlayer}{nodelayer}
			\node [style=none] (25) at (3, 1.5) {};
			\node [style=none] (26) at (4.5, 1.5) {};
			\node [style=none] (27) at (3, 0.5) {};
			\node [style=none] (28) at (4.5, 0.5) {};
			\node [style=circle] (29) at (3.75, 1) {};
			\node [style=none] (30) at (3.75, 0.25) {};
			\node [style=none] (31) at (4.25, 0.75) {$!$};
			\node [style=none] (32) at (3.5, 0.5) {};
			\node [style=none] (33) at (4, 0.5) {};
			\node [style=none] (34) at (3, -1) {$!A$};
			\node [style=none] (35) at (3.75, -0.25) {};
			\node [style=none] (36) at (4, -0.5) {};
			\node [style=none] (37) at (3.5, -0.5) {};
			\node [style=none] (38) at (4.25, -1.25) {};
			\node [style=none] (39) at (3.25, -1.25) {};
			\node [style=none] (40) at (3.25, -1.25) {};
			\node [style=none] (41) at (3.75, 0.25) {};
			\node [style=none] (42) at (4.5, -1) {$!A$};
		\end{pgfonlayer}
		\begin{pgfonlayer}{edgelayer}
			\draw (29) to (30.center);
			\draw (27.center) to (28.center);
			\draw (28.center) to (26.center);
			\draw [in=360, out=180] (26.center) to (25.center);
			\draw (25.center) to (27.center);
			\draw [bend left=75, looseness=1.25] (32.center) to (33.center);
			\draw (35.center) to (36.center);
			\draw (36.center) to (37.center);
			\draw (37.center) to (35.center);
			\draw [in=-30, out=90] (38.center) to (36.center);
			\draw [in=90, out=-150] (37.center) to (39.center);
			\draw (35.center) to (41.center);
		\end{pgfonlayer}
	\end{tikzpicture} = \begin{tikzpicture}
		\begin{pgfonlayer}{nodelayer}
			\node [style=none] (25) at (2.5, 1.5) {};
			\node [style=none] (26) at (4, 1.5) {};
			\node [style=none] (27) at (2.5, 0.5) {};
			\node [style=none] (28) at (4, 0.5) {};
			\node [style=circle] (29) at (3.25, 1) {};
			\node [style=none] (31) at (3.75, 0.75) {$!$};
			\node [style=none] (32) at (3, 0.5) {};
			\node [style=none] (33) at (3.5, 0.5) {};
			\node [style=none] (34) at (3, -1) {$!A$};
			\node [style=none] (39) at (3.25, -1.25) {};
			\node [style=none] (40) at (3.25, -1.25) {};
			\node [style=none] (43) at (4.25, 1.5) {};
			\node [style=none] (44) at (5.75, 1.5) {};
			\node [style=none] (45) at (4.25, 0.5) {};
			\node [style=none] (46) at (5.75, 0.5) {};
			\node [style=circle] (47) at (5, 1) {};
			\node [style=none] (48) at (5.5, 0.75) {$!$};
			\node [style=none] (49) at (4.75, 0.5) {};
			\node [style=none] (50) at (5.25, 0.5) {};
			\node [style=none] (51) at (4.75, -1) {$!A$};
			\node [style=none] (52) at (5, -1.25) {};
			\node [style=none] (53) at (5, -1.25) {};
		\end{pgfonlayer}
		\begin{pgfonlayer}{edgelayer}
			\draw (27.center) to (28.center);
			\draw (28.center) to (26.center);
			\draw [in=360, out=180] (26.center) to (25.center);
			\draw (25.center) to (27.center);
			\draw [bend left=75, looseness=1.25] (32.center) to (33.center);
			\draw (40.center) to (29);
			\draw (45.center) to (46.center);
			\draw (46.center) to (44.center);
			\draw [in=360, out=180] (44.center) to (43.center);
			\draw (43.center) to (45.center);
			\draw [bend left=75, looseness=1.25] (49.center) to (50.center);
			\draw (53.center) to (47);
		\end{pgfonlayer}
	\end{tikzpicture} 
	~~~~~~~~
	\begin{tikzpicture}
		\begin{pgfonlayer}{nodelayer}
			\node [style=none] (25) at (2.5, 1.5) {};
			\node [style=none] (26) at (4, 1.5) {};
			\node [style=none] (27) at (2.5, 0.5) {};
			\node [style=none] (28) at (4, 0.5) {};
			\node [style=circle] (29) at (3.25, 1) {};
			\node [style=none] (31) at (3.75, 0.75) {$!$};
			\node [style=none] (32) at (3, 0.5) {};
			\node [style=none] (33) at (3.5, 0.5) {};
			\node [style=none] (34) at (3, -0.25) {$!A$};
			\node [style=none] (39) at (3, -1.25) {};
			\node [style=none] (40) at (3.25, -1) {};
			\node [style=none] (54) at (3.5, -1.25) {};
		\end{pgfonlayer}
		\begin{pgfonlayer}{edgelayer}
			\draw (27.center) to (28.center);
			\draw (28.center) to (26.center);
			\draw [in=360, out=180] (26.center) to (25.center);
			\draw (25.center) to (27.center);
			\draw [bend left=75, looseness=1.25] (32.center) to (33.center);
			\draw (40.center) to (29);
			\draw (39.center) to (54.center);
			\draw (54.center) to (40.center);
			\draw (40.center) to (39.center);
		\end{pgfonlayer}
	\end{tikzpicture} = 1_\top  \]
	
	\vspace{-1em}

	\end{proof}
\begin{lemma} 
	In any $(!,?)$-$\dagger$-LDC any $\dagger$-linear monoid $(a, b): A \linmonw A^\dagger$ 
	and an arbitrary $\dagger$-dual $(a', b'): A \dashvv A^\dagger$ gives a $\dagger$-linear bialgebra
	$\frac{(a_!, b_?)}{(a'_!, b'_?)}:  !A \expbialgwtik ?A^\dagger$. 
\end{lemma}
\begin{proof}
	Recall that there exists a natural isomorphism $\t: (!A)^\dagger \to ?A^\dagger$ 
	because $(!,?)$ is a $\dagger$-linear functor. 	$(a_!, b_?): !A \dagmonwtik ?A^\dagger$ 
	is a $\dagger$-linear monoid because $\dagger$-linear functors preserve $\dagger$-linear monoids. 
	Hence, $(a_!, b_?): !A \dagmonwtik ?A^\dagger$  is a $\dagger$-linear monoid. 
	$(a'_!, b'_?): !A \dagcomonwtritik ?A^\dagger$ is a $\dagger$-linear comonoid because 
	$\dagger$-linear functors preserve $\dagger$-duals, and $(\Delta, \nabla)$ are $\dagger$-linear 
	tranformations i.e.,
	\[ \begin{tikzpicture}
		\begin{pgfonlayer}{nodelayer}
			\node [style=none] (0) at (3.5, -1.75) {};
			\node [style=none] (1) at (3.75, -2) {};
			\node [style=none] (2) at (3.25, -2) {};
			\node [style=none] (3) at (4, -2.5) {};
			\node [style=none] (4) at (3, -2.5) {};
			\node [style=none] (5) at (3.5, -3.5) {};
			\node [style=none] (6) at (1.5, -3.5) {};
			\node [style=none] (7) at (3.5, -2.5) {};
			\node [style=none] (8) at (1.5, -2.5) {};
			\node [style=none] (9) at (3.25, -2.5) {};
			\node [style=none] (10) at (2.75, -2.5) {};
			\node [style=none] (11) at (2, -2.75) {};
			\node [style=none] (12) at (3, -2.75) {};
			\node [style=none] (13) at (2, -0.5) {};
			\node [style=none] (14) at (1.75, -3.25) {$?$};
			\node [style=none] (15) at (4.5, -5) {};
			\node [style=none] (16) at (0.75, -5) {};
			\node [style=none] (17) at (4.5, -3.75) {};
			\node [style=none] (18) at (0.75, -3.75) {};
			\node [style=none] (19) at (4.25, -3.75) {};
			\node [style=none] (20) at (3.75, -3.75) {};
			\node [style=none] (21) at (1, -3.75) {};
			\node [style=none] (22) at (4, -3.75) {};
			\node [style=none] (23) at (4, -3.75) {};
			\node [style=none] (24) at (1, -0.5) {};
			\node [style=none] (25) at (4.25, -4.75) {$?$};
			\node [style=none] (26) at (3, -2.5) {};
			\node [style=none] (27) at (5.75, -0.5) {};
			\node [style=none] (28) at (3, -0.5) {};
			\node [style=none] (29) at (5.75, -1.5) {};
			\node [style=none] (30) at (3, -1.5) {};
			\node [style=none] (31) at (5.25, -1.5) {};
			\node [style=none] (32) at (4.75, -1.5) {};
			\node [style=none] (33) at (3.5, -1.5) {};
			\node [style=none] (34) at (5, -1.5) {};
			\node [style=none] (35) at (5.5, -1.25) {$!$};
			\node [style=none] (36) at (5, -5) {};
			\node [style=none] (37) at (2.75, -2) {$!A$};
			\node [style=none] (38) at (3.25, -3) {$A$};
			\node [style=none] (39) at (1.65, -2) {$?A^\dagger$};
			\node [style=none] (40) at (0.15, -2.5) {$?A^\dagger$};
			\node [style=none] (41) at (3.5, -4) {$A$};
			\node [style=none] (42) at (5, -1) {$A^\dagger$};
			\node [style=none] (43) at (3.5, -1) {$A$};
			\node [style=none] (44) at (5.25, -4.75) {$?A^\dagger$};
			\node [style=none] (45) at (2.5, -4.75) {$b'$};
			\node [style=none] (46) at (2.5, -3.25) {$b'$};
			\node [style=none] (47) at (4.25, -0.75) {$a'$};
			\node [style=none] (48) at (1.6, -4) {$A^\dagger$};
			\node [style=onehalfcircle] (49) at (1, -1.25) {};
			\node [style=onehalfcircle] (50) at (2, -1.25) {};
			\node [style=none] (51) at (2, -1.25) {$\t$};
			\node [style=none] (52) at (1, -1.25) {$\t$};
			\node [style=none] (53) at (0.25, -0.75) {$(!A)^\dag$};
		\end{pgfonlayer}
		\begin{pgfonlayer}{edgelayer}
			\draw (0.center) to (1.center);
			\draw (1.center) to (2.center);
			\draw (2.center) to (0.center);
			\draw [in=-30, out=90] (3.center) to (1.center);
			\draw [in=90, out=-150] (2.center) to (4.center);
			\draw [bend left=75, looseness=1.25] (9.center) to (10.center);
			\draw [bend left=90, looseness=0.75] (12.center) to (11.center);
			\draw (5.center) to (6.center);
			\draw (6.center) to (8.center);
			\draw (8.center) to (7.center);
			\draw (7.center) to (5.center);
			\draw [bend left=75, looseness=1.25] (19.center) to (20.center);
			\draw [bend left=90, looseness=0.75] (22.center) to (21.center);
			\draw (15.center) to (16.center);
			\draw (16.center) to (18.center);
			\draw (18.center) to (17.center);
			\draw (17.center) to (15.center);
			\draw (23.center) to (22.center);
			\draw (26.center) to (12.center);
			\draw (3.center) to (23.center);
			\draw [bend right=75, looseness=1.25] (31.center) to (32.center);
			\draw [bend left=270] (34.center) to (33.center);
			\draw (27.center) to (28.center);
			\draw (28.center) to (30.center);
			\draw (30.center) to (29.center);
			\draw (29.center) to (27.center);
			\draw (33.center) to (0.center);
			\draw (36.center) to (34.center);
			\draw (11.center) to (50);
			\draw (50) to (13.center);
			\draw (21.center) to (49);
			\draw (49) to (24.center);
		\end{pgfonlayer}
	\end{tikzpicture}
		= \begin{tikzpicture}
		\begin{pgfonlayer}{nodelayer}
			\node [style=none] (83) at (6.5, -3.25) {};
			\node [style=none] (84) at (6.75, -3) {};
			\node [style=none] (85) at (6.25, -3) {};
			\node [style=none] (86) at (7, -2.5) {};
			\node [style=none] (87) at (6, -2.5) {};
			\node [style=none] (88) at (6, -2.5) {};
			\node [style=none] (90) at (6.5, -5) {};
			\node [style=none] (92) at (7, -0.75) {};
			\node [style=none] (93) at (6, -0.75) {};
			\node [style=none] (94) at (7.5, -1) {$(!A)^\dag$};
			\node [style=none] (95) at (5.5, -1) {$(!A)^\dag$};
			\node [style=onehalfcircle] (96) at (6, -1.75) {};
			\node [style=onehalfcircle] (97) at (7, -1.75) {};
			\node [style=none] (99) at (6, -1.75) {$\t$};
			\node [style=none] (102) at (7, -1.75) {$\t$};
			\node [style=none] (103) at (7, -4.5) {$?A^\dag$};
		\end{pgfonlayer}
		\begin{pgfonlayer}{edgelayer}
			\draw (83.center) to (84.center);
			\draw (84.center) to (85.center);
			\draw (85.center) to (83.center);
			\draw [in=30, out=-90] (86.center) to (84.center);
			\draw [in=-90, out=150] (85.center) to (87.center);
			\draw (86.center) to (97);
			\draw (97) to (92.center);
			\draw (88.center) to (96);
			\draw (96) to (93.center);
			\draw (90.center) to (83.center);
		\end{pgfonlayer}
	\end{tikzpicture}
		= \begin{tikzpicture}
		\begin{pgfonlayer}{nodelayer}
			\node [style=none] (83) at (6.5, -2) {};
			\node [style=none] (84) at (6.75, -2.25) {};
			\node [style=none] (85) at (6.25, -2.25) {};
			\node [style=none] (86) at (7, -2.75) {};
			\node [style=none] (87) at (6, -2.75) {};
			\node [style=none] (88) at (6, -2.75) {};
			\node [style=none] (90) at (6.5, -4.75) {};
			\node [style=none] (91) at (7.25, -3.25) {$(!A)^\dagger$};
			\node [style=none] (92) at (7, -0.75) {};
			\node [style=none] (93) at (6, -0.75) {};
			\node [style=none] (96) at (6.5, -1.75) {};
			\node [style=none] (97) at (5.5, -2.75) {};
			\node [style=none] (98) at (5.5, -1.75) {};
			\node [style=none] (99) at (7.5, -1.75) {};
			\node [style=none] (100) at (7.5, -2.75) {}; 
			\node [style=none] (101) at (6.5, -2.75) {};
			\node [style=none] (102) at (6, -1.75) {};
			\node [style=none] (103) at (7, -1.75) {};
			\node [style=none] (104) at (7.25, -1) {$(!A)^\dagger$};
			\node [style=none] (105) at (5.25, -1) {$(!A)^\dagger$};
			\node [style=onehalfcircle] (106) at (6.5, -3.75) {};
			\node [style=none] (107) at (6.5, -3.75) {$\t$};
			\node [style=none] (108) at (7, -4.5) {$?A^\dag$};
		\end{pgfonlayer}
		\begin{pgfonlayer}{edgelayer}
			\draw (83.center) to (84.center);
			\draw (84.center) to (85.center);
			\draw (85.center) to (83.center);
			\draw [in=-30, out=90] (86.center) to (84.center);
			\draw [in=90, out=-150] (85.center) to (87.center);
			\draw (83.center) to (96.center);
			\draw (97.center) to (98.center);
			\draw (98.center) to (99.center);
			\draw (100.center) to (99.center);
			\draw (100.center) to (97.center);
			\draw (102.center) to (93.center);
			\draw (103.center) to (92.center);
			\draw (90.center) to (106);
			\draw (106) to (101.center);
		\end{pgfonlayer}
	\end{tikzpicture} \]
	From Lemma \ref{Lemma: !? linear monoid}, $!A \expbialgwtik ?A^\dagger$ is a linear bialgebra.
	Thereby, $!A \expbialgwtik ?A^\dag$ is a $\dagger$-linear bialgebra.
\end{proof}
	
The bialgebra structure results from the naturality of $\Delta$ and $\trianglecounit{0.55}$ over the 
functorially induced monoid structure.

Next we explore the bialgebra structure induced by the free exponential modalities and 
their connection to complementary systems.

\subsection{Sequent calculus for exponential $\dagger$-linear logic} 
\label{Sec: sequent dagger exp}

In the sequent calculus for linear logic, the rules for the exponentials $!$ and $?$ are given as in Figure \ref{Fig: exponential rules}.

\begin{figure}[h]
	\centering
    \AxiomC{$\Gamma \vdash \Delta$} 
	\LeftLabel{$(thin.L)$}
	\UnaryInfC{$\Gamma, !A \vdash \Delta$}
	\DisplayProof
	\hspace{1.5em}
	\AxiomC{$\Gamma \vdash \Delta$} 
	\LeftLabel{$(thin.R)$}
	\UnaryInfC{$\Gamma \vdash \Delta, ?B$}
	\DisplayProof

	\vspace{1em}
	
    \AxiomC{$\Gamma, A \vdash \Delta$} 
	\LeftLabel{$(der.L)$}
	\UnaryInfC{$\Gamma, !A \vdash \Delta$}
	\DisplayProof
	\hspace{1.5em}
	\AxiomC{$\Gamma \vdash \Delta, B$} 
	\LeftLabel{$(der.R)$}
	\UnaryInfC{$\Gamma \vdash \Delta, ?B$}
	\DisplayProof

	\vspace{1em}
    \AxiomC{$\Gamma, !A, !A \vdash \Delta$} 
	\LeftLabel{$(contr.L)$}
	\UnaryInfC{$\Gamma, !A \vdash \Delta$}
	\DisplayProof
	\hspace{1.5em}
	\AxiomC{$\Gamma \vdash \Delta, ?B, ?B$} 
	\LeftLabel{$(contr.R)$}
	\UnaryInfC{$\Gamma \vdash \Delta, ?B$}
	\DisplayProof	

	\vspace{1em}
    \AxiomC{$!\Gamma, A \vdash ?\Delta$} 
	\LeftLabel{$(stor.L)$}
	\UnaryInfC{$!\Gamma, ?A \vdash ?\Delta$}
	\DisplayProof
	\hspace{1.5em}
	\AxiomC{$!\Gamma \vdash ?\Delta, B$} 
	\LeftLabel{$(stor.R)$}
	\UnaryInfC{$!\Gamma \vdash ?\Delta, !B$}
	\DisplayProof	
	\caption{Sequent rules for exponential modality}
	\label{Fig: exponential rules}
\end{figure}

Categorically, the sequent rules, ($thin.L$) and ($contr.L$), means that $!A$ is a 
cocommutative comonoid. Similary, the rules,  ($thin.L$) and ($contr.L$), means that $?B$ is a commutative monoid. 
The storage rules ($stor.L$ and $stor.R$), and dereliction rules ($der.L$ and $der.R$) makes $(!,?)$
a linear functor, as shown in the figure \ref{Fig: sequent linear functor}. Moreover, the linear functor is also
a linear comonad, see \cite{BCS96} for details. 

\begin{figure}[h]
	\centering
	\AxiomC{$\Gamma, A \vdash \Delta$}
	\RightLabel{$der.L, der.R$}	
	\UnaryInfC{$!\Gamma, A \vdash ?\Delta$}
	\RightLabel{$stor.L$}
	\UnaryInfC{$!\Gamma, ?A \vdash ?\Delta$ }
	\DisplayProof
	\hspace{1em}
	\AxiomC{$\Gamma \vdash B, \Delta$}
	\RightLabel{$der.L, der.R$}	
	\UnaryInfC{$!\Gamma \vdash B, ?\Delta$}
	\RightLabel{$stor.R$}
	\UnaryInfC{$!\Gamma \vdash !B, ?\Delta$ }
	\DisplayProof
	\caption{Derivation of $(!,?)$ linear functor}
	\label{Fig: sequent linear functor}
\end{figure}

In the presence of $\dagger$, the $(!,?)$ exponential modality must satisfy the additional rules 
given in Figure \ref{Fig: dag exp rules} which $(!,?)$ a $\dagger$-linear functor. 
Derivations of the isomorphisms $\s: (?A)^\dagger \to_{\simeq} (!A)^\dagger$, and $\t: (!A)^\dag \to ?(A^\dag)$ 
for $(!,?)$ to be a $\dagger$-linear functor are shown Figures \ref{Fig: s derivation}, and \ref{Fig: t derivation}.

\begin{figure}[h]
	\centering 
	\vspace{1em}
	\AxiomC{$\Gamma, ?(A^\dag)  \vdash \Delta$} 
	\LeftLabel{$(\dagger$-$exp.L)$}
	\UnaryInfC{$\Gamma, (!A)^\dag \vdash \Delta$}     
	\DisplayProof
	\hspace{1.5em}	    
	\AxiomC{$\Gamma \vdash \Delta, !(B^\dag)$} 
	\LeftLabel{$(\dagger$-$exp.R)$}
	\UnaryInfC{$\Gamma \vdash \Delta, (?B)^\dag$}
	\DisplayProof
	\caption{Sequent rules for $\dagger$-exponential modality}
	\label{Fig: dag exp rules}
\end{figure}

\begin{figure}[h]
	\centering
	\AxiomC{$ $}
	\RightLabel{$id$}
	\UnaryInfC{$A \vdash A$}
	\RightLabel{$\iota L$}	
	\UnaryInfC{$A^{\dag \dag} \vdash A$}
	\RightLabel{$(!,?)$ linear functor}	
	\UnaryInfC{$?(A^{\dag \dag}) \vdash ?A$}
	\RightLabel{$\dagger$-$exp.L$}
	\UnaryInfC{$(!A^\dag)^\dag \vdash ?A$}
	\RightLabel{$\dagger$}
	\UnaryInfC{$(?A)^\dag \vdash (!(A^\dag))^{\dag \dag}$ }
	\RightLabel{$\iota R^{(-1)}$}
	\UnaryInfC{$(?A)^\dag \vdash !(A^\dag)$}  
	\DisplayProof
\hspace{1.5em}
\AxiomC{$ $}
\RightLabel{$id$}
\UnaryInfC{$A \vdash A$}
\RightLabel{$\dagger$}	
\UnaryInfC{$A^\dag \vdash A^\dag$}
\RightLabel{$(!,?)$ linear functor}	
\UnaryInfC{$!(A^\dag) \vdash !(A^\dag)$ }
\RightLabel{$\dagger$-$exp.R$}
\UnaryInfC{$!(A^\dag) \vdash (?A)^\dag$}     	
\DisplayProof
\caption{Derivation of the isomorphism $\s: (?A)^\dag \rightarrow !(A^\dagger)$}
\label{Fig: s derivation}
\end{figure}

\begin{figure} [h]
	\centering	
\iffalse				\AxiomC{$ $}   % PROOF WITH CUT
				\RightLabel{$id$}
				\UnaryInfC{$A \vdash A$}
				\RightLabel{$\dagger$}	
				\UnaryInfC{$A^\dag \vdash A^\dag$}
				\RightLabel{$der.R$}	
				\UnaryInfC{$A^\dag \vdash ?(A^\dag)$}
				\RightLabel{$stor.R$}	
				\UnaryInfC{$?(A^\dag) \vdash ?(A^\dag)$}
				\RightLabel{$\iota R$}	
				\UnaryInfC{$?(A^\dag) \vdash (?(A^\dag))^{\dag \dag}$}
		\AxiomC{$ $}
		\RightLabel{$id$}
		\UnaryInfC{$A \vdash A$}
		\RightLabel{$der.L$}	
		\UnaryInfC{$!A \vdash A$}
		\RightLabel{$\iota R$}	
		\UnaryInfC{$!A \vdash A^{\dag \dag}$}
		\RightLabel{$stor.R$}	
		\UnaryInfC{$!A \vdash !(A^{\dag\dag})$}
		\RightLabel{$\dagger$-$exp.R$}
		\UnaryInfC{$!A \vdash (?(A^\dag))^\dag$}
		\RightLabel{$\dagger$}
		\UnaryInfC{$(?(A^\dag))^{\dag \dag} \vdash (!A)^\dag$ }
\RightLabel{$cut$}
\BinaryInfC{$?(A^\dag) \vdash (!A)^\dag$}     	
\DisplayProof 
\vspace{1em} \fi
		\AxiomC{$ $}
		\RightLabel{$id$}
		\UnaryInfC{$A \vdash A$}
		\RightLabel{$\iota R$}	
		\UnaryInfC{$A \vdash A^{\dag \dag}$}
		\RightLabel{$(!,?)$ linear functor}	
		\UnaryInfC{$!A \vdash !(A^{\dag\dag})$}
		\RightLabel{$\dagger$-$exp.R$}
		\UnaryInfC{$!A \vdash (?(A^\dag))^\dag$}
		\RightLabel{$\dagger$}
		\UnaryInfC{$(?(A^\dag))^{\dag \dag} \vdash (!A)^\dag$ }
		\RightLabel{$\iota L^{(-1)}$}
		\UnaryInfC{$?(A^\dag) \vdash (!A)^\dag$}     	
		\DisplayProof 
\hspace{1.5em}
\AxiomC{$ $}
\RightLabel{$id$}
\UnaryInfC{$A \vdash A$}
\RightLabel{$\dagger$}	
\UnaryInfC{$A^\dag \vdash A^\dag$}
\RightLabel{$(!,?)$ linear functor}	
\UnaryInfC{$?(A^\dag) \vdash ?(A^\dag)$ }
\RightLabel{$\dagger$-$exp.L$}
\UnaryInfC{$(!A)^\dag \vdash ?(A^\dag)$}     	
\DisplayProof
\caption{Derivation of the isomorphism $\t: (!A)^\dag \rightarrow ?(A^\dagger)$}
\label{Fig: t derivation}
\end{figure}

In a $(!,?)$-$\dagger$-LDC, for any $\dagger$-dual, $A$, $!A$ is a $\dagger$-dual, because the pair $(!,?)$ is a 
$\dagger$-linear functor, and $\dagger$-linear functor preserve $\dagger$-duals. Figure \ref{Fig: exp dagger duals} 
shows the proof in sequent calculus, that, if $A$ is a $\dagger$-dual, then $!A$ is also a $\dagger$-dual.

\begin{figure}[h]
    \centering
    \AxiomC{$ $}
    \RightLabel{$id, (\top L)$}
    \UnaryInfC{ $A \vdash A$}
    \RightLabel{$\dag$-$dual.R$}
    \UnaryInfC{ $\vdash A, A^\dag$}
    \RightLabel{$(!,?)$ linear functor}
    \UnaryInfC{$ \vdash !A , ?(A^\dag)$} 
    \AxiomC{$?(A^\dag) \vdash (!A)^\dag$}
    \RightLabel{cut}
    \BinaryInfC{ $\vdash !A , (!A)^\dagger$} 
    \RightLabel{$(\top L), (\oa R)$}
    \UnaryInfC{$\top \vdash !A \oa (!A)^\dag$}
    \DisplayProof
    \hspace{1.5em}
    \AxiomC{$ $}
    \RightLabel{$id$}
    \UnaryInfC{$ A \vdash A$}
    \RightLabel{$\dag$-$dual.L$}
    \UnaryInfC{ $A^\dag, A \vdash$ }
    \RightLabel{$(!,?)$ linear functor}
    \UnaryInfC{ $?(A^\dag) , !A \vdash $}
    \RightLabel{$\dagger$-$exp$.L}
    \UnaryInfC{$(!A)^\dag , !A \vdash $}
    \RightLabel{ $(\ox L), (\bot R)$}
    \UnaryInfC{$(!A)^\dag \ox !A \vdash \bot$}
    \DisplayProof
    \caption{Sequent proof:- $!A$ is a $\dagger$-dual if $A$ is a $\dagger$-dual}
    \label{Fig: exp dagger duals}
\end{figure}

\FloatBarrier

\subsection{Complementarity from free exponential modalities}

In this section, we prove that, in a $\dagger$-isomix category with free exponentials, every $\dagger$-complementary 
system arises as a compaction of a $\dagger$-linear bialgbebra induced by the free exponentials. For this, 
we consider any complementary system in an isomix category and the linear bialgebra induced 
on the complementary system by the free exponential modalities (as in Lemma \ref{Lemma: !? linear monoid}). 
We show that the induced linear bialgebra is equipped with a sectional coring binary idempotent which splits to give the original
complementary system.  

\begin{definition}
A $(!,?)$-LDC has {\bf free exponential modalities} \cite{Laf88} if, for any object $A$, 
$(!A, \Delta_A, \trianglecounit{0.55}_A)$ is a cofree cocommutative $\ox$-comonoid and 
and $(?A, \nabla_A, \triangleunit{0.55}_A)$ is a free commutative $\oa$-monoid. This means that,  
if  $(B, d, e)$ is a cocommutative comonoid, then for all map $f: B \to A$, there exists a unique 
comonoid morphism from $f^\flat: (B, d, e) \to (!A, \Delta_A, \trianglecounit{0.65})$ such that the 
diagram $(a)$ commutes. Similarly, if $(C, m, u)$ is a commutative monoid, then for all $g: A \to C$, 
there exists a unique monoid morphism $g^\sharp: (?A, \nabla_A, \triunit{0.65}_A) \to (C, m, u)$ such that 
diagram $(b)$ commutes: 
\[ (a)~ \xymatrixcolsep{4pc} 
\xymatrix{ (B,d,e) \ar[dr]^{f} \ar@{.>}[d]_{\exists!~ f^\flat} & \\
(!A, \Delta_A, \trianglecounit{0.65}) \ar[r]_{\epsilon_A} & A} 
~~~~~~~~
(b)~ \xymatrix{ A \ar[r]^{\eta_A} \ar[dr]_{g} & 
(?A, \nabla_A, \triunit{0.65}_A)  \ar@{.>}[d]^{\exists!~ g^\sharp} \\
& (C, m , u)} \] 
\end{definition}

The universal property of free exponential modalities in a $(!,?)$-LDC implies that given any dual, 
the linear comonoid induced on the dual by the free exponentials is couniversal in the following sense:
\begin{lemma}
	\label{Lemma: free !? linear comonoid}
	In a $(!,?)$-LDC with free exponentials, if $(x,y)\!\!:\! X \lincomonbtik Y$ is a linear comonoid and 
	\[ (f,g)\!\!:\! ((x,y)\!\!:\! X \dashvv Y) \to ((a,b)\!\!:\! A \dashvv B) \]
	is a morphism of duals then, the unique map $(f^\flat, g^\sharp)$ induced by the 
	universal property of the free exponential is a morphism of linear 
	comonoids:
	\[ \xymatrix{
		(x,y): X \lincomonbtik Y \ar@{..>}[d]_{(f^\flat,g^\sharp)}  \ar[drr]^{(f,g)} \\
		(a_!,b_?):!A \lincomonwtik ?B  \ar[rr]_{(\epsilon,\eta)} & & (a,b): A \dashvv~B }  \]
\end{lemma}
\begin{proof}
Given that $(f,g)$ is a morphism of duals. We must show that $(f^\flat, g^\sharp)$ given by the 
universal property of $(!,?)$ is a morphism of duals by showing the equation on the left, below, holds.  
However, the equation on the left holds if and only if the equation on the right holds:
\[ \begin{tikzpicture}
	\begin{pgfonlayer}{nodelayer}
		\node [style=none] (0) at (-1, -0.75) {};
		\node [style=none] (1) at (-1, 0.75) {};
		\node [style=none] (2) at (0, 0.75) {};
		\node [style=none] (3) at (0, -0.75) {};
		\node [style=onehalfcircle] (4) at (-1, 0) {};
		\node [style=none] (5) at (-1, 0) {$f$};
		\node [style=none] (6) at (-0.5, 1.5) {$x$};
		\node [style=none] (7) at (-1.5, 0.75) {$X$};
		\node [style=none] (8) at (-1.5, -0.5) {$!A$};
		\node [style=none] (9) at (0.5, -0.5) {$Y$};
	\end{pgfonlayer}
	\begin{pgfonlayer}{edgelayer}
		\draw [bend left=90, looseness=1.75] (1.center) to (2.center);
		\draw (2.center) to (3.center);
		\draw (1.center) to (4);
		\draw (4) to (0.center);
	\end{pgfonlayer}
\end{tikzpicture} = \begin{tikzpicture}
	\begin{pgfonlayer}{nodelayer}
		\node [style=none] (0) at (0, -0.75) {};
		\node [style=none] (1) at (0, 0.75) {};
		\node [style=none] (2) at (-1, 0.75) {};
		\node [style=none] (3) at (-1, -0.75) {};
		\node [style=circle, scale=1.8] (4) at (0, 0) {};
		\node [style=none] (5) at (0, 0) {$g^\sharp$};
		\node [style=none] (6) at (-0.5, 1.5) {$a$};
		\node [style=none] (8) at (-1.5, -0.5) {$!A$};
		\node [style=none] (9) at (0.5, -0.5) {$Y$};
		\node [style=none] (10) at (-1.5, 0.5) {};
		\node [style=none] (11) at (0.5, 0.5) {};
		\node [style=none] (12) at (-1.5, 1.75) {};
		\node [style=none] (13) at (0.5, 1.75) {};
		\node [style=none] (14) at (-1.25, 0.5) {};
		\node [style=none] (15) at (-0.75, 0.5) {};
		\node [style=none] (16) at (0.25, 0.75) {$!$};
	\end{pgfonlayer}
	\begin{pgfonlayer}{edgelayer}
		\draw [bend right=90, looseness=1.75] (1.center) to (2.center);
		\draw (2.center) to (3.center);
		\draw (1.center) to (4);
		\draw (4) to (0.center);
		\draw (11.center) to (13.center);
		\draw (12.center) to (13.center);
		\draw (10.center) to (12.center);
		\draw (10.center) to (11.center);
		\draw [bend left=90, looseness=1.25] (14.center) to (15.center);
	\end{pgfonlayer}
\end{tikzpicture}  ~~~~~ \Leftrightarrow ~~~~~  \begin{tikzpicture}
	\begin{pgfonlayer}{nodelayer}
		\node [style=none] (3) at (0, -1) {};
		\node [style=onehalfcircle] (4) at (0, 0.5) {};
		\node [style=none] (5) at (0, 0.5) {$f^\flat$};
		\node [style=none] (18) at (0, 1.75) {};
		\node [style=none] (19) at (0.25, 1.5) {$X$};
		\node [style=none] (20) at (0.25, -0.75) {$!A$};
	\end{pgfonlayer}
	\begin{pgfonlayer}{edgelayer}
		\draw (3.center) to (4);
		\draw (4) to (18.center);
	\end{pgfonlayer}
\end{tikzpicture}
 =  \begin{tikzpicture}
	\begin{pgfonlayer}{nodelayer}
		\node [style=none] (0) at (0, -0.75) {};
		\node [style=none] (1) at (0, 0.75) {};
		\node [style=none] (2) at (-1, 0.75) {};
		\node [style=none] (3) at (-1, -1.25) {};
		\node [style=circle, scale=1.8] (4) at (0, 0) {};
		\node [style=none] (5) at (0, 0) {$g^\sharp$};
		\node [style=none] (6) at (-0.5, 1.5) {$a$};
		\node [style=none] (8) at (-1.5, -0.5) {$!A$};
		\node [style=none] (9) at (-0.25, -0.5) {$Y$};
		\node [style=none] (10) at (-1.5, 0.5) {};
		\node [style=none] (11) at (0.5, 0.5) {};
		\node [style=none] (12) at (-1.5, 1.75) {};
		\node [style=none] (13) at (0.5, 1.75) {};
		\node [style=none] (14) at (-1.25, 0.5) {};
		\node [style=none] (15) at (-0.75, 0.5) {};
		\node [style=none] (16) at (0.25, 0.75) {$!$};
		\node [style=none] (17) at (1, -0.75) {};
		\node [style=none] (18) at (1, 2) {};
		\node [style=none] (19) at (1.25, 1.75) {$X$};
	\end{pgfonlayer}
	\begin{pgfonlayer}{edgelayer}
		\draw [bend right=90, looseness=1.75] (1.center) to (2.center);
		\draw (2.center) to (3.center);
		\draw (1.center) to (4);
		\draw (4) to (0.center);
		\draw (11.center) to (13.center);
		\draw (12.center) to (13.center);
		\draw (10.center) to (12.center);
		\draw (10.center) to (11.center);
		\draw [bend left=90, looseness=1.25] (14.center) to (15.center);
		\draw [bend right=75, looseness=1.25] (0.center) to (17.center);
		\draw (17.center) to (18.center);
	\end{pgfonlayer}
\end{tikzpicture}  \]

Now, by the couniversal property of the $!$ functor, $f^\flat$ is a unique comonoid homomorpshim such that 
$f^\flat \epsilon = f$. To prove that the latter equation holds, it suffices to prove that:
\[ \begin{tikzpicture}
	\begin{pgfonlayer}{nodelayer}
		\node [style=none] (0) at (0, -0.5) {};
		\node [style=none] (1) at (0, 0.75) {};
		\node [style=none] (2) at (-1, 0.75) {};
		\node [style=none] (3) at (-1, -1) {};
		\node [style=circle, scale=1.8] (4) at (0, 0) {};
		\node [style=none] (5) at (0, 0) {$g^\sharp$};
		\node [style=none] (6) at (-0.5, 1.5) {$a$};
		\node [style=none] (8) at (-1.25, 0.25) {$!A$};
		\node [style=none] (9) at (-0.25, -0.5) {$Y$};
		\node [style=none] (10) at (-1.5, 0.5) {};
		\node [style=none] (11) at (0.5, 0.5) {};
		\node [style=none] (12) at (-1.5, 1.75) {};
		\node [style=none] (13) at (0.5, 1.75) {};
		\node [style=none] (14) at (-1.25, 0.5) {};
		\node [style=none] (15) at (-0.75, 0.5) {};
		\node [style=none] (16) at (0.25, 0.75) {$!$};
		\node [style=none] (17) at (1, -0.5) {};
		\node [style=none] (18) at (1, 1.75) {};
		\node [style=none] (19) at (1.25, 1.5) {$X$};
		\node [style=onehalfcircle] (21) at (-1, -0.25) {};
		\node [style=none] (20) at (-1, -0.25) {$\epsilon$};
		\node [style=none] (22) at (-1.25, -0.75) {$A$};
		\node [style=none] (23) at (0.5, -1.1) {$y$};
	\end{pgfonlayer}
	\begin{pgfonlayer}{edgelayer}
		\draw [bend right=90, looseness=1.75] (1.center) to (2.center);
		\draw (1.center) to (4);
		\draw (4) to (0.center);
		\draw (11.center) to (13.center);
		\draw (12.center) to (13.center);
		\draw (10.center) to (12.center);
		\draw (10.center) to (11.center);
		\draw [bend left=90, looseness=1.25] (14.center) to (15.center);
		\draw [bend right=75, looseness=1.25] (0.center) to (17.center);
		\draw (17.center) to (18.center);
		\draw (3.center) to (21);
		\draw (2.center) to (21);
	\end{pgfonlayer}
\end{tikzpicture} \stackrel{(1)}{=} \begin{tikzpicture}
	\begin{pgfonlayer}{nodelayer}
		\node [style=none] (0) at (0, -0.5) {};
		\node [style=none] (1) at (0, 1.25) {};
		\node [style=none] (2) at (-1, 1.25) {};
		\node [style=none] (3) at (-1, -1) {};
		\node [style=circle, scale=1.8] (4) at (0, 0) {};
		\node [style=none] (5) at (0, 0) {$g^\sharp$};
		\node [style=none] (6) at (-0.5, 2) {$a$};
		\node [style=none] (9) at (-0.25, -0.5) {$Y$};
		\node [style=none] (17) at (1, -0.5) {};
		\node [style=none] (18) at (1, 1.75) {};
		\node [style=none] (19) at (1.25, 1.5) {$X$};
		\node [style=none] (22) at (-1.25, -0.75) {$A$};
		\node [style=onehalfcircle] (23) at (0, 0.75) {};
		\node [style=none] (24) at (0, 0.75) {$\eta$};
		\node [style=none] (25) at (0.5, 0.5) {?B};
		\node [style=none] (26) at (0.25, 1.25) {$B$};
		\node [style=none] (27) at (0.5, -1.1) {$y$};
	\end{pgfonlayer}
	\begin{pgfonlayer}{edgelayer}
		\draw [bend right=90, looseness=1.75] (1.center) to (2.center);
		\draw (4) to (0.center);
		\draw [bend right=75, looseness=1.25] (0.center) to (17.center);
		\draw (17.center) to (18.center);
		\draw (4) to (23);
		\draw (23) to (1.center);
		\draw (3.center) to (2.center);
	\end{pgfonlayer}
\end{tikzpicture}  = \begin{tikzpicture}
	\begin{pgfonlayer}{nodelayer}
		\node [style=none] (0) at (0, -0.25) {};
		\node [style=none] (1) at (0, 1) {};
		\node [style=none] (2) at (-1, 1) {};
		\node [style=none] (3) at (-1, -1) {};
		\node [style=none] (6) at (-0.5, 1.75) {$a$};
		\node [style=none] (9) at (-0.25, 0) {$Y$};
		\node [style=none] (17) at (1, -0.25) {};
		\node [style=none] (18) at (1, 1.75) {};
		\node [style=none] (19) at (1.25, 1.5) {$X$};
		\node [style=none] (22) at (-1.25, -0.75) {$A$};
		\node [style=onehalfcircle] (23) at (0, 0.5) {};
		\node [style=none] (24) at (0, 0.5) {$g$};
		\node [style=none] (26) at (0.25, 1) {$B$};
		\node [style=none] (27) at (0.5, -0.85) {y};
	\end{pgfonlayer}
	\begin{pgfonlayer}{edgelayer}
		\draw [bend right=90, looseness=1.75] (1.center) to (2.center);
		\draw [bend right=75, looseness=1.25] (0.center) to (17.center);
		\draw (17.center) to (18.center);
		\draw (23) to (1.center);
		\draw (3.center) to (2.center);
		\draw (23) to (0.center);
	\end{pgfonlayer}
\end{tikzpicture} \stackrel{(2)}{=}  \begin{tikzpicture}
	\begin{pgfonlayer}{nodelayer}
		\node [style=none] (3) at (0, -1) {};
		\node [style=onehalfcircle] (4) at (0, 0.5) {};
		\node [style=none] (5) at (0, 0.5) {$f$};
		\node [style=none] (18) at (0, 1.75) {};
		\node [style=none] (19) at (0.25, 1.5) {$X$};
		\node [style=none] (20) at (0.25, -0.75) {$!A$};
	\end{pgfonlayer}
	\begin{pgfonlayer}{edgelayer}
		\draw (3.center) to (4);
		\draw (4) to (18.center);
	\end{pgfonlayer}
\end{tikzpicture} \]
Where $(1)$ is holds because $(\epsilon, \eta)$ is a linear transformation, while $(2)$ holds as $f$ and $g$ are mates. 
\[ \begin{tikzpicture}
	\begin{pgfonlayer}{nodelayer}
		\node [style=none] (43) at (3, -4.75) {};
		\node [style=none] (44) at (3, -2.75) {};
		\node [style=none] (45) at (2, -2.75) {};
		\node [style=circle, scale=1.8] (46) at (3, -4) {};
		\node [style=none] (47) at (3, -4) {$g^\sharp$};
		\node [style=none] (48) at (2.5, -2) {$a$};
		\node [style=none] (49) at (1.75, -4.75) {$!A$};
		\node [style=none] (50) at (2.75, -4.75) {$Y$};
		\node [style=none] (51) at (1.5, -3) {};
		\node [style=none] (52) at (3.5, -3) {};
		\node [style=none] (53) at (1.5, -1.75) {};
		\node [style=none] (54) at (3.5, -1.75) {};
		\node [style=none] (55) at (1.75, -3) {};
		\node [style=none] (56) at (2.25, -3) {};
		\node [style=none] (57) at (3.25, -2.75) {$!$};
		\node [style=none] (58) at (4, -4.75) {};
		\node [style=none] (59) at (4, -1.75) {};
		\node [style=none] (60) at (4.25, -2) {$X$};
		\node [style=none] (61) at (3.5, -5.35) {$y$};
		\node [style=none] (62) at (2, -5) {};
		\node [style=none] (63) at (1.75, -5.25) {};
		\node [style=none] (64) at (2.25, -5.25) {};
		\node [style=none] (65) at (1.5, -6) {};
		\node [style=none] (66) at (2.5, -6) {};
	\end{pgfonlayer}
	\begin{pgfonlayer}{edgelayer}
		\draw [bend right=90, looseness=1.75] (44.center) to (45.center);
		\draw (44.center) to (46);
		\draw (46) to (43.center);
		\draw (52.center) to (54.center);
		\draw (53.center) to (54.center);
		\draw (51.center) to (53.center);
		\draw (51.center) to (52.center);
		\draw [bend left=90, looseness=1.25] (55.center) to (56.center);
		\draw [bend right=75, looseness=1.25] (43.center) to (58.center);
		\draw (58.center) to (59.center);
		\draw (62.center) to (63.center);
		\draw (63.center) to (64.center);
		\draw (64.center) to (62.center);
		\draw (62.center) to (45.center);
		\draw [in=90, out=-45] (64.center) to (66.center);
		\draw [in=90, out=-135, looseness=1.25] (63.center) to (65.center);
	\end{pgfonlayer}
\end{tikzpicture} \stackrel{(1)}{=} \begin{tikzpicture}
	\begin{pgfonlayer}{nodelayer}
		\node [style=none] (0) at (-1.25, -5.25) {};
		\node [style=none] (1) at (-3.75, -2.25) {};
		\node [style=none] (2) at (-4.75, -2.25) {};
		\node [style=circle, scale=1.8] (4) at (-1.25, -4.75) {};
		\node [style=none] (5) at (-1.25, -4.75) {$g^\sharp$};
		\node [style=none] (6) at (-4.25, -1.5) {$a$};
		\node [style=none] (8) at (-5, -5.5) {$!A$};
		\node [style=none] (9) at (-1.5, -5.25) {$Y$};
		\node [style=none] (10) at (-5.25, -2.5) {};
		\node [style=none] (11) at (-3, -2.5) {};
		\node [style=none] (12) at (-5.25, -1.25) {};
		\node [style=none] (13) at (-3, -1.25) {};
		\node [style=none] (14) at (-5, -2.5) {};
		\node [style=none] (15) at (-4.5, -2.5) {};
		\node [style=none] (16) at (-3.5, -2.25) {$!$};
		\node [style=none] (17) at (0.5, -5.25) {};
		\node [style=none] (18) at (0.5, -1.5) {};
		\node [style=none] (19) at (0.75, -2) {$X$};
		\node [style=none] (23) at (-0.5, -6) {$y$};
		\node [style=none] (24) at (-1.25, -4.25) {};
		\node [style=none] (25) at (-1.5, -4) {};
		\node [style=none] (26) at (-1, -4) {};
		\node [style=none] (27) at (-1.75, -3.75) {};
		\node [style=none] (28) at (-0.75, -3.75) {};
		\node [style=none] (29) at (-4.75, -6) {};
		\node [style=none] (30) at (-0.75, -2.5) {};
		\node [style=none] (31) at (-2.25, -2.5) {};
		\node [style=none] (32) at (-1.5, -1.75) {$a$};
		\node [style=none] (33) at (-2.75, -2.75) {};
		\node [style=none] (34) at (-0.25, -2.75) {};
		\node [style=none] (35) at (-2.75, -1.5) {};
		\node [style=none] (36) at (-0.25, -1.5) {};
		\node [style=none] (37) at (-2.5, -2.75) {};
		\node [style=none] (38) at (-2, -2.75) {};
		\node [style=none] (39) at (-0.5, -2.5) {$!$};
		\node [style=none] (40) at (-2.25, -5.75) {};
		\node [style=none] (41) at (1.75, -5.5) {$!A$};
		\node [style=none] (42) at (-2.5, -5.25) {$!A$};
	\end{pgfonlayer}
	\begin{pgfonlayer}{edgelayer}
		\draw [bend right=90, looseness=1.75] (1.center) to (2.center);
		\draw (4) to (0.center);
		\draw (11.center) to (13.center);
		\draw (12.center) to (13.center);
		\draw (10.center) to (12.center);
		\draw (10.center) to (11.center);
		\draw [bend left=90, looseness=1.25] (14.center) to (15.center);
		\draw [bend right=90] (0.center) to (17.center);
		\draw (17.center) to (18.center);
		\draw (24.center) to (25.center);
		\draw (25.center) to (26.center);
		\draw (26.center) to (24.center);
		\draw [in=-90, out=45] (26.center) to (28.center);
		\draw [in=-90, out=135, looseness=1.25] (25.center) to (27.center);
		\draw (29.center) to (2.center);
		\draw [bend left=270, looseness=1.25] (30.center) to (31.center);
		\draw (34.center) to (36.center);
		\draw (35.center) to (36.center);
		\draw (33.center) to (35.center);
		\draw (33.center) to (34.center);
		\draw [bend left=90, looseness=1.25] (37.center) to (38.center);
		\draw (40.center) to (31.center);
		\draw [in=90, out=-90, looseness=0.50] (1.center) to (27.center);
		\draw (30.center) to (28.center);
		\draw (4) to (24.center);
	\end{pgfonlayer}
\end{tikzpicture}
 \stackrel{(2)}{=} \begin{tikzpicture}
	\begin{pgfonlayer}{nodelayer}
		\node [style=none] (0) at (-1.25, -5.25) {};
		\node [style=none] (1) at (-4, -2.75) {};
		\node [style=none] (2) at (-5, -2.75) {};
		\node [style=circle, scale=1.8] (4) at (-1.75, -4) {};
		\node [style=none] (5) at (-1.75, -4) {$g^\sharp$};
		\node [style=none] (6) at (-4.5, -2) {$a$};
		\node [style=none] (8) at (-5.25, -5.5) {$!A$};
		\node [style=none] (9) at (-2, -5.25) {$Y$};
		\node [style=none] (10) at (-5.5, -3) {};
		\node [style=none] (11) at (-3.25, -3) {};
		\node [style=none] (12) at (-5.5, -1.75) {};
		\node [style=none] (13) at (-3.25, -1.75) {};
		\node [style=none] (14) at (-5.25, -3) {};
		\node [style=none] (15) at (-4.75, -3) {};
		\node [style=none] (16) at (-3.75, -2.75) {$!$};
		\node [style=none] (17) at (0.5, -5.25) {};
		\node [style=none] (18) at (0.5, -1.5) {};
		\node [style=none] (19) at (0.75, -2) {$X$};
		\node [style=none] (23) at (-0.5, -6) {$y$};
		\node [style=none] (27) at (-1.75, -3.75) {};
		\node [style=none] (28) at (-0.75, -3.5) {};
		\node [style=none] (29) at (-5, -6) {};
		\node [style=none] (30) at (-0.75, -2.75) {};
		\node [style=none] (31) at (-2.25, -2.75) {};
		\node [style=none] (32) at (-1.5, -2) {$a$};
		\node [style=none] (33) at (-2.75, -3) {};
		\node [style=none] (34) at (-0.25, -3) {};
		\node [style=none] (35) at (-2.75, -1.75) {};
		\node [style=none] (36) at (-0.25, -1.75) {};
		\node [style=none] (37) at (-2.5, -3) {};
		\node [style=none] (38) at (-2, -3) {};
		\node [style=none] (39) at (-0.5, -2.75) {$!$};
		\node [style=none] (40) at (-2.25, -5.75) {};
		\node [style=none] (42) at (-2.5, -5.25) {$!A$};
		\node [style=circle, scale=1.8] (43) at (-0.75, -4) {};
		\node [style=none] (44) at (-0.75, -4) {$g^\sharp$};
		\node [style=black] (45) at (-1.25, -4.75) {};
		\node [style=black] (46) at (-1.25, -4.75) {};
	\end{pgfonlayer}
	\begin{pgfonlayer}{edgelayer}
		\draw [bend right=90, looseness=1.75] (1.center) to (2.center);
		\draw (11.center) to (13.center);
		\draw (12.center) to (13.center);
		\draw (10.center) to (12.center);
		\draw (10.center) to (11.center);
		\draw [bend left=90, looseness=1.25] (14.center) to (15.center);
		\draw [bend right=90] (0.center) to (17.center);
		\draw (17.center) to (18.center);
		\draw (29.center) to (2.center);
		\draw [bend left=270, looseness=1.25] (30.center) to (31.center);
		\draw (34.center) to (36.center);
		\draw (35.center) to (36.center);
		\draw (33.center) to (35.center);
		\draw (33.center) to (34.center);
		\draw [bend left=90, looseness=1.25] (37.center) to (38.center);
		\draw (40.center) to (31.center);
		\draw [in=105, out=-90, looseness=0.75] (1.center) to (27.center);
		\draw (30.center) to (28.center);
		\draw (45) to (0.center);
		\draw [in=165, out=-105, looseness=1.25] (4) to (45);
		\draw [in=-90, out=15, looseness=1.25] (46) to (43);
		\draw (43) to (28.center);
		\draw (4) to (27.center);
	\end{pgfonlayer}
\end{tikzpicture} = \begin{tikzpicture}
	\begin{pgfonlayer}{nodelayer}
		\node [style=none] (0) at (-0.75, -4.5) {};
		\node [style=none] (1) at (-4, -2.75) {};
		\node [style=none] (2) at (-5, -2.75) {};
		\node [style=circle, scale=1.8] (4) at (-1.75, -4) {};
		\node [style=none] (5) at (-1.75, -4) {$g^\sharp$};
		\node [style=none] (6) at (-4.5, -2) {$a$};
		\node [style=none] (8) at (-5.25, -5.5) {$!A$};
		\node [style=none] (10) at (-5.5, -3) {};
		\node [style=none] (11) at (-3.25, -3) {};
		\node [style=none] (12) at (-5.5, -1.75) {};
		\node [style=none] (13) at (-3.25, -1.75) {};
		\node [style=none] (14) at (-5.25, -3) {};
		\node [style=none] (15) at (-4.75, -3) {};
		\node [style=none] (16) at (-3.75, -2.75) {$!$};
		\node [style=none] (17) at (0.5, -4.5) {};
		\node [style=none] (18) at (0.5, -3.75) {};
		\node [style=none] (19) at (0.75, -2) {$X$};
		\node [style=none] (23) at (0, -5.75) {$y$};
		\node [style=none] (27) at (-1.75, -3.75) {};
		\node [style=none] (28) at (-0.75, -3.5) {};
		\node [style=none] (29) at (-5, -6) {};
		\node [style=none] (30) at (-0.75, -2.75) {};
		\node [style=none] (31) at (-2.25, -2.75) {};
		\node [style=none] (32) at (-1.5, -2) {$a$};
		\node [style=none] (33) at (-2.75, -3) {};
		\node [style=none] (34) at (-0.25, -3) {};
		\node [style=none] (35) at (-2.75, -1.75) {};
		\node [style=none] (36) at (-0.25, -1.75) {};
		\node [style=none] (37) at (-2.5, -3) {};
		\node [style=none] (38) at (-2, -3) {};
		\node [style=none] (39) at (-0.5, -2.75) {$!$};
		\node [style=none] (40) at (-2.25, -5.75) {};
		\node [style=none] (42) at (-2.5, -5.25) {$!A$};
		\node [style=circle, scale=1.8] (43) at (-0.75, -4) {};
		\node [style=none] (44) at (-0.75, -4) {$g^\sharp$};
		\node [style=none] (45) at (-1.75, -4.5) {};
		\node [style=none] (46) at (1.5, -4.5) {};
		\node [style=none] (47) at (1.5, -3.75) {};
		\node [style=none] (48) at (-0.75, -4) {};
		\node [style=black] (49) at (1, -2.75) {};
		\node [style=none] (50) at (1, -1.75) {};
		\node [style=none] (51) at (0, -5.1) {$y$};
	\end{pgfonlayer}
	\begin{pgfonlayer}{edgelayer}
		\draw [bend right=90, looseness=1.75] (1.center) to (2.center);
		\draw (11.center) to (13.center);
		\draw (12.center) to (13.center);
		\draw (10.center) to (12.center);
		\draw (10.center) to (11.center);
		\draw [bend left=90, looseness=1.25] (14.center) to (15.center);
		\draw [bend right=90] (0.center) to (17.center);
		\draw (17.center) to (18.center);
		\draw (29.center) to (2.center);
		\draw [bend left=270, looseness=1.25] (30.center) to (31.center);
		\draw (34.center) to (36.center);
		\draw (35.center) to (36.center);
		\draw (33.center) to (35.center);
		\draw (33.center) to (34.center);
		\draw [bend left=90, looseness=1.25] (37.center) to (38.center);
		\draw (40.center) to (31.center);
		\draw [in=105, out=-90, looseness=0.75] (1.center) to (27.center);
		\draw (30.center) to (28.center);
		\draw (43) to (28.center);
		\draw (4) to (27.center);
		\draw [bend right=90] (45.center) to (46.center);
		\draw (47.center) to (46.center);
		\draw (4) to (45.center);
		\draw (0.center) to (48.center);
		\draw [in=90, out=-150] (49) to (18.center);
		\draw [in=90, out=-30, looseness=1.25] (49) to (47.center);
		\draw (50.center) to (49);
	\end{pgfonlayer}
\end{tikzpicture} \]

Step $(1)$ is uses the fact  $(\Delta, \nabla)$ is a linear transformation, while $(2)$ holds as $g^\sharp$ is a morphism of monoids. 
Hence, $(f^\flat, g^\sharp)$ is a morphism of duals.
\end{proof}

Next we prove that in a $(!,?)$-LDC with free exponential modalities, given any monoid the 
linear bialgebra induced on the monoid by the free exponentials as in Lemma \ref{Lemma: !? linear monoid}
 is couniversal in the following sense:

\begin{lemma}
	\label{Lemma: ! monoid hom}
    In  $(!,?)$-LDC with free exponential modalities if $(X, \mulmap{1.5}{white}, \unitmap{1.5}{white},
     \tricomul{0.65},  \tricounit{0.65})$ is a $\ox$-bialgebra,  and $f: X \to A$ is a monoid morphism, then 
	$f^\flat: X \to !A$ given by the couniversal property of the $!$ is a bialgebra homomorphism
	\[ \xymatrix { (X, \mulmap{1.5}{white}, \unitmap{1.5}{white}, \comulmap{1.5}{white}, \counitmap{1.5}{white})
	  \ar@{->}[dr]^{f} \ar@{.>}[d]_{f^\flat} & \\
	  (!A, \mulmap{1.5}{black}_!, \unitmap{1.5}{black}_!, \Delta_A, \tricounit{0.65}_A) \ar[r]_{~~~~~~~~~\epsilon_A} 
	& (A, \mulmap{1.5}{black}, \unitmap{1.5}{black}) } \]
	where the multiplication and the unit for $!A$ is induced by linearity of $(!,?)$:
	\[ \mulmap{1.5}{black}_! := !A \ox !A \to^{m_\ox} !(A \ox A) \to^{!\left(\mulmap{1.5}{black} \right)} !A ~~~~~~~~~~
	\unitmap{1.5}{black}_! := \top \to^{m_\top} !\top \to^{!\left(\unitmap{1.5}{black}\right)} !A \]
\end{lemma}
\begin{proof}
Given that in a $(!,?)$-LDC with free exponential modalities, 
 $(X, \mulmap{1.5}{white}, \unitmap{1.5}{white}, \tricomul{0.65},  \tricounit{0.65})$ is a bialgebra, 
$ (A, \mulmap{1.5}{black}, \unitmap{1.5}{black})$ is a monoid, and $f: X \to A$ is a monoid 
morphism. The monoid on $A$ induces a bialgebra on 
$(!A, \mulmap{1.5}{black}_!, \unitmap{1.5}{black}_!, \Delta_A, \tricounit{0.65}_A)$.
We must show that $f^\flat$ given by the couniversal property of $(!A, \Delta_A, \trianglecounit{0.65}_A)$ 
is a bialgebra morphism, that is, $f^\flat$ is a monoid and a comonoid morphism. 

From the given couniversal diagram, $f^\flat: (X, \mulmap{1.5}{black}, \unitmap{1.5}{black})
\to (!A, \Delta_A, e_A)$, is a comonoid morphism. 
We must prove that $f^\flat: (X, \mulmap{1.5}{white}, \unitmap{1.5}{white}) \to 
(!A, \mulmap{1.5}{black}_!, \unitmap{1.5}{black}_!)$ 
is a monoid morphism, that is the following diagrams commute:
  
{\centering $(a)~~\xymatrix{
  X \ox X \ar[r]^{f^\flat \ox f^\flat} \ar[d]_{\mulmap{1.5}{white}} & !A \ox !A \ar[d]^{\mulmap{1.5}{black}_!} \\
  X \ar[r]_{f^\flat} & !A } ~~~~~~~~
  (b)~~\xymatrix{
  \top \ar[dr]^{\unitmap{1.5}{black}_!} \ar[d]_{\unitmap{1.5}{white}}  &  \\ 
  X \ar[r]_{f^\flat} & !A}$ \par }
  
  In order to prove that $(a)$ commutes, consider the following couniversal diagram.
  \[ \xymatrix{ 
    (\top, (u_\ox^L)^{-1} = (u_\ox^R)^{-1}, 1_\top) \ar@{.>}[dd]_{\left(\unitmap{1.5}{white}f \right)^\flat} 
    \ar[dr]^-{~~~\unitmap{1.5}{white}} & \\ 
    & X \ar[drr]^{f} & \\
    (!A, \Delta_A, \trianglecounit{0.65}_A) \ar[rrr]_{\epsilon_A} &  & & A} \]

  Proving that $(\unitmap{1.5}{white} f)^\flat  = \unitmap{1.5}{white} f^\flat$:

  \vspace{1em}

  It is straightforward that $(\unitmap{1.5}{white}f)^\flat \epsilon_A = \unitmap{1.5}{white} f^\flat \epsilon_A 
  = \unitmap{1.5}{white}f $. Moreover, $\unitmap{1.5}{white}f^\flat$ is a comonoid homomorphism because:
\[ \begin{tikzpicture}
   \begin{pgfonlayer}{nodelayer}
       \node [style=circle] (0) at (1, 4) {};
       \node [style=onehalfcircle] (1) at (1, 3.25) {};
       \node [style=none] (2) at (1, 2.5) {};
       \node [style=none] (3) at (0.75, 2.25) {};
       \node [style=none] (4) at (1.25, 2.25) {};
       \node [style=none] (5) at (0.25, 1.25) {};
       \node [style=none] (6) at (1.75, 1.25) {};
       \node [style=none] (7) at (1.75, 1.25) {};
       \node [style=none] (8) at (1, 3.25) {$f^\flat$};
   \end{pgfonlayer}
   \begin{pgfonlayer}{edgelayer}
       \draw (2.center) to (3.center);
       \draw (3.center) to (4.center);
       \draw (4.center) to (2.center);
       \draw [in=-15, out=90] (7.center) to (4.center);
       \draw [in=90, out=-150] (3.center) to (5.center);
       \draw (2.center) to (1);
       \draw (1) to (0);
   \end{pgfonlayer}
\end{tikzpicture} \stackrel{(1)}{=}  \begin{tikzpicture}
   \begin{pgfonlayer}{nodelayer}
       \node [style=circle] (0) at (1, 3.5) {};
       \node [style=none] (5) at (0.25, 1) {};
       \node [style=none] (7) at (1.75, 1) {};
       \node [style=circle] (9) at (1, 2.5) {};
       \node [style=onehalfcircle] (10) at (0.25, 1.75) {};
       \node [style=none] (11) at (0.25, 1.75) {$f^\flat$};
       \node [style=onehalfcircle] (12) at (1.75, 1.75) {};
       \node [style=none] (13) at (1.75, 1.75) {$f^\flat$};
   \end{pgfonlayer}
   \begin{pgfonlayer}{edgelayer}
       \draw (7.center) to (12);
       \draw [bend right=45] (12) to (9);
       \draw [bend left=45] (10) to (9);
       \draw (10) to (5.center);
       \draw (0) to (9);
   \end{pgfonlayer}
\end{tikzpicture} \stackrel{(2)}{=} \begin{tikzpicture}
   \begin{pgfonlayer}{nodelayer}
       \node [style=circle] (0) at (0.25, 3.5) {};
       \node [style=none] (5) at (0.25, 1) {};
       \node [style=none] (7) at (1.25, 1) {};
       \node [style=onehalfcircle] (10) at (0.25, 2) {};
       \node [style=none] (11) at (0.25, 2) {$f^\flat$};
       \node [style=onehalfcircle] (12) at (1.25, 2) {};
       \node [style=none] (13) at (1.25, 2) {$f^\flat$};
       \node [style=circle] (14) at (1.25, 3.5) {};
   \end{pgfonlayer}
   \begin{pgfonlayer}{edgelayer}
       \draw (7.center) to (12);
       \draw (10) to (5.center);
       \draw (12) to (14);
       \draw (10) to (0);
   \end{pgfonlayer}
\end{tikzpicture}
~~~~~~~~~~~~
\begin{tikzpicture}
   \begin{pgfonlayer}{nodelayer}
       \node [style=circle] (0) at (0.25, 3.5) {};
       \node [style=none] (5) at (0.25, 1.25) {};
       \node [style=onehalfcircle] (10) at (0.25, 2) {};
       \node [style=none] (11) at (0.25, 2) {$f^\flat$};
       \node [style=none] (12) at (0, 1) {};
       \node [style=none] (13) at (0.5, 1) {};
   \end{pgfonlayer}
   \begin{pgfonlayer}{edgelayer}
       \draw (10) to (5.center);
       \draw (10) to (0);
       \draw (5.center) to (12.center);
       \draw (12.center) to (13.center);
       \draw (13.center) to (5.center);
   \end{pgfonlayer}
\end{tikzpicture} = \begin{tikzpicture}
   \begin{pgfonlayer}{nodelayer}
       \node [style=circle] (0) at (0.25, 3) {};
       \node [style=circle] (1) at (0.25, 1) {};
   \end{pgfonlayer}
   \begin{pgfonlayer}{edgelayer}
       \draw (1) to (0);
   \end{pgfonlayer}
\end{tikzpicture} \stackrel{(3)}{=} 1_\top \]
Where $(1)$ is because $f^\flat: X \to !A$ is a comonoid morphism, and $(2)$ and $(3)$
are because $X$ is a bialgebra.

\vspace{0.7em}

Proving that $(\unitmap{1.5}{white} f)^\flat  = \counitmap{1.5}{black}_{~!}$:

\vspace{0.7em}

We have that $ \unitmap{1.5}{black}_! \epsilon_A = m_\top ! \left( \unitmap{1.5}{black} \right) 
 \epsilon_A = \unitmap{1.5}{black} = \unitmap{1.5}{white} f$ because $\epsilon_A$ is a monoidal transformation and 
 $f$ is a monoid homomorphism. Moreover, $\unitmap{1.5}{black}_!$ is a comonoid morphism 
due to the naturality of $\Delta$ and $\tricounit{0.65}$.

\vspace{0.7em}

By the uniqueness of $\unitmap{1.5}{white}^\flat$, we have that $ (\counitmap{1.5}{white} f)^\flat \unitmap{1.5}{black}_!  = 
\unitmap{1.5}{white} f^\flat$. Thereby, $f^\flat$ preserves the unit of $X$.
Thus, $f^\flat: (X, \mulmap{1.5}{white}, \unitmap{1.5}{white}) \to (!A, \mulmap{1.5}{black}_!, 
\unitmap{1.5}{black}_!)$ is a monoid homomorphism. Since $f^\flat$ is a monoid and comonoid morphism, it is a bialgebra morphism.
\end{proof}

The results discussed so far can be combined to give the more complicated observation on a linear bialgebra:
\begin{proposition} 
	\label{Prop: free !? linear bialgebra}
	In a (!,?)-LDC with free exponential modalities, let $\frac{(x,y)}{(x',y')}:X \linbialgbtik Y$ be a linear bialgebra, $(a,b): A \linmonwtik B$ a linear 
	monoid, and $(a',b'): A \dashvv B$ a dual,  then
	\[ (f^\flat,g^\sharp): \left(\frac{(x,y)}{(x',y')}:X \linbialgbtik Y \right) \to \left(\frac{(a_!,b_?)}{(a'_!,b'_?)}:!A \expbialgwtik ?B \right)\] 
	is a morphism of bialgebras, whenever $f: (X, \mulmap{1.5}{black}, \unitmap{1.5}{black}) \to 
	(A, \mulmap{1.5}{white}, \unitmap{1.5}{white})$ 
	is a morphism of monoids, and $(f,g)$ is a morphism of both duals
	 \[ (f,g)\!\!:\! ((x,y)\!\!:\! X \linmonw Y) \to ((a,b)\!\!:\! A \linmonw B) ~~~ \text{ and }~~~ (f,g)\!\!:\!\! ((x',y')\!\!:\! X \dashvv Y) \to ((a',b')\!\!:\! A \dashvv B) \] 
	\end{proposition}
	\begin{proof}
	We are given a linear monoid $(a,b)\!\!:\! A \linmonw B$, and $(a',b')\!\!:\! A \dashvv B$ is an arbitrary dual. 
	Using Lemma \ref{Lemma: !? linear monoid}, we know that  $!A \expbialgwtik ?B$ is a linear bialgebra.
	
	 We are given that $(f,g)$ is a morphism of linear monoids, and a morphism of duals as in the diagram below. 
	We must prove that $(f^\flat, g^\sharp)$ is a morphism of bialgebras:
	\[ \xymatrixcolsep{6mm} \xymatrixrowsep{8mm} \xymatrix{
		\frac{(x,y)}{(x',y')}: X \linbialgbtik Y \ar@{..>}[d]_{(f^\flat,g^\sharp)}  \ar[drr]^{(f,g)} \\
		\frac{(a_!,b_?)}{(a'_!, b'_?)}: !A \expbialgwtik ?B  \ar[rr]_{(\epsilon,\eta)~~~~~~} & & (a,b)\!\!:\! A \linmonwtik  B~; (a', b')\!\!:\! A \dashvv B } \]
	
	Since $(f,g)\!\!:\! ((x',y'): X \dashvv Y) \to ((a', b')\!\!:\! A \dashvv B)$ is a morphism of duals 
	it follows from Lemma \ref{Lemma: free !? linear comonoid} that 
	$(f^\flat, g^\sharp): ((x',y'): X \dashvv Y) \to ((a'_!, b'_!): !A \dashvv ~?B)$ is a morphism of linear comonoids. 
	
	Given that $(f,g)$ is a morphism of linear monoids.
	This means that $(f,g): ((x,y): X \dashvv Y) \to ((a,b): A \dashvv B)$ is a morphism of duals, 
	and $f: (X, \mulmap{1.5}{black}, \unitmap{1.5}{black}) \to 
	(A, \mulmap{1.5}{white}, \unitmap{1.5}{white})$ is a morphism of monoids. 
	We have to prove that $(f^\flat, g^\sharp)$ is a morphism 
	of linear monoids.  
	
	We know that (in a symmetric LDC) any dual with a $\ox$-comonoid gives a linear comonoid. Hence, 
	the dual $(x,y): X \dashvv Y$ and the $\ox$-comonoid $(X, \tricomulb{0.65}, \tricounitb{0.65})$ 
	produces a linear comonoid $(x,y): X \lincomonbtritik Y$. Now, the dual $(a,b): A \dashvv B$ 
	induces a linear comonoid $(a_!, b_?): !A \lincomonwtritik ?B$ on the exponential modalities.
	This leads to the situation as illustrated in the following diagram:
	\[ \xymatrix{
		(x,y): X \lincomonbtritik Y \ar@{..>}[d]_{(f^\flat,g^\sharp)}  \ar[drr]^{(f,g)} \\
		(a_!,b_?):!A \lincomonwtritik ?B  \ar[rr]_{(\epsilon,\eta)} & & (a,b): A \dashvv~B }  \]
	Applying Lemma \ref{Lemma: free !? linear comonoid} to the above diagram, we have that $(f^\flat, g^\sharp)$ is a morphism of duals. 
	Note that, the linear comonoids on $X$ and $!A$ in the above diagram are different 
	from the linear comonoids in the $X \linbialgbtik Y$ and $!A \linbialgwtik ?B$ bialgebras. 
	
	It remains to prove that $f^\flat: (X, \mulmap{1.5}{black}, \unitmap{1.5}{black}) \to 
	(!A, \mulmap{1.5}{white}_! , \unitmap{1.5}{white}_!)$ is a morphism of monoids.
	This is provided by the Lemma \ref{Lemma: ! monoid hom}.
	\end{proof}

\begin{corollary} \label{Corr: splitting expmod}
	In a (!,?)-LDC with free exponential modalities, 
	if $A \linbialgbtik B$ is a linear bialgebra then $(1^\flat,1^\sharp): (A \linbialgbtik B) \to (!A\expbialgbtik ?B)$ 
	is a morphism of bialgebras, making $A \linbialgbtik B$ a retract of $!A\expbialgbtik ?B$ as illustrated in the 
	diagram below:
	\[ \xymatrixcolsep{6mm} \xymatrixrowsep{8mm} \xymatrix{
		\frac{(a,b)}{(a',b')}: A \linbialgbtik B \ar@{..>}[d]_{(1^\flat,1^\sharp)}  \ar@{=}[drr] \\
		\frac{(a_!,b_?)}{(a'_!, b'_?)}: !A \expbialgbtik ?B  \ar[rr]_{(\epsilon,\eta)~~~} & & \frac{(a,b)}{(a',b')}: A \linbialgbtik B } \]
\end{corollary}

 The corollary shows that every self-linear bialgebra in an (!,?)-LDC, with free exponential modalities, induces a 
 sectional binary idempotent for the linear bialgebra induced on the exponential modalities:
 \[ \xymatrix{  !A \ar@<0.5ex>[r]^{\epsilon~~}  
	 & A \simeq B \ar@<0.5ex>[r]^{~~~\eta}  \ar@<0.5ex>[l]^{1^\flat~~}
	 & ?B \ar@<0.5ex>[l]^{~~~1^\sharp} } \]
Moreover, the idempotent splits and it is coring whenever the self-linear bialgebra resides in the core. 

Combining Corollary \ref{Corr: splitting expmod}, and Lemma \ref{Lemma: complementary idempotent}, we get: 
\begin{theorem}
	\label{Theorem: Main}
In an $(!,?)$-isomix category with free exponential modalities, every complementary system
arises as a  splitting of a sectional binary idempotent on the free exponential modalities. 
\end{theorem}

The above results extend directly to $\dagger$-linear bilagebras in $\dagger$-LDCs with free exponential modalities due to the 
$\dagger$-linearity of $(!,?)$, $(\eta, \epsilon)$, $(\Delta, \nabla)$, and $(\tricounit{0.65}, \triunit{0.65})$. 

Theorem \ref{Theorem: Main} is a new observation in quantum theory: gives a non-functorial method 
to retreive the `first quantization' from  the `second quantization'. 

%\chapter{Examples of duals, linear monoids, and comonoids} 

\chapter{Examples}
\label{Chap: free exp examples}

In this chapter, we present examples of linear monoids and comonoids, and exponential modalities for LDCs. 

\section{Duals and linear monoids}
\label{Sec: lin mon examples}

This section is dedicated to discussing  examples of  ($\dagger$-)duals, and ($\dagger$-)linear monoids. 

\subsection{Duals in $\FRel$ and $\FMat(R)$}
\label{Sec: duals examples}

In this section, we will examine the duals in $\FRel$, $\FMat(R)$, $\Rel$ and $\Mat(R)$, where $R$ is a commutative rig. 
We show that in  $\FRel$, $\FMat(R)$ every dual is also a $\dagger$-dual.
We will also prove that in $\Rel$ and $\Mat(R)$, $\dagger$-duals coincide with unitary duals. 
Before proceeding with the proofs, we quickly recap the dual, dagger, tensor and the par products 
in $\FRel$ and $\FMat(R)$.  

In $\FRel$ and $\FMat(R)$, the dual of a finiteness space, $(X, F(X), F(X)^\perp)^*$ is given by $(X, F(X)^\perp, F(X))$. 
For any finiteness relation, $R$, the dual relation $R^*$ is the converse of $R$. For any finiteness matrix, $M$, 
the dual matrix, $M^*$, is given by transposing $M$.

Both $\FRel$ and $\FMat(R)$ are conjugative $*$-isomix categories, see \ref{Lemma: conjugative cat}. The 
conjugation functor, $\overline{(-)}$ along with the $*$ gives the $\dagger$ functor: 
$(-)^\dagger = \overline{(-)^*} = \overline{(-)}^*$. In $\FRel$, the conjugation functor is the identity functor, hence, 
the $\dagger$ coincides with the $*$ functor. In $\FMat(R)$, the conjugation functor is given by the 
conjugation for the rig $R$. Hence, the $\dagger$ of a finiteness matrix is given by its conjugate transpose. 

The category ${\sf FRel}$ has a symmetric tensor product which is used to obtain a 
corresponding tensor product on ${\sf FMat}(R)$.   The tensor product for ${\sf FRel}$ is defined as follows: 
$X \ox Y = (X, F(X)) \ox (Y, F(Y)) := (X  \times Y, F(X \ox Y), F(X \ox Y)^\perp)$, where
\[ F(X \ox Y) := \downarrow \{ A \x B \mid  A \in F(X), B \in F(Y) \} \]
where $\downarrow {\cal A}:= \{ A' \mid A' \subseteq A, A \in {\cal A} \}$ is the downward closure of the set of subsets ${\cal A}$.  
The par is defined in the standard way:
\[ X \oa Y = (X^* \ox Y^*)^* \]

The units of tensor and par are the same: $\top = \bot = ( \{* \}, \{ \varnothing, \{*\} \} )$. 
Both $\FRel$ and $\FMat(R)$ are symmetric $\dagger$-$*$-isomix categories.

\begin{lemma}
In $\FRel$ and ${\sf FMat(R)}$, every dual is also a $\dagger$-dual.
\end{lemma}
\begin{proof}
We first prove the statement in $\FRel$.
Let $(X,F(X), F(X)^\perp)$ be any finiteness space in $\FRel$. 
Consider its dual, $(\eta, \epsilon): X \dashvv X^*$.
The relations $\eta:  \top \to X \oa X$, and the counit 
$\epsilon:  X \ox X \to \bot$ are same as for the duals in $\Rel$:
\[ \eta := \{ (*, (x, x)) | x \in X \} ~~~~~~~~ \epsilon := \{ ((x,x), *) | x \in X \} \]

We prove that $\eta$, and $\epsilon$ are indeed finiteness relations. 
To prove that $\eta$ is a finiteness relation, using Lemma 
\ref{character_finiteness_relation}-(iii), it suffices to prove the following:
\begin{enumerate}[(i)]
\item for all $A \in F(\top) = \{ \varnothing, \{*\}  \}$, $A \tr \eta \in F(X \oa X^\perp)$,
\item for all $b \in  X \oa X^\perp = X \times X$, $\eta \tl {b} \in  F(\top)^\perp 
= \{ \varnothing, \{*\}  \}$.
\end{enumerate}

$(ii)$ is immediate. We prove that $(i)$ holds: if $A = \emptyset$, then 
$A \tr \eta = \emptyset \in F(X \oa X^\perp)$; if $A = \{* \}$ then:
\[ A \tr \eta =  \{ (x, x)  \mid x \in X \} \]
We have to show that $A \tr \eta \in  F(X \oa X^\perp)$ However, 
$F(X \oa X^\perp) = F( X^\perp \ox X^{\perp \perp})^\perp = 
F(X^\perp \ox X)^\perp$. 
By Lemma \ref{Lemma: tensor perp characterization}, $ A\eta \in F(X^\perp \ox X)^\perp$ 
if and only if the following conditions hold:
\begin{enumerate}[(a)]
\item for all $P \in F(X)^\perp$, $P \tr S \in F(X)^\perp$
\item for all $Q \in F(X)$, $S \tl Q \in F(X)^{\perp \perp} = F(X)$
\end{enumerate}
where $S = A\eta$. However, it is clear that $P \tr S = P$, and $S \tl Q = Q$, 
for all $P \in F(X)^\perp$, $Q \in F(X)$. Thus, $S = A \tr \eta \in F(X \oa X^\perp)$. 
Thereby, $\eta$ satisfies condition $(i)$, and hence is a finiteness relation. 
Since $\epsilon$ is simply the converse of $\eta$, it is also a finiteness relation.

Finally, to prove that, in $\FRel$, every dual is a $\dagger$-dual, we must show that the following equation holds:
\[ \begin{tikzpicture}
	\begin{pgfonlayer}{nodelayer}
		\node [style=none] (0) at (-1, 1.75) {};
		\node [style=none] (1) at (0, 1.75) {};
		\node [style=none] (2) at (-1, 3) {};
		\node [style=none] (3) at (0, 3) {};
		\node [style=none] (4) at (-0.5, 3.75) {$\eta$};
		\node [style=none] (5) at (-1.75, 2) {$X^{\dag \dag}$};
		\node [style=none] (6) at (0.5, 2) {$X^\dag$};
		\node [style=onehalfcircle] (7) at (-1, 2.5) {};
		\node [style=none] (8) at (-1, 2.5) {$\iota$};
	\end{pgfonlayer}
	\begin{pgfonlayer}{edgelayer}
		\draw [bend left=90, looseness=1.75] (2.center) to (3.center);
		\draw (3.center) to (1.center);
		\draw (2.center) to (7);
		\draw (7) to (0.center);
	\end{pgfonlayer}
\end{tikzpicture} = \begin{tikzpicture}
	\begin{pgfonlayer}{nodelayer}
		\node [style=none] (0) at (-1, 2) {};
		\node [style=none] (1) at (0, 2) {};
		\node [style=none] (2) at (-1, 3.75) {};
		\node [style=none] (3) at (0, 3.75) {};
		\node [style=none] (4) at (-0.5, 3) {$\epsilon$};
		\node [style=none] (5) at (-1.5, 2) {$X^{\dag \dag}$};
		\node [style=none] (6) at (0.5, 2) {$X^\dag$};
		\node [style=none] (7) at (-1.5, 3.75) {};
		\node [style=none] (8) at (0.5, 3.75) {};
		\node [style=none] (9) at (0.5, 2.5) {};
		\node [style=none] (10) at (-1.5, 2.5) {};
		\node [style=none] (11) at (-1, 2.5) {};
		\node [style=none] (12) at (0, 2.5) {};
	\end{pgfonlayer}
	\begin{pgfonlayer}{edgelayer}
		\draw [bend right=90, looseness=1.75] (2.center) to (3.center);
		\draw (10.center) to (9.center);
		\draw (9.center) to (8.center);
		\draw (8.center) to (7.center);
		\draw (7.center) to (10.center);
		\draw (11.center) to (0.center);
		\draw (12.center) to (1.center);
	\end{pgfonlayer}
\end{tikzpicture}  \] 
However, the above equation is automatic because, in 
$\FRel$, for any finiteness space $X$, we have that $X^\dag = X^*$ and $X = X^{\dag \dag}$, 
Moreover, for any finiteness relation, $R$, the finiteness relation $R^{\dag}$ is given by the converse 
of $R$, hence $\eta = \epsilon^{\dag}$ as required.  

The same proof extends to proving that in $\FMat(R)$ every dual is a $\dagger$-dual, 
since the support of every finiteness matrix is a finiteness relation. 
\end{proof}

Observe that when $X$ is a finite set, the finiteness space is $(X, \mathcal{P}(X))$, 
which is a self-dual object. Such spaces lie in the core of ${\sf FMat(R)}$, 
and gives the subcategory ${\sf Mat}(R)$ which is a $\dagger$-compact closed category. 
Thus, every dual in ${\sf Mat}(R)$ is also $\dagger$ dual. This leads one to wonder if, 
in ${\sf Mat}(R)$, the $\dagger$-duals coincide with the conventional dagger 
duals ($\eta^\dagger = c_\ox \epsilon$).

To answer this question, we first determine the conditions under 
which a $\dagger$-dual is a unitary dual (see \ref{defn: unitary dual}) in a unitary category. 
Recall that unitary categories are  lax $\dagger$ monoidal categories 
($A \simeq A^\dagger$), and $\dagger$-monoidal categories are strict 
unitary categories $(A = A^\dagger)$. When the unitary category is a 
$\dagger$-monoidal category, the unitary duals are precisely the conventional dagger duals.  

\begin{lemma}
\label{Lemma: dag dual is unitary dual}
Let $(\eta, \epsilon): A \dagdual A^\dagger$ be  a right $\dagger$-dual in a unitary category, 
then $(\eta, \epsilon): A \dashvv A^\dagger$ is a unitary dual if one of the following conditions hold:
% Further in a dagger CC category this reduces (assuming p and q are identity) to the requirement that \eta be symmetric.  Is this true in Mat(\C)?  Seems to be so!
\begin{align}
\label{eqn: dag unitary dual}
(a)~~~ \begin{tikzpicture}
	\begin{pgfonlayer}{nodelayer}
		\node [style=circle, scale=1.5] (13) at (3, 2) {};
		\node [style=none] (14) at (3, 2) {$i$};
		\node [style=none] (15) at (3, 1.25) {};
		\node [style=none] (16) at (4.5, 1.25) {};
		\node [style=none] (17) at (2.5, 2.5) {$A$};
		\node [style=none] (18) at (4.75, 1.5) {$A^\dagger$};
		\node [style=none] (19) at (3, 2.75) {};
		\node [style=none] (20) at (4.5, 2.75) {};
		\node [style=none] (21) at (2.5, 1.5) {$A^{\dagger \dagger}$};
		\node [style=none] (22) at (3.75, 3.75) {$\eta$};
	\end{pgfonlayer}
	\begin{pgfonlayer}{edgelayer}
		\draw [bend left=90, looseness=1.50] (19.center) to (20.center);
		\draw (19.center) to (13);
		\draw (15.center) to (13);
		\draw (16.center) to (20.center);
	\end{pgfonlayer}
\end{tikzpicture}  =  \begin{tikzpicture}
	\begin{pgfonlayer}{nodelayer}
		\node [style=none] (19) at (2.75, 1.5) {};
		\node [style=none] (20) at (3.75, 1.5) {};
		\node [style=none] (23) at (2.5, 1.75) {$A^{\dag \dagger}$};
		\node [style=none] (24) at (4, 1.75) {$A^\dagger$};
		\node [style=onehalfcircle] (25) at (2.5, 2.75) {};
		\node [style=onehalfcircle] (26) at (4, 2.75) {};
		\node [style=none] (27) at (3.25, 4) {$\eta$};
		\node [style=none] (28) at (2.5, 2.75) {$\varphi$};
		\node [style=none] (29) at (4, 2.75) {$\varphi$};
	\end{pgfonlayer}
	\begin{pgfonlayer}{edgelayer}
		\draw [bend left=90, looseness=1.50] (25) to (26);
		\draw [in=90, out=-90, looseness=0.75] (26) to (19.center);
		\draw [in=90, out=-90, looseness=0.75] (25) to (20.center);
	\end{pgfonlayer}
\end{tikzpicture}  ~~~~~ (or) ~~~~~~ (b)~~~ \begin{tikzpicture}
	\begin{pgfonlayer}{nodelayer}
		\node [style=circle, scale=2] (13) at (4.25, 2) {};
		\node [style=none] (14) at (4.25, 2) {$i^{-1}$};
		\node [style=none] (15) at (4.25, 2.75) {};
		\node [style=none] (16) at (2.75, 2.75) {};
		\node [style=none] (17) at (4.75, 1.5) {$A$};
		\node [style=none] (18) at (2.25, 2.5) {$A^\dagger$};
		\node [style=none] (19) at (4.25, 1.25) {};
		\node [style=none] (20) at (2.75, 1.25) {};
		\node [style=none] (21) at (5, 2.5) {$A^{\dagger \dagger}$};
		\node [style=none] (22) at (3.5, 0.5) {$\epsilon$};
	\end{pgfonlayer}
	\begin{pgfonlayer}{edgelayer}
		\draw [bend left=90, looseness=1.25] (19.center) to (20.center);
		\draw (19.center) to (13);
		\draw (15.center) to (13);
		\draw (16.center) to (20.center);
	\end{pgfonlayer}
\end{tikzpicture} = \begin{tikzpicture}
	\begin{pgfonlayer}{nodelayer}
		\node [style=none] (19) at (3.75, 3.75) {};
		\node [style=none] (20) at (2.75, 3.75) {};
		\node [style=none] (23) at (4, 3.5) {$A^{\dag\dagger}$};
		\node [style=none] (24) at (2.5, 3.5) {$A^\dagger$};
		\node [style=twocircle] (25) at (4, 2.5) {};
		\node [style=twocircle] (26) at (2.5, 2.5) {};
		\node [style=none, scale=1.5] (27) at (3.25, 1.25) {$\epsilon$};
		\node [style=none] (28) at (4, 2.5) {$\varphi^{-1}$};
		\node [style=none] (29) at (2.5, 2.5) {$\varphi^{-1}$};
	\end{pgfonlayer}
	\begin{pgfonlayer}{edgelayer}
		\draw [bend left=90, looseness=1.50] (25) to (26);
		\draw [in=-90, out=90, looseness=0.75] (26) to (19.center);
		\draw [in=-90, out=90, looseness=0.75] (25) to (20.center);
	\end{pgfonlayer}
\end{tikzpicture}
\end{align}
\end{lemma}

Let us now examine in ${\sf Mat}(R)$ if $\dagger$-duals coincide with unitary duals. 
${\sf Mat}(R)$ is a $\dagger$-monoidal category in which every dual is also a unitary dual. Hence the 
category is $\dagger$-compact closed. 

\begin{lemma}
\label{Lemma: every dual dag dual}
In ${\sf Mat}(R)$ and {\sf Rel}, every object is its own right and left $\dagger$-dual.  Hence, every dual is a 
$\dagger$-dual.
\end{lemma}
\begin{proof}
Let us first consider ${\sf Mat}(R)$. Recall from Section \ref{Sec: Mat(R)} that in ${\sf Mat}(R)$, every object $n$ is a self-dual. Moreover, the involution, 
$\iota: n \to n^{\dagger \dagger}$ is $I_n$, the identity map. 
Given $(\eta, \epsilon) : n \dashvv n$ we have that $\eta^\dagger = \epsilon$: 
\[
	\eta^\dagger := \left(\overline{ \sum_{i=1}^n (e_i \ox e_i) } \right)^T = \sum_{i=1}^n 
	\left(\overline{(e_i \ox e_i) } \right)^T  = \sum_{i=1}^n ( \overline{e_i}^T \ox \overline{e_i}^T) = 
	\sum_{i=1}^n (e_i^T \ox e_i^T) =: \epsilon 
\] 
Thereby, every dual in $\FMat(R)$ is also a $\dagger$-dual.  
It is also useful to notice that $\Mat(R) \simeq \Core(\FMat(R))$. 
Since the statement is true in $\FMat(R)$ it must be 
true for $\Mat(R)$. 

In {\sf Rel} too, every object is a self-dual, and the involution $\iota = 1$. Observe that in {\sf Rel} also, 
for any $(\eta, \epsilon): A \dashvv A$, 
\[ \eta  := \{ ( *, (x,x)) \mid x \in A \} =   \{ ( \*, (x,x)) | ~ x \in A \}^{\dagger \dagger} = \{ ( (x,x), \{ * \}) \mid x 
\in A \}^\dagger =: \epsilon^\dagger \]
\end{proof}

Next we show that in $\Mat(R)$ and $\Rel$, the dagger duals conicide with the unitary duals:

\begin{lemma}
In ${\sf Mat}(R)$ and ${\sf Rel}$, every $\dagger$-dual is also a unitary dual.
\end{lemma}
\begin{proof}
By Lemma \ref{Lemma: every dual dag dual}, every dual in ${\sf Mat}(\C)$ and ${\sf Rel}$ are $\dagger$-duals.   
Now, observe that, in ${\sf Mat}(\C)$, for all $n \in N$, 
\[
    \begin{tikzpicture}
        \begin{pgfonlayer}{nodelayer}
            \node [style=none] (0) at (-1.25, 1.25) {};
            \node [style=none] (1) at (0.25, 1.25) {};
            \node [style=none] (2) at (-1.25, 1.25) {};
            \node [style=none] (3) at (0.25, 1.25) {};
            \node [style=none] (4) at (-1.25, 2.75) {};
            \node [style=none] (5) at (0.25, 2.75) {};
            \node [style=none] (7) at (-0.5, 3.75) {$\eta_n$};
            \node [style=none] (8) at (-1.5, 1.5) {$n$};
            \node [style=none] (9) at (0.5, 1.5) {$n$};
        \end{pgfonlayer}
        \begin{pgfonlayer}{edgelayer}
            \draw (0.center) to (4.center);
            \draw (5.center) to (1.center);
            \draw [bend left=90, looseness=1.50] (4.center) to (5.center);
        \end{pgfonlayer}
    \end{tikzpicture} = \sum_{i=1}^n (e_i \ox  e_i) = \begin{tikzpicture}
        \begin{pgfonlayer}{nodelayer}
            \node [style=none] (0) at (0.25, 1.25) {};
            \node [style=none] (1) at (-1.25, 1.25) {};
            \node [style=none] (2) at (-1.25, 3) {};
            \node [style=none] (3) at (0.25, 3) {};
            \node [style=none] (5) at (-0.5, 4) {$\eta_n$};
            \node [style=none] (6) at (-1.5, 1.5) {$n$};
            \node [style=none] (7) at (0.5, 1.5) {$n$};
        \end{pgfonlayer}
        \begin{pgfonlayer}{edgelayer}
            \draw [in=-90, out=90] (0.center) to (2.center);
            \draw [in=90, out=-90] (3.center) to (1.center);
            \draw [bend left=90, looseness=1.50] (2.center) to (3.center);
        \end{pgfonlayer}
    \end{tikzpicture} 
\]
and, $c_\ox \epsilon_n = \epsilon_n$. Hence, by Lemma \ref{Lemma: dag dual is unitary dual} in ${\sf Mat(R)}$, 
every $\dagger$-dual is also a unitary dual. 

In {\sf Rel} too, for all $(\eta, \epsilon): \{ * \} \to A \times A$, $\eta = \{ ( * , (x,x)) | ~ x \in A \} = \eta c_\ox$, 
thereby $\epsilon = c_\ox \epsilon$. Hence, every $\dagger$-dual is a unitary dual in ${\sf Rel}$.
\end{proof}

%%%%%%%%%%%%%%%%%%%%%%%%%%%%%%%%%%%%%%%%%%%%%%%%%%%%%%%%%
\iffalse
\subsection{Limits of linear monoids}

% limit of right dagger linear dual
% limit of right dagger linear monoid

In this section, we show that the limit of a diagram of $\dagger$-linear monoids is itself a $\dagger$-linear monoid.   
% TODO: A little intuitive note on limits please

We first show that, in a $\dagger$-LDC, the limit of $\dagger$-duals are $\dagger$-duals:

\begin{lemma}
\label{Lemma: L is a dagger dual}
Let $\X$ be a $\dagger$-LDC. ${\sf DagDual}(\X)$ is the category of $\dagger$-linear duals from $X$, 
and  morphisms of duals. Let $D: \D \to {\sf DagDual}(\X)$ be any diagram. Then, $L := \Lim{A \in \D} U(D(A))$, 
where $U: {\sf DagDual}(\X) \to \X$ is the underlying functor.
\end{lemma}
\begin{proof}
Given that $D : \D \to {\sf DagDual}(\X)$ is a diagram. Show that $L := \Lim{A \in \D} U(D(A))$ is a $\dagger$-dual.
i.e, $(\eta, \epsilon): L \dagdual L^\dagger$.

Note that since the $\dagger$ functor is an equivalence, it commutes with limits and colimits:  
$L^\dagger :=  (\Lim{A \in \D} U(D(A))^\dagger \simeq  \Colim{A \in \D} U(D(A))^\dagger$.
Moreover, in an LDC  the product preserves limits (in each argument) while, dually, 
tensor preserves colimits in each argument.  This is automatic in a $*$-autonomous category since 
$\_\oa X$ is right adjoint to $\_ \ox X^*$. 

\begin{description}
\item[Defining $\eta: \top \to L \oa L^\dagger$:]
\[ \eta: \top \to L \oa L^\dagger = \top \to^{\eta'}  \underset{A \in \D}{\sf Lim} \left(U(D(A)) \oa L^\dagger  
\right) \to^{\alpha}_{\simeq} L \oa L^\dagger \]

Now $\eta': \top \to \Lim{A \in \D} \left(U(D(A)) \oa L^\dagger  \right)$ is is given by the $\eta$ map of each $\dagger$-dual:
\[ \eta' := \left < \top \to^{\eta_A} U(D(A)) \oa U(D(A))^\dagger \to^{1 \oa \pi_A^\dagger } U(D(A))  
\oa L^\dagger \right >_{A \in \D} \]

\item[Defining $\epsilon: L^\dagger \ox L \to \bot$:]

Because $\ox$ preserves colimits in each argument, 
\[ \Colim{B \in \D} U(D(B))^\dagger \ox L \to^{\beta}_{\simeq} \Colim{B \in \D} \left( U(D(B))^\dagger \ox L \right) \to^{\epsilon'}  \bot \]

Now, $\epsilon':  \Colim{B \in \D} \left( U(D(B))^\dagger \ox L \right) \to \bot$ is given by the individual cup maps:
\[
\epsilon' := \left[ (U(D(B))^\dagger \ox L \to^{1 \ox \pi_B } (U(D(B))^\dagger \ox (U(D(B)) \to^{\epsilon_B}  \bot \right]_{B \in \D}
\]
\end{description}

We next prove that the snake diagrams hold which is easier to show using string diagrams.  
The maps $\eta: \top \to L \oa L^\dagger$, and $\epsilon: L^\dagger \ox L \to \bot$, 
are represented diagrammatically as follows:
\[ \eta := \eta' \alpha = \begin{tikzpicture}
	\begin{pgfonlayer}{nodelayer}
		\node [style=circle, scale=2] (0) at (2, 2.5) {};
		\node [style=none] (1) at (2, 2.5) {$\eta_A$};
		\node [style=none] (2) at (3.75, 1.25) {};
		\node [style=none] (3) at (2.25, 1.25) {};
		\node [style=none] (4) at (3.75, -0.25) {};
		\node [style=none] (5) at (2.25, -0.25) {};
		\node [style=circle, scale=2] (6) at (3, 0.5) {};
		\node [style=none] (7) at (3, 0.5) {$\pi_A$};
		\node [style=none] (8) at (2.75, 1.25) {};
		\node [style=none] (9) at (3.25, -0.25) {};
		\node [style=none] (10) at (3, 1.75) {};
		\node [style=none] (11) at (3, -0.25) {};
		\node [style=none] (12) at (3, -2) {};
		\node [style=none] (13) at (1, 1.75) {};
		\node [style=none] (14) at (3, 1.25) {};
		\node [style=none] (15) at (1, -2) {};
		\node [style=none] (16) at (-0.25, -1) {};
		\node [style=none] (17) at (4.25, -1) {};
		\node [style=none] (18) at (4.25, 3) {};
		\node [style=none] (19) at (3.5, 3) {};
		\node [style=none] (20) at (1.5, 3) {};
		\node [style=none] (21) at (-0.25, 3) {};
		\node [style=circle, fill=black] (22) at (1, -1) {};
		\node [style=none] (23) at (-0.5, -1.25) {$A \in \D$};
		\node [style=none] (24) at (3.5, 2.5) {$(DUA)^\dagger$};
		\node [style=none] (25) at (0.4, -0.25) {$DUA$};
		\node [style=none] (26) at (0.75, -2) {$L$};
		\node [style=none] (27) at (3.25, -2) {$L^\dagger$};
		\node [style=none] (28) at (3.25, -0.75) {$L^\dagger$};
	\end{pgfonlayer}
	\begin{pgfonlayer}{edgelayer}
		\draw (8.center) to (6);
		\draw (9.center) to (6);
		\draw (4.center) to (5.center);
		\draw (5.center) to (3.center);
		\draw (3.center) to (2.center);
		\draw (2.center) to (4.center);
		\draw (12.center) to (11.center);
		\draw [in=-15, out=90, looseness=1.25] (10.center) to (0);
		\draw [in=90, out=-165, looseness=1.00] (0) to (13.center);
		\draw (10.center) to (14.center);
		\draw (13.center) to (15.center);
		\draw (20.center) to (21.center);
		\draw (21.center) to (16.center);
		\draw (16.center) to (17.center);
		\draw (17.center) to (18.center);
		\draw (18.center) to (19.center);
	\end{pgfonlayer}
\end{tikzpicture} ~~~~~~~~~~~ \epsilon := \beta \epsilon' = \begin{tikzpicture}
	\begin{pgfonlayer}{nodelayer}
		\node [style=circle, scale=2] (0) at (2, -1.5) {};
		\node [style=none] (1) at (2, -1.5) {$\epsilon_B$};
		\node [style=circle, scale=2] (2) at (3, 0.5) {};
		\node [style=none] (3) at (3, 0.5) {$\pi_B$};
		\node [style=none] (4) at (3, -0.25) {};
		\node [style=none] (5) at (3, 1.25) {};
		\node [style=none] (6) at (3, -0.75) {};
		\node [style=none] (7) at (3, 1.25) {};
		\node [style=none] (8) at (3, 3) {};
		\node [style=none] (9) at (1, -0.75) {};
		\node [style=none] (10) at (3, -0.25) {};
		\node [style=none] (11) at (1, 3) {};
		\node [style=none] (12) at (4, 2) {};
		\node [style=none] (13) at (-0.5, 2) {};
		\node [style=none] (14) at (-0.5, -2) {};
		\node [style=none] (15) at (1, -2) {};
		\node [style=none] (16) at (2.5, -2) {};
		\node [style=none] (17) at (4, -2) {};
		\node [style=none] (18) at (-0.75, 2.25) {$B \in \D$};
		\node [style=none] (19) at (3.5, -1) {$DUB$};
		\node [style=none] (20) at (3.25, 2.75) {$L$};
		\node [style=none] (21) at (0.75, 2.75) {$L^\dagger$};
		\node [style=circle, fill=black] (22) at (1, 2) {};
		\node [style=none] (23) at (0.25, -1) {$(DUB)^\dagger$};
	\end{pgfonlayer}
	\begin{pgfonlayer}{edgelayer}
		\draw (4.center) to (2);
		\draw (5.center) to (2);
		\draw (8.center) to (7.center);
		\draw [in=15, out=-90, looseness=1.25] (6.center) to (0);
		\draw [in=-90, out=165, looseness=1.00] (0) to (9.center);
		\draw (6.center) to (10.center);
		\draw (9.center) to (11.center);
		\draw (16.center) to (17.center);
		\draw (17.center) to (12.center);
		\draw (12.center) to (13.center);
		\draw (13.center) to (14.center);
		\draw (14.center) to (15.center);
	\end{pgfonlayer}
\end{tikzpicture}
\]  
The individual maps are surrounded by the limit box for $\eta$, and the colimit box for $\epsilon$. The filled circles represents the isomorphisms $\alpha$, and $\beta$, given by the fact that left adjoints preserves colimits, and the right ajoints preserve limits. 

Observe that applying a projection map to a limit box, and a  coprojection to a colmit box gives the corresponding individual map:
\[ \begin{tikzpicture}
	\begin{pgfonlayer}{nodelayer}
		\node [style=circle, scale=2] (0) at (3.75, 3.75) {};
		\node [style=none] (1) at (3.75, -0.25) {};
		\node [style=none] (2) at (2.75, 4.5) {};
		\node [style=none] (3) at (4.25, 4.5) {};
		\node [style=none] (4) at (3.25, 4.5) {};
		\node [style=none] (5) at (4.75, 4.5) {};
		\node [style=none] (6) at (4.75, 2) {};
		\node [style=none] (7) at (2.75, 2) {};
		\node [style=circle, fill=black] (8) at (3.75, 2) {};
		\node [style=none] (9) at (4, 1.25) {$L$};
		\node [style=none] (10) at (3.5, 3) {$A$};
		\node [style=none] (11) at (3.75, 3.75) {$f_A$};
		\node [style=none] (12) at (3.75, 5.5) {};
		\node [style=circle, scale=2] (13) at (3.75, 0.5) {};
		\node [style=none] (14) at (3.75, 0.5) {$\pi_A$};
		\node [style=none] (15) at (4, 5.25) {$K$};
		\node [style=none] (16) at (2.5, 1.75) {$A \in \D$};
	\end{pgfonlayer}
	\begin{pgfonlayer}{edgelayer}
		\draw (4.center) to (2.center);
		\draw (2.center) to (7.center);
		\draw (7.center) to (6.center);
		\draw (6.center) to (5.center);
		\draw (5.center) to (3.center);
		\draw (12.center) to (0);
		\draw (0) to (8);
		\draw (8) to (13);
		\draw (1.center) to (13);
	\end{pgfonlayer}
\end{tikzpicture} =  \begin{tikzpicture}
	\begin{pgfonlayer}{nodelayer}
		\node [style=circle, scale=2] (0) at (3.75, 2.75) {};
		\node [style=none] (1) at (3.75, -0.25) {};
		\node [style=none] (2) at (4, 0.25) {$A$};
		\node [style=none] (3) at (3.75, 2.75) {$f_A$};
		\node [style=none] (4) at (3.75, 5.5) {};
		\node [style=none] (5) at (4, 5.25) {$K$};
	\end{pgfonlayer}
	\begin{pgfonlayer}{edgelayer}
		\draw (4.center) to (0);
		\draw (1.center) to (0);
	\end{pgfonlayer}
\end{tikzpicture}  ~~~~~~~~~ \begin{tikzpicture}
	\begin{pgfonlayer}{nodelayer}
		\node [style=circle, scale=2] (0) at (3.75, 1.5) {};
		\node [style=none] (1) at (3.75, 5.5) {};
		\node [style=none] (2) at (2.75, 0.75) {};
		\node [style=none] (3) at (4.25, 0.75) {};
		\node [style=none] (4) at (3.25, 0.75) {};
		\node [style=none] (5) at (4.75, 0.75) {};
		\node [style=none] (6) at (4.75, 3.25) {};
		\node [style=none] (7) at (2.75, 3.25) {};
		\node [style=circle, fill=black] (8) at (3.75, 3.25) {};
		\node [style=none] (9) at (4, 4) {$L^\dagger$};
		\node [style=none] (10) at (3.25, 2.25) {$B^\dag$};
		\node [style=none] (11) at (3.75, 1.5) {$g_B$};
		\node [style=none] (12) at (3.75, -0.25) {};
		\node [style=circle, scale=2.5] (13) at (3.75, 4.75) {};
		\node [style=none] (14) at (3.75, 4.75) {$\coprod_B$};
		\node [style=none] (15) at (4, 0) {$K$};
		\node [style=none] (16) at (2.5, 3.5) {$B \in \D$};
		\node [style=none] (17) at (4, 5.5) {$B$};
	\end{pgfonlayer}
	\begin{pgfonlayer}{edgelayer}
		\draw (4.center) to (2.center);
		\draw (2.center) to (7.center);
		\draw (7.center) to (6.center);
		\draw (6.center) to (5.center);
		\draw (5.center) to (3.center);
		\draw (12.center) to (0);
		\draw (0) to (8);
		\draw (8) to (13);
		\draw (1.center) to (13);
	\end{pgfonlayer}
\end{tikzpicture} = \begin{tikzpicture}
	\begin{pgfonlayer}{nodelayer}
		\node [style=circle, scale=2] (0) at (3.75, 3) {};
		\node [style=none] (1) at (3.75, 5.5) {};
		\node [style=none] (2) at (3.75, 3) {$g_B$};
		\node [style=none] (3) at (3.75, -0.25) {};
		\node [style=none] (4) at (4, 0) {$K$};
		\node [style=none] (5) at (4, 5.5) {$B$};
	\end{pgfonlayer}
	\begin{pgfonlayer}{edgelayer}
		\draw (3.center) to (0);
		\draw (1.center) to (0);
	\end{pgfonlayer}
\end{tikzpicture}
\]
Now we turn to the proof of the snake diagrams:


$ \begin{tikzpicture}
	\begin{pgfonlayer}{nodelayer}
		\node [style=circle, scale=2] (0) at (2, 2.5) {};
		\node [style=none] (1) at (2, 2.5) {$\eta'$};
		\node [style=none] (2) at (3, 1.75) {};
		\node [style=none] (3) at (3, -0.25) {};
		\node [style=none] (4) at (3, -6.5) {};
		\node [style=none] (5) at (1, 1.75) {};
		\node [style=none] (6) at (3, -0.5) {};
		\node [style=none] (7) at (1.5, 3) {};
		\node [style=none] (8) at (3.25, -6.5) {$L^\dagger$};
		\node [style=circle, scale=2] (9) at (0, -5.75) {};
		\node [style=none] (10) at (1, 1.75) {};
		\node [style=none] (11) at (1, -4.5) {};
		\node [style=none] (12) at (1, -5) {};
		\node [style=none] (13) at (-1, -5) {};
		\node [style=none] (14) at (0, -5.75) {$\epsilon'$};
		\node [style=none] (15) at (-1.25, 3.75) {$L^\dagger$};
		\node [style=none] (16) at (-1, 4) {};
		\node [style=none] (17) at (1, -4.5) {};
	\end{pgfonlayer}
	\begin{pgfonlayer}{edgelayer}
		\draw (4.center) to (3.center);
		\draw [in=-15, out=90, looseness=1.25] (2.center) to (0);
		\draw [in=90, out=-165, looseness=1.00] (0) to (5.center);
		\draw (2.center) to (6.center);
		\draw [in=15, out=-90, looseness=1.25] (12.center) to (9);
		\draw [in=-90, out=165, looseness=1.00] (9) to (13.center);
		\draw (12.center) to (17.center);
		\draw (13.center) to (16.center);
		\draw (10.center) to (11.center);
	\end{pgfonlayer}
\end{tikzpicture}
 = \begin{tikzpicture}
	\begin{pgfonlayer}{nodelayer}
		\node [style=circle, scale=2] (0) at (2, 2.5) {};
		\node [style=none] (1) at (2, 2.5) {$\eta_A$};
		\node [style=none] (2) at (3.75, 1.25) {};
		\node [style=none] (3) at (2.25, 1.25) {};
		\node [style=none] (4) at (3.75, -0.25) {};
		\node [style=none] (5) at (2.25, -0.25) {};
		\node [style=circle, scale=2] (6) at (3, 0.5) {};
		\node [style=none] (7) at (3, 0.5) {$\pi_A$};
		\node [style=none] (8) at (2.75, 1.25) {};
		\node [style=none] (9) at (3.25, -0.25) {};
		\node [style=none] (10) at (3, 1.75) {};
		\node [style=none] (11) at (3, -0.25) {};
		\node [style=none] (12) at (3, -6.5) {};
		\node [style=none] (13) at (1, 1.75) {};
		\node [style=none] (14) at (3, 1.25) {};
		\node [style=none] (15) at (1, -1.5) {};
		\node [style=none] (16) at (-0.25, -1) {};
		\node [style=none] (17) at (4.25, -1) {};
		\node [style=none] (18) at (4.25, 3) {};
		\node [style=none] (19) at (3.5, 3) {};
		\node [style=none] (20) at (1.5, 3) {};
		\node [style=none] (21) at (-0.25, 3) {};
		\node [style=circle, fill=black] (22) at (1, -1) {};
		\node [style=none] (23) at (-0.5, -1.25) {$A \in \D$};
		\node [style=none] (24) at (3.5, 2.5) {$(DUA)^\dagger$};
		\node [style=none] (25) at (0.5, -0.25) {$DUA$};
		\node [style=none] (26) at (0.75, -1.75) {$L$};
		\node [style=none] (27) at (3.25, -6.5) {$L^\dagger$};
		\node [style=none] (28) at (3.25, -0.75) {$L^\dagger$};
		\node [style=circle, scale=2] (29) at (1, -3.75) {};
		\node [style=circle, scale=2] (30) at (0, -5.75) {};
		\node [style=none] (31) at (1, -4.5) {};
		\node [style=none] (32) at (-1.75, -5.25) {$(DUB)^\dagger$};
		\node [style=none] (33) at (-2.75, -2) {$B \in \D$};
		\node [style=none] (34) at (1, -3) {};
		\node [style=none] (35) at (-2.5, -2.25) {};
		\node [style=none] (36) at (1, -3.75) {$\pi_B$};
		\node [style=none] (37) at (0.5, -6.25) {};
		\node [style=none] (38) at (1, -5) {};
		\node [style=none] (39) at (1, -1.5) {};
		\node [style=none] (40) at (-1, -5) {};
		\node [style=none] (41) at (0, -5.75) {$\epsilon_B$};
		\node [style=circle, fill=black] (42) at (-1, -2.25) {};
		\node [style=none] (43) at (2, -6.25) {};
		\node [style=none] (44) at (-1.25, 3.75) {$L^\dagger$};
		\node [style=none] (45) at (-1, -6.25) {};
		\node [style=none] (46) at (-1, 4) {};
		\node [style=none] (47) at (1, -4.5) {};
		\node [style=none] (48) at (1.5, -5.25) {$DUB$};
		\node [style=none] (49) at (-2.5, -6.25) {};
		\node [style=none] (50) at (1, -3) {};
		\node [style=none] (51) at (2, -2.25) {};
	\end{pgfonlayer}
	\begin{pgfonlayer}{edgelayer}
		\draw (8.center) to (6);
		\draw (9.center) to (6);
		\draw (4.center) to (5.center);
		\draw (5.center) to (3.center);
		\draw (3.center) to (2.center);
		\draw (2.center) to (4.center);
		\draw (12.center) to (11.center);
		\draw [in=-15, out=90, looseness=1.25] (10.center) to (0);
		\draw [in=90, out=-165, looseness=1.00] (0) to (13.center);
		\draw (10.center) to (14.center);
		\draw (13.center) to (15.center);
		\draw (20.center) to (21.center);
		\draw (21.center) to (16.center);
		\draw (16.center) to (17.center);
		\draw (17.center) to (18.center);
		\draw (18.center) to (19.center);
		\draw (31.center) to (29);
		\draw (34.center) to (29);
		\draw (39.center) to (50.center);
		\draw [in=15, out=-90, looseness=1.25] (38.center) to (30);
		\draw [in=-90, out=165, looseness=1.00] (30) to (40.center);
		\draw (38.center) to (47.center);
		\draw (40.center) to (46.center);
		\draw (37.center) to (43.center);
		\draw (43.center) to (51.center);
		\draw (51.center) to (35.center);
		\draw (35.center) to (49.center);
		\draw (49.center) to (45.center);
	\end{pgfonlayer}
	\end{tikzpicture} = \begin{tikzpicture}
	\begin{pgfonlayer}{nodelayer}
		\node [style=circle, scale=2] (0) at (2, 2.5) {};
		\node [style=none] (1) at (2, 2.5) {$\eta_A$};
		\node [style=none] (2) at (3.75, 1.25) {};
		\node [style=none] (3) at (2.25, 1.25) {};
		\node [style=none] (4) at (3.75, -0.25) {};
		\node [style=none] (5) at (2.25, -0.25) {};
		\node [style=circle, scale=2] (6) at (3, 0.5) {};
		\node [style=none] (7) at (3, 0.5) {$\pi_A$};
		\node [style=none] (8) at (2.75, 1.25) {};
		\node [style=none] (9) at (3.25, -0.25) {};
		\node [style=none] (10) at (3, 1.75) {};
		\node [style=none] (11) at (3, -0.25) {};
		\node [style=none] (12) at (3, -6.5) {};
		\node [style=none] (13) at (1, 1.75) {};
		\node [style=none] (14) at (3, 1.25) {};
		\node [style=none] (15) at (1, -1.5) {};
		\node [style=none] (16) at (-0.25, -1) {};
		\node [style=none] (17) at (4.25, -1) {};
		\node [style=none] (18) at (4.25, 3) {};
		\node [style=none] (19) at (3.5, 3) {};
		\node [style=none] (20) at (1.5, 3) {};
		\node [style=none] (21) at (-0.25, 3) {};
		\node [style=circle, fill=black] (22) at (1, -1) {};
		\node [style=none] (23) at (-0.5, -1.25) {$A \in \D$};
		\node [style=none] (24) at (3.5, 2.5) {$(DUA)^\dagger$};
		\node [style=none] (25) at (0.5, -0.25) {$DUA$};
		\node [style=none] (26) at (0.75, -1.75) {$L$};
		\node [style=none] (27) at (3.25, -6.5) {$L^\dagger$};
		\node [style=none] (28) at (3.25, -0.75) {$L^\dagger$};
		\node [style=circle, scale=2] (29) at (1, -3) {};
		\node [style=circle, scale=2] (30) at (0, -5) {};
		\node [style=none] (31) at (1, -3.75) {};
		\node [style=none] (32) at (-1.75, -4.5) {$(DUB)^\dagger$};
		\node [style=none] (33) at (-2.75, 3.5) {$B \in \D$};
		\node [style=none] (34) at (1, -2.25) {};
		\node [style=none] (35) at (-2.5, 3.25) {};
		\node [style=none] (36) at (1, -3) {$\pi_B$};
		\node [style=none] (37) at (0.5, -5.5) {};
		\node [style=none] (38) at (1, -4.25) {};
		\node [style=none] (39) at (1, -1.5) {};
		\node [style=none] (40) at (-1, -4.25) {};
		\node [style=none] (41) at (0, -5) {$\epsilon_B$};
		\node [style=circle, fill=black] (42) at (-1, 3.25) {};
		\node [style=none] (43) at (4.75, -5.5) {};
		\node [style=none] (44) at (-1.25, 3.75) {$L^\dagger$};
		\node [style=none] (45) at (-1, -5.5) {};
		\node [style=none] (46) at (-1, 4) {};
		\node [style=none] (47) at (1, -3.75) {};
		\node [style=none] (48) at (1.5, -4.5) {$DUB$};
		\node [style=none] (49) at (-2.5, -5.5) {};
		\node [style=none] (50) at (1, -2.25) {};
		\node [style=none] (51) at (4.75, 3.25) {};
	\end{pgfonlayer}
	\begin{pgfonlayer}{edgelayer}
		\draw (8.center) to (6);
		\draw (9.center) to (6);
		\draw (4.center) to (5.center);
		\draw (5.center) to (3.center);
		\draw (3.center) to (2.center);
		\draw (2.center) to (4.center);
		\draw (12.center) to (11.center);
		\draw [in=-15, out=90, looseness=1.25] (10.center) to (0);
		\draw [in=90, out=-165, looseness=1.00] (0) to (13.center);
		\draw (10.center) to (14.center);
		\draw (13.center) to (15.center);
		\draw (20.center) to (21.center);
		\draw (21.center) to (16.center);
		\draw (16.center) to (17.center);
		\draw (17.center) to (18.center);
		\draw (18.center) to (19.center);
		\draw (31.center) to (29);
		\draw (34.center) to (29);
		\draw (39.center) to (50.center);
		\draw [in=15, out=-90, looseness=1.25] (38.center) to (30);
		\draw [in=-90, out=165, looseness=1.00] (30) to (40.center);
		\draw (38.center) to (47.center);
		\draw (40.center) to (46.center);
		\draw (37.center) to (43.center);
		\draw (43.center) to (51.center);
		\draw (51.center) to (35.center);
		\draw (35.center) to (49.center);
		\draw (49.center) to (45.center);
	\end{pgfonlayer}
\end{tikzpicture} = \begin{tikzpicture}
	\begin{pgfonlayer}{nodelayer}
		\node [style=circle, scale=2] (0) at (2, 2.5) {};
		\node [style=none] (1) at (2, 2.5) {$\eta_B$};
		\node [style=none] (2) at (3.75, 1.25) {};
		\node [style=none] (3) at (2.25, 1.25) {};
		\node [style=none] (4) at (3.75, -0.25) {};
		\node [style=none] (5) at (2.25, -0.25) {};
		\node [style=circle, scale=2] (6) at (3, 0.5) {};
		\node [style=none] (7) at (3, 0.5) {$\pi_A$};
		\node [style=none] (8) at (2.75, 1.25) {};
		\node [style=none] (9) at (3.25, -0.25) {};
		\node [style=none] (10) at (3, 1.75) {};
		\node [style=none] (11) at (3, -0.25) {};
		\node [style=none] (12) at (3, -6.5) {};
		\node [style=none] (13) at (1, 1.75) {};
		\node [style=none] (14) at (3, 1.25) {};
		\node [style=none] (15) at (1, -1.5) {};
		\node [style=none] (16) at (3.5, 2.5) {$(DUB)^\dagger$};
		\node [style=none] (17) at (0.75, -1.75) {$L$};
		\node [style=none] (18) at (3.25, -6.5) {$L^\dagger$};
		\node [style=none] (19) at (3.25, -0.75) {$L^\dagger$};
		\node [style=none] (20) at (1, -3) {};
		\node [style=circle, scale=2] (21) at (0, -5) {};
		\node [style=none] (22) at (1, -3.75) {};
		\node [style=none] (23) at (-1.75, -4.5) {$(DUB)^\dagger$};
		\node [style=none] (24) at (-2.75, 3.5) {$B \in \D$};
		\node [style=none] (25) at (1, -2.25) {};
		\node [style=none] (26) at (-2.5, 3.25) {};
		\node [style=none] (27) at (1, -3) {};
		\node [style=none] (28) at (0.5, -5.5) {};
		\node [style=none] (29) at (1, -4.25) {};
		\node [style=none] (30) at (1, -1.5) {};
		\node [style=none] (31) at (-1, -4.25) {};
		\node [style=none] (32) at (0, -5) {$\epsilon_B$};
		\node [style=circle, fill=black] (33) at (-1, 3.25) {};
		\node [style=none] (34) at (4.75, -5.5) {};
		\node [style=none] (35) at (-1.25, 3.75) {$L^\dagger$};
		\node [style=none] (36) at (-1, -5.5) {};
		\node [style=none] (37) at (-1, 4) {};
		\node [style=none] (38) at (1, -3.75) {};
		\node [style=none] (39) at (1.5, -4.5) {$DUB$};
		\node [style=none] (40) at (-2.5, -5.5) {};
		\node [style=none] (41) at (1, -2.25) {};
		\node [style=none] (42) at (4.75, 3.25) {};
	\end{pgfonlayer}
	\begin{pgfonlayer}{edgelayer}
		\draw (8.center) to (6);
		\draw (9.center) to (6);
		\draw (4.center) to (5.center);
		\draw (5.center) to (3.center);
		\draw (3.center) to (2.center);
		\draw (2.center) to (4.center);
		\draw (12.center) to (11.center);
		\draw [in=-15, out=90, looseness=1.25] (10.center) to (0);
		\draw [in=90, out=-165, looseness=1.00] (0) to (13.center);
		\draw (10.center) to (14.center);
		\draw (13.center) to (15.center);
		\draw (22.center) to (20);
		\draw (25.center) to (20);
		\draw (30.center) to (41.center);
		\draw [in=15, out=-90, looseness=1.25] (29.center) to (21);
		\draw [in=-90, out=165, looseness=1.00] (21) to (31.center);
		\draw (29.center) to (38.center);
		\draw (31.center) to (37.center);
		\draw (28.center) to (34.center);
		\draw (34.center) to (42.center);
		\draw (42.center) to (26.center);
		\draw (26.center) to (40.center);
		\draw (40.center) to (36.center);
	\end{pgfonlayer}
\end{tikzpicture} = \begin{tikzpicture}
	\begin{pgfonlayer}{nodelayer}
		\node [style=none] (0) at (3.75, 1.25) {};
		\node [style=none] (1) at (2.25, 1.25) {};
		\node [style=none] (2) at (3.75, -0.25) {};
		\node [style=none] (3) at (2.25, -0.25) {};
		\node [style=circle, scale=2] (4) at (3, 0.5) {};
		\node [style=none] (5) at (3, 0.5) {$\pi_B$};
		\node [style=none] (6) at (2.75, 1.25) {};
		\node [style=none] (7) at (3.25, -0.25) {};
		\node [style=none] (8) at (3, 4) {};
		\node [style=none] (9) at (3, -0.25) {};
		\node [style=none] (10) at (3, -6.5) {};
		\node [style=none] (11) at (3, 1.25) {};
		\node [style=none] (12) at (3.95, 2.25) {$(U(D(B))^\dagger$};
		\node [style=none] (13) at (3.25, -6.5) {$L^\dagger$};
		\node [style=none] (14) at (3.25, -0.75) {$L^\dagger$};
		\node [style=none] (15) at (1.25, 3.5) {$B \in \D$};
		\node [style=none] (16) at (1.5, 3.25) {};
		\node [style=none] (17) at (3.75, -5.5) {};
		\node [style=circle, fill=black] (18) at (3, 3.25) {};
		\node [style=none] (19) at (4.75, -5.5) {};
		\node [style=none] (20) at (2.75, 3.75) {$L^\dagger$};
		\node [style=none] (21) at (2.5, -5.5) {};
		\node [style=none] (22) at (3, 4) {};
		\node [style=none] (23) at (1.5, -5.5) {};
		\node [style=none] (24) at (4.75, 3.25) {};
	\end{pgfonlayer}
	\begin{pgfonlayer}{edgelayer}
		\draw (6.center) to (4);
		\draw (7.center) to (4);
		\draw (2.center) to (3.center);
		\draw (3.center) to (1.center);
		\draw (1.center) to (0.center);
		\draw (0.center) to (2.center);
		\draw (10.center) to (9.center);
		\draw (8.center) to (11.center);
		\draw (17.center) to (19.center);
		\draw (19.center) to (24.center);
		\draw (24.center) to (16.center);
		\draw (16.center) to (23.center);
		\draw (23.center) to (21.center);
	\end{pgfonlayer}
\end{tikzpicture} = \begin{tikzpicture}
	\begin{pgfonlayer}{nodelayer}
		\node [style=none] (0) at (3, 3.25) {};
		\node [style=none] (1) at (3, -6.5) {};
		\node [style=none] (2) at (3.25, -1.25) {$L^\dagger$};
		\node [style=none] (3) at (3, 3.25) {};
	\end{pgfonlayer}
	\begin{pgfonlayer}{edgelayer}
		\draw (1.center) to (0.center);
	\end{pgfonlayer}
\end{tikzpicture}
$
\vspace{0.8em}

The second step of the proof is valid since par preserve colimits in each argument. The next step applies the 
projection map to the limit box. Similarly, the final step applies the coprojection to the colimit box. 
The other snake diagram can be proven similarly. Thus, $(\eta, \epsilon): L \dashvv L^\dagger$ is a linear dual. 

$(\eta, \epsilon): L \dashvv L^\dagger$ is also right $\dagger$-linear dual:
\[
\begin{tikzpicture}
	\begin{pgfonlayer}{nodelayer}
		\node [style=circle, scale=2] (0) at (2, 2.5) {};
		\node [style=none] (1) at (2, 2.5) {$\eta_A$};
		\node [style=none] (2) at (3.75, 1.25) {};
		\node [style=none] (3) at (2.25, 1.25) {};
		\node [style=none] (4) at (3.75, -0.25) {};
		\node [style=none] (5) at (2.25, -0.25) {};
		\node [style=circle, scale=2] (6) at (3, 0.5) {};
		\node [style=none] (7) at (3, 0.5) {$\pi_A$};
		\node [style=none] (8) at (2.75, 1.25) {};
		\node [style=none] (9) at (3.25, -0.25) {};
		\node [style=none] (10) at (3, 1.75) {};
		\node [style=none] (11) at (3, -0.25) {};
		\node [style=none] (12) at (3, -3) {};
		\node [style=none] (13) at (1, 1.75) {};
		\node [style=none] (14) at (3, 1.25) {};
		\node [style=none] (15) at (-0.25, -1) {};
		\node [style=none] (16) at (4.25, -1) {};
		\node [style=none] (17) at (4.25, 3) {};
		\node [style=none] (18) at (3.5, 3) {};
		\node [style=none] (19) at (1.5, 3) {};
		\node [style=none] (20) at (-0.25, 3) {};
		\node [style=circle, fill=black] (21) at (1, -1) {};
		\node [style=none] (22) at (-0.5, -1.25) {$A \in \D$};
		\node [style=none] (23) at (3.5, 2.5) {$U(D(A))^\dagger$};
		\node [style=none] (24) at (0.5, -0.25) {$U(D(A))$};
		\node [style=none] (25) at (0.5, -1.5) {$L$};
		\node [style=none] (26) at (3.25, -3) {$L^\dagger$};
		\node [style=none] (27) at (3.25, -0.75) {$L^\dagger$};
		\node [style=circle, scale=2] (28) at (1, -2) {};
		\node [style=none] (29) at (1, -2) {$\iota$};
		\node [style=none] (30) at (1, -3) {};
		\node [style=none] (31) at (0.25, -3) {$L^{\dagger \dagger}$};
	\end{pgfonlayer}
	\begin{pgfonlayer}{edgelayer}
		\draw (8.center) to (6);
		\draw (9.center) to (6);
		\draw (4.center) to (5.center);
		\draw (5.center) to (3.center);
		\draw (3.center) to (2.center);
		\draw (2.center) to (4.center);
		\draw (12.center) to (11.center);
		\draw [in=-15, out=90, looseness=1.25] (10.center) to (0);
		\draw [in=90, out=-165, looseness=1.00] (0) to (13.center);
		\draw (10.center) to (14.center);
		\draw (19.center) to (20.center);
		\draw (20.center) to (15.center);
		\draw (15.center) to (16.center);
		\draw (16.center) to (17.center);
		\draw (17.center) to (18.center);
		\draw (30.center) to (28);
		\draw (13.center) to (28);
	\end{pgfonlayer}
\end{tikzpicture} = \begin{tikzpicture}
	\begin{pgfonlayer}{nodelayer}
		\node [style=circle, scale=2] (0) at (2, 2.5) {};
		\node [style=none] (1) at (2, 2.5) {$\eta_A$};
		\node [style=none] (2) at (3.75, 1.25) {};
		\node [style=none] (3) at (2.25, 1.25) {};
		\node [style=none] (4) at (3.75, -0.25) {};
		\node [style=none] (5) at (2.25, -0.25) {};
		\node [style=circle, scale=2] (6) at (3, 0.5) {};
		\node [style=none] (7) at (3, 0.5) {$\pi_A$};
		\node [style=none] (8) at (2.75, 1.25) {};
		\node [style=none] (9) at (3.25, -0.25) {};
		\node [style=none] (10) at (3, 1.75) {};
		\node [style=none] (11) at (3, -0.25) {};
		\node [style=none] (12) at (3, -3) {};
		\node [style=none] (13) at (1, 1.75) {};
		\node [style=none] (14) at (3, 1.25) {};
		\node [style=none] (15) at (-0.5, -1) {};
		\node [style=none] (16) at (4.25, -1) {};
		\node [style=none] (17) at (4.25, 3) {};
		\node [style=none] (18) at (3.5, 3) {};
		\node [style=none] (19) at (1.5, 3) {};
		\node [style=none] (20) at (-0.5, 3) {};
		\node [style=circle, fill=black] (21) at (1, -1) {};
		\node [style=none] (22) at (-0.75, -1.25) {$A \in \D$};
		\node [style=none] (23) at (3.5, 2.5) {$(DUA)^\dagger$};
		\node [style=none] (24) at (0.5, 1.25) {$DUA$};
		\node [style=none] (25) at (3.25, -3) {$L^\dagger$};
		\node [style=none] (26) at (3.25, -0.75) {$L^\dagger$};
		\node [style=circle, scale=2] (27) at (1, 0.5) {};
		\node [style=none] (28) at (1, 0.5) {$\iota$};
		\node [style=none] (29) at (1, -3) {};
		\node [style=none] (30) at (0.25, -3) {$L^{\dagger \dagger}$};
		\node [style=none] (31) at (0.25, -0.5) {$DUA^{\dagger \dagger}$};
	\end{pgfonlayer}
	\begin{pgfonlayer}{edgelayer}
		\draw (8.center) to (6);
		\draw (9.center) to (6);
		\draw (4.center) to (5.center);
		\draw (5.center) to (3.center);
		\draw (3.center) to (2.center);
		\draw (2.center) to (4.center);
		\draw (12.center) to (11.center);
		\draw [in=-15, out=90, looseness=1.25] (10.center) to (0);
		\draw [in=90, out=-165, looseness=1.00] (0) to (13.center);
		\draw (10.center) to (14.center);
		\draw (19.center) to (20.center);
		\draw (20.center) to (15.center);
		\draw (15.center) to (16.center);
		\draw (16.center) to (17.center);
		\draw (17.center) to (18.center);
		\draw (29.center) to (27);
		\draw (13.center) to (27);
	\end{pgfonlayer}
\end{tikzpicture} = \begin{tikzpicture}
	\begin{pgfonlayer}{nodelayer}
		\node [style=none] (0) at (3.75, -0) {};
		\node [style=none] (1) at (2.25, -0) {};
		\node [style=none] (2) at (3.75, -1.5) {};
		\node [style=none] (3) at (2.25, -1.5) {};
		\node [style=circle, scale=2] (4) at (3, -0.75) {};
		\node [style=none] (5) at (3, -0.75) {$\pi_A$};
		\node [style=none] (6) at (2.75, -0) {};
		\node [style=none] (7) at (3.25, -1.5) {};
		\node [style=none] (8) at (3, -1.5) {};
		\node [style=none] (9) at (3, -3) {};
		\node [style=none] (10) at (3, -0) {};
		\node [style=none] (11) at (0, -2.25) {};
		\node [style=none] (12) at (4.25, -2.25) {};
		\node [style=none] (13) at (4.25, 3) {};
		\node [style=none] (14) at (3.5, 3) {};
		\node [style=none] (15) at (1.5, 3) {};
		\node [style=none] (16) at (0, 3) {};
		\node [style=circle, fill=black] (17) at (1, -2.25) {};
		\node [style=none] (18) at (-0.25, -2.5) {$A \in \D$};
		\node [style=none] (19) at (3.25, -3) {$L^\dagger$};
		\node [style=none] (20) at (3.25, -2) {$L^\dagger$};
		\node [style=none] (21) at (1, -3) {};
		\node [style=none] (22) at (0.25, -3) {$L^{\dagger \dagger}$};
		\node [style=none] (23) at (3, 2.75) {};
		\node [style=none] (24) at (1, 2.75) {};
		\node [style=circle, scale=2] (25) at (2, 1.5) {};
		\node [style=none] (26) at (2, 1.5) {$\epsilon_A$};
		\node [style=none] (27) at (0.75, 2.75) {};
		\node [style=none] (28) at (0.75, 1) {};
		\node [style=none] (29) at (3.25, 1) {};
		\node [style=none] (30) at (3.25, 2.75) {};
		\node [style=none] (31) at (1, 1) {};
		\node [style=none] (32) at (3, 1) {};
	\end{pgfonlayer}
	\begin{pgfonlayer}{edgelayer}
		\draw (6.center) to (4);
		\draw (7.center) to (4);
		\draw (2.center) to (3.center);
		\draw (3.center) to (1.center);
		\draw (1.center) to (0.center);
		\draw (0.center) to (2.center);
		\draw (9.center) to (8.center);
		\draw (15.center) to (16.center);
		\draw (16.center) to (11.center);
		\draw (11.center) to (12.center);
		\draw (12.center) to (13.center);
		\draw (13.center) to (14.center);
		\draw [in=-90, out=165, looseness=1.00] (25) to (24.center);
		\draw [in=15, out=-90, looseness=1.25] (23.center) to (25);
		\draw (27.center) to (28.center);
		\draw (28.center) to (29.center);
		\draw (29.center) to (30.center);
		\draw (30.center) to (27.center);
		\draw (32.center) to (10.center);
		\draw (21.center) to (31.center);
	\end{pgfonlayer}
\end{tikzpicture} = \begin{tikzpicture}
	\begin{pgfonlayer}{nodelayer}
		\node [style=none] (0) at (3.75, 3) {};
		\node [style=none] (1) at (0.5, 3) {};
		\node [style=none] (2) at (3.75, -1.5) {};
		\node [style=none] (3) at (0.5, -1.5) {};
		\node [style=circle, scale=2] (4) at (3, 1.5) {};
		\node [style=none] (5) at (3, 1.5) {$\pi_A$};
		\node [style=none] (6) at (3, 3) {};
		\node [style=none] (7) at (3, 0.5) {};
		\node [style=none] (8) at (3, -1.5) {};
		\node [style=none] (9) at (3, -3) {};
		\node [style=none] (10) at (0, -2.25) {};
		\node [style=none] (11) at (4.25, -2.25) {};
		\node [style=none] (12) at (4.25, 3.5) {};
		\node [style=none] (13) at (3.5, 3.5) {};
		\node [style=none] (14) at (1.5, 3.5) {};
		\node [style=none] (15) at (0, 3.5) {};
		\node [style=circle, fill=black] (16) at (1, -2.25) {};
		\node [style=none] (17) at (-0.25, -2.5) {$A \in \D$};
		\node [style=none] (18) at (3.25, -3) {$L^\dagger$};
		\node [style=none] (19) at (3.25, -2) {$L^\dagger$};
		\node [style=none] (20) at (1, -3) {};
		\node [style=none] (21) at (0.25, -3) {$L^{\dagger \dagger}$};
		\node [style=none] (22) at (3, 0.5) {};
		\node [style=none] (23) at (1, 0.5) {};
		\node [style=circle, scale=2] (24) at (2, -0.75) {};
		\node [style=none] (25) at (2, -0.75) {$\epsilon_A$};
		\node [style=none] (26) at (1, -1.5) {};
		\node [style=none] (27) at (1, 3) {};
	\end{pgfonlayer}
	\begin{pgfonlayer}{edgelayer}
		\draw (6.center) to (4);
		\draw (7.center) to (4);
		\draw (2.center) to (3.center);
		\draw (3.center) to (1.center);
		\draw (1.center) to (0.center);
		\draw (0.center) to (2.center);
		\draw (9.center) to (8.center);
		\draw (14.center) to (15.center);
		\draw (15.center) to (10.center);
		\draw (10.center) to (11.center);
		\draw (11.center) to (12.center);
		\draw (12.center) to (13.center);
		\draw [in=-90, out=165, looseness=1.00] (24) to (23.center);
		\draw [in=15, out=-90, looseness=1.25] (22.center) to (24);
		\draw (20.center) to (26.center);
		\draw (23.center) to (27.center);
	\end{pgfonlayer}
\end{tikzpicture} = \begin{tikzpicture}
	\begin{pgfonlayer}{nodelayer}
		\node [style=none] (0) at (4.75, 4) {};
		\node [style=none] (1) at (-0.5, 4) {};
		\node [style=none] (2) at (4.75, -1.5) {};
		\node [style=none] (3) at (-0.5, -1.5) {};
		\node [style=circle, scale=2] (4) at (3, 2.25) {};
		\node [style=none] (5) at (3, 2.25) {$\pi_A$};
		\node [style=none] (6) at (3, 4) {};
		\node [style=none] (7) at (3, 1.25) {};
		\node [style=none] (8) at (3, -1.5) {};
		\node [style=none] (9) at (3, -3) {};
		\node [style=none] (10) at (0, -0.5) {};
		\node [style=none] (11) at (4.25, -0.5) {};
		\node [style=none] (12) at (4.25, 3.5) {};
		\node [style=none] (13) at (2.75, 3.5) {};
		\node [style=none] (14) at (1.5, 3.5) {};
		\node [style=none] (15) at (0, 3.5) {};
		\node [style=none] (16) at (0.25, -0.75) {$A \in \D$};
		\node [style=none] (17) at (3.25, -3) {$L^\dagger$};
		\node [style=none] (18) at (1, -3) {};
		\node [style=none] (19) at (0.5, -3) {$L^{\dagger \dagger}$};
		\node [style=none] (20) at (3, 1.25) {};
		\node [style=none] (21) at (1, 1.25) {};
		\node [style=circle, scale=2] (22) at (2, -0) {};
		\node [style=none] (23) at (2, -0) {$\epsilon_A$};
		\node [style=none] (24) at (1, -1.5) {};
		\node [style=none] (25) at (1, 4) {};
		\node [style=circle, fill=black] (26) at (1, 3.5) {};
	\end{pgfonlayer}
	\begin{pgfonlayer}{edgelayer}
		\draw (6.center) to (4);
		\draw (7.center) to (4);
		\draw (2.center) to (3.center);
		\draw (3.center) to (1.center);
		\draw (1.center) to (0.center);
		\draw (0.center) to (2.center);
		\draw (9.center) to (8.center);
		\draw (14.center) to (15.center);
		\draw (15.center) to (10.center);
		\draw (10.center) to (11.center);
		\draw (11.center) to (12.center);
		\draw (12.center) to (13.center);
		\draw [in=-90, out=165, looseness=1.00] (22) to (21.center);
		\draw [in=15, out=-90, looseness=1.25] (20.center) to (22);
		\draw (21.center) to (25.center);
		\draw (24.center) to (18.center);
	\end{pgfonlayer}
\end{tikzpicture}
\]
Step 2 is true because $(\eta_A, \epsilon_A): U(D(A)) \dagdual U(D(A))^\dagger$ is a right $\dagger$-linear dual. The final step is given by the fact that the dagger functor is an equivalence. Thus, $(\eta, \epsilon): L \dagdual L^\dagger$ is a right $\dagger$-linear dual.
\end{proof}

% @Robin: we need to discuss the proof and the statement @Priyaa is this OK now?
\begin{lemma}
Let $\X$ be a symmetric $\dagger$-LDC. Let ${\sf DagMon}(\X)$ be the category of $\dagger$-linear monoids, and linear monoid homomorphisms. Let $D: \D \to {\sf DagMon}(\X)$ be any diagram. Then, $\underset{ A \in \D}{\text{Lim }} U(D(A)) $ is a $\dagger$-linear monoid where $U: {\sf DagMon}(\X) \to \X$ is the underlying functor.
\end{lemma}
\begin{proof}
Let $L := \Lim{A \in \D} U(D(A))$. Then, $L^\dagger  =  \left( \Lim{A \in \D} U(D(A )) \right)^\dagger \simeq \Colim{A \in \D} U(D(A))^\dagger$.

By Lemma \ref{Lemma: dagger monoid}, every $\dagger$-linear monoid is isomorphic to a right $\dagger$-linear monoid. By Lemma
 \ref{Lemma: dagmon dual}, every right $\dagger$-linear monoid, $A \whitedag{(\iota, 1)} A^\dagger$,  gives a right $\dagger$-linear dual, 
 $A \dagdual A^\dagger$ with a monoid structure on $A$, and vice versa.  By Lemma \ref{Lemma: L is a dagger dual}, $L \dagdual L^\dagger$ is a right $\dagger$-linear dual. Hence, to prove that $L \whitedag{(\iota,1)} L^\dagger$ is a $\dagger$-linear monoid, it suffices to prove that that $L$ is a monoid. 
 
The multiplication and the unit maps for $L$ are given by the unique maps in the following limit diagrams:
\[
\xymatrixrowsep{8mm}
\xymatrix{
 &    L \ox L \ar[ld]_{\pi_A \ox \pi_A} \ar[dr]^{\pi_B \ox \pi_B} \ar@{.>}[dd]    & \\ 
 A \ox A \ar[dd]_{m_A} \ar[rr]_>>>>>>>>>{f \ox f}|\hole &  & B \ox B \ar[dd]^{m_B} \\ 
  & L  \ar[ld]_{\pi_A}  \ar[rd]^{\pi_B} \\
 A  \ar[rr]_{f} & & B} ~~~~~~~~ 
 \xymatrix{
 & \top \ar@{.>}[d] \ar@/_1pc/[ldd]_{u_A} \ar@/^1pc/[rdd]^{u_B} & \\
 & L \ar[ld]_{\pi_A} \ar[rd]^{\pi_B}& \\
 A \ar[rr]_{f} & & B
 } 
\]
The multiplication and the unit maps can be drawn as follows:
\[L \ox L \to^{m} L := \left< (\pi_A \ox \pi_A) m_A \right>_{A \in \D} = \begin{tikzpicture}
	\begin{pgfonlayer}{nodelayer}
		\node [style=none] (0) at (4.5, 3.25) {};
		\node [style=none] (1) at (3, 3.25) {};
		\node [style=circle, scale=2] (2) at (3, 2) {};
		\node [style=circle, scale=2] (3) at (4.5, 2) {};
		\node [style=none] (4) at (3, 2) {$\pi_A$};
		\node [style=none] (5) at (4.5, 2) {$\pi_A$};
		\node [style=circle] (6) at (3.75, 1) {};
		\node [style=none] (7) at (3.75, -0.5) {};
		\node [style=none] (8) at (2.5, 2.75) {};
		\node [style=none] (9) at (4.25, 2.75) {};
		\node [style=none] (10) at (3.25, 2.75) {};
		\node [style=none] (11) at (5, 2.75) {};
		\node [style=none] (12) at (5, 0.25) {};
		\node [style=none] (13) at (2.5, 0.25) {};
		\node [style=circle, fill=black] (14) at (3.75, 0.25) {};
		\node [style=none] (15) at (4, -0.5) {$L$};
		\node [style=none] (16) at (2.75, 3.25) {$L$};
		\node [style=none] (17) at (4.75, 3.25) {$L$};
	\end{pgfonlayer}
	\begin{pgfonlayer}{edgelayer}
		\draw [bend right, looseness=1.00] (6) to (3);
		\draw [bend left, looseness=1.25] (6) to (2);
		\draw (6) to (7.center);
		\draw (2) to (1.center);
		\draw (0.center) to (3);
		\draw (10.center) to (8.center);
		\draw (8.center) to (13.center);
		\draw (13.center) to (12.center);
		\draw (12.center) to (11.center);
		\draw (11.center) to (9.center);
	\end{pgfonlayer}
\end{tikzpicture} ~~~~~~ u := \left< u_A \right>_{A \in \D} = 
\begin{tikzpicture}
	\begin{pgfonlayer}{nodelayer}
		\node [style=circle] (0) at (3.75, 2.25) {};
		\node [style=none] (1) at (3.75, -0.5) {};
		\node [style=none] (2) at (2.5, 2.75) {};
		\node [style=none] (3) at (4.25, 2.75) {};
		\node [style=none] (4) at (3.25, 2.75) {};
		\node [style=none] (5) at (5, 2.75) {};
		\node [style=none] (6) at (5, 0.25) {};
		\node [style=none] (7) at (2.5, 0.25) {};
		\node [style=circle, fill=black] (8) at (3.75, 0.25) {};
		\node [style=none] (9) at (4, -0.5) {$L$};
		\node [style=none] (10) at (3.5, 1.25) {$A$};
	\end{pgfonlayer}
	\begin{pgfonlayer}{edgelayer}
		\draw (4.center) to (2.center);
		\draw (2.center) to (7.center);
		\draw (7.center) to (6.center);
		\draw (6.center) to (5.center);
		\draw (5.center) to (3.center);
		\draw (0) to (1.center);
	\end{pgfonlayer}
\end{tikzpicture}
\]
Using string diagrams, we can prove that the unit law and the associate law holds. 

Thus,  we have that $L$ is a $\dagger$-linear monoid.
\end{proof}
\fi 


\subsection{Linear monoids in ${\sf FHilb}$}
\label{Sec: linear monoid examples}

In this section we provide distinguishing examples of linear monoids 
and Frobenius algebras in ${\sf FHilb}$ as illustrated by the following Venn diagram.

\[ \begin{tikzpicture}[scale=1.5]
	\begin{pgfonlayer}{nodelayer}
		\node [style=circle, scale=15] (0) at (-2, -6) {};
		\node [style=circle, scale=15] (1) at (1, -6) {};
		\node [style=circle] (2) at (-3, -4.75) {$\ref{example: CFA}$};
		%\node [style=circle] (3) at (-3, -7) {$\ref{example: DFA}$};
		\node [style=circle] (4) at (-0.5, -4.75) {$\ref{example: CDFA}$};
		\node [style=circle] (5) at (-0.5, -7) {$\ref{example: DFA}$};
		\node [style=circle] (6) at (2, -4.75) {$\ref{example: CDLM}$};
		\node [style=circle] (7) at (2, -7) {$\ref{example: DLM}$};
		\node [style=none] (8) at (5.25, -4.5) {};
		\node [style=none] (9) at (2.75, -2) {};
		\node [style=none] (10) at (5.25, -7.5) {};
		\node [style=none] (11) at (2.75, -10) {};
		\node [style=none] (12) at (-4, -10) {};
		\node [style=none] (13) at (-6.25, -7.5) {};
		\node [style=none] (14) at (-6.25, -4.5) {};
		\node [style=none] (15) at (-3.75, -2) {};
		\node [style=none] (16) at (-6.25, -6) {};
		\node [style=none] (17) at (5.25, -6) {};
		\node [style=none] (18) at (-0.5, -9.75) {non-commutative};
		\node [style=none] (19) at (-0.5, -2.25) {commutative};
		\node [style=none] (20) at (4.5, -5.75) {LM};
		\node [style=none] (21) at (2, -5.75) {$\dag$-LM};
		\node [style=none] (22) at (-0.5, -5.75) {$\dag$-FA};
		\node [style=none] (23) at (-3, -5.75) {FA};
		\node [style=none] (24) at (8.25, -5.5) {LM : linear monoid};
		\node [style=none] (25) at (8.6, -6.5) {FA : Frobenius algbera};
		\node [style=circle] (26) at (-4.5, -3) {$\ref{example: CLM}$};
		\node [style=circle] (27) at (-4.5, -8.75) {$\ref{example: LM}$};
	\end{pgfonlayer}
	\begin{pgfonlayer}{edgelayer}
		\draw [bend right=45, looseness=1.25] (15.center) to (14.center);
		\draw (14.center) to (13.center);
		\draw [bend right=45] (13.center) to (12.center);
		\draw (12.center) to (11.center);
		\draw [bend right=45] (11.center) to (10.center);
		\draw (10.center) to (8.center);
		\draw [bend right=45] (8.center) to (9.center);
		\draw (9.center) to (15.center);
		\draw (16.center) to (17.center);
	\end{pgfonlayer}
\end{tikzpicture} \]

\bigskip

The numbers within each section of the diagram refer to the examples listed below.

\begin{example}\cite{CPV12}
    \label{example: CFA}
     Every special commutative Frobenius Algebra (SCFA) in $\FHilb$ corresponds precisely to an arbitrary copyable basis.
\end{example}
By copyable basis, we mean that the comultiplication of the corresponding SCFA copies (duplicates) 
the basis elements. Similary, given a SCFA, the corresponding basis is given by 
those vectors which the comultiplication of SCFA duplicates.

\begin{example} \cite{CPV12}
    \label{example: CDFA}
    Every commutative $\dagger$-Frobenius Algebra in $\FHilb$ corresponds precisely to an 
    orthogonal copyable basis.
\end{example}

\begin{example}
\label{example: DFA}
The pants algebra are non-commutative $\dagger$-Frobenius algebra. Recall that in $\FHilb$, 
pants algebra are given by the algebra of $n \times n$ complex matrices, See Section 
\ref{Sec: observables}. 
\end{example}

The algebra of polynomials over a field $K$ give a wealth of examples for linear monoids. 
\begin{example}
    \label{example: CDLM}
    In $\FHilb$, the basic Weil algebra, $\C[x]/x^2=0$, is a commutative $\dagger$-linear monoid.
\end{example}
\begin{proof}
    Consider the two-dimensional basic Weil algebra:  $\C[x]/x^2=0$ 
    The multiplication, $m$, for the algebra can be represented as a matrix, $M$, 
    with 4 rows and 2 columns. The rows are indexed from top to bottom as, $1.1$, $1.x$, $x.1$ and 
    $x.x$,. The columns are indexed from left to right as $1$ and $x$. The multiplication 
    matrix is given as follows: 
    \[ M := \begin{bmatrix}
        1 & 0 \\
        0 & 1 \\
        0 & 1 \\
        0 & 0 
    \end{bmatrix} \] 
    It is evident that $M^* = M^\dag$, that is the transpose of $M$ is the same as its conjugate transpose.
    Moreover, the multiplication is commutative. Hence, $\C[x]/x^2=0$ is a commutative $\dagger$-linear monoid. 
    
    Its not a $\dagger$-Frobenius algebra since the algebra is not generated by a copyable basis. The proof is as follows. 
        For contradiction, let us assume that $\C[x]/x^2=0$ is generated by a copyable basis,  and  $b_i = a + bx$ 
    be a basis element. Since the basis is copyable, we have that $(b_i \ox b_i) m = b_i$. 
    Using matrix representation of $b_i$ we get the following equation:
    \[ \left[ \left( \begin{matrix} a & b \end{matrix} \right) \ox 
    \left( \begin{matrix} a & b \end{matrix} \right) \right] M = 
    \left( \begin{matrix} a^2 & 2ab \end{matrix} \right) = 
    \left( \begin{matrix} a & b \end{matrix} \right) \]
    
    The possible solutions to the above equation are: $b_i = \left( \begin{matrix} 1 & 0 \end{matrix} \right)$, 
    $ b_i = \left( \begin{matrix} -1 & 0 \end{matrix} \right)$, and 
    $b_i = \left( \begin{matrix} 0 & 0 \end{matrix} \right)$. 
    The matrices clearly do not produce a basis for the algebra. 
    Hence, the given Weil algebra is not Frobenius.
\end{proof}

\begin{example}
    \label{example: DLM}
In $\FHilb$, The  polynomial algebra $ \C[x,y,z]/x^2 = 0, y^2 = 0, z^2 = 0, xy = iz, yx = -iz, xz = 0, zx = 0, yz = 0, zy = 0$ is 
a non-commutative $\dagger$-linear monoid.
\end{example}
\begin{proof}
    Consider the four-dimensional polynomial algebra,
    \[ \C[x,y,z]/x^2 = 0, y^2 = 0, z^2 = 0, xy = iz, yx = -iz, xz = 0, zx = 0, yz = 0, zy = 0 \]
    Consider the multiplication map $m$ for the algebra. 
    Note that for all $i,j \in \{ 1, x, y, z \}$, the conjugate of the multiplication 
    coincides with the multiplication map:
    \[ \overline{m}(i,j) \stackrel{(1)}{=} \overline{m(j,i)} \stackrel{(2)}{=} m(i, j) \]
    The step $(1)$ is true because conjugation reverses the order of the types. 
    For all $i,j$, the step $(2)$ is true for the given algebra.  Since the 
    given monoid coincides with the conjugate monoid (equivalently 
    $\dagger$ of the monoid coincides with dual of the monoid), and the 
    multiplication is non-commutative, we have a non-commutative $\dagger$-linear monoid. 
    
    As in the previous section, we prove that the algebra is not generated by a copyable basis. 
    Assuming there exists a copyable basis for the algebra, consider any basis element $e_i = (a + bx + cy + dz)$. 
    Solving the equation $e_i . e_i = e_i$ leads to the following situation:
      \[  a^2 = 0 ~~~~~~~~~ ab + ba = b ~~~~~~~~ ac + ca = c ~~~~~~~~~~ ad + da = d \]
    The solutions for the above equations are: $a = 1, -1, 0$; $b = c = d = 0$, which is not a basis. 
    This leads to a contradiction. Hence, the given algebra is not $\dagger$-Frobenius. 
\end{proof}

\begin{example}
\label{example: CLM}
In $\FHilb$, the following algebra is a commutative linear monoid:
\[C[x,y,z]/x^2 = 0, y^2 = 0, z^2 = 0, xy = iz, yx = iz, xz = 0, zx = 0, yz = 0, zy = 0 \]
\end{example}
\begin{proof}
    Recall that $\FHilb$ is a symmetric KCC. Every object in the category comes with a chosen dual and 
    the duals are symmetric. Hence, every monoid in this category is automatically a linear monoid. 
    Note that the monoid is commutative. Hence, the given algebra is a commutative linear monoid. 
    
    It is not a commutative Frobenius algebra since there does not exist a copyable basis for the algebra:
    For contradiction, assume that $e_i = a + bx + cy + dz + e(yz)$ be an element of the copyable basis. 
    Since the basis is copyable, the following must hold: $(e_i \ox e_i)m = e_i$. Then we have that, 
    \[ a^2 = a; ~~~~~~~~~ 2ab = b ~~~~~~~~~ 2ac = c ~~~~~~~~~ 2(ad + i bc) = d \]
    The solutions for the above equations are: $a = 1, -1, 0$; $b = c = d = 0$, which is not a basis set. 
    Hence, the given algebra cannot be Frobenius.
\end{proof}

\begin{example}
    \label{example: LM}
    In $\FHilb$, the following algebra is a non-commutative linear monoid:
    \[ \C[x,y,z]/x^2 = 0, y^2 = 0, z^2 = 0, xy = iz, yx = z, xz = 0, zx = 0, yz = 0, zy = 0 \] 
\end{example}
Proven similarly as the above examples.

\iffalse TODO: COME BACK TO THIS AT THE END
\subsection{Clifford algebras in ${\sf FFVec_\C}$}
\label{Sec: Clifford}

Clifford algebras \cite{Hes12, LuS09, DoL03} play an important role in physics due to their intimate connection to geometry. 
Every Clifford operation can be viewed as a series of geometric transformations on a spatial structure 
and every geometric transformation can be written as a series of Clifford operations. 
Clifford algebras are considered as a unifying language for quantum theory and relativity \cite{Hes12, VaR19}. 
The Pauli $I, X, Y, Z$ matrices which are quite significant for quantum mechanics form a Clifford algebra. 
The algebra generated by the gamma matrices of quantum field theory form a Clifford algebra.   
In this section, we study Clifford algebras in $\FHilb$. Quite strikingly, we note that in $\FHilb$, every Clifford algebra is a symmetric $\dagger$-Frobenius algebra. 

Let $K$ be a field. A Clifford algebra for any $K$-vector space, $V$ with a symmetric biliniear form $B: V \ox V \to K$ is defined as 
follows:
\begin{definition}
The {\bf Clifford algebra} for any $K$-vector space with a symmetric bilinear from is the tensor algebra 
$(T(V) := \oa_n V^{\ox n}, \mulmap{1.5}{white}, \unitmap{1.5}{white})$ of $V$ satisfying the condition:
\[ \text{for all } v \in V, ~~~~  \begin{tikzpicture}[scale=1.3]
	\begin{pgfonlayer}{nodelayer}
		\node [style=none] (0) at (-2.25, -2.75) {};
		\node [style=none] (1) at (-2.75, -3.5) {};
		\node [style=none] (2) at (-1.75, -3.5) {};
		\node [style=none] (3) at (-0.75, -2.75) {};
		\node [style=none] (4) at (-1.25, -3.5) {};
		\node [style=none] (5) at (-0.25, -3.5) {};
		\node [style=none] (6) at (-2.25, -3.5) {};
		\node [style=none] (7) at (-0.75, -3.5) {};
		\node [style=circle] (8) at (-1.5, -4.5) {};
		\node [style=none] (9) at (-1.5, -5.5) {};
		\node [style=none] (10) at (-2.25, -3.25) {$v$};
		\node [style=none] (11) at (-0.75, -3.25) {$v$};
		\node [style=none] (12) at (-2.25, -5.25) {$T(V)$};
		\node [style=none] (13) at (-3, -4) {$T(V)$};
	\end{pgfonlayer}
	\begin{pgfonlayer}{edgelayer}
		\draw (0.center) to (1.center);
		\draw (1.center) to (2.center);
		\draw (2.center) to (0.center);
		\draw (3.center) to (4.center);
		\draw (4.center) to (5.center);
		\draw (5.center) to (3.center);
		\draw [in=-90, out=165] (8) to (6.center);
		\draw [in=-90, out=15] (8) to (7.center);
		\draw (8) to (9.center);
	\end{pgfonlayer}
\end{tikzpicture} =  b(v,v) \begin{tikzpicture}[scale=1.3]
        \begin{pgfonlayer}{nodelayer}
            \node [style=none] (3) at (-0.75, -2.75) {};
            \node [style=none] (4) at (-1.25, -3.5) {};
            \node [style=none] (5) at (-0.25, -3.5) {};
            \node [style=none] (7) at (-0.75, -3.5) {};
            \node [style=none] (11) at (-0.75, -3.25) {$v$};
            \node [style=none] (12) at (-0.75, -5.5) {};
        \end{pgfonlayer}
        \begin{pgfonlayer}{edgelayer}
            \draw (3.center) to (4.center);
            \draw (4.center) to (5.center);
            \draw (5.center) to (3.center);
            \draw (7.center) to (12.center);
        \end{pgfonlayer}
    \end{tikzpicture}  \]
\end{definition}
Note that $V$ is a subspace of $T(V)$. 

In \cite[Sec. 2.2]{LuS09}, Lundholm and Svensson give a generic construction for a Clifford Algebra, $Cl(B, R, r)$, from an 
over finite set $B$ with $R$ being a commutative ring, and $r: B \to R$ being an arbitrary map. In this 
section, we describe their construction for $R = \C$, and  $r: B \to \C$ defined as, for all $b \in B$, $r(b) := 1$.  

Given an arbitrary finite set $B$, the basis for the Clifford algebra, $Cl(B, \C, r)$, is given by $\{ e_X \}_{X \in \mathcal{P}(B)}$
where $\mathcal{P}(B)$ the power set of $B$. The multiplication on the elements of the Clifford basis is defined 
as follows: for all $X, Y \in \mathcal{P}(B)$,

\begin{equation}
    \label{Eqn: clifford mult}
e_X e_Y := \tau_{X,Y} ~ e_{X \Delta Y ^\text{\footnote{$X \Delta Y = (X - Y) \cup (Y - X)$}}}  
\end{equation}

such that 
\begin{itemize}
\item $\tau_{X,Y} \in \{ 1, -1 \} $
\item $\tau_{\{  \} \{ i \} } =  1 = \tau_{\{ i \} \{  \} }$ for all $i \in B$
\item $\tau_{\{ i \} \{ i \} } = 1$ for all $i \in B$
\item $\tau_{\{ i \} \{ j \} } = - \tau_{ \{ j \} \{ i \} }$ for all $i, j \in B$, $i \neq j$
\item $\tau_{X, Y \Delta Z} = \tau_{X,Y} \tau_{X \Delta Y, Z}$ 
for all $A, B, C  \in \mathcal{P}(B)$
\end{itemize}

$\tau_{X,Y}$ is the cefficient of the multiplication. In the rest of the article, 
we will write $Cl(B, \C, r)$ in short as $Cl(B)$. 

We demonstrate an example of $Cl(B)$ with $B = \{ 1, 2, 3 \}$. The elements of the Clifford basis are 
given by $\{ e_X \}_{X \in \mathcal{P}(B)}$. For brevity, we will write element of the Clifford basis, 
say $e_{\{1, 2 \}}$ as $e_{12}$, and $e_{\{\}} = 1$.  Similary, $\tau_{\{1,2\},\{ 2,3 \}} := \tau_{12, 23}$.
Now, the Clifford multiplication on $\{ e_X \}_{X \in \mathcal{P}(\{ 1, 2, 3 \})}$
is defined inductively as follows:

Assuming an ordering on the elements of $B$, say, $e_1 < e_2 < e_3$. For all
 $i, j \in \{1,2,3\}$ and $i \neq j$, define $\tau_{i,j} = 1$, if $e_i < e_j$.  
 Hence, $e_1 e_2 = \tau_{1,2} e_{12} = e_{12}$. Let us consider multiplication of $e_{12}$ and $e_3$: 
$e_{12} e_{3} = (e_1 e_2) e_3 = e_{123}$. Hence, $\tau_{12,3} = 1$. Consider, 
\begin{align*}
e_{12} e_{13} &= (e_1 e_2) (e_1 e_3)  = (( e_1 e_2) e_1) e_3 = ((e_1) (e_2 e_1)) e_3 \\
&= - (e_1 (e_1 e_2)) e_3 = -((e_1 e_1) e_2) e_3 = - (e_{ \{ \} } e_2) e_3 = - e_{23} 
\end{align*}

Hence, $\tau_{12, 13} = -1$.  
The multiplication is associative. 
The multiplication table for the Clifford algebra $Cl(\{ e_1, e_2 \})$ generated by 
$\{ e_0, e_1 \}$ is as follows:

\begin{center}
\begin{tabular}{c|cccc}
$.$ & $1$ & $e_1$ & $e_2$ & $e_{12}$ \\
\hline
$1$ & $e_1$ & $e_2$ & $e_3$ & $e_{12}$ \\
$e_1$ & $e_1$ & $1$  & $e_{12}$ & $e_2$ \\
$e_2$ & $e_2$ & $-e_{12}$ & $1$ & $-e_1$ \\
$e_{12}$ & $e_{12}$ & $-e_2$ & $e_1$ & $-1$ 
\end{tabular} 
\end{center}

\begin{lemma}
\label{Lemma: keyresult}
Let $Cl(\C^n)$ be a Clifford algebra generated from an orthogonal basis $B = \{ e_0, e_1, \cdots, e_n \}$.
Then, for all $X, Y \in \mathcal{P}(B)$, $\tau_{(X \Delta Y,X \Delta Y)} = 
 \tau_{X, X} \tau_{X,Y} \tau_{Y,X} \tau_{Y,Y}$
\end{lemma}
\begin{proof}
Proof by induction on the size of the basis set $B$.

\begin{description}
\item[{\bf Base case:}] $|B| = 0$, and $|B| = 1$

Let $B = \emptyset$. Its is straightforward that the statement holds. 

Let $B = \{ x \}$. Then, $\mathcal{P}(B) = \{ \{ \}, \{x\} \}$. Let $X = \{ x \}, J = \{ \} = \emptyset$

$ \tau_{\emptyset,\{x\}} = 1$

$ \tau_{\emptyset,\{x\}} \tau_{\{x\},\emptyset} \tau_{\{x\},\{x\}}, \tau_{\emptyset,\emptyset} = (1)(1)(1)(1)
= 1$

It is strightforward that, the statement holds when $X = Y$ for all $X \in \mathcal{P}(B)$.

\item[{\bf Inductive hypothesis:}] There exists $n \in \N$ such that for all $0 \leq k \leq n$, if 
$|B| = k$, then the statement holds.

\item[{\bf Inductive step:}] Assume $|B'| = n+1$, $B' = B \cup Z$, where $Z = \{z\}$

For $X \in \mathcal{P}(B)$, let $XZ$ refer to the set $X \cup \{z\}$.

\begin{description}
    \item[Case 1:] 
    
    Prove that $\tau_{XZ \Delta Y, XZ \Delta Y} = \tau_{XZ,XZ} \tau_{XZ, Y}, \tau_{Y, XZ}, \tau_{Y,Y}$
 
    
    Left hand side (LHS):
    \[ \tau_{XZ \Delta Y, XZ \Delta Y} = (-1)^{Y \Delta X} \tau_{X \Delta Y, X \Delta Y} \]

    Right hand side (RHS):
    \begin{align*}
	\tau_{XZ,XZ} \tau_{XZ,Y} \tau_{Y,XZ} \tau_{Y,Y} &= (-1)^{|X|} \tau_{X,X} (-1)^|Y| \tau_{X,Y} 
	\tau_{Y,X} \tau_{Y,Y}\\
    &= (-1)^{|X| + |Y|} \tau_{X,X} \tau_{X,Y} \tau_{Y,X} \tau_{Y,Y}\\
    &= (-1)^{|X| + |Y|} \tau_{X \Delta Y, X \Delta Y}
    \end{align*}

    Now, $(-1)^{|X| + |Y|} = (-1)^{Y \Delta X}$ because, $|X \Delta Y| = |X| + |Y| - 2 |X \cap Y|$. Hence, 
    the statement holds.

    \item[Case 2:] $\tau_{X \Delta YZ, X \Delta YZ} = \tau_{X,X} \tau_{X, YZ}, \tau_{YZ, X}, \tau_{YZ,YZ}$
     
    Left hand side (LHS):
    \[ \tau_{X \Delta YZ, X \Delta YZ} = (-1)^{X \Delta Y} \tau_{X \Delta Y, X \Delta Y} \]

    Right hand side (RHS):
    \begin{align*}
	\tau_{X,X} \tau_{X, YZ}, \tau_{YZ, X}, \tau_{YZ,YZ} &= \tau_{X,X} \tau_{X,Y} (-1)^{|X|} 
	\tau_{Y,X} (-1)^|Y| \tau_{Y,Y}\\
    &= (-1)^{|X| + |Y|} \tau_{X,X} \tau_{X,Y} \tau_{Y,X} \tau_{Y,Y}\\
    &= (-1)^{|X| + |Y|} \tau_{X \Delta Y, X \Delta Y}
    \end{align*}
    
    LHS = RHS because, $(-1)^{|X| + |Y|} = (-1)^{Y \Delta X}$.

    \item [Case 3:]
    
	Prove that $\tau_{XZ \Delta YZ, XZ \Delta YZ} = \tau_{XZ,XZ} \tau_{XZ, YZ}, \tau_{YZ, XZ}, 
	\tau_{YZ,YZ}$

    Left hand side (LHS):
    \[ \tau_{XZ \Delta YZ, XZ \Delta YZ} = \tau_{X \Delta Y, X \Delta Y} \]

    Right hand side (RHS):
    \begin{align*}
	\tau_{XZ,XZ} \tau_{XZ, YZ}, \tau_{YZ, XZ}, \tau_{YZ,YZ} &= (-1)^{X} \tau_{X,X} (-1)^|Y| 
	\tau_{X,Y} (-1)^{Y,X} (-1)^Y \tau_{Y,Y}\\
    &= (-1)^2(|X| + |Y| ) \tau_{X \Delta Y, X \Delta Y} \\
    &= \tau_{X \Delta Y, X \Delta Y}
    \end{align*}

\end{description}
\end{description}
\end{proof}

Now we are ready to prove that every Clifford algbera in ${\sf FHilb}$ is a non-commutative dagger Frobenius Algebra.

Next, we consider Clifford Algebras over finite-dimensional complex Hilbert spaces. 
We prove that Clifford Algebras in ${\sf FHilb}$ that the multiplication for Clifford Algebras 
is self-conjugate, hence by Lemma \cite[Defn. 5.25]{HeV19} , they are 
$\dagger$-Frobenius.

\begin{lemma}
    \label{Lemma: clifford} 
    Every Clifford algebra in ${\sf FHilb}$, $Cl(\C^n)$, generated by an orthogonal basis for $\C^n$
     is a non-commutative dagger Frobenius Algebra.
\end{lemma}
\begin{proof}
Consider any finite dimensional Hilbert Space $\C^n$. Let $B=(e_0, e_1, \cdots, e_n)$ be an 
orthogonal basis of $\C^n$. Let $Cl(\C^n)$ be a Clifford Algebra corresponding to the basis $B$. The basis for the 
Clifford Algebra $Cl(\C^n)$ is given by the powerset of $B$, $\mathcal{P}(B)$. 
By Lemma \ref{Lemma: involutive}, $Cl(\C^n)$ is a dagger-FA if and only if 
the multiplication map coincides with its conjugate, that is, 
 for the multiplication, $m: Cl(\C^n) \ox Cl(\C^n) \to Cl(\C^n)$, and 
 for all the elements, $e_X$, $e_Y$ in the Clifford basis, it must be the case that: 
\[ e_X . e_Y := (e_X \ox e_Y) m 
= (\overline{e_X} \ox \overline{e_Y} ) \overline{m} 
= \overline{e_Y \ox e_X } \overline m
= \overline{(e_Y \ox e_X) m}
= \overline{e_Y . e_X} \]

The multiplication for $Cl(\C^n)$ as given in equation \ref{Eqn: clifford mult} 
is not self-conjugate per se for the basis 
$\mathcal{P}(B)$. Hence, we perform a basis transformation on $Cl(\C^n)$ as follows:
\[ f: Cl(\C^n) \to Cl(\C^n) ~ ; ~ e_{X \subseteq \mathcal{P}(B)} \mapsto \gamma_X e_X'\]
where $\gamma_{\{ \}} := 1$, $\gamma_X := \sqrt[+]{\tau_{X,X}} \gamma^2_X$, and $\tau_{X, X} \in \{1,i\}$.

It is clear that, for all $I \in \mathcal{P}(B)$, $\gamma_X = \{ 1, i, -1, -i \}$. 
Now, we prove that under the basis transformation $f$, the multiplication map is self-conjugate 
for the newly defined basis, that is, for all $X, Y \subseteq \mathcal{P}(B)$, $e'_X e'_Y =
\overline{e_X' e_Y'}$.

We know that $f(e_X) := \gamma_X e_X'$. Then, $e_I' = 1 / \gamma_X f(e_X)$. Multiplying basis 
elements $e_X'$ and  $e_Y'$, we have:

\begin{align*}
e_I' e_J' &:= \frac{1}{\gamma_X} f(e_X) \frac{1}{\gamma_J} f(e_Y) \\
&= \frac{1}{\gamma_X \gamma_Y} f(e_X e_Y) \\
&= \frac{1}{\gamma_X \gamma_Y} f(\tau_{I, J} e_{X \Delta Y}) \\
&= \frac{\tau_{X, Y}}{\gamma_X \gamma_Y} f(e_{X \Delta Y}) \\
&= \frac{\tau_{X, Y} \gamma_{X \Delta Y}}{\gamma_X \gamma_Y} e_{X \Delta Y}' 
\end{align*}

Hence, we have that:
\[ 
    e_X' e_Y' = \Gamma_{X,Y} e'_{X \Delta Y}, \text{ where, } \Gamma_{X,Y} = 
    \frac{\tau_{X, Y} \gamma_{X \Delta Y}}{\gamma_X \gamma_Y}
\]

To prove that multiplication is self-confugate for the transformed basis, we have to prove 
that for all $X, Y \subseteq \mathcal{P}(B)$, $\Gamma_{X,Y} = \overline{\Gamma_{Y,X}}$. We
give the proof case by case:


\begin{description}
\item[Case 1: $X = \emptyset$ or $Y = \emptyset$] 
This case represents the first row $(X = \emptyset)$ and the first column $(Y = \emptyset)$
 of the multiplication table. If $X = \emptyset$, then $ \Gamma_{X, Y} = \frac{\tau_{X, Y} 
 \gamma_{X \Delta Y}}{\gamma_X \gamma_Y} = \frac{1 \gamma_X}{1 \gamma_Y} = 1$. 
 Clearly, $\Gamma_{X, Y} = \overline{\Gamma_{Y, X}}$. 

 \item[Case 2: $X = \emptyset$ and $Y = \emptyset$]
This case represents the diagonal elements of the multiplication table. It is straightforward
that $\gamma_{ \{ \}, \{ \}} = 1$.

\item[Case 3: $X \neq Y \neq \emptyset$] This case represents the off-diagonal elements of the 
multiplication table.
We know that:
\[ \Gamma_{X,Y} = \frac{\tau_{X, Y} \gamma_{X \Delta Y}}{\gamma_X \gamma_Y} = \frac{\tau_{X,Y} 
\sqrt[+]{\tau_{X \Delta Y, X \Delta Y}}}{\sqrt[+]{\tau_{X,X}} \sqrt[+]{\tau_{Y,Y}}} 
= \tau_{X, Y} \kappa \]
\[ \Gamma_{Y,X} = \frac{\tau_{Y, X} \gamma_{X \Delta Y}}{\gamma_X \gamma_Y} = \frac{\tau_{Y,X} 
\sqrt[+]{\tau_{X \Delta Y, X \Delta Y}}}{\sqrt[+]{\tau_{X,X}} \sqrt[+]{\tau_{Y,Y}}}
=  \tau_{Y, X} \kappa \]

Observe that $\kappa \in \{1, -1, i, -i \}$. We also know that for all $A,B \in 
\mathcal{P}(B)$, we know that $\tau_{X, Y} \in \{ 1, -1 \}$.
Now, in order to show that $\Gamma_{X, Y} = \overline{\Gamma_{Y, X}}$, it suffices to prove
that:
 \begin{itemize}
    \item $\kappa \in \{ 1, -1 \}$ , if and only if $\tau_{X,Y} = \tau_{Y,X}$, and 
    \item if $\kappa \in \{ i, -i \}$ if and only if $\tau_{X,Y} = -\tau_{Y,X}$
 \end{itemize}

The conditions listed above hold because, by Lemma \ref{Lemma: keyresult}
$\tau_{X \Delta Y, X \Delta Y} = \tau_{X,X} \tau_{Y,Y} \tau_{X,Y} \tau_{Y,X}$.

Replacing for $\tau_{X \Delta Y, X \Delta Y}$ in $\kappa$, we get
\[ 
\kappa = \frac{\tau_{Y,X} \sqrt[+]{\tau_{X \Delta Y, X \Delta Y}}}
{\sqrt[+]{\tau_{X,X}} \sqrt[+]{\tau_{Y,Y}}}
= \frac{\tau_{Y,X} \sqrt[+]{\tau_{X,X} \tau_{Y,Y} \tau_{X,Y} \tau_{Y,X}}}
{\sqrt[+]{\tau_{X,X}} \sqrt[+]{\tau_{Y,Y}}} 
= \sqrt[+]{\tau_{X,Y}} \sqrt[+]{\tau_{Y,AX}} 
\]

In that case,
$\kappa \in \{ 1, -1 \}$ if and only if $\tau_{X,Y} = \tau_{Y,X} = 1$ OR 
$\tau_{X,Y} = \tau_{Y,X} = -1$. 
$\kappa \in \{ i, -i \}$ if and only if $\tau_{A,B} = -1$ and $\tau_{Y,X} = 1$, OR 
$\tau_{X,Y} = 1$ and $\tau_{Y,X4} = -1$.
\end{description}
\end{proof}
\fi

%%%%%%%%%%%%%%%%%%%%%%%%%%%%%%%%%%%%%%%%%%%%%%%%%%%%%%%%%

\section{Exponential Modalities}
\label{Section: exp mod examples}

In this section we give a few examples of LDCs with free exponential modalities. \cite{HyS03, Lem19}
discusses more examples of categories with free exponential modalities. Explicit constructions for 
free exponential modalities are discussed in \cite{MTT18, Sla17}.

\subsection{The free exponential modalities for $\Rel$}

We know that  $\Rel$, the category of sets and relations is a $\dagger$-compact closed category, See Section \ref{Sec: relations}.
Thus, $\Rel$ is a compact LDC with $\ox = \oa$. The category $\Rel$ comes with the free exponential modalities which are 
defined as follows:

\begin{enumerate}[(i)]
\item  For a set $X$, $!X$ is the set of all finite multisets of $X$:
\[ !X := \{ \llbracket x_1, \cdots, x_n \rrbracket \ | x_i \in X \} \]
For a relation $R: X \to Y$, $!R$ relates multisets with same number of distinct elements 
such that the $i^{th}$ distinct element of each multi-set is related by $R$:
\[ !R := \{  (\llbracket x_1, \cdots, x_n \rrbracket, \llbracket y_1, \cdots, y_n \rrbracket) | 
(x_i, y_i) \in R \} \]

In $\Rel$, since each object is self-dual, $?X = (!X)^{\dag*} = !X$. Similary, for any relation $R$, 
\[ ?R = (!R)^{\dag*} = (?R^\circ)^\dag = !(R^{\circ \circ}) \] 
where $R^\circ$ is the converse relation. 

\item $\delta_X: !X \to !!X$ relates a finite multiset of $X$ to the set of all possible splittings of the multiset. 
\[ \delta_X := \{ K, \llbracket K_1, K_2, \cdots, K_n \rrbracket ) | K_1 \cup K_2 \cup \cdots \cup K_n = B \} 
\subseteq !X \times !!X \]
Note that an element of $!!X$ is a finite multiset whose elements are finite multisets of $X$. 
$\mu_X: ?X \to ??X$ is the converse of $\delta_X$.

\item $\epsilon_X: !X \to X$ relates multisets of a single element in $X$ to the element itself. $\eta_X$ is the 
converse of $\epsilon_X$. 
\[ \epsilon_X := \{ (\llbracket x \rrbracket, x) | x \in X\} \subseteq !X \times X \] 

\item $(m_\ox)_{X,Y} : !X \times !Y \to !(X \times Y) $ is the relation which relates the finite multisets of $X$ and $Y$ 
with equal number of distinct elements to the multiset given by their cartesian product:

\bigskip

$ (m_\ox)_{X,Y} := \{ ((\llbracket x_1, \cdots, x_n \rrbracket, \llbracket y_1, \cdots, y_n \rrbracket),
\llbracket (x_i, y_j)) | 0 < i, j \leq n \rrbracket | x_i \in X, y_j \in Y \}$
$~~~~~\subseteq (!X \times !Y) \times !(X \times Y) $

\bigskip

$(n_\ox)_{X,Y}: ?(X \times Y) \to ?X \times ?Y$ is the converse of $(m_\ox)_{X,Y}$.

\item $m_I = \{*\} \to !\{*\}$ is simply the cartesian product, $\{ * \} \times !\{ * \}$, that is, $m_I$ relates 
$*$ to every element of $!\{*\}$. The relation $n_I: ?\{*\} \to \{*\}$ is the converse of $m_I$

\item $\Delta_X : !X \to !X \times !X$  relates a finite multiset of $X$ to pairs of finte multisets such that 
the pair is a splitting of the original multiset:
\[ \Delta_X := \{ K, (K_1, K_2) | K, K_1, K_2 \in !X,  K_1 \cup K_2 = K \} \subseteq !X \times (!X \times !X )\]
$\nabla_X: ?X \ox ?X \to ?X$ is the converse of $\Delta_X$. 

\item $\tricounit{0.65}_X: !X \to \{ * \} $ relates the empty set from $!X$ to $*$. $\triunit{0.65}_X: \{ * \} \to ?X$ 
is the converse of $\tricounit{0.65}_X$.
\[ \tricounit{0.65}_X := \{ (\emptyset, *) \} \subseteq !X \times \{*\} \]

\end{enumerate}

\subsection{The free exponential modalities for ${\sf FRel}$ and ${\sf FMat}(R)$}

In  \cite{Ehr05} the exponential modalities for ${\sf FRel}$ and, more generally, ${\sf FMat}(R)$ are described.  Furthermore, Christine Tasson in her PhD. thesis, \cite{tasson}, showed that this modality in ${\sf FRel}$ is {\em free\/} in the sense of Yves LaFont.    The purpose of this section is to provide a review of these results and to establish also that the modality in 
${\sf Mat}(R)$ is also free. This means that, by Theorem \ref{Theorem: Main}, in the MUC $\Mat(\C) \hookrightarrow \FMat(\C)$, 
every complementary system within the unitary core arise as a
compaction of a $\dagger$-linear bialgebra on the free exponentials.

In a symmetric monoidal category, we saw that, $\X$, an {\bf coalgebra comodality} is given by a comonad 
\[ (!: \X \to \X, !(A) \to^{\varepsilon} A,!(A) \to^{\delta} !(!(A))) \]
 in which each object $!(X)$ carries the structure of a cocommutative comonoid naturally
\[ \top \from^e !(X) \to^\Delta !(A) \ox !(A) \]
such that  $\delta: !(A) \to !(!(A))$ is a morphism of these comonoids.   

In a SMC with finite products and terminal object $\top$, a coalgebra comodality has {\bf Seely isomorphisms} \cite{Bie95, BCS15} 
if the following natural transformations are isomorphisms:
\[ s_\top:  !\top \to^{\tricounit{0.65}} I ~~~~~~~~  s_\ox: !(A \x B) \to^{\Delta} !(A \times B) \ox !(A \times B) \to^{!(\pi_0) \ox !(\pi_1)} !A \ox !B \]
Hence, $!\top \simeq I$ and $!(A \times B) \simeq !A \ox !B$.

In a SMC with finite products and terminal object, a coalgebra modality is monoidal if \cite{Bie95} and only if \cite{BCS15} it has the Seely isomorphisms:
Often it is easier to exhibit Seely isomorphisms than to exhibit the monoidal structure directly.

 In the category of sets and relations, ${\sf Rel}$, it is well-known that the free exponential modality on a set $X$ 
 is $!(X) := {\sf Bag}(X)$, that is, 
 the set of all multisets of $X$ (a multiset or a bag is a set which allows multiple 
 instances of elements).  Thus, in ${\sf FRel}$ we expect the 
 web of the free exponential modality to be ${\sf Bag}(X)$ the delicacy is define the finiteness structure.  This was described in \cite{Ehr05} as:
\[ F(!(X)) := \{ S \subseteq {\sf Bag}(X)| S^\cup \in F(X) \} \]
where $S^\cup$ is the union of all the bags in $S$. Thus, a finitary set of $!(X)$ is a set of bags, 
$S \subseteq {\sf Bag}(X)$, the union of which is a finitary set of $X$.  It is not obvious that this is an orthogonal closed subset:

\begin{lemma}
As defined above $F(!(X))=F(!(X))^{\perp\perp}$.
\end{lemma}

\begin{proof}
Suppose that $S \in F(!(X))^{\perp\perp}$ we aim to show that it follows that $S \in F(!(X))$.   Suppose, for contradiction, that 
$S \notin F(!(X))$ then there is a $w \in F(X)^\perp$ with $w' := S^\cup \cap w$ infinite.  This means there is a function $f: w' \to S$ 
which associates to each $a \in w'$ an $f(a) \in S$ with $a \in f(a)$.   Notice that ${\sf cod}(f)$ is infinite as $f$ is finitely fibred because 
each bag contains only finitely many elements.  

However, ${\sf cod}(f) \in F(!(X))^\perp$ as for each $T \in F(!(X))$ we have ${\sf cod}(f) \cap T$ finite as
\[ f^{-1}({\sf cod}(f) \cap T) \subseteq f^{-1}({\sf cod}(f)) \cap T^\cup \subseteq S^\cup \cap w \cap T^\cup  \subseteq w \cap T^\cup \]
where $w \cap T^\cup$ is finite as $T \in F(!(X))$.  But ${\sf cod}(f) \cap S = {\sf cod}(f)$ is infinite so $S$ is not in $F(!(X))^{\perp\perp}$ and we are done.
\end{proof}

We shall prove following \cite{tasson}, that $!(X)$, so defined, is a free exponential modality.  An exponential modality is {\bf free} in case given any cocommutative comonoid 
\[ \top \from^u X  \to^d X \ox X \]
and map $f: X \to A$ there is a unique homomorphism of comonoids $f^\flat: X \to !(A)$ such that 
$f^\flat \varepsilon = f: X \to A$, that is, the following diagrams commute:
\[ \xymatrix{ & X \ar@{..>}[d]_{f^\flat} \ar[dl]_{f} \ar[r]^d & X \ox X \ar@{..>}[d]^{f^\flat \ox f^\flat}   & X \ar@{..>}[d]_{f^\flat} \ar[dr]^u \\
                   A & !(A) \ar[l]^{\varepsilon} \ar[r]_{\Delta} & !(A) \ox !(A)  & !(A) \ar[r]_e & \top} \]
Recall that this property also provides the coherences required of an exponential modality by:
\[ \xymatrix{!(A) \ar[dr]^{\varepsilon_A f} \ar[d]_{!(f)} \\ !(B) \ar[r]_{\varepsilon_B} & B} 
~~~~~ \xymatrix{ !(A) \ar@{=}[dr] \ar[d]_\delta \\ !(!(A)) \ar[r]_{\varepsilon_{!(A)}} & !(A)} 
~~~~~  \xymatrix{!(A) \ox !(B) \ar[dr]^{\varepsilon_A \ox \varepsilon_B} \ar[d]_{m_\ox}\\ !(A \ox B) \ar[r]_{\varepsilon_{A \ox B}} & A \ox B } 
~~~~~ \xymatrix{ \top \ar@{=}[dr] \ar[d]_{m_\top} \\ !(\top) \ar[r]_{\varepsilon_\top} & \top} \] 
The map is the monoidal map and is defined using the cocommutative monoid with comultiplication
\[ !(A) \ox !(B) \to^{\Delta \ox \Delta} !(A) \ox !(A) \ox !(B) \ox !(B) \to^{1 \ox c_\ox \ox 1} !(A) \ox !(B) \ox !(A) \ox !(B) \]
and unit with comultiplication $\top \to^{u_\ox^{_1}} \top \ox \top$.

Thus, once one establishes that a modality has this universal property all the required coherence properties are automatic. 
The fact that the exponential modalities are free in ${\sf FRel}$ is proven in \cite{tasson}:

\begin{lemma}
The exponential modality in ${\sf FRel}$ is free.
\end{lemma}

\begin{proof}
From ${\sf Rel}$ we know that: 
\[ f^\flat :=\{ (x,\mu) | \exists n \text{ such that } x d^n (x_1,...,x_n), ~ \mu = (a_1,..,a_n), \forall i \leq n ~ x_i f a_i \}\] 
where $d^n: X \to X^{\ox^n}$, is the unique relation making the diagram commute.  We must show that $f^\flat$ is a finiteness relation.

Towards this end consider $f^\flat \triangleleft\{ (a_1,...,a_n) \}$ this the same as $d^n (f \ox ... \ox f) \triangleleft\{ (a_1,...,a_n) \}$ which is a cofinitary set of $X$ as $d^n$ and $f$ are finiteness relations.
  
It remain to show that finitary relations are preserved by $f^\flat$. Let $W \in F(X)$ then we must show $W \triangleright f^\flat \in F(!(A))$.  But $W \triangleright f^\flat \in F(!(A))$ if and only if $(W \triangleright f^\flat)^\cup \in F(A)$.  But $a \in (W \triangleright f^\flat)^\cup$ if and only if there is a $w \in W$ with $(w,(a_1,...,a_n)) \in f^\flat$ and $a=a_i$.  Further decomposing this gives  $(w,(a_1, ...,a_n)) \in f^\flat$ if and only if $(w, (x_1,...,x_n)) \in d^n$ and $(x_i, a) \in f$.  As $d$ is commutative we may assume $(x_1,a) \in f$ and 
also, using associativity, that $(w,(x_1,x_2)) \in d$ and $(x_1,a) \in f$.   But that makes $(W \triangleright f^\flat)^\cup = (W \triangleright f) \cup \pi_0(W \triangleright d (f \ox f))$ which is 
a finitary set as required.
\end{proof}
 
Having free exponentials is a strong property but is certainly not unique to ${\sf FRel}$ (for example see \cite{CEPT}).   Our next task is to show that ${\sf FMat}(R)$ also has 
free exponentials.   It is useful to first observe that, in $\FRel$ there is a always transformation from the exponential 
modality to the modality of linear monoid (see Sec \ref{Sec: ! linear monoid}) because $!A$ is a comonoid:
\[   \xymatrix{& & !(X) \ar[rr]^\Delta \ar[dll]_{\varepsilon_X} \ar@{.>}[d]_{\gamma_!} & & !(X) \ox !(X) \ar[rr]^{\sf mix} & & !(X) \oa !(X) \ar[d]^{\gamma_! \oa \gamma_!} \\
     X & &  !_w(X) \ar[ll]^{\pi_1} \ar[rrrr]_{\Delta'}  & & & &  !_w(X) \oa !_w(X)  } \]
Note that $\gamma_!$ is a bijic map as it underlies to the identity in ${\sf Rel}$.

\begin{theorem}
 ${\sf FMat}(R)$ has free exponential modalities.
 \end{theorem}
 
 \begin{proof}
 First we note the object $?(A)$ is a $\oa$-monoid as where the monoid maps $\nabla: ?(A) \oa ?(A )\to ?(A)$ and $\bot \to^u ?(A)$ are given by the characteristic maps of those in ${\sf FRel}$.  This means there is a linear transformation given by the universal map $?(A) \to^{\gamma_?} ?(A)$ induced by preceding $\nabla: ?A \oa ?A \to ?A$ by the mixor.  
Now suppose we are given $f: A \to X$ and a par commutative monoid $m: X \oa X \to X \from \bot: e$ then we have:
 \[  \xymatrix{?_w(A) \ox ?_w(A) \ar[d]_{\gamma_? \ox \gamma_?}  \ar[rr]^{\nabla'} & & ?_w(A) \ar[d]^{\gamma_?} & & A \ar[ll]_{\sigma_1}  \ar[dll]^\eta \ar[ddll]^f \\
                     ?(A) \ox ?(A) \ar@{..>}[d]_{\beta \ox \beta} \ar[r]_{\sf mix} & ?(A) \oa ?(A) \ar@{..>}[d]_{\beta\oa\beta} \ar[r]_{\nabla} & ?(A) \ar@{..>}[d]^\beta \\
                     X \ox X \ar[r]_{\sf mix} & X \oa X \ar[r]_m & X} \]
which shows, as $\gamma_?$ and the mixor are bijective, that if there is a map $\beta$ making the bottom right square commute, it must certainly be unique.

Now $\gamma_?\beta$ exists by the universal property of $?_w$ and $\gamma_?$ is the identity matrix (although significantly the finiteness topology is weakened).  So it suffices to 
show that $\gamma_?\beta$ induces a map $?(A) \to X$: this amounts to showing that the support of $\gamma_?\beta$ is a finitary set of $?(A)$.  However, by translating the diagram
into ${\sf FRel}$, where such a $\beta$ exists, shows that the support is certainly finitary for $?(A)$.  Thus, $\beta$ is well defined in ${\sf Mat}(R)$.
 \end{proof}

 In \cite{Ehr05} the exponential modality of ${\sf FMat}(R)$ is approached by providing an explicit description of $!(R)$ for a finiteness matrix $R: X \to Y$ in  ${\sf Mat}(R)$.  
 The description is not so simple and, thus, it requires some work to show that the exponential gives a functor.  The explicit construction can be reconstructed from our approach 
 using the behaviour of the maps on the projections $!(A) \to A^{\ox^n}\backslash n!$:
 \[  \xymatrix{!(A) \ar[d]^{\pi_n} \ar[rr]^{!(R)}  && !(B) \ar[d]^{\pi_n} \\
      A^{\ox^n}\backslash n! \ar[rr]_{R^{\ox^n}\backslash n!} \ar@{>->}[d]_{\nu_n} & & B^{\ox^n} \backslash n! \ar@{>->}[d]^{\nu}  \\
      A^{\ox^n} \ar[rr]_{R^{\ox^n}} && B^{\ox^n} }  \]
The formula using the notation in \cite{Ehr05}:
\[ !(R)_{u,v} = \sum_{\sigma \in L(u,v)} \left[ \begin{matrix} u \\ \sigma  \end{matrix} \right] R^\sigma \]
where $\sigma \in L(u,v) \subseteq {\sf Bag}(A \x B)$ is such that $v(a) = \sum_{b \in B} \sigma_{a,b}$ and $u(b) = \sum_{a \in A} \sigma_{a,b}$ (here we are viewing bags as finitely supported maps to the natural numbers), $R^\sigma := \prod_{(a,b) \in A \x B} R_{a,b}^{\sigma_{a,b}}$ and $\left[ \begin{matrix} v \\ \sigma  \end{matrix} \right]  := 
\prod_{b \in B} v(b)!/\prod_{a \in A,b \in B} \sigma_{a,b}!$.  

\medskip

For bags of length $n$ this is recaptured (when the rig has division by natural numbers) 
as $\nu R^{\ox^n} \nu^\Box$.  It is interesting to note that in \cite{Ehr05} 
the functor was described as  $!(R)_{u,v} = \sum_{\sigma \in L(u,v)} \left[ \begin{matrix} v \\ 
	\sigma  \end{matrix} \right] R^\sigma$ (note the partition counting term is now the codomain).  
	This form of the functor is a result of using the other splitting  (see Remark \ref{alternate-splitting})
	 which is available when one assumes (as was the case in that paper) that the rig $R$ is a field. 
	  In particular, this means, when $R$ has inverses of natural numbers, that there is a non-trivial 
	  automorphism $\Box: !(A) \to !(A)$ which shifts the combinatorial weighting.

\subsection{The free exponential modalities in Chu spaces}

\iffalse
We saw that if a symmetric monoidal closed category, $\X$, admits a 
cofree cocommutative comonoid $!A$ for every object $A$, then $\X$ 
has a free exponential modality \cite{Laf88}. Barr applied this result to prove that Chu spaces, 
at times, are equipped with free exponential modalities, when the 
parent category has free exponential modalities. We review this example of Barr's \cite{Bar91} 
in this section.

A directed set $(S, \leq)$ is a perorder $(S, \leq)$ in which for each pair of elements  $a$ and $b$ in the set, there exists an 
element $c$ such that $a \leq b$, and $a \leq c$.  A {\bf directed category} is a category in which every finite diagram 
has a colimit. Directed categories, a.k.a filtered categories, are categorification of directed sets : in a directed 
category there exists an `upper bound' for every pair of objects (binary coproduct) and for every pair of maps 
(coequalizer).  A {\bf directed colimit} is a colimit of a functor $D: I \to \C$ where $I$ is a directed category.

\begin{lemma}
A category is directed if and only the colimits over the category commute with finite limits over Sets. 
\end{lemma}

An object $X$ in a category $\X$ is finitely presented if the functor ${\sf Hom}(X, -): \X \to {\sf Set}$ 
preserves filtered colimits in $\X$.  A category $\C$ is locally finitely presentable if it has all small colimits, 
the subcategory $\C_{fp}$ of finitely presentable objects is small, 
and every  object in $\X$ is presented as a filtered colimit of a diagram over $\C_{fp}$. 
This means that any object in the category $X$ can be obtained by `glueing' together adequate 
finitely presented objects (generators) while taking into account the maps between those objects. 
The term {\em locally presentable} refers to the fact that a category is presented in terms of generating 
objects and maps. 

Suppose $\X$ is symmetric monoidal closed category with a factorization system $(\mathcal{E}, \mathcal{M})$ which  
is compatible with the internal hom, that is, if $f: V \to V' \in \mathcal{M}$ and $g: W \to W' \in \mathcal{E}$, then 
$f \lollipop g \in \mathcal{M}: V \lollipop W' \to V' \lollipop W$. An object $D \in \X$ is {\bf internal cogenerator} with 
respect to $(\mathcal{E}, \mathcal{M})$ if for all objects $V \in \X$, the natural transformation $ V \to V^{**}$ is in $\mathcal{M}$, 
where $V^* := V \lollipop I$. Cogenerator of a monoidal closed category $\X$ is an object $D$ such that 
for any two parallel morphisms, $f,g: A \to B$, if for all $x: B \to X$, fx = gx then $f =g$, in other words, ${\sf Hom}(-, D): \C^\op 
\to {\sf Set}$ is faithful. For an internal cogenerator, the external hom functor is replaced by the interal Hom, $( -  \lollipop D)$.  
\fi 

Barr proved that: 
\begin{theorem} \cite[Theorem 4.8]{Bar91}
If $\X$ is a symmetric monoidal closed category that is locally presentable and $D$ is any object in the category, then 
$\Chu_\X(D)$ is a model of linear logic with free exponential modalities. 
\end{theorem}

The category of vector spaces over complex numbers and linear maps, ${\sf Vec}(\C)$, is a symmetric monoidal closed category 
which is locally presentable. Notably,  $\Chus_{{\sf Vec}_\C}(\C)$ is a $\dagger$-isomix category, see Section \ref{Sec: CHU MUC}. 
We conjecture that the free exponential modalities of  $\Chus_{{\sf Vec}_\C}(\C)$ are $\dagger$-exponentials, 
in other words,  $\Chus_{{\sf Vec}_\C}(\C)$ is a $(!,?)$-$\dagger$-isomix category . 

\section{The free infinite linear monoids in ${\sf FRel}$ and ${\sf FMat}(R)$} 
\label{Sec: ! linear monoid}

In this section, we describe a construction for free infinite linear monoids 
in isomix categories with certain limits and colimits using a standard 
construction for {\em exponential modalities} (see Section \ref{Sec: exp modalities}) in LDCs.

A standard way to attempt to build universal exponential modalities in an LDC is to use the following formulae (due to Michael Barr \cite{barr}):
\[  !_w(A) := \prod_{n \in \N} A^{\ox^n}\backslash n! ~~~~\mbox{and}~~~~ ?_w(A) := \coprod_{n \in \N} A^{\oa^n}/n!.\]
Here, $A^{\ox^n} \backslash n!$ indicates the equalizer of all the $n!$ permutations of $A^{\ox^n}$.  
Dually $A^{\oa^n}/n!$ indicates the coequalizer of all the $n_!$ permutations of $A^{\ox^n}$. 
In the category of vector spaces over a field $K$, $!_w(V)$ is the free symmetric algebra, $\bigoplus_{n \in \N} V^{\ox^n}$ 
over a vector space $V$.

For $!_w$ and $?_w$ to be exponential modalities, along with the other conditions, we require 
for all objects $A$ in the LDC, $!_w(A)$ to be a cofree $\ox$-coalgebra, and $?_w(A)$ to 
be a free $\oa$-algebra. In order to define $\Delta_A: !_w(A) \to !_w(A) \ox !_w(A) $ and 
$\nabla_A: ?_w(A) \oa ?_w(A) \to ?_w(A)$ it is required that $\ox$ distributes 
over the product, that is $X \ox \prod_{n \in \N} A^{\ox^n} = \prod_{n \in \N} X \ox A^{\ox^n}$, 
and $\oa$ over the coproduct respectively -- while the usual natural distributions (i.e. those which are guaranteed) 
are of the tensor over the coproduct and the par over the product. This construction --  and when it fails to produce an exponential modality -- is discussed in detail in \cite{MTT18}.   

Nonetheless, it does sometimes happen that the unexpected distributions do hold: 
that is, the tensor distributes over (countable) products and the par over (countable) coproducts.   
For example, in ${\sf Rel}$, where tensor is the same as par, and is inherited from the product in sets, 
the distribution over the product and coproduct is inherited from the distribution of products over 
coproducts in sets, this also works in suplattice (see \cite{barr}).  However, even given the close 
relation of ${\sf FRel}$ to ${\sf Rel}$, this construction does {\em not\/} work in ${\sf FRel}$.  
However, it does provide a modallity (i.e. linear functor) which produces natural $\ox$-monoids 
(and natural $\oa$-comonoids) which furthermore have a universal property.   This, in particular, 
provides us with a ready source of infinite linear monoids. 
in ${\sf FRel}$ and ${\sf Mat}(R)$.   Furthermore, as this is a construction which works in 
{\em all\/}  isomix categories with the appropriate limits and colimits it provides us with a very general 
source of infinite linear monoids which can be exploited.

One aspect of this construction which is immediate clear is that, in the presence of a duality, $!_w(X) \simeq (?_w(X^{*}))^{*}$ as:
\begin{align*}
 (?_w(A^*))^* & := \left(\coprod_{n \in \N} (A^*)^{\oa^n}/n!\right)^*= \prod_{n \in \N} \left( (A^*)^{\oa^n}/n!\right)^* \\
 & = \prod_{n \in \N} \left( (A^*)^{\oa^n}\right)^*\backslash n! =  \prod_{n \in \N} (A^{**})^{\ox^n} \backslash n! \\
 & = \prod_{n \in \N} A^{\ox^n} \backslash n! = !_w(A)
 \end{align*}
 This means, importantly, that $ (!_w,?_w)$ form a linear functor pair provided $!_w(A)$ 
 is a monoidal functor with respect to the tensor (or, equivalently, $?_w(A)$ is a 
 comonoidal functor with respect to par).  However, notice that, as $!_w$ is built using limits,
 it is {\em automatically\/} monoidal 
 and so $(!_w,?_w)$, if it exists, is always a linear functor.

The reason for wanting the distributive law of $\oa$ over the coproduct is that this enables 
one to define a natural multiplication $\nabla_2: ?_w(A) \oa ?_w(A) \to ?_w(A)$ in the absence of a 
distributive law, however, one cannot define such a multiplication.  However, there {\em is\/} a 
multiplication $\nabla'_2: ?_w(A) \ox ?_w(A) \to ?_w(A)$ as we do have the appropriate distributive law 
for tensor, thus, define the multiplication: 
\[  \infer{\left( \coprod_{n \in \N} A^{\oa^n}/n! \right) \ox \left( \coprod_{n \in \N} A^{\oa^n}/n! \right)  \to_{\nabla'_2} \coprod_{n \in \N} A^{\oa^n}/n!
       }{ \infer{\coprod_{n \in \N} \coprod_{i+j =n} (A^{\oa^i}/i!) \ox (A^{\oa^j}/j!) \to \coprod_{n \in \N} A^{\oa^n}/n! 
       }{ \left\{ (A^{\oa^i}/i!) \ox (A^{\oa^j}/j!)  \to^{\sf mix} (A^{\oa^i}/i!) \oa (A^{\oa^j}/j!)  \to A^{\oa^{i+j}}/(i+j)! \to^{\sigma_{i+j}}  \coprod_{n \in \N} A^{\oa^n}/n! \right\}_{n \in \N} }} \]
 The unit for the multiplication is $u:= \sigma_0: \top=\bot \to^u \coprod_{n \in \N} A^{\oa^n}/n!$.  
 
 This is clearly not a free $\ox$-commutative monoid in general but it is a linear functorial
  construction which exists for very general reasons and is present in isomix categories which are 
 complete and cocomplete (and for which colimits distribute over tensor and limits distribute over par
  -- which, of course, is immediate in *-autonomous categories).
 
  Note that the forumalae for ${!}_w$ uses equalizers and countable products, dually, the formulae for 
  $?_w$ uses coequalizers and countable coproducts. In ${\sf FRel}$ and ${\sf FMat}(R)$ 
  not all limits and colimits exist so we have to be a little careful
  in constructing ${!}_w$. We know arbitrary products and coproducts are present, we need only 
  to discuss the formation of the coequalizer  $A^{\oa^n}/n!$ this is the coequalizer 
  \begin{equation} 
	\label{eqn: complex coeq}
 \xymatrix{A^{\oa^n} \ar@{}[rr]|{\vdots} \ar@/_1pc/[rr]_{\rho_{n!}} \ar@/^1pc/[rr]^{\rho_{1}} 
 & & A^{\oa^n} \ar@{->>}[rr]^{\nu_n}  & & A^{\oa^n}/n! } 
  \end{equation}
 where $\rho_i$ the $i^{\rm th}$ permutation of the par power.  When the category is additively enriched (which ${\sf FRel}$ and ${\sf FMat}(R)$ are) -- and when the rig, $R$, admits division by  positive natural numbers -- we may replace this coequalizer by the simpler coequalizer 
 \begin{equation}
	\label{eqn: simple coeq}
 \xymatrix{A^{\oa^n}  \ar@/_1pc/[rr]_{(\sum_{i=1}^{n!} \rho_i)/n!} 
 \ar@{=}@/^1pc/[rr] & & A^{\oa^n} \ar@{->>}[rr]^{\nu_n} & & A^{\oa^n}/n! } 
 \end{equation}
 The ability to divide by $n!$ is necessary for this simplification.  
Note that in ${\sf FRel}$,  $m = 1 \cup ...\cup 1 =1$ so it admits division by positive numbers.  
Generally in ${\sf FMat}(R)$ one needs the rig, $R$, to admits division by natural numbers 
in order to reach this simplification: clearly this is the case for all fields and $\mathbb{Q}$-algebras.
 
 We show that the simpler parallel pair of maps - $1$ , $(\sum_{i=1}^{n!} \rho_i)/n!$ - 
 in formula \ref{eqn: simple coeq} has the same coequalizer as for the maps $\rho_1, \cdots, \rho_{n!}$ 
 in formula \ref{eqn: complex coeq}. 
Suppose $\nu_n$ is the coequalizer of $1$ and $(\sum_{i=1}^{n!} \rho_i)/n!$ in \ref{eqn: simple coeq}, 
that is, $1_{A^{\oa^n}} \nu_n = (\sigma_i^{n!} \rho_i)/n!$, then $\nu_n$ coequalizes 
all pairs of $\rho_i$ and $\rho_j$ in formula \ref{eqn: complex coeq}, in particular, 
we have that for all $i \in n!$, $\rho_i \nu_n = 1_{A^{\oa^n}} \nu_n = \rho_1 \nu_n$.
 \[ \rho_i \nu_n = \rho_i 1_{A^{\oa^n}} \nu^n = \rho_i \left( \sum_{j=1}^{n!} \rho_j/n! \right) \nu_n 
 = \left( \sum_{i=1}^{n!} \rho_i\rho_j/n! \right) \nu_n \stackrel{(*)}{=} \left( \sum_{i=1}^{n!} \rho_i/n! \right) \nu_n 
 = 1_{A^{\oa^n}} \nu_n = \rho_1 \nu_n \]
 The step $(*)$ is true because the set of all permutations form a group. 
 Similarly, if $\lambda:  A^{\oa^n} \to Y$ coequalizes all the permutations $\rho_1, \cdots, \rho_{n!}$,
then $\lambda$ coequalizes $1$ and $(\sum_{i=1}^{n!} \rho_i)/n!$:
 \[ \left( \sum_{i=1}^{n!} \rho_i/n! \right) \lambda = 
 \left( \sum_{i=1}^{n!} \rho_i\lambda \right)/n! = 
 \left( \sum_{i=1}^{n!} \lambda \right)/n! = 
 \left( \sum_{i=    1}^{n!} 1_{A^{\oa^n}} \right)/n! \lambda = 1_{A^{\oa^n}} \lambda \]
 so the coequalizer of the formula in \ref{eqn: simple coeq} is precisely the coequalizer of the permutations.
 
 It is also clear that $(\sum_{i=1}^{n!} \rho_i)/n!$ is an idempotent as:
 \[ \left( \sum_{i=1}^{n!} \rho_i/n!\right)  \left( \sum_{i,j=1}^{n!} \rho_j/n!\right) = 
 \sum_{i=1}^{n!} \sum_{j=1}^{n!} \rho_i\rho_j/(n!)^2 =  
 \sum_{j=1}^{n!} n! \rho_j/(n!)^2 =  \sum_{j=1}^{n!} \rho_j/n! \]
 This means that the coequalizer (and indeed the equalizer) we seek, in this case, 
 is obtained by splitting this idempotent (and such splittings are preserved by all functors).   
 Thus, when we have division by positive natural numbers, we have the coequalizer 
 we seek provided (these) idempotents split.  Even though this is a very mild requirement, 
 recall that in ${\sf FRel}$ not all idempotents split so we must examine this particular 
 idempotent in more detail.
 
 In ${\sf Rel}$ the coequalization of permutations of product powers is given by the surjection 
 $\nu_n: A^n \to {\sf Bag}(A)_n$, where ${\sf Bag}(A)_n$ is the set of bags of $A$ 
 with exactly $n$-elements.   The section is just the converse of this map 
 $\nu_n^\circ: {\sf Bag}(A)_n \to A^n$ (it is monic exactly because 
 it is the converse of an epic!).   In ${\sf SRel}$ on the web the splitting must take the form 
 it has in ${\sf Rel}$, however, we must show that the finiteness structures are compatible
  with this.  We will deal with the par and so we shall argue using the cofinitary sets.
 
 $F(A^{\oa^n})^\perp = \downarrow \{ A_1 \x ... \x A_n \mid A_i \in F(A)^\perp \} = \downarrow \{ A_0^n \mid A_0 \in F(A)^\perp \}$, 
 where the last set dominates the previous one because  $A_1 \x ... \x  A_n \subseteq A_0^n$ where $A^0 := \bigcup_{i=1}^n A_i$.  
 
 We set $F(A^{\oa^n}/n{!})^\perp = \{  X \mid \nu_n \triangleleft X \in F(A^{\oa^n})^\perp \} = \{ X \mid \nu_n \triangleleft X \subset A_0^n, A_0 \in F(A)^\perp \} 
 = \{ X \mid X^\cup \in F(A)^\perp \}$ where by $X^\cup$  we indicate the union of all the elements in the bags in $X \subseteq {\sf Bag}(A)_n$.  
 We must show that this does provide a well-defined finiteness space on ${\sf Bag}(A)_n$.  
 Toward this end note that $\nu_n$ is a map so $\{a_1 \oa...\oa a_n\} \triangleright \nu_n$ 
 is a singleton so a finitary set in ${\sf Bag}(A)_n$.  Furthermore, $\nu_n$ reflects cofinitary sets by definition.  
 On the otherhand, $\nu_\circ$ still has $\{ b \} \triangleright \nu_n^\circ$ finite and, again, by definition preserves cofinitary sets.

This all means that the coequalizer exist in ${\sf FRel}$ and is given by splitting an idempotent.  To transpose this into a coequalizer in ${\sf FMat}(R)$ we take the characteristic function of $\nu_n$, however, for $\nu_n^{\circ}$ we use the matrix (which we denote $\nu^\Box$) with its entries determined by: 
\[ \nu^\Box: {\sf Bag}(A)_n  \x A^n \to R; (y,x) \mapsto  \left\{ \begin{array}{ll} \frac{r_1{!} \cdot ..\cdot r_p{!}}{n!} & \mbox{when $y = \nu_n(x)$ and $y = r_1\cdot a_1 + ...+ r_p \cdot a_p$ }\\ 0 & \mbox{otherwise} \end{array} \right. \]
where this formula takes into account the repetitions in the bag: a repetition reduces the number of sequences corresponding to the bag.  We now have:

\begin{lemma}
In both ${\sf FRel}$ and ${\sf FMat}(R)$, where $R$ is a rig with division by positive natural numbers, the idempotents $(\sum_{i=1}^{n{!}} \rho_i)/n{!} $ on both $A^{\oa^n}$ and $A^{\ox^n}$ split.
\end{lemma}

In fact, the coequalizer $\nu_n: A ^{\oa^n} \to A^{\oa^n}\!\!/n{!}$ exists in ${\sf FMat}(R)$, for an arbitrary rig $R$.  
It, however, will not be given by a splitting of an idempotent in general.   
The point is that the map $\nu_n: A^{\oa^n} \to A^{\oa^n}\!\!/n{!}={\sf Bag}_n(A)$ is always present as the 
characteristic function of $\nu_n$.   Its universal property is given by its finiteness relational universality.  
This means that we have all the components of the proposed construction of the modality $({!}_w, {?}_w)$ in all these categories:

\begin{proposition}
In both ${\sf FRel}$ and ${\sf FMat}(R)$, where $R$ is any rig, the modality $({!}_w,{?}_w)$ described above exists.  Furthermore, it produces 
infinite linear monoids:  thus, for any object $A$, $?_w(A) \whitelin {!}_w(A^*)$ is a linear monoid.
\end{proposition}

The linear functor $({!}_w,{?}_w)$ has a universal property but for commutative ``mixed" $\oa$-monoids and $\ox$-comonoids: suppose $m: X \oa X \to X \from \bot: u$ is a commutative monoid and there is a map $f: A \to X$ then there is a unique map $f^{\oa^r} m_r: A^{\oa^r}/r{!} \to X$ determined by the $n$-fold multiplication $m_r:X^{\oa^r}/r{!} \to X$ this then gives 
a map from the coproduct $f^\sharp: ?_w(A) \to X$ such that:
\[ \xymatrix{ ?_w(A) \ox ?_w(A) \ar[d]_{f^\flat \ox f^\sharp} \ar[rr]^{\nabla'_2} & & ?_w(A) \ar[d]_{f^\sharp}  & & A \ar[ll]_{\sigma_1} \ar[dll]^{f} \\X \ox X \ar[r]_{\sf mix} & X \oa X \ar[r]_m & X} \]
where $f^\flat:= \{ A^{\oa^n}/n! \to^{f^{\oa^n}/n{!}} X^{\oa^n}/n{!} \to^{m_n} X \}_{n \in \mathbb{N}}$ is the comparison map from the coproduct.  The map $f^\flat$  is unique whenever the 
mixor ${\sf mix}: A \ox B \to A \oa B$ is epic -- which,  notably, is so in ${\sf FRel}$ and ${\sf Fmat}(R)$ as it underlies to the identity in ${\sf Rel}$ -- because:
\[ \xymatrix{A^{\oa^p}/p{!} \ox A^{\oa^q}/q{!} \ar[rr]^{\sf mix} \ar[d]_{\mu_p \ox \mu_q} 
                & & A^{\oa^p}/p{!} \oa A^{\oa^q}/q{!} \ar[rr]^{\nabla'_{n,m}}  \ar[d]_{\mu_p \oa \mu_q} 
                & & A^{\oa^{p+q}}/(p+q){!} \ar[d]_{\mu_{p+q}}  \\
                       X \ox X \ar[rr]_{\sf mix} && X \oa X \ar[rr]_m && X} \]
whenever the outer square commutes, because the left square commutes, it follows that the righthand square must commute 
(assuming the mixor is epic).   However, the righthand square is determined by the colimit to have $\mu_r = f^{\oa^r} m_r$ 
so $\{ \mu_r \}_{r \in \mathbb{N}} =f^\flat$.  This gives:

\begin{proposition}
In any isomix category which has products (which distribute over par) and coproducts (which distribute over tensor) and  for which the mixor is bijic (that is both epic and monic) the weak exponential has the universal property that given any commutative monoid $X \oa X \to^m X \from^u \bot$ and $f: A \to X$ there is a unique morphism to the mixed monoid $f^\sharp: ?_w(A) \to X$ such that:
\[ \xymatrix{ ?_w(A) \ox ?_w(A) \ar@{..>}[d]_{f^\sharp \ox f^\sharp} \ar[rr]^{\nabla'_2} & & ?_w(A) \ar@{..>}[d]_{f^\sharp}  & & A \ar[ll]_{\sigma_1} \ar[dll]^{f} \\X \ox X \ar[r]_{\sf mix} & X \oa X \ar[r]_m & X} ~~~~~\xymatrix{\bot \ar[rr]^{\nabla'_0} \ar[rrd]_u & &?_w(A) \ar@{..>}[d]^{f^\sharp} \\ & & X} \]
Dually, given a cocommutative comonoid $Y \ox Y \from^d Y \to^e \top$  and map $g:Y \to B$ there is a unique morphism form the mixed comonoid $f^\flat: Y \to !_w(B)$ such that 
$f^\flat d ~{\sf mix} = \Delta' (f^\flat \oa f^\flat)$.
\end{proposition}

\begin{remark} \label{alternate-splitting}
In a ${\sf FMat}(R)$, in which $R$ has division by natural numbers, there is a second canonical way to split the idempotent as $\omega^\Box_n: A^{\oa^n} \to A^{\oa^n}\!\!/n!$ 
with the section $\omega:A^{\oa^n}\!\!/n{!} \to A^{\oa^n}$ being the characteristic function.  Thus, 
\[ \omega^\Box_n([a_1,..,a_n],r_1\cdot a_{r_1}+...+ r_p \cdot a_r = \lbag a_1,..,a_n\rbag = r_1{!}...f_p!/n{!}. \]
This alternate splitting shifts the weights from the section onto the retraction.  The splitting given by $\nu_n$ and $\nu^\Box_n$ has the advantage that the map $\nu_n$ exists 
for every rig $R$ -- this is why we can form the weak exponential for all rigs.
\end{remark}




\chapter{Summary}
\label{Chap: part 2 summary}

The second part of this thesis focused on formulating the structures fundamental to CQM, 
namely completely positive maps, $\dagger$-Frobenius algebras, and complementary bialgebras, in MUCs. 
Recall that a MUC, $M: \U \to \C$, is given by a $\dagger$-isomix functor 
$M$ from a unitary category $\U$ into a $\dagger$-isomix category $\C$. Moreover, $M$ factors through the core of $\C$. 

Chapter \ref{Chap: positivity} generalized Coecke and Heunen's $\CP^\infty$ construction \cite{CoH16} 
on $\dagger$-SMCs to MUCs. In a MUC setting, the auxiliary wire for Kraus maps must be unitary. 
Similar to the $\CP^\infty$ construction on $\dagger$-SMCs, we characterize the $\CP^\infty$ construction 
on MUCs using environment structures. Unlike in $\dagger$-SMCs, an environment structure for 
MUC requires discarding maps only within the unitary core. A MUC which has an environment structure with purification is
isomorphic to  the $\CP^\infty$ category of some other MUC. Our $\CP^\infty$ construction on a MUC 
produces an isomix category. The construction preserves dagger when the base category 
has unitary duals. 

Chapter \ref{Chapter: complementarity} discussed  modelling measurement 
and complementarity in MUCs using structures developed in Chapter \ref{Chapter: compaction}. 
In $\dagger$-SMCs, Coecke and Pavlovic \cite{CoP07} described a demolition measurement 
as a map $f: A \to X$ where $X$ is a special commutative $\dagger$-Frobenius algebra 
and $f^\dag f = 1_X$. Coecke and Pavlovic's measurement formula cannot be directly 
applied in a MUC setting because in a $\dagger$-isomix category an 
object $A$ is not always isomorphic to $A^\dagger$ except in the unitary core. Hence, in a MUC setting, 
measuement is done in two steps: an object is first {\em compacted} into the unitary core 
followed by a demolition measurement.  Interestingly, compaction produces a coring $\dagger$-binary 
idempotent, and a coring $\dagger$-binary idempotent, when split, produces an object in the 
canonical unitary core.  The details are discussed in the Section \ref{Sec: compaction}. 
 
A component of Coecke and Pavlovic's measurement forumla  
is a special commutative $\dagger$-Frobenius algebra ($\dagger$-FA).  In LDCs, FAs 
are generalized by linear monoids containing a $\ox$-monoid, say on $A$, and a dual 
$\oa$-comonoid on $B$.  Note that, the monoid and the comonoid are not usually 
defined on the same object which differentiates linear monoids from Frobenius algebras. 
In a compact setting, linear monoids on isomorphic objects collapse to  FAs when the isomorphism 
satisfy a certain condition. We show that these results extend to a $\dagger$ settings too, see 
Section \ref{Section: linear monoids}. 

Even though expected, it is suprising that the linear monoids lead to the notion of  
linear comonoids consists of a $\ox$-comonoid and a $\oa$-monoid, introduced in 
Section \ref{Sec: linear comonoid}. Exponential modalities 
for LDCs is a significant source of examples for linear comonoids.  In an LDC, linear monoids 
and linear comonoids interact bialgebraically to produce a pair of bialgebras - one 
on the tensor and the other one on the par product. Linear bialgebras 
are fundamental to complementary sytems which are discussed in Section \ref{Sec: linear bialgebra}. 
When a self-linear bialgebra satisfies certain conditions, its $\ox$ and $\oa$ bialgebra are Hopf algebras 
with a certain antipode; such self-linear bialgebras are referred to as complementary systems, see Section
\ref{Sec: complementary systems}. 

In the presence of the free exponential modalities $({!},{?})$, every complementary system 
in an isomix category induces a linear bialgebra on the free exponential modalities - the 
linear comonoid of the induced $({!},{?})$-linear bialgebra is 
given by the free canonical cocommutative $\ox$-comonoid of the $!$. The universal property of the 
free $({!},{?})$ induces a binary idempotent on the $({!},{?})$-linear bialgebra which splits to produce the 
original complementary system. Thus, {\em every complementary system arises 
as a splitting of a binary idempotent on the linear bialgebra induced on the 
free exponential modalities}, see Section \ref{Sec: exp modalities}. 
The results extend directly to $\dagger$-complementary systems in $\dagger$-isomix 
categories with a free $(!,?)$. $\Rel$, $\FRel$, $\FMat(R)$ where $R$ is commutative rig and certain Chu catgories are 
examples of $\dagger$-isomix categories with free exponential modalities, see Section \ref{Section: exp mod examples}. 

