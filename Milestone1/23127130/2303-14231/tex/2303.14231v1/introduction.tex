% !TEX root = thesis.tex

\chapter{Introduction}

The programme of Categorical Quantum Mechanics (CQM) \cite{CoK17,HeV19} initiated by 
Abramsky and Coecke's seminal paper \cite{AC04} employs a graphical calculus 
to study quantum processes within the $\dag$-compact closed category ($\dag$-KCC)
of finite dimensional Hilbert spaces ($\FHilb$). From the perspective of logic, 
the graphical calculus is the proof theory of a compact fragment of multiplicative $\dagger$-linear logic \cite{Dun06}.
Thus, CQM provides a novel approach to quantum mechanics with its use of a graphical calculus backed by the 
rigor of Categorical Proof Theory.

 A well-known limitation of compact closed categories is that they model finite dimensional 
 Hilbert spaces, but they do not model infinite dimensional spaces \cite{Heunen16}.  
A categorical generalization of compact closed categories, in which infinite dimensional spaces 
 can be modelled are $*$-autonomous categories. 
 This can be taken a step further by generalizing to linearly 
 distributive categories (LDCs) in which the existence of dual objects is not assumed. 
 These linear settings come with a proof theory (for non-compact multiplicative linear logic) which is
 a graphical calculus. Thus, one does not abandon this attractive feature of CQM in these more general settings. In fact, 
 the graphical calculus of linearly distributive and $*$-autonomous categories subsume the 
 graphical calculus of compact closed categories. Thus, from a categorical perspective, an inviting direction to  
 accommodate infinite dimensional  quantum processes is to utilize $*$-autonomous and 
 linearly distributive categories as extensions of the compact closed settings used in CQM. The advantage 
of such a foundational approach is that it encompasses many of the earlier approaches and allows for a structural 
positioning of the different approaches.

 The aim of this thesis is to lay the foundations of generalization in this direction. Towards this aim, 
the first part of this thesis develops the categorical proof theory of non-compact $\dagger$-linear logic 
using linearly distributive and $*$-autonomous categories, and shows that one can always 
extract the usual settings for compact $\dagger$-linear logic from this new framework 
of $\dagger$-isomix categories. The first part also describes models for the new 
framework. The second part of this thesis explores the applicability of $\dagger$-isomix 
categories to CQM and studying quantum processes of arbitrary dimensions. 

%how is the dagger functor expressed in the $*$-autonomous and the LDCs, and how the key structures of CQM namely 
%completely positive maps, $\dagger$-Frobenius algebras, strongly complementary systems 
%generalize in these settings. 
%In this light, this thesis is divided into two parts with the first part focusing on the former question and 
%the second part focusing on the latter. 


\section{From linear logic to quantum mechanics}

\subsection{Linear logic}
Linear logic was introduced by Girard in \cite{Gir87}  as a logic of resources manipulation. Unlike classical 
logic which treats logical statements as truth values, linear logic treats logical statements as resources 
which cannot be duplicated or destroyed.  For example, consider the following statements, $p$ and $q$:
\begin{align*}
    p &:  \text{to spend a dollar} \\
    q &: \text{to buy an apple} 
\end{align*}

Then, in linear logic, the compound statement ``$p \Rightarrow q$" has the meaning that if a dollar is spent then an apple 
can be bought. This means that a person can either have {\em a dollar} or {\em an apple} at a given time but not both. 
The word ``linear" refers to this resource sensitivity of the 
logic: thus, a proof of a statement in linear logic may be regarded as a series of resource transformations.

The types (formulae) of linear logic can be defined inductively as follows (``$|$" is to be read as ``or''):
\begin{align*}
A :=&  ~p ~|~ A^{\perp} \\
&|~ A \ox  A ~|~ 1 ~|~ A \parr A ~|~ \bot \\ 
&|~ A \with A ~|~ \top ~|~ A  \oa A ~|~ 0 \\
&|~ !A ~|~ ?A     
\end{align*}

In the inductive definition of $A$, $p$ is an atomic formula such as 
``to spend a dollar" and $p^\perp$ is the negation of the formula which in this case will be ``to 
receive a dollar".  The {\bf negation} is an involution, hence, $(p^\perp)^\perp = p$.

The connectives $\ox$, $1$, $\parr$, and $\bot$ are called the {\bf multiplicatives}. A statement, 
$A \otimes B$ (read as A {\em tensor} B) 
allows the resources $A$ and $B$ to be available at the same time.  For example, consider the statement $r : $ {\em an orange}. Then, $p \Rightarrow q \otimes r$ refers to the fact that 
spending a dollar buys an apple and an orange at the same time. The formula $1$ is the multiplicative truth, hence, $A \otimes 1 = 1 \otimes A = A$. The connective, $\parr$, read as {\em par}, is the multiplicative disjunction and is dual to tensor. This means that,
$(A^\perp \otimes B^\perp)^\perp = A \parr B$. Smililary, $1$ and $\top$ are duals to one another $(1^\perp = \top)$.

The connectives $\with$, $\top$, $\oa$, and $0$ are called the {\bf additives}. The statement, $A \with B$ 
(read as $A$ {\em with} B), means that either $A$ or $B$ is available at a time.  For example, the statement, $p \Rightarrow q \with r$, refers to the fact that 
spending a dollar buys either an apple or an orange with the freedom to choose either. In computer science $\with$, 
represents non-determinism. The formula $0$ is the additive false: $ A \parr 0 = 0 \parr A = A$.
The additive disjunction $\parr$ (read as the {\em plus}) and $\with$ are dual to one another. Similary, 
 $0$ and $\bot$ are duals. 

Finally, the {\bf exponential operators} - $!$ read as {\em of course} or {\em bang} and $?$ read as {\em why not} or {\em whimper} - allow 
for the duplication and destruction of resources. For any resource $A$, $!A$ models an {\em unlimited} store from which the resource $A$ 
can be extracted 0 or more times. Morever, the storage itself can be duplicated or even destroyed.  The {\em why not} operator, $?$ encodes the notion of infinite demand, and is dual to the $!$, 
that is, $(?A)^\perp = !(A^\perp)$. 

Interested readers can refer to \cite{Gir95, Gir87} for more details on linear logic. 

\iffalse
\subsection{ Linearly distributive categories and monoidal categories}

TODO: Write here why we are interested in categorical semantics of linear logic

Linearly distributive categories (LDCs) \cite{CS97} provide a categorical semantics 
for the multiplicative fragment of linear logic (MLL) which include the multiplicative conjunction, disjunction, 
truth and false. LDCs come equipped with two distinct tensor products called the ``tensor'', $\ox$, and the ``par'', $\oplus$ 
corresponding to the multiplicative conjunction and disjunction of linear logic respectively. 
 In the linear logic community the par is often denoted by $\parr$, however
we follow the convention in \cite{CS97} and use $\oplus$. Under the same convention, we write the unit of  $\ox$ 
as top $\top$ rather than $1$ as in the linear logic community. 
The unit of $\oa$ is a bottom, $\bot$. The tensor and the par are related by a left linear distributor, 
$\partial^L: (A \oa B) \ox C \to A \oa (B \ox C) $, and a right linear distributor, $ \partial^R: A \ox (B \oa C) 
\to (A \ox B) \oa C$,  which are natural in all three arguments. Note that the distributors are not isomorphisms. 

In an LDC, a dual of an object, if it exists, represents its negation.  An LDC in which every object has a dual is a 
$*$-autonomous category \cite{Bar91}. 

LDCs which come with a map from the multiplicative false, to the multiplicative truth, called the `mix map', $\m: \bot \to \top$, satisfying 
the ``mix law" \cite{BCS00} are knows as {\bf mix categories}. The mix law provides a natural transformation, 
$\mx: A \ox B \to A \oa B$, from the multiplicative conjunction to the multiplicative disjunction, called the {\bf mixor}.  
An object $U$ for which the natural transformations, $\mx_{(-, U)}$ and $\mx_{(U, -)}$ are isomorphisms
(that is for any object $A$, $A \ox U \simeq A \oa U$ and $U \ox A \simeq U \oa A$) is said to be in the {\bf core} 
of the mix category. The core of a mix category determined by such objects forms a full sub category.  

An {\bf isomix category} is a mix category with the mix map, $\m$, being an isomorphism. 
 An isomix category in which the mixor is a natural isomorphism ($\ox \simeq \oa$) is called a {\bf compact} LDC. 
 The core an isomix category is a compact LDC. Compact LDCs are linearly equivalent to a monoidal categories.  
 Conversely, monoidal categories \cite{Mac13} can be viewed as being degenerate or 
 compact LDCs in which the mixor is the identity map ($\ox = \oa$).  Thus, monoidal categories provide a categorical semantics of 
 a compact version linear logic in which the two multiplicative connectives of the logic coincide. A monoidal category in which every 
 object has a dual, is a compact closed category \cite{Kel64}. 

This sub section will be expanded with relevant details in Section \ref{Sec: LDC}.

\fi

\subsection{Categorical semantics for linear logic}

Linear logic, being a logic of resources, emphasizes the structure of proofs rather than provability, that is, one is more 
interested to know how a statement can be proved, rather than, merely if the statement is provable. Proofs in sequent calculus 
often contain extraneous details due the sequential nature of the calculus. For example, in a sequent proof, in order to apply a set of rules 
to a sequent, one must choose an order for the application of the rules, even if the rules are independent (that is, the order in which 
the rules are applied does not affect the final result). 

In order to remove such unuseful information,  Girard \cite{Gir87} introduced proof-nets to 
represent proofs of linear logic, specifically for the multiplicative fragment without units 
i.e., only the $\ox$ and  $\parr$. Proof-nets are formalized as circuits. The study of algorithms to decide if a given circuit corresponds to a valid 
proof-net paved the way to the study of the categorical semantics for multiplicative linear logic (MLL). 
An overview of the categorical proof theories for different fragments of MLL is provided in the table below: 

  \begin{table}[h]
    \centering
	\begin{tabular}{ | l | l | }
    \hline
	 {\bf Linear logic fragment} & {\bf Categorical proof theory}  \\  
   \hline 
   \hline
	MLL & Linearly distributive categories  \cite{CS97} \\  
  \hline
	MLL with negation  & $*$-autonomous categories  \cite{Bar91b} \\ 
  \hline
	Compact MLL ($\ox = \parr$, $1 = \bot$) & Monoidal categories \cite{Kel64} \\ 
  \hline
	Compact MLL with negation & Compact closed categories \cite{Kel80} \\ 
  \hline
	\end{tabular}
  \caption{Categorical semantics for multiplicative linear logic}
  \label{Table: MLL} 
\end{table}

\FloatBarrier

The above listed semantics are sound and complete in the sense that  
there exists a one-to-one correspondence between the proof-nets of a fragment and the morphisms 
of its corresponding categorical setting.  The proof-nets provide a graphical calculus to these categorical settings, 
thereby, enabling a string diagrammatic presentation of the morphisms in these categories. 

The proof theory of compact MLL based on the graphical calculus 
of monoidal categories \cite{Sel10}  is used to derive an elegant description 
of quantum mechanics \cite{CoK17, HeV19} .

\subsection{Categorical quantum mechanics (CQM)}

Traditionally, Hilbert spaces \cite{NiC10} or more generally von Neumann Algebras \cite{Dix11} are used as the mathematical 
framework of quantum mechanics \cite{Neu18} . While these frameworks support detailed computation, they 
do not support an intuition for the problem: this leads to an approach described as ``shut up and calculate'' \cite{Mer89}. 
The programme of Categorical Quantum Mechanics (CQM) \cite{AC04, CoK17, HeV19}  emerged out of the desire 
to develop a more intuitive framework to represent and reason about quantum processes.

Linear logic captures \cite{Pra93, AbD04, AC04, Dun06, Bol19} the essence of quantum mechanics owing to its 
resource-sensitive character. In particular, linear logic does not allow duplication of an arbitrary type: in quantum mechanics, this is 
referred to as the {\em no-cloning theorem} \cite{NiC10} which states that it is impossible to duplicate an arbitrary quantum state. 
Motivated by this connection, CQM uses the compact multiplicative linear logic as the base framework for its purpose, 
and added the notion of `dagger' to this fragment, thus giving rise to compact $\dagger$-linear logic.  The `dagger' abstracts the notion of `adjoint' 
which is crucial to quantum mechanics: measurable properties of a quantum system are 
given by self-adjoint operators on a separable Hilbert space, that is, a Hilbert space with 
countable orthonormal basis.  

While monoidal categories provide the semantics for compact MLL, $\dagger$-monoidal categories provide the categorical semantics for 
compact $\dagger$-linear logic.  CQM uses $\dagger$-monoidal categories,  specifically, $\dagger$-compact closed 
categories (See Chapter \ref{Chap: CQM}) to develop a high-level, intuitive, formal language for quantum mechanics 
by abstracting the standard, traditionally-used, analytical framework of Hilbert spaces, as illustrated in Figure \ref{Fig: layers}.

\begin{figure}[h]
\centering
	\begin{tikzpicture}[scale=1.25]
	\begin{pgfonlayer}{nodelayer}
		\node [style=none] (0) at (-4, -2) {};
		\node [style=none] (1) at (-1, 0) {};
		\node [style=none] (2) at (4, 0) {};
		\node [style=none] (3) at (1, -2) {};
		\node [style=none] (4) at (-4, -1) {};
		\node [style=none] (5) at (-1, 1) {};
		\node [style=none] (6) at (4, 1) {};
		\node [style=none] (7) at (1, -1) {};
		\node [style=none] (8) at (-4, 0) {};
		\node [style=none] (9) at (-1, 2) {};
		\node [style=none] (10) at (4, 2) {};
		\node [style=none] (11) at (1, 0) {};
		\node [style=none, scale=1.25] (12) at (7.75, 2) {$\dagger$-compact closed categories};
		\node [style=none, scale=1.25] (13) at (8, 1) { Finite dimensional Hilbert spaces};
		\node [style=none, scale=1.25] (14) at (8.65, 0) {Finite dimensional Quantum mechanics};
	\end{pgfonlayer}
	\begin{pgfonlayer}{edgelayer}
		\draw [style=filled, fill=red!80] (3.center)
			 to (0.center)
			 to (1.center)
			 to (2.center)
			 to cycle;
		\draw [style=filled, fill=red!60] (7.center)
			 to (4.center)
			 to (5.center)
			 to (6.center)
			 to cycle;
		\draw [style=filled, fill=red!40] (11.center)
			 to (8.center)
			 to (9.center)
			 to (10.center)
			 to cycle;
	\end{pgfonlayer}
\end{tikzpicture} 
\caption{$\dagger$-compact closed categories for quantum mechanics}
\label{Fig: layers}
\end{figure}

In 2004, Abramsky and Coecke \cite{AC04} described the fundamental axioms of quantum mechanics 
within the framework of $\dagger$-compact closed categories ($\dagger$-KCCs). 
This was quite significant as it meant that the proof theory 
based on string diagrams of monoidal categories \cite{Sel10}, 
could be deployed to reason about quantum processes. For example, in CQM, 
physical systems are represented as wires and processes as circles.  
The label of a wire represents its type. Diagram $(a)$ represents a system $A$, and 
diagram $(b)$ represents a transformation from system $A$ to system $B$.
 Processes can composed sequentially by connecting the wires 
with matching types. Note that the string diagrams are to be read 
from top to bottom (following the direction of gravity), and from left to right. 

\[ (a) ~~~ \begin{tikzpicture}
	\begin{pgfonlayer}{nodelayer}
		\node [style=none] (4) at (2.5, 7.25) {};
		\node [style=none] (5) at (2.5, 3) {};
		\node [style=none] (7) at (2, 5.25) {$A$};
	\end{pgfonlayer}
	\begin{pgfonlayer}{edgelayer}
		\draw (5.center) to (4.center);
	\end{pgfonlayer}
\end{tikzpicture} ~~~~~~~~~~
 (b) ~~~ \begin{tikzpicture}
	\begin{pgfonlayer}{nodelayer}
		\node [style=circle, scale=2] (3) at (2.5, 5.25) {};
		\node [style=none] (4) at (2.5, 7.25) {};
		\node [style=none] (5) at (2.5, 3) {};
		\node [style=none] (6) at (2.5, 5.25) {$f$};
		\node [style=none] (7) at (2, 7) {$A$};
		\node [style=none] (8) at (2, 3.5) {$B$};
	\end{pgfonlayer}
	\begin{pgfonlayer}{edgelayer}
		\draw (4.center) to (3);
		\draw (3) to (5.center);
	\end{pgfonlayer}
\end{tikzpicture}
~~~~~~~~~~ (c) ~~~ \begin{tikzpicture}
	\begin{pgfonlayer}{nodelayer}
		\node [style=circle, scale=2] (0) at (0, 5.5) {};
		\node [style=none] (1) at (0, 7) {};
		\node [style=none] (2) at (0, 2.5) {};
		\node [style=none] (3) at (0, 5.5) {$f$};
		\node [style=none] (4) at (-0.5, 6.75) {$A$};
		\node [style=none] (5) at (-0.5, 4.75) {$B$};
		\node [style=circle, scale=2] (7) at (0, 4) {};
		\node [style=none] (6) at (0, 4) {$g$};
		\node [style=none] (8) at (-0.5, 2.75) {$C$};
	\end{pgfonlayer}
	\begin{pgfonlayer}{edgelayer}
		\draw (1.center) to (0);
		\draw (7) to (0);
		\draw (2.center) to (7);
	\end{pgfonlayer}
\end{tikzpicture}
  \]

Moreover, the wires and the boxes can be composed in parallel leading to processes as shown in diagrams 
$(e)$ and $(f)$. Morever, the wires are allowed to cross one another another as shown in diagram $(g)$.
\[ (e)~~~ \begin{tikzpicture}
	\begin{pgfonlayer}{nodelayer}
		\node [style=circle, scale=2] (0) at (0, 4.75) {};
		\node [style=none] (1) at (-0.75, 6.5) {};
		\node [style=none] (2) at (-0.75, 3) {};
		\node [style=none] (4) at (-0.75, 6.75) {$A_1$};
		\node [style=none] (5) at (0, 6.5) {...};
		\node [style=none] (6) at (0.75, 6.5) {};
		\node [style=none] (7) at (0.75, 3) {};
		\node [style=none] (8) at (0, 3) {...};
		\node [style=none] (9) at (0.75, 6.75) {$A_n$};
		\node [style=none] (10) at (0, 4.75) {$f$};
		\node [style=none] (11) at (-0.75, 2.75) {$B_1$};
		\node [style=none] (12) at (0.75, 2.75) {$B_m$};
	\end{pgfonlayer}
	\begin{pgfonlayer}{edgelayer}
		\draw [in=150, out=-90] (1.center) to (0);
		\draw [in=-150, out=90] (2.center) to (0);
		\draw [in=-30, out=90] (7.center) to (0);
		\draw [in=30, out=-90] (6.center) to (0);
	\end{pgfonlayer}
\end{tikzpicture}
~~~~~~~~~~ (g) ~~~ \begin{tikzpicture}
	\begin{pgfonlayer}{nodelayer}
		\node [style=none] (1) at (-1, 6.5) {};
		\node [style=none] (2) at (-1, 3) {};
		\node [style=none] (4) at (-1, 6.75) {$A$};
		\node [style=none] (6) at (0.25, 6.5) {};
		\node [style=none] (7) at (0.25, 3) {};
		\node [style=none] (9) at (0.25, 6.75) {$C$};
		\node [style=none] (11) at (-1, 2.75) {$B$};
		\node [style=none] (12) at (0.25, 2.75) {$D$};
		\node [style=circle, scale=2] (13) at (-1, 4.75) {};
		\node [style=circle, scale=2] (14) at (0.25, 4.75) {};
		\node [style=none] (15) at (-1, 4.75) {$f$};
		\node [style=none] (16) at (0.25, 4.75) {$g$};
	\end{pgfonlayer}
	\begin{pgfonlayer}{edgelayer}
		\draw (6.center) to (14);
		\draw (14) to (7.center);
		\draw (13) to (2.center);
		\draw (13) to (1.center);
	\end{pgfonlayer}
\end{tikzpicture}
~~~~~~~~~~ (h) ~~~
\begin{tikzpicture}
	\begin{pgfonlayer}{nodelayer}
		\node [style=none] (1) at (-1.25, 6.5) {};
		\node [style=none] (2) at (-1.25, 3) {};
		\node [style=none] (4) at (-1.25, 6.75) {$A$};
		\node [style=none] (6) at (0.25, 6.5) {};
		\node [style=none] (7) at (0.25, 3) {};
		\node [style=none] (9) at (0.25, 6.75) {$B$};
		\node [style=none] (11) at (-1.25, 2.75) {$B$};
		\node [style=none] (12) at (0.25, 2.75) {$A$};
	\end{pgfonlayer}
	\begin{pgfonlayer}{edgelayer}
		\draw [in=90, out=-90, looseness=1.25] (1.center) to (7.center);
		\draw [in=90, out=-90, looseness=1.25] (6.center) to (2.center);
	\end{pgfonlayer}
\end{tikzpicture} \]

%In their paper, Abramsky and Coecke demonstrated 
%the proof of quantum teleporation \cite{Wat18} and a few other key quantum protocols purely in the graphical calculus 
%\cite{Sel10} of monoidal categories. 

It is far simpler to reason about processes using string diagrams since the human brain is good at processing visual information. 
For example, it is quite easy to see that the diagrams below represent the same process: one can prove the diagrams equal by 
fixing the ends of the wires and moving the circles. 

\[ \begin{tikzpicture}
	\begin{pgfonlayer}{nodelayer}
		\node [style=circle, scale=2] (0) at (-0.75, 4.75) {};
		\node [style=none] (1) at (1.25, 6) {};
		\node [style=none] (2) at (1.25, 2.25) {};
		\node [style=none] (10) at (-0.75, 4.75) {$f$};
		\node [style=circle, scale=2] (11) at (0.25, 4.75) {};
		\node [style=none] (12) at (0.25, 7) {};
		\node [style=none] (13) at (-1, 2.25) {};
		\node [style=circle, scale=2] (14) at (1.25, 4.75) {};
		\node [style=none] (15) at (0, 2.25) {};
		\node [style=none] (16) at (-0.75, 7) {};
		\node [style=none] (17) at (1.25, 7) {};
		\node [style=none] (18) at (0.25, 4.75) {$g$};
		\node [style=none] (19) at (1.25, 4.75) {$h$};
	\end{pgfonlayer}
	\begin{pgfonlayer}{edgelayer}
		\draw [in=90, out=-90, looseness=1.25] (1.center) to (0);
		\draw [in=-90, out=90] (2.center) to (0);
		\draw [in=90, out=-90] (12.center) to (11);
		\draw [in=-90, out=90, looseness=0.75] (13.center) to (11);
		\draw [in=-90, out=90, looseness=1.25] (15.center) to (14);
		\draw [in=-90, out=90] (14) to (16.center);
		\draw (17.center) to (1.center);
	\end{pgfonlayer}
\end{tikzpicture} = \begin{tikzpicture}
	\begin{pgfonlayer}{nodelayer}
		\node [style=circle, scale=2] (0) at (1, 4.75) {};
		\node [style=none] (1) at (1, 6) {};
		\node [style=none] (2) at (1, 2.25) {};
		\node [style=none] (10) at (1, 4.75) {$f$};
		\node [style=circle, scale=2] (11) at (-1, 4.75) {};
		\node [style=none] (12) at (0, 7) {};
		\node [style=none] (13) at (-1, 2.25) {};
		\node [style=circle, scale=2] (14) at (0, 4.75) {};
		\node [style=none] (15) at (0, 2.25) {};
		\node [style=none] (16) at (-1, 7) {};
		\node [style=none] (17) at (1, 7) {};
		\node [style=none] (18) at (-1, 4.75) {$g$};
		\node [style=none] (19) at (0, 4.75) {$h$};
	\end{pgfonlayer}
	\begin{pgfonlayer}{edgelayer}
		\draw [in=90, out=-90, looseness=1.25] (1.center) to (0);
		\draw [in=-90, out=90] (2.center) to (0);
		\draw [in=90, out=-90] (12.center) to (11);
		\draw [in=-90, out=90, looseness=0.75] (13.center) to (11);
		\draw [in=-90, out=90, looseness=1.25] (15.center) to (14);
		\draw [in=-90, out=90] (14) to (16.center);
		\draw (17.center) to (1.center);
	\end{pgfonlayer}
\end{tikzpicture} = \begin{tikzpicture}
	\begin{pgfonlayer}{nodelayer}
		\node [style=circle, scale=2] (0) at (1, 4.75) {};
		\node [style=none] (1) at (1, 6) {};
		\node [style=none] (2) at (1, 2.25) {};
		\node [style=none] (10) at (1, 4.75) {$f$};
		\node [style=circle, scale=2] (11) at (0, 4.75) {};
		\node [style=none] (12) at (0, 7) {};
		\node [style=none] (13) at (-1, 2.25) {};
		\node [style=circle, scale=2] (14) at (-1, 4.75) {};
		\node [style=none] (15) at (0, 2.25) {};
		\node [style=none] (16) at (-1, 7) {};
		\node [style=none] (17) at (1, 7) {};
		\node [style=none] (18) at (0, 4.75) {$g$};
		\node [style=none] (19) at (-1, 4.75) {$h$};
	\end{pgfonlayer}
	\begin{pgfonlayer}{edgelayer}
		\draw [in=90, out=-90, looseness=1.25] (1.center) to (0);
		\draw [in=-90, out=90] (2.center) to (0);
		\draw [in=90, out=-90] (12.center) to (11);
		\draw [in=-90, out=90] (13.center) to (11);
		\draw [in=-90, out=90, looseness=1.25] (15.center) to (14);
		\draw [in=-90, out=90] (14) to (16.center);
		\draw (17.center) to (1.center);
	\end{pgfonlayer}
\end{tikzpicture} \]

In compact closed categories, one can additionally bend wires into a cap and a cup as follows, thus adding to the expressive power of the language: 
\[  	\begin{tikzpicture}
		\begin{pgfonlayer}{nodelayer}
			\node [style=none] (0) at (-6, 5) {};
			\node [style=none] (1) at (-4.5, 5) {};
			\node [style=none] (2) at (-6, 6.25) {};
			\node [style=none] (3) at (-4.5, 6.25) {};
		\end{pgfonlayer}
		\begin{pgfonlayer}{edgelayer}
			\draw (0.center) to (2.center);
			\draw [bend left=90, looseness=1.50] (2.center) to (3.center);
			\draw (3.center) to (1.center);
		\end{pgfonlayer}
	\end{tikzpicture} ~~~~~~~~ \begin{tikzpicture}
		\begin{pgfonlayer}{nodelayer}
			\node [style=none] (0) at (-6, 6.25) {};
			\node [style=none] (1) at (-4.5, 6.25) {};
			\node [style=none] (2) at (-6, 5) {};
			\node [style=none] (3) at (-4.5, 5) {};
		\end{pgfonlayer}
		\begin{pgfonlayer}{edgelayer}
			\draw (0.center) to (2.center);
			\draw [bend right=90, looseness=1.50] (2.center) to (3.center);
			\draw (3.center) to (1.center);
		\end{pgfonlayer}
	\end{tikzpicture} \]

%See Section \ref{Sec: dag mon cat} of this thesis for a detailed presentation of monoidal categories and the graphical calculus.

Around the same time that Coecke and Absramsky applied the graphical calculus of monoidal categories to study quantum mechanics, 
Selinger used structures in monoidal categories for designing a quantum programming language \cite{Sel04}. 
In 2007, Selinger \cite{Sel07} refined and extended  
the framework of Abramsky and Coecke to account for mixed quantum states, 
and coined the term `Dagger($\dagger$) compact closed categories' for the categorical framework of quantum mechanics.  
The $\dagger$-compact closed categories faithfully abstract the structure of finite-dimensional Hilbert spaces, 
thereby enabling a diagrammatic but rigorous reasoning technique for quantum processes and protocols
within the category of finite-dimensional Hilbert spaces and linear maps, $\FHilb$. The category of 
all Hilbert spaces and linear maps, ${\sf Hilb}$, is $\dagger$-monoidal but not compact closed. 

\subsection{Towards arbitrary dimensions in CQM}

CQM has been applied to study problems in areas such as quantum foundations, 
quantum information theory and quantum computing. CQM techniques have been used to study 
%toy theories \cite{CoE11, Bac15b, HeT15, Gog17, GoS18} to better understand quantum phenomena, 
causality \cite{KiU19, Coe14} and non-locality \cite{CDKW12, CBS11} in quantum foundations. 
In quantum information theory, it has been used to construct structures 
like quantum Latin squares \cite{MuV16} which are crucial to many protocols in quantum information theory. Perhaps, the most significant 
outcome of CQM is in the field of quantum computation namely the ZX-calculus \cite{CoD11} which is a fine-grained 
diagrammatic calculus providing a set of generators and rewrite rules for designing and optimizing quantum circuits.  

Because the compact closed setting is restricted to finite dimensional Hilbert spaces, practical applications of CQM, 
so far, have been limited to the areas such as quantum computing and quantum information theory 
which are off-shoots of finite dimensional quantum mechanics. This is because the base framework of CQM namely the
compact closed categories impose finite dimensionality on the Hilbert Spaces \cite{Heunen16}.  Hence, the category of 
all Hilbert spaces is $\dagger$-monoidal and not $\dagger$-compact closed. The success of CQM in 
the study of finite-dimensional processes has inspired researchers to explore strategies for extending CQM to infinite dimensional processes. 

One approach was to identify the algebraic structures which can characterize the key components of quantum mechanics 
without imposing any restriction on dimensionality. For example, in CQM, dagger Frobenius algebras (See Section \ref{Sec: observables})
provide a precise algebraic characterization of quantum observables 
in the category of finite dimensional Hilbert Spaces  \cite{CPV12}  (A quantum observable is a measurable physical property of a quantum system).  

With the aim of extending this idea to quantum observables of arbitrary dimensions,  
Abramsky and Heunen, in \cite{AbH12}, showed how Ambrose's $H^*$-algebras \cite{Amb45} 
could be used to characterize orthonormal bases, hence quantum observables, in infinite dimensional Hilbert spaces. 
However, the move from $\dagger$-Frobenius to $H^*$-algebras comes at a cost -  an
$H^*$-algebra is modelled as a semi-Frobenius algebra, that is, Frobenius algebras without the units.

%and required a special property 
%(which they appropriately called $H^*$) on the maps from the unit.  

Gogioso and Genovese \cite{GG17} proposed an interesting approach to reinstating the units using techniques from non-standard analysis \cite{Rob96}.  
They considered $~^\star${\bf Hilb}, the category of non-standard separable Hilbert Spaces and linear maps.  This they claimed is a 
$\dagger$-compact closed category, in which, among other things, the semi-Frobenius algebras of Abramsky and Heunen can be modelled.  
Furthermore, the counit can be reinstated because formal infinite sums are permitted.  

In \cite{CoH16}, Coecke and Heunen, in order to model infinite dimensional  quantum processes, 
took the simple step of dropping the requirement that the category is compact closed and worked in dagger 
symmetric monoidal categories ($\dagger$-SMCs). The category of all Hilbert 
spaces, $\Hilb$, is the prototypical example of a $\dagger$-SMC.  
In CQM, quantum processes are modelled as completely positive maps in the 
category of finite-dimensional Hilbert Spaces. Coecke and Heunen  
showed how to build a category of completely positive maps for  
an arbitrary $\dagger$-SMC. A downside of this approach is that by moving to $\dagger$-SMCs, 
one loses the structural richness provided by duals.

In \cite{HeR18}, Heunen and Reyes considered a different $\dagger$-SMC, namely, the category of Hilbert $\C^*$-modules.  Its objects can be equivalently viewed as 
bundles of Hilbert spaces over a locally compact Hausdorff space.  They characterized the special commutative $\dagger$-Frobenius algebras in this
category as bundles of finite dimensional Hilbert spaces (with dimensions that are uniformly bounded).  These objects, while being far from finite, do retain a 
(uniform) locally finite nature.  This example,  by using vector bundles and ideas from differential geometry, enters the domain of traditional 
theoretical physics, and serves as a reminder that $\dagger$-Frobenius algebras are not only of interest for Hilbert spaces. 

%\subsection{The exponential modalities of linear logic}

 In \cite{Vic08}, Vicary attempted to model a quantum harmonic oscillator, 
 which is inherently an infinite-dimensional 
quantum system in $\dagger$-monoidal categories using free exponential modalities. 
He proposed a notion of $\dagger$-exponentials in 
$\dagger$-symmetric monoidal categories with $\dagger$-biproducts and 
used it to derive an abstract Fock Space and the ladder operators of Fock Spaces. 
He conjectured that the category of countable dimensional inner product spaces and 
everywhere-defined linear maps is a model for this categorical setting. 

%Fock spaces model the 
%state space of a quantum harmonic oscillator. Vicary noted in his discussion that it is far from evident how one can 
%capture the dynamics of a quantum harmonic oscilator with the existing tools in CQM \cite{Vic08}. 

%Interestingly, outside the context of CQM, Blute and Panandagen \cite{BPS94} also used the 
%exponential operator $!$ of linear logic to model and examine Bosonic Fock Spaces 
%in different categories such as vector spaces over complex numbers and Banach spaces.

%In this work, he introduced compactly 
%accessible categories which allows for infinite-dimensional types. The category of Hilbert spaces and 
%bounded linear maps, $\Hilb$, is compactly accessible. Compactly accessible categories 
%come equipped with a free $!$ exponential modality which is then used in the modelling of unbounded storage.

%Note that, Heunen and Vicary used compact fragments of linear logic, hence the $?$ operator coincides 
%with its dual operator $!$ in their respective categorical settings. 

\subsection{Non-compact multiplicative $\dagger$-linear logic}

While various strategies have been tried for modelling systems of arbitrary dimensions, 
it is a general consensus in the CQM community, that the 
dimensionality constraint of the structures used in CQM is yet to be satisfactorily addressed. 
In this thesis, a different approach is taken by shifting focus beyond Hilbert spaces and other models, 
and by moving to non-compact $\dagger$-linear logic.  

Rather than insisting that the infinite dimensional structures are concretely related to Hilbert spaces, 
it is considered that there may be a system of formal types which extend the existing compact logic of CQM.  
An example of such a setting is the embedding of the category of finite dimensional Hilbert spaces  (which is $\dagger$-compact)
into the ($\dagger$-)$*$-autonomous category of Chu Spaces over complex vector spaces, $\Chus_{\sf Vec(\C)}(I)$, 
where $I$, the tensor unit of ${\sf Vec}(\C)$, is the dualizing object (See Section \ref{Section: Chu}). Yet another example is the embedding 
of finite dimensional complex matrices into the  ($\dagger$-)$*$-autonomous 
category of finiteness spaces and finiteness matrices over complex numbers \cite{Ehr05} (see Section \ref{Sec: Finiteness matrices}).

We begin our explorations with linearly distributive categories (LDCs) and $*$-autonomous categories rather 
than monoidal categories to first obtain a semantics for non-compact $\dagger$-linear logic. 
This idea is not new.  Models for quantum mechanics in $*$-autonomous categories are often described 
as ``toy models'' \cite{Abr12} and were, in particular discussed by Pavlovic \cite{Pav11} where some very similar 
directions were advocated.   Indeed, Egger \cite{Egg11}, in initiating the development of ``involutive'' categories, 
implicitly suggested that a dagger functor is not stationary ($A \neq A^\dag$) on objects in an LDC setting. 

%Another example is provided by the 
%extension of finite dimensional complex matrices to  finiteness matrices.  
%Finiteness matrices are matrices of infinite dimensions, however, multiplying 
%two such matrices produces only finite sums (see Section \ref{Sec: Finiteness matrices}}), hence 
%these matrices can be composed. 

\subsection{Linearly distributive categories}
\label{Sec: ldc-intro}

Linearly distributive categories (LDCs) \cite{CS97} provide the categorical semantics (so they are the proof theory)
for  the non-compact multiplicative fragment of linear logic (MLL) containing the 
$\ox$, $1$, $\parr$, and $\bot$. The sequent rules for the multiplicatives are shown in Figure \ref{Fig: multiplicative rules}.
In the linear logic community the multiplicative disjunction 
is often denoted by $\parr$, however this thesis follows the convention in \cite{CS97} and shall use $\oplus$. 
Under the same convention, we write the unit of  $\ox$ as top $\top$ rather than $1$ as in the linear logic community. 
The unit of $\oa$ is a bottom, $\bot$. 

\begin{figure}[h]
	\centering
		\AxiomC{$\Gamma_1, \Gamma_2  \vdash \Delta$}
		\LeftLabel{$(\top L)$}
		\UnaryInfC{$\Gamma_1, \textcolor{blue}{\top}, \Gamma_2 \vdash \Delta$}
	   \DisplayProof 
        \hspace{1.5 em}
		\AxiomC{$ $}
		\LeftLabel{$(\top R)$}
		\UnaryInfC{$\vdash \textcolor{blue}{\top}$} 
		\DisplayProof 

		\vspace{1em}
		
		\AxiomC{$\Gamma_1, \textcolor{blue}{A, B}, \Gamma_2 \vdash \Delta$}
		\LeftLabel{$(\ox  L)$}
		\UnaryInfC{$\Gamma_1, \textcolor{blue}{A \ox B}, \Gamma_2 \vdash \Delta$}
	   \DisplayProof
	   \hspace{1.5em}
		\AxiomC{$\Gamma_1 \vdash \Gamma_2, \textcolor{blue}{A}, \Gamma_3 ~~~~ \Delta_1 \vdash \Delta_2, \textcolor{blue}{B}, \Delta_3$}
		\LeftLabel{$(\ox  R)$}
		\UnaryInfC{$\Gamma_1, \Delta_1 \vdash \Gamma_2, \Delta_2, \textcolor{blue}{A \ox B}, \Gamma_2, \Delta_3$}
		\DisplayProof

		\vspace{1em}

		\AxiomC{$ $}
		\LeftLabel{$(\bot  L)$}
		\UnaryInfC{$\textcolor{blue}{\bot} \vdash $}
	    \DisplayProof 
        \hspace{1.5 em}
		\AxiomC{$\Gamma \vdash \Delta_1, \Delta_2$}
		\LeftLabel{$(\bot  R)$}
		\UnaryInfC{$\Gamma \vdash \Delta_1, \textcolor{blue}{\bot}, \Delta_2$}
	    \DisplayProof 
        
		\vspace{1 em}		
		
		\AxiomC{$\Gamma_1, \textcolor{blue}{A}, \Gamma_2 \vdash \Gamma_3 ~~~~ \Delta_1, \textcolor{blue}{B}, \Delta_2 \vdash \Delta_3$}
		\LeftLabel{$(\oa  L)$}
		\UnaryInfC{$\Gamma_1, \Delta_1,  \textcolor{blue}{A \oa B},  \Gamma_2, \Delta_2 \vdash \Gamma_3, \Delta_3$}
	    \DisplayProof 
        \hspace{1.5 em}		
		\AxiomC{$\Gamma \vdash \Delta_1, \textcolor{blue}{A, B}, \Delta_2$}
		\LeftLabel{$(\oa  R)$}
		\UnaryInfC{$\Gamma \vdash  \Delta_1, \textcolor{blue}{A \oa B}, \Delta_2$}
	    \DisplayProof 
\caption{Sequent rules for multiplicatives}
\label{Fig: multiplicative rules}
\end{figure}

\FloatBarrier

We assume that the sequent calculus is commutative (the order of premises and antecedents does not matter). 
This is accommodated by the exchange rules which allows the neighboring premises and antecedents to be swapped: 
\[ 
\AxiomC{$\Gamma_1, \textcolor{blue}{A, B}, \Gamma_2 \vdash \Delta$} 
\LeftLabel{$(exch.L)$}
\UnaryInfC{$\Gamma_1, \textcolor{blue}{B, A}, \Gamma_2 \vdash \Delta$}
\DisplayProof 
\hspace{1.5em}
\AxiomC{$\Gamma \vdash \Delta_1, \textcolor{blue}{C,D}, \Delta_2$} 
\LeftLabel{$(exch.R)$}
\UnaryInfC{$\Gamma \vdash \Delta_1, \textcolor{blue}{D, C}, \Delta_2$} 
\DisplayProof 
\]

Since LDCs are the proof theory of MLL, they come equipped with two distinct tensor products 
called the ``tensor'', $\ox$, and the ``par'', $\oplus$ 
corresponding to the multiplicative conjunction and disjunction of linear logic respectively. 
The tensor and the par are related by two natural transformations called linear distributors: 
\[ \partial^L: A \ox (B \oa C) \to (A \ox B) \oa C~~~~~~~~ \partial^R: (A \oa B) \ox C \to A \oa (B \ox C) \] 
The distributors are not isomorphisms. The following is the sequent derivation of $\partial^L$:

\vspace{0.5em}

\begin{center}
    \AxiomC{$ $}
    \RightLabel{$id$, $\oa L$, cut}
    \UnaryInfC{$B \oa C \vdash B, C$}
    
    \AxiomC{$ $}
    \RightLabel{$id$, $\ox R$, cut}
    \UnaryInfC{$A,B \vdash A \ox B$}
    
    \RightLabel{cut}
    \BinaryInfC{$A, B \oa C \vdash A \ox B, C$}
    \RightLabel{$(\ox L), (\oa R)$}
    \UnaryInfC{$A \ox (B \oa C) \vdash (A \ox B) \oa C$}
    \DisplayProof
\end{center} 

The multiplicative fragment with negation ($A^\perp$) has its categorical semantics in $*$-autonomous categories.
The sequent rules for negation allow premises to be flipped to the opposite side of the entailment:

\begin{figure}[h]
\centering
\AxiomC{$\Gamma, \textcolor{blue}{B^\perp} \vdash \Delta$}
\LeftLabel{$(Neg.L)$}
\UnaryInfC{$\Gamma \vdash \Delta, \textcolor{blue}{B}$}
\DisplayProof 
\hspace{1.5 em}
\AxiomC{$\Gamma \vdash \Delta, \textcolor{blue}{B}$}
\LeftLabel{$(Neg.R)$}
\UnaryInfC{$\Gamma, \textcolor{blue}{B^\perp}  \vdash \Delta$}
\DisplayProof 
\caption{Sequent rules for negation}
\label{Fig: negation rules}
\end{figure}


In an LDC, negation of an object is given by its categorical dual. An object $A$ has a dual $A^\perp$ (negation of $A$) 
if there exists two maps, $\eta: \top \to A \oa A^\perp$, 
and $A^\perp \ox A \to \bot$ satisfying the `snake' equations, see Definition \ref{defn: duals}.
 An LDC in which every object has a chosen dual is a $*$-autonomous category \cite{Bar91}, 
 see Section \ref{Sec: *-autonomous}. 
 The following proofs show the derivations of the $\eta$ and the $\epsilon$ maps respectively
 from the negation rules:

 \vspace{0.5em}

 \begin{center}
	\AxiomC{$ $}
	 \RightLabel{(id)}
	 \UnaryInfC{$A \vdash A$}
	 \RightLabel{Neg.R}
	 \UnaryInfC{$\vdash A^\perp, A$}
	 \RightLabel{$(\oa R)$}
	 \UnaryInfC{$ \vdash A \oa A^\perp $}
	 \RightLabel{$(\top L)$}
	 \UnaryInfC{$\top \vdash A \oa A^\perp$}	
	 \DisplayProof
	 \qquad
	  \AxiomC{$ $}
	 \RightLabel{(id)}
	 \UnaryInfC{$A \vdash A$}
	 \RightLabel{Neg.L}
	 \UnaryInfC{$A^\perp, A \vdash $}
	 \RightLabel{$(\ox L)$}
	 \UnaryInfC{$A^\perp \ox A \vdash $}
	 \RightLabel{$(\bot R)$}
	 \UnaryInfC{$A^\perp \ox A \vdash \bot$}
	 \DisplayProof
 
	 \vspace{0.75 em}
 
	 Proof for $\eta$ ~~~~~~~~~~~~~~~~ Proof for $\epsilon$
 
  \end{center}

An {\bf isomix category} is an LDC with an isomorphism, 
called the mix map, $\m: \bot \to \top$, satisfying the `mix law' \cite[Definition 6.2]{BCS00}, 
see Section \ref{Sec: mix, isomix, compact LDC}. The mix law provides a natural transformation, $\mx: A \ox B \to A \oa B$, 
from the multiplicative conjunction to the multiplicative disjunction, called the {\bf mixor}, see Section \ref{Sec: mix, isomix, compact LDC}. 
The mix law corresponds to the sequent rules shown in Figure \ref{Fig: mix map} (a)-(b). 

\begin{figure}[h]
	\centering
	$(a)~~~$ 
	\AxiomC{$\Gamma \vdash \Delta$}
	\AxiomC{$\Gamma' \vdash \Delta'$}
	\BinaryInfC{$\Gamma, \Gamma' \vdash \Delta, \Delta'$}
	\DisplayProof
	$~~~~~~(b)~~$ \AxiomC{$ $} 
	\UnaryInfC{$\top \vdash \bot$}
	\DisplayProof
	\caption{(a) Binary mix axiom (b) Nullary mix axiom}
	\label{Fig: mix map}
\end{figure}

	\FloatBarrier

In the presence of the cut rule, the rule $(a)$ is equivalent to 
the axiom $\bot \vdash \top$ \cite[Lemma 6.1]{CS97a}, see below. 

\begin{center}
    \AxiomC{$\Gamma \vdash \Delta$}
    \RightLabel{$(\bot L)$}
    \UnaryInfC{$\Gamma \vdash \Delta, \bot$}

    \AxiomC{$ $}
    \RightLabel{$id$}
    \UnaryInfC{$\bot \vdash \top$}

    \RightLabel{cut}
    \BinaryInfC{$\Gamma \vdash \Delta, \top$}

    \AxiomC{$\Gamma' \vdash \Delta'$}
    \RightLabel{ $(\top L)$}
    \UnaryInfC{$\top, \Gamma' \vdash \Delta'$}

    \RightLabel{cut}
    \BinaryInfC{$\Gamma, \Gamma' \vdash \Delta, \Delta'$}

    \DisplayProof

	\vspace{0.75em}

    Derivation of the binary mix axiom in the presence of the cut rule
\end{center}

In an isomix category, an object $U$ for which the natural transformations, $\mx_{(-, U)}$ and $\mx_{(U, -)}$ are isomorphisms
(that is for any object $A$, $A \ox U \simeq A \oa U$ and $U \ox A \simeq U \oa A$) is said to be in the {\bf core} 
of the category. The core of an isomix category determined by such objects forms a full subcategory.  

An isomix category in which the mixor is a natural isomorphism ($\ox \simeq \oa$) is called a {\bf compact} LDC. 
 The core of an isomix category is always a compact LDC. Compact LDCs are linearly equivalent to a monoidal categories.  
 Conversely, {\bf monoidal} categories \cite{Mac13} can be viewed as being degenerate  
 compact LDCs in which the mixor is the identity map ($\ox = \oa$). 
 A monoidal category in which every object has a dual, is a compact closed category \cite{Kel80}. 
 From this perspective, a compact closed category can be viewed as a compact $*$-autonomous category.  

%Monoidal categories provide a categorical semantics of 
% a compact version linear logic in which the two multiplicative connectives of the logic coincide. 

%CQM uses $\dagger$-monoidal categories and $\dagger$-compact closed categories to study 
%quantum processes. 
%In this thesis we propose the categorical proof theory for non-compact $\dagger$-linear logic using 
%$\dagger$-isomix categories and generalize the existing algebraic structures of CQM to the new non-compact framework. 


\section{Thesis outline}

The aim of this thesis is to lay the categorical foundations for non-compact $\dagger$-linear logic and to apply it 
to categorical quantum mechanics. With this goal in mind, the thesis is divided into 
two parts: 

\begin{description} 
	\item[Part 1:] composed of chapters \ref{chap: LDC} - \ref{chap: part 1 summary} 
is focused on developing the {\bf categorical semantics} of non-compact $\dagger$-linear logic. 
     \item[Part 2:] composed of chapters \ref{Chap: CQM} - \ref{Chap: part 2 summary}, is focused on {\bf developing structures 
for CQM} in the categorical setting developed in Part 1.
\end{description}

The rest of the section outlines the contents of this thesis and indicates the contributions. 

\subsection{Part I: Dagger linear logic}

 This first part begins with Chapter  \ref{chap: LDC} which provides an introduction to 
LDCs and its variants, linear functors and transformations. This chapter also discusses the 
Ehrhard's Finiteness spaces in detail, which is used as an example throughout this thesis. 
Chapters \ref{Chap: dagger-LDC} and \ref{Chap: MUCs} 
chapters are derived from the article titled `Dagger linear logic for categorical quantum mechanics' \cite{CCS18}:
it was presented at the Symposium on Compositional Structures (SYCO I) in Birmingham (U.K.) and as a poster
 at the 15th international conference in Quantum Physics and Logic (QPL), Halifax, Canada. 
The contributions of chapters \ref{Chap: dagger-LDC} and \ref{Chap: MUCs} are outlined below. 
Chapter \ref{chap: part 1 summary} summarizes the first part of this thesis. 

 \vspace{1em}

\noindent{\bf Dagger linearly distributive categories}

\nopagebreak

\vspace{0.5em}

\nopagebreak

It is standard in CQM to interpret the dagger as a contravariant functor which is 
stationary on objects ($A = A^\dagger$) and an involution for maps ($ f^{\dag \dag} = f$). 
However, in an LDC with two tensor products - 
a tensor $\ox$, and a par $\oa$ - the dagger has to flip the tensor and par so that 
$(A \oa B)^\dagger = A^\dagger \ox B^\dagger$. Without such a flip, daggering the linear distributor would 
produce a non-permissible map. 

This implies that the dagger can no longer be viewed as being stationary on objects in an LDC. 
A non-stationary dagger implies that one has to address the coherence issues  determining how the 
dagger interacts with the structures of an LDC.  Moreover, one can replace the equality 
above by a natural isomorphism $ \lambda_\ox: A^\dag \ox B^\dag \to (A \oa B)^\dag$, and 
the involution by a natural isomorphism $\iota_A: A \to A^{\dag \dag}$. 
We deal with these natural isomorphisms and define $\dagger$-LDC, 
$\dagger$-mix and $\dagger$-isomix categories in Section \ref{Section: dagger LDC}. The sequent 
calculus for $\dagger$-linear logic is discussed in Section \ref{Sec: dagger sequent rules}.

\vspace{1em}

\noindent{\bf Unitary isomorphisms for $\dagger$-isomix categories}  

\nopagebreak

\vspace{0.5em}

\nopagebreak

In CQM, the dagger functor determines the notion of a unitary isomorphism which 
is an isomorphism $f: A \to B$ such that $f^\dag = f^{-1}$. These isomorphisms are particularly 
important since they model the unitary evolution of a quantum system. Applying this idea 
directly to $\dagger$-LDCs is not feasible, because the maps, $f^{-1}: B \to A$ and 
$f^\dag: B^\dag \to A^\dag$ now have different types, hence cannot be directly equal. 
However, minimally, if there exist isomorphisms, $\varphi_A: A \simeq A^\dag$ and $\varphi_B: B \simeq B^\dag$, 
then one can define a unitary isomorphism to be a map $f$ satisfying the following commuting diagram: 
\[  \xymatrix{
	A \ar[r]^{f} \ar[d]_{\varphi_A} & B \ar[d]^{\varphi_B} \\
	A^\dag \ar@{<-}[r]_{f^\dag} & B^\dag } \] 
Note that, in a $\dagger$-monoidal category, the isomorphisms $\varphi_A$ 
and $\varphi_B$ are simply the identity maps. 

In a $\dagger$-isomix category, the isomorphism $\varphi_A$ is referred to as a
{\em unitary structure} map  if $A$ resides in the core, and behaves coherently 
with the natural isomorphisms of the category. In this case, $A$ is referred to as unitary objects. 
The unitary objects of $\dagger$-isomix category forms a sub compact-$\dagger$-isomix category. 
Such a compact $\dagger$-isomix category in which every object is unitary is called a {\em unitary category}. 
Unitary categories are $\dagger$-linearly equivalent to a $\dagger$-monoidal categories. Moreover, 
when unitary categories have unitary duals, the category is $\dagger$-linearly equivalent to 
$\dagger$-compact closed categories. These ideas are developed in Section \ref{Sec: unitary}.

 \vspace{1em}

 \noindent {\bf Unitary construction} 

 \nopagebreak

 \vspace{0.5em}
 
 \nopagebreak
 
 Having defined the $\dagger$-LDCs and the unitary structure for $\dagger$-isomix categories, our next objective is to 
extract a unitary category, that is the traditional CQM setting, from any $\dagger$-isomix category.
For this, one collects the ``pre-unitary'' objects  of a $\dagger$-isomix category. An object $A$ 
is  {\bf pre-unitary} if $A$ is in the core and there exists an isomorphism $\varphi_A: A \to A^\dag$ and the isomorphism 
behaves coherently with the involutor $\iota$ as shown in the following commuting diagram: 
\[ \xymatrix{
	A \ar[d]_{\varphi_A} \ar[dr]^{\iota} & \\ 
	A^\dag \ar[r]_{\varphi_A^{-1 \dag}} & A^{\dag \dag}} \]
The unitary category thus extracted using the `unitary contruction' is referred to as the {\bf canonical unitary core} 
of the $\dagger$-isomix category. See Section \ref{Sec: unitary construction} 
for details. 

\vspace{1em}

\noindent{\bf Mixed unitary categories} 

\nopagebreak

\vspace{0.5em}

\nopagebreak

Collecting the structures we have developed so far, namely the $\dagger$-isomix categories and 
the unitary categories, we arrive at a general framework of `Mixed Unitary categories' (MUCs). 
MUCs embed a unitary category in a larger $\dagger$-isomix category via a $\dagger$-isomix-functor 
from the unitary category to the core of the $\dagger$-isomix category: 
\[ M: \U \to \Core(\C) \hookrightarrow \C \]  
where $\U$ is the unitary category and $\C$ is the $\dagger$-isomix category. 
Thus, a MUC is a general framework which encompasses the traditional CQM framework within its core. 
The unitary category of a MUC acting on the larger $\dagger$-isomix category is analogous to a field $K$ 
acting on a $K$-algebra as scalars. Sections \ref{Sec: MUCs} and \ref{Sec: MUC examples} are dedicated 
towards developing these ideas.

\subsection{Part II: Application to dagger linear logic to CQM}

The second part of this thesis is concerned with applying the MUC framework to quantum mechanics. 
To this end, we generalize the key algebraic structures of CQM to the MUC setting 
and study the implications of this generalization.  

This part begins with Chapter \ref{Chap: CQM} which provides an introduction to $\dagger$-monoidal categories and 
the other key algebraic structures used by CQM. Chapter \ref{Chap: positivity} is derived from 
my coauthored article titled `Complete positivity for Mixed Unitary Categories' \cite{CS19}, which was presented as a talk at the QPL 2019 held in California, USA, and 
at SYCO II held in Glasgow, UK.  Chapters \ref{Chapter: compaction} and \ref{Chapter: complementarity} are 
derived from my coauthored paper titled `Exponential modalities and complementarity' \cite{CS21}. This work 
was presented at the 4th International Conference on Applied Category Theory held in July 2021. 
Chapter \ref{Chap: free exp examples} presents examples for the structures introduced in Chapters 
\ref{Chapter: compaction} and \ref{Chapter: complementarity}. Chapter \ref{Chap: part 2 summary} summarizes the second part of the thesis. 

The contributions of chapters \ref{Chap: positivity} - \ref{Chapter: complementarity} are outlined below. 

\vspace{1em}

\noindent{\bf Completely positive maps in MUCs}

\nopagebreak

\vspace{0.5em}

\nopagebreak

In a $\dagger$-monoidal settings, completely positive maps abstract the notion of quantum processes.
Coecke and Heunen \cite{CoH16} developed the $\CP^\infty$ construction which produces the 
category of completely positive maps when applied to any $\dagger$-symmetric monoidal category.
We define the completely positive maps and the $\CP^\infty$ construction for a MUC based on the 
Coecke and Heunen's construction \cite{CoH16}. We also characterize the $\CP^\infty$ construction on MUCs using the 
notion of environment structures and purification. The characterization of $\CP^\infty$ construction for MUCs led to 
an elegant observation that it suffices for the unitary core to have environment maps to characterize 
the $\CP^\infty$ construction for MUCs. Chapter \ref{Chap: positivity} covers this discussion. 

\vspace{1em}

\noindent{\bf Measurement in MUCs} 

\nopagebreak

\vspace{0.5em}

\nopagebreak

Coecke and Pavlovic provided an algebraic description of quantum measurements in $\dagger$-monoidal categories  \cite{CoP07}.
In a MUC, measurement happens in two steps: first the system to be measured must be {\em compacted} into the 
unitary core (i.e., the traditional CQM core), and then the usual measurement process \cite{CoP07} 
as described by Coecke and Pavlocic must be applied within the core. In order to characterize the compaction process, 
we introduce {\em binary idempotents} and {\em $\dagger$}-binary idempotents. 
Compaction of an object in a $\dagger$-isomix category precisely corresponds to the splitting of certain 
$\dagger$-binary idempotents on the object, 
See Section \ref{Sec: compaction}. The $\dagger$-binary idempotents generalize Selinger's \cite{Sel08}  
$\dagger$-idempotents in $\dagger$-monoidal categories which $\dagger$-splits to produce classical types.  

\vspace{1em}

\noindent{\bf Complementary systems in isomix categories}

\nopagebreak

\vspace{0.5em}

\nopagebreak

The notion of complementary observables is central to quantum mechanics. An observable is a measurable property 
of a quantum system. Two observables are complementary if measuring (knowing) the value of one observable 
increases the uncertainty of the value of the other. A classic example of complementary observables is the 
position and momentum of an electron. 

In CQM, quantum observables are algebraically presented as special commutative $\dagger$-Frobenius algebras \cite{CPV12} in $\dagger$-monoidal categories. Moreover, 
complementary observables are two such $\dagger$-Frobenius algebras on the same object interacting bialgebraically to produce 
two Hopf algebras\footnote{Thus is often referred to as {\em strong} complementarity in CQM.} \cite{CoD11}. 

In LDCs, linear monoids with a $\ox$-monoid and a $\oa$-monoid provide a general version of Frobenius algebras. In fact, Frobenius 
algebras in monoidal categories can be viewed as linear monoids satifying an extra property.  
Linear monoids lead to the new notion of a ``linear comonoid", which can interact bialgebraically 
with a linear monoid to give a ``linear bialgebra".  Using these structures in a ($\dagger$-)isomix category, one can 
define a complementary system as a ($\dagger$-)linear bialgebra satisfying a few extra equations. 
Chapter \ref{Chapter: compaction} is dedicated to developing linear comonoids and linear bialgebras. 
Complementary systems in isomix categories are described in Section \ref{Sec: complementary systems}.

\vspace{1em}

\noindent{\bf Relating complementarity and exponential modalities} 

\nopagebreak

\vspace{0.5em}

\nopagebreak

A final but a significant contribution of this thesis is to establish the connection between exponential modalities 
of linear logic and complementarity of quantum mechanics using our MUC framework. For this, we define 
$(!,?)$-$\dagger$-LDCs, i.e., a $\dagger$-LDC with exponential modalities, and also provide the sequent rules 
for the corresponding logic, see Sections \ref{Sec: dagger exp modalities}, and \ref{Sec: sequent dagger exp}. 
We prove that every complementary 
system in a ($\dagger$-)isomix category arises as a compaction of a ($\dagger$-)linear bialgebra induced on the {\bf free} 
exponential modalities, see Section \ref{Sec: exp modalities}. 

\vspace{1em} 

Chapter \ref{Chap: conclusion} concludes this thesis and dicusses future directions. 

\iffalse
\section{Thesis structure}

This thesis is divided into two parts. The first part composed of Chapters \ref{chap: LDC} - \ref{chap: part 1 summary} is dedicated to developing the categorical semantics of non-compact 
$\dagger$-linear logic providing relevant examples as we develop the structures. The second part composed of Chapters \ref{Chap: CQM} - \ref{Chap: part 2 summary} 
of this thesis is dedicated to apply our framework to Categorical Quantum Mechanics. Every chapter of this thesis has an Examples section 
presenting examples of structures discussed in that chapter. 

\begin{description}
\item[Chapter 2:] This is a background chapter providing an introduction to LDCs and its variants, linear functors and transformations. This chapter 
also discusses the example of Ehrard's Finiteness spaces in detail. 
\item[Chapter 3:] This chapter provides the coherence conditions for $\dagger$-LDCs and its variants, $\dagger$-linear functors, and  
discusses the relationship between the dagger, the dual and the conjugation functors. 
\item[Chapter 4:]  This chapter describes unitary structure for $\dagger$-isomix categories and mixed unitary categories. 
\item[Chapter 5:] This chapter summarizes the first part of this thesis
\item[Chapter 6:] This is a background chapter discussing $\dagger$-monoidal categories, and structures fundamental to Categorical Quantum Mechanics
\item[Chapter 7:] This chapter deals with completely positive maps in MUCs and environment structures.
\item[Chapter 8:]  This chapter discusses measurement in MUCs in its first section. The later sections introduce the notion of linear comonoid
and proves that, in monoidal categories, Fr\"obenius algebras are linear monoids and linear comonoids satisfying an extra property.
\item[Chapter 9:] In this chapter, we define with linear bialgebras,  complementary systems in MUCs,  $\dagger$-LDCs with 
exponential modalities, and prove a connection between exponential modalities and complementary systems.
\item[Chapter 10:] This chapter summarizes the second part of this thesis. 
\item[Chapter 11:]  This chapter concludes this thesis and discusses future directions. 
\end{description}
\fi

\section{Prerequisites}
 We assume that the reader has a basic understanding of category theory including the definition of 
categories, functors, natural transformations, duality, isomorphisms, and adjunctions. We refer the reader 
to a few excellent sources \cite{Awo10, BaC90, Bor94, Coe08,  Bae10, Mac13}, particularly, references \cite{Coe08,  Bae10} 
are well-suited for someone with a Physics background while \cite{Bor94} is geared towards audience in computer science.

\section{Notation}

Composition is written in diagrammatic  order: $fg$ means apply $f$ followed by $g$.
The string diagrams are to be read from top to bottom (following the direction of gravity) and left to right. 

\iffalse
\section{Outline}

As the first setp in our journey, we set out to discover the coherence conditions  for a $\dagger$-linearly distributive category. 
It is standard in CQM to interpret the dagger as a contravariant functor which is stationary on objects $(A = A^\dagger)$ 
and is an involution (for a map $f$, $f^{\dag \dag} = f$). However, in an LDC setting with two tensor products - 
a tensor product $\ox$, and a par $\oa$ - the dagger is expected to flip the tensor and par so that 
$(A \oa B)^\dagger = A^\dagger \ox B^\dagger$. Without such a flip, daggering the linear distributor would 
produce an illegal map in the category (see below), unless the tensor and the par coincides: 
\[ \infer{\partial^L: A \ox (B \oa C) \to (A \ox B) \oa C}{(\partial^L)^\dagger:  (A \ox B) \oa C \to A \ox (B \oa C)} \]
This implies that the dagger can no longer be viewed as being stationary on objects. A non-stationary dagger implies that one has to 
address the coherence issues  to determine how the dagger interacts with the structures which determine an LDC. 
We deal with these coherence issues and define a $\dagger$-LDC in Section \ref{Section: dagger LDC}. 
A few examples are discussed in Section \ref{subsection: Dagger LDC examples}.

In a $\dagger$-compact closed category, the $\dagger$ commutes with the dual functor $($-$)^*$ to 
produce a conjugation functor, $\overline{(\text{-})}$ (also written as $($-$)_*$ in the literature). Such relationship 
between the dagger, the dual, and the conjugation functor needs to be preserved for the $\dagger$-LDCs too. 
Sections \ref{daggers-duals-conjugation} and \ref{Sec: Examples Dagger and conjugation} are dedicated to verifying 
this relationship between the functors, and to discuss relevant examples. 

In CQM, the dagger functor determines the notion of a unitary isomorphism which 
is an isomorphism $f: A \to B$ such that $f^\dag = f^{-1}$. These isomorphisms are particularly 
important since they model the unitary evolution of a quantum system. Defining unitary 
isomorphisms  in $\dagger$-LDCs is complicated, because the maps, $f^{-1}: B \to A$ and 
$f^\dag: B^\dag \to A^\dag$ now have different types, hence cannot be directly equal. 
To resolve this issue, we introduce the notion of unitary objects in Section \ref{Sec: unitary}. 

In a $\dagger$-isomix category, a unitary object is an object, $A$, with an isomorphism, $\varphi_A: A \to^{\simeq} A^\dag$, 
referred to as a {\em unitary structure} map which behaves coherently with the dagger and the other structures. 
Unitary isomorphisms can be defined for unitary objects as follows.  
A unitary isomorphism $f: A \to B$ in an LDC is an isomorphism between unitary objects 
which is (twisted) natural with respect to the unitary structures, $\varphi_A$ and $\varphi_B$,
in the sense that  the following diagram is rendered commutative:
\[ \xymatrix{ A \ar[d]_{\varphi_A} \ar[rr]^f & & B \ar[d]^{\varphi_B} \\ 
A^\dagger  && B^\dagger  \ar[ll]^{f^\dagger} } \]
Note that, when the  unitary structure is the identity map, we recover the usual notion of unitary 
isomorphism. These ideas are developed in Section \ref{Sec: unitary} and their examples in Section 
\ref{Sec: Examples The unitary construction}. Our formulation of unitary isomorphisms is not completely original as a lively 
discussion of whether $\dagger$-categories were ``evil''  led Peter Lumsdaine to suggest in 
the math overflow forum \cite{Lums15} how they might be made a little less evil.   

Resolving the coherence requirements of a unitary structure map leads to a pleasant observation that, 
in a $\dagger$-isomix category, for all unitary objects, all the coherence isomorphisms  including the 
associators, unitors for the tensor and the par, the mixor $\mx: A \ox B \to A \oa B$ are unitary isomorphisms. 
This means that, 

The sub-LDC of the $\dagger$-isomix category determined by the unitary objects, is a compact $\dagger$-isomix 
category, since in this sub-LDC, all the coherence isomorphisms such as the associators, the unitors, the mixor 
are unitary isomorphisms.  This leads to the notion of unitary categories. A {\bf unitary category} is simply a 
$\dagger$-isomix category in which every object is unitary. Unitary categories 
are rather special. In fact, we show in Section \ref{subsection: unitary categories} that they are $\dagger$-linearly equivalent to the  
standard CQM structure of a $\dagger$-SMC -- and, furthermore, that a unitary category with unitary 
duals is $\dagger$-linearly equivalent to a $\dagger$-compact  closed category.   

Having defined the $\dagger$-LDCs and the unitary structure for $\dagger$-isomix categories, our next objective is to 
extract the traditional CQM setting of $\dagger$-SMC from any $\dagger$-isomix category.
For this, we introduce the unitary construction which chooses ``pre-unitary'' objects from the core of 
the $\dagger$-isomix category. The unitary category thus extracted is referred to as the {\bf canonical unitary core} 
of the $\dagger$-isomix category. Collecting all these structures, we arrive at a general 
framework termed as `Mixed Unitary categories'(MUCs) which contain a unitary category in a 
larger $\dagger$-isomix category via a $\dagger$-isomix-functor from the unitary category to the 
core of the $\dagger$-isomix category. The unitary category acting on the 
larger $\dagger$-isomix category is analogous to a field $K$ acting on a $K$-algebra as scalars.   
Sections \ref{Sec: MUCs} and \ref{Sec: MUC examples} are dedicated to developing MUCs and 
examples for MUCs.

An important example of a MUC, closely related to traditional CQM, is the $*$-autonomous category of ``finiteness matrices'', 
${\sf FMat}(\C)$, over the complex numbers \cite{Ehr05} (see Example \ref{Sec: Finiteness matrices}).  
 ${\sf FMat}(\C)$ forms a $\dagger$-isomix category which, furthermore, is $\dagger$-$*$-autonomous.  
 The core of this category is isomorphic to the category of finite  matrices over complex numbers which is 
 a well-studied category in CQM. 

Another source of examples (see Example \ref{Section: Chu}) for a MUC is from the Chu construction, \cite{Bar06}, 
where the dualizing object is set to the tensor unit.  The Chu construction over complex vector 
spaces  forms a $\dagger$-isomix category.  From there one can obtain a non-trivial MUC by 
using our unitary construction, or, more directly, by using the fact that the category of Hilbert spaces 
embeds into this category.  To obtain a MUC one must then restrict this last embedding to the finite dimensional Hilbert spaces.

With the MUC framework in place, our next aim is to test the applicability of the MUC framework for quantum mechanics. 
To this end, we generalize the key algebraic structures in CQM to the MUC setting and study the implications of this 
generalization. First, we define the completely positive maps and the $\CP^\infty$ construction for a MUC based on the 
$\CP^\infty$ construction of Coecke and Heunen \cite{CoH16}. We also characterize our $\CP^\infty$ construction with the 
notion of environment structures and purification. The characterization of $\CP^\infty$ construction for MUCs led to 
an elegant observation that it sufficies for the unitary core to have environment maps to characterize 
the $\CP^\infty$ construction for MUCs. 

In the next step, we generalize $\dagger$-Frobenius algebras from the $\dagger$-SMCs to $\dagger$-LDCs using 
$\dagger$-linear monoids. This generalization leads to an interesting effect: owing to the presence of 
two tensor products, $\dagger$-linear monoids leads to a dual notion of $\dagger$-linear comonoids. While linear 
monoids are equipped with a $\ox$-monoid and $\oa$-comonoid, a linear comonoid is equipped with a $\oa$-monoid 
and a $\ox$-comonoid. In a compact LDC, both these structures correspond to Fr\"obenius algebras under certain conditions.
Armed with these general $\dagger$-Frobenius algebras, we investigate the notion of measurement in MUCs. We show 
that in a MUC measurement happens in two steps: first the system to be measured must be compacted into the 
unitary core (i.e., the traditional CQM core), and then apply the usual measurement process \cite{CoP07} within the core. 
The compaction process precisely corresponds to the splitting of certain ``$\dagger$-binary idempotents". 
This is remarkable since the $\dagger$-binary idempotents generalize Selinger's \cite{Sel08}  
$\dagger$-idempotents in $\dagger$-monoidal categories which $\dagger$-splits to produce classical types. 

Our final objective is to capture the notion of complementary measurements within a MUC framework. 
To acheive this objective, we forumulate a bialgebraic interaction between a linear monoid and a linear comonoid. 
This forumlation leads to a structure called a linear bialgebra composed of a linear monoid and a linear comonoid 
interacting to produce a $\ox$-bialgebra and a $\oa$-bialgebra. Note that, in a $\dagger$-monoidal setting, 
such a bialgebraic interaction is considered between two $\dagger$-Frobenius algebras. Working with distinct tensor 
products allows one to see different structures - a linear monoid and a linear comonoid - at play.  
Complementary measurements are modelled by linear bialgebras residing in the (finite-dimensional) unitary core 
and satisfying a few conditions.  Using our formulation of complementary systems, we arrive at a  
significant new result that in a $\dagger$-isomix category, every complementary system inside the canonical unitary core 
arise from an arbitrary-dimensional linear bialgebra (outside the canonical unitary core) induced on the exponential modalities. 
\fi 

\iffalse
The first part of this thesis titled ``Dagger linear logic" includes Chapters \ref{chap: LDC}, \ref{Chap: dagger-LDC}, and \ref{Chap: MUCs} along with their summary in 
Chapter \ref{chap: part 1 summary}. The second part of this thesis titled ``Applications of dagger linear logic to categorical 
quantum mechanics" includes chapters \ref{Chap: CQM}, \ref{Chap: positivity}, \ref{Chapter: compaction}, 
\ref{Chapter: complementarity}, and a summary in \ref{Chap: part 2 summary}. The thesis is concluded and possible 
future directions are discussed in \ref{Chap: conclusion}.
\fi

