% !TEX root = thesis.tex
%Marker words: forward-reference, revisit

\chapter{Categorical semantics for linear logic}
\label{chap: LDC}

Linear logic \cite{Gir87} was introduced by Girard in 1987 as a resource sensitive logic in which 
logical statements were treated as resources. Hence, proving statements in linear logic involves manipulating these 
resources, most of which cannot be duplicated or destroyed. Linear logic has been considered to be the 
logic of quantum information theory due to its resource sensitivity \cite{BaM11, Dun06}. 
In this chapter, a categorical semantics of linear logic is reviewed using linearly distributive categories. We also provide an 
interpretation of the linear logic structures in finiteness spaces \cite{Ehr05} which is used a running example in the 
rest of the thesis. 

\section{Linearly Distributive Categories (LDCs)}
\label{Sec: LDC}

Linearly distributive categories, introduced by Cockett and Seely \cite{CS97} in 1997, provide a categorical semantics of Multiplicative Linear Logic (MLL). 
These categories were originally referred to as Weakly distributive categories and were later renamed 
to linearly distributive categories (LDCs). 

\subsection{Linearly distributive categories}
\label{Subsec: LDC}

Informally, a linearly distributive category is a category having two monoidal structures 
linked by a linear distributor. We first recall the definition of monoidal categories before 
moving on to LDCs:

\begin{definition}
	\label{defn: monoidal}
A {\bf monoidal category} $(\X, \otimes, I, a_\otimes, u_\ox^l, u_\ox^r)$ is a category, $\X$, consisting of:
\begin{itemize}
\item a bifunctor, $\otimes: \X \times \X \rightarrow \X$, called tensor product;
\item a designated object, $I \in \X$, called the unit object;
\item a natural isomorphism, $(a_\otimes)_{A,B,C}: A \otimes (B \otimes C) \xrightarrow{\simeq} (A \otimes B) \otimes C$, called the associator;
\item a natural isomorphism, $ (u_\ox^l)_A: I \otimes A \xrightarrow{\simeq} A$, called the left unitor;
\item a natural isomorphism, $(u_\ox^r)_A: A \otimes I \xrightarrow{\simeq} A$, called the right unitor;
\end{itemize}
such that the following coherence diagrams commute \cite{Kel64}:
\begin{itemize}
\item Maclane's pentagon diagram:
\[ \xymatrixrowsep{8mm} \xymatrix{
 & A\otimes(B \otimes(C \otimes D)) \ar[dl]_{1 \otimes a_\otimes} \ar[dr]^{a_\otimes} &  \\
A \otimes ( (B \otimes C) \otimes D) \ar[d]_{a_\otimes} & &  (A \otimes B) \otimes (C \otimes D) \ar[d]^{a_\otimes} \\
 (A \otimes (B \otimes C)) \otimes D \ar[rr]_{a_\otimes \otimes 1} & &   ((A \otimes B) \otimes C) \otimes D }  \]

\item Kelly's unit diagram:
 \[ \xymatrix{
(A \otimes I) \otimes B \ar[d]_{a_\otimes} \ar[drr]^{u_\otimes^r \otimes 1_B}  \\
A \otimes (I \otimes B) \ar[rr]_{1_A \otimes u_\otimes^l} & & (A \otimes B)} \]
\end{itemize}
\end{definition}

A monoidal category in which the associator, the left unitor and the right unitor are identity arrows is called a {\bf strict monoidal category}. 

A {\bf symmetric monoidal category} (SMC) is a monoidal category with a natural isomorphism:
\[ (c_\otimes)_{A,B}: A \otimes B \xrightarrow{\simeq} B \otimes A \]
such that the following equations hold:
\begin{itemize}
\item (Hexagon law) $a_\ox c_\ox a_\ox = (1 \ox c_\ox) a_\ox (c_\ox \ox 1)$ 
\[ \xymatrix{
	A \otimes (B \otimes C) \ar[r]_{a_\otimes} \ar[d]^{1 \otimes c_\otimes} 
	& (A \otimes B) \otimes C  \ar[r]_{c_\otimes} 
	& C \otimes (A \otimes B) \ar[d]^{a_\ox}\\ 
	\otimes (C \otimes B) \ar[r]_{ a_\ox} 
	& (A \otimes C) \otimes B \ar[r]_{a_\ox (c_\ox \ox 1)} 
	& (C \otimes A) \otimes B 
} \]
\item (Inverse law) $(c_\ox)_{A, B} = (c_\ox)_{B,A}^{-1}$
 \[ \xymatrix{
	A\otimes B \ar@{=}[dr] \ar[d]_{(c_\otimes)_{A,B}}  \\
	B \otimes A  \ar[r]_{(c_\otimes)_{A,B}} & A \otimes  B}\]
\item (Unit law) $c_\ox u_\ox^l = u_\ox^r$
\[ \xymatrix{
	A \otimes I \ar[r]^{c_\otimes} \ar[dr]_{u^r} & I \otimes A \ar[d]^{u^l} \\
	& A & } \]
\end{itemize}

\iffalse
such that the following diagrams commute.
\begin{enumerate}[(a)]
\item Hexagon diagram:
\begin{equation}
\label{eqn: hexagon}
\xymatrix{
& A \otimes (B \otimes C) \ar[dl]_{a_\otimes} \ar[dr]^{1 \otimes c_\otimes} & \\
(A \otimes B) \otimes C \ar[d]_{c_\otimes} &  & A \otimes (C \otimes B) \ar[d]^{a_\otimes} \\
C \otimes (A \otimes B) \ar[dr]_{a_\otimes} & & (A \otimes C) \otimes B \ar[dl]^{c_\otimes \otimes 1} \\
& (C \otimes A) \otimes B &
}
\end{equation}

\item Inverse law:
\begin{equation}
\label{eqn: triangle}
\xymatrix{
A\otimes B \ar@{=}[dr] \ar[d]_{(c_\otimes)_{A,B}}  \\
B \otimes A  \ar[r]_{(c_\otimes)_{A,B}} & A \otimes  }
\end{equation}

\item Unit coherence:
\begin{equation}
\label{unit triangle}
\xymatrix{
A \otimes I \ar[r]^{c_\otimes} \ar[dr]_{u^r} & I \otimes A \ar[d]^{u^l} \\
& A & }
\end{equation}
\end{enumerate}
\fi 

In an LDC, there exists two tensor products which are related by natural transformations called linear distributors:
\begin{definition} \cite{CS97}
A {\bf linearly distributive category}, $(\X, \ox, \top, \oa, \bot)$ is a category $\X$ 
consisting of:

\begin{itemize}

\item a monoidal structure, $(\ox, \top, a_\ox, u_\ox^L, u_\ox^R)$

($\ox$ is referred to as `the tensor' and its unit, $\top$, `the top')

\item a monoidal structure, $(\oa, \bot, a_\oa, u_\oa^L, u_\oa^R)$

($\oa$ is referred to as `the par' and its unit, $\bot$, `the bottom')

\item The tensor and the par are related by the following natural transformations 
which are called the left and the right linear {\bf distributors} respectively:
\begin{align*}
& \partial^l: A \ox (B \oa  C) \to (A \ox B) \oa C \\
& \partial^r: (A \oa B) \ox C \to A \oa (B \ox C) 
\end{align*}
\end{itemize}

satisfying the following coherence conditions:

\begin{itemize}

\item The assosicators and the unitors for the $\ox$ and the $\oa$ satisfy Maclane's pentagon diagram 
and Kelly's unit diagram, See Definition \ref{defn: monoidal}.

\item Coherence conditions for unit natural transformations and linear distributors:

\begin{enumerate}[LDC. 1]
\item \begin{enumerate}[(a)]
\item $\partial^l (u_\ox^l \oa 1_B) = u_\ox^l$

\[ \xymatrixcolsep{4pc} \xymatrix{
\top \ox (A \oa B) \ar[d]_{\partial^l} \ar[dr]^{u_\ox^l} \\
(\top \ox A) \oa B \ar[r]_{u_\ox^l \oa 1_B} & A \oa B } \]

\item $u_\ox^r = \partial^r; 1 \oa u_\ox^l$
\item $ \partial^r u_\oa^l = u_\oa^l \ox 1_B $
\item $1 \ox u_\oa^r = \partial^l; u_\oa^l$

\[ \xymatrixcolsep{4pc}
\xymatrix{ 	(\bot \oa A) \ox B \ar[r]^{\partial^r} \ar[dr]_{u_\oa^l \ox 1_B} & \bot \oa 
(A \ox B) \ar[d]^{u_\oa^l} \\
& A \ox B } \] 

\end{enumerate}
\end{enumerate}

\item Coherences for associativity natural transformations and the distributors:

\begin{enumerate}[LDC. 2]
\item \begin{enumerate}[(a)]
\item $a_\ox (1_A \ox \partial^l) \delta^l = \delta^l (a_\ox \oa 1_D)$


\[ \xymatrixcolsep{4pc} \xymatrix{
(A \ox B) \ox (C \oa D) \ar[r]^{a_\ox} \ar[dd]_{\partial^l} & A \ox (B \ox ( C \oa D )) \ar[d]^{1_A \ox \partial^l} \\
 & A \ox ((B \ox C) \oa D) \ar[d]^{\partial^l} \\
 ((A \ox B) \ox C) \oa D \ar[r]_{a_\ox \oa 1_D} & (A \ox (B \ox C)) \oa D } \]

\item $\partial^l (a_\ox \oa 1) = a_\ox (1 \ox \partial^l) \partial^l$

\item  $\partial^r a_\oa = (a_\oa \ox 1_D) \partial^r (1_A \ox \partial^r)$

\item $(1 \ox a_\oa) \partial^l = \partial^l (1 \oa \partial^l) a_\oa$
\end{enumerate}
\end{enumerate}

\item Coherences between the left and the right linear distributors: 

\begin{enumerate}[LDC. 3]
\item \begin{enumerate}[(a)]
\item $\partial^r (1_A \oa \partial^l) = \partial^l (\partial^r \oa 1_D) a_\ox$
\[ \xymatrix{
 & (A \oa B) \ox (C \oa D) \ar[ld]_{\partial^l} \ar[dr]^{\partial^r} & \\
((A \oa B) \ox C) \oa D) \ar[d]_{\partial^r \oa 1_D} &  & A \oa ( B \ox (C \oa D)) \ar[d]^{1_A \oa \partial^l} \\
(A \oa (B \ox C)) \oa D \ar[rr]_{a_\ox} &  & A \oa ((B \ox C) \oa D) } \]
\item $a_\ox (1 \ox \partial^r) \partial^l = (\partial^l \ox 1) \partial^r$
\end{enumerate}
\end{enumerate}
\end{itemize}

\end{definition}
A {\bf symmetric LDC} is an LDC in which both the tensor products are symmetric,  
with symmetry maps $c_\ox$ and $c_\oa$, such that
$\partial^R = c_\ox (1 \ox c_\oa) \partial^L (c_\ox \oa 1) c_\oa$. 
For a symmetric LDC, the left linear distributor determines the right linear distributor and vice versa.

%TODO: Write a little bit about linear logic here 
Linearly distributive categories (LDCs) provide a categorical semantics for Multiplicative Linear 
Logic (MLL). The two tensor products, $\ox$ and $\oa$, in an LDC corresponds to the multiplicative 
conjunction and multiplicative disjunction in linear logic respectively.
 
%TODO: Mention simple examples here, possibly reference them to a later section.
The following are some examples of LDCs:
\begin{itemize}
\item Every monoidal category is also an LDC where the tensor and the par coincide, and the distributor is 
the associator natural isomorphism. An LDC with the linear distributors being isomorphisms is not 
necessarily monoidal. The next two examples elucidate the point. 

\item A {\bf bounded distributive lattice} is a lattice $(L, \leq, \wedge, \top, \vee, \bot)$ with a greatest element, 
$\top$, and a least element, $\bot$, such that for all $a \in L$, $\bot \leq a \leq \top$, and the join 
$(\wedge)$ and meet $(\vee)$ operations distribute over one another:
\[ a \wedge (b \vee c) = (a \wedge b) \vee (a \wedge c) ~~~~~~~~~~~
   a \vee (b \wedge c) = (a \vee b) \wedge (a \vee c) \]
A distributive lattice regarded as a category whose objects are lattice elements and the maps 
given by the preorder, is an LDC. The tensor is given by $\wedge$ with unit object $\top$,
 and the par is given by $\vee$ with unit object $\bot$. Both the tensor products are 
symmetric. The right linear distributor is given as follows:
%\[ a \vee (b \wedge c) = (a \vee b) \wedge (a \vee c) \leq (a \vee b) \wedge c	\]
\[ (a \vee b) \wedge c = (a \wedge c) \vee (b \wedge c) \leq a \vee (b \wedge c) \]
Any Boolean algebra is an example of a distributive lattice. 

\item Let $M$ be any set. A {\bf shift monoid} is a commutative monoid, $(M, +, 0)$ with a designated 
element $s$ such that there exists an inverse to $s$ i.e, $s - s = 0$. 
A second multiplication can be defined on  the set $M$ as follows: for all $x,y \in M$, $x \circ y = (x + y) - s$. The 
unit of the second multiplication is $s$. A shift monoid considered as a discrete category 
(the elements of the monoid are the objects and the maps are identity maps) is an LDC with 
$\ox := +$, and $\oa := \circ$. The unit objects are given by the units of the respective multiplications. 
The linear distributor is given by the following equality:
\[ x \circ (y + z) := (x + (y + z)) - s = ((x + y ) -s) + z = (x \circ y) + z \]
Note that the distributors are identity maps but $\ox$ and $\oa$ are distinct.

\item $*$-autonomous categories are LDCs with a dualizing object. See Section \ref{Sec: *-autonomous}.

\item The category of bialgebra modules and module homomorphisms of a $*$-autonomous category is an LDC. 
The tensor products are inherited from the base category \cite{CS97}. The category of Hopf modules and 
module homomorphisms from a $*$-autonomous category is $*$-autonomous \cite{PaS09}. We discuss the category of 
Hopf Modules in section \ref{Sec: HModx}. 

\item Girard's Coherence spaces \cite{Gir87}, Ehrhard's finiteness spaces \cite{Ehr05}, 
and Chu Spaces \cite{Bar06} are linearly distributive categories that are also $*$-autonomous. 
 Indeed, $\FRel$, the category of finiteness spaces and finiteness relations, and $\FMat(\C)$, 
 the category of finiteness spaces and finiteness matrices are used as primary examples for 
 structures introduced in this thesis. See Section \ref{Sec: motivating examples} for discussion of these categories. 

\item  Bicompleting a monoidal category (adding arbitrary limits and colimits) gives an LDC. 
A procedure for bicompletion of monoidal categories has been described by Joyal in \cite{Joy95}.  
Joyal also proved in the same article that if the base category is compact closed, then the resulting category is $*$-autonomous.
\end{itemize}


\subsection{Graphical calculus}
\label{Sec: graphical}
LDCs come equipped with a graphical calculus \cite{BCST96} that contains 
the calculus for monoidal categories. Every sequent rule and derivation in MLL corresponds to a 
circuit in the graphical calculus of LDCs, and vice versa. In this section, we review the fundamentals of 
the graphical calculus for LDCs. For detailed exposition, see \cite{BCST96, CS97}.  
The following are the generators of LDC circuits: wires represent objects and circles represent maps. The input wires of a map are tensored (with $\ox$), and the output wires are ``par''ed (with $\oa$). The following diagram represents a map $f: A \ox B \to C \oa D$. 
 \[ \begin{tikzpicture}
	\begin{pgfonlayer}{nodelayer}
		\node [style=circle, scale=2] (0) at (-0.25, -0.5) {};
		\node [style=none] (1) at (-0.25, -0.5) {$f$};
		\node [style=none] (2) at (-1, 0.5) {$A$};
		\node [style=none] (3) at (0.5, 0.5) {$B$};
		\node [style=none] (4) at (-1, -1.5) {$C$};
		\node [style=none] (5) at (0.5, -1.5) {$D$};
		\node [style=ox] (6) at (-0.25, 1) {};
		\node [style=oa] (7) at (-0.25, -2) {};
		\node [style=none] (8) at (-0.25, -2.75) {};
		\node [style=none] (9) at (-0.25, 1.75) {};
		\node [style=none] (10) at (-0.5, -1.65) {};
		\node [style=none] (11) at (0, -1.65) {};
		\node [style=none] (12) at (0, 0.65) {};
		\node [style=none] (13) at (-0.5, 0.65) {};
		\node [style=none] (14) at (-0.25, -3) {$f: A \ox B \to C \oa D$};
	\end{pgfonlayer}
	\begin{pgfonlayer}{edgelayer}
		\draw [bend right=45, looseness=1.25] (6) to (0);
		\draw [bend left=45, looseness=1.25] (6) to (0);
		\draw [bend left=45, looseness=1.25] (7) to (0);
		\draw [bend right=45, looseness=1.25] (7) to (0);
		\draw (9.center) to (6);
		\draw (7) to (8.center);
	\end{pgfonlayer}
\end{tikzpicture}  \]

The $\ox$-associator, the $\oa$-associator, the left linear distributor, and the right linear distributors are, respectively, drawn as follows:
\[(a) ~~~ \begin{tikzpicture}
	\begin{pgfonlayer}{nodelayer}
		\node [style=ox] (0) at (1, -2.5) {};
		\node [style=ox] (1) at (0.5, -0) {};
		\node [style=ox] (2) at (1.75, -1.5) {};
		\node [style=ox] (3) at (1.5, 1.25) {};
		\node [style=none] (4) at (1.5, 2) {};
		\node [style=none] (5) at (1, -3.25) {};
		\node [style=none] (6) at (0.75, -0.5) {};
		\node [style=none] (7) at (0.25, -0.5) {};
		\node [style=none] (8) at (1.25, 1) {};
		\node [style=none] (9) at (1.75, 1) {};
		\node [style=none] (10) at (2.75, 1.75) {$A \ox (B \ox C)$};
		\node [style=none] (11) at (2.25, -3) {$(A \ox B) \ox C$};
		\node [style=none] (12) at (0.5, 1) {$A \ox B$};
		\node [style=none] (13) at (2.25, 0.75) {$C$};
		\node [style=none] (14) at (2.25, -2.25) {$B \ox C$};
		\node [style=none] (15) at (0, -2) {$A$};
	\end{pgfonlayer}
	\begin{pgfonlayer}{edgelayer}
		\draw (5.center) to (0);
		\draw [bend right, looseness=1.00] (0) to (2);
		\draw [in=-120, out=160, looseness=1.00] (0) to (1);
		\draw (3) to (4.center);
		\draw [in=-150, out=75, looseness=0.75] (1) to (3);
		\draw [in=-60, out=120, looseness=1.25] (2) to (1);
		\draw [in=-30, out=75, looseness=1.00] (2) to (3);
	\end{pgfonlayer}
\end{tikzpicture} ~~~~~~~~ (b)~~~~ \begin{tikzpicture}
	\begin{pgfonlayer}{nodelayer}
		\node [style=oa] (0) at (1.5, 0.75) {};
		\node [style=oa] (1) at (2, -1.75) {};
		\node [style=oa] (2) at (0.75, -0.25) {};
		\node [style=oa] (3) at (1, -3) {};
		\node [style=none] (4) at (1, -3.75) {};
		\node [style=none] (5) at (1.5, 1.5) {};
		\node [style=none] (6) at (1.8, -1.35) {};
		\node [style=none] (7) at (2.15, -1.35) {};
		\node [style=none] (8) at (1.35, -2.65) {};
		\node [style=none] (9) at (0.75, -2.65) {};
		\node [style=none] (10) at (2.25, -3.5) {$A \oa (B \oa C)$};
		\node [style=none] (11) at (2.75, 1.25) {$(A \oa B) \oa C$};
		\node [style=none] (12) at (0, 0.5) {$A \oa B$};
		\node [style=none] (13) at (2.25, -2.5) {$B \oa C$};
		\node [style=none] (14) at (1.25, -1.25) {$B$};
		\node [style=none] (15) at (2.5, -0) {$C$};
		\node [style=none] (16) at (0.25, -2.5) {$A$};
	\end{pgfonlayer}
	\begin{pgfonlayer}{edgelayer}
		\draw (5.center) to (0);
		\draw [bend right, looseness=1.00] (0) to (2);
		\draw [in=60, out=-20, looseness=1.00] (0) to (1);
		\draw (3) to (4.center);
		\draw [in=30, out=-105, looseness=0.75] (1) to (3);
		\draw [in=120, out=-60, looseness=1.25] (2) to (1);
		\draw [in=135, out=-130, looseness=1.00] (2) to (3);
	\end{pgfonlayer}
\end{tikzpicture} ~~~~~~~~ (c) ~~~~ \begin{tikzpicture}
	\begin{pgfonlayer}{nodelayer}
		\node [style=ox] (0) at (1.25, 0.75) {};
		\node [style=ox] (1) at (0.5, -1.25) {};
		\node [style=oa] (2) at (2, -0.25) {};
		\node [style=oa] (3) at (1.25, -2.5) {};
		\node [style=none] (4) at (1.25, -3.25) {};
		\node [style=none] (5) at (1.25, 1.5) {};
		\node [style=none] (6) at (1.5, -2.25) {};
		\node [style=none] (7) at (1, -2.25) {};
		\node [style=none] (8) at (1, 0.5) {};
		\node [style=none] (9) at (1.5, 0.5) {};
		\node [style=none] (10) at (2.5, 1.25) {$A \ox (B \oa C)$};
		\node [style=none] (11) at (0.25, -0) {$A$};
		\node [style=none] (12) at (2.55, 0.5) {$B \oa C$};
		\node [style=none] (13) at (1.25, -0.5) {$B$};
		\node [style=none] (14) at (2.5, -1) {$C$};
		\node [style=none] (15) at (0, -2) {$A \ox B$};
		\node [style=none] (16) at (2.55, -3.25) {$(A \ox B) \oa C$};
	\end{pgfonlayer}
	\begin{pgfonlayer}{edgelayer}
		\draw (5.center) to (0);
		\draw [bend left, looseness=1.00] (0) to (2);
		\draw [in=90, out=-150, looseness=1.00] (0) to (1);
		\draw (3) to (4.center);
		\draw [in=150, out=-75, looseness=1.00] (1) to (3);
		\draw [in=60, out=-120, looseness=1.25] (2) to (1);
		\draw [in=30, out=-75, looseness=1.00] (2) to (3);
	\end{pgfonlayer}
\end{tikzpicture} ~~~~~~~~ (d) ~~~~\begin{tikzpicture}
	\begin{pgfonlayer}{nodelayer}
		\node [style=ox] (0) at (1.3, 0.75) {};
		\node [style=ox] (1) at (2.05, -1.25) {};
		\node [style=oa] (2) at (0.55, -0.25) {};
		\node [style=oa] (3) at (1.3, -2.5) {};
		\node [style=none] (4) at (1.3, -3.25) {};
		\node [style=none] (5) at (1.3, 1.5) {};
		\node [style=none] (6) at (1.05, -2.25) {};
		\node [style=none] (7) at (1.55, -2.25) {};
		\node [style=none] (8) at (1.55, 0.5) {};
		\node [style=none] (9) at (1.05, 0.5) {};
		\node [style=none] (10) at (0, 1.25) {$(A \oa B) \ox C$};
		\node [style=none] (11) at (2.3, -0) {$C$};
		\node [style=none] (12) at (0, 0.5) {$A \oa B$};
		\node [style=none] (13) at (1.3, -0.5) {$B$};
		\node [style=none] (14) at (0.04999995, -1) {$A$};
		\node [style=none] (15) at (2.55, -2) {$B \ox C$};
		\node [style=none] (16) at (0, -3.25) {$A \oa (B \ox C)$};
	\end{pgfonlayer}
	\begin{pgfonlayer}{edgelayer}
		\draw (5.center) to (0);
		\draw [bend right, looseness=1.00] (0) to (2);
		\draw [in=90, out=-30, looseness=1.00] (0) to (1);
		\draw (3) to (4.center);
		\draw [in=30, out=-105, looseness=1.00] (1) to (3);
		\draw [in=120, out=-60, looseness=1.25] (2) to (1);
		\draw [in=150, out=-105, looseness=1.00] (2) to (3);
	\end{pgfonlayer}
\end{tikzpicture}  \]

$\begin{tikzpicture}
	\begin{pgfonlayer}{nodelayer}
		\node [style=oa] (0) at (1, -3) {};
		\node [style=none] (1) at (1, -3.75) {};
		\node [style=none] (2) at (1.25, -2.75) {};
		\node [style=none] (3) at (0.75, -2.75) {};
		\node [style=none] (7) at (0.5, -2) {};
		\node [style=none] (8) at (1.5, -2) {};
	\end{pgfonlayer}
	\begin{pgfonlayer}{edgelayer}
		\draw (0) to (1.center);
		\draw [in=-90, out=150, looseness=1.00] (0) to (7.center);
		\draw [in=-90, out=30, looseness=1.00] (0) to (8.center);
	\end{pgfonlayer}
\end{tikzpicture}$ is the $\oa$-introduction ($\oa I$) rule, $\begin{tikzpicture}
	\begin{pgfonlayer}{nodelayer}
		\node [style=ox] (0) at (1, -3) {};
		\node [style=none] (1) at (1, -3.75) {};
		\node [style=none] (2) at (1.25, -2.75) {};
		\node [style=none] (3) at (0.75, -2.75) {};
		\node [style=none] (7) at (0.5, -2) {};
		\node [style=none] (8) at (1.5, -2) {};
	\end{pgfonlayer}
	\begin{pgfonlayer}{edgelayer}
		\draw (0) to (1.center);
		\draw [in=-90, out=150, looseness=1.00] (0) to (7.center);
		\draw [in=-90, out=30, looseness=1.00] (0) to (8.center);
	\end{pgfonlayer}
\end{tikzpicture}$ is $\ox$-introduction ($\ox I$) rule, $\begin{tikzpicture}
	\begin{pgfonlayer}{nodelayer}
		\node [style=ox] (0) at (1, -2.75) {};
		\node [style=none] (1) at (1, -2) {};
		\node [style=none] (2) at (1.25, -3) {};
		\node [style=none] (3) at (0.75, -3) {};
		\node [style=none] (7) at (0.5, -3.75) {};
		\node [style=none] (8) at (1.5, -3.75) {};
	\end{pgfonlayer}
	\begin{pgfonlayer}{edgelayer}
		\draw (0) to (1.center);
		\draw [in=90, out=-150, looseness=1.00] (0) to (7.center);
		\draw [in=90, out=-30, looseness=1.00] (0) to (8.center);
	\end{pgfonlayer}
\end{tikzpicture}$ is the $\ox$-elimination ($\ox E$) rule, $\begin{tikzpicture}
	\begin{pgfonlayer}{nodelayer}
		\node [style=oa] (0) at (1, -2.75) {};
		\node [style=none] (1) at (1, -2) {};
		\node [style=none] (2) at (0.5, -3.75) {};
		\node [style=none] (3) at (1.5, -3.75) {};
	\end{pgfonlayer}
	\begin{pgfonlayer}{edgelayer}
		\draw (0) to (1.center);
		\draw [in=90, out=-150, looseness=1.00] (0) to (2.center);
		\draw [in=90, out=-30, looseness=1.00] (0) to (3.center);
	\end{pgfonlayer}
\end{tikzpicture}$ is the $\oa$-elimination ($\oa E$) rule. As shown below, the rules, $(\ox I)$ and $(\ox E)$, 
correspond to the sequent rules for tensor introduction, $\ox L$ and $\ox R$ in Figure \ref{Fig: multiplicative rules}: 
\[ \infer[(\ox L)]{\Gamma, A \ox B, \Gamma' \vdash \Delta }{\Gamma, A, B, \Gamma' \vdash \Delta} 
~~~~~~~~
 \infer[(\ox R)]{\Gamma_1, \Delta_1 \vdash \Gamma_2, \Delta_2, A \ox B, \Gamma_3, \Delta_3 }
{\Gamma_1  \vdash \Gamma_2, A, \Gamma_3 & \Delta_1  \vdash \Delta_2, B, \Delta_3}\]
Similarly, $\oa$I and $\oa$E correspond to $\oa$L and $\oa$R respectively, see Figure \ref{Fig: multiplicative rules}.

The unitors are drawn as follows:
\[ (a) ~~~~ \begin{tikzpicture}
	\begin{pgfonlayer}{nodelayer}
		\node [style=circle] (0) at (0, -0) {$\top$};
		\node [style=none] (1) at (0, -2) {};
		\node [style=none] (2) at (0.75, -0) {};
		\node [style=none] (3) at (0.75, -2) {};
		\node [style=none] (4) at (1, -1) {$A$};
		\node [style=none] (5) at (0, -2.5) {$(u_\ox^L)^{-1}: A \to \top \ox A$};
	\end{pgfonlayer}
	\begin{pgfonlayer}{edgelayer}
		\draw (0) to (1.center);
	\end{pgfonlayer}
\end{tikzpicture} ~~~~~~~~ (b) ~~~~ \begin{tikzpicture}
	\begin{pgfonlayer}{nodelayer}
		\node [style=circle] (0) at (0, -0) {$\top$};
		\node [style=none] (1) at (0.75, 0.75) {};
		\node [style=none] (2) at (0.75, -2) {};
		\node [style=none] (3) at (1, -0.25) {$A$};
		\node [style=none] (4) at (0.25, -2.5) {$u_\ox^L: \top \ox A \to A$};
		\node [style=circle, scale=0.6] (5) at (0.75, -1.25) {};
		\node [style=none] (6) at (0, 0.75) {};
	\end{pgfonlayer}
	\begin{pgfonlayer}{edgelayer}
		\draw (2.center) to (1.center);
		\draw [dotted, bend right, looseness=1.25] (0) to (5);
		\draw (6.center) to (0);
	\end{pgfonlayer}
\end{tikzpicture} ~~~~~~~~ (c) ~~~ \begin{tikzpicture}
	\begin{pgfonlayer}{nodelayer}
		\node [style=circle] (0) at (0, -2.5) {$\bot$};
		\node [style=none] (1) at (0.75, -3.5) {};
		\node [style=none] (2) at (0.75, -0.5) {};
		\node [style=none] (3) at (1, -0.75) {$A$};
		\node [style=none] (4) at (0.5, -3.75) {$(u_\oa^L)^{-1}:  A \to \bot \oa A$};
		\node [style=circle, scale=0.6] (5) at (0.75, -1.5) {};
		\node [style=none] (6) at (0, -3.5) {};
	\end{pgfonlayer}
	\begin{pgfonlayer}{edgelayer}
		\draw (1.center) to (2.center);
		\draw [dotted, bend left, looseness=1.25, dotted] (0) to (5);
		\draw (6.center) to (0);
	\end{pgfonlayer}
\end{tikzpicture} ~~~~~~~ (d) ~~~ \begin{tikzpicture}
	\begin{pgfonlayer}{nodelayer}
		\node [style=circle] (0) at (0, -2.75) {$\bot$};
		\node [style=none] (1) at (0.75, -0.75) {};
		\node [style=none] (2) at (0.75, -3) {};
		\node [style=none] (3) at (1, -2.75) {$A$};
		\node [style=none] (4) at (0.25, -3.5) {$u_\oa^L:  \bot \oa A \to A$};
		\node [style=none] (5) at (0, -0.75) {};
	\end{pgfonlayer}
	\begin{pgfonlayer}{edgelayer}
		\draw (5.center) to (0);
	\end{pgfonlayer}
\end{tikzpicture}  \]

Diagram $(a)$ is called the left $\top$-introduction, $(b)$ is called the left $\top$-elimination, 
$(c)$ is the left $\bot$-introduction, and $(d)$ is the left $\bot$-elimination which correspond 
to the sequent rules $(\top R)$,  $(\top L)$,  $(\bot R)$, $(\bot L)$ in Figure \ref{Fig: multiplicative rules} respectively. 
The unit $\top$ is introduced, and the counit $\bot$ is eliminated using the thinning links 
which are shown using dotted wires in the diagrams. See \cite[Section 2.3]{BCST96} for details 
on the thinning links.

The following are a set of circuit equalities (which when oriented become reduction rewrite rules):
\[ [Reduction]: ~~~ \begin{tikzpicture}
	\begin{pgfonlayer}{nodelayer} 
		\node [style=circle] (0) at (0, -1) {$\top$};
		\node [style=none] (1) at (0.75, 0.25) {};
		\node [style=none] (2) at (0.75, -3) {};
		\node [style=none] (3) at (1, -1.25) {$A$};
		\node [style=circle, scale = 0.4] (4) at (0.75, -2.25) {};
		\node [style=circle] (5) at (0, -0) {$\top$};
	\end{pgfonlayer}
	\begin{pgfonlayer}{edgelayer}
		\draw (1.center) to (2.center);
		\draw [dotted, bend right, looseness=1.25] (0) to (4);
		\draw (5) to (0);
	\end{pgfonlayer}
\end{tikzpicture} =  \begin{tikzpicture}
	\begin{pgfonlayer}{nodelayer}
		\node [style=none] (0) at (2.25, -1.5) {$A$};
		\node [style=none] (1) at (2, 0.25) {};
		\node [style=none] (2) at (2, -3) {};
	\end{pgfonlayer}
	\begin{pgfonlayer}{edgelayer}
		\draw (1.center) to (2.center);
	\end{pgfonlayer}
\end{tikzpicture}
~~~~~~~~
\begin{tikzpicture}
	\begin{pgfonlayer}{nodelayer}
		\node [style=circle] (0) at (0, -1.75) {$\bot$};
		\node [style=none] (1) at (0.75, -3) {};
		\node [style=none] (2) at (0.75, 0.25) {};
		\node [style=none] (3) at (1, -1.5) {$A$};
		\node [style=circle, scale=0.4] (4) at (0.75, -0.5) {};
		\node [style=circle] (5) at (0, -2.75) {$\bot$};
	\end{pgfonlayer}
	\begin{pgfonlayer}{edgelayer}
		\draw (1.center) to (2.center);
		\draw [dotted, bend left, looseness=1.25] (0) to (4);
		\draw (5) to (0);
	\end{pgfonlayer}
\end{tikzpicture} = \begin{tikzpicture}
	\begin{pgfonlayer}{nodelayer}
		\node [style=none] (0) at (2.25, -1.5) {$A$};
		\node [style=none] (1) at (2, 0.25) {};
		\node [style=none] (2) at (2, -3) {};
	\end{pgfonlayer}
	\begin{pgfonlayer}{edgelayer}
		\draw (1.center) to (2.center);
	\end{pgfonlayer}
\end{tikzpicture}
~~~~~~~~
\begin{tikzpicture}
	\begin{pgfonlayer}{nodelayer}
		\node [style=ox] (0) at (2, -0) {};
		\node [style=ox] (1) at (2, -1) {};
		\node [style=none] (2) at (1.5, -2) {};
		\node [style=none] (3) at (2.5, -2) {};
		\node [style=none] (4) at (1.5, 1) {};
		\node [style=none] (5) at (2.5, 1) {};
		\node [style=none] (6) at (1.25, 0.75) {$A$};
		\node [style=none] (7) at (2.75, 0.75) {$B$};
		\node [style=none] (8) at (1.25, -1.75) {$A$};
		\node [style=none] (9) at (2.75, -1.75) {$B$};
	\end{pgfonlayer}
	\begin{pgfonlayer}{edgelayer}
		\draw (0) to (1);
		\draw [in=90, out=-135, looseness=1.00] (1) to (2.center);
		\draw [in=90, out=-45, looseness=1.00] (1) to (3.center);
		\draw [in=-90, out=135, looseness=1.00] (0) to (4.center);
		\draw [in=-90, out=45, looseness=1.00] (0) to (5.center);
	\end{pgfonlayer}
\end{tikzpicture} = \begin{tikzpicture}
	\begin{pgfonlayer}{nodelayer}
		\node [style=none] (0) at (1.5, -2) {};
		\node [style=none] (1) at (2, -2) {};
		\node [style=none] (2) at (1.5, 1) {};
		\node [style=none] (3) at (2, 1) {};
		\node [style=none] (4) at (1.25, 0.75) {$A$};
		\node [style=none] (5) at (2.25, 0.75) {$B$};
		\node [style=none] (6) at (1.25, -1.75) {$A$};
		\node [style=none] (7) at (2.25, -1.75) {$B$};
	\end{pgfonlayer}
	\begin{pgfonlayer}{edgelayer}
		\draw (2.center) to (0.center);
		\draw (3.center) to (1.center);
	\end{pgfonlayer}
\end{tikzpicture}
~~~~~~~~
\begin{tikzpicture}
	\begin{pgfonlayer}{nodelayer}
		\node [style=oa] (0) at (2, -0) {};
		\node [style=oa] (1) at (2, -1) {};
		\node [style=none] (2) at (1.5, -2) {};
		\node [style=none] (3) at (2.5, -2) {};
		\node [style=none] (4) at (1.5, 1) {};
		\node [style=none] (5) at (2.5, 1) {};
		\node [style=none] (6) at (1.25, 0.75) {$A$};
		\node [style=none] (7) at (2.75, 0.75) {$B$};
		\node [style=none] (8) at (1.25, -1.75) {$A$};
		\node [style=none] (9) at (2.75, -1.75) {$B$};
	\end{pgfonlayer}
	\begin{pgfonlayer}{edgelayer}
		\draw (0) to (1);
		\draw [in=90, out=-135, looseness=1.00] (1) to (2.center);
		\draw [in=90, out=-45, looseness=1.00] (1) to (3.center);
		\draw [in=-90, out=135, looseness=1.00] (0) to (4.center);
		\draw [in=-90, out=45, looseness=1.00] (0) to (5.center);
	\end{pgfonlayer}
\end{tikzpicture} = \begin{tikzpicture}
	\begin{pgfonlayer}{nodelayer}
		\node [style=none] (0) at (1.5, -2) {};
		\node [style=none] (1) at (2, -2) {};
		\node [style=none] (2) at (1.5, 1) {};
		\node [style=none] (3) at (2, 1) {};
		\node [style=none] (4) at (1.25, 0.75) {$A$};
		\node [style=none] (5) at (2.25, 0.75) {$B$};
		\node [style=none] (6) at (1.25, -1.75) {$A$};
		\node [style=none] (7) at (2.25, -1.75) {$B$};
	\end{pgfonlayer}
	\begin{pgfonlayer}{edgelayer}
		\draw (2.center) to (0.center);
		\draw (3.center) to (1.center);
	\end{pgfonlayer}
\end{tikzpicture} \]
The following are also circuit equalities (and when oriented become expansion rules:)
\[ [Expansion]: ~~~ \begin{tikzpicture}
	\begin{pgfonlayer}{nodelayer}
		\node [style=ox] (0) at (0, -0.75) {};
		\node [style=ox] (1) at (0, -2) {};
		\node [style=none] (2) at (0, -3) {};
		\node [style=none] (3) at (0, -0) {};
	\end{pgfonlayer}
	\begin{pgfonlayer}{edgelayer}
		\draw [bend right=60, looseness=1.50] (0) to (1);
		\draw [bend right=60, looseness=1.50] (1) to (0);
		\draw (3.center) to (0);
		\draw (1) to (2.center);
	\end{pgfonlayer}
\end{tikzpicture} = \begin{tikzpicture}
	\begin{pgfonlayer}{nodelayer}
		\node [style=none] (0) at (1, -0) {};
		\node [style=none] (1) at (1, -3) {};
		\node [style=none] (2) at (1.65, -1.75) {$A \ox B$};
	\end{pgfonlayer}
	\begin{pgfonlayer}{edgelayer}
		\draw (0.center) to (1.center);
	\end{pgfonlayer}
\end{tikzpicture}
~~~~~~~~
\begin{tikzpicture}
	\begin{pgfonlayer}{nodelayer}
		\node [style=oa] (0) at (0, -0.75) {};
		\node [style=oa] (1) at (0, -2) {};
		\node [style=none] (2) at (0, -3) {};
		\node [style=none] (3) at (0, -0) {};
	\end{pgfonlayer}
	\begin{pgfonlayer}{edgelayer}
		\draw [bend right=60, looseness=1.50] (0) to (1);
		\draw [bend right=60, looseness=1.50] (1) to (0);
		\draw (3.center) to (0);
		\draw (1) to (2.center);
	\end{pgfonlayer}
\end{tikzpicture} = \begin{tikzpicture}
	\begin{pgfonlayer}{nodelayer}
		\node [style=none] (0) at (1, -0) {};
		\node [style=none] (1) at (1, -3) {};
		\node [style=none] (2) at (1.65, -1.75) {$A \oa B$};
	\end{pgfonlayer}
	\begin{pgfonlayer}{edgelayer}
		\draw (0.center) to (1.center);
	\end{pgfonlayer}
\end{tikzpicture}
~~~~~~~~~
\begin{tikzpicture}
	\begin{pgfonlayer}{nodelayer}
		\node [style=circle] (0) at (0, -0.75) {$\top$};
		\node [style=circle] (1) at (-1, -1.5) {$\top$};
		\node [style=none] (2) at (-1, -3.25) {};
		\node [style=none] (3) at (0, 0) {};
		\node [style=circle, scale=0.4] (4) at (-1, -2.5) {};
	\end{pgfonlayer}
	\begin{pgfonlayer}{edgelayer}
		\draw (3.center) to (0);
		\draw (1) to (2.center);
		\draw [dotted, in=-90, out=30, looseness=1.25] (4) to (0);
	\end{pgfonlayer}
\end{tikzpicture} = \begin{tikzpicture}
	\begin{pgfonlayer}{nodelayer}
		\node [style=none] (0) at (1, -0) {};
		\node [style=none] (1) at (1, -3.25) {};
		\node [style=none] (2) at (1.25, -1.75) {$\top$};
	\end{pgfonlayer}
	\begin{pgfonlayer}{edgelayer}
		\draw (0.center) to (1.center);
	\end{pgfonlayer}
\end{tikzpicture}
~~~~~~~~~
\begin{tikzpicture}
	\begin{pgfonlayer}{nodelayer}
		\node [style=circle] (0) at (0, -2.5) {$\bot$};
		\node [style=circle] (1) at (-1, -1.75) {$\bot$};
		\node [style=none] (2) at (-1, 0) {};
		\node [style=none] (3) at (0, -3.25) {};
		\node [style=circle, scale=0.4] (4) at (-1, -0.75) {};
	\end{pgfonlayer}
	\begin{pgfonlayer}{edgelayer}
		\draw (3.center) to (0);
		\draw (1) to (2.center);
		\draw [dotted, in=90, out=-30, looseness=1.25] (4) to (0);
	\end{pgfonlayer}
\end{tikzpicture} = \begin{tikzpicture}
	\begin{pgfonlayer}{nodelayer}
		\node [style=none] (0) at (1, -0) {};
		\node [style=none] (1) at (1, -3.25) {};
		\node [style=none] (2) at (1.25, -1.75) {$\bot$};
	\end{pgfonlayer}
	\begin{pgfonlayer}{edgelayer}
		\draw (0.center) to (1.center);
	\end{pgfonlayer}
\end{tikzpicture} \]

%A given $\ox$-$\oa$-circuit is valid is precislely when all the circuit components and edges form a connected directed acyclic graph. This is called the boxing criterion \cite{BCST96} and a valid circuit is also called a proof net. 

As in linear logic, not all circuit diagrams constructed from these basic components represent a valid LDC circuit.  In his seminal paper on linear logic, \cite{Gir87}, Girard introduced a criterion for the correctness of his representation of proofs using proof nets based on switching links.  A valid proof structure must be connected and acyclic for all the switching link choices.  Using this correctness criterion has the disadvantage of requiring exponential time in the number of switching links.  Danos and Regnier \cite{DaR89} improved this situation significantly by providing an algorithm for correctness which takes linear time (see \cite{Gue99})
%S. Guerrini
%Correctness of multiplicative proof nets is linear
%14th Annual IEEE Symposium on Logic in Computer Science, LICS 1999, IEEE Computer Society, Trento, Italy (1999), pp. 454-463 
on the size of the circuit.  To verify the validity of the circuit diagrams of LDCs, Blute et.al. \cite{BCST96}, provided a boxing algorithm which was based on Danos and Regnier's more efficient algorithm which we now describe.

In order to verify that an LDC circuit is valid, circuit components are 
``boxed''  using the rules below. The primitive generating maps are 
automatically boxed. 
\[ (a_1)~~~\begin{tikzpicture}
	\begin{pgfonlayer}{nodelayer}
		\node [style=ox] (0) at (0, 2) {};
		\node [style=none] (1) at (0, 1.25) {};
		\node [style=none] (2) at (-0.5, 2.75) {};
		\node [style=none] (3) at (0.5, 2.75) {};
	\end{pgfonlayer}
	\begin{pgfonlayer}{edgelayer}
		\draw (0) to (1.center);
		\draw [in=150, out=-90, looseness=1.00] (2.center) to (0);
		\draw [in=-90, out=30, looseness=1.00] (0) to (3.center);
	\end{pgfonlayer}
\end{tikzpicture} \Rightarrow \begin{tikzpicture}
	\begin{pgfonlayer}{nodelayer}
		\node [style=none] (0) at (0, 1) {};
		\node [style=none] (1) at (-0.25, 2.75) {};
		\node [style=none] (2) at (0.25, 2.75) {};
		\node [style=none] (3) at (-0.5, 1.5) {};
		\node [style=none] (4) at (0.5, 1.5) {};
		\node [style=none] (5) at (-0.5, 2.25) {};
		\node [style=none] (6) at (0.5, 2.25) {};
		\node [style=none] (7) at (0.25, 2.25) {};
		\node [style=none] (8) at (-0.25, 2.25) {};
		\node [style=none] (9) at (0, 1.5) {};
	\end{pgfonlayer}
	\begin{pgfonlayer}{edgelayer}
		\draw (3.center) to (5.center);
		\draw (5.center) to (6.center);
		\draw (6.center) to (4.center);
		\draw (4.center) to (3.center);
		\draw (0.center) to (9.center);
		\draw (8.center) to (1.center);
		\draw (7.center) to (2.center);
	\end{pgfonlayer}
\end{tikzpicture}
~~~~~~~~
(a_2)~~~\begin{tikzpicture}[yscale=-1]
	\begin{pgfonlayer}{nodelayer}
		\node [style=oa] (0) at (0, 2) {};
		\node [style=none] (1) at (0, 1.25) {};
		\node [style=none] (2) at (-0.5, 2.75) {};
		\node [style=none] (3) at (0.5, 2.75) {};
	\end{pgfonlayer}
	\begin{pgfonlayer}{edgelayer}
		\draw (0) to (1.center);
		\draw [in=150, out=-90, looseness=1.00] (2.center) to (0);
		\draw [in=-90, out=30, looseness=1.00] (0) to (3.center);
	\end{pgfonlayer}
\end{tikzpicture} \Rightarrow \begin{tikzpicture}[yscale=-1]
	\begin{pgfonlayer}{nodelayer}
		\node [style=none] (0) at (0, 1) {};
		\node [style=none] (1) at (-0.25, 2.75) {};
		\node [style=none] (2) at (0.25, 2.75) {};
		\node [style=none] (3) at (-0.5, 1.5) {};
		\node [style=none] (4) at (0.5, 1.5) {};
		\node [style=none] (5) at (-0.5, 2.25) {};
		\node [style=none] (6) at (0.5, 2.25) {};
		\node [style=none] (7) at (0.25, 2.25) {};
		\node [style=none] (8) at (-0.25, 2.25) {};
		\node [style=none] (9) at (0, 1.5) {};
	\end{pgfonlayer}
	\begin{pgfonlayer}{edgelayer}
		\draw (3.center) to (5.center);
		\draw (5.center) to (6.center);
		\draw (6.center) to (4.center);
		\draw (4.center) to (3.center);
		\draw (0.center) to (9.center);
		\draw (8.center) to (1.center);
		\draw (7.center) to (2.center);
	\end{pgfonlayer}
\end{tikzpicture}
~~~~~~~~
(b_1)~~~  \begin{tikzpicture}
	\begin{pgfonlayer}{nodelayer}
		\node [style=none] (0) at (0, 1) {};
		\node [style=none] (1) at (-1.25, 1.5) {};
		\node [style=none] (2) at (1.25, 1.5) {};
		\node [style=none] (3) at (-1.25, 2.25) {};
		\node [style=none] (4) at (1.25, 2.25) {};
		\node [style=none] (5) at (0.5, 2.25) {};
		\node [style=none] (6) at (-0.5, 2.25) {};
		\node [style=none] (7) at (0, 1.5) {};
		\node [style=ox] (8) at (0, 2.75) {};
		\node [style=none] (9) at (0, 3.25) {};
		\node [style=none] (10) at (0.75, 3.25) {};
		\node [style=none] (11) at (0.75, 2.25) {};
		\node [style=none] (12) at (-0.75, 3.25) {};
		\node [style=none] (13) at (-0.75, 2.25) {};
		\node [style=none] (14) at (1, 2.25) {};
		\node [style=none] (15) at (1, 3.25) {};
		\node [style=none] (16) at (-1, 2.25) {};
		\node [style=none] (17) at (-1, 3.25) {};
		\node [style=none] (18) at (0.25, 1) {};
		\node [style=none] (19) at (0.25, 1.5) {};
	\end{pgfonlayer}
	\begin{pgfonlayer}{edgelayer}
		\draw (1.center) to (3.center);
		\draw (3.center) to (4.center);
		\draw (4.center) to (2.center);
		\draw (2.center) to (1.center);
		\draw (0.center) to (7.center);
		\draw [in=90, out=-165, looseness=1.25] (8) to (6.center);
		\draw [in=90, out=-15, looseness=1.25] (8) to (5.center);
		\draw (8) to (9.center);
		\draw (11.center) to (10.center);
		\draw (13.center) to (12.center);
		\draw (14.center) to (15.center);
		\draw (16.center) to (17.center);
		\draw (18.center) to (19.center);
	\end{pgfonlayer}
\end{tikzpicture} \Rightarrow \begin{tikzpicture}
	\begin{pgfonlayer}{nodelayer}
		\node [style=none] (0) at (0, 1) {};
		\node [style=none] (1) at (-1, 1.5) {};
		\node [style=none] (2) at (1, 1.5) {};
		\node [style=none] (3) at (-1, 2.75) {};
		\node [style=none] (4) at (1, 2.75) {};
		\node [style=none] (5) at (0, 2.75) {};
		\node [style=none] (6) at (-0.75, 2.75) {};
		\node [style=none] (7) at (0, 1.5) {};
		\node [style=none] (8) at (0, 3.25) {};
		\node [style=none] (9) at (0.75, 3.25) {};
		\node [style=none] (10) at (0.75, 2.75) {};
		\node [style=none] (11) at (-0.75, 3.25) {};
		\node [style=none] (12) at (-0.5, 2.75) {};
		\node [style=none] (13) at (-0.5, 3.25) {};
		\node [style=none] (14) at (0.5, 2.75) {};
		\node [style=none] (15) at (0.5, 3.25) {};
		\node [style=none] (16) at (-0.25, 1) {};
		\node [style=none] (17) at (-0.25, 1.5) {};
	\end{pgfonlayer}
	\begin{pgfonlayer}{edgelayer}
		\draw (1.center) to (3.center);
		\draw (3.center) to (4.center);
		\draw (4.center) to (2.center);
		\draw (2.center) to (1.center);
		\draw (0.center) to (7.center);
		\draw (10.center) to (9.center);
		\draw (8.center) to (5.center);
		\draw (11.center) to (6.center);
		\draw (13.center) to (12.center);
		\draw (15.center) to (14.center);
		\draw (17.center) to (16.center);
	\end{pgfonlayer}
\end{tikzpicture}
~~~~~~~~
(b_2)~~~  \begin{tikzpicture}[yscale=-1]
	\begin{pgfonlayer}{nodelayer}
		\node [style=none] (0) at (0, 1) {};
		\node [style=none] (1) at (-1.25, 1.5) {};
		\node [style=none] (2) at (1.25, 1.5) {};
		\node [style=none] (3) at (-1.25, 2.25) {};
		\node [style=none] (4) at (1.25, 2.25) {};
		\node [style=none] (5) at (0.5, 2.25) {};
		\node [style=none] (6) at (-0.5, 2.25) {};
		\node [style=none] (7) at (0, 1.5) {};
		\node [style=oa] (8) at (0, 2.75) {};
		\node [style=none] (9) at (0, 3.25) {};
		\node [style=none] (10) at (0.75, 3.25) {};
		\node [style=none] (11) at (0.75, 2.25) {};
		\node [style=none] (12) at (-0.75, 3.25) {};
		\node [style=none] (13) at (-0.75, 2.25) {};
		\node [style=none] (14) at (1, 2.25) {};
		\node [style=none] (15) at (1, 3.25) {};
		\node [style=none] (16) at (-1, 2.25) {};
		\node [style=none] (17) at (-1, 3.25) {};
		\node [style=none] (18) at (0.25, 1) {};
		\node [style=none] (19) at (0.25, 1.5) {};
	\end{pgfonlayer}
	\begin{pgfonlayer}{edgelayer}
		\draw (1.center) to (3.center);
		\draw (3.center) to (4.center);
		\draw (4.center) to (2.center);
		\draw (2.center) to (1.center);
		\draw (0.center) to (7.center);
		\draw [in=90, out=-165, looseness=1.25] (8) to (6.center);
		\draw [in=90, out=-15, looseness=1.25] (8) to (5.center);
		\draw (8) to (9.center);
		\draw (11.center) to (10.center);
		\draw (13.center) to (12.center);
		\draw (14.center) to (15.center);
		\draw (16.center) to (17.center);
		\draw (18.center) to (19.center);
	\end{pgfonlayer}
\end{tikzpicture} \Rightarrow \begin{tikzpicture}[yscale=-1]
	\begin{pgfonlayer}{nodelayer}
		\node [style=none] (0) at (0, 1) {};
		\node [style=none] (1) at (-1, 1.5) {};
		\node [style=none] (2) at (1, 1.5) {};
		\node [style=none] (3) at (-1, 2.75) {};
		\node [style=none] (4) at (1, 2.75) {};
		\node [style=none] (5) at (0, 2.75) {};
		\node [style=none] (6) at (-0.75, 2.75) {};
		\node [style=none] (7) at (0, 1.5) {};
		\node [style=none] (8) at (0, 3.25) {};
		\node [style=none] (9) at (0.75, 3.25) {};
		\node [style=none] (10) at (0.75, 2.75) {};
		\node [style=none] (11) at (-0.75, 3.25) {};
		\node [style=none] (12) at (-0.5, 2.75) {};
		\node [style=none] (13) at (-0.5, 3.25) {};
		\node [style=none] (14) at (0.5, 2.75) {};
		\node [style=none] (15) at (0.5, 3.25) {};
		\node [style=none] (16) at (-0.25, 1) {};
		\node [style=none] (17) at (-0.25, 1.5) {};
	\end{pgfonlayer}
	\begin{pgfonlayer}{edgelayer}
		\draw (1.center) to (3.center);
		\draw (3.center) to (4.center);
		\draw (4.center) to (2.center);
		\draw (2.center) to (1.center);
		\draw (0.center) to (7.center);
		\draw (10.center) to (9.center);
		\draw (8.center) to (5.center);
		\draw (11.center) to (6.center);
		\draw (13.center) to (12.center);
		\draw (15.center) to (14.center);
		\draw (17.center) to (16.center);
	\end{pgfonlayer}
\end{tikzpicture}
\]
\[ (c) ~~~ \begin{tikzpicture}
	\begin{pgfonlayer}{nodelayer}
		\node [style=none] (0) at (-0.25, 1) {};
		\node [style=none] (1) at (1, 1) {};
		\node [style=none] (2) at (-0.25, 1.75) {};
		\node [style=none] (3) at (1, 1.75) {};
		\node [style=none] (4) at (0.5, 0.5) {};
		\node [style=none] (5) at (1.75, 0.5) {};
		\node [style=none] (6) at (1.75, -0.25) {};
		\node [style=none] (7) at (0.5, -0.25) {};
		\node [style=none] (8) at (0.5, 1) {};
		\node [style=none] (9) at (1, 0.5) {};
		\node [style=none] (10) at (0.25, 1.75) {};
		\node [style=none] (11) at (0.5, 1.75) {};
		\node [style=none] (12) at (1.5, 0.5) {};
		\node [style=none] (13) at (1.25, 0.5) {};
		\node [style=none] (14) at (0, 1) {};
		\node [style=none] (15) at (0.25, 1) {};
		\node [style=none] (16) at (0, -0.75) {};
		\node [style=none] (17) at (0.25, -0.75) {};
		\node [style=none] (18) at (1.25, 2.25) {};
		\node [style=none] (19) at (1.5, 2.25) {};
		\node [style=none] (20) at (0.25, 2.25) {};
		\node [style=none] (21) at (0.5, 2.25) {};
		\node [style=none] (22) at (1, -0.75) {};
		\node [style=none] (23) at (1.25, -0.75) {};
		\node [style=none] (24) at (1, -0.25) {};
		\node [style=none] (25) at (1.25, -0.25) {};
	\end{pgfonlayer}
	\begin{pgfonlayer}{edgelayer}
		\draw (0.center) to (2.center);
		\draw (2.center) to (3.center);
		\draw (3.center) to (1.center);
		\draw (1.center) to (0.center);
		\draw (7.center) to (4.center);
		\draw (4.center) to (5.center);
		\draw (5.center) to (6.center);
		\draw (6.center) to (7.center);
		\draw (14.center) to (16.center);
		\draw (15.center) to (17.center);
		\draw (8.center) to (9.center);
		\draw (18.center) to (13.center);
		\draw (19.center) to (12.center);
		\draw (20.center) to (10.center);
		\draw (21.center) to (11.center);
		\draw (24.center) to (22.center);
		\draw (25.center) to (23.center);
	\end{pgfonlayer}
\end{tikzpicture} \Rightarrow \begin{tikzpicture}
	\begin{pgfonlayer}{nodelayer}
		\node [style=none] (0) at (-0.25, 1.75) {};
		\node [style=none] (1) at (1.5, 1.75) {};
		\node [style=none] (2) at (1.5, -0.25) {};
		\node [style=none] (3) at (-0.25, -0.25) {};
		\node [style=none] (4) at (0, 1.75) {};
		\node [style=none] (5) at (0.25, 1.75) {};
		\node [style=none] (6) at (1.25, 1.75) {};
		\node [style=none] (7) at (1, 1.75) {};
		\node [style=none] (8) at (0, -0.25) {};
		\node [style=none] (9) at (0.25, -0.25) {};
		\node [style=none] (10) at (0, -0.75) {};
		\node [style=none] (11) at (0.25, -0.75) {};
		\node [style=none] (12) at (1, 2.25) {};
		\node [style=none] (13) at (1.25, 2.25) {};
		\node [style=none] (14) at (0, 2.25) {};
		\node [style=none] (15) at (0.25, 2.25) {};
		\node [style=none] (16) at (1, -0.75) {};
		\node [style=none] (17) at (1.25, -0.75) {};
		\node [style=none] (18) at (1, -0.25) {};
		\node [style=none] (19) at (1.25, -0.25) {};
	\end{pgfonlayer}
	\begin{pgfonlayer}{edgelayer}
		\draw (0.center) to (1.center);
		\draw (2.center) to (3.center);
		\draw (8.center) to (10.center);
		\draw (9.center) to (11.center);
		\draw (12.center) to (7.center);
		\draw (13.center) to (6.center);
		\draw (14.center) to (4.center);
		\draw (15.center) to (5.center);
		\draw (18.center) to (16.center);
		\draw (19.center) to (17.center);
		\draw (0.center) to (3.center);
		\draw (1.center) to (2.center);
	\end{pgfonlayer}
\end{tikzpicture}
~~~~~~~~
(d_1) ~~~ \begin{tikzpicture}
	\begin{pgfonlayer}{nodelayer}
		\node [style=circle, scale=2] (0) at (0, -0) {};
		\node [style=none] (1) at (0, -0) {$\bot$};
		\node [style=none] (2) at (0, 2) {};
	\end{pgfonlayer}
	\begin{pgfonlayer}{edgelayer}
		\draw (2.center) to (0);
	\end{pgfonlayer}
\end{tikzpicture} \Rightarrow \begin{tikzpicture}
	\begin{pgfonlayer}{nodelayer}
		\node [style=none] (0) at (0, 2) {};
		\node [style=none] (1) at (-0.75, -0) {};
		\node [style=none] (2) at (0.75, -0) {};
		\node [style=none] (3) at (0.75, 1.25) {};
		\node [style=none] (4) at (-0.75, 1.25) {};
		\node [style=none] (5) at (0, 1.25) {};
	\end{pgfonlayer}
	\begin{pgfonlayer}{edgelayer}
		\draw (0.center) to (5.center);
		\draw (4.center) to (3.center);
		\draw (3.center) to (2.center);
		\draw (2.center) to (1.center);
		\draw (1.center) to (4.center);
	\end{pgfonlayer}
\end{tikzpicture}
~~~~~~~~
(d_2) ~~~ \begin{tikzpicture}[yscale=-1]
	\begin{pgfonlayer}{nodelayer}
		\node [style=circle, scale=2] (0) at (0, -0) {};
		\node [style=none] (1) at (0, -0) {$\bot$};
		\node [style=none] (2) at (0, 2) {};
	\end{pgfonlayer}
	\begin{pgfonlayer}{edgelayer}
		\draw (2.center) to (0);
	\end{pgfonlayer}
\end{tikzpicture} \Rightarrow \begin{tikzpicture} [yscale=-1]
	\begin{pgfonlayer}{nodelayer}
		\node [style=none] (0) at (0, 2) {};
		\node [style=none] (1) at (-0.75, -0) {};
		\node [style=none] (2) at (0.75, -0) {};
		\node [style=none] (3) at (0.75, 1.25) {};
		\node [style=none] (4) at (-0.75, 1.25) {};
		\node [style=none] (5) at (0, 1.25) {};
	\end{pgfonlayer}
	\begin{pgfonlayer}{edgelayer}
		\draw (0.center) to (5.center);
		\draw (4.center) to (3.center);
		\draw (3.center) to (2.center);
		\draw (2.center) to (1.center);
		\draw (1.center) to (4.center);
	\end{pgfonlayer}
\end{tikzpicture}
~~~~~~~~
(d_3) ~~~\begin{tikzpicture}
	\begin{pgfonlayer}{nodelayer}
		\node [style=none] (0) at (-0.5, 1.5) {};
		\node [style=none] (1) at (-0.5, 2.75) {};
		\node [style=none] (2) at (-0.5, 3.25) {};
		\node [style=none] (3) at (-0.5, 2.75) {};
		\node [style=none] (4) at (-0.5, 1) {};
		\node [style=none] (5) at (-0.5, 1.5) {};
	\end{pgfonlayer}
	\begin{pgfonlayer}{edgelayer}
		\draw (1.center) to (0.center);
		\draw (3.center) to (2.center);
		\draw (5.center) to (4.center);
	\end{pgfonlayer}
\end{tikzpicture} \Rightarrow \begin{tikzpicture}
	\begin{pgfonlayer}{nodelayer}
		\node [style=none] (0) at (-1, 1.5) {};
		\node [style=none] (1) at (0, 1.5) {};
		\node [style=none] (2) at (-1, 2.75) {};
		\node [style=none] (3) at (0, 2.75) {};
		\node [style=none] (4) at (-0.5, 3.25) {};
		\node [style=none] (5) at (-0.5, 2.75) {};
		\node [style=none] (6) at (-0.5, 1) {};
		\node [style=none] (7) at (-0.5, 1.5) {};
	\end{pgfonlayer}
	\begin{pgfonlayer}{edgelayer}
		\draw (0.center) to (2.center);
		\draw (2.center) to (3.center);
		\draw (3.center) to (1.center);
		\draw (1.center) to (0.center);
		\draw (5.center) to (4.center);
		\draw (7.center) to (6.center);
	\end{pgfonlayer}
\end{tikzpicture}  
\] \[ (e_1) ~~~ \begin{tikzpicture}
	\begin{pgfonlayer}{nodelayer}
		\node [style=none] (0) at (-2, 2) {};
		\node [style=none] (1) at (-2, -1) {};
		\node [style=circle, scale=0.5] (2) at (-2, -0) {};
		\node [style=circle, scale=1.5] (3) at (-1, 1.25) {};
		\node [style=none] (4) at (-1, 1.25) {$\top$};
		\node [style=none] (5) at (-1, 2) {};
	\end{pgfonlayer}
	\begin{pgfonlayer}{edgelayer}
		\draw (0.center) to (1.center);
		\draw [dotted, in=-90, out=15, looseness=1.25] (2) to (3);
		\draw (5.center) to (3);
	\end{pgfonlayer}
\end{tikzpicture} \Rightarrow \begin{tikzpicture}
	\begin{pgfonlayer}{nodelayer}
		\node [style=none] (0) at (-2.25, 2) {};
		\node [style=none] (1) at (-2, -1) {};
		\node [style=none] (2) at (-2.5, 1) {};
		\node [style=none] (3) at (-2.5, -0) {};
		\node [style=none] (4) at (-1.5, -0) {};
		\node [style=none] (5) at (-1.5, 1) {};
		\node [style=none] (6) at (-2.25, 1) {};
		\node [style=none] (7) at (-2, -0) {};
		\node [style=none] (8) at (-1.75, 1) {};
		\node [style=none] (9) at (-1.75, 2) {};
	\end{pgfonlayer}
	\begin{pgfonlayer}{edgelayer}
		\draw (2.center) to (5.center);
		\draw (5.center) to (4.center);
		\draw (4.center) to (3.center);
		\draw (3.center) to (2.center);
		\draw (0.center) to (6.center);
		\draw (7.center) to (1.center);
		\draw (9.center) to (8.center);
	\end{pgfonlayer}
\end{tikzpicture}
~~~~~~~~
(e_2)~~~~
\begin{tikzpicture} [xscale=-1]
	\begin{pgfonlayer}{nodelayer}
		\node [style=none] (0) at (-2, 2) {};
		\node [style=none] (1) at (-2, -1) {};
		\node [style=circle, scale=0.5] (2) at (-2, -0) {};
		\node [style=circle, scale=1.5] (3) at (-1, 1.25) {};
		\node [style=none] (4) at (-1, 1.25) {$\top$};
		\node [style=none] (5) at (-1, 2) {};
	\end{pgfonlayer}
	\begin{pgfonlayer}{edgelayer}
		\draw (0.center) to (1.center);
		\draw [dotted, in=-90, out=15, looseness=1.25] (2) to (3);
		\draw (5.center) to (3);
	\end{pgfonlayer}
\end{tikzpicture}
 \Rightarrow \begin{tikzpicture}
	\begin{pgfonlayer}{nodelayer}
		\node [style=none] (0) at (-2.25, 2) {};
		\node [style=none] (1) at (-2, -1) {};
		\node [style=none] (2) at (-2.5, 1) {};
		\node [style=none] (3) at (-2.5, -0) {};
		\node [style=none] (4) at (-1.5, -0) {};
		\node [style=none] (5) at (-1.5, 1) {};
		\node [style=none] (6) at (-2.25, 1) {};
		\node [style=none] (7) at (-2, -0) {};
		\node [style=none] (8) at (-1.75, 1) {};
		\node [style=none] (9) at (-1.75, 2) {};
	\end{pgfonlayer}
	\begin{pgfonlayer}{edgelayer}
		\draw (2.center) to (5.center);
		\draw (5.center) to (4.center);
		\draw (4.center) to (3.center);
		\draw (3.center) to (2.center);
		\draw (0.center) to (6.center);
		\draw (7.center) to (1.center);
		\draw (9.center) to (8.center);
	\end{pgfonlayer}
\end{tikzpicture}
~~~~~~~~
(e_3)~~~~
\begin{tikzpicture} [yscale=-1]
	\begin{pgfonlayer}{nodelayer}
		\node [style=none] (0) at (-2, 2) {};
		\node [style=none] (1) at (-2, -1) {};
		\node [style=circle, scale=0.5] (2) at (-2, -0) {};
		\node [style=circle, scale=1.5] (3) at (-1, 1.25) {};
		\node [style=none] (4) at (-1, 1.25) {$\top$};
		\node [style=none] (5) at (-1, 2) {};
	\end{pgfonlayer}
	\begin{pgfonlayer}{edgelayer}
		\draw (0.center) to (1.center);
		\draw [dotted, in=-90, out=15, looseness=1.25] (2) to (3);
		\draw (5.center) to (3);
	\end{pgfonlayer}
\end{tikzpicture} \Rightarrow \begin{tikzpicture}
	\begin{pgfonlayer}{nodelayer}
		\node [style=none] (0) at (-2.25, -1) {};
		\node [style=none] (1) at (-2, 2) {};
		\node [style=none] (2) at (-2.5, 0) {};
		\node [style=none] (3) at (-2.5, 1) {};
		\node [style=none] (4) at (-1.5, 1) {};
		\node [style=none] (5) at (-1.5, 0) {};
		\node [style=none] (6) at (-2.25, 0) {};
		\node [style=none] (7) at (-2, 1) {};
		\node [style=none] (8) at (-1.75, 0) {};
		\node [style=none] (9) at (-1.75, -1) {};
	\end{pgfonlayer}
	\begin{pgfonlayer}{edgelayer}
		\draw (2.center) to (5.center);
		\draw (5.center) to (4.center);
		\draw (4.center) to (3.center);
		\draw (3.center) to (2.center);
		\draw (0.center) to (6.center);
		\draw (7.center) to (1.center);
		\draw (9.center) to (8.center);
	\end{pgfonlayer}
\end{tikzpicture}
~~~~~~~~
(e_4)~~~~
\begin{tikzpicture} [xscale=-1, yscale=-1]
	\begin{pgfonlayer}{nodelayer}
		\node [style=none] (0) at (-2, 2) {};
		\node [style=none] (1) at (-2, -1) {};
		\node [style=circle, scale=0.5] (2) at (-2, -0) {};
		\node [style=circle, scale=1.5] (3) at (-1, 1.25) {};
		\node [style=none] (4) at (-1, 1.25) {$\top$};
		\node [style=none] (5) at (-1, 2) {};
	\end{pgfonlayer}
	\begin{pgfonlayer}{edgelayer}
		\draw (0.center) to (1.center);
		\draw [dotted, in=-90, out=15, looseness=1.25] (2) to (3);
		\draw (5.center) to (3);
	\end{pgfonlayer}
\end{tikzpicture}\Rightarrow\begin{tikzpicture}
	\begin{pgfonlayer}{nodelayer}
		\node [style=none] (0) at (-2.25, -1) {};
		\node [style=none] (1) at (-2, 2) {};
		\node [style=none] (2) at (-2.5, 0) {};
		\node [style=none] (3) at (-2.5, 1) {};
		\node [style=none] (4) at (-1.5, 1) {};
		\node [style=none] (5) at (-1.5, 0) {};
		\node [style=none] (6) at (-2.25, 0) {};
		\node [style=none] (7) at (-2, 1) {};
		\node [style=none] (8) at (-1.75, 0) {};
		\node [style=none] (9) at (-1.75, -1) {};
	\end{pgfonlayer}
	\begin{pgfonlayer}{edgelayer}
		\draw (2.center) to (5.center);
		\draw (5.center) to (4.center);
		\draw (4.center) to (3.center);
		\draw (3.center) to (2.center);
		\draw (0.center) to (6.center);
		\draw (7.center) to (1.center);
		\draw (9.center) to (8.center);
	\end{pgfonlayer}
\end{tikzpicture} \]

Double lines refer to multiple number of wires. The boxes contain circuit components including maps. $\ox$-introduction and $\oa$-elimination are boxed in $(a_1)$ and $(a_2)$ respectively. 
In $(b_1)$, it is shown how a box `eats' the $\ox$-elimination: in $(b_2)$ the dual rule shows a $\oa$-introduction being eaten. 
$(c)$ shows how boxes can be amalgamated when they are connected by a single wire. $\bot$-elimination, 
$\top$-introduction, and identity maps are boxed in $(d_1)$, $(d_2)$, and $(d_3)$ respectively. In $(e_1)$-$(e_4)$, 
it is shown how the thinning links can be boxed. By progressively enclosing the components of the circuit in boxes using these rules, 
if we end up with a single box (or a wire), precisely when the circuit is valid. As an example, we verify the validity of the left linear distributor:
\[ \begin{tikzpicture}
	\begin{pgfonlayer}{nodelayer}
		\node [style=ox] (0) at (1.25, 0.75) {};
		\node [style=ox] (1) at (0.5, -1.25) {};
		\node [style=oa] (2) at (2, -0.25) {};
		\node [style=oa] (3) at (1.25, -2.5) {};
		\node [style=none] (4) at (1.25, -3.25) {};
		\node [style=none] (5) at (1.25, 1.5) {};
	\end{pgfonlayer}
	\begin{pgfonlayer}{edgelayer}
		\draw (5.center) to (0);
		\draw [bend left, looseness=1.00] (0) to (2);
		\draw [in=90, out=-150, looseness=1.00] (0) to (1);
		\draw (3) to (4.center);
		\draw [in=150, out=-75, looseness=1.00] (1) to (3);
		\draw [in=60, out=-120, looseness=1.25] (2) to (1);
		\draw [in=30, out=-75, looseness=1.00] (2) to (3);
	\end{pgfonlayer}
\end{tikzpicture} \stackrel{a_1,a_2}{\Rightarrow} \begin{tikzpicture}
	\begin{pgfonlayer}{nodelayer}
		\node [style=ox] (0) at (1.25, 0.75) {};
		\node [style=ox] (1) at (0.5, -1.25) {};
		\node [style=oa] (2) at (2, -0.25) {};
		\node [style=oa] (3) at (1.25, -2.5) {};
		\node [style=none] (4) at (1.25, -3.25) {};
		\node [style=none] (5) at (1.25, 1.5) {};
		\node [style=none] (6) at (1.5, 0.25) {};
		\node [style=none] (7) at (2.5, 0.25) {};
		\node [style=none] (8) at (2.5, -1) {};
		\node [style=none] (9) at (1.5, -1) {};
		\node [style=none] (10) at (1.25, -0.5) {};
		\node [style=none] (11) at (0, -0.5) {};
		\node [style=none] (12) at (0, -1.75) {};
		\node [style=none] (13) at (1.25, -1.75) {};
	\end{pgfonlayer}
	\begin{pgfonlayer}{edgelayer}
		\draw (5.center) to (0);
		\draw [bend left, looseness=1.00] (0) to (2);
		\draw [in=90, out=-150, looseness=1.00] (0) to (1);
		\draw (3) to (4.center);
		\draw [in=150, out=-75, looseness=1.00] (1) to (3);
		\draw [in=60, out=-120, looseness=1.25] (2) to (1);
		\draw [in=30, out=-75, looseness=1.00] (2) to (3);
		\draw (9.center) to (8.center);
		\draw (8.center) to (7.center);
		\draw (7.center) to (6.center);
		\draw (6.center) to (9.center);
		\draw (11.center) to (12.center);
		\draw (12.center) to (13.center);
		\draw (13.center) to (10.center);
		\draw (10.center) to (11.center);
	\end{pgfonlayer}
\end{tikzpicture} \stackrel{c}{\Rightarrow} \begin{tikzpicture}
	\begin{pgfonlayer}{nodelayer}
		\node [style=ox] (0) at (1.25, 0.75) {};
		\node [style=ox] (1) at (0.5, -1.25) {};
		\node [style=oa] (2) at (2, -0.25) {};
		\node [style=oa] (3) at (1.25, -2.5) {};
		\node [style=none] (4) at (1.25, -3.25) {};
		\node [style=none] (5) at (1.25, 1.5) {};
		\node [style=none] (6) at (2.5, 0.25) {};
		\node [style=none] (7) at (2.5, -1.75) {};
		\node [style=none] (8) at (0, 0.25) {};
		\node [style=none] (9) at (0, -1.75) {};
	\end{pgfonlayer}
	\begin{pgfonlayer}{edgelayer}
		\draw (5.center) to (0);
		\draw [bend left, looseness=1.00] (0) to (2);
		\draw [in=90, out=-150, looseness=1.00] (0) to (1);
		\draw (3) to (4.center);
		\draw [in=150, out=-75, looseness=1.00] (1) to (3);
		\draw [in=60, out=-120, looseness=1.25] (2) to (1);
		\draw [in=30, out=-75, looseness=1.00] (2) to (3);
		\draw (7.center) to (6.center);
		\draw (8.center) to (9.center);
		\draw (8.center) to (6.center);
		\draw (9.center) to (7.center);
	\end{pgfonlayer}
\end{tikzpicture} \stackrel{b_1,b_2}{\Rightarrow} \begin{tikzpicture}
	\begin{pgfonlayer}{nodelayer}
		\node [style=ox] (0) at (1.25, 0.75) {};
		\node [style=ox] (1) at (0.5, -1.25) {};
		\node [style=oa] (2) at (2, -0.25) {};
		\node [style=oa] (3) at (1.25, -2.5) {};
		\node [style=none] (4) at (1.25, -3.25) {};
		\node [style=none] (5) at (1.25, 1.5) {};
		\node [style=none] (6) at (2.5, 1.25) {};
		\node [style=none] (7) at (2.5, -3) {};
		\node [style=none] (8) at (0, 1.25) {};
		\node [style=none] (9) at (0, -3) {};
	\end{pgfonlayer}
	\begin{pgfonlayer}{edgelayer}
		\draw (5.center) to (0);
		\draw [bend left, looseness=1.00] (0) to (2);
		\draw [in=90, out=-150, looseness=1.00] (0) to (1);
		\draw (3) to (4.center);
		\draw [in=150, out=-75, looseness=1.00] (1) to (3);
		\draw [in=60, out=-120, looseness=1.25] (2) to (1);
		\draw [in=30, out=-75, looseness=1.00] (2) to (3);
		\draw (7.center) to (6.center);
		\draw (8.center) to (9.center);
		\draw (8.center) to (6.center);
		\draw (9.center) to (7.center);
	\end{pgfonlayer}
\end{tikzpicture} \]
% Examples here
In the first step the $\ox$-introduction and $\oa$-elimination are boxed. In the second step the boxes are amalgamated along the 
single wire joining them. In the third step,  the box absorbs the $\ox$-elimination and $\oa$-introduction.

In contrast, we now show that the reverse of the linear distributor is invalid as the boxing process gets stuck (there are 
no rules to box $\ox$-elimination and $\oa$-introduction):
\[ \begin{tikzpicture}
	\begin{pgfonlayer}{nodelayer}
		\node [style=ox] (0) at (1.25, -2.5) {};
		\node [style=ox] (1) at (2, -0.25) {};
		\node [style=oa] (2) at (0.5, -1.25) {};
		\node [style=oa] (3) at (1.25, 0.75) {};
		\node [style=none] (4) at (1.25, 1.5) {};
		\node [style=none] (5) at (1.25, -3.25) {};
	\end{pgfonlayer}
	\begin{pgfonlayer}{edgelayer}
		\draw (5.center) to (0);
		\draw [in=-83, out=157, looseness=1.00] (0) to (2);
		\draw [in=-90, out=30, looseness=1.00] (0) to (1);
		\draw (3) to (4.center);
		\draw [in=-15, out=90, looseness=1.00] (1) to (3);
		\draw [in=-120, out=60, looseness=1.25] (2) to (1);
		\draw [in=-150, out=90, looseness=1.00] (2) to (3);
	\end{pgfonlayer}
\end{tikzpicture} \stackrel{a_1,a_2}{\Rightarrow} \begin{tikzpicture}
	\begin{pgfonlayer}{nodelayer}
		\node [style=ox] (0) at (1.25, -2.5) {};
		\node [style=ox] (1) at (2, -0.5) {};
		\node [style=oa] (2) at (0.5, -1.5) {};
		\node [style=oa] (3) at (1.25, 0.75) {};
		\node [style=none] (4) at (1.25, 1.5) {};
		\node [style=none] (5) at (1.25, -3.25) {};
		\node [style=none] (6) at (0.5, 1.25) {};
		\node [style=none] (7) at (2, 1.25) {};
		\node [style=none] (8) at (2, 0.25) {};
		\node [style=none] (9) at (0.5, 0.25) {};
		\node [style=none] (10) at (0.5, -2) {};
		\node [style=none] (11) at (2, -2) {};
		\node [style=none] (12) at (2, -3) {};
		\node [style=none] (13) at (0.5, -3) {};
	\end{pgfonlayer}
	\begin{pgfonlayer}{edgelayer}
		\draw (5.center) to (0);
		\draw [in=-83, out=157, looseness=1.00] (0) to (2);
		\draw [in=-90, out=30, looseness=1.00] (0) to (1);
		\draw (3) to (4.center);
		\draw [in=-15, out=90, looseness=1.00] (1) to (3);
		\draw [in=-120, out=60, looseness=1.25] (2) to (1);
		\draw [in=-150, out=90, looseness=1.00] (2) to (3);
		\draw (9.center) to (6.center);
		\draw (6.center) to (7.center);
		\draw (7.center) to (8.center);
		\draw (9.center) to (8.center);
		\draw (12.center) to (13.center);
		\draw (13.center) to (10.center);
		\draw (10.center) to (11.center);
		\draw (11.center) to (12.center);
	\end{pgfonlayer}
\end{tikzpicture}  \] 

%%%%%%%%%%%%%%%%%%%%%%%%%%%%%%%%%%%%%%%%%%%%%%%%%

\subsection{Mix, isomix and compact LDCs}
\label{Sec: mix, isomix, compact LDC}

In this thesis, we are predominately concerned with LDCs which have a mix map:

\begin{definition} \cite{CS97a}
	\label{Defn: mix cat}
A {\bf mix category} is an LDC with a {\bf mix map} ${\sf m}:\bot\to\top$ such that:
\[
\xymatrixcolsep{4pc}
\xymatrix{
A \ox B \ar[r]^{1 \ox u_\oa^{L^{-1}}} \ar[d]_{(u_\oa^R)^{-1} \ox 1} \ar@{.>}[ddrr]^{\mx_{A,B}} & A \ox (\bot \oa B) 
\ar[r]^{1 \ox (\m \oa 1)} & A \ox ( \top \oa B) \ar[d]^{\partial^L} \\
(A \oa \bot) \ox B \ar[d]_{\partial^R} & & ( A \ox \top ) \oa B  \ar[d]^{u_\ox^R \oa 1} \\
A \oa (\bot \ox B) \ar[r]_{1 \oa (\m \ox 1)} & A \oa (\top \ox B) \ar[r]_{1 \oa u_\ox^L} &  A \oa B
}
\]
\end{definition}

The map $\mx_{A,B}$ is a natural transformation and is called the {\bf mixor}. The coherence condition for the mix map has the following form in string diagrams (where the mix map is represented by an empty box):

{ \centering $\mx_{A,B}:=
\begin{tikzpicture}
	\begin{pgfonlayer}{nodelayer}
		\node [style=ox] (0) at (0, 0.2500001) {};
		\node [style=circ] (1) at (0.5000001, -0.2500001) {};
		\node [style=circ] (2) at (0, -1) {$\bot$};
		\node [style=map] (3) at (0, -1.75) {~};
		\node [style=circ] (4) at (0, -2.5) {$\top$};
		\node [style=circ] (5) at (-0.5000001, -3.25) {};
		\node [style=oa] (6) at (0, -3.75) {};
		\node [style=nothing] (7) at (0, 0.7499999) {};
		\node [style=nothing] (8) at (0, -4.25) {};
	\end{pgfonlayer}
	\begin{pgfonlayer}{edgelayer}
		\draw (7) to (0);
		\draw (0) to (1);
		\draw [in=45, out=-60, looseness=1.00] (1) to (6);
		\draw [in=120, out=-135, looseness=1.00] (0) to (5);
		\draw (5) to (6);
		\draw (6) to (8);
		\draw [densely dotted, in=-90, out=45, looseness=1.00] (5) to (4);
		\draw (4) to (3);
		\draw (3) to (2);
		\draw [densely dotted, in=-135, out=90, looseness=1.00] (2) to (1);
	\end{pgfonlayer}
\end{tikzpicture}
=
\begin{tikzpicture}
	\begin{pgfonlayer}{nodelayer}
		\node [style=circ] (0) at (-0.5000001, -0.2500001) {};
		\node [style=circ] (1) at (0, -1) {$\bot$};
		\node [style=map] (2) at (0, -1.75) {~};
		\node [style=circ] (3) at (0, -2.5) {$\top$};
		\node [style=circ] (4) at (0.5000001, -3.25) {};
		\node [style=nothing] (5) at (0, 0.7499999) {};
		\node [style=nothing] (6) at (0, -4.25) {};
		\node [style=oa] (7) at (0, -3.75) {};
		\node [style=ox] (8) at (0, 0.2500001) {};
	\end{pgfonlayer}
	\begin{pgfonlayer}{edgelayer}
		\draw [densely dotted, in=-90, out=150, looseness=1.00] (4) to (3);
		\draw (3) to (2);
		\draw (2) to (1);
		\draw [densely dotted, in=-45, out=90, looseness=1.00] (1) to (0);
		\draw (8) to (5);
		\draw (8) to (0);
		\draw [in=135, out=-120, looseness=1.00] (0) to (7);
		\draw (7) to (6);
		\draw (7) to (4);
		\draw [in=-45, out=60, looseness=1.00] (4) to (8);
	\end{pgfonlayer}
\end{tikzpicture} $ \par}

In a mix category, the associator, the distributor and the mix maps interact as follows. See Lemma 2, and proposition 3 in \cite{BCS00} for a proof.
\begin{equation*}
\mbox{\bf [mix.]}~~~~~~~ \xymatrix{
(A \oa B) \ox C \ar[r]^{\delta^R} \ar[d]_{\mx} \ar@{}[dr]|{(a)} &  A \oa (B \ox C) \ar[d]^{1 \oa \mx} \\
(A \oa B) \oa C \ar[r]_{a_\oa} & A \oa (B \oa C)
} ~~~~~~~~~~~~~
\xymatrix{
(A \ox B) \ox C \ar[r]^{\mx} \ar[d]_{a_\ox} \ar@{}[dr]|{(b)} & A \oa ( B \ox C ) \ar@{<-}[d]^{\delta^L} \\
A \ox (B \ox C) \ar[r]_{1 \ox \mx} & A \ox (B \oa C)
}
\end{equation*}
\[  ~~~~~~~~~~~~ \xymatrix{
C \ox (A \oa B) \ar[r]^{\delta^L} \ar[d]_{\mx} \ar@{}[dr]|{(c)} &  (C \ox A) \oa B \ar[d]^{ \mx \oa 1} \\
C \oa (A \oa B) \ar[r]_{a_\oa^{-1}} & (C \oa A) \oa B
} ~~~~~~~~~~~~~
\xymatrix{
A \ox (B \ox C) \ar[r]^{\mx} \ar[d]_{a_\ox^{-1}} \ar@{}[dr]|{(d)} & A \oa ( B \ox C ) \ar@{<-}[d]^{\delta^R} \\
(A \ox B) \ox C \ar[r]_{\mx \ox 1} & (A \oa B) \ox C
} \]

There are many examples of mix categories including coherence spaces \cite{Gir87}, 
and finiteness spaces \cite{Ehr05}.

\begin{definition}
An LDC with a mix map, ${\sf m: \bot \to \top}$ which is an isomorphism is said to be an 
{\bf isomix category}. 
\end{definition}

When ${\sf m}$ is an isomorphism, the coherence requirement for the mixor is automatically 
satisfied (see \cite[Lemma 6.6]{CS97a}). Moreover, the mix map, $\m$, being an isomorphism does
 not imply that the mixor, $\mx$, is an isomorphism. Finiteness spaces \cite{Ehr05} and Chu spaces 
 with the tensor unit as the dualizing object  \cite{Bar06}  provide examples of isomix categories.

 \begin{definition}
 A {\bf compact LDC} is an isomix category in which each mixor, $\mx_{A,B}$ is an isomorphism.  
 \end{definition}

 An important way in which compact LDCs arise is from the ``core" of an isomix category

 \begin{definition} \cite{BCS00} An object $U$ is in the {\bf core} of a mix category if and only if 
	the following natural transformations are isomorphisms: 
    \[ U \ox (\_) \to^{\mx_{U,(\_)}} U \oa (\_) ~~~~\mbox{and}~~~~ 
    (\_) \ox  U \to^{\mx_{(\_),U}} (\_) \oa U \] 
    \end{definition}
    Therefore, the core of a mix category, $\Core(\X) \subseteq \X$, is the full subcategory of $\X$ 
	with the mixor being an isomorphism.  It follows that $\Core(\X)$ is a compact LDC. 

    \begin{proposition} \cite[Proposition 3]{BCS00} 
        If $\X$ is a mix-LDC and $A,B \in \Core(\X)$ then $A \oa B$ and 
        $A \ox B \in \Core(\X)$ (and $A \oa B \simeq A \ox B$).  If $\X$ is an isomix-LDC, 
        then $\top, \bot \in \Core(\X)$.  
        \end{proposition}
        
		A monoidal category is a compact LDC with the mix, and the mixor maps coinciding with the 
		identity. Hence, the tensor and the par coincide in a monoidal category. In fact, any compact LDC is linearly equivalent 
		to a monoidal category. A detailed description of this linear equivalence is given in the section on
		linear functors and transformations. 
		
		The following schematic diagram summarizes different properties of LDCs:
		\begin{center}
		\begin{figure}[h]
			\centering
		\begin{tikzpicture}[scale=1.8]
			\begin{pgfonlayer}{nodelayer}
				\node [style=circle, scale=2, color=black, fill=red] (0) at (-5.75, 2.75) {};
				\node [style=circle, scale=2, color=black, fill=red!70] (1) at (-3.5, 2.75) {};
				\node [style=circle, scale=2, color=black, fill=red!60] (2) at (-1, 2.75) {};
				\node [style=circle, scale=2, color=black, fill=red!40] (3) at (1.75, 2.75) {};
				\node [style=circle, scale=2, color=black, fill=red!20] (5) at (4, 2.75) {};
				\node [style=none] (4) at (-7.75, 2.75) {};
				\node [style=none] (6) at (6, 2.75) {};
				\node [style=none] (7) at (-5.75, 2) {LDC};
				\node [style=none] (8) at (-3.5, 4) {Mix category};
				\node [style=none] (9) at (-3.5, 3.5) {$\m: \bot \to \top$};
				\node [style=none] (10) at (-1, 2) {Isomix category};
				\node [style=none] (11) at (1.75, 4.25) {Compact LDC};
				\node [style=none] (12) at (1.75, 3.65) {$A \ox B \to^{\mx}_{\simeq} A \oa B$};
				\node [style=none] (13) at (4, 2) {Monoidal category};
				\node [style=none] (14) at (-1, 1.4) {$\bot \to^{\m}_{\simeq} \top$};
				\node [style=none] (15) at (4, 1.5) {$\m = 1$, $\mx=1$};
				\node [style=none] (16) at (-5.75, 1.5) {$(\X, \ox, \top)$};
				\node [style=none] (17) at (-5.75, 1) {$(\X, \oa, \bot)$};
			\end{pgfonlayer}
			\begin{pgfonlayer}{edgelayer}
				\draw [dotted] (4.center) to (0);
				\draw (0) to (1);
				\draw (1) to (2);
				\draw (2) to (3);
				\draw (3) to (5);
				\draw [dotted] (5) to (6.center);
			\end{pgfonlayer}
		\end{tikzpicture}
		\caption{Schematic diagram of LDC properties}
		\label{Fig: LDCs}
	\end{figure}
\end{center}

		\subsection{$*$-autonomous categories}
		\label{Sec: *-autonomous}

		A key notion in the theory of LDCs is the notion of a linear adjoint \cite{CKS00}.  
		Here we shall refer to linear adjoints as ``duals'' in order to avoid any confusion 
		with an adjunction of linear functors.   
		
		\begin{definition} 
			\label{defn: duals}
			Suppose $\mathbb{X}$ is a LDC and $A,B \in\X$, then $B$ is {\bf left dual}  
			(or left linear adjoint) to $A$ -- or $A$ is {\bf right dual} (right linear adjoint) 
			to $B$ -- written $(\eta, \epsilon): B \dashv \!\!\!\!\! \dashv  A$, if there 
			exists a unit map, $\eta: \top \rightarrow B \oa A$ and 
			a counit map, $\epsilon: A \ox B \rightarrow \bot$ such that the following diagrams commute:
		\[
		\xymatrix{
		B \ar[r]^{(u_\ox^L)^{-1}} \ar@{=}[d] 
		& \top \ox B \ar[r]^{\eta \ox 1} 
		& (B \oa A) \ox B \ar[d]^{\partial_R} \\
		B 
		& B \oa \bot \ar[l]^{u_\oa^R} 
		& B \oa (A \ox B) \ar[l]^{1 \oa \epsilon}
		}
		~~~~~
		\xymatrix{
		A \ar[r]^{(u_\ox^R)^{-1}} \ar@{=}[d] 
		& A \ox \top  \ar[r]^{1 \ox \eta} 
		& A  \ox  (B \oa A)\ar[d]^{\partial_L} \\
		A
		& \bot \oa A \ar[l]^{u_\oa^L} 
		& (A \ox B) \oa A   \ar[l]^{ \epsilon \oa 1} }
		\]
		\end{definition}
		
	The unit map unit and the counit maps of a dual are drawn in string diagrams as a cap and a cup: 
  \[ \eta := \begin{tikzpicture}
	\begin{pgfonlayer}{nodelayer}
		\node [style=none] (0) at (-1, 1.25) {};
		\node [style=none] (1) at (1, 1.25) {};
		\node [style=none] (2) at (-1, 2.5) {};
		\node [style=none] (3) at (1, 2.5) {};
		\node [style=none] (6) at (-1.25, 1.5) {$A$};
		\node [style=none] (7) at (1.25, 1.5) {$B$};
		\node [style=none] (8) at (0, 3.5) {$\eta$};
	\end{pgfonlayer}
	\begin{pgfonlayer}{edgelayer}
		\draw (1.center) to (3.center);
		\draw (2.center) to (0.center);
		\draw [bend left=90, looseness=1.25] (2.center) to (3.center);
	\end{pgfonlayer}
\end{tikzpicture}  ~~~~~~~\text{ and }~~~~~~~ 
\epsilon := \begin{tikzpicture}
	\begin{pgfonlayer}{nodelayer}
		\node [style=none] (0) at (-1, 3.5) {};
		\node [style=none] (1) at (1, 3.5) {};
		\node [style=none] (2) at (-1, 2.25) {};
		\node [style=none] (3) at (1, 2.25) {};
		\node [style=none] (6) at (-1.25, 3.25) {$B$};
		\node [style=none] (7) at (1.25, 3.25) {$A$};
		\node [style=none, scale=1.5] (8) at (0, 1) {$\epsilon$};
	\end{pgfonlayer}
	\begin{pgfonlayer}{edgelayer}
		\draw (1.center) to (3.center);
		\draw (2.center) to (0.center);
		\draw [bend right=90, looseness=1.50] (2.center) to (3.center);
	\end{pgfonlayer}
\end{tikzpicture} \]
		
	The commuting diagrams are called often referred to as ``snake diagrams'' because of their 
	shape in graphical calculus:
		\[	\begin{tikzpicture}
			\begin{pgfonlayer}{nodelayer}
				\node [style=none] (6) at (1, 0) {};
				\node [style=none] (7) at (1, 1) {};
				\node [style=none] (8) at (2, 1) {};
				\node [style=none] (9) at (2, 0.75) {};
				\node [style=none] (10) at (3, 0.75) {};
				\node [style=none] (11) at (3, 2) {};
				\node [style=none] (12) at (1.5, 1.75) {$\eta$};
				\node [style=none] (13) at (2.5, 0) {$\epsilon$};
				\node [style=none] (14) at (0.75, 0.25) {$A$};
				\node [style=none] (15) at (3.25, 1.75) {$A$};
			\end{pgfonlayer}
			\begin{pgfonlayer}{edgelayer}
				\draw (6.center) to (7.center);
				\draw [bend left=90, looseness=1.50] (7.center) to (8.center);
				\draw (8.center) to (9.center);
				\draw [bend right=90, looseness=1.50] (9.center) to (10.center);
				\draw (10.center) to (11.center);
			\end{pgfonlayer}
		\end{tikzpicture} = \begin{tikzpicture}
		  \draw (0,2.5) -- (0,0);
		\end{tikzpicture} ~~~~~~~~~~
		\begin{tikzpicture}
			\begin{pgfonlayer}{nodelayer}
				\node [style=none] (6) at (3, 0) {};
				\node [style=none] (7) at (3, 1) {};
				\node [style=none] (8) at (2, 1) {};
				\node [style=none] (9) at (2, 0.75) {};
				\node [style=none] (10) at (1, 0.75) {};
				\node [style=none] (11) at (1, 2) {};
				\node [style=none] (12) at (2.5, 1.75) {$\eta$};
				\node [style=none] (13) at (1.5, 0) {$\epsilon$};
				\node [style=none] (14) at (3.25, 0.25) {$B$};
				\node [style=none] (15) at (0.75, 1.75) {$B$};
			\end{pgfonlayer}
			\begin{pgfonlayer}{edgelayer}
				\draw (6.center) to (7.center);
				\draw [bend right=90, looseness=1.50] (7.center) to (8.center);
				\draw (8.center) to (9.center);
				\draw [bend left=90, looseness=1.50] (9.center) to (10.center);
				\draw (10.center) to (11.center);
			\end{pgfonlayer}
		\end{tikzpicture} = \begin{tikzpicture}
		  \draw (0,2.5) -- (0,0);
		\end{tikzpicture} \]
		The linear distributor is hidden in the above circuit diagrams due to the use to circuit expansion, 
		and circuit reduction rules, see Section \ref{Sec: graphical}.
		
		\begin{lemma} \cite{BCS00}
		\begin{enumerate}[(i)]
		\item In an LDC if $(\eta,\epsilon): B \dashvv A$ and $(\eta',\epsilon'): C \dashvv A$, then $B$ and $C$ are isomorphic;
		\item In a symmetric LDC $(\eta, \epsilon): B \dashvv A$ if and only if $(\eta c_\oa, c_\ox \epsilon): A \dashvv B$;
		\item In a mix category if $B \in \Core(\X)$ and $B \dashvv A$, then $A \in \Core(\X)$.
		\end{enumerate}
		\end{lemma}
		
    	In a monoidal category, duals coincide with the usual notion of duals. 
		Next we define the homomorphism of duals:
		
		%homomorphism of duals
		\begin{definition}
			A homomorphism of duals, $(f,f'): (\eta,  \epsilon) \to (\tau, \gamma)$, is given by a pair of maps
			\[ \xymatrix{ A \ar[d]^f \ar@{-||}[r]^{(\eta,  \epsilon)}  &  B \\ A' \ar@{-||}[r]_{(\tau, \gamma)} &  B' \ar[u]_{f'}} \] 
		such that the following equations hold:  \[ (a) ~~~  \begin{tikzpicture}
			\begin{pgfonlayer}{nodelayer}
				\node [style=none] (0) at (-1, 2) {};
				\node [style=none] (1) at (0.5, 3) {};
				\node [style=none] (2) at (-1, 3) {};
				\node [style=none] (3) at (-0.25, 4) {$\tau$};
				\node [style=none] (4) at (0.5, 2) {};
				\node [style=none] (5) at (-1.25, 2.25) {$A'$};
				\node [style=none] (6) at (1, 3.5) {$B'$};
				\node [style=circle, scale=1.5] (7) at (0.5, 2.75) {};
				\node [style=none] (8) at (0.5, 2.75) {$f'$};
				\node [style=none] (9) at (1, 2.25) {$B$};
			\end{pgfonlayer}
			\begin{pgfonlayer}{edgelayer}
				\draw (4.center) to (7);
				\draw (1.center) to (7);
				\draw (2.center) to (0.center);
				\draw [bend left=90, looseness=1.75] (2.center) to (1.center);
			\end{pgfonlayer}
		\end{tikzpicture}
		= \begin{tikzpicture}
			\begin{pgfonlayer}{nodelayer}
				\node [style=none] (0) at (0.5, 2) {};
				\node [style=none] (1) at (-1, 3) {};
				\node [style=none] (2) at (0.5, 3) {};
				\node [style=none] (3) at (-0.25, 4) {$\eta$};
				\node [style=none] (4) at (-1, 2) {};
				\node [style=none] (5) at (0.75, 2.25) {$B$};
				\node [style=none] (6) at (-1.5, 3.5) {$A$};
				\node [style=circle, scale=1.5] (7) at (-1, 2.75) {};
				\node [style=none] (8) at (-1, 2.75) {$f$};
				\node [style=none] (9) at (-1.5, 2.25) {$A'$};
			\end{pgfonlayer}
			\begin{pgfonlayer}{edgelayer}
				\draw (4.center) to (7);
				\draw (1.center) to (7);
				\draw (2.center) to (0.center);
				\draw [bend right=90, looseness=1.75] (2.center) to (1.center);
			\end{pgfonlayer}
		\end{tikzpicture}
		  ~~~~~~(b)~~~ \begin{tikzpicture}
			\begin{pgfonlayer}{nodelayer}
				\node [style=none] (0) at (0.5, 3.75) {};
				\node [style=none] (1) at (-1, 2.75) {};
				\node [style=none] (2) at (0.5, 2.75) {};
				\node [style=none] (3) at (-0.25, 1.75) {$\gamma$};
				\node [style=none] (4) at (-1, 3.75) {};
				\node [style=none] (5) at (0.75, 3.5) {$B'$};
				\node [style=none] (6) at (-1.5, 2.25) {$A'$};
				\node [style=circle, scale=1.5] (7) at (-1, 3) {};
				\node [style=none] (8) at (-1, 3) {$f$};
				\node [style=none] (9) at (-1.5, 3.5) {$A$};
			\end{pgfonlayer}
			\begin{pgfonlayer}{edgelayer}
				\draw (4.center) to (7);
				\draw (1.center) to (7);
				\draw (2.center) to (0.center);
				\draw [bend right=90, looseness=1.75] (1.center) to (2.center);
			\end{pgfonlayer}
		\end{tikzpicture}
		 =  \begin{tikzpicture}
			\begin{pgfonlayer}{nodelayer}
				\node [style=none] (0) at (-1, 3.75) {};
				\node [style=none] (1) at (0.5, 2.75) {};
				\node [style=none] (2) at (-1, 2.75) {};
				\node [style=none] (3) at (-0.25, 1.75) {$\epsilon$};
				\node [style=none] (4) at (0.5, 3.75) {};
				\node [style=none] (5) at (-1.25, 3.5) {$A$};
				\node [style=none] (6) at (1, 2.25) {$B$};
				\node [style=circle, scale=1.5] (7) at (0.5, 3) {};
				\node [style=none] (8) at (0.5, 3) {$f'$};
				\node [style=none] (9) at (1, 3.5) {$B'$};
			\end{pgfonlayer}
			\begin{pgfonlayer}{edgelayer}
				\draw (4.center) to (7);
				\draw (1.center) to (7);
				\draw (2.center) to (0.center);
				\draw [bend right=90, looseness=1.75] (2.center) to (1.center);
			\end{pgfonlayer}
		\end{tikzpicture} \]
		\end{definition}
		
		Notice that a morphism of duals is determined completely by either of the pair of maps ($f$ 
		completely determines $f'$, and vice versa), and are referred to as Australian mates \cite{CKS00}. :
        \[ f' := \begin{tikzpicture}
			\begin{pgfonlayer}{nodelayer}
				\node [style=circle, scale=2] (0) at (0, 2) {};
				\node [style=none] (1) at (0, 2) {$f$};
				\node [style=none] (2) at (0, 2.5) {};
				\node [style=none] (3) at (1, 2.5) {};
				\node [style=none] (4) at (1, 0.75) {};
				\node [style=none] (5) at (0, 1.5) {};
				\node [style=none] (6) at (-1, 1.5) {};
				\node [style=none] (7) at (-1, 3.5) {};
				\node [style=none] (8) at (-1.25, 3) {$B'$};
				\node [style=none] (9) at (1.25, 1.25) {$B$};
			\end{pgfonlayer}
			\begin{pgfonlayer}{edgelayer}
				\draw (2.center) to (0);
				\draw (0) to (5.center);
				\draw [bend left=90, looseness=1.75] (5.center) to (6.center);
				\draw (6.center) to (7.center);
				\draw [bend left=90, looseness=2.00] (2.center) to (3.center);
				\draw (3.center) to (4.center);
			\end{pgfonlayer}
		\end{tikzpicture}
		\]
		
		The map $f$ is an isomorphism if and only if $f'$ is an isomorphism. Also, notice that 
		if $f$ or $f'$ is an isomorphism, equations $(a)$ and $(b)$ imply one another in the 
		definition of homomorphism of duals. 
		
		A dual, $(\eta, \epsilon): A \dashvv B$ such that $A$ is both left and right dual of $B$, 
		is called a {\bf cyclic dual}. 
		In a symmetric LDC, every dual $(\eta, \epsilon): A \dashvv B$ gives another 
		dual $(\eta c_\oa, c_\ox \epsilon): B \dashvv A$, which is obtained by twisting 
		the wires using the symmetry map. Thus, in a symmetric LDC, every dual is a cyclic dual. 
		
		\iffalse
		Cyclic duals are closely related to the notion of linear monoids. On the one 
		hand, every linear monoid gives a cyclic dual, and, on the other hand,  
		every cyclic dual has an associated ``pants'' linear monoid. 
		
		In Section \ref{Sec: Linear monoids}, we recall the definition of a linear monoid and show how 
		every cyclic dual gives rise to a pants linear monoid. 
		See \cite{CKS00} for a more detailed exposition of this topic.
		\fi
		
		\begin{definition}
		An LDC in which every object has a chosen left and right dual, 
		respectively $(\eta{*},\epsilon{*}): A^{*}  \dashvv A$ and 
		$({*}\eta,{*}\epsilon): A  \dashvv \!~^{*}A$, is a {\bf $*$-autonomous category}.   		
	    \end{definition}

		In the symmetric case a left dual gives a right dual using the symmetry: 
		thus, it is standard to assume the existence of just the left dual with the 
		right being the same object with the unit and counit given by symmetry 
		(as above). 

		In a $*$-autonomous category, taking the left (or right) linear dual of an object 
		extends to a linear functor, see Section \ref{Sec: dduals}. 

		Just as compact LDCs are linearly equivalent to monoidal categories so 
		compact $*$-autonomous categories are linearly equivalent to a compact 
		closed categories. The equivalence is given by ${\sf Mx}_\uparrow$ which 
		spreads the par onto  two tensor structures (or, indeed, by 
		${\sf Mx}_{\downarrow}$ which shows how to spread out a compact closed 
		structure on the tensor), see Section \ref{Sec: mix-functor}.
		
		%  Do we want to move the cyclors to where they are used?
		
		In a symmetric $*$-autonomous category the left dual of an object is always 
		canonically isomorphic to the right dual.  Moreover, even in non-symmetric 
		$*$-autonomous categories, it is often the case that the two duals are 
		coherently isomorphic:
		
		\begin{definition}\cite{EggMcCurd12}
		A {\bf cyclor} in a $*$-autonomous category $(\X, \ox, \top, \oa, \bot, ~^{*}(\_), (\_)^*)$ is a natural isomorphism $A^* \to^{\psi} \!~^{*}A$ satisfying the following coherence conditions:
		\[
		\xymatrix{
		\bot^* \ar[rr]^{\psi} \ar[dr] & \ar@{}[d]|{\mbox{\tiny \bf [C.1]}} & ~^{*}\bot \ar[ld]^{} \\
		& \top &
		} ~~~~~~~~ \xymatrix{
		 & A \ar[ld] \ar[dr] \ar@{}[d]|{\mbox{\tiny \bf [C.2]}} & \\
		( ~^{*}A)^* \ar[r]_{\psi^*} & (A^*)^* \ar[r]_{\psi_{A^*}} & ~^{*}(A^*)
		} ~~~~~~~~ \xymatrix{
		\top^* \ar[rr]^{\psi} \ar[dr] & \ar@{}[d]|{\mbox{\tiny \bf [C.3]}} & ~^{*}\top \ar[ld]^{} \\
		& \bot &
		}
		\]
		
		\[
		\xymatrixcolsep{4pc}
		\xymatrix{
		(A \ox B)^* \ar[r]^{\psi} \ar[d]_{} \ar@{}[dr]|{\mbox{\tiny \bf [C.4]}} & ~^{*}(A \ox B) \ar[d]^{} \\
		(B^* \oa A^*) \ar[r]_{\psi \oa \psi} & ~^{*}B \oa ~^{*}A
		} ~~~~~~~~~ \xymatrix{
		(A \oa B)^* \ar[r]^{\psi} \ar[d]_{} \ar@{}[dr]|{\mbox{\tiny \bf [C.5]}}  & ~^{*}(A \oa B) \ar[d]^{} \\
		(B^* \ox A^*) \ar[r]_{\psi \ox \psi} & ~^{*}B \ox ~^{*}A
		}
		\]
		A $*$-autonomous category with a cyclor is said to be {\bf cyclic}.
		\end{definition}
		
		The coherence conditions are not independent of each other: 
		being cyclic is equivalent to any one of the following four pairs of coherences: 
		({\bf [C.1]}, {\bf [C.5]}), ({\bf [C.2]}, {\bf [C.5]}),  ({\bf [C.4]}, {\bf [C.2]}) and 
		({\bf [C.4]}, {\bf [C.3]}) \cite{EggMcCurd12}.
		
		Condition {\bf [C.2]} which is used extensively in Section \ref{daggers-duals-conjugation} is represented graphically as follows:
		\[ \begin{tikzpicture}
			\begin{pgfonlayer}{nodelayer}
				\node [style=circle] (0) at (-2, 1) {$\psi_{A^*}$};
				\node [style=circle] (1) at (-0.75, 1) {$\psi_A$};
				\node [style=none] (2) at (-2, 1.75) {};
				\node [style=none] (3) at (-0.75, 1.75) {};
				\node [style=none] (4) at (-2, -1) {};
				\node [style=none] (5) at (-0.75, -0) {};
				\node [style=none] (6) at (0.5, -0) {};
				\node [style=none] (7) at (0.5, 3) {};
				\node [style=none] (8) at (-1.5, 2.75) {$\eta*$};
				\node [style=none] (9) at (0, -0.75) {$*\epsilon$};
				\node [style=none] (10) at (-2.5, 1.75) {$A^{**}$};
				\node [style=none] (11) at (-2.7, -0.7) {$~^*(A^*)$};
				\node [style=none] (12) at (-0.5, 2) {$A^*$};
				\node [style=none] (13) at (-0.45, 0.25) {$~^*A$};
				\node [style=none] (14) at (0.75, 2.5) {$A$};
			\end{pgfonlayer}
			\begin{pgfonlayer}{edgelayer}
				\draw (0) to (4.center);
				\draw (2.center) to (0);
				\draw [bend left=90, looseness=2.00] (2.center) to (3.center);
				\draw (3.center) to (1);
				\draw (1) to (5.center);
				\draw [bend right=90, looseness=1.50] (5.center) to (6.center);
				\draw (6.center) to (7.center);
			\end{pgfonlayer}
		\end{tikzpicture}  = \begin{tikzpicture}
			\begin{pgfonlayer}{nodelayer}
				\node [style=none] (0) at (-2, 0.25) {};
				\node [style=none] (1) at (-0.75, 0.25) {};
				\node [style=none] (2) at (-2, 3) {};
				\node [style=none] (3) at (-0.75, 2) {};
				\node [style=none] (4) at (0.5, 2) {};
				\node [style=none] (5) at (0.5, -1) {};
				\node [style=none] (6) at (-0.25, 2.75) {$*\eta$};
				\node [style=none] (7) at (-1.5, -0.75) {$\epsilon*$};
				\node [style=none] (8) at (1, -0.7) {$~^*(A^*)$};
				\node [style=none] (9) at (-0.5, 1.15) {$A^*$};
				\node [style=none] (10) at (-2.25, 2.75) {$A$};
			\end{pgfonlayer}
			\begin{pgfonlayer}{edgelayer}
				\draw [bend right=90, looseness=2.00] (0.center) to (1.center);
				\draw [bend left=90, looseness=1.50] (3.center) to (4.center);
				\draw (4.center) to (5.center);
				\draw (3.center) to (1.center);
				\draw (2.center) to (0.center);
			\end{pgfonlayer}
		\end{tikzpicture}
		\]
		
		%  You must get the equallities in the right order so you can see the reasoning!!!!!
	    The maps in {\bf [C.2]} are invertible:
		\[  \text{\bf [C.2]$^{-1}$} 
		  \begin{tikzpicture}
			\begin{pgfonlayer}{nodelayer}
				\node [style=circle] (0) at (0, 1) {$\psi^{-1}_A$};
				\node [style=circle] (1) at (2, 1) {$\psi^{-1}_{A^*}$};
				\node [style=none] (2) at (0, -0) {};
				\node [style=none] (3) at (2, -0) {};
				\node [style=none] (4) at (2, 3) {};
				\node [style=none] (5) at (0, 2) {};
				\node [style=none] (6) at (-1, 2) {};
				\node [style=none] (7) at (-1, -1.25) {};
				\node [style=none] (8) at (1, -1.1) {$\epsilon*$};
				\node [style=none] (9) at (-0.5, 2.75) {$*\eta$};
				\node [style=none] (10) at (2.5, 0.25) {$A^{**}$};
				\node [style=none] (11) at (-0.5, 0.25) {$A^*$};
				\node [style=none] (12) at (2.5, 2.75) {$~^*(A^*)$};
				\node [style=none] (13) at (-0.5, 1.75) {$~^*A$};
				\node [style=none] (14) at (-1.25, -0.75) {$A$};
			\end{pgfonlayer}
			\begin{pgfonlayer}{edgelayer}
				\draw (0) to (2.center);
				\draw [bend right=90, looseness=1.50] (2.center) to (3.center);
				\draw (3.center) to (1);
				\draw (4.center) to (1);
				\draw (5.center) to (0);
				\draw [bend right=90, looseness=2.00] (5.center) to (6.center);
				\draw (6.center) to (7.center);
			\end{pgfonlayer}
		\end{tikzpicture} =  \left(
		\begin{tikzpicture}
			\begin{pgfonlayer}{nodelayer}
				\node [style=circle] (0) at (-2, 1) {$\psi_{A^*}$};
				\node [style=circle] (1) at (-0.75, 1) {$\psi_A$};
				\node [style=none] (2) at (-2, 1.75) {};
				\node [style=none] (3) at (-0.75, 1.75) {};
				\node [style=none] (4) at (-2, -1) {};
				\node [style=none] (5) at (-0.75, -0) {};
				\node [style=none] (6) at (0.5, -0) {};
				\node [style=none] (7) at (0.5, 3) {};
				\node [style=none] (8) at (-1.5, 2.75) {$\eta*$};
				\node [style=none] (9) at (0, -0.75) {$*\epsilon$};
				\node [style=none] (10) at (-2.5, 1.75) {$A^{**}$};
				\node [style=none] (11) at (-2.7, -0.7) {$~^*(A^*)$};
				\node [style=none] (12) at (-0.5, 2) {$A^*$};
				\node [style=none] (13) at (-0.45, 0.25) {$~^*A$};
				\node [style=none] (14) at (0.75, 2.5) {$A$};
			\end{pgfonlayer}
			\begin{pgfonlayer}{edgelayer}
				\draw (0) to (4.center);
				\draw (2.center) to (0);
				\draw [bend left=90, looseness=2.00] (2.center) to (3.center);
				\draw (3.center) to (1);
				\draw (1) to (5.center);
				\draw [bend right=90, looseness=1.50] (5.center) to (6.center);
				\draw (6.center) to (7.center);
			\end{pgfonlayer}
		\end{tikzpicture} \right)^{-1} = \left( \begin{tikzpicture}
			\begin{pgfonlayer}{nodelayer}
				\node [style=none] (0) at (-2, 0.25) {};
				\node [style=none] (1) at (-0.75, 0.25) {};
				\node [style=none] (2) at (-2, 3) {};
				\node [style=none] (3) at (-0.75, 2) {};
				\node [style=none] (4) at (0.5, 2) {};
				\node [style=none] (5) at (0.5, -1) {};
				\node [style=none] (6) at (-0.25, 2.75) {$*\eta$};
				\node [style=none] (7) at (-1.5, -0.75) {$\epsilon*$};
				\node [style=none] (8) at (1, -0.7) {$~^*(A^*)$};
				\node [style=none] (9) at (-0.5, 1.15) {$A^*$};
				\node [style=none] (10) at (-2.25, 2.75) {$A$};
			\end{pgfonlayer}
			\begin{pgfonlayer}{edgelayer}
				\draw [bend right=90, looseness=2.00] (0.center) to (1.center);
				\draw [bend left=90, looseness=1.50] (3.center) to (4.center);
				\draw (4.center) to (5.center);
				\draw (3.center) to (1.center);
				\draw (2.center) to (0.center);
			\end{pgfonlayer}
		\end{tikzpicture} \right)^{-1}
		= \begin{tikzpicture}
			\begin{pgfonlayer}{nodelayer}
				\node [style=none] (0) at (-2, 0.25) {};
				\node [style=none] (1) at (-0.75, 0.25) {};
				\node [style=none] (2) at (-2, 3) {};
				\node [style=none] (3) at (-0.75, 2) {};
				\node [style=none] (4) at (0.5, 2) {};
				\node [style=none] (5) at (0.5, -1) {};
				\node [style=none] (6) at (-0.25, 2.75) {$\eta*$};
				\node [style=none] (7) at (-1.5, -0.75) {$*\epsilon$};
				\node [style=none] (8) at (-2.75, 2.75) {$~^*(A^*)$};
				\node [style=none] (9) at (-0.5, 1) {$A^*$};
				\node [style=none] (10) at (0.75, -0.75) {$A$};
			\end{pgfonlayer}
			\begin{pgfonlayer}{edgelayer}
				\draw [bend right=90, looseness=2.00] (0.center) to (1.center);
				\draw [bend left=90, looseness=1.50] (3.center) to (4.center);
				\draw (4.center) to (5.center);
				\draw (3.center) to (1.center);
				\draw (2.center) to (0.center);
			\end{pgfonlayer}
		\end{tikzpicture}
		\]
		
		Symmetric $*$-autonomous categories always have a canonical cyclor:
		
		\[
		\begin{tikzpicture}
			\begin{pgfonlayer}{nodelayer}
				\node [style=none] (0) at (-0.25, 2.25) {};
				\node [style=none] (1) at (-0.25, 0.5) {};
				\node [style=none] (12) at (-0.85, 1.7) {$*\eta$};
				\node [style=none] (2) at (-1.5, 0.5) {};
				\node [style=none] (3) at (-1.5, 1) {};
				\node [style=none] (4) at (-0.5, 1) {};
				\node [style=none] (34) at (-0.85, -0.4) {$\epsilon*$};
				\node [style=none] (5) at (-0.5, -1) {};
				\node [style=none] (6) at (0, 2) {$A^{*}$};
				\node [style=none] (7) at (-0.15, -0.75) {${~^*A}$};
			\end{pgfonlayer}
			\begin{pgfonlayer}{edgelayer}
				\draw (0.center) to (1.center);
				\draw [bend left=90, looseness=1.75] (1.center) to (2.center);
				\draw (2.center) to (3.center);
				\draw [bend left=90, looseness=1.75] (3.center) to (4.center);
				\draw (4.center) to (5.center);
			\end{pgfonlayer}
		\end{tikzpicture} 
		\]
		
		%forward reference
		We shall use the cyclor in Section \ref{daggers-duals-conjugation} to show how conjugation 
		and dagger are related in the presence of dualization.
		
%%%%%%%%%%%%%%%%%%%%%%%%%%%%%%%%%%%%%%%%%%%%%%%%%%%%%
\section{Linear functors and transformations}
\label{Sec: linear functor}
%%%%%%%%%%%%%%%%%%%%%%%%%%%%%%%%%%%%%%%%%%%%%%%%%%

Functors between LDCs are referred to as linear functors \cite{CS99}. Following the same 
pattern that LDCs generalize monoidal categories, linear functors generalize monoidal functors. 
Moreover, these general functors also account for the structures 
in linear logic such as exponential modalities $({!}, {?})$ and additive connectives $(+, \with)$ 
which are linear functors satisfying additional properties. 

\subsection{Linear functors and transformations}

In order to introduce linear functors, we first recall the definition of monoidal functors.
A functor $F: \X \to \Y$ between monoidal categories is a monoidal functor if its equipped with natural transformations $m_\ox: F(A) \ox F(B) \to F(A \ox B)$ and $m_I: I \to F(I)$ such that the following diagrams commute:

\[ \xymatrixcolsep{15mm}
\xymatrix{
(F(A) \ox F(B)) \ox F(C) \ar[d]_{a_\ox} \ar[r]^{m_\ox \ox 1} & F(A \ox B) \ox F(C) \ar[r]^{m_\ox} & F((A \ox B) \ox C) \ar[d]^{F(a_\ox)} \\
F(A) \ox (F(B) \ox F(C)) \ar[r]_{1 \ox m_\ox} & F(A) \ox F(B \ox C) \ar[r]_{m_\ox} & F(A \ox (B \ox C)) 
} \]

\medskip

\[ \xymatrix{
F(A) \ox I \ar[drr]^{u_\ox^R} \ar[d]_{1 \ox m_I} \\
F(A) \ox F(I) \ar[r]_{m_\ox} & F(A \ox I) \ar[r]_{F(u_\ox^L)} & F(A)}  ~~~~~~~~~~~~~~
\xymatrix{
I \ox F(A) \ar[drr]^{u_\ox^L} \ar[d]_{m_I \ox 1} \\
F(I) \ox F(A) \ar[r]_{m_\ox} & F(I \ox A) \ar[r]_{F(u_\ox^L)} & F(A)
}
\]

The first diagram for the monoidal functor is the associative law, 
and the other two diagrams are the right and the left unit laws respectively. 
Dually, a functor $(F, n_\ox, n_I): \X \to Y$ between monoidal categories is 
comonoidal if  $(F, n_\ox, n_I): \X^{\op} \to \Y^{\op}$ is monoidal. 
Monoidal functors preserve monoids and the comonoidal functors perserve comonoids.

\begin{definition} \cite[Definition 1]{CS99}
Given linearly distributive categories $\X$ and $\Y$, a {\bf linear functor} $F: \X \to \Y$ consists of 
\begin{enumerate}[(i)]
\item a pair of functors $F = (F_\ox, F_\oa)$; $(F_\ox, m_\ox, m_\top): \X \to \Y$ which is 
monoidal with respect to $\ox$ and $(F_\oa, n_\oa, n_\bot): \X \to \Y$ which is comonoidal 
with respect to $\oa$. We refer to $m_\ox$ and $n_\oa$ as {\bf tensor laxors}, 
and $m_\top$ and $n_\bot$ as {\bf unit laxors}. 

\item natural transformations:
\begin{align*}
\nu_\ox^R &: F_\ox(A \oa B) \to F_\oa(A) \oa F_\ox(B) \\
\nu_\ox^L &: F_\ox(A \oa B) \to F_\ox(A) \oa F_\oa(B) \\
\nu_\oa^R &: F_\ox(A) \ox F_\oa(B) \to F_\oa(A \ox B) \\
\nu_\oa^L &: F_\oa(A) \ox F_\ox(B) \to F_\oa( A \ox B)
\end{align*}
\end{enumerate}
such that the following coherence conditions hold:
\begin{enumerate}[{\bf \small [LF.1]}]
\item 
\begin{enumerate}[(a)]
\item $F_\ox(u^L_\oa) = \nu^R_\ox (n_\bot \oa 1) u^L_\oa$
\[ \xymatrix{
F_\ox( \bot \oa A) \ar[r]^{F_\ox(u_\oa^L)} \ar[d]_{\nu_\ox^R} & F_\ox(A) \\
F_\oa(\bot) \oa F_\ox(A) \ar[r]_{n_\bot \oa 1} & \bot \oa F_\ox(A) \ar[u]_{u_\oa^L}
} \]
\item $\nu_\ox^L ( 1 \oa n_\bot ) u_\oa^R = F_\ox( u_\oa^R) $
\item $(u_\ox^L)^{-1} (m_\top \ox 1) \nu_\oa^R = F_\oa((u_\ox^L)^{-1})$
\item $(u_\ox^R)^{-1} (m_\top \ox 1) \nu_\oa^L = F_\oa((u_\ox^R)^{-1}) $
\end{enumerate} 
\item 
\begin{enumerate}[(a)]
\item $F_\ox(a_\oa) \nu^R_\ox (1 \oa \nu^R_\ox) = \nu^R_\ox (n_\oa \oa 1) a_\oa$
\[ \xymatrix{
F_\ox((A \oa B) \oa C) \ar[r]^{F_\ox(a_\oa)} \ar[r]^{F_\ox(a_\oa)} \ar[d]_{\nu_\ox^R} & 
F_\ox(A \oa (B \oa C)) \ar[d]^{\nu_\ox^R} \\
F_\oa(A \oa B) \oa F_\ox(C) \ar[d]_{n_\oa \oa 1} &
F_\oa(A) \oa F_\ox (B \oa C) \ar[d]^{1 \oa \nu_\ox^R} \\
(F_\oa (A) \oa F_\oa(B)) \oa F_\ox(C) \ar[r]_{a_\oa} &
F_\oa(A) \oa (F_\oa(B) \oa F_\oa(C))
} \]
\item $F_\ox(a_\oa) \nu_\ox^L (1 \oa n_\oa ) = \nu^L_\oa (\nu^L \oa 1 ) a_\oa$
\item $(m_\ox \ox 1) \nu_\oa^R F_\oa(a_\ox) = a_\ox (1 \ox \nu_\oa^R) \nu_\oa^R$
\item $(\nu^R_\oa \ox 1) \nu_\oa^L F_\oa(a_\ox) = a_\ox (1 \ox m_\ox) \nu_\oa^L$
\end{enumerate}
\item 
\begin{enumerate}[(a)]
\item $F_\ox(a_\oa)\nu^R_\ox(1 \oa \nu^L_\ox) = \nu_\ox^L (\nu^R_\ox \oa 1) a_\oa$
\[ \xymatrix{
F_\ox((A \oa B) \oa C) \ar[r]^{F_\ox(a_\oa)} \ar[d]_{\nu_\ox^L} & 
F_\ox(A \oa (B \oa C))  \ar[d]^{\nu_\ox^R} \\
F_\ox(A \oa B) \oa F_\oa(C) \ar[d]_{\nu_\ox^R \oa 1} &
F_\oa(A) \oa F_\ox (B \oa C) \ar[d]^{1 \oa \nu_\ox^L} \\
(F_\oa (A) \oa F_\ox(B)) \oa F_\oa(C) \ar[r]_{a_\oa}&
F_\oa(A) \oa (F_\ox(B) \oa F_\oa(C))}
\]
\item $(\nu^R_\oa \ox 1) \nu^L_\oa F_\oa(a_\ox) = a_\ox (1 \ox \nu_\oa^L) \nu_\oa^R$
\end{enumerate}
\item 
\begin{enumerate}[(a)]
\item $(1 \ox \nu^R_\ox) \partial^L (\nu^R_\oa \oa 1) = m_\ox F_\ox(\partial^L) \nu^R_\ox$
\[ \xymatrix{
F_\ox(A) \ox F_\ox(B \oa C) \ar[r]^{1 \ox \nu_\ox^R} \ar[d]_{m_\ox} &
F_\ox(A) \ox (F_\oa(B) \oa F_\ox(C)) \ar[d]^{\partial^L} \\
F_\ox(A \ox (B \oa C)) \ar[d]_{F_\ox(\partial^L)} &
(F_\ox(A) \ox F_\oa(B)) \oa F_\ox(C) \ar[d]^{\nu_\oa^R \oa 1} \\
F_\ox((A \ox B) \oa C) \ar[r]_{\nu_\ox^R} &
F_\oa(A \oa B) \oa F_\ox(C)
} \]
\item $(\nu_\ox^L \ox 1) \partial^R (1 \oa \nu_\oa^L) = m_\ox F_\ox(\partial^R) \nu_\ox^L$
\item $(1 \ox \nu_\ox^L) \partial^L (\nu_\oa^L \oa 1) = \nu_\oa^L F_\oa(\partial^L) n_\oa$
\item $(\nu_\ox^R \ox 1) \partial^R (1 \oa \nu_\oa^R) = \nu_\oa^R F_\oa(\partial^R) n_\oa $
\end{enumerate}
\item 
\begin{enumerate}[(a)]
\item $(1 \ox \nu^L_\ox) \partial^L(m_\ox \oa 1) = m_\ox F_\ox(\partial^L) \nu^L_\ox$ 
\[ \xymatrix{
F_\ox(A) \ox F_\ox(B \oa C) \ar[r]^{1 \ox \nu_\ox^L} \ar[d]_{m_\ox} &
F_\ox(A) \ox (F_\ox(B) \oa F_\oa(C)) \ar[d]^{\partial^L} \\
F_\ox(A \ox (B \oa C)) \ar[d]_{F_\ox(\partial^L)} &
(F_\ox(A) \ox F_\ox(B)) \oa F_\oa(C) \ar[d]^{m_\ox \oa 1} \\
F_\ox((A \ox B) \oa C) \ar[r]_{\nu_\ox^L} &
F_\ox(A \ox B) \oa F_\ox(C)
}\]
\item $(\nu_\ox^R \ox 1) \partial^R (1 \oa m_\ox) = m_\ox F_\ox(\partial^R) \nu_\ox^R$
\item $(1 \ox n_\oa) \partial^L (\nu_\oa^R \oa 1) = \nu_\oa^R F_\oa(\partial^L) n_\oa $
\item $(n_\oa \ox 1) \partial^R (1 \oa \nu_\oa^L) = \nu_\oa^L F_\oa(\partial^R) n_ \oa $
\end{enumerate}
\end{enumerate}
\end{definition}

In the graphical calculus, functors are represented by linear functor boxes \cite{CS99}. 
A linear functor box can either be monoidal or comonoidal. When the functor box is monoidal $(F_\ox)$, 
it has one principal output wire (of $F_\ox$ type) represented by a port where the wire exits the box and 
the other wires (of $F_\oa$ type) are auxiliary. When the box is comonoidal ($F_\oa$), 
it has one principal input wire with a port (of $F_\oa$ type) and the other wires (of $F_\ox$ type) are auxiliary. 
The functor boxes are subject to a very natural ``box eats box'' calculus described in \cite{CS99}.  
A box can eat another box only when a ported wire meets an auxiliary wire. The linear strengths are drawn in the graphical calculus as follows:
\[\nu_\oa^L = \begin{tikzpicture}
	\begin{pgfonlayer}{nodelayer}
		\node [style=none] (0) at (-2, 2) {};
		\node [style=none] (1) at (-1.5, 2) {};
		\node [style=none] (2) at (-2.25, 2) {};
		\node [style=none] (3) at (-2.25, 1) {};
		\node [style=none] (4) at (-0.75, 1) {};
		\node [style=none] (5) at (-0.75, 2) {};
		\node [style=none] (6) at (-2, 2.75) {};
		\node [style=none] (7) at (-1, 2.75) {};
		\node [style=none] (61) at (-2.75, 2.75) {$F_\oa(A)$};
		\node [style=none] (71) at (-0.25, 2.75) {$F_\ox(B)$};
		\node [style=none] (8) at (-1.5, 0.25) {};
		\node [style=none] (81) at (-2.25, 0) {$F_\oa(A \ox B) $};
		\node [style=ox] (9) at (-1.5, 1.5) {};
		\node [style=none] (10) at (-2, 1.25) {$F_\oa$};
	\end{pgfonlayer}
	\begin{pgfonlayer}{edgelayer}
		\draw [in=-90, out=-90, looseness=1.25] (0.center) to (1.center);
		\draw [bend right=15, looseness=1.00] (6.center) to (9);
		\draw [bend right=15, looseness=0.75] (9) to (7.center);
		\draw (2.center) to (3.center);
		\draw (3.center) to (4.center);
		\draw (4.center) to (5.center);
		\draw (5.center) to (2.center);
		\draw (9) to (8.center);
	\end{pgfonlayer}
\end{tikzpicture} ~~~~~~~~ \nu_\oa^R = \begin{tikzpicture}
	\begin{pgfonlayer}{nodelayer}
		\node [style=none] (0) at (-1.5, 2) {};
		\node [style=none] (1) at (-1, 2) {};
		\node [style=none] (2) at (-2.25, 2) {};
		\node [style=none] (3) at (-2.25, 1) {};
		\node [style=none] (4) at (-0.75, 1) {};
		\node [style=none] (5) at (-0.75, 2) {};
		\node [style=none] (6) at (-2, 2.75) {};
		\node [style=none] (7) at (-1, 2.75) {};
		\node [style=none] (61) at (-2.75, 2.75) {$F_\ox(A)$};
		\node [style=none] (71) at (-0.25, 2.75) {$F_\oa(B)$};
		\node [style=none] (8) at (-1.5, 0.25) {};
		\node [style=none] (81) at (-2.25, 0) {$F_\oa(A \ox B) $};
		\node [style=ox] (9) at (-1.5, 1.5) {};
		\node [style=none] (10) at (-2, 1.25) {$F_\oa$};
	\end{pgfonlayer}
	\begin{pgfonlayer}{edgelayer}
		\draw [in=-90, out=-90, looseness=1.25] (0.center) to (1.center);
		\draw [bend right=15, looseness=1.00] (6.center) to (9);
		\draw [bend right=15, looseness=0.75] (9) to (7.center);
		\draw (2.center) to (3.center);
		\draw (3.center) to (4.center);
		\draw (4.center) to (5.center);
		\draw (5.center) to (2.center);
		\draw (9) to (8.center);
	\end{pgfonlayer}
\end{tikzpicture}  ~~~~~~~~ \nu_\ox^L = \begin{tikzpicture}
	\begin{pgfonlayer}{nodelayer}
		\node [style=none] (0) at (-2.25, 1) {};
		\node [style=none] (1) at (-2.25, 2) {};
		\node [style=none] (2) at (-0.75, 2) {};
		\node [style=none] (3) at (-0.75, 1) {};
		\node [style=none] (4) at (-2, 0.25) {};
		\node [style=none] (5) at (-1, 0.25) {};
		\node [style=none] (6) at (-1.5, 2.75) {};
		\node [style=none] (41) at (-2.75, 0.25) {$F_\ox(A)$};
		\node [style=none] (51) at (-0.25, 0.25) {$F_\oa(B)$};
		\node [style=none] (61) at (-2, 3) {$F_\ox(A \oa B)$};
		\node [style=oa] (7) at (-1.5, 1.5) {};
		\node [style=none] (8) at (-2, 1.75) {$F_\ox$};
		\node [style=none] (9) at (-2, 1) {};
		\node [style=none] (10) at (-1.5, 1) {};
	\end{pgfonlayer}
	\begin{pgfonlayer}{edgelayer}
		\draw [bend left=15, looseness=1.00] (4.center) to (7);
		\draw [bend left=15, looseness=0.75] (7) to (5.center);
		\draw (0.center) to (1.center);
		\draw (1.center) to (2.center);
		\draw (2.center) to (3.center);
		\draw (3.center) to (0.center);
		\draw (7) to (6.center);
		\draw [in=90, out=90, looseness=1.25] (9.center) to (10.center);
	\end{pgfonlayer}
\end{tikzpicture} ~~~~~~~~~ \nu_\ox^R = \begin{tikzpicture}
	\begin{pgfonlayer}{nodelayer}
		\node [style=none] (0) at (-2.25, 1) {};
		\node [style=none] (1) at (-2.25, 2) {};
		\node [style=none] (2) at (-0.75, 2) {};
		\node [style=none] (3) at (-0.75, 1) {};
		\node [style=none] (4) at (-2, 0.25) {};
		\node [style=none] (5) at (-1, 0.25) {};
		\node [style=none] (41) at (-2.75, 0.25) {$F_\oa(A)$};
		\node [style=none] (51) at (-0.25, 0.25) {$F_\ox(B)$};
		\node [style=none] (61) at (-2, 3) {$F_\ox(A \oa B)$};
		\node [style=none] (6) at (-1.5, 2.75) {};
		\node [style=oa] (7) at (-1.5, 1.5) {};
		\node [style=none] (8) at (-2, 1.75) {$F_\ox$};
		\node [style=none] (9) at (-1.5, 1) {};
		\node [style=none] (10) at (-1, 1) {};
	\end{pgfonlayer}
	\begin{pgfonlayer}{edgelayer}
		\draw [bend left=15, looseness=1.00] (4.center) to (7);
		\draw [bend left=15, looseness=0.75] (7) to (5.center);
		\draw (0.center) to (1.center);
		\draw (1.center) to (2.center);
		\draw (2.center) to (3.center);
		\draw (3.center) to (0.center);
		\draw (7) to (6.center);
		\draw [in=90, out=90, looseness=1.25] (9.center) to (10.center);
	\end{pgfonlayer}
\end{tikzpicture} \]
\[ m_\ox = \begin{tikzpicture}
	\begin{pgfonlayer}{nodelayer}
		\node [style=none] (0) at (-2.25, 2) {};
		\node [style=none] (1) at (-2.25, 1) {};
		\node [style=none] (2) at (-0.75, 1) {};
		\node [style=none] (3) at (-0.75, 2) {};
		\node [style=none] (4) at (-2, 2.75) {};
		\node [style=none] (5) at (-1, 2.75) {};
		\node [style=none] (41) at (-2.75, 2.75) {$F_\ox(A)$};
		\node [style=none] (51) at (-0.25, 2.75) {$F_\ox(B)$};
		\node [style=none] (6) at (-1.5, 0.25) {};
		\node [style=none] (61) at (-2, 0) {$F_\ox(A \ox B)$};
		\node [style=ox] (7) at (-1.5, 1.5) {};
		\node [style=none] (8) at (-2, 1.25) {$F_\ox$};
		\node [style=none] (9) at (-1.75, 1) {};
		\node [style=none] (10) at (-1.25, 1) {};
	\end{pgfonlayer}
	\begin{pgfonlayer}{edgelayer}
		\draw [bend right=15, looseness=1.00] (4.center) to (7);
		\draw [bend right=15, looseness=0.75] (7) to (5.center);
		\draw (0.center) to (1.center);
		\draw (1.center) to (2.center);
		\draw (2.center) to (3.center);
		\draw (3.center) to (0.center);
		\draw (7) to (6.center);
		\draw [in=90, out=90, looseness=1.25] (9.center) to (10.center);
	\end{pgfonlayer}
\end{tikzpicture} ~~~~~ m_\top =  \begin{tikzpicture}
	\begin{pgfonlayer}{nodelayer}
	      \node [style=none] (8) at (0, 1) {};
		\node [style=circle] (0) at (0, -0) {$\top$};
		\node [style=none] (1) at (0.75, -0.5) {};
		\node [style=none] (2) at (2.75, -0.5) {};
		\node [style=none] (3) at (0.75, -1.5) {};
		\node [style=none] (4) at (2.75, -1.5) {};
		\node [style=none] (5) at (1.75, -2.5) {};
		\node [style=circle] (6) at (1.75, -1) {$\top$};
		\node [style=circle, scale=0.5] (7) at (1.75, -2) {};
	\end{pgfonlayer}
	\begin{pgfonlayer}{edgelayer}
		\draw (1.center) to (2.center);
		\draw (2.center) to (4.center);
		\draw (4.center) to (3.center);
		\draw (3.center) to (1.center);
		\draw (6) to (5.center);
		\draw [dotted, in=-165, out=-90, looseness=1.25] (0) to (7);
		\draw (0) to (8.center);
	\end{pgfonlayer}
\end{tikzpicture} = \begin{tikzpicture}
	\begin{pgfonlayer}{nodelayer}
	      \node [style=none] (8) at (2.75, 1) {};
		\node [style=circle] (0) at (2.75, -0) {$\top$};
		\node [style=none] (1) at (2, -0.5) {};
		\node [style=none] (2) at (0, -0.5) {};
		\node [style=none] (3) at (2, -1.5) {};
		\node [style=none] (4) at (0, -1.5) {};
		\node [style=none] (5) at (1, -2.5) {};
		\node [style=circle] (6) at (1, -1) {$\top$};
		\node [style=circle, scale=0.5] (7) at (1, -2) {};
	\end{pgfonlayer}
	\begin{pgfonlayer}{edgelayer}
		\draw (1.center) to (2.center);
		\draw (2.center) to (4.center);
		\draw (4.center) to (3.center);
		\draw (3.center) to (1.center);
		\draw (6) to (5.center);
		\draw (0) to (8.center);
		\draw [dotted, in=-15, out=-90, looseness=1.25] (0) to (7);
	\end{pgfonlayer}
\end{tikzpicture} ~~~~~ n_\oa =  \begin{tikzpicture}
	\begin{pgfonlayer}{nodelayer}
		\node [style=none] (0) at (-2.25, 1) {};
		\node [style=none] (1) at (-2.25, 2) {};
		\node [style=none] (2) at (-0.75, 2) {};
		\node [style=none] (3) at (-0.75, 1) {};
		\node [style=none] (4) at (-2, 0.25) {};
		\node [style=none] (5) at (-1, 0.25) {};
		\node [style=none] (41) at (-2.75, 0.25) {$F_\oa(A)$};
		\node [style=none] (51) at (-0.25, 0.25) {$F_\oa(B)$};
		\node [style=none] (61) at (-2, 3) {$F_\oa(A \oa B)$};
		\node [style=none] (6) at (-1.5, 2.75) {};
		\node [style=oa] (7) at (-1.5, 1.5) {};
		\node [style=none] (8) at (-2, 1.75) {$F_\oa$};
		\node [style=none] (9) at (-1.25, 2) {};
		\node [style=none] (10) at (-1.75, 2) {};
	\end{pgfonlayer}
	\begin{pgfonlayer}{edgelayer}
		\draw [bend left=15, looseness=1.00] (4.center) to (7);
		\draw [bend left=15, looseness=0.75] (7) to (5.center);
		\draw (0.center) to (1.center);
		\draw (1.center) to (2.center);
		\draw (2.center) to (3.center);
		\draw (3.center) to (0.center);
		\draw (7) to (6.center);
		\draw [in=-90, out=-90, looseness=1.25] (9.center) to (10.center);
	\end{pgfonlayer}
\end{tikzpicture}
~~~~~ n_\bot = 
\begin{tikzpicture}
	\begin{pgfonlayer}{nodelayer}
		\node [style=none] (0) at (-3, 2) {};
		\node [style=none] (1) at (-3, 1) {};
		\node [style=none] (2) at (-1, 2) {};
		\node [style=none] (3) at (-1, 1) {};
		\node [style=circle] (4) at (-2, 1.5) {$\bot$};
		\node [style=circle] (5) at (0, 1) {$\bot$};
		\node [style=none] (6) at (-2, 3) {};
		\node [style=circle, scale=0.5] (7) at (-2, 2.5) {};
		\node [style=none] (8) at (0, 0) {};
	\end{pgfonlayer}
	\begin{pgfonlayer}{edgelayer}
		\draw (0.center) to (1.center);
		\draw (1.center) to (3.center);
		\draw (3.center) to (2.center);
		\draw (2.center) to (0.center);
		\draw (6.center) to (4);
		\draw (8.center) to (5);
		\draw [dotted, bend left=45, looseness=1.25] (7) to (5);
	\end{pgfonlayer}
\end{tikzpicture} = \begin{tikzpicture}
	\begin{pgfonlayer}{nodelayer}
		\node [style=none] (0) at (0, 2) {};
		\node [style=none] (1) at (0, 1) {};
		\node [style=none] (2) at (-2, 2) {};
		\node [style=none] (3) at (-2, 1) {};
		\node [style=circle] (4) at (-1, 1.5) {$\bot$};
		\node [style=circle] (5) at (-3, 1) {$\bot$};
		\node [style=none] (6) at (-1, 3) {};
		\node [style=circle, scale=0.5] (7) at (-1, 2.5) {};
		\node [style=none] (8) at (-3, 0) {};
	\end{pgfonlayer}
	\begin{pgfonlayer}{edgelayer}
		\draw (0.center) to (1.center);
		\draw (1.center) to (3.center);
		\draw (3.center) to (2.center);
		\draw (2.center) to (0.center);
		\draw (6.center) to (4);
		\draw (8.center) to (5);
		\draw [dotted, bend right=45, looseness=1.25] (7) to (5);
	\end{pgfonlayer}
\end{tikzpicture} \]

When working in the categorical doctrine of {\em symmetric} LDCs we will expect the linear functors to preserve the symmetry.   
Thus, a {\bf symmetric linear functor} is a linear functor $F= (F_\ox,F_\oa)$ 
which satisfies in addition:
\[ \xymatrix{F_\ox(A) \ox F_\ox(B) \ar[d]_{c_\ox} \ar[rr]^{m_\ox} & & F_\ox(A \ox B) \ar[d]^{F_\ox(c_\ox)} \\
                   F_\ox(B) \ox F_\ox(A) \ar[rr]_{m_\ox} & & F_\ox(B \ox A) }
    ~~~~~~
   \xymatrix{F_\oa(A \oa B)  \ar[d]_{F_\oa(c_\oa)} \ar[rr]^{n_\ox} & & F_\oa(A) \oa F_\oa(B)\ar[d]^{c_\oa} \\
                   F_\oa(B \ox A) \ar[rr]_{n_\oa} & &  F_\oa(B) \oa F_\oa(A) } \]

Linear functors preserve duals:

\begin{lemma} \cite{CKS00}
\label{Lemma: linear adjoints}
Linear functors preserve duals: when $F: \X \to \Y$ is a linear functor and $(\eta, \epsilon): A \dashvv B \in \X$, then $F_\ox(A) \dashvv F_\oa(B)$ and $F_\oa(A) \dashvv F_\ox(B)$.
\end{lemma}
\begin{proof}
The unit and counit of the adjunction $(\eta', \epsilon'): F_\ox(A) \dashvv F_\oa(B)$ is given as follows:
\[ \eta' := \top \xrightarrow{m_\top} F_\ox(\top) \xrightarrow{F_\ox(\eta)} F_\ox( A \oa B) \xrightarrow{\nu_\ox^L} F_\ox(A) \oa F_\oa(B) =
\begin{tikzpicture}
	\begin{pgfonlayer}{nodelayer}
		\node [style=none] (0) at (0.5, -0.75) {};
		\node [style=none] (1) at (-0.5, -2) {};
		\node [style=none] (2) at (0.5, -2) {};
		\node [style=none] (7) at (-1, 0) {};
		\node [style=none] (8) at (1, 0) {};
		\node [style=none] (9) at (1, -1.25) {};
		\node [style=none] (10) at (-1, -1.25) {};
		\node [style=none] (11) at (-0.75, -1.25) {};
		\node [style=none] (12) at (-0.25, -1.25) {};
		\node [style=none] (13) at (0.75, -0.25) {$F$};
		\node [style=none] (14) at (-0.5, -0.75) {};
		\node [style=none] (15) at (0, -0.25) {$\eta$};
	\end{pgfonlayer}
	\begin{pgfonlayer}{edgelayer}
		\draw (0.center) to (2.center);
		\draw (7.center) to (8.center);
		\draw (8.center) to (9.center);
		\draw (9.center) to (10.center);
		\draw (10.center) to (7.center);
		\draw [bend left=90, looseness=1.25] (11.center) to (12.center);
		\draw [bend left=90, looseness=1.25] (14.center) to (0.center);
		\draw [style=filled] (14.center) to (1.center);
	\end{pgfonlayer}
\end{tikzpicture} \]
\[ \epsilon' := F_\oa(B) \ox F_\ox(A) \xrightarrow{\nu_\oa^L} F_\oa(B \ox A) \xrightarrow{F_\oa(\epsilon)} F_\oa(\bot) \xrightarrow{n_\bot} \bot =
\begin{tikzpicture}
	\begin{pgfonlayer}{nodelayer}
		\node [style=none] (0) at (0.5, -1) {};
		\node [style=none] (1) at (-0.5, 0.25) {};
		\node [style=none] (2) at (0.5, 0.25) {};
		\node [style=none] (7) at (-1, -1.75) {};
		\node [style=none] (8) at (1, -1.75) {};
		\node [style=none] (9) at (1, -0.5) {};
		\node [style=none] (10) at (-1, -0.5) {};
		\node [style=none] (11) at (-0.75, -0.5) {};
		\node [style=none] (12) at (-0.25, -0.5) {};
		\node [style=none] (13) at (0.75, -1.5) {$F$};
		\node [style=none] (14) at (-0.5, -1) {};
		\node [style=none] (15) at (0, -1.5) {$\epsilon$};
	\end{pgfonlayer}
	\begin{pgfonlayer}{edgelayer}
		\draw (0.center) to (2.center);
		\draw (7.center) to (8.center);
		\draw (8.center) to (9.center);
		\draw (9.center) to (10.center);
		\draw (10.center) to (7.center);
		\draw [bend right=90, looseness=1.25] (11.center) to (12.center);
		\draw [bend right=90, looseness=1.25] (14.center) to (0.center);
		\draw (14.center) to (1.center);
	\end{pgfonlayer}
\end{tikzpicture}	 \]

The unit and counit of the other adjunction is given similarly.
\end{proof}			

Natural transformations between linear functors also break into two components linking 
respectively the tensor functors by a monoidal transformation and, in the opposite  direction, 
the par functors by a comonoidal transformation.

A monoidal transformation $\alpha: F \Rightarrow G$ between two monoidal functors is a natural 
transformation $\alpha: F \Rightarrow G$ such that the following diagrams commute: 
\[ \xymatrixcolsep{12mm} \xymatrix{ 
	F(A) \ox F(B) \ar[r]^{\alpha_A \ox \alpha_B} \ar[d]_{m_\ox^F} & G(A) \ox G(B) \ar[d]^{m_\ox^G} \\
	F(A \ox B) \ar[r]_{\alpha_{A \ox B}} & G(A \ox B) } ~~~~~~~~
\xymatrix{
	I \ar[d]_{m_I^F} \ar[dr]^{m_I^F} \\
	F(I) \ar[r]_{\alpha_I} & G(I) } \] 
The coherences for a comonoidal transformation are precisely the mirror images of the above coherences.

\begin{definition} \cite[Definition 3]{CS99}
A {\bf linear (natural) transformation}\footnote{In this thesis, a linear transformation 
is a natural transformation between linear functors, and is different 
from the linear transformations of linear algebra. We drop the word ``natural" for brevity.},  $\alpha: F \to G$,  between parallel linear functors $F,G: \X \to \Y$  consists of a pair of natural transformations $\alpha = (\alpha_\ox,\alpha_\oa)$ such that $\alpha_\ox: F_\ox \to G_\ox$ is a monoidal transformation and $\alpha_\oa: G_\oa \to F_\oa$ is a comonoidal transformation satisfying the following coherence conditions:
\begin{enumerate}[{\bf \small [LT.1]}]
\item $a_\ox \nu^R_\ox (a_\oa \oa 1) = \nu^R_\ox (1 \oa a_\ox)$
\[ \xymatrix{
F_\ox (A \oa B) \ar[rr]^{\alpha_\ox} \ar[d]_{\nu_\ox^R} & & G_\ox(A \oa B) \ar[d]^{\nu_\ox^R} \\
F_\oa(A) \oa F_\ox(B) \ar[dr]_{1 \oa \alpha_\ox} & & G_\oa(A) \oa G_\ox(B) \ar[ld]^{\alpha_\oa \oa 1} \\
& F_\oa(A) \oa G_\ox(B)&
} \]
\item $\alpha_\ox \nu_\ox^L (1 \oa \alpha_\oa) = \nu_\ox^L (\alpha_\ox \oa 1)$
\item $(1 \ox \alpha_\ox) \nu_\oa^L (\alpha_\oa) = (\alpha_\oa \ox 1) \nu_\oa^L$
\item $(\alpha_\ox \ox 1) \nu_\oa^R \alpha_\oa = (1 \ox \alpha_\oa) \nu_\oa^R$
\end{enumerate}
\end{definition}

Conditions {\bf [LT.1]} - {\bf [LT.4]} are represented graphically as follows:
\[
\mbox{\small\bf [LT.1]}~ \begin{tikzpicture} %nat-a
	\begin{pgfonlayer}{nodelayer}
		\node [style=none] (0) at (-0.5, -0.75) {};
		\node [style=none] (1) at (-0.5, 0.25) {};
		\node [style=none] (2) at (-2.5, 0.25) {};
		\node [style=none] (3) at (-2.5, -0.75) {};
		\node [style=oa] (4) at (-1.5, -0.25) {};
		\node [style=none] (5) at (-1, -2) {};
		\node [style=none] (6) at (-2, -2) {};
		\node [style=none] (7) at (-1.5, 1.25) {};
		\node [style=circle, scale=2] (8) at (-1, -1.25) {};
		\node [style=none] (9) at (-1, -1.25) {$\alpha_\ox$};
		\node [style=none] (10) at (-0.75, -0) {$F$};
		\node [style=none] (11) at (-1.25, -0.75) {};
		\node [style=none] (12) at (-0.75, -0.75) {};
	\end{pgfonlayer}
	\begin{pgfonlayer}{edgelayer}
		\draw (0.center) to (1.center);
		\draw (1.center) to (2.center);
		\draw (2.center) to (3.center);
		\draw (0.center) to (3.center);
		\draw [in=90, out=-150, looseness=1.00] (4) to (6.center);
		\draw (4) to (7.center);
		\draw (5.center) to (8);
		\draw [in=-30, out=90, looseness=1.00] (8) to (4);
		\draw [bend left=90, looseness=1.25] (11.center) to (12.center);
	\end{pgfonlayer}
\end{tikzpicture} 
 = \begin{tikzpicture} %nat-b
	\begin{pgfonlayer}{nodelayer}
		\node [style=none] (0) at (-0.5, -0.75) {};
		\node [style=none] (1) at (-0.5, 0.25) {};
		\node [style=none] (2) at (-2.5, 0.25) {};
		\node [style=none] (3) at (-2.5, -0.75) {};
		\node [style=oa] (4) at (-1.5, -0.25) {};
		\node [style=none] (5) at (-1, -2) {};
		\node [style=none] (6) at (-2, -2) {};
		\node [style=none] (7) at (-2, -1.25) {$\alpha_\oa$};
		\node [style=circle, scale=2] (8) at (-1.5, 1) {};
		\node [style=none] (9) at (-1.5, 2) {};
		\node [style=none] (10) at (-1.5, 1) {$\alpha_\ox$};
		\node [style=none] (11) at (-0.75, -0) {$G$};
		\node [style=circle, scale=2] (12) at (-2, -1.25) {};
		\node [style=none] (13) at (-1.5, -0.75) {};
		\node [style=none] (14) at (-1, -0.75) {};
	\end{pgfonlayer}
	\begin{pgfonlayer}{edgelayer}
		\draw (0.center) to (1.center);
		\draw (1.center) to (2.center);
		\draw (2.center) to (3.center);
		\draw (0.center) to (3.center);
		\draw (9.center) to (8);
		\draw (8) to (4);
		\draw [in=90, out=-44, looseness=0.75] (4) to (5.center);
		\draw [bend left=90, looseness=1.25] (13.center) to (14.center);
		\draw (6.center) to (12);
		\draw [bend left, looseness=1.00] (12) to (4);
	\end{pgfonlayer}
\end{tikzpicture}  ~~~\mbox{\small\bf [LT.2]}~ \begin{tikzpicture}
	\begin{pgfonlayer}{nodelayer}
		\node [style=none] (0) at (-2.5, -0.75) {};
		\node [style=none] (1) at (-2.5, 0.25) {};
		\node [style=none] (2) at (-0.5, 0.25) {};
		\node [style=none] (3) at (-0.5, -0.75) {};
		\node [style=oa] (4) at (-1.5, -0.25) {};
		\node [style=none] (5) at (-2, -2) {};
		\node [style=none] (6) at (-1, -2) {};
		\node [style=none] (7) at (-1.5, 1.25) {};
		\node [style=circle, scale=2] (8) at (-2, -1.25) {};
		\node [style=none] (9) at (-2, -1.25) {$\alpha_\ox$};
		\node [style=none] (10) at (-0.75, 0) {$F$};
		\node [style=none] (11) at (-1.75, -0.75) {};
		\node [style=none] (12) at (-2.25, -0.75) {};
	\end{pgfonlayer}
	\begin{pgfonlayer}{edgelayer}
		\draw (0.center) to (1.center);
		\draw (1.center) to (2.center);
		\draw (2.center) to (3.center);
		\draw (0.center) to (3.center);
		\draw [in=90, out=-15, looseness=1.00] (4) to (6.center);
		\draw (4) to (7.center);
		\draw (5.center) to (8);
		\draw [in=-165, out=90, looseness=1.25] (8) to (4);
		\draw [bend right=90, looseness=1.25] (11.center) to (12.center);
	\end{pgfonlayer}
\end{tikzpicture} = \begin{tikzpicture}
	\begin{pgfonlayer}{nodelayer}
		\node [style=none] (0) at (-2.5, -0.75) {};
		\node [style=none] (1) at (-2.5, 0.25) {};
		\node [style=none] (2) at (-0.5, 0.25) {};
		\node [style=none] (3) at (-0.5, -0.75) {};
		\node [style=oa] (4) at (-1.5, -0.25) {};
		\node [style=none] (5) at (-2, -2) {};
		\node [style=none] (6) at (-1, -2) {};
		\node [style=none] (7) at (-1, -1.25) {$\alpha_\oa$};
		\node [style=circle, scale=2] (8) at (-1.5, 1) {};
		\node [style=none] (9) at (-1.5, 2) {};
		\node [style=none] (10) at (-1.5, 1) {$\alpha_\ox$};
		\node [style=none] (11) at (-0.75, -0) {$G$};
		\node [style=circle, scale=2] (12) at (-1, -1.25) {};
		\node [style=none] (13) at (-1.5, -0.75) {};
		\node [style=none] (14) at (-2, -0.75) {};
	\end{pgfonlayer}
	\begin{pgfonlayer}{edgelayer}
		\draw (0.center) to (1.center);
		\draw (1.center) to (2.center);
		\draw (2.center) to (3.center);
		\draw (0.center) to (3.center);
		\draw (9.center) to (8);
		\draw (8) to (4);
		\draw [in=90, out=-135, looseness=0.75] (4) to (5.center);
		\draw [bend right=90, looseness=1.25] (13.center) to (14.center);
		\draw (6.center) to (12);
		\draw [bend right, looseness=1.00] (12) to (4);
	\end{pgfonlayer}
\end{tikzpicture} ~~~ \mbox{\small\bf [LT.3]}~ \begin{tikzpicture}
	\begin{pgfonlayer}{nodelayer}
		\node [style=none] (0) at (-0.5, 0) {};
		\node [style=none] (1) at (-0.5, -1) {};
		\node [style=none] (2) at (-2.5, -1) {};
		\node [style=none] (3) at (-2.5, -0) {};
		\node [style=ox] (4) at (-1.5, -0.5) {};
		\node [style=none] (5) at (-1, 1.25) {};
		\node [style=none] (6) at (-2, 1.25) {};
		\node [style=none] (7) at (-1.5, -2) {};
		\node [style=circle, scale=2] (8) at (-1, 0.5) {};
		\node [style=none] (9) at (-1, 0.5) {$\alpha_\oa$};
		\node [style=none] (10) at (-0.75, -0.5) {$F$};
		\node [style=none] (11) at (-1.25, 0) {};
		\node [style=none] (12) at (-0.75, 0) {};
	\end{pgfonlayer}
	\begin{pgfonlayer}{edgelayer}
		\draw (0.center) to (1.center);
		\draw (1.center) to (2.center);
		\draw (2.center) to (3.center);
		\draw (0.center) to (3.center);
		\draw [in=-90, out=165, looseness=1.00] (4) to (6.center);
		\draw (4) to (7.center);
		\draw (5.center) to (8);
		\draw [in=15, out=-90, looseness=1.25] (8) to (4);
		\draw [bend right=90, looseness=1.25] (11.center) to (12.center);
	\end{pgfonlayer}
\end{tikzpicture} = \begin{tikzpicture}
	\begin{pgfonlayer}{nodelayer}
		\node [style=none] (0) at (-0.5, 0.75) {};
		\node [style=none] (1) at (-0.5, -0.25) {};
		\node [style=none] (2) at (-2.5, -0.25) {};
		\node [style=none] (3) at (-2.5, 0.75) {};
		\node [style=ox] (4) at (-1.5, 0.25) {};
		\node [style=none] (5) at (-1, 2) {};
		\node [style=none] (6) at (-2, 2) {};
		\node [style=none] (7) at (-2, 1.25) {$\alpha_\ox$};
		\node [style=circle, scale=2] (8) at (-1.5, -1) {};
		\node [style=none] (9) at (-1.5, -2) {};
		\node [style=none] (10) at (-1.5, -1) {$\alpha_\oa$};
		\node [style=none] (11) at (-0.75, 0.5) {$F$};
		\node [style=circle, scale=2] (12) at (-2, 1.25) {};
		\node [style=none] (13) at (-1.5, 0.75) {};
		\node [style=none] (14) at (-1, 0.75) {};
	\end{pgfonlayer}
	\begin{pgfonlayer}{edgelayer}
		\draw (0.center) to (1.center);
		\draw (1.center) to (2.center);
		\draw (2.center) to (3.center);
		\draw (0.center) to (3.center);
		\draw (9.center) to (8);
		\draw (8) to (4);
		\draw [in=-90, out=44, looseness=0.75] (4) to (5.center);
		\draw [bend right=90, looseness=1.25] (13.center) to (14.center);
		\draw (6.center) to (12);
		\draw [bend right, looseness=1.00] (12) to (4);
	\end{pgfonlayer}
\end{tikzpicture}~~~ \mbox{\small\bf [LT.4]}~ \begin{tikzpicture}
	\begin{pgfonlayer}{nodelayer}
		\node [style=none] (0) at (-2.5, 0) {};
		\node [style=none] (1) at (-2.5, -1) {};
		\node [style=none] (2) at (-0.5, -1) {};
		\node [style=none] (3) at (-0.5, 0) {};
		\node [style=ox] (4) at (-1.5, -0.5) {};
		\node [style=none] (5) at (-2, 1.25) {};
		\node [style=none] (6) at (-1, 1.25) {};
		\node [style=none] (7) at (-1.5, -2) {};
		\node [style=circle, scale=2] (8) at (-2, 0.5) {};
		\node [style=none] (9) at (-2, 0.5) {$\alpha_\oa$};
		\node [style=none] (10) at (-0.75, -0.25) {$G$};
		\node [style=none] (11) at (-1.75, 0) {};
		\node [style=none] (12) at (-2.25, 0) {};
	\end{pgfonlayer}
	\begin{pgfonlayer}{edgelayer}
		\draw (0.center) to (1.center);
		\draw (1.center) to (2.center);
		\draw (2.center) to (3.center);
		\draw (0.center) to (3.center);
		\draw [in=-90, out=15, looseness=1.00] (4) to (6.center);
		\draw (4) to (7.center);
		\draw (5.center) to (8);
		\draw [in=165, out=-90, looseness=1.25] (8) to (4);
		\draw [bend left=90, looseness=1.25] (11.center) to (12.center);
	\end{pgfonlayer}
\end{tikzpicture} = \begin{tikzpicture}
	\begin{pgfonlayer}{nodelayer}
		\node [style=none] (0) at (-2.5, 0.75) {};
		\node [style=none] (1) at (-2.5, -0.25) {};
		\node [style=none] (2) at (-0.5, -0.25) {};
		\node [style=none] (3) at (-0.5, 0.75) {};
		\node [style=ox] (4) at (-1.5, 0.25) {};
		\node [style=none] (5) at (-2, 2) {};
		\node [style=none] (6) at (-1, 2) {};
		\node [style=none] (7) at (-1, 1.25) {$\alpha_\ox$};
		\node [style=circle, scale=2] (8) at (-1.5, -1) {};
		\node [style=none] (9) at (-1.5, -2) {};
		\node [style=none] (10) at (-1.5, -1) {$\alpha_\oa$};
		\node [style=none] (11) at (-0.75, 0.5) {$F$};
		\node [style=circle, scale=2] (12) at (-1, 1.25) {};
		\node [style=none] (13) at (-1.5, 0.75) {};
		\node [style=none] (14) at (-2, 0.75) {};
	\end{pgfonlayer}
	\begin{pgfonlayer}{edgelayer}
		\draw (0.center) to (1.center);
		\draw (1.center) to (2.center);
		\draw (2.center) to (3.center);
		\draw (0.center) to (3.center);
		\draw (9.center) to (8);
		\draw (8) to (4);
		\draw [in=-90, out=136, looseness=0.75] (4) to (5.center);
		\draw [bend left=90, looseness=1.00] (13.center) to (14.center);
		\draw (6.center) to (12);
		\draw [bend left, looseness=1.00] (12) to (4);
	\end{pgfonlayer}
\end{tikzpicture} 
\]

An adjunction of linear functors,  $(\eta, \epsilon): F \dashv G$ is an adjunction in the usual sense 
(i.e. satisfying the triangle equalities) in the 2-category of LDCs with linear functors and 
linear natural transformations.   In particular, such an adjunction yields a pair of 
adjunctions: $(\eta_\ox, \epsilon_\ox): F_\ox \dashv G_\ox$ which is a monoidal adjunction, 
and $(\epsilon_\oa, \eta_\oa): G_\oa \dashv F_\oa$ which is a comonoidal adjunction.  
By Kelly's results \cite{Kel97}, a functor with a right adjoint is comonoidal if and only if its 
right adjoint is monoidal. This leads to the observation that:

\begin{lemma}
\label{Lemma: strong adjunction}
If $(\eta, \epsilon): F \dashvv G$ is an adjunction of linear functors, then $F_\ox$ is iso-monoidal 
(or strong) with respect to $\ox$ and $F_\oa$ is iso-comonoidal making the 
linear functor $F$ strong.
\end{lemma}
\begin{proof}
Since $(\eta_\ox, \epsilon_\ox): F_\ox \dashv G_\ox$ is a monoidal adjunction, 
the left adjoint $(F_\ox, m_\ox, m_\top)$ is a strong monoidal functor. Similarly, 
since $(\epsilon_\oa, \eta_\oa): G_\oa \dashv F_\oa$ is a comonoidal adjunction, the 
right adjoint $(F_\oa, n_\oa, n_\bot)$ is a strong comonoidal functor.
\end{proof}

A {\bf linear equivalence} is a linear adjunction in which the unit and counit 
are linear natural isomorphisms.

\subsection{Linear functors for isomix categories}
\label{Sec: mix-functor}

Any isomix category, $(\X,\ox, \oa)$ always has two linear functors ${\sf Mx}_{\downarrow}: 
(\X,\ox, \ox) \to (\X,\ox,\oa)$  and ${\sf Mx}_\uparrow: (\X,\oa, \oa) \to (\X,\ox,\oa)$  
given by the  identity functor, that is $({\sf Mx}_\uparrow)_\ox = ({\sf Mx}_\uparrow)_\oa = 
{\sf Id} = ({\sf Mx}_{\downarrow})_\ox = ({\sf Mx}_{\downarrow})_\oa$.   
The linear strengths and monoidal maps are given by the inverse of the mix map and  the mixor.  
These mix functors take the degenerate linear structure on the tensor (respectively the par) 
and spread it out over both the tensor structures. 

\begin{lemma} \label{mix-functor}
For any isomix category $\X$ the functors ${\sf Mx}_\downarrow: (\X,\ox,\ox) \to (\X,\ox, \oa)$ 
and ${\sf Mx}_\uparrow: (\X,\oa,\oa) \to (\X,\ox, \oa)$ are linear functors.
\end{lemma}

\begin{proof}
We show that ${\sf Mx}_{{\downarrow}}: (\X,\ox, \ox) \to (\X,\ox,\oa)$ is a linear functor: 
the monoidal and comonoidal components of the functor are given by  
$(1, 1, 1)$ and $( 1, \mx, \m^{-1})$ respectively. The linear strengths are $\nu_\ox^L = \nu_\ox^R: A \ox B \to 
A \oa B := \mx$ and $\nu_\oa^L = \nu_\oa^R: A \oa B \to A \oa B := 1$.

First we show $( 1, \mx, \m^{-1}): (\X,\ox,\ox) \to (\X,\ox, \oa)$ is a monoidal functor:

\begin{itemize}
\item The associative law for monoidal functors, $(\mx \ox 1)~\mx~a_\oa = a_\ox~(1 \ox \mx)~\mx$, 
is satisfied:
\[
\begin{tikzpicture} %asso1
\begin{pgfonlayer}{nodelayer}
\node [style=circle, scale=0.5] (0) at (-0.5, 2) {};
\node [style=circle, scale=0.5] (1) at (0.5, 1.25) {};
\node [style=map] (2) at (0, 1.75) {};
\node [style=ox] (3) at (0, 2.5) {};
\node [style=oa] (4) at (0, 0.75) {};
\node [style=ox] (5) at (1, 3) {};
\node [style=oa] (6) at (0, -1) {};
\node [style=oa] (7) at (1, -1.75) {};
\node [style=none] (8) at (1, -3) {};
\node [style=none] (9) at (1, 3.75) {};
\node [style=none] (10) at (-0.75, -3) {};
\node [style=circle, scale=0.5] (11) at (1.5, -1.25) {};
\node [style=map] (12) at (0.75, -0.5) {};
\node [style=circle, scale=0.5] (13) at (0, -0) {};
\end{pgfonlayer}
\begin{pgfonlayer}{edgelayer}
\draw [dotted, in=90, out=-15, looseness=1.25] (0) to (2);
\draw [dotted, in=165, out=-90, looseness=1.25] (2) to (1);
\draw (9.center) to (5);
\draw [bend left=45, looseness=1.00] (5) to (7);
\draw [bend right=15, looseness=1.00] (5) to (3);
\draw [bend right=15, looseness=1.00] (6) to (7);
\draw (7) to (8.center);
\draw [in=90, out=-126, looseness=1.00] (6) to (10.center);
\draw [bend right=60, looseness=1.25] (3) to (4);
\draw (4) to (6);
\draw [bend left=60, looseness=1.25] (3) to (4);
\draw [dotted, in=90, out=0, looseness=1.50] (13) to (12);
\draw [dotted, in=165, out=-90, looseness=1.25] (12) to (11);
\end{pgfonlayer}
\end{tikzpicture} = \begin{tikzpicture} %asso2
\begin{pgfonlayer}{nodelayer}
\node [style=circle, scale=0.5] (0) at (0.25, 1) {};
\node [style=circle, scale=0.5] (1) at (1.75, -0.5) {};
\node [style=map] (2) at (1, 0.5) {};
\node [style=ox] (3) at (-0.25, 1.75) {};
\node [style=ox] (4) at (1, 3) {};
\node [style=oa] (5) at (1, -1.75) {};
\node [style=none] (6) at (1, -3) {};
\node [style=none] (7) at (1, 3.75) {};
\node [style=none] (8) at (-1, -3) {};
\node [style=circle, scale=0.5] (9) at (0.5, -0.75) {};
\node [style=map] (10) at (-0.25, -0) {};
\node [style=circle, scale=0.5] (11) at (-0.75, 1) {};
\end{pgfonlayer}
\begin{pgfonlayer}{edgelayer}
\draw [dotted, in=90, out=-15, looseness=1.25] (0) to (2);
\draw [dotted, in=165, out=-90, looseness=1.25] (2) to (1);
\draw (7.center) to (4);
\draw [bend left=45, looseness=1.00] (4) to (5);
\draw [bend right=15, looseness=1.00] (4) to (3);
\draw (5) to (6.center);
\draw [dotted, in=90, out=0, looseness=1.50] (11) to (10);
\draw [dotted, in=165, out=-90, looseness=1.25] (10) to (9);
\draw [in=90, out=-165, looseness=0.50] (3) to (8.center);
\draw [in=135, out=-45, looseness=1.00] (3) to (5);
\end{pgfonlayer}
\end{tikzpicture} 
  = \begin{tikzpicture} %asso3
\begin{pgfonlayer}{nodelayer}
\node [style=circle, scale=0.5] (0) at (0.25, 2.5) {};
\node [style=circle, scale=0.5] (1) at (1.5, -0.5) {};
\node [style=map] (2) at (1, 0.5) {};
\node [style=ox] (3) at (-0.25, 1.75) {};
\node [style=ox] (4) at (1, 3) {};
\node [style=oa] (5) at (1, -1.25) {};
\node [style=none] (6) at (1, -3) {};
\node [style=none] (7) at (1, 3.75) {};
\node [style=none] (8) at (-1, -3) {};
\node [style=circle, scale=0.5] (9) at (1, -2.25) {};
\node [style=map] (10) at (-0.25, -0) {};
\node [style=circle, scale=0.5] (11) at (-0.75, 1) {};
\end{pgfonlayer}
\begin{pgfonlayer}{edgelayer}
\draw [dotted, in=90, out=-15, looseness=1.25] (0) to (2);
\draw [dotted, in=165, out=-90, looseness=1.25] (2) to (1);
\draw (7.center) to (4);
\draw [bend left=45, looseness=1.00] (4) to (5);
\draw [bend right=15, looseness=1.00] (4) to (3);
\draw (5) to (6.center);
\draw [dotted, in=90, out=0, looseness=1.50] (11) to (10);
\draw [dotted, in=165, out=-90, looseness=1.25] (10) to (9);
\draw [in=90, out=-165, looseness=0.50] (3) to (8.center);
\draw [in=135, out=-45, looseness=1.00] (3) to (5);
\end{pgfonlayer}
\end{tikzpicture}
\]

\item The unit laws for monoidal functors hold.   Here is the pictorial proof of $(1 \ox \m^{-1}) \mx = u_\ox^L(u_\oa^L)^{-1}$, where
the filled rectangles represent $\m^{-1}$:
\[
\begin{tikzpicture}
\begin{pgfonlayer}{nodelayer}
\node [style=map, fill=black] (0) at (0, 2) {};
\node [style=circle, scale=0.5] (1) at (0, 1.5) {};
\node [style=circle, scale=0.5] (2) at (-1.75, -0) {};
\node [style=map] (3) at (-1, 0.75) {};
\node [style=none] (4) at (0, 3.5) {};
\node [style=none] (5) at (0, -1) {};
\node [style=none] (6) at (-1.75, -1) {};
\node [style=none] (7) at (-1.75, 3.5) {};
\end{pgfonlayer}
\begin{pgfonlayer}{edgelayer}
\draw [dotted, in=90, out=-165, looseness=1.00] (1) to (3);
\draw [dotted, in=30, out=-90, looseness=1.25] (3) to (2);
\draw (4.center) to (0);
\draw (7.center) to (6.center);
\draw (0) to (5.center);
\end{pgfonlayer}
\end{tikzpicture} = \begin{tikzpicture}
\begin{pgfonlayer}{nodelayer}
\node [style=map, fill=black] (0) at (0, 2.25) {};
\node [style=circle, scale=0.5] (1) at (0, 1.5) {};
\node [style=circle, scale=0.5] (2) at (-1.75, -0.5) {};
\node [style=map] (3) at (-1, 0.25) {};
\node [style=none] (4) at (0, 3.75) {};
\node [style=none] (5) at (0.5, -1) {};
\node [style=none] (6) at (-1.75, -1) {};
\node [style=none] (7) at (-1.75, 3.75) {};
\node [style=circle] (8) at (0, 3) {$\top$};
\node [style=circle] (9) at (0, 0.25) {$\bot$};
\node [style=circle] (10) at (0.5, -0.5) {$\bot$};
\node [style=circle, scale=0.5] (11) at (0, 1) {};
\end{pgfonlayer}
\begin{pgfonlayer}{edgelayer}
\draw [dotted, dotted, in=90, out=-165, looseness=1.00] (1) to (3);
\draw [dotted, dotted, in=30, out=-90, looseness=1.25] (3) to (2);
\draw (7.center) to (6.center);
\draw (8) to (0);
\draw (0) to (1);
\draw (1) to (9);
\draw (10) to (5.center);
\draw [bend left, looseness=1.00, dotted] (11) to (10);
\draw (4.center) to (8);
\end{pgfonlayer}
\end{tikzpicture} 
  = \begin{tikzpicture}
\begin{pgfonlayer}{nodelayer}
\node [style=map, fill=black] (0) at (0, 2.25) {};
\node [style=circle, scale=0.5] (1) at (0, 1.5) {};
\node [style=circle, scale=0.5] (2) at (-1.75, 0.5) {};
\node [style=map] (3) at (-1, 1) {};
\node [style=none] (4) at (0, 3.75) {};
\node [style=none] (5) at (0, -1) {};
\node [style=none] (6) at (-1.75, -1) {};
\node [style=none] (7) at (-1.75, 3.75) {};
\node [style=circle] (8) at (0, 3) {$\top$};
\node [style=circle] (9) at (0, 0.75) {$\bot$};
\node [style=circle] (10) at (0, -0.25) {$\bot$};
\node [style=circle, scale=0.5] (11) at (-1.75, -0) {};
\end{pgfonlayer}
\begin{pgfonlayer}{edgelayer}
\draw [dotted, dotted, in=90, out=-165, looseness=1.00] (1) to (3);
\draw [dotted, dotted, in=30, out=-90, looseness=1.25] (3) to (2);
\draw (7.center) to (6.center);
\draw (8) to (0);
\draw (0) to (1);
\draw (1) to (9);
\draw (10) to (5.center);
\draw [bend left, looseness=1.00, dotted] (11) to (10);
\draw (4.center) to (8);
\end{pgfonlayer}
\end{tikzpicture} = \begin{tikzpicture}
\begin{pgfonlayer}{nodelayer}
\node [style=circle, scale=0.5] (0) at (-1.75, 1.75) {};
\node [style=none] (1) at (0, 3.75) {};
\node [style=none] (2) at (0, -1) {};
\node [style=none] (3) at (-1.75, -1) {};
\node [style=none] (4) at (-1.75, 3.75) {};
\node [style=circle] (5) at (0, 3) {$\top$};
\node [style=circle] (6) at (0, -0.25) {$\bot$};
\node [style=circle, scale=0.5] (7) at (-1.75, 0.75) {};
\end{pgfonlayer}
\begin{pgfonlayer}{edgelayer}
\draw (4.center) to (3.center);
\draw (6) to (2.center);
\draw [dotted, bend left, looseness=1.00] (7) to (6);
\draw (1.center) to (5);
\draw [dotted, bend left=45, looseness=1.00] (5) to (0);
\end{pgfonlayer}
\end{tikzpicture}
\]
The other unit law holds similarly.
\end{itemize}

${\sf Mx}_\downarrow: (\X,\ox, \ox) \to (\X,\ox,\oa)$ satisfies all the coherence requirements of a linear functor:
{\bf [LF.1]}, {\bf [LF.2]}, and {\bf [LF.3]} hold because $({\sf Mx}_{\downarrow})_\ox$ and $({\sf Mx}_\downarrow)_\oa$ are monoidal and comonoidal respectively, {\bf [LF.4]}(a) becomes 
$\mx a_\oa^{-1} = \partial^L (\mx \oa 1)$ and holds because:
\[
\begin{tikzpicture} %dist1
\begin{pgfonlayer}{nodelayer}
\node [style=oa] (0) at (0, 2.25) {};
\node [style=oa] (1) at (0, 1) {};
\node [style=oa] (2) at (0, -1) {};
\node [style=oa] (3) at (-1, -2) {};
\node [style=none] (4) at (-1, -3) {};
\node [style=none] (5) at (0.5, -3) {};
\node [style=none] (6) at (-2, 3) {};
\node [style=none] (7) at (0, 3) {};
\node [style=map] (8) at (-1.25, 0.5) {};
\node [style=circle, scale=0.5] (9) at (0, -0.25) {};
\node [style=circle, scale=0.5] (10) at (-2, 1.5) {};
\end{pgfonlayer}
\begin{pgfonlayer}{edgelayer}
\draw (6.center) to (10);
\draw (7.center) to (0);
\draw [bend left=60, looseness=1.25] (0) to (1);
\draw [bend right=60, looseness=1.25] (0) to (1);
\draw (1) to (2);
\draw [in=90, out=-45, looseness=1.00] (2) to (5.center);
\draw [in=30, out=-150, looseness=1.50] (2) to (3);
\draw [in=-89, out=135, looseness=1.00] (3) to (10);
\draw (3) to (4.center);
\draw [in=90, out=-15, looseness=1.25,dotted] (10) to (8);
\draw [in=180, out=-90, looseness=1.25, dotted] (8) to (9);
\end{pgfonlayer}
\end{tikzpicture} 
  = \begin{tikzpicture}
\begin{pgfonlayer}{nodelayer}
\node [style=oa] (0) at (0, 2.25) {};
\node [style=oa] (1) at (0, -0) {};
\node [style=oa] (2) at (0, -1) {};
\node [style=oa] (3) at (-1, -2) {};
\node [style=none] (4) at (-1, -3) {};
\node [style=none] (5) at (0.5, -3) {};
\node [style=none] (6) at (-2, 3) {};
\node [style=none] (7) at (0, 3) {};
\node [style=map] (8) at (-1.25, 1.5) {};
\node [style=circle,scale=0.5] (9) at (-0.75, 0.75) {};
\node [style=circle, scale=0.5] (10) at (-2, 2.25) {};
\end{pgfonlayer}
\begin{pgfonlayer}{edgelayer}
\draw (6.center) to (10);
\draw (7.center) to (0);
\draw [bend left=60, looseness=1.25] (0) to (1);
\draw [bend right=60, looseness=1.25] (0) to (1);
\draw (1) to (2);
\draw [in=90, out=-45, looseness=1.00] (2) to (5.center);
\draw [in=30, out=-150, looseness=1.50] (2) to (3);
\draw [in=-89, out=135, looseness=1.00] (3) to (10);
\draw (3) to (4.center);
\draw [dotted, in=90, out=-15, looseness=1.25] (10) to (8);
\draw [dotted, in=180, out=-90, looseness=1.25] (8) to (9);
\end{pgfonlayer}
\end{tikzpicture} 
  = \begin{tikzpicture}
\begin{pgfonlayer}{nodelayer}
\node [style=oa] (0) at (0, 2.25) {};
\node [style=oa] (1) at (-1, -2) {};
\node [style=none] (2) at (-1, -3) {};
\node [style=none] (3) at (0.5, -3) {};
\node [style=none] (4) at (-2, 3) {};
\node [style=none] (5) at (0, 3) {};
\node [style=map] (6) at (-1.25, 1) {};
\node [style=circle, scale=0.5] (7) at (-0.75, -0.5) {};
\node [style=circle, scale=0.5] (8) at (-2, 2.25) {};
\end{pgfonlayer}
\begin{pgfonlayer}{edgelayer}
\draw (4.center) to (8);
\draw (5.center) to (0);
\draw [in=-89, out=135, looseness=1.00] (1) to (8);
\draw (1) to (2.center);
\draw [dotted, in=90, out=-15, looseness=1.25] (8) to (6);
\draw [dotted, in=150, out=-90, looseness=1.00] (6) to (7);
\draw [in=90, out=-45, looseness=0.50] (0) to (3.center);
\draw [in=60, out=-150, looseness=0.75] (0) to (1);
\end{pgfonlayer}
\end{tikzpicture}
\]
{\bf [LF.4]} (b) - (d) and {\bf [LF.5]} (a) - (d) are satisfied similarly. 

Thus, ${\sf Mx}_{\downarrow}$ is a linear functor. 

The proof that ${\sf Mx}_\uparrow$ is a linear functor is (linearly) dual
\end{proof}

\begin{corollary} \label{compact-mix-functor}
	When $\X$ is a compact LDC,  the mix functors, ${\sf Mx}_{\downarrow}$ and ${\sf Mx}_\uparrow$, are linear isomorphisms. 
	Consequently, compact LDCs are linearly equivalent to monoidal categories.
\end{corollary}
	
We shall denote the inverse of ${\sf Mx}_{\downarrow}$ by 
${\sf Mx}^{*}_\downarrow: (\X,\ox,\oa) \to (\X,\oa,\oa)$: this is the identity functor as a mere functor, 
strict on the par structure, and on the tensor structure having as the unit laxor ${\sf m}$ and as the tensor laxor ${\sf mx}^{-1}$.   
Similarly, we shall denote the inverse of ${\sf Mx}_\uparrow$ by ${\sf Mx}^{*}_\uparrow$. 

The linear functors $\Mx_{\downarrow}$ and $\Mx_\uparrow$ are examples of {\em  isomix Frobenius functors\/}, 
which we shall introduce formally in the next section.

%%%%%%%%%%%%%%%%%%%%%%%%%%%%%%%%%%%%%%%%%%%%%%%%%%

\subsection{Frobenius functors}
\label{Sec: frobenius functors}

In this thesis, we will be interested in linear functors between LDCs called the Frobenius 
functors which come in various flavours, including mix functors and isomix functors, 
as illustrated in Figure \ref{linear-functor-family}.  These functors are directly related 
to the Frobenius monoidal functors of \cite{DP08} and they are referred to as degenerate 
linear functors in \cite{BPS12}.  Furthermore, we have already seen two rather 
basic examples, namely, ${\sf Mx}_\uparrow$ and ${\sf Mx}_{\downarrow}$.

\begin{figure}[ht]
\begin{center}
\begin{tikzpicture} [scale=1.5]
	\begin{pgfonlayer}{nodelayer}
		\node [style=none] (0) at (-3, 2) {};
		\node [style=none] (1) at (-3, -0) {};
		\node [style=none] (2) at (1, -0) {};
		\node [style=none] (3) at (1, 2) {};
		\node [style=none] (4) at (-4, 3) {};
		\node [style=none] (5) at (-4, -1) {};
		\node [style=none] (6) at (2, -1) {};
		\node [style=none] (7) at (2, 3) {};
		\node [style=none] (8) at (-5, -2) {};
		\node [style=none] (9) at (3, -2) {};
		\node [style=none] (10) at (3, 4) {};
		\node [style=none] (11) at (-5, 4) {};
		\node [style=none] (12) at (-6, 5) {};
		\node [style=none] (13) at (4, 5) {};
		\node [style=none] (14) at (4, -3) {};
		\node [style=none] (15) at (-6, -3) {};
		\node [style=none] (16) at (-1, 4.5) {Linear functors};
		\node [style=none] (17) at (-1, 3.5) {Frobenius functors};
		\node [style=none] (18) at (-1, 2.5) {Mix functors};
		\node [style=none] (19) at (-1, 1.5) {Isomix functors};
		\node [style=none] (34) at (-1, 1) {(Normal functors)};
		\node [style=none] (20) at (-1, 0.5) {$m_\top F(\m^{-1})n_\bot = \m^{-1}$};
		\node [style=none] (21) at (-1, -0.5) {$n_\bot\m m_\top = F(\m)$};
		\node [style=none] (22) at (-1, -1.5) {$F_\ox = F_\oa; m_\ox = \nu_\oa^L = \nu_\oa^R; n_\oa = \nu_\ox^L = \nu_\ox^R$};
		\node [style=none] (23) at (-1, -2.75) {};
		\node [style=none] (24) at (0, 6) {};
		\node [style=none] (25) at (0, -4) {};
		\node [style=none] (26) at (5.75, 6) {};
		\node [style=none] (27) at (5.75, -4) {};
		\node [style=none] (28) at (-0.9999999, 7) {};
		\node [style=none] (29) at (-0.9999999, -5) {};
		\node [style=none] (30) at (6.75, -5) {};
		\node [style=none] (31) at (6.75, 7) {};
		\node [style=none] (32) at (3, 6.75) {cyclic functors};
		\node [style=none] (33) at (3, 5.5) {symmetric functors};
	\end{pgfonlayer}
	\begin{pgfonlayer}{edgelayer}
		\draw (0.center) to (1.center);
		\draw (1.center) to (2.center);
		\draw (2.center) to (3.center);
		\draw (3.center) to (0.center);
		\draw (4.center) to (5.center);
		\draw (5.center) to (6.center);
		\draw (6.center) to (7.center);
		\draw (7.center) to (4.center);
		\draw (11.center) to (10.center);
		\draw (10.center) to (9.center);
		\draw (9.center) to (8.center);
		\draw (8.center) to (11.center);
		\draw (12.center) to (13.center);
		\draw (13.center) to (14.center);
		\draw (12.center) to (15.center);
		\draw (15.center) to (14.center);
		\draw[dotted]  (28.center) to (29.center);
		\draw[dotted] (29.center) to (30.center);
		\draw[dotted]  (30.center) to (31.center);
		\draw[dotted]  (31.center) to (28.center);
		\draw[dotted]  (24.center) to (25.center);
		\draw[dotted]  (25.center) to (27.center);
		\draw[dotted]  (27.center) to (26.center);
		\draw[dotted] (26.center) to (24.center);
	\end{pgfonlayer}
\end{tikzpicture}
\end{center}
\caption{Linear functor family}
\label{linear-functor-family}
\end{figure}

Frobenius functors preserve duals and with an additional coherence 
condition they preserve the mix map.  The coherence requirements for a dagger 
on an LDC are implied by requiring that the dagger functor be a Frobenius 
involutive equivalence.  Once the dagger  is understood we can consider $\dagger$-mix 
categories and their functors which we shall take to be mix Frobenius functors 
with a further requirement concerning the preservation of the dagger.

\begin{definition}
Let $\X$ and $\Y$ to LDCs. A {\bf Frobenius functor} is a linear functor $F: \X \to \Y$ such that:
\begin{enumerate}[{\bf \small [FLF.1]}]
\item $F_\ox = F_\oa $
\item $m_\ox = \nu_\oa^R = \nu_\oa^L $ 
\item $n_\oa = \nu_\ox^L = \nu_\ox^R$
\end{enumerate}
\end{definition}

The left and right linear strengths of $\ox$ and $\oa$ coinciding with the 
$m_\ox$ and $n_\oa$ respectively means that in the diagrammatic 
calculus, ports can be moved around freely:
\[
\begin{tikzpicture}
	\begin{pgfonlayer}{nodelayer}
		\node [style=none] (0) at (-2, 2) {};
		\node [style=none] (1) at (-1.5, 2) {};
		\node [style=none] (2) at (-2.25, 2) {};
		\node [style=none] (3) at (-2.25, 1) {};
		\node [style=none] (4) at (-0.75, 1) {};
		\node [style=none] (5) at (-0.75, 2) {};
		\node [style=none] (6) at (-2, 2.75) {};
		\node [style=none] (7) at (-1, 2.75) {};
		\node [style=none] (61) at (-2.75, 2.75) {$F_\oa(A)$};
		\node [style=none] (71) at (-0.25, 2.75) {$F_\ox(B)$};
		\node [style=none] (8) at (-1.5, 0.25) {};
		\node [style=none] (81) at (-2.25, 0) {$F_\oa(A \ox B) $};
		\node [style=ox] (9) at (-1.5, 1.5) {};
		\node [style=none] (10) at (-2, 1.25) {$F$};
	\end{pgfonlayer}
	\begin{pgfonlayer}{edgelayer}
		\draw [in=-90, out=-90, looseness=1.25] (0.center) to (1.center);
		\draw [bend right=15, looseness=1.00] (6.center) to (9);
		\draw [bend right=15, looseness=0.75] (9) to (7.center);
		\draw (2.center) to (3.center);
		\draw (3.center) to (4.center);
		\draw (4.center) to (5.center);
		\draw (5.center) to (2.center);
		\draw (9) to (8.center);
	\end{pgfonlayer}
\end{tikzpicture} = \begin{tikzpicture}
	\begin{pgfonlayer}{nodelayer}
		\node [style=none] (0) at (-1.5, 2) {};
		\node [style=none] (1) at (-1, 2) {};
		\node [style=none] (2) at (-2.25, 2) {};
		\node [style=none] (3) at (-2.25, 1) {};
		\node [style=none] (4) at (-0.75, 1) {};
		\node [style=none] (5) at (-0.75, 2) {};
		\node [style=none] (6) at (-2, 2.75) {};
		\node [style=none] (7) at (-1, 2.75) {};
		\node [style=none] (61) at (-2.75, 2.75) {$F_\ox(A)$};
		\node [style=none] (71) at (-0.25, 2.75) {$F_\oa(B)$};
		\node [style=none] (8) at (-1.5, 0.25) {};
		\node [style=none] (81) at (-2.25, 0) {$F_\oa(A \ox B) $};
		\node [style=ox] (9) at (-1.5, 1.5) {};
		\node [style=none] (10) at (-2, 1.25) {$F$};
	\end{pgfonlayer}
	\begin{pgfonlayer}{edgelayer}
		\draw [in=-90, out=-90, looseness=1.25] (0.center) to (1.center);
		\draw [bend right=15, looseness=1.00] (6.center) to (9);
		\draw [bend right=15, looseness=0.75] (9) to (7.center);
		\draw (2.center) to (3.center);
		\draw (3.center) to (4.center);
		\draw (4.center) to (5.center);
		\draw (5.center) to (2.center);
		\draw (9) to (8.center);
	\end{pgfonlayer}
\end{tikzpicture} = \begin{tikzpicture}
	\begin{pgfonlayer}{nodelayer}
		\node [style=none] (0) at (-2.25, 2) {};
		\node [style=none] (1) at (-2.25, 1) {};
		\node [style=none] (2) at (-0.75, 1) {};
		\node [style=none] (3) at (-0.75, 2) {};
		\node [style=none] (4) at (-2, 2.75) {};
		\node [style=none] (5) at (-1, 2.75) {};
		\node [style=none] (41) at (-2.75, 2.75) {$F_\ox(A)$};
		\node [style=none] (51) at (-0.25, 2.75) {$F_\ox(B)$};
		\node [style=none] (6) at (-1.5, 0.25) {};
		\node [style=none] (61) at (-2, 0) {$F_\ox(A \ox B)$};
		\node [style=ox] (7) at (-1.5, 1.5) {};
		\node [style=none] (8) at (-2, 1.25) {$F$};
		\node [style=none] (9) at (-1.75, 1) {};
		\node [style=none] (10) at (-1.25, 1) {};
	\end{pgfonlayer}
	\begin{pgfonlayer}{edgelayer}
		\draw [bend right=15, looseness=1.00] (4.center) to (7);
		\draw [bend right=15, looseness=0.75] (7) to (5.center);
		\draw (0.center) to (1.center);
		\draw (1.center) to (2.center);
		\draw (2.center) to (3.center);
		\draw (3.center) to (0.center);
		\draw (7) to (6.center);
		\draw [in=90, out=90, looseness=1.25] (9.center) to (10.center);
	\end{pgfonlayer}
\end{tikzpicture}  ~~~~~~~~ \begin{tikzpicture}
	\begin{pgfonlayer}{nodelayer}
		\node [style=none] (0) at (-2.25, 1) {};
		\node [style=none] (1) at (-2.25, 2) {};
		\node [style=none] (2) at (-0.75, 2) {};
		\node [style=none] (3) at (-0.75, 1) {};
		\node [style=none] (4) at (-2, 0.25) {};
		\node [style=none] (5) at (-1, 0.25) {};
		\node [style=none] (6) at (-1.5, 2.75) {};
		\node [style=none] (41) at (-2.75, 0.25) {$F_\ox(A)$};
		\node [style=none] (51) at (-0.25, 0.25) {$F_\oa(B)$};
		\node [style=none] (61) at (-2, 3) {$F_\ox(A \oa B)$};
		\node [style=oa] (7) at (-1.5, 1.5) {};
		\node [style=none] (8) at (-2, 1.75) {$F$};
		\node [style=none] (9) at (-2, 1) {};
		\node [style=none] (10) at (-1.5, 1) {};
	\end{pgfonlayer}
	\begin{pgfonlayer}{edgelayer}
		\draw [bend left=15, looseness=1.00] (4.center) to (7);
		\draw [bend left=15, looseness=0.75] (7) to (5.center);
		\draw (0.center) to (1.center);
		\draw (1.center) to (2.center);
		\draw (2.center) to (3.center);
		\draw (3.center) to (0.center);
		\draw (7) to (6.center);
		\draw [in=90, out=90, looseness=1.25] (9.center) to (10.center);
	\end{pgfonlayer}
\end{tikzpicture} = \begin{tikzpicture}
	\begin{pgfonlayer}{nodelayer}
		\node [style=none] (0) at (-2.25, 1) {};
		\node [style=none] (1) at (-2.25, 2) {};
		\node [style=none] (2) at (-0.75, 2) {};
		\node [style=none] (3) at (-0.75, 1) {};
		\node [style=none] (4) at (-2, 0.25) {};
		\node [style=none] (5) at (-1, 0.25) {};
		\node [style=none] (41) at (-2.75, 0.25) {$F_\oa(A)$};
		\node [style=none] (51) at (-0.25, 0.25) {$F_\ox(B)$};
		\node [style=none] (61) at (-2, 3) {$F_\ox(A \oa B)$};
		\node [style=none] (6) at (-1.5, 2.75) {};
		\node [style=oa] (7) at (-1.5, 1.5) {};
		\node [style=none] (8) at (-2, 1.75) {$F$};
		\node [style=none] (9) at (-1.5, 1) {};
		\node [style=none] (10) at (-1, 1) {};
	\end{pgfonlayer}
	\begin{pgfonlayer}{edgelayer}
		\draw [bend left=15, looseness=1.00] (4.center) to (7);
		\draw [bend left=15, looseness=0.75] (7) to (5.center);
		\draw (0.center) to (1.center);
		\draw (1.center) to (2.center);
		\draw (2.center) to (3.center);
		\draw (3.center) to (0.center);
		\draw (7) to (6.center);
		\draw [in=90, out=90, looseness=1.25] (9.center) to (10.center);
	\end{pgfonlayer}
\end{tikzpicture} = \begin{tikzpicture}
	\begin{pgfonlayer}{nodelayer}
		\node [style=none] (0) at (-2.25, 1) {};
		\node [style=none] (1) at (-2.25, 2) {};
		\node [style=none] (2) at (-0.75, 2) {};
		\node [style=none] (3) at (-0.75, 1) {};
		\node [style=none] (4) at (-2, 0.25) {};
		\node [style=none] (5) at (-1, 0.25) {};
		\node [style=none] (41) at (-2.75, 0.25) {$F_\oa(A)$};
		\node [style=none] (51) at (-0.25, 0.25) {$F_\oa(B)$};
		\node [style=none] (61) at (-2, 3) {$F_\oa(A \oa B)$};
		\node [style=none] (6) at (-1.5, 2.75) {};
		\node [style=oa] (7) at (-1.5, 1.5) {};
		\node [style=none] (8) at (-2, 1.75) {$F$};
		\node [style=none] (9) at (-1.25, 2) {};
		\node [style=none] (10) at (-1.75, 2) {};
	\end{pgfonlayer}
	\begin{pgfonlayer}{edgelayer}
		\draw [bend left=15, looseness=1.00] (4.center) to (7);
		\draw [bend left=15, looseness=0.75] (7) to (5.center);
		\draw (0.center) to (1.center);
		\draw (1.center) to (2.center);
		\draw (2.center) to (3.center);
		\draw (3.center) to (0.center);
		\draw (7) to (6.center);
		\draw [in=-90, out=-90, looseness=1.25] (9.center) to (10.center);
	\end{pgfonlayer}
\end{tikzpicture}
\]
\[ \nu_\oa^L = \nu_\oa^R = m_\ox  ~~~~~~~~~~~~~~~~~~~~~~~~~~~~~~~~~~~~  \nu_\ox^L = \nu_\ox^R = n_\oa \]
This implies that the ports can be omitted in the circuits.

A Frobenius functor is {\bf symmetric} if as a linear functor it preserves 
the symmetries of the tensor and par.  

\begin{lemma}
\label{Lemma: Frobenius}
Suppose $\X$ and $\Y$ are LDCs. The following are equivalent:
\begin{enumerate}[(a)]
\item $F: \X \to \Y$ is a Frobenius linear functor.
\item $F$ is $\ox$-monoidal and $\oa$-comonoidal such that 
\[ \xymatrixcolsep{2.5pc}
\xymatrix{
F(A) \ox F(B \oa C) \ar[r]^{1 \ox n_\oa} \ar[d]_{m_\ox} \ar@{}[dr]|{\tiny{\bf [F.1]}} & 
F(A) \ox (F(B) \oa F(C)) \ar[d]^{\delta^L} \\
F(A \ox (B \oa C)) \ar[d]_{F(\delta^L)} & (F(A) \ox F(B)) \oa F(C) \ar[d]^{m_\ox \oa 1} \\
F((A \oa B) \oa C) \ar[r]_{n_\oa} & F(A \oa B) \oa F(C)
}\] \[ \xymatrixcolsep{2.5pc} \xymatrix{
F( A \oa B) \ox F(C) \ar[r]^{n_\oa \ox 1}  \ar[d]_{m_\ox} \ar@{}[dr]|{\tiny{\bf [F.2]}} & 
(F(A) \oa F(B)) \ox F(C) \ar[d]^{\delta^R} \\
F( (A \oa B) \ox C) \ar[d]_{F(\delta^R)} & F(A) \oa (F(B) \ox F(C)) \ar[d]^{1 \oa m_\ox} \\
F(A \oa (B \ox C)) \ar[r]_{n_\oa} & F(A) \oa F(B \ox C) 
} \]
\end{enumerate}
\end{lemma}
\begin{proof}
For (a)$\Rightarrow$(b), fix $F := F_\ox = F_\oa$,  then $F$ is $\ox$-monoidal and $\oa$-comonoidal. 
Conditions {\bf \small [F.1]} and {\bf \small [F.2]} are given by {\bf  \small [LF.5]-(a)} and 
{\bf \small [LF.5]-(b)}. For the other direction, define $F_\ox = F_\oa := F$. Then it is 
straightforward to check that all the axioms of Frobenius linear functors are 
satisfied by $(F_\ox, F_\oa)$.
\end{proof}

Conditions {\bf \small  [F.1]} and {\bf \small [F.2]} in Lemma \ref{Lemma: Frobenius} 
are diagrammatically represented as follows:
\[
{\bf [F.1]}~~~~~~~~
\begin{tikzpicture}
	\begin{pgfonlayer}{nodelayer}
		\node [style=oa] (0) at (-3, 2) {};
		\node [style=ox] (1) at (-4, 1) {};
		\node [style=none] (2) at (-4.5, 3) {};
		\node [style=none] (3) at (-3, 3) {};
		\node [style=none] (4) at (-2.5, -0) {};
		\node [style=none] (5) at (-4, -0) {};
		\node [style=none] (6) at (-4.75, 2.5) {};
		\node [style=none] (7) at (-4.75, 0.5) {};
		\node [style=none] (8) at (-2.25, 0.5) {};
		\node [style=none] (9) at (-2.25, 2.5) {};
		\node [style=none] (10) at (-2.5, 2.25) {$F$};
	\end{pgfonlayer}
	\begin{pgfonlayer}{edgelayer}
		\draw [in=15, out=-150, looseness=1.00] (0) to (1);
		\draw [in=-90, out=135, looseness=1.00] (1) to (2.center);
		\draw (0) to (3.center);
		\draw [in=90, out=-45, looseness=1.00] (0) to (4.center);
		\draw (1) to (5.center);
		\draw (6.center) to (9.center);
		\draw (9.center) to (8.center);
		\draw (8.center) to (7.center);
		\draw (7.center) to (6.center);
	\end{pgfonlayer}
\end{tikzpicture} = \begin{tikzpicture}
	\begin{pgfonlayer}{nodelayer}
		\node [style=oa] (0) at (-3, 2.5) {};
		\node [style=ox] (1) at (-4, 1) {};
		\node [style=none] (2) at (-4.5, 3.5) {};
		\node [style=none] (3) at (-3, 3.5) {};
		\node [style=none] (4) at (-2.5, -0) {};
		\node [style=none] (5) at (-4, -0) {};
		\node [style=none] (6) at (-4.75, 1.5) {};
		\node [style=none] (7) at (-4.75, 0.5) {};
		\node [style=none] (8) at (-3.25, 0.5) {};
		\node [style=none] (9) at (-3.25, 1.5) {};
		\node [style=none] (10) at (-3.75, 2) {};
		\node [style=none] (11) at (-2.25, 2) {};
		\node [style=none] (12) at (-2.25, 3) {};
		\node [style=none] (13) at (-3.75, 3) {};
		\node [style=none] (14) at (-2.5, 2.75) {$F$};
		\node [style=none] (15) at (-3.5, 1.25) {$F$};
	\end{pgfonlayer}
	\begin{pgfonlayer}{edgelayer}
		\draw [in=60, out=-150, looseness=1.00] (0) to (1);
		\draw [in=-90, out=135, looseness=1.00] (1) to (2.center);
		\draw (0) to (3.center);
		\draw [in=90, out=-45, looseness=1.00] (0) to (4.center);
		\draw (1) to (5.center);
		\draw (6.center) to (9.center);
		\draw (9.center) to (8.center);
		\draw (8.center) to (7.center);
		\draw (7.center) to (6.center);
		\draw (13.center) to (10.center);
		\draw (10.center) to (11.center);
		\draw (11.center) to (12.center);
		\draw (12.center) to (13.center);
	\end{pgfonlayer}
\end{tikzpicture} 
~~~~~~~~~~~~~~~~~~ {\bf [F.2]}~~~~~~~~ \begin{tikzpicture}
	\begin{pgfonlayer}{nodelayer}
		\node [style=oa] (0) at (-3.25, 1) {};
		\node [style=ox] (1) at (-4, 1.75) {};
		\node [style=none] (2) at (-4, 3) {};
		\node [style=none] (3) at (-2.75, 3) {};
		\node [style=none] (4) at (-3.25, -0) {};
		\node [style=none] (5) at (-4.5, -0) {};
		\node [style=none] (6) at (-4.75, 2.5) {};
		\node [style=none] (7) at (-4.75, 0.5) {};
		\node [style=none] (8) at (-2.25, 0.5) {};
		\node [style=none] (9) at (-2.25, 2.5) {};
		\node [style=none] (10) at (-2.5, 2.25) {$F$};
	\end{pgfonlayer}
	\begin{pgfonlayer}{edgelayer}
		\draw [in=-45, out=165, looseness=1.25] (0) to (1);
		\draw [in=-90, out=90, looseness=1.00] (1) to (2.center);
		\draw [in=-90, out=45, looseness=1.00] (0) to (3.center);
		\draw [in=90, out=-90, looseness=1.00] (0) to (4.center);
		\draw [in=90, out=-150, looseness=0.75] (1) to (5.center);
		\draw (6.center) to (9.center);
		\draw (9.center) to (8.center);
		\draw (8.center) to (7.center);
		\draw (7.center) to (6.center);
	\end{pgfonlayer}
\end{tikzpicture} = \begin{tikzpicture}
	\begin{pgfonlayer}{nodelayer}
		\node [style=oa] (0) at (-3.25, 0.5) {};
		\node [style=ox] (1) at (-4, 1.75) {};
		\node [style=none] (2) at (-4, 3) {};
		\node [style=none] (3) at (-2.75, 3) {};
		\node [style=none] (4) at (-3.25, -0.5) {};
		\node [style=none] (5) at (-4.5, -0.5) {};
		\node [style=none] (6) at (-2.5, 1) {};
		\node [style=none] (7) at (-2.5, -0) {};
		\node [style=none] (8) at (-3.75, -0) {};
		\node [style=none] (9) at (-3.75, 1) {};
		\node [style=none] (10) at (-2.5, 2.25) {};
		\node [style=none] (11) at (-3.25, 2.25) {};
		\node [style=none] (12) at (-3.25, 1.25) {};
		\node [style=none] (13) at (-4.5, 2.25) {};
		\node [style=none] (14) at (-4.5, 1.25) {};
		\node [style=none] (15) at (-3.4, 1.5) {$F$};
		\node [style=none] (16) at (-2.75, 0.25) {$F$};
	\end{pgfonlayer}
	\begin{pgfonlayer}{edgelayer}
		\draw [in=-45, out=135, looseness=1.25] (0) to (1);
		\draw [in=-90, out=90, looseness=1.00] (1) to (2.center);
		\draw [in=-90, out=45, looseness=1.00] (0) to (3.center);
		\draw [in=90, out=-90, looseness=1.00] (0) to (4.center);
		\draw [in=90, out=-150, looseness=0.75] (1) to (5.center);
		\draw (6.center) to (9.center);
		\draw (9.center) to (8.center);
		\draw (8.center) to (7.center);
		\draw (7.center) to (6.center);
		\draw (11.center) to (13.center);
		\draw (13.center) to (14.center);
		\draw (14.center) to (12.center);
		\draw (12.center) to (11.center);
	\end{pgfonlayer}
\end{tikzpicture}
\]

Frobenius functors compose: the composition is defined as the usual composition 
of linear functors \cite{CS97}. 

It is immediate from Lemma \ref{Lemma: linear adjoints} that Frobenius functors preserve linear duals.
 In fact if $F: \X \to \Y$ is a Frobenius functor and $A \dashvv B$ is a linear dual, 
 as the duals $F_\ox(A) \dashvv F_\oa(B)$ and $F_\oa(A) \dashvv F_\ox(B)$ now coincide,  
 we just obtain the one dual $F(A) \dashvv F(B)$.   In the case when the Frobenius functor 
 is between cyclic $*$-autonomous categories we expect the 
functor to be  {\bf cyclor-preserving} in the following sense:
\[ \mbox{\bf [CFF]} ~~~~~\xymatrix{ F(X^{*}) \ar[d]_{\cong} \ar[rr]^{F(\psi)} & & 
 F(\!\!~^{*}X) \ar[d]^{\cong} \\
        F(X)^{*} \ar[rr]_{\psi} & & \!\!~^{*}F(X) } \]
where the left and right vertical arrows are respectively the maps:
\[ (u^R_\ox)^{-1} (\eta* \ox 1) \delta^R (1 \oa (m^F_\ox F(\epsilon*) n^F_\bot) u^R_\oa 
       ~~~\mbox{and}~~~(u^R)^{-1} (1 \ox *\eta) \delta^L (m^F_\ox \oa 1)((F(*\epsilon)n_\bot^F) \oa 1) u^L_\oa \]
The cyclor preserving condition maybe pictorially represented as follows:
\[ %newcoh1
\begin{tikzpicture}
	\begin{pgfonlayer}{nodelayer}
		\node [style=none] (0) at (-1, 4) {};
		\node [style=none] (1) at (3, 4) {};
		\node [style=none] (2) at (-1, 2) {};
		\node [style=none] (3) at (3, 2) {};
		\node [style=none] (4) at (0, 3.25) {};
		\node [style=none] (5) at (2, 3.25) {};
		\node [style=circle, scale=2] (6) at (-2, 1) {};
		\node [style=none] (7) at (-2, -0.5) {};
		\node [style=none] (8) at (2, 6) {};
		\node [style=none] (9) at (0, 5) {};
		\node [style=none] (10) at (-2, 5) {};
		\node [style=none] (11) at (-2, 1) {$\psi$};
		\node [style=none] (12) at (2.7, 2.3) {$F$};
		\node [style=none] (13) at (2.7, 5.75) {$F(X^*)$};
		\node [style=none] (14) at (2.35, 3.5) {$X^*$};
		\node [style=none] (15) at (1, 2.25) {$\epsilon*$};
		\node [style=none] (16) at (-0.25, 3.5) {$X$};
		\node [style=none] (17) at (-1, 6.25) {$\eta*$};
		\node [style=none] (18) at (-2.7, 2) {$F(X)^*$};
		\node [style=none] (19) at (-2.7, -0.3) {$~^*F(X)$};
	\end{pgfonlayer}
	\begin{pgfonlayer}{edgelayer}
		\draw (0.center) to (2.center);
		\draw (2.center) to (3.center);
		\draw (3.center) to (1.center);
		\draw (1.center) to (0.center);
		\draw (8.center) to (5.center);
		\draw [bend left=90, looseness=1.25] (5.center) to (4.center);
		\draw (4.center) to (9.center);
		\draw [bend left=90, looseness=1.50] (10.center) to (9.center);
		\draw (10.center) to (6);
		\draw (6) to (7.center);
	\end{pgfonlayer}
\end{tikzpicture} =  %newcoh2
\begin{tikzpicture}
	\begin{pgfonlayer}{nodelayer}
		\node [style=none] (0) at (1.5, 5) {};
		\node [style=none] (1) at (-2.5, 5) {};
		\node [style=none] (2) at (1.5, 2) {};
		\node [style=none] (3) at (-2.5, 2) {};
		\node [style=none] (4) at (0.5, 3.25) {};
		\node [style=none] (5) at (-1.5, 3.25) {};
		\node [style=none] (6) at (-1.5, 6) {};
		\node [style=none] (7) at (0.5, 5) {};
		\node [style=none] (8) at (2.5, 5) {};
		\node [style=none] (9) at (1.2, 2.3) {$F$};
		\node [style=none] (10) at (-2.2, 5.75) {$F(X^*)$};
		\node [style=none] (11) at (-1.85, 3.5) {$~^*X$};
		\node [style=none] (12) at (-0.5, 2.25) {$*\epsilon$};
		\node [style=none] (13) at (0.85, 3.5) {$X$};
		\node [style=none] (14) at (1.5, 6.15) {$*\eta$};
		\node [style=none] (15) at (3.2, -0.25) {$~^*F(X)$};
		\node [style=circle, scale=2] (16) at (-1.5, 4.25) {};
		\node [style=none] (17) at (-1.5, 4.25) {$\psi$};
		\node [style=none] (18) at (2.5, -0.5) {};
	\end{pgfonlayer}
	\begin{pgfonlayer}{edgelayer}
		\draw (0.center) to (2.center);
		\draw (2.center) to (3.center);
		\draw (3.center) to (1.center);
		\draw (1.center) to (0.center);
		\draw [bend right=90, looseness=1.25] (5.center) to (4.center);
		\draw (4.center) to (7.center);
		\draw [bend right=90, looseness=1.50] (8.center) to (7.center);
		\draw (6.center) to (16);
		\draw (16) to (5.center);
		\draw (8.center) to (18.center);
	\end{pgfonlayer}
\end{tikzpicture}
\]

\begin{lemma}
\label{Lemma: cyclic Frob}
Suppose $F$ is a cyclor preserving Frobenius linear functor, then
\[\begin{tikzpicture}
	\begin{pgfonlayer}{nodelayer}
		\node [style=circle, scale=2.5] (0) at (2.5, -4.75) {};
		\node [style=circle, scale=2.5] (1) at (4, -3.25) {};
		\node [style=none] (2) at (2.5, -6) {};
		\node [style=none] (3) at (4, -4.25) {};
		\node [style=none] (4) at (5.75, -4.25) {};
		\node [style=none] (5) at (5.75, -1) {};
		\node [style=none] (6) at (2.5, -2) {};
		\node [style=none] (7) at (4, -2) {};
		\node [style=none] (8) at (3.25, -2) {};
		\node [style=none] (9) at (3.25, -5.5) {};
		\node [style=none] (10) at (6.5, -5.5) {};
		\node [style=none] (11) at (6.5, -2) {};
		\node [style=none] (12) at (6.25, -5) {$F$};
		\node [style=none] (13) at (6.25, -1.75) {$F(X)$};
		\node [style=none] (14) at (5, -5) {$*\epsilon$};
		\node [style=none] (15) at (4, -3.25) {$\psi$};
		\node [style=none] (16) at (2.5, -4.75) {$\psi$};
		\node [style=none] (17) at (3.25, -1) {$\eta*$};
		\node [style=none] (18) at (6, -2.75) {$X$};
		\node [style=none] (19) at (4.5, -4) {$~^*X$};
		\node [style=none] (20) at (4.5, -2.5) {$X^*$};
		\node [style=none] (21) at (4.75, -1.75) {$F(X^*)$};
		\node [style=none] (22) at (1.75, -2.25) {$F(X^*)^*$};
		\node [style=none] (23) at (1.75, -5.75) {$~^*F(X^*)$};
	\end{pgfonlayer}
	\begin{pgfonlayer}{edgelayer}
		\draw (6.center) to (0);
		\draw (0) to (2.center);
		\draw [bend left=90, looseness=1.50] (6.center) to (7.center);
		\draw (7.center) to (1);
		\draw (1) to (3.center);
		\draw [bend right=90] (3.center) to (4.center);
		\draw (4.center) to (5.center);
		\draw (8.center) to (9.center);
		\draw (9.center) to (10.center);
		\draw (10.center) to (11.center);
		\draw (11.center) to (8.center);
	\end{pgfonlayer}
\end{tikzpicture}  = \begin{tikzpicture}
	\begin{pgfonlayer}{nodelayer}
		\node [style=none] (0) at (1.5, -3.75) {};
		\node [style=none] (1) at (0.5, -3.75) {};
		\node [style=none] (2) at (0.5, -1) {};
		\node [style=none] (3) at (0, -1.5) {$F(X)$};
		\node [style=none] (4) at (0.25, -3.25) {$X$};
		\node [style=none] (5) at (1.5, -2.25) {};
		\node [style=none] (6) at (3, -2.25) {};
		\node [style=none] (7) at (-0.25, -2.5) {};
		\node [style=none] (8) at (-0.25, -5) {};
		\node [style=none] (9) at (2.5, -5) {};
		\node [style=none] (10) at (2.5, -2.5) {};
		\node [style=none] (11) at (2.25, -4.75) {$F$};
		\node [style=none] (12) at (3, -6) {};
		\node [style=none] (13) at (3.75, -5.5) {$~^*F(X^*)$};
		\node [style=none] (14) at (3, -2.25) {};
		\node [style=none] (15) at (2.25, -1.25) {$*\eta$};
		\node [style=none] (16) at (1, -4.5) {$\epsilon*$};
	\end{pgfonlayer}
	\begin{pgfonlayer}{edgelayer}
		\draw [bend left=90, looseness=1.75] (0.center) to (1.center);
		\draw (1.center) to (2.center);
		\draw [bend right=90, looseness=1.50] (6.center) to (5.center);
		\draw (7.center) to (10.center);
		\draw (10.center) to (9.center);
		\draw (9.center) to (8.center);
		\draw (8.center) to (7.center);
		\draw (14.center) to (12.center);
		\draw (5.center) to (0.center);
	\end{pgfonlayer}
\end{tikzpicture} \]
\end{lemma}
\begin{proof}~
$\begin{tikzpicture}
	\begin{pgfonlayer}{nodelayer}
		\node [style=circle, scale=2.5] (0) at (2, -4.75) {};
		\node [style=circle, scale=2.5] (1) at (4, -3.25) {};
		\node [style=none] (2) at (2, -7) {};
		\node [style=none] (3) at (4, -4.25) {};
		\node [style=none] (4) at (5.999999, -4) {};
		\node [style=none] (5) at (5.999999, 0.9999999) {};
		\node [style=none] (6) at (2, -2) {};
		\node [style=none] (7) at (4, -2) {};
		\node [style=none] (8) at (3.25, -2) {};
		\node [style=none] (9) at (3.25, -5.5) {};
		\node [style=none] (10) at (6.75, -5.5) {};
		\node [style=none] (11) at (6.75, -2) {};
		\node [style=none] (12) at (6.499999, -5) {$F$};
		\node [style=none] (13) at (6.499999, -0.25) {$F(X)$};
		\node [style=none] (14) at (5, -5) {$*\epsilon$};
		\node [style=none] (15) at (4, -3.25) {$\psi$};
		\node [style=none] (16) at (2, -4.75) {$\psi$};
		\node [style=none] (17) at (3, -0.7499999) {$\eta*$};
		\node [style=none] (18) at (6.25, -2.75) {$X$};
		\node [style=none] (19) at (4.5, -4) {$~^*X$};
		\node [style=none] (20) at (4.5, -2.5) {$X^*$};
		\node [style=none] (21) at (4.25, -1.25) {$F(X^*)$};
		\node [style=none] (22) at (1.25, -2.25) {$F(X^*)^*$};
		\node [style=none] (23) at (1.25, -6.25) {$~^*F(X^*)$};
	\end{pgfonlayer}
	\begin{pgfonlayer}{edgelayer}
		\draw (6.center) to (0);
		\draw (0) to (2.center);
		\draw [bend left=90, looseness=1.50] (6.center) to (7.center);
		\draw (7.center) to (1);
		\draw (1) to (3.center);
		\draw [bend right=90, looseness=1.25] (3.center) to (4.center);
		\draw (4.center) to (5.center);
		\draw (8.center) to (9.center);
		\draw (9.center) to (10.center);
		\draw (10.center) to (11.center);
		\draw (11.center) to (8.center);
	\end{pgfonlayer}
\end{tikzpicture} = %cyclor2
\begin{tikzpicture}
	\begin{pgfonlayer}{nodelayer}
		\node [style=circle, scale=2.5] (0) at (2, -4.75) {};
		\node [style=circle, scale=2.5] (1) at (6.75, -4.25) {};
		\node [style=none] (2) at (2, -7.5) {};
		\node [style=none] (3) at (6.75, -5.25) {};
		\node [style=none] (4) at (8.75, -5) {};
		\node [style=none] (5) at (8.75, 1) {};
		\node [style=none] (6) at (2, -2) {};
		\node [style=none] (7) at (4, -2) {};
		\node [style=none] (8) at (3.25, -2) {};
		\node [style=none] (9) at (3.25, -6.5) {};
		\node [style=none] (10) at (9.500001, -6.5) {};
		\node [style=none] (11) at (9.500001, -2) {};
		\node [style=none] (12) at (9.25, -6) {$F$};
		\node [style=none] (13) at (9.25, -0.25) {$F(X)$};
		\node [style=none] (14) at (7.75, -6) {$*\epsilon$};
		\node [style=none] (15) at (6.75, -4.25) {$\psi$};
		\node [style=none] (16) at (2, -4.75) {$\psi$};
		\node [style=none] (17) at (3, -0.7499999) {$\eta*$};
		\node [style=none] (18) at (9.000001, -3.75) {$X$};
		\node [style=none] (19) at (7.25, -5) {$~^*X$};
		\node [style=none] (20) at (7.25, -3.5) {$X^*$};
		\node [style=none] (21) at (4.25, -1.25) {$F(X^*)$};
		\node [style=none] (22) at (1.25, -2.25) {$F(X^*)^*$};
		\node [style=none] (23) at (1.25, -6.25) {$~^*F(X^*)$};
		\node [style=none] (24) at (4, -5) {};
		\node [style=none] (25) at (5.499999, -5) {};
		\node [style=none] (26) at (5.499999, -3.5) {};
		\node [style=none] (27) at (6.75, -3.5) {};
		\node [style=none] (28) at (5.999999, -2.75) {$\eta*$};
		\node [style=none] (29) at (4.75, -6) {$\epsilon*$};
	\end{pgfonlayer}
	\begin{pgfonlayer}{edgelayer}
		\draw (6.center) to (0);
		\draw (0) to (2.center);
		\draw [bend left=90, looseness=1.50] (6.center) to (7.center);
		\draw (1) to (3.center);
		\draw [bend right=90, looseness=1.25] (3.center) to (4.center);
		\draw (4.center) to (5.center);
		\draw (8.center) to (9.center);
		\draw (9.center) to (10.center);
		\draw (10.center) to (11.center);
		\draw (11.center) to (8.center);
		\draw (7.center) to (24.center);
		\draw [bend right=90, looseness=1.75] (24.center) to (25.center);
		\draw (26.center) to (25.center);
		\draw [bend right=90, looseness=1.50] (27.center) to (26.center);
		\draw (27.center) to (1);
	\end{pgfonlayer}
\end{tikzpicture} = %cyclor3
\begin{tikzpicture}
	\begin{pgfonlayer}{nodelayer}
		\node [style=circle, scale=2.5] (0) at (2, -4.75) {};
		\node [style=circle, scale=2.5] (1) at (6.75, -1.25) {};
		\node [style=none] (2) at (2, -7.5) {};
		\node [style=none] (3) at (6.75, -2.25) {};
		\node [style=none] (4) at (8.75, -2) {};
		\node [style=none] (5) at (8.75, 1.75) {};
		\node [style=none] (6) at (2, -2) {};
		\node [style=none] (7) at (4, -2) {};
		\node [style=none] (8) at (3.25, -3.75) {};
		\node [style=none] (9) at (3.25, -6.5) {};
		\node [style=none] (10) at (5.999999, -6.5) {};
		\node [style=none] (11) at (5.75, -6) {$F$};
		\node [style=none] (12) at (9.25, 1.5) {$F(X)$};
		\node [style=none] (13) at (7.75, -3.25) {$*\epsilon$};
		\node [style=none] (14) at (6.75, -1.25) {$\psi$};
		\node [style=none] (15) at (2, -4.75) {$\psi$};
		\node [style=none] (16) at (3, -0.7499999) {$\eta*$};
		\node [style=none] (17) at (9.000001, -0.7499999) {$X$};
		\node [style=none] (18) at (7.25, -2) {$~^*X$};
		\node [style=none] (19) at (7.25, -0.4999999) {$X^*$};
		\node [style=none] (20) at (4.25, -1.25) {$F(X^*)$};
		\node [style=none] (21) at (1.25, -2.25) {$F(X^*)^*$};
		\node [style=none] (22) at (1.25, -6.25) {$~^*F(X^*)$};
		\node [style=none] (23) at (4, -5) {};
		\node [style=none] (24) at (5.499999, -5) {};
		\node [style=none] (25) at (5.499999, -0.7499999) {};
		\node [style=none] (26) at (6.75, -0.4999999) {};
		\node [style=none] (27) at (6.25, 0.25) {$\eta*$};
		\node [style=none] (28) at (4.75, -6) {$\epsilon*$};
		\node [style=none] (29) at (5.999999, -3.75) {};
		\node [style=none] (30) at (5.25, 0.7499999) {};
		\node [style=none] (31) at (5.25, -3.5) {};
		\node [style=none] (32) at (9.500001, -3.5) {};
		\node [style=none] (33) at (9.500001, 0.7499999) {};
		\node [style=none] (34) at (9.25, -3.25) {$F$};
	\end{pgfonlayer}
	\begin{pgfonlayer}{edgelayer}
		\draw (6.center) to (0);
		\draw (0) to (2.center);
		\draw [bend left=90, looseness=1.50] (6.center) to (7.center);
		\draw (1) to (3.center);
		\draw [bend right=90, looseness=1.25] (3.center) to (4.center);
		\draw (4.center) to (5.center);
		\draw (8.center) to (9.center);
		\draw (9.center) to (10.center);
		\draw (7.center) to (23.center);
		\draw [bend right=90, looseness=1.75] (23.center) to (24.center);
		\draw (25.center) to (24.center);
		\draw [bend right=90, looseness=1.50] (26.center) to (25.center);
		\draw (26.center) to (1);
		\draw (10.center) to (29.center);
		\draw (29.center) to (8.center);
		\draw (30.center) to (33.center);
		\draw (33.center) to (32.center);
		\draw (32.center) to (31.center);
		\draw (31.center) to (30.center);
	\end{pgfonlayer}
\end{tikzpicture} \stackrel{\tiny {\bf [CFF]}}{=} %cyclor4
\begin{tikzpicture}
	\begin{pgfonlayer}{nodelayer}
		\node [style=circle, scale=2.5] (0) at (6.75, -2.25) {};
		\node [style=none] (1) at (6.75, -2.75) {};
		\node [style=none] (2) at (8.75, -2.75) {};
		\node [style=none] (3) at (8.75, 0) {};
		\node [style=none] (4) at (9.25, -0.25) {$F(X)$};
		\node [style=none] (5) at (7.75, -3.75) {$*\epsilon$};
		\node [style=none] (6) at (6.75, -2.25) {$\psi$};
		\node [style=none] (7) at (9, -1.75) {$X$};
		\node [style=none] (8) at (7.25, -2.75) {$~^*X$};
		\node [style=none] (9) at (7.25, -1.5) {$X^*$};
		\node [style=none] (10) at (5.5, -1.75) {};
		\node [style=none] (11) at (6.75, -1.75) {};
		\node [style=none] (12) at (6, -1) {$\eta*$};
		\node [style=none] (13) at (5.25, -0.75) {};
		\node [style=none] (14) at (5.25, -4) {};
		\node [style=none] (15) at (9.5, -4) {};
		\node [style=none] (16) at (9.5, -0.75) {};
		\node [style=none] (17) at (9.25, -3.75) {$F$};
		\node [style=none] (18) at (4.5, -5.25) {};
		\node [style=none] (19) at (7, -5.5) {};
		\node [style=none] (20) at (6.25, -7.5) {$\epsilon*$};
		\node [style=none] (21) at (7.75, -8) {};
		\node [style=none] (22) at (8.75, -8.5) {};
		\node [style=none] (23) at (9.5, -7.75) {$~^*F(X^*)$};
		\node [style=none] (24) at (4.5, -8) {};
		\node [style=none] (25) at (8.75, -5.5) {};
		\node [style=none] (26) at (7, -6.75) {};
		\node [style=none] (27) at (7.75, -5.25) {};
		\node [style=none] (28) at (7.75, -4.5) {$*\eta$};
		\node [style=circle, scale=2.5] (29) at (5.5, -6.25) {};
		\node [style=none] (30) at (5.5, -6.25) {$\psi$};
		\node [style=none] (31) at (7.5, -7.75) {$F$};
		\node [style=none] (32) at (5.5, -6.75) {};
	\end{pgfonlayer}
	\begin{pgfonlayer}{edgelayer}
		\draw (0) to (1.center);
		\draw [bend right=90, looseness=1.25] (1.center) to (2.center);
		\draw (2.center) to (3.center);
		\draw [bend right=90, looseness=1.50] (11.center) to (10.center);
		\draw (11.center) to (0);
		\draw (13.center) to (16.center);
		\draw (16.center) to (15.center);
		\draw (15.center) to (14.center);
		\draw (14.center) to (13.center);
		\draw (25.center) to (22.center);
		\draw (18.center) to (27.center);
		\draw [bend right=90, looseness=1.50] (25.center) to (19.center);
		\draw (21.center) to (24.center);
		\draw (24.center) to (18.center);
		\draw (19.center) to (26.center);
		\draw (27.center) to (21.center);
		\draw (29) to (10.center);
		\draw [bend right=75, looseness=1.25] (32.center) to (26.center);
		\draw (29) to (32.center);
	\end{pgfonlayer}
\end{tikzpicture} =$ 
$\begin{tikzpicture}
	\begin{pgfonlayer}{nodelayer}
		\node [style=circle, scale=2.5] (33) at (0, -2.25) {};
		\node [style=none] (34) at (0, -2.75) {};
		\node [style=none] (35) at (2, -2.75) {};
		\node [style=none] (36) at (2, 0) {};
		\node [style=none] (37) at (2.5, -0.25) {$F(X)$};
		\node [style=none] (38) at (1, -3.75) {$*\epsilon$};
		\node [style=none] (39) at (0, -2.25) {$\psi$};
		\node [style=none] (40) at (2.25, -1.75) {$X$};
		\node [style=none] (41) at (0.500001, -2.75) {$~^*X$};
		\node [style=none] (42) at (0.500001, -1.5) {$X^*$};
		\node [style=none] (43) at (-1.25, -1.75) {};
		\node [style=none] (44) at (0, -1.75) {};
		\node [style=none] (45) at (-0.499999, -1) {$\eta*$};
		\node [style=none] (46) at (-2.25, -0.75) {};
		\node [style=none] (47) at (-2.25, -4.25) {};
		\node [style=none] (48) at (2.75, -4.25) {};
		\node [style=none] (49) at (2.75, -0.75) {};
		\node [style=none] (50) at (2.5, -4) {$F$};
		\node [style=none] (51) at (-2.25, -5.5) {};
		\node [style=none] (52) at (0.25, -5.75) {};
		\node [style=none] (53) at (-0.499999, -7) {$*\epsilon$};
		\node [style=none] (54) at (1, -7.5) {};
		\node [style=none] (55) at (2, -8.5) {};
		\node [style=none] (56) at (2.75, -6.75) {$~^*F(X^*)$};
		\node [style=none] (57) at (-2.25, -7.5) {};
		\node [style=none] (58) at (2, -5.75) {};
		\node [style=none] (59) at (0.25, -6.25) {};
		\node [style=none] (60) at (1, -5.5) {};
		\node [style=none] (61) at (1, -4.75) {$*\eta$};
		\node [style=circle, scale=2.5] (62) at (-1.25, -2.25) {};
		\node [style=none] (63) at (-1.25, -2.25) {$\psi$};
		\node [style=none] (64) at (0.75, -7.25) {$F$};
		\node [style=none] (65) at (-1.25, -6.25) {};
	\end{pgfonlayer}
	\begin{pgfonlayer}{edgelayer}
		\draw (33) to (34.center);
		\draw [bend right=90, looseness=1.25] (34.center) to (35.center);
		\draw (35.center) to (36.center);
		\draw [bend right=90, looseness=1.50] (44.center) to (43.center);
		\draw (44.center) to (33);
		\draw (46.center) to (49.center);
		\draw (49.center) to (48.center);
		\draw (48.center) to (47.center);
		\draw (47.center) to (46.center);
		\draw (58.center) to (55.center);
		\draw (51.center) to (60.center);
		\draw [bend right=90, looseness=1.50] (58.center) to (52.center);
		\draw (54.center) to (57.center);
		\draw (57.center) to (51.center);
		\draw (52.center) to (59.center);
		\draw (60.center) to (54.center);
		\draw (62) to (43.center);
		\draw [bend right=75, looseness=1.25] (65.center) to (59.center);
		\draw (62) to (65.center);
	\end{pgfonlayer}
\end{tikzpicture} \stackrel{F({\bf [C.2]})}{=}  %cyclor6
\begin{tikzpicture}
	\begin{pgfonlayer}{nodelayer}
		\node [style=none] (66) at (7.25, -3.25) {};
		\node [style=none] (67) at (5.75, -3.25) {};
		\node [style=none] (68) at (5.75, 0) {};
		\node [style=none] (69) at (5.25, -0.25) {$F(X)$};
		\node [style=none] (70) at (5.5, -2.25) {$X$};
		\node [style=none] (71) at (7.25, -2.5) {};
		\node [style=none] (72) at (8.75, -2.5) {};
		\node [style=none] (73) at (5, -1) {};
		\node [style=none] (74) at (5, -4.75) {};
		\node [style=none] (75) at (9.5, -4.75) {};
		\node [style=none] (76) at (9.5, -1) {};
		\node [style=none] (77) at (9.25, -4.5) {$F$};
		\node [style=none] (78) at (7.75, -5) {};
		\node [style=none] (79) at (10.25, -5.25) {};
		\node [style=none] (80) at (9.5, -6.5) {$*\epsilon$};
		\node [style=none] (81) at (11, -7) {};
		\node [style=none] (82) at (11.75, -8) {};
		\node [style=none] (83) at (12.5, -6.75) {$~^*F(X^*)$};
		\node [style=none] (84) at (7.75, -7) {};
		\node [style=none] (85) at (11.75, -5.25) {};
		\node [style=none] (86) at (10.25, -5.75) {};
		\node [style=none] (87) at (11, -5) {};
		\node [style=none] (88) at (11, -4.25) {$*\eta$};
		\node [style=none] (89) at (10.75, -6.75) {$F$};
		\node [style=none] (90) at (8.75, -5.75) {};
		\node [style=none] (91) at (6.5, -4.25) {$\epsilon*$};
		\node [style=none] (92) at (8, -1.5) {$*\eta$};
	\end{pgfonlayer}
	\begin{pgfonlayer}{edgelayer}
		\draw [bend left=90, looseness=1.75] (66.center) to (67.center);
		\draw (67.center) to (68.center);
		\draw [bend right=90, looseness=1.50] (72.center) to (71.center);
		\draw (73.center) to (76.center);
		\draw (76.center) to (75.center);
		\draw (75.center) to (74.center);
		\draw (74.center) to (73.center);
		\draw (85.center) to (82.center);
		\draw (78.center) to (87.center);
		\draw [bend right=90, looseness=1.50] (85.center) to (79.center);
		\draw (81.center) to (84.center);
		\draw (84.center) to (78.center);
		\draw (79.center) to (86.center);
		\draw (87.center) to (81.center);
		\draw [bend right=75, looseness=1.25] (90.center) to (86.center);
		\draw (72.center) to (90.center);
		\draw (71.center) to (66.center);
	\end{pgfonlayer}
\end{tikzpicture} = %cyclor7
\begin{tikzpicture}
	\begin{pgfonlayer}{nodelayer}
		\node [style=none] (0) at (7.25, -2) {};
		\node [style=none] (1) at (5.25, -1.75) {};
		\node [style=none] (2) at (5.25, 3.5) {};
		\node [style=none] (3) at (4.75, 3.25) {$F(X)$};
		\node [style=none] (4) at (5, -0.4999999) {$X$};
		\node [style=none] (5) at (7.25, 1.5) {};
		\node [style=none] (6) at (9.500001, 1.25) {};
		\node [style=none] (7) at (4.5, 0.7499999) {};
		\node [style=none] (8) at (4.5, -3.5) {};
		\node [style=none] (9) at (8.25, -3.5) {};
		\node [style=none] (10) at (8.25, 0.7499999) {};
		\node [style=none] (11) at (7.999999, -3.25) {$F$};
		\node [style=none] (12) at (9.500001, -4.75) {};
		\node [style=none] (13) at (10.5, -4.25) {$~^*F(X^*)$};
		\node [style=none] (14) at (9.500001, 1.25) {};
		\node [style=none] (15) at (8.25, 2.75) {$*\eta$};
		\node [style=none] (16) at (6.25, -3) {$\epsilon*$};
		\node [style=none] (17) at (4.25, -4) {};
	\end{pgfonlayer}
	\begin{pgfonlayer}{edgelayer}
		\draw [bend left=90, looseness=1.25] (0.center) to (1.center);
		\draw (1.center) to (2.center);
		\draw [bend right=90, looseness=1.50] (6.center) to (5.center);
		\draw (7.center) to (10.center);
		\draw (10.center) to (9.center);
		\draw (9.center) to (8.center);
		\draw (8.center) to (7.center);
		\draw (14.center) to (12.center);
		\draw (5.center) to (0.center);
	\end{pgfonlayer}
\end{tikzpicture}$
\end{proof}

\begin{definition}
Suppose $\X$ and $\Y$ are mix categories. $F: \X \to \Y$ is a {\bf mix functor} if it is a Frobenius functor such that 
\[
\mbox{\bf{[mix-FF]}}~~~~~
\xymatrix{
F(\bot) \ar@/_2pc/[rrr]_{F(\m)} \ar[r]^{n_\bot} & \bot \ar[r]^{\m} & \top \ar[r]^{m_\top} & F(\top) \\
}
\]
\end{definition}

The equation \mbox{\bf{[mix-FF]}} is diagrammatically given as follows:
\[\begin{tikzpicture}
	\begin{pgfonlayer}{nodelayer}
		\node [style=none] (0) at (-3, 2) {};
		\node [style=none] (1) at (-3, 1) {};
		\node [style=none] (2) at (-1, 2) {};
		\node [style=none] (3) at (-1, 1) {};
		\node [style=circle] (4) at (-2, 1.5) {$\bot$};
		\node [style=circle] (5) at (0, 1.3) {$\bot$};
		\node [style=none] (6) at (0, 0.5) {};
		\node [style=circle] (7) at (0, -0.3) {$\top$};
		\node [style=none] (8) at (-2, 3) {};
		\node [style=none] (9) at (0.75, -0.5) {};
		\node [style=none] (10) at (2.75, -0.5) {};
		\node [style=none] (11) at (0.75, -1.5) {};
		\node [style=none] (12) at (2.75, -1.5) {};
		\node [style=none] (13) at (1.75, -2.5) {};
		\node [style=circle] (14) at (1.75, -1) {$\top$};
		\node [style=map] (15) at (0, 0.5) {};
		\node [style=circle, scale=0.5] (16) at (1.75, -2) {};
		\node [style=circle, scale=0.5] (17) at (-2, 2.5) {};
	\end{pgfonlayer}
	\begin{pgfonlayer}{edgelayer}
		\draw (0.center) to (1.center);
		\draw (1.center) to (3.center);
		\draw (3.center) to (2.center);
		\draw (2.center) to (0.center);
		\draw (9.center) to (10.center);
		\draw (10.center) to (12.center);
		\draw (12.center) to (11.center);
		\draw (11.center) to (9.center);
		\draw (8.center) to (4);
		\draw (14) to (13.center);
		\draw [dotted, bend left=45, looseness=1.25] (17) to (5);
		\draw [dotted, in=-165, out=-90, looseness=1.25] (7) to (16);
		\draw (5) to (6.center);
		\draw (6.center) to (7);
	\end{pgfonlayer}
\end{tikzpicture}  = \begin{tikzpicture}
	\begin{pgfonlayer}{nodelayer}
		\node [style=circle] (0) at (0, 1.5) {$\bot$};
		\node [style=none] (1) at (0, 0.5) {};
		\node [style=circle] (2) at (0, -0.5) {$\top$};
		\node [style=map] (3) at (0, 0.5) {};
		\node [style=none] (4) at (0, -1.75) {};
		\node [style=none] (5) at (0, 2.5) {};
		\node [style=none] (6) at (-1.25, 2.2) {};
		\node [style=none] (7) at (-1.25, -1.2) {};
		\node [style=none] (8) at (1.25, 2.2) {};
		\node [style=none] (9) at (1.25, -1.2) {};
		\node [style=none] (10) at (-1, 1.8) {$F$};
	\end{pgfonlayer}
	\begin{pgfonlayer}{edgelayer}
		\draw (0) to (1.center);
		\draw (1.center) to (2);
		\draw (5.center) to (0);
		\draw (2) to (4.center);
		\draw (6.center) to (8.center);
		\draw (8.center) to (9.center);
		\draw (9.center) to (7.center);
		\draw (7.center) to (6.center);
	\end{pgfonlayer}
\end{tikzpicture} \]

\begin{lemma}
\label{Lemma: Mix Frobenius linear functor}
Mix functors preserve the mix map:
\[
\xymatrix{
F(A) \ox F(B) \ar[r]^{\mx} \ar[d]_{m_\ox} \ar[r]^{\mx} & F(A) \oa F(B) \\
F(A \ox B) \ar[r]_{F(\mx)} \ar[r]_{F(\mx)} & F(A \oa B) \ar[u]_{n_\oa}
}
\]
\end{lemma}
\begin{proof}
\[
\begin{tikzpicture} %dag7
	\begin{pgfonlayer}{nodelayer}
		\node [style=none] (0) at (-2.25, 2.5) {};
		\node [style=none] (1) at (-2.25, -2.25) {};
		\node [style=none] (2) at (1.5, 2.5) {};
		\node [style=none] (3) at (1.5, -2.25) {};
		\node [style=none] (4) at (-1, -4.5) {};
		\node [style=circle] (5) at (0, 1.5) {$\bot$};
		\node [style=none] (6) at (0, 0.5) {};
		\node [style=circle] (7) at (0, -0.5) {$\top$};
		\node [style=none] (8) at (-1, 4.5) {};
		\node [style=none] (9) at (1, -4.5) {};
		\node [style=none] (10) at (1, 4.5) {};
		\node [style=map] (11) at (0, 0.5) {};
		\node [style=circle, scale=0.5] (12) at (1, -2) {};
		\node [style=circle, scale=0.5] (13) at (-1, 2.25) {};
		\node [style=none] (14) at (-1.75, 2) {$F$};
		\node [style=none] (15) at (-1.5, 4.5) {$F(A)$};
		\node [style=none] (16) at (1.5, 4.5) {$F(B)$};
	\end{pgfonlayer}
	\begin{pgfonlayer}{edgelayer}
		\draw (0.center) to (1.center);
		\draw (1.center) to (3.center);
		\draw (3.center) to (2.center);
		\draw (2.center) to (0.center);
		\draw (8.center) to (4.center);
		\draw (10.center) to (9.center);
		\draw [dotted, in=90, out=3, looseness=1.00] (13) to (5);
		\draw [dotted, in=180, out=-90, looseness=1.00] (7) to (12);
		\draw (5) to (6.center);
		\draw (6.center) to (7);
	\end{pgfonlayer}
\end{tikzpicture} =  \begin{tikzpicture} %dag8
	\begin{pgfonlayer}{nodelayer}
		\node [style=none] (0) at (-2.25, 3) {};
		\node [style=none] (1) at (-2.25, 1.5) {};
		\node [style=none] (2) at (0.5000002, 3) {};
		\node [style=none] (3) at (0.5000002, 1.5) {};
		\node [style=none] (4) at (-0.9999997, -4) {};
		\node [style=circle] (5) at (0, 2) {$\bot$};
		\node [style=circle] (6) at (0, -1.5) {$\top$};
		\node [style=none] (7) at (-0.9999997, 4.5) {};
		\node [style=none] (8) at (0.9999997, -4) {};
		\node [style=none] (9) at (0.9999997, 4.5) {};
		\node [style=map] (10) at (0, 0.2500001) {};
		\node [style=circle, scale=0.5] (11) at (0.9999997, -2.25) {};
		\node [style=circle, scale=0.5] (12) at (-0.9999997, 2.75) {};
		\node [style=none] (13) at (-1.75, 2.5) {};
		\node [style=none] (14) at (-0.5000002, -1) {};
		\node [style=none] (15) at (2, -1) {};
		\node [style=none] (16) at (-0.5000002, -2.75) {};
		\node [style=none] (17) at (2, -2.75) {};
		\node [style=none] (18) at (1.5, -2.5) {};
		\node [style=none] (19) at (-0.5000002, 0.75) {};
		\node [style=none] (20) at (-0.5000002, -0.2500001) {};
		\node [style=none] (21) at (0.5000002, -0.2500001) {};
		\node [style=none] (22) at (0.5000002, 0.75) {};
		\node [style=none] (23) at (-1.75, 1.75) {$F$};
		\node [style=none] (24) at (0, -2.5) {$F$};
		\node [style=none] (25) at (-0.3, -0.1) {$F$};
	\end{pgfonlayer}
	\begin{pgfonlayer}{edgelayer}
		\draw (0.center) to (1.center);
		\draw (1.center) to (3.center);
		\draw (3.center) to (2.center);
		\draw (2.center) to (0.center);
		\draw (7.center) to (4.center);
		\draw (9.center) to (8.center);
		\draw [dotted, in=90, out=3, looseness=1.00] (12) to (5);
		\draw [dotted, in=180, out=-90, looseness=1.00] (6) to (11);
		\draw (14.center) to (16.center);
		\draw (16.center) to (17.center);
		\draw (17.center) to (15.center);
		\draw (15.center) to (14.center);
		\draw (5) to (10);
		\draw (10) to (6);
		\draw (19.center) to (20.center);
		\draw (20.center) to (21.center);
		\draw (21.center) to (22.center);
		\draw (22.center) to (19.center);
	\end{pgfonlayer}
\end{tikzpicture} \stackrel{\text{ {\bf[mix-FF]}}}{=}
\begin{tikzpicture} %dag5
	\begin{pgfonlayer}{nodelayer}
		\node [style=none] (0) at (-2.25, 2.25) {};
		\node [style=none] (1) at (-4, 2.25) {};
		\node [style=none] (2) at (-2.25, 1) {};
		\node [style=none] (3) at (-4, 1) {};
		\node [style=none] (4) at (-3.5, 3) {};
		\node [style=none] (5) at (-3.5, -6.25) {};
		\node [style=circle, scale=0.5] (6) at (-3.5, 2) {};
		\node [style=circle] (7) at (-2.75, 1.5) {$\bot$};
		\node [style=circle] (8) at (-1.5, -2) {$\top$};
		\node [style=circle] (9) at (-1.5, -0.25) {$\bot$};
		\node [style=none] (10) at (-0.75, -2) {};
		\node [style=none] (11) at (0.25, -2) {};
		\node [style=none] (12) at (0.25, -3) {};
		\node [style=none] (13) at (-3.25, -0) {};
		\node [style=none] (14) at (-2.25, -1) {};
		\node [style=circle] (15) at (-0.25, -2.5) {$\top$};
		\node [style=none] (16) at (-0.75, -3) {};
		\node [style=none] (17) at (-2.25, -0) {};
		\node [style=circle] (18) at (-2.75, -0.5) {$\bot$};
		\node [style=circle, scale=0.5] (19) at (-2.75, 0.5) {};
		\node [style=none] (20) at (-3.25, -1) {};
		\node [style=circle, scale=0.5] (21) at (-0.25, -3.5) {};
		\node [style=map] (22) at (-1.5, -1.25) {};
		\node [style=none] (23) at (1.25, -4) {};
		\node [style=none] (24) at (0.75, -6.25) {};
		\node [style=none] (25) at (-0.75, -4) {};
		\node [style=circle] (26) at (-0.25, -4.5) {$\top$};
		\node [style=none] (27) at (1.25, -5.25) {};
		\node [style=none] (28) at (0.75, 3) {};
		\node [style=none] (29) at (-0.75, -5.25) {};
		\node [style=circle, scale=0.5] (30) at (0.75, -5) {};
	\end{pgfonlayer}
	\begin{pgfonlayer}{edgelayer}
		\draw (0.center) to (1.center);
		\draw (1.center) to (3.center);
		\draw (3.center) to (2.center);
		\draw (2.center) to (0.center);
		\draw (4.center) to (5.center);
		\draw [bend left, looseness=1.25, dotted] (6) to (7);
		\draw (13.center) to (20.center);
		\draw (20.center) to (14.center);
		\draw (14.center) to (17.center);
		\draw (17.center) to (13.center);
		\draw (10.center) to (11.center);
		\draw (11.center) to (12.center);
		\draw (12.center) to (16.center);
		\draw (16.center) to (10.center);
		\draw [dotted, bend left=45, looseness=1.25] (19) to (9);
		\draw [dotted, in=-165, out=-90, looseness=1.25] (8) to (21);
		\draw (9) to (22);
		\draw (22) to (8);
		\draw (29.center) to (27.center);
		\draw (27.center) to (23.center);
		\draw (23.center) to (25.center);
		\draw (25.center) to (29.center);
		\draw (24.center) to (28.center);
		\draw [dotted, bend left=45, looseness=1.25] (30) to (26);
		\draw (15) to (26);
		\draw (7) to (18);
	\end{pgfonlayer}
\end{tikzpicture} = \begin{tikzpicture} %dag9
	\begin{pgfonlayer}{nodelayer}
		\node [style=none] (0) at (-2.25, -0.25) {};
		\node [style=none] (1) at (-4, -0.25) {};
		\node [style=none] (2) at (-2.25, -2.75) {};
		\node [style=none] (3) at (-4, -2.75) {};
		\node [style=none] (4) at (-3.5, 2.5) {};
		\node [style=none] (5) at (-3.5, -6.5) {};
		\node [style=circle, scale=0.5] (6) at (-3.5, -0.5) {};
		\node [style=circle] (7) at (-2.75, -1.25) {$\bot$};
		\node [style=circle] (8) at (-1.5, -2) {$\top$};
		\node [style=circle] (9) at (-1.5, -0.25) {$\bot$};
		\node [style=circle] (10) at (-0.25, -1.25) {$\top$};
		\node [style=circle] (11) at (-2.75, -2.25) {$\bot$};
		\node [style=circle, scale=0.5] (12) at (-3.5, 0.5) {};
		\node [style=circle, scale=0.5] (13) at (0.75, -4.25) {};
		\node [style=map] (14) at (-1.5, -1.25) {};
		\node [style=none] (15) at (1.25, -0.75) {};
		\node [style=none] (16) at (0.75, -6.5) {};
		\node [style=none] (17) at (-0.75, -0.75) {};
		\node [style=circle] (18) at (-0.25, -2.25) {$\top$};
		\node [style=none] (19) at (1.25, -3.5) {};
		\node [style=none] (20) at (0.75, 2.5) {};
		\node [style=none] (21) at (-0.75, -3.5) {};
		\node [style=circle, scale=0.5] (22) at (0.75, -3.25) {};
		\node [style=none] (23) at (-0.5, -3.25) {$F$};
		\node [style=none] (24) at (-3.75, -2.5) {$F$};
	\end{pgfonlayer}
	\begin{pgfonlayer}{edgelayer}
		\draw (0.center) to (1.center);
		\draw (1.center) to (3.center);
		\draw (3.center) to (2.center);
		\draw (2.center) to (0.center);
		\draw (4.center) to (5.center);
		\draw [bend left, looseness=1.25, dotted] (6) to (7);
		\draw [dotted, bend left, looseness=1.25] (12) to (9);
		\draw [dotted, in=-165, out=-90, looseness=1.25] (8) to (13);
		\draw (9) to (14);
		\draw (14) to (8);
		\draw (21.center) to (19.center);
		\draw (19.center) to (15.center);
		\draw (15.center) to (17.center);
		\draw (17.center) to (21.center);
		\draw (16.center) to (20.center);
		\draw [dotted, bend left=45, looseness=1.25] (22) to (18);
		\draw (10) to (18);
		\draw (7) to (11);
	\end{pgfonlayer}
\end{tikzpicture} =
\begin{tikzpicture}
	\begin{pgfonlayer}{nodelayer}
		\node [style=none] (0) at (-0.9999997, -4) {};
		\node [style=circle] (1) at (0, 2) {$\bot$};
		\node [style=circle] (2) at (0, -1.5) {$\top$};
		\node [style=none] (3) at (-0.9999997, 4.5) {};
		\node [style=none] (4) at (0.9999997, -4) {};
		\node [style=none] (5) at (0.9999997, 4.5) {};
		\node [style=map] (6) at (0, 0.2500001) {};
		\node [style=circle, scale=0.5] (7) at (0.9999997, -2.25) {};
		\node [style=circle, scale=0.5] (8) at (-0.9999997, 2.75) {};
		\node [style=none] (9) at (-1.5, 4.25) {$F(A)$};
		\node [style=none] (10) at (1.5, 4.25) {$F(B)$};
	\end{pgfonlayer}
	\begin{pgfonlayer}{edgelayer}
		\draw (3.center) to (0.center);
		\draw (5.center) to (4.center);
		\draw [dotted, in=90, out=3, looseness=1.00] (8) to (1);
		\draw [dotted, in=180, out=-90, looseness=1.00] (2) to (7);
		\draw (1) to (6);
		\draw (6) to (2);
	\end{pgfonlayer}
\end{tikzpicture}
\]
\end{proof}

Linear natural isomorphisms between Frobenius functors $(\alpha_\ox, \alpha_\oa): F \to G$ often take a special form with $\alpha_\ox = \alpha_\oa^{-1}$: this allows the coherence 
requirements to be simplified.  The next results describe some basic circumstances in which this happens:
 
\begin{lemma}
\label{Lemma: Frobenius linear transformation}
Suppose $F: \X \to \Y$ are Frobenius linear functors and $\alpha := (\alpha_\ox, \alpha_\oa): F \Rightarrow G$ is a linear natural transformation.  Then, the following are equivalent:
\begin{enumerate}[(i)]
\item One of {\bf [nat.1](a)} or {\bf [nat.1](b)} holds, and one of  $\alpha_\ox$ or $\alpha_\oa$ is an isomorphism. \[
\mbox{\bf \small [nat.1]} ~~~
\xymatrixcolsep{5pc}
\xymatrix{
\top \ar[r]^{m_\top} \ar[dr]_{m_\top} & G(\top) \ar[d]^{\alpha_\oa} \ar@{}[dl]|(.35){\tiny{(a)}} \\
& F(\top)
} ~~~~~ \text{ or }~~~~~  \xymatrix{
F(\bot) \ar[r]^{\alpha_\ox} \ar[dr]_{n_\bot} & G(\bot) \ar[d]^{n_\bot} \ar@{}[dl]|(.35){\tiny{(b)}} \\
& \bot
}
\] 


\item One of {\bf [nat.1](a)} or {\bf [nat.1](b)} holds and one of the following commuting diagrams holds.
\[ \mbox{\bf \small [nat.2]}~~~~ 
\xymatrix{
G(A) \ox F(B) \ar[r]^{1 \ox \alpha_\ox} \ar[d]_{\alpha_\oa \ox 1} \ar@{}[ddr]|{\tiny{(a)}} & G(A) \ox G(B) \ar[dd]^{m_\ox^G} \\
F(A) \ox F(B) \ar[d]_{m_\ox^F} & \\
F(A \ox B) \ar[r]_{\alpha_\ox} & G(A \ox B)
} ~~~ \text{ or }~~~ \xymatrix{
F(A) \ox G(B) \ar[r]^{\alpha_\ox \ox 1} \ar[d]_{1 \ox \alpha_\oa} \ar@{}[ddr]|{\tiny{(b)}}  & G(A) \ox G(B) \ar[dd]^{m_\ox^G} \\
F(A) \ox F(B) \ar[d]_{m_\ox^F} & \\
F(A \ox B) \ar[r]_{\alpha_\ox} & G(A \ox B)
}
\]
\[
\text{or}~~~~~ 
\xymatrix{
G(A \oa B) \ar[r]^{n_\oa^G} \ar[d]_{\alpha_\oa} \ar@{}[ddr]|{\tiny{(c)}}  & G(A) \oa G(B) \ar[dd]^{1 \oa \alpha_\oa} \\
F(A \oa B) \ar[d]_{n_\oa^F} & \\
F(A) \oa F(B) \ar[r]_{\alpha_\ox \oa 1} & G(A) \oa F(B)
} ~~~\text{or}~~~
\xymatrix{
G(A \oa B) \ar[r]^{n_\oa^G} \ar[d]_{\alpha_\oa} \ar@{}[ddr]|{\tiny{(d)}} & G(A) \oa G(B) \ar[dd]^{\alpha_\oa \oa 1} \\
F(A \oa B) \ar[d]_{n_\oa} & \\
F(A) \oa F(B) \ar[r]_{1 \ox \alpha_\ox} & F(A) \oa G(B)
}
\] 
\item $\alpha_\ox^{-1} = \alpha_\oa$
\item $\alpha' := (\alpha_\oa, \alpha_\ox): G \Rightarrow F$ is a linear transformation.
\end{enumerate}
\end{lemma}

Conditions {\bf [nat.2]} are as follows in the graphical calculus:
\[ \mbox{\small (a)}~~
\begin{tikzpicture}%nat2
\begin{pgfonlayer}{nodelayer}
\node [style=none] (0) at (-2.5, -0) {};
\node [style=none] (1) at (-2.5, -1) {};
\node [style=none] (2) at (-0.5, -1) {};
\node [style=none] (3) at (-0.5, -0) {};
\node [style=ox] (4) at (-1.5, -0.5) {};
\node [style=none] (5) at (-2, 1.25) {};
\node [style=none] (6) at (-1, 1.25) {};
\node [style=none] (7) at (-1.5, -2) {};
\node [style=circle, scale=2] (8) at (-2, 0.5) {};
\node [style=none] (9) at (-2, 0.5) {$\alpha_\ox$};
\node [style=none] (10) at (-0.75, -0.25) {$G$};
\end{pgfonlayer}
\begin{pgfonlayer}{edgelayer}
\draw (0.center) to (1.center);
\draw (1.center) to (2.center);
\draw (2.center) to (3.center);
\draw (0.center) to (3.center);
\draw [in=-90, out=30, looseness=1.00] (4) to (6.center);
\draw (4) to (7.center);
\draw (5.center) to (8);
\draw [in=150, out=-90, looseness=1.00] (8) to (4);
\end{pgfonlayer}
\end{tikzpicture} = \begin{tikzpicture}
\begin{pgfonlayer}{nodelayer}
\node [style=none] (0) at (-0.5, 0.75) {};
\node [style=none] (1) at (-0.5, -0.25) {};
\node [style=none] (2) at (-2.5, -0.25) {};
\node [style=none] (3) at (-2.5, 0.75) {};
\node [style=ox] (4) at (-1.5, 0.25) {};
\node [style=none] (5) at (-1, 2) {};
\node [style=none] (6) at (-2, 2) {};
\node [style=circle, scale=2] (7) at (-1, 1.25) {};
\node [style=none] (8) at (-1, 1.25) {$\alpha_\oa$};
\node [style=circle, scale=2] (9) at (-1.5, -1) {};
\node [style=none] (10) at (-1.5, -2) {};
\node [style=none] (11) at (-1.5, -1) {$\alpha_\ox$};
\node [style=none] (12) at (-0.75, 0.5) {$F$};
\end{pgfonlayer}
\begin{pgfonlayer}{edgelayer}
\draw (0.center) to (1.center);
\draw (1.center) to (2.center);
\draw (2.center) to (3.center);
\draw (0.center) to (3.center);
\draw [in=-90, out=150, looseness=1.00] (4) to (6.center);
\draw (5.center) to (7);
\draw [in=30, out=-90, looseness=1.00] (7) to (4);
\draw (10.center) to (9);
\draw (9) to (4);
\end{pgfonlayer}
\end{tikzpicture}~~~~~~ \mbox{\small (b)} ~~\begin{tikzpicture}
\begin{pgfonlayer}{nodelayer}
\node [style=none] (0) at (-0.5, -0) {};
\node [style=none] (1) at (-0.5, -1) {};
\node [style=none] (2) at (-2.5, -1) {};
\node [style=none] (3) at (-2.5, -0) {};
\node [style=ox] (4) at (-1.5, -0.5) {};
\node [style=none] (5) at (-1, 1.25) {};
\node [style=none] (6) at (-2, 1.25) {};
\node [style=none] (7) at (-1.5, -2) {};
\node [style=circle, scale=2] (8) at (-1, 0.5) {};
\node [style=none] (9) at (-1, 0.5) {$\alpha_\ox$};
\node [style=none] (10) at (-0.75, -0.25) {$G$};
\end{pgfonlayer}
\begin{pgfonlayer}{edgelayer}
\draw (0.center) to (1.center);
\draw (1.center) to (2.center);
\draw (2.center) to (3.center);
\draw (0.center) to (3.center);
\draw [in=-90, out=150, looseness=1.00] (4) to (6.center);
\draw (4) to (7.center);
\draw (5.center) to (8);
\draw [in=30, out=-90, looseness=1.00] (8) to (4);
\end{pgfonlayer}
\end{tikzpicture} = \begin{tikzpicture}
\begin{pgfonlayer}{nodelayer}
\node [style=none] (0) at (-2.5, 0.75) {};
\node [style=none] (1) at (-2.5, -0.25) {};
\node [style=none] (2) at (-0.5, -0.25) {};
\node [style=none] (3) at (-0.5, 0.75) {};
\node [style=ox] (4) at (-1.5, 0.25) {};
\node [style=none] (5) at (-2, 2) {};
\node [style=none] (6) at (-1, 2) {};
\node [style=circle, scale=2] (7) at (-2, 1.25) {};
\node [style=none] (8) at (-2, 1.25) {$\alpha_\oa$};
\node [style=circle, scale=2] (9) at (-1.5, -1) {};
\node [style=none] (10) at (-1.5, -2) {};
\node [style=none] (11) at (-1.5, -1) {$\alpha_\ox$};
\node [style=none] (12) at (-2.25, 0.5) {$F$};
\end{pgfonlayer}
\begin{pgfonlayer}{edgelayer}
\draw (0.center) to (1.center);
\draw (1.center) to (2.center);
\draw (2.center) to (3.center);
\draw (0.center) to (3.center);
\draw [in=-90, out=30, looseness=1.00] (4) to (6.center);
\draw (5.center) to (7);
\draw [in=150, out=-90, looseness=1.00] (7) to (4);
\draw (10.center) to (9);
\draw (9) to (4);
\end{pgfonlayer}
\end{tikzpicture}  ~~~~~~~~~ \mbox{\small (c)}~~\begin{tikzpicture}
\begin{pgfonlayer}{nodelayer}
\node [style=none] (0) at (-0.5, -0.75) {};
\node [style=none] (1) at (-0.5, 0.25) {};
\node [style=none] (2) at (-2.5, 0.25) {};
\node [style=none] (3) at (-2.5, -0.75) {};
\node [style=oa] (4) at (-1.5, -0.25) {};
\node [style=none] (5) at (-1, -2) {};
\node [style=none] (6) at (-2, -2) {};
\node [style=none] (7) at (-1.5, 1.25) {};
\node [style=circle, scale=2] (8) at (-1, -1.25) {};
\node [style=none] (9) at (-1, -1.25) {$\alpha_\oa$};
\node [style=none] (10) at (-0.75, -0) {$G$};
\end{pgfonlayer}
\begin{pgfonlayer}{edgelayer}
\draw (0.center) to (1.center);
\draw (1.center) to (2.center);
\draw (2.center) to (3.center);
\draw (0.center) to (3.center);
\draw [in=90, out=-150, looseness=1.00] (4) to (6.center);
\draw (4) to (7.center);
\draw (5.center) to (8);
\draw [in=-30, out=90, looseness=1.00] (8) to (4);
\end{pgfonlayer}
\end{tikzpicture} = \begin{tikzpicture}
\begin{pgfonlayer}{nodelayer}
\node [style=none] (0) at (-2.5, -0.75) {};
\node [style=none] (1) at (-2.5, 0.25) {};
\node [style=none] (2) at (-0.5, 0.25) {};
\node [style=none] (3) at (-0.5, -0.75) {};
\node [style=oa] (4) at (-1.5, -0.25) {};
\node [style=none] (5) at (-2, -2) {};
\node [style=none] (6) at (-1, -2) {};
\node [style=circle, scale=2] (7) at (-2, -1.25) {};
\node [style=none] (8) at (-2, -1.25) {$\alpha_\ox$};
\node [style=circle, scale=2] (9) at (-1.5, 1) {};
\node [style=none] (10) at (-1.5, 2) {};
\node [style=none] (11) at (-1.5, 1) {$\alpha_\oa$};
\node [style=none] (12) at (-0.75, -0) {$F$};
\end{pgfonlayer}
\begin{pgfonlayer}{edgelayer}
\draw (0.center) to (1.center);
\draw (1.center) to (2.center);
\draw (2.center) to (3.center);
\draw (0.center) to (3.center);
\draw [in=90, out=-30, looseness=1.00] (4) to (6.center);
\draw (5.center) to (7);
\draw [in=-150, out=90, looseness=1.00] (7) to (4);
\draw (10.center) to (9);
\draw (9) to (4);
\end{pgfonlayer}
\end{tikzpicture} ~~~~~~~~ \mbox{\small (d)}~~\begin{tikzpicture}
\begin{pgfonlayer}{nodelayer}
\node [style=none] (0) at (-2.5, -0.75) {};
\node [style=none] (1) at (-2.5, 0.25) {};
\node [style=none] (2) at (-0.5, 0.25) {};
\node [style=none] (3) at (-0.5, -0.75) {};
\node [style=oa] (4) at (-1.5, -0.25) {};
\node [style=none] (5) at (-2, -2) {};
\node [style=none] (6) at (-1, -2) {};
\node [style=none] (7) at (-1.5, 1.25) {};
\node [style=circle, scale=2] (8) at (-2, -1.25) {};
\node [style=none] (9) at (-2, -1.25) {$\alpha_\oa$};
\node [style=none] (10) at (-0.75, -0) {$G$};
\end{pgfonlayer}
\begin{pgfonlayer}{edgelayer}
\draw (0.center) to (1.center);
\draw (1.center) to (2.center);
\draw (2.center) to (3.center);
\draw (0.center) to (3.center);
\draw [in=90, out=-30, looseness=1.00] (4) to (6.center);
\draw (4) to (7.center);
\draw (5.center) to (8);
\draw [in=-150, out=90, looseness=1.00] (8) to (4);
\end{pgfonlayer}
\end{tikzpicture} = \begin{tikzpicture}
\begin{pgfonlayer}{nodelayer}
\node [style=none] (0) at (-0.5, -0.75) {};
\node [style=none] (1) at (-0.5, 0.25) {};
\node [style=none] (2) at (-2.5, 0.25) {};
\node [style=none] (3) at (-2.5, -0.75) {};
\node [style=oa] (4) at (-1.5, -0.25) {};
\node [style=none] (5) at (-1, -2) {};
\node [style=none] (6) at (-2, -2) {};
\node [style=circle,scale=2] (7) at (-1, -1.25) {};
\node [style=none] (8) at (-1, -1.25) {$\alpha_\ox$};
\node [style=circle, scale=2] (9) at (-1.5, 1) {};
\node [style=none] (10) at (-1.5, 2) {};
\node [style=none] (11) at (-1.5, 1) {$\alpha_\oa$};
\node [style=none] (12) at (-0.75, -0) {$F$};
\end{pgfonlayer}
\begin{pgfonlayer}{edgelayer}
\draw (0.center) to (1.center);
\draw (1.center) to (2.center);
\draw (2.center) to (3.center);
\draw (0.center) to (3.center);
\draw [in=90, out=-150, looseness=1.00] (4) to (6.center);
\draw (5.center) to (7);
\draw [in=-30, out=90, looseness=1.00] (7) to (4);
\draw (10.center) to (9);
\draw (9) to (4);
\end{pgfonlayer}
\end{tikzpicture}
\]




\begin{proof}
\begin{description}
\item{$(i) \Rightarrow (iii)$:} Here is the proof assuming {\bf \small [nat.1](a)} that $\alpha_\ox \alpha_\oa = 1$:
\[ \hspace{-0.75cm}
\begin{tikzpicture} %dag15
	\begin{pgfonlayer}{nodelayer}
		\node [style=none] (0) at (0.5, 3.5) {};
		\node [style=none] (1) at (0.5, -2) {};
		\node [style=none] (2) at (0, 3.25) {$F(X)$};
	\end{pgfonlayer}
	\begin{pgfonlayer}{edgelayer}
		\draw (0.center) to (1.center);
	\end{pgfonlayer}
\end{tikzpicture} =
\begin{tikzpicture} %dag14
	\begin{pgfonlayer}{nodelayer}
		\node [style=none] (0) at (0.5, 3.5) {};
		\node [style=none] (1) at (0.5, -2) {};
		\node [style=none] (2) at (0, 2.5) {};
		\node [style=none] (3) at (0, -1) {};
		\node [style=none] (4) at (1.5, -1) {};
		\node [style=none] (5) at (1.5, 2.5) {};
		\node [style=none] (6) at (1.25, -0.5) {$F$};
		\node [style=none] (7) at (0.25, -1) {};
		\node [style=none] (8) at (0.75, -1) {};
		\node [style=none] (9) at (0, 3.25) {$F(X)$};
	\end{pgfonlayer}
	\begin{pgfonlayer}{edgelayer}
		\draw (0.center) to (1.center);
		\draw (2.center) to (5.center);
		\draw (5.center) to (4.center);
		\draw (4.center) to (3.center);
		\draw (3.center) to (2.center);
		\draw [bend left=90, looseness=1.25] (7.center) to (8.center);
	\end{pgfonlayer}
\end{tikzpicture} =
\begin{tikzpicture} %dag13
	\begin{pgfonlayer}{nodelayer}
		\node [style=circle] (0) at (1.75, 1.25) {$\top$};
		\node [style=circle] (1) at (1.75, -0) {$\top$};
		\node [style=none] (2) at (0.5, 3.5) {};
		\node [style=none] (3) at (0.5, -2) {};
		\node [style=circle, scale=0.5] (4) at (0.5, -0.75) {};
		\node [style=none] (5) at (0, 2) {};
		\node [style=none] (6) at (0, -1.5) {};
		\node [style=none] (7) at (2.5, -1.5) {};
		\node [style=none] (8) at (2.5, 2) {};
		\node [style=none] (9) at (2.25, -1) {$F$};
		\node [style=none] (10) at (0.25, -1.5) {};
		\node [style=none] (11) at (0.75, -1.5) {};
		\node [style=none] (12) at (0, 3.25) {$F(X)$};
	\end{pgfonlayer}
	\begin{pgfonlayer}{edgelayer}
		\draw (2.center) to (3.center);
		\draw [dotted, in=0, out=-90, looseness=1.25] (1) to (4);
		\draw (5.center) to (8.center);
		\draw (8.center) to (7.center);
		\draw (7.center) to (6.center);
		\draw (6.center) to (5.center);
		\draw [bend left=90, looseness=1.25] (10.center) to (11.center);
		\draw (0) to (1);
	\end{pgfonlayer}
\end{tikzpicture} =
\begin{tikzpicture} %dag12
	\begin{pgfonlayer}{nodelayer}
		\node [style=circle] (0) at (2, 2) {$\top$};
		\node [style=circle] (1) at (2, -0) {$\top$};
		\node [style=none] (2) at (0.5, 3.5) {};
		\node [style=none] (3) at (0.5, -2) {};
		\node [style=circle, scale=0.5] (4) at (0.5, -0.75) {};
		\node [style=none] (5) at (0, 0.5) {};
		\node [style=none] (6) at (0, -1.5) {};
		\node [style=none] (7) at (2.5, -1.5) {};
		\node [style=none] (8) at (2.5, 0.5) {};
		\node [style=none] (9) at (1.5, 2.5) {};
		\node [style=none] (10) at (3, 2.5) {};
		\node [style=none] (11) at (1.5, 1.25) {};
		\node [style=none] (12) at (3, 1.25) {};
		\node [style=none] (13) at (2.25, -1.25) {$F$};
		\node [style=none] (14) at (0.25, -1.5) {};
		\node [style=none] (15) at (0.75, -1.5) {};
		\node [style=none] (16) at (1.75, 1.25) {};
		\node [style=none] (17) at (2.25, 1.25) {};
		\node [style=none] (18) at (0, 3.25) {$F(X)$};
		\node [style=none] (19) at (2.75, 1.5) {$F$};
	\end{pgfonlayer}
	\begin{pgfonlayer}{edgelayer}
		\draw (2.center) to (3.center);
		\draw [dotted, in=0, out=-90, looseness=1.25] (1) to (4);
		\draw (5.center) to (8.center);
		\draw (8.center) to (7.center);
		\draw (7.center) to (6.center);
		\draw (6.center) to (5.center);
		\draw (9.center) to (11.center);
		\draw (11.center) to (12.center);
		\draw (12.center) to (10.center);
		\draw (10.center) to (9.center);
		\draw [bend left=90, looseness=1.25] (14.center) to (15.center);
		\draw [bend left=90, looseness=1.25] (16.center) to (17.center);
		\draw (0) to (1);
	\end{pgfonlayer}
\end{tikzpicture}
=
\begin{tikzpicture} %dag10
	\begin{pgfonlayer}{nodelayer}
		\node [style=circle] (0) at (2, 1.25) {$\alpha_\oa$};
		\node [style=circle] (1) at (2, 2.75) {$\top$};
		\node [style=circle] (2) at (2, -0.25) {$\top$};
		\node [style=none] (3) at (0.5, 3.5) {};
		\node [style=none] (4) at (0.5, -2) {};
		\node [style=circle, scale=0.5] (5) at (0.5, -1) {};
		\node [style=none] (6) at (0, 0.25) {};
		\node [style=none] (7) at (0, -1.5) {};
		\node [style=none] (8) at (2.5, -1.5) {};
		\node [style=none] (9) at (2.5, 0.25) {};
		\node [style=none] (10) at (1.5, 3.25) {};
		\node [style=none] (11) at (3, 3.25) {};
		\node [style=none] (12) at (1.5, 2) {};
		\node [style=none] (13) at (3, 2) {};
		\node [style=none] (14) at (2.25, -1.25) {$F$};
		\node [style=none] (15) at (0.25, -1.5) {};
		\node [style=none] (16) at (0.75, -1.5) {};
		\node [style=none] (17) at (1.75, 2) {};
		\node [style=none] (18) at (2.25, 2) {};
		\node [style=none] (19) at (2.75, 2.25) {$G$};
		\node [style=none] (20) at (0, 3.25) {$F(X)$};
	\end{pgfonlayer}
	\begin{pgfonlayer}{edgelayer}
		\draw (1) to (0);
		\draw (0) to (2);
		\draw (3.center) to (4.center);
		\draw [dotted, in=0, out=-90, looseness=1.25] (2) to (5);
		\draw (6.center) to (9.center);
		\draw (9.center) to (8.center);
		\draw (8.center) to (7.center);
		\draw (7.center) to (6.center);
		\draw (10.center) to (12.center);
		\draw (12.center) to (13.center);
		\draw (13.center) to (11.center);
		\draw (11.center) to (10.center);
		\draw [bend left=105, looseness=1.00] (15.center) to (16.center);
		\draw [bend left=90, looseness=1.25] (17.center) to (18.center);
	\end{pgfonlayer}
\end{tikzpicture}=
\begin{tikzpicture} %dag10
	\begin{pgfonlayer}{nodelayer}
		\node [style=circle] (0) at (2, 1.25) {$\alpha_\oa$};
		\node [style=circle] (1) at (2, 2.75) {$\top$};
		\node [style=circle] (2) at (2, -0.25) {$\top$};
		\node [style=none] (3) at (0.5, 3.5) {};
		\node [style=none] (4) at (0.5, -2) {};
		\node [style=circle, scale=0.5] (5) at (0.5, -1) {};
		\node [style=none] (6) at (0, 0.5) {};
		\node [style=none] (7) at (0, -1.5) {};
		\node [style=none] (8) at (2.5, -1.5) {};
		\node [style=none] (9) at (2.5, 0.5) {};
		\node [style=none] (10) at (1.5, 3.25) {};
		\node [style=none] (11) at (3, 3.25) {};
		\node [style=none] (12) at (1.5, 2) {};
		\node [style=none] (13) at (3, 2) {};
		\node [style=none] (14) at (2.25, -1.25) {$F$};
		\node [style=none] (15) at (1.75, 0.5) {};
		\node [style=none] (16) at (2.25, 0.5) {};
		\node [style=none] (17) at (1.75, 2) {};
		\node [style=none] (18) at (2.25, 2) {};
		\node [style=none] (19) at (2.75, 2.25) {$G$};
		\node [style=none] (20) at (0, 3.25) {$F(X)$};
	\end{pgfonlayer}
	\begin{pgfonlayer}{edgelayer}
		\draw (1) to (0);
		\draw (0) to (2);
		\draw (3.center) to (4.center);
		\draw [dotted, in=0, out=-90, looseness=1.25] (2) to (5);
		\draw (6.center) to (9.center);
		\draw (9.center) to (8.center);
		\draw (8.center) to (7.center);
		\draw (7.center) to (6.center);
		\draw (10.center) to (12.center);
		\draw (12.center) to (13.center);
		\draw (13.center) to (11.center);
		\draw (11.center) to (10.center);
		\draw [bend right=105, looseness=1.00] (15.center) to (16.center);
		\draw [bend left=90, looseness=1.25] (17.center) to (18.center);
	\end{pgfonlayer}
\end{tikzpicture} = 
\begin{tikzpicture} %dag11
	\begin{pgfonlayer}{nodelayer}
		\node [style=circle] (0) at (1.75, 3) {$\top$};
		\node [style=circle] (1) at (1.75, 0.75) {$\top$};
		\node [style=none] (2) at (0.5, 3.5) {};
		\node [style=none] (3) at (0.5, -2.5) {};
		\node [style=circle, scale=0.5] (4) at (0.5, -0) {};
		\node [style=none] (5) at (0, 1.5) {};
		\node [style=none] (6) at (0, -0.75) {};
		\node [style=none] (7) at (2.25, -0.75) {};
		\node [style=none] (8) at (2.25, 1.5) {};
		\node [style=none] (9) at (1.25, 3.5) {};
		\node [style=none] (10) at (2.75, 3.5) {};
		\node [style=none] (11) at (1.25, 2) {};
		\node [style=none] (12) at (2.75, 2) {};
		\node [style=none] (13) at (2, -0.5) {$G$};
		\node [style=circle, scale=2] (14) at (0.5, 2.5) {};
		\node [style=circle, scale=2] (15) at (0.5, -1.5) {};
		\node [style=none] (16) at (0.5, 2.5) {$\alpha_\ox$};
		\node [style=none] (17) at (0.5, -1.5) {$\alpha_\oa$};
		\node [style=none] (18) at (2.5, 2.25) {$F$};
		\node [style=none] (19) at (1.5, 2) {};
		\node [style=none] (20) at (2, 2) {};
		\node [style=none] (21) at (-0.3, 3.25) {$F(X)$};
		\node [style=none] (24) at (-0.3, -2) {$F(X)$};
		\node [style=none] (25) at (2, 1.5) {};
		\node [style=none] (26) at (1.5, 1.5) {};
	\end{pgfonlayer}
	\begin{pgfonlayer}{edgelayer}
		\draw [dotted, in=0, out=-105, looseness=1.25] (1) to (4);
		\draw (5.center) to (8.center);
		\draw (8.center) to (7.center);
		\draw (7.center) to (6.center);
		\draw (6.center) to (5.center);
		\draw (9.center) to (11.center);
		\draw (11.center) to (12.center);
		\draw (12.center) to (10.center);
		\draw (10.center) to (9.center);
		\draw (0) to (1);
		\draw (2.center) to (14);
		\draw (14) to (4);
		\draw (4) to (15);
		\draw (15) to (3.center);
		\draw [bend left=75, looseness=1.50] (19.center) to (20.center);
		\draw [bend right=75, looseness=1.50] (26.center) to (25.center);
	\end{pgfonlayer}
\end{tikzpicture} =
\begin{tikzpicture} %dag11
	\begin{pgfonlayer}{nodelayer}
		\node [style=circle] (0) at (1.75, 3) {$\top$};
		\node [style=circle] (1) at (1.75, 0.75) {$\top$};
		\node [style=none] (2) at (0.5, 3.5) {};
		\node [style=none] (3) at (0.5, -2.5) {};
		\node [style=circle, scale=0.5] (4) at (0.5, -0) {};
		\node [style=none] (5) at (0, 1.5) {};
		\node [style=none] (6) at (0, -0.75) {};
		\node [style=none] (7) at (2.25, -0.75) {};
		\node [style=none] (8) at (2.25, 1.5) {};
		\node [style=none] (9) at (1.25, 3.5) {};
		\node [style=none] (10) at (2.75, 3.5) {};
		\node [style=none] (11) at (1.25, 2) {};
		\node [style=none] (12) at (2.75, 2) {};
		\node [style=none] (13) at (2, -0.5) {$G$};
		\node [style=circle, scale=2] (14) at (0.5, 2.5) {};
		\node [style=circle, scale=2] (15) at (0.5, -1.5) {};
		\node [style=none] (16) at (0.5, 2.5) {$\alpha_\ox$};
		\node [style=none] (17) at (0.5, -1.5) {$\alpha_\oa$};
		\node [style=none] (18) at (2.5, 2.25) {$G$};
		\node [style=none] (19) at (1.5, 2) {};
		\node [style=none] (20) at (2, 2) {};
		\node [style=none] (21) at (-0.3, 3.25) {$F(X)$};
		\node [style=none] (24) at (-0.3, -2) {$F(X)$};
		\node [style=none] (25) at (0.75, -0.75) {};
		\node [style=none] (26) at (0.25, -0.75) {};
	\end{pgfonlayer}
	\begin{pgfonlayer}{edgelayer}
		\draw [dotted, in=0, out=-105, looseness=1.25] (1) to (4);
		\draw (5.center) to (8.center);
		\draw (8.center) to (7.center);
		\draw (7.center) to (6.center);
		\draw (6.center) to (5.center);
		\draw (9.center) to (11.center);
		\draw (11.center) to (12.center);
		\draw (12.center) to (10.center);
		\draw (10.center) to (9.center);
		\draw (0) to (1);
		\draw (2.center) to (14);
		\draw (14) to (4);
		\draw (4) to (15);
		\draw (15) to (3.center);
		\draw [bend left=75, looseness=1.50] (19.center) to (20.center);
		\draw [bend left=75, looseness=1.50] (26.center) to (25.center);
	\end{pgfonlayer}
\end{tikzpicture} = \begin{tikzpicture} %dag16
	\begin{pgfonlayer}{nodelayer}
		\node [style=none] (0) at (0.5, 3.5) {};
		\node [style=none] (1) at (0.5, -2) {};
		\node [style=none] (2) at (0, 3.25) {$F(X)$};
		\node [style=circle] (3) at (0.5, 2) {$\alpha_\ox$};
		\node [style=circle] (4) at (0.5, -0) {$\alpha_\oa$};
	\end{pgfonlayer}
	\begin{pgfonlayer}{edgelayer}
		\draw (0.center) to (3);
		\draw (3) to (4);
		\draw (4) to (1.center);
	\end{pgfonlayer}
\end{tikzpicture}
\]
if either $\alpha_\ox$ or $\alpha_\oa$ are isomorphisms this implies $\alpha_\oa \alpha_\ox =1$.
\item{$(ii) \Rightarrow (iii)$:}
The assumption of {\bf [nat.1](a)} or {\bf (b)} yields, as above, that $\alpha_\ox \alpha_\oa =1$. 
Using {\bf [nat.2](c)} for example gives $\alpha_\oa \alpha_\ox =1$:
\[
\begin{tikzpicture} %dag10
	\begin{pgfonlayer}{nodelayer}
		\node [style=none] (0) at (0.5, 3.5) {};
		\node [style=none] (1) at (0.5, -2) {};
		\node [style=none] (2) at (0, 3.25) {$G(X)$};
	\end{pgfonlayer}
	\begin{pgfonlayer}{edgelayer}
		\draw (0.center) to (1.center);
	\end{pgfonlayer}
\end{tikzpicture}=
\begin{tikzpicture} %dag14
	\begin{pgfonlayer}{nodelayer}
		\node [style=none] (0) at (0.5, 3.5) {};
		\node [style=none] (1) at (0.5, -2) {};
		\node [style=none] (2) at (0, 2.5) {};
		\node [style=none] (3) at (0, -1) {};
		\node [style=none] (4) at (1.5, -1) {};
		\node [style=none] (5) at (1.5, 2.5) {};
		\node [style=none] (6) at (1.25, -0.5) {$G$};
		\node [style=none] (7) at (0.25, 2.5) {};
		\node [style=none] (8) at (0.75, 2.5) {};
		\node [style=none] (9) at (0, 3.25) {$G(X)$};
	\end{pgfonlayer}
	\begin{pgfonlayer}{edgelayer}
		\draw (0.center) to (1.center);
		\draw (2.center) to (5.center);
		\draw (5.center) to (4.center);
		\draw (4.center) to (3.center);
		\draw (3.center) to (2.center);
		\draw [bend right=90, looseness=1.25] (7.center) to (8.center);
	\end{pgfonlayer}
\end{tikzpicture} = 
\begin{tikzpicture} %dag12b
	\begin{pgfonlayer}{nodelayer}
		\node [style=circle] (0) at (1.75, -1) {$\bot$};
		\node [style=circle] (1) at (1.75, 1) {$\bot$};
		\node [style=none] (2) at (0.5, -2.5) {};
		\node [style=none] (3) at (0.5, 3.5) {};
		\node [style=circle, scale=0.5] (4) at (0.5, 1.75) {};
		\node [style=none] (5) at (0, 0.5) {};
		\node [style=none] (6) at (0, 2.5) {};
		\node [style=none] (7) at (2.5, 2.5) {};
		\node [style=none] (8) at (2.5, 0.5) {};
		\node [style=none] (9) at (1.25, -1.5) {};
		\node [style=none] (10) at (2.75, -1.5) {};
		\node [style=none] (11) at (1.25, -0.25) {};
		\node [style=none] (12) at (2.75, -0.25) {};
		\node [style=none] (13) at (2.25, 0.75) {$G$};
		\node [style=none] (14) at (0.25, 2.5) {};
		\node [style=none] (15) at (0.75, 2.5) {};
		\node [style=none] (16) at (1.5, -0.25) {};
		\node [style=none] (17) at (2, -0.25) {};
		\node [style=none] (18) at (2.5, -1.25) {$G$};
		\node [style=none] (19) at (0, 3.25) {$G(X)$};
		\node [style=none] (20) at (0, -2) {$G(X)$};
	\end{pgfonlayer}
	\begin{pgfonlayer}{edgelayer}
		\draw [dotted, in=0, out=90, looseness=1.25] (1) to (4);
		\draw (5.center) to (8.center);
		\draw (8.center) to (7.center);
		\draw (7.center) to (6.center);
		\draw (6.center) to (5.center);
		\draw (9.center) to (11.center);
		\draw (11.center) to (12.center);
		\draw (12.center) to (10.center);
		\draw (10.center) to (9.center);
		\draw [bend right=90, looseness=1.50] (14.center) to (15.center);
		\draw [bend right=90, looseness=1.25] (16.center) to (17.center);
		\draw (3.center) to (4);
		\draw (4) to (2.center);
		\draw (1) to (0);
	\end{pgfonlayer}
\end{tikzpicture} =
\begin{tikzpicture} %dag11b
	\begin{pgfonlayer}{nodelayer}
		\node [style=circle] (0) at (1.75, -1.75) {$\bot$};
		\node [style=circle] (1) at (1.75, 1) {$\bot$};
		\node [style=none] (2) at (0.5, -2.5) {};
		\node [style=none] (3) at (0.5, 3.5) {};
		\node [style=circle, scale=0.5] (4) at (0.5, 1.75) {};
		\node [style=none] (5) at (0, 0.5) {};
		\node [style=none] (6) at (0, 2.5) {};
		\node [style=none] (7) at (2.5, 2.5) {};
		\node [style=none] (8) at (2.5, 0.5) {};
		\node [style=none] (9) at (1.25, -2.25) {};
		\node [style=none] (10) at (2.75, -2.25) {};
		\node [style=none] (11) at (1.25, -1) {};
		\node [style=none] (12) at (2.75, -1) {};
		\node [style=none] (13) at (2.25, 0.75) {$G$};
		\node [style=none] (14) at (0.25, 2.5) {};
		\node [style=none] (15) at (0.75, 2.5) {};
		\node [style=none] (16) at (1.5, -1) {};
		\node [style=none] (17) at (2, -1) {};
		\node [style=none] (18) at (2.5, -2) {$F$};
		\node [style=none] (19) at (0, 3.25) {$G(X)$};
		\node [style=none] (20) at (0, -2) {$G(X)$};
		\node [style=circle] (21) at (1.75, -0.25) {$\alpha_\oa$};
	\end{pgfonlayer}
	\begin{pgfonlayer}{edgelayer}
		\draw [dotted, in=0, out=90, looseness=1.25] (1) to (4);
		\draw (5.center) to (8.center);
		\draw (8.center) to (7.center);
		\draw (7.center) to (6.center);
		\draw (6.center) to (5.center);
		\draw (9.center) to (11.center);
		\draw (11.center) to (12.center);
		\draw (12.center) to (10.center);
		\draw (10.center) to (9.center);
		\draw [bend right=90, looseness=1.50] (14.center) to (15.center);
		\draw [bend right=90, looseness=1.25] (16.center) to (17.center);
		\draw (3.center) to (4);
		\draw (4) to (2.center);
		\draw (1) to (21);
		\draw (21) to (0);
	\end{pgfonlayer}
\end{tikzpicture}  =
\begin{tikzpicture} %dag10b
	\begin{pgfonlayer}{nodelayer}
		\node [style=circle] (0) at (1.75, -1.75) {$\bot$};
		\node [style=circle] (1) at (1.75, 0.5) {$\bot$};
		\node [style=none] (2) at (0.5, -2.5) {};
		\node [style=none] (3) at (0.5, 3.5) {};
		\node [style=circle, scale=0.5] (4) at (0.5, 1.25) {};
		\node [style=none] (5) at (0, -0) {};
		\node [style=none] (6) at (0, 2) {};
		\node [style=none] (7) at (2.5, 2) {};
		\node [style=none] (8) at (2.5, -0) {};
		\node [style=none] (9) at (1.25, -2.25) {};
		\node [style=none] (10) at (2.75, -2.25) {};
		\node [style=none] (11) at (1.25, -1) {};
		\node [style=none] (12) at (2.75, -1) {};
		\node [style=none] (13) at (2.25, 0.25) {$F$};
		\node [style=none] (14) at (0.25, 2) {};
		\node [style=none] (15) at (0.75, 2) {};
		\node [style=none] (16) at (1.5, -1) {};
		\node [style=none] (17) at (2, -1) {};
		\node [style=circle] (18) at (0.5, 2.75) {$\alpha_\oa$};
		\node [style=circle] (19) at (0.5, -1.25) {$\alpha_\ox$};
		\node [style=none] (20) at (2.5, -2) {$F$};
		\node [style=none] (21) at (0, 3.5) {$G(X)$};
		\node [style=none] (22) at (0, -2) {$G(X)$};
	\end{pgfonlayer}
	\begin{pgfonlayer}{edgelayer}
		\draw [dotted, in=0, out=90, looseness=1.25] (1) to (4);
		\draw (5.center) to (8.center);
		\draw (8.center) to (7.center);
		\draw (7.center) to (6.center);
		\draw (6.center) to (5.center);
		\draw (9.center) to (11.center);
		\draw (11.center) to (12.center);
		\draw (12.center) to (10.center);
		\draw (10.center) to (9.center);
		\draw [bend right=90, looseness=1.50] (14.center) to (15.center);
		\draw [bend right=90, looseness=1.25] (16.center) to (17.center);
		\draw (1) to (0);
		\draw (3.center) to (18);
		\draw (18) to (4);
		\draw (4) to (19);
		\draw (19) to (2.center);
	\end{pgfonlayer}
\end{tikzpicture} =  \begin{tikzpicture} %dag16
	\begin{pgfonlayer}{nodelayer}
		\node [style=none] (0) at (0.5, 3.5) {};
		\node [style=none] (1) at (0.5, -2) {};
		\node [style=none] (2) at (0, 3.25) {$G(X)$};
		\node [style=circle] (3) at (0.5, 2) {$\alpha_\oa$};
		\node [style=circle] (4) at (0.5, -0) {$\alpha_\ox$};
		\node [style=none] (5) at (0, -1.5) {$G(X)$};
	\end{pgfonlayer}
	\begin{pgfonlayer}{edgelayer}
		\draw (0.center) to (3);
		\draw (3) to (4);
		\draw (4) to (1.center);
	\end{pgfonlayer}
\end{tikzpicture}
\]
Since, $\alpha_\ox \alpha_\oa = 1$ and $\alpha_\ox \alpha_\oa = 1$ we have $\alpha_\ox = \alpha_\oa^{-1}$. The other combinations of rules are used in similar fashion.
\item{$(iii) \Rightarrow (iv)$:}
If $\alpha_\ox = \alpha_\oa^{-1}$, then 

$(\alpha_\oa \ox \alpha_\oa) m_\ox^F = m_\ox^G \alpha_\ox: G(A) \ox G(B) \to F(A \ox B)$
\[
\begin{tikzpicture}
	\begin{pgfonlayer}{nodelayer}
		\node [style=none] (0) at (-2, 1.25) {};
		\node [style=none] (1) at (-2, 0.25) {};
		\node [style=none] (2) at (0.25, 0.25) {};
		\node [style=none] (3) at (0.25, 1.25) {};
		\node [style=ox] (4) at (-1, 0.75) {};
		\node [style=circle] (5) at (-1.75, 2) {$\alpha_\oa$};
		\node [style=circle] (6) at (-0.25, 2) {$\alpha_\oa$};
		\node [style=none] (7) at (-1.75, 2.75) {};
		\node [style=none] (8) at (-0.25, 2.75) {};
		\node [style=none] (9) at (-1, -1.25) {};
		\node [style=none] (10) at (0, 0.5) {};
		\node [style=none] (11) at (-1.25, 0.25) {};
		\node [style=none] (12) at (-0.75, 0.25) {};
		\node [style=none] (15) at (0, 0.5) {$F$};
	\end{pgfonlayer}
	\begin{pgfonlayer}{edgelayer}
		\draw (7.center) to (5);
		\draw (8.center) to (6);
		\draw (0.center) to (3.center);
		\draw (3.center) to (2.center);
		\draw (2.center) to (1.center);
		\draw (1.center) to (0.center);
		\draw [bend right=15, looseness=1.00] (5) to (4);
		\draw [bend right=15, looseness=1.00] (4) to (6);
		\draw (4) to (9.center);
		\draw [bend left=75, looseness=1.00] (11.center) to (12.center);
	\end{pgfonlayer}
\end{tikzpicture} = \begin{tikzpicture} %mon2
	\begin{pgfonlayer}{nodelayer}
		\node [style=none] (0) at (-2, 1.25) {};
		\node [style=none] (1) at (-2, 0.25) {};
		\node [style=none] (2) at (0.25, 0.25) {};
		\node [style=none] (3) at (0.25, 1.25) {};
		\node [style=ox] (4) at (-1, 0.75) {};
		\node [style=circle] (5) at (-1.75, 2) {$\alpha_\oa$};
		\node [style=circle] (6) at (-0.25, 2) {$\alpha_\oa$};
		\node [style=none] (7) at (-1.75, 2.75) {};
		\node [style=none] (8) at (-0.25, 2.75) {};
		\node [style=none] (9) at (-1, -2.5) {};
		\node [style=none] (10) at (0, 0.5) {};
		\node [style=none] (11) at (-1.25, 0.25) {};
		\node [style=none] (12) at (-0.75, 0.25) {};
		\node [style=circle] (13) at (-1, -0.5) {$\alpha_\ox$};
		\node [style=circle] (14) at (-1, -1.5) {$\alpha_\oa$};
		\node [style=none] (15) at (0, 0.5) {$F$};
	\end{pgfonlayer}
	\begin{pgfonlayer}{edgelayer}
		\draw (7.center) to (5);
		\draw (8.center) to (6);
		\draw (0.center) to (3.center);
		\draw (3.center) to (2.center);
		\draw (2.center) to (1.center);
		\draw (1.center) to (0.center);
		\draw [bend right=15, looseness=1.00] (5) to (4);
		\draw [bend right=15, looseness=1.00] (4) to (6);
		\draw [bend left=75, looseness=1.00] (11.center) to (12.center);
		\draw (4) to (13);
		\draw (13) to (14);
		\draw (14) to (9.center);
	\end{pgfonlayer}
\end{tikzpicture} =
\begin{tikzpicture} %mon3
	\begin{pgfonlayer}{nodelayer}
		\node [style=none] (0) at (-2, 0.25) {};
		\node [style=none] (1) at (-2, -0.75) {};
		\node [style=none] (2) at (0.25, -0.75) {};
		\node [style=none] (3) at (0.25, 0.25) {};
		\node [style=ox] (4) at (-1, -0.25) {};
		\node [style=circle] (5) at (-1.75, 2) {$\alpha_\oa$};
		\node [style=circle] (6) at (-0.25, 2) {$\alpha_\oa$};
		\node [style=none] (7) at (-1.75, 2.75) {};
		\node [style=none] (8) at (-0.25, 2.75) {};
		\node [style=none] (9) at (-1, -2.5) {};
		\node [style=none] (10) at (0, -0.5) {};
		\node [style=none] (11) at (-1.25, -0.75) {};
		\node [style=none] (12) at (-0.75, -0.75) {};
		\node [style=circle] (13) at (-1, -1.5) {$\alpha_\oa$};
		\node [style=circle] (14) at (-1.75, 1) {$\alpha_\ox$};
		\node [style=circle] (15) at (-0.25, 1) {$\alpha_\ox$};
		\node [style=none] (16) at (0, -0.5) {$G$};
	\end{pgfonlayer}
	\begin{pgfonlayer}{edgelayer}
		\draw (7.center) to (5);
		\draw (8.center) to (6);
		\draw (0.center) to (3.center);
		\draw (3.center) to (2.center);
		\draw (2.center) to (1.center);
		\draw (1.center) to (0.center);
		\draw [bend left=75, looseness=1.00] (11.center) to (12.center);
		\draw (13) to (9.center);
		\draw (5) to (14);
		\draw (6) to (15);
		\draw [bend right, looseness=1.00] (14) to (4);
		\draw [bend right, looseness=1.00] (4) to (15);
		\draw (4) to (13);
	\end{pgfonlayer}
\end{tikzpicture} =
\begin{tikzpicture} %mon4
	\begin{pgfonlayer}{nodelayer}
		\node [style=none] (0) at (-2, 1.25) {};
		\node [style=none] (1) at (-2, 0.25) {};
		\node [style=none] (2) at (0.25, 0.25) {};
		\node [style=none] (3) at (0.25, 1.25) {};
		\node [style=ox] (4) at (-1, 0.75) {};
		\node [style=none] (5) at (-1.75, 2.75) {};
		\node [style=none] (6) at (-0.25, 2.75) {};
		\node [style=none] (7) at (-1, -2.5) {};
		\node [style=none] (8) at (-1.25, 0.25) {};
		\node [style=none] (9) at (-0.75, 0.25) {};
		\node [style=circle] (10) at (-1, -1.5) {$\alpha_\oa$};
	\end{pgfonlayer}
	\begin{pgfonlayer}{edgelayer}
		\draw (0.center) to (3.center);
		\draw (3.center) to (2.center);
		\draw (2.center) to (1.center);
		\draw (1.center) to (0.center);
		\draw [bend left=75, looseness=1.00] (8.center) to (9.center);
		\draw (10) to (7.center);
		\draw [bend right=15, looseness=1.00] (5.center) to (4);
		\draw [bend right=15, looseness=1.00] (4) to (6.center);
		\draw (4) to (10);
	\end{pgfonlayer}
\end{tikzpicture}
\]
$m_\top^F = m_\top^G \alpha_\oa: \top \to F(\top)$
\[
\begin{tikzpicture}
	\begin{pgfonlayer}{nodelayer}
		\node [style=circle] (0) at (1.25, 2.25) {$\top$};
		\node [style=none] (1) at (2, 1.5) {};
		\node [style=none] (2) at (2, 3) {};
		\node [style=none] (3) at (0.5, 3) {};
		\node [style=none] (4) at (0.5, 1.5) {};
		\node [style=circle, scale=0.5] (5) at (1.25, 1) {};
		\node [style=circle] (6) at (-0.75, 3.25) {$\top$};
		\node [style=none] (7) at (1.5, 1.5) {};
		\node [style=none] (8) at (1, 1.5) {};
		\node [style=none] (9) at (1.25, -2) {};
		\node [style=none] (10) at (1.75, 1.75) {};
		\node [style=none] (11) at (1.75, 1.75) {$F$};
		\node [style=none] (12) at (-0.75, 4.25) {};
	\end{pgfonlayer}
	\begin{pgfonlayer}{edgelayer}
		\draw (1.center) to (2.center);
		\draw (2.center) to (3.center);
		\draw (3.center) to (4.center);
		\draw (4.center) to (1.center);
		\draw (5) to (0);
		\draw [in=-90, out=180, looseness=1.25, dotted] (5) to (6);
		\draw [bend right=75, looseness=1.25] (7.center) to (8.center);
		\draw (5) to (9.center);
		\draw (6) to (12.center);
	\end{pgfonlayer}
\end{tikzpicture}=
\begin{tikzpicture} %mon7b
	\begin{pgfonlayer}{nodelayer}
		\node [style=circle] (0) at (1.25, 2.25) {$\top$};
		\node [style=none] (1) at (2, 1.5) {};
		\node [style=none] (2) at (2, 3) {};
		\node [style=none] (3) at (0.5, 3) {};
		\node [style=none] (4) at (0.5, 1.5) {};
		\node [style=circle, scale=0.5] (5) at (1.25, 1) {};
		\node [style=circle] (6) at (-0.75, 3.25) {$\top$};
		\node [style=circle] (7) at (1.25, 0.25) {$\alpha_\ox$};
		\node [style=none] (8) at (1.5, 1.5) {};
		\node [style=none] (9) at (1, 1.5) {};
		\node [style=circle] (10) at (1.25, -1) {$\alpha_\oa$};
		\node [style=none] (11) at (1.75, 1.75) {$F$};
		\node [style=none] (12) at (1.25, -2) {};
		\node [style=none] (13) at (-0.75, 4.25) {};
	\end{pgfonlayer}
	\begin{pgfonlayer}{edgelayer}
		\draw (1.center) to (2.center);
		\draw (2.center) to (3.center);
		\draw (3.center) to (4.center);
		\draw (4.center) to (1.center);
		\draw (5) to (0);
		\draw [in=-90, out=180, looseness=1.25, dotted] (5) to (6);
		\draw [bend right=75, looseness=1.25] (8.center) to (9.center);
		\draw (10) to (7);
		\draw (7) to (5);
		\draw (10) to (12.center);
		\draw (6) to (13.center);
	\end{pgfonlayer}
\end{tikzpicture} =
\begin{tikzpicture} %mon8b
	\begin{pgfonlayer}{nodelayer}
		\node [style=circle] (0) at (1.25, 3) {$\top$};
		\node [style=none] (1) at (2, 2.25) {};
		\node [style=none] (2) at (2, 3.75) {};
		\node [style=none] (3) at (0.5, 3.75) {};
		\node [style=none] (4) at (0.5, 2.25) {};
		\node [style=circle, scale=0.5] (5) at (1.25, 1.75) {};
		\node [style=circle] (6) at (-0.75, 4) {$\top$};
		\node [style=none] (7) at (1.25, -1) {};
		\node [style=none] (8) at (1.5, 2.25) {};
		\node [style=none] (9) at (1, 2.25) {};
		\node [style=circle] (10) at (1.25, -0.25) {$\alpha_\oa$};
		\node [style=none] (11) at (1.75, 2.5) {$G$};
		\node [style=none] (12) at (-0.75, 5) {};
	\end{pgfonlayer}
	\begin{pgfonlayer}{edgelayer}
		\draw (1.center) to (2.center);
		\draw (2.center) to (3.center);
		\draw (3.center) to (4.center);
		\draw (4.center) to (1.center);
		\draw (5) to (0);
		\draw [dotted, in=-90, out=180, looseness=1.25] (5) to (6);
		\draw [bend right=75, looseness=1.25] (8.center) to (9.center);
		\draw (7.center) to (10);
		\draw (10) to (5);
		\draw (6) to (12.center);
	\end{pgfonlayer}
\end{tikzpicture}
\]
Thus, $\alpha_\ox$ is comonoidal. Similarly, it can be proven that $\alpha_\oa$ is monoidal. The axioms {\bf \small [LT.4] (a)}-{\bf \small (d)} 
for a linear transformation are satisfied for $(\alpha_\oa, \alpha_\ox)$ because $\alpha_\oa = \alpha_\ox^{-1}$. 

\item{$(iv) \Rightarrow (i)$ and $(ii)$:} The axioms {\bf \small [nat.1]} and {\bf \small [nat.2]} are given by the fact 
that $(\alpha_\oa, \alpha_\ox)$ is a linear transformation. 
%\item[(b) $\Rightarrow$ (d):] Given that $\alpha_\ox = \alpha_\oa^{-1}$. It has been already shown that $(\alpha_\oa, \alpha_\ox)$ is a linear transformation. Thereby, $(\alpha_\ox, \alpha_\oa)^{-1} = (\alpha_\oa, \alpha_\ox)$.
\end{description}
\end{proof}

Frobenius functors between isomix categories are especially important in the development of dagger 
linearly distributive categories and they often satisfy an additional property:

\begin{definition}
A Frobenius functor between isomix categories is an {\bf isomix functor} in case it is a mix functor which satisfies, in addition, the following diagram:
\[ \mbox{\bf{[isomix-FF]}}~~~~~\xymatrix{ \top \ar@/^/[rrr]^{{\sf m}^{-1}} \ar[dr]_{m_\top} & & & \bot \\ & F(\top) \ar[r]_{F({\sf m}^{-1})} & F(\bot) \ar[ur]_{n_\top} } \]
\end{definition}

Recall that a linear functor is {\bf normal} in case both $m_\top$ and $n_\bot$ are isomorphisms.  We observe:

\begin{lemma} 
\label{Lemma: isomix functor}
For a mix Frobenius functor, $F: \X \to \Y$, between isomix categories the following are equivalent:
\begin{enumerate}[(i)]
\item $n_\bot: F(\bot) \to \bot$  or $m_\top: \top \to F(\top)$ is an isomorphism;
\item $F$ is a normal functor;
\item $F$ is an isomix functor.
\end{enumerate}
\end{lemma}

\begin{proof}~
\begin{description}
\item{$(i) \Rightarrow (ii)$:}  Note that, as $F$ is a mix functor $F({\sf m}) = n_\bot {\sf m}~m_\top$.  As the mix map ${\sf m}$ is an isomorphism so is $F({\sf m})$ which implies that if  $n_\bot$ is an isomorphism then $m_\top$ must be an isomorphism and vice versa.  Thus, $F$ will be a normal functor. 
\item{$(ii) \Rightarrow (iii)$:} If $F$ is normal then $n_\bot$ and $m_\top$ are isomorphisms and so 
\[ \infer={m_\top F({\sf m}^{-1}) n_\bot = {\sf m}^{-1}}{\infer={F({\sf m}^{-1}) = m_\top^{-1} {\sf m}^{-1} n_\bot^{-1}}{F({\sf m}) = n_\bot {\sf m}~m_\top}} \]
\item{$(iii) \Rightarrow (i)$:} The mix-preservation for $F$ makes  $n_\bot$ a section (and $m_\top$ a retraction) while the isomix-preservation makes $m_\bot$ a retraction (and $m_\top$ a section. 
This means $n_\bot$ is an isomorphism ($m_\top$ is an isomorphism).                                                                                                                                                                                                                                                                                                                                                                                                                                                                         
\end{description}
\end{proof}


\begin{corollary} \label{Corollary: normal-nat-iso}
$\alpha := (\alpha_\ox, \alpha_\oa)$ is a linear natural isomorphism between isomix Frobenius linear functors if and only if $\alpha_\ox = \alpha_\oa^{-1}$.
\end{corollary}

\begin{proof}
Note  that if we can establish {\bf [nat.1](a)} or {\bf (b)} then we can prove that $\alpha_\ox\alpha_\oa =1$ and, as $\alpha_\ox$ is an isomorphism it follows that $\alpha_\oa\alpha_\ox =1$.
Thus, it suffices to show that {\bf [nat.1](a)} holds:
\[ m_\top \alpha_\oa G({\sf m}^{-1}) n_\bot = m_\top F({\sf m}^{-1})\alpha_\oa  n_\bot = m_\top F({\sf m}^{-1})  n_\bot = {\sf m}^{-1} = m_\top G({\sf m}^{-1}) n_\bot  \]
However, as $G({\sf m}^{-1}) n_\bot$ is an isomorphism, it follows that $m_\top \alpha_\oa = m_\top$.
\end{proof}

% rewrite this! - done

Lemma \ref{Lemma: Frobenius linear transformation} and Corollary \ref{Corollary: normal-nat-iso} are generalizations of \cite[Proposition 7]{DP08}. \cite[Proposition 7]{DP08} states the following:

Let $\X$ and $\Y$ be monoidal categories and $(\eta, \epsilon): A \dashv B \in \X$. If $F,G: \X \to \Y$ are Frobenius monoidal functors with a natural transformation $\alpha: F \Rightarrow G$ which is both monoidal and comonoidal, then $\alpha_A$ is invertible. 

In Lemma \ref{Lemma: Frobenius linear transformation}, when $A \dashvv B \in \X$, then $\alpha_\oa$ is defined as follows:
\[
\alpha_\oa: G(A) \to F(A) = \begin{tikzpicture}
	\begin{pgfonlayer}{nodelayer}
		\node [style=circle] (0) at (-0.25, -0) {$\alpha_\ox$};
		\node [style=none] (1) at (1.25, -2.75) {};
		\node [style=none] (2) at (-0.25, -1) {};
		\node [style=none] (3) at (-1.75, -1) {};
		\node [style=none] (4) at (-1.75, 2.75) {};
		\node [style=circle] (5) at (-1, -1.75) {$\epsilon$};
		\node [style=none] (6) at (-0.25, 1) {};
		\node [style=none] (7) at (1.25, 1) {};
		\node [style=circle] (8) at (0.5, 1.75) {$\eta$};
		\node [style=none] (9) at (1.5, 2.5) {};
		\node [style=none] (10) at (1.5, 1) {};
		\node [style=none] (11) at (-0.5, 1) {};
		\node [style=none] (12) at (-0.5, 2.5) {};
		\node [style=none] (13) at (-2, -0.75) {};
		\node [style=none] (14) at (0, -0.75) {};
		\node [style=none] (15) at (0, -2.5) {};
		\node [style=none] (16) at (-2, -2.5) {};
		\node [style=none] (17) at (-2.25, 2.5) {$G(A)$};
		\node [style=none] (18) at (1.75, -2.5) {$F(A)$};
		\node [style=none] (19) at (0.25, 0.65) {$F(B)$};
		\node [style=none] (20) at (0.5, -0.5) {$G(B)$};
		\node [style=none] (21) at (1.25, 2.25) {$F$};
		\node [style=none] (22) at (-0.25, -2.25) {$G$};
	\end{pgfonlayer}
	\begin{pgfonlayer}{edgelayer}
		\draw (0) to (2.center);
		\draw (3.center) to (4.center);
		\draw [in=0, out=-90, looseness=1.25] (2.center) to (5);
		\draw [in=180, out=-90, looseness=1.25] (3.center) to (5);
		\draw (11.center) to (10.center);
		\draw (12.center) to (11.center);
		\draw (10.center) to (9.center);
		\draw (9.center) to (12.center);
		\draw (7.center) to (1.center);
		\draw [in=90, out=0, looseness=1.25] (8) to (7.center);
		\draw (6.center) to (0);
		\draw [in=180, out=90, looseness=1.25] (6.center) to (8);
		\draw (13.center) to (16.center);
		\draw (16.center) to (15.center);
		\draw (15.center) to (14.center);
		\draw (14.center) to (13.center);
	\end{pgfonlayer}
\end{tikzpicture}
\]

For these special linear isomorphisms with $\alpha_\ox = \alpha_\oa^{-1}$ we can simplify the coherence requirements:

\begin{lemma} \label{simplifying-coherences}
Suppose $F$ and  $G$ are Frobenius functors and $\alpha: F \to G$ is a natural isomorphism then:
\begin{enumerate}[(i)]
\item If $\alpha: F \to G$ is $\ox$-monoidal and $\oa$-comonoidal then $(\alpha,\alpha^{-1})$ is a linear transformation;
\item If $F$ and $G$ are strong Frobenius functors and $\alpha$ is $\ox$-monoidal and $\oa$-monoidal then $(\alpha,\alpha^{-1})$ is a linear transformation.
\end{enumerate}
\end{lemma}

\begin{proof}~
\begin{enumerate}[{\em (i)}]
\item If $\alpha$ is $\ox$-monoidal and $\oa$-comonoidal then so is $\alpha^{-1}$ supporting the possibility that it is a component of a linear transformation.
Considering  {\bf [LT.1]} we show that $(\alpha,\alpha^{-1})$ satisfies this requirement as:
\[ \xymatrix{ F(A \oa B) \ar[rr]^{\alpha_\ox}  \ar[d]_{n_\oa = \nu^R_\ox} & & G(A \oa B) \ar[d]^{n_\oa = \nu^R_\ox} \\
                    F(A) \oa F(B) \ar[dr]_{1 \oa \alpha} \ar[rr]^{\alpha \oa \alpha} & & G(A) \oa G(B)  \ar[dl]^{\alpha^{-1} \oa 1} \\
                    & F(A) \oa G(B) } \]
The remaining requirements follow similarly.   
\item  When the laxors for the functors are isomorphisms then being monoidal implies being comonoidal.   \qedhere
\end{enumerate}
\end{proof}

%%%%%%%%%%%%%%%%%%%%%%%%%%%%%%%%%%%%%%%%%%%%%%%%

\section{Motivating examples}
\label{Sec: motivating examples}

In Section \ref{Sec: LDC}, we listed a few examples of linearly distributive categories.
In this section, we present the categories of finiteness relations, ${\sf FRel}$ and finiteness matrices
over a commutative rig $R$, $\FMat(R)$, and discuss their properties. These isomix categories are closely related 
to categorical quantum mechanics because the core of these categories are used as the standard 
categorical models to study quantum processes in finite dimensions: the core of $\FMat(R)$ is equivalent 
to the category finite dimensional matrices over a commutative rig $R$, and the core of $\FRel$ is the category of finite sets and 
relations.  

The categories $\FRel$ and $\FMat(R)$ are used as running examples 
throughout this thesis. Consequently, the purpose of this section 
is to revisit the results of \cite{Ehr05} in which these categories were introduced.

\subsection{Finiteness relations ${\sf FRel}$ and finiteness matrices ${\sf FMat}(R)$}

Finiteness spaces were introduced by Ehrhard \cite{Ehr05} as a categorical model of Girard's linear logic \cite{Gir87}.  
A finiteness space $(X,{\cal A},{\cal B})$ may be regarded as a set, $X$, called the {\bf web} of $X$, equipped 
with two sets of subsets, ${\cal A} \subseteq {\cal P}(X)$ the {\bf finitary} sets of $X$, and ${\cal B} \subseteq {\cal P}(X)$ 
the {\bf cofinitary} sets of $X$.  The finitary and cofinitary sets must be finiteness complements of each other 
(as described below) and so determine each other.  This means that a finiteness space is often presented 
with just the finitary sets.  Morphisms of a finiteness spaces are relations between the webs that preserve 
the finitary sets and reflect the cofinitary sets  The category of finiteness spaces with finiteness relations is 
an isomix $*$-autonomous category.  In this section, we aim to give a reasonably self-contained account of $\FRel$, 
the category of finitary relations, and of $\FMat(R)$ the category of finiteness matrices with entries in a commutative rig $R$, 
and to establish that these categories are symmetric $\dagger$-$*$-isomix categories. 

Let $X$ be any set and $A, B \subseteq X$ then $A$ is {\bf finiteness orthogonal} \cite{Ehr05}
\cite[Section 1]{Ehr05} to $B$, written $A \perp_f B$, in case $A \cap B$ is finite.   
Given a set of subsets of $X$, ${\cal A} \subseteq {\cal P}(X)$ set,
\[ {\cal A}^\perp = \{ B \subseteq X \mid \forall A \in {\cal A}, B \perp_f A\} \]
Thus ${\cal A}^\perp$ is the set of all subsets of $X$ which intersects with all the sets in ${\cal A}$ finitely.  Observe that:
 
 \begin{lemma} For ${\cal A}, {\cal B} \subset {\cal P}(X)$:
\begin{enumerate}[(i)]
\item ${\cal A} \subseteq {\cal B} \Rightarrow {\cal B}^{\perp} \subseteq {\cal A}^{\perp}$;
\item ${\cal A} \subseteq {\cal A}^{\perp \perp}$;
\item ${\cal A}^{\perp \perp \perp} = {\cal A}^{\perp}$;
\item ${\cal A}^\perp$ is downset closed and closed under finite unions;
\item if ${\cal A} = {\cal A}^{\perp\perp}$ then ${\cal P}_f(X) \subseteq {\cal A}$ (that is ${\cal A}$ contains all finite subsets);
\item If ${\cal A}_i= {\cal A}_i^{\perp\perp}$ for $i \in I$ then $\bigcap_{i \in I} {\cal A}_i = (\bigcup_{i \in I} {\cal A}^\perp_i)^{\perp}$ \\
(so $\bigcap_{i \in I} {\cal A}_i =  (\bigcap_{i \in I} {\cal A}_i)^{\perp\perp}$).
 \end{enumerate}
 \end{lemma}
 
The first two observations establish a Galois connection on the subsets of ${\cal P}(X)$ from which the next observation is standard.  The last two are easy consequences of the form of the finiteness orthogonality.
 
\begin{definition}
A {\bf finiteness space} is a triple $(X, {\cal A},{\cal B})$ where $X$ is a set, 
called the {\bf web} of the space, and ${\cal A}, {\cal B}  \subseteq {\cal P}(X)$, 
such that the two subsets are finiteness complements, that is 
${\cal A}^\perp = {\cal B}$ and ${\cal B}^\perp = {\cal A}$.   
Elements of ${\cal A}$ are the {\bf finitary} sets of the finiteness space, 
while elements of ${\cal B}$ are called the {\bf cofinitary} sets. 
\end{definition}

Because the cofinitary sets are completely determined by the finitary sets it is often convenient 
to write a finiteness space as $(X,F(X),F(X)^\perp)$, 
making clear that the structure is completely determined by the web and the finitary sets.  
The finitary and cofinitary sets are downward closed, that is $F(X) = {\downarrow}F(X)$, 
and are closed under arbitrary intersections and finite unions.  As finite sets always intersect 
finitely both with the finitary and cofinitary sets, the finitary and cofinitary sets of every 
finiteness space must always include all the finite subsets.  In particular this means that a 
finiteness set with a finite web must have every subset both finitary and cofinitary.  Furthermore, 
every set, $X$, always carries two ``trivial'' finiteness structures (which coincide for a finite set): 
$(X,{\cal P}_f(X),{\cal P}(X))$ and  $(X,{\cal P}(X),{\cal P}_f(X))$, where ${\cal P}_f(X)$ 
is the set of finite subsets of $X$. On the other hand, as we now 
observe, an infinite set always has infinitely many non-trivial finiteness structures: 

\begin{lemma}~
\label{Lemma: finite finite}
\begin{enumerate}[(i)]
\item  A finiteness space, $(X, F(X), F(X)^\perp)$, with a finite web, 
has its finitary and cofinitary sets completely determined to be 
$F(X) = F(X)^\perp = {\cal P}(X) = {\cal P}_f(X)$;
\item  Every infinite set carries infinitely many non-trivial finiteness structures;
\item The set of finiteness structures carried by $X$ form a complete (non-distributive) lattice with $(X,{\cal A},{\cal B}) \leq (X,{\cal A}',{\cal B}')$ if and only if ${\cal A} \subseteq {\cal A}'$ 
(equivalently ${\cal B}' \subseteq {\cal B}$).  The lattice structure is given by:
\begin{align*}
\top & := (X,{\cal P}(X),{\cal P}_f(X)) &
\bot & := (X,{\cal P}_f(X),{\cal P}(X)) \\
\bigwedge_i A_i & := \left(X, \bigcap_i {\cal A}_i, \left(\bigcup_i {\cal A}_i \right)^{\perp\perp} \right) &
\bigvee_{i \in I} A_i & := \left(X, \left(\bigcup_i {\cal A}_i \right)^{\perp\perp}, \bigcap_i {\cal B}_i \right) 
\end{align*}
\end{enumerate}
\end{lemma}

\begin{proof}~
\begin{enumerate}[{\em (i)}]
\item This is immediate.
\item There are, of course, many different possible finiteness structures for an infinite set: to see this consider the finiteness space $(X,\{ A\}^{\perp\perp},\{ A \}^\perp)$, generated by insisting that an infinite subset, $A \subseteq X$, whose complement, $\neg A$, is also infinite, is finitary.  Notice that, in this case:
\[ \{ A \}^\perp = \{ Y \mid Y \cap A~\mbox{is finite}\} = \{ Y \mid Y \subseteq \neg A \cup F~\mbox{where $F$ is finite}\} \]
\[ \{ A\}^{\perp\perp} = \{ X \mid X \subset A \cup F~\mbox{where $F$ is finite}\} \]
It is then sufficient to show that there are infinitely many of these sets which are distinct by more that a finite set:  for this note that having chosen an infinite set one can split it again 
recursively in the same manner.
\item It suffices to show that the two sets $\bigwedge_{i \in I} {\cal A}_i$ and $\bigvee_{i \in I}{\cal B}_i$ are complementary.  Note that each set $B_i \in {\cal B}_i$ is in $\bigvee_{i \in I}{\cal B}_i$ so any set in the complement must be in the complement of each ${\cal B}_i$ and so must be in $\bigwedge_{i \in I} {\cal A}_i$.  Conversely, a set in $\bigwedge_{i \in I} {\cal A}_i$ is necessarily orthogonal to each finite union $\bigcup_{i \in F} B_i$ as it intersects with each $B_i$ finitely.  Thus, $(\bigvee_{i \in I}{\cal B}_i)^\perp =\bigwedge_{i \in I} {\cal A}_i$.
\end{enumerate}
\end{proof}

To complete the description of the category ${\sf FRel}$, we describe the morphisms between these spaces:
\begin{definition}
A {\bf finiteness relation} $R: (X, F(X),F(X)^\perp) \to (Y, F(Y),F(Y)^\perp)$ between two finiteness spaces, is a relation $X \to^{R} Y$ such that:
\begin{enumerate}[(i)]
\item for all $A \in F(X), ~~ A \triangleright R := \{ y \mid \exists x \in A \text{ such that } xRy \} \in F(Y)$
\item for all $B \in {F(Y)}^\perp, ~~ R \triangleleft B := \{ x \mid \exists y \in B \text{ such that } xRy \} \in F(X)^\perp$ 
\end{enumerate}
\end{definition}

Here we are using a notation which can be translated into allegorical notation, 
see \cite{FrS89}, where it is possible to do most of these proofs more generally 
and so can be made formal for general categories of relations: we, of course, 
will have in mind sets and relations.  In this notation a finitary set in $F(X)$ 
is a subrelation of the identity relation $A \subseteq 1_{X}$ 
while a cofinitary set, $B \in  F(X)^\perp$, is a subrelation $B \subseteq 1_{Y}$.  
The orthogonality relation $A \perp_f A'$ is then the requirement that 
$A \cap A' = AA'$ is finite.  A finiteness relation is a map $R: X \to Y$ with 
$A \triangleright R:= {\sf cod}(AR) \in F(Y)$, for every $A \in F(X)$, and 
$R \triangleleft B:= {\sf dom}(RB) \in F(X)^\perp$ for every $B \in F(Y)^\perp$ 
(recall that ${\sf dom}(R) = 1_{X} \cap RR^\circ$ and ${\sf cod}(R) = 
1_{Y} \cap R^\circ R$, where $R^\circ$ is the converse of $R$).  

Finite relations are determined by having finite domains and codomains: 
so that $R: X \to Y$ is finite if and only if ${\sf cod}(R) \subseteq_f 1_{Y}$ 
and ${\sf dom}(R) \subseteq_f 1_{X}$.  Thus $R$ is a finiteness relation if and 
only if $ARB$ is a finite relation for every $A \in F(X)$ and $B \in F(Y)^\perp$.

Notice that once we require that $A \triangleright R \in F(Y)$ for every 
$A \in F(X)$ then each $B \in F(Y)^\perp$ has a finite intersection with 
$A \triangleright R$ so that 
there are finitely many $b \in (A\triangleright R)\cap B$ and whence
\begin{align*}
 ARB & = AR((A \triangleright R)\cap B)  =  AR\bigcup_{b \in A \triangleright R \cap B} \{ b\} \\
 & =  \bigcup_{b \in A \triangleright R \cap B} AR\{b\}  =  
 \bigcup_{b \in A \triangleright R \cap B} (A \cap R \triangleleft \{b\}) R \{b\} 
 \end{align*}
This means that so long as $R \triangleleft \{b\} \in F(X)^\perp$ each $A \cap R \triangleleft \{b\}$ will be finite, so, as the union over $b \in A \triangleright R \cap B$ is a finite union, this ensures that $ARB$ is a finite relation and, thus, that $R$ is a finiteness relation.

This discussion gives the following important simplifications (see Ehrhard \cite{Ehr05}) of the conditions of being a finiteness relation:

\begin{lemma} \label{character_finiteness_relation}
The following  are equivalent:
\begin{enumerate}[{(i)}]
\item $R: X \to Y$ is a finiteness relation;
\item $ARB$ is a finite relation for all $A \in F(X)$ and $B \in F(Y)^\perp$;
\item $R: X \to Y$ has $A \triangleright R \in F(Y)$ for all $A \in F(X)$ and 
for every $y \in Y$, $R \triangleleft \{ y \} \in F(X)^\perp$;
\item $R: X \to Y$ has $R \triangleleft B \in F(X)^\perp$ for all 
$B \in F(Y)^\perp$ and for every $x \in X$, $\{ x \} \triangleright R \in F(Y)$.
\end{enumerate}
\end{lemma}

\begin{proof}
The above discussion proves that {\em (i)\/} is equivalent to {\em (ii)\/}. That {\em (ii)\/} implies {\em (iii)\/} is now immediate (by converse symmetry this also gives {\em (ii)\/} implies {\em (iv)\/}).  Finally we have shown above that {\em (iii)} implies {\em (ii)\/} (and by converse symmetry that {\em (iv)\/} implies {\em (ii)\/}).
\end{proof}

The category of finiteness relations, ${\sf FRel}$, has objects finiteness spaces and maps finiteness relations with composition as in the category of relations:

\begin{lemma}
Finiteness spaces with finiteness relations, ${\sf FRel}$,  forms a category with composition 
the usual composition of relations and identities the diagonal relations.
\end{lemma}

\begin{proof}
Suppose $X \to^{R} Y$, and $Y \to^{S} Z$ are finiteness relations.  Then, for $A \in F(X)$ we know $A \triangleright R \in F(Y)$ 
and whence $A\triangleright (RS) = (A \triangleright R) \triangleright S  \in F(Z)$.  Similarly for $B \in F(Z)^\perp$ we have $S \triangleleft B \in F(Y)^\perp$ and, 
therefore, $(RS) \triangleleft B = (R \triangleleft (S \triangleleft B)) \in F(X)^\perp$. 

Clearly the identity relation is always a finiteness relation.
\end{proof}

We record an obvious yet important fact: there is a faithful functor $|\_|: {\sf FRel} \to {\sf Rel}$ 
which takes a finiteness space to its web and a finiteness relation to its underlying relation.  
This functor, we shall see, preserves the structure of ${\sf FRel}$ as an isomix *-autonomous category 
on the nose so that we can use the known coherence properties of ${\sf Rel}$ to 
obtain the corresponding coherence properties of ${\sf FRel}$.

With finiteness relations in hand one can build the category of finiteness matrices, 
${\sf FMat}(R)$, for an arbitrary commutative rig, $R$. While $R$ can be any commutative rig, 
we are most interested in the case when $R$ is the complex numbers, and, thus, in the 
category ${\sf FMat}(\C)$. 

The category ${\sf FMat}(R)$, where $R$ is any commutative rig is defined as follows:
\begin{description}
\item[Objects:]  Finiteness spaces;
\item[Maps:]  Finiteness matrices $M: X \to Y$ where 
$M = [M_{i ,j}]_{i \in X, j \in Y}$ where each $M_{i,j} \in R$ and the 
support, $|M|$, is a finiteness relation, that is $|M| =\{ (i,j) \mid M_{i,j} \neq 0\}$ is a finiteness relation;
\item[Composition:] Matrix multiplication -- where we note all the non-zero sums are finite (see Lemma \ref{Lemma: finiteness matrix} below),  $\sum_j r_{i,j}s_{j,k} = \sum_{j \in \{i\}|r| \cap |s|\{ k \}}r_{i,j}s_{j,k} .$
\item[Identities:]  Identity matrices.
\end{description}

Observe that, although finiteness matrices may be infinite dimensional, to form the composition of two finiteness matrices only requires the ability to multiply a finite number of entries and, thus, to sum only a finite number of elements.  This is because the support of each matrix is always a finiteness relation, thus, for composition one only need compute the products of non-zero elements and these lie in the intersection of a finite and cofinite set, so is finite.  This explicitly is the observation:

\begin{lemma}
\label{Lemma: finiteness matrix}
If $X \to^{R} Y \to^S Z$ is a finiteness relation,  then 
\begin{enumerate}[(i)]
\item for all $x \in X$, $\{ x \} \triangleright R \in F(Y)$, and 
\item for all $y \in Y$, $S \tl \{ z \} \in F(Y)^\perp$
\item for all $x \in X$ and $z \in Z$, $\{ x \} \tr R \perp_f S \tl \{ z \}$, that is $\{ x \} \tr R \cap S \tl \{ z \}$ is finite.
\end{enumerate}
\end{lemma}

In a finiteness matrix $X \to^{M} Y$,  each row of the matrix has support a finitary set of $Y$, and each column has a cofinitary set of $X$ as shown in Figure \ref{Diagram: finiteness matrix}. 

\begin{figure}
\centering
\begin{tikzpicture}[scale=1.5]
	\begin{pgfonlayer}{nodelayer}
		\node [style=none] (0) at (-1.5, 2) {};
		\node [style=none] (1) at (-1.5, -1) {};
		\node [style=none] (2) at (1.25, -1) {};
		\node [style=none] (3) at (1.25, 2) {};
		\node [style=none] (4) at (-0.75, 2) {};
		\node [style=none] (5) at (-0.25, 2) {};
		\node [style=none] (6) at (-0.75, -1) {};
		\node [style=none] (7) at (-0.25, -1) {};
		\node [style=none] (8) at (-1.5, 1) {};
		\node [style=none] (9) at (-1.5, 0.5) {};
		\node [style=none] (10) at (1.25, 1) {};
		\node [style=none] (11) at (1.25, 0.5) {};
		\node [style=none] (12) at (2, 0.75) {$\in F(Y)$};
		\node [style=none] (13) at (-0.5, 2.75) {$\in F(X)^\perp$};
		\node [style=none] (14) at (0.5, 1.5) {$\cdots$};
		\node [style=none] (15) at (-1, 1.5) {$\cdots$};
		\node [style=none] (16) at (0.5, -0) {$\cdots$};
		\node [style=none] (17) at (-1, -0) {$\cdots$};
		\node [style=none] (18) at (-2, 0.5) {$|X|$};
		\node [style=none] (19) at (0, -1.5) {$|Y|$};
		\node [style=none] (20) at (0.5, -1.25) {};
		\node [style=none] (21) at (1.25, -1.25) {};
		\node [style=none] (22) at (-0.5, -1.25) {};
		\node [style=none] (23) at (-1.5, -1.25) {};
		\node [style=none] (24) at (-2, 1) {};
		\node [style=none] (25) at (-2, 2) {};
		\node [style=none] (26) at (-2, -0) {};
		\node [style=none] (27) at (-2, -1) {};
		\node [style=none] (28) at (-0.5, 2.15) {};
		\node [style=none] (29) at (-0.5, 2.45) {};
	\end{pgfonlayer}
	\begin{pgfonlayer}{edgelayer}
		\draw (8.center) to (10.center);
		\draw (9.center) to (11.center);
		\draw (0.center) to (1.center);
		\draw (1.center) to (2.center);
		\draw (3.center) to (2.center);
		\draw (3.center) to (0.center);
		\draw (4.center) to (6.center);
		\draw (7.center) to (5.center);
		\draw [->] (24.center) to (25.center);
		\draw [->] (26.center) to (27.center);
		\draw [->] (22.center) to (23.center);
		\draw [->] (20.center) to (21.center);
		\draw[->] (28.center) to (29.center);
	\end{pgfonlayer}
\end{tikzpicture}
\caption{Finiteness matrix}
\label{Diagram: finiteness matrix}
\end{figure}


\begin{lemma}
Finiteness spaces with finiteness matrices over any commutative rig, $R$, form a category, ${\sf FMat}(R)$.
\end{lemma}

\begin{proof}
We must show that finiteness matrices compose.   We already know that, even though finiteness matrices can be infinite-dimensional, two such matrices can be multiplied using only finite sums of elements because their supports are finiteness relations. However, we still have to show that the product of two finiteness matrices is a finiteness matrix.

 Suppose $X \to^{P} Y$ and $Y \to^{Q} Z$, $[PQ]_{x,z} = \sum_{y \in Y} [P]_{x,y} [Q]_{x,y}$  can only be non-zero if $\exists ~ y \in Y$, such that $[P]_{x,y}$, and $[Q]_{y,z}$ are non-zero, although, even if there exists $y \in Y$, such that $[P]_{x,y}$ and $[Q]_{y,z}$ are non-zero, $[PQ]_{x,z}$ may still be zero. This means the support $|PQ| \subseteq |P||Q|$. Since, any subset of a finitary relation is finitary, $PQ$ is a finiteness matrix.
\end{proof}

The proof shows that taking the support of matrices gives, in general, a colax 2-functor $|\_|: {\sf FMat}(R) \to {\sf FRel}$  as $|MN| \leq |M||N|$.  In general if $f: R \to S$ is a morphism of rigs then we obtain a functor ${\sf FMat}(f): {\sf FMat}(R) \to {\sf FMat}(S)$: if $S$ is an ordered rig and $f$ is suitably colax then ${\sf FMat}(f)$ is a colax functor.  Recall that the two element lattice $\mathbbm{2}$ with join as addition and meet as multiplication has ${\sf FMat}(\mathbbm{2}) = {\sf FRel}$. 


%%%%%%%%%%%%%%%%%%%%%%%%%%%%%%%%%%%%%%%%%%%%%%%%%%%%%%%%%%%%%%%%%%%%%%%%%%%%%
\subsection{Products, coproducts and biproducts in ${\sf FRel}$ and ${\sf FMat}(R)$}

We now begin to explore the properties of ${\sf FRel}$, and $\FMat(R)$.  First we observe that there is an obvious involution on ${\sf FRel}$ given by swapping the finitary and cofinitary sets:
\[ (\_)^{*}: {\sf FRel}^{\rm op} \to {\sf FRel}; \begin{matrix} \xymatrix{ (X,{\cal A},{\cal B}) \ar@{|->}[rr] \ar[dd]_R  & & X^{*} = (X,{\cal B},{\cal A})  \\ 
                                                                                                                ~~~~~~~~~~~~ \ar@{|->}[rr] & &~~~~~~~~~~~ \\
                                                                                                                (X',{\cal A}',{\cal B}' ) \ar@{|->}[rr] & & {X'}^{*} = (X',{\cal B}',{\cal A}' ) \ar[uu]_{R^\circ} } \end{matrix} \] 
For ${\sf FMat}(R)$ this involution is given by transposing the matrix:
\[ (\_)^{*}: {\sf FMat}(R)^{\rm op} \to {\sf Mat}(R); \begin{matrix} \xymatrix{ (X,{\cal A},{\cal B}) \ar@{|->}[rr] \ar[dd]_{\left(r_{i,j}\right)_{i \in X,j \in X'}} & & X^{*} = (X,{\cal B},{\cal A})  \\ 
                                                                                                                ~~~~~~~~~~~~ \ar@{|->}[rr] & &~~~~~~~~~~~ \\
                                                                                                                (X',{\cal A}',{\cal B}' ) \ar@{|->}[rr] & & {X'}^{*} = (X',{\cal B}',{\cal A}' ) \ar[uu]_{\left(r_{j,i}\right)_{j \in X', i \in X}} } \end{matrix} \] 
This involution clearly has the property that   $(X^*)^* = X$.  This symmetry means that  we can take a ``one-sided'' approach to establishing structures using the involution to automatically produce the dual structure.  Shortly we will see that this involution is actually the duality of the $*$-autonomous structure of both categories.                                                                                                   

The first observation is:

\begin{lemma}
\label{Lemma: prod and coprod}
 ${\sf FRel}$ and ${\sf FMat}(R)$, for any commutative rig $R$, have arbitrary products and coproducts.
\end{lemma}

\begin{proof}
Let $(X_i, F(X_i), F(X_i)^\perp)$ be a set of finiteness spaces. 
We shall work for the coproduct as the argument for the product is dual.  The coproduct  $\coprod_{i \in I} X_i$ of finiteness spaces 
has its web given by the disjoint union of the webs with injections 
$\sigma_i: X_i \to \sqcup_{i \in I} X_i$.  
Its finitary sets are given by ${\cal A} = 
\{ \bigcup_{i \in I'} A_i\sigma_i \mid A_i \in F(X_i) ~\&~ I' \subseteq_f I \}$ 
while the cofinitary sets are given by ${\cal B} = \{ \bigcup_{i \in I} B_i \sigma_i \mid B_i \in F(X_i)^\perp \}$. 
First note that the injections $\sigma_i: X_i \to \sqcup_{i \in I} X_i$ are certainly finiteness relations.  As a relation the comparison map for a family of relations, $R_i: X_i \to Y$, is unique 
and given by $\bigcup_{i \in I} \sigma_i^\circ R_i : \sqcup_{i \in I} X_i \to Y$: we must check this is a finiteness relation assuming each $R_i$ is.  Note that if $B \in F(Y)^\perp$ then 
\[ B \left( \bigcup_{i \in I} \sigma_i^\circ R_i \right)^\circ = \bigcup_{i \in I} B R_i^\circ \sigma_i \]
where $B R_i^\circ \in F(X_i)^\perp$ so this is in ${\cal B}$.  Conversely, given $\bigcup_{i \in I'} A_i\sigma_i \in {\cal A}$ mapping this forward gives:
\[ \left(\bigcup_{i \in I'} A_i\sigma_i \right) \left( \bigcup_{i \in I} \sigma_i^\circ R_i \right) = \bigcup_{i \in I'} A_i R_i \]
which is a finite union of finitary sets of $Y$ and so a finitary set.

Finally, we must check that, as we have defined it, $\coprod_{i \in I} X_i$ is a finiteness set, that is ${\cal A}^\perp = {\cal B}$ and ${\cal B}^\perp = {\cal A}$.  Toward this end 
suppose $H \subseteq \bigsqcup_{i \in I} X_i$ is orthogonal to ${\cal B}$ then, 
restricting to each $X_i$, $H \sigma_i^\circ \cap A_i$ is finite for each $A_i \in F(X_i)^\perp$ and 
there can only be finitely $i$ for which this is non-empty.  
But this means $H \in {\cal A}$.  Conversely, suppose $K \subseteq \bigsqcup_{i \in I} X_i$ is orthogonal to ${\cal A}$ then 
restricting to $X_i$ means that each $K \sigma_i^\circ \in F(X_i)^\perp$ which means 
$K = \bigcup_{i \in I} B_i \sigma$ and so is in ${\cal B}$.

For ${\sf FMat}(R)$ the coproduct has the same underlying finiteness space: it is then straightforward to check that all the required maps have the appropriate support.
\end{proof}

Note that in both ${\sf FRel}$ and ${\sf FMat}(R)$ finite products are the same as finite coproducts. 
Thus, both categories have biproducts and so are additively enriched (that is, they are enriched in commutative monoids). 

\begin{corollary}
Both ${\sf FRel}$ and ${\sf FMat}(R)$ have (finite) biproducts.
\end{corollary}

%We denote the biproduct of two objects by $A \bip B$ as explained earlier.   
In ${\sf FRel}$ the additive enrichment is given by the union of the underlying relations.  In ${\sf FMat}(R)$ the additive enrichment is given by the pointwise addition of matrices.

%%%%%%%%%%%%%%%%%%%%%%%%%%%%%%%%%%%%%%%%%%%%%%%%%%%%%%%%%%%%%%%%%%%%%%%%%%%%%

\subsection{The tensor structure in ${\sf FRel}$ and ${\sf FMat}(R)$}

${\sf FRel}$ has a symmetric tensor product and this can be used to obtain a corresponding tensor product on ${\sf FMat}(R)$.   We shall start by focusing on ${\sf FRel}$ and define 
$X \ox Y := (X \times Y, F(X \ox Y), F(X \ox Y)^\perp)$, where
\[
F(X \ox Y) := {\downarrow} \{ A \x B \mid  A \in F(X), B \in F(Y) \}
\]
where ${\downarrow} {\cal A}:= \{ A' \mid A' \subseteq A, A \in {\cal A} \}$ is the downward closure of the set of subsets ${\cal A}$.  To show this is well-defined we must prove  that $F(X \ox Y) = F(X \ox Y)^{\perp\perp}$ (this is essentially \cite[Lemma 2]{Ehr05}): 

\begin{lemma} \label{Lemma: tensor characterization}
For any finiteness spaces $X$ and $Y$,  $X \ox Y = (X,F(X \ox Y),F(X \ox Y)^\perp)$, as defined above, is a well-defined finiteness space.
\end{lemma}

\begin{proof}
We must show that any $Q \subseteq X \x Y$ which intersects finitely with any $P \subseteq X \x Y$ 
which, in turn, intersects  finitely with all $R \in F(X \ox Y)$ must already be in 
$F(X \ox Y)$.   Suppose for contradiction that $Q$ is not in $F(X \ox Y)$, as defined;  
this means there is no $A \in F(X)$ and $B \in F(Y)$ such that $Q \subseteq A \x B$.  
This, in turn, means that ${\sf dom}(Q) \not\in F(X)$ or ${\sf cod}(Q) \not\in F(Y)$. 
Without the loss of generality, we may assume the former: 
this means that there is a $C \in F(X)^\perp$ with $C \cap {\sf dom}(Q)$ infinite.  
We may ``thicken'' $C \cap {\sf dom}(Q)$ to $C': X \to Y $ by choosing for each 
$c \in C \cap {\sf dom}(Q)$ a $y_c \in Y$ so that $(c,y_c) \in Q$.  
Then $C' \cap Q$ is certainly infinite, however, $C' \in F(X \ox Y)$ as 
intersecting with any $A \x B$ with $A \in F(X)$ and $B \in F(Y)$ is finite as $C \cap A$ is finite.   
But this means $Q$ cannot be in $F(X \ox Y)^{\perp\perp}$.
\end{proof}

By the discussion above Lemma \ref{character_finiteness_relation}, as $Q \in F(X \cap Y)^\perp$ precisely when $Q \cap A \x B$ is finite for all $A \in F(X)$ and $B \in F(Y)$, $Q$ may be  characterized as a finiteness relation $Q: X^{*} \to Y$ or $Q^\circ: Y^{*} \to X$.  This immediately gives:

\begin{lemma}
\label{Lemma: tensor perp characterization}
Let $X$ and $Y$ be a finiteness spaces then $P \subseteq X \x Y$ then the following are equivalent:
\begin{enumerate}[(i)]
\item $P \in F(X \ox Y)^\perp$;
\item $P \cap A \x B$ is finite for all $A \in F(X)$ and $B \in F(Y)$;
\item For all $A \in F(X)$,  $A \triangleright P  \in F(Y)^\perp$ and,  for all $B \in {F(Y)}$, $P \triangleleft B \in F(X)^\perp$;
\item For all $A \in F(X)$,  $A \triangleright P  \in F(Y)^\perp$ and, for every $y \in Y$, $P \triangleleft \{ y\} \in F(X)^\perp$;
\item For every $x \in X$, $\{ x\} \triangleright P \in F(Y)^\perp$ and, for all $B \in {F(Y)}$, $P \triangleleft B \in F(X)^\perp$.
\end{enumerate}
\end{lemma}

The unit of tensor is given by the one element set with only finiteness structure possible:  $\top :=  (\{ \star \}, {\cal P}(\{ \star \}), {\cal P}(\{ \star \}) )$.

Given finiteness relations, $X \to^{R} X'$, and $Y \to^{S} Y'$, then: 
\[ R \ox S := \{ ((x,y), (x',y')) \mid xRx', ySy' \}\]  
as in ${\sf Rel}$, is clearly a finiteness relation.  The associator and unitors for this tensor product are the same as in $\Rel$.  
Thus, the associator is the following relation: 
\[ a_\ox := {(((x,y),z), (x,(y,z))) \mid x \in X, y \in Y, z \in Z}\] 
and this clearly gives a bijection between 
$F((X \ox Y) \ox Z) = {\downarrow}\{ ((a,b),c) \mid a \in A \in F(X), b \in B \in F(Y), c \in C \in F(Z) \}$, and $F(X \ox (Y \ox Z)) = \downarrow\{ ((a,(b,c)) \mid a \in A \in F(X), b \in B \in F(Y), c \in C \in F(Z) \}$. 

The tensor product is also symmetric, with the symmetry relation as in ${\Rel}$.

\medskip

 If $M$ and $N$ are finiteness matrices, $M \ox N$ is given by the outer product,  
 while the finiteness matrices for the associators, and the unitors are given by the 
 characteristic matrices (where the entries are either 0 or 1) of the corresponding 
 coherence relations in ${\sf FRel}$. The commutativity of the rig 
 makes the tensor products symmetric. 

\begin{lemma}
\label{Lemma:tensors_FRel}
$\FRel$ and ${\sf FMat}(R)$ have a symmetric tensor product $\_ \ox \_$ and by de Morgan duality a further symmetric tensor product, called the {\bf par}, $\_ \oa \_$ where $X \oa Y := (X^* \ox Y^*)^*$.
\end{lemma}

The tensor and par do not coincide in general, however, note that, as
\begin{align*} F(X \oa Y) & := F((X^* \ox Y^*)^*) = F(X^* \ox Y^*)^\perp = \{ A' \x B' \mid A' \in F(X)^\perp, B' \in F(Y)^\perp \}^\perp \\
& = \{ G \mid \forall A' \in F(X)^\perp, B' \in F(Y)^\perp. G \cap A' \x B'~\mbox{is finite} \} 
\end{align*}
for $A \in F(X)$ and $B \in F(Y)$ we therefore have $A \x B \in F(X \oa Y)$ as $A \x B \cap A' \x B'$ is finite.  This means the map 
${\sf mx}: A \ox Y \to A \oa Y$, called the mix map,  which is the identity map on the webs is a finiteness relation.  It is, as we shall see,  not generally an isomorphism.  
The unit for the par is $\bot := \top^{*} =  (\{ \star \}, {\cal P}(\{ \star \}),{\cal P}(\{ \star \})) = \top$.  Thus, the units for the two tensors coincide.  This suggests the structure of an isomix category.  Next, we show that there is a linear distributor in both $\FRel$ and ${\sf FMat}(R)$. Preliminary to this we observe:

\begin{lemma}
\label{Lemma: perp perp} 
For any finiteness spaces $X$, $Y$, and $Z$, the following hold:
\begin{enumerate}[(i)]
\item $F(X \ox Y) \subseteq F(X \oa Y)$;
\item $F(X^* \ox Y^*) \subseteq  F(X \ox Y)^{\perp}$;
\item  $F((X^* \ox Y^*) \ox Z) \subseteq F((X \ox Y)^* \ox Z)$.
\end{enumerate}
\end{lemma}
\begin{proof}~
\begin{enumerate}[{\em (i)}]
\item As discussed above.
\item $F(X \ox Y)^{\perp} = F((X \ox Y)^*) = F((X^{**} \ox Y^{**})^*) = F(X^* \oa Y^*) \supseteq F(X^* \ox Y^*)$. 
\item $F((X^* \ox Y^*) \ox Z) = F((X^* \ox Y^*)^{**} \ox Z) = F((X \oa Y)^* \ox Z) \subseteq F((X \ox Y)^* \ox Z)$.
\end{enumerate}
\end{proof}

\begin{proposition}
\label{Lemma: FRel is a LDC}
{\sf FRel}, and $\FMat(R)$ are symmetric LDCs.
\end{proposition}
\begin{proof}
The linear distributor for $\FRel$, $\partial^L : X \ox (Y \oa Z) \to (X \ox Y) \oa Z$ on the webs is the associator in ${\sf Rel}$.  To show this is a finiteness relation (although it is not an isomorphism) it is sufficient to prove that if $A \in F( X \ox (Y \oa Z))$ then ${\sf cod}(A a_\x) \in F( (X \ox Y) \oa Z)$ as $a_\x$ certainly has the preimage of singleton sets in $F( X \ox (Y \oa Z))^\perp$ as they are singleton sets.

Observe that
\begin{align*}
F( X \ox (Y \oa Z))  &= F(X \ox (Y^* \ox Z^*)^*) \\
&= F((X)^{**} \ox  (Y^* \ox Z^*)^*)   \tag{as $X^{**} = X$ } \\
&\subseteq  F(X^* \ox (Y^* \ox Z^*))^\perp  \tag{ Lemma  \ref{Lemma: perp perp}{\em (ii)\/}} \\
&\simeq   F( (X^* \ox Y^*) \ox Z^*)^\perp  \tag{ Associativity of tensor } \\
&\subseteq  F( (X \ox Y)^*  \ox Z^*)^\perp  \tag{ Lemma \ref{Lemma: perp perp}{\em (iii)\/}}\\
&= F( (X \ox Y) \oa Z) \tag{ as $(X, F(X), F(X)^\perp)^* = (X^*, F(X^*), F(X^*)^\perp) := (X, F(X)^\perp, F(X))$ }
\end{align*}

For $\FMat(R)$, the left distributor is the characteristic matrix of $\partial^L$. 
\end{proof}

\begin{proposition}
$\FRel$, and $\FMat(R)$ are symmetric isomix $*$-autonomous categories with 
respect to the involution $(\_)^{*}$.
\end{proposition}
\begin{proof}
By Lemma \ref{Lemma: FRel is a LDC}, $\FRel$ is an LDC. It is an isomix category 
because $\m : \top \to \bot = 1_{\{ \star \}}$ is an  isomorphism.

To show this we must demonstrate that we have a ``cap''  $\top \to A \oa A^*$ and a ``cup'' $A^* \ox A \to \bot$: on the webs these are the 
corresponding maps in ${\sf Rel}$ given by the diagonal subset seen as a relation in two ways.  It suffices to show that these maps (which are dual) 
are finiteness relations.  As the required duality identities are satisfied by the underlying relations, they are also satisfied in ${\sf FRel}$.

Consider $\eta: \top \to A \ox A^*$ where $\eta:= \{ (\star,(a,a)) \mid a \in A\}$  as the finitary sets of $\top$ are just $\{\}$ and $\{ \star \}$ and $\eta^\circ$ certainly has preimages of cofinitary sets cofinitary it remain only to show that $\Delta_A := \{ (a,a) \mid a \in A\}$ is finitary in $A \ox A^{*}$.  However, this is so if and only if $\Delta_A$ is a finiteness relation from $A \to A$ (by Lemma \ref{Lemma: tensor perp characterization} suitably dualized) ... which it certainly is as it is the identity relation!

This argument translates into ${\sf FMat}(R)$ using the characteristic matrices of these relations and the fact that all the coherence maps are given by the corresponding characteristic matrices.
\end{proof}

As $\FRel$ is *-autonomous, it is self-enriched. The internal hom is given as usual by 
$X \lollipop Y =  X^* \ox Y = (X \ox Y^*)^*$. Thus,  $X \lollipop Y = 
(X \times Y, F(X \otimes Y^\perp)^\perp, F(X \otimes Y^\perp))$. 
The same argument applies to $\FMat(R)$.

\begin{lemma}
 Let $X$,  and $Y$ be  finiteness spaces, then $A \in F( X \lollipop Y)$ 
 if and only if  $A: X \to Y$ is a finiteness relation.
\end{lemma}


\subsection{The core of ${\sf FRel}$ and ${\sf FMat}(R)$}
\label{Sec: core of FMat}

Recall that, for any isomix category $\X$,  the {\bf core} of $\X$, ${\sf Core}(\X) 
\subseteq \X$, is the full subcategory consisting of objects $U$ such that for 
any object $X \in \X$, $\mx : U \ox X \to U \oa X$ (and $\mx: X \ox U \to X \oa U$) 
is an isomorphism.  The core is always closed to tensor and par and the units. Thus, 
the core of an isomix category is an isomix category with 
${\sf mx}: U \ox V \to U  \ox V$ is a natural isomorphism. 

In ${\sf FRel}$, the mix map is given by the identity map on the webs and in ${\sf FMat}(R)$ it is given by the characteristic function of this identity map.
Our objective in this subsection is to give a complete description of the core of  ${\sf FRel}$ and ${\sf FMat}(R)$:

\begin{proposition}
\label{Lemma: Core is finite}
$U \in \Core(\FRel)$  if and only if $U$ is finite.  Similarly, 
$U \in \Core(\FMat(R))$ if and only if $U$ is finite.
\end{proposition}

\begin{proof}
Suppose $U$ is a finite set then $U = U^{*}$ as $F(U)=F(U)^\perp = 
{\cal P}(U)$.  We shall prove that $F(U \ox Y) = F(U \ox Y)$ for all $Y$. 
We first prove this for $\FRel$:

\begin{description}
\item[$U$ finite $\Rightarrow$ $U \in \Core(\FRel)$:]
It suffices to show that $R \in  F(U \oa Y)$ then $R \in F(U \ox Y)$.  
However, $R \in F(U \oa Y) = F(U^{*} \oa Y)$ if and only if $R: U \to Y$ is a 
finiteness relation.  As $R$ is a finiteness relation, because 
$U \in F(U)$, ${\sf cod}(R) \in F(Y)$ but then 
$R \subseteq U \x {\sf cod}(R)$ showing $R \in  F(U \ox Y)$. 
\item[$U \in \Core(\FRel)$ $\Rightarrow$  $U$ is finite:] 
We prove the converse statement: if $U$ is infinite set, 
that there is a finiteness space, $Y$, such that $F(X \ox Y) 
\subseteq F(X \oa Y)$. Choose $Y = (Y, \mathcal{P}(Y), \mathcal{P}_f(Y))$ 
such that there is an injective function  $\alpha: U \to Y$: this forces $Y$ to be infinite.  
Because $|U|$ is infinite, there is an infinite $Q \subseteq U$ such that $Q \in F(U)^\perp$ or $Q \in F(U)$.  
Without loss of generality, we may assume that $Q \in F(U)^\perp$.

Define $R := \{ (x,\alpha(x)) \mid x \in Q \}$, then $R \in F(U \ox Y)$ because the following two conditions hold:
\begin{enumerate}[(i)]
\item For all $A \in F(U)^\perp$, $A \tr R \in F(Y)$ as $F(Y) \in \mathcal{P}(Y)$.
\item For all $y \in Y$, $R \tl \{y\} \in F(X)^\perp$, as $\alpha$ is a monic, either $R\tl \{y\} = \varnothing$, 
or $R\tr \{y\} = \{ x \}$ where $\alpha(x) = y$. Since, $R \tr \{y\}$ is a finite, $R \tr \{y\} \in F(X)^\perp$. 
\end{enumerate} 

Next we show, to complete the proof, that $R \notin F(X \ox Y)$. $R \in F(X \ox Y)$ if and only $R \subseteq A \x B$, where $A \in F(U)$ and $B \in F(Y)$.   However, ${\sf dom}(R)= Q \in F(U)^\perp$. Since, $Q$ is infinite, $Q$ cannot also be a member of $F(U)$.  Thus, $Q \notin F(X \ox Y)$.
\end{description}

The same proof applies to $\FMat(R)$ by transposing to the characteristic functions of the finiteness relationships.
\end{proof}

\begin{corollary} ~
\label{Corollary: core}
\begin{enumerate}[(i)]
\item $\Core({\FRel})$ is the category of finite sets and relations.
\item $\Core({\FMat(R)})$ is (equivalent to) the category of finite-dimensional matrices over the rig $R$.
\end{enumerate}
\end{corollary}
