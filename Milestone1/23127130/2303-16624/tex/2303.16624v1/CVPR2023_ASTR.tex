% CVPR 2023 Paper Template
% based on the CVPR template provided by Ming-Ming Cheng (https://github.com/MCG-NKU/CVPR_Template)
% modified and extended by Stefan Roth (stefan.roth@NOSPAMtu-darmstadt.de)

\documentclass[10pt,twocolumn,letterpaper]{article}
% \documentclass{llncs}

%%%%%%%%% PAPER TYPE  - PLEASE UPDATE FOR FINAL VERSION
% \usepackage[review]{cvpr}      % To produce the REVIEW version
% \usepackage{cvpr}              % To produce the CAMERA-READY version
% \usepackage{epsfig}
\usepackage[pagenumbers]{cvpr} % To force page numbers, e.g. for an arXiv version

% Include other packages here, before hyperref.
\usepackage{graphicx}
\usepackage{amsmath}
\usepackage{amssymb}
\usepackage{booktabs}

\usepackage{amsthm}
\usepackage{multirow}
% \usepackage{hyperref}
% \usepackage{authblk}

% It is strongly recommended to use hyperref, especially for the review version.
% hyperref with option pagebackref eases the reviewers' job.
% Please disable hyperref *only* if you encounter grave issues, e.g. with the
% file validation for the camera-ready version.
%
% If you comment hyperref and then uncomment it, you should delete
% ReviewTempalte.aux before re-running LaTeX.
% (Or just hit 'q' on the first LaTeX run, let it finish, and you
%  should be clear).
\usepackage[pagebackref,breaklinks,colorlinks]{hyperref}

\usepackage[accsupp]{axessibility}  % Improves PDF readability for those with disabilities.

% Support for easy cross-referencing
\usepackage[capitalize]{cleveref}
\crefname{section}{Sec.}{Secs.}
\Crefname{section}{Section}{Sections}
\Crefname{table}{Table}{Tables}
\crefname{table}{Tab.}{Tabs.}

% \cvprfinalcopy % *** Uncomment this line for the final submission

%%%%%%%%% PAPER ID  - PLEASE UPDATE
\def\cvprPaperID{5645} % *** Enter the CVPR Paper ID here
% \def\httilde{\mbox{\tt\raisebox{-.5ex}{\symbol{126}}}}
\def\confName{CVPR}
\def\confYear{2023}

% \makeatletter
% \newcommand{\printfnsymbol}[1]{%
%   \textsuperscript{\@fnsymbol{#1}}%
% }
% \makeatother

\begin{document}

%%%%%%%%% TITLE - PLEASE UPDATE
\title{Adaptive Spot-Guided Transformer for Consistent Local Feature Matching}

\author{Jiahuan Yu\footnotemark[1], Jiahao Chang\footnotemark[1], Jianfeng He, Tianzhu Zhang\footnotemark[2], Feng Wu\\
{University of Science and Technology of China}
% {\tt\small yjh.cs.1998@gmail.com}
% For a paper whose authors are all at the same institution,
% omit the following lines up until the closing ``}''.
% Additional authors and addresses can be added with ``\and'',
% just like the second author.
% To save space, use either the email address or home page, not both
% \and
% Jiahao Chang\\
% University of Science and Technology of China\\
% {\tt\small secondauthor@i2.org}
% {\small \textsuperscript{1}University of Science and Technology of China, \textsuperscript{2}Deep Space Exploration Lab, \textsuperscript{3}China Academy of Space Technology}
}


% \renewcommand{\thefootnote}{\fnsymbol{footnote}}
% \footnotetext[1]{test}
% \footnotetext[2]{test}

% University of Science and Technology of China\\

% \maketitle
% \begin{figure*}[t]
%     \centering
%     \includegraphics[width=1.0\linewidth]{graph/ASTR_motivation.pdf}
%     \caption{
%     The architecture of our ASTR consists of two major components, including the spot-guided aggregation module and the adaptive scaling modules.
%     }\label{fig:motivation}
% \end{figure*}
% \maketitle

\twocolumn[{%
    \renewcommand\twocolumn[1][]{#1}%
	\maketitle
	\begin{center}
	\includegraphics[width=1.0\linewidth]{graph/ASTR_motivation.pdf}
	\captionof{figure}{The visualization of the cross attention heatmaps and matching results.
    We sample two similar adjacent points in the reference image (a), marked with green and red.
    (b) are two heatmaps of the linear cross attention in LoFTR~\cite{sun2021loftr} when green and red pixels are queries.
    (c) are two heatmaps obtained from the vanilla cross attention.
    (d) are two heatmaps generated by our spot-guided attention.
    (e) are the comparison of the final matching results produced by LoFTR~\cite{sun2021loftr} (top) and our method (down).}
	\label{fig:motivation}
	\end{center}
}]

% \maketitle
% \begin{figure*}[t]
%     \centering
%     \includegraphics[width=1.0\linewidth]{graph/ASTR_motivation.pdf}
% 	\captionof{figure}{The visualization of the cross attention heatmaps and matching results.
%     We sample two similar adjacent points in the reference image (a), marked with green and red.
%     (b) are two heatmaps of the linear cross attention in LoFTR~\cite{sun2021loftr} when green and red pixels are queries.
%     (c) are two heatmaps obtained from the vanilla cross attention.
%     (d) are two heatmaps generated by our spot-guided attention.
%     (e) are the comparison of the final matching results produced by LoFTR~\cite{sun2021loftr} (top) and our method (down).}
% 	\label{fig:motivation}
% \end{figure*}

\renewcommand{\thefootnote}{\fnsymbol{footnote}} %将脚注符号设置为fnsymbol类型,即特殊符号表示
\footnotetext[1]{Equal Contribution} %对应脚注[1]
\footnotetext[2]{Corresponding Author} %对应脚注[2]

%%%%%%%%% ABSTRACT
\begin{abstract}


Over the past few years, there has been a significant amount of research focused on studying the ReLU activation function, with the aim of achieving neural network convergence through over-parametrization. However, recent developments in the field of Large Language Models (LLMs) have sparked interest in the use of exponential activation functions, specifically in the attention mechanism.

Mathematically, we define the neural function $F: \R^{d \times m} \times  \mathbb{R}^d \rightarrow \mathbb{R}$ using an exponential activation function. Given a set of data points with labels $\{(x_1, y_1), (x_2, y_2), \dots, (x_n, y_n)\} \subset \mathbb{R}^d \times \mathbb{R}$ where $n$ denotes the number of the data. Here $F(W(t),x)$ can be expressed as $F(W(t),x) := \sum_{r=1}^m a_r \exp(\langle w_r, x \rangle)$, where $m$ represents the number of neurons, and $w_r(t)$ are weights at time $t$. It's standard in literature that $a_r$ are the fixed weights and it's never changed during the training. We initialize the weights $W(0) \in \mathbb{R}^{d \times m}$ with random Gaussian distributions, such that $w_r(0) \sim \mathcal{N}(0, I_d)$ and initialize $a_r$ from random sign distribution for each $r \in [m]$.

Using the gradient descent algorithm, we can find a weight $W(T)$ such that $\| F(W(T), X) - y \|_2 \leq \epsilon$ holds with probability $1-\delta$, where $\epsilon \in (0,0.1)$ and $m = \Omega(n^{2+o(1)}\log(n/\delta))$. To optimize the over-parametrization bound $m$, we employ several tight analysis techniques from previous studies [Song and Yang arXiv 2019, Munteanu, Omlor, Song and Woodruff ICML 2022]. 

 

\end{abstract}

%%%%%%%%% BODY TEXT
\section{Introduction}
\label{sec:intro}
\section{Introduction}

The increasing complexity of source code poses a key challenge to the reliability of large-scale software systems. Software bugs in these systems can lead to safety issues~\cite{bug_safety} for users around the world as well as cause non-negligible financial losses~\cite{bug_loss}. As such, developers have to spend a large amount of time and effort on bug fixing. Consequently, \aprfull (\apr), designed to automatically generate patches to fix software bugs, has attracted wide attention from both academia and industry~\cite{long2016prophet, legoues2012genprog, long2015spr, lou2020can, tufano2018empstudy}. 


To achieve \apr, one popular approach is known as Generate-and-Validate (G\&V)~\cite{qi2015gv, ghanbari2019prapr, lou2020can, le2016hdrepair, legoues2012genprog, wen2018capgen, hua2018sketchfix, martinez2016astor, koyuncu2020fixminder, liu2019tbar, liu2019avatar}, which is typically based on the following pipeline: First, fault localization techniques~\cite{wong2016fl, abreu2007ochiai, zhang2013injecting, papadakis2015metallaxis, li2019deepfl, li2017transforming} are applied to determine the suspicious locations in programs where bugs are likely to exist. Then, the buggy locations are used by the \apr tools to generate a list of patches that replace buggy lines with correct lines. Afterward, each patch is validated against the original test suite to identify any \emph{plausible patches} (i.e., passing all tests in the test suite). Finally, to determine the \emph{correct patches}, developers examine the list of plausible patches to see if any of them can correctly fix the bug. 

Traditional \apr tools can mainly be categorized into heuristic-based~\cite{legoues2012genprog, le2016hdrepair, wen2018capgen}, constraint-based~\cite{mechtaev2016angelix, le2017s3, demacro2014nopol, long2015spr} and \template~\cite{ghanbari2019prapr, hua2018sketchfix, martinez2016astor, liu2019tbar, liu2019avatar}. Among these traditional tools, \template \apr tools~\cite{ghanbari2019prapr, liu2019tbar, benton2020effectiveness} have been able to achieve state-of-the-art results. \Template \apr tools typically leverage pre-defined templates (e.g., adding a nullness check) for bug fixing. However, since these fix templates are typically handcrafted, the number and types of bugs they are able to fix can be limited. 



To address the limitations of traditional \apr, researchers have proposed various \learning \apr tools~\cite{li2020dlfix, chen2018sequencer, jiang2021cure, lutellier2020coconut, zhu2021recoder, ye2022rewardrepair} based on the \nmtfull (\nmt) architecture~\cite{sutskever2014mt} where the input is the buggy code snippets and the goal is to translate the buggy code snippets into a fixed version. To accomplish this, \learning \apr tools require supervised training datasets with pairs of both buggy and fixed code snippets in order to learn how to perform this translation step. These training data are usually obtained by mining historical bug fixes using heuristics/keywords~\cite{dallmeier2007benchmark}, which can be imprecise for identifying bug-fixing commits; even the actual bug-fixing commits can include irrelevant code changes, leading to further pollution in the dataset~\cite{xia2022alpharepair}.
% 
Moreover, it can be hard for such \apr tools to generalize and fix bug types unseen during training. 



To better leverage recent advances in \plmfull{s} (\plm{s}), researchers~\cite{xia2022alpharepair, xia2023repairstudy, kolak2022patch, prenner2021codexws} have directly applied \plm{s} to generate patches without bug-fixing datasets. These \llm-based \apr tools work by either directly generating a complete code function~\cite{prenner2021codexws, xia2023repairstudy} or predict/infill the correct code snippet given its surrounding context~\cite{xia2022alpharepair, xia2023repairstudy}. By directly using \llm{s} that are pre-trained on billions of open-source code snippets, \llm-based \apr tools can achieve state-of-the-art performance on many repair datasets~\cite{xia2022alpharepair}. 


% 
%
%

Traditional \apr tools have long used the insight of the \emph{plastic surgery hypothesis}~\cite{barr2014plastic} where it states that the code ingredients to fix a bug already exist within the same project. Traditional \apr tools have manually designed pattern-~\cite{ghanbari2019prapr, saha2017elixir} or heuristic-based~\cite{jiang2018simfix, legoues2012genprog} approaches to finding and using such relevant code ingredients to generate fixes for bugs. However, the plastic surgery hypothesis has been largely ignored in \llm-based \apr. In fact, \llm provides a unique opportunity to fully automate the plastic surgery hypothesis idea via fine-tuning (learning project-specific information via model updates from the buggy project) and prompting (directly providing relevant code ingredients to the model), and make it directly applicable to different languages (since the \llm{s} are typically multi-lingual).%
Moreover, despite the intensive manual efforts involved, traditional \apr tools still cannot fully leverage project-specific information due to large search space for leveraging/composing existing code ingredients. In contrast, the project-specific information can effectively leveraged by \llm{s} due to their power in code understanding/vectorization, e.g., even partial/imprecise information may still guide \llm{s} in correct patch generation!
 To this end, we ask the question: \emph{How useful is the plastic surgery hypothesis in the era of \plm{s}}?








\mypara{Our Work.} To answer the question, we present \ourtech{\xspace} -- a \llm-based approach that automatically utilizes the plastic surgery hypothesis by systematically combining multiple fine-tuning and prompting strategies for \apr. \ourtech fine-tunes \plm{s} using two novel domain-specific training strategies: \textbf{\epfinetune} -- we fine-tune using the original buggy project by aggressively masking out a high percentage of tokens, which allows \plm to learn project-specific code tokens and programming styles; and \textbf{\rofinetune} -- which only masks out a single continuous code sequence per training sample, allowing the model to get used to the final \csapr task of predicting a single continuous code sequence. Furthermore, we directly leverage the ability for \plm{s} to understand natural language instructions and introduce a novel prompting strategy, \textbf{\idprompting}, which uses information retrieval and static analysis to obtain a list of relevant identifiers for the buggy lines. While such relevant identifiers are critical for fixing some difficult bugs, they may not be seen by the \llm during inference due to limited context window size. Through the use of prompting, we directly tell the model to use these extracted identifiers (relevant code ingredients) to generate the correct code. Finally, to perform repair, we combine all four model variants (including the base model, both fine-tuned models and the base model with prompting) for the final repair.





While our insight of leveraging the plastic surgery hypothesis for \llm-based \apr is generalizable across different types of \plm{s}, to implement \ourtech, we choose a recent \plm{\xspace}, \ctfive~\cite{wang2021codet5}, which is pre-trained on millions of open-source code snippets. \ctfive is an encoder-decoder model trained using \mspfull (\msp) objective where a percentage of tokens are masked out and each continuous masked token sequence is referred to as a masked span. Also, although we only extract relevant identifiers from the current buggy project (since this paper focuses on the plastic surgery hypothesis), our work can be easily extended to obtain other code information (such as relevant statements or functions) from other sources, such as  the massive pre-training corpora~\cite{husain2020codesearchnet} or historical bug-fixing datasets~\cite{jiang2019infer}, which can provide more coding knowledge for \llm{s}. Besides, although we mainly focus on using traditional string comparison algorithms for information retrieval in this paper, these techniques can be easily replaced by other frequency-based retrieval~\cite{robertson2009probabilistic} and neural search (or embedding-based search)~\cite{reimers2019sentence}.
  In summary, this paper makes the following contributions:


%


\begin{itemize}[noitemsep, leftmargin=*, topsep=0pt]
    \item \textbf{Dimension.} This paper is the first to revisit the important plastic surgery hypothesis in the era of \llm{s}. It opens up a new dimension for \llm-based \apr to incorporate previously neglected information from the buggy project itself to boost \apr performance. Furthermore, it demonstrates the promising future of retrieval-based prompting for modern \llm-based \apr.
    \item \textbf{Implementation.} We implement \ourtech based on the recent \ctfive model. We augment the model using two novel fine-tuning strategies: \epfinetune and \rofinetune, along with a novel prompting strategy based on information retrieval and static analysis: \idprompting. We combine the patches generated by all four models together and perform patch ranking to speed up \apr.% 
    \item \textbf{Evaluation Study.} We conduct an extensive evaluation against state-of-the-art \apr tools. On the widely studied \dfj 1.2 and 2.0 datasets~\cite{just2014dfj}, \ourtech is able to achieve the new state-of-the-art results of 89 and 44 correct bug fixes (15 and 8 more than best baseline) respectively.  Furthermore, we perform a broad ablation study to justify our design. \ourtech demonstrates for the first time that the plastic surgery hypothesis can substantially boost \llm-based \apr and advance state-of-the-art \apr, while being fully automated and general. Moreover, even partial/imprecise code ingredients may still effectively guide \llm{s} for \apr!
\end{itemize}



%------------------------------------------------------------------------
\section{Related Work}
\label{sec:rw}
\vspace{-1.5ex}
\section{Related works}
\vspace{-1ex}
\noindent\textbf{Quantization} has been widely used for efficiency in deployment. Extensive efforts can be classified into post-training quantization (PTQ)~\cite{datafreequant,4bitptq} and quantization-aware training (QAT)~\cite{8bitquant,relaxedquant,lsq,lsqplus}. QAT generally outperforms PTQ in quantizing compact DNNs to  8bit and very low-bit (2, 3, 4bit) by finetuning the quantized weights. Despite their success, traditional quantization methods focus on minimizing accuracy loss for a given pre-trained model, but ignore the real-world inference efficiency. 
% In our work, we utilize LSQ+~\citep{lsqplus}, an extension of LSQ~\citep{lsq}, that works well on INT8 quantization with trainable scale and offset parameters. 
%Our work considers both the  accuracy and on-device latency. 
%only INT8 quantization as most commercial edge devices support only for integer 8 or floating-point inference. 
 

% PTQ~\citep{datafreequant,4bitptq} directly quantizes a pre-trained FP32 model~\citep{8bitquant,relaxedquant,lsq}, while QAT generally outperforms PTQ with finetuning the quantized weights.  While SOTA QAT work well on typical 8bit and very low-bit (2, 3, 4bit) quantization, little attention is paid on studying the real-world inference latency of quantized networks. In practical, only 8bit has been widely supported on commercial edge devices~\citep{}, and many works~\citep{apq,haq} report simulated latency on specialized hardware like FPGA. 


%\textbf{Neural Architecture Search}. Introduce NAS algorithms, and then introduce existing quantization-aware NAS to search layer bit-width. 


%Recent studies combine quantization and NAS to automatically search for layer bit-width with given architecture or search for operations with given bit-width. HAQ~\citep{haq} searches for mixed-bits for a given network architecture. SPOS~\citep{spos} trains a quantized one-shot supernet to search for bit-width. APQ~\citep{apq} builds upon a full precision one-shot supernet and  build a proxy quantized accuracy predictor. Most existing methods fall into two categories. The first optimizes for a singled weighted
%MPS~\citep{mps}

%DNAS~\citep{dnas}

%Recently, formulated mixed-precision network quantization as an instance of NAS. 


%only 8bit has been widely supported on commercial edge devices~\citep{}.


%have achieved great process in 8bit and other extremely-low bits like 4,2,1 bits. However, 

% There have been extensive works on compressing DNNs by quantization, that can be classified into post-training methods  ~\citep{8bitquant,4-2bitquant} propose to 

% INT8 quantization has been widely used for efficiency in deployment. 

%The quantization community has shown low bits as 

%it was possible to quantize to 8-bit integer or even lower as 2-bit integers with minimal accuracy loss. 

%There exists two main methods of quantizing pre-trained NNs: QAT and post-training quantization (PTQ). 

%https://arxiv.org/pdf/2111.03759.pdf



\noindent\textbf{NAS for Quantization}. %Early NAS works focus on automating  network design for SOTA accuracy. Recent hardware-aware NAS methods~\cite{proxylessnas,ofa,fbnetv3} consider both  accuracy and  efficiency by introducing latency predictors.However, these works consider the latency of FP32 models, leading to a big gap and performance degradation for  quantized models.  
Early works~\cite{haq,dnas,spos,apq,mps} formulates mixed-precision problem into NAS to search layer bit-width with a given architecture. Recently,  ~\cite{oqa,batchquant}  train a quantized-for-all supernet to search both architecture and bit-width. The searched models can be directly evaluated with comparable accuracy to train-from-scratch. However, little attention is paid on optimizing quantized  latency on real-world devices. Through searching quantization-friendly search space, our discovered quantized models can achieve both high accuracy and low latency.

 


%On the other hand, ~\citep{haq,dnas,spos,apq,mps} formulates mixed-precision problem into NAS to search layer bit-width with given architecture or search for operations with given bit-width. Recently, ~\citep{oqa,batchquant} leverage NAS techniques in ~\citep{ofa,bignas,sandwichrule} and train a quantization-for-all supernet for quantized model search. However, current quantization-aware NAS focus on mixed-precision quantization, 
%that requires specialized hardware support (e.g., FPGA) or kernel implementation~\citep{hawqv3}.  In our work, we consider INT8 quantization as it's widely supported on commercial edge devices. Our approach optimizes for a quantization-friendly search space, and is orthogonal to these methods.  
 

\noindent\textbf{Search Space Design}. % A good search space is crucial to the NAS performance\citep{regnet,attentivenas}.
 Starting from ~\cite{proxylessnas}, the manually-designed MBConv-based space becomes the dominant in most NAS works~\cite{proxylessnas,ofa,bignas,attentivenas}.
 RegNet~\cite{regnet} is the first to present  standard guidelines to optimize a search space by each dimension.  Recently, ~\cite{angle-based,asap,pcnas,padnas,nse,autoformerv2} propose to  shrink to a better compact search space by either pruning unimportant operators or configurations.  
  However, these works focus on optimizing the accuracy and little attention is paid on quantization-friendly search space design. Our work is the first lightweight solution towards this direction.
  %cannot be applied to search a quantization-friendly  space for two reasons.
% Firstly, none of them search the space for both operator type and configurations, which are the crucial factors impacting INT8 latency; Secondly, 
%each iteration requires training a candidate space from scratch to evaluate its quality,
% the current search is constrained within 10 iterations due to the huge training cost.  Our work is the first lightweight solution for hardware-friendly search space automation.
 %However, they focus on optimizing model accuracy and the exploration of hardware efficient search space is limited. Moreover, none of them can efficiently search both operator and configuration on large datasets (e.g., ImageNet). Our work is the first  along this direction with a feasible search cost.
 
 %However, these works focus on optimizing the accuracy and cannot be applied to search a quantization-friendly  space for two reasons.
 %Firstly, none of them search the space for both operator type and configurations (e.g., the channel width for each layer), which are the crucial factors impacting INT8 latency; Secondly, 
 % each iteration requires training a candidate space from scratch to evaluate its quality,  the current search is constrained within 10 iterations due to the huge training cost.

% none of them can efficiently search both operator and configuration for a quantization-friendly space on large datasets (e.g., ImageNet).
%Recently, several works~\citep{angle-based,asap,pcnas,padnas,nse}  automatically shrink a better compact search space and achieve better accuracy. They prune unimportant operators by the importance ranking during  model architecture search.  While they search only for the operators, S3~\citep{autoformerv2} is the first work that searches the space configurations for vision transformer by search dimension decomposition and linear parameterization. However, none of them can efficiently search both operator and configuration for a quantization-friendly space on large datasets (e.g., ImageNet). In contrast, {\algname} is the first lightweight search space search work that automatically leverages hardware optimizations like INT8 quantization. Moreover, our method decouples space search and architecture search, so that the searched space can apply to many existing NAS algorithms. 



%S3~\cite{autoformerv2} is the first work that searches the configurations for vision transformer search space by 

%add hurriance, discuss which work search operator, which search configurations. what are their search cost

%sota search spaces 

%auto search space design
%\cite{regnet,autoformerv2,padnas,nse}

%~\cite{hurricane}


%\lz{~\cite{pcnas}: to solve the weight sharing issue: posterior fading: as the number of models in the supergraph increases, the kl-divergence between true parameter posterior and proxy posterior also increases. Solution: divide the training of supergraph into several intervals and maintain a pool of high potential partial models and progressively update this pool. }

%\lz{~\cite{asap}, xnas: pcnas: "introduce pruning during the training of over-parameterized networks.  Similar to these approaches, we start with an overparameterized network and then reduce the search space to
%	derive the optimized architecture. Instead of focusing on
%	the speed-up of training, we further improve the rankings of
%	models and evaluate operators directly on validation set"}

%\lz{~\cite{asap}: a differentiable search space that allows the annealing of architecture weights, while gradually pruning inferior operations. It allows gradual pruning of weak weights, which reduces the number of computed connections through the search. Search space: darts and enas, cifar10. use the operator weights to reduce operators.}

%\lz{xnas: dynamically wipe out inferior architectures and enhance superior ones. cifar10}

%\lz{angle-based: weight-sharing nas, during the search process.  existing nas usually use accuracy-based metric or magnitude-based metric to guide the shrinking process. This paper propose a novel angle-based metric to guide the shrinking process. Save heavy computation overhead, higher stability and ranking correlation. Prune operators. small dataset}

%We: decouple the neural architecture search and search space search. 




%-------------------------------------------------------------------------
\section{Our Approach}
\label{sec:method}
\section{Method}
\label{sec:method}

% \ml{``Inconsistent'' to ``large variation''}

% In this section, we propose our methods based on the observations in Section \ref{sec:motivation}.
In this section, we propose two techniques to further enhance the strong baseline to capture the variation of activation distributions better.
We first introduce spatial re-scaling to adapt the network to pixel-to-pixel variation.
We then propose channel-wise shifting and re-scaling to better capture the channel-to-channel variation.
Meanwhile, as both of the two methods are image-dependent, the image-to-image variation can be captured naturally.
By combining the two methods with our strong baseline, we build our enhanced BNN for SR, named EBSR.

% Because the activation distributions among pixels, channels and images have large variations \red{**are highly inconsistent} in SR networks, we introduce spatial re-scaling to adapt to pixel-wise variations and channel shift and re-scaling to adapt to channel-wise variations. And both of them are image-dependent to adapt to image-wise variations, which means during inference our network re-scales and shifts the distributions of activations flexibly for different input images. Based on these methods, we build an enhanced binary neural network for image super-resolution (EBSR).

% According to [3], the difference of activation magnitudes indicates different scaling factors are needed for each pixel.

\subsection{Spatial Re-scaling}
% It is better to use different scaling factors for different pixels to reduce the quantization error and retain more detailed information for image super-resolution. 

% \ml{In the main method, we do not need to introduce the previous works but can focus on introducing our own method. Channel rescaling in Real-to-binary Net is not relevant in this context.}

% Re-scaling the output of binary convolutions was proposed at the birth of BNN in XNOR-Net \cite{rastegari2016xnor} to reduce quantization error and improve accuracy for image classification tasks.
% It is computed as below:
% \begin{equation}
% \mathcal{A} * \mathcal{W} \approx(\operatorname{sign}(\mathcal{A}) \circledast \operatorname{sign}(\mathcal{W})) \odot \mathcal{K} \alpha
% \label{eq:xnor-net rescale}
% \end{equation}
% where $\circledast$ denotes the binary convolution and $\odot$ denotes the element-wise multiplication.
% $\mathcal{A}$, $\mathcal{W}$, $\alpha$, and $\mathcal{K}$ denote the activation, weight, weight scaling factor, and activation scaling factor, respectively.
%  Later in XNOR-Net++ \cite{bulat2019xnor}, Bulat et al. fuse the activation and weight scaling factors into a single one that is learned end-to-end based on gradients and this improves the classification accuracy on ImageNet dataset.

% % It is computed as Eq.~\ref{eq:xnor-net rescale}, where $\circledast$ denotes 
% %  the binary convolution and $\odot$ denotes the element-wise multiplication. The binary convolution of $\mathcal{A}$ and $\mathcal{W}$ is rescaled by the weight scaling factor $\alpha$ and the activation scaling factor $\mathcal{K}$, both of which are calculated analytically.


% \zc{Similarly, you should explain the meaning of A, W and the operators $\circledast$ in the formula}
% Then in Real-to-binary Net \cite{martinez2020training}, Martinez et al. used a data-driven channel re-scaling module that takes the pre-convolution activations as input to predict the activation scaling factor. Unlike that in XNOR-Net++ \cite{bulat2019xnor}, these scaling factors are not fixed during inference but rather inferred from data. By doing this, they further improved the classification accuracy on ImageNet over XNOR-Net++. 
As is shown in Figure \ref{fig:pixel}, activation distributions have large pixel-to-pixel variation in SR networks
and the difference of activation magnitudes indicates different scaling factors are preferred for different pixels.
Inspired by \cite{martinez2020training}, we propose spatial re-scaling to better adapt the network to the spatial variation
of activation distributions in SR networks.
% fit the various pixel-wise distributions in SR networks.
We take the real-valued activations $A$ before convolution as input and predict pixel-wise scaling factors $S(A)$, which re-scale the binary convolution output. Spatial re-scaling process can be formulated as follows:
\begin{equation}
A * W \approx(\operatorname{sign}(A) \circledast \operatorname{sign}(W)) \odot \alpha \odot S(A)
\label{eq:spatial rescale}
\end{equation}
where $\circledast$ denotes 
the binary convolution and $\odot$ denotes the element-wise multiplication. $A$, $W$, $\alpha$, and $S\left(A\right)$ denote real-valued activations, weights, the scaling factor of weights, and the spatial-wise scaling factor of activations respectively. $S\left(A\right) \in \mathbb{R}^{1\times H\times W}$ can be calculated with a convolution and a sigmoid function.
% as $\sigma\left( CONV\left(A\right)\right)$. 
As shown in Figure \ref{fig:method}(a), real-valued activations first go through a convolution layer,
which has an input channel of $C$ and an output channel of 1, 
and then pass through a sigmoid function to produce the scaling factors $S(A)$ along the spatial dimension.
During inference, the scaling factor will change dynamically according to different input feature maps.
By re-scaling binary convolution output using $S(A)$, we can reduce the quantization error and the original pixel-wise information in FP activation
will be preserved much better.
Spatial re-scaling leads to a large PSNR improvement of 0.24 dB (from 30.30 dB to 31.54 dB) on Set5 and 0.22 dB (from 25.09 dB to 25.31 dB)
on Urban100 compared with our strong baseline. 

\subsection{Channel-wise Shifting and Re-scaling}

\begin{table}[!tb]
\centering
\caption{Comparison between whether to fuse channel-wise shifting and re-scaling or not based on our baseline with spatial re-scaling. }
\label{tab:fusing}

\scalebox{0.65}{
\begin{tabular}{c|cc|cc|cc}
\hline
\multirow{2}{*}{Method}     & \multirow{2}{*}{OPs} & \multirow{2}{*}{Params} & \multicolumn{2}{c|}{Set5} & \multicolumn{2}{c}{Urban100} \\ \cline{4-7} 
                            &                      &                         & PSNR        & SSIM        & PSNR          & SSIM         \\ \hline
Baseline + spatial re-scale & 2.16G                & 0.05M                   & 31.54       & 0.883       & 25.31         & 0.759        \\
+ channel-wise shift and re-scale             & 2.34G                & 0.09M                   & 31.61       & 0.885       & 25.35         & 0.761        \\
+ w/ fusing                   & 2.27G                & 0.08M                   & \textbf{31.64}       & \textbf{0.885}       & \textbf{25.36}         & \textbf{0.761}        \\ \hline
\end{tabular}
}
\end{table}

In SR networks, activation distributions exhibit larger channel-to-channel variation (Figure \ref{fig:chl}).
Both the mean and magnitude of the activation distributions vary significantly across channels.
% Thus we use channel-wise shifting and re-scaling to adapt to various channel-wise distributions. 
\cite{martinez2020training} has proposed the data-driven channel re-scaling, 
but our method differs from them in further introducing data-driven thresholds to handle the channel-wise variation of both mean and magnitude.
Since the blocks to generate the scaling factors and thresholds are very similar, we further propose to fuse them into one module.
% and fusing channel-wise shifting and re-scaling into one module.
We evaluate the effect of fusing the two blocks in Table \ref{tab:fusing}.
With channel-wise shifting and re-scaling fused, our models have fewer operations and parameters overhead and slightly higher performance.

For the specific process, we take the real-valued activations as input and predict different thresholds and scaling factors for each channel. They are also image dependent, e.g., $\beta_{i}$ in Eq.\ref{eq:act_binarize} is no longer fixed during inference but generated according to different input feature maps. Channel-wise shifting and re-scaling can be formulated as follows:
\begin{equation}
A * W \approx(\operatorname{sign}(A-C_s(A)) \circledast \operatorname{sign}(W)) \odot \alpha \odot C_r(A)
\label{eq:channel-wise_shift_and_rescale}
\end{equation}
where $\circledast$ denotes 
the binary convolution and $\odot$ denotes the element-wise multiplication. $C_s(A), C_r(A) \in \mathbb{R}^{C\times1\times1}$ denote the channel-wise threshold and scaling factor, respectively. 
We show the block diagram in Figure \ref{fig:method}(b).
The real-valued input feature map is first squeezed to a ${C\times1\times1}$ vector by a global average pooling (GAP) layer.
The subsequent fully connected layers and ReLU learn the channel-wise information and output a ${2C\times1\times1}$ vector.
Then the ${2C\times1\times1}$ vector is split into two ${C\times1\times1}$ vectors.
We use the first $C$ channels as the channel-wise bias and pass the last $C$ channels through a sigmoid layer 
as the channel-wise scaling factor, which are used to shift the real-valued activations and re-scale the binary convolution output, respectively. 


% \ml{We can mention previously, channel-wise re-scale has been proposed. We propose to fuse them. Add the comparison between fuse v.s. no fuse.}

\begin{figure}[!tbp]%
  \centering
    \includegraphics[width=0.4\textwidth]{fig/methods.png}
  
% \subfloat[channel-wise shifting\&re-scale]{
%     \label{subfig:channel-wise shifting and re-scale}
%     \includegraphics[width=0.2\textwidth]{fig/chl shift and rescale.png}
%   }

  \caption{Block diagram for spatial re-scaling, and channel-wise shifting and re-scaling.} 
  % Input A is the real-valued activation tensor and C, H, and W denote its dimension. GAP stands for global average pooling. The reduction ratio r is set to 16 for a better trade-off between the performance and the number of operations and parameters.}
  \label{fig:method}
\end{figure}


\subsection{Network Structure}

Combining the spatial re-scaling and the channel-wise shifting and re-scaling methods, we construct the enhanced convolution layer (E-Conv).
Then we build our EBSR model based on E-Conv.
In Figure \ref{fig:E-conv}, we compare the binary convolution layer used in the baseline network and our proposed E-Conv.
We use spatial and channel-wise scaling factors to re-scale the binary convolution output,
and use channel-wise shifting to learn appropriate thresholds for each channel before binarization.
The scaling factors and threshold used in E-Conv are learnable and depend on the real-valued input activations.
In this way, our proposed EBSR can adapt to pixel-to-pixel, channel-to-channel, and image-to-image variations
to reduce the large binarization error and preserve more details.
% In this way, our proposed E-Conv reduces the large quantization error caused by binarization and keeps the original information of input feature maps to a large extent.


\begin{figure}[!tb]%
  \centering

    \includegraphics[width=0.5\textwidth]{fig/E-conv.png}

  \caption{Comparison of (a) the binary convolution layer with a skip connection used in our baseline network and (b) the proposed E-Conv.}
  \label{fig:E-conv}
\end{figure}


Figure \ref{fig:network} shows the basic block based on the E-Conv and our EBSR composed of the basic blocks. Following existing works, the convolution layers in the head and tail modules are not binarized. We choose the lightweight EDSR which has 16 basic blocks and 64 channels, and EDSR which has 32 basic blocks and 256 channels as our backbones, which correspond to EBSR-light and EBSR, respectively.

\begin{figure}[!tb]%
  \centering
  {
    \includegraphics[width=0.35\textwidth]{fig/network.png}
  }
  
  \caption{The structure of our proposed EBSR.  Convolution layers in purple are real-valued vanilla 3x3 convolutions.}
  \label{fig:network}
\end{figure}

%-------------------------------------------------------------------------
\section{Experiments}
\label{sec:expr}
In this section, we evaluate our ASTR with extensive experiments.
First of all, we introduce implementation details, followed by experiments on five benchmarks and some visualizations.
Finally, we conduct a series of ablation studies to verify the effectiveness of each component.

\subsection{Implementation Details}\label{4.1}
We implement the proposed model in Pytorch~\cite{paszke2019pytorch}.
Our ASTR is trained on the MegaDepth dataset~\cite{li2018megadepth}.
In the training phase, we input images with the size of $832 \times 832$ for training.
The CNN extractor is a deepened ResNet-18~\cite{he2016deep} with features at $1/32$ resolution.
In spot-guided attention, we set the kernel size of local region $l$ to 5 and $k$ to 4 in $\mathrm{topk}$.
Threshold $\theta_c$ in coarse matching is chosen to 0.2.
% Temperature coefficient $\tau$ is 10.
At the fine stage, window size $s_i$ in the reference image is fixed to 5, and window size $s_j$ in the source image will be adaptively calculated according to the depth information.
In particular, $s_j/s_i$ is clamped into $[1, 3]$.
Our network is trained for 15 epochs with a batch size of 8 by Adam~\cite{kingma2014adam} optimizer.
The initial learning rate is $1 \times 10^{-3}$.
In order to establish spot-guided attention efficiently, we implement a highly optimized general sparse attention operator based on CUDA.
Please refer to the Supplementary Material for more details about the operator.
\begin{table}[ht]
	\centering
	\small
	\caption{Evaluation on HPatches~\cite{balntas2017hpatches} for homography estimation.}
	\label{tab:HPatches_result}
	\vspace{-3mm}
	\scalebox{0.75}{
		\begin{tabular}{c l c c c c}
			\hline
            \multirow{2}{*}{Category} &\multicolumn{1}{c}{\multirow{2}{*}{Method}} &\multicolumn{3}{c}{Homography est. AUC} &\multirow{2}{*}{matches} \\
			% after \\: \hline or \cline{col1-col2} \cline{col3-col4} ...
			\cline{3-5}
                    & &{@3px}  &{@5px}  &{@10px} \\
			\hline
            \multirow{5}{*}{Detector-based} &D2Net~\cite{dusmanu2019d2}+NN &23.2 &35.9 &53.6 &0.2K \\

			                                &R2D2~\cite{r2d2}+NN &50.6 &63.9 &76.8 &0.5K \\

                                            &DISK~\cite{tyszkiewicz2020disk}+NN &52.3 &64.9 &78.9 &1.1K \\

                                            &SP~\cite{detone2018superpoint}+SuperGlue~\cite{sarlin2020superglue} &53.9 &68.3 &81.7 &0.6K \\

                                            &Patch2Pix~\cite{zhou2021patch2pix} &46.4 &59.2 &73.1 & 1.0k \\
            \hline
            \multirow{6}{*}{Detector-free} &Sparse-NCNet~\cite{rocco2020efficient} &48.9 &54.2 &67.1 &1.0K \\ 

                                            & COTR~\cite{jiang2021cotr} &41.9 &57.7 &74.0 &1.0K \\
                                            
                                            &DRC-Net~\cite{li20dualrc} &50.6 &56.2 &68.3 &1.0K \\
                                            
                                            &LoFTR~\cite{sun2021loftr} &65.9 &75.6 &84.6 &1.0K \\           

                                            &PDC-Net+~\cite{truong2023pdc} &66.7 & 76.8 & 85.8 & 1.0k \\

			                                &\textbf{ASTR(ours)}       &\bf 71.7 &\bf 80.3 &\bf 88.0 &1.0K   \\
			\hline
		\end{tabular}
	}
\end{table}

\subsection{Homography Estimation}\label{4.2}
\textbf{Dataset and Metric.}
HPatches~\cite{balntas2017hpatches} is a popular benchmark for image matching.
Following~\cite{dusmanu2019d2} , we choose 56 sequences under significant viewpoint changes and 52 sequences with large illumination variation to evaluate the performance of our ASTR trained on MegaDepth~\cite{li2018megadepth}.
We use the same evaluation protocol as LoFTR~\cite{sun2021loftr}.
% \textbf{Metric.} 
We report the area under the cumulative curve (AUC) of the corner error distance up to 3, 5, and 10 pixels, respectively.
We limit the maximum number of output matches to 1k.

\textbf{Results.}
In Table~\ref{tab:HPatches_result}, we can see that our ASTR achieves new state-of-the-art performance on HPatches~\cite{balntas2017hpatches} under all error thresholds, which strongly proves the effectiveness of our method.
ASTR outperforms the best method before ({PDC-net+}~\cite{truong2023pdc}), achieving a large margin of $\textbf{4.4\%}$ under 3 pixels, $\textbf{3.5\%}$ under 5 pixels, and $\textbf{2.5\%}$ under 10 pixels.
Thanks to the proposed spot-guided aggregation module and adaptive scaling module, our method can yield more accurate matches under extreme viewpoint and illumination variations.
% limiting the number of matches.  
% Table~\ref{tab:HPatches_result} summarizes the performance comparison between our ASTR and state-of-the-art image matching methods on HPatches dataset.

\subsection{Relative Pose Estimation}\label{4.3}
\textbf{Dataset and Metric.}
We use MegaDepth~\cite{li2018megadepth} and ScanNet~\cite{dai2017scannet} to demonstrate the performance of our ASTR in relative pose estimation.
MegaDepth~\cite{li2018megadepth} is a large-scale outdoor dataset that contains 1 million internet images of 196 different outdoor scenes.
Each scene is reconstructed by COLMAP~\cite{schonberger2016structure}. 
Depth maps as intermediate results can be converted to ground truth matches.
We sample the same 1500 pairs as ~\cite{sun2021loftr} for testing.
All test images are resized such that their longer dimensions are 1216.
ScanNet~\cite{dai2017scannet} is usually used to validate the performance of indoor pose estimation.
It is composed of monocular sequences with ground truth poses and depth maps.
Wide baselines and extensive textureless regions in image pairs make ScanNet~\cite{dai2017scannet} challenging.
For a fair comparison, we follow the same testing pairs and evaluation protocol as ~\cite{sun2021loftr}.
And all test images are resized to $640 \times 480$.
Note that we use our ASTR trained on MegaDepth~\cite{li2018megadepth} to evaluate its performance on ScanNet~\cite{dai2017scannet}.   
% \textbf{Metric.}
We report the AUC of the pose error at thresholds $(5^{\circ}, 10^{\circ}, 20^{\circ})$, where pose error is the maximum angular error in rotation and translation.
The angular error is computed between the ground truth pose and the predicted pose.

\begin{table}[t]
	\centering
	\small
	\caption{Evaluation on MegaDepth~\cite{li2018megadepth} for outdoor relative position estimation.}
	\label{tab:MegaDepth_result}
	\vspace{-3mm}
	\scalebox{0.75}{
		\begin{tabular}{c l c c c c}
			\hline
            \multirow{2}{*}{Category} &\multicolumn{1}{c}{\multirow{2}{*}{Method}} &\multicolumn{3}{c}{Pose estimation AUC} \\
			% after \\: \hline or \cline{col1-col2} \cline{col3-col4} ...
			\cline{3-5}
                    & &{@$5^\circ$}  &{@$10^\circ$}  &{@$20^\circ$} \\
			\hline
            \multirow{2}{*}{Detector-based} &SP~\cite{detone2018superpoint}+SuperGlue~\cite{sarlin2020superglue} &42.2 &59.0 &73.6 \\
            
                                            &SP~\cite{detone2018superpoint}+SGMNet~\cite{chen2021learning} &40.5 &59.0 &73.6 \\
            \hline
            \multirow{7}{*}{Detector-free} &DRC-Net~\cite{li20dualrc} &27.0 &42.9 &58.3 \\
                                            
                                            &PDC-Net+(H)~\cite{truong2023pdc}  &43.1  &61.9  &76.1 \\
                                            
                                            &LoFTR~\cite{sun2021loftr} &52.8 &69.2 &81.2 \\                          
                                            
                                            &MatchFormer~\cite{wang2022matchformer} &53.3 &69.7 &81.8 \\
                                            
                                            &QuadTree~\cite{tang2022quadtree} & 54.6 & 70.5 & 82.2 \\
                                            
                                            &ASpanFormer~\cite{chen2022aspanformer} &55.3 &71.5 &83.1 \\

			                                &\textbf{ASTR(ours)}       &\textbf{58.4} & \textbf{73.1} &\ \textbf{83.8}  \\
			\hline
		\end{tabular}
	}
\end{table}

\begin{table}[t]
	\centering
	\small
	\caption{Evaluation on ScanNet~\cite{dai2017scannet} for indoor relative position estimation. * indicates models trained on MegaDepth~\cite{li2018megadepth}.}
	\label{tab:ScanNet_result}
	\vspace{-3mm}
	\scalebox{0.75}{
		\begin{tabular}{c l c c c c}
			\hline
            \multirow{2}{*}{Category} &\multicolumn{1}{c}{\multirow{2}{*}{Method}} &\multicolumn{3}{c}{Pose estimation AUC} \\
			% after \\: \hline or \cline{col1-col2} \cline{col3-col4} ...
			\cline{3-5}
                    & &{@$5^\circ$}  &{@$10^\circ$}  &{@$20^\circ$} \\
			\hline
            \multirow{3}{*}{Detector-based} &D2-Net~\cite{dusmanu2019d2}+NN &5.3 &14.5 &28.0 \\

											&SP~\cite{detone2018superpoint}+OANet~\cite{zhang2019learning} &11.8 &26.9 &43.9 \\
											
											&SP~\cite{detone2018superpoint}+SuperGlue~\cite{sarlin2020superglue} &16.2 &33.8 &51.8 \\
            \hline
            \multirow{5}{*}{Detector-free}  &DRC-Net~\cite{li20dualrc}* &7.7 &17.9 &30.5 \\                         
            
                                            &MatchFormer~\cite{wang2022matchformer}* &15.8  &32.0   &48.0 \\

											&LoFTR-OT~\cite{sun2021loftr}* &16.9 &33.6 &50.6 \\
                                            
                                            &Quadtree~\cite{tang2022quadtree}* &19.0 &37.3  &53.5 \\

			                                &\textbf{ASTR(ours)*}       & \textbf{19.4} & \textbf{37.6} & \textbf{54.4}  \\
			\hline
		\end{tabular}
	}
\end{table}

\begin{figure}[t]
    \centering
    \includegraphics[width=1.0\linewidth]{graph/ASTR_comparision_exp.pdf}
    \vspace{-6mm}
	\caption{
	Qualitative results of dense matching on MegaDepth~\cite{li2018megadepth} and ScanNet~\cite{dai2017scannet}.
	% The green color indicates epipolar error within $5 \times 10^{-4}$ for indoor scenes and $1 \times 10^{-4}$ for outdoor scenes (in the normalized image coordinates).
    }\label{fig:qualitative}
    \vspace{-3mm}
\end{figure}

\textbf{Results.}
As shown in Table~\ref{tab:MegaDepth_result}, our ASTR outperforms other state-of-the-art methods on MegaDepth~\cite{li2018megadepth}.
In particular, our ASTR improves by $\textbf{3.1\%}$ in AUC$@5^{\circ}$ and $\textbf{1.6\%}$ in AUC$@10^{\circ}$.
Table~\ref{tab:ScanNet_result} summarizes the performance comparison between the proposed ASTR and state-of-the-art methods on ScanNet~\cite{dai2017scannet}.
Our ASTR ranks first when only considering models not trained on ScanNet~\cite{dai2017scannet}, indicating the impressive generalization of our method.
Thanks to the proposed spot-guided aggregation module and adaptive scaling module, our method can yield more correct matches, resulting in more accurate pose estimation.
In order to further demonstrate the effectiveness of our ASTR, in Figure~\ref{fig:qualitative}, we visually demonstrate the comparison with other methods on the matching result.
Notably, our methods can better handle the challenges such as textureless areas, repetitive patterns, and scale variations.

% Comparison results in Fig.~\ref{fig:comparison} verify that our ASTR follows the consistency principle to obtain accurate matching results.

% our spot-guided aggreagtion module has the ability to focus the interaction area of most pixels to the correct area.
% In contrast, the regions of linear attention are very scattered.
\subsection{Visual Localization}\label{4.4}
\textbf{Dataset and Metric.}
In this experiment, InLoc~\cite{taira2018inloc} and Aachen Day-Night v1.1~\cite{zhang2021reference} are used to verify the ability of our ASTR in visual localization.
InLoc~\cite{taira2018inloc} is an indoor dataset with 9972 RGBD images, of which 329 RGB images are employed as queries for visual localization.
The challenge of InLoc~\cite{taira2018inloc} mainly comes from textureless regions and repetitive patterns under large viewpoint changes.
In Aachen Day-Night v1.1~\cite{zhang2021reference}, 824 day-time images and 191 night-time images are chosen as queries for outdoor visual localization.
Large illumination and viewpoint changes pose challenges for Aachen~\cite{zhang2021reference}.
For both benchmarks, we evaluate the performance of our ASTR trained on MegaDepth~\cite{li2018megadepth} in the same way as ~\cite{sun2021loftr}.
% \textbf{Metric.}
The metrics of Inloc~\cite{taira2018inloc} and Aachen~\cite{zhang2021reference} are the same, which measure the percentage of images registered within given error thresholds.

\begin{table}[t]
	\centering
	\small
	\caption{Visual localization evaluation on the InLoc~\cite{taira2018inloc} benchmark.}
	\label{tab:Inloc_result}
	\vspace{-3mm}
	\scalebox{0.75}{
		\begin{tabular}{l c c}
			\hline
            \multicolumn{1}{c}{\multirow{2}{*}{Method}} & DUC1 & DUC2 \\
			% after \\: \hline or \cline{col1-col2} \cline{col3-col4} ...
			\cline{2-3}
                     & \multicolumn{2}{c}{$\left(0.25m, 10^\circ\right)$ / $\left(0.5m, 10^\circ\right)$ / $\left(1m, 10^\circ\right)$} \\
			\hline
            % \multirow{1}{*}{Detector-based} & SP~\cite{detone2018superpoint}+SuperGlue~\cite{sarlin2020superglue} &  49.0 / 68.7 / 80.8 & 53.4 / 77.1 / 82.4  \\
            Patch2Pix~\cite{zhou2021patch2pix}(w.SP~\cite{sarlin2020superglue}+CAPS~\cite{wang2020learning}) & 42.4 / 62.6 / 76.3 & 43.5 / 61.1 / 71.0 \\
            LoFTR~\cite{sun2021loftr} & 47.5 / 72.2 / 84.8 & 54.2 / 74.8 / \textbf{85.5} \\
            MatchFormer~\cite{wang2022matchformer} & 46.5 / 73.2 / 85.9 & \textbf{55.7} / 71.8 / 81.7 \\
            ASpanFormer~\cite{chen2022aspanformer} & 51.5 / \textbf{73.7} / 86.4 & 55.0 / 74.0 / 81.7 \\
            \textbf{ASTR(ours)} & \textbf{53.0} / \textbf{73.7} / \textbf{87.4} & 52.7 / \textbf{76.3} / 84.0 \\
			\hline
		\end{tabular}
	}
\end{table}

\begin{table}[t]
	\centering
	\small
	\caption{Visual localization evaluation on the Aachen Day-Night benchmark v1.1~\cite{zhang2021reference}.}
	\label{tab:Aachen_result}
	\vspace{-3mm}
	\scalebox{0.75}{
		\begin{tabular}{ l c c}
			\hline
            \multicolumn{1}{c}{\multirow{2}{*}{Method}} & Day & Night \\
			% after \\: \hline or \cline{col1-col2} \cline{col3-col4} ...
			\cline{2-3}
                     & \multicolumn{2}{c}{$\left(0.25m, 2^\circ\right)$ / $\left(0.5m, 5^\circ\right)$ / $\left(1m, 10^\circ\right)$} \\
			\hline
			\multicolumn{3}{l}{\textbf{Localization with matching pairs provided in dataset}} \\
			\hline
            R2D2~\cite{r2d2}+NN & - & 71.2 / 86.9 / 98.9 \\
            ASLFeat~\cite{luo2020aslfeat}+NN & - & 72.3 / 86.4 / 97.9 \\
            SP~\cite{detone2018superpoint}+SuperGlue~\cite{sarlin2020superglue} & - & 73.3 / 88.0 / 98.4 \\
            SP~\cite{detone2018superpoint}+SGMNet~\cite{chen2021learning} & - & 72.3 / 85.3 / 97.9 \\
			\hline
			\multicolumn{3}{l}{\textbf{Localization with matching pairs generated by HLoc}} \\
			\hline
            % SP~\cite{detone2018superpoint}+SuperGlue~\cite{sarlin2020superglue} & 89.8 / 96.1 / 99.4 & 77.0 / 90.6 / 100.0 \\
            LoFTR~\cite{sun2021loftr} & 88.7 / 95.6 / 99.0 & 78.5 / 90.6 / 99.0 \\
            ASpanFormer~\cite{chen2022aspanformer} & 89.4 / 95.6 / 99.0 & 77.5 / 91.6 / 99.0 \\
            AdaMatcher~\cite{huang2022adaptive} & 89.2 / \textbf{95.9}  / \textbf{99.2} & \textbf{79.1} / \textbf{92.1} / \textbf{99.5} \\
			\textbf{ASTR(ours)} & \textbf{89.9} / 95.6 / \textbf{99.2} & 76.4 / \textbf{92.1} / \textbf{99.5}    \\
			\hline
		\end{tabular}
	}
        \vspace{-3mm}
\end{table}

\textbf{Results.}
For InLoc~\cite{taira2018inloc} benchmark, our method achieves the best performance on DUC1 and is on par with state-of-the-art methods on DUC2 (in Tabel~\ref{tab:Inloc_result}).
% For Aachen~\cite{zhang2021reference} benchmark, our ASTR outperforms other methods on Day scenes and performs comparative with others on Night scenes (in Tabel~\ref{tab:Aachen_result}).
For Aachen~\cite{zhang2021reference} benchmark, our ASTR performs comparative with others on Day and Night scenes (in Tabel~\ref{tab:Aachen_result}).
Overall, our method exhibits strong generalization ability in visual localization.

\subsection{Ablation Study}\label{4.5}

\begin{table}[t]
	\centering
	\small
	\caption{Ablation Study of each component on MegaDepth~\cite{li2018megadepth}.}
	\label{tab:ablation_study}
	\vspace{-3mm}
	\scalebox{0.75}{
		\begin{tabular}{c c c c c c c}
    		\hline
    		% \multirow{2}{*}{Method} & 
			\multirow{2}{*}{Index} & \multirow{2}{*}{Multi-Level} & \multirow{1}{*}{Spot-Guided} & \multirow{2}{*}{Scaling}
		    &\multicolumn{3}{c}{Pose estimation AUC} \\
			% after \\: \hline or \cline{col1-col2} \cline{col3-col4} ...
			\cline{5-7}
                & &($l=5,k=4$) & & {@$5^\circ$}  &{@$10^\circ$}  &{@$20^\circ$} \\
            % \hline
		    % LoFTR~\cite{sun2021loftr} & & & & & 45.9 & 63.3 & 76.8 \\
		    \hline
		    % \multirow{4}{*}{ASTR(ours)}
		      1 & & & & 45.6 & 62.2 & 75.3 \\
		      2 & \checkmark & & & 46.7 & 63.1 & 76.3 \\
		      3 & \checkmark & \checkmark & & 47.7 & 64.5 & 77.4\\
		      4 & \checkmark & \checkmark & \checkmark & \bf 48.3 & \bf 65.0 & \bf 77.7\\
		     \hline
		\end{tabular}
	}
\end{table}

To deeply analyze the proposed method, we perform detailed ablation studies on MegaDepth~\cite{li2018megadepth} to evaluate the effectiveness of each component in ASTR.
Here, we use images with a size of 544 for training and evaluation.
As shown in Table~\ref{tab:ablation_study}, we intend to gradually add these components to the baseline.
The baseline (Index-1) we used is slightly different from LoFTR~\cite{sun2021loftr}.
More details can be found in Supplementary Material.

\begin{figure}[t]
    \centering
    \includegraphics[width=1.0\linewidth]{graph/ASTR_heatmaps.pdf}
    \vspace{-6mm}
	\caption{
	Visualization of vanilla and spot-guided cross attention maps on MegaDepth~\cite{li2018megadepth} (outdoor) and ScanNet~\cite{dai2017scannet} (indoor).
    }\label{fig:spot_vis}
\end{figure}

\textbf{Effectiveness of Spot-Guided Aggregation Module.}
We divide the spot-guided aggregation module into multi-level cross attention and spot-guided attention for ablation studies.
We first add vanilla cross attention layers at 1/32 resolution to the baseline (Index-2 in Table~\ref{tab:ablation_study}).
Comparing the results of Index-2 and Index-1, we conclude that 1/32 resolution global interaction across images is beneficial for image matching.
Then, in Index-3, linear attention layers at 1/8 resolution are substituted for the spot-guided attention layers.
The performance of Index-3 is improved compared with Index-2, which verifies the effectiveness of our spot-guided attention.
In Figure~\ref{fig:spot_vis}, we visualize vanilla and our spot-guided cross attention maps for contrast, showing that spot-guided attention can indeed avoid interference from unrelated areas.
% Please refer to Supplementary Material for ablation studies on values of $l$ and $k$.

\begin{table}[t]
	\centering
	\small
	\caption{Ablation Study with different $k$ and $l$ in spot-guided attention on MegaDepth~\cite{li2018megadepth}.}
	\label{tab:k_l_result}
	\vspace{-3mm}
	\scalebox{0.75}{
		\begin{tabular}{c c c c}
    		\hline
    		% \multirow{2}{*}{Method} & 
			\multirow{2}{*}{$k$($l=5$)} &\multicolumn{3}{c}{Pose estimation AUC} \\
			% after \\: \hline or \cline{col1-col2} \cline{col3-col4} ...
			\cline{2-4}
                & {@$5^\circ$}  &{@$10^\circ$}  &{@$20^\circ$} \\
            % \hline
		    % LoFTR~\cite{sun2021loftr} & & & & & 45.9 & 63.3 & 76.8 \\
		    \hline
		    % \multirow{4}{*}{ASTR(ours)}
		      1  & 46.0 & 62.7 & 76.2 \\
			  2  & 47.5 & 64.0 & 77.1\\
		      3  & 47.3 & 63.8 & 76.7 \\
		    %   3  & 47.5 & 64.0 & 77.1\\
		      4  & \bf 47.7 & \bf 64.5 & \bf 77.4\\
			  5  & 47.1 & 63.7 & 77.0\\
			  6  & 46.9 & 63.6 & 76.6\\
		     \hline
		\end{tabular}
	}
	\scalebox{0.85}{
		\begin{tabular}{c c c c}
    		\hline
    		% \multirow{2}{*}{Method} & 
			\multirow{2}{*}{$l$($k=4$)} &\multicolumn{3}{c}{Pose estimation AUC} \\
			% after \\: \hline or \cline{col1-col2} \cline{col3-col4} ...
			\cline{2-4}
                & {@$5^\circ$}  &{@$10^\circ$}  &{@$20^\circ$} \\
            % \hline
		    % LoFTR~\cite{sun2021loftr} & & & & & 45.9 & 63.3 & 76.8 \\
		    \hline
		    % \multirow{4}{*}{ASTR(ours)}
		      3  & 46.7 & 63.2 & 76.1 \\
			  5  & \bf 47.7 & \bf 64.5 & \bf 77.4\\
		      7  & 47.2 & 63.4 & 76.8 \\
		      9  & 43.0 & 60.5 & 74.8\\
		     \hline
		\end{tabular}
	}
\end{table}

% \begin{table}[t]
% 	\centering
% 	\small
% 	\caption{\textbf{Ablation Study with different $l$ in spot-guided attention.} Higher is better.}
% 	\label{tab:MegaDepth_result}
% 	\scalebox{1.0}{
% 		\begin{tabular}{c c c c}
%     		\hline
%     		% \multirow{2}{*}{Method} & 
% 			\multirow{2}{*}{$l$($k=4$)} &\multicolumn{3}{c}{Pos estimation AUC} \\
% 			% after \\: \hline or \cline{col1-col2} \cline{col3-col4} ...
% 			\cline{2-4}
%                 & {@$5^\circ$}  &{@$10^\circ$}  &{@$20^\circ$} \\
%             % \hline
% 		    % LoFTR~\cite{sun2021loftr} & & & & & 45.9 & 63.3 & 76.8 \\
% 		    \hline
% 		    % \multirow{4}{*}{ASTR(ours)}
% 		      3  & 46.7 & 63.2 & 76.1 \\
% 			  5  & \bf 47.7 & \bf 64.5 & \bf 77.4\\
% 		      7  & 47.2 & 63.4 & 76.8 \\
% 		      9  & 43.0 & 60.5 & 74.8\\
% 		     \hline
% 		\end{tabular}
% 	}
% \end{table}

To maximize the effectiveness of our spot-guided attention, we explore how to set suitable parameters $l$ and $k$.
First, in the setting of Index-3, we fix $l=5$ and vary $k$ from 1 to 6.
After observing the results in Table~\ref{tab:k_l_result}, the performance drops when $k$ is smaller than 4 or larger than 4.
Then, we fix $k=4$ and vary $l$ from 3 to 9.
As shown in Table~\ref{tab:k_l_result}, we find that the model achieves the best performance at $l=5$.
The reason may be that the spot area is too small to provide sufficient information from another image when using small $k$ or $l$.
With large $k$ or $l$, for a certain pixel, some matching areas of low confidence or dissimilar points will damage its feature aggregation.

\begin{figure}[t]
    \centering
    \includegraphics[width=1.0\linewidth]{graph/ASTR_grids.pdf}
    \vspace{-6mm}
	\caption{
	Visualization of grids from adaptive scaling module on MegaDepth~\cite{li2018megadepth} (outdoor) and ScanNet~\cite{dai2017scannet} (indoor).
    }\label{fig:grids_vis}
\end{figure}


\textbf{Effectiveness of Adaptive Scaling Module.}
As shown in Table~\ref{tab:ablation_study}, comparing the results of Index-4 and Index-3, we can see that the performance is improved, which indicates that coarse-level matching results are better refined with adaptive scaling module.
In Figure~\ref{fig:grids_vis}, we visualize the cropped grids from adaptive scaling module, indicating that our adaptive scaling module can adaptively crop grids of different sizes according to scale variations.



%------------------------------------------------------------------------
\section{Conclusion}
\label{sec:conclusion}
\section{Conclusion}\label{sec:conclusion}
In this work, we focus on addressing the fundamental challenge of OOD detection tasks, which is how to fully understand the semantic discrepancy between the ID/OOD samples. We reveal that the key to success in the realistic SCOOD task is to allocate as many ID samples in the unlabeled set correctly as possible. To this end, we propose a novel uncertainty-aware optimal transport scheme that introduces class-specific energy scores as guidance for effective label assignment. Experimental results show that our method achieves better performance than previous state-of-the-art methods on SCOOD benchmarks.

\textbf{Limitations.} In addition to temperature scaling, other techniques such as feature clipping applied in ReAct~\cite{sun2021react} also enhance the performance of energy score, so how to obtain an OOD score that best fits the SCOOD task can be further explored. Moreover, a setting highly related to SCOOD has been proposed in \cite{katz2022training} and formulated as a constrained optimization problem. We will also theoretically analyze these practical OOD settings in our feature work.

% \section*{Acknowledgments}
\textbf{Acknowledgments.} 
This work is supported by National Key R\&D Program of China under Grant 2020AAA0105701, National Natural Science Foundation of China (NSFC) under Grants 61872327, Major Special Science and Technology Project of Anhui, National Natural Science Foundation of China (62033012) and Ant Group through Ant Research Intern Program.


\clearpage

%%%%%%%%% TITLE - PLEASE UPDATE
\twocolumn[{
\begin{center}
\textbf{
\Large Adaptive Spot-Guided Transformer for Consistent Local Feature Matching\\
\rule[3pt]{1.0cm}{0.1em}Supplementary Material\rule[3pt]{1.0cm}{0.1em}
}
\end{center}
}]
% \title{
% Adaptive Spot-Guided Transformer for Consistent Local Feature Matching\\
% \rule[3pt]{1.0cm}{0.1em}Supplementary Material\rule[3pt]{1.0cm}{0.1em}
% }

% \maketitle
% \maketitle
In this supplementary material, we first introduce the general sparse attention operator in Section~\ref{sec:1}.
In Section~\ref{sec:2}, we provide some details about our experiment.
In Section~\ref{sec:3}, we show additional visualizations about the spot-guided attention and adaptive scaling modules.

\begin{figure}[ht]
  \centering
  \includegraphics[width=1.0\linewidth]{graph/sparse_attention.pdf}
  \vspace{-6mm}
\caption{
  The illustration of our general sparse attention operator.
  }\label{fig:sparse}
\end{figure}

%%%%%%%%% BODY TEXT
\section{General Sparse Attention Operator}
\label{sec:1}

% Please follow the steps outlined below when submitting your manuscript to the IEEE Computer Society Press.
% This style guide now has several important modifications (for example, you are no longer warned against the use of sticky tape to attach your artwork to the paper), so all authors should read this new version.

% Due to irregular key/value token number for each query in Sopt Attention, naive implementation by PyTorch is not efficient for memory and computation, which use a mask to set unwanted values in the attention map to $0$.
% Instead, inspired by PointNet and Stratified Transformer, we implement a memory and computationally efficient Spot Attention operator using CUDA, which only compute the necessary attention between much less query/key tokens.

% We can divide an vanilla attention operator into 3 steps.
% First, compute attention map $A=QK^T$.
% Then, compute softmax $S=\mathrm{softmax}(A/\sqrt{d_k})$,
% Finally, compute result $O=SV$.

% In step 1, because only few results in $A$ are needed for Spot Attention, we do not directly compute the full $A$.
% Instead, suppose we actually need to compute $L_m$ dot products between all query and key tokens.
% Suppose two arrays $M_q$ and $M_k$, whose length are both $L_m$, and $\left( M_q[i], M_k[i] \right)$ indicate a dot production between $Q[M_q[i]]$ and $K[M_k[i]]$ is needed.
% So we only compute $L_m$ times dot productions and the result is $attn$, which is also an array of $L_m$ length and $attn[i]=Q[M_q[i]]K[M_k[i]]^T$.

% In step 2, we apply scatter $\mathrm{softmax}$ on the elements in $attn$ with the same $M_q[i]$, denote the result as $\mathrm{attn\_softmax}$.

% In step 3, we compute the result
% \begin{equation}\label{eq:ep1}
% \noindent O[q]=\sum_{M_q[i]=q}{\mathrm{attn\_softmax}[i] \cdot V[M_k[i]]}.
% \end{equation}
% All of three steps is implemented in CUDA, see our github repository for more details.

% Comparing to the naive implementation using PyTorch, our highly optimized implementation using CUDA reduce the memory and time complexity from $\mathcal{O}(N_q \cdot N_k \cdot N_h \cdot N_d)$ to $\mathcal{O}(L_m \cdot N_h \cdot N_d)$, where $N_q$ and $N_k$ is the number of query and key tokens, $N_h$ is the number of attention heads, and $N_d$ is the dimension of each head, $N_s$ is the number of token in the spot area.
% Considering $L_m \ll N_q \cdot N_k$, our implementation is much more efficient than naive implementation.
% Furthermore, this implementation can not only be used for Spot Attention, but also to accelerate the case where where the numbers of key/value corresponding to queries is not the same.

Due to irregular key/value token number for each query in Spot Attention, the naive implementation by PyTorch~\cite{paszke2019pytorch} is not efficient for memory and computation, which uses a mask to set unwanted values in the attention map to $0$.
More generally, the same problem also exists when the numbers of key corresponding to queries are not the same.
Inspired by PointNet~\cite{qi2017pointnet} and Stratified Transformer~\cite{lai2022stratified},
%{\color{red}we implement a memory and computationally efficient general sparse attention operator using CUDA,
%%
%which only compute the necessary attention between much less query/key tokens.}
%
%
%{\color{red}
%We can divide an vanilla attention operator into 3 steps.
%First, compute attention map $A=QK^T$.
%Then, compute softmax $S=\mathrm{softmax}(A/\sqrt{d_k})$,
%Finally, compute result $O=SV$.}
%
%
%{\color{red}
%In step 1, because only few results in $A$ are needed for Sparse Attention, we do not directly compute the full $A$.
%Instead, suppose we actually need to compute $L_m$ dot products between all query and key tokens.
%Suppose two arrays $M_q$ and $M_k$, whose length are both $L_m$, and $\left( M_q[i], M_k[i] \right)$ indicate a dot production between $Q[M_q[i]]$ and $K[M_k[i]]$ is needed.
%So we only compute $L_m$ times dot productions and the result is $attn$, which is also an array of $L_m$ length and $attn[i]=Q[M_q[i]]K[M_k[i]]^T$.
%
%
%
%
%In step 2, we apply scatter $\mathrm{softmax}$ on the elements in $attn$ with the same $M_q[i]$, denote the result as $\mathrm{{attn}_s}$.
%
%In step 3, we compute the result
%\begin{equation}\label{eq:ep1}
%\noindent O[q]=\sum_{M_q[i]=q}{{attn}_s[i] \cdot V[M_k[i]]}.
%\end{equation}
%All of three steps is implemented in CUDA.
%
%Comparing to the naive implementation using PyTorch~\cite{paszke2019pytorch}, our highly optimized implementation using CUDA reduce the memory and time complexity from $\mathcal{O}(N_q \cdot N_k \cdot N_h \cdot N_d)$ to $\mathcal{O}(L_m \cdot N_h \cdot N_d)$, where $N_q$ and $N_k$ is the number of query and key tokens, $N_h$ is the number of attention heads, and $N_d$ is the dimension of each head, $N_s$ is the number of token in the spot area or other general sparse attention.
%Considering $L_m \ll N_q \cdot N_k$, our implementation is much more efficient than naive implementation.
%
%In particular, we also calulate the matching matrix in spot-guided attention in this way and set the probability of unrelated pixels to 0, which can greatly reduce the memory and computation costs.
%}
we implement a general sparse attention operator 
using CUDA that is efficient in terms of memory and computation.
We attempt to only compute the necessary attention between much less query/key tokens.


We can divide a vanilla attention operator into 3 steps.
Inputs are grouped as query $Q$, key $K$ and value $V$.
First, the attention map $A$ is computed by dot production as $A=QK^T$.
Then, a softmax operator is performed on the attention map: $A_s=\mathrm{softmax}(A/\sqrt{d_k})$.
Finally, the updated query $O$ can be obtained by $O=A_s V$.
We optimize  these three steps separately.

In the step 1, because only a few results in $A$ are useful for sparse attention, we do not need to compute the full $A$.
Instead, we compute the dot productions between $L_m$ pairs of query and key.
% Given two index arrays $M_q$ and $M_k$, whose length are both $L_m$, and $\left( M_q[i], M_k[i] \right)$ indicate that a dot production between $Q[M_q[i]]$ and $K[M_k[i]]$ is needed.
$M_q$ and $M_k$ record the indexes of query and key tokens whose dot productions are needed.
The length of $M_q$ and $M_k$ are both $L_m$.
% , so we only compute $L_m$ dot productions.
%  and the result is $attn$, which is also an array of $L_m$ length and $attn[i]=Q[M_q[i]]K[M_k[i]]^T$.
Here, we denote the sparse attention map as $attn$, which is calculated by
\begin{equation}\label{eq:ep1}
\noindent attn[i]=Q[M_q[i]]K[M_k[i]]^T, \; i=0,1,\cdots, L_m-1.
\end{equation}

% In step 2, we apply grouped $\mathrm{softmax}$ on the elements in $attn$ with the same $M_q[i]$, denote the result as $\mathrm{{attn}_s}$.
In the step 2, we group the elements in $attn$ with the same query index and apply $\mathrm{softmax}$ on each group.
The result is denoted as $attn_s$.


In the step 3, we compute the updated query
\begin{equation}\label{eq:ep2}
O[q]=\sum_{M_q[i]=q}{{attn}_s[i] \cdot V[M_k[i]]}.
\end{equation}

All of three steps are implemented in CUDA.
% see our github repository for more details.

Compared with the naive implementation using PyTorch~\cite{paszke2019pytorch}, our highly optimized implementation reduces the memory and time complexity from $\mathcal{O}(N_q \cdot N_k \cdot N_h \cdot N_d^2)$ to $\mathcal{O}(L_m \cdot N_h \cdot N_d^2)$, where $N_q$, $N_k$ and $N_h$ are separately the numbers of query tokens, key tokens and attention heads, and $N_d$ is the dimension of each head.
Considering $L_m \ll N_q \cdot N_k$, our implementation is much more efficient than the naive implementation.

In particular, we also calculate the matching matrix in spot-guided attention in this way and set the probability of unrelated pixels to 0, which can greatly reduce the memory and computation cost.


\begin{figure}[t]
  \centering
  \includegraphics[width=0.75\linewidth]{graph/learning_rate.pdf}
  \vspace{-3mm}
\caption{
Learning rate curve while training on MegaDepth~\cite{li2018megadepth}.
  }\label{fig:learning_rate}
\end{figure}

\begin{figure}[t]
  \centering
  \includegraphics[width=1.0\linewidth]{graph/ASTR_spot_supp.pdf}
  \vspace{-6mm}
\caption{
Visualizations of vanilla and spot-guided attention maps on MegaDepth~\cite{li2018megadepth} (outdoor) and ScanNet~\cite{dai2017scannet} (indoor).
  }\label{fig:spot}
\end{figure}

\section{Experimental Details}
\label{sec:2}

\subsection{Training Details}\label{sec:2.1}
To reduce the GPU memory, we randomly sample $50\%$ of ground truth matches to supervise the matching matrix at the coarse stage.
And we sample $20\%$ of the maximum number of coarse-level possible matches at the fine stage.
We train ASTR on MegaDepth~\cite{li2018megadepth} for 15 epochs.
The initial learning rate is $1 \times 10^{-3}$, with a linear learning rate warm-up for 15000 iterations.
The learning rate curve is shown in Figure~\ref{fig:learning_rate}.

\begin{figure}[t]
  \centering
  \includegraphics[width=1.0\linewidth]{graph/ASTR_scale_supp.pdf}
  \vspace{-6mm}
\caption{
Visualizations of grids from adaptive scaling module and corresponding depth maps on MegaDepth~\cite{li2018megadepth}.
Note that we use depth values with scale uncertainty to compose the depth maps.
  }\label{fig:scale}
\end{figure}


\subsection{Differences between Baseline and LoFTR}\label{sec:2.2}
%
%
There are two main differences between our baseline and LoFTR~\cite{sun2021loftr}.

\textbf{(1) Normalized Positional Encoding.}
LoFTR~\cite{sun2021loftr} adopts the absolute sinusoidal positional encoding by following~\cite{carion2020end}:
\begin{equation}\label{eq:ep2}
  % Conf(p_i) = \mathop{max}\limits_{i}\mathop{softmax}\limits_{i}(\{\langle F^{1/8}_{Ref.}(p_i), F^{1/8}_{Src.}(p_j) \rangle\}_{p_j \in R(p_i)}),
\mathrm{PE}_i(x,y) = \left\{
  \begin{array}{ll}
  \mathrm{sin}(w_k \cdot x), & i = 4k \\
  \mathrm{cos}(w_k \cdot x), & i = 4k + 1 \\
  \mathrm{sin}(w_k \cdot y), & i = 4k + 2 \\
  \mathrm{cos}(w_k \cdot y), & i = 4k + 3 \\
  \end{array},
\right.
\end{equation}
where $w_k = \frac{1}{10000^{2k/d}}$, $d$ denotes the number of feature channels and $i$ is the index for feature channels.
Considering the gap in image resolution between training and testing, we utilize the normalized positional encoding as~\cite{chen2022aspanformer}, which is proven to mitigate the impact of image resolution changes in~\cite{chen2022aspanformer}.
The normalized positional encoding $\mathrm{NPE}_i(\cdot,\cdot)$ can be expressed as
\begin{equation}\label{eq:ep2}
\mathrm{NPE}_i(x,y) = \mathrm{PE}_i(x * \frac{W_{train}}{W_{test}}, y * \frac{H_{train}}{H_{test}}),
\end{equation}
where $W_{train/test}$ and $H_{train/test}$ are width and height of training/testing images.

\textbf{(2) Convolution in Attention.}
Chen et al.~\cite{chen2022aspanformer} find that replacing the self attention with convolution can improve the performance.
Hence, we deprecate self attention and MLP,  and utilize a $3 \times 3$ convolution in our ASTR.

\subsection{CNN Backbone}\label{sec:2.3}
Here we leverage a deepened version of Feature Pyramid Network (FPN)~\cite{lin2017feature}, which achieves a minimum resolution of 1/32.
% The number of feature channels at each stage is [128,128,196,256,256,256].
The initial dimension for the stem is still 128 as LoFTR~\cite{sun2021loftr}, and the number of feature channels for subsequent stages is [128, 196, 256, 256, 256].

\section{Visualization Results}\label{sec:3}

In Figure~\ref{fig:spot}, we pick up two similar adjacent pixels as queries and visualize the corresponding attention maps of vanilla and our spot-guided attention for comparison.
%
The vanilla attention mechanism is vulnerable to repetitive textures, while our spot-guided attention can focus on the correct areas in these repeated texture regions.
%
Because large scale variation occurs frequently on outdoor datasets,
%
we mainly visualize the grids from the adaptive scaling module and corresponding depth maps on MegaDepth~\cite{li2018megadepth}.
% With adaptive scaling module, our ASTR can handle large scale changes well.
% Our adaptive scaling module can produce aligned grids according to depth information.
As shown in Figure~\ref{fig:scale}, our adaptive scaling module can adjust the size of grids according to depth information.


%%%%%%%%% REFERENCES
{\small
\bibliographystyle{ieee_fullname}
\bibliography{CVPR2023bib}
}

\end{document}
