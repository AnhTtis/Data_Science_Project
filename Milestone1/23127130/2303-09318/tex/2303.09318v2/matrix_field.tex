%% LyX 2.3.7 created this file.  For more info, see http://www.lyx.org/.
%% Do not edit unless you really know what you are doing.
\documentclass[english]{article}
\renewcommand{\familydefault}{\rmdefault}
\usepackage[LGR,T1]{fontenc}
\usepackage[latin9]{inputenc}
\usepackage{geometry}
\geometry{verbose,tmargin=3cm,bmargin=2.5cm,lmargin=3cm,rmargin=2cm}
\usepackage{color}
\usepackage{babel}
\usepackage{refstyle}
\usepackage{float}
\usepackage{fancybox}
\usepackage{calc}
\usepackage{amsmath}
\usepackage{amsthm}
\usepackage{amssymb}
\usepackage{graphicx}
\usepackage{wasysym}
\usepackage[all]{xy}
\PassOptionsToPackage{normalem}{ulem}
\usepackage{ulem}
\usepackage[unicode=true,
 bookmarks=true,bookmarksnumbered=false,bookmarksopen=false,
 breaklinks=false,pdfborder={0 0 1},backref=false,colorlinks=true]
 {hyperref}
\hypersetup{pdftitle={The Conservative Matrix Field},
 pdfauthor={Ofir David},
 pdfkeywords={generalized continued fraction, irrationality}}

\makeatletter

%%%%%%%%%%%%%%%%%%%%%%%%%%%%%% LyX specific LaTeX commands.

\AtBeginDocument{\providecommand\secref[1]{\ref{sec:#1}}}
\AtBeginDocument{\providecommand\subsecref[1]{\ref{subsec:#1}}}
\AtBeginDocument{\providecommand\claimref[1]{\ref{claim:#1}}}
\AtBeginDocument{\providecommand\lemref[1]{\ref{lem:#1}}}
\AtBeginDocument{\providecommand\thmref[1]{\ref{thm:#1}}}
\AtBeginDocument{\providecommand\corref[1]{\ref{cor:#1}}}
\AtBeginDocument{\providecommand\exaref[1]{\ref{exa:#1}}}
\AtBeginDocument{\providecommand\defref[1]{\ref{def:#1}}}
\AtBeginDocument{\providecommand\remref[1]{\ref{rem:#1}}}
\AtBeginDocument{\providecommand\eqref[1]{\ref{eq:#1}}}
\AtBeginDocument{\providecommand\appref[1]{\ref{app:#1}}}
\AtBeginDocument{\providecommand\enuref[1]{\ref{enu:#1}}}
\AtBeginDocument{\providecommand\factref[1]{\ref{fact:#1}}}
\DeclareRobustCommand{\greektext}{%
  \fontencoding{LGR}\selectfont\def\encodingdefault{LGR}}
\DeclareRobustCommand{\textgreek}[1]{\leavevmode{\greektext #1}}
\ProvideTextCommand{\~}{LGR}[1]{\char126#1}

\RS@ifundefined{subsecref}
  {\newref{subsec}{name = \RSsectxt}}
  {}
\RS@ifundefined{thmref}
  {\def\RSthmtxt{theorem~}\newref{thm}{name = \RSthmtxt}}
  {}
\RS@ifundefined{lemref}
  {\def\RSlemtxt{lemma~}\newref{lem}{name = \RSlemtxt}}
  {}


%%%%%%%%%%%%%%%%%%%%%%%%%%%%%% Textclass specific LaTeX commands.
\theoremstyle{plain}
\newtheorem{thm}{\protect\theoremname}
\theoremstyle{remark}
\newtheorem{rem}[thm]{\protect\remarkname}
\theoremstyle{definition}
\newtheorem{defn}[thm]{\protect\definitionname}
\theoremstyle{plain}
\newtheorem{lem}[thm]{\protect\lemmaname}
\theoremstyle{plain}
\newtheorem{claim}[thm]{\protect\claimname}
\theoremstyle{plain}
\newtheorem{cor}[thm]{\protect\corollaryname}
\theoremstyle{definition}
\newtheorem{example}[thm]{\protect\examplename}
\theoremstyle{definition}
\newtheorem{notation}[thm]{\protect\notationname}
\theoremstyle{plain}
\newtheorem{fact}[thm]{\protect\factname}

\@ifundefined{date}{}{\date{}}
%%%%%%%%%%%%%%%%%%%%%%%%%%%%%% User specified LaTeX commands.
\hypersetup{pdfstartview=}
%
\newref{def}{name=Definition~}
\newref{ex}{name=Example~}
\newref{eq}{name=Equation~}
\newref{exa}{name=Example~}
\newref{sub}{name=Subsection~}
\newref{cor}{name=Corollary~}
\newref{rem}{name=Remark~}
\newref{prop}{name=Proposition~}
\newref{lem}{name=Lemma~}
\newref{thm}{name=Theorem~}
\newref{sec}{name=Section~}
\newref{claim}{name=Claim~}
\newref{app}{name=Appendix~}

\hypersetup{citecolor=blue}

\newcommand{\xyR}[1]{
  \xydef@\xymatrixrowsep@{#1}}
\newcommand{\xyC}[1]{
  \xydef@\xymatrixcolsep@{#1}}

\makeatother

\providecommand{\claimname}{Claim}
\providecommand{\corollaryname}{Corollary}
\providecommand{\definitionname}{Definition}
\providecommand{\examplename}{Example}
\providecommand{\factname}{Fact}
\providecommand{\lemmaname}{Lemma}
\providecommand{\notationname}{Notation}
\providecommand{\remarkname}{Remark}
\providecommand{\theoremname}{Theorem}

\begin{document}
\global\long\def\call{\mathcal{L}}%
\global\long\def\nn{\mathcal{N}}%
\global\long\def\ff{\mathcal{F}}%
\global\long\def\aa{\mathcal{A}}%
\global\long\def\RR{\mathbb{R}}%
\global\long\def\EE{\mathbb{E}}%
\global\long\def\CC{\mathbb{C}}%
\global\long\def\QQ{\mathbb{Q}}%
\global\long\def\ZZ{\mathbb{Z}}%
\global\long\def\NN{\mathbb{N}}%
\global\long\def\KK{\mathbb{K}}%
\global\long\def\SL{\mathrm{SL}}%
\global\long\def\GL{\mathrm{GL}}%
\global\long\def\ds{\mathrm{ds}}%
\global\long\def\dnu{\mathrm{d\nu}}%
\global\long\def\dmu{\mathrm{d\mu}}%
\global\long\def\dt{\mathrm{dt}}%
\global\long\def\dw{\mathrm{dw}}%
\global\long\def\dx{\mathrm{dx}}%
\global\long\def\dy{\mathrm{dy}}%
\global\long\def\norm#1{\left\Vert #1\right\Vert }%
\global\long\def\limfi#1{{\displaystyle \lim_{#1\to\infty}}}%
\global\long\def\arrfi#1{\overset{#1\to\infty}{\longrightarrow}}%
\global\long\def\flr#1{\left\lfloor #1\right\rfloor }%
\global\long\def\lcm{\mathrm{lcm}}%

\title{\textbf{\huge{}The conservative matrix field}}
\author{Ofir David\\
The Ramanujan Machine team, Faculty of Electrical and Computer Engineering,\\
Technion - Israel Institute of Technology, Haifa 3200003, Israel}
\maketitle
\begin{abstract}
We present a new structure called the \textquotedbl conservative
matrix field,\textquotedbl{} initially developed to elucidate and
provide insight into the methodologies employed by Ap\'ery in his
proof of the irrationality of \textgreek{z}(3). This framework is
also applicable to other well known mathematical constants, including
$e,\;\pi,\;\ln\left(2\right)$, and more, and can be used to study
their properties. Moreover, the conservative matrix field exhibits
inherent connections to various ideas and techniques in number theory,
thereby indicating promising avenues for further applications and
investigations.
\end{abstract}

\section{Introduction}

The Riemann zeta function $\zeta\left(s\right)$ is a complex valued
function that plays a crucial role in mathematics. It is defined as
$\zeta\left(s\right)=\sum_{n=1}^{\infty}\frac{1}{n^{s}}$ for complex
numbers $s$ with $Re\left(s\right)>1$ and can be extended analytically
to all of $\CC$ with a simple pole at $s=1$. In particular, the
values $\zeta\left(d\right)$ for integers $d\geq2$ have significant
implications in a number of areas, e.g. $\frac{1}{\zeta\left(d\right)}$
is the probability of choosing a random integer which is not divisible
by $m^{d}$ for some integer $m\geq1$. While the even evaluations
$\zeta\left(2d\right)$ are well understood, with $\zeta\left(2\right)=\frac{\pi^{2}}{6}$
and more generally $\frac{\zeta\left(2d\right)}{\pi^{2d}}$ are rational
numbers, the behavior of odd evaluations $\zeta\left(2d+1\right)$
remains largely unknown.

One of the main results about these odd evaluations was in 1978 where
Ap\'ery showed that $\zeta\left(3\right)$ is irrational \cite{apery_irrationalite_1979}.
As Ap\'ery's proof was complicated, subsequent attempts were made
to explain and simplify it, see for example van der Poorten \cite{van_der_poorten_proof_1979},
and others tried to reprove it all together, e.g. Beukers in \cite{beukers_note_2013}.
Further research \cite{rivoal_fonction_2000,zudilin_one_2001} has
also shown that there are infinitely many odd integers for which $\zeta\left(2d+1\right)$
is irrational, and in particular at least one of $\zeta\left(5\right),\zeta\left(7\right),\zeta\left(9\right)$,
and $\zeta\left(11\right)$ is irrational.

One of the main goal of this paper is to provide some motivation to
the steps used in Ap\'ery's proof through the creation of a novel
mathematical structure referred to as the \textbf{conservative matrix
field}. This structure will allow us to understand Ap\'ery's original
proof and provide a framework for studying other natural constants
such as $\zeta\left(2\right),\pi$, and $e$, with the potential to
uncover new relationships and properties among them. Moreover, while
we will mainly work over the integers, this structure seems to have
natural generalizations for general metric fields.

\medskip{}

The conservative matrix field structure is based on generalized continued
fractions, which are number presentations of the form
\[
\KK_{1}^{\infty}\frac{b_{k}}{a_{k}}:=\frac{b_{1}}{a_{1}+\frac{b_{2}}{a_{2}+\frac{b_{3}}{a_{3}+\ddots}}}=\limfi n\frac{b_{1}}{a_{1}+\frac{b_{2}}{a_{2}+\frac{b_{3}}{\ddots+\frac{b_{n}}{a_{n}+0}}}}\qquad a_{i},b_{i}\in\CC,
\]
where the numbers in the sequence on the left for which we look for
the limit are called the \textbf{convergents} of that expansion.

Their much more well known cousins, the simple continued fractions
where $b_{k}=1$ and $a_{k}\geq1$ are integers, have been studied
extensively and are connected to many research areas in mathematics
and in general. In particular, the original goal of these continued
fractions was to find the ``best'' rational approximations for a
given irrational number, which are given by the convergents defined
above.

While irrational numbers have a unique simple continued fraction expansion,
and rationals have two expansions, there can be many presentations
in the generalized version (more details in \secref{Generalized-continued-fractions}).
The uniqueness in the simple continued fraction expansion allows us
to extract a lot of information from the expansion, and while we lose
this property in the generalized form, what we gain is the option
to find ``nice'' generalized continued fractions which are easier
to work with. In particular, we are interested in \textbf{polynomial
continued fractions} where $a_{k}=a\left(k\right),\;b_{k}=b\left(k\right)$
with $a,b\in\ZZ\left[x\right]$.

For example, in the $\zeta\left(3\right)$ case, the simple continued
fraction is 
\[
\zeta\left(3\right)=[1;4,1,18,1,1,1,4,1,9,...]=1+\frac{1}{4+\frac{1}{1+\frac{1}{18+\frac{1}{1+\ddots}}}},
\]
where the coefficients $1,4,1,18,1,...$ don't seem to have any usable
pattern. However, it has a much simpler generalized continued fraction
form
\[
\zeta\left(3\right)=\frac{1}{1+\KK_{1}^{\infty}\frac{-i^{6}}{i^{3}+\left(1+i\right)^{3}}}=\frac{1}{1-\frac{1^{6}}{1^{3}+2^{3}-\frac{2^{6}}{2^{3}+3^{3}-\frac{3^{6}}{3^{3}+4^{3}-{\scriptscriptstyle \ddots}}}}},
\]
where the convergents in this expansion are the standard approximations
$\sum_{1}^{n}\frac{1}{k^{3}}$ for $\zeta\left(3\right)$. Moreover,
this abundance of presentations allows us to find many presentations
for $\zeta\left(3\right)$ which can be combined together to find
a ``good enough'' presentation where the convergents converge fast
enough to prove that $\zeta\left(3\right)$ is irrational. In particular,
in Ap\'ery's original proof, and in our, we eventually show that
\[
\zeta\left(3\right)=\frac{6}{5+\KK_{1}^{\infty}\frac{-k^{6}}{17\left(k^{3}+\left(1+k\right)^{3}\right)-12\left(k+\left(1+k\right)\right)}}.
\]
\smallskip{}

The irrationality proof uses a very elementary argument (see \secref{Irrationality-testing})
that shows that if $\frac{p_{n}}{q_{n}}\to L$ where $\frac{p_{n}}{q_{n}}$
are reduced rational numbers with $\left|q_{n}\right|\to\infty$,
and $\left|L-\frac{p_{n}}{q_{n}}\right|=o\left(\frac{1}{\left|q_{n}\right|}\right)$,
then $L$ must be irrational. Moreover, we can measure how irrational
$L$ is by looking for $\delta>0$ such that $\left|L-\frac{p_{n}}{q_{n}}\right|\sim\frac{1}{\left|q_{n}\right|^{1+\delta}}$.
The main object of this study, the \textbf{conservative matrix field}
defined in \secref{Definition-properties-CMF},\textbf{ }is an algebraic
object that collects infinitely many related such approximations $\frac{p_{n,m}}{q_{n,m}}$
arranged on the integer lattice in the positive quadrant. Computing
$\delta_{n,m}$ for each approximation, namely $\delta_{n,m}=-1-\frac{\ln\left|L-\frac{p_{n,m}}{q_{n,m}}\right|}{\ln\left|q_{n,m}\right|}$
where the rational $\frac{p_{n,m}}{q_{n,m}}$ is reduced, and plotting
them as a heat map we get the following

\begin{figure}[H]
\begin{centering}
\includegraphics[scale=0.25]{heat_map_zeta3.jpeg}
\par\end{centering}
\caption{(Figure by Rotem Elimelech) The gradient color from red$\to$white$\to$blue
correspond to $\delta_{n,m}=-1-\log_{q_{n,m}}\left|L-\frac{p_{n,m}}{q_{n,m}}\right|$
going from positive$\to$zero$\to$negative.}
\end{figure}

As we shall see, the $X$-axis and $Y$-axis correspond more or less
to the standard approximations of $\zeta\left(3\right)$, namely $\sum_{1}^{n}\frac{1}{k^{3}}$,
which do not converge fast enough to show irrationality, while on
the diagonal we get the expansion mentioned above used by Ap\'ery
to prove the irrationality.

\medskip{}

The conservative matrix field structure is not only a way to understand
Ap\'ery's original proof, but seems to have a much broader range
of applications. There are many places that study generalized continued
fraction and in particular polynomial continued fractions (see for
example \cite{apostol_introduction_1998,jones_continued_1980,pincherle_delle_1894,laughlin_real_2004})
. This paper originated in the Ramanujan machine project \cite{raayoni_generating_2021}
which aimed to find polynomial continued fraction presentations to
interesting mathematical constants using computer automation. With
the goal of trying to prove many of the conjectures discovered by
the computer, and along the way understand Ap\'ery's proof, this
conservative matrix field structure was found. These computer conjectures
suggest that there is still much to be explored in this field and
that this new structure is just a step towards a deeper understanding
of mathematical constants and their relations.

\subsection{Structure of the paper}

We present an overview of the paper's structure to delineate the development
of the conservative matrix field. The construction itself is in \secref{Definition-properties-CMF},
and while you don't necessarily need the constructions and results
in the previous sections, we believe that they provide interesting
background and motivate the conservative matrix field structure.

\textbf{Irrationality Testing (}\secref{Irrationality-testing}):
We begin by exploring the concept of good rational approximations
and their role in irrationality testing. These approximations are
closely linked to simple continued fractions, but also still close
to other continued fraction versions.

\textbf{The polynomial continued fractions (}\secref{Generalized-continued-fractions}):
We provide definitions for various types of continued fractions, with
a specific focus on polynomial continued fractions. We also recall
the Mobius action, a key tool for studying these continued fractions
and utilizing them in irrationality proofs. In \subsecref{Euler's-formula}
we employ Euler's method to construct polynomial continued fraction
expansions for well-known constants. These \textquotedbl Euler continued
fractions\textquotedbl{} serve as the fundamental building blocks
of the conservative matrix field, where by combining infinitely many
such continued fractions, we generate new rational approximations.

\textbf{Generalization to Polynomial Matrices} (\secref{The-most-general}):
Here, we extend the concept of polynomial continued fractions to encompass
general polynomial matrices. This broader approach enables the study
of families of polynomial continued fractions and their interrelations,
eventually leading to the definition of the conservative matrix field.

\textbf{The conservative matrix field }(\secref{Definition-properties-CMF}):
After defining the conservative matrix field, we present an intriguing
matrix field construction in \subsecref{matrix-field-construction}
with several interesting properties, where in particular each such
matrix field is accompanied by a dual matrix field, elaborated upon
in \subsecref{The-dual-matrix-field}.

\textbf{The $\zeta\left(3\right)$ matrix field} (\secref{The-z3-case}):
In this section, we explore $\zeta\left(3\right)$ matrix field's
structure, its intriguing properties, and its connections to the irrationality
proof of \textgreek{z}(3). The natural questions arising from this
study often parallel ideas presented in Ap\'ery's proof.

\textbf{Future Directions and Connections} (\secref{On-future-fractions}):
As a conclusion, we propose potential avenues for further development
of the conservative matrix field, and its potential connections to
other fields within number theory, hinting at exciting possibilities
for future research.

\newpage{}

\section{\label{sec:Irrationality-testing}Rational Approximations and Irrationality
Testing}

The simple continued fractions were first introduced in order to find
good rational approximations for a given number, and in a sense also
the best rational approximations. Interestingly enough, the numbers
which have ``very good'' rational approximations are exactly the
irrational numbers. This idea of proving irrationality through good
rational approximations is quite general and will be one of the main
applications to polynomial continued fractions, and the conservative
matrix field which we introduce in this paper. As such, we being with
this irrationality test.

\medskip{}

Any real number $L$ can be approximated by rational numbers. More
over, for any denominator $q\in\NN$ there is a numerator $p\in\ZZ$
such that $\left|L-\frac{p}{q}\right|\leq\frac{1}{q}$. Looking for
approximations where the error is much smaller than $\frac{1}{q}$
is the starting point of the study of Diophantine approximations.
One of the basic results in this field is Dirichlet's theorem, which
states that if we can choose the denominator intelligently, then there
are approximations with much smaller error.
\begin{thm}[Dirichlet's theorem for Diophantine approximation]
Given any number $L\in\RR$, there are infinitely many $p_{n},q_{n}\in\ZZ$
with $q_{n}\to\infty$ such that 
\[
\left|L-\frac{p_{n}}{q_{n}}\right|\leq\frac{1}{q_{n}^{2}}.
\]
\end{thm}

\begin{rem}
One of the important properties of simple continued fractions, is
that they can generate such approximations as in Dirichlet's theorem.
\end{rem}

Dirichlet's theorem is trivial when $L=\frac{p}{q}$ is rational by
simply taking $\left(p_{n},q_{n}\right)=\left(np,nq\right)$. However if we also require that the $\frac{p_{n}}{q_{n}}$ are
distinct, then this theorem is no longer trivial, or even true. In
this case, once $L=\frac{p}{q}\neq\frac{p_{n}}{q_{n}}$ we get that
\[
\left|L-\frac{p_{n}}{q_{n}}\right|=\left|\frac{p\cdot q_{n}-q\cdot p_{n}}{q\cdot q_{n}}\right|\geq\frac{1}{\left|q\cdot q_{n}\right|},
\]
where the emphasize is that $q$ is constant, so that $\left|q_{n}L-p_{n}\right|\geq\frac{1}{q}$
is bounded from below. When $L$ is irrational, it is easy to see
that the approximations $\frac{p_{n}}{q_{n}}$ cannot become constant
at any point, leading us to:

\noindent\doublebox{\begin{minipage}[t]{1\columnwidth - 2\fboxsep - 7.5\fboxrule - 1pt}%
\begin{align*}
L\in\QQ\; & \Rightarrow\quad\text{There is }C>0\text{ such that }\left|qL-p\right|\geq C\text{ for any }\frac{p}{q}\neq L\\
L\notin\QQ\; & \Rightarrow\quad\text{There are }\frac{p_{n}}{q_{n}}\in\QQ\text{ distinct such that }\left|q_{n}L-p_{n}\right|\to0.
\end{align*}
%
\end{minipage}}

In general, when looking for rational approximations with $p_{n},q_{n}$
as above, they are not going to be coprime. Letting $\tilde{q}_{n}=\frac{q_{n}}{\gcd\left(p_{n},q_{n}\right)},\;\tilde{p}_{n}=\frac{p_{n}}{\gcd\left(p_{n},q_{n}\right)}$
, we see that 
\[
\gcd\left(q_{n},p_{n}\right)\left|\tilde{q}_{n}L-\tilde{p}_{n}\right|=\left|q_{n}L-p_{n}\right|,
\]
so it is enough that $\left|q_{n}L-p_{n}\right|=o\left(\gcd\left(p_{n},q_{n}\right)\right)$
in order to get the condition $\left|\tilde{q}_{n}L-\tilde{p}_{n}\right|\to0$
above to hold. Since $\frac{\tilde{p}_{n}}{\tilde{q}_{n}}=\frac{p_{n}}{q_{n}}$,
we obtain the following irrationality test:
\begin{thm}
\label{thm:improved-rationality-test}Suppose that $p_{n},q_{n}\in\ZZ$
where $\frac{p_{n}}{q_{n}}$ is not eventually constant. If $\left|q_{n}L-p_{n}\right|=o\left(\gcd\left(p_{n},q_{n}\right)\right)$,
then $L$ is irrational.
\end{thm}

The importance of rational approximations derived from polynomial
continued fractions, as we shall see in \claimref{upper-bound}, is
that it gives us an upper bound on the error on the one hand, and
an easy way to check that $\frac{p_{n}}{q_{n}}$ is not eventually
constant on the other hand.

\medskip{}


\subsection*{The $\zeta\left(3\right)$ case:}

As an example, let us consider $\zeta\left(3\right)$ which Apery
has shown to be irrational. We start with its standard rational approximations
\[
\frac{p_{n}}{q_{n}}:=\sum_{1}^{n}\frac{1}{k^{3}}\overset{n\to\infty}{\longrightarrow}\zeta\left(3\right).
\]

Multiplying the denominators, we can write $q_{n}=\left(n!\right)^{3}$
and $p_{n}=\sum_{1}^{n}\left(\frac{n!}{k}\right)^{3}$, both of which
in $\ZZ$. Trying to apply the irrationality test from the previous
section, we get that 
\[
\left|\zeta\left(3\right)-\frac{p_{n}}{q_{n}}\right|=\sum_{n+1}^{\infty}\frac{1}{k^{3}}=\Theta\left(\frac{1}{n^{2}}\right).
\]
This is, of course, far from what we need to prove irrationality,
since $\frac{1}{n^{2}}$ is much larger than $\frac{1}{q_{n}}=\frac{1}{\left(n!\right)^{3}}$.
Indeed, with denominator $q_{n}=\left(n!\right)^{3}$ we can always
find $p_{n}'$ such that $\left|\zeta\left(3\right)-\frac{p_{n}'}{q_{n}}\right|\leq\frac{1}{\left(n!\right)^{3}}\ll\frac{1}{n^{2}}$.
Hence, while the rational approximations choice above is easy to use
in general, it is not good enough to show irrationality. 

One way to improve the approximation, as mentioned before, is by moving
to a reduced form of $\frac{p_{n}}{q_{n}}$. Taking instead the common
denominator in the sum above, we get that $\tilde{q}_{n}=\lcm\left[n\right]^{3}$,
where $\lcm\left[n\right]:=\lcm\left\{ 1,2,...,n\right\} $ and then
$\tilde{p}_{n}=\sum_{1}^{n}\left(\frac{\lcm\left[n\right]}{k}\right)^{3}$.
It is well known that $\ln\left(\lcm\left[n\right]\right)=n+o\left(n\right)$
(it follows from the prime number theorem, see \cite{apostol_introduction_1998}),
so that the new denominator $\tilde{q}_{n}=\lcm\left[n\right]^{3}\sim e^{3n}$
is much smaller than $\left(n!\right)^{3}$ and we basically get a
factorial reduction.

However, the error is still too big $\frac{1}{n^{2}}\gg\frac{1}{e^{3n}}$,
so even with this improvement, it is still not enough. One way to
improve this approximation even further, is to consider $\frac{p_{n}}{q_{n}}=\left(\sum_{1}^{n-1}\frac{1}{k^{3}}\right)+\frac{1}{2n^{2}}$.
While it doesn't change the common denominator too much, the error
becomes smaller, since
\[
\left|\zeta\left(3\right)-\frac{p_{n}}{q_{n}}\right|=\left|\left(\sum_{n}^{\infty}\frac{1}{k^{3}}\right)-\frac{1}{2n^{2}}\right|=\left|\left(\sum_{n}^{\infty}\frac{1}{k^{3}}\right)-\int_{n}^{\infty}\frac{1}{x^{3}}\dx\right|\leq\frac{1}{n^{3}}.
\]

While there is still a lot of room for improvement, there is reason
for optimism. The challenge lies in automating the process indefinitely
for all $r\in\NN$, aiming to reduce the error to $\frac{1}{n^{r}}$
for arbitrarily large $r$, with the hope of somewhere in the limit
getting the error to be small enough for our irrationality test. 

The polynomial continued fractions become crucial at this juncture.
They not only furnish rational approximations but also offer iterative
means to refine them repeatedly, where in some cases leading to instances
where these improved approximations are good enough to prove the desired
irrationality.

\newpage{}

\section{\label{sec:Generalized-continued-fractions}The polynomial continued
fraction}

\subsection{\label{subsec:The-definitions}The definitions and main tools}

We start with a generalization of the simple continued fractions,
which unsurprisingly, is called generalized continued fractions. These
can be defined over any topological field, though here we focus on
the complex field with its standard Euclidean metric, and more specifically
when the numerators and denominators are integers.
\begin{defn}[\textbf{(Generalized) continued fractions}]
Let $a_{n},b_{n}$ be a sequence of complex numbers. We will write
\[
a_{0}+\KK_{1}^{n}\frac{b_{i}}{a_{i}}:=a_{0}+\cfrac{b_{1}}{a_{1}+\cfrac{b_{2}}{a_{2}+\cfrac{b_{3}}{\ddots+\cfrac{b_{n}}{a_{n}+0}}}}\in\CC\cup\left\{ \infty\right\} ,
\]
and if the limit as $n\to\infty$ exists, we also will write
\[
a_{0}+\KK_{1}^{\infty}\frac{b_{i}}{a_{i}}:=\limfi n\left(a_{0}+\KK_{1}^{n}\frac{b_{i}}{a_{i}}\right).
\]

We call this type of expansion (both finite and infinite) a \textbf{continued
fraction expansion. }In case that $a_{0}+\KK_{1}^{\infty}\frac{b_{i}}{a_{i}}$
exists, we call the finite part $a_{0}+\KK_{1}^{n-1}\frac{b_{i}}{a_{i}}$
the \textbf{$n$-th convergents} for that expansion.

A continued fraction is called \textbf{simple continued fraction},
if $b_{i}=1$ for all $i$, $a_{0}\in\ZZ$ and $1\leq a_{i}\in\ZZ$
are positive integers for $i\geq1$.

A continued fraction is called \textbf{polynomial continued fraction},
if $b_{i}=b\left(i\right),\;a_{i}=a\left(i\right)$ for some polynomials
$a\left(x\right),b\left(x\right)\in\CC\left[x\right]$ and all $i\geq1$.

\medskip{}
\end{defn}

Simple continued fraction expansion is one of the main and basic tools
used in number theory when studying rational approximations of numbers,
and their convergents satisfy the inequality in Dirichlet's theorem
that we mentioned before (for more details, see chapter 3 in \cite{einsiedler_ergodic_2013}).
The coefficients $a_{i}$ in that expansion can be found using a generalized
Euclidean division algorithm, and it is well known that a number is
rational if and only if its simple continued fraction expansion is
finite. However, while we have an algorithm to find the (almost) unique
expansion, in general they can be very complicated without any known
patterns, even for ``nice'' numbers, for example:

\[
\pi=[3;7,15,1,292,1,1,1,2,1,3,1,14,2,1,1,2,2,2,2,1,84,...]=3+\cfrac{1}{7+\cfrac{1}{15+\cfrac{1}{1+\ddots}}}.
\]

When moving to generalized continued fractions, even when we assume
that both $a_{i}$ and $b_{i}$ are integers, we lose the uniqueness
property, and the rational if and only if finite property. What we
gain in return are more presentations for each number, where some
of them can be much simpler to use. For example, $\pi$ can be written
as 
\[
\pi=3+\KK_{1}^{\infty}\frac{\left(2n-1\right)^{2}}{6}.
\]
We want to study these presentations, and (hopefully) use them to
show interesting properties, e.g. prove irrationality for certain
numbers.

\medskip{}

One of the main tools used to study continued fractions are \textbf{Mobius
transformations}. Recall that given a $2\times2$ invertible matrix $M=\left(\begin{smallmatrix}a & b\\
c & d
\end{smallmatrix}\right)\in\GL_{2}\left(\CC\right)$ and a $z\in\CC$, the Mobius action is defined by
\[
M\left(z\right)=\frac{az+b}{cz+d}.
\]

In other words, we apply the standard matrix multiplication $\left(\begin{smallmatrix}a & b\\
c & d
\end{smallmatrix}\right)\left(\begin{smallmatrix}z\\
1
\end{smallmatrix}\right)=\left(\begin{smallmatrix}az+b\\
cz+d
\end{smallmatrix}\right)$ and project it onto $R^{1}\CC$ by dividing the $x$-coordinate by
the $y$-coordinate.

By this definition, it is easy to see that 
\[
\KK_{1}^{n}\frac{b_{i}}{a_{i}}=\frac{b_{1}}{a_{1}+\frac{b_{2}}{a_{2}+\frac{\ddots}{\frac{b_{n}}{a_{n}+0}}}}=\left(\begin{smallmatrix}0 & b_{1}\\
1 & a_{1}
\end{smallmatrix}\right)\left(\begin{smallmatrix}0 & b_{2}\\
1 & a_{2}
\end{smallmatrix}\right)\cdots\left(\begin{smallmatrix}0 & b_{n}\\
1 & a_{n}
\end{smallmatrix}\right)\left(0\right)
\]

This Mobius presentation allows us to show an interesting recurrence
relation on the numerators and denominators of the convergents, which
generalizes the well known recurrence on simple continued fractions.
\begin{lem}
\label{lem:gcf-recursion}Let $a_{n},b_{n}$ be a sequence of integers.
Define $M_{n}=\left(\begin{smallmatrix}0 & b_{n}\\
1 & a_{n}
\end{smallmatrix}\right)$ and set $\left(\begin{smallmatrix}p_{n}\\
q_{n}
\end{smallmatrix}\right)=\left(\prod_{1}^{n-1}M_{i}\right)\left(\begin{smallmatrix}0\\
1
\end{smallmatrix}\right)$. Then $\frac{p_{n}}{q_{n}}=\prod_{1}^{n-1}M_{i}\left(0\right)=\KK_{1}^{n-1}\frac{b_{i}}{a_{i}}$
are the convergents of the generalized continued fraction presentation.
More over, we have that $\left(\begin{smallmatrix}p_{n-1} & p_{n}\\
q_{n-1} & q_{n}
\end{smallmatrix}\right)=\prod_{1}^{n-1}M_{i}$, implying the same recurrence relation on $p_{n}$ and $q_{n}$ given
by 
\begin{align*}
p_{n+1} & =a_{n}p_{n}+b_{n}p_{n-1}\\
q_{n+1} & =a_{n}q_{n}+b_{n}q_{n-1},
\end{align*}
with starting condition $p_{0}=1,\;p_{1}=0$ and $q_{0}=0,\;q_{1}=1$.
\end{lem}

\begin{proof}
Left as an exercise.
\end{proof}
Now that we have this basic result, we can apply it to our irrationality
test as follows:
\begin{claim}
\label{claim:upper-bound}Let $a_{n},b_{n},p_{n},q_{n}$ be sequence
of integers satisfying the recurrence from \lemref{gcf-recursion}.
Then:
\begin{enumerate}
\item If $\frac{p_{n}}{q_{n}}\to L$, then 
\[
\left|L-\frac{p_{n}}{q_{n}}\right|=\left|\sum_{k=n}^{\infty}\frac{\prod_{1}^{k}b_{i}}{q_{k}q_{k+1}}\right|\leq\sum_{k=n}^{\infty}\frac{\prod_{1}^{k}\left|b_{i}\right|}{\left|q_{k}q_{k+1}\right|}.
\]
\item Suppose in addition that the $b_{n}$ are nonzero. Then $\frac{p_{n}}{q_{n}}$
is not eventually constant, so that $\left|q_{n}L-p_{n}\right|=o\left(\gcd\left(p_{n},q_{n}\right)\right)$
implies that $L$ is irrational.
\end{enumerate}
\end{claim}

\begin{proof}
\begin{enumerate}
\item Let $M_{n}=\begin{pmatrix}0 & b_{n}\\
1 & a_{n}
\end{pmatrix}$ so that $\prod_{1}^{n-1}M_{i}=\left(\begin{smallmatrix}p_{n-1} & p_{n}\\
q_{n-1} & q_{n}
\end{smallmatrix}\right).$ For all $m\geq n$ we have
\[
L-\frac{p_{n}}{q_{n}}=L-\frac{p_{m+1}}{q_{m+1}}+\sum_{k=n}^{m}\left(\frac{p_{k+1}}{q_{k+1}}-\frac{p_{k}}{q_{k}}\right)=L-\frac{p_{m+1}}{q_{m+1}}-\sum_{k=n}^{m}\frac{\det\left(\begin{smallmatrix}p_{k} & p_{k+1}\\
q_{k} & q_{k+1}
\end{smallmatrix}\right)}{q_{k}q_{k+1}}.
\]
Under the assumption that $L-\frac{p_{m+1}}{q_{m+1}}\to0$ as $m\to\infty$,
we conclude that 
\[
\left|L-\frac{p_{n}}{q_{n}}\right|\leq\sum_{k=n}^{\infty}\left|\frac{\det\left(\begin{smallmatrix}p_{k} & p_{k+1}\\
q_{k} & q_{k+1}
\end{smallmatrix}\right)}{q_{k}q_{k+1}}\right|=\sum_{k=n}^{\infty}\frac{\prod_{1}^{k}\left|\det\left(M_{i}\right)\right|}{\left|q_{k}q_{k+1}\right|}=\sum_{k=n}^{\infty}\frac{\prod_{1}^{k}\left|b_{i}\right|}{\left|q_{k}q_{k+1}\right|}.
\]
\item In case that the $b_{k}\neq0$ for all $k$, then
\[
\left|\frac{p_{n+1}}{q_{n+1}}-\frac{p_{n}}{q_{n}}\right|=\left|\frac{\det\left(\begin{smallmatrix}p_{n} & p_{n+1}\\
q_{n} & q_{n+1}
\end{smallmatrix}\right)}{q_{n}q_{n+1}}\right|=\frac{\prod_{1}^{n}\left|b_{k}\right|}{\left|q_{n}q_{n+1}\right|}\neq0.
\]
It follows that the $\frac{p_{n}}{q_{n}}$ is not eventually constant,
so we can apply \thmref{improved-rationality-test} to prove this
claim.
\end{enumerate}
\end{proof}
The recursion relation of the $q_{i}$ suggests that the larger the
$\left|b_{i}\right|$ are, the faster the growth of $\left|q_{i}\right|$
is, and in general we expect it to be fast enough so that $\sum_{k=1}^{\infty}\frac{\prod_{1}^{k+1}\left|b_{i}\right|}{\left|q_{k}q_{k+1}\right|}$
will converge. However, it might still not be in $o\left(\frac{1}{\left|q_{n}\right|}\right)$.
Hopefully, if the $gcd\left(p_{n},q_{n}\right)$ is large enough,
then it is in $o\left(\frac{\gcd\left(p_{n},q_{n}\right)}{\left|q_{n}\right|}\right)$,
which is enough to prove irrationality.

\newpage{}


\subsection{\label{subsec:Euler's-formula}Euler's formula}

The Euclidean division algorithm allows us to determine the simple
continued fraction expansion of any given number. Generalized continued
fractions, however, lack a unique representation, rendering a single
algorithm impractical. Nevertheless, this flexibility permits us to
seek convenient presentations, facilitating our ability to switch
between them - a vital tool in our endeavor to reprove Ap\'ery's
theorem. This section commences our exploration by introducing a fascinating
family of polynomial continued fractions that are computationally
straightforward.

One of the most elementary and useful continued fraction presentation
was introduced by Euler who found a way to convert standard finite
sums (and their infinite sum limits) to generalized continued fraction. 
\begin{thm}[Euler's formula]
 Let $r_{i}\in\CC$ for $i\geq1$. Then
\[
1+r_{1}+r_{1}r_{2}+\cdots+r_{1}\cdots r_{n}=\sum_{k=0}^{n}\left(\prod_{i=1}^{k}r_{i}\right)=\frac{1}{1+\KK_{1}^{n}\frac{-r_{i}}{1+r_{i}}}.
\]
By taking the limit (if exists), we have that 
\[
\KK_{1}^{\infty}\frac{-r_{i}}{1+r_{i}}=\frac{1}{\sum_{k=0}^{\infty}\prod_{i=1}^{k}r_{i}}-1.
\]
\end{thm}

\begin{proof}
This is a standard induction, which we leave as an exercise to the
reader.
\end{proof}
For more details and applications of this formula, the reader is referred
to \cite{jones_continued_1980}. 

Euler's formula implies that whenever $a_{i}+b_{i}=1$ we can go
back from generalized continued fractions $\KK_{1}^{\infty}\frac{b_{i}}{a_{i}}$
to infinite sums, where we have many more tools at our disposal. However,
in general this condition doesn't hold, though fortunately there is
a trick to move to equivalent presentations where it might.
\begin{lem}[The equivalence transformation]
\label{lem:equivalence-transformation} Let $a_{i},b_{i}\in\CC$
be two sequences and $0\neq c_{i}\in\CC$ another sequence with nonzero
elements. Then
\[
\KK_{1}^{n}\frac{b_{i}}{a_{i}}=\frac{1}{c_{0}}\KK_{1}^{n}\frac{c_{i-1}c_{i}b_{i}}{c_{i}a_{i}}.
\]
\end{lem}

\begin{proof}
Intuitively, this lemma follows from the fact that $\frac{b_{i}}{a_{i}+x}=\frac{c_{i}b_{i}}{c_{i}a_{i}+c_{i}x}$
plus induction. For example
\[
\frac{4}{11}=\frac{1}{2+\frac{1}{1+\frac{1}{3}}}=\frac{1}{{\color{red}\boldsymbol{c_{0}}}}\cdot\frac{{\color{red}\boldsymbol{c_{0}}}}{2+\frac{1}{1+\frac{1}{3}}}=\frac{1}{c_{0}}\cdot\frac{c_{0}{\color{red}\boldsymbol{c_{1}}}}{2{\color{red}\boldsymbol{c_{1}}}+\frac{{\color{red}\boldsymbol{c_{1}}}}{1+\frac{1}{3}}}=\frac{1}{c_{0}}\cdot\frac{c_{0}c_{1}}{2c_{1}+\frac{c_{1}{\color{red}\boldsymbol{c_{2}}}}{{\color{red}\boldsymbol{c_{2}}}+\frac{{\color{red}\boldsymbol{c_{2}}}}{3}}}=\frac{1}{c_{0}}\cdot\frac{c_{0}c_{1}}{2c_{1}+\frac{c_{1}c_{2}}{c_{2}+\frac{c_{2}{\color{red}\boldsymbol{c_{3}}}}{3{\color{red}\boldsymbol{c_{3}}}}}}.
\]
More precisely, recall that $\KK_{1}^{n}\frac{b_{i}}{a_{i}}=\left[\prod_{1}^{n}\left(\begin{smallmatrix}0 & b_{i}\\
1 & a_{i}
\end{smallmatrix}\right)\right]\left(0\right)$. As Mobius transformations defined by scalar matrices are the identity,
we get that
\begin{align*}
\KK_{1}^{n}\frac{b_{i}}{a_{i}}=\left[\prod_{1}^{n}\left(\begin{smallmatrix}0 & b_{i}\\
1 & a_{i}
\end{smallmatrix}\right)\right]\left(0\right) & =\left[\prod_{1}^{n}\left(\begin{smallmatrix}0 & b_{i}\\
1 & a_{i}
\end{smallmatrix}\right)\cdot c_{i}I\right]\left(0\right)=\left[\prod_{1}^{n}\left(\begin{smallmatrix}0 & b_{i}\\
1 & a_{i}
\end{smallmatrix}\right)\cdot\left(\begin{smallmatrix}1 & 0\\
0 & c_{i}
\end{smallmatrix}\right)\left(\begin{smallmatrix}c_{i} & 0\\
0 & 1
\end{smallmatrix}\right)\right]\left(0\right)\\
 & =\left(\begin{smallmatrix}c_{0}^{-1} & 0\\
0 & 1
\end{smallmatrix}\right)\left[\prod_{1}^{n}\left(\left(\begin{smallmatrix}c_{i-1} & 0\\
0 & 1
\end{smallmatrix}\right)\left(\begin{smallmatrix}0 & b_{i}\\
1 & a_{i}
\end{smallmatrix}\right)\left(\begin{smallmatrix}1 & 0\\
0 & c_{i}
\end{smallmatrix}\right)\right)\right]\left(\begin{smallmatrix}c_{n} & 0\\
0 & 1
\end{smallmatrix}\right)\left(0\right)\\
 & =\left(\begin{smallmatrix}c_{0}^{-1} & 0\\
0 & 1
\end{smallmatrix}\right)\left[\prod_{1}^{n}\left(\begin{smallmatrix}0 & c_{i-1}c_{i}b_{i}\\
1 & c_{i}a_{i}
\end{smallmatrix}\right)\right]\left(\begin{smallmatrix}c_{n} & 0\\
0 & 1
\end{smallmatrix}\right)\left(0\right)
\end{align*}
Since $\left(\begin{smallmatrix}c_{n} & 0\\
0 & 1
\end{smallmatrix}\right)\left(\begin{smallmatrix}0\\
1
\end{smallmatrix}\right)=\left(\begin{smallmatrix}0\\
1
\end{smallmatrix}\right)$ , we conclude that
\[
\KK_{1}^{n}\frac{b_{i}}{a_{i}}=\frac{1}{c_{0}}\KK_{1}^{n}\frac{c_{i-1}c_{i}b_{i}}{c_{i}a_{i}}.
\]
\end{proof}
\begin{rem}
In the last lemma we basically moved from the matrices $\left(\begin{smallmatrix}0 & b_{i}\\
1 & a_{i}
\end{smallmatrix}\right)$ to $\left(c_{n-1}U_{n-1}\right)^{-1}M_{n}U_{n}$ with $U_{n}=\left(\begin{smallmatrix}1 & 0\\
0 & c_{n}
\end{smallmatrix}\right)$. This type of equivalence can be generalized, as we shall see it
later in \secref{The-most-general}.
\end{rem}

Combining this lemma and Euler's formula, we are led to look for $c_{n}$
satisfying
\[
c_{n}a_{n}+c_{n-1}c_{n}b_{n}=1.
\]
If we can find such $c_{n}$, then we have the following.
\begin{cor}
\label{cor:c-n-conversion}Let $a_{i},b_{i}\in\CC$ be any sequences
and suppose that we can find a solution to 
\[
c_{i}a_{i}+c_{i-1}c_{i}b_{i}=1
\]
with nonzero $c_{i}$. Then
\[
\KK_{1}^{n}\frac{b_{i}}{a_{i}}=\frac{1}{c_{0}}\KK_{1}^{n}\frac{\left(c_{i-1}c_{i}b_{i}\right)}{\left(c_{i}a_{i}\right)}=\frac{1}{c_{0}}\left(\frac{1}{\sum_{k=0}^{n}\left(-1\right)^{k}\prod_{i=1}^{k}\left(c_{i-1}c_{i}b_{i}\right)}-1\right),
\]
or equivalently
\[
\sum_{k=0}^{n}\left(-1\right)^{k}\prod_{i=1}^{k}\left(c_{i-1}c_{i}b_{i}\right)=\frac{1}{1+c_{0}\cdot\KK_{1}^{n}\frac{b_{i}}{a_{i}}}=\left(\begin{smallmatrix}0 & 1\\
1 & c_{0}
\end{smallmatrix}\right)\left(\KK_{1}^{n}\frac{b_{i}}{a_{i}}\right).
\]
\end{cor}

\begin{example}[The exponential function]
\label{exa:(The-exponential-function)} Given some $x\in\RR$, we start with the standard Taylor expansion
for $e^{x}$:
\[
e^{x}=1+x+\frac{x^{2}}{2}+\frac{x^{3}}{3!}+\cdots=\sum_{0}^{\infty}\frac{x^{n}}{n!}=\sum_{0}^{\infty}\prod_{1}^{n}\left(\frac{x}{i}\right).
\]
Taking $r_{i}=\frac{x}{i}$ in Euler's formula we get that 
\[
e^{x}=\frac{1}{1+\KK_{1}^{\infty}\frac{-x/i}{1+x/i}}.
\]
Using the equivalence transformation with $c_{i}=i$ starting from
the second index (to avoid division by zero) gives
\[
e^{x}=\frac{1}{1-\frac{x}{1+x+\KK_{1}^{\infty}\frac{-ix}{1+i+x}}}\;\Rightarrow\;\frac{1+x-e^{x}}{e^{x}-1}=\KK_{1}^{\infty}\frac{-ix}{1+i+x}.
\]

In particular, for $x=1$ and $x=-1$ we get that 
\[
\frac{2-e}{e-1}=\KK_{1}^{\infty}\frac{-i}{2+i}\quad,\quad\frac{1}{e-1}=\KK_{1}^{\infty}\frac{i}{i}.
\]

Similar computation can be done with other functions like $\sin\left(x\right),\cos\left(x\right),\ln\left(1+x\right)$
etc.\medskip{}
\end{example}

Finding $c_{n}$ which satisfy the relation in \corref{c-n-conversion}
above is equivalent to solving the recurrence
\[
c_{i}:=\frac{1}{a_{i}+c_{i-1}b_{i}}=\left(\begin{smallmatrix}0 & 1\\
b_{i} & a_{i}
\end{smallmatrix}\right)\left(c_{i-1}\right).
\]

Of course, the hard part is not to find some sequence $c_{i}$ satisfying
this relation, but a ``nice enough'' such sequence for which we
can compute $\sum_{k=0}^{n}\left(-1\right)^{k}\prod_{i=1}^{k}\left(c_{i-1}c_{i}b_{i}\right)$.
The matrix above is the transpose of our original polynomial continued
fraction matrix. Transposing it, and thinking of $c_{i}=\frac{F_{i}}{F_{i+1}}$
as the projection of $\left(F_{i},F_{i+1}\right)$, we get the recursion:
\[
\left(\begin{smallmatrix}F_{i-1} & F_{i}\end{smallmatrix}\right)\left(\begin{smallmatrix}0 & b_{i}\\
1 & a_{i}
\end{smallmatrix}\right)=\left(\begin{smallmatrix}F_{i} & F_{i+1}\end{smallmatrix}\right).
\]
This is exactly the recurrence satisfied by $p_{n}$ and $q_{n}$
we saw in \lemref{gcf-recursion} (so that both $c_{i}=\frac{p_{i}}{p_{i+1}}$
and $c_{i}=\frac{q_{i}}{q_{i+1}}$ solve the recurrence above). Thus,
in a sense, finding one such ``nice'' solution to the recurrsion,
let us find both $p_{n}$ and $q_{n}$.

Next, we try to give simple conditions on $a_{i},b_{i}$ where we
can find ``nice'' solution for the recurrence above. A good starting point is when $a_{i},b_{i}$ are fixed $a_{i}\equiv a,\;b_{i}\equiv b$,
so that our matrix is $M_{i}=M=\left(\begin{smallmatrix}0 & b\\
1 & a
\end{smallmatrix}\right)$, and a solution $F_{n}$ can be found by looking at $M^{n}$. This
is a standard recursion where $F_{n}$ will be a combination of a
polynomial $f\left(n\right)$ and exponential $h^{n}$, where $h$
is one of the two roots $h_{1},h_{2}$ for the characteristic polynomial
$x^{2}-ax-b=0$, namely $h_{1}\cdot h_{2}=-b$ and $h_{1}+h_{2}=a$. 

More generally, our recurrence depends on $i$, and the ``right''
way to think about exponential is more like factorial, so we should
look for $F_{n}$ of the form $f\left(n\right)\cdot\prod_{1}^{n}h\left(k\right)$
for some polynomials $f,h$ , or in the $c_{n}$ notation we have
$c_{n}=\frac{F_{n}}{F_{n+1}}=\frac{f\left(n\right)}{f\left(n+1\right)h\left(n+1\right)}$. 

With this quadratic intuition in mind, we have the following family
of continued fractions which have this sort of solution to their corresponding
recurrence relation. Special cases of these continued fractions appear
in many places (in particular, see for example \cite{brier_note_2022,bowman_polynomial_2018}),
though we did not see this exact formulation in the literature.
\begin{thm}
\label{thm:recurrence-roots}Let $h_{1},h_{2},f:\CC\to\CC$ be any
functions, and define $a,b:\CC\to\CC$ such that 
\begin{align*}
b\left(x\right) & =-h_{1}\left(x\right)h_{2}\left(x\right)\\
f\left(x\right)a\left(x\right) & =f\left(x-1\right)h_{1}\left(x\right)+f\left(x+1\right)h_{2}\left(x+1\right)
\end{align*}
Then taking $F_{n}=f\left(n\right)\cdot\prod_{1}^{n}h_{2}\left(k\right)$
solves the recurrence relation
\[
F_{n}a\left(n\right)+F_{n-1}b\left(n\right)=F_{n+1},
\]
and we get that
\[
\KK_{1}^{n}\frac{b_{i}}{a_{i}}=\frac{f\left(1\right)h_{2}\left(1\right)}{f\left(0\right)}\left(\frac{1}{\sum_{k=0}^{n}\frac{f\left(0\right)f\left(1\right)}{f\left(k\right)f\left(k+1\right)}\prod_{i=1}^{k}\left(\frac{h_{1}\left(i\right)}{h_{2}\left(i+1\right)}\right)}-1\right).
\]
\end{thm}

\begin{proof}
Simply putting the definition of $a,b,F$ in the recurrence gives
us
\[
F_{n}a\left(n\right)+F_{n-1}b\left(n\right)-F_{n+1}=\left[\prod_{1}^{n}h_{2}\left(k\right)\right]\left(f\left(n\right)a\left(n\right)-f\left(n-1\right)h_{1}\left(n\right)-f\left(n+1\right)h_{2}\left(n+1\right)\right)=0.
\]
Using \corref{c-n-conversion} and taking $c_{n}=\frac{F_{n}}{F_{n+1}}$
we get that
\[
\KK_{1}^{n}\frac{b_{i}}{a_{i}}=\frac{F_{1}}{F_{0}}\left(\frac{1}{\sum_{k=0}^{n}\prod_{i=1}^{k}\left(\frac{F_{i-1}}{F_{i+1}}h_{1}\left(i\right)h_{2}\left(i\right)\right)}-1\right)=\frac{f\left(1\right)h_{2}\left(1\right)}{f\left(0\right)}\left(\frac{1}{\sum_{k=0}^{n}\frac{f\left(0\right)f\left(1\right)}{f\left(k\right)f\left(k+1\right)}\prod_{i=1}^{k}\left(\frac{h_{1}\left(i\right)}{h_{2}\left(i+1\right)}\right)}-1\right).
\]
\end{proof}
\begin{rem}
We leave it as an exercise to show that solutions to $c\left(n\right)a\left(n\right)+c\left(n-1\right)c\left(n\right)b\left(n\right)=1$
where $a,b\in\CC\left[x\right]$, and $c\left(x\right)\in\CC\left(x\right)$
is a rational function, imply that $a,b$ must have the form as in
the theorem above where $f,h_{1},h_{2}$ are all polynomials in themselves
and $c\left(x\right)=\frac{f\left(x\right)}{f\left(x+1\right)h_{1}\left(x+1\right)}$.
\end{rem}

While the theorem above is applicable to any functions $h_{1},h_{2}$
and $f$, it is probably most useful when they are polynomials over
$\ZZ$, in which case the product $\prod_{i=1}^{k}\left(\frac{h_{1}\left(i\right)}{h_{2}\left(i+1\right)}\right)$
can become much simpler.

\newpage{}
\begin{example}[The trivial Euler family]
\label{exa:Euler-family}\footnote{It has come to our attention that Euler doesn't have enough mathematical
objects named after him.} In \thmref{recurrence-roots}, consider the case where $f\left(x\right)=1$:
\begin{align*}
b\left(x\right) & =-h_{1}\left(x\right)h_{2}\left(x\right)\\
a\left(x\right) & =h_{1}\left(x\right)+h_{2}\left(x+1\right).
\end{align*}
In this case we have 
\[
\KK_{1}^{n}\frac{b_{i}}{a_{i}}=h_{2}\left(1\right)\left(\frac{1}{\sum_{k=0}^{n}\prod_{i=1}^{k}\left(\frac{h_{1}\left(i\right)}{h_{2}\left(i+1\right)}\right)}-1\right),
\]
or alternatively
\[
\sum_{k=0}^{n}\prod_{i=1}^{k}\left(\frac{h_{1}\left(i\right)}{h_{2}\left(i+1\right)}\right)=\left(\frac{1}{h_{2}\left(1\right)}\KK_{1}^{n}\frac{b_{i}}{a_{i}}+1\right)^{-1}.
\]
In the following example we denote $P_{k}:=\prod_{i=1}^{k}\left(\frac{h_{1}\left(i\right)}{h_{2}\left(i+1\right)}\right)$
so that $\KK_{1}^{n}\frac{b_{i}}{a_{i}}=h_{2}\left(1\right)\left(\left(\sum_{0}^{n}P_{k}\right)^{-1}-1\right)$.\begin{enumerate}
\item Let $h_{1}\left(x\right)=1$ and $h_{2}\left(x\right)=x$. We then
have that $P_{k}=\frac{1}{\left(k+1\right)!}$, so that
\[
\KK_{1}^{\infty}\frac{-n}{n+2}=\frac{1}{\sum_{k=0}^{\infty}\frac{1}{\left(k+1\right)!}}-1=\frac{1}{e-1}-1,
\]
which we already saw in \exaref{(The-exponential-function)}.
\item Let $h_{1}\left(x\right)=h_{2}\left(x\right)=x^{d}$ for some $d\geq2$.
Then $P_{k}=\prod_{i=1}^{k}\frac{i^{d}}{\left(i+1\right)^{d}}=\frac{1}{\left(k+1\right)^{d}}$,
so that
\[
\KK_{1}^{\infty}\frac{b_{n}}{a_{n}}=\frac{1}{\sum_{k=0}^{\infty}\frac{1}{\left(k+1\right)^{d}}}-1=\frac{1}{\zeta\left(d\right)}-1.
\]
\end{enumerate}
\end{example}

In the examples above we only looked at ``trivial'' solutions where
$f\equiv1$. In general, there are solutions where $f\neq1$, however,
they are all part of the trivial family in disguise. Indeed, if
\begin{align*}
b\left(x\right) & =-h_{1}\left(x\right)h_{2}\left(x\right)\\
f\left(x\right)a\left(x\right) & =f\left(x-1\right)h_{1}\left(x\right)+f\left(x+1\right)h_{2}\left(x+1\right)
\end{align*}
as in the theorem, then using the equivalence transformation from
\lemref{equivalence-transformation} with $c_{n}=f\left(n\right)$,
we get that 
\[
\KK_{1}^{n}\frac{b\left(n\right)}{a\left(n\right)}=\frac{1}{f\left(0\right)}\KK_{1}^{n}\frac{\tilde{b}\left(n\right)}{\tilde{a}\left(n\right)}.
\]
where this new continued fraction is in the trivial Euler family:
\begin{align*}
\tilde{h}_{1}\left(x\right) & =h_{1}\left(x\right)f\left(x-1\right)\qquad;\qquad\tilde{h}_{2}\left(x\right)=h_{2}\left(x\right)f\left(x\right)\\
\tilde{b}\left(x\right) & =f\left(x-1\right)f\left(x\right)b\left(x\right)=-\tilde{h}_{1}\left(x\right)\times\tilde{h}_{2}\left(x\right)\\
\tilde{a}\left(x\right) & =f\left(x\right)a\left(x\right)=\tilde{h}_{1}\left(x\right)+\tilde{h}_{2}\left(x+1\right).
\end{align*}

\newpage{}

\section{\label{sec:The-most-general}The most general of generalized continued
fractions}

The utilization of polynomial matrices in the study of polynomial
continued fractions, along with their equivalence transformation presented
in matrix notation, suggests that maybe the right framework is to
simply study polynomial matrices. Moreover, one of our goals, was
to start with a given continued fraction expansion for some constant
(e.g. $\zeta\left(3\right)=\left(\begin{smallmatrix}0 & 1\\
1 & 1
\end{smallmatrix}\right)\left(\KK_{1}^{\infty}\frac{-n^{6}}{n^{3}+\left(1+n\right)^{3}}\right)$), and create new continued fractions, in which the approximation
error goes to zero quicker than the denominator goes to infinity.
This new matrix notation will give this process an extra structure
to utilize.
\begin{defn}
Let $M\left(i\right)\in\GL_{2}\left(\CC\right)$ be a sequence of
$2\times2$ matrices. For $z\in\CC$ we will denote 
\[
\left[\prod_{1}^{\infty}M\left(i\right)\right]\left(z\right)=\limfi n\left[\prod_{1}^{n}M\left(i\right)\right]\left(z\right).
\]
\end{defn}

In the definition above, it is possible that the limit converges for
one $z$ and diverges for another. For example, if we take $M\left(i\right)=\left(\begin{smallmatrix}-1 & 0\\
0 & 1
\end{smallmatrix}\right)$, then $\left[\prod_{1}^{\infty}M\left(i\right)\right]\left(0\right)=0$
while $\left[\prod_{1}^{n}M\left(i\right)\right]\left(1\right)=\left(-1\right)^{n}$
doesn't converge. However, in many ``natural'' sequences we have
a very strong convergence behavior. 
\begin{example}
\begin{enumerate}
\item If $M_{i}=\left(\begin{smallmatrix}1 & a_{i}\\
0 & 1
\end{smallmatrix}\right)$ are upper triangular, then 
\[
\left(\prod_{1}^{n}M_{i}\right)\left(z\right)=\left(\begin{smallmatrix}1 & \sum_{1}^{n}a_{i}\\
0 & 1
\end{smallmatrix}\right)\left(z\right)=z+\sum_{1}^{n}a_{i}.
\]
If $\sum_{1}^{\infty}a_{i}=\infty$, then $\left(\prod_{1}^{n}M_{i}\right)\left(z\right)\to\infty$
for all $z$. If we also add an $M_{0}$ matrix which takes $\infty\to w\in\CC$,
then $\left(\prod_{0}^{n}M_{i}\right)\left(z\right)\to w$ for all
$z$.
\item If $M_{i}=\left(\begin{smallmatrix}0 & b_{i}\\
1 & a_{i}
\end{smallmatrix}\right)$ has the continued fraction form, then $\left(\prod_{1}^{n}M_{i}\right)\left(\infty\right)=\left(\prod_{1}^{n-1}M_{i}\right)\left(0\right)$
since $M_{i}\left(\infty\right)=0$. It follows that we have convergence
at $0$ if and only if we have convergence at $\infty$. Writing $\prod_{1}^{n-1}M_{i}=\left(\begin{smallmatrix}p_{n-1} & p_{n}\\
q_{n-1} & q_{n}
\end{smallmatrix}\right)$ , this limit will simply be $\limfi n\frac{p_{n}}{q_{n}}$. \\
For $x\in\left(0,\infty\right)$ we have that 
\[
\left[\prod_{1}^{n-1}M_{i}\right]\left(x\right)=\left(\begin{smallmatrix}p_{n-1} & p_{n}\\
q_{n-1} & q_{n}
\end{smallmatrix}\right)\left(x\right)=\frac{p_{n-1}x+p_{n}}{q_{n-1}x+q_{n}}
\]
and in case the $p_{n}$ and $q_{n}$ sequences are positive, it is
an exercise to show that $\frac{p_{n-1}x+p_{n}}{q_{n-1}x+q_{n}}$
is in the segment defined by the endpoints $\left\{ \frac{p_{n}}{q_{n}},\frac{p_{n-1}}{q_{n-1}}\right\} $.
It follows that when applying $\left[\prod_{1}^{n-1}M_{i}\right]$
to any element in $\left[0,\infty\right]$, the sequence converges
and to the same limit. \\
The condition on the $p_{n},q_{n}$ is true, for example, if the $a_{i},b_{i}$
are all positive, which is the case for simple continued fractions. 
\end{enumerate}
\end{example}


\subsection*{Potential and coboundary equivalence}

\begin{flushleft}
In our previous discussion about continued fractions, where $M_{i}=\left(\begin{smallmatrix}0 & b_{i}\\
1 & a_{i}
\end{smallmatrix}\right)$, we were specially interested in $P_{n}:=\prod_{1}^{n}M_{i}=\left(\begin{smallmatrix}p_{n-1} & p_{n}\\
q_{n-1} & q_{n}
\end{smallmatrix}\right)$ which contained the numerators and denominators of the convergents.
These products can be defined for any sequence $M_{i}$ of matrices,
and we think of them as \textbf{potential matrices} where we move
from the potential at point $i$ to the potential at point $i+1$
via the matrix $M_{i}$, or more formally $P_{i}M_{i}=P_{i+1}$. We
can also restrict them to row vectors, instead of full $2\times2$
matrices. In particular, for the continued fraction matrices, each
such vector sequence will satisfy $\left(F_{i-1},F_{i}\right)M_{i}=\left(F_{i},F_{i+1}\right)$,
where $F_{i}$ solves the recurrence relation 
\[
F_{i+1}=F_{i}a_{i}+F_{i-1}b_{i}
\]
that we already encountered before in \subsecref{Euler's-formula}.

Attempting to move from the potential of one sequence to another is
quite natural, and this type of question is usually asked in cohomology
theory (see for example chapter 4 in \cite{brown_cohomology_2012}),
where such sequences $M_{i}$ of matrices should and can be called
``\textbf{cocycles}''. We will not go too much into this theory
here, since on the one hand this cocycle structure is in a sense trivial
here, and on the other hand, the theory is usually much more geared
into commutative rings, unlike our noncommutative matrix setting.
However, the question about natural conversion between the potentials
exists in this theory and is called coboundary equivalence, and this
will come up a lot in our study. More specifically we want to move
from one potential $P_{i}^{\left(1\right)}$ to the second $P_{i}^{\left(2\right)}$
using a nice transformation $P_{i}^{\left(1\right)}U_{i}=P_{i}^{\left(2\right)}$,
producing for us this commutative diagram:
\[
\xymatrix{P_{1}^{\left(1\right)}\ar[r]^{M_{1}^{\left(1\right)}}\ar[d]^{U_{1}} & P_{2}^{\left(1\right)}\ar[r]^{M_{2}^{\left(1\right)}}\ar[d]^{U_{2}} & P_{3}^{\left(1\right)}\ar[r]^{M_{3}^{\left(1\right)}}\ar[d]^{U_{3}} & \cdots\ar[r]^{M_{n-1}^{\left(1\right)}} & P_{n}^{\left(1\right)}\ar[d]^{U_{n}}\\
P_{1}^{\left(2\right)}\ar[r]^{M_{1}^{\left(2\right)}} & P_{2}^{\left(2\right)}\ar[r]^{M_{2}^{\left(2\right)}} & P_{3}^{\left(2\right)}\ar[r]^{M_{3}^{\left(2\right)}} & \cdots\ar[r]^{M_{n-1}^{\left(2\right)}} & P_{n}^{\left(2\right)}
}
\]
More formally, we have the following.
\begin{defn}
Two matrix sequences $M_{i}^{\left(1\right)},\;M_{i}^{\left(2\right)}$
are called \textbf{$U_{i}$-coboundary equivalent} for a sequence
of invertible matrices $U_{i}$ if $M_{i}^{\left(1\right)}U_{i+1}=U_{i}M_{i}^{\left(2\right)}$
for all $i$, or in a commutative diagram form:
\[
\xymatrix{\left(*\right)\ar[r]^{M_{i}^{\left(2\right)}} & \left(*\right)\\
\left(*\right)\ar[r]_{M_{i}^{\left(1\right)}}\ar[u]^{U_{i}} & \left(*\right)\ar[u]_{U_{i+1}}
}
.
\]
\end{defn}

\begin{rem}
In the world of standard, non indexed matrices, this coboundary equivalence
is simply matrix conjugation, and as we shall see some of the results
for this coboundary equivalence are just ``indexed'' version of
what we expect from matrix conjugacy.
\end{rem}

The commutativity condition in the coboundary definition can be extended
to products of the $M_{i}$, and in particular for our potential matrices.
Indeed, a simple inductions shows that with the notations as in the
definition, for all $m\leq n$ we have
\[
U_{m}\left[\prod_{m}^{n}M_{i}^{\left(2\right)}\right]=\left[\prod_{m}^{n}M_{i}^{\left(1\right)}\right]U_{n+1}.
\]

Every two matrix sequences $M_{i}^{\left(1\right)},\;M_{i}^{\left(2\right)}$
(invertible) are coboundary equivalent for some $U_{i}$. Indeed,
once we choose $U_{1}$, we can recursively define $U_{i+1}=\left(M_{i}^{\left(2\right)}\right)^{-1}U_{i}M_{i}^{\left(1\right)}$.
However, what will matter to us later on is that $U_{i}$ is simple
enough to work with. For example, it can be defined over $\ZZ$, triangular,
diagonal, etc. In particular, we want to work with the Mobius maps
induced by the matrices, and the upper triangular (resp. lower triangular)
are exactly the matrices which take infinity to itself (resp. zero
to itself).

This idea of coboundary equivalent sequences is very useful, and we
have already seen one such important example. In the ``equivalence
transformation'' for continued fractions in \lemref{equivalence-transformation}
we used 
\[
M_{n}^{\left(1\right)}=\left(\begin{smallmatrix}0 & b_{n}\\
1 & a_{n}
\end{smallmatrix}\right),\quad,M_{n}^{\left(2\right)}=\left(\begin{smallmatrix}0 & c_{n-1}c_{n}b_{n}\\
1 & c_{n}a_{n}
\end{smallmatrix}\right),\quad U_{n}=\left(\begin{smallmatrix}1 & 0\\
0 & c_{n-1}
\end{smallmatrix}\right)
\]
so that $M_{n}^{\left(1\right)}U_{n+1}=c_{n-1}U_{n}M_{n}^{\left(2\right)}$,
and since we deal with Mobius transformation, where scalar matrices
act as the identity, we have $M_{n}^{\left(1\right)}U_{n+1}\equiv U_{n}M_{n}^{\left(2\right)}$.

\medskip{}

Another important example is polynomial continued fractions in the
Euler family.
\begin{example}
\label{exa:Euler_to_triangular}Consider the polynomial continued fraction matrix of the form
\begin{align*}
b\left(x\right) & =-h_{1}\left(x\right)h_{2}\left(x\right)\\
f\left(x\right)a\left(x\right) & =f\left(x-1\right)h_{1}\left(x\right)+f\left(x+1\right)h_{2}\left(x+1\right)\\
M\left(x\right) & =\begin{pmatrix}0 & b\left(x\right)\\
1 & a\left(x\right)
\end{pmatrix}.
\end{align*}
Define 
\[
U\left(x\right)=\begin{pmatrix}\left(f\left(x\right)h_{2}\left(x\right)\right)^{-1} & 0\\
f\left(x-1\right) & f\left(x\right)h_{2}\left(x\right)
\end{pmatrix}.
\]
Then $M\left(n\right)$ is $U\left(n\right)$-coboundary equivalent
to 
\begin{align*}
M'\left(n\right):=U\left(n\right)M\left(n\right)U\left(n+1\right)^{-1} & =\begin{pmatrix}h_{1}\left(x\right) & \frac{-1}{f\left(x\right)f\left(x+1\right)}\frac{h_{2}\left(x\right)}{h_{2}\left(x+1\right)}\\
0 & h_{2}\left(x\right)
\end{pmatrix}.
\end{align*}
It then follows that 
\[
U\left(1\right)\left[\prod_{1}^{n}M\left(i\right)\right]\left(0\right)=\left[\prod_{1}^{n}M'\left(i\right)\right]U\left(n+1\right)\left(0\right)=\left[\prod_{1}^{n}M'\left(i\right)\right]\left(0\right).
\]
In other words, to find the limit of a polynomial continued fraction
in the Euler form, it is enough to find the product of upper triangular
matrices which is much easier to compute. We leave it as an exercise
to show that it actually reaches the same limit as in \subsecref{Euler's-formula}.
\end{example}

\par\end{flushleft}

\newpage{}

\global\long\def\call{\mathcal{L}}%
\global\long\def\nn{\mathcal{N}}%
\global\long\def\ff{\mathcal{F}}%
\global\long\def\aa{\mathcal{A}}%
\global\long\def\RR{\mathbb{R}}%
\global\long\def\EE{\mathbb{E}}%
\global\long\def\CC{\mathbb{C}}%
\global\long\def\QQ{\mathbb{Q}}%
\global\long\def\ZZ{\mathbb{Z}}%
\global\long\def\NN{\mathbb{N}}%
\global\long\def\KK{\mathbb{K}}%
\global\long\def\SL{\mathrm{SL}}%
\global\long\def\GL{\mathrm{GL}}%
\global\long\def\ds{\mathrm{ds}}%
\global\long\def\dnu{\mathrm{d\nu}}%
\global\long\def\dmu{\mathrm{d\mu}}%
\global\long\def\dt{\mathrm{dt}}%
\global\long\def\dw{\mathrm{dw}}%
\global\long\def\dx{\mathrm{dx}}%
\global\long\def\dy{\mathrm{dy}}%
\global\long\def\norm#1{\left\Vert #1\right\Vert }%
\global\long\def\limfi#1{{\displaystyle \lim_{#1\to\infty}}}%
\global\long\def\arrfi#1{\overset{#1\to\infty}{\longrightarrow}}%
\global\long\def\flr#1{\left\lfloor #1\right\rfloor }%
\global\long\def\lcm{\mathrm{lcm}}%


\section{\label{sec:Definition-properties-CMF}The conservative matrix field
- definition and properties}

Up until now we mainly looked at a single continued fractions $\KK_{1}^{\infty}\frac{b_{i}}{a_{i}}$,
and in particular where $a_{i}=a\left(i\right),b_{i}=b\left(i\right)$
with $a,b\in\ZZ\left[x\right]$. In this section we define the\textbf{
conservative matrix field}, which is a collection of such continued
fractions with interesting connections between them. 
\begin{defn}
A pair of matrices $M_{X}\left(x,y\right),M_{Y}\left(x,y\right)$
is called a \textbf{conservative matrix field} (or just \textbf{matrix
field} for simplicity), if
\begin{enumerate}
\item The entries of $M_{X}\left(x,y\right),M_{Y}\left(x,y\right)$ are
polynomial in $x,y$ ,
\item The matrices satisfy the conservativeness relation
\[
M_{X}\left(x,y\right)M_{Y}\left(x+1,y\right)=M_{Y}\left(x,y\right)M_{X}\left(x,y+1\right)\;\forall x,y,
\]
or in commutative diagram form:
\[
\xyR{.5pc}\xyC{0.5pc}\xymatrix{\left(x,y+1\right)\ar[rr]^{M_{X}\left(x,y+1\right)} &  & \left(x+1,y+1\right)\\
 & {\Huge\leftturn}\\
\left(x,y\right)\ar[rr]_{M_{X}\left(x,y\right)}\ar[uu]^{M_{Y}\left(x,y\right)} &  & \left(x+1,y\right)\ar[uu]_{M_{Y}\left(x+1,y\right)}
}
\xyR{1pc}\xyC{1pc}
\]
\end{enumerate}
\end{defn}

The name \emph{conservative matrix field} arose from the resemblence
to \emph{conservative vector field}. When visualizing the corners
of the commutative diagram as points in the plane, the notion is that
traveling along the bottom and then right edge or the left and then
top edge yields the same product, which essentially is the behaviour
of standard conservative vector fields (and indeed, both are 1-cocycles
with the appropriate groups). Retaining this intuitive connection,
led to the adoption of the name conservative matrix field. 

One of the main differences, is that moving left or down along the
matrix field amounts to multiplying by $M_{X}\left(x,y\right)^{-1},M_{Y}\left(x,y\right)^{-1}$
respectively, which aren't necessarily invertible. However, they will
be invertible in most cases, and with that in mind, we define:
\begin{defn}
Given a matrix field $M_{X},M_{Y}$, and initial position $\left(n_{0},m_{0}\right)\in\ZZ^{2}$,
we define the potential $S\left(n,m\right)=S_{n_{0},m_{0}}\left(n,m\right)$
matrix for $n\geq n_{0},m\geq m_{0}$ by

\[
S\left(n,m\right)=\prod_{n_{0}}^{n-1}M_{X}\left(k,m_{0}\right)\cdot\prod_{m_{0}}^{m-1}M_{Y}\left(n,k\right).
\]
Note that the potential is independent of the choice of path from
$\left(n_{0},m_{0}\right)$ to $\left(n,m\right)$.
\end{defn}

\medskip{}

Given a pathes to infinity $\left(n_{i},m_{i}\right)$, it is natural
to ask whether $\lim S\left(n_{i},m_{i}\right)\left(0\right)$ converge,
are how does changing the path affect the limit. In particular, if
$\alpha=\lim S\left(n_{i},m_{i}\right)\left(0\right)$ along some
path, we can look at the rest of the matrix field for other properties
of $\alpha$. For example, in \secref{The-z3-case} we will construct
such a matrix field for $\zeta\left(3\right)$, starting from its
Euler continued fraction (from \exaref{Euler-family}) on the $Y=0$
line, then see that the limits on the $Y=m$ lines converge to $\sum_{m}^{\infty}\frac{1}{k^{3}}$,
and the diagonal line $X=Y$ can be used to define another continued
fraction presentation where the convergents converge to $\zeta\left(3\right)$
is fast enough to prove its irrationality.

With this intuition in mind, we start with a construction for specific
family of matrix fields with many interesting properties in \subsecref{matrix-field-construction},
where in particular each row is a polynomial continued fraction (and
therefore coboundary equivalent via the $M_{Y}$ matrices). We then
``twist'' it a little bit, to get a matrix field which is easier
to work with. Then in \subsecref{The-dual-matrix-field} we find out
how every such matrix field comes with its dual, which is in a sense
a reflection through the $x=y$ line. Once we have this dual matrix
field, we study the numerators and denominators of the continued fractions
in that matrix field, and in particular find their greatest common
divisors. Finally we show how to put everything together in \secref{The-z3-case}
to show that $\zeta\left(3\right)$ is irrational.

\newpage{}

\subsection{\label{subsec:matrix-field-construction}The construction}

We begin with an interesting construction of a family of matrix fields.
\begin{defn}
\label{def:conjugate}We say that two polynomial $f,\bar{f}\in\CC\left[x,y\right]$
are \textbf{conjugate}, if they satisfy:
\begin{enumerate}
\item \textbf{\uline{Linear condition}}: 
\[
f\left(x+1,y-1\right)-\bar{f}\left(x,y-1\right)=f\left(x,y\right)-\bar{f}\left(x+1,y\right).
\]
When this condition holds, we denote the expression above by $a_{f,\bar{f}}\left(x,y\right):=a\left(x,y\right)$.
\item \textbf{\uline{Quadratic condition}}: 
\[
\left(f\bar{f}\right)\left(x,y\right)-\left(f\bar{f}\right)\left(0,y\right)=\left(f\bar{f}\right)\left(x,0\right)-\left(f\bar{f}\right)\left(0,0\right).
\]
In other words, there are no mixed monomial $x^{n}y^{m}$ where $n,m>0$
in $\left(f\bar{f}\right)\left(x,y\right)$.\\
When this condition holds, we denote the expression above by $b_{f,\bar{f}}\left(x\right):=b\left(x\right)$,
which only depends on $x$. We will usually also have that $\left(f\bar{f}\right)\left(0,0\right)=0$,
in which case $b\left(x\right)=\left(f\bar{f}\right)\left(x,0\right)$.
With this in mind, we will also denote $b_{X}\left(x\right):=b\left(x\right)$
and $b_{Y}\left(y\right):=\left(f\bar{f}\right)\left(0,y\right)$
\end{enumerate}
Given two such conjugate polynomials, we define 
\begin{align*}
M_{X}^{cf}\left(x,y\right) & =\left(\begin{smallmatrix}0 & b\left(x\right)\\
1 & a\left(x,y\right)
\end{smallmatrix}\right)\\
M_{Y}^{cf}\left(x,y\right) & =\left(\begin{smallmatrix}\bar{f}\left(x,y\right) & b\left(x\right)\\
1 & f\left(x,y\right)
\end{smallmatrix}\right).
\end{align*}
The $cf$ indicates that $M_{X}^{cf}$ is in a continued fraction
form. We will shortly change it a little bit and remove these $cf$.
\end{defn}

\medskip{}

\begin{rem}
\label{rem:ff(0)=00003D0}If $\left(f\bar{f}\right)\left(0,0\right)=0$,
then the $y=1$ is the continued fraction with $b_{i}=\left(f\bar{f}\right)\left(i,0\right)$
and $a_{i}=f\left(i+1,0\right)-\bar{f}\left(i,0\right)$, which is
in the trivial Euler family defined in \exaref{Euler-family}, where
$h_{1}\left(x\right)=f\left(x,0\right)$ and $h_{2}\left(x\right)=-\bar{f}\left(x,0\right)$.
Using the second presentation of $a\left(x,y\right)$, the $y=1$
line is a continued fraction with $b_{i}=\left(f\bar{f}\right)\left(i,0\right)$
and $a_{i}=f\left(i+1,0\right)-\bar{f}\left(i,0\right)$, which is
similarly in the trivial Euler family.\\
Also, recall that finding an ``Euler'' presentation for a continued
fraction is in a sense a generalization of solving a quadratic equation
$x^{2}+ax+b=0$, where the roots $\lambda_{1},\lambda_{2}$ satisfy
$\lambda_{1}\lambda_{2}=b$ and $-\left(\lambda_{1}+\lambda_{2}\right)=a$.
The structure defined above should be considered as an even further
generalization of this concept. Indeed, starting with $b\left(x\right)$
and $a\left(x,y\right)$, we look for $f,\bar{f}$ such that 
\begin{align*}
b\left(x\right) & =f\left(x,0\right)\bar{f}\left(x,0\right)\\
a\left(x,y\right) & =f\left(x,y\right)-\bar{f}\left(x+1,y\right).
\end{align*}

With this point of view, the term ``conjugates'' should be more
natural, since in a way $f,\bar{f}$ are roots of a quadratic equation
(though with polynomials coefficients).
\end{rem}

\newpage{}
\begin{example}[\textbf{The $\zeta\left(3\right)$ matrix field}]
\label{exa:zeta-3-matrix-field}The main example that we should have in mind is a matrix field for
$\zeta\left(3\right)$ defined by
\begin{align*}
f\left(x,y\right) & =x^{3}+2x^{2}y+2xy^{2}+y^{3}=\frac{y^{3}-x^{3}}{y-x}\left(y+x\right)\\
\bar{f}\left(x,y\right) & =-x^{3}+2x^{2}y-2xy^{2}+y^{3}=\frac{y^{3}+x^{3}}{y+x}\left(y-x\right)=f\left(-x,y\right)\\
\left(f\bar{f}\right)\left(x,y\right) & =y^{6}-x^{6}\\
b\left(x\right) & =\left(f\bar{f}\right)\left(x,0\right)=-x^{6}\\
a\left(x,y\right) & =x^{3}+\left(1+x\right)^{3}+2y\left(y-1\right)\left(2x+1\right).
\end{align*}
In particular, for $y=0,1$ we have the polynomial continued fraction
$b\left(n\right)=-n^{6}$ and $a\left(n,0\right)=n^{3}+\left(1+n\right)^{3}$,
which is exactly the Euler continued fraction which converges to $\frac{1}{\zeta\left(3\right)}-1$,
as we saw in \exaref{Euler-family}. We will see in \secref{The-z3-case}
that for any fixed integer $y=m\geq1$, the continued fraction with
$b_{n}=b\left(n\right)$ and $a_{n}=a\left(n,m\right)$ converges
to $\frac{1}{\sum_{m}^{\infty}\frac{1}{k^{3}}}-1$.

The polynomial matrices in this matrix field are 
\begin{align*}
M_{X}^{cf}\left(x,y\right) & =\left(\begin{smallmatrix}0 & b\left(x\right)\\
1 & a\left(x,y\right)
\end{smallmatrix}\right)=\left(\begin{smallmatrix}0 & -x^{6}\\
1 & x^{3}+\left(1+x\right)^{3}+2y\left(y-1\right)\left(2x+1\right)
\end{smallmatrix}\right)\\
M_{Y}^{cf}\left(x,y\right) & =\left(\begin{smallmatrix}\bar{f}\left(x,y\right) & b\left(x\right)\\
1 & f\left(x,y\right)
\end{smallmatrix}\right)=\left(\begin{smallmatrix}\frac{y^{3}+x^{3}}{y+x}\left(y-x\right) & -x^{6}\\
1 & \frac{y^{3}-x^{3}}{y-x}\left(y+x\right)
\end{smallmatrix}\right)
\end{align*}
and the first few of them are 
\begin{align*}
\xyR{1.5pc}\xyC{1.5pc}\xymatrix{\left(*\right) &  &  & \left(*\right) &  &  & \left(*\right) &  & \left(*\right)\\
\\
*+[F]{\left(1,3\right)}\ar[rrr]_{\left(\begin{smallmatrix}0 & -1\\
1 & 45
\end{smallmatrix}\right)}\ar[uu]^{\left(\begin{smallmatrix}14 & -1\\
1 & 52
\end{smallmatrix}\right)} &  &  & *+[F]{\left(2,3\right)}\ar[rrr]_{\left(\begin{smallmatrix}0 & -2^{3}\\
1 & 95
\end{smallmatrix}\right)}\ar[uu]^{\left(\begin{smallmatrix}7 & -1\\
1 & 95
\end{smallmatrix}\right)} &  &  & *+[F]{\left(3,3\right)}\ar[rr]_{\left(\begin{smallmatrix}0 & -3^{3}\\
1 & 175
\end{smallmatrix}\right)}\ar[uu]^{\left(\begin{smallmatrix}0 & -1\\
1 & 168
\end{smallmatrix}\right)} &  & \left(*\right)\\
\\
*+[F]{\left(1,2\right)}\ar[rrr]_{\left(\begin{smallmatrix}0 & -1\\
1 & 21
\end{smallmatrix}\right)}\ar[uu]^{\left(\begin{smallmatrix}3 & -1\\
1 & 21
\end{smallmatrix}\right)} &  &  & *+[F]{\left(2,2\right)}\ar[rrr]_{\left(\begin{smallmatrix}0 & -2^{3}\\
1 & 55
\end{smallmatrix}\right)}\ar[uu]^{\left(\begin{smallmatrix}0 & -1\\
1 & 48
\end{smallmatrix}\right)} &  &  & *+[F]{\left(3,2\right)}\ar[rr]_{\left(\begin{smallmatrix}0 & -3^{3}\\
1 & 119
\end{smallmatrix}\right)}\ar[uu]^{\left(\begin{smallmatrix}-7 & -1\\
1 & 95
\end{smallmatrix}\right)} &  & \left(*\right)\\
\\
*+[F]{\left(1,1\right)}\ar[rrr]_{\left(\begin{smallmatrix}0 & -1\\
1 & 9
\end{smallmatrix}\right)}\ar[uu]^{\left(\begin{smallmatrix}0 & -1\\
1 & 6
\end{smallmatrix}\right)} &  &  & *+[F]{\left(2,1\right)}\ar[rrr]_{\left(\begin{smallmatrix}0 & -2^{3}\\
1 & 35
\end{smallmatrix}\right)}\ar[uu]^{\left(\begin{smallmatrix}-3 & -1\\
1 & 21
\end{smallmatrix}\right)} &  &  & *+[F]{\left(3,1\right)}\ar[rr]_{\left(\begin{smallmatrix}0 & -3^{3}\\
1 & 91
\end{smallmatrix}\right)}\ar[uu]^{\left(\begin{smallmatrix}-14 & -1\\
1 & 52
\end{smallmatrix}\right)} &  & \left(*\right)
}
\end{align*}
\end{example}

\newpage{}

We continue to show that this general construction produces a conservative
matrix field.
\begin{thm}
\label{thm:matrix-field-structure}Given polynomials $f,\bar{f},a,b$
as in \defref{conjugate} and the matrices
\begin{align*}
M_{X}^{cf}\left(x,y\right) & =\left(\begin{smallmatrix}0 & b\left(x\right)\\
1 & a\left(x,y\right)
\end{smallmatrix}\right)\\
M_{Y}^{cf}\left(x,y\right) & =\left(\begin{smallmatrix}\bar{f}\left(x,y\right) & b\left(x\right)\\
1 & f\left(x,y\right)
\end{smallmatrix}\right),
\end{align*}
then the following hold:
\begin{enumerate}
\item The matrices satisfy the coboundary equivalence condition
\[
M_{X}^{cf}\left(x,y\right)M_{Y}^{cf}\left(x+1,y\right)=M_{Y}^{cf}\left(x,y\right)M_{X}^{cf}\left(x,y+1\right)\;\forall x,y.
\]
\item The determinants of $M_{X}^{cf}\left(x,y\right),M_{Y}^{cf}\left(x,y\right)$
are only functions of $x,y$ respectively, and more specifically:
\begin{align*}
\det\left(M_{X}^{cf}\left(x,y\right)\right) & =-b\left(x\right)=-b_{X}\left(x\right)\\
\det\left(M_{Y}^{cf}\left(x,y\right)\right) & =\left(f\cdot\bar{f}\right)\left(0,y\right)=b_{Y}\left(y\right).
\end{align*}
\end{enumerate}
\end{thm}

\begin{proof}
\begin{enumerate}
\item From the assumption on our functions we know that for all $x,y$ we
have
\begin{align}
a\left(x,y\right) & =f\left(x,y\right)-\bar{f}\left(x+1,y\right)\\
a\left(x,y+1\right) & =f\left(x+1,y\right)-\bar{f}\left(x,y\right)\\
b\left(x+1\right)-b\left(x\right) & =f\left(x,y\right)a\left(x,y+1\right)-a\left(x,y\right)f\left(x+1,y\right)
\end{align}
Using conditions (1) and (2) we get that
\begin{align*}
M_{X}^{cf}\left(x,y\right)M_{Y}^{cf}\left(x+1,y\right) & =\left(\begin{smallmatrix}0 & b\left(x\right)\\
1 & a\left(x,y\right)
\end{smallmatrix}\right)\left(\begin{smallmatrix}\bar{f}\left(x+1,y\right) & b\left(x+1\right)\\
1 & f\left(x+1,y\right)
\end{smallmatrix}\right)\overset{\left(1\right)}{=}\left(\begin{smallmatrix}b\left(x\right) & b\left(x\right)f\left(x+1,y\right)\\
f\left(x,y\right) & b\left(x+1\right)+a\left(x,y\right)f\left(x+1,y\right)
\end{smallmatrix}\right)\\
M_{Y}^{cf}\left(x,y\right)M_{X}^{cf}\left(x,y+1\right) & =\left(\begin{smallmatrix}\bar{f}\left(x,y\right) & b\left(x\right)\\
1 & f\left(x,y\right)
\end{smallmatrix}\right)\left(\begin{smallmatrix}0 & b\left(x\right)\\
1 & a\left(x,y+1\right)
\end{smallmatrix}\right)\overset{\left(2\right)}{=}\left(\begin{smallmatrix}b\left(x\right) & b\left(x\right)f\left(x+1,y\right)\\
f\left(x,y\right) & b\left(x\right)+f\left(x,y\right)a\left(x,y+1\right)
\end{smallmatrix}\right)
\end{align*}
The two matrices are the same using (3) from above.
\item Simple computation.
\end{enumerate}
\end{proof}
\begin{example}
\label{exa:vector-field-examples}There are many examples of conservative
matrix fields, and we give some of them below.

For each pair $f,\bar{f}$, we also add the $b\left(x\right),a\left(x,y\right)$
appearing as the continued fraction on the horizontal lines. In particular,
as we saw in \remref{ff(0)=00003D0}, when $\left(f\bar{f}\right)\left(0,0\right)=0$,
the $y=1$ line is in the Euler Family from \exaref{Euler-family},
namely $b\left(n\right)=-h_{1}\left(n\right)\times h_{2}\left(n\right)$
and $a\left(n\right)=h_{1}\left(n\right)+h_{2}\left(n+1\right)$.
In these cases we can convert it to an infinite sum and hopefully
use it to compute the value of the continued fraction, which we add
in the examples below (up to a Mobius map). Further more, in many
cases we think of $\bar{f}$ as an image under some nice linear map
$g\mapsto\bar{g}$ of $f$, and when this is the case, we will give
this linear map instead of $\bar{f}$.
\begin{enumerate}
\item When both $f,\bar{f}$ are linear themselves, solving the linear and
quadratic conditions in \defref{conjugate} is elementary (which we
leave as exercise). There are two families of solutions
\begin{align*}
f\left(x,y\right) & =A\left(x+y\right)+C\\
\bar{f}\left(x,y\right) & =\bar{A}\left(x-y\right)+\bar{C}
\end{align*}
or 
\begin{align*}
f\left(x,y\right) & =Ax+By+C\\
\bar{f}\left(x,y\right) & =-Ax+By+\bar{C}
\end{align*}
where $A,B,C,\bar{A},\bar{C}$ above are the parameters of the families.
\begin{enumerate}
\item Taking $f\left(x,y\right)=x+y$ and $\bar{f}\left(x,y\right)=x-y$,
we get $b\left(x\right)=x^{2}$ and $a\left(x,y\right)=2y-1$. In
$y=1$ we get the continued fraction
\[
\KK_{1}^{\infty}\frac{n^{2}}{1}=\KK_{1}^{\infty}\frac{-\left(-n\right)\times n}{\left(-n\right)+\left(n+1\right)}=\frac{1}{\sum_{k=0}^{\infty}\prod_{i=1}^{k}\frac{-i}{i+1}}-1=\frac{1}{\sum_{k=0}^{\infty}\frac{\left(-1\right)^{k}}{k+1}}-1=\frac{1-\ln\left(2\right)}{\ln\left(2\right)}.
\]
Taking $\bar{f}\left(x,y\right)=y-x$ instead, we get $b\left(x\right)=-x^{2}$
and $a\left(x,y\right)=2x+1$. Since $a$ is independent of $y$,
all the horizontal lines in the matrix field are the same, so in a
sense it is degenerate. Moreover, trying to compute the continued
fraction produces
\[
\KK_{1}^{\infty}\frac{-n^{2}}{2n+1}=\KK_{1}^{\infty}\frac{-n\times n}{n+\left(n+1\right)}=\frac{1}{\sum_{k=0}^{\infty}\prod_{i=1}^{k}\frac{i}{i+1}}-1=\frac{1}{\sum_{k=0}^{\infty}\frac{1}{k+1}}-1=-1,
\]
since the harmonic sum $\sum_{0}^{\infty}\frac{1}{k+1}$ diverges
to infinity.
\item For $f\left(x,y\right)=x+y$ and $\bar{f}\left(x,y\right)=1$ (which
we can think of as $\frac{\partial f}{\partial x}=\frac{\partial f}{\partial y}=\bar{f}$),
we get $b\left(x\right)=x,\;a\left(x,y\right)=x+y-1$, and in the
$y=1$ case we get 
\[
\KK_{1}^{\infty}\frac{n}{n}=\KK_{1}^{\infty}\frac{-\left(\left(-1\right)\times n\right)}{\left(-1\right)+\left(n+1\right)}=\frac{1}{\sum_{k=0}^{\infty}\prod_{i=1}^{k}\frac{-1}{i+1}}-1=\frac{1}{e-1}
\]
which we already saw in \exaref{(The-exponential-function)}.
\end{enumerate}
\item When $f,\bar{f}$ have degree at most 2, then we have the following
families of examples (as function of $C$):{\tiny{}
\[
\begin{array}{c|c|c|c|c}
\text{operation} & f\left(x,y\right) & a\left(x,y\right) & b\left(x\right) & \text{Euler family}\;\left(a\left(x,1\right)\right)\\
\hline \bar{g}\left(x,y\right)=-g\left(-x,y\right) & x^{2}+xy+\frac{y^{2}}{2}+C\left(x+y\right) & \left(x+1\right)^{2}+x^{2}+y\left(y-1\right)+C\left(2y-1\right) & -x^{2}\left(x^{2}-C^{2}\right) & \left(x+1\right)\left(x+1+C\right)+x\left(x-C\right)\\
\bar{g}\left(x,y\right)=g\left(-x,y\right) & x^{2}+2xy+2y^{2}+C\left(2y-x\right) & \left(2x+1\right)\left(2y-1+C\right) & x^{2}\left(x^{2}-C^{2}\right) & \left(x+1\right)\left(x+1+C\right)-x\left(x-C\right)\\
\bar{g}\left(x,y\right)=g\left(x,-y\right) & x^{2}+2xy+2y^{2}+C\left(x+y\right) & \left(2x+1+C\right)\left(2y-1\right) & x^{2}\left(x+C\right)^{2} & \left(x+1\right)\left(x+C+1\right)-x\left(x+C\right)\\
\bar{g}\left(x,y\right)=-g\left(x,-y\right) & \frac{2x^{2}+2xy+y^{2}+C\left(2x+y\right)}{2} & C\left(2x+1\right)+x^{2}+\left(x+1\right)^{2}+y\left(y-1\right) & -x^{2}\left(x+C\right)^{2} & \left(x+1\right)\left(x+C+1\right)+x\left(x+C\right)
\end{array}
\]
}In particular, when taking $C=0$, the $y=1$ line is either $b\left(x\right)=-x^{4}$
and $a\left(x\right)=x^{2}+\left(1+x\right)^{2}$, or $b\left(x\right)=x^{4}$
and $a\left(x\right)=\left(x+1\right)^{2}-x^{2}$. The continued fraction
will eventually be transformed (after the right Mobius action) to
the sums $\sum_{1}^{\infty}\frac{1}{n^{2}}$ and $\sum_{1}^{\infty}\frac{\left(-1\right)^{n}}{n^{2}}$
which are $\zeta\left(2\right)$ and $\frac{1}{2}\zeta\left(2\right)$
respectively.
\item In degree 3, with the action $\bar{g}\left(x,y\right)\mapsto g\left(-x,y\right)$,
we have the family 
\begin{align*}
f\left(x,y\right) & =x^{3}+2x^{2}\left(y-C\right)+2x\left(y-C\right)^{2}+\left(y-C\right)^{3}-\left(x+y-C\right)C^{2}\\
b\left(x\right) & =-x^{2}\left(x-C\right)^{2}\left(x+C\right)^{2}\\
a\left(x,y\right) & =x\left(x-C\right)^{2}+\left(x+1\right)\left(x+1+C\right)^{2}+\left(1+2x\right)\left(y-1-2C\right)2y.
\end{align*}
When $y=1$ we get a continued fraction in the Euler family with $h_{1}\left(x\right)=x\left(x-C\right)^{2}$
and $h_{2}\left(x\right)=x\left(x+C\right)^{2}$. In particular, in
the case where $C=0$ we simply get the matrix field for $\zeta\left(3\right)$
mentioned in \exaref{zeta-3-matrix-field}.
\end{enumerate}
\end{example}

\begin{rem}
Once we have a pair of conjugate polynomials $f,\bar{f}$, there are
several ways to generate more such pairs. The simplest way is just
to take $cf,c\bar{f}$ for some $0\neq c\in\CC$. Another less trivial
way is to look at the pair $\left(f\left(y,x\right),\;-\bar{f}\left(y,x\right)\right)$.
We shall see in \subsecref{The-dual-matrix-field} how this new pair
is hidden in the same conservative matrix field.
\end{rem}

Right now, while the $M_{X}^{cf}$ matrix has the known continued
fraction form, the $M_{Y}^{cf}$ matrices have this new unkown form
$\left(\begin{smallmatrix}\bar{f}\left(x,y\right) & b\left(x\right)\\
1 & f\left(x,y\right)
\end{smallmatrix}\right)$. However, as we shall soon see, there are hidden continued fraction
in $M_{Y}^{cf}$ as well, are both $M_{X}^{cf},M_{Y}^{cf}$ are defined
very similarly. For that, we use the following notations.
\begin{notation}
We define:
\[
U_{\alpha}=\left(\begin{smallmatrix}1 & \alpha\\
0 & 1
\end{smallmatrix}\right)\qquad D_{\alpha}=\left(\begin{smallmatrix}\alpha & 0\\
0 & 1
\end{smallmatrix}\right)\qquad\tau=\left(\begin{smallmatrix}0 & 1\\
1 & 0
\end{smallmatrix}\right).
\]
For any matrix $M$, we will write the isomorphism $M\mapsto M^{\tau}=\tau M\tau^{-1}$
(and note that $\tau^{2}=Id$, so that $\tau^{-1}=\tau$). More specifically,
we have that $\left(\begin{smallmatrix}a & b\\
c & c
\end{smallmatrix}\right)^{\tau}=\left(\begin{smallmatrix}d & c\\
b & a
\end{smallmatrix}\right)$ is just switching the rows and switching the columns, and in particular
$U_{\alpha}^{\tau}=U_{\alpha}^{tr}$.
\end{notation}

\medskip{}

With these notations we get:
\begin{alignat*}{2}
M_{X}^{cf}\left(x,y\right) & =\left(\begin{smallmatrix}0 & b\left(x\right)\\
1 & f\left(x,y\right)-\bar{f}\left(x+1,y\right)
\end{smallmatrix}\right) &  & =D_{b_{X}\left(x\right)}\cdot\tau\cdot U_{f\left(x,y\right)}\cdot U_{-\bar{f}\left(x+1,y\right)}\\
M_{Y}^{cf}\left(x,y\right) & =\left(\begin{smallmatrix}\bar{f}\left(x,y\right) & b\left(x\right)\\
1 & f\left(x,y\right)
\end{smallmatrix}\right) &  & =U_{\bar{f}\left(x,y\right)}\cdot D_{-b_{Y}\left(y\right)}\cdot\tau\cdot U_{f\left(x,y\right)},
\end{alignat*}
so that $M_{X}^{cf}$ and $M_{Y}^{cf}$ are ``almost'' the same.
There is some ``cyclic permutation'' and after it they have a similar
structure, with related parameters, and in particular the $M_{Y}^{cf}$
is also a continued fraction sequence, after a simple coboundary equivalence
(via the matrices $U_{\bar{f}\left(x,y\right)}$).

As mentioned in \remref{ff(0)=00003D0}, if $\left(f\bar{f}\right)\left(0\right)=0$,
then the $Y=0$ and $Y=1$ lines are in the trivial Euler family.
Similarly, on the $X=0$ line we have 
\[
M_{Y}^{cf}\left(0,y\right)=\left(\begin{smallmatrix}\bar{f}\left(0,y\right) & 0\\
1 & f\left(0,y\right)
\end{smallmatrix}\right),
\]
which is even simpler to work with (recall that the continued fraction
in the trivial Euler family, are in essence upper triangular in disguise).
However, on the $X=0$ line we have that $M_{X}^{cf}\left(0,y\right)=\left(\begin{smallmatrix}0 & 0\\
1 & f\left(0,y\right)-\bar{f}\left(1,y\right)
\end{smallmatrix}\right)$ are not invertible. With this in mind, we do a slight change of parameters,
which will solve this problem, and we will see is more natural.
\begin{defn}
Let $f,\bar{f}$ be conjugate polynomials and $M_{X},M_{Y}$ as in
\defref{conjugate}. Define
\begin{align*}
M_{X}\left(x,y\right) & :=D_{b\left(x\right)}^{-1}M_{X}^{cf}\left(x,y\right)D_{b\left(x+1\right)}=\tau U_{f\left(x,y\right)}U_{-\bar{f}\left(x+1,y\right)}D_{b\left(x+1\right)}=\left(\begin{smallmatrix}0 & 1\\
b\left(x+1\right) & f\left(x,y\right)-\bar{f}\left(x+1,y\right)
\end{smallmatrix}\right)\\
M_{Y}\left(x,y\right) & :=D_{b\left(x\right)}^{-1}M_{Y}^{cf}\left(x,y\right)D_{b\left(x\right)}=U_{f\left(x,y\right)}^{\tau}\tau D_{-\left(f\bar{f}\right)\left(0,y\right)}U_{\bar{f}\left(x,y\right)}^{\tau}=\left(\begin{smallmatrix}\bar{f}\left(x,y\right) & 1\\
b\left(x\right) & f\left(x,y\right)
\end{smallmatrix}\right)
\end{align*}
\smallskip{}
\end{defn}

There are three main reasons why this is a bit better way to view
our matrix fields:
\begin{enumerate}
\item If we start the continued fraction on the $y$ line at $x=0$ with
the previous $M_{X}^{cf}$ matrices, we get 
\[
M_{X}^{cf}\left(0,y\right)M_{X}^{cf}\left(1,y\right)M_{X}^{cf}\left(2,y\right)\cdots=\left[D_{b\left(0\right)}\tau U_{f\left(0,y\right)}U_{-\bar{f}\left(0+1,y\right)}\right]\left[D_{b\left(1\right)}\tau U_{f\left(1,y\right)}U_{-\bar{f}\left(1+1,y\right)}\right]\cdots
\]
where as we mentioned before $D_{b\left(0\right)}=\left(\begin{smallmatrix}0 & 0\\
0 & 1
\end{smallmatrix}\right)$ is singular which can cause problems. This means that we have to
start with $x=1$, and therefore ``lose'' the information from $\tau U_{f\left(0,y\right)}U_{-\bar{f}\left(0+1,y\right)}$.
With our new matrices $M_{X}$ we instead get
\[
M_{X}\left(0,y\right)M_{X}\left(1,y\right)M_{X}\left(2,y\right)\cdots=\left[\tau U_{f\left(0,y\right)}U_{-\bar{f}\left(0+1,y\right)}D_{b\left(1\right)}\right]\left[\tau U_{f\left(1,y\right)}U_{-\bar{f}\left(1+1,y\right)}D_{b\left(2\right)}\right]\cdots
\]
so we start exactly after the problematic matrix $D_{b\left(0\right)}$.
\item With this new definition, where we start at $x=0$, we get that $M_{Y}\left(0,y\right)$
is upper triangular, since
\[
M_{Y}\left(0,y\right)=\left(\begin{smallmatrix}\bar{f}\left(0,y\right) & 1\\
b\left(0\right) & f\left(0,y\right)
\end{smallmatrix}\right)=\left(\begin{smallmatrix}\bar{f}\left(0,y\right) & 1\\
0 & f\left(0,y\right)
\end{smallmatrix}\right).
\]
\item Finally, as we shall see below, the limits for each horizontal line
are more natural, namely 
\[
\limfi N\left[\prod_{n=0}^{N}M_{X}\left(n,m\right)\right]\left(0\right)=\tau U_{a\left(0,m\right)}\limfi N\left[\prod_{n=1}^{N}M_{X}^{cf}\left(n,m\right)\right]\left(0\right)=\left(a\left(0,m\right)+\left(\KK_{1}^{\infty}\frac{b\left(n\right)}{a\left(n,m\right)}\right)\right)^{-1}.
\]
 In particular, in the new matrix field for our $\zeta\left(3\right)$
example, we will get the limits $\sum_{m}^{\infty}\frac{1}{k^{3}}$.
This will simplify the arguments when trying to find the denominators
and numerators of the convergents.
\end{enumerate}
With this in mind we rewrite \thmref{matrix-field-structure} and
expand it with this new matrices.
\begin{thm}
\label{thm:normalized-matrix-field}Let $f,\bar{f},a,b$ be polynomials
as in \defref{conjugate}. We set 
\begin{align*}
M_{X}\left(x,y\right) & :=\tau U_{f\left(x,y\right)}U_{-\bar{f}\left(x+1,y\right)}D_{b\left(x+1\right)}=\left(\begin{smallmatrix}0 & 1\\
b\left(x+1\right) & f\left(x,y\right)-\bar{f}\left(x+1,y\right)
\end{smallmatrix}\right)\\
M_{Y}\left(x,y\right) & :=U_{f\left(x,y\right)}^{\tau}\tau D_{-\left(f\bar{f}\right)\left(0,y\right)}U_{\bar{f}\left(x,y\right)}^{\tau}=\left(\begin{smallmatrix}\bar{f}\left(x,y\right) & 1\\
b\left(x\right) & f\left(x,y\right)
\end{smallmatrix}\right)
\end{align*}
Then
\begin{enumerate}
\item The matrices form a conservative matrix field, namely
\[
M_{X}\left(x,y\right)M_{Y}\left(x+1,y\right)=M_{Y}\left(x,y\right)M_{X}\left(x,y+1\right).
\]
\item The determinants of $M_{X}\left(x,y\right),M_{Y}\left(x,y\right)$
are only functions of $x,y$ respectively, and more specifically:
\begin{align*}
\det\left(M_{X}\left(x,y\right)\right) & =-b\left(x+1\right)=\left(f\cdot\bar{f}\right)\left(0,0\right)-\left(f\cdot\bar{f}\right)\left(x+1,0\right)=-b_{X}\left(x+1\right)\\
\det\left(M_{Y}\left(x,y\right)\right) & =\left(f\cdot\bar{f}\right)\left(x,y\right)-b\left(x\right)=\left(f\cdot\bar{f}\right)\left(0,y\right)=b_{Y}\left(y\right).
\end{align*}
\item For $x=0$ , the matrices $M_{Y}\left(0,y\right)$ are upper triangular
\[
M_{Y}\left(0,y\right)=\left(\begin{smallmatrix}\bar{f}\left(0,y\right) & 1\\
0 & f\left(0,y\right)
\end{smallmatrix}\right).
\]
\end{enumerate}
\end{thm}

\begin{proof}
This follows directly from \thmref{matrix-field-structure} .
\end{proof}
Next, we use the fact that the $Y=1$ line in the original conservative
matrix field is in the trivial Euler family, to find a simple presentation
for the $Y=1$ line in our new matrix field.
\begin{lem}
\label{lem:First-row}Suppose that $\left(f\bar{f}\right)\left(0,0\right)=0$.
Then
\begin{align*}
U_{\bar{f}\left(0,0\right)}^{\tau}\left[\prod_{0}^{n-1}M_{X}\left(k,1\right)\right]U_{-\bar{f}\left(n,0\right)}^{\tau} & =\left(\begin{smallmatrix}\left(-1\right)^{n}\prod_{1}^{n}\bar{f}\left(k,0\right) & c_{n}\\
0 & \prod_{1}^{n}f\left(k,0\right)
\end{smallmatrix}\right)\\
c_{n} & =\sum_{k=1}^{n}\left(-1\right)^{k-1}\left(\prod_{i=1}^{k-1}\bar{f}\left(i,0\right)\right)\left(\prod_{i=k+1}^{n}f\left(i,0\right)\right).
\end{align*}
\end{lem}

\begin{proof}
Similar to \exaref{Euler_to_triangular}, assuming that $\left(f\bar{f}\right)\left(0,0\right)=0$,
at the $Y=1$ we have 
\begin{align*}
U_{\bar{f}\left(x,0\right)}^{\tau}M_{X}\left(x,1\right)U_{-\bar{f}\left(x+1,0\right)}^{\tau} & =\left(\begin{smallmatrix}1 & 0\\
\bar{f}\left(x,0\right) & 1
\end{smallmatrix}\right)\left(\begin{smallmatrix}0 & 1\\
\left(f\bar{f}\right)\left(x+1,0\right) & f\left(x+1,0\right)-\bar{f}\left(x,0\right)
\end{smallmatrix}\right)\left(\begin{smallmatrix}1 & 0\\
-\bar{f}\left(x+1,0\right) & 1
\end{smallmatrix}\right)=\left(\begin{smallmatrix}-\bar{f}\left(x+1,0\right) & 1\\
0 & f\left(x+1,0\right)
\end{smallmatrix}\right).
\end{align*}
It then follows that 
\[
U_{\bar{f}\left(0,0\right)}^{\tau}\left[\prod_{0}^{n-1}M_{X}\left(k,1\right)\right]U_{-\bar{f}\left(n,0\right)}=\prod_{1}^{n}\left(\begin{smallmatrix}-\bar{f}\left(k,0\right) & 1\\
0 & f\left(k,0\right)
\end{smallmatrix}\right),
\]
and a standard induction on product of upper triangular matrices will
show that we get
\[
\left(\begin{smallmatrix}\left(-1\right)^{n}\prod_{1}^{n}\bar{f}\left(k,0\right) & c_{n}\\
0 & \prod_{1}^{n}f\left(k,0\right)
\end{smallmatrix}\right)\quad,\quad c_{n}=\sum_{k=1}^{n}\left(-1\right)^{k-1}\left(\prod_{i=1}^{k-1}\bar{f}\left(k,0\right)\right)\left(\prod_{i=k+1}^{n}f\left(k,0\right)\right).
\]
\end{proof}
\begin{rem}
Note in particular that when $\bar{f}\left(0,0\right)=0$ in the lemma
above, then $U_{\bar{f}\left(0,0\right)}^{\tau}=I$ is simply the
identity matrix.
\end{rem}


\subsection{\label{subsec:The-dual-matrix-field}The dual conservative matrix
field}

With \thmref{normalized-matrix-field} and \lemref{First-row} in
the previous section, we see that we understand quite well both the
$Y=1$ and $X=0$ lines. More over, we already saw that both the horizontal
and the vertical lines in the matrix field are more or less continued
fractions, namely
\begin{align*}
M_{X}\left(x,y\right) & :=\tau U_{f\left(x,y\right)}U_{-\bar{f}\left(x+1,y\right)}D_{b\left(x+1\right)}=\left(\begin{smallmatrix}0 & 1\\
b\left(x+1\right) & f\left(x,y\right)-\bar{f}\left(x+1,y\right)
\end{smallmatrix}\right)\\
M_{Y}\left(x,y\right) & :=U_{f\left(x,y\right)}^{\tau}\tau D_{-\left(f\bar{f}\right)\left(0,y\right)}U_{\bar{f}\left(x,y\right)}^{\tau}=\left(\begin{smallmatrix}\bar{f}\left(x,y\right) & 1\\
b\left(x\right) & f\left(x,y\right)
\end{smallmatrix}\right)
\end{align*}
The next goal is to use this almost symmetry with the hope of eventually
saying something about the diagonal line $X=Y$.
\begin{defn}[\textbf{The dual matrix field}]
Let $f\left(x,y\right),\bar{f}\left(x,y\right)$ be conjugate polynomial,
and let $M_{X},M_{Y}$ be as above. We define the dual matrix field
to be
\begin{align*}
\hat{M}_{Y}\left(y,x\right) & =U_{\bar{f}\left(x-1,y\right)}^{\tau}M_{X}\left(x-1,y+1\right)U_{-\bar{f}\left(x,y\right)}^{\tau}=U_{f\left(x,y\right)}^{\tau}\tau D_{b\left(x\right)}U_{-\bar{f}\left(x,y\right)}^{\tau}\\
\hat{M}_{X}\left(y,x\right) & =U_{\bar{f}\left(x-1,y\right)}^{\tau}M_{Y}\left(x-1,y+1\right)U_{-\bar{f}\left(x-1,y+1\right)}^{\tau}=\tau U_{\bar{f}\left(x-1,y\right)}U_{f\left(x-1,y+1\right)}D_{-b_{Y}\left(y+1\right)}
\end{align*}
This new matrix field corresponds to the conjugate polynomials 
\begin{align*}
\hat{f}\left(x,y\right) & =f\left(y,x\right)\\
\bar{\hat{f}}\left(x,y\right) & =-\bar{f}\left(y,x\right)\\
\hat{a}\left(x,y\right) & =\hat{f}\left(x,y\right)-\bar{\hat{f}}\left(x+1,y\right)=f\left(y,x\right)+\bar{f}\left(y,x+1\right)\\
\hat{b}\left(x\right) & =\left(\hat{f}\bar{\hat{f}}\right)\left(x,0\right)=-\left(f\bar{f}\right)\left(0,x\right)
\end{align*}
\end{defn}

\begin{example}
In the $\zeta\left(3\right)$ matrix field mentioned in \exaref{zeta-3-matrix-field}
we have a special case where
\[
f\left(x,y\right)=\frac{y^{3}-x^{3}}{y-x}\left(y+x\right)\qquad;\qquad\bar{f}\left(x,y\right)=\frac{y^{3}+x^{3}}{y+x}\left(y-x\right),
\]
satisfy $f\left(x,y\right)=f\left(y,x\right)$ and $\bar{f}\left(x,y\right)=-\bar{f}\left(y,x\right)$,
so that $\hat{f}=f$ and $\bar{\hat{f}}=\bar{f}$.

In the one of the $\zeta\left(2\right)$ matrix field from \exaref{vector-field-examples},
we have
\[
f\left(x,y\right)=2x^{2}+2xy+y^{2}\qquad;\qquad\bar{f}\left(x,y\right)=-2x^{2}+2xy-y^{2}
\]
so that 
\[
\hat{f}\left(x,y\right)=x^{2}+2xy+2y^{2}\qquad;\qquad\bar{\hat{f}}\left(x,y\right)=x^{2}-2xy+2y^{2}
\]
and therefore
\begin{align*}
\hat{a}\left(x,y\right) & =\left(x^{2}+2xy+2y^{2}\right)-\left(\left(x+1\right)^{2}-2\left(x+1\right)y+2y^{2}\right)=\left(2y-1\right)\left(2x+1\right)\\
\hat{b}_{X}\left(x\right) & =x^{4}.
\end{align*}

\newpage{}
\end{example}

This dual matrix field construction not only gives us free of charge
another conservative matrix field for every one that we find, but
they are also closely related. In the matrix field with $M_{X},M_{Y}$
, the horizontal lines are (almost) polynomial continued fractions,
and we wish to study how the numerators and denominator behave there.
By definition, the horizontal lines of the dual matrix field correspond
to vertical line in the original matrix field, so understanding the
full matrix field is equivalent to understand these continued fractions.
More precisely, since
\[
M_{Y}\left(x,y\right)=U_{-\bar{f}\left(x,y-1\right)}^{\tau}\hat{M}_{X}\left(y-1,x+1\right)U_{\bar{f}\left(x,y\right)}^{\tau},
\]
we get that
\begin{equation}
\prod_{k=1}^{n}M_{Y}\left(y,k\right)=U_{-\bar{f}\left(y,0\right)}^{\tau}\left[\prod_{k=0}^{n-1}\hat{M}_{X}\left(k,y+1\right)\right]U_{\bar{f}\left(y,n\right)}^{\tau}\label{eq:dual-row-column}
\end{equation}

With this dualic structure we turn to study the rational approximation
given by the different points on the matrix field, and more concretely
how far the standard rational presentation is from being a reduced
rational presentation. 
\begin{defn}
\label{def:deno-nume}For every $n\geq0$ define the polynomial vectors
\begin{align*}
\left(\begin{smallmatrix}p_{n}\left(y\right)\\
q_{n}\left(y\right)
\end{smallmatrix}\right) & =\left[\prod_{0}^{n-1}M_{X}\left(k,y\right)\right]e_{2}\\
\left(\begin{smallmatrix}\hat{p}_{n}\left(y\right)\\
\hat{q}_{n}\left(y\right)
\end{smallmatrix}\right) & =\left[\prod_{0}^{n-1}\hat{M}_{X}\left(k,y\right)\right]e_{2}.
\end{align*}
\end{defn}

For example, the first few values of $p_{n}\left(m\right),q_{n}\left(m\right)$
are arranged as :

\begin{align*}
\xyR{1pc}\xyC{1pc}\xymatrix{\left(*\right) &  &  & \left(*\right) &  &  & \left(*\right) &  &  & \left(*\right)\\
\\
{\left(\begin{smallmatrix}p_{0}\left(3\right)\\
q_{0}\left(3\right)
\end{smallmatrix}\right)}\ar[rrr]_{M_{X}\left(0,3\right)}\ar@{..>}[uu]^{M_{Y}\left(0,3\right)} &  &  & {\left(\begin{smallmatrix}p_{1}\left(3\right)\\
q_{1}\left(3\right)
\end{smallmatrix}\right)}\ar[rrr]_{M_{X}\left(1,3\right)}\ar@{..>}[uu]^{M_{Y}\left(1,3\right)} &  &  & {\left(\begin{smallmatrix}p_{2}\left(3\right)\\
q_{2}\left(3\right)
\end{smallmatrix}\right)}\ar[rrr]_{M_{X}\left(2,3\right)}\ar@{..>}[uu]^{M_{Y}\left(2,3\right)} &  &  & \left(*\right)\\
\\
{\left(\begin{smallmatrix}p_{0}\left(2\right)\\
q_{0}\left(2\right)
\end{smallmatrix}\right)}\ar[rrr]_{M_{X}\left(0,2\right)}\ar@{..>}[uu]^{M_{Y}\left(0,2\right)} &  &  & {\left(\begin{smallmatrix}p_{1}\left(2\right)\\
q_{1}\left(2\right)
\end{smallmatrix}\right)}\ar[rrr]_{M_{X}\left(1,2\right)}\ar@{..>}[uu]^{M_{Y}\left(1,2\right)} &  &  & {\left(\begin{smallmatrix}p_{2}\left(2\right)\\
q_{2}\left(2\right)
\end{smallmatrix}\right)}\ar[rrr]_{M_{X}\left(2,2\right)}\ar@{..>}[uu]^{M_{Y}\left(2,2\right)} &  &  & \left(*\right)\\
\\
{\left(\begin{smallmatrix}p_{0}\left(1\right)\\
q_{0}\left(1\right)
\end{smallmatrix}\right)}\ar[rrr]_{M_{X}\left(0,1\right)}\ar@{..>}[uu]^{M_{Y}\left(0,1\right)} &  &  & {\left(\begin{smallmatrix}p_{1}\left(1\right)\\
q_{1}\left(1\right)
\end{smallmatrix}\right)}\ar[rrr]_{M_{X}\left(1,1\right)}\ar@{..>}[uu]^{M_{Y}\left(1,1\right)} &  &  & {\left(\begin{smallmatrix}p_{2}\left(1\right)\\
q_{2}\left(1\right)
\end{smallmatrix}\right)}\ar[rrr]_{M_{X}\left(2,1\right)}\ar@{..>}[uu]^{M_{Y}\left(2,1\right)} &  &  & \left(*\right)
}
\end{align*}

\begin{rem}
Note that since $M_{X}\left(k,y\right)e_{1}=b\left(k+1\right)e_{2}$
and $\hat{M}_{X}\left(k,y\right)e_{1}=-b_{Y}\left(k\right)$, we have
for $n\geq1$
\begin{align*}
\left(\begin{smallmatrix}p_{n-1}\left(y\right) & p_{n}\left(y\right)\\
q_{n-1}\left(y\right) & q_{n}\left(y\right)
\end{smallmatrix}\right)D_{b_{X}\left(n\right)} & =\prod_{0}^{n-1}M_{X}\left(k,y\right)\\
\left(\begin{smallmatrix}\hat{p}_{n-1}\left(y\right) & \hat{p}_{n}\left(y\right)\\
\hat{q}_{n-1}\left(y\right) & \hat{q}_{n}\left(y\right)
\end{smallmatrix}\right)D_{-b_{Y}\left(n\right)} & =\prod_{0}^{n-1}\hat{M}_{X}\left(k,y\right).
\end{align*}
\end{rem}

To study these polynomials $p_{n}\left(m\right)$ and $q_{n}\left(m\right)$,
we use the conservative matrix field structure to see what happens
when we increase $n$ or increase $m$, and also what is the connections
between them and their duals $\hat{p}_{m}\left(n\right)$ and $\hat{q}_{m}\left(n\right)$.

\newpage{}
\begin{claim}
\label{claim:dual-field-identities}Let $f,\bar{f}\in\ZZ\left[x,y\right]$
be conjugate polynomials such that $\left(f\bar{f}\right)\left(0,0\right)=0$
and let $p_{n},q_{n},\hat{p}_{m},\hat{q}_{m}$ as in \defref{deno-nume}
above. Then
\begin{enumerate}
\item \label{enu:first-line-polynomials}\textbf{\uline{Bottom line}}\textbf{:
}Evaluating the polynomial $p_{n},q_{n}$ at $m=1$ we have
\begin{align*}
p_{n}\left(1\right) & =\sum_{k=1}^{n}\left(-1\right)^{k-1}\left(\prod_{i=1}^{k-1}\bar{f}\left(k,0\right)\right)\left(\prod_{i=k+1}^{n}f\left(k,0\right)\right)\\
q_{n}\left(1\right) & =\prod_{1}^{n}f\left(k,0\right)-\bar{f}\left(0,0\right)p_{n}\left(1\right).
\end{align*}
\item \label{enu:Increase-n}\textbf{\uline{Horizontal lines}}\textbf{:
}When increasing $n$ we get 3-term reccurence
\begin{align*}
\left(\begin{smallmatrix}p_{n+1}\left(y\right)\\
q_{n+1}\left(y\right)
\end{smallmatrix}\right) & =\left(\begin{smallmatrix}p_{n-1}\left(y\right) & p_{n}\left(y\right)\\
q_{n-1}\left(y\right) & q_{n}\left(y\right)
\end{smallmatrix}\right)\left(\begin{smallmatrix}b\left(n\right)\\
a\left(n,y\right)
\end{smallmatrix}\right).
\end{align*}
\item \label{enu:increase-m}\textbf{\uline{Vertical lines}}\textbf{:}
When $\left(f\bar{f}\right)\left(0,m\right)\neq0$, increasing $m$
follows the recurrence 
\begin{align*}
\left(\begin{smallmatrix}p_{n}\left(m+1\right)\\
q_{n}\left(m+1\right)
\end{smallmatrix}\right) & =\frac{1}{\left(f\bar{f}\right)\left(0,m\right)}\left(\begin{smallmatrix}f\left(0,m\right) & -1\\
0 & \bar{f}\left(0,m\right)
\end{smallmatrix}\right)\left(\begin{smallmatrix}p_{n-1}\left(m\right) & p_{n}\left(m\right)\\
q_{n-1}\left(m\right) & q_{n}\left(m\right)
\end{smallmatrix}\right)\left(\begin{smallmatrix}\left(f\bar{f}\right)\left(n,0\right)\\
f\left(n,m\right)
\end{smallmatrix}\right)
\end{align*}
\item \label{enu:reciprocal-polynomials}\textbf{\uline{Conservativeness}}\textbf{:}
Suppose that $\bar{f}\left(0,0\right)=0$. Then $p_{n},q_{n}$ and
$\hat{p}_{m},\hat{q}_{m}$ are related via the equation
\[
\left(\begin{smallmatrix}\prod_{k=1}^{m}\bar{f}\left(0,k\right) & \hat{p}_{m}\left(1\right)\\
0 & \hat{q}_{m}\left(1\right)
\end{smallmatrix}\right)\left(\begin{smallmatrix}p_{n}\left(m+1\right)\\
q_{n}\left(m+1\right)
\end{smallmatrix}\right)=\left(\begin{smallmatrix}\left(-1\right)^{n}\prod_{k=1}^{n}\bar{f}\left(k,0\right) & p_{n}\left(1\right)\\
0 & q_{n}\left(1\right)
\end{smallmatrix}\right)\left(\begin{smallmatrix}\hat{p}_{m}\left(n+1\right)\\
\hat{q}_{m}\left(n+1\right)
\end{smallmatrix}\right),
\]
and in particular we have that $\hat{q}_{m}\left(1\right)q_{n}\left(m+1\right)=q_{n}\left(1\right)\hat{q}_{m}\left(n+1\right)$.
\end{enumerate}
\end{claim}

\begin{proof}
\begin{enumerate}
\item Applying \lemref{First-row} we get 
\begin{align*}
\left(\begin{smallmatrix}p_{n}\left(1\right)\\
q_{n}\left(1\right)
\end{smallmatrix}\right) & =\left[\prod_{0}^{n-1}M_{X}\left(k,1\right)\right]e_{2}=U_{-\bar{f}\left(0,0\right)}^{\tau}\left(\begin{smallmatrix}\left(-1\right)^{n}\prod_{1}^{n}\bar{f}\left(k,0\right) & c_{n}\\
0 & \prod_{1}^{n}f\left(k,0\right)
\end{smallmatrix}\right)U_{\bar{f}\left(n,0\right)}^{\tau}e_{2}\\
 & =U_{-\bar{f}\left(0,0\right)}^{\tau}\left(\begin{smallmatrix}\left(-1\right)^{n}\prod_{1}^{n}\bar{f}\left(k,0\right) & c_{n}\\
0 & \prod_{1}^{n}f\left(k,0\right)
\end{smallmatrix}\right)e_{2}=U_{-\bar{f}\left(0,0\right)}^{\tau}\left(\begin{smallmatrix}c_{n}\\
\prod_{1}^{n}f\left(k,0\right)
\end{smallmatrix}\right)
\end{align*}
where
\begin{align*}
c_{n} & =\sum_{k=1}^{n}\left(-1\right)^{k-1}\left(\prod_{i=1}^{k-1}\bar{f}\left(k,0\right)\right)\left(\prod_{i=k+1}^{n}f\left(k,0\right)\right).
\end{align*}
It follows that 
\begin{align*}
p_{n}\left(1\right) & =e_{1}^{tr}\left(\begin{smallmatrix}p_{n}\left(1\right)\\
q_{n}\left(1\right)
\end{smallmatrix}\right)=e_{1}^{tr}U_{-\bar{f}\left(0,0\right)}^{\tau}\left(\begin{smallmatrix}c_{n}\\
\prod_{1}^{n}f\left(k,0\right)
\end{smallmatrix}\right)=c_{n}\\
q_{n}\left(1\right) & =e_{2}^{tr}\left(\begin{smallmatrix}p_{n}\left(1\right)\\
q_{n}\left(1\right)
\end{smallmatrix}\right)=\left(-\bar{f}\left(0,0\right),1\right)\left(\begin{smallmatrix}c_{n}\\
\prod_{1}^{n}f\left(k,0\right)
\end{smallmatrix}\right)=\prod_{1}^{n}f\left(k,0\right)-\bar{f}\left(0,0\right)c_{n}\\
 & =\sum_{k=0}^{n}\left(-1\right)^{k}\left(\prod_{i=0}^{k-1}\bar{f}\left(k,0\right)\right)\left(\prod_{i=k+1}^{n}f\left(k,0\right)\right).
\end{align*}
\item This is the standard recursion for continued fractions, and it follows
from
\begin{align*}
\left(\begin{smallmatrix}p_{n+1}\left(y\right)\\
q_{n+1}\left(y\right)
\end{smallmatrix}\right) & =\left[\prod_{0}^{n}M_{X}\left(k,y\right)\right]e_{2}=\left[\prod_{0}^{n-1}M_{X}\left(k,y\right)\right]M_{X}\left(n,y\right)e_{2}\\
 & =\left(\begin{smallmatrix}p_{n-1}\left(y\right) & p_{n}\left(y\right)\\
q_{n-1}\left(y\right) & q_{n}\left(y\right)
\end{smallmatrix}\right)D_{b_{X}\left(n\right)}M_{X}\left(n,y\right)e_{2}=\left(\begin{smallmatrix}p_{n-1}\left(y\right) & p_{n}\left(y\right)\\
q_{n-1}\left(y\right) & q_{n}\left(y\right)
\end{smallmatrix}\right)\left(\begin{smallmatrix}b_{X}\left(n\right)\\
a\left(n,y\right)
\end{smallmatrix}\right).
\end{align*}
\item This follows from the coboundary condition of the conservative matrix
field structure
\begin{align*}
\left(\begin{smallmatrix}p_{n}\left(m+1\right)\\
q_{n}\left(m+1\right)
\end{smallmatrix}\right) & =\left[\prod_{0}^{n-1}M_{X}\left(k,m+1\right)\right]e_{2}=M_{Y}\left(0,m\right)^{-1}\left[\prod_{0}^{n-1}M_{X}\left(k,m\right)\right]M_{Y}\left(n,m\right)e_{2}\\
 & =\left(\begin{smallmatrix}\bar{f}\left(0,m\right) & 1\\
0 & f\left(0,m\right)
\end{smallmatrix}\right)^{-1}\left(\begin{smallmatrix}p_{n-1}\left(m\right) & p_{n}\left(m\right)\\
q_{n-1}\left(m\right) & q_{n}\left(m\right)
\end{smallmatrix}\right)D_{b\left(n\right)}\cdot\left(\begin{smallmatrix}1\\
f\left(n,m\right)
\end{smallmatrix}\right)\\
 & =\frac{1}{\left(f\bar{f}\right)\left(0,m\right)}\left(\begin{smallmatrix}f\left(0,m\right) & -1\\
0 & \bar{f}\left(0,m\right)
\end{smallmatrix}\right)\left(\begin{smallmatrix}p_{n-1}\left(m\right) & p_{n}\left(m\right)\\
q_{n-1}\left(m\right) & q_{n}\left(m\right)
\end{smallmatrix}\right)\left(\begin{smallmatrix}\left(f\bar{f}\right)\left(n,0\right)\\
f\left(n,m\right)
\end{smallmatrix}\right)
\end{align*}
\item We compute the matrix in the $\left(n,m+1\right)$ position in the
matrix field in two different ways - first by moving along the $X=0$
line and then $Y=m+1$ line, and second by moving along the $Y=1$
line and then the $X=n$ line, namely 
\[
\prod_{1}^{m}M_{Y}\left(0,k\right)\prod_{0}^{n-1}M_{X}\left(k,m+1\right)=\prod_{0}^{n-1}M_{X}\left(k,1\right)\prod_{1}^{m}M_{Y}\left(n,k\right).
\]
Converting the $M_{Y}$ into $\hat{M}_{X}$ via \eqref{dual-row-column}
we have
\[
U_{-\bar{f}\left(0,0\right)}^{\tau}\left[\prod_{k=0}^{m-1}\hat{M}_{X}\left(k,1\right)\right]U_{\bar{f}\left(0,m\right)}^{\tau}\prod_{0}^{n-1}M_{X}\left(k,m+1\right)=\prod_{0}^{n-1}M_{X}\left(k,1\right)U_{-\bar{f}\left(n,0\right)}^{\tau}\left[\prod_{k=0}^{m-1}\hat{M}_{X}\left(k,n+1\right)\right]U_{\bar{f}\left(n,m\right)}^{\tau}.
\]
Multiplying both side by $e_{2}$, we get 
\[
U_{-\bar{f}\left(0,0\right)}^{\tau}\left[\prod_{k=0}^{m-1}\hat{M}_{X}\left(k,1\right)\right]U_{\bar{f}\left(0,m\right)}^{\tau}\left(\begin{smallmatrix}p_{n}\left(m+1\right)\\
q_{n}\left(m+1\right)
\end{smallmatrix}\right)=\left[\prod_{0}^{n-1}M_{X}\left(k,1\right)\right]U_{-\bar{f}\left(n,0\right)}^{\tau}\left(\begin{smallmatrix}\hat{p}_{m}\left(n+1\right)\\
\hat{q}_{m}\left(n+1\right)
\end{smallmatrix}\right).
\]
Next, we use \lemref{First-row} as in the previous part, and the
fact that $U_{\bar{f}\left(0,0\right)}^{\tau}=Id$ to get that 
\begin{align*}
\left[\prod_{0}^{n-1}M_{X}\left(k,1\right)\right]U_{-\bar{f}\left(n,0\right)}^{\tau} & =\left(\begin{smallmatrix}\left(-1\right)^{n}\prod_{1}^{n}\bar{f}\left(k,0\right) & p_{n}\left(1\right)\\
0 & q_{n}\left(1\right)
\end{smallmatrix}\right)\\
\left[\prod_{0}^{m-1}\hat{M}_{X}\left(k,1\right)\right]U_{\bar{f}\left(0,m\right)}^{\tau} & =\left(\begin{smallmatrix}\prod_{1}^{m}\bar{f}\left(0,k\right) & \hat{p}_{m}\left(1\right)\\
0 & \hat{q}_{m}\left(1\right)
\end{smallmatrix}\right).
\end{align*}
Putting everything together, we get 
\[
\left(\begin{smallmatrix}\prod_{1}^{m}\bar{f}\left(0,k\right) & \hat{p}_{m}\left(1\right)\\
0 & \hat{q}_{m}\left(1\right)
\end{smallmatrix}\right)\left(\begin{smallmatrix}p_{n}\left(m+1\right)\\
q_{n}\left(m+1\right)
\end{smallmatrix}\right)=\left(\begin{smallmatrix}\left(-1\right)^{n}\prod_{1}^{n}\bar{f}\left(k,0\right) & p_{n}\left(1\right)\\
0 & q_{n}\left(1\right)
\end{smallmatrix}\right)\left(\begin{smallmatrix}\hat{p}_{m}\left(n+1\right)\\
\hat{q}_{m}\left(n+1\right)
\end{smallmatrix}\right),
\]
which is what we wanted to show.
\end{enumerate}
\end{proof}
If $\bar{f}\left(0,0\right)=0$ as in the last part of the claim above,
then $\frac{q_{n}\left(m+1\right)}{q_{n}\left(1\right)}=\frac{\hat{q}_{m}\left(n+1\right)}{\hat{q}_{m}\left(1\right)}$
. Fixing $m$ and letting $n\to\infty$, the numbers $q_{n}\left(m\right)$
are simply the denominators for the continued fraction on the matrix
field's $m$'th row. As we already know how to compute $q_{n}\left(1\right)$,
to understand these denominators, we would need to understand $\frac{\hat{q}_{m}\left(n+1\right)}{\hat{q}_{m}\left(1\right)}$.
Since $m$ is fixed, the function $n\mapsto\hat{q}_{m}\left(n+1\right)$
is just polynomial in $n$, and we divide it by the constant $\hat{q}_{m}\left(1\right)$.
This already tells us a lot about these denominators.

Eventually we would want to find and show that $\gcd\left(q_{n}\left(m+1\right),p_{n}\left(m+1\right)\right)$
is large. In particular, it would be helpful to know if $q_{n}\left(1\right)\mid q_{n}\left(m+1\right)$
for all $n$, which we just saw is equivalent to $\frac{\hat{q}_{m}\left(n+1\right)}{\hat{q}_{m}\left(1\right)}$
always being an integer. Hence, we are left with the problem of checking
if all the evaluations of a polynomial $\hat{q}_{m}$ at integer points
are divisible by the same number $\hat{q}_{m}\left(1\right)$. The
solution to this type of question is well known, and it not hard to
show that this holds exactly when $\hat{q}_{m}\left(1\right)\mid\hat{q}_{m}\left(n+1\right)$
for $\deg\left(\hat{q}_{m}\right)+1$ consecutive integers (see \appref{integer-values}).
This suggest an induction like process to show that this holds for
all of the denominators, and in particular we will show it for the
matrix field of $\zeta\left(3\right)$.

\newpage{}

\section{\label{sec:The-z3-case}The $\zeta\left(3\right)$ case}

We now apply the dual matrix field identities for the $\zeta\left(3\right)$
matrix field. Recall that in this case we have that 
\begin{align*}
f\left(x,y\right) & =\frac{y^{3}-x^{3}}{y-x}\left(y+x\right)=y^{3}+2y^{2}x+2yx^{2}+x^{3}\\
\bar{f}\left(x,y\right) & =\frac{y^{3}+x^{3}}{y+x}\left(y-x\right)=y^{3}-2y^{2}x+2yx^{2}-x^{3}\\
a\left(x,y\right) & =x^{3}+\left(1+x\right)^{3}+2y\left(y-1\right)\left(2x+1\right)\\
b\left(x\right) & =-x^{6}.
\end{align*}

Our first goal is showing that $\gcd\left(q_{n}\left(m\right),p_{n}\left(m\right)\right)$
is large as $n$ and $m$ increase, and we will use the results from
\claimref{dual-field-identities}. Once we understand these polynomials
and their gcd, which are defined for each row separately, we will
combine them together to understand the general numerators and denominators
appearing in any route on the matrix field, starting at the bottom
left corner. In particular, investigating the diagonal route, we will
show that both the approximations converge fast enough, and the gcd
grows fast enough to conclude at the end that $\zeta\left(3\right)$
is irrational.

This $\zeta\left(3\right)$ matrix field has several properties making
it easier to work with, which will come into play later:
\begin{fact}
\label{fact:zeta_3}
\begin{enumerate}
\item The matrix field is its own dual, since $f\left(y,x\right)=f\left(x,y\right)$
and $-\bar{f}\left(y,x\right)=\bar{f}\left(x,y\right)$. In particular
we get that $p=\hat{p}$ and $q=\hat{q}$.
\item We have that $f\left(0,0\right)=\bar{f}\left(0,0\right)=0$.
\item All the $f\left(n,0\right),f\left(0,n\right),\bar{f}\left(n,0\right),\bar{f}\left(0,n\right)$
are the same up to a sign (and therefore also $\hat{f}$ and $\bar{\hat{f}}$),
namely these are $n^{3}$. Furthermore, they all divide $f\left(n,n\right)=\hat{f}\left(n,n\right)=6n^{3}$.
\item We can write $a\left(x,y\right)$ as $a\left(x,y\right)=A_{1}\left(x\right)+y\left(y-1\right)A_{2}\left(x\right)$,
so in particular $a\left(x,1-y\right)=a\left(x,y\right)$.
\end{enumerate}
\end{fact}

Next, we want to show that $\gcd\left(q_{n}\left(m\right),p_{n}\left(m\right)\right)$,
is almost divisible by $\left(n!\right)^{3}$. We already know that
fixing $m$ and only increasing $n$, namely running on horizontal
lines in the matrix field, we get ``nice'' continued fractions which
should have factorial reduction. The next lemma shows that these factorial
reduction are in a sense synchronized between the different horizontal
lines.

In the following, we will use \textbf{$\lcm\left[n\right]$ }for $\lcm\left\{ 1,2,...,n\right\} $
where $n\geq1$ and also set $\lcm\left[0\right]=1$.
\begin{lem}
\label{lem:q-n-lemma}For all $n\geq0$ and $m\in\ZZ$ we have $\left(n!\right)^{3}\mid q_{n}\left(m\right)$
(with equality for $m=1$) and $\left(\frac{n!}{\lcm\left[n\right]}\right)^{3}\mid p_{n}\left(m\right)$.
In particular we have that $\left(\frac{n!}{\lcm\left[n\right]}\right)^{3}\mid\gcd\left(p_{n}\left(m\right),q_{n}\left(m\right)\right)$.
\end{lem}

\begin{proof}
We will prove this claim by induction on $n$, but before that, we
first consider the case where $m=1$ and $n$ is arbitrary (the bottom
horizontal line). By part \ref{enu:first-line-polynomials} in \claimref{dual-field-identities}
we get that 
\begin{align*}
p_{n}\left(1\right) & =\sum_{k=1}^{n}\left(-1\right)^{k-1}\left(\prod_{i=1}^{k-1}\bar{f}\left(k,0\right)\right)\left(\prod_{i=k+1}^{n}f\left(k,0\right)\right)=\sum_{1}^{n}\left(\frac{n!}{k}\right)^{3}\\
q_{n}\left(1\right) & =\prod_{1}^{n}f\left(k,0\right)=\left(n!\right)^{3}.
\end{align*}
These are exactly the numerator and denominator of $\sum_{1}^{n}\frac{1}{k^{3}}$
when taking the product of the denominators. Since we can also instead
take the least common multiple of the denominators, we see that $\left(\frac{n!}{\lcm\left[n\right]}\right)^{3}\mid p_{n}\left(1\right)$
as required. Of course the $\left(n!\right)^{3}\mid q_{n}\left(1\right)$
is trivial since $\left(n!\right)^{3}=q_{n}\left(1\right)$, but more
over it allows us to think of the general conditions as $q_{n}\left(1\right)\mid q_{n}\left(m\right)$
and $q_{1}\left(n\right)\mid\lcm\left[n\right]^{3}p_{n}\left(m\right)$.\\

We prove the rest of this lemma using induction on $n$. The induction
hypothesis will go as follows - assuming that the claim is true for
$\left(n-1,m\right)$ for a given $n$ and all $m\in\ZZ$, we show:
\begin{enumerate}
\item From part \enuref{reciprocal-polynomials} in \claimref{dual-field-identities},
we show that the claim is true for $\left(n,m\right)$ with $1\leq m\leq n$.
\item From part \enuref{increase-m} in \claimref{dual-field-identities},
if the claim is true for $\left(n-1,n\right)$ and $\left(n,n\right)$,
then it is true for $\left(n,n+1\right)$.
\item Our polynomials satisfy $q_{n}\left(y\right)=q_{n}\left(1-y\right)$
and $p_{n}\left(y\right)=p_{n}\left(1-y\right)$, so the claim is
true for $\left(n,m\right)$ with $-n\leq m\leq n+1$, which are $2\left(n+1\right)$
consecutive integers.
\item These polynomials have degree $\leq2n+1$, so this is enough to show
the claim for $\left(n,m\right)$ for all $m$.
\end{enumerate}

When $n=0$ we have $q_{0}\left(m\right)\equiv1$ and $p_{0}\left(m\right)\equiv0$
which are divisible by $\left(0!\right)^{3}=1$ and $\frac{0!}{\lcm\left[0\right]}=1$
respectively.

Suppose now that the claim is true for $\left(k,m\right)$ with $k\leq n-1$
and all $m$ and we prove for $\left(n,m\right)$ and all $m$. We
prove first for the denominators, which is easier.\\

\textbf{\uline{Denominators:}}

By using identity \enuref{reciprocal-polynomials} from \claimref{dual-field-identities},
together with the facts in \factref{zeta_3} about the matrix field
we get 
\[
\left(\begin{smallmatrix}q_{m}\left(1\right) & p_{m}\left(1\right)\\
0 & q_{m}\left(1\right)
\end{smallmatrix}\right)\left(\begin{smallmatrix}p_{n}\left(m+1\right)\\
q_{n}\left(m+1\right)
\end{smallmatrix}\right)=\left(\begin{smallmatrix}q_{n}\left(1\right) & p_{n}\left(1\right)\\
0 & q_{n}\left(1\right)
\end{smallmatrix}\right)\left(\begin{smallmatrix}p_{m}\left(n+1\right)\\
q_{m}\left(n+1\right)
\end{smallmatrix}\right).
\]
For the denominators, this implies that 
\[
\frac{q_{n}\left(m+1\right)}{q_{n}\left(1\right)}=\frac{q_{m}\left(n+1\right)}{q_{m}\left(1\right)}.
\]
By the induction hypothesis, for $0\leq m\leq n-1$ the right hand
of this equation is an integer, so that $q_{n}\left(1\right)\mid q_{n}\left(m+1\right)$.
Using part \enuref{increase-m} in \claimref{dual-field-identities}
with $n=m$ we have

\begin{align*}
\left(\begin{smallmatrix}p_{n}\left(n+1\right)\\
q_{n}\left(n+1\right)
\end{smallmatrix}\right) & =\frac{1}{\left(f\bar{f}\right)\left(0,n\right)}\left(\begin{smallmatrix}f\left(0,n\right) & -1\\
0 & \bar{f}\left(0,n\right)
\end{smallmatrix}\right)\left(\begin{smallmatrix}p_{n-1}\left(n\right) & p_{n}\left(n\right)\\
q_{n-1}\left(n\right) & q_{n}\left(n\right)
\end{smallmatrix}\right)\left(\begin{smallmatrix}\left(f\bar{f}\right)\left(n,0\right)\\
f\left(n,n\right)
\end{smallmatrix}\right)
\end{align*}
so for the denominators we get
\[
q_{n}\left(n+1\right)=q_{n-1}\left(n\right)f\left(n,0\right)\frac{\bar{f}\left(n,0\right)}{f\left(0,n\right)}+q_{n}\left(n\right)\frac{f\left(n,n\right)}{f\left(0,n\right)}=q_{n-1}\left(n\right)n^{3}\left(-1\right)+q_{n}\left(n\right)\cdot6.
\]
By the induction hypothesis $\left(n-1\right)!^{3}\mid q_{n-1}\left(n\right)$
and from the argument above $n!^{3}\mid q_{n}\left(n\right)$, so
we conclude that $n!^{3}\mid q_{n}\left(n+1\right)$. At this point,
we know the claim for $\left(n,m\right)$ with $1\leq m\leq n+1$.

Using the fact that $a\left(x,y\right)$ can be written as $A_{1}\left(x\right)+y\left(y-1\right)A_{2}\left(x\right)$
, we get that $a\left(x,y\right)=a\left(x,1-y\right)$. Since
\begin{align*}
q_{n}\left(y\right) & =e_{2}^{tr}\left[\prod_{0}^{n-1}M_{X}\left(k,y\right)\right]e_{2}=e_{2}^{tr}\left[\prod_{0}^{n-1}\left(\begin{smallmatrix}0 & 1\\
b\left(k+1\right) & a\left(k,y\right)
\end{smallmatrix}\right)\right]e_{2}
\end{align*}
we also get that $q_{n}\left(1-y\right)=q_{n}\left(y\right)$ and
$\deg_{y}\left(q_{n}\right)\leq2n$. From this we conclude that $q_{n}\left(1\right)\mid q_{n}\left(m\right)$
for all $-n\leq m\leq n+1$, which is a total of $2n+2\geq\deg_{y}\left(q_{n}\right)+1$
consecutive integers. Finally, using \lemref{binomial-polynomials}
from \appref{integer-values} we conclude that $q_{n}\left(1\right)\mid q_{n}\left(m\right)$
for all $m$, thus proving the induction step for the denominators.\\
\newpage{}

\textbf{\uline{Numerators:}}

The proof for the numerators is similar, but requires a bit more computations.
Part \enuref{reciprocal-polynomials} from \claimref{dual-field-identities}
\[
q_{m}\left(1\right)p_{n}\left(m+1\right)+p_{m}\left(1\right)q_{n}\left(m+1\right)=q_{n}\left(1\right)p_{m}\left(n+1\right)+p_{n}\left(1\right)q_{m}\left(n+1\right)
\]
can be rewritten as 
\[
\frac{\lcm\left[n\right]^{3}p_{n}\left(m+1\right)}{q_{n}\left(1\right)}=\overbrace{\frac{\lcm\left[n\right]^{3}p_{m}\left(n+1\right)}{q_{m}\left(1\right)}}^{\left(1\right)}+\overbrace{\frac{\lcm\left[n\right]^{3}p_{n}\left(1\right)}{q_{n}\left(1\right)}}^{\left(2\right)}\cdot\overbrace{\frac{q_{m}\left(n+1\right)}{q_{m}\left(1\right)}}^{\left(3\right)}-\overbrace{\frac{\lcm\left[n\right]^{3}p_{m}\left(1\right)}{q_{m}\left(1\right)}}^{\left(4\right)}\cdot\overbrace{\frac{q_{n}\left(m+1\right)}{q_{n}\left(1\right)}}^{\left(5\right)}.
\]
To show that the expression on the left is an integer, it is enough
to show that $\left(1\right)-\left(5\right)$ on the right are integers. 
\begin{itemize}
\item Expression $\left(2\right)$ is on the first row of the matrix field,
and we saw in the beginning of the proof that it is an integer.
\item Expressions $\left(3\right)$ and $\left(5\right)$ follows from the
claim about the denominators (which is independent of this proof about
the numerators).
\item Expressions $\left(1\right)$ and $\left(4\right)$ are true if $0\leq m\leq n-1$
using the induction hypothesis, and the fact that $\lcm\left[m\right]\mid\lcm\left[n\right]$
in that case.
\end{itemize}
To conclude, we just saw that the claim is true for $\left(n,m\right)$
when $1\leq m\leq n$.

Using part \enuref{increase-m} in \claimref{dual-field-identities}
with $n=m$ for the numerators we get 
\begin{align*}
p_{n}\left(n+1\right) & =\frac{1}{\left(f\bar{f}\right)\left(0,n\right)}\left[f\left(0,n\right)\left(\left(f\bar{f}\right)\left(n,0\right)p_{n-1}\left(n\right)+f\left(n,n\right)p_{n}\left(n\right)\right)-\left(\left(f\bar{f}\right)\left(n,0\right)q_{n-1}\left(n\right)+f\left(n,n\right)q_{n}\left(n\right)\right)\right]\\
 & =\left(-n^{3}p_{n-1}\left(n\right)+6p_{n}\left(n\right)\right)+\left(q_{n-1}\left(n\right)-\frac{6}{n^{3}}q_{n}\left(n\right)\right).
\end{align*}
Since the claim is true for $\left(n-1,n\right)$ and $\left(n,n\right)$,
we get that
\begin{align*}
\left(\frac{n!}{\lcm\left[n\right]}\right)^{3} & \mid\left(-n^{3}p_{n-1}\left(n\right)+6p_{n}\left(n\right)\right)\\
\left(n-1\right)!^{3} & \mid\left(q_{n-1}\left(n\right)-\frac{6}{n^{3}}q_{n}\left(n\right)\right).
\end{align*}
Since $n\mid\lcm\left[n\right]$ , it follows that $\frac{n!}{\lcm\left[n\right]}\mid\left(n-1\right)!$,
so everything together shows that 
\[
\left(\frac{n!}{\lcm\left[n\right]}\right)^{3}\mid p_{n}\left(n+1\right).
\]
At this point we know that $\left(\frac{n!}{\lcm\left[n\right]}\right)^{3}\mid p_{n}\left(m\right)$
for $1\leq m\leq n+1$. The same trick as with the denominators show
that $p_{n}\left(m\right)=p_{n}\left(1-m\right)$, so the claim is
true for $-n\leq m\leq n+1$, and using \lemref{binomial-polynomials}
again we conclude that it is true for all $m$, thus finishing the
proof for the induction step, and therefore the original claim.
\end{proof}
Up until now we looked at each row separately. We now move to the
whole matrix field.
\begin{defn}
Given $n\geq0$ and $m\geq1$, define
\[
\left(\begin{smallmatrix}P\left(n,m\right)\\
Q\left(n,m\right)
\end{smallmatrix}\right):=\left[\prod_{k=1}^{m-1}M_{Y}\left(0,k\right)\right]\left[\prod_{k=0}^{n-1}M_{X}\left(k,m\right)\right]e_{2}.
\]
In particular, as Mobius transformations we get that 
\[
\left[\prod_{k=1}^{m-1}M_{Y}\left(0,k\right)\right]\left[\prod_{k=0}^{n-1}M_{X}\left(k,m\right)\right]\left(0\right)=\frac{P\left(n,m\right)}{Q\left(n,m\right)}.
\]
\end{defn}

\begin{rem}
Note that for the general matrix field with $\bar{f}\left(0,0\right)=0$,
we can use \enuref{reciprocal-polynomials} in \claimref{dual-field-identities}
to get 
\[
\left(\begin{smallmatrix}P\left(n,m\right)\\
Q\left(n,m\right)
\end{smallmatrix}\right)=\left(\begin{smallmatrix}\prod_{k=1}^{m}\bar{f}\left(0,k\right) & \hat{p}_{m}\left(1\right)\\
0 & \hat{q}_{m}\left(1\right)
\end{smallmatrix}\right)\left(\begin{smallmatrix}p_{n}\left(m+1\right)\\
q_{n}\left(m+1\right)
\end{smallmatrix}\right)=\left(\begin{smallmatrix}\left(-1\right)^{n}\prod_{k=1}^{n}\bar{f}\left(k,0\right) & p_{n}\left(1\right)\\
0 & q_{n}\left(1\right)
\end{smallmatrix}\right)\left(\begin{smallmatrix}\hat{p}_{m}\left(n+1\right)\\
\hat{q}_{m}\left(n+1\right)
\end{smallmatrix}\right).
\]
In particular, in our $\zeta\left(3\right)$ case we have 
\[
\left(\begin{smallmatrix}P\left(n,m\right)\\
Q\left(n,m\right)
\end{smallmatrix}\right)=\left(\begin{smallmatrix}q_{m}\left(1\right) & p_{m}\left(1\right)\\
0 & q_{m}\left(1\right)
\end{smallmatrix}\right)\left(\begin{smallmatrix}p_{n}\left(m+1\right)\\
q_{n}\left(m+1\right)
\end{smallmatrix}\right)=\left(\begin{smallmatrix}q_{n}\left(1\right) & p_{n}\left(1\right)\\
0 & q_{n}\left(1\right)
\end{smallmatrix}\right)\left(\begin{smallmatrix}p_{m}\left(n+1\right)\\
q_{m}\left(n+1\right)
\end{smallmatrix}\right).
\]
\end{rem}

With this new notation, we have the new factorial reduction for these
numerators and denominators.
\begin{cor}
\label{cor:factorial-reduction}For all $n,m\geq0$ we have that 
\[
q_{m}\left(1\right)q_{n}\left(1\right)\mid Q\left(n,m+1\right),
\]
\begin{align*}
\frac{q_{m}\left(1\right)q_{n}\left(1\right)}{\lcm\left[\max\left(m,n\right)\right]^{3}} & \mid\gcd\left(P\left(n,m+1\right),Q\left(n,m+1\right)\right).
\end{align*}
In particular for $n=m$ we get that 
\begin{align*}
\left(\frac{n!}{\lcm\left[n\right]}\cdot n!\right)^{3}=\frac{\left(q_{n}\left(1\right)\right)^{2}}{\lcm\left[n\right]^{3}} & \mid\gcd\left(P\left(n,n+1\right),Q\left(n,n+1\right)\right).
\end{align*}
\end{cor}

\begin{proof}
Using the presentation from the remark above
\[
\left(\begin{smallmatrix}P\left(n,m+1\right)\\
Q\left(n,m+1\right)
\end{smallmatrix}\right)=\left(\begin{smallmatrix}q_{m}\left(1\right) & p_{m}\left(1\right)\\
0 & q_{m}\left(1\right)
\end{smallmatrix}\right)\left(\begin{smallmatrix}p_{n}\left(m+1\right)\\
q_{n}\left(m+1\right)
\end{smallmatrix}\right),
\]
and \lemref{q-n-lemma} we get that 
\begin{align*}
q_{m}\left(1\right)q_{n}\left(1\right) & \mid q_{m}\left(1\right)q_{n}\left(m+1\right)=Q\left(n,m+1\right)\\
\frac{q_{m}\left(1\right)q_{n}\left(1\right)}{\lcm\left[\max\left(m,n\right)\right]^{3}} & \mid q_{m}\left(1\right)p_{n}\left(m+1\right)+p_{m}\left(1\right)q_{n}\left(m+1\right)=P\left(n,m+1\right).
\end{align*}
\end{proof}
This factorial reduction property, will help us in the end to show
that $\zeta\left(3\right)$ is irrational, but it can also be used
to show more general properties of the matrix field, as follows.
\begin{thm}
\label{thm:all-directions-converge}Let $n_{i},m_{i}\geq1$ be any
sequence such that $\max\left(n_{i},m_{i}\right)\to\infty$. Then
$\frac{P\left(n_{i},m_{i}\right)}{Q\left(n_{i},m_{i}\right)}\to\zeta\left(3\right)$.
\end{thm}

Before we prove this theorem, here is an interesting corollary for
using this theorem for fixed $m$.
\begin{cor}
For any $m\geq1$, the limit for the $Y=m$ line is$\limfi n\frac{p_{n}\left(m\right)}{q_{n}\left(m\right)}=\sum_{m}^{\infty}\frac{1}{n^{3}}.$
\end{cor}

\begin{proof}
Using the notation $\left(\begin{smallmatrix}P\left(n,m-1\right)\\
Q\left(n,m-1\right)
\end{smallmatrix}\right)=\left(\begin{smallmatrix}q_{m-1}\left(1\right) & p_{m-1}\left(1\right)\\
0 & q_{m-1}\left(1\right)
\end{smallmatrix}\right)\left(\begin{smallmatrix}p_{n}\left(m\right)\\
q_{n}\left(m\right)
\end{smallmatrix}\right)$, we get that 
\[
\frac{P\left(n,m-1\right)}{Q\left(n,m-1\right)}=\frac{p_{n}\left(m\right)}{q_{n}\left(m\right)}+\frac{p_{m-1}\left(1\right)}{q_{m-1}\left(1\right)}.
\]
By \thmref{all-directions-converge} we know that $\limfi n\frac{P\left(n,m-1\right)}{Q\left(n,m-1\right)}=\zeta\left(3\right)$,
and we have already seen that $\frac{p_{m-1}\left(1\right)}{q_{m-1}\left(1\right)}=\sum_{1}^{m-1}\frac{1}{n^{3}}$,
so together we get that $\limfi n\frac{p_{n}\left(m\right)}{q_{n}\left(m\right)}=\sum_{m}^{\infty}\frac{1}{n^{3}}$.
\end{proof}
And now for the proof of the theorem.
\begin{proof}[Proof of \thmref{all-directions-converge}]
\textbf{\uline{The \mbox{$m_{i}$} bounded case}}\textbf{: }Suppose
first that $m_{i}$ is bounded, and by splitting the sequence to finitely
many subsequence, we may assume that $m_{i}=m$ is constant. We use
the presentation 
\[
\left(\begin{smallmatrix}P\left(n_{i},m\right)\\
Q\left(n_{i},m\right)
\end{smallmatrix}\right):=\left(\begin{smallmatrix}q_{n_{i}}\left(1\right) & p_{n_{i}}\left(1\right)\\
0 & q_{n_{i}}\left(1\right)
\end{smallmatrix}\right)\left(\begin{smallmatrix}\hat{p}_{m}\left(n_{i}+1\right)\\
\hat{q}_{m}\left(n_{i}+1\right)
\end{smallmatrix}\right)
\]
so that 
\[
\frac{P\left(n_{i},m\right)}{Q\left(n_{i},m\right)}=\frac{p_{m}\left(n_{i}+1\right)}{q_{m}\left(n_{i}+1\right)}+\frac{p_{n_{i}}\left(1\right)}{q_{n_{i}}\left(1\right)}.
\]

We already know that $\limfi n\frac{p_{n}\left(1\right)}{q_{n}\left(1\right)}=\zeta\left(3\right)$,
so if we can show that $\deg\left(q_{m}\right)>\deg\left(p_{m}\right)$,
then $\limfi n\frac{p_{m}\left(n\right)}{q_{m}\left(n\right)}=0$.
Indeed, recall that 
\[
\left(\begin{smallmatrix}p_{m}\left(y\right)\\
q_{m}\left(y\right)
\end{smallmatrix}\right)=\left[\prod_{0}^{m-1}M_{X}\left(k,y\right)\right]e_{2}
\]
where $M_{X}\left(x,y\right)=\left(\begin{smallmatrix}0 & 1\\
b\left(x+1\right) & a\left(x,y\right)
\end{smallmatrix}\right)$, and 
\[
a\left(x,y\right)=x^{3}+\left(1+x\right)^{3}+2y\left(y-1\right)\left(2x+1\right).
\]
This means that for every $k\geq0$ we have that $\deg_{y}\left(a\left(k,y\right)\right)=2$,
and by induction $\deg\left(p_{m}\right)=2m-2$ while $\deg\left(q_{m}\right)=2m$.
Hence $\deg\left(q_{m}\right)>\deg\left(p_{m}\right)$ and we are
done.\\

\textbf{\uline{The \mbox{$m_{i}$} unbounded case}}\textbf{:} Here
we will use the second presentation of $P$ and $Q$, namely
\[
\left(\begin{smallmatrix}P\left(n,m\right)\\
Q\left(n,m\right)
\end{smallmatrix}\right)=\left(\begin{smallmatrix}q_{m}\left(1\right) & p_{m}\left(1\right)\\
0 & q_{m}\left(1\right)
\end{smallmatrix}\right)\left(\begin{smallmatrix}p_{n}\left(m+1\right)\\
q_{n}\left(m+1\right)
\end{smallmatrix}\right)
\]
and therefore
\[
\frac{P\left(n,m\right)}{Q\left(n,m\right)}=\frac{p_{n}\left(m+1\right)}{q_{n}\left(m+1\right)}+\frac{p_{m}\left(1\right)}{q_{m}\left(1\right)}.
\]

Fix some $\varepsilon>0$. Since $\limfi m\frac{p_{m}\left(1\right)}{q_{m}\left(1\right)}=\zeta\left(3\right)$,
for all $m$ large enough we have $\left|\frac{p_{m}\left(1\right)}{q_{m}\left(1\right)}-\zeta\left(3\right)\right|\leq\frac{\varepsilon}{2}$.
As we shall see below, for all $m$ large enough we also have that
$\left|\frac{p_{n}\left(m+1\right)}{q_{n}\left(m+1\right)}\right|\leq\frac{\varepsilon}{2}$
independent of $n$. Hence, we can find $M=M_{\varepsilon}$ so that
for $m\geq M_{\varepsilon}$ we have $\left|\frac{P\left(n,m\right)}{Q\left(n,m\right)}-\zeta\left(3\right)\right|\leq\varepsilon$.
Thus, if we look at the two subsequence of $\left(n_{i},m_{i}\right)$
, where $m_{i}\geq M_{\varepsilon}$ and where $m_{i}\leq M_{\varepsilon}$,
we get that $\left|\frac{P\left(n_{i},m_{i}\right)}{Q\left(n_{i},m_{i}\right)}-\zeta\left(3\right)\right|\leq\varepsilon$
on the first subsequence, and from the previous case, if the second
subsequence is infinite, so that $n_{i}\to\infty$, we have $\left|\frac{P\left(n_{i},m_{i}\right)}{Q\left(n_{i},m_{i}\right)}-\zeta\left(3\right)\right|\leq\varepsilon$
for all $i$ large enough.

We are left to show that $\left|\frac{p_{n}\left(m\right)}{q_{n}\left(m\right)}\right|\leq\frac{\varepsilon}{2}$
for all $m$ large enough (independent of $n\geq0$). 

Note that the result in the corollary above that $\limfi n\frac{p_{n}\left(m\right)}{q_{n}\left(m\right)}=\sum_{m}^{\infty}\frac{1}{n^{3}}$
only depends on the $m$ bounded case, so we already fully proved
it. This is a tail of a convergent series, so it will be small for
all large enough $m$. With this motivation (without the result itself
from the corollary) we show that as $m$ increases , $\left|\frac{p_{n}\left(m\right)}{q_{n}\left(m\right)}\right|$
becomes bounded by similar such tail and therefore is as small as
we want.

Recall that 
\[
\left(\begin{smallmatrix}p_{n-1}\left(y\right) & p_{n}\left(y\right)\\
q_{n-1}\left(y\right) & q_{n}\left(y\right)
\end{smallmatrix}\right)D_{b_{X}\left(n\right)}=\prod_{0}^{n-1}M_{X}\left(k,y\right),
\]
so taking the determinant, we get that 
\[
\left(p_{n-1}\left(y\right)q_{n}\left(y\right)-p_{n}\left(y\right)q_{n-1}\left(y\right)\right)b\left(n\right)=\prod_{1}^{n}\left(-b\left(k\right)\right),
\]
which we can rewrite as
\[
\frac{p_{n}\left(y\right)}{q_{n}\left(y\right)}=\frac{p_{n-1}\left(y\right)}{q_{n-1}\left(y\right)}-\frac{\left(-1\right)^{n}\prod_{1}^{n-1}b\left(k\right)}{q_{n-1}\left(y\right)q_{n}\left(y\right)}.
\]

Using the fact that $p_{0}\left(y\right)=0$ , $q_{0}\left(y\right)=1$
and $\prod_{1}^{j-1}\left|b\left(k\right)\right|=\left(\left(j-1\right)!\right)^{6}=\left|q_{j-1}\left(1\right)\right|^{2}$,
we get that 
\[
\left|\frac{p_{n}\left(y\right)}{q_{n}\left(y\right)}\right|=\left|\sum_{j=1}^{n}\frac{\left(-1\right)^{j}\prod_{1}^{j-1}b\left(k\right)}{q_{j-1}\left(y\right)q_{j}\left(y\right)}\right|\leq\sum_{j=1}^{\infty}\left|\frac{\prod_{1}^{j-1}b\left(k\right)}{q_{j-1}\left(y\right)q_{j}\left(y\right)}\right|.
\]
By \lemref{q-n-lemma} we have that $\prod_{1}^{n-1}b\left(k\right)=\left(n-1\right)!^{6}=q_{n-1}\left(1\right)^{2}$,
and also $q_{n-1}\left(1\right)\mid q_{n-1}\left(m\right)$ and $q_{n-1}\left(1\right)n^{3}=q_{n}\left(n\right)\mid q_{n}\left(m\right)$
so that 
\[
\left|\frac{\prod_{1}^{j-1}b\left(k\right)}{q_{j-1}\left(m\right)q_{j}\left(m\right)}\right|=\left|\frac{q_{j-1}\left(1\right)}{q_{j-1}\left(m\right)}\cdot\frac{q_{j-1}\left(1\right)j^{3}}{q_{j}\left(m\right)}\cdot\frac{1}{j^{3}}\right|\leq\frac{1}{j^{3}}.
\]
This already shows that $\left|\frac{p_{n}\left(m\right)}{q_{n}\left(m\right)}\right|\leq\sum_{j=1}^{\infty}\frac{1}{j^{3}}$,
which is of course not enough, as instead of the tail, we got the
full sum. To solve this, we note that each one of the $q_{j}\left(y\right)$
for fixed $j\geq1$ are nonconstant polynomials of $y$ (of degree
$2j$) so that $\limfi y\left|\frac{q_{j-1}\left(1\right)}{q_{j-1}\left(y\right)}\cdot\frac{q_{j-1}\left(1\right)}{q_{j}\left(y\right)}\right|=0$.
Fixing $\varepsilon>0$ and $N>0$, we can find $M=M_{\varepsilon,N}$
large enough such that $\left|\frac{q_{j-1}\left(1\right)}{q_{j-1}\left(y\right)}\cdot\frac{q_{j-1}\left(1\right)}{q_{j}\left(y\right)}\right|<\frac{\varepsilon}{N}$
for all $y>M$ and $1\leq j<N$. In particular, for any such $y>M$
we have that 
\[
\left|\frac{p_{n}\left(m\right)}{q_{n}\left(m\right)}\right|\leq\sum_{j=1}^{\infty}\left|\frac{q_{j-1}\left(1\right)}{q_{j-1}\left(m\right)}\cdot\frac{q_{j-1}\left(1\right)}{q_{j}\left(m\right)}\right|\leq\frac{\varepsilon}{N}N+\sum_{j=N}^{\infty}\left|\frac{q_{j-1}\left(1\right)}{q_{j-1}\left(m\right)}\cdot\frac{q_{j-1}\left(1\right)}{q_{j}\left(m\right)}\right|\leq\varepsilon+\sum_{j=N}^{\infty}\frac{1}{j^{3}}.
\]
Since $\sum_{1}^{\infty}\frac{1}{j^{3}}<\infty$ converges, we can
find $N$ large enough so that $\sum_{N}^{\infty}\frac{1}{j^{3}}\leq\varepsilon$
also, so together we get that for all $y$ big enough (independent
of $n$) we have $\left|\frac{p_{n}\left(y\right)}{q_{n}\left(y\right)}\right|\leq2\varepsilon$
which is what we wanted to prove.
\end{proof}
Finally, we combine all of the results to show that $\zeta\left(3\right)$
is irrational.
\begin{thm}
\label{thm:zeta-3-irrational}The number $\zeta\left(3\right)$ is
irrational.
\end{thm}

\begin{proof}
Consider the diagonal direction on the $\zeta\left(3\right)$ matrix
field where $m=n+1$. From \thmref{all-directions-converge} we have
that 
\[
\limfi n\frac{P\left(n,n+1\right)}{Q\left(n,n+1\right)}=\zeta\left(3\right).
\]

\textbf{\uline{The main idea:}}

Let us denote $Q_{n}=Q\left(n,n+1\right),\;P_{n}=P\left(n,n+1\right)$
and $\tilde{Q}_{n}=\frac{Q_{n}}{gcd\left(Q_{n},P_{n}\right)}$, $\tilde{P}_{n}=\frac{P_{n}}{gcd\left(Q_{n},P_{n}\right)}$
so that $\limfi n\frac{P_{n}}{Q_{n}}=\limfi n\frac{\tilde{P}_{n}}{\tilde{Q}_{n}}=\zeta\left(3\right)$.
Recall that in \thmref{improved-rationality-test} we showed that
if $\frac{P_{n}}{Q_{n}}$ is not eventually constant and $\left|Q_{n}\zeta\left(3\right)-P_{n}\right|=o\left(gcd\left(P_{n},Q_{n}\right)\right)$,
then $\zeta\left(3\right)$ is irrational. We will begin by showing
that the diagonal is also a polynomial continued fraction in disguise.
Once we know that we can use it to approximate the errors and denominators.

Setting $\lambda_{+}=\left(1+\sqrt{2}\right)^{4}$, we will first
show that given any $\varepsilon>0$ we have:
\[
\left.\begin{array}{c}
\left|\zeta\left(3\right)-\frac{P_{n}}{Q_{n}}\right|=O\left(\frac{1}{\left(\lambda_{+}-\varepsilon\right)^{2n}}\right)\\
Q_{n}=O\left(\left(n!\right)^{6}\left(\lambda_{+}+\varepsilon\right)^{n}\right)
\end{array}\right\} \;\Rightarrow\;\left|Q_{n}\zeta\left(3\right)-P_{n}\right|=O\left(\frac{\left(n!\right)^{6}\left(\lambda_{+}+\varepsilon\right)^{n}}{\left(\lambda_{+}-\varepsilon\right)^{2n}}\right)\sim\frac{\left(n!\right)^{6}}{\lambda_{+}^{n}}.
\]
We then use the well known result that $\lcm\left[n\right]=O\left(\left(e+\varepsilon\right)^{n}\right)$
(it follows from the prime number theorem, see \cite{apostol_introduction_1998})
together with \corref{factorial-reduction} to get that 
\[
\frac{\left(n!\right)^{6}}{e^{3n}}\sim\left(\frac{n!}{\lcm\left[n\right]}\cdot n!\right)^{3}\mid\gcd\left(P_{n},Q_{n}\right).
\]
Finally, since $20.08\sim e^{3}<\lambda_{+}=\left(1+\sqrt{2}\right)^{4}\sim33.97$,
we could choose $\varepsilon>0$ small enough to use the irrationality
test and show that $\zeta\left(3\right)$ is irrational.

\medskip{}

\newpage{}

\textbf{\uline{Step 1: Find recursion relation for \mbox{$Q_{n}$}:}}

With this main idea, we are left to find the growth rate of $Q_{n}$
and how fast $\left|\zeta\left(3\right)-\frac{P_{n}}{Q_{n}}\right|$
goes to zero. 

Using the coboundary condition on the matrix field, we get that
\begin{align*}
\left(\begin{smallmatrix}P_{n}\\
Q_{n}
\end{smallmatrix}\right)=\left(\begin{smallmatrix}P\left(n,n+1\right)\\
Q\left(n,n+1\right)
\end{smallmatrix}\right) & =\left[\prod_{k=1}^{n}M_{Y}\left(0,k\right)\right]\left[\prod_{k=0}^{n-1}M_{X}\left(k,n+1\right)\right]e_{2}\\
 & =\left[\prod_{k=1}^{n}M_{X}\left(k-1,k\right)M_{Y}\left(k,k\right)\right]e_{2},
\end{align*}
where 
\begin{align*}
M_{X}\left(k-1,k\right)M_{Y}\left(k,k\right) & =\left(\begin{smallmatrix}0 & 1\\
b\left(k\right) & f\left(k-1,k\right)-\bar{f}\left(k,k\right)
\end{smallmatrix}\right)\left(\begin{smallmatrix}\bar{f}\left(k,k\right) & 1\\
b\left(k\right) & f\left(k,k\right)
\end{smallmatrix}\right)\\
 & =\left(\begin{smallmatrix}b\left(k\right) & f\left(k,k\right)\\
f\left(k-1,k\right)b\left(k\right)\; & \left(f\bar{f}\right)\left(k,0\right)+f\left(k,k\right)f\left(k-1,k\right)-\left(f\bar{f}\right)\left(k,k\right)
\end{smallmatrix}\right)\\
 & =\left(\begin{smallmatrix}-k^{6} & 6k^{3}\\
-f\left(k-1,k\right)k^{6}\; & 6k^{3}f\left(k-1,k\right)-k^{6}
\end{smallmatrix}\right)=k^{3}\left(\begin{smallmatrix}-k^{3} & 6\\
-f\left(k-1,k\right)k^{3}\; & 6f\left(k-1,k\right)-k^{3}
\end{smallmatrix}\right).
\end{align*}

Fortunately, the last matrix is also a polynomial continued matrix
in disguise (coboundary equivalent). Indeed, setting $U\left(k\right)=\left(\begin{smallmatrix}0 & 6\\
1 & \left(6f\left(k-1,k\right)-k^{3}\right)\;
\end{smallmatrix}\right)$ we get 
\[
M\left(k\right):=U\left(k\right)^{-1}\left(\begin{smallmatrix}-k^{3} & 6\\
-f\left(k-1,k\right)k^{3}\; & 6f\left(k-1,k\right)-k^{3}
\end{smallmatrix}\right)U\left(k+1\right)=\begin{pmatrix}0 & -k^{6}\\
1 & 6f\left(k,k+1\right)-k^{3}-\left(1+k\right)^{3}
\end{pmatrix},
\]
and therefore 
\begin{align*}
\frac{1}{\left(n!\right)^{3}}\left(\begin{smallmatrix}P_{n}\\
Q_{n}
\end{smallmatrix}\right) & =U\left(1\right)\left[\prod_{k=1}^{n}M\left(k\right)\right]U\left(n+1\right)^{-1}e_{2}=\left(\begin{smallmatrix}0 & 6\\
1 & 5
\end{smallmatrix}\right)\left[\prod_{k=1}^{n}M\left(k\right)\right]e_{1}=\left(\begin{smallmatrix}0 & 6\\
1 & 5
\end{smallmatrix}\right)\left[\prod_{k=1}^{n-1}M\left(k\right)\right]e_{2}.
\end{align*}

Setting as usuall $\begin{pmatrix}Q'_{n}\\
P'_{n}
\end{pmatrix}=\prod_{k=1}^{n-1}M\left(k\right)e_{2}$, we get that $u_{n}:=\frac{Q_{n}}{\left(n!\right)^{3}}=Q'_{n}+5P'_{n}$,
and since both $P_{n}',Q_{n}'$ satisfy the same reccurence, then
so does $u_{n}$, and we get that 

\begin{align*}
u_{n+1} & =u_{n}\left(6f\left(n,n+1\right)-\left(n+1\right)^{3}-n^{3}\right)-u_{n-1}n^{6},
\end{align*}
where $u_{0}=\frac{Q_{0}}{0!^{3}}=1$ and $u_{1}=\frac{Q_{1}}{1!^{3}}=6f\left(0,1\right)-1=5$.
Denote
\begin{align*}
F\left(n\right) & =6f\left(n,n+1\right)-\left(n+1\right)^{3}-n^{3}=34n^{3}+51n^{2}+27n+5,
\end{align*}
so the recurrence can be written as $u_{n+1}=F\left(n\right)u_{n}-n^{6}u_{n-1}$.

It is also interesting to note that $\frac{P_{n}}{\left(n!\right)^{3}}$
satisfies the same recurrence and $\frac{P_{0}}{\left(0!\right)^{3}}=0,\;\frac{P_{1}}{\left(1!\right)^{3}}=6$,
so that
\[
\zeta\left(3\right)=\limfi n\frac{P_{n}}{Q_{n}}=\left(\begin{smallmatrix}0 & 6\\
1 & 5
\end{smallmatrix}\right)\left[\prod_{1}^{n}\left(\begin{smallmatrix}0 & -k^{6}\\
1 & F\left(k\right)
\end{smallmatrix}\right)\right]\left(0\right)=\left(\begin{smallmatrix}0 & 6\\
1 & 5
\end{smallmatrix}\right)\left(\KK_{1}^{\infty}\frac{-k^{6}}{F\left(k\right)}\right),
\]
or alternatively
\[
\frac{6}{\zeta\left(3\right)}-5=\KK_{1}^{\infty}\frac{-k^{6}}{F\left(k\right)}.
\]
\medskip{}

\newpage{}

\textbf{\uline{Step 2: Analyze the recurrence to find the growth
rate of \mbox{$Q_{n}$}:}}

By \corref{factorial-reduction} we know that $\left(n!\right)^{6}\mid Q_{n}$,
so that $v_{n}=\frac{Q_{n}}{\left(n!\right)^{6}}=\frac{u_{n}}{\left(n!\right)^{3}}$
are integers which satisfy
\[
v_{n+1}\left(n+1\right)^{3}=F\left(n\right)v_{n}-n^{3}v_{n-1},
\]
where $v_{0}=\frac{Q_{0}}{0!^{6}}=1$ and $v_{1}=\frac{Q_{1}}{1!^{6}}=5$.
Equivalently, we can write 
\[
v_{n+1}=\frac{F\left(n\right)}{n^{3}}v_{n}-\frac{n^{3}}{\left(1+n\right)^{3}}v_{n-1}.
\]
Taking the limit only for the coefficients, we get the ``limit''
recurrence $v_{n+1}'=34v_{n}'-v_{n-1}'$. This correspons to the quadratic
equation $x^{2}-34x+1=0$ with the roots 
\[
\lambda_{\pm}=\frac{34\pm\sqrt{1156-4}}{2}=\frac{34\pm24\sqrt{2}}{2}=17\pm12\sqrt{2}=\left(1\pm\sqrt{2}\right)^{4},
\]
so a standard computation shows that $\frac{v_{n}'}{v_{n-1}'}\to\lambda_{+}=\left(1+\sqrt{2}\right)^{4}$.
In the original recurrence with the nonconstant coefficients, the
same holds, but needs a bit more explanation. As with the standard
recurrence with constant coefficients, we expect the general solution
to behave like $v_{n}\sim\lambda_{+}^{n}$, though there is a specific
starting position for which $v_{n}\sim\lambda_{-}^{n}$. Since $\lambda_{-}=\left(1-\sqrt{2}\right)^{4}\sim0.03$
, this is highly unlikely to happen, since we deal with integer values.
More sepcifically, the first few elements in $v_{i}$ are $1,5,73,1445,33001,...$
which is an increasing sequence of positive integers, and since $\frac{F\left(n\right)}{n^{3}}\geq11$
for $n\geq3$, it is not hard to show by induction that 
\[
v_{n+1}=\frac{F\left(n\right)}{n^{3}}v_{n}-\frac{n^{3}}{\left(1+n\right)^{3}}v_{n-1}\geq11v_{n}-v_{n-1}\geq10v_{n},
\]
so at least we get that $v_{n}\geq10^{n}$ grows much faster than
the very special case of $\lambda_{-}^{n}$. This is enough to show
that for every $\varepsilon>0$ and for any $n$ large enough, we
have
\[
\left(\lambda_{+}-\varepsilon\right)^{n}\leq v_{n}\leq\left(\lambda_{+}+\varepsilon\right)^{n}.
\]
The rest of the details are standard computations, and we leave it
to the reader.

\medskip{}

\textbf{\uline{Step 3: Analyze the approximation error:}}

The sequence of $\frac{P_{n}}{\left(n!\right)^{3}},u_{n}=\frac{Q_{n}}{\left(n!\right)^{3}}$
are the numerators and denominators of the continued fraction $\KK_{1}^{\infty}\frac{-n^{6}}{F\left(n\right)}$.
Using \claimref{upper-bound} we get that for all $n$ large enough
\[
\left|\zeta\left(3\right)-\frac{P_{n}}{Q_{n}}\right|\leq\sum_{k=n}^{\infty}\frac{\left(k!\right)^{6}}{\left|u_{k}u_{k+1}\right|}=\sum_{k=n}^{\infty}\frac{1}{\left(k+1\right)^{3}\left|v_{k}v_{k+1}\right|}\leq\sum_{n}^{\infty}\frac{1}{\left(\lambda_{+}-\varepsilon\right)^{2k+1}}=O\left(\frac{1}{\left(\lambda_{+}-\varepsilon\right)^{2n}}\right).
\]

These are the growth rate for $Q_{n}=\left(n!\right)^{6}v_{n}$ and
the error for $\left|\zeta\left(3\right)-\frac{P_{n}}{Q_{n}}\right|$
that we needed in the beginning, thus completing the proof.
\end{proof}

\newpage{}

\section{\label{sec:On-future-fractions}On future fractions}

The main goal of this paper was to introduce this new mathematical
object of conservative matrix field, and as an application use it
to reprove Apery's result about the irrationality of $\zeta\left(3\right)$.
As can be seen in \secref{The-z3-case}, the final proof as it is
right now is very specific to the matrix field of $\zeta\left(3\right)$,
which has several nice properties, and doesn't hold for other examples
of matrix fields. However it might be possible that some of the results
hold in a more general setting.

While this irrationality result is already interesting by itself,
the conservative matrix field object also seems to have many interesting
properties. Among others, it is a natural generalization of quadratic
equations, and it involves a bit of noncommutative cohomology theory
in the form of cocycles and coboundaries.

So far, the conservative matrix fields that we managed to find where
$f,\bar{f}$ are polynomials of degree 4 or more seem to always be
degenerate, namely $a\left(x,y\right)=f\left(x,y\right)-\bar{f}\left(x+1,y\right)$
doesn't depend on $y$. This might be related to the fact that we
work over $2\times2$ matrix, which might bound the possible matrix
fields. Whether this is the case or not, this leads to several possible
interesting generalizations of this theory, which are standard in
the theory of continued fractions and in number theory in general.
\begin{enumerate}
\item While many of the results mentioned in this paper are true for general
continued fractions over $\CC$ (and even other fields), the irrationality
of $\zeta\left(3\right)$ relied heavily on the fact that the defining
polynomials $f,\bar{f}$ were in $\ZZ\left[x,y\right]$. This leads
naturally to the question of what happens when we use other integer
rings in algebraic extensions, e.g. $\ZZ\left[i\right]$ or $\ZZ\left[\sqrt{2}\right]$.
Both in the $\zeta\left(2\right)$ and $\zeta\left(3\right)$ matrix
fields case we can find in the background algebraic numbers of degree
$2$ (namely $1+i$ and $\zeta_{3}=e^{\frac{2\pi}{3}i}$ respectively).
This type of field extension, with the right definition of generalized
continued fraction might add many more interesting examples.
\item In the proof of the irrationality of $\zeta\left(3\right)$ we had
two main results that we needed to show. One was to find the error
rate and how fast it converges to zero, and the second was to find
$gcd\left(P_{n},Q_{n}\right)$ and hope that it grows to infinity
fast enough. As it is usually the case in number theoretic problems,
the first result lives in the standard Euclidean geometry, where we
needed to show that some sequence goes to zero in the $\left|\cdot\right|_{\infty}$
norm, and the second result can be seen as showing that the $p$-adic
norms of $\left|gcd\left(P_{n},Q_{n}\right)\right|_{p}$ all go to
zero as well. This suggests a more general approach where the matrix
field lives over the Adeles, and the convergence in the real and $p$-adic
places together prove irrationality.
\item All the results in this paper were for $2\times2$ matrices, and a
natural generalization would be by going to a higher dimension matrices.
There are many suggestions for what should be the generalization of
continued fractions to higher dimensions, however probably one of
the best approaches is to change the language all together from continued
fractions to lattices in $\RR^{d}$. The subject of lattices is well
studied in the literature with many connections to other subjects.
With this approach, the question should be what is the right way to
formulate the results about general continued fraction as results
on lattices, and what can we say in higher dimension.
\end{enumerate}
These three types of generalization of changing the field, the norm,
or the dimension, can also be combined. Of course, there are more
tools available already in ``standard'' $2\times2$ matrices over
the integers to study polynomial continued fraction. However, it seems
that the conservative matrix field holds some interesting structure
which might reveal itself to be very useful not only to prove results
about continued fractions, but to other subjects as well.


\newpage{}

\part{Appendix}


\appendix


\section{\label{app:integer-values}The algebra of integer valued polynomial}

One of the tools we need along the way were rational polynomial $f\in\QQ\left[x\right]$
where $f\left(\ZZ\right)\subseteq\ZZ$, which are called \textbf{integer
valued polynomials}. Of course, if we can write $f\left(x\right)\in\ZZ\left[x\right]$,
then $f$ is integer valued, but the other direction is not true.
For example, every integer value of $g\left(x\right)=x\left(x+1\right)$
is divisible by $2$, so that $f\left(x\right)=\frac{x\left(x+1\right)}{2}$
is an integer valued polynomial which is not in $\ZZ\left[x\right]$.
There are many more such polynomials, which we write as follows:
\begin{defn}
For $n\in\NN$ we define the polynomial $\binom{x}{n}=\prod_{0}^{n-1}\frac{\left(x-i\right)}{i+1}=\frac{x\cdot\left(x-1\right)\cdots\left(x-n+1\right)}{n!}$.
This is a polynomial of degree $n$ in $\QQ\left[x\right]$, so that
$\left\{ \binom{x}{n}\right\} _{0}^{\infty}$ is a $\QQ$-basis for
$\QQ\left[x\right]$. In particular, for a nonnegative integers $x$,
we simply get the binomials.
\end{defn}

The class of integer valued polynomials was fully described by P�lya
in \cite{polya_uber_1915}, and it was shown to contain exactly the
integer combinations of the $\binom{x}{n}$ above. For the ease of
the reader, we add the proof for this result here.
\begin{lem}
For every integer $m$, we have that $\binom{m}{n}\in\ZZ$. 
\end{lem}

\begin{proof}
We first note that Pascal's identity holds for the polynomial. Indeed,
taking 
\[
p\left(x\right)=\binom{x}{n}-\binom{x-1}{n-1}-\binom{x-1}{n},
\]
we get a finite degree polynomial where $p\left(m\right)=0$ for all
$m\geq n$, so that $p\left(x\right)\equiv0$ as a polynomial.

In order to prove that $\binom{m}{n}\in\ZZ$ for all $n,m\in\ZZ$,
we use induction, first on $n$ and then on $m$.

For $n=0$ we simply get that $\binom{m}{0}=1$ and for $n=1$ we
get $\binom{m}{1}=m$ for all $m$, so we are done.

Assume now the claim for $n-1$ and we prove for $n\geq2$. By Pascal's
identity we have $\binom{m}{n}=\binom{m-1}{n-1}+\binom{m-1}{n}$ and
since $\binom{m-1}{n-1}$ is always an integer by the induction hypothesis,
then $\binom{m}{n}$ is an integer if and only if $\binom{m-1}{n}$
is an integer. In other words, we only need to show this for a single
$m$. Taking $m=0$ we get $\binom{0}{n}=0$ and we are done.
\end{proof}
In the following, when we write $d\mid q$ for $d\in\ZZ$ and $q\in\QQ$,
we mean that $q$ has to be an integer and is divisible by $d$.
\begin{lem}
\label{lem:binomial-polynomials}Given a general polynomial $f\left(x\right)=\sum_{0}^{d}a_{n}\binom{x}{n}\in\QQ\left[x\right]$
and an integer $k$ the following are equivalent:
\begin{enumerate}
\item For all $0\leq n\leq d$ we have $k\mid a_{n}$ ,
\item For all $m\in\ZZ$ we have $k\mid f\left(m\right)$ ,
\item For $m=0,1,...,d$ , we have $k\mid f\left(m\right)$ ,
\item There exists $m_{0}$ such that $k\mid f\left(m_{0}+m\right)$ for
$m=0,1,...,d$, and
\end{enumerate}
\end{lem}

\begin{proof}
Note first that considering the polynomial $\frac{f\left(x\right)}{k}$
instead, it is enough to prove the lemma for $k=1$. Namely, we just
need to show that the coefficients\textbackslash evaluation are integers.
\begin{itemize}
\item $\left(1\right)\Rightarrow\left(2\right)$: follows from the fact
that $\binom{m}{n}$ are integers for all $m$.
\item $\left(2\right)\Rightarrow\left(3\right)$: is trivial.
\item $\left(3\right)\Rightarrow\left(1\right)$: Since $\binom{n}{n}=1$
and $\binom{m}{n}=0$ when $0\leq m<n$, it follows that for $0\leq m\leq d$
we have 
\[
f\left(m\right)=a_{m}+\sum_{n=0}^{m-1}a_{n}\binom{m}{n},
\]
which we can also write as 
\[
a_{m}=f\left(m\right)-\sum_{n=0}^{m-1}a_{n}\binom{m}{n}.
\]
So if $a_{n}\in\ZZ$ for $0\leq n<m$, then since $f\left(m\right)\in\ZZ$
by assumption and $\binom{m}{n}\in\ZZ$, we conclude that $a_{m}\in\ZZ$.
Thus, by induction we get that $a_{n}\in\ZZ$ for all $0\leq n\leq d$,
namely we get $\left(1\right)$.
\item $\left(2\right)\Rightarrow\left(4\right)$: is trivial.
\item $\left(4\right)\Rightarrow\left(2\right)$: If $f\left(m_{0}+m\right)\in\ZZ$
for $m=0,...,d$, then setting $g\left(m\right)=f\left(m_{0}+m\right)$
we see that $g\left(m\right)\in\ZZ$ for $m=0,...,d$. By the $\left(3\right)\Rightarrow\left(2\right)$
direction for the degree $d$ polynomial $g$ we get that $g\left(m\right)=f\left(m_{0}+m\right)\in\ZZ$
for all $m$, which is exactly condition $\left(2\right)$ for the
polynomial $f$ and we are done.
\end{itemize}
\end{proof}

\newpage{}

\bibliographystyle{plain}
\bibliography{apery}

\end{document}
