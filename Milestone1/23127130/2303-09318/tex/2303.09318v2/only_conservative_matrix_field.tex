\global\long\def\call{\mathcal{L}}%
\global\long\def\nn{\mathcal{N}}%
\global\long\def\ff{\mathcal{F}}%
\global\long\def\aa{\mathcal{A}}%
\global\long\def\RR{\mathbb{R}}%
\global\long\def\EE{\mathbb{E}}%
\global\long\def\CC{\mathbb{C}}%
\global\long\def\QQ{\mathbb{Q}}%
\global\long\def\ZZ{\mathbb{Z}}%
\global\long\def\NN{\mathbb{N}}%
\global\long\def\KK{\mathbb{K}}%
\global\long\def\SL{\mathrm{SL}}%
\global\long\def\GL{\mathrm{GL}}%
\global\long\def\ds{\mathrm{ds}}%
\global\long\def\dnu{\mathrm{d\nu}}%
\global\long\def\dmu{\mathrm{d\mu}}%
\global\long\def\dt{\mathrm{dt}}%
\global\long\def\dw{\mathrm{dw}}%
\global\long\def\dx{\mathrm{dx}}%
\global\long\def\dy{\mathrm{dy}}%
\global\long\def\norm#1{\left\Vert #1\right\Vert }%
\global\long\def\limfi#1{{\displaystyle \lim_{#1\to\infty}}}%
\global\long\def\arrfi#1{\overset{#1\to\infty}{\longrightarrow}}%
\global\long\def\flr#1{\left\lfloor #1\right\rfloor }%
\global\long\def\lcm{\mathrm{lcm}}%


\section{\label{sec:Definition-properties-CMF}The conservative matrix field
- definition and properties}

Up until now we mainly looked at a single continued fractions $\KK_{1}^{\infty}\frac{b_{i}}{a_{i}}$,
and in particular where $a_{i}=a\left(i\right),b_{i}=b\left(i\right)$
with $a,b\in\ZZ\left[x\right]$. In this section we define the\textbf{
conservative matrix field}, which is a collection of such continued
fractions with interesting connections between them. 
\begin{defn}
A pair of matrices $M_{X}\left(x,y\right),M_{Y}\left(x,y\right)$
is called a \textbf{conservative matrix field} (or just \textbf{matrix
field} for simplicity), if
\begin{enumerate}
\item The entries of $M_{X}\left(x,y\right),M_{Y}\left(x,y\right)$ are
polynomial in $x,y$ ,
\item The matrices satisfy the conservativeness relation
\[
M_{X}\left(x,y\right)M_{Y}\left(x+1,y\right)=M_{Y}\left(x,y\right)M_{X}\left(x,y+1\right)\;\forall x,y,
\]
or in commutative diagram form:
\[
\xyR{.5pc}\xyC{0.5pc}\xymatrix{\left(x,y+1\right)\ar[rr]^{M_{X}\left(x,y+1\right)} &  & \left(x+1,y+1\right)\\
 & {\Huge\leftturn}\\
\left(x,y\right)\ar[rr]_{M_{X}\left(x,y\right)}\ar[uu]^{M_{Y}\left(x,y\right)} &  & \left(x+1,y\right)\ar[uu]_{M_{Y}\left(x+1,y\right)}
}
\xyR{1pc}\xyC{1pc}
\]
\end{enumerate}
\end{defn}

The name \emph{conservative matrix field} arose from the resemblence
to \emph{conservative vector field}. When visualizing the corners
of the commutative diagram as points in the plane, the notion is that
traveling along the bottom and then right edge or the left and then
top edge yields the same product, which essentially is the behaviour
of standard conservative vector fields (and indeed, both are 1-cocycles
with the appropriate groups). Retaining this intuitive connection,
led to the adoption of the name conservative matrix field. 

One of the main differences, is that moving left or down along the
matrix field amounts to multiplying by $M_{X}\left(x,y\right)^{-1},M_{Y}\left(x,y\right)^{-1}$
respectively, which aren't necessarily invertible. However, they will
be invertible in most cases, and with that in mind, we define:
\begin{defn}
Given a matrix field $M_{X},M_{Y}$, and initial position $\left(n_{0},m_{0}\right)\in\ZZ^{2}$,
we define the potential $S\left(n,m\right)=S_{n_{0},m_{0}}\left(n,m\right)$
matrix for $n\geq n_{0},m\geq m_{0}$ by

\[
S\left(n,m\right)=\prod_{n_{0}}^{n-1}M_{X}\left(k,m_{0}\right)\cdot\prod_{m_{0}}^{m-1}M_{Y}\left(n,k\right).
\]
Note that the potential is independent of the choice of path from
$\left(n_{0},m_{0}\right)$ to $\left(n,m\right)$.
\end{defn}

\medskip{}

Given a pathes to infinity $\left(n_{i},m_{i}\right)$, it is natural
to ask whether $\lim S\left(n_{i},m_{i}\right)\left(0\right)$ converge,
are how does changing the path affect the limit. In particular, if
$\alpha=\lim S\left(n_{i},m_{i}\right)\left(0\right)$ along some
path, we can look at the rest of the matrix field for other properties
of $\alpha$. For example, in \secref{The-z3-case} we will construct
such a matrix field for $\zeta\left(3\right)$, starting from its
Euler continued fraction (from \exaref{Euler-family}) on the $Y=0$
line, then see that the limits on the $Y=m$ lines converge to $\sum_{m}^{\infty}\frac{1}{k^{3}}$,
and the diagonal line $X=Y$ can be used to define another continued
fraction presentation where the convergents converge to $\zeta\left(3\right)$
is fast enough to prove its irrationality.

With this intuition in mind, we start with a construction for specific
family of matrix fields with many interesting properties in \subsecref{matrix-field-construction},
where in particular each row is a polynomial continued fraction (and
therefore coboundary equivalent via the $M_{Y}$ matrices). We then
``twist'' it a little bit, to get a matrix field which is easier
to work with. Then in \subsecref{The-dual-matrix-field} we find out
how every such matrix field comes with its dual, which is in a sense
a reflection through the $x=y$ line. Once we have this dual matrix
field, we study the numerators and denominators of the continued fractions
in that matrix field, and in particular find their greatest common
divisors. Finally we show how to put everything together in \secref{The-z3-case}
to show that $\zeta\left(3\right)$ is irrational.

\newpage{}

\subsection{\label{subsec:matrix-field-construction}The construction}

We begin with an interesting construction of a family of matrix fields.
\begin{defn}
\label{def:conjugate}We say that two polynomial $f,\bar{f}\in\CC\left[x,y\right]$
are \textbf{conjugate}, if they satisfy:
\begin{enumerate}
\item \textbf{\uline{Linear condition}}: 
\[
f\left(x+1,y-1\right)-\bar{f}\left(x,y-1\right)=f\left(x,y\right)-\bar{f}\left(x+1,y\right).
\]
When this condition holds, we denote the expression above by $a_{f,\bar{f}}\left(x,y\right):=a\left(x,y\right)$.
\item \textbf{\uline{Quadratic condition}}: 
\[
\left(f\bar{f}\right)\left(x,y\right)-\left(f\bar{f}\right)\left(0,y\right)=\left(f\bar{f}\right)\left(x,0\right)-\left(f\bar{f}\right)\left(0,0\right).
\]
In other words, there are no mixed monomial $x^{n}y^{m}$ where $n,m>0$
in $\left(f\bar{f}\right)\left(x,y\right)$.\\
When this condition holds, we denote the expression above by $b_{f,\bar{f}}\left(x\right):=b\left(x\right)$,
which only depends on $x$. We will usually also have that $\left(f\bar{f}\right)\left(0,0\right)=0$,
in which case $b\left(x\right)=\left(f\bar{f}\right)\left(x,0\right)$.
With this in mind, we will also denote $b_{X}\left(x\right):=b\left(x\right)$
and $b_{Y}\left(y\right):=\left(f\bar{f}\right)\left(0,y\right)$
\end{enumerate}
Given two such conjugate polynomials, we define 
\begin{align*}
M_{X}^{cf}\left(x,y\right) & =\left(\begin{smallmatrix}0 & b\left(x\right)\\
1 & a\left(x,y\right)
\end{smallmatrix}\right)\\
M_{Y}^{cf}\left(x,y\right) & =\left(\begin{smallmatrix}\bar{f}\left(x,y\right) & b\left(x\right)\\
1 & f\left(x,y\right)
\end{smallmatrix}\right).
\end{align*}
The $cf$ indicates that $M_{X}^{cf}$ is in a continued fraction
form. We will shortly change it a little bit and remove these $cf$.
\end{defn}

\medskip{}

\begin{rem}
\label{rem:ff(0)=00003D0}If $\left(f\bar{f}\right)\left(0,0\right)=0$,
then the $y=1$ is the continued fraction with $b_{i}=\left(f\bar{f}\right)\left(i,0\right)$
and $a_{i}=f\left(i+1,0\right)-\bar{f}\left(i,0\right)$, which is
in the trivial Euler family defined in \exaref{Euler-family}, where
$h_{1}\left(x\right)=f\left(x,0\right)$ and $h_{2}\left(x\right)=-\bar{f}\left(x,0\right)$.
Using the second presentation of $a\left(x,y\right)$, the $y=1$
line is a continued fraction with $b_{i}=\left(f\bar{f}\right)\left(i,0\right)$
and $a_{i}=f\left(i+1,0\right)-\bar{f}\left(i,0\right)$, which is
similarly in the trivial Euler family.\\
Also, recall that finding an ``Euler'' presentation for a continued
fraction is in a sense a generalization of solving a quadratic equation
$x^{2}+ax+b=0$, where the roots $\lambda_{1},\lambda_{2}$ satisfy
$\lambda_{1}\lambda_{2}=b$ and $-\left(\lambda_{1}+\lambda_{2}\right)=a$.
The structure defined above should be considered as an even further
generalization of this concept. Indeed, starting with $b\left(x\right)$
and $a\left(x,y\right)$, we look for $f,\bar{f}$ such that 
\begin{align*}
b\left(x\right) & =f\left(x,0\right)\bar{f}\left(x,0\right)\\
a\left(x,y\right) & =f\left(x,y\right)-\bar{f}\left(x+1,y\right).
\end{align*}

With this point of view, the term ``conjugates'' should be more
natural, since in a way $f,\bar{f}$ are roots of a quadratic equation
(though with polynomials coefficients).
\end{rem}

\newpage{}
\begin{example}[\textbf{The $\zeta\left(3\right)$ matrix field}]
\label{exa:zeta-3-matrix-field}The main example that we should have in mind is a matrix field for
$\zeta\left(3\right)$ defined by
\begin{align*}
f\left(x,y\right) & =x^{3}+2x^{2}y+2xy^{2}+y^{3}=\frac{y^{3}-x^{3}}{y-x}\left(y+x\right)\\
\bar{f}\left(x,y\right) & =-x^{3}+2x^{2}y-2xy^{2}+y^{3}=\frac{y^{3}+x^{3}}{y+x}\left(y-x\right)=f\left(-x,y\right)\\
\left(f\bar{f}\right)\left(x,y\right) & =y^{6}-x^{6}\\
b\left(x\right) & =\left(f\bar{f}\right)\left(x,0\right)=-x^{6}\\
a\left(x,y\right) & =x^{3}+\left(1+x\right)^{3}+2y\left(y-1\right)\left(2x+1\right).
\end{align*}
In particular, for $y=0,1$ we have the polynomial continued fraction
$b\left(n\right)=-n^{6}$ and $a\left(n,0\right)=n^{3}+\left(1+n\right)^{3}$,
which is exactly the Euler continued fraction which converges to $\frac{1}{\zeta\left(3\right)}-1$,
as we saw in \exaref{Euler-family}. We will see in \secref{The-z3-case}
that for any fixed integer $y=m\geq1$, the continued fraction with
$b_{n}=b\left(n\right)$ and $a_{n}=a\left(n,m\right)$ converges
to $\frac{1}{\sum_{m}^{\infty}\frac{1}{k^{3}}}-1$.

The polynomial matrices in this matrix field are 
\begin{align*}
M_{X}^{cf}\left(x,y\right) & =\left(\begin{smallmatrix}0 & b\left(x\right)\\
1 & a\left(x,y\right)
\end{smallmatrix}\right)=\left(\begin{smallmatrix}0 & -x^{6}\\
1 & x^{3}+\left(1+x\right)^{3}+2y\left(y-1\right)\left(2x+1\right)
\end{smallmatrix}\right)\\
M_{Y}^{cf}\left(x,y\right) & =\left(\begin{smallmatrix}\bar{f}\left(x,y\right) & b\left(x\right)\\
1 & f\left(x,y\right)
\end{smallmatrix}\right)=\left(\begin{smallmatrix}\frac{y^{3}+x^{3}}{y+x}\left(y-x\right) & -x^{6}\\
1 & \frac{y^{3}-x^{3}}{y-x}\left(y+x\right)
\end{smallmatrix}\right)
\end{align*}
and the first few of them are 
\begin{align*}
\xyR{1.5pc}\xyC{1.5pc}\xymatrix{\left(*\right) &  &  & \left(*\right) &  &  & \left(*\right) &  & \left(*\right)\\
\\
*+[F]{\left(1,3\right)}\ar[rrr]_{\left(\begin{smallmatrix}0 & -1\\
1 & 45
\end{smallmatrix}\right)}\ar[uu]^{\left(\begin{smallmatrix}14 & -1\\
1 & 52
\end{smallmatrix}\right)} &  &  & *+[F]{\left(2,3\right)}\ar[rrr]_{\left(\begin{smallmatrix}0 & -2^{3}\\
1 & 95
\end{smallmatrix}\right)}\ar[uu]^{\left(\begin{smallmatrix}7 & -1\\
1 & 95
\end{smallmatrix}\right)} &  &  & *+[F]{\left(3,3\right)}\ar[rr]_{\left(\begin{smallmatrix}0 & -3^{3}\\
1 & 175
\end{smallmatrix}\right)}\ar[uu]^{\left(\begin{smallmatrix}0 & -1\\
1 & 168
\end{smallmatrix}\right)} &  & \left(*\right)\\
\\
*+[F]{\left(1,2\right)}\ar[rrr]_{\left(\begin{smallmatrix}0 & -1\\
1 & 21
\end{smallmatrix}\right)}\ar[uu]^{\left(\begin{smallmatrix}3 & -1\\
1 & 21
\end{smallmatrix}\right)} &  &  & *+[F]{\left(2,2\right)}\ar[rrr]_{\left(\begin{smallmatrix}0 & -2^{3}\\
1 & 55
\end{smallmatrix}\right)}\ar[uu]^{\left(\begin{smallmatrix}0 & -1\\
1 & 48
\end{smallmatrix}\right)} &  &  & *+[F]{\left(3,2\right)}\ar[rr]_{\left(\begin{smallmatrix}0 & -3^{3}\\
1 & 119
\end{smallmatrix}\right)}\ar[uu]^{\left(\begin{smallmatrix}-7 & -1\\
1 & 95
\end{smallmatrix}\right)} &  & \left(*\right)\\
\\
*+[F]{\left(1,1\right)}\ar[rrr]_{\left(\begin{smallmatrix}0 & -1\\
1 & 9
\end{smallmatrix}\right)}\ar[uu]^{\left(\begin{smallmatrix}0 & -1\\
1 & 6
\end{smallmatrix}\right)} &  &  & *+[F]{\left(2,1\right)}\ar[rrr]_{\left(\begin{smallmatrix}0 & -2^{3}\\
1 & 35
\end{smallmatrix}\right)}\ar[uu]^{\left(\begin{smallmatrix}-3 & -1\\
1 & 21
\end{smallmatrix}\right)} &  &  & *+[F]{\left(3,1\right)}\ar[rr]_{\left(\begin{smallmatrix}0 & -3^{3}\\
1 & 91
\end{smallmatrix}\right)}\ar[uu]^{\left(\begin{smallmatrix}-14 & -1\\
1 & 52
\end{smallmatrix}\right)} &  & \left(*\right)
}
\end{align*}
\end{example}

\newpage{}

We continue to show that this general construction produces a conservative
matrix field.
\begin{thm}
\label{thm:matrix-field-structure}Given polynomials $f,\bar{f},a,b$
as in \defref{conjugate} and the matrices
\begin{align*}
M_{X}^{cf}\left(x,y\right) & =\left(\begin{smallmatrix}0 & b\left(x\right)\\
1 & a\left(x,y\right)
\end{smallmatrix}\right)\\
M_{Y}^{cf}\left(x,y\right) & =\left(\begin{smallmatrix}\bar{f}\left(x,y\right) & b\left(x\right)\\
1 & f\left(x,y\right)
\end{smallmatrix}\right),
\end{align*}
then the following hold:
\begin{enumerate}
\item The matrices satisfy the coboundary equivalence condition
\[
M_{X}^{cf}\left(x,y\right)M_{Y}^{cf}\left(x+1,y\right)=M_{Y}^{cf}\left(x,y\right)M_{X}^{cf}\left(x,y+1\right)\;\forall x,y.
\]
\item The determinants of $M_{X}^{cf}\left(x,y\right),M_{Y}^{cf}\left(x,y\right)$
are only functions of $x,y$ respectively, and more specifically:
\begin{align*}
\det\left(M_{X}^{cf}\left(x,y\right)\right) & =-b\left(x\right)=-b_{X}\left(x\right)\\
\det\left(M_{Y}^{cf}\left(x,y\right)\right) & =\left(f\cdot\bar{f}\right)\left(0,y\right)=b_{Y}\left(y\right).
\end{align*}
\end{enumerate}
\end{thm}

\begin{proof}
\begin{enumerate}
\item From the assumption on our functions we know that for all $x,y$ we
have
\begin{align}
a\left(x,y\right) & =f\left(x,y\right)-\bar{f}\left(x+1,y\right)\\
a\left(x,y+1\right) & =f\left(x+1,y\right)-\bar{f}\left(x,y\right)\\
b\left(x+1\right)-b\left(x\right) & =f\left(x,y\right)a\left(x,y+1\right)-a\left(x,y\right)f\left(x+1,y\right)
\end{align}
Using conditions (1) and (2) we get that
\begin{align*}
M_{X}^{cf}\left(x,y\right)M_{Y}^{cf}\left(x+1,y\right) & =\left(\begin{smallmatrix}0 & b\left(x\right)\\
1 & a\left(x,y\right)
\end{smallmatrix}\right)\left(\begin{smallmatrix}\bar{f}\left(x+1,y\right) & b\left(x+1\right)\\
1 & f\left(x+1,y\right)
\end{smallmatrix}\right)\overset{\left(1\right)}{=}\left(\begin{smallmatrix}b\left(x\right) & b\left(x\right)f\left(x+1,y\right)\\
f\left(x,y\right) & b\left(x+1\right)+a\left(x,y\right)f\left(x+1,y\right)
\end{smallmatrix}\right)\\
M_{Y}^{cf}\left(x,y\right)M_{X}^{cf}\left(x,y+1\right) & =\left(\begin{smallmatrix}\bar{f}\left(x,y\right) & b\left(x\right)\\
1 & f\left(x,y\right)
\end{smallmatrix}\right)\left(\begin{smallmatrix}0 & b\left(x\right)\\
1 & a\left(x,y+1\right)
\end{smallmatrix}\right)\overset{\left(2\right)}{=}\left(\begin{smallmatrix}b\left(x\right) & b\left(x\right)f\left(x+1,y\right)\\
f\left(x,y\right) & b\left(x\right)+f\left(x,y\right)a\left(x,y+1\right)
\end{smallmatrix}\right)
\end{align*}
The two matrices are the same using (3) from above.
\item Simple computation.
\end{enumerate}
\end{proof}
\begin{example}
\label{exa:vector-field-examples}There are many examples of conservative
matrix fields, and we give some of them below.

For each pair $f,\bar{f}$, we also add the $b\left(x\right),a\left(x,y\right)$
appearing as the continued fraction on the horizontal lines. In particular,
as we saw in \remref{ff(0)=00003D0}, when $\left(f\bar{f}\right)\left(0,0\right)=0$,
the $y=1$ line is in the Euler Family from \exaref{Euler-family},
namely $b\left(n\right)=-h_{1}\left(n\right)\times h_{2}\left(n\right)$
and $a\left(n\right)=h_{1}\left(n\right)+h_{2}\left(n+1\right)$.
In these cases we can convert it to an infinite sum and hopefully
use it to compute the value of the continued fraction, which we add
in the examples below (up to a Mobius map). Further more, in many
cases we think of $\bar{f}$ as an image under some nice linear map
$g\mapsto\bar{g}$ of $f$, and when this is the case, we will give
this linear map instead of $\bar{f}$.
\begin{enumerate}
\item When both $f,\bar{f}$ are linear themselves, solving the linear and
quadratic conditions in \defref{conjugate} is elementary (which we
leave as exercise). There are two families of solutions
\begin{align*}
f\left(x,y\right) & =A\left(x+y\right)+C\\
\bar{f}\left(x,y\right) & =\bar{A}\left(x-y\right)+\bar{C}
\end{align*}
or 
\begin{align*}
f\left(x,y\right) & =Ax+By+C\\
\bar{f}\left(x,y\right) & =-Ax+By+\bar{C}
\end{align*}
where $A,B,C,\bar{A},\bar{C}$ above are the parameters of the families.
\begin{enumerate}
\item Taking $f\left(x,y\right)=x+y$ and $\bar{f}\left(x,y\right)=x-y$,
we get $b\left(x\right)=x^{2}$ and $a\left(x,y\right)=2y-1$. In
$y=1$ we get the continued fraction
\[
\KK_{1}^{\infty}\frac{n^{2}}{1}=\KK_{1}^{\infty}\frac{-\left(-n\right)\times n}{\left(-n\right)+\left(n+1\right)}=\frac{1}{\sum_{k=0}^{\infty}\prod_{i=1}^{k}\frac{-i}{i+1}}-1=\frac{1}{\sum_{k=0}^{\infty}\frac{\left(-1\right)^{k}}{k+1}}-1=\frac{1-\ln\left(2\right)}{\ln\left(2\right)}.
\]
Taking $\bar{f}\left(x,y\right)=y-x$ instead, we get $b\left(x\right)=-x^{2}$
and $a\left(x,y\right)=2x+1$. Since $a$ is independent of $y$,
all the horizontal lines in the matrix field are the same, so in a
sense it is degenerate. Moreover, trying to compute the continued
fraction produces
\[
\KK_{1}^{\infty}\frac{-n^{2}}{2n+1}=\KK_{1}^{\infty}\frac{-n\times n}{n+\left(n+1\right)}=\frac{1}{\sum_{k=0}^{\infty}\prod_{i=1}^{k}\frac{i}{i+1}}-1=\frac{1}{\sum_{k=0}^{\infty}\frac{1}{k+1}}-1=-1,
\]
since the harmonic sum $\sum_{0}^{\infty}\frac{1}{k+1}$ diverges
to infinity.
\item For $f\left(x,y\right)=x+y$ and $\bar{f}\left(x,y\right)=1$ (which
we can think of as $\frac{\partial f}{\partial x}=\frac{\partial f}{\partial y}=\bar{f}$),
we get $b\left(x\right)=x,\;a\left(x,y\right)=x+y-1$, and in the
$y=1$ case we get 
\[
\KK_{1}^{\infty}\frac{n}{n}=\KK_{1}^{\infty}\frac{-\left(\left(-1\right)\times n\right)}{\left(-1\right)+\left(n+1\right)}=\frac{1}{\sum_{k=0}^{\infty}\prod_{i=1}^{k}\frac{-1}{i+1}}-1=\frac{1}{e-1}
\]
which we already saw in \exaref{(The-exponential-function)}.
\end{enumerate}
\item When $f,\bar{f}$ have degree at most 2, then we have the following
families of examples (as function of $C$):{\tiny{}
\[
\begin{array}{c|c|c|c|c}
\text{operation} & f\left(x,y\right) & a\left(x,y\right) & b\left(x\right) & \text{Euler family}\;\left(a\left(x,1\right)\right)\\
\hline \bar{g}\left(x,y\right)=-g\left(-x,y\right) & x^{2}+xy+\frac{y^{2}}{2}+C\left(x+y\right) & \left(x+1\right)^{2}+x^{2}+y\left(y-1\right)+C\left(2y-1\right) & -x^{2}\left(x^{2}-C^{2}\right) & \left(x+1\right)\left(x+1+C\right)+x\left(x-C\right)\\
\bar{g}\left(x,y\right)=g\left(-x,y\right) & x^{2}+2xy+2y^{2}+C\left(2y-x\right) & \left(2x+1\right)\left(2y-1+C\right) & x^{2}\left(x^{2}-C^{2}\right) & \left(x+1\right)\left(x+1+C\right)-x\left(x-C\right)\\
\bar{g}\left(x,y\right)=g\left(x,-y\right) & x^{2}+2xy+2y^{2}+C\left(x+y\right) & \left(2x+1+C\right)\left(2y-1\right) & x^{2}\left(x+C\right)^{2} & \left(x+1\right)\left(x+C+1\right)-x\left(x+C\right)\\
\bar{g}\left(x,y\right)=-g\left(x,-y\right) & \frac{2x^{2}+2xy+y^{2}+C\left(2x+y\right)}{2} & C\left(2x+1\right)+x^{2}+\left(x+1\right)^{2}+y\left(y-1\right) & -x^{2}\left(x+C\right)^{2} & \left(x+1\right)\left(x+C+1\right)+x\left(x+C\right)
\end{array}
\]
}In particular, when taking $C=0$, the $y=1$ line is either $b\left(x\right)=-x^{4}$
and $a\left(x\right)=x^{2}+\left(1+x\right)^{2}$, or $b\left(x\right)=x^{4}$
and $a\left(x\right)=\left(x+1\right)^{2}-x^{2}$. The continued fraction
will eventually be transformed (after the right Mobius action) to
the sums $\sum_{1}^{\infty}\frac{1}{n^{2}}$ and $\sum_{1}^{\infty}\frac{\left(-1\right)^{n}}{n^{2}}$
which are $\zeta\left(2\right)$ and $\frac{1}{2}\zeta\left(2\right)$
respectively.
\item In degree 3, with the action $\bar{g}\left(x,y\right)\mapsto g\left(-x,y\right)$,
we have the family 
\begin{align*}
f\left(x,y\right) & =x^{3}+2x^{2}\left(y-C\right)+2x\left(y-C\right)^{2}+\left(y-C\right)^{3}-\left(x+y-C\right)C^{2}\\
b\left(x\right) & =-x^{2}\left(x-C\right)^{2}\left(x+C\right)^{2}\\
a\left(x,y\right) & =x\left(x-C\right)^{2}+\left(x+1\right)\left(x+1+C\right)^{2}+\left(1+2x\right)\left(y-1-2C\right)2y.
\end{align*}
When $y=1$ we get a continued fraction in the Euler family with $h_{1}\left(x\right)=x\left(x-C\right)^{2}$
and $h_{2}\left(x\right)=x\left(x+C\right)^{2}$. In particular, in
the case where $C=0$ we simply get the matrix field for $\zeta\left(3\right)$
mentioned in \exaref{zeta-3-matrix-field}.
\end{enumerate}
\end{example}

\begin{rem}
Once we have a pair of conjugate polynomials $f,\bar{f}$, there are
several ways to generate more such pairs. The simplest way is just
to take $cf,c\bar{f}$ for some $0\neq c\in\CC$. Another less trivial
way is to look at the pair $\left(f\left(y,x\right),\;-\bar{f}\left(y,x\right)\right)$.
We shall see in \subsecref{The-dual-matrix-field} how this new pair
is hidden in the same conservative matrix field.
\end{rem}

Right now, while the $M_{X}^{cf}$ matrix has the known continued
fraction form, the $M_{Y}^{cf}$ matrices have this new unkown form
$\left(\begin{smallmatrix}\bar{f}\left(x,y\right) & b\left(x\right)\\
1 & f\left(x,y\right)
\end{smallmatrix}\right)$. However, as we shall soon see, there are hidden continued fraction
in $M_{Y}^{cf}$ as well, are both $M_{X}^{cf},M_{Y}^{cf}$ are defined
very similarly. For that, we use the following notations.
\begin{notation}
We define:
\[
U_{\alpha}=\left(\begin{smallmatrix}1 & \alpha\\
0 & 1
\end{smallmatrix}\right)\qquad D_{\alpha}=\left(\begin{smallmatrix}\alpha & 0\\
0 & 1
\end{smallmatrix}\right)\qquad\tau=\left(\begin{smallmatrix}0 & 1\\
1 & 0
\end{smallmatrix}\right).
\]
For any matrix $M$, we will write the isomorphism $M\mapsto M^{\tau}=\tau M\tau^{-1}$
(and note that $\tau^{2}=Id$, so that $\tau^{-1}=\tau$). More specifically,
we have that $\left(\begin{smallmatrix}a & b\\
c & c
\end{smallmatrix}\right)^{\tau}=\left(\begin{smallmatrix}d & c\\
b & a
\end{smallmatrix}\right)$ is just switching the rows and switching the columns, and in particular
$U_{\alpha}^{\tau}=U_{\alpha}^{tr}$.
\end{notation}

\medskip{}

With these notations we get:
\begin{alignat*}{2}
M_{X}^{cf}\left(x,y\right) & =\left(\begin{smallmatrix}0 & b\left(x\right)\\
1 & f\left(x,y\right)-\bar{f}\left(x+1,y\right)
\end{smallmatrix}\right) &  & =D_{b_{X}\left(x\right)}\cdot\tau\cdot U_{f\left(x,y\right)}\cdot U_{-\bar{f}\left(x+1,y\right)}\\
M_{Y}^{cf}\left(x,y\right) & =\left(\begin{smallmatrix}\bar{f}\left(x,y\right) & b\left(x\right)\\
1 & f\left(x,y\right)
\end{smallmatrix}\right) &  & =U_{\bar{f}\left(x,y\right)}\cdot D_{-b_{Y}\left(y\right)}\cdot\tau\cdot U_{f\left(x,y\right)},
\end{alignat*}
so that $M_{X}^{cf}$ and $M_{Y}^{cf}$ are ``almost'' the same.
There is some ``cyclic permutation'' and after it they have a similar
structure, with related parameters, and in particular the $M_{Y}^{cf}$
is also a continued fraction sequence, after a simple coboundary equivalence
(via the matrices $U_{\bar{f}\left(x,y\right)}$).

As mentioned in \remref{ff(0)=00003D0}, if $\left(f\bar{f}\right)\left(0\right)=0$,
then the $Y=0$ and $Y=1$ lines are in the trivial Euler family.
Similarly, on the $X=0$ line we have 
\[
M_{Y}^{cf}\left(0,y\right)=\left(\begin{smallmatrix}\bar{f}\left(0,y\right) & 0\\
1 & f\left(0,y\right)
\end{smallmatrix}\right),
\]
which is even simpler to work with (recall that the continued fraction
in the trivial Euler family, are in essence upper triangular in disguise).
However, on the $X=0$ line we have that $M_{X}^{cf}\left(0,y\right)=\left(\begin{smallmatrix}0 & 0\\
1 & f\left(0,y\right)-\bar{f}\left(1,y\right)
\end{smallmatrix}\right)$ are not invertible. With this in mind, we do a slight change of parameters,
which will solve this problem, and we will see is more natural.
\begin{defn}
Let $f,\bar{f}$ be conjugate polynomials and $M_{X},M_{Y}$ as in
\defref{conjugate}. Define
\begin{align*}
M_{X}\left(x,y\right) & :=D_{b\left(x\right)}^{-1}M_{X}^{cf}\left(x,y\right)D_{b\left(x+1\right)}=\tau U_{f\left(x,y\right)}U_{-\bar{f}\left(x+1,y\right)}D_{b\left(x+1\right)}=\left(\begin{smallmatrix}0 & 1\\
b\left(x+1\right) & f\left(x,y\right)-\bar{f}\left(x+1,y\right)
\end{smallmatrix}\right)\\
M_{Y}\left(x,y\right) & :=D_{b\left(x\right)}^{-1}M_{Y}^{cf}\left(x,y\right)D_{b\left(x\right)}=U_{f\left(x,y\right)}^{\tau}\tau D_{-\left(f\bar{f}\right)\left(0,y\right)}U_{\bar{f}\left(x,y\right)}^{\tau}=\left(\begin{smallmatrix}\bar{f}\left(x,y\right) & 1\\
b\left(x\right) & f\left(x,y\right)
\end{smallmatrix}\right)
\end{align*}
\smallskip{}
\end{defn}

There are three main reasons why this is a bit better way to view
our matrix fields:
\begin{enumerate}
\item If we start the continued fraction on the $y$ line at $x=0$ with
the previous $M_{X}^{cf}$ matrices, we get 
\[
M_{X}^{cf}\left(0,y\right)M_{X}^{cf}\left(1,y\right)M_{X}^{cf}\left(2,y\right)\cdots=\left[D_{b\left(0\right)}\tau U_{f\left(0,y\right)}U_{-\bar{f}\left(0+1,y\right)}\right]\left[D_{b\left(1\right)}\tau U_{f\left(1,y\right)}U_{-\bar{f}\left(1+1,y\right)}\right]\cdots
\]
where as we mentioned before $D_{b\left(0\right)}=\left(\begin{smallmatrix}0 & 0\\
0 & 1
\end{smallmatrix}\right)$ is singular which can cause problems. This means that we have to
start with $x=1$, and therefore ``lose'' the information from $\tau U_{f\left(0,y\right)}U_{-\bar{f}\left(0+1,y\right)}$.
With our new matrices $M_{X}$ we instead get
\[
M_{X}\left(0,y\right)M_{X}\left(1,y\right)M_{X}\left(2,y\right)\cdots=\left[\tau U_{f\left(0,y\right)}U_{-\bar{f}\left(0+1,y\right)}D_{b\left(1\right)}\right]\left[\tau U_{f\left(1,y\right)}U_{-\bar{f}\left(1+1,y\right)}D_{b\left(2\right)}\right]\cdots
\]
so we start exactly after the problematic matrix $D_{b\left(0\right)}$.
\item With this new definition, where we start at $x=0$, we get that $M_{Y}\left(0,y\right)$
is upper triangular, since
\[
M_{Y}\left(0,y\right)=\left(\begin{smallmatrix}\bar{f}\left(0,y\right) & 1\\
b\left(0\right) & f\left(0,y\right)
\end{smallmatrix}\right)=\left(\begin{smallmatrix}\bar{f}\left(0,y\right) & 1\\
0 & f\left(0,y\right)
\end{smallmatrix}\right).
\]
\item Finally, as we shall see below, the limits for each horizontal line
are more natural, namely 
\[
\limfi N\left[\prod_{n=0}^{N}M_{X}\left(n,m\right)\right]\left(0\right)=\tau U_{a\left(0,m\right)}\limfi N\left[\prod_{n=1}^{N}M_{X}^{cf}\left(n,m\right)\right]\left(0\right)=\left(a\left(0,m\right)+\left(\KK_{1}^{\infty}\frac{b\left(n\right)}{a\left(n,m\right)}\right)\right)^{-1}.
\]
 In particular, in the new matrix field for our $\zeta\left(3\right)$
example, we will get the limits $\sum_{m}^{\infty}\frac{1}{k^{3}}$.
This will simplify the arguments when trying to find the denominators
and numerators of the convergents.
\end{enumerate}
With this in mind we rewrite \thmref{matrix-field-structure} and
expand it with this new matrices.
\begin{thm}
\label{thm:normalized-matrix-field}Let $f,\bar{f},a,b$ be polynomials
as in \defref{conjugate}. We set 
\begin{align*}
M_{X}\left(x,y\right) & :=\tau U_{f\left(x,y\right)}U_{-\bar{f}\left(x+1,y\right)}D_{b\left(x+1\right)}=\left(\begin{smallmatrix}0 & 1\\
b\left(x+1\right) & f\left(x,y\right)-\bar{f}\left(x+1,y\right)
\end{smallmatrix}\right)\\
M_{Y}\left(x,y\right) & :=U_{f\left(x,y\right)}^{\tau}\tau D_{-\left(f\bar{f}\right)\left(0,y\right)}U_{\bar{f}\left(x,y\right)}^{\tau}=\left(\begin{smallmatrix}\bar{f}\left(x,y\right) & 1\\
b\left(x\right) & f\left(x,y\right)
\end{smallmatrix}\right)
\end{align*}
Then
\begin{enumerate}
\item The matrices form a conservative matrix field, namely
\[
M_{X}\left(x,y\right)M_{Y}\left(x+1,y\right)=M_{Y}\left(x,y\right)M_{X}\left(x,y+1\right).
\]
\item The determinants of $M_{X}\left(x,y\right),M_{Y}\left(x,y\right)$
are only functions of $x,y$ respectively, and more specifically:
\begin{align*}
\det\left(M_{X}\left(x,y\right)\right) & =-b\left(x+1\right)=\left(f\cdot\bar{f}\right)\left(0,0\right)-\left(f\cdot\bar{f}\right)\left(x+1,0\right)=-b_{X}\left(x+1\right)\\
\det\left(M_{Y}\left(x,y\right)\right) & =\left(f\cdot\bar{f}\right)\left(x,y\right)-b\left(x\right)=\left(f\cdot\bar{f}\right)\left(0,y\right)=b_{Y}\left(y\right).
\end{align*}
\item For $x=0$ , the matrices $M_{Y}\left(0,y\right)$ are upper triangular
\[
M_{Y}\left(0,y\right)=\left(\begin{smallmatrix}\bar{f}\left(0,y\right) & 1\\
0 & f\left(0,y\right)
\end{smallmatrix}\right).
\]
\end{enumerate}
\end{thm}

\begin{proof}
This follows directly from \thmref{matrix-field-structure} .
\end{proof}
Next, we use the fact that the $Y=1$ line in the original conservative
matrix field is in the trivial Euler family, to find a simple presentation
for the $Y=1$ line in our new matrix field.
\begin{lem}
\label{lem:First-row}Suppose that $\left(f\bar{f}\right)\left(0,0\right)=0$.
Then
\begin{align*}
U_{\bar{f}\left(0,0\right)}^{\tau}\left[\prod_{0}^{n-1}M_{X}\left(k,1\right)\right]U_{-\bar{f}\left(n,0\right)}^{\tau} & =\left(\begin{smallmatrix}\left(-1\right)^{n}\prod_{1}^{n}\bar{f}\left(k,0\right) & c_{n}\\
0 & \prod_{1}^{n}f\left(k,0\right)
\end{smallmatrix}\right)\\
c_{n} & =\sum_{k=1}^{n}\left(-1\right)^{k-1}\left(\prod_{i=1}^{k-1}\bar{f}\left(i,0\right)\right)\left(\prod_{i=k+1}^{n}f\left(i,0\right)\right).
\end{align*}
\end{lem}

\begin{proof}
Similar to \exaref{Euler_to_triangular}, assuming that $\left(f\bar{f}\right)\left(0,0\right)=0$,
at the $Y=1$ we have 
\begin{align*}
U_{\bar{f}\left(x,0\right)}^{\tau}M_{X}\left(x,1\right)U_{-\bar{f}\left(x+1,0\right)}^{\tau} & =\left(\begin{smallmatrix}1 & 0\\
\bar{f}\left(x,0\right) & 1
\end{smallmatrix}\right)\left(\begin{smallmatrix}0 & 1\\
\left(f\bar{f}\right)\left(x+1,0\right) & f\left(x+1,0\right)-\bar{f}\left(x,0\right)
\end{smallmatrix}\right)\left(\begin{smallmatrix}1 & 0\\
-\bar{f}\left(x+1,0\right) & 1
\end{smallmatrix}\right)=\left(\begin{smallmatrix}-\bar{f}\left(x+1,0\right) & 1\\
0 & f\left(x+1,0\right)
\end{smallmatrix}\right).
\end{align*}
It then follows that 
\[
U_{\bar{f}\left(0,0\right)}^{\tau}\left[\prod_{0}^{n-1}M_{X}\left(k,1\right)\right]U_{-\bar{f}\left(n,0\right)}=\prod_{1}^{n}\left(\begin{smallmatrix}-\bar{f}\left(k,0\right) & 1\\
0 & f\left(k,0\right)
\end{smallmatrix}\right),
\]
and a standard induction on product of upper triangular matrices will
show that we get
\[
\left(\begin{smallmatrix}\left(-1\right)^{n}\prod_{1}^{n}\bar{f}\left(k,0\right) & c_{n}\\
0 & \prod_{1}^{n}f\left(k,0\right)
\end{smallmatrix}\right)\quad,\quad c_{n}=\sum_{k=1}^{n}\left(-1\right)^{k-1}\left(\prod_{i=1}^{k-1}\bar{f}\left(k,0\right)\right)\left(\prod_{i=k+1}^{n}f\left(k,0\right)\right).
\]
\end{proof}
\begin{rem}
Note in particular that when $\bar{f}\left(0,0\right)=0$ in the lemma
above, then $U_{\bar{f}\left(0,0\right)}^{\tau}=I$ is simply the
identity matrix.
\end{rem}


\subsection{\label{subsec:The-dual-matrix-field}The dual conservative matrix
field}

With \thmref{normalized-matrix-field} and \lemref{First-row} in
the previous section, we see that we understand quite well both the
$Y=1$ and $X=0$ lines. More over, we already saw that both the horizontal
and the vertical lines in the matrix field are more or less continued
fractions, namely
\begin{align*}
M_{X}\left(x,y\right) & :=\tau U_{f\left(x,y\right)}U_{-\bar{f}\left(x+1,y\right)}D_{b\left(x+1\right)}=\left(\begin{smallmatrix}0 & 1\\
b\left(x+1\right) & f\left(x,y\right)-\bar{f}\left(x+1,y\right)
\end{smallmatrix}\right)\\
M_{Y}\left(x,y\right) & :=U_{f\left(x,y\right)}^{\tau}\tau D_{-\left(f\bar{f}\right)\left(0,y\right)}U_{\bar{f}\left(x,y\right)}^{\tau}=\left(\begin{smallmatrix}\bar{f}\left(x,y\right) & 1\\
b\left(x\right) & f\left(x,y\right)
\end{smallmatrix}\right)
\end{align*}
The next goal is to use this almost symmetry with the hope of eventually
saying something about the diagonal line $X=Y$.
\begin{defn}[\textbf{The dual matrix field}]
Let $f\left(x,y\right),\bar{f}\left(x,y\right)$ be conjugate polynomial,
and let $M_{X},M_{Y}$ be as above. We define the dual matrix field
to be
\begin{align*}
\hat{M}_{Y}\left(y,x\right) & =U_{\bar{f}\left(x-1,y\right)}^{\tau}M_{X}\left(x-1,y+1\right)U_{-\bar{f}\left(x,y\right)}^{\tau}=U_{f\left(x,y\right)}^{\tau}\tau D_{b\left(x\right)}U_{-\bar{f}\left(x,y\right)}^{\tau}\\
\hat{M}_{X}\left(y,x\right) & =U_{\bar{f}\left(x-1,y\right)}^{\tau}M_{Y}\left(x-1,y+1\right)U_{-\bar{f}\left(x-1,y+1\right)}^{\tau}=\tau U_{\bar{f}\left(x-1,y\right)}U_{f\left(x-1,y+1\right)}D_{-b_{Y}\left(y+1\right)}
\end{align*}
This new matrix field corresponds to the conjugate polynomials 
\begin{align*}
\hat{f}\left(x,y\right) & =f\left(y,x\right)\\
\bar{\hat{f}}\left(x,y\right) & =-\bar{f}\left(y,x\right)\\
\hat{a}\left(x,y\right) & =\hat{f}\left(x,y\right)-\bar{\hat{f}}\left(x+1,y\right)=f\left(y,x\right)+\bar{f}\left(y,x+1\right)\\
\hat{b}\left(x\right) & =\left(\hat{f}\bar{\hat{f}}\right)\left(x,0\right)=-\left(f\bar{f}\right)\left(0,x\right)
\end{align*}
\end{defn}

\begin{example}
In the $\zeta\left(3\right)$ matrix field mentioned in \exaref{zeta-3-matrix-field}
we have a special case where
\[
f\left(x,y\right)=\frac{y^{3}-x^{3}}{y-x}\left(y+x\right)\qquad;\qquad\bar{f}\left(x,y\right)=\frac{y^{3}+x^{3}}{y+x}\left(y-x\right),
\]
satisfy $f\left(x,y\right)=f\left(y,x\right)$ and $\bar{f}\left(x,y\right)=-\bar{f}\left(y,x\right)$,
so that $\hat{f}=f$ and $\bar{\hat{f}}=\bar{f}$.

In the one of the $\zeta\left(2\right)$ matrix field from \exaref{vector-field-examples},
we have
\[
f\left(x,y\right)=2x^{2}+2xy+y^{2}\qquad;\qquad\bar{f}\left(x,y\right)=-2x^{2}+2xy-y^{2}
\]
so that 
\[
\hat{f}\left(x,y\right)=x^{2}+2xy+2y^{2}\qquad;\qquad\bar{\hat{f}}\left(x,y\right)=x^{2}-2xy+2y^{2}
\]
and therefore
\begin{align*}
\hat{a}\left(x,y\right) & =\left(x^{2}+2xy+2y^{2}\right)-\left(\left(x+1\right)^{2}-2\left(x+1\right)y+2y^{2}\right)=\left(2y-1\right)\left(2x+1\right)\\
\hat{b}_{X}\left(x\right) & =x^{4}.
\end{align*}

\newpage{}
\end{example}

This dual matrix field construction not only gives us free of charge
another conservative matrix field for every one that we find, but
they are also closely related. In the matrix field with $M_{X},M_{Y}$
, the horizontal lines are (almost) polynomial continued fractions,
and we wish to study how the numerators and denominator behave there.
By definition, the horizontal lines of the dual matrix field correspond
to vertical line in the original matrix field, so understanding the
full matrix field is equivalent to understand these continued fractions.
More precisely, since
\[
M_{Y}\left(x,y\right)=U_{-\bar{f}\left(x,y-1\right)}^{\tau}\hat{M}_{X}\left(y-1,x+1\right)U_{\bar{f}\left(x,y\right)}^{\tau},
\]
we get that
\begin{equation}
\prod_{k=1}^{n}M_{Y}\left(y,k\right)=U_{-\bar{f}\left(y,0\right)}^{\tau}\left[\prod_{k=0}^{n-1}\hat{M}_{X}\left(k,y+1\right)\right]U_{\bar{f}\left(y,n\right)}^{\tau}\label{eq:dual-row-column}
\end{equation}

With this dualic structure we turn to study the rational approximation
given by the different points on the matrix field, and more concretely
how far the standard rational presentation is from being a reduced
rational presentation. 
\begin{defn}
\label{def:deno-nume}For every $n\geq0$ define the polynomial vectors
\begin{align*}
\left(\begin{smallmatrix}p_{n}\left(y\right)\\
q_{n}\left(y\right)
\end{smallmatrix}\right) & =\left[\prod_{0}^{n-1}M_{X}\left(k,y\right)\right]e_{2}\\
\left(\begin{smallmatrix}\hat{p}_{n}\left(y\right)\\
\hat{q}_{n}\left(y\right)
\end{smallmatrix}\right) & =\left[\prod_{0}^{n-1}\hat{M}_{X}\left(k,y\right)\right]e_{2}.
\end{align*}
\end{defn}

For example, the first few values of $p_{n}\left(m\right),q_{n}\left(m\right)$
are arranged as :

\begin{align*}
\xyR{1pc}\xyC{1pc}\xymatrix{\left(*\right) &  &  & \left(*\right) &  &  & \left(*\right) &  &  & \left(*\right)\\
\\
{\left(\begin{smallmatrix}p_{0}\left(3\right)\\
q_{0}\left(3\right)
\end{smallmatrix}\right)}\ar[rrr]_{M_{X}\left(0,3\right)}\ar@{..>}[uu]^{M_{Y}\left(0,3\right)} &  &  & {\left(\begin{smallmatrix}p_{1}\left(3\right)\\
q_{1}\left(3\right)
\end{smallmatrix}\right)}\ar[rrr]_{M_{X}\left(1,3\right)}\ar@{..>}[uu]^{M_{Y}\left(1,3\right)} &  &  & {\left(\begin{smallmatrix}p_{2}\left(3\right)\\
q_{2}\left(3\right)
\end{smallmatrix}\right)}\ar[rrr]_{M_{X}\left(2,3\right)}\ar@{..>}[uu]^{M_{Y}\left(2,3\right)} &  &  & \left(*\right)\\
\\
{\left(\begin{smallmatrix}p_{0}\left(2\right)\\
q_{0}\left(2\right)
\end{smallmatrix}\right)}\ar[rrr]_{M_{X}\left(0,2\right)}\ar@{..>}[uu]^{M_{Y}\left(0,2\right)} &  &  & {\left(\begin{smallmatrix}p_{1}\left(2\right)\\
q_{1}\left(2\right)
\end{smallmatrix}\right)}\ar[rrr]_{M_{X}\left(1,2\right)}\ar@{..>}[uu]^{M_{Y}\left(1,2\right)} &  &  & {\left(\begin{smallmatrix}p_{2}\left(2\right)\\
q_{2}\left(2\right)
\end{smallmatrix}\right)}\ar[rrr]_{M_{X}\left(2,2\right)}\ar@{..>}[uu]^{M_{Y}\left(2,2\right)} &  &  & \left(*\right)\\
\\
{\left(\begin{smallmatrix}p_{0}\left(1\right)\\
q_{0}\left(1\right)
\end{smallmatrix}\right)}\ar[rrr]_{M_{X}\left(0,1\right)}\ar@{..>}[uu]^{M_{Y}\left(0,1\right)} &  &  & {\left(\begin{smallmatrix}p_{1}\left(1\right)\\
q_{1}\left(1\right)
\end{smallmatrix}\right)}\ar[rrr]_{M_{X}\left(1,1\right)}\ar@{..>}[uu]^{M_{Y}\left(1,1\right)} &  &  & {\left(\begin{smallmatrix}p_{2}\left(1\right)\\
q_{2}\left(1\right)
\end{smallmatrix}\right)}\ar[rrr]_{M_{X}\left(2,1\right)}\ar@{..>}[uu]^{M_{Y}\left(2,1\right)} &  &  & \left(*\right)
}
\end{align*}

\begin{rem}
Note that since $M_{X}\left(k,y\right)e_{1}=b\left(k+1\right)e_{2}$
and $\hat{M}_{X}\left(k,y\right)e_{1}=-b_{Y}\left(k\right)$, we have
for $n\geq1$
\begin{align*}
\left(\begin{smallmatrix}p_{n-1}\left(y\right) & p_{n}\left(y\right)\\
q_{n-1}\left(y\right) & q_{n}\left(y\right)
\end{smallmatrix}\right)D_{b_{X}\left(n\right)} & =\prod_{0}^{n-1}M_{X}\left(k,y\right)\\
\left(\begin{smallmatrix}\hat{p}_{n-1}\left(y\right) & \hat{p}_{n}\left(y\right)\\
\hat{q}_{n-1}\left(y\right) & \hat{q}_{n}\left(y\right)
\end{smallmatrix}\right)D_{-b_{Y}\left(n\right)} & =\prod_{0}^{n-1}\hat{M}_{X}\left(k,y\right).
\end{align*}
\end{rem}

To study these polynomials $p_{n}\left(m\right)$ and $q_{n}\left(m\right)$,
we use the conservative matrix field structure to see what happens
when we increase $n$ or increase $m$, and also what is the connections
between them and their duals $\hat{p}_{m}\left(n\right)$ and $\hat{q}_{m}\left(n\right)$.

\newpage{}
\begin{claim}
\label{claim:dual-field-identities}Let $f,\bar{f}\in\ZZ\left[x,y\right]$
be conjugate polynomials such that $\left(f\bar{f}\right)\left(0,0\right)=0$
and let $p_{n},q_{n},\hat{p}_{m},\hat{q}_{m}$ as in \defref{deno-nume}
above. Then
\begin{enumerate}
\item \label{enu:first-line-polynomials}\textbf{\uline{Bottom line}}\textbf{:
}Evaluating the polynomial $p_{n},q_{n}$ at $m=1$ we have
\begin{align*}
p_{n}\left(1\right) & =\sum_{k=1}^{n}\left(-1\right)^{k-1}\left(\prod_{i=1}^{k-1}\bar{f}\left(k,0\right)\right)\left(\prod_{i=k+1}^{n}f\left(k,0\right)\right)\\
q_{n}\left(1\right) & =\prod_{1}^{n}f\left(k,0\right)-\bar{f}\left(0,0\right)p_{n}\left(1\right).
\end{align*}
\item \label{enu:Increase-n}\textbf{\uline{Horizontal lines}}\textbf{:
}When increasing $n$ we get 3-term reccurence
\begin{align*}
\left(\begin{smallmatrix}p_{n+1}\left(y\right)\\
q_{n+1}\left(y\right)
\end{smallmatrix}\right) & =\left(\begin{smallmatrix}p_{n-1}\left(y\right) & p_{n}\left(y\right)\\
q_{n-1}\left(y\right) & q_{n}\left(y\right)
\end{smallmatrix}\right)\left(\begin{smallmatrix}b\left(n\right)\\
a\left(n,y\right)
\end{smallmatrix}\right).
\end{align*}
\item \label{enu:increase-m}\textbf{\uline{Vertical lines}}\textbf{:}
When $\left(f\bar{f}\right)\left(0,m\right)\neq0$, increasing $m$
follows the recurrence 
\begin{align*}
\left(\begin{smallmatrix}p_{n}\left(m+1\right)\\
q_{n}\left(m+1\right)
\end{smallmatrix}\right) & =\frac{1}{\left(f\bar{f}\right)\left(0,m\right)}\left(\begin{smallmatrix}f\left(0,m\right) & -1\\
0 & \bar{f}\left(0,m\right)
\end{smallmatrix}\right)\left(\begin{smallmatrix}p_{n-1}\left(m\right) & p_{n}\left(m\right)\\
q_{n-1}\left(m\right) & q_{n}\left(m\right)
\end{smallmatrix}\right)\left(\begin{smallmatrix}\left(f\bar{f}\right)\left(n,0\right)\\
f\left(n,m\right)
\end{smallmatrix}\right)
\end{align*}
\item \label{enu:reciprocal-polynomials}\textbf{\uline{Conservativeness}}\textbf{:}
Suppose that $\bar{f}\left(0,0\right)=0$. Then $p_{n},q_{n}$ and
$\hat{p}_{m},\hat{q}_{m}$ are related via the equation
\[
\left(\begin{smallmatrix}\prod_{k=1}^{m}\bar{f}\left(0,k\right) & \hat{p}_{m}\left(1\right)\\
0 & \hat{q}_{m}\left(1\right)
\end{smallmatrix}\right)\left(\begin{smallmatrix}p_{n}\left(m+1\right)\\
q_{n}\left(m+1\right)
\end{smallmatrix}\right)=\left(\begin{smallmatrix}\left(-1\right)^{n}\prod_{k=1}^{n}\bar{f}\left(k,0\right) & p_{n}\left(1\right)\\
0 & q_{n}\left(1\right)
\end{smallmatrix}\right)\left(\begin{smallmatrix}\hat{p}_{m}\left(n+1\right)\\
\hat{q}_{m}\left(n+1\right)
\end{smallmatrix}\right),
\]
and in particular we have that $\hat{q}_{m}\left(1\right)q_{n}\left(m+1\right)=q_{n}\left(1\right)\hat{q}_{m}\left(n+1\right)$.
\end{enumerate}
\end{claim}

\begin{proof}
\begin{enumerate}
\item Applying \lemref{First-row} we get 
\begin{align*}
\left(\begin{smallmatrix}p_{n}\left(1\right)\\
q_{n}\left(1\right)
\end{smallmatrix}\right) & =\left[\prod_{0}^{n-1}M_{X}\left(k,1\right)\right]e_{2}=U_{-\bar{f}\left(0,0\right)}^{\tau}\left(\begin{smallmatrix}\left(-1\right)^{n}\prod_{1}^{n}\bar{f}\left(k,0\right) & c_{n}\\
0 & \prod_{1}^{n}f\left(k,0\right)
\end{smallmatrix}\right)U_{\bar{f}\left(n,0\right)}^{\tau}e_{2}\\
 & =U_{-\bar{f}\left(0,0\right)}^{\tau}\left(\begin{smallmatrix}\left(-1\right)^{n}\prod_{1}^{n}\bar{f}\left(k,0\right) & c_{n}\\
0 & \prod_{1}^{n}f\left(k,0\right)
\end{smallmatrix}\right)e_{2}=U_{-\bar{f}\left(0,0\right)}^{\tau}\left(\begin{smallmatrix}c_{n}\\
\prod_{1}^{n}f\left(k,0\right)
\end{smallmatrix}\right)
\end{align*}
where
\begin{align*}
c_{n} & =\sum_{k=1}^{n}\left(-1\right)^{k-1}\left(\prod_{i=1}^{k-1}\bar{f}\left(k,0\right)\right)\left(\prod_{i=k+1}^{n}f\left(k,0\right)\right).
\end{align*}
It follows that 
\begin{align*}
p_{n}\left(1\right) & =e_{1}^{tr}\left(\begin{smallmatrix}p_{n}\left(1\right)\\
q_{n}\left(1\right)
\end{smallmatrix}\right)=e_{1}^{tr}U_{-\bar{f}\left(0,0\right)}^{\tau}\left(\begin{smallmatrix}c_{n}\\
\prod_{1}^{n}f\left(k,0\right)
\end{smallmatrix}\right)=c_{n}\\
q_{n}\left(1\right) & =e_{2}^{tr}\left(\begin{smallmatrix}p_{n}\left(1\right)\\
q_{n}\left(1\right)
\end{smallmatrix}\right)=\left(-\bar{f}\left(0,0\right),1\right)\left(\begin{smallmatrix}c_{n}\\
\prod_{1}^{n}f\left(k,0\right)
\end{smallmatrix}\right)=\prod_{1}^{n}f\left(k,0\right)-\bar{f}\left(0,0\right)c_{n}\\
 & =\sum_{k=0}^{n}\left(-1\right)^{k}\left(\prod_{i=0}^{k-1}\bar{f}\left(k,0\right)\right)\left(\prod_{i=k+1}^{n}f\left(k,0\right)\right).
\end{align*}
\item This is the standard recursion for continued fractions, and it follows
from
\begin{align*}
\left(\begin{smallmatrix}p_{n+1}\left(y\right)\\
q_{n+1}\left(y\right)
\end{smallmatrix}\right) & =\left[\prod_{0}^{n}M_{X}\left(k,y\right)\right]e_{2}=\left[\prod_{0}^{n-1}M_{X}\left(k,y\right)\right]M_{X}\left(n,y\right)e_{2}\\
 & =\left(\begin{smallmatrix}p_{n-1}\left(y\right) & p_{n}\left(y\right)\\
q_{n-1}\left(y\right) & q_{n}\left(y\right)
\end{smallmatrix}\right)D_{b_{X}\left(n\right)}M_{X}\left(n,y\right)e_{2}=\left(\begin{smallmatrix}p_{n-1}\left(y\right) & p_{n}\left(y\right)\\
q_{n-1}\left(y\right) & q_{n}\left(y\right)
\end{smallmatrix}\right)\left(\begin{smallmatrix}b_{X}\left(n\right)\\
a\left(n,y\right)
\end{smallmatrix}\right).
\end{align*}
\item This follows from the coboundary condition of the conservative matrix
field structure
\begin{align*}
\left(\begin{smallmatrix}p_{n}\left(m+1\right)\\
q_{n}\left(m+1\right)
\end{smallmatrix}\right) & =\left[\prod_{0}^{n-1}M_{X}\left(k,m+1\right)\right]e_{2}=M_{Y}\left(0,m\right)^{-1}\left[\prod_{0}^{n-1}M_{X}\left(k,m\right)\right]M_{Y}\left(n,m\right)e_{2}\\
 & =\left(\begin{smallmatrix}\bar{f}\left(0,m\right) & 1\\
0 & f\left(0,m\right)
\end{smallmatrix}\right)^{-1}\left(\begin{smallmatrix}p_{n-1}\left(m\right) & p_{n}\left(m\right)\\
q_{n-1}\left(m\right) & q_{n}\left(m\right)
\end{smallmatrix}\right)D_{b\left(n\right)}\cdot\left(\begin{smallmatrix}1\\
f\left(n,m\right)
\end{smallmatrix}\right)\\
 & =\frac{1}{\left(f\bar{f}\right)\left(0,m\right)}\left(\begin{smallmatrix}f\left(0,m\right) & -1\\
0 & \bar{f}\left(0,m\right)
\end{smallmatrix}\right)\left(\begin{smallmatrix}p_{n-1}\left(m\right) & p_{n}\left(m\right)\\
q_{n-1}\left(m\right) & q_{n}\left(m\right)
\end{smallmatrix}\right)\left(\begin{smallmatrix}\left(f\bar{f}\right)\left(n,0\right)\\
f\left(n,m\right)
\end{smallmatrix}\right)
\end{align*}
\item We compute the matrix in the $\left(n,m+1\right)$ position in the
matrix field in two different ways - first by moving along the $X=0$
line and then $Y=m+1$ line, and second by moving along the $Y=1$
line and then the $X=n$ line, namely 
\[
\prod_{1}^{m}M_{Y}\left(0,k\right)\prod_{0}^{n-1}M_{X}\left(k,m+1\right)=\prod_{0}^{n-1}M_{X}\left(k,1\right)\prod_{1}^{m}M_{Y}\left(n,k\right).
\]
Converting the $M_{Y}$ into $\hat{M}_{X}$ via \eqref{dual-row-column}
we have
\[
U_{-\bar{f}\left(0,0\right)}^{\tau}\left[\prod_{k=0}^{m-1}\hat{M}_{X}\left(k,1\right)\right]U_{\bar{f}\left(0,m\right)}^{\tau}\prod_{0}^{n-1}M_{X}\left(k,m+1\right)=\prod_{0}^{n-1}M_{X}\left(k,1\right)U_{-\bar{f}\left(n,0\right)}^{\tau}\left[\prod_{k=0}^{m-1}\hat{M}_{X}\left(k,n+1\right)\right]U_{\bar{f}\left(n,m\right)}^{\tau}.
\]
Multiplying both side by $e_{2}$, we get 
\[
U_{-\bar{f}\left(0,0\right)}^{\tau}\left[\prod_{k=0}^{m-1}\hat{M}_{X}\left(k,1\right)\right]U_{\bar{f}\left(0,m\right)}^{\tau}\left(\begin{smallmatrix}p_{n}\left(m+1\right)\\
q_{n}\left(m+1\right)
\end{smallmatrix}\right)=\left[\prod_{0}^{n-1}M_{X}\left(k,1\right)\right]U_{-\bar{f}\left(n,0\right)}^{\tau}\left(\begin{smallmatrix}\hat{p}_{m}\left(n+1\right)\\
\hat{q}_{m}\left(n+1\right)
\end{smallmatrix}\right).
\]
Next, we use \lemref{First-row} as in the previous part, and the
fact that $U_{\bar{f}\left(0,0\right)}^{\tau}=Id$ to get that 
\begin{align*}
\left[\prod_{0}^{n-1}M_{X}\left(k,1\right)\right]U_{-\bar{f}\left(n,0\right)}^{\tau} & =\left(\begin{smallmatrix}\left(-1\right)^{n}\prod_{1}^{n}\bar{f}\left(k,0\right) & p_{n}\left(1\right)\\
0 & q_{n}\left(1\right)
\end{smallmatrix}\right)\\
\left[\prod_{0}^{m-1}\hat{M}_{X}\left(k,1\right)\right]U_{\bar{f}\left(0,m\right)}^{\tau} & =\left(\begin{smallmatrix}\prod_{1}^{m}\bar{f}\left(0,k\right) & \hat{p}_{m}\left(1\right)\\
0 & \hat{q}_{m}\left(1\right)
\end{smallmatrix}\right).
\end{align*}
Putting everything together, we get 
\[
\left(\begin{smallmatrix}\prod_{1}^{m}\bar{f}\left(0,k\right) & \hat{p}_{m}\left(1\right)\\
0 & \hat{q}_{m}\left(1\right)
\end{smallmatrix}\right)\left(\begin{smallmatrix}p_{n}\left(m+1\right)\\
q_{n}\left(m+1\right)
\end{smallmatrix}\right)=\left(\begin{smallmatrix}\left(-1\right)^{n}\prod_{1}^{n}\bar{f}\left(k,0\right) & p_{n}\left(1\right)\\
0 & q_{n}\left(1\right)
\end{smallmatrix}\right)\left(\begin{smallmatrix}\hat{p}_{m}\left(n+1\right)\\
\hat{q}_{m}\left(n+1\right)
\end{smallmatrix}\right),
\]
which is what we wanted to show.
\end{enumerate}
\end{proof}
If $\bar{f}\left(0,0\right)=0$ as in the last part of the claim above,
then $\frac{q_{n}\left(m+1\right)}{q_{n}\left(1\right)}=\frac{\hat{q}_{m}\left(n+1\right)}{\hat{q}_{m}\left(1\right)}$
. Fixing $m$ and letting $n\to\infty$, the numbers $q_{n}\left(m\right)$
are simply the denominators for the continued fraction on the matrix
field's $m$'th row. As we already know how to compute $q_{n}\left(1\right)$,
to understand these denominators, we would need to understand $\frac{\hat{q}_{m}\left(n+1\right)}{\hat{q}_{m}\left(1\right)}$.
Since $m$ is fixed, the function $n\mapsto\hat{q}_{m}\left(n+1\right)$
is just polynomial in $n$, and we divide it by the constant $\hat{q}_{m}\left(1\right)$.
This already tells us a lot about these denominators.

Eventually we would want to find and show that $\gcd\left(q_{n}\left(m+1\right),p_{n}\left(m+1\right)\right)$
is large. In particular, it would be helpful to know if $q_{n}\left(1\right)\mid q_{n}\left(m+1\right)$
for all $n$, which we just saw is equivalent to $\frac{\hat{q}_{m}\left(n+1\right)}{\hat{q}_{m}\left(1\right)}$
always being an integer. Hence, we are left with the problem of checking
if all the evaluations of a polynomial $\hat{q}_{m}$ at integer points
are divisible by the same number $\hat{q}_{m}\left(1\right)$. The
solution to this type of question is well known, and it not hard to
show that this holds exactly when $\hat{q}_{m}\left(1\right)\mid\hat{q}_{m}\left(n+1\right)$
for $\deg\left(\hat{q}_{m}\right)+1$ consecutive integers (see \appref{integer-values}).
This suggest an induction like process to show that this holds for
all of the denominators, and in particular we will show it for the
matrix field of $\zeta\left(3\right)$.

\newpage{}

\section{\label{sec:The-z3-case}The $\zeta\left(3\right)$ case}

We now apply the dual matrix field identities for the $\zeta\left(3\right)$
matrix field. Recall that in this case we have that 
\begin{align*}
f\left(x,y\right) & =\frac{y^{3}-x^{3}}{y-x}\left(y+x\right)=y^{3}+2y^{2}x+2yx^{2}+x^{3}\\
\bar{f}\left(x,y\right) & =\frac{y^{3}+x^{3}}{y+x}\left(y-x\right)=y^{3}-2y^{2}x+2yx^{2}-x^{3}\\
a\left(x,y\right) & =x^{3}+\left(1+x\right)^{3}+2y\left(y-1\right)\left(2x+1\right)\\
b\left(x\right) & =-x^{6}.
\end{align*}

Our first goal is showing that $\gcd\left(q_{n}\left(m\right),p_{n}\left(m\right)\right)$
is large as $n$ and $m$ increase, and we will use the results from
\claimref{dual-field-identities}. Once we understand these polynomials
and their gcd, which are defined for each row separately, we will
combine them together to understand the general numerators and denominators
appearing in any route on the matrix field, starting at the bottom
left corner. In particular, investigating the diagonal route, we will
show that both the approximations converge fast enough, and the gcd
grows fast enough to conclude at the end that $\zeta\left(3\right)$
is irrational.

This $\zeta\left(3\right)$ matrix field has several properties making
it easier to work with, which will come into play later:
\begin{fact}
\label{fact:zeta_3}
\begin{enumerate}
\item The matrix field is its own dual, since $f\left(y,x\right)=f\left(x,y\right)$
and $-\bar{f}\left(y,x\right)=\bar{f}\left(x,y\right)$. In particular
we get that $p=\hat{p}$ and $q=\hat{q}$.
\item We have that $f\left(0,0\right)=\bar{f}\left(0,0\right)=0$.
\item All the $f\left(n,0\right),f\left(0,n\right),\bar{f}\left(n,0\right),\bar{f}\left(0,n\right)$
are the same up to a sign (and therefore also $\hat{f}$ and $\bar{\hat{f}}$),
namely these are $n^{3}$. Furthermore, they all divide $f\left(n,n\right)=\hat{f}\left(n,n\right)=6n^{3}$.
\item We can write $a\left(x,y\right)$ as $a\left(x,y\right)=A_{1}\left(x\right)+y\left(y-1\right)A_{2}\left(x\right)$,
so in particular $a\left(x,1-y\right)=a\left(x,y\right)$.
\end{enumerate}
\end{fact}

Next, we want to show that $\gcd\left(q_{n}\left(m\right),p_{n}\left(m\right)\right)$,
is almost divisible by $\left(n!\right)^{3}$. We already know that
fixing $m$ and only increasing $n$, namely running on horizontal
lines in the matrix field, we get ``nice'' continued fractions which
should have factorial reduction. The next lemma shows that these factorial
reduction are in a sense synchronized between the different horizontal
lines.

In the following, we will use \textbf{$\lcm\left[n\right]$ }for $\lcm\left\{ 1,2,...,n\right\} $
where $n\geq1$ and also set $\lcm\left[0\right]=1$.
\begin{lem}
\label{lem:q-n-lemma}For all $n\geq0$ and $m\in\ZZ$ we have $\left(n!\right)^{3}\mid q_{n}\left(m\right)$
(with equality for $m=1$) and $\left(\frac{n!}{\lcm\left[n\right]}\right)^{3}\mid p_{n}\left(m\right)$.
In particular we have that $\left(\frac{n!}{\lcm\left[n\right]}\right)^{3}\mid\gcd\left(p_{n}\left(m\right),q_{n}\left(m\right)\right)$.
\end{lem}

\begin{proof}
We will prove this claim by induction on $n$, but before that, we
first consider the case where $m=1$ and $n$ is arbitrary (the bottom
horizontal line). By part \ref{enu:first-line-polynomials} in \claimref{dual-field-identities}
we get that 
\begin{align*}
p_{n}\left(1\right) & =\sum_{k=1}^{n}\left(-1\right)^{k-1}\left(\prod_{i=1}^{k-1}\bar{f}\left(k,0\right)\right)\left(\prod_{i=k+1}^{n}f\left(k,0\right)\right)=\sum_{1}^{n}\left(\frac{n!}{k}\right)^{3}\\
q_{n}\left(1\right) & =\prod_{1}^{n}f\left(k,0\right)=\left(n!\right)^{3}.
\end{align*}
These are exactly the numerator and denominator of $\sum_{1}^{n}\frac{1}{k^{3}}$
when taking the product of the denominators. Since we can also instead
take the least common multiple of the denominators, we see that $\left(\frac{n!}{\lcm\left[n\right]}\right)^{3}\mid p_{n}\left(1\right)$
as required. Of course the $\left(n!\right)^{3}\mid q_{n}\left(1\right)$
is trivial since $\left(n!\right)^{3}=q_{n}\left(1\right)$, but more
over it allows us to think of the general conditions as $q_{n}\left(1\right)\mid q_{n}\left(m\right)$
and $q_{1}\left(n\right)\mid\lcm\left[n\right]^{3}p_{n}\left(m\right)$.\\

We prove the rest of this lemma using induction on $n$. The induction
hypothesis will go as follows - assuming that the claim is true for
$\left(n-1,m\right)$ for a given $n$ and all $m\in\ZZ$, we show:
\begin{enumerate}
\item From part \enuref{reciprocal-polynomials} in \claimref{dual-field-identities},
we show that the claim is true for $\left(n,m\right)$ with $1\leq m\leq n$.
\item From part \enuref{increase-m} in \claimref{dual-field-identities},
if the claim is true for $\left(n-1,n\right)$ and $\left(n,n\right)$,
then it is true for $\left(n,n+1\right)$.
\item Our polynomials satisfy $q_{n}\left(y\right)=q_{n}\left(1-y\right)$
and $p_{n}\left(y\right)=p_{n}\left(1-y\right)$, so the claim is
true for $\left(n,m\right)$ with $-n\leq m\leq n+1$, which are $2\left(n+1\right)$
consecutive integers.
\item These polynomials have degree $\leq2n+1$, so this is enough to show
the claim for $\left(n,m\right)$ for all $m$.
\end{enumerate}

When $n=0$ we have $q_{0}\left(m\right)\equiv1$ and $p_{0}\left(m\right)\equiv0$
which are divisible by $\left(0!\right)^{3}=1$ and $\frac{0!}{\lcm\left[0\right]}=1$
respectively.

Suppose now that the claim is true for $\left(k,m\right)$ with $k\leq n-1$
and all $m$ and we prove for $\left(n,m\right)$ and all $m$. We
prove first for the denominators, which is easier.\\

\textbf{\uline{Denominators:}}

By using identity \enuref{reciprocal-polynomials} from \claimref{dual-field-identities},
together with the facts in \factref{zeta_3} about the matrix field
we get 
\[
\left(\begin{smallmatrix}q_{m}\left(1\right) & p_{m}\left(1\right)\\
0 & q_{m}\left(1\right)
\end{smallmatrix}\right)\left(\begin{smallmatrix}p_{n}\left(m+1\right)\\
q_{n}\left(m+1\right)
\end{smallmatrix}\right)=\left(\begin{smallmatrix}q_{n}\left(1\right) & p_{n}\left(1\right)\\
0 & q_{n}\left(1\right)
\end{smallmatrix}\right)\left(\begin{smallmatrix}p_{m}\left(n+1\right)\\
q_{m}\left(n+1\right)
\end{smallmatrix}\right).
\]
For the denominators, this implies that 
\[
\frac{q_{n}\left(m+1\right)}{q_{n}\left(1\right)}=\frac{q_{m}\left(n+1\right)}{q_{m}\left(1\right)}.
\]
By the induction hypothesis, for $0\leq m\leq n-1$ the right hand
of this equation is an integer, so that $q_{n}\left(1\right)\mid q_{n}\left(m+1\right)$.
Using part \enuref{increase-m} in \claimref{dual-field-identities}
with $n=m$ we have

\begin{align*}
\left(\begin{smallmatrix}p_{n}\left(n+1\right)\\
q_{n}\left(n+1\right)
\end{smallmatrix}\right) & =\frac{1}{\left(f\bar{f}\right)\left(0,n\right)}\left(\begin{smallmatrix}f\left(0,n\right) & -1\\
0 & \bar{f}\left(0,n\right)
\end{smallmatrix}\right)\left(\begin{smallmatrix}p_{n-1}\left(n\right) & p_{n}\left(n\right)\\
q_{n-1}\left(n\right) & q_{n}\left(n\right)
\end{smallmatrix}\right)\left(\begin{smallmatrix}\left(f\bar{f}\right)\left(n,0\right)\\
f\left(n,n\right)
\end{smallmatrix}\right)
\end{align*}
so for the denominators we get
\[
q_{n}\left(n+1\right)=q_{n-1}\left(n\right)f\left(n,0\right)\frac{\bar{f}\left(n,0\right)}{f\left(0,n\right)}+q_{n}\left(n\right)\frac{f\left(n,n\right)}{f\left(0,n\right)}=q_{n-1}\left(n\right)n^{3}\left(-1\right)+q_{n}\left(n\right)\cdot6.
\]
By the induction hypothesis $\left(n-1\right)!^{3}\mid q_{n-1}\left(n\right)$
and from the argument above $n!^{3}\mid q_{n}\left(n\right)$, so
we conclude that $n!^{3}\mid q_{n}\left(n+1\right)$. At this point,
we know the claim for $\left(n,m\right)$ with $1\leq m\leq n+1$.

Using the fact that $a\left(x,y\right)$ can be written as $A_{1}\left(x\right)+y\left(y-1\right)A_{2}\left(x\right)$
, we get that $a\left(x,y\right)=a\left(x,1-y\right)$. Since
\begin{align*}
q_{n}\left(y\right) & =e_{2}^{tr}\left[\prod_{0}^{n-1}M_{X}\left(k,y\right)\right]e_{2}=e_{2}^{tr}\left[\prod_{0}^{n-1}\left(\begin{smallmatrix}0 & 1\\
b\left(k+1\right) & a\left(k,y\right)
\end{smallmatrix}\right)\right]e_{2}
\end{align*}
we also get that $q_{n}\left(1-y\right)=q_{n}\left(y\right)$ and
$\deg_{y}\left(q_{n}\right)\leq2n$. From this we conclude that $q_{n}\left(1\right)\mid q_{n}\left(m\right)$
for all $-n\leq m\leq n+1$, which is a total of $2n+2\geq\deg_{y}\left(q_{n}\right)+1$
consecutive integers. Finally, using \lemref{binomial-polynomials}
from \appref{integer-values} we conclude that $q_{n}\left(1\right)\mid q_{n}\left(m\right)$
for all $m$, thus proving the induction step for the denominators.\\
\newpage{}

\textbf{\uline{Numerators:}}

The proof for the numerators is similar, but requires a bit more computations.
Part \enuref{reciprocal-polynomials} from \claimref{dual-field-identities}
\[
q_{m}\left(1\right)p_{n}\left(m+1\right)+p_{m}\left(1\right)q_{n}\left(m+1\right)=q_{n}\left(1\right)p_{m}\left(n+1\right)+p_{n}\left(1\right)q_{m}\left(n+1\right)
\]
can be rewritten as 
\[
\frac{\lcm\left[n\right]^{3}p_{n}\left(m+1\right)}{q_{n}\left(1\right)}=\overbrace{\frac{\lcm\left[n\right]^{3}p_{m}\left(n+1\right)}{q_{m}\left(1\right)}}^{\left(1\right)}+\overbrace{\frac{\lcm\left[n\right]^{3}p_{n}\left(1\right)}{q_{n}\left(1\right)}}^{\left(2\right)}\cdot\overbrace{\frac{q_{m}\left(n+1\right)}{q_{m}\left(1\right)}}^{\left(3\right)}-\overbrace{\frac{\lcm\left[n\right]^{3}p_{m}\left(1\right)}{q_{m}\left(1\right)}}^{\left(4\right)}\cdot\overbrace{\frac{q_{n}\left(m+1\right)}{q_{n}\left(1\right)}}^{\left(5\right)}.
\]
To show that the expression on the left is an integer, it is enough
to show that $\left(1\right)-\left(5\right)$ on the right are integers. 
\begin{itemize}
\item Expression $\left(2\right)$ is on the first row of the matrix field,
and we saw in the beginning of the proof that it is an integer.
\item Expressions $\left(3\right)$ and $\left(5\right)$ follows from the
claim about the denominators (which is independent of this proof about
the numerators).
\item Expressions $\left(1\right)$ and $\left(4\right)$ are true if $0\leq m\leq n-1$
using the induction hypothesis, and the fact that $\lcm\left[m\right]\mid\lcm\left[n\right]$
in that case.
\end{itemize}
To conclude, we just saw that the claim is true for $\left(n,m\right)$
when $1\leq m\leq n$.

Using part \enuref{increase-m} in \claimref{dual-field-identities}
with $n=m$ for the numerators we get 
\begin{align*}
p_{n}\left(n+1\right) & =\frac{1}{\left(f\bar{f}\right)\left(0,n\right)}\left[f\left(0,n\right)\left(\left(f\bar{f}\right)\left(n,0\right)p_{n-1}\left(n\right)+f\left(n,n\right)p_{n}\left(n\right)\right)-\left(\left(f\bar{f}\right)\left(n,0\right)q_{n-1}\left(n\right)+f\left(n,n\right)q_{n}\left(n\right)\right)\right]\\
 & =\left(-n^{3}p_{n-1}\left(n\right)+6p_{n}\left(n\right)\right)+\left(q_{n-1}\left(n\right)-\frac{6}{n^{3}}q_{n}\left(n\right)\right).
\end{align*}
Since the claim is true for $\left(n-1,n\right)$ and $\left(n,n\right)$,
we get that
\begin{align*}
\left(\frac{n!}{\lcm\left[n\right]}\right)^{3} & \mid\left(-n^{3}p_{n-1}\left(n\right)+6p_{n}\left(n\right)\right)\\
\left(n-1\right)!^{3} & \mid\left(q_{n-1}\left(n\right)-\frac{6}{n^{3}}q_{n}\left(n\right)\right).
\end{align*}
Since $n\mid\lcm\left[n\right]$ , it follows that $\frac{n!}{\lcm\left[n\right]}\mid\left(n-1\right)!$,
so everything together shows that 
\[
\left(\frac{n!}{\lcm\left[n\right]}\right)^{3}\mid p_{n}\left(n+1\right).
\]
At this point we know that $\left(\frac{n!}{\lcm\left[n\right]}\right)^{3}\mid p_{n}\left(m\right)$
for $1\leq m\leq n+1$. The same trick as with the denominators show
that $p_{n}\left(m\right)=p_{n}\left(1-m\right)$, so the claim is
true for $-n\leq m\leq n+1$, and using \lemref{binomial-polynomials}
again we conclude that it is true for all $m$, thus finishing the
proof for the induction step, and therefore the original claim.
\end{proof}
Up until now we looked at each row separately. We now move to the
whole matrix field.
\begin{defn}
Given $n\geq0$ and $m\geq1$, define
\[
\left(\begin{smallmatrix}P\left(n,m\right)\\
Q\left(n,m\right)
\end{smallmatrix}\right):=\left[\prod_{k=1}^{m-1}M_{Y}\left(0,k\right)\right]\left[\prod_{k=0}^{n-1}M_{X}\left(k,m\right)\right]e_{2}.
\]
In particular, as Mobius transformations we get that 
\[
\left[\prod_{k=1}^{m-1}M_{Y}\left(0,k\right)\right]\left[\prod_{k=0}^{n-1}M_{X}\left(k,m\right)\right]\left(0\right)=\frac{P\left(n,m\right)}{Q\left(n,m\right)}.
\]
\end{defn}

\begin{rem}
Note that for the general matrix field with $\bar{f}\left(0,0\right)=0$,
we can use \enuref{reciprocal-polynomials} in \claimref{dual-field-identities}
to get 
\[
\left(\begin{smallmatrix}P\left(n,m\right)\\
Q\left(n,m\right)
\end{smallmatrix}\right)=\left(\begin{smallmatrix}\prod_{k=1}^{m}\bar{f}\left(0,k\right) & \hat{p}_{m}\left(1\right)\\
0 & \hat{q}_{m}\left(1\right)
\end{smallmatrix}\right)\left(\begin{smallmatrix}p_{n}\left(m+1\right)\\
q_{n}\left(m+1\right)
\end{smallmatrix}\right)=\left(\begin{smallmatrix}\left(-1\right)^{n}\prod_{k=1}^{n}\bar{f}\left(k,0\right) & p_{n}\left(1\right)\\
0 & q_{n}\left(1\right)
\end{smallmatrix}\right)\left(\begin{smallmatrix}\hat{p}_{m}\left(n+1\right)\\
\hat{q}_{m}\left(n+1\right)
\end{smallmatrix}\right).
\]
In particular, in our $\zeta\left(3\right)$ case we have 
\[
\left(\begin{smallmatrix}P\left(n,m\right)\\
Q\left(n,m\right)
\end{smallmatrix}\right)=\left(\begin{smallmatrix}q_{m}\left(1\right) & p_{m}\left(1\right)\\
0 & q_{m}\left(1\right)
\end{smallmatrix}\right)\left(\begin{smallmatrix}p_{n}\left(m+1\right)\\
q_{n}\left(m+1\right)
\end{smallmatrix}\right)=\left(\begin{smallmatrix}q_{n}\left(1\right) & p_{n}\left(1\right)\\
0 & q_{n}\left(1\right)
\end{smallmatrix}\right)\left(\begin{smallmatrix}p_{m}\left(n+1\right)\\
q_{m}\left(n+1\right)
\end{smallmatrix}\right).
\]
\end{rem}

With this new notation, we have the new factorial reduction for these
numerators and denominators.
\begin{cor}
\label{cor:factorial-reduction}For all $n,m\geq0$ we have that 
\[
q_{m}\left(1\right)q_{n}\left(1\right)\mid Q\left(n,m+1\right),
\]
\begin{align*}
\frac{q_{m}\left(1\right)q_{n}\left(1\right)}{\lcm\left[\max\left(m,n\right)\right]^{3}} & \mid\gcd\left(P\left(n,m+1\right),Q\left(n,m+1\right)\right).
\end{align*}
In particular for $n=m$ we get that 
\begin{align*}
\left(\frac{n!}{\lcm\left[n\right]}\cdot n!\right)^{3}=\frac{\left(q_{n}\left(1\right)\right)^{2}}{\lcm\left[n\right]^{3}} & \mid\gcd\left(P\left(n,n+1\right),Q\left(n,n+1\right)\right).
\end{align*}
\end{cor}

\begin{proof}
Using the presentation from the remark above
\[
\left(\begin{smallmatrix}P\left(n,m+1\right)\\
Q\left(n,m+1\right)
\end{smallmatrix}\right)=\left(\begin{smallmatrix}q_{m}\left(1\right) & p_{m}\left(1\right)\\
0 & q_{m}\left(1\right)
\end{smallmatrix}\right)\left(\begin{smallmatrix}p_{n}\left(m+1\right)\\
q_{n}\left(m+1\right)
\end{smallmatrix}\right),
\]
and \lemref{q-n-lemma} we get that 
\begin{align*}
q_{m}\left(1\right)q_{n}\left(1\right) & \mid q_{m}\left(1\right)q_{n}\left(m+1\right)=Q\left(n,m+1\right)\\
\frac{q_{m}\left(1\right)q_{n}\left(1\right)}{\lcm\left[\max\left(m,n\right)\right]^{3}} & \mid q_{m}\left(1\right)p_{n}\left(m+1\right)+p_{m}\left(1\right)q_{n}\left(m+1\right)=P\left(n,m+1\right).
\end{align*}
\end{proof}
This factorial reduction property, will help us in the end to show
that $\zeta\left(3\right)$ is irrational, but it can also be used
to show more general properties of the matrix field, as follows.
\begin{thm}
\label{thm:all-directions-converge}Let $n_{i},m_{i}\geq1$ be any
sequence such that $\max\left(n_{i},m_{i}\right)\to\infty$. Then
$\frac{P\left(n_{i},m_{i}\right)}{Q\left(n_{i},m_{i}\right)}\to\zeta\left(3\right)$.
\end{thm}

Before we prove this theorem, here is an interesting corollary for
using this theorem for fixed $m$.
\begin{cor}
For any $m\geq1$, the limit for the $Y=m$ line is$\limfi n\frac{p_{n}\left(m\right)}{q_{n}\left(m\right)}=\sum_{m}^{\infty}\frac{1}{n^{3}}.$
\end{cor}

\begin{proof}
Using the notation $\left(\begin{smallmatrix}P\left(n,m-1\right)\\
Q\left(n,m-1\right)
\end{smallmatrix}\right)=\left(\begin{smallmatrix}q_{m-1}\left(1\right) & p_{m-1}\left(1\right)\\
0 & q_{m-1}\left(1\right)
\end{smallmatrix}\right)\left(\begin{smallmatrix}p_{n}\left(m\right)\\
q_{n}\left(m\right)
\end{smallmatrix}\right)$, we get that 
\[
\frac{P\left(n,m-1\right)}{Q\left(n,m-1\right)}=\frac{p_{n}\left(m\right)}{q_{n}\left(m\right)}+\frac{p_{m-1}\left(1\right)}{q_{m-1}\left(1\right)}.
\]
By \thmref{all-directions-converge} we know that $\limfi n\frac{P\left(n,m-1\right)}{Q\left(n,m-1\right)}=\zeta\left(3\right)$,
and we have already seen that $\frac{p_{m-1}\left(1\right)}{q_{m-1}\left(1\right)}=\sum_{1}^{m-1}\frac{1}{n^{3}}$,
so together we get that $\limfi n\frac{p_{n}\left(m\right)}{q_{n}\left(m\right)}=\sum_{m}^{\infty}\frac{1}{n^{3}}$.
\end{proof}
And now for the proof of the theorem.
\begin{proof}[Proof of \thmref{all-directions-converge}]
\textbf{\uline{The \mbox{$m_{i}$} bounded case}}\textbf{: }Suppose
first that $m_{i}$ is bounded, and by splitting the sequence to finitely
many subsequence, we may assume that $m_{i}=m$ is constant. We use
the presentation 
\[
\left(\begin{smallmatrix}P\left(n_{i},m\right)\\
Q\left(n_{i},m\right)
\end{smallmatrix}\right):=\left(\begin{smallmatrix}q_{n_{i}}\left(1\right) & p_{n_{i}}\left(1\right)\\
0 & q_{n_{i}}\left(1\right)
\end{smallmatrix}\right)\left(\begin{smallmatrix}\hat{p}_{m}\left(n_{i}+1\right)\\
\hat{q}_{m}\left(n_{i}+1\right)
\end{smallmatrix}\right)
\]
so that 
\[
\frac{P\left(n_{i},m\right)}{Q\left(n_{i},m\right)}=\frac{p_{m}\left(n_{i}+1\right)}{q_{m}\left(n_{i}+1\right)}+\frac{p_{n_{i}}\left(1\right)}{q_{n_{i}}\left(1\right)}.
\]

We already know that $\limfi n\frac{p_{n}\left(1\right)}{q_{n}\left(1\right)}=\zeta\left(3\right)$,
so if we can show that $\deg\left(q_{m}\right)>\deg\left(p_{m}\right)$,
then $\limfi n\frac{p_{m}\left(n\right)}{q_{m}\left(n\right)}=0$.
Indeed, recall that 
\[
\left(\begin{smallmatrix}p_{m}\left(y\right)\\
q_{m}\left(y\right)
\end{smallmatrix}\right)=\left[\prod_{0}^{m-1}M_{X}\left(k,y\right)\right]e_{2}
\]
where $M_{X}\left(x,y\right)=\left(\begin{smallmatrix}0 & 1\\
b\left(x+1\right) & a\left(x,y\right)
\end{smallmatrix}\right)$, and 
\[
a\left(x,y\right)=x^{3}+\left(1+x\right)^{3}+2y\left(y-1\right)\left(2x+1\right).
\]
This means that for every $k\geq0$ we have that $\deg_{y}\left(a\left(k,y\right)\right)=2$,
and by induction $\deg\left(p_{m}\right)=2m-2$ while $\deg\left(q_{m}\right)=2m$.
Hence $\deg\left(q_{m}\right)>\deg\left(p_{m}\right)$ and we are
done.\\

\textbf{\uline{The \mbox{$m_{i}$} unbounded case}}\textbf{:} Here
we will use the second presentation of $P$ and $Q$, namely
\[
\left(\begin{smallmatrix}P\left(n,m\right)\\
Q\left(n,m\right)
\end{smallmatrix}\right)=\left(\begin{smallmatrix}q_{m}\left(1\right) & p_{m}\left(1\right)\\
0 & q_{m}\left(1\right)
\end{smallmatrix}\right)\left(\begin{smallmatrix}p_{n}\left(m+1\right)\\
q_{n}\left(m+1\right)
\end{smallmatrix}\right)
\]
and therefore
\[
\frac{P\left(n,m\right)}{Q\left(n,m\right)}=\frac{p_{n}\left(m+1\right)}{q_{n}\left(m+1\right)}+\frac{p_{m}\left(1\right)}{q_{m}\left(1\right)}.
\]

Fix some $\varepsilon>0$. Since $\limfi m\frac{p_{m}\left(1\right)}{q_{m}\left(1\right)}=\zeta\left(3\right)$,
for all $m$ large enough we have $\left|\frac{p_{m}\left(1\right)}{q_{m}\left(1\right)}-\zeta\left(3\right)\right|\leq\frac{\varepsilon}{2}$.
As we shall see below, for all $m$ large enough we also have that
$\left|\frac{p_{n}\left(m+1\right)}{q_{n}\left(m+1\right)}\right|\leq\frac{\varepsilon}{2}$
independent of $n$. Hence, we can find $M=M_{\varepsilon}$ so that
for $m\geq M_{\varepsilon}$ we have $\left|\frac{P\left(n,m\right)}{Q\left(n,m\right)}-\zeta\left(3\right)\right|\leq\varepsilon$.
Thus, if we look at the two subsequence of $\left(n_{i},m_{i}\right)$
, where $m_{i}\geq M_{\varepsilon}$ and where $m_{i}\leq M_{\varepsilon}$,
we get that $\left|\frac{P\left(n_{i},m_{i}\right)}{Q\left(n_{i},m_{i}\right)}-\zeta\left(3\right)\right|\leq\varepsilon$
on the first subsequence, and from the previous case, if the second
subsequence is infinite, so that $n_{i}\to\infty$, we have $\left|\frac{P\left(n_{i},m_{i}\right)}{Q\left(n_{i},m_{i}\right)}-\zeta\left(3\right)\right|\leq\varepsilon$
for all $i$ large enough.

We are left to show that $\left|\frac{p_{n}\left(m\right)}{q_{n}\left(m\right)}\right|\leq\frac{\varepsilon}{2}$
for all $m$ large enough (independent of $n\geq0$). 

Note that the result in the corollary above that $\limfi n\frac{p_{n}\left(m\right)}{q_{n}\left(m\right)}=\sum_{m}^{\infty}\frac{1}{n^{3}}$
only depends on the $m$ bounded case, so we already fully proved
it. This is a tail of a convergent series, so it will be small for
all large enough $m$. With this motivation (without the result itself
from the corollary) we show that as $m$ increases , $\left|\frac{p_{n}\left(m\right)}{q_{n}\left(m\right)}\right|$
becomes bounded by similar such tail and therefore is as small as
we want.

Recall that 
\[
\left(\begin{smallmatrix}p_{n-1}\left(y\right) & p_{n}\left(y\right)\\
q_{n-1}\left(y\right) & q_{n}\left(y\right)
\end{smallmatrix}\right)D_{b_{X}\left(n\right)}=\prod_{0}^{n-1}M_{X}\left(k,y\right),
\]
so taking the determinant, we get that 
\[
\left(p_{n-1}\left(y\right)q_{n}\left(y\right)-p_{n}\left(y\right)q_{n-1}\left(y\right)\right)b\left(n\right)=\prod_{1}^{n}\left(-b\left(k\right)\right),
\]
which we can rewrite as
\[
\frac{p_{n}\left(y\right)}{q_{n}\left(y\right)}=\frac{p_{n-1}\left(y\right)}{q_{n-1}\left(y\right)}-\frac{\left(-1\right)^{n}\prod_{1}^{n-1}b\left(k\right)}{q_{n-1}\left(y\right)q_{n}\left(y\right)}.
\]

Using the fact that $p_{0}\left(y\right)=0$ , $q_{0}\left(y\right)=1$
and $\prod_{1}^{j-1}\left|b\left(k\right)\right|=\left(\left(j-1\right)!\right)^{6}=\left|q_{j-1}\left(1\right)\right|^{2}$,
we get that 
\[
\left|\frac{p_{n}\left(y\right)}{q_{n}\left(y\right)}\right|=\left|\sum_{j=1}^{n}\frac{\left(-1\right)^{j}\prod_{1}^{j-1}b\left(k\right)}{q_{j-1}\left(y\right)q_{j}\left(y\right)}\right|\leq\sum_{j=1}^{\infty}\left|\frac{\prod_{1}^{j-1}b\left(k\right)}{q_{j-1}\left(y\right)q_{j}\left(y\right)}\right|.
\]
By \lemref{q-n-lemma} we have that $\prod_{1}^{n-1}b\left(k\right)=\left(n-1\right)!^{6}=q_{n-1}\left(1\right)^{2}$,
and also $q_{n-1}\left(1\right)\mid q_{n-1}\left(m\right)$ and $q_{n-1}\left(1\right)n^{3}=q_{n}\left(n\right)\mid q_{n}\left(m\right)$
so that 
\[
\left|\frac{\prod_{1}^{j-1}b\left(k\right)}{q_{j-1}\left(m\right)q_{j}\left(m\right)}\right|=\left|\frac{q_{j-1}\left(1\right)}{q_{j-1}\left(m\right)}\cdot\frac{q_{j-1}\left(1\right)j^{3}}{q_{j}\left(m\right)}\cdot\frac{1}{j^{3}}\right|\leq\frac{1}{j^{3}}.
\]
This already shows that $\left|\frac{p_{n}\left(m\right)}{q_{n}\left(m\right)}\right|\leq\sum_{j=1}^{\infty}\frac{1}{j^{3}}$,
which is of course not enough, as instead of the tail, we got the
full sum. To solve this, we note that each one of the $q_{j}\left(y\right)$
for fixed $j\geq1$ are nonconstant polynomials of $y$ (of degree
$2j$) so that $\limfi y\left|\frac{q_{j-1}\left(1\right)}{q_{j-1}\left(y\right)}\cdot\frac{q_{j-1}\left(1\right)}{q_{j}\left(y\right)}\right|=0$.
Fixing $\varepsilon>0$ and $N>0$, we can find $M=M_{\varepsilon,N}$
large enough such that $\left|\frac{q_{j-1}\left(1\right)}{q_{j-1}\left(y\right)}\cdot\frac{q_{j-1}\left(1\right)}{q_{j}\left(y\right)}\right|<\frac{\varepsilon}{N}$
for all $y>M$ and $1\leq j<N$. In particular, for any such $y>M$
we have that 
\[
\left|\frac{p_{n}\left(m\right)}{q_{n}\left(m\right)}\right|\leq\sum_{j=1}^{\infty}\left|\frac{q_{j-1}\left(1\right)}{q_{j-1}\left(m\right)}\cdot\frac{q_{j-1}\left(1\right)}{q_{j}\left(m\right)}\right|\leq\frac{\varepsilon}{N}N+\sum_{j=N}^{\infty}\left|\frac{q_{j-1}\left(1\right)}{q_{j-1}\left(m\right)}\cdot\frac{q_{j-1}\left(1\right)}{q_{j}\left(m\right)}\right|\leq\varepsilon+\sum_{j=N}^{\infty}\frac{1}{j^{3}}.
\]
Since $\sum_{1}^{\infty}\frac{1}{j^{3}}<\infty$ converges, we can
find $N$ large enough so that $\sum_{N}^{\infty}\frac{1}{j^{3}}\leq\varepsilon$
also, so together we get that for all $y$ big enough (independent
of $n$) we have $\left|\frac{p_{n}\left(y\right)}{q_{n}\left(y\right)}\right|\leq2\varepsilon$
which is what we wanted to prove.
\end{proof}
Finally, we combine all of the results to show that $\zeta\left(3\right)$
is irrational.
\begin{thm}
\label{thm:zeta-3-irrational}The number $\zeta\left(3\right)$ is
irrational.
\end{thm}

\begin{proof}
Consider the diagonal direction on the $\zeta\left(3\right)$ matrix
field where $m=n+1$. From \thmref{all-directions-converge} we have
that 
\[
\limfi n\frac{P\left(n,n+1\right)}{Q\left(n,n+1\right)}=\zeta\left(3\right).
\]

\textbf{\uline{The main idea:}}

Let us denote $Q_{n}=Q\left(n,n+1\right),\;P_{n}=P\left(n,n+1\right)$
and $\tilde{Q}_{n}=\frac{Q_{n}}{gcd\left(Q_{n},P_{n}\right)}$, $\tilde{P}_{n}=\frac{P_{n}}{gcd\left(Q_{n},P_{n}\right)}$
so that $\limfi n\frac{P_{n}}{Q_{n}}=\limfi n\frac{\tilde{P}_{n}}{\tilde{Q}_{n}}=\zeta\left(3\right)$.
Recall that in \thmref{improved-rationality-test} we showed that
if $\frac{P_{n}}{Q_{n}}$ is not eventually constant and $\left|Q_{n}\zeta\left(3\right)-P_{n}\right|=o\left(gcd\left(P_{n},Q_{n}\right)\right)$,
then $\zeta\left(3\right)$ is irrational. We will begin by showing
that the diagonal is also a polynomial continued fraction in disguise.
Once we know that we can use it to approximate the errors and denominators.

Setting $\lambda_{+}=\left(1+\sqrt{2}\right)^{4}$, we will first
show that given any $\varepsilon>0$ we have:
\[
\left.\begin{array}{c}
\left|\zeta\left(3\right)-\frac{P_{n}}{Q_{n}}\right|=O\left(\frac{1}{\left(\lambda_{+}-\varepsilon\right)^{2n}}\right)\\
Q_{n}=O\left(\left(n!\right)^{6}\left(\lambda_{+}+\varepsilon\right)^{n}\right)
\end{array}\right\} \;\Rightarrow\;\left|Q_{n}\zeta\left(3\right)-P_{n}\right|=O\left(\frac{\left(n!\right)^{6}\left(\lambda_{+}+\varepsilon\right)^{n}}{\left(\lambda_{+}-\varepsilon\right)^{2n}}\right)\sim\frac{\left(n!\right)^{6}}{\lambda_{+}^{n}}.
\]
We then use the well known result that $\lcm\left[n\right]=O\left(\left(e+\varepsilon\right)^{n}\right)$
(it follows from the prime number theorem, see \cite{apostol_introduction_1998})
together with \corref{factorial-reduction} to get that 
\[
\frac{\left(n!\right)^{6}}{e^{3n}}\sim\left(\frac{n!}{\lcm\left[n\right]}\cdot n!\right)^{3}\mid\gcd\left(P_{n},Q_{n}\right).
\]
Finally, since $20.08\sim e^{3}<\lambda_{+}=\left(1+\sqrt{2}\right)^{4}\sim33.97$,
we could choose $\varepsilon>0$ small enough to use the irrationality
test and show that $\zeta\left(3\right)$ is irrational.

\medskip{}

\newpage{}

\textbf{\uline{Step 1: Find recursion relation for \mbox{$Q_{n}$}:}}

With this main idea, we are left to find the growth rate of $Q_{n}$
and how fast $\left|\zeta\left(3\right)-\frac{P_{n}}{Q_{n}}\right|$
goes to zero. 

Using the coboundary condition on the matrix field, we get that
\begin{align*}
\left(\begin{smallmatrix}P_{n}\\
Q_{n}
\end{smallmatrix}\right)=\left(\begin{smallmatrix}P\left(n,n+1\right)\\
Q\left(n,n+1\right)
\end{smallmatrix}\right) & =\left[\prod_{k=1}^{n}M_{Y}\left(0,k\right)\right]\left[\prod_{k=0}^{n-1}M_{X}\left(k,n+1\right)\right]e_{2}\\
 & =\left[\prod_{k=1}^{n}M_{X}\left(k-1,k\right)M_{Y}\left(k,k\right)\right]e_{2},
\end{align*}
where 
\begin{align*}
M_{X}\left(k-1,k\right)M_{Y}\left(k,k\right) & =\left(\begin{smallmatrix}0 & 1\\
b\left(k\right) & f\left(k-1,k\right)-\bar{f}\left(k,k\right)
\end{smallmatrix}\right)\left(\begin{smallmatrix}\bar{f}\left(k,k\right) & 1\\
b\left(k\right) & f\left(k,k\right)
\end{smallmatrix}\right)\\
 & =\left(\begin{smallmatrix}b\left(k\right) & f\left(k,k\right)\\
f\left(k-1,k\right)b\left(k\right)\; & \left(f\bar{f}\right)\left(k,0\right)+f\left(k,k\right)f\left(k-1,k\right)-\left(f\bar{f}\right)\left(k,k\right)
\end{smallmatrix}\right)\\
 & =\left(\begin{smallmatrix}-k^{6} & 6k^{3}\\
-f\left(k-1,k\right)k^{6}\; & 6k^{3}f\left(k-1,k\right)-k^{6}
\end{smallmatrix}\right)=k^{3}\left(\begin{smallmatrix}-k^{3} & 6\\
-f\left(k-1,k\right)k^{3}\; & 6f\left(k-1,k\right)-k^{3}
\end{smallmatrix}\right).
\end{align*}

Fortunately, the last matrix is also a polynomial continued matrix
in disguise (coboundary equivalent). Indeed, setting $U\left(k\right)=\left(\begin{smallmatrix}0 & 6\\
1 & \left(6f\left(k-1,k\right)-k^{3}\right)\;
\end{smallmatrix}\right)$ we get 
\[
M\left(k\right):=U\left(k\right)^{-1}\left(\begin{smallmatrix}-k^{3} & 6\\
-f\left(k-1,k\right)k^{3}\; & 6f\left(k-1,k\right)-k^{3}
\end{smallmatrix}\right)U\left(k+1\right)=\begin{pmatrix}0 & -k^{6}\\
1 & 6f\left(k,k+1\right)-k^{3}-\left(1+k\right)^{3}
\end{pmatrix},
\]
and therefore 
\begin{align*}
\frac{1}{\left(n!\right)^{3}}\left(\begin{smallmatrix}P_{n}\\
Q_{n}
\end{smallmatrix}\right) & =U\left(1\right)\left[\prod_{k=1}^{n}M\left(k\right)\right]U\left(n+1\right)^{-1}e_{2}=\left(\begin{smallmatrix}0 & 6\\
1 & 5
\end{smallmatrix}\right)\left[\prod_{k=1}^{n}M\left(k\right)\right]e_{1}=\left(\begin{smallmatrix}0 & 6\\
1 & 5
\end{smallmatrix}\right)\left[\prod_{k=1}^{n-1}M\left(k\right)\right]e_{2}.
\end{align*}

Setting as usuall $\begin{pmatrix}Q'_{n}\\
P'_{n}
\end{pmatrix}=\prod_{k=1}^{n-1}M\left(k\right)e_{2}$, we get that $u_{n}:=\frac{Q_{n}}{\left(n!\right)^{3}}=Q'_{n}+5P'_{n}$,
and since both $P_{n}',Q_{n}'$ satisfy the same reccurence, then
so does $u_{n}$, and we get that 

\begin{align*}
u_{n+1} & =u_{n}\left(6f\left(n,n+1\right)-\left(n+1\right)^{3}-n^{3}\right)-u_{n-1}n^{6},
\end{align*}
where $u_{0}=\frac{Q_{0}}{0!^{3}}=1$ and $u_{1}=\frac{Q_{1}}{1!^{3}}=6f\left(0,1\right)-1=5$.
Denote
\begin{align*}
F\left(n\right) & =6f\left(n,n+1\right)-\left(n+1\right)^{3}-n^{3}=34n^{3}+51n^{2}+27n+5,
\end{align*}
so the recurrence can be written as $u_{n+1}=F\left(n\right)u_{n}-n^{6}u_{n-1}$.

It is also interesting to note that $\frac{P_{n}}{\left(n!\right)^{3}}$
satisfies the same recurrence and $\frac{P_{0}}{\left(0!\right)^{3}}=0,\;\frac{P_{1}}{\left(1!\right)^{3}}=6$,
so that
\[
\zeta\left(3\right)=\limfi n\frac{P_{n}}{Q_{n}}=\left(\begin{smallmatrix}0 & 6\\
1 & 5
\end{smallmatrix}\right)\left[\prod_{1}^{n}\left(\begin{smallmatrix}0 & -k^{6}\\
1 & F\left(k\right)
\end{smallmatrix}\right)\right]\left(0\right)=\left(\begin{smallmatrix}0 & 6\\
1 & 5
\end{smallmatrix}\right)\left(\KK_{1}^{\infty}\frac{-k^{6}}{F\left(k\right)}\right),
\]
or alternatively
\[
\frac{6}{\zeta\left(3\right)}-5=\KK_{1}^{\infty}\frac{-k^{6}}{F\left(k\right)}.
\]
\medskip{}

\newpage{}

\textbf{\uline{Step 2: Analyze the recurrence to find the growth
rate of \mbox{$Q_{n}$}:}}

By \corref{factorial-reduction} we know that $\left(n!\right)^{6}\mid Q_{n}$,
so that $v_{n}=\frac{Q_{n}}{\left(n!\right)^{6}}=\frac{u_{n}}{\left(n!\right)^{3}}$
are integers which satisfy
\[
v_{n+1}\left(n+1\right)^{3}=F\left(n\right)v_{n}-n^{3}v_{n-1},
\]
where $v_{0}=\frac{Q_{0}}{0!^{6}}=1$ and $v_{1}=\frac{Q_{1}}{1!^{6}}=5$.
Equivalently, we can write 
\[
v_{n+1}=\frac{F\left(n\right)}{n^{3}}v_{n}-\frac{n^{3}}{\left(1+n\right)^{3}}v_{n-1}.
\]
Taking the limit only for the coefficients, we get the ``limit''
recurrence $v_{n+1}'=34v_{n}'-v_{n-1}'$. This correspons to the quadratic
equation $x^{2}-34x+1=0$ with the roots 
\[
\lambda_{\pm}=\frac{34\pm\sqrt{1156-4}}{2}=\frac{34\pm24\sqrt{2}}{2}=17\pm12\sqrt{2}=\left(1\pm\sqrt{2}\right)^{4},
\]
so a standard computation shows that $\frac{v_{n}'}{v_{n-1}'}\to\lambda_{+}=\left(1+\sqrt{2}\right)^{4}$.
In the original recurrence with the nonconstant coefficients, the
same holds, but needs a bit more explanation. As with the standard
recurrence with constant coefficients, we expect the general solution
to behave like $v_{n}\sim\lambda_{+}^{n}$, though there is a specific
starting position for which $v_{n}\sim\lambda_{-}^{n}$. Since $\lambda_{-}=\left(1-\sqrt{2}\right)^{4}\sim0.03$
, this is highly unlikely to happen, since we deal with integer values.
More sepcifically, the first few elements in $v_{i}$ are $1,5,73,1445,33001,...$
which is an increasing sequence of positive integers, and since $\frac{F\left(n\right)}{n^{3}}\geq11$
for $n\geq3$, it is not hard to show by induction that 
\[
v_{n+1}=\frac{F\left(n\right)}{n^{3}}v_{n}-\frac{n^{3}}{\left(1+n\right)^{3}}v_{n-1}\geq11v_{n}-v_{n-1}\geq10v_{n},
\]
so at least we get that $v_{n}\geq10^{n}$ grows much faster than
the very special case of $\lambda_{-}^{n}$. This is enough to show
that for every $\varepsilon>0$ and for any $n$ large enough, we
have
\[
\left(\lambda_{+}-\varepsilon\right)^{n}\leq v_{n}\leq\left(\lambda_{+}+\varepsilon\right)^{n}.
\]
The rest of the details are standard computations, and we leave it
to the reader.

\medskip{}

\textbf{\uline{Step 3: Analyze the approximation error:}}

The sequence of $\frac{P_{n}}{\left(n!\right)^{3}},u_{n}=\frac{Q_{n}}{\left(n!\right)^{3}}$
are the numerators and denominators of the continued fraction $\KK_{1}^{\infty}\frac{-n^{6}}{F\left(n\right)}$.
Using \claimref{upper-bound} we get that for all $n$ large enough
\[
\left|\zeta\left(3\right)-\frac{P_{n}}{Q_{n}}\right|\leq\sum_{k=n}^{\infty}\frac{\left(k!\right)^{6}}{\left|u_{k}u_{k+1}\right|}=\sum_{k=n}^{\infty}\frac{1}{\left(k+1\right)^{3}\left|v_{k}v_{k+1}\right|}\leq\sum_{n}^{\infty}\frac{1}{\left(\lambda_{+}-\varepsilon\right)^{2k+1}}=O\left(\frac{1}{\left(\lambda_{+}-\varepsilon\right)^{2n}}\right).
\]

These are the growth rate for $Q_{n}=\left(n!\right)^{6}v_{n}$ and
the error for $\left|\zeta\left(3\right)-\frac{P_{n}}{Q_{n}}\right|$
that we needed in the beginning, thus completing the proof.
\end{proof}

\newpage{}

\section{\label{sec:On-future-fractions}On future fractions}

The main goal of this paper was to introduce this new mathematical
object of conservative matrix field, and as an application use it
to reprove Apery's result about the irrationality of $\zeta\left(3\right)$.
As can be seen in \secref{The-z3-case}, the final proof as it is
right now is very specific to the matrix field of $\zeta\left(3\right)$,
which has several nice properties, and doesn't hold for other examples
of matrix fields. However it might be possible that some of the results
hold in a more general setting.

While this irrationality result is already interesting by itself,
the conservative matrix field object also seems to have many interesting
properties. Among others, it is a natural generalization of quadratic
equations, and it involves a bit of noncommutative cohomology theory
in the form of cocycles and coboundaries.

So far, the conservative matrix fields that we managed to find where
$f,\bar{f}$ are polynomials of degree 4 or more seem to always be
degenerate, namely $a\left(x,y\right)=f\left(x,y\right)-\bar{f}\left(x+1,y\right)$
doesn't depend on $y$. This might be related to the fact that we
work over $2\times2$ matrix, which might bound the possible matrix
fields. Whether this is the case or not, this leads to several possible
interesting generalizations of this theory, which are standard in
the theory of continued fractions and in number theory in general.
\begin{enumerate}
\item While many of the results mentioned in this paper are true for general
continued fractions over $\CC$ (and even other fields), the irrationality
of $\zeta\left(3\right)$ relied heavily on the fact that the defining
polynomials $f,\bar{f}$ were in $\ZZ\left[x,y\right]$. This leads
naturally to the question of what happens when we use other integer
rings in algebraic extensions, e.g. $\ZZ\left[i\right]$ or $\ZZ\left[\sqrt{2}\right]$.
Both in the $\zeta\left(2\right)$ and $\zeta\left(3\right)$ matrix
fields case we can find in the background algebraic numbers of degree
$2$ (namely $1+i$ and $\zeta_{3}=e^{\frac{2\pi}{3}i}$ respectively).
This type of field extension, with the right definition of generalized
continued fraction might add many more interesting examples.
\item In the proof of the irrationality of $\zeta\left(3\right)$ we had
two main results that we needed to show. One was to find the error
rate and how fast it converges to zero, and the second was to find
$gcd\left(P_{n},Q_{n}\right)$ and hope that it grows to infinity
fast enough. As it is usually the case in number theoretic problems,
the first result lives in the standard Euclidean geometry, where we
needed to show that some sequence goes to zero in the $\left|\cdot\right|_{\infty}$
norm, and the second result can be seen as showing that the $p$-adic
norms of $\left|gcd\left(P_{n},Q_{n}\right)\right|_{p}$ all go to
zero as well. This suggests a more general approach where the matrix
field lives over the Adeles, and the convergence in the real and $p$-adic
places together prove irrationality.
\item All the results in this paper were for $2\times2$ matrices, and a
natural generalization would be by going to a higher dimension matrices.
There are many suggestions for what should be the generalization of
continued fractions to higher dimensions, however probably one of
the best approaches is to change the language all together from continued
fractions to lattices in $\RR^{d}$. The subject of lattices is well
studied in the literature with many connections to other subjects.
With this approach, the question should be what is the right way to
formulate the results about general continued fraction as results
on lattices, and what can we say in higher dimension.
\end{enumerate}
These three types of generalization of changing the field, the norm,
or the dimension, can also be combined. Of course, there are more
tools available already in ``standard'' $2\times2$ matrices over
the integers to study polynomial continued fraction. However, it seems
that the conservative matrix field holds some interesting structure
which might reveal itself to be very useful not only to prove results
about continued fractions, but to other subjects as well.
