%%%%%%%%%%%%%%%%%%%%%%%%%%%%%%%%%%%%%%%%%%%%%%%%%%%%%%%%%%%%%%%%%%%%%%%%%%%%%%%
%2345678901234567890123456789012345678901234567890123456789012345678901234567890
%        1         2         3         4         5         6         7         8

\documentclass[letterpaper, 10 pt, conference]{ieeeconf}  % Comment this line out if you need a4paper

%\documentclass[a4paper, 10pt, conference]{ieeeconf}      % Use this line for a4 paper

\IEEEoverridecommandlockouts                              % This command is only needed if 
                                                          % you want to use the \thanks command

\overrideIEEEmargins                                      % Needed to meet printer requirements.

%In case you encounter the following error:
%Error 1010 The PDF file may be corrupt (unable to open PDF file) OR
%Error 1000 An error occurred while parsing a contents stream. Unable to analyze the PDF file.
%This is a known problem with pdfLaTeX conversion filter. The file cannot be opened with acrobat reader
%Please use one of the alternatives below to circumvent this error by uncommenting one or the other
%\pdfobjcompresslevel=0
%\pdfminorversion=4

% See the \addtolength command later in the file to balance the column lengths
% on the last page of the document

% The following packages can be found on http:\\www.ctan.org
\usepackage{graphics} % for pdf, bitmapped graphics files
\usepackage{graphicx} % for pdf, bitmapped graphics files
%\usepackage{subfigure}
%\usepackage{caption}
% \usepackage{subcaption} % for subfigures
% \captionsetup[figure]{font=footnotesize} % To use the original size (subcaption increases the caption sizes). Spacing seems slightly different

% For subfigures with IEEE font specifications
\makeatletter
\let\MYcaption\@makecaption
\makeatother

\usepackage[font=footnotesize]{subcaption}

\makeatletter
\let\@makecaption\MYcaption
\makeatother

%\usepackage{epsfig} % for postscript graphics files
%\usepackage{mathptmx} % assumes new font selection scheme installed
%\usepackage{times} % assumes new font selection scheme installed
\usepackage{amsmath} % assumes amsmath package installed
\usepackage{amssymb}  % assumes amsmath package installed
\usepackage{algorithm, algorithmic} % for writing algorithms
\usepackage{bm}
\usepackage{xcolor}
\usepackage{hyperref}
\usepackage{dblfloatfix}
\usepackage{multirow}
\usepackage{cite} % For compressed ranges (works with IEEETran)
\usepackage[textsize=tiny]{todonotes}

% Uncomment only one of the next two lines to show/hoide \todo{} items
%\newcommand{\todo}[1]{{\color{red} #1}}
%\newcommand{\todo}[1]{}

% squeeze length
% \renewcommand{\baselinestretch}{0.98}

\title{\LARGE \bf
%A Mixed-Integer Programming Model for Multi-Agent Planning on Dynamic Topological Graphs
%A Mixed-Integer Programming Model for Covert Multi-Robot Reconnaissance (?)
%A Mixed-Integer Programming Model for Covert Multi-Robot Reconnaissance on Dynamic Topological Graphs
%Covert Multi-Robot Reconnaissance on Dynamic Topological Graphs via Mixed-Integer Programming
Multi-Robot Planning %for Covert Reconnaissance 
on Dynamic Topological Graphs using Mixed-Integer Programming
}


\author{Cora A. Dimmig$^{1,2}$, Kevin C. Wolfe$^{1}$, and Joseph Moore$^{1,2}$% <-this % stops a space
%\thanks{Funding comment...}% <-this % stops a space
\thanks{$^{1}$Johns Hopkins University Applied Physics Laboratory, Laurel, MD
	20723, USA.}%
\thanks{$^{2}$Department of Mechanical Engineering, Johns Hopkins University, Baltimore, MD 21218, USA.  Email: {\tt\small Cora.Dimmig@jhuapl.edu, Joseph.Moore@jhuapl.edu}}%
}

\begin{document}



\maketitle
\thispagestyle{empty}
\pagestyle{empty}


%%%%%%%%%%%%%%%%%%%%%%%%%%%%%%%%%%%%%%%%%%%%%%%%%%%%%%%%%%%%%%%%%%%%%%%%%%%%%%%%
\begin{abstract}

Planning for multi-robot teams in complex environments is a challenging problem, especially when these teams must coordinate to accomplish a common objective. 
In general, optimal solutions to these planning problems are computationally intractable, since the decision space grows exponentially with the number of robots. In this paper, we present a novel approach for multi-robot planning on topological graphs using mixed-integer programming. Central to our approach is the notion of a dynamic topological graph, where edge weights vary dynamically based on the locations of the robots in the graph. We construct this graph using the critical features of the planning problem and the relationships between robots; we then leverage mixed-integer programming to minimize a shared cost that depends on the paths of all robots through the graph. 
% To improve computational tractability, our formulation uses an objective function with a fully convex relaxation. %and also eliminates the exponential dependence on the number of robots.
To improve computational tractability, we formulated an objective function with a fully convex relaxation and designed our decision space around eliminating the exponential dependence on the number of robots. 
% In this paper, we present a novel approach for multi-robot planning using dynamic topological graphs and mixed-integer programming. 
% In particular, we construct our topological graph to capture the relationships between the most important features of the planning problem and leverage mixed-integer programming to minimize a cost which depends on the paths of all robots through the graph. 
% Central to our approach is the notion of a dynamic topological graph, where edge weights vary based on the locations of the robots in the graph, and a mixed-integer programming formulation that relies on an objective function with a convex relaxation to improve computational tractability. 
%
%In this paper, we present a novel approach for multi-robot planning using  mixed-integer programming to compute optimal solutions in seconds. 
%Central to our approach is the notion of a dynamic topological graph, which we construct using the critical features of the planning problem and the relationships between robots.
%The edge weights of our graph are dynamic and vary based on the locations of the robots in the graph.
%Our mixed-integer programming approach minimizes costs that depend on the paths of all robots through the graph. 
%We present a formulation of our objective function with a fully convex relaxation to improve computational tractability.
%
We test our approach on a multi-robot %covert 
reconnaissance scenario, where robots must coordinate to minimize detectability and maximize safety while gathering information. % of adversary positions. 
We demonstrate that our approach is able to scale to a series of representative scenarios %while remaining feasible for online computation 
and is capable of computing optimal %enabling 
coordinated strategic behaviors for autonomous multi-robot teams in seconds.

%Covert multi-robot reconnaissance is a challenging space requiring coordination across robots to minimize detectability, maximize safety, and efficiently achieve scenario goals. 
%In this work, we formulate the covert multi-robot reconnaissance problem and present a novel representation of the problem in the form of planning on dynamic topological graphs. 
%We derive a compact mixed-integer program with an objective function that has a convex relaxation to solve large scale problems, that would otherwise be computationally intractable, in seconds. 
%%Our proposed approach does not require approximations of the state space and the size of our model is independent of the number of robots. 
%We consider explicit collaboration between robots with our topological graphs evolving dynamically based on the positions of the robots. 
%This enables finding optimal paths for multi-robot teams to work together, resulting in complex tactical maneuvers for navigating through challenging environments.
%% to move through a danagerous environment and achieve complex objectives.
%We demonstrate our approach on a series of representative scenarios. 
%Our method develops coordinated strategic behaviors for autonomous multi-robot teams, while remaining fully interpretable and extensible to varying operational conditions.  



\end{abstract}


%%%%%%%%%%%%%%%%%%%%%%%%%%%%%%%%%%%%%%%%%%%%%%%%%%%%%%%%%%%%%%%%%%%%%%%%%%%%%%%%
\section{INTRODUCTION}

Achieving unified coordination in complex real-world environments is a fundamental challenge for multi-robot systems. The robots must often achieve dynamic task allocation under geometric and temporal constraints. The problem is further complicated if teaming is required in the presence of uncertainty, such as might arise from inaccurate motion models or imperfect communication. In many circumstances, the combinatorial complexity of these multi-robot planning problems leads to computational intractability. This is especially true as the number of robots increases. 

In this paper, we present a novel approach for multi-robot planning on topological graphs using mixed-integer programming (MIP). In particular, our approach leverages the notion of a \emph{dynamic} topological graph, where the edge weights vary with the state of the robot team. To construct our graphs, we use a problem-specific embedding that captures the important features of the planning problem. We then use MIP to formulate a compact optimization problem which calculates paths for multi-robot teams that collaborate to minimize a shared cost function. We employ a fully convex relaxation of our objective function to be able to compute solutions to challenging real world scenarios in seconds, overcoming the common computational limitations of MIP.
These rapid solve times have the potential to enable frequent online re-planning in evolving conditions.
Additionally, by using MIP, our results are interpretable and extensible, and we can calculate a solution with guaranteed optimality. 

%In this paper, we present a novel approach for multi-robot planning on dynamic topological graphs using mixed-integer programming (MIP). 
%Our algorithm calculates paths for multi-robot teams to explicitly collaborate to complete scenario objectives.
% Our approach relies on 
%We designed a dynamic topological graph structure that captures the key components of the planning problem and dynamically evolves with the state of the robot team. 
%Based on this representation of the problem, we use MIP to formulate a compact optimization problem. 
%We derive a fully convex relaxation of our objective function to be able to compute solutions to challenging real world scenarios in seconds, overcoming the common computational limitations of MIP.
%These rapid solve times enable frequent online re-planning in evolving conditions.
%Additionally, by using MIP, our results are fully interpretable and extensible and we calculate a solution with guaranteed optimality. 

The dynamic topological graph structure we propose also facilitates the transfer of problem specific elements to new application domains. For this reason, we believe our approach has the potential to be generally applicable to a broad set of complex scenarios.
In this work, we demonstrate our approach on a set of sample reconnaissance test cases, such as in Fig.~\ref{fig:built_side_problem}, where teams of autonomous robots will need to coordinate to maneuver through complex environments while minimizing detection. 
% Developing strong tactical behaviors requires considering
% We consider 
% the nature of robot interactions and team collaborations to reduce the risk of a scenario, such as by overseeing team member movements,
% moving in formations to stay cognizant of the environment, and moving in groups for redundancy and robustness to unforeseen circumstances. 
%
\begin{figure}[t]
	\vspace*{2mm}
	\centering
	\includegraphics[trim={0cm 1.2cm 0cm 1.2cm},clip, width=1.0\columnwidth]{figures/built_side_cropped}
	\caption{A depiction of our proposed dynamic topological graph applied to a reconnaissance problem. 
    The nodes are in forested regions of cover. Ten robot scouts start at node 1 with the goal of getting at least one robot to node 2, across the meadow. 
    The edges of the graph have cost for transitioning between nodes that is a function of their distance, detectability, and vulnerability. %This cost depends dynamically on the state of the robots in the graph. 
    This cost can be reduced by moving in teams and through overwatch conditions, where robots at a particular node can oversee the movement of robots along a corresponding edge to help mitigate some of the risk of traversing. Overwatch conditions are indicated with an arrow from the overwatch node pointing toward the monitored edge. 
    The robot team routes through the graph, solved for with our proposed method, represent a solution to this problem.
	}
	\label{fig:built_side_problem}
	\vspace*{-3mm}
\end{figure}
%
% Fig.~\ref{fig:built_side_problem} depicts an example scenario in a meadow environment from \cite{NatureManufacture}. 
% Our approach aims to minimize detection as robots move through a challenging environment. 
% We build a topological graph with nodes in regions of cover and edges weighted based on the detectability, vulnerability, and distance to traverse between the regions of cover. 
% The edges of the graph have costs for transitioning between nodes that depend dynamically on the state of the robots in the graph. 
% We introduce the concept of overwatch conditions where robots at a particular node can oversee the movement of robots along a corresponding edge to help mitigate some of the risk of traversing. 
% Additionally, we introduce metrics to reward robots moving together to enable the potential for movement in formations. 
% As seen in Fig.~\ref{fig:built_side_problem}, our approach results in robust multi-robot plans for robots to work together to safely traverse a complex environment.



% Achieving unified coordination in complex real-world environments is a fundamental challenge for multi-robot systems. 
% % Many current approaches to multi-robot planning problems consider resolving conflicts between robots working towards independent goals or to support existing plans. 
% Robots are often considered independently to achieve a set of goals. For example, multi-robot path planning and task allocation applications often consider mitigating conflicts between robot plans or assigning goals to robots in a team, but often do not consider any further coordination between robots for collaboration.  
% %Robots are often considered independently to achieve a set of goals, either through exclusively mitigating conflicts, planning for sub-teams independently, or in scheduling and task allocation applications where interactions are not expected. 
% Planning for a team of robots to collaborate in working toward a common goal introduces new challenges. 
% %Covert multi-robot reconnaissance introduces new challenges to multi-robot planning. 
% % Addressing these challenges is particularly important in the context of future military operations, where teams of autonomous robots will need to coordinate maneuvering through complex battlefield environments to serve as forward deployed reconnaissance agents. 
% % Addressing these challenges is particularly important in the context of %covert
% In this work, we consider addressing these challenges in the context of
% reconnaissance operations, where teams of autonomous robots will need to coordinate maneuvering through complex environments. 
% % Developing strong tactical behaviors requires considering a multitude of factors, including detectability, challenging terrain and vantage points, the ability to support friendly units through the concept known as overwatch, teaming to enable movement in formations, and ultimately reaching a set of desired goal points. 
% Developing strong tactical behaviors requires considering the nature of robot interactions and team collaborations to reduce the risk of a scenario, such as by overseeing team member movements, %through the concept known as overwatch, 
% moving in formations to stay cognizant of the environment, and moving in groups for redundancy and robustness to unforeseen circumstances. 

% \begin{figure}[t]
% 	% \vspace*{-3mm}
% 	\centering
% 	\includegraphics[trim={0cm 1.2cm 0cm 1.2cm},clip, width=1.0\columnwidth]{figures/built_side_cropped}
% 	\caption{A depiction of our proposed dynamic topological graph applied to a reconnaissance problem. 
%     The nodes are in forested regions of cover. Ten robot scouts start at node 1 with the goal of getting at least one robot to node 2, across the meadow. 
%     The edges of the graph have cost for transitioning between nodes that is a function of their distance, detectability, and vulnerability. This cost can be reduced based on overwatch (indicated with an arrow from the overwatch node pointing toward the monitored edge) and moving in teams. 
%     The robot team routes through the graph, solved for with our proposed method, represent a solution to this problem.
% 	}
% 	\label{fig:built_side_problem}
% 	\vspace*{-3mm}
% \end{figure}

% In this work, we present a novel approach for multi-robot planning using dynamic topological graphs and Mixed-Integer Programming (MIP).
% Our algorithm calculates paths for multi-robot teams to explicitly collaborate to complete scenario objectives.
% % (1) a formulation of the covert multi-robot reconnaissance problem, (2) it's conversion to a planning problem on a topological graph with dynamic weights, and (3) a Mixed-Integer Programming (MIP) optimization approach to solving this problem. 
% We use MIP to formulate a compact optimization problem 
% % Our MIP formulation is compact 
% with an objective function that has a fully convex relaxation, yielding solutions to challenging real world scenarios in seconds, overcoming the common computational limitations of MIP.
% %Our computationally tractable approach for large-scale, dynamic coordination 
% These rapid solve times enable frequent online re-planning in evolving conditions.
% % Many current approaches to multi-robot planning problems consider resolving conflicts between robots working towards independent goals or to support existing plans. 
% By using MIP, our results are fully interpretable and extensible and we find a solution with guaranteed optimality. 

% Our approach is based on dynamic topological graph structures, which allows problem specific elements to be transferred to new application spaces. 
% We believe our approach is generally applicable to a broad set of complex scenarios.
% We demonstrate our approach on a set of sample reconnaissance test cases. 

% Fig.~\ref{fig:built_side_problem} depicts a example scenario in a meadow environment from \cite{NatureManufacture}. 
% Our approach aims to minimize detection as robots move through a challenging environment. 
% Our approach utilizes graphs built with nodes often in regions of cover and edges weighted based on the detectability, vulnerability, and distance to traverse between the regions of cover. 
% The edges of the graph have costs for transitioning between nodes that depend dynamically on the state of the robots in the graph. 
% We introduce the concept of overwatch conditions where robots at a particular node can oversee the movement of robots along a corresponding edge to help mitigate some of the risk of traversing. We present metrics to reward robots moving together to enable the potential for movement in formations. Ultimately resulting in robust multi-robot plans.

% %This computationally tractable approach for large-scale, dynamic coordination enables frequent online re-planning as information is gathered. 
% %We believe our approach is generally applicable to a broad set of complex scenarios.


%%%%%%%%%%%%%%%%%%%%%%%%%%%%%%%%%%%%%%%%%%%%%%%%%%%%%%%%%%%%%%%%%%%%%%%%%%%%%%%%
\section{RELATED WORK}

Multi-robot coordination \cite{Yan2013, Verma2021} and cooperative multi-agent planning \cite{Torreno2017} %, such as for scheduling, task allocation, routing, or path planning, 
are computationally challenging spaces. % that have been %investigated through a wide variety of techniques and algorithms. 
%widely investigated.
%For multi-robot planning with explicit cooperation, we need an approach that can 
%(1) be solved end-to-end without dividing the problem into sub-problems since that could hinder opportunities for cooperation, 
%(2) be solved without significant simplifying assumptions for the approach to be generally applicable,
%(3) can be solved online to efficiently re-plan as the scenario and environment changes,
%(4) yield interpretable results that generalize to new problems and guarantee optimality,
%and (5) encourage explicit cooperation between robots towards a goal rather than considering the robots independently.
%The following explores some of the prominent methods for multi-robot planning.

\subsection{General Methods}

For multi-agent path finding, Conflict Based Search (CBS) methods \cite{Sharon2015} plan optimal paths for all agents and then resolve conflicts in the paths, % through a conflict tree, 
% ultimately considering each agents' goals independently. 
ultimately planning to avoid interactions between agents. 
%
Multi-agent task allocation approaches often consider robots working independently toward a shared objective, such as with auction-based methods \cite{Lagoudakis2004} and game theoretic approaches \cite{Park2021}. 
% 
We look to encourage explicit cooperation between robots towards a goal, which requires considering their interactions more directly than these approaches allow.

Multi-agent games require cooperation between agents to be considered, though often necessitate decomposing the problem into smaller local-games, as in \cite{Shishika2018}, %such as with the game theoretic approach in \cite{Shishika2018}, 
which hinders finding a fully optimal solution to the overarching problem.

%Game theoretic approaches have explored cooperation for multi-agent perimeter defense problems by decomposing the problem into smaller local-games \cite{Shishika2018}. 
%Additionally, for multi-robot task allocation, game theory is applied to revise task selections in response to changes in the environment in \cite{Park2021}.

%Auction-based methods have been considered as a distributed approach that scales well for multi-robot routing \cite{Lagoudakis2004}. Performance bounds \cite{Lagoudakis2005} and efficiency \cite{Schneider2015} of these approaches have been studied. 

%For multi-agent path finding, Conflict Based Search (CBS) was a significant development in the field \cite{Sharon2015}. CBS plans optimal paths for all agents and then resolves conflicts in the paths through a conflict tree. In \cite{Choudhury2022} the CBS framework is extended for dynamically allocating tasks under uncertainty. 

\subsection{Machine Learning}

Recently techniques utilizing Machine Learning (ML) have emerged,
%been applied to multi-agent planning problems, 
most commonly using Graph Neural Networks (GNNs) \cite{Zhou2020, Kortvelesy2021}, such as for 
% for applications such as 
scheduling problems \cite{Wang2022}, %\cite{Wang2020, Wang2022} 
% coordination using communication graphs \cite{Kortvelesy2021},
%
% Approaches are presented 
handling agent interactions \cite{Liu2020}, and considering coordination in uncertain and adversarial environments \cite{Zhou2021}.
%For handling agent interactions, methods like reported in \cite{Liu2020} investigate training an attention network for interactions between agents to simplify learning multi-agent policies. %In \cite{Li2022}, collaborative prediction units are used to investigate predicting future statuses of multiple correlated agents based on a collaborative graph. 
%
%The work in \cite{Zhou2021} considers coordination in uncertain and adversarial environments for risk-aware and resilient actions.
%
%
One of the largest challenges with ML techniques is generalization to scenarios that were not seen during training. Some recent work with Deep Reinforcement Learning have shown progress in this space \cite{Almasan2022, Paul2022}, however are still limited and these approaches cannot guarantee optimality.
%with a GNN \cite{Almasan2022} or with a Graph Capsule Convolutional Neural Network (GCAPCN), a class of GNNs, \cite{Paul2022}. 

Some work has been done toward DRL algorithms with optimality guarantees using Multi-agent Markov Decision Processes \cite{Qu2019}. However categorically, ML approaches have not been able to guarantee optimality, and while they aid in overcoming computational limitations of conventional approaches, they suffer from low interpretability. 

\subsection{Mixed-Integer Programming}

A Mixed-Integer Programming (MIP) problem is an optimization problem in which decision variables may be binary or integer valued. This enables efficient, interpretable problem formulations that can be solved to optimality. %The relaxation of a MIP removes these integrality constraints.
%
%MIP has been used for trajectory generation \cite{Kushleyev2013}, with exciting results from \cite{Marcucci2021} to find globally optimal paths in a large graph.

%MIP has been used for many multi-robot coordination applications. 
%
%Early work exploring the computational complexity of MIP for multi-vehicle cooperative control in an adversarial game is presented in \cite{Earl2007} showing the challenges of solving large problems.
%, detecting and controlling regions of interest in an unknown environment \cite{Atay2006}
%
In \cite{Yu2016}, complete algorithms for multi-robot path planning are reported with heuristics to improve computational performance. 
%
In these contexts, multi-robot path planning often considers planning for robots to achieve set goals without conflicts with other robots, but does not investigate the complexity of explicit collaboration between robots. 
%
Similarly, MIP is used in recharging or timed delivery scenarios, as in \cite{Mathew2013, Kamra2015}, 
which considers planning for delivery robots around task robots with fixed locations or known paths. 
%Considering only one side of collaborating robots significantly reduces the computational complexity. 
%which often consider the task robots to be fixed or on a known path and operating on a fixed graph. This significantly reduces the computational complexity for planning with the delivery robots. 
%
In our proposed approach we simultaneously optimize paths of collaborating robots. Explicit collaboration in our method introduces the dynamic element to our topological graphs and enables devising optimal plans across the entire team.

MIP is NP-complete, which can inherently lead to long solve times. 
Many MIP applications can accommodate longer solve times, such as for 
scheduling and task allocation \cite{Koes2005} and communication provisioning \cite{Flushing2017}. 
%MIP has been used to combine scheduling and task allocation, with path planning problems in \cite{Koes2005} and communication provisioning in \cite{Flushing2017}. In these applications, 
%longer MIP solve times can be accommodated. %, such as optimizing for up to an hour without reaching full optimality \cite{Flushing2017}. In \cite{Fatemi-Anaraki2023}, speed up constraints are considered to improve the optimality gap when constraining solve times to 120 seconds to enable dynamic scheduling. 
%
In \cite{Asfora2020}, the authors explore intercepting a moving target for short missions, though their approach requires computation offline for an optimal solution. 
%
In our application space, these long solve times would not be practical.
%
%Many exciting results have been shown with MIP, however, 
%MIP is NP-complete, which can inherently lead to long solve times. 
Fortunately, modern solving methods, e.g. as implemented in Gurobi \cite{GurobiOptimization2023}, and efficient problem formulations, e.g. as developed in \cite{Marcucci2021}, have helped improve these solve times drastically. 
%
In the work reported herein, we have used these results and formulated our problem with a cost function that has a fully convex relaxation. We demonstrate solving large scale, complex problems entirely to optimality in seconds, enabling online applications and re-planning as operational conditions evolve. 


%%%%%%%%%%%%%%%%%%%%%%%%%%%%%%%%%%%%%%%%%%%%%%%%%%%%%%%%%%%%%%%%%%%%%%%%%%%%%%%%
\section{PROBLEM STATEMENT}
\label{sec:problem_statement}

Our objective is to plan tactical maneuvers for a multi-robot team that minimize detection, maximize safe navigation between regions of cover, and give rise to tactical behaviors such as “bounding overwatch,” where robots alternate movement across dangerous areas to oversee the other teams movement and mitigate risk.
% To formulate this problem, 
We consider three foundational constructs for reducing the risk of traversing a path:
\begin{enumerate}
	\item \textbf{Overwatch conditions}: When one team is at a vantage point to oversee the movement of another team, they are said to be providing overwatch, which reduces the risk of movement for the traversing team.
	\item \textbf{Formations in areas of high vulnerability}: When an area to traverse is particularly dangerous, moving with a greater number of robots allows moving in a formation to increase awareness. 
	\item \textbf{General teaming}: Moving in teams offers a higher level of protection to the robots.
\end{enumerate}
We will refer to these constructs as overwatch, vulnerability, and teaming conditions, respectively.

To achieve risk-sensitive coordinated reconnaissance %while minimizing detectability, 
with a multi-robot team, we propose representing this problem as a topological graph with dynamic edge weights. 
Nodes in the graph represent regions of cover and/or essential locations, while edges represent traversable paths between the nodes. 
Our three foundational constructs set the weights on the edges in our graphs dynamically, based on the locations of the robots in the graph. 
The base edge weight is a function of detectability and distance. Then if a particular overwatch position is occupied, corresponding edges will have reduced weights. 
The cost of the edge is increased if a desired number of robots for a vulnerable edge is not met and edge cost is reduced based on the number of robots traversing the edge as an incentive for moving as a team.
%A graphical depiction of a sample problem can be found in Fig.~\ref{fig:built_side_problem}.
Fig.~\ref{fig:built_side_problem} depicts a sample problem using a dynamic topological graph.


\subsection{Combinatorial Considerations}

For a total number of robots/agents $n_A$ and total number of locations $n_L$ (nodes, $n_V$, and edges, $n_E$), all possible states of robots at locations in a graph for a particular time step can be enumerated as
\begin{align}
%n_L (n_L - 1) (2^{n_A - 1} - 1) + n_L. 
{n_L}^{n_A}
\label{eq:combinatorics}
\end{align}
For large numbers of robots and locations, (\ref{eq:combinatorics}) grows exponentially.
% This grows exponentially for larger numbers of robots and locations. % as seen in Table~\ref{tab:combinatorics}. 
%Two nodes connected by a single edge would be 3 locations. Fig.~\ref{fig:toy_problem_solution} is an example of 17 locations when both directions of the edges are counted. 
%
%\begin{table}[tbh]
%	\centering
%	\caption{Total Possible States in One Time Step}
%	\label{tab:combinatorics}
%	\begin{center}
%	\renewcommand{\arraystretch}{1.3}
%	\begin{tabular}{ c || c | c | c | c }
%		\multirow{2}{*}{\textbf{Locations, $n_L$}} & \multicolumn{4}{c}{\textbf{Robots, $n_A$}} \\
%%		\vspace*{0.1mm} \\
%		\cline{2-5}
%		& \textit{2} & \textit{5} & \textit{10} & \textit{20} \\
%		\hline \hline
%		\textit{3}	& 9		& 93	& 3,069 & 3,145,725 \\
%		\textit{17}	& 289	& 4,097	& 139,009 & 142,606,081 \\
%		\textit{50}	& 2,500	& 36,800	& 1,252,000 & 1,284,503,200 \\
%		\textit{100}	& 10,000	& 148,600 & 	5,059,000 & 5,190,441,400 \\
%		\textit{250}	& 62,500	& 934,000 & 	31,810,000 & 32,636,866,000 \\
%	\end{tabular}
%	\end{center}
%\end{table}
%
Thus, our problem would quickly become computationally intractable for methods that seek to apply dynamic programming in this discrete state space. Techniques that rely on using function approximation to overcome the curse of dimensionality would likely suffer from poor generalizability and explainability.

To improve computational tractability, 
we developed a compact MIP %Mixed-Integer Programming (MIP) 
formulation to solve the multi-robot planning problem. 
In our formulation, we were able to drastically reduce the size of the decision space. 
%Instead considering $n_L$ locations (which includes nodes and both directions of each edge for an undirected graph) and 
When considering $n_O$ overwatch conditions and $n_T$ time steps, our total number of variables scales by (\ref{eq:vars_scaling}), which is linear in the parameters. %as follows. 
\begin{align}
n_T (1 + n_L + 2 n_E + n_O). \label{eq:vars_scaling}
\end{align}
%we have $n_L$ integer variables, $1 + n_E$ binary variables, and $n_E + n_O$ continuous variables at each time step. 
%Our MIP formulation 
% Which grows linearly in the parameters. 
% With this formulation, we removed the dependence on the number of robots from our decision space. 
Additionally, we were able to remove the dependence on the number of robots from our decision space. In this work, we assume a homogeneous team of robots, though this approach could scale by the number of different types of robots for heterogeneous teams. 
%In an illustrative example, as shown in Fig.~\ref{fig:toy_problem_solution}, with 5 nodes, 12 directed edges, 4 overwatch conditions (considering each directed edge separately), and 10 robots, we have 46 total variables for one time step in comparison to the 139,009 total possible states using (\ref{eq:combinatorics}). %multiplied by 10 time steps. 
In this paper, we demonstrate that our MIP formulation stays tractable for larger graph sizes. Furthermore, our approach is easily explainable for interpreting the results and applicable to new problem spaces. 


\section{MIXED-INTEGER PROGRAMMING APPROACH}

The objective of our mixed-integer program is to position a group of robots at a set of target locations within a specified time horizon while minimizing time and cost due to traversing edges. 
In our scenario, the costs of the edges represent the risk associated with being detected or neutralized while traversing. % those edges. 
Using our foundational constructs from Section \ref{sec:problem_statement}, certain nodes will provide overwatch of particular edges; this condition reduces the cost of traversing that edge since traversal risk has been reduced. 
We consider reduced risk/cost when multiple robots are traversing a particular edge, and, for particularly vulnerable edges, we further incentivize more robots to move together. 

% \subsection{Mathematical Formulation}

Table~\ref{tab:parameters} lists the parameters we use to formulate the covert multi-robot reconnaissance problem
and construct our decision variables. 
%We construct our decision variables using these parameters. 
%
%To formulate the covert multi-robot reconnaissance problem, we consider a scenario with $n_A$ robots with a time horizon of $n_T$ time steps. 
%We consider a graph with $n_E$ edges and $n_V$ vertices. For undirected graphs, we consider each direction as a separate edge. A list of locations $L$ is composed of the edges and each vertex, $v$, written in the form $(v,v)$. Thus the total number of locations is $n_L = n_E + n_V$. 
%%This formulation assumes an undirected graph (such that both directions of an edge are traversable), though could be seamlessly applied to a directed graph, in which case $n_L = n_E + n_V$. \todo{Could write more generally throughout for a directed or undirected graph - currently assuming undirected}
%We consider $n_O$ overwatch conditions of node-edge pairs, where robots at node $l_o$ can provide overwatch to robots on edge $l_k$. 
%
%With these parameters and for location numbers $l \in [1, n_L]$, time steps $t \in [1, n_T]$, edges $e \in [1,n_E]$, and overwatch conditions $c \in [1, n_O]$, we construct our decision variables. 
Table \ref{tab:decision_vars} shows each group of decision variables, their type (integer, binary, or continuous), lower bounds (LB), upper bounds (UB), and descriptions.

% \begin{table}
% 	\centering
% 	\caption{MIP Covert Reconnaissance Parameters}
% 	\label{tab:parameters}
% 	\begin{center}
% 		\renewcommand{\arraystretch}{1.3}
% 		\begin{tabular}{ c | c | p{4.8cm} }
% 			%	\hline
% 			%	\multicolumn{6}{|c|}{Agent} \\
% 			%	\hline
% 			%Variable Group Name & 
% 			\textbf{Category} & \textbf{Var} & \textbf{Description} \\
% 			\hline
% 			\hline
% 			\multirow{8}{4em}{Problem Size}
% 			& $n_A$ & Number of agents/robots \\
% 			& $n_T$ & Number of time steps in time horizon \\ 
% 			& $n_O$ & Number of overwatch conditions \\
% 			& $n_V$ & Number of nodes/vertices \\
% 			& $n_E$ & Number of edges, both directions \\
% 			& $n_L$ & Number of locations ($n_V + n_E$) \\
% 			& $n_S$ & Number of start locations \\
% 			& $n_G$ & Number of goal locations \\
% 			\hline
% 			\multirow{12}{4em}{Indices and Sets} 
% 			& $t$ & Time step from $1$ to $n_T$ \\
% 			& $c$ & Overwatch condition from $1$ to $n_O$ \\
% 			& $v$ & Node/vertex number from $1$ to $n_V$ \\
% 			& $e$ & Edge number from $1$ to $n_E$ \\
% 			& $l$ & Location number from $1$ to $n_L$ \\
% 			& $l_e$ & Location number for edge $e$ \\
% 			& $l_s$ & Start location for $s$ from $1$ to $n_S$ \\
% 			& $l_g$ & Goal location for $g$ from $1$ to $n_G$ \\
% 			& $(l_o, l_k)$ & Overwatch condition: node $l_o$ and corresponding edge $l_k$ \\
% 			& $L$ & List of locations: edges, e.g. $(v_1, v_2)$, and vertices, e.g. $(v_1,v_1)$ \\
% 			\hline
% 			\multirow{4}{4em}{Cost of Traversing} 
% 			& $w_e$ & Base cost to traverse $e$ \\
% 			& $a_e$ & Minimum desired number of robots on $e$ \\ 
% 			& $m_e$ & Additional cost for robots on $e$ before $a_e$ \\
% 			& $r_e$ & Cost reduction on $e$ for robots over $a_e$ \\
% 			\hline
% 			\multirow{4}{4em}{Cost of Overwatch} 
% 			& $\omega_c$ & Benefit of overwatch for $c$ \\
% 			& $a_c$ & Number of robots for full overwatch for $c$ \\ 
% 			& $r_c$ & Reward for each additional agent for $c$ over $a_c$ \\
% 			\hline
% 			\multirow{2}{4em}{Start/Goal Locations} 
% 			& $n_{l_s}$ & Number of robots at start location $l_s$ \\
% 			& $n_{l_g}$ & Number of robots at goal location $l_g$ \\ 
% 		\end{tabular}
% 	\end{center}
%     \vspace*{4mm}
% \end{table}
% %
% \begin{table}[tbh]
% 	% \centering
% 	\caption{MIP Covert Reconnaissance Decision Variables}
% 	\label{tab:decision_vars}
% 	\begin{center}
% 		\renewcommand{\arraystretch}{1.3}
% 		\begin{tabular}{ c | c | c | c | p{3cm} }
% 			%	\hline
% 			%	\multicolumn{6}{|c|}{Agent} \\
% 			%	\hline
% 			%Variable Group Name & 
% 			\textbf{Var} & \textbf{Type} & \textbf{LB} & \textbf{UB} & \textbf{Description} \\
% 			\hline
% 			\hline
% 			%Location Occupancy & 
% 			$p_{l,t}$ & Integer & 0 & $n_A$ & Number of robots at location $l$ at time $t$ \\ 
% 			%	Overwatch & $o_{c,t}$ & Binary & 0 & 1 & Whether overwatch condition $c$ is being met at time $t$ \\
% 			%	Overwatch Agent & $\Omega_{c,t}$ & Integer & 0 & $n_A$ & Number of robots at node for overwatch condition $c$ at time $t$ \\ 
% 			%		Edge Used &
% 			$\phi_{e,t}$ & Binary & 0 & 1 & Whether robots are on edge $e$ at time $t$ \\
% 			%Time Used & 
% 			$\tau_{t}$ & Binary & 0 & 1 & Whether robots have moved at time $t$ \\
% 			%Cost of Traversing & 
% 			$C_{W_{e,t}}$ & Cont. & 0 & $\infty$ & Cost of traversing edge $e$ at time $t$ \\
% 			%Cost of Overwatch & 
% 			$C_{O_{c,t}}$ & Cont. & $-\infty$ & 0 & Cost of overwatch condition $c$ at time $t$
% 			%	\hline
% 		\end{tabular}
% 	\end{center}
%     % \vspace*{-3mm}
% \end{table}

\begin{table}
	\centering
	\caption{MIP Parameters}
	\vspace*{-2mm}
    \label{tab:parameters}
	\begin{center}
		\renewcommand{\arraystretch}{1.3}
		\begin{tabular}{ c | c | p{4.9cm} }
			%	\hline
			%	\multicolumn{6}{|c|}{Agent} \\
			%	\hline
			%Variable Group Name & 
			\textbf{Category} & \textbf{Var} & \textbf{Description} \\
			\hline
			\hline
			\multirow{8}{4em}{Problem Size}
			& $n_A$ & Number of agents/robots \\
			& $n_T$ & Number of time steps in the time horizon \\ 
			& $n_O$ & Number of overwatch conditions \\
			& $n_V$ & Number of nodes/vertices \\
			& $n_E$ & Number of edges, both directions \\
			& $n_L$ & Number of locations ($n_V + n_E$) \\
			& $n_S$ & Number of start locations \\
			& $n_G$ & Number of goal locations \\
			\hline
			\multirow{12}{4em}{Indices and Sets} 
			& $t$ & Time step from $1$ to $n_T$ \\
			& $c$ & Overwatch condition from $1$ to $n_O$ \\
			& $v$ & Node/vertex number from $1$ to $n_V$ \\
			& $e$ & Edge number from $1$ to $n_E$ \\
			& $l$ & Location number from $1$ to $n_L$ \\
			& $l_e$ & Location number for edge $e$ \\
			& $l_s$ & Start location for $s$ from $1$ to $n_S$ \\
			& $l_g$ & Goal location for $g$ from $1$ to $n_G$ \\
			& $(l_o, l_k)$ & Overwatch condition pair: node $l_o$ can overwatch corresponding edge $l_k$ \\
			& $L$ & List of locations: edges, e.g. $(v_1, v_2)$, and vertices, e.g. $(v_1,v_1)$ \\
			\hline
			\multirow{4}{4em}{Cost of Traversing} 
			& $w_e$ & Base cost to traverse $e$ \\
			& $a_e$ & Minimum desired number of robots on $e$ \\ 
			& $m_e$ & Additional cost for robots on $e$ before $a_e$ \\
			& $r_e$ & Cost reduction on $e$ for robots over $a_e$ \\
			\hline
			\multirow{3}{4em}{Cost of Overwatch} 
			& $\omega_c$ & Benefit of full overwatch for $c$ \\
			& $a_c$ & Number of robots for full overwatch for $c$ \\ 
			& $r_c$ & Reward for overwatch robots over $a_c$ for $c$ \\
			\hline
			\multirow{2}{4em}{Start/Goal Locations} 
			& $n_{l_s}$ & Number of robots at start location $l_s$ \\
			& $n_{l_g}$ & Number of robots at goal location $l_g$ \\ 
		\end{tabular}
	\end{center}
    % \vspace*{-1mm}
    \bigskip 
% \end{table}
%
% \begin{table}[tbh]
	\centering
	\caption{MIP Decision Variables (At Time $t$)}
	\label{tab:decision_vars}
    \vspace*{-2mm}
	\begin{center}
		\renewcommand{\arraystretch}{1.3}
		\begin{tabular}{ c | c | c | c | p{3.5cm} }
			%	\hline
			%	\multicolumn{6}{|c|}{Agent} \\
			%	\hline
			%Variable Group Name & 
			\textbf{Var} & \textbf{Type} & \textbf{LB} & \textbf{UB} & \textbf{Description} \\
			\hline
			\hline
			%Location Occupancy & 
			$p_{l,t}$ & Integer & 0 & $n_A$ & Number of robots at location $l$ \\ % at time $t$ \\ 
			%	Overwatch & $o_{c,t}$ & Binary & 0 & 1 & Whether overwatch condition $c$ is being met at time $t$ \\
			%	Overwatch Agent & $\Omega_{c,t}$ & Integer & 0 & $n_A$ & Number of robots at node for overwatch condition $c$ at time $t$ \\ 
			%		Edge Used &
			$\phi_{e,t}$ & Binary & 0 & 1 & Whether robots are on edge $e$ \\ % at time $t$ \\
			%Time Used & 
			$\tau_{t}$ & Binary & 0 & 1 & Whether robots have moved \\ % at time $t$ \\
			%Cost of Traversing & 
			$C_{W_{e,t}}$ & Cont. & 0 & $\infty$ & Cost of traversing edge $e$ \\ % at time $t$ \\
			%Cost of Overwatch & 
			$C_{O_{c,t}}$ & Cont. & $-\infty$ & 0 & Cost of overwatch condition $c$ \\ % at time $t$
			%	\hline
		\end{tabular}
	\end{center}
    % \vspace*{-3mm}
    \bigskip \bigskip

% \begin{figure}[b]
%     \vspace*{-5mm}
% 	\centering
	\begin{subtable}[t]{0.49\columnwidth}
        \renewcommand{\thetable}{\arabic{table}} % NOTE: This is only working since table 2 is right above figure 2
        \footnotesize
		\centering
		\includegraphics[width=\textwidth]{figures/cost_of_traversing_thicker_lines.png}
        \vspace*{0mm}
        \captionsetup{font=scriptsize}
		\caption{Cost of traversing edge~$e$ at time~$t$ versus number of robots on the edge}
		\label{fig:cost_of_traversing}
	\end{subtable}
	\hfill \hfill
	\begin{subtable}[t]{0.49\columnwidth}
        \renewcommand{\thetable}{\arabic{table}} % NOTE: This is only working since table 2 is right above figure 2
        \footnotesize
		\centering
		\includegraphics[width=\textwidth]{figures/cost_of_overwatch_thicker_lines.png}
        \vspace*{0mm}
        \captionsetup{font=scriptsize}
		\caption{Cost of overwatch condition~$c$ at time~$t$ versus number of overwatch robots}
		\label{fig:cost_of_overwatch}
	\end{subtable}%1.4cm 1.4cm 1cm 0.5
    % \vspace*{-2mm}
	\captionof{figure}{Piecewise-Linear Cost Functions}
	\label{fig:pwl_costs}
% \end{figure}

\end{table}

%%%%%%%%%%%%%%%%%%%%%%%%%%%%%%%%%%%%%%%%%%
\subsection{Cost Function}
%%%%%%%%%%%%%%%%%%%%%%%%%%%%%%%%%%%%%%%%%%

Our goal is to minimize the overall cost of traversing and the total time to reach a set of target locations. 
%This is broken into piecewise-linear cost functions for variables $p_{l,t}$ and $\Omega_{c,t}$, and linear cost for $\tau_t$. 

\subsubsection{Cost of Traversing}

% \begin{figure}[b]
%     \vspace*{-5mm}
% 	\centering
% 	\begin{subfigure}[t]{0.49\columnwidth}
% 		\centering
% 		\includegraphics[width=\textwidth]{figures/cost_of_traversing_thicker_lines.png}
% 		\caption{Cost of traversing edge versus number of robots on the edge}
% 		\label{fig:cost_of_traversing}
% 	\end{subfigure}
% 	\hfill
% 	\begin{subfigure}[t]{0.49\columnwidth}
% 		\centering
% 		\includegraphics[width=\textwidth]{figures/cost_of_overwatch_thicker_lines.png}
% 		\caption{Cost of overwatch versus number of overwatch robots for a particular overwatch condition}
% 		\label{fig:cost_of_overwatch}
% 	\end{subfigure}%1.4cm 1.4cm 1cm 0.5
% 	\caption{Piecewise-Linear Cost Functions}
% 	\label{fig:pwl_costs}
% \end{figure}

Using the parameters from Table~\ref{tab:parameters}, we consider a base cost to traverse an edge $e$ as a positive $w_e$. We encode vulnerability information for that edge by specifying a minimum desired number of robots $a_e$ on the edge and additional positive cost $m_e$ for each robot until the minimum is met.
%For each edge location $l_e$, a base cost of traversing the edge $e$ is specified as a positive $w_e$. Additionally, we specify a minimum desired number of robots $a_e$ on the edge and additional positive cost $m_e$ for each robot until the minimum is met. This is used to encode vulnerability information of the edge. 
More robots on a vulnerable edge would enable moving in a formation for greater awareness. 
%It may be desirable for a more vulnerable edge to include a greater number of robots, such that the team can move in a formation. 
As an incentive for further teaming, for each robot over the minimum desired, an additional reduction can be specified as a positive $r_e$. Thus, as depicted in Fig.~\ref{fig:pwl_costs}a, the piecewise-linear cost of traversing an edge at a particular time, $C_{W_{e,t}}$, 
can be expressed in terms of the number of robots on the edge location at that time, $p_{l_e, t}$, as follows.
\begin{align}
C_{W_{e,t}} = \begin{cases}
0, & p_{l_e, t} = 0 \\
w_e + m_e (a_e - p_{l_e, t}), & 0 < p_{l_e, t} \leq a_e \\
w_e - r_e(p_{l_e, t} - a_e), & a_e \leq p_{l_e, t} \leq n_A
\end{cases}
\label{eq:pwl_traversing}
\end{align}

%\begin{tabular}{ c|c  }
%	%	\hline
%	%	\multicolumn{6}{|c|}{Robot} \\
%	%	\hline
%	Variable & Purpose \\
%	\hline
%	\hline
%	$e$ & Edge number \\
%	$w_e$ & Cost to traverse \\
%	$a_e$ & Minimum desired number of robots \\ 
%	$m_e$ & Additional cost for robots before $a_e$ \\
%	$r_e$ & Cost reduction for robots over $a_e$ \\
%	%	\hline
%\end{tabular}
%%
%\newline

When our integrality constraints are relaxed, this cost is not convex due to the zero point. 
We can restate this cost
% This can be written 
using perspective functions, as described in \cite{Marcucci2021}, and a variable for whether an edge is used, $\phi_{e,t}$, for the relaxation to be convex. This allows the first two cases to be combined. 
%Then the cost can be expressed as follows
\begin{align}
C_{W_{e,t}} = \begin{cases}
-m_e p_{l_e, t} + (w_e + m_e a_e) \phi_{e,t}, & 0 \leq p_{l_e, t} \leq a_e \\
w_e - r_e(p_{l_e, t} - a_e), & a_e \leq p_{l_e, t} \leq n_A
\end{cases}
\label{eq:pwl_traversing_convex}
\end{align}

When $p_{l_e, t} = 0$ then $\phi_{e,t} = 0$ by definition, and this cost is zero. Otherwise, for any $p_{l_e, t} > 0$ then $\phi_{e,t} = 1$ and the first case in (\ref{eq:pwl_traversing_convex}) is identical to the second case in (\ref{eq:pwl_traversing}). 

The cost in (\ref{eq:pwl_traversing_convex}) is convex when we select $m_e \geq r_e$ and relax our integrality constraints. Due to this convexity,
% Finally, since the piecewise-linear cost in (\ref{eq:pwl_traversing_convex}) is convex when integrality constraints are relaxed, 
we can equivalently express this cost in our optimization problem with a linear term in the cost function %(\ref{eq:cost_function}) 
that scales with our decision variables $C_{W_{e,t}}$ and two linear constraints. %, (\ref{eq:trav_constraint_1}) and (\ref{eq:trav_constraint_2}). 
\begin{align}
C_{W_{e,t}} &\geq - m_e p_{l_e, t} + (w_e + m_e a_e) \phi_{e,t} \label{eq:cost:trav1} \\
C_{W_{e,t}} &\geq - r_e p_{l_e, t} + (w_e + r_e a_e) \phi_{e,t} \label{eq:cost:trav2} 
\end{align}

Since we are optimizing for minimum cost, $C_{W_{e,t}}$ will be tight to the piecewise-linear cost function (\ref{eq:pwl_traversing_convex}). 

\subsubsection{Cost of Overwatch}

For each edge with an overwatch condition, we consider having overwatch while traversing to offer a cost reduction. When robots are traversing a corresponding edge, any number of robots at the overwatch node results in overwatch, but more robots may provide a greater reward. 
Using the parameters from Table~\ref{tab:parameters}, we specify a positive $\omega_c$ as the benefit of overwatch condition $c$, $a_c$ is the number of robots needed for full overwatch (i.e. to receive the full reward), and a positive $r_c$ is a further reward for additional robots past $a_c$ (i.e. a further teaming incentive). We consider $\Omega_{c,t}$ to be the number of overwatch robots for condition $c$ at time $t$. Thus, as depicted in Fig.~\ref{fig:pwl_costs}b, for a particular overwatch condition the ``cost'', $C_{O_{c,t}}$, is %can be expressed.
\begin{align}
C_{O_{c,t}} = \begin{cases}
-\frac{\omega_c}{a_c} \Omega_{c,t}, & 0 \leq \Omega_{c,t} \leq a_c \\
-\omega_c - r_c (\Omega_{c,t} - a_c), & a_c \leq \Omega_{c,t} \leq n_A
\end{cases}
\label{eq:pwl_overwatch}
\end{align}

%\begin{tabular}{ c|c }
%	%	\hline
%	%	\umn{6}{|c|}{Robot} \\
%	%	\hline
%	Variable & Purpose \\
%	\hline
%	\hline
%	$c$ & Overwatch condition \\
%	$\omega_c$ & Benefit of overwatch \\
%	$a_c$ & Number of robots for full overwatch \\ 
%	$r_c$ & Reward for each additional robot over $a_c$ \\
%	%	\hline
%\end{tabular}
%\newline

%Note there currently is not any cost for staying at a vertex. 

This ``cost'' will always be negative since it is rewarding overwatch. Similar to the cost for traversing, since this piecewise-linear cost is convex when integrality constraints are relaxed and $\omega_c / a_c \geq r_c$, we can express the cost, $C_{O_{c,t}}$, linearly in our cost function %(\ref{eq:cost_function}) 
with %the following 
two linear constraints. 
%
%\begin{align}
%C_{O_{c,t}} \geq -\omega_c (\Omega_{c,t} / a_c) \\
%C_{O_{c,t}} \geq -\omega_c - r_c (\Omega_{c,t} - a_c)
%\label{eq:overwatch_constr_initial}
%\end{align}
%
Additionally, we can remove the overwatch robot variables, $\Omega_{c,t}$, in these constraints. 
We consider the number of robots providing overwatch to be equal to the number of robots at the node, $l_o$, if there are robots on the edge, $l_k$.  
%To represent this, in (\ref{eq:overwatch_constr_initial}), 
We set $\Omega_{c,t} = p_{l_o, t}$, resulting in the following constraints.
%, resulting in constraints (\ref{eq:overwatch_constraint_1}) and (\ref{eq:overwatch_constraint_2}) in our optimization problem. %, and add another constraint with a steeper slope. 
%
\begin{align}
C_{O_{c,t}} &\geq - \frac{\omega_c}{a_c} p_{l_o, t} \label{eq:cost:overwatch1} \\
C_{O_{c,t}} &\geq - \omega_c - r_c (p_{l_o, t} - a_c) 
\label{eq:cost:overwatch2}
\end{align}
%
Then we add another constraint dependent on $p_{l_k, t}$, the number of robots on the corresponding edge, to ensure there are robots traversing the this edge. 
\begin{align}
C_{O_{c,t}} \geq - \frac{\omega_c}{a_c} n_A p_{l_k, t} 
\label{eq:cost:overwatch3}
\end{align}
This constraint %, (\ref{eq:overwatch_constraint_3}), 
has a steeper slope (due to scaling by $n_A$) than
% (\ref{eq:overwatch_constraint_1}) and (\ref{eq:overwatch_constraint_2}) 
(\ref{eq:cost:overwatch1}) and (\ref{eq:cost:overwatch2})
when there are robots on the edge ($p_{l_k, t}~>~0$) to assure it is the least restrictive. %, which we can guarantee for convex (\ref{eq:pwl_overwatch}). 
%with a steeper slope, (\ref{eq:overwatch_constraint_3}), we scale $p_{l_k, t}$ by $n_A$ to assure it is the least restrictive constraint when $p_{l_k, t} > 0$. We can guarantee it is the least restrictive, since the piecewise-linear cost function in (\ref{eq:pwl_overwatch}) is convex.
For $p_{l_o, t} > 0$, %the last constraint in (\ref{eq:overwatch_constr_full}) 
%(\ref{eq:overwatch_constraint_3}) 
(\ref{eq:cost:overwatch3})
is the most restrictive overwatch constraint when $p_{l_k, t} = 0$, forcing the cost $C_{O_{c,t}}$ to be zero, since there is not an overwatch benefit when there are not robots traversing the edge. Similarly, when $p_{l_o} = 0$, $C_{O_{c,t}} = 0$ since there are not robots providing overwatch. 

\subsubsection{Cost of Time}

The final cost, $C_T$, is for minimizing the time to achieve the goal.
We formulate a cost that scales with time for each time step robots are moving and thus rewards achieving the goal as quickly as possible. This cost uses the binary decision variables $\tau_{t}$ that represent whether robots have moved at time $t$. 
\begin{align}
C_T = \sum_{t = 1}^{n_T} t \tau_t
\end{align}

%This ``cost'' is a reward for each time step robots are at the goals. 
%\begin{align}
%C_T = - \frac{n_T}{2} \sum_{g = 1}^{n_G} \sum_{t = 1}^{n_T} \frac{p_{l_g, t}}{t}
%\end{align}
%For each goal location $l_g$, the number of robots at the goal is summed over time. This is a reward for each time step robots are at the goal nodes. 
%Additionally, this reward is an incentive for more robots to reach the goal, as a way to encourage teaming. 
%We scale this reward by the inverse of the time step so that the reward is greater for earlier time steps and we add a scaling factor $\frac{n_T}{2}$ to increase the magnitude of this cost.

\subsubsection{Overall Cost}

The overall cost to minimize 
%is expressed in (\ref{eq:cost_function}).
can be expressed as follows.
\begin{align}
C = C_T + \sum_{t = 1}^{n_T} \bigg( \sum_{e = 1}^{n_E} C_{W_{e,t}} + \sum_{c = 1}^{n_O} C_{O_{c,t}}\bigg) \label{eq:overall_cost}
\end{align}
These terms could be weighted depending on the priority of minimizing traversing cost versus minimizing time. 

%%%%%%%%%%%%%%%%%%%%%%%%%%%%%%%%%%%%%%%%%%
\subsection{Constraints}
%%%%%%%%%%%%%%%%%%%%%%%%%%%%%%%%%%%%%%%%%%

We add constraints to set support variables used in our cost functions and restrict movement to the %dynamic
topological graph.

\subsubsection{Edge Used Variables}

The following constraint sets
%Constraint (\ref{eq:constraint_edge_used_vars}) is used to set
the variables tracking if an edge is used, $\phi_{e,t}$, to true if there are robots on that edge, and false otherwise. This constraint assumes that setting this variable to true will never reduce the cost. 
%It is formulated similarly to the time variables constraint and is (\ref{eq:constraint_edge_used_vars}) in the optimization problem. 
For each time step $t$ and edge $e$, 
\begin{align}
\phi_{e,t} \geq \frac{1}{n_A} p_{l_e,t}. \label{eq:constr:edge_used}
\end{align}

\subsubsection{Time Variables}

We add binary time variables, $\tau_t$, to track if there are robots on the edges of the graph. Assuming the edges all have weight, %in (\ref{eq:constraint_time_vars}), 
we sum the number of robots on the edges for each time step to track whether robots are still moving since robots cannot go instantaneously between nodes. 
This constraint sets $\tau_t = 1$ if the sum is nonzero and $\tau_t = 0$ otherwise, since this variable contributes to increasing the cost.
For each time step $t$, %(\ref{eq:constraint_time_vars}) sets 
\begin{align}
\tau_t \geq \frac{1}{n_A} \sum_{e = 1}^{n_E} p_{l_e,t}. \label{eq:constr:time_vars}
\end{align}
%This constraint sets 
% to the cost function and thus the algorithm is incentivized to minimize cost). 

\subsubsection{Start Locations}

We add constraints for the start locations of each robot. For each start location $l_s$, with $n_{l_s}$ robots at that location, we set %the following constraint is added. 
\begin{align}
p_{l_s, 1} = n_{l_s} \label{eq:constr:start}
\end{align}
%constraint (\ref{eq:constraint_start}) is added.
% $p_{l_s, 1} = n_{l_s}$.

\subsubsection{Goal Locations}

For each goal location $l_g$ which requires at least $n_{l_g}$ robots, we set
%the following constraint is added. 
\begin{align}
p_{l_g, n_T} \geq n_{l_g} \label{eq:constr:goal}
\end{align}
%constraint (\ref{eq:constraint_goal}) is added.
% $p_{l_g, n_T} \geq n_{l_g}$.

\subsubsection{Maximum Robots}

%Constraint (\ref{eq:constraint_max_robots})
To ensure that only the maximum number of robots can exist across all locations at a particular time, 
for each time $t$, we set
%the following constraint is added.
\begin{align}
\sum_{l = 1}^{n_L} p_{l,t} = n_A. \label{eq:constr:max_robots}
\end{align}

\subsubsection{Sequential Flow}

For each node, the number of robots in the node and flowing into the node must be equal to the number of robots in the node and flowing out of the node in the next time step.
%, as expressed in (\ref{eq:constraint_sequential}). 
For each time $t \in [2, n_T]$ and node $v$, 
%this is expressed in (\ref{eq:constraint_sequential}).
\begin{align}
\sum_{\substack{l_\alpha | (u,v) \in L}} p_{l_\alpha,t-1} = \sum_{\substack{l_\beta | (v,u) \in L}} p_{l_\beta,t} \label{eq:constr:sequential}
\end{align}
The first sum %in (\ref{eq:constraint_sequential}) 
considers all locations of the form $(u,v)$ and the second sum considers all locations of the form $(v,u)$ for a fixed node $v$, where $l_\alpha$ and $l_\beta$ are the corresponding location numbers. 
%The second sum considers all locations of the form $(v,u)$ for a fixed node $v$, where $l_\beta$ is the corresponding location number. 
This constraint assumes a robot would never stay on an edge (due to the cost). This is enforced by the cost function as long as the cost of all edges is greater than zero and overwatch is never provided from an edge. 
This constraint allows robots to move from one edge to the next without stopping at the node for a time step.

%%%%%%%%%%%%%%%%%%%%%%%%%%%%%%%%%%%%%%%%%%
\subsection{Optimization Problem}
\label{sec:optimization_problem}
%%%%%%%%%%%%%%%%%%%%%%%%%%%%%%%%%%%%%%%%%%

Combining our objective function and constraints from the previous sections, our overall MIP optimization problem is expressed in Table~\ref{tab:optimization_problem}.
%can be expressed as follows. 
%We will isolate each component in the following sections. 

% \begin{align}
% \min \quad & C_T + \sum_{t = 1}^{n_T} \bigg( \sum_{e = 1}^{n_E} C_{W_{e,t}} + \sum_{c = 1}^{n_O} C_{O_{c,t}}\bigg) \label{eq:cost_function} \\
% \textrm{s.t.} \multirow{5}{*}{\rotatebox{90}{Cost Constraints~}} \quad 
% & C_{W_{e,t}} \geq - m_e p_{l_e, t} + (w_e + m_e a_e) \phi_{e,t}, ~\forall e, t \label{eq:trav_constraint_1}\\
% & C_{W_{e,t}} \geq - r_e p_{l_e, t} + (w_e + r_e a_e) \phi_{e,t}, ~\forall e, t \label{eq:trav_constraint_2}\\
% & C_{O_{c,t}} \geq - \frac{\omega_c}{a_c} p_{l_o, t}, ~\forall c, t \label{eq:overwatch_constraint_1}\\
% & C_{O_{c,t}} \geq - \omega_c - r_c (p_{l_o, t} - a_c), ~\forall c, t \label{eq:overwatch_constraint_2}\\
% & C_{O_{c,t}} \geq - \frac{\omega_c}{a_c} n_A p_{l_k, t}, ~\forall c, t \label{eq:overwatch_constraint_3} \\
% \cline{2-3}
% \multirow{6}{*}{\rotatebox{90}{Constraints\quad\quad\quad}} \quad 
% & \phi_{e,t} \geq \frac{1}{n_A} p_{l_e,t}, ~\forall e, t \label{eq:constraint_edge_used_vars} \\
% & \tau_t \geq \frac{1}{n_A} \sum_{e = 1}^{n_E} p_{l_e,t}, ~\forall t \label{eq:constraint_time_vars} \\
% & p_{l_s, 1} = n_{l_s}, ~\forall l_s \label{eq:constraint_start}\\
% & p_{l_g, n_T} \geq n_{l_g}, ~\forall l_g \label{eq:constraint_goal}\\
% & \sum_{l = 1}^{n_L} p_{l,t} = n_A, ~\forall t \label{eq:constraint_max_robots}\\
% & \sum_{\substack{l_\alpha | (u,v) \in L}} p_{l_\alpha,t-1} = \sum_{\substack{l_\beta | (v,u) \in L}} p_{l_\beta,t}, ~\forall v, ~t \in [2, n_T] \label{eq:constraint_sequential}
% \end{align}

\begin{table}[b]
\centering
    \vspace*{-2mm}
	\caption{MIP Optimization Problem}
    \vspace*{-4mm}
	\label{tab:optimization_problem}
	\begin{center}
		\renewcommand{\arraystretch}{2.0}
		\begin{tabular}{ l p{4.8cm} p{1.35cm} | c }
			%	\hline
			%	\multicolumn{6}{|c|}{Agent} \\
			%	\hline
			%Variable Group Name & 
			\multicolumn{3}{c|}{\textbf{Optimization Problem}} & \textbf{Eq.} \\
			\hline
			\hline
			\multicolumn{3}{l|}{$\min~C_T + \sum\limits_{t = 1}^{n_T} \bigg( \sum\limits_{e = 1}^{n_E} C_{W_{e,t}} + \sum\limits_{c = 1}^{n_O} C_{O_{c,t}}\bigg)$ subject to} & (\ref{eq:overall_cost})\\
            \hline
            \multirow{5}{*}{\rotatebox{90}{Cost Constraints~}} 
            & $C_{W_{e,t}} \geq - m_e p_{l_e, t} + (w_e + m_e a_e) \phi_{e,t},$ &$\forall e, t$ & (\ref{eq:cost:trav1})\\
            & $C_{W_{e,t}} \geq - r_e p_{l_e, t} + (w_e + r_e a_e) \phi_{e,t},$ &$\forall e, t$ & (\ref{eq:cost:trav2})\\
            & $C_{O_{c,t}} \geq - \frac{\omega_c}{a_c} p_{l_o, t},$ &$\forall c, t $ & (\ref{eq:cost:overwatch1})\\
            & $C_{O_{c,t}} \geq - \omega_c - r_c (p_{l_o, t} - a_c),$ &$\forall c, t $ & (\ref{eq:cost:overwatch2})\\
            & $C_{O_{c,t}} \geq - \frac{\omega_c}{a_c} n_A p_{l_k, t},$ &$\forall c, t $ & (\ref{eq:cost:overwatch3})\\
            \hline
            \multirow{6}{*}{\rotatebox{90}{Constraints\quad\quad\quad}} 
            & $\phi_{e,t} \geq \frac{1}{n_A} p_{l_e,t},$ & $\forall e, t $ & (\ref{eq:constr:edge_used}) \\
            & $\tau_t \geq \frac{1}{n_A} \sum\limits_{e = 1}^{n_E} p_{l_e,t},$ &$\forall t $ & (\ref{eq:constr:time_vars}) \\
            & $p_{l_s, 1} = n_{l_s},$ &$\forall l_s $ & (\ref{eq:constr:start})\\
            & $p_{l_g, n_T} \geq n_{l_g},$ &$\forall l_g $ & (\ref{eq:constr:goal})\\
            & $\sum\limits_{l = 1}^{n_L} p_{l,t} = n_A,$ &$\forall t$ & (\ref{eq:constr:max_robots})\\
            & $\sum\limits_{l_\alpha | (u,v) \in L} p_{l_\alpha,t-1} = \sum\limits_{l_\beta | (v,u) \in L} p_{l_\beta,t},$ &$\forall v, \newline t \in [2, n_T]$ & (\ref{eq:constr:sequential}) \\
            % & & \\
            % & & \\
			%	\hline
		\end{tabular}
	\end{center}
    \vspace*{-3mm}
\end{table}

By capturing the piecewise-linear cost structures with linear constraints due to the convexity of the costs (with relaxed integrality constraints), we can use mixed-integer linear programming to solve our problem since we have a linear objective function with linear constraints. 

%%%%%%%%%%%%%%%%%%%%%%%%%%%%%%%%%%%%%%%%%%
\subsection{Cost Formulation Considerations}
%%%%%%%%%%%%%%%%%%%%%%%%%%%%%%%%%%%%%%%%%%

We assume a graph without negative cycles.
Overwatch for one edge can come from multiple nodes, but the weight of the edge (resulting from the cost of traversing, benefit of overwatch, and cost reductions from vulnerability/teaming) cannot reduce to 0 or below. 
The formulation proposed herein allows robots at one node to overwatch robots traversing multiple edges. 
%
Intuitively, for an overwatch position to be used, the cost reduction from overwatch needs to be more than the cost to get to the overwatch position. % for it to be used. 
Otherwise, %the algorithm will keep all 
the robots will stay together for teaming benefits.

While our solutions are always guaranteed to be optimal, the optimal solution is not guaranteed to be unique. Small problems with simple numbers can result in 
%a large quantity of solutions with 
many equivalent cost solutions and cause computation time to increase. 
For example, with teaming, without any other factors such as vulnerability and overwatch, if the $r_e$ cost reductions are equivalent on all edges and there are multiple routes being taken, more robots will go on the longest route, which is logical from a risk mitigation perspective. If route lengths are equal, multiple optimal solutions will arise. More realistic values for the cost parameters will diminish these challenges. In the future, we plan to use reinforcement learning to generate appropriate weights for operational scenarios.


%Adding more robots need to be sure weights cannot reduce below 0

%Etc.

%Directed and undirected graphs, without negative cycles

%Negative weights are handled through validation checks when setting the initial weights. 


%%%%%%%%%%%%%%%%%%%%%%%%%%%%%%%%%%%%%%%%%%%%%%%%%%%%%%%%%%%%%%%%%%%%%%%%%%%%%%%%
\section{COMPUTATIONAL RESULTS AND DISCUSSION}

For scenarios of interest, we create topological graphs and solve our MIP optimization problem from Table~\ref{tab:optimization_problem}. We then process this solution through an assignment routine to determine paths for each robot through the graph. In this section, we present a few sample scenarios. 

\subsection{Illustrative Example}

\begin{figure}
    \vspace*{2mm}
	\centering
	\begin{subfigure}[t]{0.65\columnwidth}
		\centering
		%\fbox{}
		\includegraphics[width=\textwidth]{figures/toy1/graph_1_scene}
		\caption{Time = 0, Total Cost = 0}
		\label{fig:toy_problem_solution_0}
	\end{subfigure}%
	
	%	\hfill
	\begin{subfigure}[t]{0.5\columnwidth}
		\centering
		\includegraphics[trim={1.7cm 1.9cm 0.9cm 0.3cm},clip, width=\textwidth]{figures/toy1/toy1_reduction23True_NotRelaxed_1}
		\caption{Time = 1, Total Cost = 11}
		\label{fig:toy_problem_solution_1}
	\end{subfigure}%
	%	\hfill
	\begin{subfigure}[t]{0.5\columnwidth}
		\centering
		\includegraphics[trim={1.7cm 1.9cm 0.9cm 0.3cm},clip, width=\textwidth]{figures/toy1/toy1_reduction23True_NotRelaxed_2}
		\caption{Time = 2, Total Cost = 65}
		\label{fig:toy_problem_solution_2}
	\end{subfigure}
	
	\begin{subfigure}[t]{0.5\columnwidth}
		\centering
		\includegraphics[trim={1.7cm 1.9cm 0.9cm 0.3cm},clip, width=\textwidth]{figures/toy1/toy1_reduction23True_NotRelaxed_3}
		\caption{Time = 3, Total Cost = 131}
		\label{fig:toy_problem_solution_3}
	\end{subfigure}%1.4cm 1.4cm 1cm 0.5
	%	\hspace{.05\textwidth}%
	\begin{subfigure}[t]{0.5\columnwidth}
		\centering
		\includegraphics[trim={1.7cm 1.9cm 0.9cm 0.3cm},clip, width=\textwidth]{figures/toy1/toy1_reduction23True_NotRelaxed_4}
		\caption{Time = 4, Total Cost = 131}
		\label{fig:toy_problem_solution_4}
	\end{subfigure}%
	%	\begin{subfigure}[t]{0.32\textwidth}
	%		\centering
	%		\includegraphics[trim={1.4cm 2.0cm 0.5cm 0.5cm},clip, width=\textwidth]{figures/toy1/toy1_reduction23True_NotRelaxed_paths}
	%		\caption{Visualization of team paths}
	%		\label{fig:toy_problem_solution_paths}
	%	\end{subfigure}%
	\caption{MIP problem solution to an illustrative multi-robot reconnaissance scenario, sketched in 
    % Fig.~\ref{fig:toy_problem_solution_0}. 
    (a).
    Ten robots start at node 1 with a goal of at least one robot reaching node 5 to observe the adversary units. The light green regions represent areas of cover. The color of the edges indicate the threat level for transitioning between nodes from low to high (LTH): blue, yellow, red. Edge (3,5) has the highest weight due to its visibility by the adversary. 
	In each subplot, the edge labels indicate the edge cost under the current conditions and show in brackets the desired number of robots, $a_e$, due to vulnerability. Overwatch conditions are shown with a pink arrow from the overwatch node pointing to the edge that can be monitored. At each time step, the position of each robot scout is shown.}
	\label{fig:toy_problem_solution}
    \vspace*{-4mm}
\end{figure}

To demonstrate our results, we first consider the %illustrative 
example in Fig.~\ref{fig:toy_problem_solution} which portrays a simple reconnaissance scenario. %with 10 robots starting at node 1 and requiring at least one robot to reach node 5. 
%Fig.~\ref{fig:toy_problem_solution} steps through a solution to our MIP problem. 
In Fig.~\ref{fig:toy_problem_solution_0}, the edge costs are shown as the maximum cost for one robot to traverse without any overwatch, teaming, or vulnerability cost reductions. 
%The edge labels in each subplot show the edge cost and the number of robots desired in brackets ([]). The color of the edge indicates threat level blue (lowest), yellow, or red (highest). 
To encourage teaming, on each edge, each additional robot reduces the cost by 1. Both directions of edges (1,3), (3,5), and (4,5) are considered vulnerable and at least four robots are desired; the cost reduction for each robot up to four is 10. Overwatch conditions are indicated by the pink arrows. Overwatch can be provided from node 2 for both directions of edge (2,4) and node 3 for both directions of edge (4,5), which can reduce the cost by up to 20 or 60, respectively, when two robots are providing overwatch. %, and an additional 2 for each additional robot. 
Each additional robot providing overwatch after the first two would reduce the cost by 2. The dynamic costs incurred by each edge are updated in the graphs in Fig.~\ref{fig:toy_problem_solution} to reflect the teaming, vulnerability, and overwatch conditions being met. In this example we scale the time cost by 10 to encourage reaching the goal in minimal time. The total accumulated cost is tracked in the captions. 
%As a summary,
%Fig.~\ref{fig:toy_problem_solution_paths} visualizes the paths of each robot team. The base graph is shown in blue and time steps are labeled along the paths.

In this solution, we see the robots break into three teams to work together for a team to safely traverse to the target node. The robots all move together on edge (1,2) to get the largest benefit from teaming. Two robots then remain at node 2 to provide overwatch to the four robots traversing edge (2,4) and the remaining four continue toward node 3 to be prepared to provide overwatch in the next step. The greatest benefits from overwatch at node 2 and 3 are with 2 robots. Four robots (rather than two) go to node 3 to also get a larger teaming benefit on edge (2,3) and additional rewards for overwatch in the next step. It would have been equivalent cost for two of those robots to continue on with robots 4-7. In time step 3, robots 4-7 traverse a vulnerable edge, getting the largest reward for having four robots and enabling moving in a formation. In the last step, robots 4-7 reach the goal node. The cost does not increase since the robots moved to a node in this step and there is no longer time cost since there are not any robots remaining on edges. 

By solving our problem to optimality we are guaranteed to have minimized the total cost. However, this depends greatly on the weights/cost assigned to the problem and the prioritization of minimizing traversing cost versus minimizing time. 
For example, if we did not scale our time cost by 10, the resulting solution would keep more robots together to receive more teaming benefits for moving together and from additional overwatch cost reductions. As in Fig.~\ref{fig:toy_problem_solution_1}, all robots would go to node 2. Then the robots would split into only two teams: a team for overwatch that moves from node 2 to 3 and the team being monitored on edges (2,4) and (4,5). Ultimately requiring one additional time step. 

%, however that does not guarantee that the solution is unique. 
% Not a valid example since the time cost incentivizes more robots to reach the goal:
%However, there can be multiple optimal solutions with the same cost. For example, if at time 2 in Fig.~\ref{fig:toy_problem_solution_2}, two additional robots stayed at node 2, the maximum cost reductions for overwatch and vulnerability would still be met. At node 2, those robots would provide an additional overwatch reduction of 2 versus continuously on they would provide an additional 

\subsection{Bounding Overwatch Example}

\begin{figure}[tbh]
	\vspace*{-3mm}
	\centering
	\includegraphics[trim={0cm 0.3cm 0cm 0.5cm},clip, width=\columnwidth]{figures/graph_5_paths}
	\caption{Example solution demonstrating bounding overwatch behavior as all robots move from node 1 to 11. Robot team routes are shown with each time step labeled. The overwatch conditions, indicated by the pink arrows, are met at time steps 2, 3, 6, and 7. The two teams alternate providing overwatch as they move through the environment. 
	}
	\label{fig:problem4}
	% \vspace*{1mm}
\end{figure}


As a demonstration of a larger graph that particularly lends itself to the bounding overwatch paradigm, Fig.~\ref{fig:problem4} shows a solution to our MIP problem for the graph shown in blue. We set the start point for all robots to be node 1 and the goal for all robots is node 11. The solution yields two robot teams alternating traversing and providing overwatch. Each team's route is visualized with time steps to enable correlating the teams' positions and the bounding overwatch behavior. % as the teams alternate traversing and providing overwatch. 

\subsection{Real World Scenarios}

\begin{figure}[tbh]
	\vspace*{1.5mm}
	\centering
	\includegraphics[trim={1.9cm 0.4cm 1.9cm 0cm},clip, width=\columnwidth]{figures/built_aerial_cropped_larger_font_wider}
	\caption{Aerial map of a meadow environment showing a sample scenario. Nodes are in regions of cover. Vulnerable edges due to crossing roads are shown in dark blue. The largest cost reductions on these edges come from having four robots. In the solution routes shown, robot teams split up and form new teams as they move through the terrain providing overwatch. 
	}
	\label{fig:built_graph}
	\vspace*{-3mm}
\end{figure}

In Figures~\ref{fig:built_graph} and \ref{fig:built_side_problem}, we demonstrate applying this approach to complex real world scenarios. We consider traversing between areas of forested cover in a high risk environment, using the meadow environment from \cite{NatureManufacture} as a representative example. We show results with all robots reaching a designated goal and a subset of robots reaching the goal. %with goals for some or all robot scouts reaching a designated goal.

\subsubsection{Meadow Map 1} 

In Fig.~\ref{fig:built_graph}, we consider 10 robots starting at node 1 with a goal of all robots reaching node 2 within 10 time steps. We see three distinct paths emerge. Through time step 3, six robots maneuver together (orange and red teams) alternating providing overwatch with another team (purple) of four robots. At time step 4, two robots (red) remain at node 4 to provide overwatch for the purple team, while orange maneuvers toward the next overwatch position. This allows the purple and red teams to combine and proceed alternating overwatch with the orange team through the rest of the graph toward the goal node. 

\subsubsection{Meadow Map 2}

Fig.~\ref{fig:built_side_problem} is an example with further subdivision  
of the robots in the solution. Again we consider 10 robots starting at node 1. 
In this scenario, the goal is for at least one robot to reach node 2 within 12 time steps.
% This time with a goal of at least one robot reaching node 2 within 12 time steps (to accommodate the larger map). 
In this case, we see four teams emerge. One team of four (orange) that traverses through the map toward goal, providing overwatch as needed, and three smaller supporting teams (red, purple, brown) providing overwatch for orange and each other. Since all robots do not need to reach the goal node, this example highlights the trade off between the cost of traversing and the benefit of overwatch. Teams will not continue towards the goal if the overall cost of their movement is not less than the benefit of the overwatch they would provide to a primary team moving towards the goal. This concept is consistent with expectations in an operational setting: the risk outweighs the reward. Ultimately, this is the objective of our method, to be able to determine strategic tactical maneuvers %for reconnaissance 
in complex environments. 

\subsection{Ablation Study}

\begin{figure}
	\centering
	\begin{subfigure}[t]{0.5\columnwidth}
		\centering
		\includegraphics[trim={0cm 0.5cm 0cm 0cm},clip,width=\textwidth]{figures/ablation/wo_ovt_TeamingPWL_convex_constraints_wo_overwatch_vars/toy1_reduction23True_NotRelaxed_paths}
		\caption{Without overwatch, vulnerability, \\and teaming}
		\label{fig:toy1_without_overwatch_vul_teaming}
	\end{subfigure}%
	\begin{subfigure}[t]{0.5\columnwidth}
		\centering
		\includegraphics[trim={0cm 0.5cm 0cm 0cm},clip,width=\textwidth]{figures/ablation/wo_vt_TeamingPWL_convex_constraints_wo_overwatch_vars/toy1_reduction23True_NotRelaxed_paths}
		\caption{With overwatch and without vulnerability and teaming}
		\label{fig:toy1_without_vul_teaming}
	\end{subfigure}%1.4cm 1.4cm 1cm 0.5
	
	\begin{subfigure}[t]{0.5\columnwidth}
		\centering
		\includegraphics[trim={0cm 0.5cm 0cm 0cm},clip,width=\textwidth]{figures/ablation/wo_t_TeamingPWL_convex_constraints_wo_overwatch_vars/toy1_reduction23True_NotRelaxed_paths}
		\caption{With overwatch and vulnerability \\and without teaming}
		\label{fig:toy1_without_teaming}
	\end{subfigure}%
	\begin{subfigure}[t]{0.5\columnwidth}
		\centering
		\includegraphics[trim={0cm 0.5cm 0cm 0cm},clip,width=\textwidth]{figures/ablation/TeamingPWL_convex_constraints_wo_overwatch_vars/toy1_reduction23True_NotRelaxed_paths}
		\caption{Full solution (with overwatch, vulnerability, and teaming)}
		\label{fig:toy_problem_solution_paths}
	\end{subfigure}%
	
	\begin{subfigure}[t]{\columnwidth}
		\centering
		\includegraphics[width=\textwidth]{figures/ablation/horizontal_legend}
	\end{subfigure}%
	\caption{Ablation study illustrating the impact of each of our main constructs: overwatch conditions, edges with high vulnerability, and incentives for moving as a team. Robot team paths are shown in each subplot with components of our formulation incrementally added.}
	\label{fig:toy1_ablation}
    \vspace*{-3mm}
\end{figure}

We performed an ablation study to assess the impact of the 
% various components of our algorithm, specifically, the 
constructs we label overwatch, vulnerability, and teaming in our algorithm. Fig.~\ref{fig:toy1_ablation} shows the solutions to our illustrative example when removing these components and incrementally adding them back in. 
%For this we use formulations of our algorithm suited to the reduced complexity of the problem, this includes a time cost based on whether there are robots on the edges rather than scaling by the number of robots at the goal to remove any incentive for teaming. 
The problem is setup as in Fig.~\ref{fig:toy_problem_solution}, with all robots starting at node 1 and an overall goal of at least one robot reaching node~5. 

When we remove the overwatch, vulnerability, and teaming conditions, in Fig.~\ref{fig:toy1_without_overwatch_vul_teaming}, the problem essentially becomes a shortest path problem. The weights in the graph are fixed and one robot takes the path with the least overall cost due to the edge weights and time. It would be an equivalent cost solution for any number of robots to take this path. 

In tactical maneuvers, overwatch allows minimizing detection and maximizing safety. When we add our formulation of overwatch conditions in Fig.~\ref{fig:toy1_without_vul_teaming}, the edge weights now vary based on the overwatch conditions being met and the optimal solution includes robots moving to the two overwatch positions while one robot traverses to the goal, overall making the operation safer. 

As an incentive to travel in a formation on edges that are particularly dangerous, or vulnerable, we add our vulnerability condition back in Fig.~\ref{fig:toy1_without_teaming}, making edges (1,3), (3,5), and (4,5) more costly to traverse alone. We see in the solution that the desired minimum of 4 robots traverse edge (4,5). The robot providing overwatch at node 3 moves with other robots on edge (1,2) and then alone on edge (2,3). This is ultimately the same cost as if four robots were to traverse the vulnerable edge (1,3). Both solutions are optimal. As in the previous cases, the extra robots stay at the start node. 

In a tactical scenario, more robots moving together would provide more data, potentially reducing the overall threat level. Additionally, the redundancy of having more robots yields a more robust solution. 
When we add our teaming conditions back in,  Fig.~\ref{fig:toy_problem_solution_paths}, we see all robots moving for the first time, no longer leaving robots at the start node, since moving as a team provides cost reductions on each edge and greater cost reductions when providing overwatch. 
%there are greater cost reductions when more robots provide overwatch. 
Ultimately, the overwatch, vulnerability, and teaming conditions yield practical plans for tactical maneuvers that minimize detection and maximize robustness and safe navigation.

\subsection{Computation Time}

\begin{table}[tbh]
    \vspace*{3mm}
	\centering
	\caption{Example Graphs' Problem Size and Computation Time} %across Example Topological Graphs}
	\label{tab:computation_time}
    \vspace*{-2mm}
	\begin{center}
		\renewcommand{\arraystretch}{1.3}
		\begin{tabular}{ c || c | c | c | c }
			% & \multicolumn{4}{c}{\textbf{Topological Graph}} \\
			& \textbf{Illustr.} & \textbf{Bounding} & \textbf{Map 1} & \textbf{Map 2} \\
			\hline \hline
			%			\multicolumn{2}{c ||}{\textbf{Locations, $n_L$}} & 17 & 43 & 32 & 51 \\
			%			\multicolumn{2}{c ||}{\textbf{Overwatch, $n_O$}} & 6 & 8 & 18 & 32 \\
			%			\multicolumn{2}{c ||}{\textbf{Time Horizon, $n_T$}} & 10 & 10 & 10 & 12 \\
			%			\hline
			%			\multicolumn{2}{c ||}{\textbf{Total Variables}} & 470 & 1150 & 980 & 1860 \\
			%			\hline
			%			\multirow{4}{2.3em}{\textbf{Mean Solve Time \\(sec)}} & w/o O, V, T & 0.024 & 0.005 & 0.007 & 0.336 \\
			%			& w/o V, T & 0.088 & 0.092 & 0.262 & 2.060 \\
			%			& w/o T & 0.080 & 0.087 & 0.217 & 1.788 \\
			%			\cline{2-6}
			%			& Full & 0.081 & 0.100 & 0.249 & 3.702 \\	
			\textbf{Locations, $n_L$} & 17 & 43 & 32 & 51 \\
			\textbf{Overwatch, $n_O$} & 4 & 8 & 18 & 32 \\
			\textbf{Time Horizon, $n_T$} & 10 & 10 & 10 & 12 \\
			\hline
			\textbf{Total Variables} & 460 & 1160 & 990 & 1872 \\
			\hline
			\textbf{Mean Solve Time (s)} & 0.081 &	0.100 &	0.249 &	3.702 \\	
		\end{tabular}
	\end{center}
    \vspace*{-4mm}
\end{table}

The common limitation of MIP is computation time. The high level planning we propose would need to be computed rapidly (on the order of magnitude of minutes) for robots to be able to act on the plan and re-plan as new data is gathered. Table~\ref{tab:computation_time} shows the parameters that affect the problem size for the graphs discussed in this paper (number of locations, number of overwatch conditions, and time horizon), the total variables in our optimization problem due to those parameters, and the resulting solve time for our MIP problem in each case. 
The graphs are labeled Illustrative, Bounding, Map 1, and Map 2, which correspond to Figures~\ref{fig:toy_problem_solution} and \ref{fig:toy1_ablation}, Fig.~\ref{fig:problem4}, Fig.~\ref{fig:built_graph}, and Fig.~\ref{fig:built_side_problem}, respectively. 

We used the Gurobi optimizer \cite{GurobiOptimization2023} on an Intel® Core™ i7-10875H CPU @ 2.30GHz × 16 and averaged the solve times across 100 trials. In all cases the number of robots was 10. Varying the number of robots does not impact the total number of variables in our problem. %, which has a significant effect on solve time. %, but would impact the search space for~$p_{l,t}$ and the conditioning of the problem. 
%This results show solve times in seconds for realistic operational scenarios. 
%A greater number of robots could introduce negative cycles if the cost reductions of an edge are greater than the edge's original cost. The graphs are first evaluated for the potential for negative cycles before solving the MIP. 
%\todo{How scales with number of robots? Show examples? Or give a value of how much the computation time varies (mean, std?) as we change the number of robots?}

These results show computation times in seconds for realistic operational scenarios. Demonstrating that the problem stays computational tractable to   
%The solve times increase for the more realistic graphs, Map 1 and Map 2, however the problem still stays computationally tractable; 
solve complicated routes on large graphs for multi-robot teams. % in less than 4 seconds. 
% We expect these graph sizes and time horizons to be realistic for operational scenarios. 
% In even larger scenarios, if necessary, 
% Additionally, we can use a receding horizon approach to keep the time horizon small. 

%\subsubsection{Ablation Study Computation Time}
%
%\begin{figure}[tbh]
%	% \vspace*{-3mm}
%	\centering
%	\includegraphics[trim={0cm 0cm 0cm 1.3cm},clip, width=\columnwidth]{figures/ablation_computation_time}
%	\caption{Ablation study computation times for each example topological graph. We use logarithmic scale to show the relative solve time changes due to removing overwatch (O), vulnerability (V), and teaming (T) components of our algorithm.}
%	\label{fig:ablation_computation_time}
%	% \vspace*{-3mm}
%\end{figure}
%
%Fig.~\ref{fig:ablation_computation_time} shows the relative changes in the computation times with components of our algorithm removed, as discussed in the ablation study, to show the impact of these constructs to the overall problem's complexity. %though the number of variables may vary slightly depending on the reduced formulation. 
%
%When removing the overwatch, vulnerability, and teaming conditions, the reduced problem has significantly faster solve times, as expected. Depending on the objectives of a particular scenario, one of these reduced algorithms could be used, particularly if minimal solve time is favored over complex behaviors. 
%%
%For the illustrative and bounding graphs, the solve times were similar. Notably the in these cases the solve time without teaming is higher than with teaming. 
%%This is likely due to the difference in time cost formulation. The time cost without teaming introduces new variables and the rest of the formulation remains the same, except with $r_e = r_c = 0$ and $a_c = 1$. 
%This is likely due to the quantity of solutions. While our solutions are always guaranteed to be optimal, particularly with simple examples such as these two, the optimal solution is not guaranteed to be unique. Removing teaming, there is not any change in the cost for robots to stay together or not. This results in a large quantity of solutions with equivalent cost function values. 
%When vulnerability is removed as well the piecewise-linear costs become strictly linear, removing the need for the cost decision variables, $C_{W_{e,t}}$ and $C_{O_{c,t}}$, and overall more drastically reducing the complexity of the problem. 
%
%%The solve times increase for the more realistic graphs, Map 1 and Map 2, however the problem still stays computationally tractable; solving complicated routes on large graphs for multi-robot teams in only 3 seconds. We expect these graph sizes and time horizons to be realistic for operational scenarios. In even larger scenarios, if necessary, we can use a receding horizon approach to keep the time horizon small. 

%%%%%%%%%%%%%%%%%%%%%%%%%%%%%%%%%%%%%%%%%%%%%%%%%%%%%%%%%%%%%%%%%%%%%%%%%%%%%%%%
%\section{DISCUSSION}


%%%%%%%%%%%%%%%%%%%%%%%%%%%%%%%%%%%%%%%%%%%%%%%%%%%%%%%%%%%%%%%%%%%%%%%%%%%%%%%%
\section{CONCLUSION}

In this work, we demonstrated expressing complicated scenarios compactly with dynamic topological graphs and MIP, significantly reducing the overall state space. 
We considered minimizing detectability in sample reconnaissance problems, introducing the concepts of overwatch, vulnerability, and teaming, and analyzing the trade-offs of these constructs. 
% We investigate planning on dynamic topological graphs and the trade-offs of the overwatch, vulnerability, and teaming conditions we introduce. 
We show results solving complicated, real world scenarios in seconds, resulting in full tactical maneuvers for multi-robot teams with explicit collaboration. We have removed the dependence on the number of robots in our state space, allowing this approach to easily scale to large teams of robots. Additionally, for problems on larger graphs, we can plan with a receding horizon and as the environment changes we can continuously re-plan. 

By expressing this problem using MIP, the state space, costs, and constraints are fully comprehensible, and yield explainable results. We hypothesize that these results have the potential to generalize to larger classes of problems, such as search and rescue applications over treacherous terrain. We hope to extend this formulation for distributed planning and heterogeneous teams. Additionally, since we have formulated our problem with a convex objective function when integrality constraints are relaxed, we hope to find a tight relaxation to further decrease solve times. 

%main points of the paper,
%elaborate on the importance of the work,
%suggest applications and extensions

%\addtolength{\textheight}{-12cm}   % This command serves to balance the column lengths
% on the last page of the document manually. It shortens
% the textheight of the last page by a suitable amount.
% This command does not take effect until the next page
% so it should come on the page before the last. Make
% sure that you do not shorten the textheight too much.

%%%%%%%%%%%%%%%%%%%%%%%%%%%%%%%%%%%%%%%%%%%%%%%%%%%%%%%%%%%%%%%%%%%%%%%%%%%%%%%%



%%%%%%%%%%%%%%%%%%%%%%%%%%%%%%%%%%%%%%%%%%%%%%%%%%%%%%%%%%%%%%%%%%%%%%%%%%%%%%%%



%%%%%%%%%%%%%%%%%%%%%%%%%%%%%%%%%%%%%%%%%%%%%%%%%%%%%%%%%%%%%%%%%%%%%%%%%%%%%%%%
%\section*{APPENDIX}


%%%%%%%%%%%%%%%%%%%%%%%%%%%%%%%%%%%%%%%%%%%%%%%%%%%%%%%%%%%%%%%%%%%%%%%%%%%%%%%%
\section*{ACKNOWLEDGMENT}

We gratefully acknowledge the support of the Army Research Laboratory under grant W911NF-22-2-0241. 

%%%%%%%%%%%%%%%%%%%%%%%%%%%%%%%%%%%%%%%%%%%%%%%%%%%%%%%%%%%%%%%%%%%%%%%%%%%%%%%%

\bibliographystyle{IEEEtran}
\bibliography{references}

\end{document}