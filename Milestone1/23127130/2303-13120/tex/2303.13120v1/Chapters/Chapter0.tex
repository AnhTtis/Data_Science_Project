% Chapter 0

\chapter{Motivation to the study of parallels, random geometry and quantum gravity} % Main chapter title

\label{Chapter0} % For referencing the chapter elsewhere, use \ref{Chapter1} 

%----------------------------------------------------------------------------------------

% Define some commands to keep the formatting separated from the content 
\newcommand{\keyword}[1]{\textbf{#1}}
\newcommand{\tabhead}[1]{\textbf{#1}}
\newcommand{\code}[1]{\texttt{#1}}
\newcommand{\file}[1]{\texttt{\bfseries#1}}
\newcommand{\option}[1]{\texttt{\itshape#1}}

\definecolor{Gray}{gray}{0.85}
\definecolor{LightCyan}{rgb}{0.88,1,1}

\newcolumntype{a}{>{\columncolor{Gray}}c}
\newcolumntype{b}{>{\columncolor{white}}c}

\newcommand{\mc}[2]{\multicolumn{#1}{c}{#2}}
%----------------------------------------------------------------------------------------
\textit{"You must not attempt this approach to the parallels. I know this way to the very end. I have traversed this bottomless night, which extinguished all light and joy of my life. I entreat you, to leave the science of parallels alone. For God’s sake, please give it up. Fear it no less than the sensual passion, because it, too, may take up all your time and deprive you of your health, peace of mind, and happiness in life. I thought I would sacrifice myself for the sake of truth. I was ready to become a martyr who would remove the flaw from geometry and return it purified to mankind. I accomplished monstrous, enormous labors: my creations are far better than those of others and yet I have not achieved complete satisfaction. I turned back when I saw no man can reach the bottom of this night. I turned back unconsolidated, pitying myself and all mankind. Learn from my example: I wanted to know about parallels. I remain ignorant, this has taken all the flowers of my life and all my time from me...."} - \textbf{A letter of Bolyai Farkas to his son Bolyai János}\\\\


Geometry, the mathematical study of shapes, always interested humans. From the ancient Greeks till today's science various topics related to geometry are key aspects to mathematics and natural sciences. Euclid laid down five axioms, which became the foundations of mathematics. At that time, mathematics was postulated in terms of words and rarely graphics, but not equations. The postulates of Euclid~\cite{postulates}, based on his axioms, defined geometry until the $19^{th}$ century. His postulates were:

\begin{itemize}
    \item  A straight line segment may be drawn from any given point to any other.
    \item A straight line may be extended to any finite length.
    \item A circle may be described with any given point as its center and any distance as its radius.
    \item All right angles are congruent.
    \item If a straight line intersects two other straight lines, and so makes the two interior angles on one side of it together less than two right angles, then the other straight lines will meet at a point if extended far enough on the side on which the angles are less than two right angles.
\end{itemize}

None dared to question the truth of these postulates as any sane person could check their truths by drawing those lines and not finding any which doesn't fit. This was true until some questioned whether it is possible to draw triangles on various non-flat shapes such that the sum of their inner angles is different than that of $\frac{\pi}{2}$. This is exactly what Bolyai Farkas is talking about in his letter to his son. He discovered that parallels can meet sometimes, but did not manage to describe the phenomena in its entirety even though he worked in that field for his whole life. Thus he warned his son not to pursue geometry and its parallels. But his son, János had his own ideas, and years later he constructed the basics of non-Euclidean geometry. Had he listened to his father, the topic of my doctoral thesis would be probably significantly different. Bolyai pursued a non-mainstream topic of mathematics and reached success with it. Later Riemann based his work on the work of Bolyai (and others), which was then used by Einstein when he worked out the general theory of relativity. The science of fundamental physics brings forth our knowledge of nature, if we wouldn't walk off-road in the theory space, but only follow the mainstreams, we wouldn't be able to solve the hardest problems of science.\\

At the beginning of the twentieth century, the appearance of two theories gave us an enormous leap toward understanding nature. The paradigm shift which is related on one hand to the curving relativistic four-dimensional spacetime described by general relativity (\textbf{GR}) and on the other hand to the discreteness of nature as it is seen by quantum mechanics turned science into science-fiction in the eye of the scientifically not educated people. Math and physics, needed in order to understand it, started to be so complex and demanding,  that scientific results became non-trivial. Many physical theories are validated or falsified via mathematical derivations and many cannot be accessed because of their mathematical complexity. The work presented in this thesis belongs to a similar off-road field, which is strongly related to parallels and geometry. Quantum gravity is the field where quantum mechanics and general relativity meets. Quantum mechanics is the theory that describes the smallest scales, the tiny fluctuations of matter, and the rules of nature that escape everyday experience, and gravity is the theory that describes the physics of the largest scales, the orbiting of planets, and even the earliest history of the Universe. Their intersection should be the theory of quantum gravity, the theory which describes how the attraction between bodies behaves on the smallest scales, on the scales where other forces of nature dominate and bodies fall apart to their components. As we advanced in our understanding of the world and the Universe it turned out that quantum gravity could potentially explain also the largest scales and the earliest moments of history. It could tell us why we have such a large-scale structure of galaxies that we see, could explain why visible matter constitutes only four percent of everything, could hint at whether we live in a closed or an open Universe, and foreshadow a potential cold death at the end of times. The quantum theory of gravity has the potential to explain the nature and the structure of space-time, to resolve singularities of GR, and furthermore to explain or disprove the theories regarding dark matter and dark energy.\\

After Einstein introduced GR many scientists tried to find the theory of quantum gravity without success. The first attempt to describe quantum gravity was a naive application of perturbative methods of QFT to GR, but it failed. The treatment of infinities by perturbative renormalization techniques, which can be used in the case of the standard model physics, cannot be applied to gravity, which turned out perturbatively non-renormalizable \cite{non_renorm_g}. However, S. Weinberg conjectured that gravity may adhere to an Asymptotic Safety (\textbf{AS}) scenario \cite{Weinberg, asym, asym_critiq}, where using Renormalization Group Flow (\textbf{RG}) techniques one may find a \textit{fixed point}, where there exist only a finite number of coupling constants needed to describe the full quantum  theory in a non-perturbative way. In a lattice formulation of a quantum theory, fixed points are typically connected to phase transitions, and the hypothesis is that there is at least one non-trivial fixed point for gravity related to the ultraviolet (\textbf{UV}) regime, which necessarily requires the existence of a higher order phase transition. Such a phase transition can be typically recognized from the diverging correlation length  and related scaling exponents.\\

By the end of the $20^{th}$ century, with the increasing available computational power, numerical algorithms became widely used. One of the most notable computer-based techniques in physics is related to Lattice Quantum Chromodynamics (LQCD) \cite{lqcd1,lqcd2}, which was developed in parallel with the physical experiments. The basic idea is to discretize the continuum theory such that the field variables are located at the vertices of a regular $D$-dimensional lattice ($D$ depends on the dimensionality of the discussed model). The lattice spacing $a$, which is the length between two adjacent vertices of the lattice, should be sent to zero while keeping the relevant physical observables constant, in order to reach the continuum limit within the numerical simulations. Since the beginning of the development of lattice theories, many physically relevant observations were derived from numerical simulations, e.g., related to phase transitions \cite{lqcd_chiral}, physical masses of particles \cite{lqcd_masses}, and many other phenomena. In contrast to  the LQCD, lattice quantum gravity is special in the sense that the lattice connectivity itself encodes the geometric degrees of freedom and therefore provides information about the distinct features of gravitational physics on the quantum level. In order to create a physically relevant model of  lattice quantum gravity one also has to be able to include matter fields, e.g., scalar fields or gauge fields \cite{2d_cdt_m,gauge_2d_cdt}.\\

This document is a guide to a collection of articles published in the past years and constituting my doctoral thesis. All of the publications were published in peer-reviewed journals. \\ 

The structure of this document is as follows: The introduction to Causal Dynamical Triangulations (CDT), which is a non-perturbative approach in the quest of quantizing gravity, is the topic of chapter two. In chapter three, I discuss some details of numerical implementation and Monte Carlo simulation methods used to study CDT. The fourth and fifth chapters discuss respectively the results of my studies obtained for empty Universes (pure gravity) and Universes with matter content (gravity coupled to scalar fields). Afterward, all publications which constitute my thesis are briefly discussed in chapter six, together with information about my contribution to them. The published papers are attached at the very end of chapter seven in the following order:

\begin{enumerate}
    \item[\cite{pub1}]  J. Ambjorn G. Czelusta et al. “The higher-order phase transition in toroidal CDT”. In: J. of High Energ. Phys. 2020 (5), p. 30.\\DOI: 10.1007/JHEP05(2020)030
    \item[\cite{pub2}] J. Ambjorn et al. “Towards an UV fixed point in CDT gravity”. In: Journal of High Energy Physics 2019 (7), p. 166.\\ DOI: 10.1007/JHEP07(2019)166
    \item[\cite{pub3}]  J. Ambjorn et al. “Topology induced first-order phase transitions in lattice quantum gravity”. In: Journal of High Energy Physics 2022 (4), p. 103.\\ DOI: 10.1007/JHEP04(2022)103.
    \item[\cite{pub4}] J.Ambjorn et al. “Cosmic voids and filaments from quantum gravity”. In: The European Physical Journal C 81 (8 2021), p. 708.\\ DOI: 10.1140/epjc/s10052-021-09468-z
    \item[\cite{pub5}] J. Ambjorn et al. “Matter-Driven Change of Spacetime Topology”. In: Phys. Rev. Lett. 127 (16 Oct. 2021), p. 161301.\\ DOI: 10.1103/PhysRevLett.127161301
    \item[\cite{pub6}]  J. Ambjorn et al. “Scalar fields in causal dynamical triangulations”. In: Classical and Quantum Gravity 38 (19 Sept. 2021), p. 195030.\\ DOI: 10.1088/1361-6382/ac2135
\end{enumerate}
