% Chapter 3

\chapter{Empty Universes} % Main chapter title
\label{chapter3}  

\textit{This chapter gives a brief summary of the following articles: \cite{pub1, pub2, pub3} which are presented as publications in the last chapter}.\\\\

\section{About criticality at phase transitions}
\textit{"My memory for figures, otherwise tolerably accurate, always lets me down when I am counting beer glasses”} - \textbf{Ludwig Boltzmann}\\\\

Transitions between phases can typically be described by simple models. The general idea is that one has to find some (macroscopic) properties of a given physical system that characterize the phases, and track their changes by varying the coupling constants of a given theory. For finite size systems, as the ones observed in numerical MC simulations,  one cannot observe true phase transitions but only pseudo phase transitions, i.e. cross-overs, as finite size systems have finite thermodynamic potentials and also all derivatives of such potentials are finite.  Anyway, one can observe that order parameters, related to some derivatives of the thermodynamic potential, become more and more singular with increasing system size (lattice volume), and by taking the volume to infinity one encounters a true phase transition. There are several phase transitions that exhibit similar behavior and can be characterized the same way, thus they will belong to the same universality class. Phase transitions belonging to the same universality class will show the same type of finite volume scaling properties. It manifests itself by universal values of scaling exponents, which can be used to measure the order of a phase transition. The notion of an order of a phase transition was introduced by Ehrenfest, who characterized  phase transitions using derivatives of thermodynamic potentials (e.g. free energy, entropy, chemical potential... etc). If the $n^{th}$ order derivative diverges at the transition point then one has an $n^{th}$ order phase transition. This picture was refined by Landau by introducing the notion of local order parameters (OP) (in the context of the Ising model), and Ginzburg \cite{ginz} improved Landau's theory by adding fluctuations to the model. Since then the classification of phase transitions shifted towards distinguishing between two types of phase transitions: first-order, which has a divergent first-order derivative of the thermodynamic potential, and  higher-order (also called continuous), where the second- or higher-order derivative diverges. Furthermore, in the above classification, there is a relation between the order of a transition and the correlation length. For a first-order transition one typically has finite correlation lengths, while divergent correlation length signals a continuous phase transition. In the lattice approach, as those discussed in the thesis, finding a phase transition where the correlation length diverges is crucial as only then can the lattice spacing be taken to zero to reach the continuum limit, while keeping the physical quantities fixed. One should also note, that recent models of solid state physics revealed, that the Landau-Ginzburg characterization can also fail, when a phase cannot be characterized by a local order parameter, but rather by long-range entanglement, called topological order \cite{topord}. If the nature itself exhibits such phenomena where the traditional description of phase transitions fails then we cannot take it for granted that such a description works for a model of quantum gravity. Nevertheless, in this chapter we will stick to the Landau approach. As we will later see, most CDT transitions falls in this category, however some of them show atypical features having  relation to phase transitions involving the topology of the underlying manifold.\\

We will present the idea of critical exponents by taking an example of the Ising model \cite{Ising}. The Ising model is one of the simplest lattice models of spin chains with nearest-neighbor interactions. In one dimension it is literally a chain of spins, in two dimensions the spins are placed in vertices %the grid points 
of a regular lattice. The Hamiltonian of the model is:

\begin{equation}
    \mathcal{H}_{I} = -J \sum_{i\leftrightarrow j} \sigma_i \sigma_j,
\end{equation}
where $J$ is a coupling constant, $\sigma_i = \pm 1$ is a spin and the sum is over nearest neighbors in the lattice. Including an external magnetic field $h$ one may write the partition function including the magnetization ($M = \sum_i \sigma_i$) as

\begin{equation}
    \mathcal{Z}(T,h) = \sum_{\{\sigma_i\}} e^{-\beta(\mathcal{H}_{I} -hM)},
\end{equation}
where $\beta = \frac{1}{k_b T}$ is the inverse temperature and the sum is over all possible spin configurations. An example order parameter is $\mathcal M$, the average magnetization:

\begin{equation}
    \langle \mathcal{M} \rangle = \frac{\partial \mathcal{Z}}{\partial(\beta h)} = \frac{1}{\mathcal{Z}}\sum_{\{\sigma_i\}} M e^{-\beta(\mathcal{H}_{I} -hM)}.
\end{equation}
The susceptibility ($\chi$) is the first-order derivative of the magnetization:

\begin{equation}
    \chi(T,h) = \frac{1}{V}\frac{\partial \langle \mathcal{M} \rangle }{\partial h},
\end{equation}
where $V$ is the volume of the system. The relation which follows from this is then:

\begin{equation}
    \frac{V\chi}{\beta}% = \text{var}(M) 
    = \langle \mathcal{M}^2 \rangle - \langle \mathcal{M} \rangle^2,
\end{equation}
so the susceptibility $\chi$ is related to the magnetization variance. Taking the continuum limit, i.e., the lattice spacing $a\to 0$ and the lattice size $N \to \infty$  such that the physical volume  $V=a^d N $ (where $d$ is the  dimension of the system) remains constant, one can compute a two-point correlation function, where the susceptibility will depend on the spatial distance of two points in the following way

\begin{equation}
    k_bT\chi = \frac{1}{V}\int dx \int dx' [\langle m(x)m(x')\rangle - \langle m(x) \rangle \langle m(x') \rangle] = \int dx \langle m(x) m(0) \rangle_c, 
\end{equation}
where $\langle m(x) m(0) \rangle_c$ denotes the connected correlator $G_c$, which  typically decays exponentially with some characteristic correlation length $\xi$. In case of $|x| < \xi$ the susceptibility will behave as:

\begin{equation}
    k_B T\chi < g \xi^d,
\end{equation}
where $g$ is a constant, yielding the correlation length divergent in case of  the  divergent susceptibility. The correlation function can be measured with respect to the change of the temperature yielding

\begin{equation}\label{eq:scalrel}
    \xi(T, H = 0) \propto  |T-T_{crit}|^{-\nu},
\end{equation}
which means that the correlation length scales  with the critical exponent $\nu$ as $T$ approaches the critical temperature $T_{crit}$.\\

In the lattice MC measurements, the largest available correlation length is controlled by the lattice size $N$, i.e., $\xi(N) \sim V^{1/d} =  a N^{1/d}$. Using equation (\ref{eq:scalrel}) it follows that the (pseudo-) critical temperature, or in the general the (pseudo-) critical coupling constant, which triggers the phase transition, will show the following finite-size scaling dependence:
\begin{equation}
    T^{crit}(N) = T^{crit}(\infty)+ \text{const} \times {N}^{-\frac{1}{\gamma}},
\end{equation}
where $T^{crit}(\infty)=T^{crit}$ is the (true) critical temperature in the thermodynamical limit ($N\to \infty$), and $\gamma=\nu d$ is the critical scaling  exponent. The above scaling relation  was used in the studies presented in this chapter. One should note that for a higher-order transition one expects the scaling exponent $\gamma>1$, while for a first-order transition  one typically has $\gamma=1$. 

\section{Order parameters and the internal structure of the configurations}
\textit{"Einfach wie möglich aber nicht einfacher." / "Everything should be made as simple as possible, but not simpler. "} - \textbf{Albert Einstein}\\\\

\label{sec:OPs}

The idea behind Monte Carlo numerical simulations is quite simple. As discussed in the Chapter \ref{chapter2}, one can generate a set of  (almost)  statistically independent configurations using a Markov chain of "moves" applied randomly with a proper transition probability, and then use it to estimate expectation values or correlators of observables, such as order parameters related to  phase transitions. The Regge action of CDT, see eq. (\ref{eq:ation_kappa}), contains  a linear combination of the total number of vertices and simplices of various types, weighted by the bare coupling constants. When changing the couplings the (averaged) values of the above mentioned {\it global} numbers, and also other characteristics of the triangulations, change as well. Therefore, these observables  can be used to define the order parameters of CDT.\\

In the four-dimensional pure gravity CDT model we have three coupling constants, thus we can use them to parametrize the phase-diagram. As we will see in this section, the numerical MC simulations used in CDT reveal four distinct regions  (phases) in the CDT parameter space. In order to be able to perform the MC simulations we fix the (average) lattice volume $\bar{N}_{41}$. The volume fixing means that, throughout the MC simulation, the observed lattice volume ${N}_{41}$ will perform fluctuations around the fixed value $\bar N_{41}$, also restricting the values of related quantities, such as coordination numbers of various sub-simplices. It also corresponds to fixing the total spatial volume $\sum_t V_3(t)=\frac{1}{2} N_{41}$ (total number of spatial tetrahedra in slices with integer lattice time coordinate $t$). Using different values of $\bar{N}_{41}$ one is able to perform the finite volume scaling analysis, where the change in the order parameters (OPs) can be related to the change in the lattice size, as discussed in the previous section. This way one can track the approach to the thermodynamical limit. In order to enforce fluctuations of the lattice volume $N_{41}$ around $\bar{N}_{41}$ it is also necessary to tune the bare cosmological coupling constant $\kappa_4\to \kappa_4^{c}(\kappa_0, \Delta,\bar N_{41})$. This way one trades the  $\kappa_4$ coupling for the  $\bar{N}_{41}$ volume fixing. The fixing slices-off a two-dimensional hyper-surface $\kappa_4(\kappa_0,\Delta)$ from the full parameter space for fixed $\bar{N}_{41}$.\\

Already, before starting any deeper analysis, one can  look at the freedom of the {\it global} numbers characterizing a CDT triangulation and appearing in the bare Regge action (\ref{eq:ation_kappa}), i.e., $N_0$, $N_{41}$ and $N_{32}$. A single triangulation (a path in the path integral) is itself physically not meaningful, however if some features of a given triangulation repeat in the ensemble of generic triangulations observed in a given phase, then in such a case it makes sense to discuss these features of a particular triangulation, as they will also appear in the expectation values (averages) of the measured observables. As all global numbers ($N_0, N_{41}$ and $N_{32}$) are independent of each other\footnote{Although there are theoretical lower and upper limits, which in itself features an unsolved mathematical problem.}, a configuration with a given fixed $N_{41}$ can have small or high number of vertices or other (higher-dimensional) sub-simplices, which will result in a significantly different distribution of these numbers in different phases. One can also imagine that even if all global numbers $N_0, N_{41}$ and $N_{32}$ were constant, the local distribution of vertices and (sub-)simplices within a configuration can be not homogeneous. Even though every simplex has exactly 5 neighbors, every vertex has a different number of simplices connected to it, which gives rise to the possibility of non-trivial vertex coordination number distributions, where some vertices are shared by only a few simplices, but some other vertices will have a large coordination number. \\


The (ratios of) global numbers of a configuration are natural OPs as they are the simplest degrees of freedom in our geometric setup. Thus the first two OPs  can be defined as:
 
 \begin{equation}
\mathcal{O}_1 = \frac{N_0}{N_{41}} \quad , \quad    \mathcal{O}_2 = \frac{N_{32}}{N_{41}}.
\end{equation}
There are also some OPs which are not global in the sense that they are related to a local distribution of (sub-)simplices in a triangulation. For example, as we have a foliation, we can measure the distribution of vertices as a function of the lattice time coordinate $N_0(t)$. We can also measure similar distributions for the  4-simplices, but, as the above simplices  are four-dimensional objects, instead of talking about fixed $t$ we rather talk about the four-dimensional {\it slab} (part of the triangulation between $t$ and $t+1$), and denote the number of 4-simplices in the slab by $N_{41}(t)$ and $N_{32}(t)$, respectively. For example, the (three-dimensional) volume profile, introduced in Chapter~\ref{Chapter1}, is simply given by: $V_3(t)=\frac{1}{2} N_{41}(t)$. If the adjacent spatial slices contain similar number of tetrahedra, then $N_{41}(t)$ will be a flat function but if the volume profile has a non-trivial shape, then the difference between the adjacent slices will be larger. Therefore, one can define the third OP which quantifies this:

\begin{equation}
\mathcal{O}_3 = \frac{1}{N_{41}} \sum_{t}%^T
(N_{41}(t)-N_{41}(t+1))^2.
\end{equation}
The shape function $\langle V_3(t) \rangle$ (the volume profile) could potentially be also used as an order parameter. For example, in the case of  spherical CDT the $\langle V_3(t) \rangle \approx \cos^3{(t)}$ \cite{sphere} and in the toroidal CDT it is $\langle V_3(t) \rangle = \bar{N}_{41}$ in the semi-classical phase~($C$), while it has a completely different shape in other phases. An example of a local OP is $\mathcal{O}_4$, defined by the highest vertex coordination number among the set of vertices: 
\begin{equation}
    \mathcal{O}_4 = \frac{1}{N_{41}} \argmax_v (coord(v)), 
\end{equation}
where $v$ is a set of all vertices in a  triangulation. One can as well measure the distribution of this quantity in the lattice time $t$.\\

Additional OPs can  also be useful. For example, one can measure the total number of $type_1$-type of simplices neighboring $type_2$-type of simplices in a triangulation, where $type$ refers to a general 4-simplex. The various types of these numbers are summarized in Table \ref{table:ABCD}.

\begin{center}
\begin{tabular}[h]{ a c c c c c }
\rowcolor{Gray}
\mc{1}{}  & \mc{1}{$s_{41}$} & \mc{1}{$s_{32}$} & \mc{1}{$s_{23}$} & \mc{1}{$s_{14}$} & \mc{1}{sums to} \\
 $s_{41}$ & $A_1$ & $C_1$ & 0 & $E$ & $\rightarrow 5\cdot N_{41}$ \\ 
 $s_{32}$ & $C_1$ & $B_{1a} + B_{1b}$ & $D$ & 0 & $\rightarrow 5\cdot N_{32}$ \\  
 $s_{23}$ & 0 & $D$ & $B_{2a} + B_{2b}$ & $C_2$ & $\rightarrow 5\cdot N_{23}$ \\
 $s_{14}$ & $E$ & 0 & $C_2$ & $A_2$ & $\rightarrow 5\cdot N_{14}$  
\end{tabular}
\captionof{table}{The table summarizes the numbers related to the adjacency relations of 4-simplices. All rows and columns sum up to the global numbers $N_{41}$ or $N_{32}$.}
\label{table:ABCD}
\end{center}
The rows and columns of Table \ref{table:ABCD} denote the adjacent $type_1$ and $type_2$ simplices, e.g.,  $A_1$ is the total number of common faces (tetrahedra) between two $s_{41}$ simplices in a given triangulation, while $C_1$ counts the total number of tetrahedra connecting the $s_{41}$ and $s_{32}$ simplices. The parameter $B_1$ (and $B_2$), which  measures the self connectivity between the $s_{32}$ (or respectively $s_{23}$) simplices, can additionally be split into two sub-categories, depending on the type of a connection between the sub-simplices.\footnote{See discussion in Chapter \ref{chapter2}.} Subscript $a$ denotes the connectivity via a spatial tetrahedron ($s_{31}$) and subscript $b$ via a time-like tetrahedron ($s_{22}$). Even though Table \ref{table:ABCD} contains in general $10$  different additional parameters characterizing a CDT triangulation, one can show that only some of these parameters are  independent, but surprisingly not all can be expressed via the \textit{global} numbers. Taking also into account all different types of sub-simplices in a triangulation (e.g. vertices, space-like links, time-like links, spatial triangles, time-like triangles,..., etc.) the topological constraints of the CDT manifolds restrict the total number of independent parameters (including the elements of Table \ref{table:ABCD}.) to 8. The derivation of the relations is presented in Appendix \ref{AppendixA}.\\

During a Monte Carlo simulation, the topology of the triangulations, i.e., their Euler characteristic $\chi$, is fixed and to perform a simulation one also fixes the coupling constants $\Delta$ and $\kappa_0$, and tunes $\kappa_4$ to the critical value corresponding to a given lattice volume $\bar{N}_{41}$. There are then two independent global parameters that can change freely\footnote{Strictly speaking, $N_{41}$ also changes as it fluctuates around the fixed $\bar{N}_{41}$.}:  the total number of vertices $N_{0}$ and the total number of $s_{32}$ plus $s_{23}$  simplices $N_{32}$. Apart from the above mentioned global parameters, there are still three independent  parameters left, one can choose, e.g., $C_1, C_2$ and $D$. Statistically $\langle{C}_1\rangle \approx \langle{C}_2 \rangle$, therefore one can effectively increase the number of order parameters by two, defining:

\begin{equation}
\mathcal{O}_5 = \frac{C_1+C_2}{N_{41}}= \frac{C}{N_{41}},
\end{equation}
and 
\begin{equation}
\mathcal{O}_6 = \frac{D}{N_{41}}.    
\end{equation}


%The extension of the model with the new parameters will not be discussed further in the thesis but 
In next section we will show how to use the OPs to analyze the phase-diagram of the CDT model.
 



\section{Phase transitions}

\textit{"God does not play dice with the Universe!"} - \textbf{Albert Einstein}\\\\

%---------------------------------------------------------------------------------------

Albert Einstein once criticized quantum mechanics and he said: "God does not play dice with the Universe!", maybe Gods don't but we do within our numerical simulations. As it was discussed in Chapters \ref{Chapter1} and \ref{chapter2}, CDT aims to study the lattice regularized path integral of  quantum gravity using numerical MC methods. In the simplest case one deals with triangulated empty "universes", i.e., pure gravity models, without additional matter fields. The properties of CDT emerge as a result of interplay between the bare Regge action:
\begin{equation}
    S_{R} = -(\kappa_0+6\Delta) N_0 + \kappa_4 (N_{41}+N_{32}) + \Delta N_{41}, \\
\end{equation}
and the entropy of states, i.e., the number of triangulations with the same value of the bare action in the partition function (\ref{eq:partfun}). Due to this entropic nature there are several phases which can be visualized in the two-dimensional parameter space\footnote{As explained above, the third coupling constant $\kappa_4$ is tuned to $\kappa^c_4(\kappa_0, \Delta, \bar N_{41})$ corresponding to the fixed lattice volume $\bar{N}_{41}$ of a MC simulation.} $(\kappa_0,\Delta)$. As we have a two-dimensional coupling-constant space $(\kappa_0,\Delta)$, sometimes we will refer to the coupling constants as coordinates in the phase-diagram. The four phases of CDT are presented in Fig. \ref{fig:phase_diag}.\\

\begin{figure}[h]
    \centering
    \includegraphics[width = 0.8\textwidth]{Figures/Phase_structure.pdf}
    \caption{The phase-diagram of CDT, which shows four different phases: $A$ (branched polymer), $B$ (collapsed), $C$ (de Sitter) and $C_b$ (bifurcation).}
    \label{fig:phase_diag}
\end{figure}

Even though the  CDT model is simple in its construction, the resulting complexity arises in the variety of possible configurations. For very large   (inverse) bare gravitational coupling  $\kappa_0$ one recovers  phase $A$, which is characterised by a vanishing kinetic term in the effective action of CDT, parametrized by the spatial volume $V_3(t)$ (or alternatively by the scale factor) \cite{transfer_matrix}. The internal dynamics between the simplices results in an emerging geometry with a branched-polymer structure. For low enough asymmetry parameter $\Delta$ phase $B$ can be observed. It is characterized by the vanishing time-extent of the generic geometric configurations. All  spatial tetrahedra ($3-$volume) gather in one spatial slice, and each of the two adjacent  slices features a vertex with an almost full connectivity to the 4-simplices containing these tetrahedra. The occurrence of this phase is understandable in the context of the, so-called, balls-in-boxes model \cite{balls_in_boxes, condens_model}. The most interesting region of the phase-diagram is phase $C$, also called the de Sitter\footnote{Technically the name "de Sitter" should be used only for the spherical CDT case, as the toroidal CDT volume profile is constant and does not resemble any de Sitter-like solution.} \cite{nonperturb_desitter} or the semi-classical phase, which can be mostly observed for positive $\Delta$ and medium range of $\kappa_0$. In the case of toroidal CDT, the spatial volume profile $V_3(t)$ of generic phase $C$ triangulations is constant while in the case of the spherical CDT a de Sitter-like blob with the shape $V_3(t)\approx cos^3(t)$ forms. Last but not least, the remaining phase is the, so-called, bifurcation phase or shortly phase $C_b$. The phase is characterized by the appearance of vertices of high coordination number in every second spatial slice and the formation of a blob (different from that of phase $C$) in the volume profile both in the spherical and the toroidal CDT. As the de Sitter phase is  physically the most interesting one, the phase transitions surrounding this region were studied the most, especially as the perspective UV fixed point of quantum gravity should lie at the border of this region. It was found, that the lattice spacing decreases with increasing $\kappa_0$ and slightly decreases with decreasing $\Delta$ \cite{towardcontinum}, thus the part of the phase-diagram nearby the $C-B$ phase transition is of great interest, as the two "triple" points where the phase transition lines meet are natural candidates to be the UVFP of the theory. Due to this, it is very important to analyze the scaling exponents related to the phase transitions around the triple points. This is the reason why in this section we will present results related to the three-phase transitions: $A-B$, $C-B$, and $C_b - B$. If any of them turns out to be higher-order then it will support the possibility of the existence of the UVFP. However it is also known, that first-order phase transition lines may  end at a higher-order point (e.g., in the phase diagram of water).\\

\newpage

The typical way to find a phase transition is to fix one coupling constant, which will be either $\kappa_0$ or $\Delta$ in the case of 4-dimensional CDT, and then start a set of MC simulations for various values of the other coupling constant. To show the behavior of the order parameters, defined in Section, \ref{sec:OPs}., we present Fig. \ref{fig:ops} and Fig. \ref{fig:op56}, where the OPs were measured in CDT with toroidal spatial topology for fixed $\Delta= 0.02$, total lattice volume  $\bar{N}_{41} = 160k$ and length of the (periodic) lattice time coordinate (number of spatial slices) $T = 4$. 
\begin{figure}
    \centering
    \includegraphics[width = 0.8\textwidth]{Figures/k4_k0.pdf}
    \caption{Values of $\kappa_4$ in the function of $\kappa_0$. Slight discontinuities in the function $\kappa_4(\kappa_0)$  signal the phase transitions, which is related to the change in entropy of the configurations on the two sides of the phase transitions. Between the vertical lines the corresponding phases are shown.}
    \label{fig:k4k0}
\end{figure}
One of the parameters that strongly depend on the volume is the bare cosmological coupling constant, that has to be tuned for each $\bar{N}_{41}$, however, its value also depends on the selected average volume. $\kappa_4(\kappa_0,\Delta)$ is a function of the other coupling constants, thus fixing one of it one may find how it changes in the function of the other (see Fig. \ref{fig:k4k0}). 

\begin{figure}[h]
    \centering
    \includegraphics[width = 0.45\textwidth]{Figures/op_1.pdf}
    \includegraphics[width = 0.45\textwidth]{Figures/op_2.pdf}
    \includegraphics[width = 0.45\textwidth]{Figures/op_3.pdf}
    \includegraphics[width = 0.45\textwidth]{Figures/op_4.pdf}
    \caption{Example of the measured OPs from $\mathcal{O}_1$ (top left) to $\mathcal{O} 4$ (bottom rigt). The most left region on the plots is  phase $B$ (collapsed), next to it is phase $C_b$ (bifurcation), then phase $C$ (de Sitter), and next to it phase $A$ (branched polymer).}
    \label{fig:ops}
\end{figure}

Red vertical lines visible in figures Fig. \ref{fig:k4k0} - \ref{fig:ops} show locations where the behavior of (at least some) OPs changes, thus they signal the phase transitions. In Fig. \ref{fig:k4k0} between the vertical red lines each corresponding phase is written. The slight discontinuities in the function $\kappa_4(\kappa_0,\Delta = fix)$ signal the locations of the phase transitions, which originate from the different entropy on the two sides.\\

Not every OP signals all of the phase transitions, but using different OPs one can find them. For example, $\mathcal{O}_1$, $\mathcal{O}_2$ and $\mathcal{O}_3$ distinguish seemingly three different regions, while $\mathcal{O}_4$ seems to be sensitive to all four phases. The behavior of the two new OPs, $\mathcal{O}_5$ and $\mathcal{O}_6$, is similar to $\mathcal{O}_2$, however, their crossing point additionally signals the $C_b - C$  phase-transition, as presented in Fig.  \ref{fig:op56}. This becomes apparent when one looks at the  susceptibility of $(\mathcal{O}_6-\mathcal{O}_5)$, as shown in Fig.  \ref{fig:varop56}. In the figure we plot the susceptibility ${\chi}(\mathcal{O}_6-\mathcal{O}_5)$, i.e., the variance of  $(\mathcal{O}_6 -  \mathcal{O}_5)$, normalized by its expectation value $\langle\mathcal{O}_6-\mathcal{O}_5\rangle$, which shows a clear peak at the $C_b - C$ transition point.
\begin{figure}[h]
    \centering
    \includegraphics[width = 0.45\textwidth]{Figures/op_CD.pdf}
    \includegraphics[width = 0.45\textwidth]{Figures/op_DmC.pdf}
    \caption{The new OPs $\mathcal{O}_5$ and $\mathcal{O}_6$ are shown in the left plot. Their behavior is similar to $\mathcal{O}_2$, but the crossing point additionally signals the $C_b - C$ phase-transition. The difference $\mathcal{O}_6-\mathcal{O}_5 $ is presented in the right plot, where the dashed horizontal line is at value zero.}
    \label{fig:op56}
\end{figure}
Fig. \ref{fig:op56} shows the two new OPs, and their difference, which is close to zero in phase $A$ and $B$, positive in phase $C_b$ and  negative in phase $C$, and thus is useful in recognizing all four phases of CDT. The above observations show, that it is not enough to look at one OP but rather a set of OPs should be used while analyzing phase transitions. \\

\begin{figure}[h]
    \centering
    \includegraphics[width = 0.8\textwidth]{Figures/chiO6m5.pdf}
    \caption{The figure shows the (normalized) susceptibility ${\bar{\chi}}( \mathcal{O}_6-\mathcal{O}_5 ) / \langle \mathcal{O}_6-\mathcal{O}_5 \rangle $. The peak of the susceptibility, i.e., the   variance of $( \mathcal{O}_6-\mathcal{O}_5)$, at the  $C_b-C$  transition is a clear signal of the phase transition.}
    \label{fig:varop56}
\end{figure}

Similar plots can be drawn if one measures the OPs in the function of the coupling constant $\Delta$ for fixed $\kappa_0$, or even when choosing  both $(\kappa_0, \Delta)$ values on some straight but not vertical nor horizontal line in the CDT phase-diagram. 

\newpage

\subsection{Finite volume scaling analysis} 

Even though, as explained in Section \ref{sec:OPs}.,  CDT phase transition signals observed for any fixed lattice volume $\bar{N}_{41}$ are not real phase transitions, as for any finite volume the free energy is finite and formally one just observes a cross-over, nevertheless using finite-size scaling analysis one can investigate the (real) phase transitions and draw conclusions about critical exponents in the thermodynamical limit. By extrapolating the scaling relations to $\bar N_{41}\to \infty$ one can as well find the (infinite volume) critical values of the coupling constants $\kappa_0^\infty$ and $ \Delta^\infty$ and of the order parameters $OP^\infty$. Thus a typical finite-size scaling relation of a coupling $\mathcal{C}$ corresponding to the transition point will be described by a function

\begin{equation}
\label{eq:scaling}
    \mathcal{C}^{crit}({\bar{N}_{41}}) = \mathcal{C}^\infty - \alpha \bar{N}_{41}^{-\frac{1}{\gamma}},
\end{equation}
where $\gamma$ is the critical exponent, whose value may be used to distinguish between the first-order ($\gamma=1$) and the higher-order ($\gamma>1$) phase transition.
\\

At the thermodynamical limit of the higher-order phase transition one can perspectively find a UVFP, however doing so in lattice simulations is not an easy task as it was shown in  \cite{rg_flow1, rg_flow2}. Previous findings did not give convincing evidence for the existence of the UVFP of CDT, however, since these measurements, there were many improvements both in the CDT code (making the MC simulations more efficient) and computer technology. Therefore now, using the new possibilities, one can try to re-investigate this issue in more detail.

\subsection{$A - B$ phase-transition}
\label{sec:AB_trans}

\textit{This subsection is based on the publication \cite{pub3}.}\\\\

Since the appearance of the path integral formalism of Feynman, we know that not only those paths should be taken into account which can be imagined classically but also non-classical ones too. In the path integral formalism,  for example, a point particle takes all possible paths when traveling from point A to B including also classically forbidden paths when it tunnels through potential walls or simply goes outside of the light cone. The contribution of most such paths however cancels out and the classical trajectory can be computed as an average of all paths. A generic triangulation of phase B is characterized by the following pattern: there is a vertex with almost maximal coordination number in a spatial slice with time coordinate  $t-1$, almost all spatial tetrahedra (3-volume) are gathered in slice $t$, and again there is a vertex with almost maximal coordination number in slice $t+1$. All spatial slices with time coordinate different than $t$ has spatial volume close to the minimal allowed cutoff. As the configurations of phase, A are characteristically different, i.e., they can be characterized by branched polymers,  there is a difference in entropy of the configurations between the two phases, which results in a phase-transition between them. 

\begin{figure}[h]
    \centering
    \includegraphics[width = 0.7\textwidth]{Figures/AB_Dcrit.pdf}
    \caption{(Pseudo-) critical values of $\Delta^{crit}(\bar{N}_{41})$ measured for $\kappa_0 = 4.8$ (green), $\kappa_0 = 4.6$ (blue),
and $\kappa_0 = 4.5$ (red) together with the fits of eq. (\ref{eq:scaling}). The solid curves were fitted with the critical exponent fixed to $\gamma = 1$ for all three data sets.}
    \label{fig:AB_Dcrit}
\end{figure}

The fist-order nature of the A-B transition is obvious when one looks at the finite-size scaling of the critical coupling $\Delta^{crit}$ in the function of the lattice volume $\bar{N}_{41}$ presented in Fig. \ref{fig:AB_Dcrit}. Using eq. (\ref{eq:scaling}) one can fit the critical exponent $\gamma$.  The best fits resulted with critical exponent values $\gamma_{4.5} = 1.151 \pm 0.379$, $\gamma_{4.6} = 1.029 \pm 0.178$ and $\gamma_{4.8} = 1.088 \pm 0.101$ for three independent series of measurements with fixed $\kappa_0= 4.5, 4.6$ and $4.8$, respectively.  All three scaling exponents are in agreement with $\gamma = 1$ characteristic for the first-order transition. The same conclusion can be drawn if one looks at finite size scaling of $\mathcal{O}_2$. The fits of such scaling relations are presented in Fig. \ref{fig:AB_o2}. The value of $\mathcal{O}_2$ is very small in both phases. In the infinite volume limit it approaches zero in phase $B$ and for higher $\kappa_0$ also in phase $A$.

\begin{figure}[h]
    \centering
    \includegraphics[width = 0.8\textwidth]{Figures/AB_OP2.pdf}
    \caption{The running of $\mathcal{O}_2$ for $\kappa_0 = 4.8$, $\kappa_0 = 4.6$ and $\kappa_0 = 4.5$.  Blue colors correspond to data measured in phase $B$ and red in phase $A$ closest to the phase transition point, and the darker the color the lower the corresponding $\kappa_0$ coupling, i.e., the closer to the $A\!-\!B\!-\!C$ triple point. The error bars are smaller than the size of the data-points. The solid curves correspond to the fits of a relation similar to eq. (\ref{eq:scaling}) with the critical exponent fixed to be $\gamma = 1$ for all  data sets. }
    \label{fig:AB_o2}
\end{figure}

The conclusion is that phases $A$ and $B$ are thought to be non-physical in the sense of an emergent semi-classical geometry. It is well explained by the fact of the decreasing  connectivity between two adjacent spatial geometries, represented by the vanishing $\mathcal{O}_2$ parameter, although this phenomenon may be possibly related to the, so-called, "asymptotic silence" \cite{asymp_sil}. 


\subsection{$C_b - B$ phase-transition} 
\label{sec:cb_trans}

\textit{This subsection is based on the publication \cite{pub1}.}\\\\

The $B$ and $C_b$ phases are not that different from each other, as in both phases there are vertices which connect to almost every tetrahedron on the adjacent spatial slices, and such high connectivity structure makes these phases to be effectively infinite-dimensional. Both spectral and Hausdorff dimensions differ from the topological value  4, thus these phases do not describe a four-dimensional Universe. Even though  their seemingly non-physical nature the phase transition becomes important to be studied as its endpoint leads to a candidate of the UVFP of the theory, to the $B-C-C_b$ triple-point.\\

The volume profile of the bifurcation phase is presented in Fig. \ref{fig:cb_prof}. It looks the same in the spherical and toroidal version of CDT.

\begin{figure}[h]
\centering
\includegraphics[width=0.4\linewidth]{figbifSphere.pdf}
\includegraphics[width=0.4\linewidth]{figbifTorus.pdf}
\caption{\small The (re-scaled) average spatial volume profiles $\langle V_3(t)\rangle $ observed in the bifurcation phase $C_b$ in the spherical (left plot) and the toroidal (right plot) CDT. In both plots the spatial volume profiles were presented with respect to the center of the volume, set at $t=0$, and shifted  by a (constant  $V_3^0$) volume measured in the {\it stalk} range ($|t|>\sim 10$),  $V_3^0$ being different for each volume profile (in general  $V_3^0$ is bigger in the toroidal CDT where discretization effects are larger). Data measured for various total $\bar N_{41}$ lattice volumes and different $T$ were rescaled by $V_4=\sum_t(\langle V_3(t)\rangle - V_3^0$), i.e., in agreement with the Hausdorff dimension $d_H=\infty$.}
\label{fig:cb_prof}
\end{figure}

The $C_b-B$  transition was analyzed for fixed $\kappa_0= 2.0$. Starting in phase $C_b$ and decreasing $\Delta$ close to $\Delta^{crit }\approx 0$ one finds the transition to phase $B$. The main difference between the two phases is the time extent of phase $C_b$, where the volume profile $V_3(t)$ resembles that of the spherical CDT in phase $C$, while in phase $B$ it is mostly collapsed to a single spatial slice. Although, as already mentioned, there are also many similarities between the two phases, for example, the Hausdorff dimension of generic geometries is very large or even infinite in the large volume limit. Probably due to these similarities, the $C_b-B$ phase transition was found to be the higher-order transition. Not only the fits to the finite size scaling relation of eq. (\ref{eq:scaling}) yielded a solution that was in disagreement with $\gamma = 1$ (as shown in Fig. \ref{fig:cb_scaling}) but also the order parameters showed a smooth transition between the two phases.
\begin{figure}[h]
\centering
\includegraphics[width=0.8\linewidth]{ScalingFit.pdf}
\caption{\small Lattice volume dependence of the pseudo-critical $\Delta^c(N_{41})$ values in CDT with toroidal spatial topology measured for fixed $\kappa_0=2.2$ together with the fit of the finite-size scaling relation (\ref{eq:scaling}) with critical exponent $\gamma=2.51 \pm 0.03$ (orange solid line) and the same fit with a forced value of  $\gamma=1$ (blue dashed line). }
\label{fig:cb_scaling}
\end{figure}
Furthermore, the Binder cumulants tend to vanish with increasing lattice volume, which is also characteristic of a higher-order transition \cite{bind}.\\ 

Summing up, in publication \cite{pub1}, the $C_b-B$ transition was shown to be a higher-order phase transition in CDT with the toroidal spatial topology. This is an important result in the quest for the UVFP of CDT. Due to the strong hysteresis observed in the toroidal CDT, the $C_b-C$ transition bordering the semi-classical phase was classified to be a first-order transition \cite{phase_structure_torus}.\footnote{In contrast to this, in the spherical CDT model the $C_b-C$ transition was shown to be a higher order transition \cite{cb1}.} The finding that the $C_b-B$  transition is higher-order provides a hope that its endpoint (i.e., the $B-C-C_b$ triple point) is also higher-order, yielding it a possible candidate  for the UVFP of the theory.\\


\subsection{$C - B$ phase-transition} 
\label{sec:CB_trans}

\textit{This subsection is based on the publications \cite{pub2} and \cite{pub3}.}\\\\


Fixing the value of $\kappa_0$ in the range [$3.5 : 4.5$] and changing $\Delta$ one can cross the $C-B$ phase transition. In the case of toroidal spatial topology the volume profile $V_3(t)$ of phase $C$ is almost constant, and thus invariant under the translation in time, but crossing to phase $B$ this symmetry of the generic configurations is broken to a "collapsed" volume profile. Phase $C$ is also characterized by quite homogeneous and isotropic  geometry in sufficiently large scales, but as one traverses to phase $B$ one can immediately observe that vertices with very high coordination numbers appear, which breaks the above homogeneity and isotropy. As a result, one observes a strong hysteresis  around the phase transition line, as presented
in Fig. \ref{fig:hist}.

\begin{figure}[h]
    \centering
    \includegraphics[width = 0.7\textwidth]{Figures/hysteresisATG.pdf}
    \caption{The plot illustrates the hysteresis measured during simulations for the lattice volume $\bar N_{41}=160\mathrm{k}$. The green and blue dots correspond to the location of the phase C side of the phase transition,  while the red and black dots correspond to the location of the phase B side of the phase transition.}
    \label{fig:hist}
\end{figure}

In order to encounter the $C-B$ transition, instead of fixing $\kappa_0$ in range [$3.5 : 4.5$] and changing $\Delta$, one can as well fix $\Delta$ in range [$-0.04:0.00$] and change $\kappa_0$. Therefore, the phase transition study, including the finite-volume scaling analysis, was performed for two different fixed $\kappa_0$ values ($\kappa_0 = 4.0$ and $4.2$) and for two different fixed $\Delta$ values ($\Delta = 0$ and $-0.02$). The  transition was determined to be the first-order  phase transition, but a rather atypical one. Although the values of  order parameters measured at both  sides of the hysteresis region do not converge to a common value, which itself signals a first-order transition, the size of the hysteresis region    shrinks when the lattice volume $\bar N_{41}$ is increased, which  can signal a higher-order transition in the thermodynamical limit. In order to resolve this inconsistency, in publication \cite{pub2} we measured the critical exponent resulting from the scaling relation of eq. (\ref{eq:scaling}), which turned out to be $\gamma = 1.62 \pm 0.25$, suggesting a higher-order transition. Nevertheless, in publication \cite{pub3} we repeated the finite-size scaling analysis using  much bigger data statistics and also  additional locations in the phase diagram. We also used a slightly modified finite-size scaling relation in the form:

\begin{equation}
\label{eq:scaling_m1}
    \mathcal{C}^{crit}({\bar{N}_{41}}) = \mathcal{C}^\infty - \alpha (\bar{N}_{41}-{c})^{-\frac{1}{\gamma}},
\end{equation}

where ${c}$ is a discretization correction. We found that the critical exponents are consistent with $\gamma=1$, see Fig. \ref{fig:scaling_couplings}, which signals the first-order transition. We, therefore, concluded that the $C-B$ phase transition is a first-order transition in the case of toroidal CDT.

\begin{figure}[ht!]
\centering
\includegraphics[width = 0.45\textwidth]{dcrit_vertical.pdf}
\includegraphics[width = 0.45\textwidth]{k0crit_horizontal.pdf}
\caption{Finite-volume scaling of the coupling constants $\Delta^{crit}$ (left panel) and $\kappa_0^{crit}$ (right panel) in the $C-B$ transition. The dashed and solid curves represent fits eq. (\ref{eq:scaling}) and eq. (\ref{eq:scaling_m1}), respectively. In the left panel (vertical measurements), green data points are for fixed $\kappa_0 = 4.0$ and blue are for $\kappa_0 = 4.2$. In the right panel (horizontal measurements), green data points are for fixed $\Delta = -0.02$ and  blue are for $\Delta = 0$. }
\label{fig:scaling_couplings}
\end{figure}

\subsection{Summary} 

The three-phase transitions discussed in this chapter were the $A-B$, $C-B$, and the $C_b - B$ transitions, out of which only the last one turned out to be a higher-order (continuous) phase transition in CDT with the toroidal spatial topology. The $A-B$ and $C-B$ phase transitions were not yet analyzed in the case of the spherical CDT. What is more, the existence of the direct $C-B$ transition was not even known before the article \cite{phase_structure_torus} was published. The reason was that the region of the phase diagram analyzed in detail in this thesis was at that time thought to be unreachable via MC  simulations. In theory, the phase diagram of the CDT model does not have to be the same for different spatial topologies, thus a future analysis may potentially find that the $C-B$ transition is continuous in the spherical CDT.  An argument for this phase transition being continuous in the spherical CDT is related to the \textit{effective} topology of the phases. In the case when the topology of the spatial slices is chosen to be $S^3$ and the time direction is compactified to $S^1$, which is the case in the MC simulations, the full topology of the triangulations is $S^3 \times S^1$. However, in the semi-classical phase $C$, the emergent (Euclidean) de Sitter-like geometry, i.e.,  the four-sphere, transforms the {\it effective} topology to $S^4$. To clarify the notion of the {\it effective} topology, let us explain that, by definition, the imposed $S^3 \times S^1$ topology of the triangulations is not changed in the MC simulations, thus the two furthest (time-like) points (poles) of the  four- sphere are  connected by a thin \textit{stalk} of cutoff size, which can be treated as a lattice artifact. The {\it stalk} is necessary to preserve the imposed topological conditions numerically, but if only it was allowed by the MC algorithm it would completely disappear, yielding the change of  topology from  $S^3 \times S^1$ to $S^4$. It illustrates the fact that not only the space-time effective dimensionality but also  the effective topology are emergent concepts on the quantum level. Thus our conjecture, formulated in \cite{pub3}, is the following: \\

\noindent \textit{phase transitions which involve a change in effective topology will be first-order transitions}.\\

\noindent The argument is that if there are two adjacent phases separated by a phase transition, and these phases have different genera then the phase transition cannot happen smoothly resulting in the first-order  transition.\\ 

In the case of CDT with toroidal spatial slices, phase $C_b$ has, similarly to phase $C$ in the spherical CDT case, an effective topology $S^4$. At the same time, the semi-classical phase $C$ has the toroidal effective topology $T^4$. As a result of the phase transition the effective topology changes, yielding the $C-B$  transition first-order. Contrary to that, as explained above, in CDT with spherical spatial slices the effective topology of the semi-classical phase is $S^4$, so it does not change under the phase transition, potentially making the $C-B$ transition higher-order. \\  

So far all analysis was done in the case of empty Universes (pure gravity models). In the next chapter, we will discuss how the CDT model changes in the presence of matter fields.


