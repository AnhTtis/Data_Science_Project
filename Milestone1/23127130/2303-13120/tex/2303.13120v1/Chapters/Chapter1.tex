% Chapter 1

\chapter{Causal Dynamical Triangulations} % Main chapter title

\label{Chapter1} % For referencing the chapter elsewhere, use \ref{Chapter1} 

%---------------------------------------------------------------------------------------

\section{Introduction to Quantum Gravity}

\textit{"The beauty and clearness of the dynamical theory, which asserts heat and light to be modes of motion, is at present obscured by two clouds..."} - \textbf{Lord Kelvin}\\\\

Lord Kelvin wrongly predicted the end of physics in the late nineteenth century. The two clouds mentioned were the problem of heat and radiation, more precisely the theorized material that fills everything called "ether" and the black body radiation. When we mention modern physics, we refer to the time when the solutions to these two "clouds" were presented in the form of special relativity and quantum mechanics. The start of the twentieth century brought us an explosion of physical theories, as special relativity led to general relativity, which is extensively studied today in relation to astrophysical and cosmological models or technologies, such as GPS tracking devices. At the same time, quantum mechanics evolved into quantum field theory, and later our technological advancements led to the ability to measure the properties of particles. The standard model of particle physics is one of the greatest achievements in physics, as it gives an explanation of the fundamental nature of matter. The biggest problem of modern physics is that the theory of matter and the theory of gravity cannot be matched into a unified framework together. Many physicists tried in the past hundred years to describe the theory of quantum gravity, which led to many different research projects, such as Loop Quantum Gravity (LQG), String Theory (ST), Causal Sets (CS), Group Field Theory (GFT), Non-Commutative Geometry (NCG), Canonical Quantum Gravity (CQG), Hořava–Lifshitz Gravity (HLG), Asymptotic Safety (AS), Euclidean Dynamical Triangulations (EDT), Causal Dynamical Triangulations (CDT) and many other approaches. \\

\subsection{(Non-)renormalizability of quantum gravity and the need for non-perturbative approaches}

Merging the quantum theory with gravity is not a trivial task. Quantum field theory (QFT) predicts fluctuations of  fields, and according to GR and Einstein's field equations, where there is energy density, there is curvature. These fluctuations can at very high energies produce such a large energy density in a small volume that the naive application of Einstein's equations would predict the appearance of black holes\cite{scale_grav,scale_grav2}. The problems with UV-completion of quantum gravity become apparent in the perturbative expansion of a QFT based on GR. Such a formulation is perturbatively non-renormalizable \cite{non_renorm}, which means, that the naive application of the perturbation theory would result in infinitely many parameters and coupling constants appearing in the theory, that cannot be eliminated via renormalization thus yielding the theory to be un-predictive. \\

It is well known, that the couplings appearing in QFTs are scale-dependent, this scale dependence is referred to as "running of the couplings". In the case of the full theory, where one integrates from zero to infinite momenta (or alternatively zero distances) many models exhibit infinite divergences, the solution to which is provided by some cutoff $\Lambda$ introduced to the high energy regime. Up to this cutoff, the theory is predictive, and the aim is to remove the cutoff and avoid the appearance of infinities. The UV completeness of a QFT is provided by the existence of fixed points of the renormalization group flow in the coupling constant space: as the energy scale changes, the running coupling constants approach some fixed point value. The microscopic theory is defined in such fixed points, thus finding them is a crucial part of any theory based on QFT language. Let  $g$ be a coupling constant of a given theory, then the so-called "beta function" $\beta(g)$ will define the scale dependence, or running of the coupling. The fixed points are defined by  zeros of $\beta(g)$, which can result in a free or interactive theory. The free theory is achieved when the zero of the beta function corresponds to zero values of the couplings, which is called "asymptotic freedom" and such a fixed point is called trivial or Gaussian. If instead zeros of the beta function are achieved for a finite number of non-zero couplings, it is called "asymptotic safety", where one has non-trivial fixed points corresponding to an interactive theory \cite{nontriv_fp}. A fixed point corresponding to high energy, or short scale, is called the "ultraviolet" (UV) fixed point, while the "infrared " (IR) fixed point will correspond to the low energy, or large scale theory.\\

A QFT description of GR means, that one treats the metric tensor $g_{\mu\nu}$ as the field of gravitation and defines an action in terms of geometric invariants obtained from the metric tensor, such as e.g., $R, R^2, R_{\mu\nu}R^{\mu\nu}$, etc. The most important couplings in the case of gravity are Newton's coupling $G$, and the cosmological constant $\Lambda$. The theory of gravity is perturbatively non-renormalizable, as applying perturbation theory in every order one has to introduce  infinitely many counter-terms and the corresponding new couplings, which renders the theory to be non-predictive. Nevertheless, according to the "asymptotic safety" conjecture, formulated by Steven Weinberg \cite{Weinberg},  most of the (potentially infinitely many) couplings appearing in such a theory become irrelevant at the non-trivial UV fixed point, and there will be only a finite number of relevant couplings rendering the theory non-perturbatively renormalizable, i.e., UV-complete and predictive to arbitrarily large energy scale. Therefore a non-perturbative description of quantum gravity is needed which can be done with the help of numerical simulations. The non-perturbative approach discussed in this thesis  is called Causal Dynamical Triangulations (CDT) and it is based on Regge calculus and Feynman path integral formulation.



\subsection{Regge calculus}

Before jumping into the description of CDT, it is necessary to discuss the mathematical formulation that led to it. This formulation was introduced by Regge and is called Regge calculus \cite{Regge}. The aim of Regge was to approximate space-times, which are solutions to the Einstein field equations, via piecewise-flat manifolds.\footnote{Often the name {\it piecewise-linear manifold} instead of {\it piecewise-flat manifold}  is used.} The approximation is done with the help of internally flat triangular building blocks (simplices) glued together in a non-trivial way, hence the name "triangulation". The simplices in a 2-dimensional triangulation are triangles, in 3-dimensions are tetrahedra, and in 4-dimensions are pentachora. All simplices in a triangulation are glued to each other via their $(d-1)$ dimensional faces (links for $d = 2$, triangles for $d = 3$, and tetrahedra for $d = 4$). These $(d-1)$ dimensional sub-simplices are also connected via "hinges", also called  "bones", which are $(d-2)$ dimensional objects. The hinges play a crucial role, as  curvature can be defined there locally. The curvature is related to the angular difference (deficit angle) at a given bone. Let's imagine a triangulation consisting of $n$ equilateral triangles glued together along edges (links) around a single point (vertex). If $n = 6$ then one can place it on a flat 2-dimensional surface. If $n = 5$ then one can place it only if it is cut along one edge, and it will be visible that a triangle is "missing". The angle associated with the missing (or for $n > 6$ extra) triangles is the deficit angle.\\

Let us consider the simplest (nontrivial) case of a three-dimensional Riemannian manifold which is well approximated by a fine triangulation. Following the approach of Regge \cite{intro_regge_calc}, the discretized curvature is obtained by considering the parallel transport of a vector around a bone. Many simplices (in this case tetrahedra) touch each other at the bone forming a bundle $p$. One can associate the number of simplices in the bundle with bone density $\rho$ at $p$, which is equal to the number of simplices divided by a unit area. The deficit angle ($\epsilon_p$) associated with the bone is a measure of a dihedral angle:

\begin{equation}
    \epsilon_p = 2\pi - \sum_n \theta_n,
\end{equation}
$\theta_n$ being the dihedral angle of the $n$-th simplex at the bone. One can alternatively define $\epsilon = \frac{1}{N}\epsilon_p$, which is the deficit angle of the bone smeared on its simplices. Now, let's take a loop $a$ with area $\Sigma$ around the bundle and parallel transport a vector $\Vec{A}$ around the loop. If $n_\Sigma$ is a unit vector orthogonal to $\Sigma$, then one can define:

\begin{equation}
    \Vec{\Sigma} = \Sigma n_\Sigma,
\end{equation}
which is an area vector associated with the loop. Parallel transporting a vector around the bundle will rotate $\Vec{A}$ by an angle $\sigma$ due to the process of the parallel transport. One can associate a vector of length $\sigma$ to the rotation, and let this vector  be parallel to the bone, so it  will be defined by: 

\begin{equation}
    \Vec{\sigma} = \sigma n,
\end{equation}
where $n$ is the unit vector parallel to the bone. Rotating $\Vec{A}$ by an angle $\sigma$  will produce the vector $\Vec{A'}$ $=\Vec{A}+\delta \Vec{A}$. The infinitesimal change $ \delta \Vec{A}$ will be equal to the product $ \delta \Vec{A} = \Vec{\sigma} \times \Vec{A}$. The rotation angle $\sigma$ is proportional to the number of simplices ($N$) visited by the loop $a$ circumventing the bone $p$, thus :

\begin{equation}
     \sigma = N \epsilon, 
\end{equation}
where $N$ can be expressed in terms of the
bone density $\rho$, the oriented area vector $\Sigma$ and the unit vector parallel to the bone $n$:

\begin{equation}
    N = \rho \, n \cdot \Vec{\Sigma}.
\end{equation}
Putting all the expressions together the infinitesimal change $\delta \vec{A}$ is given by:

\begin{equation}
    \delta \Vec{A} = \rho \epsilon (n \Vec{\Sigma}) \cdot (n \times \Vec{A}).
\end{equation}
Using coordinate (vector component) notation:

\begin{equation}
\delta A_\mu = \rho \epsilon (n^\nu \Sigma_\nu) ( \varepsilon_{\mu \alpha\beta} n^{\alpha}  A^\beta), 
\end{equation}
where:  $\varepsilon_{\mu \alpha\beta}$ is the Levi-Civita symbol. Now, one can express the $n$ and $\Vec{\Sigma}$ vectors in the dual space, i.e., the space of two-forms:

\begin{equation}
    n_\nu = \frac{1}{2}\varepsilon_{\nu\rho\sigma}n^{\rho\sigma}, 
\end{equation}
and 
\begin{equation}
    \Sigma_\nu = \frac{1}{2}\varepsilon_{\nu\alpha\beta}\Sigma^{\alpha\beta}.
\end{equation}
Using the fact that $n^{\nu\lambda} = -n^{\lambda\nu}$, the infinitesimal change $\delta \vec A$ can be now written as:

\begin{equation}
    \delta A_\mu = \frac{1}{4}\rho \epsilon  (\varepsilon_{\nu \rho\sigma} n^{\rho \sigma} \frac{1}{2}\varepsilon^{\nu\alpha\beta} \Sigma_{\alpha \beta}) (2n_{\gamma\mu}) A^\gamma = \frac{1}{2}(\rho \epsilon n_{\alpha \beta} n_{\gamma \mu})\Sigma^{\alpha\beta}A^\gamma.
\end{equation}
Using the continuous counterpart of the same equation with the help of the Riemann tensor one can write:

\begin{equation}
    \delta A_\mu = \frac{1}{2}{R^\gamma}_{\mu\alpha \beta} \Sigma ^{\alpha \beta} A_\gamma.
\end{equation}
Comparing the two equations one can recognize the discretized Riemann curvature tensor. The Ricci tensor can be then defined by index contraction:

\begin{equation}
    {{R^\alpha}_{\mu\alpha\nu}} = R_{\mu\nu} = \rho \epsilon (\delta_{\mu\nu} - n_{\mu} n_{\nu}),
\end{equation}
where we switched back to the unit vector $n$. And with further index contraction, one can get the Ricci scalar:

\begin{equation}
    R = {R^{\alpha}}_{\alpha} = \rho \epsilon ({\delta^\alpha}_\alpha - n^\alpha n_\alpha) = 2 \rho \epsilon,
\end{equation}
which gives a direct connection between the curvature of continuous Riemannian manifolds and their discretized approximations. The above formula can be generalized to more  dimensions as well as to pseudo-Riemannian manifolds. \\

Using Regge calculus, the  Regge action $S_{R}$, i.e., the gravitational action for a piecewise-flat triangulation, can be formulated. The starting point of this is the Einstein-Hilbert action:

\begin{equation}
    \frac{1}{16\pi G}\int d^dx \sqrt{-g} (R - 2\Lambda), 
\end{equation}
where $G$ is the Newton's constant, $R$ is the scalar curvature and $\Lambda$ is the cosmological constant. Writing the curvature in terms of Regge calculus one gets the form:

\begin{equation}
\frac{1}{16\pi G}\int d^dx\sqrt{-g}R = \frac{1}{8\pi G}\int d^dx\sqrt{-g} \rho \epsilon = \kappa \sum_{n_{(d-2)}} k_n\epsilon_n,   
\label{eq:curvature}
\end{equation}
where $\kappa=(8 \pi G)^{-1}$ is the  (inverse) bare  gravitational constant, $k_n$ denotes the volume of the $(d-2)$-dimensional hinge, $\epsilon_n$ is the deficit angle associated with  the hinge and the summation is over $(d-2)$-dimensional simplices, denoted by $n_{(d-2)}$. The term including cosmological constant reads:

\begin{equation}
\frac{1}{16\pi G}\int d^dx\sqrt{-g} (-2 \Lambda) = \frac{-2\Lambda}{16\pi G}\int d^dx\sqrt{-g} = \lambda \sum_{n_d} V_{n_d}, 
\end{equation}
where $\lambda=-\Lambda \kappa$ is the bare cosmological constant, $V_{n_d}$ is the volume of the $d$-dimensional simplices building up the triangulation and the summation is over $d$-dimensional simplices. This leads to the full Regge action:

\begin{equation}
    S_R = \kappa \sum_{n_{(d-2)}} k_n\epsilon_n + \lambda \sum_{n_d} V_n,
    \label{eq:curvature2}
\end{equation}
which holds in any dimension.  
One should note that the Regge form of the gravitational action (\ref{eq:curvature2}) is not expressed in terms of the metric tensor, but in terms of numbers of simplices and sub-simplices. Expressing the Regge action for a particular triangulation can lead to a complicated form, however applying certain constraints can simplify the expressions. 

\section{Causal Dynamical Triangulations}

\textit{"The more success the quantum theory has, the sillier it looks. How nonphysicists would scoff if they were able to follow the odd course of developments!"} - \textbf{Albert Einstein}\\\\

Following the ideas of Weinberg and assuming the existence of a UV fixed point for gravity the properties of quantum gravity can be analyzed using non-perturbative methods. As fixed points were found in other QFT-based theories, such as Quantum Chronodynamics (QCD) \cite{as_freed}, theorists turned towards lattice formulations (e.g. Lattice Quantum Chronodynamics (LQCD)). The simplest lattice theory of GR is called Dynamical Triangulations (DT). In DT, one can use the Regge action straight away. The spacetime is constructed by gluing $d$-dimensional simplicial building blocks: triangles, tetrahedra, and pentachora. The triangulation does not play a role in the physics of the model, as it serves the purpose of regularization, providing a UV  cutoff related to lattice spacing $a$, which should be removed from the continuum limit if it exists. A huge difference between the DT approach from other techniques based on the Regge calculus, such as Quantum Regge Calculus \cite{quantumregge} or some versions of LQG \cite{lqg_regge}, is that the edge length ($a$) of all the simplices is kept fixed and thus piecewise-flat manifolds are constructed from identical equilateral simplices. Transforming the metric signature with the Wick rotation one gets a Euclidean description which allows studying the (regularized) path integral of quantum gravity using statistical methods. In the  DT there is no difference between space and time, however, CDT twists the picture via the introduction of a foliation and thus the notion of time is restored as the causal evolution of the leaves of the foliation. The decomposition of the four-dimensional space-time into space and time is similar to that of the Arnowitt-Deser-Misner (ADM) formalism \cite{adm_f}. Thus, the 4-dimensional space-time is assumed to be globally hyperbolic and each ($d-1$)-dimensional hypersurface ("leaf" of the foliation) has the same fixed topology.  The word "causal" in the name of CDT refers to the time-slicing of the triangulation, as opposed to usual DT, and "dynamical" points at the difference between CDT and traditional lattice approaches, as in CDT the lattice connectivity is not fixed and it encodes  the gravitational degrees of freedom. For example, in LQCD there is a fixed and regular lattice, on which the theory is defined, but in CDT the different lattice configurations correspond to the different trajectories (histories) in the gravitational path integral. Therefore a single configuration (single trajectory) is non-physical, and one has to compute a suitable average over an ensemble of such configurations. \\

In a $d$-dimensional CDT triangulation, by construction, every (sub-)simplex lies in a $d$-dimensional {\it slab} (part of the triangulation) between lattice (discrete) time $t$ and $t+1$. Different types of simplicial building blocks (simplices) $s_{\alpha \beta}$ can be defined by indicating the number $\alpha$ of their vertices in $t$ and the number $\beta$ of  vertices in $t+1$. In 2 dimensions there are two types of building blocks, i.e., triangles: $s_{21}$ and $s_{12}$. In 3 dimensions there are three different types of building blocks, i.e., tetrahedra: $s_{22}$, $s_{31}$, and the mirror reflection $s_{13}$. Finally, in 4 dimensions there are 4 types of such simplices: $s_{41}$ with its mirror-reflection $s_{14}$ and $s_{32}$ with its mirror-reflection $s_{23}$. Due to this construction and the symmetry of the action, as we will see, CDT exhibits a time reflection symmetry as well. Thanks to a small number of different categories of simplices appearing in the four-dimensional CDT and due to  topological constraints of the triangulated manifolds, see  Appendix \ref{AppendixA}, the Regge action (\ref{eq:curvature2}), which governs the dynamics of the model, can be expressed in terms of these $4$-dimensional simplices and  vertices in a triangulation $\cal T$ \cite{nonperturb}:


\begin{equation}
S_{R} = - (\kappa_0 + 6\Delta) N_0 + \kappa_4 (N_{41} + N_{32}) + \Delta N_{41},
\label{eq:ation_kappa}
\end{equation}
where $N_0= \sum s_{10}$ is the total  number of vertices, while $N_{41}= \sum (s_{41} + s_{14})$ and $N_{32} = \sum (s_{32} + s_{23})$ are the total numbers of the various types of simplices in the triangulation $\cal T$. The three bare coupling constants are $\kappa_0$, the bare inverse  Newton constant, $\kappa_4$, the bare cosmological constant, and $\Delta$, related to the asymmetry between  lengths of space-like and {time-like links in the lattice}. From now on we will refer to $N_0$, $N_{41}$ and $N_{32}$ as \textit{global numbers}.\\

The path integral of quantum gravity is formally defined as:

\begin{equation}
    \mathcal{Z}_{QG} = \int D[g_{\mu\nu}]e^{iS_{EH}[g_{\mu\nu}]} \to^{reg}\to \sum_\mathcal{T} \frac{1}{C_\mathcal{T}}e^{iS_R[\mathcal{T}]} = \mathcal{Z}_a,
\end{equation}
where $D$ is the measure term, which enables one to integrate over geometries, i.e., diffeomorphism invariant equivalence classes of smooth metrics $g_{\mu \nu}$, and $S_{EH}$ is the Einstein - Hilbert action. After the lattice regularization ($\to^{reg}$) the path integral is replaced by a sum over all possible triangulations with a measure $1/C_\mathcal{T}$, the size of the automorphism group of $\mathcal{T}$. The index $a$ in $\mathcal{Z}_a$ refers to the lattice regulator, which is the edge length of the simplices and $S_R$ is the Regge action (\ref{eq:ation_kappa}), which is the lattice-regularized version of the Einstein-Hilbert action. The distinction of space and time introduced by the foliation is also present in the edge lengths, as the time-like edge lengths $a_t$ and the space-like edge lengths $a_s$ are not necessarily the same, which gives rise to a degree of freedom, called the asymmetry parameter $\alpha$, where $-\alpha a_t^2 = a^2_s$ in the Lorentzian setting. The aim of CDT is to define the gravitational path integral, or at least approximate it as close as it gets.  All  possible triangulations  $\mathcal{T}$ include only such triangulations which respect the foliation structure and some additional topological constraints. To be able to treat the model with methods of statistical physics a Wick rotation has to be applied to the partition function to change the metrics from Lorentzian to Euclidean signature. Due to the imposed global foliation, the  "Euclideanization" of the path integral via the Wick rotation is well defined and is related to the analytic continuation of the Regge action to negative values of $\alpha$  in the lower half of the complex $\alpha$ plane. Performing it one turns the path integral into the partition function: 

\begin{equation}
\mathcal{Z}_{R} = \sum_{\mathcal{T}} \frac{1}{C_\mathcal{T}}e^{-S_{R}[\mathcal{T}]} ,
\label{eq:partfun}
\end{equation}
where, for a simpler notation, we kept the same symbol $S_R$ for the (now) Euclidean Regge action. The Wick rotation allows for the application of statistical physics methods on the model, for example, one can compute the expectation values of observables as:

\begin{equation}
\langle \mathcal{O}\rangle = \frac{1}{\mathcal{Z}} \sum_\mathcal{T} \frac{1}{\mathcal{C}_{\mathcal{T}}}\mathcal{O} e^{-S_{R}[\mathcal{T}]}    .
\end{equation}

One of the benefits of the Wick rotation is that the model became suitable for numerical Monte Carlo (MC) simulations, where the partition function can be approximated by an ensemble of configurations generated in such simulations. The past twenty years of  numerical studies of the 4-dimensional CDT model led to many interesting and important results.


\subsection{Most important previous results of CDT}

CDT was formulated in the beginning of the 21st century and became recognized by the quantum gravity community in the following years. The introduction of the foliation to the triangulation allowed for the addition of the asymmetry parameter between space and time, which was promoted to a new coupling constant $\Delta$ in the action (\ref{eq:ation_kappa}). This particular change had a huge impact on the properties of the CDT model, compared to DT, as due to the enforced causality constraint the ensemble of triangulations present in the partition function (defined by eq. (\ref{eq:partfun})) became significantly reduced. At the same time, the third coupling constant ($\Delta$) allowed for an extended view on the phase diagram of simplicial quantum gravity. There were only two phases in DT, one phase where a link of the generic triangulation gathered a significant number of simplices around itself, and its end vertices experienced a huge coordination number\footnote{The coordination number of a vertex is defined as the number of  four-simplices which share the vertex.}, comparable to the system size, thus the name "collapsed phase". The generic geometries of the other phase could be described by the so-called, branched polymers \cite{branchedpoli}, hence the name "branched polymer phase". The analogs of these phases \cite{phase_struct_cdt} are present in CDT\footnote{Phase $B$ is the collapsed phase and phase $A$ is the branched polymer phase}, however, the topological restriction related to the foliation resulted in the appearance of two new phases \cite{ccc,cb1,cb2}. This became apparent when new observables were used related to the newly introduced time foliation. The number of spatial tetrahedra at a given CDT foliation leaf (with integer lattice time $t$) can be computed and it is, by definition, proportional to the spatial three-volume at $t$, which defines the so-called, volume profile $V_3(t)$, shown in Fig. \ref{fig:volprofs}.


\begin{figure}[ht!]
\centering
\frame{\includegraphics[width=0.23\textwidth]{fazaAs.pdf}}
\frame{\includegraphics[width=0.23\textwidth]{fazaBs.pdf}}
\frame{\includegraphics[width=0.23\textwidth]{fazaCbs.pdf}}
\frame{\includegraphics[width=0.23\textwidth]{fazaCs.pdf}}\\
\frame{\includegraphics[width=0.23\textwidth]{fazaAt.pdf}}
\frame{\includegraphics[width=0.23\textwidth]{fazaBt.pdf}}
\frame{\includegraphics[width=0.23\textwidth]{fazaCbt.pdf}}
\frame{\includegraphics[width=0.23\textwidth]{fazaCt.pdf}}\\
\caption{Spatial volume profiles of generic CDT configurations in different phases. Top: Spherical CDT: $A$, $B$, $C_b$, $C$; Bottom: Toroidal CDT:  $A$, $B$, $C_b$, $C$, respectively.}
\label{fig:volprofs}
\end{figure}

Apart from a "collapsed" volume profile of phase $B$ (where all three-volume is concentrated in one spatial "slice", i.e., the 3-dimensional  foliation leaf of integer $t$), and the heavily fluctuating volume profile of the "branched polymer" phase $A$ (independent number of tetrahedra in each spatial slice) there are new phases where the  volume profiles averaged over MC configurations follow a particular smooth function. The most interesting new phase is phase $C$ , where, in the case of the fixed spherical topology of spatial slices, the resulting average volume profile behaves as $cos^3(t)$, which corresponds to the (Euclidean) de Sitter solution of GR \cite{nonperturb_desitter}. Therefore phase $C$ is also called the de Sitter or the semi-classical phase and it is related to the IR limit of quantum gravity. The fourth phase, which is called the bifurcation phase ($C_b$), exhibits a smooth volume profile in case of large enough fixed total volumes (lattice sizes). The volume profile in phase $C_b$ is similar to the volume profile in  phase $C$ measured for the spherical spatial topology, however, it scales in a non-canonical  way when the lattice volume is increased. Furthermore, in phase, $C_b$ every second  spatial slice of integer lattice time coordinate contains a vertex with a macroscopically large coordination number, similar to "high-order" vertices encountered in phase $B$.\\

By analyzing  fluctuations of the spatial volume it was possible to derive an effective action\cite{impact_top} of CDT parametrized by the spatial volume, or alternatively by the scale factor. The effective action in the de Sitter phase ($C$) \cite{transfer_matrix} turned out to be consistent with the Hartle-Hawking minisuperspace model \cite{mini1,mini2,mini3}. This result is non-trivial, as in the case of CDT the scale factor is obtained after "integrating out" all other geometric degrees of freedom present in the lattice, while in the minisuperspace model, where spacetime isotropy and homogeneity are put in by hand, the scale factor is the only degree of freedom. Therefore, this feature of CDT is fully emergent. One could also show that the notion of effective dimension of spacetime first measured in the case of 2D CDT\cite{eff_dim_2d} and also in the case of Locally Causal Dynamical Triangulations (LCDT) \cite{spectraldim} was extended to higher dimensions. In the case of 4-dim CDT in phase $C$ it was measured to be consistent with the topological dimension four. This was not so obvious as the effective dimension measured in other phases of CDT (and earlier in DT) was different than four\cite{scaling_in_4d_grav}. Both the so-called, Hausdorff dimension \cite{cdt_desit_fi}, related to the scaling of an area and volume, and the spectral dimension \cite{spect_dim_uni_scale}, defined by a heat kernel of the Laplace operator, were measured. Additionally, the spectral dimension was shown to exhibit a non-trivial scale dependence changing from four in large scales (comparable to the size of the configuration) to approximately two in short scales and also in the presence of matter fields\cite{spectral} it can deviate from the classical values. The above phenomenon  of "dimensional reduction" was also confirmed in many other approaches to quantum gravity (e.g., in ST \cite{dim_red_st}, NCG \cite{dim_red_noncommgeom},  HLG \cite{dim_red_hlg}, AS  \cite{dim_red_1,dim_red_2,dim_red_3} and LQG \cite{dim_red_4}). \\

Most of the phase transitions present in the CDT model with spherical spatial topology were analyzed, and the $A-C$ phase transition was found to be first-order \cite{ac,pts_in_cdt,phase_struct_cdt}, while the $B-C_b$ and the $C-C_b$ turned out to be continuous  \cite{pts_in_cdt,cb1}. The existence of higher order (continuous)  phase transitions is an important result in view of the perspective existence of the UV fixed point of quantum gravity\footnote{As explained in Chapter \ref{chapter3}, such a fixed point should appear as a higher order transition point in CDT. }, however, the first study of the RG trajectories in CDT \cite{rg_flow1,rg_flow2} did not show convincing evidence for the existence of the UV fixed point. One of the issues was that a part of the phase diagram was out of reach due to computational difficulties, thus the analysis of some phase transitions was not possible. Also at that time, the available computational power was significantly smaller than presently. With the help of modern technology, much larger system sizes can be analyzed nowadays within available computational resources.\\ 

Most of the results presented above were obtained for the CDT model with the fixed spherical topology of the spatial slices. As the spatial topology choice is one of the free parameters of the model, in the past few years the main focus of the 4-dimensional CDT research was on models with  toroidal spatial topology. It was found that the phase diagram is almost invariant under the change of the topology, as all observed phases were present in both cases \cite{phase_structure_torus}. A huge difference between the spherical and the toroidal case is visible in the volume profile of phase $C$, see Fig. \ref{fig:volprofs}, and it is related to the potential term appearing in the effective action of CDT, which is different in the two cases \cite{impact_top}. In the spherical case, the potential term can be interpreted as coming from GR and it is consistent with the minisuperspace model, which also contains such a potential term for the scale factor. However, in the case of the toroidal CDT, one does not have a classical analog of the measured  potential, thus it can be interpreted as a quantum correction. Using the spatial topology of a three-torus allowed for an introduction of many new methods of analyzing the lattice-regularized quantum geometries, as it will be presented in Chapter \ref{chapter4}. It was also possible to investigate the region of the phase-diagram which was thought to be not available in the spherical CDT, see Chapter \ref{chapter3}. \\

In this thesis, we will present results related to the toroidal CDT: the study of the remaining phase transitions, including critical phenomena at the phase-transition lines. Then we will also discuss how to add scalar fields to the model of CDT, and either use them as semi-classical maps defining a coordinate system on the geometry or couple them to the geometry and analyze the effects of their back-reaction. But first, let us turn our attention to the numerical implementation of the CDT model, which is the topic of the next chapter.