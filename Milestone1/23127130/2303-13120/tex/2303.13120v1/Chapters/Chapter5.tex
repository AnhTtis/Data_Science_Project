% Chapter 5

\chapter{Conclusions} % Main chapter title
\label{chapter5}
\textit{This chapter contains a summary and a description of my contribution to the publications included in Chapter \ref{chapter6}.}\\ 


In this section, we will summarize the discussion presented in the previous chapters. This thesis is supposed to be a collection of publications done during my Ph.D. studies and a guide to the presented articles together with some theoretical introduction and some additional thoughts that cannot be found elsewhere. This includes the discussion of the MC moves using a "colored dots" graph representing the discretized geometry of the $t+\frac{1}{2}$ foliation leaf, the possibility of introducing new MC moves presented in the Appendix \ref{AppendixA}, the discussion of topological relations between the triangulation parameters in the Appendix \ref{AppendixB}, and the results related to the Hausdorff dimension calculated from the scalar field distribution. All other figures and results were taken from the publications.\\

For all of the works presented in the previous chapters, I performed a significant amount of numerical simulations and did a large part of the numerical data analysis. The phase transition studies were challenging as they required simulations that lasted for several months, due to the prolonged thermalization time related to the nature of the problem. Furthermore, many measurements had to be repeated due to various technical difficulties.\\

In Chapter \ref{chapter3} (section \ref{sec:AB_trans})  I discussed the study of the $A-B$ phase transition, presented in the publication \cite{pub3}. In the case of this phase transition, I was the main contributor to the study. I decided on the analysis of this particular part of the CDT phase diagram and selected the methods and the MC simulation parameters necessary to perform the study (e.g., values of the coupling constants, values of $\bar{N}_{41}$, etc.).  Data coming from the simulations was shared between the members of the CDT group and the conclusions and results were discussed on regular group meetings. Finally, I had a large contribution to the editing of the text of the publication \cite{pub3}.\\  

In the case of the publication \cite{pub1}, described in Chapter \ref{chapter3} (section \ref{sec:cb_trans}), my main contribution was the finding that the volume profile $V_3(t)$ of the $C_b$ phase in the case of toroidal spatial topology contains an emergent blob, seemingly similar to that observed for the case of the spherical CDT. I collected evidence for that behavior and performed analysis to calculate the Hausdorff dimension of the observed triangulations.\\

After the finding that there is a direct phase transition between the $B$ and $C$ phases \cite{phase_structure_torus}, the study of the $C-B$ phase transition, discussed in Chapter \ref{chapter3} (section \ref{sec:CB_trans}) became one of the priorities of my Ph.D. research. The analysis of the $C-B$ phase transition (presented in \cite{pub2,pub3}) was one of the most demanding works of my Ph.D. It required  a large number of computer simulations  (several hundreds of MC runs) which had to be performed  in order to achieve the published results. Each of these simulations had to be overseen one by one on a regular basis and then data had to be analyzed. I had also an important contribution to  editing the text of the publications \cite{pub2} and \cite{pub3}. The findings of \cite{pub2} were unclear, as signals of the phase transition were mixed and one couldn't find its order with 100\% accuracy. Performing numerical simulations in additional locations in the phase diagram and increasing the statistics yielded similar results \cite{pub3}, however, we managed to show that introducing a discretization correction to scaling relations gives the fits compatible with the scaling exponent corresponding to a first-order  transition. The description of the nature of this phase transition, and as we understand it now also other CDT phase transitions, was facilitated by another study related to the scalar fields (publications \cite{pub4,pub5,pub6}), where the notion of emergent topology became apparent. Similarly to the effective dimensionality, discussed throughout the thesis, the effective topology of the quantum universe seems to be an emergent phenomenon, and according to our conjecture, for which we seem to find evidence, whenever a phase transition occurs between phases of different effective topology then it should be a first-order transition.\\

All the phase transitions mentioned above were analyzed in the case of empty CDT universes, which means that there was no additional matter content, only the gravitational degrees of freedom. As discussed in Chapter \ref{chapter4}, the simplest form of a matter field that can be included in our model is a massless scalar field. For all of the scalar field-related publications \cite{pub4,pub5,pub6} I contributed by performing MC simulations, data analysis, result interpretation and co-editing the articles. Additionally, I was the corresponding author of publication \cite{pub4}. The classical scalar fields, described in publications \cite{pub4,pub6}, were used as a tool to introduce a coordinate system to the CDT triangulations. Such coordinates are a (quantum) analogue of the harmonic (de Donder) gauge fixing in GR. The massless scalar fields are harmonic maps, enabling one to visualize the non-trivial fractal structure of the underlying quantum geometries. Using the mapping, the regions of the triangulation with under- and over- 4-volume density are visible, which makes it possible to observe structures resembling cosmic voids and filaments similar to the large scale structure of the Universe. One may  think of these structures, coming from the quantum fluctuations of pure gravity, as the source of  initial inhomogeneities in the matter content of the early Universe, but this idea requires further studies. These maps can be measured in all CDT phases and they reveal important differences in the geometric structures of generic triangulations observed in each phase. This observation, in particular, lead to the notion of the effective topology discussed above. The back-reaction of the scalar fields was added to the simulations and discussed in publications \cite{pub5,pub6}, where I contributed by performing the MC simulations and data analysis. Adding scalar fields with non-trivial "jump" conditions resulted in a phase transition observed for some value of the jump amplitude, see Chapter \ref{chapter4}. In the case where the field  was winding around the time direction, the phase transition led to the volume profile $V_3(t)$  consistent with a $cos(t)$ function, resulting from the minisuperspace-type approximation discussed in Appendix 3. of \cite{pub6}. In the case where three scalar fields were winding around spatial directions, the phase transition led to the "pinching" of the geometry in these directions and consequently to the effective spatial topology change from the toroidal to the spherical one. This in turn resulted in the  de Sitter type, i.e., $cos^3(t)$, volume profile, leading to  further effective topology change to that of the  four-sphere. \\ 


There is still a lot to be investigated in the CDT phase-diagram. Especially, the open question remains if there exists the UVFP of CDT. Without such a UVFP CDT can be at most treated as some effective quantum gravity model, valid only to some energy scale, but not a fundamental non-perturbatively renormalizable theory of quantum gravity. Potentially some kind of extension of the model  is needed to be able to obtain such a UVFP. An extension may come from the introduction of new parameters to the bare Regge action $S_{R}$, discussed in Appendix \ref{AppendixB}, although such a change should be well motivated, and some physical quantities related to the new parameters have to be found. Another  extension, which may possibly yield the UVFP, can potentially come by adding various matter content. For example, adding gauge fields is currently the topic of an ongoing study, but it is at a preliminary stage and therefore it will not be discussed in this thesis. \\


Summing up, there is still  plenty of directions which future research of CDT can follow in the quest for understanding quantum gravity. All I can say is that I am proud that, with the results presented in this thesis, I could participate in the development of the theory which has the potential to become  widely accepted theory of quantum gravity.\\ 

\chapter{Publications} % Main chapter title
\label{chapter6}

\textit{This chapter contains publications constituting the main part of the PhD thesis. The order of publications, as it was mentioned in Chapter \ref{Chapter0}., is as follows:}\\ 

\begin{enumerate}
    \item[\cite{pub1}]  J. Ambjorn G. Czelusta et al. “The higher-order phase transition in toroidal CDT”. In: J. of High Energ. Phys. 2020 (5), p. 30.\\DOI: 10.1007/JHEP05(2020)030
    \item[\cite{pub2}] J. Ambjorn et al. “Towards an UV fixed point in CDT gravity”. In: Journal of High Energy Physics 2019 (7), p. 166.\\ DOI: 10.1007/JHEP07(2019)166
    \item[\cite{pub3}]  J. Ambjorn et al. “Topology induced first-order phase transitions in lattice quantum gravity”. In: Journal of High Energy Physics 2022 (4), p. 103.\\ DOI: 10.1007/JHEP04(2022)103.
    \item[\cite{pub4}] J.Ambjorn et al. “Cosmic voids and filaments from quantum gravity”. In: The European Physical Journal C 81 (8 2021), p. 708.\\ DOI: 10.1140/epjc/s10052-021-09468-z
    \item[\cite{pub5}] J. Ambjorn et al. “Matter-Driven Change of Spacetime Topology”. In: Phys. Rev. Lett. 127 (16 Oct. 2021), p. 161301.\\ DOI: 10.1103/PhysRevLett.127161301
    \item[\cite{pub6}]  J. Ambjorn et al. “Scalar fields in causal dynamical triangulations”. In: Classical and Quantum Gravity 38 (19 Sept. 2021), p. 195030.\\ DOI: 10.1088/1361-6382/ac2135
\end{enumerate}

Pub. \cite{pub1}: Discovery of a scientific result published in the paper. Performing numerical simulations, analyzing the data, and discussing the results. Estimated contribution: 20\%.\\

Pub. \cite{pub2}: Conducting the main research, including the maintenance of numerical simulations, performing the analysis of the data, discussing the results, and writing the initial version of the publication. Estimated contribution: 70\%.\\

Pub. \cite{pub3}: Conducting the main research, including the maintenance of numerical simulations, performing the analysis of the data, discussing the results, and writing the publication together with coauthors. Estimated contribution: 75\%.\\

Pub. \cite{pub4}: Performing numerical simulations, providing data to the collaborators, discussing the results, and writing the initial version of the publication. I was the corresponding author of the paper. Estimated contribution: 20\%.\\

Pub. \cite{pub5}: Performing numerical simulations, providing data to the collaborators, doing some part of the data analysis, and discussing the results. Estimated contribution: 30\%.\\

Pub \cite{pub6}: Performing numerical simulations, providing data to the collaborators, doing some part of the data analysis, and discussing the results. Estimated contribution: 30\%.\\

\section{Publications}

%\includepdf[pages=-]{Pubs/pub1.pdf}
%\includepdf[pages=-]{Pubs/pub2.pdf}
%%\includepdf[pages=-]{Pubs/pub3.pdf}
%\includepdf[pages=-]{Pubs/pub4.pdf}
%\includepdf[pages=-]{Pubs/pub5.pdf}
%\includepdf[pages=-]{Pubs/pub6.pdf}