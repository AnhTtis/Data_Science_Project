% Chapter 4

\chapter{Universes with matter fields} % Main chapter title
\label{chapter4}  
\textit{This chapter gives a brief summary of the following articles: \cite{pub4,pub5,pub6}}.\\\\

\section{Scalar fields as coordinates}
\textit{"If people do not believe that mathematics is simple, it is only because they do not realize how complicated life is."} - \textbf{Neumann János}\\\\ 

\textit{This section is based on the publications \cite{pub4,pub5}.}\\\\

Life (the actual physical phenomenon) is indeed complicated, much more complicated than any  model designed in order to describe it. However, usually  simple models are the only ones that can be solved. Also in most physical theories vacuum solutions are the simplest, easiest and first ones to be found. The same is true in GR as most of known solutions of Einstein's field equations are vacuum solutions. It is also the case of CDT, where the first  twenty years of studies  were dominated by pure gravity models  (empty Universes) discussed above. As it was mentioned in Chapters \ref{Chapter1} and \ref{chapter2}, CDT is formulated in  a coordinate-free way, except from the time direction where one has a natural global proper-time coordinate $t$, consistent with the introduced foliation. It would be therefore beneficial to introduce some notion of coordinates, making contact with other gravitational research.\\

The simplest extension of the CDT model is the addition of massless scalar fields. As will be shown, such scalar fields can also play role of "clocks" and "rods", enabling one to define a coordinate system in the triangulated manifold, being an analogue of the harmonic (de Donder) gauge fixing in GR. As we will discuss below, in order to apply this method of defining coordinates in CDT one also has to make a proper choice of the target space of the scalar fields. As an example, take a (smooth) Riemannian manifold $\mathcal{M}$ equipped with a metric tensor $g_{\mu\nu}$ and another Riemannian manifold $\mathcal{N}$ with a trivial flat metric $h_{\alpha\beta}$. A harmonic map $\mathcal{M} \to \mathcal{N}$ can be defined with the help of a four-component scalar field $\phi^\alpha$, with $\alpha = 1, 2, 3, 4$. In case of our setup, if $\mathcal{M}$ has a topology of the four-torus $T^4$, then we also choose $\mathcal{N}$ to have the same toroidal topology, and each component $\phi^\alpha(x)$ is a map $\mathcal{M} \to S^1$, which minimizes the action

\begin{equation}
\label{eq:scalar_action}
S_M[\phi]	= \frac{1}{2} \int \mathrm{d}^{4}x \sqrt{g(x)} \;g^{\mu \nu} (x) \; h_{\alpha \beta}(\phi^\gamma(x)) \;\partial_\mu \phi^\alpha(x) \partial_\nu \phi^\beta(x).
\end{equation}

Due to our   choice of  the trivial metric $h_{\alpha \beta}$ on $\mathcal{N}$, the four-component scalar field can be decomposed into four independent components, later denoted $\phi^x$, $\phi^y$, $\phi^z$, and $\phi^t$. Due to this, it is enough to discuss the case of a single component (let's call it  $\phi$). The Euler-Lagrange equations for the field resulting from eq. (\ref{eq:scalar_action}) give rise to the Laplace equation:

\begin{equation}
\Delta_x \phi (x) =0, \quad 
\Delta_x = \frac{1}{\sqrt{g(x)}} \;\frac{\partial}{\partial x^\mu} 
\Big(\sqrt{g(x)} \,
g^{\mu\nu}(x)\Big) \frac{\partial}{\partial x^\nu},
\quad \phi(x) \in S^1.
\label{eq:laplace}
\end{equation}

In the case when $\mathcal{M}$ is closed, if we chose the target space of the scalar field $\phi$ to be $\mathbb{R}$ then the constant zero-mode of the Laplacian would be the only solution to the equation $\Delta_x \phi(x) = 0$. If instead, as we do, one chooses a nontrivial target space of the field to be $S^1$ (with circumference $\delta$) then one can obtain a nontrivial solution for the scalar field. Technically, the  condition $\phi(x)\in S^1$ can be obtained by considering a scalar field with the target space $\mathbb R$ and identifying 

\begin{equation}
\phi(x) \equiv \phi(x) + n \, \delta,\quad n \in \mathbb{Z}.
\end{equation}
The situation of interest is when we have the toroidal manifold $\mathcal M$ which can be thought of as an elementary cell  periodically repeating in all four directions. In such a case one can define four non-equivalent boundaries of the elementary cell,  i.e. 3-dimensional connected hypersurfaces $H(\alpha), \alpha = \{x,y,z,t\}$. Let us consider  the case when each component of the field $\phi^\alpha \in S^1$ winds around the circle once as we go around any non-contractible loop in  $\mathcal M$ that crosses a boundary in direction $\alpha$. In that case the field $\phi$ is a continuous function  except when one crosses the hypersurface $H(\alpha)$, where a jump of the field with amplitude $\delta$ happens, and the Laplace equation (\ref{eq:laplace}) acquires a nontrivial boundary term leading to a non-trivial solution for the field $\phi$. A corresponding function that is continuous despite the jump, will be a map

\begin{equation}
\phi \to \psi = \frac{\delta}{2\pi}\;e^{2\pi i \phi/\delta},
\label{eq:psi}
\end{equation}
which maps the scalar field $\phi$ to a circle in the complex plane. The interesting point is that, for a given direction $\alpha$, the  map $\psi$ does not depend on the exact choice of the boundary $H(\alpha)$ of the elementary cell.\footnote{Formally it depends only trivially, i.e., a continuous deformation  (a "shift") of the boundary $H(\alpha)$ will only cause a shift of the phase in the complex function $\psi$ by some constant.} \\

In CDT we  consider a discretization of the action (\ref{eq:scalar_action}) and the corresponding Laplace equation (\ref{eq:laplace}), where the field is localized in the center of simplices. We therefore consider a (discretized) Laplacian defined on the dual lattice, i.e., the graph whose  vertices represent the 4-simplices of the original CDT triangulation, and links represent the common interfaces between the 4-simplices in the triangulation. The Laplacian on the dual lattice can be defined via the adjacency matrix $A_{ij}$:

\begin{equation}
A_{ij} =
\begin{cases}
	1 & \textrm{if (the link } i \leftrightarrow j) \in \textrm{dual lattice},\\
	0 & \textrm{otherwise},
\end{cases}
\end{equation}
where $\leftrightarrow$ refers to the adjacency relation of simplex $i$ and $j$. The discrete Laplacian can be defined as:

\begin{equation}
L =  D - A,
\end{equation}
where $A$ is the adjacency matrix and $D$ is a diagonal matrix with $i$-th diagonal element containing the number of neighbors of a simplex labelled $i$. As, in the four-dimensional CDT, each simplex in the  triangulation has exactly 5 neighbors, the dual lattice of any triangulation is a five-valent graph, and therefore
\begin{equation}
    D = 5 \cdot I,
\end{equation}
with $I$ being the identity matrix of size $N_4 \times N_4$, where $N_4$ is the number of all 4-simplices in the triangulation. The discretized form of the scalar field action is then given by:

\begin{equation}
\label{eq:scalar_discrete}
S_M^{CDT}[\{\phi\},\mathcal{T}]	= \frac{1}{2} \sum_{i \leftrightarrow j} (\phi_i - \phi_j)^2 = \sum_{i,j} \phi_i L_{ij} \phi_j \equiv \phi^T L \phi, 
\end{equation}
where $\mathcal{T}$ underlines the impact of the triangulation on the Laplacian matrix $L$. The discrete analog of the Laplace eq. (\ref{eq:laplace}) is then: 

\begin{equation}
    L\phi = 0.
\end{equation}
The above equation has the same issue as before, i.e., if the target space of the field was chosen to be $\mathbb R$ then it would only have a trivial solution  $\phi = const$. Non-trivial solutions can be  found by choosing the field to take values in $S^1$ with circumference $\delta$, which winds around the circle once as one goes around any non-contractible loop in the dual lattice. In order to do  that one  identifies:

\begin{equation}
\phi_i \equiv \phi_i + n \cdot \delta, 
\quad n \in \mathbb{Z}. %\quad \forall i \in {\cal T},
\end{equation}
This can be achieved by adding a jump condition when crossing a boundary  hyper-surface $H(\alpha)$ in direction, $\alpha$. The way of  introducing such boundary hypersurfaces to CDT was proposed in \cite{boundaries}. As already mentioned, the exact position of the (four non-equivalent) boundaries $H(\alpha)$ in the triangulation is not important as it has only a trivial impact on our solutions, thus the boundaries are non-physical. Technically, one can define the "jump" condition by introducing the boundary jump matrix $B_{ij}$:

\begin{equation}
B_{ij} =
\begin{cases}
	+1 & \textrm{if the dual link i} \rightarrow \textrm{j crosses the boundary $H(\alpha)$ in the \textit{positive} direction},\\
	-1 & \textrm{if the dual link i} \rightarrow \textrm{j crosses the boundary $H(\alpha)$ in the \textit{negative} direction},\\
	0 & \textrm{otherwise}
\end{cases}
\end{equation}
and defining
\begin{equation}
V = \frac{1}{2}  \sum_{ij} B_{ij}^2= \frac{1}{2} \sum_i | b_i |, 
\end{equation}
where $b_i = \sum_j B_{ij}$ is the boundary jump vector, and it measures the occasions when a tetrahedral face of a simplex $i$ appears on the boundary. To accommodate to the jump condition we modify the discretized matter action:

\begin{multline}
\label{eq:scalar_w_boundary}
S_M^{CDT}[\{\phi\}, \mathcal{T}]= \frac{1}{2} \sum_{i \leftrightarrow j} (\phi_i - \phi_j - \delta B_{ij})^2 = \sum_{i,j} \phi_i L_{ij} \phi_j - 2 \delta \sum_i \phi_i b_i +\delta^2 V \\ \equiv \phi^T L \phi - 2 \delta \phi^T b + \delta^2 V.
\end{multline}
Now, the Euler-Lagrange equation for the field $\phi$ yields:

\begin{equation}
 L \phi	= \delta \, b,
\end{equation}
so it acquires a non-trivial boundary term: $\delta \ b$.
The classical solution to the scalar field distribution  is formally given by 

\begin{equation}
    \phi^{classical} =\delta \ L^{-1}  b.
\end{equation}
The practical problem is that the Laplacian has zero modes but, fortunately, one can find a solution in the subspace orthogonal to the zero modes. The solution strongly depends on the underlying triangulated geometry and it smoothly interpolates between the boundaries of the (toroidal) elementary cell. In the publication \cite{pub4} we proposed to treat the harmonic map $\phi^{classical}$, or rather the resulting map $\psi^{classical}$, see eq. (\ref{eq:psi}), as a coordinate in the direction $\alpha$. This way one can introduce a coordinate system for every triangulation generated in the MC simulations. The coordinates can be used to visualize the differences between generic triangulations of different CDT phases. It is worth mentioning that the harmonic maps (coordinates) described above have a very good property of smoothly interpolating between the 4-simplices in the geometric  outgrowths, which commonly appear in the CDT triangulations forming fractal structures. Imagine such an outgrowth consisting of many simplices and linked to the rest of the triangulation by only  a few simplices. Due to  the properties of harmonic maps, all simplices in the outgrowth will have almost the same value of the field $\phi^{\alpha}_i$ in all $\alpha$-directions. Therefore the outgrowths should appear as the field condensations in the harmonic maps.


\begin{figure}[h]
    \centering
    \includegraphics[width = 0.6\textwidth]{Figures/maps.pdf}
    \caption{The 4-volume density map projected on two spatial ("$x$" and "$y$") directions measured in phase $C$ ($\kappa_0 = 4.0, \Delta= 0.2, T = 20, \bar{N}_{41} = 720k$). Each point on the plot represents a 4-simplex having the scalar field values (coordinates) $(\phi^x,\phi^y)$. The colors encode the time coordinate $t$ of the original CDT foliation. The dense regions are geometric fractal     outgrowths in the triangulation. The outgrows structure resembles cosmic voids and filaments of the real Universe.}
    \label{fig:density_map}
\end{figure}

A typical map (projected on two spatial directions: "$x$" and "$y$") measured in the semi-classical phase $C$ of the toroidal CDT is presented in Fig. \ref{fig:density_map}. Looking at Fig. \ref{fig:density_map} it becomes apparent that the phase $C$ generic triangulations represent a homogeneous and isotropic geometry on macroscopic scales. However, exactly as it is observed in the real Universe, there are local density fluctuations (geometric outgrows in the case of CDT) showing very non-trivial voids and filaments structure. One should note that in this context this is the emerging feature of pure quantum gravity, as the scalar fields discussed above do not have any impact (back-reaction) on the geometry, and are simply introduced for visualization purposes. Similar maps, obtained for generic triangulations of other CDT phases have completely different shapes, as discussed in publication \cite{pub6}.\\ 

Using the scalar fields as coordinates one can also measure the scaling of 4-volume in a triangulation by picking a (random) center and following a diffusion wave from that center and observing the growth in the volume of the diffusion shell. Looking at the scaling of the volume with radius one can measure the, so-called, Hausdorff dimension, associated with the harmonic coordinates. This was measured for the following fixed lattice volumes $\bar{N}_{41} = \{80k, 160k, 200k, 240k, 300k, $ $360k,400k, 480k, 560k, 600k, 720k\}$. The 4-volume contained in a box (window) of size $\Delta\phi^x\times\Delta\phi^y\times\Delta\phi^z \times \Delta \phi^t$, denoted $\Delta N_{win}$, normalized by the total volume $N_{tot}$ can be used to measure the Hausdorff dimension. It was found that in   phase $C$ one obtains a universal behavior, as presented in Fig. \ref{fig:inc_vol_prof}. The fitted Hausdorff dimension is consistent with $d_H = 4$. 

 
\begin{figure}[h]
    \centering
    \includegraphics[width = 0.8\textwidth]{Figures/4D_varvols_C.pdf}
    \caption{The figure shows the ratio of $\Delta N_{win}$ (4-volume inside the box of size $\Delta\phi^x\times\Delta\phi^y\times\Delta\phi^z \times \Delta \phi^t$) and $N_{tot}$ (total volume) in the function of the normalized size of the box (Radius). This function was measured in measured in phase $C$ ($\kappa_0 = 4.0, \Delta = 0.2, T = 20$). The various thin lines denote measurements for different lattice volumes $\bar N_{41}$, the solid blue line is a fit of the function $a r^b$ to their average.}
    \label{fig:inc_vol_prof}
\end{figure}

\newpage

\section{Dynamical scalar fields}

\textit{"The effort to understand the Universe is one of the very few things that lifts human life a little above the level of farce, and gives it some of the grace of tragedy."} - \textbf{Steven Weinberg}\\\\ 

\textit{This section is based on the publications \cite{pub5,pub6}.}\\\\

So far the back-reaction of the matter field on the purely geometric degrees of freedom was not taken into account. Including back-reaction of quantum (later also called dynamical) scalar fields can lead to nontrivial changes in the geometry. In the results presented below, the scalar fields are massless scalar fields with the (discretized) action (\ref{eq:scalar_w_boundary}), minimally coupled to the geometric (Regge) action (\ref{eq:ation_kappa}). Including such fields in the MC, simulations mean that not only do the field values have to be generated - in the MC simulations the heat bath method \cite{hb1,hb2} was used - but also that the matter action will affect the probability of performing the purely geometric moves. Depending on the parameters of a simulation, either the geometric or the matter part of the action dominates, thus one can expect a phase transition of some sort when moving in the  parameter space, now also including a new coupling constant corresponding to the circumference $\delta$ of the $S^1$ target space (or alternatively the jump amplitude) of the scalar field. \\
\begin{figure}[h]
    \centering
    \includegraphics[width = 0.45\textwidth]{Figures/jump_time.pdf}
    \includegraphics[width = 0.45\textwidth]{Figures/jump_space.pdf}
    \caption{The volume profile in the presence of one field winding around the time direction (left) and three fields winding around the non-equivalent spatial directions (right).}
    \label{fig:field_prof}
\end{figure}

The choice of the $\delta$ value is not the only additional parameter, as one can also choose the number of $\phi$-fields, as well as the number and type (time- or space-like) of  non-equivalent winding directions for the scalar field(s). Adding a field with $\delta = 0$ has already a visible but small effect, as it shifts some characteristics of generic triangulations appearing in the path integral, for example, it lowers the ratio of ${N}_{32}/N_{41}$, and adding more fields the effect is larger. A much stronger effect is observed for large jump magnitude $\delta$. As already discussed, if no scalar fields are added, the measured volume profile of the toroidal CDT model in the semi-classical  phase $C$ is a constant function. This is also the case of CDT coupled to the scalar fields with zero or small jump magnitude $\delta$, but for large $\delta$ one observes a dramatic change in the volume profile, as it is shown in Fig. \ref{fig:field_prof}. In the case with one scalar field winding around the time direction, using a simple minisuperspace-like model presented in Appendix  3 of \cite{pub6}, one can expect to observe a "pinched"  volume profile, turning the constant function into a $cos(t)$ function, as seen in the left panel of Fig. \ref{fig:field_prof}.  
On the other hand, as can be seen on the right panel of Fig. \ref{fig:field_prof}, the jump condition introduced only in the spatial directions will also trigger, for large-enough $\delta$ values,  a  kind of "pinched" volume profile, however, the reason behind it is different. In that case, the fitted volume profile is given by a  $cos^3(t)$ function, which corresponds to the volume profile of the (Euclidean) de Sitter sphere.

\begin{figure}[h]
    \centering
    \includegraphics[width = 0.6\textwidth]{Figures/maps_dyn.pdf}
    \caption{The 4-volume density map projected on two spatial ("$x$" and "$y$") directions measured in phase $C$ ($\kappa_0 = 2.2 \Delta = 0.6, T = 4, \bar{N}_{41} = 160k$) in the presence of 3 scalar fields winding around  non-equivalent spatial directions ($\delta = 1.0$). Each point on the plot represents a 4-simplex having the scalar field values (coordinates) $(\phi^x,\phi^y)$.  The 4-volume is concentrated in the center of the plot, and the low-density region around it shows the "pinching"  effect, leading to the effective  spatial topology change.}
    \label{fig:density_map_din}
\end{figure}

Qualitatively, the same kind of "pinching" happens in the spatial directions, leading to the effective topology change from toroidal to spherical. By the effective topology change we mean a situation where there is still a remnant of the original CDT topology (which by definition cannot change in the MC simulations), but, due to the "pinching", the toroidal part is of  cutoff size, and the dominating geometry has (almost) spherical topology. So effectively, the triangulations start to behave as if the topology of spatial slices was spherical instead of toroidal. It triggers an additional effect, also observed in the spherical CDT, leading to the non-trivial de Sitter-like  volume profile of $\cos^3(t)$, i.e., it causes a "pinching" in the time direction, changing the effective topology to that of $S^4$. Consequently, the toroidal CDT model with  scalar fields winding around spatial directions  behaves effectively as the spherical CDT model. \\ 

Summarizing, the presence of the dynamical scalar fields with a non-trivial  jump  condition (or alternatively a nontrivial target space $S^1$) can trigger a phase transition, which effectively changes the topology of the CDT configurations. Fig. \ref{fig:density_map_din} shows the 4-volume density map (projected to the "$x$" and "$y$" spatial directions) of a generic triangulation in phase $C$ in the presence of 3 scalar fields with large jump magnitude ($\delta = 1.0$) winding around 3 non-equivalent spatial directions.  Most simplices are concentrated in the center of the plot and at the edge of the plot the density becomes much smaller. This is exactly the "pinching" effect, leading to a formation of a single large geometric outgrowth, where almost all 4-volume is concentrated, and therefore changing the effective topology from the toroidal to the spherical one. The geometry looks considerably different than that of the pure gravity model, presented in Fig. \ref{fig:density_map}, where the scalar fields were used only as maps and did not have any back-reaction impact on the underlying manifold. 