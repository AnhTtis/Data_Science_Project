% Chapter 2

\chapter{Numerical Simulations} % Main chapter title

\label{chapter2}  

%---------------------------------------------------------------------------------------

\section{The Numerical Setup}

\textit{"The student should not lose any opportunity of exercising himself in numerical calculation and particularly in the use of logarithmic tables. His power of applying mathematics to questions of practical utility is in direct proportion to the facility which he possesses in computation."  - \textbf{Augustus De Morgan}}\\\\\\

In the case of the four-dimensional CDT, there is no analytical solution, however, certain numerical methods provide useful tools in the quest to find out more about the nature of the model. One of those tools is a Monte Carlo (MC) simulation \cite{tellerteller}. In an MC simulation one attempts to numerically approximate the path integral or rather, in the Euclidean formulation, the partition function of eq. (\ref{eq:partfun}), and estimate expectation values or correlators of various observables based on a sample of independent configurations generated with a probability proportional to the Boltzmann weight: $\exp(-S)$. There are various algorithms enabling to the realization of this goal. In this discussion, we will present the Metropolis Algorithm, as this is the one that is used in the case of four-dimensional CDT. One starts from any initial state of the model (in the CDT case any allowed triangulation with a given fixed topology\footnote{In practice one usually uses an initial configuration which is easy to be constructed "by hand".}), and applies a set of local changes (moves) transforming state $A$ to $B$. In order to ensure that the probability of generating a state converges to the required equilibrium probability $\propto \exp(-S)$, the probability of performing the move has to satisfy the detailed balance condition:

\begin{equation}
\mathcal{P}(A)\mathcal{W}(A \to B) = \mathcal{P}(B)\mathcal{W}(B \to A),     
\end{equation}
where $\mathcal{P}\propto e^{-S}$ is the probability distribution of a state and $\mathcal{W}$ is the transition probability from one state to another. Additionally, the moves have to be selected in such a way, that all possible states can be reached with a finite number of performed moves, in other words, the configuration space should be closed with respect to the selected moves. This condition provides ergodicity, which is crucial to ensure meaningful statistics. In the Metropolis algorithm the transition probability $\mathcal{W}$ is chosen to be  

\begin{equation}
    \mathcal{W}(A \to B) = min\{1, e^{-\Delta S}\},
\end{equation}
where $\Delta S = S(B)-S(A)$ is the change of the action by the move.\footnote{In the case of CDT the transition probability $\mathcal{W}$ depends also on a "geometric" factor related to the number of possible locations in a triangulation where the move and its inverse can be performed.} As already mentioned,  after the so-called {\it thermalization} period, the probability distribution of configurations generated by the Metropolis algorithm reaches an equilibrium defined by the  partition function (the action $S$). This is the point when one can start collecting a sample of configurations, which has to be large enough to ensure good statistics of the measured observables.\\

CDT is perfectly suitable for numerical simulations due to its relatively simple construction. The foliated space-times (MC states) are constructed by gluing the four-dimensional simplicial building blocks, presented in Fig. \ref{fig:building-blocks}, to each other, fulfilling some global and local constraints (discussed later in detail).


\begin{figure}[ht]
    \centering
    \includegraphics[width = 0.8\textwidth]{Figures/4d-simplices.pdf}
    \caption{Two building blocks of the triangulation. The left simplex is $s_{41}$ and the right one is $s_{32}$. The other two types, $s_{14}$ and $s_{23}$, are mirrored-symmetric versions of them.}
    \label{fig:building-blocks}
\end{figure}

 Due to the nature of the triangulation, every simplex has exactly five neighbors, thus the local neighborhood of a hinge (i.e. a triangle in the four-dimensional CDT) can be simply discussed. The MC moves used in CDT are based on the so-called Pachner (or Alexander) moves \cite{pachner}, modified in such a way that, as shown later, the foliation structure with the fixed  topology of each spatial slice is conserved. Within our simulation code, we keep track of the vertices forming the 4-simplices and adjacency relations between the simplices. This information is enough to reconstruct the whole triangulation. Nevertheless, in order to optimize and speed up the code, we also keep track of some additional information, e.g., some specific types of sub-simplices or their coordination numbers.\footnote{The coordination number  measures how many $4$-dimensional simplices meet at a given vertex, link or triangle.} When we perform a measurement we usually have to calculate observables from the actual adjacency relations or other data that we store. As Fig. \ref{fig:building-blocks} shows, the graphical representation of simplices on a 2-dimensional figure is difficult. Due to this reason, we will present the idea of the moves of 4-dimensional CDT by showing how they impact the triangulation at the $t+\frac{1}{2}$ plane. Technically the $t+\frac{1}{2}$ plane describes the connectivity structure of the triangulation in a {\it slab}, defined by all simplices between spatial slices at (integer) lattice time $t$ and $t+1$. This treatment simplifies the discussion as it reduces the dimensionality of the problem by one, because on the $t+\frac{1}{2}$ plane a slab of the 4-dimensional triangulation is mapped to a 3-dimensional graph decorated by colors. The construction of the building blocks of this 3- dimensional graph is analogous to the method used in \cite{abab} in the case of three-dimensional CDT. Specifically, the color or, in other words, the type of the links (solid black/grey or dashed) is important, as only links of the same type can be connected. When referring to the links of such a graph, the words "color" or "type" will be used interchangeably.

\begin{figure}[ht]
    \centering
    \includegraphics[width = 0.8\textwidth]{Figures/simplices_thalf.pdf}
    \caption{The figure shows the representations of $s_{41}$ (left) and $s_{32}$ (right) simplices in  the $t + \frac{1}{2}$ plane. The simplex $s_{41}$ is a single-colored tetrahedron, $s_{32}$ is a bi-colored prism with two triangular and three rectangular faces. The other two types of simplices $s_{14}$ and $s_{23}$ are mirror-reflections.}
    \label{fig:projection}
\end{figure}

In the $t+\frac{1}{2}$ plane, instead of the 4-simplices,  we now have 3-dimensional objects:  tetrahedra and prisms, see Fig. \ref{fig:projection}. In order to distinguish between the $s_{41}$ and the $s_{14}$ simplices we  attribute colors to the tetrahedra, such that a tetrahedron belonging to the $s_{41}$ simplex will have all-black (triangular) faces, and all triangles of a  tetrahedron belonging to the  $s_{14}$ simplex will be  grey. Even though each 4-simplex  has formally 5 neighbors, the tetrahedra have only four, which means that (for better clarity of the graphs) we omit the neighbors belonging to the previous/next slab. The prisms with black/grey triangular faces and transparent rectangular sides represent the $s_{32}/s_{23}$ simplices, respectively. Due to the topological constraints imposed on the CDT triangulations (the fixed spatial topology must be preserved  in all  layers interpolating between the spatial slices at $t$ and $t+1$)  the discretized geometry of the $t+\frac{1}{2}$ layer must be also connected in a specific way, meaning that a black triangle can be glued only to a black triangle, a grey triangle to a grey triangle, and a  transparent rectangle to a  transparent rectangle. It reflects the fact that the $s_{41}$ simplex can be adjacent only to  $s_{41}$ or $s_{32}$ simplices\footnote{Here we disregard the connections to the $s_{14}$ simplices of the previous slab.}, the $s_{32}$ simplex can be adjacent only to $s_{41}$, $s_{32}$ or $s_{23}$ simplices, etc.  


\begin{figure}[h]
    \centering
    \includegraphics[width = 0.45\textwidth]{Figures/32.pdf}
    \includegraphics[width =0.45\textwidth]{Figures/40.pdf}
    \caption{The figure presents the
    3-dimensional elements of the $t + 1/2$ plane. The prism with triangular bases and rectangular sides (left panel) comes from a $s_{32}$ simplex. In the graphical representation, it will be a blue dot with two solid black and three dashed legs. The tetrahedron  (right panel) comes from a $s_{41}$ simplex. In the graphical representation, it will be a black dot with four solid black legs (connections to neighboring slabs are omitted). Similarly, one has a red dot with two solid grey and three dashed legs, and a grey dot with four solid grey legs, coming from the mirror-reflected $s_{23}$ and $s_{14}$ simplices, respectively, which are not shown in the plot.}
    \label{fig:P_types}
\end{figure}

In order to simplify notation we will represent the black/grey tetrahedra by the black/grey dots, and the prisms by the blue/red dots, such that a blue dot represents a prism with two black triangles (and three transparent rectangles) and a red dot is a prism with  two grey triangles (and also three transparent rectangles). In the 4-dimensional context, the black/grey dots will correspond to the $s_{41}$ / $s_{14}$ simplices in the slab, and the blue/red dots will correspond to the $s_{32}$ / $s_{23}$ simplices, respectively. The dots will be connected by "legs" of various types, representing the different types of connections (through colored triangles or rectangles) in the $t+\frac{1}{2} $ plane. Thus a solid black/grey  leg will represent a black/grey triangle, and the dashed leg will be a transparent rectangle, see Fig. \ref{fig:P_types}. In order to preserve the topological restrictions, only the legs of the same color/type can be connected. All possible  connections between colored dots are presented in Fig \ref{fig:connectivities} (up to mirror-reflections). 
\begin{figure}[h!]
    \centering
    \includegraphics[width =0.8\textwidth]{Figures/Connection_types.pdf}
    \caption{The figure presents  possible connections between various objects of the $t+\frac{1}{2}$ plane. Black dots (tetrahedra) can be connected to each other and to blue dots (prisms) via solid black legs (triangles). Similarly, blue dots can be connected to black and blue dots via solid black legs, but they can be also connected to blue and red dots via dashed legs (rectangles). The red dots have two solid grey legs, which can be connected to other red or grey dots, which are not shown in the figure.}
    \label{fig:connectivities}
\end{figure}
As, by definition, the manifold constraints of the original triangulation are not violated, and the description of the triangulation in the $t+\frac{1}{2}$ plane is still a manifold (a three-dimensional one), it is in one-to-one correspondence with the transition tensor of the triangulation from slice $t$ to $t + 1$. An example (part of the) $t+\frac{1}{2}$ slice of a CDT triangulation and the corresponding graph with colored dots and various types of legs is presented 
in Fig.~\ref{fig:bbbbr_example}. \\
\begin{figure}[h]
    \centering
    \includegraphics[width =0.8\textwidth]{Figures/example_11.pdf}
    \caption{An example of a possible connection between four $s_{41}$, three $s_{32}$ and one $s_{23}$ simplices in the $t+\frac{1}{2} $ plane (left panel) and the corresponding graphical representation (right panel). A solid black loop in the graphical representation  is a spatial link in the  CDT triangulation. I deleted here a sentence with "red triangles" - you did not introduce such triangles in the description - it becomes a mess}
    \label{fig:bbbbr_example}
\end{figure}

One should note that if, in the original CDT triangulation, two $s_{41}$ simplices are connected to the same vertex at $t+1$ then these simplices correspond necessarily to two adjacent tetrahedra in the $t+\frac{1}{2}$ plane, or in the graphical representation two black dots connected by a solid black line. The same is of course true for the mirror-reflected $s_{14}$ simplices and thus the grey dots connected by a solid grey line. Additionally, using the graphical representation one can recognize the links of the original CDT triangulation as closed loops in the colored dot graphs. Closed solid loops are spatial links (black on slice $t$ and grey on slice $t+1$), while closed dashed loops are time-like links of the original triangulation. Then, the coordination number of a link in the original triangulation is related to the number of dots along that loop. Another important feature of this graphical representation is, that the vertices of the original triangulation are represented as 3-dimensional objects defined by the surrounding colored dots and closed loops. As it was already mentioned, the above graphical representation contains only elements of the $t + \frac{1}{2}$ plane of a slab, therefore the true coordination number of spatial links will actually also depend on a similar graph in the adjacent slab.\\

As the CDT moves are local, i.e., they change only the interior of a small region in a CDT triangulation, the connection to the outside region of the triangulation is preserved, which, in the graphical representation, manifests itself by the fact that the type and number of  external legs remain unchanged when the move is performed. \\

Now, we are ready to discuss the moves with the graphical representation defined above. In the following discussion, if only black or black-and-blue dots are shown, then recoloring black to grey and blue to red will lead to the mirror-reflected version of the movie. 
 
\subsection{Move-2}


\begin{figure}[h]
    \centering
    \includegraphics[width =0.45\textwidth]{Figures/move2.pdf}
   \includegraphics[width = 0.45\textwidth]{Figures/move2v2.pdf}
    \caption{Move-2: version-1 (left) and version-2 (right). In the CDT triangulation, it replaces a (tetrahedral) interface between 4-simplices with a link, creating additional three 4-simplices.}
    \label{fig:m2}
\end{figure}

"Move-2" is a move that changes the interface between two (black-blue or blue-red) dots and increases the number of dots by two.  It exists in two versions. Version one can be done between a black and a blue dot. The move removes two dots and replaces them with four, see Fig. \ref{fig:m2}. After the move, the black dot will be connected to the external leg, which was earlier connected to the blue dot, and, at the same time, all of the original black dot's external legs will become the external legs of the three new blue dots. These blue dots are also connected via dashed legs. The second version of the movie can be done between a blue and a red dot. The move replaces the dashed line between the original blue and red dots with four dashed lines between the  blue and red dots. These new blue and red dots are connected to the external dashed legs of the original configuration.

\subsection{Move-3}



\begin{figure}[h]
    \centering
    \includegraphics[width = 0.45\textwidth]{Figures/move3.pdf}
    \includegraphics[width = 0.45\textwidth]{Figures/move3_v2.pdf}
    \caption{Move-3: version 1 (left) and version 2 (right). In the CDT triangulation, it replaces the triangular interface with a dual one. }
    \label{fig:m3}
\end{figure}

The next move is "move-3", shown in Fig. \ref{fig:m3}, which is an analog of the "flip" move used in the two-dimensional CDT. It also comes in two versions. In version one, it replaces one blue and two red dots with one red and two blue dots, which corresponds to replacing an $s_{12}$ triangle with an $s_{21}$ in the CDT triangulation. The second version removes two adjacent blue dots connected with the black dot and places them on the other side, i.e. connects them to two external legs of the original black dot. At the same time, the black dot gets connected to the two external legs originally connected to the blue dots. The move does not change any of the global numbers in the triangulation.

\subsection{Move-4}

The "move-4" is one of the simplest ones, and is shown in Fig. \ref{fig:m4}. Move-4 and its inverse are effectively a special case of a "split-merge" move. It removes a black dot and replaces it with a fully connected set of four black dots. The four dots are also connected to the external legs of the original configuration, one by one.

\begin{figure}[h]
    \centering
    \includegraphics[width = 0.5\textwidth]{Figures/move4.pdf}
    \caption{Move-4 replaces a black dot with four fully connected black dots, connecting each of them to the external link of the original configuration. In the CDT triangulation, it adds a vertex inside an $s_{41}$ simplex, replacing the simplex with four new $s_{41}$ simplices.}
    \label{fig:m4}
\end{figure}

As every solid loop in the graphical representation corresponds to a spatial link, and as it is visible in the right panel of Fig. \ref{fig:m4} there are four such solid loops, thus in the real triangulation four new spatial links are created. As all the four black dots are adjacent to each other, it can happen only if they share a vertex, thus the move creates a vertex in the original triangulation, this vertex has coordination number  four.\footnote{In fact, the coordination number is eight, as there are additional $s_{14}$ simplices in the previous slab.} 

\subsection{Move-5}
The last move is "move-5", shown in Fig. \ref{fig:m5}.

\begin{figure}[h]
    \centering
    \includegraphics[width = 0.5\textwidth]{Figures/move5.pdf}
    \caption{Move-5 replaces two adjacent black dots with three black dots. In the CDT triangulation, it creates a spatial link with coordination number three. The link is signaled by a solid black loop in the graphical representation.}
    \label{fig:m5}
\end{figure}

The move takes two adjacent black dots (tetrahedra) and replaces the triangular interface formed by the three common vertices with a link that connects the remaining two vertices. The move creates a link with coordination number three\footnote{In fact, the coordination number is six, as there are additional $s_{14}$ simplices in the previous slab.}, signaled by the solid black loop connecting the three black dots on the right panel in Fig. \ref{fig:m5}. The inverse move requires a link with coordination number three.\\ 

One should also note, that in the  CDT code, we use the full four-dimensional triangulation. In the graphical representation, it could be achieved by  adding a single solid external leg to each black/grey dot. This way  each black (grey) dot of the $t+\frac{1}{2}$ plane would be connected to a grey (black) dot of the previous (next) slab, defined by the plane at $t+\frac{1}{2}-1$ ($t+\frac{1}{2}+1$). Move-4 and move-5 are the only moves that are affected by the neighboring slabs, and the connected grey/black dots of the adjacent slabs would behave exactly the same way as the black/grey dots behave in the above graphical representation description. \\

So far we discussed the moves which are currently used in the MC simulations of the four-dimensional  CDT. In principle, one could try to define some new moves, but doing so is a hard task as they must be efficient numerically and their required components (a vertex/link/triangle with a given coordination number) must be easy to be tracked during the simulations, e.g., a vertex with a given coordination number is easy to track, but it is not the case for more complex structures. In appendix \ref{AppendixB} we discuss some proposals for new moves with the help of the  graphical representation defined above. 

%----------------------------------------------------------------------------------------
