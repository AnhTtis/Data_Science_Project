%%%%%%%%%%%%%%%%%%%%%%%%%%%%%%%%%%%%%%%%%
% Masters/Doctoral Thesis 
% LaTeX Template
% Version 2.5 (27/8/17)
%
% This template was downloaded from:
% http://www.LaTeXTemplates.com
%
% Version 2.x major modifications by:
% Vel (vel@latextemplates.com)
%
% This template is based on a template by:
% Steve Gunn (http://users.ecs.soton.ac.uk/srg/softwaretools/document/templates/)
% Sunil Patel (http://www.sunilpatel.co.uk/thesis-template/)
%
% Template license:
% CC BY-NC-SA 3.0 (http://creativecommons.org/licenses/by-nc-sa/3.0/)
%
%%%%%%%%%%%%%%%%%%%%%%%%%%%%%%%%%%%%%%%%%

%----------------------------------------------------------------------------------------
%	PACKAGES AND OTHER DOCUMENT CONFIGURATIONS
%----------------------------------------------------------------------------------------



\documentclass[
11pt, % The default document font size, options: 10pt, 11pt, 12pt
%oneside, % Two side (alternating margins) for binding by default, uncomment to switch to one side
english, % ngerman for German
singlespacing, % Single line spacing, alternatives: onehalfspacing or doublespacing
%draft, % Uncomment to enable draft mode (no pictures, no links, overfull hboxes indicated)
%nolistspacing, % If the document is onehalfspacing or doublespacing, uncomment this to set spacing in lists to single
%liststotoc, % Uncomment to add the list of figures/tables/etc to the table of contents
%toctotoc, % Uncomment to add the main table of contents to the table of contents
%parskip, % Uncomment to add space between paragraphs
%nohyperref, % Uncomment to not load the hyperref package
headsepline, % Uncomment to get a line under the header
%chapterinoneline, % Uncomment to place the chapter title next to the number on one line
%consistentlayout, % Uncomment to change the layout of the declaration, abstract and acknowledgements pages to match the default layout
]{MastersDoctoralThesis} % The class file specifying the document structure

\usepackage[utf8]{inputenc} % Required for inputting international characters
\usepackage[T1]{fontenc} % Output font encoding for international characters
\usepackage{pdfpages}
\usepackage{mathpazo} % Use the Palatino font by default

\usepackage[style=numeric, natbib=true, sorting=none]{biblatex} % Use the bibtex backend with the authoryear citation style (which resembles APA)


\addbibresource{main.bib} % The filename of the bibliography

\usepackage[autostyle=true]{csquotes} % Required to generate language-dependent quotes in the bibliography

\usepackage{graphicx}

\usepackage{caption}
\usepackage{xcolor,colortbl}

\usepackage{amsmath}

\usepackage{soul}
\newcommand{\jgs}[1]{\textcolor{red}{#1}}
\newcommand{\jgsc}[1]{ \textcolor{blue}{<COMMENT: #1>}}


\DeclareMathOperator*{\argmax}{argmax}

%----------------------------------------------------------------------------------------
%	MARGIN SETTINGS
%----------------------------------------------------------------------------------------

\geometry{
	paper=a4paper, % Change to letterpaper for US letter
	inner=2.5cm, % Inner margin
	outer=3.8cm, % Outer margin
	bindingoffset=.5cm, % Binding offset
	top=1.5cm, % Top margin
	bottom=1.5cm, % Bottom margin
	%showframe, % Uncomment to show how the type block is set on the page
}

%----------------------------------------------------------------------------------------
%	THESIS INFORMATION
%----------------------------------------------------------------------------------------

\thesistitle{Studies of Critical Phenomena in Causal Dynamical Triangulations on a Torus} % Your thesis title, this is used in the title and abstract, print it elsewhere with \ttitle
\supervisor{Prof. dr hab.  Jerzy \textsc{Jurkiewicz} \\ Dr. hab.  Jakub \textsc{Gizbert-Studnicki}} % Your supervisor's name, this is used in the title page, print it elsewhere with \supname
\examiner{} % Your examiner's name, this is not currently used anywhere in the template, print it elsewhere with \examname
\degree{Doctor of Philosophy} % Your degree name, this is used in the title page and abstract, print it elsewhere with \degreename
\author{Dániel \textsc{Németh}} % Your name, this is used in the title page and abstract, print it elsewhere with \authorname
\addresses{} % Your address, this is not currently used anywhere in the template, print it elsewhere with \addressname

%\subject{Biological Sciences} % Your subject area, this is not currently used anywhere in the template, print it elsewhere with \subjectname
\keywords{} % Keywords for your thesis, this is not currently used anywhere in the template, print it elsewhere with \keywordnames
\university{\href{https://en.uj.edu.pl/}{Jagiellonian University}} % Your university's name and URL, this is used in the title page and abstract, print it elsewhere with \univname
\department{\href{https://fais.uj.edu.pl/en_GB/the-institute-of-physics}{Jagiellonian University}} % Your department's name and URL, this is used in the title page and abstract, print it elsewhere with \deptname
\group{\href{.}{Faculty of Physics, Astronomy and Applied Computer Science}} 

% Your research group's name and URL, this is used in the title page, print it elsewhere with \groupname
\faculty{\href{https://fais.uj.edu.pl/}{Faculty of Phyiscs, Astronomy and Applied Computer Science}} % Your faculty's name and URL, this is used in the title page and abstract, print it elsewhere with \facname

\AtBeginDocument{
\hypersetup{pdftitle=\ttitle} % Set the PDF's title to your title
\hypersetup{pdfauthor=\authorname} % Set the PDF's author to your name
\hypersetup{pdfkeywords=\keywordnames} % Set the PDF's keywords to your keywords
}


\usepackage{xpatch} % also loads expl3
%%START
\makeatletter
\xpatchcmd{\@bibitem}
  {\item}
  {\item[\@biblabel{\changekey{#1}}]}
  {}{}
\xpatchcmd{\@bibitem}
  {\the\value{\@listctr}}
  {\changekey{#1}}
  {}{}
\makeatother

\ExplSyntaxOn
\cs_new:Npn \changekey #1
 {
  \str_case:nVF {#1} \g_changekey_list_tl { ?? }
 }
\cs_new_protected:Npn \setchangekey #1 #2
 {
  \tl_gput_right:Nn \g_changekey_list_tl { {#1}{#2} }
 }
\tl_new:N \g_changekey_list_tl
\cs_generate_variant:Nn \str_case:nnF { nV }
\ExplSyntaxOff


\setchangekey{pub1}{1}
\setchangekey{pub2}{2}
\setchangekey{pub3}{3}
\setchangekey{pub4}{4}
\setchangekey{pub5}{5}
\setchangekey{pub6}{6}



\begin{document}

\frontmatter % Use roman page numbering style (i, ii, iii, iv...) for the pre-content pages

\pagestyle{plain} % Default to the plain heading style until the thesis style is called for the body content

%----------------------------------------------------------------------------------------
%	TITLE PAGE
%----------------------------------------------------------------------------------------

\begin{titlepage}
\begin{center}

\vspace*{.06\textheight}
{\scshape\LARGE \univname\par}\vspace{1.5cm} % University name
\textsc{\Large Doctoral Thesis}\\[0.5cm] % Thesis type

\HRule \\[0.4cm] % Horizontal line
{\huge \bfseries \ttitle\par}\vspace{0.4cm} % Thesis title
\HRule \\[1.5cm] % Horizontal line
 
\begin{minipage}[t]{0.4\textwidth}
\begin{flushleft} \large
\emph{Author:}\\
\href{http://www.johnsmith.com}{\authorname} % Author name - remove the \href bracket to remove the link
\end{flushleft}
\end{minipage}
\begin{minipage}[t]{0.4\textwidth}
\begin{flushright} \large
\emph{Supervisors:} \\
\href{http://www.jamessmith.com}{\supname} % Supervisor name - remove the \href bracket to remove the link  
\end{flushright}
\end{minipage}\\[3cm]
 
\vfill

\large \textit{A thesis submitted in fulfillment of the requirements\\ for the degree of \degreename}\\[0.3cm] % University requirement text
\textit{in the}\\[0.4cm]
\groupname\\\deptname\\[2cm] % Research group name and department name
 
\vfill

%{\large \today}\\[4cm] % Date
%\includegraphics{Logo} % University/department logo - uncomment to place it
 
\vfill
\end{center}
\end{titlepage}

%----------------------------------------------------------------------------------------
%	DECLARATION PAGE
%----------------------------------------------------------------------------------------

\begin{declaration}
\addchaptertocentry{\authorshipname} % Add the declaration to the table of contents
\vspace{1cm}
Ja, niżej podpisany, \authorname, (nr indeksu: 1159847), doktorant Wydziału Fizyki, Astronomii i Informatyki Stosowanej Uniwersytetu Jagiellońskiego oświadczam, że przedłożona przeze mnie rozprawa doktorska pt. "Studies of Critical Phenomena in Causal
Dynamical Triangulations on a Torus" jest oryginalna i przedstawia wyniki badań wykonanych przeze mnie osobiście, pod kierunkiem prof. dr hab. Jerzego Jurkiewicza i dr hab. Jakuba Gizbert-Studnickiego. Pracę napisałem samodzielnie.\vspace{0.35cm} \\

Oświadczam, że moja rozprawa doktorska została opracowana zgodnie z Ustawą o prawie autorskim i prawach pokrewnych z dnia 4 lutego 1994 r. (Dziennik Ustaw 1994 nr 24 poz. 83 wraz z późniejszymi zmianami). \vspace{0.35cm}\\

Jestem świadomy, że niezgodność niniejszego oświadczenia z prawdą ujawniona w dowolnym czasie, niezależnie od skutków prawnych wynikających z ww. ustawy, może spowodować unieważnienie stopnia nabytego na podstawie tej rozprawy.
\vspace{1cm}\\
 
\noindent Kraków, ............................... \hfill .......................................................... \\
\hspace{1.7cm}(data)\hfill (podpis doktoranta)
%\rule[0.5em]{25em}{0.5pt} \hfill % This prints a line for the signature
 
%\rule[0.5em]{25em}{0.5pt} % This prints a line to write the date
\end{declaration}

\cleardoublepage

%----------------------------------------------------------------------------------------
%	QUOTATION PAGE
%----------------------------------------------------------------------------------------

%\vspace*{0.2\textheight}

%\noindent\enquote{\itshape Thanks to my solid academic training, today I can write hundreds of %words on virtually any topic without possessing a shred of information, which is how I got a %good job in journalism.}\bigbreak

%\hfill Dave Barry

%----------------------------------------------------------------------------------------
%	ABSTRACT PAGE
%----------------------------------------------------------------------------------------

\begin{abstract}
\addchaptertocentry{\abstractname} 
This document contains my publications and results based on  research done as a member of the Causal  Dynamical Triangulations (CDT) group at the Jagiellonian University during my PhD studies. The field of my research was the four-dimensional CDT, which is a lattice regularization of the theory of quantum gravity, based on the formalism of Regge Calculus and Feynman path integrals. Due to  mathematical complexity, analytical solutions to the model exists only in two dimensions. The four-dimensional theory is analyzed by numerical simulations. Earlier discoveries include  dynamically emergent quantum de Sitter universes with emergent four-dimensional properties, scale-dependent spectral dimensions and a complex phase structure in which first- and higher-order phase transitions were shown to exist. The document describes the nature of previously not yet analyzed phase transitions, new ways to analyze triangulations and the impact of classical and dynamical (quantum) scalar fields in four-dimensional CDT with toroidal spatial topology. The main results of the dissertation are the six publications attached to the last chapter. This document is intended as an introduction to CDT and serves as a guide to the papers comprising the doctoral thesis. 
\nocite{pub1,pub2,pub3,pub4,pub5,pub6}
\end{abstract}

\begin{abstract}
\addchaptertocentry{\abstractname} 
Niniejszy dokument zawiera moje publikacje i wyniki oparte na badaniach prowadzonych jako członek grupy Causal Dynamical Triangulations (CDT) na Uniwersytecie Jagiellońskim podczas studiów doktoranckich. Obszarem moich badań był czte- rowymiarowy model CDT, który stanowi sieciową regularyzację teorii kwantowej grawitacji, opartą na formalizmach rachunku Regge i całek po trajektoriach Feynmana. Ze względu na złożoność matematyczną, rozwiązania analityczne tego modelu istnieją tylko w dwóch wymiarach. Czterowymiarowa teoria jest analizowana przez symulacje numeryczne. Wcześniejsze odkrycia obejmują dynamicznie pojawiające się kwantowe wszechświaty de Sittera z emergentnymi właściwościami czterowymiarowymi, zależne od skali wymiary spektralne oraz skomplikowaną strukturę fazową, w której istnieją przejścia fazowe pierwszego i wyższego rzędu. Dokument zawiera opis natury nie analizowanych dotychczas przejść fazowych, nowych sposobów analizy triangulacji oraz wpływu klasycznych i dynamicznych (kwantowych) pól skalarnych w czterowymiarowym CDT o toroidalnej topologii przestrzennej. Głównymi wynikami rozprawy jest sześć publikacji załączonych w ostatnim rozdziale. Dokument ten ma na celu wprowadzenie do CDT i stanowi przewodnik po artykułach składających się na rozprawę doktorską.
\end{abstract}


%----------------------------------------------------------------------------------------
%	LIST OF CONTENTS/FIGURES/TABLES PAGES
%----------------------------------------------------------------------------------------

\tableofcontents % Prints the main table of contents

%\listoffigures % Prints the list of figures

%\listoftables % Prints the list of tables

%----------------------------------------------------------------------------------------
%	ABBREVIATIONS
%----------------------------------------------------------------------------------------

\begin{abbreviations}{ll} % Include a list of abbreviations (a table of two columns)

\textbf{ADM} & \textbf{A}rnowitt-\textbf{D}eser-\textbf{M}isner\\
\textbf{AS} & \textbf{A}symptotic \textbf{S}afety\\
\textbf{CDT} & \textbf{C}ausal \textbf{D}ynamical \textbf{T}riangulations\\
\textbf{CFT} & \textbf{C}onformal \textbf{F}ield \textbf{T}heory\\
\textbf{EDT} & \textbf{E}uclidean \textbf{D}ynamical \textbf{T}riangulations\\
\textbf{GFT} & \textbf{G}roup \textbf{F}ield \textbf{T}heory\\
\textbf{HLG} & \textbf{H}ořava–\textbf{L}ifshitz \textbf{G}ravity\\
\textbf{IR} & \textbf{I}nfra \textbf{R}ed \\
\textbf{LCDT} & \textbf{L}ocally \textbf{C}ausal \textbf{D}ynamical \textbf{T}riangulations \\
\textbf{LQCD} & \textbf{L}attice \textbf{Q}uantum \textbf{C}hrono \textbf{D}ynamics \\
\textbf{LQG} & \textbf{L}oop \textbf{Q}uantum \textbf{G}ravity\\
\textbf{MC} & \textbf{M}onte \textbf{C}arlo \\
\textbf{NCG} & \textbf{N}on-\textbf{C}ommutative \textbf{G}eometry \\
\textbf{OP} & \textbf{O}rder \textbf{P}arameter \\
\textbf{QCD} & \textbf{Q}uantum \textbf{C}hrono \textbf{D}ynamics \\
\textbf{QFT} & \textbf{Q}uantum \textbf{F}ield \textbf{T}heory \\
\textbf{QM} & \textbf{Q}uantum \textbf{M}echanics \\
\textbf{RG} & \textbf{R}enormalization \textbf{G}roup \\
\textbf{ST} & \textbf{S}tring \textbf{T}heory \\
\textbf{UV} & \textbf{U}ltra \textbf{V}iolet \\
\textbf{UVFP} & \textbf{U}ltra \textbf{V}iolet \textbf{F}ixed \textbf{P}oint \\


\end{abbreviations}

%----------------------------------------------------------------------------------------
%	PHYSICAL CONSTANTS/OTHER DEFINITIONS
%----------------------------------------------------------------------------------------

%\begin{constants}{lr@{${}={}$}l} % The list of physical constants is a three column table

% The \SI{}{} command is provided by the siunitx package, see its documentation for instructions on how to use it

%Speed of Light & $c_{0}$ & \SI{2.99792458e8}{\meter\per\second} (exact)\\
%Constant Name & $Symbol$ & $Constant Value$ with units\\

%\end{constants}

%----------------------------------------------------------------------------------------
%	SYMBOLS
%----------------------------------------------------------------------------------------

%\begin{symbols}{lll} % Include a list of Symbols (a three column table)

%$a$ & distance & \si{\meter} \\
%$P$ & power & \si{\watt} (\si{\joule\per\second}) \\
%Symbol & Name & Unit \\

%\addlinespace % Gap to separate the Roman symbols from the Greek

%$\omega$ & angular frequency & \si{\radian} \\

%\end{symbols}

%----------------------------------------------------------------------------------------
%	DEDICATION
%----------------------------------------------------------------------------------------

\dedicatory{To my daughter who cannot read yet\ldots} 

%----------------------------------------------------------------------------------------
%	THESIS CONTENT - CHAPTERS
%----------------------------------------------------------------------------------------

\mainmatter % Begin numeric (1,2,3...) page numbering

\pagestyle{thesis} % Return the page headers back to the "thesis" style

% Include the chapters of the thesis as separate files from the Chapters folder
% Uncomment the lines as you write the chapters

% Chapter 0

\chapter{Motivation to the study of parallels, random geometry and quantum gravity} % Main chapter title

\label{Chapter0} % For referencing the chapter elsewhere, use \ref{Chapter1} 

%----------------------------------------------------------------------------------------

% Define some commands to keep the formatting separated from the content 
\newcommand{\keyword}[1]{\textbf{#1}}
\newcommand{\tabhead}[1]{\textbf{#1}}
\newcommand{\code}[1]{\texttt{#1}}
\newcommand{\file}[1]{\texttt{\bfseries#1}}
\newcommand{\option}[1]{\texttt{\itshape#1}}

\definecolor{Gray}{gray}{0.85}
\definecolor{LightCyan}{rgb}{0.88,1,1}

\newcolumntype{a}{>{\columncolor{Gray}}c}
\newcolumntype{b}{>{\columncolor{white}}c}

\newcommand{\mc}[2]{\multicolumn{#1}{c}{#2}}
%----------------------------------------------------------------------------------------
\textit{"You must not attempt this approach to the parallels. I know this way to the very end. I have traversed this bottomless night, which extinguished all light and joy of my life. I entreat you, to leave the science of parallels alone. For God’s sake, please give it up. Fear it no less than the sensual passion, because it, too, may take up all your time and deprive you of your health, peace of mind, and happiness in life. I thought I would sacrifice myself for the sake of truth. I was ready to become a martyr who would remove the flaw from geometry and return it purified to mankind. I accomplished monstrous, enormous labors: my creations are far better than those of others and yet I have not achieved complete satisfaction. I turned back when I saw no man can reach the bottom of this night. I turned back unconsolidated, pitying myself and all mankind. Learn from my example: I wanted to know about parallels. I remain ignorant, this has taken all the flowers of my life and all my time from me...."} - \textbf{A letter of Bolyai Farkas to his son Bolyai János}\\\\


Geometry, the mathematical study of shapes, always interested humans. From the ancient Greeks till today's science various topics related to geometry are key aspects to mathematics and natural sciences. Euclid laid down five axioms, which became the foundations of mathematics. At that time, mathematics was postulated in terms of words and rarely graphics, but not equations. The postulates of Euclid~\cite{postulates}, based on his axioms, defined geometry until the $19^{th}$ century. His postulates were:

\begin{itemize}
    \item  A straight line segment may be drawn from any given point to any other.
    \item A straight line may be extended to any finite length.
    \item A circle may be described with any given point as its center and any distance as its radius.
    \item All right angles are congruent.
    \item If a straight line intersects two other straight lines, and so makes the two interior angles on one side of it together less than two right angles, then the other straight lines will meet at a point if extended far enough on the side on which the angles are less than two right angles.
\end{itemize}

None dared to question the truth of these postulates as any sane person could check their truths by drawing those lines and not finding any which doesn't fit. This was true until some questioned whether it is possible to draw triangles on various non-flat shapes such that the sum of their inner angles is different than that of $\frac{\pi}{2}$. This is exactly what Bolyai Farkas is talking about in his letter to his son. He discovered that parallels can meet sometimes, but did not manage to describe the phenomena in its entirety even though he worked in that field for his whole life. Thus he warned his son not to pursue geometry and its parallels. But his son, János had his own ideas, and years later he constructed the basics of non-Euclidean geometry. Had he listened to his father, the topic of my doctoral thesis would be probably significantly different. Bolyai pursued a non-mainstream topic of mathematics and reached success with it. Later Riemann based his work on the work of Bolyai (and others), which was then used by Einstein when he worked out the general theory of relativity. The science of fundamental physics brings forth our knowledge of nature, if we wouldn't walk off-road in the theory space, but only follow the mainstreams, we wouldn't be able to solve the hardest problems of science.\\

At the beginning of the twentieth century, the appearance of two theories gave us an enormous leap toward understanding nature. The paradigm shift which is related on one hand to the curving relativistic four-dimensional spacetime described by general relativity (\textbf{GR}) and on the other hand to the discreteness of nature as it is seen by quantum mechanics turned science into science-fiction in the eye of the scientifically not educated people. Math and physics, needed in order to understand it, started to be so complex and demanding,  that scientific results became non-trivial. Many physical theories are validated or falsified via mathematical derivations and many cannot be accessed because of their mathematical complexity. The work presented in this thesis belongs to a similar off-road field, which is strongly related to parallels and geometry. Quantum gravity is the field where quantum mechanics and general relativity meets. Quantum mechanics is the theory that describes the smallest scales, the tiny fluctuations of matter, and the rules of nature that escape everyday experience, and gravity is the theory that describes the physics of the largest scales, the orbiting of planets, and even the earliest history of the Universe. Their intersection should be the theory of quantum gravity, the theory which describes how the attraction between bodies behaves on the smallest scales, on the scales where other forces of nature dominate and bodies fall apart to their components. As we advanced in our understanding of the world and the Universe it turned out that quantum gravity could potentially explain also the largest scales and the earliest moments of history. It could tell us why we have such a large-scale structure of galaxies that we see, could explain why visible matter constitutes only four percent of everything, could hint at whether we live in a closed or an open Universe, and foreshadow a potential cold death at the end of times. The quantum theory of gravity has the potential to explain the nature and the structure of space-time, to resolve singularities of GR, and furthermore to explain or disprove the theories regarding dark matter and dark energy.\\

After Einstein introduced GR many scientists tried to find the theory of quantum gravity without success. The first attempt to describe quantum gravity was a naive application of perturbative methods of QFT to GR, but it failed. The treatment of infinities by perturbative renormalization techniques, which can be used in the case of the standard model physics, cannot be applied to gravity, which turned out perturbatively non-renormalizable \cite{non_renorm_g}. However, S. Weinberg conjectured that gravity may adhere to an Asymptotic Safety (\textbf{AS}) scenario \cite{Weinberg, asym, asym_critiq}, where using Renormalization Group Flow (\textbf{RG}) techniques one may find a \textit{fixed point}, where there exist only a finite number of coupling constants needed to describe the full quantum  theory in a non-perturbative way. In a lattice formulation of a quantum theory, fixed points are typically connected to phase transitions, and the hypothesis is that there is at least one non-trivial fixed point for gravity related to the ultraviolet (\textbf{UV}) regime, which necessarily requires the existence of a higher order phase transition. Such a phase transition can be typically recognized from the diverging correlation length  and related scaling exponents.\\

By the end of the $20^{th}$ century, with the increasing available computational power, numerical algorithms became widely used. One of the most notable computer-based techniques in physics is related to Lattice Quantum Chromodynamics (LQCD) \cite{lqcd1,lqcd2}, which was developed in parallel with the physical experiments. The basic idea is to discretize the continuum theory such that the field variables are located at the vertices of a regular $D$-dimensional lattice ($D$ depends on the dimensionality of the discussed model). The lattice spacing $a$, which is the length between two adjacent vertices of the lattice, should be sent to zero while keeping the relevant physical observables constant, in order to reach the continuum limit within the numerical simulations. Since the beginning of the development of lattice theories, many physically relevant observations were derived from numerical simulations, e.g., related to phase transitions \cite{lqcd_chiral}, physical masses of particles \cite{lqcd_masses}, and many other phenomena. In contrast to  the LQCD, lattice quantum gravity is special in the sense that the lattice connectivity itself encodes the geometric degrees of freedom and therefore provides information about the distinct features of gravitational physics on the quantum level. In order to create a physically relevant model of  lattice quantum gravity one also has to be able to include matter fields, e.g., scalar fields or gauge fields \cite{2d_cdt_m,gauge_2d_cdt}.\\

This document is a guide to a collection of articles published in the past years and constituting my doctoral thesis. All of the publications were published in peer-reviewed journals. \\ 

The structure of this document is as follows: The introduction to Causal Dynamical Triangulations (CDT), which is a non-perturbative approach in the quest of quantizing gravity, is the topic of chapter two. In chapter three, I discuss some details of numerical implementation and Monte Carlo simulation methods used to study CDT. The fourth and fifth chapters discuss respectively the results of my studies obtained for empty Universes (pure gravity) and Universes with matter content (gravity coupled to scalar fields). Afterward, all publications which constitute my thesis are briefly discussed in chapter six, together with information about my contribution to them. The published papers are attached at the very end of chapter seven in the following order:

\begin{enumerate}
    \item[\cite{pub1}]  J. Ambjorn G. Czelusta et al. “The higher-order phase transition in toroidal CDT”. In: J. of High Energ. Phys. 2020 (5), p. 30.\\DOI: 10.1007/JHEP05(2020)030
    \item[\cite{pub2}] J. Ambjorn et al. “Towards an UV fixed point in CDT gravity”. In: Journal of High Energy Physics 2019 (7), p. 166.\\ DOI: 10.1007/JHEP07(2019)166
    \item[\cite{pub3}]  J. Ambjorn et al. “Topology induced first-order phase transitions in lattice quantum gravity”. In: Journal of High Energy Physics 2022 (4), p. 103.\\ DOI: 10.1007/JHEP04(2022)103.
    \item[\cite{pub4}] J.Ambjorn et al. “Cosmic voids and filaments from quantum gravity”. In: The European Physical Journal C 81 (8 2021), p. 708.\\ DOI: 10.1140/epjc/s10052-021-09468-z
    \item[\cite{pub5}] J. Ambjorn et al. “Matter-Driven Change of Spacetime Topology”. In: Phys. Rev. Lett. 127 (16 Oct. 2021), p. 161301.\\ DOI: 10.1103/PhysRevLett.127161301
    \item[\cite{pub6}]  J. Ambjorn et al. “Scalar fields in causal dynamical triangulations”. In: Classical and Quantum Gravity 38 (19 Sept. 2021), p. 195030.\\ DOI: 10.1088/1361-6382/ac2135
\end{enumerate}

% Chapter 1

\chapter{Causal Dynamical Triangulations} % Main chapter title

\label{Chapter1} % For referencing the chapter elsewhere, use \ref{Chapter1} 

%---------------------------------------------------------------------------------------

\section{Introduction to Quantum Gravity}

\textit{"The beauty and clearness of the dynamical theory, which asserts heat and light to be modes of motion, is at present obscured by two clouds..."} - \textbf{Lord Kelvin}\\\\

Lord Kelvin wrongly predicted the end of physics in the late nineteenth century. The two clouds mentioned were the problem of heat and radiation, more precisely the theorized material that fills everything called "ether" and the black body radiation. When we mention modern physics, we refer to the time when the solutions to these two "clouds" were presented in the form of special relativity and quantum mechanics. The start of the twentieth century brought us an explosion of physical theories, as special relativity led to general relativity, which is extensively studied today in relation to astrophysical and cosmological models or technologies, such as GPS tracking devices. At the same time, quantum mechanics evolved into quantum field theory, and later our technological advancements led to the ability to measure the properties of particles. The standard model of particle physics is one of the greatest achievements in physics, as it gives an explanation of the fundamental nature of matter. The biggest problem of modern physics is that the theory of matter and the theory of gravity cannot be matched into a unified framework together. Many physicists tried in the past hundred years to describe the theory of quantum gravity, which led to many different research projects, such as Loop Quantum Gravity (LQG), String Theory (ST), Causal Sets (CS), Group Field Theory (GFT), Non-Commutative Geometry (NCG), Canonical Quantum Gravity (CQG), Hořava–Lifshitz Gravity (HLG), Asymptotic Safety (AS), Euclidean Dynamical Triangulations (EDT), Causal Dynamical Triangulations (CDT) and many other approaches. \\

\subsection{(Non-)renormalizability of quantum gravity and the need for non-perturbative approaches}

Merging the quantum theory with gravity is not a trivial task. Quantum field theory (QFT) predicts fluctuations of  fields, and according to GR and Einstein's field equations, where there is energy density, there is curvature. These fluctuations can at very high energies produce such a large energy density in a small volume that the naive application of Einstein's equations would predict the appearance of black holes\cite{scale_grav,scale_grav2}. The problems with UV-completion of quantum gravity become apparent in the perturbative expansion of a QFT based on GR. Such a formulation is perturbatively non-renormalizable \cite{non_renorm}, which means, that the naive application of the perturbation theory would result in infinitely many parameters and coupling constants appearing in the theory, that cannot be eliminated via renormalization thus yielding the theory to be un-predictive. \\

It is well known, that the couplings appearing in QFTs are scale-dependent, this scale dependence is referred to as "running of the couplings". In the case of the full theory, where one integrates from zero to infinite momenta (or alternatively zero distances) many models exhibit infinite divergences, the solution to which is provided by some cutoff $\Lambda$ introduced to the high energy regime. Up to this cutoff, the theory is predictive, and the aim is to remove the cutoff and avoid the appearance of infinities. The UV completeness of a QFT is provided by the existence of fixed points of the renormalization group flow in the coupling constant space: as the energy scale changes, the running coupling constants approach some fixed point value. The microscopic theory is defined in such fixed points, thus finding them is a crucial part of any theory based on QFT language. Let  $g$ be a coupling constant of a given theory, then the so-called "beta function" $\beta(g)$ will define the scale dependence, or running of the coupling. The fixed points are defined by  zeros of $\beta(g)$, which can result in a free or interactive theory. The free theory is achieved when the zero of the beta function corresponds to zero values of the couplings, which is called "asymptotic freedom" and such a fixed point is called trivial or Gaussian. If instead zeros of the beta function are achieved for a finite number of non-zero couplings, it is called "asymptotic safety", where one has non-trivial fixed points corresponding to an interactive theory \cite{nontriv_fp}. A fixed point corresponding to high energy, or short scale, is called the "ultraviolet" (UV) fixed point, while the "infrared " (IR) fixed point will correspond to the low energy, or large scale theory.\\

A QFT description of GR means, that one treats the metric tensor $g_{\mu\nu}$ as the field of gravitation and defines an action in terms of geometric invariants obtained from the metric tensor, such as e.g., $R, R^2, R_{\mu\nu}R^{\mu\nu}$, etc. The most important couplings in the case of gravity are Newton's coupling $G$, and the cosmological constant $\Lambda$. The theory of gravity is perturbatively non-renormalizable, as applying perturbation theory in every order one has to introduce  infinitely many counter-terms and the corresponding new couplings, which renders the theory to be non-predictive. Nevertheless, according to the "asymptotic safety" conjecture, formulated by Steven Weinberg \cite{Weinberg},  most of the (potentially infinitely many) couplings appearing in such a theory become irrelevant at the non-trivial UV fixed point, and there will be only a finite number of relevant couplings rendering the theory non-perturbatively renormalizable, i.e., UV-complete and predictive to arbitrarily large energy scale. Therefore a non-perturbative description of quantum gravity is needed which can be done with the help of numerical simulations. The non-perturbative approach discussed in this thesis  is called Causal Dynamical Triangulations (CDT) and it is based on Regge calculus and Feynman path integral formulation.



\subsection{Regge calculus}

Before jumping into the description of CDT, it is necessary to discuss the mathematical formulation that led to it. This formulation was introduced by Regge and is called Regge calculus \cite{Regge}. The aim of Regge was to approximate space-times, which are solutions to the Einstein field equations, via piecewise-flat manifolds.\footnote{Often the name {\it piecewise-linear manifold} instead of {\it piecewise-flat manifold}  is used.} The approximation is done with the help of internally flat triangular building blocks (simplices) glued together in a non-trivial way, hence the name "triangulation". The simplices in a 2-dimensional triangulation are triangles, in 3-dimensions are tetrahedra, and in 4-dimensions are pentachora. All simplices in a triangulation are glued to each other via their $(d-1)$ dimensional faces (links for $d = 2$, triangles for $d = 3$, and tetrahedra for $d = 4$). These $(d-1)$ dimensional sub-simplices are also connected via "hinges", also called  "bones", which are $(d-2)$ dimensional objects. The hinges play a crucial role, as  curvature can be defined there locally. The curvature is related to the angular difference (deficit angle) at a given bone. Let's imagine a triangulation consisting of $n$ equilateral triangles glued together along edges (links) around a single point (vertex). If $n = 6$ then one can place it on a flat 2-dimensional surface. If $n = 5$ then one can place it only if it is cut along one edge, and it will be visible that a triangle is "missing". The angle associated with the missing (or for $n > 6$ extra) triangles is the deficit angle.\\

Let us consider the simplest (nontrivial) case of a three-dimensional Riemannian manifold which is well approximated by a fine triangulation. Following the approach of Regge \cite{intro_regge_calc}, the discretized curvature is obtained by considering the parallel transport of a vector around a bone. Many simplices (in this case tetrahedra) touch each other at the bone forming a bundle $p$. One can associate the number of simplices in the bundle with bone density $\rho$ at $p$, which is equal to the number of simplices divided by a unit area. The deficit angle ($\epsilon_p$) associated with the bone is a measure of a dihedral angle:

\begin{equation}
    \epsilon_p = 2\pi - \sum_n \theta_n,
\end{equation}
$\theta_n$ being the dihedral angle of the $n$-th simplex at the bone. One can alternatively define $\epsilon = \frac{1}{N}\epsilon_p$, which is the deficit angle of the bone smeared on its simplices. Now, let's take a loop $a$ with area $\Sigma$ around the bundle and parallel transport a vector $\Vec{A}$ around the loop. If $n_\Sigma$ is a unit vector orthogonal to $\Sigma$, then one can define:

\begin{equation}
    \Vec{\Sigma} = \Sigma n_\Sigma,
\end{equation}
which is an area vector associated with the loop. Parallel transporting a vector around the bundle will rotate $\Vec{A}$ by an angle $\sigma$ due to the process of the parallel transport. One can associate a vector of length $\sigma$ to the rotation, and let this vector  be parallel to the bone, so it  will be defined by: 

\begin{equation}
    \Vec{\sigma} = \sigma n,
\end{equation}
where $n$ is the unit vector parallel to the bone. Rotating $\Vec{A}$ by an angle $\sigma$  will produce the vector $\Vec{A'}$ $=\Vec{A}+\delta \Vec{A}$. The infinitesimal change $ \delta \Vec{A}$ will be equal to the product $ \delta \Vec{A} = \Vec{\sigma} \times \Vec{A}$. The rotation angle $\sigma$ is proportional to the number of simplices ($N$) visited by the loop $a$ circumventing the bone $p$, thus :

\begin{equation}
     \sigma = N \epsilon, 
\end{equation}
where $N$ can be expressed in terms of the
bone density $\rho$, the oriented area vector $\Sigma$ and the unit vector parallel to the bone $n$:

\begin{equation}
    N = \rho \, n \cdot \Vec{\Sigma}.
\end{equation}
Putting all the expressions together the infinitesimal change $\delta \vec{A}$ is given by:

\begin{equation}
    \delta \Vec{A} = \rho \epsilon (n \Vec{\Sigma}) \cdot (n \times \Vec{A}).
\end{equation}
Using coordinate (vector component) notation:

\begin{equation}
\delta A_\mu = \rho \epsilon (n^\nu \Sigma_\nu) ( \varepsilon_{\mu \alpha\beta} n^{\alpha}  A^\beta), 
\end{equation}
where:  $\varepsilon_{\mu \alpha\beta}$ is the Levi-Civita symbol. Now, one can express the $n$ and $\Vec{\Sigma}$ vectors in the dual space, i.e., the space of two-forms:

\begin{equation}
    n_\nu = \frac{1}{2}\varepsilon_{\nu\rho\sigma}n^{\rho\sigma}, 
\end{equation}
and 
\begin{equation}
    \Sigma_\nu = \frac{1}{2}\varepsilon_{\nu\alpha\beta}\Sigma^{\alpha\beta}.
\end{equation}
Using the fact that $n^{\nu\lambda} = -n^{\lambda\nu}$, the infinitesimal change $\delta \vec A$ can be now written as:

\begin{equation}
    \delta A_\mu = \frac{1}{4}\rho \epsilon  (\varepsilon_{\nu \rho\sigma} n^{\rho \sigma} \frac{1}{2}\varepsilon^{\nu\alpha\beta} \Sigma_{\alpha \beta}) (2n_{\gamma\mu}) A^\gamma = \frac{1}{2}(\rho \epsilon n_{\alpha \beta} n_{\gamma \mu})\Sigma^{\alpha\beta}A^\gamma.
\end{equation}
Using the continuous counterpart of the same equation with the help of the Riemann tensor one can write:

\begin{equation}
    \delta A_\mu = \frac{1}{2}{R^\gamma}_{\mu\alpha \beta} \Sigma ^{\alpha \beta} A_\gamma.
\end{equation}
Comparing the two equations one can recognize the discretized Riemann curvature tensor. The Ricci tensor can be then defined by index contraction:

\begin{equation}
    {{R^\alpha}_{\mu\alpha\nu}} = R_{\mu\nu} = \rho \epsilon (\delta_{\mu\nu} - n_{\mu} n_{\nu}),
\end{equation}
where we switched back to the unit vector $n$. And with further index contraction, one can get the Ricci scalar:

\begin{equation}
    R = {R^{\alpha}}_{\alpha} = \rho \epsilon ({\delta^\alpha}_\alpha - n^\alpha n_\alpha) = 2 \rho \epsilon,
\end{equation}
which gives a direct connection between the curvature of continuous Riemannian manifolds and their discretized approximations. The above formula can be generalized to more  dimensions as well as to pseudo-Riemannian manifolds. \\

Using Regge calculus, the  Regge action $S_{R}$, i.e., the gravitational action for a piecewise-flat triangulation, can be formulated. The starting point of this is the Einstein-Hilbert action:

\begin{equation}
    \frac{1}{16\pi G}\int d^dx \sqrt{-g} (R - 2\Lambda), 
\end{equation}
where $G$ is the Newton's constant, $R$ is the scalar curvature and $\Lambda$ is the cosmological constant. Writing the curvature in terms of Regge calculus one gets the form:

\begin{equation}
\frac{1}{16\pi G}\int d^dx\sqrt{-g}R = \frac{1}{8\pi G}\int d^dx\sqrt{-g} \rho \epsilon = \kappa \sum_{n_{(d-2)}} k_n\epsilon_n,   
\label{eq:curvature}
\end{equation}
where $\kappa=(8 \pi G)^{-1}$ is the  (inverse) bare  gravitational constant, $k_n$ denotes the volume of the $(d-2)$-dimensional hinge, $\epsilon_n$ is the deficit angle associated with  the hinge and the summation is over $(d-2)$-dimensional simplices, denoted by $n_{(d-2)}$. The term including cosmological constant reads:

\begin{equation}
\frac{1}{16\pi G}\int d^dx\sqrt{-g} (-2 \Lambda) = \frac{-2\Lambda}{16\pi G}\int d^dx\sqrt{-g} = \lambda \sum_{n_d} V_{n_d}, 
\end{equation}
where $\lambda=-\Lambda \kappa$ is the bare cosmological constant, $V_{n_d}$ is the volume of the $d$-dimensional simplices building up the triangulation and the summation is over $d$-dimensional simplices. This leads to the full Regge action:

\begin{equation}
    S_R = \kappa \sum_{n_{(d-2)}} k_n\epsilon_n + \lambda \sum_{n_d} V_n,
    \label{eq:curvature2}
\end{equation}
which holds in any dimension.  
One should note that the Regge form of the gravitational action (\ref{eq:curvature2}) is not expressed in terms of the metric tensor, but in terms of numbers of simplices and sub-simplices. Expressing the Regge action for a particular triangulation can lead to a complicated form, however applying certain constraints can simplify the expressions. 

\section{Causal Dynamical Triangulations}

\textit{"The more success the quantum theory has, the sillier it looks. How nonphysicists would scoff if they were able to follow the odd course of developments!"} - \textbf{Albert Einstein}\\\\

Following the ideas of Weinberg and assuming the existence of a UV fixed point for gravity the properties of quantum gravity can be analyzed using non-perturbative methods. As fixed points were found in other QFT-based theories, such as Quantum Chronodynamics (QCD) \cite{as_freed}, theorists turned towards lattice formulations (e.g. Lattice Quantum Chronodynamics (LQCD)). The simplest lattice theory of GR is called Dynamical Triangulations (DT). In DT, one can use the Regge action straight away. The spacetime is constructed by gluing $d$-dimensional simplicial building blocks: triangles, tetrahedra, and pentachora. The triangulation does not play a role in the physics of the model, as it serves the purpose of regularization, providing a UV  cutoff related to lattice spacing $a$, which should be removed from the continuum limit if it exists. A huge difference between the DT approach from other techniques based on the Regge calculus, such as Quantum Regge Calculus \cite{quantumregge} or some versions of LQG \cite{lqg_regge}, is that the edge length ($a$) of all the simplices is kept fixed and thus piecewise-flat manifolds are constructed from identical equilateral simplices. Transforming the metric signature with the Wick rotation one gets a Euclidean description which allows studying the (regularized) path integral of quantum gravity using statistical methods. In the  DT there is no difference between space and time, however, CDT twists the picture via the introduction of a foliation and thus the notion of time is restored as the causal evolution of the leaves of the foliation. The decomposition of the four-dimensional space-time into space and time is similar to that of the Arnowitt-Deser-Misner (ADM) formalism \cite{adm_f}. Thus, the 4-dimensional space-time is assumed to be globally hyperbolic and each ($d-1$)-dimensional hypersurface ("leaf" of the foliation) has the same fixed topology.  The word "causal" in the name of CDT refers to the time-slicing of the triangulation, as opposed to usual DT, and "dynamical" points at the difference between CDT and traditional lattice approaches, as in CDT the lattice connectivity is not fixed and it encodes  the gravitational degrees of freedom. For example, in LQCD there is a fixed and regular lattice, on which the theory is defined, but in CDT the different lattice configurations correspond to the different trajectories (histories) in the gravitational path integral. Therefore a single configuration (single trajectory) is non-physical, and one has to compute a suitable average over an ensemble of such configurations. \\

In a $d$-dimensional CDT triangulation, by construction, every (sub-)simplex lies in a $d$-dimensional {\it slab} (part of the triangulation) between lattice (discrete) time $t$ and $t+1$. Different types of simplicial building blocks (simplices) $s_{\alpha \beta}$ can be defined by indicating the number $\alpha$ of their vertices in $t$ and the number $\beta$ of  vertices in $t+1$. In 2 dimensions there are two types of building blocks, i.e., triangles: $s_{21}$ and $s_{12}$. In 3 dimensions there are three different types of building blocks, i.e., tetrahedra: $s_{22}$, $s_{31}$, and the mirror reflection $s_{13}$. Finally, in 4 dimensions there are 4 types of such simplices: $s_{41}$ with its mirror-reflection $s_{14}$ and $s_{32}$ with its mirror-reflection $s_{23}$. Due to this construction and the symmetry of the action, as we will see, CDT exhibits a time reflection symmetry as well. Thanks to a small number of different categories of simplices appearing in the four-dimensional CDT and due to  topological constraints of the triangulated manifolds, see  Appendix \ref{AppendixA}, the Regge action (\ref{eq:curvature2}), which governs the dynamics of the model, can be expressed in terms of these $4$-dimensional simplices and  vertices in a triangulation $\cal T$ \cite{nonperturb}:


\begin{equation}
S_{R} = - (\kappa_0 + 6\Delta) N_0 + \kappa_4 (N_{41} + N_{32}) + \Delta N_{41},
\label{eq:ation_kappa}
\end{equation}
where $N_0= \sum s_{10}$ is the total  number of vertices, while $N_{41}= \sum (s_{41} + s_{14})$ and $N_{32} = \sum (s_{32} + s_{23})$ are the total numbers of the various types of simplices in the triangulation $\cal T$. The three bare coupling constants are $\kappa_0$, the bare inverse  Newton constant, $\kappa_4$, the bare cosmological constant, and $\Delta$, related to the asymmetry between  lengths of space-like and {time-like links in the lattice}. From now on we will refer to $N_0$, $N_{41}$ and $N_{32}$ as \textit{global numbers}.\\

The path integral of quantum gravity is formally defined as:

\begin{equation}
    \mathcal{Z}_{QG} = \int D[g_{\mu\nu}]e^{iS_{EH}[g_{\mu\nu}]} \to^{reg}\to \sum_\mathcal{T} \frac{1}{C_\mathcal{T}}e^{iS_R[\mathcal{T}]} = \mathcal{Z}_a,
\end{equation}
where $D$ is the measure term, which enables one to integrate over geometries, i.e., diffeomorphism invariant equivalence classes of smooth metrics $g_{\mu \nu}$, and $S_{EH}$ is the Einstein - Hilbert action. After the lattice regularization ($\to^{reg}$) the path integral is replaced by a sum over all possible triangulations with a measure $1/C_\mathcal{T}$, the size of the automorphism group of $\mathcal{T}$. The index $a$ in $\mathcal{Z}_a$ refers to the lattice regulator, which is the edge length of the simplices and $S_R$ is the Regge action (\ref{eq:ation_kappa}), which is the lattice-regularized version of the Einstein-Hilbert action. The distinction of space and time introduced by the foliation is also present in the edge lengths, as the time-like edge lengths $a_t$ and the space-like edge lengths $a_s$ are not necessarily the same, which gives rise to a degree of freedom, called the asymmetry parameter $\alpha$, where $-\alpha a_t^2 = a^2_s$ in the Lorentzian setting. The aim of CDT is to define the gravitational path integral, or at least approximate it as close as it gets.  All  possible triangulations  $\mathcal{T}$ include only such triangulations which respect the foliation structure and some additional topological constraints. To be able to treat the model with methods of statistical physics a Wick rotation has to be applied to the partition function to change the metrics from Lorentzian to Euclidean signature. Due to the imposed global foliation, the  "Euclideanization" of the path integral via the Wick rotation is well defined and is related to the analytic continuation of the Regge action to negative values of $\alpha$  in the lower half of the complex $\alpha$ plane. Performing it one turns the path integral into the partition function: 

\begin{equation}
\mathcal{Z}_{R} = \sum_{\mathcal{T}} \frac{1}{C_\mathcal{T}}e^{-S_{R}[\mathcal{T}]} ,
\label{eq:partfun}
\end{equation}
where, for a simpler notation, we kept the same symbol $S_R$ for the (now) Euclidean Regge action. The Wick rotation allows for the application of statistical physics methods on the model, for example, one can compute the expectation values of observables as:

\begin{equation}
\langle \mathcal{O}\rangle = \frac{1}{\mathcal{Z}} \sum_\mathcal{T} \frac{1}{\mathcal{C}_{\mathcal{T}}}\mathcal{O} e^{-S_{R}[\mathcal{T}]}    .
\end{equation}

One of the benefits of the Wick rotation is that the model became suitable for numerical Monte Carlo (MC) simulations, where the partition function can be approximated by an ensemble of configurations generated in such simulations. The past twenty years of  numerical studies of the 4-dimensional CDT model led to many interesting and important results.


\subsection{Most important previous results of CDT}

CDT was formulated in the beginning of the 21st century and became recognized by the quantum gravity community in the following years. The introduction of the foliation to the triangulation allowed for the addition of the asymmetry parameter between space and time, which was promoted to a new coupling constant $\Delta$ in the action (\ref{eq:ation_kappa}). This particular change had a huge impact on the properties of the CDT model, compared to DT, as due to the enforced causality constraint the ensemble of triangulations present in the partition function (defined by eq. (\ref{eq:partfun})) became significantly reduced. At the same time, the third coupling constant ($\Delta$) allowed for an extended view on the phase diagram of simplicial quantum gravity. There were only two phases in DT, one phase where a link of the generic triangulation gathered a significant number of simplices around itself, and its end vertices experienced a huge coordination number\footnote{The coordination number of a vertex is defined as the number of  four-simplices which share the vertex.}, comparable to the system size, thus the name "collapsed phase". The generic geometries of the other phase could be described by the so-called, branched polymers \cite{branchedpoli}, hence the name "branched polymer phase". The analogs of these phases \cite{phase_struct_cdt} are present in CDT\footnote{Phase $B$ is the collapsed phase and phase $A$ is the branched polymer phase}, however, the topological restriction related to the foliation resulted in the appearance of two new phases \cite{ccc,cb1,cb2}. This became apparent when new observables were used related to the newly introduced time foliation. The number of spatial tetrahedra at a given CDT foliation leaf (with integer lattice time $t$) can be computed and it is, by definition, proportional to the spatial three-volume at $t$, which defines the so-called, volume profile $V_3(t)$, shown in Fig. \ref{fig:volprofs}.


\begin{figure}[ht!]
\centering
\frame{\includegraphics[width=0.23\textwidth]{fazaAs.pdf}}
\frame{\includegraphics[width=0.23\textwidth]{fazaBs.pdf}}
\frame{\includegraphics[width=0.23\textwidth]{fazaCbs.pdf}}
\frame{\includegraphics[width=0.23\textwidth]{fazaCs.pdf}}\\
\frame{\includegraphics[width=0.23\textwidth]{fazaAt.pdf}}
\frame{\includegraphics[width=0.23\textwidth]{fazaBt.pdf}}
\frame{\includegraphics[width=0.23\textwidth]{fazaCbt.pdf}}
\frame{\includegraphics[width=0.23\textwidth]{fazaCt.pdf}}\\
\caption{Spatial volume profiles of generic CDT configurations in different phases. Top: Spherical CDT: $A$, $B$, $C_b$, $C$; Bottom: Toroidal CDT:  $A$, $B$, $C_b$, $C$, respectively.}
\label{fig:volprofs}
\end{figure}

Apart from a "collapsed" volume profile of phase $B$ (where all three-volume is concentrated in one spatial "slice", i.e., the 3-dimensional  foliation leaf of integer $t$), and the heavily fluctuating volume profile of the "branched polymer" phase $A$ (independent number of tetrahedra in each spatial slice) there are new phases where the  volume profiles averaged over MC configurations follow a particular smooth function. The most interesting new phase is phase $C$ , where, in the case of the fixed spherical topology of spatial slices, the resulting average volume profile behaves as $cos^3(t)$, which corresponds to the (Euclidean) de Sitter solution of GR \cite{nonperturb_desitter}. Therefore phase $C$ is also called the de Sitter or the semi-classical phase and it is related to the IR limit of quantum gravity. The fourth phase, which is called the bifurcation phase ($C_b$), exhibits a smooth volume profile in case of large enough fixed total volumes (lattice sizes). The volume profile in phase $C_b$ is similar to the volume profile in  phase $C$ measured for the spherical spatial topology, however, it scales in a non-canonical  way when the lattice volume is increased. Furthermore, in phase, $C_b$ every second  spatial slice of integer lattice time coordinate contains a vertex with a macroscopically large coordination number, similar to "high-order" vertices encountered in phase $B$.\\

By analyzing  fluctuations of the spatial volume it was possible to derive an effective action\cite{impact_top} of CDT parametrized by the spatial volume, or alternatively by the scale factor. The effective action in the de Sitter phase ($C$) \cite{transfer_matrix} turned out to be consistent with the Hartle-Hawking minisuperspace model \cite{mini1,mini2,mini3}. This result is non-trivial, as in the case of CDT the scale factor is obtained after "integrating out" all other geometric degrees of freedom present in the lattice, while in the minisuperspace model, where spacetime isotropy and homogeneity are put in by hand, the scale factor is the only degree of freedom. Therefore, this feature of CDT is fully emergent. One could also show that the notion of effective dimension of spacetime first measured in the case of 2D CDT\cite{eff_dim_2d} and also in the case of Locally Causal Dynamical Triangulations (LCDT) \cite{spectraldim} was extended to higher dimensions. In the case of 4-dim CDT in phase $C$ it was measured to be consistent with the topological dimension four. This was not so obvious as the effective dimension measured in other phases of CDT (and earlier in DT) was different than four\cite{scaling_in_4d_grav}. Both the so-called, Hausdorff dimension \cite{cdt_desit_fi}, related to the scaling of an area and volume, and the spectral dimension \cite{spect_dim_uni_scale}, defined by a heat kernel of the Laplace operator, were measured. Additionally, the spectral dimension was shown to exhibit a non-trivial scale dependence changing from four in large scales (comparable to the size of the configuration) to approximately two in short scales and also in the presence of matter fields\cite{spectral} it can deviate from the classical values. The above phenomenon  of "dimensional reduction" was also confirmed in many other approaches to quantum gravity (e.g., in ST \cite{dim_red_st}, NCG \cite{dim_red_noncommgeom},  HLG \cite{dim_red_hlg}, AS  \cite{dim_red_1,dim_red_2,dim_red_3} and LQG \cite{dim_red_4}). \\

Most of the phase transitions present in the CDT model with spherical spatial topology were analyzed, and the $A-C$ phase transition was found to be first-order \cite{ac,pts_in_cdt,phase_struct_cdt}, while the $B-C_b$ and the $C-C_b$ turned out to be continuous  \cite{pts_in_cdt,cb1}. The existence of higher order (continuous)  phase transitions is an important result in view of the perspective existence of the UV fixed point of quantum gravity\footnote{As explained in Chapter \ref{chapter3}, such a fixed point should appear as a higher order transition point in CDT. }, however, the first study of the RG trajectories in CDT \cite{rg_flow1,rg_flow2} did not show convincing evidence for the existence of the UV fixed point. One of the issues was that a part of the phase diagram was out of reach due to computational difficulties, thus the analysis of some phase transitions was not possible. Also at that time, the available computational power was significantly smaller than presently. With the help of modern technology, much larger system sizes can be analyzed nowadays within available computational resources.\\ 

Most of the results presented above were obtained for the CDT model with the fixed spherical topology of the spatial slices. As the spatial topology choice is one of the free parameters of the model, in the past few years the main focus of the 4-dimensional CDT research was on models with  toroidal spatial topology. It was found that the phase diagram is almost invariant under the change of the topology, as all observed phases were present in both cases \cite{phase_structure_torus}. A huge difference between the spherical and the toroidal case is visible in the volume profile of phase $C$, see Fig. \ref{fig:volprofs}, and it is related to the potential term appearing in the effective action of CDT, which is different in the two cases \cite{impact_top}. In the spherical case, the potential term can be interpreted as coming from GR and it is consistent with the minisuperspace model, which also contains such a potential term for the scale factor. However, in the case of the toroidal CDT, one does not have a classical analog of the measured  potential, thus it can be interpreted as a quantum correction. Using the spatial topology of a three-torus allowed for an introduction of many new methods of analyzing the lattice-regularized quantum geometries, as it will be presented in Chapter \ref{chapter4}. It was also possible to investigate the region of the phase-diagram which was thought to be not available in the spherical CDT, see Chapter \ref{chapter3}. \\

In this thesis, we will present results related to the toroidal CDT: the study of the remaining phase transitions, including critical phenomena at the phase-transition lines. Then we will also discuss how to add scalar fields to the model of CDT, and either use them as semi-classical maps defining a coordinate system on the geometry or couple them to the geometry and analyze the effects of their back-reaction. But first, let us turn our attention to the numerical implementation of the CDT model, which is the topic of the next chapter.
% Chapter 2

\chapter{Numerical Simulations} % Main chapter title

\label{chapter2}  

%---------------------------------------------------------------------------------------

\section{The Numerical Setup}

\textit{"The student should not lose any opportunity of exercising himself in numerical calculation and particularly in the use of logarithmic tables. His power of applying mathematics to questions of practical utility is in direct proportion to the facility which he possesses in computation."  - \textbf{Augustus De Morgan}}\\\\\\

In the case of the four-dimensional CDT, there is no analytical solution, however, certain numerical methods provide useful tools in the quest to find out more about the nature of the model. One of those tools is a Monte Carlo (MC) simulation \cite{tellerteller}. In an MC simulation one attempts to numerically approximate the path integral or rather, in the Euclidean formulation, the partition function of eq. (\ref{eq:partfun}), and estimate expectation values or correlators of various observables based on a sample of independent configurations generated with a probability proportional to the Boltzmann weight: $\exp(-S)$. There are various algorithms enabling to the realization of this goal. In this discussion, we will present the Metropolis Algorithm, as this is the one that is used in the case of four-dimensional CDT. One starts from any initial state of the model (in the CDT case any allowed triangulation with a given fixed topology\footnote{In practice one usually uses an initial configuration which is easy to be constructed "by hand".}), and applies a set of local changes (moves) transforming state $A$ to $B$. In order to ensure that the probability of generating a state converges to the required equilibrium probability $\propto \exp(-S)$, the probability of performing the move has to satisfy the detailed balance condition:

\begin{equation}
\mathcal{P}(A)\mathcal{W}(A \to B) = \mathcal{P}(B)\mathcal{W}(B \to A),     
\end{equation}
where $\mathcal{P}\propto e^{-S}$ is the probability distribution of a state and $\mathcal{W}$ is the transition probability from one state to another. Additionally, the moves have to be selected in such a way, that all possible states can be reached with a finite number of performed moves, in other words, the configuration space should be closed with respect to the selected moves. This condition provides ergodicity, which is crucial to ensure meaningful statistics. In the Metropolis algorithm the transition probability $\mathcal{W}$ is chosen to be  

\begin{equation}
    \mathcal{W}(A \to B) = min\{1, e^{-\Delta S}\},
\end{equation}
where $\Delta S = S(B)-S(A)$ is the change of the action by the move.\footnote{In the case of CDT the transition probability $\mathcal{W}$ depends also on a "geometric" factor related to the number of possible locations in a triangulation where the move and its inverse can be performed.} As already mentioned,  after the so-called {\it thermalization} period, the probability distribution of configurations generated by the Metropolis algorithm reaches an equilibrium defined by the  partition function (the action $S$). This is the point when one can start collecting a sample of configurations, which has to be large enough to ensure good statistics of the measured observables.\\

CDT is perfectly suitable for numerical simulations due to its relatively simple construction. The foliated space-times (MC states) are constructed by gluing the four-dimensional simplicial building blocks, presented in Fig. \ref{fig:building-blocks}, to each other, fulfilling some global and local constraints (discussed later in detail).


\begin{figure}[ht]
    \centering
    \includegraphics[width = 0.8\textwidth]{Figures/4d-simplices.pdf}
    \caption{Two building blocks of the triangulation. The left simplex is $s_{41}$ and the right one is $s_{32}$. The other two types, $s_{14}$ and $s_{23}$, are mirrored-symmetric versions of them.}
    \label{fig:building-blocks}
\end{figure}

 Due to the nature of the triangulation, every simplex has exactly five neighbors, thus the local neighborhood of a hinge (i.e. a triangle in the four-dimensional CDT) can be simply discussed. The MC moves used in CDT are based on the so-called Pachner (or Alexander) moves \cite{pachner}, modified in such a way that, as shown later, the foliation structure with the fixed  topology of each spatial slice is conserved. Within our simulation code, we keep track of the vertices forming the 4-simplices and adjacency relations between the simplices. This information is enough to reconstruct the whole triangulation. Nevertheless, in order to optimize and speed up the code, we also keep track of some additional information, e.g., some specific types of sub-simplices or their coordination numbers.\footnote{The coordination number  measures how many $4$-dimensional simplices meet at a given vertex, link or triangle.} When we perform a measurement we usually have to calculate observables from the actual adjacency relations or other data that we store. As Fig. \ref{fig:building-blocks} shows, the graphical representation of simplices on a 2-dimensional figure is difficult. Due to this reason, we will present the idea of the moves of 4-dimensional CDT by showing how they impact the triangulation at the $t+\frac{1}{2}$ plane. Technically the $t+\frac{1}{2}$ plane describes the connectivity structure of the triangulation in a {\it slab}, defined by all simplices between spatial slices at (integer) lattice time $t$ and $t+1$. This treatment simplifies the discussion as it reduces the dimensionality of the problem by one, because on the $t+\frac{1}{2}$ plane a slab of the 4-dimensional triangulation is mapped to a 3-dimensional graph decorated by colors. The construction of the building blocks of this 3- dimensional graph is analogous to the method used in \cite{abab} in the case of three-dimensional CDT. Specifically, the color or, in other words, the type of the links (solid black/grey or dashed) is important, as only links of the same type can be connected. When referring to the links of such a graph, the words "color" or "type" will be used interchangeably.

\begin{figure}[ht]
    \centering
    \includegraphics[width = 0.8\textwidth]{Figures/simplices_thalf.pdf}
    \caption{The figure shows the representations of $s_{41}$ (left) and $s_{32}$ (right) simplices in  the $t + \frac{1}{2}$ plane. The simplex $s_{41}$ is a single-colored tetrahedron, $s_{32}$ is a bi-colored prism with two triangular and three rectangular faces. The other two types of simplices $s_{14}$ and $s_{23}$ are mirror-reflections.}
    \label{fig:projection}
\end{figure}

In the $t+\frac{1}{2}$ plane, instead of the 4-simplices,  we now have 3-dimensional objects:  tetrahedra and prisms, see Fig. \ref{fig:projection}. In order to distinguish between the $s_{41}$ and the $s_{14}$ simplices we  attribute colors to the tetrahedra, such that a tetrahedron belonging to the $s_{41}$ simplex will have all-black (triangular) faces, and all triangles of a  tetrahedron belonging to the  $s_{14}$ simplex will be  grey. Even though each 4-simplex  has formally 5 neighbors, the tetrahedra have only four, which means that (for better clarity of the graphs) we omit the neighbors belonging to the previous/next slab. The prisms with black/grey triangular faces and transparent rectangular sides represent the $s_{32}/s_{23}$ simplices, respectively. Due to the topological constraints imposed on the CDT triangulations (the fixed spatial topology must be preserved  in all  layers interpolating between the spatial slices at $t$ and $t+1$)  the discretized geometry of the $t+\frac{1}{2}$ layer must be also connected in a specific way, meaning that a black triangle can be glued only to a black triangle, a grey triangle to a grey triangle, and a  transparent rectangle to a  transparent rectangle. It reflects the fact that the $s_{41}$ simplex can be adjacent only to  $s_{41}$ or $s_{32}$ simplices\footnote{Here we disregard the connections to the $s_{14}$ simplices of the previous slab.}, the $s_{32}$ simplex can be adjacent only to $s_{41}$, $s_{32}$ or $s_{23}$ simplices, etc.  


\begin{figure}[h]
    \centering
    \includegraphics[width = 0.45\textwidth]{Figures/32.pdf}
    \includegraphics[width =0.45\textwidth]{Figures/40.pdf}
    \caption{The figure presents the
    3-dimensional elements of the $t + 1/2$ plane. The prism with triangular bases and rectangular sides (left panel) comes from a $s_{32}$ simplex. In the graphical representation, it will be a blue dot with two solid black and three dashed legs. The tetrahedron  (right panel) comes from a $s_{41}$ simplex. In the graphical representation, it will be a black dot with four solid black legs (connections to neighboring slabs are omitted). Similarly, one has a red dot with two solid grey and three dashed legs, and a grey dot with four solid grey legs, coming from the mirror-reflected $s_{23}$ and $s_{14}$ simplices, respectively, which are not shown in the plot.}
    \label{fig:P_types}
\end{figure}

In order to simplify notation we will represent the black/grey tetrahedra by the black/grey dots, and the prisms by the blue/red dots, such that a blue dot represents a prism with two black triangles (and three transparent rectangles) and a red dot is a prism with  two grey triangles (and also three transparent rectangles). In the 4-dimensional context, the black/grey dots will correspond to the $s_{41}$ / $s_{14}$ simplices in the slab, and the blue/red dots will correspond to the $s_{32}$ / $s_{23}$ simplices, respectively. The dots will be connected by "legs" of various types, representing the different types of connections (through colored triangles or rectangles) in the $t+\frac{1}{2} $ plane. Thus a solid black/grey  leg will represent a black/grey triangle, and the dashed leg will be a transparent rectangle, see Fig. \ref{fig:P_types}. In order to preserve the topological restrictions, only the legs of the same color/type can be connected. All possible  connections between colored dots are presented in Fig \ref{fig:connectivities} (up to mirror-reflections). 
\begin{figure}[h!]
    \centering
    \includegraphics[width =0.8\textwidth]{Figures/Connection_types.pdf}
    \caption{The figure presents  possible connections between various objects of the $t+\frac{1}{2}$ plane. Black dots (tetrahedra) can be connected to each other and to blue dots (prisms) via solid black legs (triangles). Similarly, blue dots can be connected to black and blue dots via solid black legs, but they can be also connected to blue and red dots via dashed legs (rectangles). The red dots have two solid grey legs, which can be connected to other red or grey dots, which are not shown in the figure.}
    \label{fig:connectivities}
\end{figure}
As, by definition, the manifold constraints of the original triangulation are not violated, and the description of the triangulation in the $t+\frac{1}{2}$ plane is still a manifold (a three-dimensional one), it is in one-to-one correspondence with the transition tensor of the triangulation from slice $t$ to $t + 1$. An example (part of the) $t+\frac{1}{2}$ slice of a CDT triangulation and the corresponding graph with colored dots and various types of legs is presented 
in Fig.~\ref{fig:bbbbr_example}. \\
\begin{figure}[h]
    \centering
    \includegraphics[width =0.8\textwidth]{Figures/example_11.pdf}
    \caption{An example of a possible connection between four $s_{41}$, three $s_{32}$ and one $s_{23}$ simplices in the $t+\frac{1}{2} $ plane (left panel) and the corresponding graphical representation (right panel). A solid black loop in the graphical representation  is a spatial link in the  CDT triangulation. I deleted here a sentence with "red triangles" - you did not introduce such triangles in the description - it becomes a mess}
    \label{fig:bbbbr_example}
\end{figure}

One should note that if, in the original CDT triangulation, two $s_{41}$ simplices are connected to the same vertex at $t+1$ then these simplices correspond necessarily to two adjacent tetrahedra in the $t+\frac{1}{2}$ plane, or in the graphical representation two black dots connected by a solid black line. The same is of course true for the mirror-reflected $s_{14}$ simplices and thus the grey dots connected by a solid grey line. Additionally, using the graphical representation one can recognize the links of the original CDT triangulation as closed loops in the colored dot graphs. Closed solid loops are spatial links (black on slice $t$ and grey on slice $t+1$), while closed dashed loops are time-like links of the original triangulation. Then, the coordination number of a link in the original triangulation is related to the number of dots along that loop. Another important feature of this graphical representation is, that the vertices of the original triangulation are represented as 3-dimensional objects defined by the surrounding colored dots and closed loops. As it was already mentioned, the above graphical representation contains only elements of the $t + \frac{1}{2}$ plane of a slab, therefore the true coordination number of spatial links will actually also depend on a similar graph in the adjacent slab.\\

As the CDT moves are local, i.e., they change only the interior of a small region in a CDT triangulation, the connection to the outside region of the triangulation is preserved, which, in the graphical representation, manifests itself by the fact that the type and number of  external legs remain unchanged when the move is performed. \\

Now, we are ready to discuss the moves with the graphical representation defined above. In the following discussion, if only black or black-and-blue dots are shown, then recoloring black to grey and blue to red will lead to the mirror-reflected version of the movie. 
 
\subsection{Move-2}


\begin{figure}[h]
    \centering
    \includegraphics[width =0.45\textwidth]{Figures/move2.pdf}
   \includegraphics[width = 0.45\textwidth]{Figures/move2v2.pdf}
    \caption{Move-2: version-1 (left) and version-2 (right). In the CDT triangulation, it replaces a (tetrahedral) interface between 4-simplices with a link, creating additional three 4-simplices.}
    \label{fig:m2}
\end{figure}

"Move-2" is a move that changes the interface between two (black-blue or blue-red) dots and increases the number of dots by two.  It exists in two versions. Version one can be done between a black and a blue dot. The move removes two dots and replaces them with four, see Fig. \ref{fig:m2}. After the move, the black dot will be connected to the external leg, which was earlier connected to the blue dot, and, at the same time, all of the original black dot's external legs will become the external legs of the three new blue dots. These blue dots are also connected via dashed legs. The second version of the movie can be done between a blue and a red dot. The move replaces the dashed line between the original blue and red dots with four dashed lines between the  blue and red dots. These new blue and red dots are connected to the external dashed legs of the original configuration.

\subsection{Move-3}



\begin{figure}[h]
    \centering
    \includegraphics[width = 0.45\textwidth]{Figures/move3.pdf}
    \includegraphics[width = 0.45\textwidth]{Figures/move3_v2.pdf}
    \caption{Move-3: version 1 (left) and version 2 (right). In the CDT triangulation, it replaces the triangular interface with a dual one. }
    \label{fig:m3}
\end{figure}

The next move is "move-3", shown in Fig. \ref{fig:m3}, which is an analog of the "flip" move used in the two-dimensional CDT. It also comes in two versions. In version one, it replaces one blue and two red dots with one red and two blue dots, which corresponds to replacing an $s_{12}$ triangle with an $s_{21}$ in the CDT triangulation. The second version removes two adjacent blue dots connected with the black dot and places them on the other side, i.e. connects them to two external legs of the original black dot. At the same time, the black dot gets connected to the two external legs originally connected to the blue dots. The move does not change any of the global numbers in the triangulation.

\subsection{Move-4}

The "move-4" is one of the simplest ones, and is shown in Fig. \ref{fig:m4}. Move-4 and its inverse are effectively a special case of a "split-merge" move. It removes a black dot and replaces it with a fully connected set of four black dots. The four dots are also connected to the external legs of the original configuration, one by one.

\begin{figure}[h]
    \centering
    \includegraphics[width = 0.5\textwidth]{Figures/move4.pdf}
    \caption{Move-4 replaces a black dot with four fully connected black dots, connecting each of them to the external link of the original configuration. In the CDT triangulation, it adds a vertex inside an $s_{41}$ simplex, replacing the simplex with four new $s_{41}$ simplices.}
    \label{fig:m4}
\end{figure}

As every solid loop in the graphical representation corresponds to a spatial link, and as it is visible in the right panel of Fig. \ref{fig:m4} there are four such solid loops, thus in the real triangulation four new spatial links are created. As all the four black dots are adjacent to each other, it can happen only if they share a vertex, thus the move creates a vertex in the original triangulation, this vertex has coordination number  four.\footnote{In fact, the coordination number is eight, as there are additional $s_{14}$ simplices in the previous slab.} 

\subsection{Move-5}
The last move is "move-5", shown in Fig. \ref{fig:m5}.

\begin{figure}[h]
    \centering
    \includegraphics[width = 0.5\textwidth]{Figures/move5.pdf}
    \caption{Move-5 replaces two adjacent black dots with three black dots. In the CDT triangulation, it creates a spatial link with coordination number three. The link is signaled by a solid black loop in the graphical representation.}
    \label{fig:m5}
\end{figure}

The move takes two adjacent black dots (tetrahedra) and replaces the triangular interface formed by the three common vertices with a link that connects the remaining two vertices. The move creates a link with coordination number three\footnote{In fact, the coordination number is six, as there are additional $s_{14}$ simplices in the previous slab.}, signaled by the solid black loop connecting the three black dots on the right panel in Fig. \ref{fig:m5}. The inverse move requires a link with coordination number three.\\ 

One should also note, that in the  CDT code, we use the full four-dimensional triangulation. In the graphical representation, it could be achieved by  adding a single solid external leg to each black/grey dot. This way  each black (grey) dot of the $t+\frac{1}{2}$ plane would be connected to a grey (black) dot of the previous (next) slab, defined by the plane at $t+\frac{1}{2}-1$ ($t+\frac{1}{2}+1$). Move-4 and move-5 are the only moves that are affected by the neighboring slabs, and the connected grey/black dots of the adjacent slabs would behave exactly the same way as the black/grey dots behave in the above graphical representation description. \\

So far we discussed the moves which are currently used in the MC simulations of the four-dimensional  CDT. In principle, one could try to define some new moves, but doing so is a hard task as they must be efficient numerically and their required components (a vertex/link/triangle with a given coordination number) must be easy to be tracked during the simulations, e.g., a vertex with a given coordination number is easy to track, but it is not the case for more complex structures. In appendix \ref{AppendixB} we discuss some proposals for new moves with the help of the  graphical representation defined above. 

%----------------------------------------------------------------------------------------
 
% Chapter 3

\chapter{Second-order ordinary differential equation}\label{ch 3} % Main chapter title

\label{Chapter3} % For referencing the chapter elsewhere, use \ref{Chapter3} 

%----------------------------------------------------------------------------------------

% Define some commands to keep the formatting separated from the content 

%----------------------------------------------------------------------------------------

\section{Variational formulation for $\partial_{tt}u+ \mu u=f$}\label{sect var form}
As a model problem we consider the following second-order linear equation: 
\begin{equation}\label{eq ode}
\partial_{tt}u(t)+ \mu u(t)=f(t), \quad \text{for} \ t \in (0,T), \quad u(0)=\partial_{t}u(t)_{|t=0}=0,
\end{equation}
where $\mu >0$.

The variational formulation of \eqref{eq ode} reads as follows:
\begin{equation}\label{var ode}
\begin{cases}
\text{Find} \ u \in H^1_{0,*}(0,T) \quad \text{such that}\\
a(u,v)=\langle f,v \rangle_{(0,T)} \quad  \forall v \in H^1_{*,0}(0,T),
\end{cases}
\end{equation}
where $T>0$ and $f \in [H^1_{*,0}(0,T)]' $ are given, and where the bilinear form $a(\cdot,\cdot): H^1_{0,*}(0,T) \times H^1_{*,0}(0,T) \longrightarrow \mathbb{R}$ is 
\begin{equation}\label{bil ode}
a(u,v):=-\langle \partial_{t}u, \partial_{t}v \rangle_{L^2(0,T)}+ \mu \langle u,v \rangle_{L^2(0,T)},
\end{equation} 
for all $u \in H^1_{0,*}(0,T)$, $v \in H^1_{*,0}(0,T)$. The notation $\langle \cdot,\cdot \rangle_{(0,T)}$ denotes the duality pairing as extension of the inner product in $L^2(0,T)$, and the Sobolev spaces $H^1_{0,*}(0,T), H^1_{*,0}(0,T)$ are introduced in Section \ref{sec sobolev}. Note that the first initial condition $u(0)=0$ is incorporated in the solution space $H^1_{0,*}(0,T)$, whereas the second initial condition $\partial_{t}u(t)_{|t=0}=0$ is considered as a natural
condition in the variational formulation.

Thanks to the Cauchy-Schwarz inequality and the Poincaré inequality \eqref{Poinc}, the bilinear form $a(\cdot,\cdot)$ is bounded with
\begin{equation}\label{cont of a}
|a(u,v)| \leq \Bigg(1+\frac{4 T^2 \mu }{\pi^2} \Bigg) |u|_{H^1(0,T)}|v|_{H^1(0,T)},
\end{equation}
for all $(u,v) \in H^1_{0,*}(0,T) \times H^1_{*,0}(0,T)$. 

In order to prove the well-posedness of \eqref{var ode} we consider an equivalent variational problem. Thus, we introduce the isomorphism
\begin{gather}\label{HT}
\overline{\mathcal{H}}_T: H^1_{0,*}(0,T) \longrightarrow H^1_{*,0}(0,T)\\
u \mapsto u(T)-u(\cdot),
\end{gather}
where its inverse is given by
\begin{gather*}
\overline{\mathcal{H}}_T^{-1}: H^1_{*,0}(0,T) \longrightarrow H^1_{0,*}(0,T)\\
v \mapsto v(0)-v(\cdot).
\end{gather*}
Since the above mappings are actually isometries with respect to $|\cdot|_{H^1(0,T)}$, then the well-posedness of  \eqref{var ode} is equivalent to the well-posedness (with the same stability constant) of the following problem
\begin{equation}\label{var ode equiv}
\begin{cases}
\text{Find} \ u \in H^1_{0,*}(0,T) \ \text{such that}\\
a(u,\overline{\mathcal{H}}_Tv)=\langle f,\overline{\mathcal{H}}_Tv \rangle_{(0,T)} \quad  \forall v \in H^1_{0,*}(0,T),
\end{cases}
\end{equation}
where the solution and test space coincide. Note that the continuity of the bilinear form in \eqref{var ode equiv} is an immediate consequence of estimate \eqref{cont of a}. Unfortunately, at least for $\mu$ sufficiently large, the bilinear form of \eqref{var ode equiv} is not coercive, hence, we cannot rely on the classical Lax-Milgram Lemma \ref{lax-milgram}, but we need more specific tools. The branch of functional analysis that we exploit is called \textit{Fredholm theory} and studies compact perturbations of linear bounded operators. In particular, there holds the following Theorem.

\begin{theorem}[\cite{Zank2020}]\label{zank}
	For a given $f \in [H^1_{*,0}(0,T)]'$, there exists a unique solution $u \in H^1_{0,*}(0,T)$ of the variational formulation \eqref{var ode equiv}, and the following a priori estimate holds
	\begin{equation}\label{stab ode}
	|u|_{H^1(0,T)}\leq \Bigg(\frac{2+\sqrt{\mu}T}{2}\Bigg) \|f\|_{[H^1_{*,0}(0,T)]'}.
	\end{equation}
	In addition, \eqref{stab ode} is optimal with respect to the order of $\mu$ and $T$.
\end{theorem}
The well-posedness of \eqref{var ode equiv} is an immediate consequence of the results presented in Section \ref{sec prel ode}. Indeed, defining two bounded linear operators $\mathcal{B},\mathcal{D}:H^1_{0,*}(0,T) \rightarrow [H^1_{0,*}(0,T)]'$ by
\begin{gather*}
\langle \mathcal{B}u,v \rangle _{(0,T)} := \langle \partial_tu,\partial_t v\rangle_{L^2(0,T)}, \quad u,v \in H^1_{0,*}(0,T)\\
\langle \mathcal{D}u,v \rangle _{(0,T)} := \langle u,v(T)-v \rangle_{L^2(0,T)}, \quad u,v \in H^1_{0,*}(0,T),
\end{gather*}
 the map $\mathcal{B}+\mu \mathcal{D}: H^1_{0,*}(0,T) \rightarrow [H^1_{0,*}(0,T)]'$ is an injective Fredholm operator of index 0, where the compactness of $\mathcal{D}$ follows from the properties of trace operators. However, estimate \eqref{stab ode} gives an explicit dependency relation of its stability constant on $T,\mu$. In order to prove \eqref{stab ode}, we use the notation $\mathcal{A}$ for the bounded linear operator related to the bilinear form \eqref{bil ode}, i.e.,
\begin{gather*}
\mathcal{A}:  H^1_{0,*}(0,T) \rightarrow [H^1_{*,0}(0,T)]' \quad \text{s.t.}\\
\langle \mathcal{A}u,v \rangle _{(0,T)}:=a(u,v), \quad u \in H^1_{0,*}(0,T), v \in H^1_{*,0}(0,T).
\end{gather*}
Since $\mathcal{A}$ is an isomorphism, its adjoint operator $\mathcal{A}^*:H^1_{*,0}(0,T)\rightarrow  [H^1_{0,*}(0,T)]' $ is an isomorphism with $(\mathcal{A}^*)^{-1}=(A^{-1})^*$, see (\cite{adjoint}). As a consequence, given $g \in [H^1_{0,*}(0,T)]'$, the adjoint problem 
\begin{equation}\label{adj dim}
\begin{cases}
\text{Find} \ z \in H^1_{*,0}(0,T) \quad \text{such that}\\
a(w,z)=\langle g,w \rangle_{(0,T)} \quad \forall w \in H^1_{0,*}(0,T)
\end{cases}
\end{equation}
is well-posed. In particular, it is possible to compute the exact solution of problem \eqref{adj dim} using a Green's function, if the right-hand side  $g \in [H^1_{0,*}(0,T)]'$ depends on a fixed $u \in H^1_{0,*}(0,T)$; for more details we refer to (\cite{Zank2020}). For the optimality of the estimate we refer to Theorem 4.2.6 of (\cite{Zank2020}).

\section{Isogeometric discretization}\label{sec iga}
As discrete spaces for the Galerkin discretization of \eqref{var ode} we choose to consider spline spaces of degree two with maximal regularity, i.e., piecewise polynomials of degree two with global $C^1$ regularity.

Given a positive integer $N$, let $\Xi:=\{t_1, \ldots, t_{N+3}\}$ be an open knot vector in $[0,T]$ with the interior knots that appear just once, i.e., $0=t_1=t_2=t_3 < \ldots  <t_{N+1}=t_{N+2}=t_{N+3}=T$. By means of Cox-de Boor recursion formulas \eqref{cox de boor} we define the quadratic univariate B-spline basis functions ${b}_{i,2}: [0,T] \rightarrow \mathbb{R}$ for $i=1, \ldots, N$ (we omit the subscript $\Xi$ to lighten the notation). Thus, the univariate spline space is defined as 
\begin{equation*}
{S}^2_h:= \text{span}\{{b}_{i,2}\}_{i=1}^N,
\end{equation*}
where $h$ is the mesh-size, i.e., $h:=\max{\{|t_{i+1}-t_i|\ : \ i=1,\ldots,N+2}\}$. 

We introduce the spline space with initial conditions as
\begin{equation}\label{iga space}
\begin{split}
V^h_{0,*} &:= {S}^2_h \cap H^1_{0,*}(0,1)
= \{v_h \in {S}^2_h \ : \ {v}_h(0)=0 \}\\
&= \text{span}\{{b}_{i,2}\}_{i=2}^N,
\end{split}
\end{equation}
which is our isogeometric solution space, and the spline space 
\begin{equation}\label{iga test space}
\begin{split}
{V}^h_{*,0} &:= {S}^2_h \cap H^1_{*,0}(0,1)
= \{ {v}_h \in {S}^2_h \ : \ {v}_h(T)=0 \}\\
&= \text{span}\{ {b}_{i,2}\}_{i=1}^{N-1},
\end{split}
\end{equation}
which is our isogeometric test space. Note that $V^h_{0,*}=\text{span}\{{b}_{i,2}\}_{i=2}^N$  and ${V}^h_{*,0}=\text{span}\{ {b}_{i,2}\}_{i=1}^{N-1}$ since the first and last B-spline basis functions of an open knot vector are interpolating functions.

\begin{oss}
	Typically, in isogeometric discretizations, and in the GeoPDEs library that we are going to use for the numerical experiments, one considers the parametric domain $[0,1]^n$ and the splines/NURBS on this domain. Once these spaces are constructed, 
	one maps, via a NURBS function $F$,	
	the parametric domain to the physical domain of interest, \begin{minipage}[r]{0.6\linewidth} 
		\includegraphics[width=\textwidth]{Figures/isogeo}
	\end{minipage} \\ and the discrete solution and test spaces are the pushforward by $F$ of splines/NURBS spaces  on $[0,1]^n$. \\
	In our case, the map $F$ would be $F: [0,1] \rightarrow [0,T]$ s.t. $\tau \mapsto T\tau$. Therefore, doing the pushforward of the spline spaces on the parametric interval, we would obtain spline spaces on $[0,T]$ with B-spline basis functions that are the pushforward of the B-spline basis functions on $[0,1]$. It is then sufficient for us to construct the discrete spaces directly on $[0,T]$.
\end{oss}

A conforming Galerkin-Petrov isogeometric discretization
of \eqref{var ode} is the following problem
\begin{equation}\label{iga ode}
\begin{cases}
\text{Find} \ u_h \in V^h_{0,*} \quad \text{such that}\\
a(u_h,v_h)=\langle f,v_h \rangle_{(0,T)} \quad  \forall v_h \in V^h_{*,0}.
\end{cases}
\end{equation}
As in the continuous framework, the restricted operator
\begin{gather*}
\overline{\mathcal{H}}_{T_{|V^h_{0,*}}}: V^h_{0,*} \longrightarrow V^h_{*,0}\\
u_h \mapsto u_h(T)-u_h(\cdot)
\end{gather*}
is actually an isometric isomorphism with respect to $|\cdot|_{H^1(0,T)}$. Therefore, the well-posedness of \eqref{iga ode} is  equivalent to the well-posedness (with the same stability constant) of the conforming Galerkin-Bubnov isogeometric discretization of \eqref{var ode equiv}:
\begin{equation}\label{iga ode equiv}
\begin{cases}
\text{Find} \ u_h \in V^h_{0,*} \ \text{such that}\\
a(u_h,\overline{\mathcal{H}}_Tv_h)=\langle f,\overline{\mathcal{H}}_Tv_h \rangle_{(0,T)} \quad  \forall v_h \in V^h_{0,*}.
\end{cases}
\end{equation}
The isogeometric spaces $V^h_{0,*}$ define a dense discrete sequence in $H^1_{0,*}(0,T)$ (as we will see in Section \ref{approx properties spline}), directed in the real parameter $h$. Hence, we can conclude the following Theorem.

\begin{theorem}\label{cond well-posed iga}
	There exists two constants $C(T,\mu),\overline{h}>0$ such that, if $h\leq \overline{h}$, problem \eqref{iga ode equiv} is well-posed, with the stability estimate
	\begin{equation*}
	|u_h|_{H^1(0,T)} \leq C(T,\mu) \|f\|_{[H^1_{*,0}(0,T)]'} \quad f \in [H^1_{*,0}(0,T)]',
	\end{equation*}
	where $u_h$ is the unique solution of \eqref{iga ode equiv}.
	Moreover, if $h\leq \overline{h}$, a quasi-optimality estimate holds
	\begin{equation*}
	|u-u_h|_{H^1(0,T)} \leq \Big(1+C(T,\mu) \|\mathcal{A}\| \Big) \inf_{v_h \in V^h_{0,*}} |u-v_h|_{H^1(0,T)},
	\end{equation*}
	where $u \in H^1_{0,*}(0,T)$ is the unique solution of \eqref{var ode equiv}, and where $\|\cdot\|$ is the norm of the \textit{solution-to-data} operator $\mathcal{A}:H^1_{0,*}(0,T) \rightarrow [H^1_{0,*}(0,T)]'$, related to the bilinear form of the variational formulation \eqref{var ode equiv}.
\end{theorem}
\begin{proof}
	Theorem \ref{cond well-posed iga} is a straightforward consequence of the results presented in Section \ref{sec prel ode}, i.e., Proposition \ref{prop infsup comp}, Corollary \ref{well-pos comp dis} and Corollary \ref{céa comp}.
\end{proof}

\subsection{Approximation properties of spline spaces with an initial boundary condition}\label{approx properties spline}
So far we have shown that, if the IGA discretization is \textit{sufficiently fine}, problem \eqref{iga ode equiv} is well posed, stability holds and we eventually reach convergence. However, we would like to make explicit the threshold on the mesh-size and (possibly) the stability and quasi-optimality constants. In order to get these results we need a priori error estimate in the Sobolev semi-norm $|\cdot|_{H^1(0,T)}$
for approximation in spline spaces of maximal smoothness on grids defined by arbitrary break points. As pointed out in the paper (\cite{n-width}), classical error estimates for spline approximation are expressed in terms of:
\begin{itemize}
	\item [1.] A power of the mesh-size.
	\item [2.] An appropriate semi-norm of the function to be approximated.
	\item [3.] A constant which is independent of the previous quantities.
\end{itemize}

However, we are interested in estimates with the constant of point 3 made explicit. In this respect, article (\cite{n-width}) is relevant to us, since we slightly modify the construction of this work in order to obtain the estimates we need for test and trial spaces of our interest. In particular, the authors study the approximation properties of spline spaces, without boundary conditions, and of periodic spline spaces. We partially extend their work by including an initial boundary condition.

Let $S_\Xi^p(0,T)$ be the spline space of degree $p \geq 0$ and maximal regularity, where $\Xi$ is the open knot vector that defines the B-spline basis functions of $S_\Xi^p(0,T)$. We firstly observe that we use the terms \textit{knot vector} and \textit{break points} as introduced in Section \ref{sec splines}, unlike (\cite{n-width}) in which the sequence of break points is called knot vector. Let $Q_p^q: H^q(0,T) \rightarrow S_\Xi^p(0,T)$, $q=0,\ldots,p$, be a sequence of bounded linear operators such that 
\begin{gather}
Q^0_p:=P_p \quad \text{is the $L^2(0,T)$ orthogonal projection on} \ S^p_\Xi(0,T) \label{Qp0}, \\
\begin{split}
(Q^q_p&u)(t):=u(0) + \int_0^t(Q_{p-1}^{q-1}\partial_t u)(s) \ ds \label{Qpq},\\
& \quad \ 1 \leq q \leq p, \ \forall u \in H^q(0,T),
\end{split}
\end{gather}
%where $c(u)$ is a constant such that
%\begin{equation}
%\int_{0}^{T} (Q_p^qu)(t)dt = \int_0^T u(t) dt, 
%\end{equation}
%i.e., $c(u)$ is the mean of the function $u(\cdot)-\int_0^{(\cdot)}(Q_{p-1}^{q-1}\partial u)(s) ds $
%on the interval $[0,T]$. 
Firstly, let now observe the reason why the operators $Q^p_q$ maps $H^q(0,T)$ into $S^p_\Xi(0,T)$. Let $K$ be the integral operator such that, for $f \in L^2(0,T)$, 
\begin{equation}\label{K}
Kf(t):=\int_0^t f(s) \ ds.
\end{equation}
Recall from Theorem 17 of (\cite{chapter1}) that the spline space $S_\Xi^p(0,T)$ satisfies 
\begin{equation}\label{der spline}
\partial^q_tS_\Xi^p(0,T)=S_\Xi^{p-q}(0,T), \quad \text{for any} \ q=0,\ldots,p,
\end{equation}
where, by abuse of notation, we denote by the letter $\Xi$ both the open knot vector $\{t_1, \ldots, t_{N+p+1}\}$ (with $0=t_1=\ldots=t_{p+1} <\ldots  <t_{N+1}=\ldots =t_{N+p+1}=T$) and the one obtained from $\{t_1, \ldots, t_{N+p+1}\}$ by reducing the external nodes from $p+1$ to $p-q+1$. 
Thus, as a consequence of \eqref{der spline} and of the Fundamental Theorem of Calculus, there holds
\begin{equation}
S_\Xi^p(0,T)=\mathbb{P}_0 + K\Big(S_\Xi^{p-1}(0,T)\Big), \quad \forall p \geq 1
\end{equation}
where $\mathbb{P}_0$ is the space of constant functions. Hence, by an inductive argument, the range of $Q_p^q$ is a subspace of $S_\Xi^p(0,T)$. The linearity and continuity of $Q^p_q$ are straightforward.


\begin{prop}
	The maps $Q_p^q$ defined in \eqref{Qp0} and \eqref{Qpq} are projection operators with range $S_\Xi^p(0,T)$. They also satisfies
	\begin{equation}\label{commut Q}
	\partial_t Q_p^q = Q^{q-1}_{p-1}\partial_t, \quad \text{for all} \ p \geq 1, \ q=1,\ldots,p.
	\end{equation}
\end{prop}

\begin{proof} 
	Equality \eqref{commut Q} is satisfied by definition. 
	
	If $q=0$, $Q^q_p$ is a projection operator by definition. We use an inductive argument in order to prove that $Q^q_p\big(Q^q_pu \big) = Q^q_p u$ for all $u \in H^q(0,T)$ if $p \geq 1$ and $q=1,\ldots,p$. Let $p=q=1$, then
	\begin{equation*}
	\begin{split}
	Q^1_1\big(Q^1_1u \big)(t)&=Q^1_1\Big( u(0) +\int_0^{(\cdot)}(Q_0^0\partial_t u)(s) \ ds \Big)(t) \\
	&=Q^1_1 \big( u(0) \big) + Q^1_1\Big(\int_0^{(\cdot)}(Q_0^0\partial_t u)(s) \ ds \Big) (t) \quad \text{(linearity of $Q^p_q$)} \\
	&=u(0)+ \int_0^t Q^0_0 \Big(\partial_t \int_0^{(\cdot)}(Q_0^0\partial_t u)(s) \ ds \Big)(\tau) \ d\tau \quad \text{(definition of $Q^p_q$)} \\
	&\overset{\eqref{commut Q}}=u(0)+ \int_0^t Q^0_0 \Big(\partial_t \int_0^{(\cdot)}(\partial_t Q_1^1u)(s) \ ds \Big)(\tau) \ d\tau\\
	&=u(0)+ \int_0^t Q^0_0 \big(\partial_t Q^1_1u(\tau)-\partial_t(Q^1_1u(0)) \big) \ d\tau \quad \text{(Fundamental Th. of Calculus)}\\
	&\overset{\eqref{commut Q}}=u(0)+ \int_0^t Q^0_0 \big(Q^0_0\partial_t u(\tau)\big) \ d\tau \\
	&=u(0)+\int_0^t (Q^0_0 \partial_t u)(\tau) \ d\tau \quad \text{($Q^0_0$ is $L^2(0,T)$-projection)}.
	\end{split}
	\end{equation*}
    Let now assume that the statement is true for $p \geq 1, \ q=1,\ldots,p$, and let now consider $p+1, \ q=1,\ldots,p+1$. Hence, as before, there hold the following identities
	\begin{equation*}
	\begin{split}
	Q^q_{p+1}\big(Q^q_{p+1}u \big)(t)&=Q^q_{p+1}\Big( u(0) +\int_0^{(\cdot)}(Q_p^{q-1}\partial_t u)(s) \ ds \Big)(t) \\
	&=u(0)+ \int_0^t Q_p^{q-1} \Big(\partial_t \int_0^{(\cdot)}(Q_p^{q-1}\partial u)(s) \ ds \Big)(\tau) \ d\tau \\
	&=u(0)+\int_0^t (Q_p^{q-1} \partial_t u)(\tau) \ d\tau.
	\end{split}	
	\end{equation*}
If $q=0$, the range of $Q^q_p$ is clearly $S^p_\Xi(0,T)$. Indeed $Q^0_p$ is the $L^2(0,T)$-projection on $S^p_\Xi(0,T)$, hence, in particular, ${Q^0_p}_{|S^p_\Xi(0,T)} \equiv Id$. If $p \geq 1$ and $q=1,\ldots,p$, the fact that the range of $Q^q_p$ is $S^p_\Xi(0,T)$ is straightforward. Indeed, by using an inductive argument and \eqref{der spline}, one can prove that ${Q^q_p}_{|S^p_\Xi(0,T)} \equiv Id$ also for $p \geq 1$ and $q=1,\ldots,p$.
\end{proof}

Let now recall Theorem $1$ of (\cite{n-width}).

\begin{theorem}[E. Sande, C. Manni, H. Speleers]\label{sande,manni,speleers}
	For any sequence of break points that defines the open knot vector $\Xi$ with maximal regularity, let $h$ denote its maximal knot distance, and let $P_p$ denote the $L^2(0,T)$ orthogonal projection onto the spline space $S^p_\Xi(0,T)$. Then, for any function $u \in H^r(0,T)$ with $r \geq 1$,
	\begin{equation}\label{proj L2}
	\|u-P_pu\|_{L^2(0,T)} \leq \Big ( \frac{h}{\pi} \Big)^r |u|_{H^r(0,T)},
	\end{equation}
	for all $p \geq r-1$.
\end{theorem}

\begin{oss}
	Let $P$ be the $L^2(0,T)$ orthogonal projection onto a finite dimensional subspace $\mathcal{X}$ of $L^2(0,T)$. For $A \subseteq L^2(0,T)$ we define
	\begin{equation*}
	E(A,\mathcal{X})=\sup_{u \in A}{\|u-Pu\|_{L^2(0,T)}},
	\end{equation*}
	i.e., $E(A,\mathcal{X})$ is the ``maximal of the minimal distances'' between $A$ and the projection space $\mathcal{X}$.
	The \textit{Kolmogorov $L^2$ n-width} of $A$ is defined as 
	\begin{equation*}
	d_n(A)=\inf_{\mathcal{X} \subset L^2(0,T), dim\mathcal{X}=n} E(A,\mathcal{X}).
	\end{equation*}
	The projection space $\mathcal{X}$ is called an optimal subspace for $d_n(A)$ if $d_n(A)=E(A,\mathcal{X})$.
	
Let $r \geq 1$ and let $A=\{u \in H^r(0,T): \|\partial^r_t{u}\|_{L^2(0,T)} \leq 1 \}$. 
	Clearly, for any finite subspace $\mathcal{X}$ of $L^2(0,T)$ the following estimate holds
	\begin{equation*}
	\|u-Pu\|_{L^2(0,T)} \leq E(A,\mathcal{X})\|\partial^r_t{u}\|_{L^2(0,T)} \quad \forall u \in H^r(0,T),
	\end{equation*}
	and, in (\cite{n-width}), an error estimate of the form
	\begin{equation}\label{proj}
	\|u-Pu\|_{L^2(0,T)} \leq C\|\partial^r_t{u}\|_{L^2(0,T)} \quad \forall u \in H^r(0,T)
	\end{equation}
	is said to be \textit{sharp} if 
	\begin{equation*}
	C=E(A,\mathcal{X}).
	\end{equation*}
Also, in (\cite{n-width}), a projection error estimate of the form \eqref{proj} is said to be \textit{optimal} if the subspace we project onto is optimal for the Kolmogorov $L^2$ n-width of $A$ and the projection error estimate is sharp.

In (\cite{n-width}) the following results are proven.
\begin{itemize}
	\item If $r=1$ and $p=0$ and the sequence of break points that defines $\Xi$ is uniform, then the estimate \eqref{proj L2} is optimal.
	\item If $r=1$ and $p>0$ and the sequence of break points that defines $\Xi$ is uniform, then the estimate \eqref{proj L2} is asymptotically $\big($with respect to the dimension of the spline space $S^p_\Xi(0,T)$ $\big)$ optimal.
\end{itemize}
In general, the authors conjecture that:
\begin{center}
 for all degree $p \geq 0$ there exists a sequence of break points such that for at least an $r=1,\ldots,p+1$, the estimate \eqref{proj L2} is optimal.
 \end{center}
\end{oss}

As a consequence of Theorem \ref{sande,manni,speleers} there holds the following result, partially extending Theorem 3 of (\cite{n-width}).

\begin{theorem}
	Let $u \in H^r(0,T)$ for $r \geq 2$. For any $q=1,\ldots,r-1$ and any sequence of break points that defines the open knot vector $\Xi$ with maximal regularity, let $h$ denote its maximal knot distance, and let $Q^q_p$ be the projection onto $S^p_\Xi(0,T)$ defined in \eqref{Qpq}. Then,
	\begin{equation}\label{err spline}
	\big |u-Q^q_pu \big |_{H^q(0,T)}\leq \Big(\frac{h}{\pi}\Big)^{r-q}\big |u \big |_{H^r(0,T)},
	\end{equation}
	for all $p \geq r-1$.
\end{theorem}

\begin{proof}
Firstly, as a consequence of the Fundamental Theorem of Calculus for absolutely continuous functions, observe that the space $H^r(0,T)$, with $r \geq 1$, satisfies
\begin{equation}\label{Hr}
H^r(0,T)=\mathbb{P}_0+K\big(H^{r-1}(0,T)\big)=\ldots=\mathbb{P}_{r-1}+K^r\big(H^0(0,T)\big),
\end{equation}
where $\mathbb{P}_{r-1}$ is the space of polynomials of degree at most $r-1$ and $K$ is the integral operator defined in \eqref{K}.

From \eqref{Hr} we know that $u \in H^r(0,T)$ can be written as $u=g+K^qv$, for $g \in \mathbb{P}_{q-1} \subset S^{p}_\Xi$, with $q \geq 1$, and $v \in H^{r-q}(0,T)$. Since the projection operator $Q^q_p: H^q(0,T) \rightarrow S^p_\Xi(0,T)$ is surjective and $\partial^q Q^q_p = Q^{q-q}_{p-q}\partial^q=P_{p-q}\partial^q$, as a consequence of Theorem \ref{sande,manni,speleers} there hold the following relations
\begin{equation*}
\begin{split}
\big\|\partial^q\big(u-Q^q_pu \big) \big \|_{L^2(0,T)}&=\big\| v-P_{p-q}v \big \|_{L^2(0,T)} \\
&\overset{(\text{if} \ p-q \geq r-q-1 \geq 0)}\leq \Big(\frac{h}{\pi}\Big)^{r-q} \big \| \partial^{r-q}v \big \|_{L^2(0,T)}=\big \|\partial^r u \big \|_{L^2(0,T)}, 
\end{split}
\end{equation*}
$q \leq r-1$ and $p \geq r-1$. Since $q = 1, \ldots, r-1$, the last inequality holds for $r \geq 2$.
\end{proof}


\begin{oss}
	The density of the family of spline spaces $(V^h_{0,*})_h$ in $H^1_{0,*}(0,T)$ is a consequence of the result \eqref{err spline} with $p=r=2$, $q=1$, and of the density of $C^\infty_c(0,T]$ in $H^1_{0,*}(0,T)$ observed in Section \ref{sec sobolev}. 
	\begin{proof}
		Let $u \in H^1_{0,*}(0,T)$ and let $\epsilon >0$ be fixed. As a consequence of the density of $C^\infty_c(0,T]$ in $H^1_{0,*}(0,T)$, there exists $\phi \in C^\infty_c(0,T]$ such that\\ $|u-\phi|_{H^1(0,T)} \leq \frac{\epsilon}{2}$. As a consequence of \eqref{err spline}, there exists $\overline{h}>0$ such that \\ $\big |\phi-Q^1_2\phi \big |_{H^1(0,T)} \leq \frac{\epsilon}{2}$ for all $h \leq \overline{h}$. We then obtain:
		\begin{equation*}
		\begin{split}
		\inf_{v_h \in V^h_{0,*}}|u-v_h|_{H^1(0,T)} &\leq |u-Q^1_2\phi \big |_{H^1(0,T)} \\
		&\leq |u-\phi|_{H^1(0,T)} + |\phi-Q^1_2\phi \big |_{H^1(0,T)} \leq \epsilon \quad \forall h \leq \overline{h}.
		\end{split}
		\end{equation*}
	\end{proof}
\end{oss}

Let us recall that we have modified the projection operator of (\cite{n-width}) in order to get a projection operator whose restriction to $H^1_{0,*}(0,T) \cap H^q(0,T)$ ($q \geq 1$) assumes values in $V^h_{0,*}$ defined in \eqref{iga space}.

\subsection{Bound on the mesh-size, stability and quasi-optimality constants}
In this Section we give two results of conditioned stability \textit{with an explicit bound on the mesh-size} and we also make explicit the stability and quasi-optimality constants.
\subsubsection{Extension to quadratic IGA with maximal regularity of conditioned stability for piecewise continuous linear FEM}
A first result is an extension to the IGA discretization of Theorem 4.7 of (\cite{Coercive}).

\begin{theorem}\label{teo stab IGA zank}
	Let 
	\begin{equation}\label{h bound}
	h \leq \frac{\pi^2}{\sqrt{2}(2+\sqrt{\mu}T)\mu T}
	\end{equation}
	be satisfied. Then, the bilinear form $a(\cdot,\cdot)$ as defined in \eqref{bil ode} satisfies the inf-sup condition
	\begin{equation}\label{infsup zank}
	\frac{2 \pi^2}{(2+\sqrt{\mu}T)^2(\pi^2+4\mu T^2)} |u_h|_{H^1(0,T)} \leq \sup_{0 \neq v_h \in V^h_{0,*}} \frac{a(u_h,\overline{\mathcal{H}}_T v_h)}{|v_h|_{H^1(0,T)}},
	\end{equation}
	for all $u_h \in V^h_{0,*}$.
\end{theorem}

\begin{proof}
	Let $u_h \in V^h_{0,*}$. As a consequence of Lax-Milgram Lemma \ref{lax-milgram}, let $w \in H^1_{0,*}(0,T)$ be the unique solution of the variational problem
	\begin{equation}\label{1 var}
	-\int_0^T \partial_t w(t) \partial_t(\overline{\mathcal{H}}_T v)(t) \ dt = -\mu \int_0^T u_h(t)(\overline{\mathcal{H}}_T v)(t) \ dt  \quad \forall v \in H^1_{0,*}(0,T).
	\end{equation}
	Hence, recalling that $(\overline{\mathcal{H}}_T v)(t)=v(T)-v(t)$, 
	\begin{equation}\label{auh-w}
	\begin{split}
		a(u_h,\overline{\mathcal{H}}_T(u_h-w))&=-\int_0^T \partial_t u_h(t) \partial_t[(\overline{\mathcal{H}}_T u_h)(t)-(\overline{\mathcal{H}}_T w)(t)] \ dt \\
		 & \quad + \mu \int_0^T u_h(t) [(\overline{\mathcal{H}}_T u_h)(t)-(\overline{\mathcal{H}}_T w)(t)] \ dt\\
	&\overset{\eqref{1 var}}= \int_0^T \partial_t u_h(t) \partial_t[u_h(t)- w(t)] \ dt - \int_0^T \partial_t w(t) [\partial_t u_h(t)-\partial_t w(t)] \ dt\\
	&= \int_0^T [\partial_t u_h(t)-\partial_t w(t)]^2 \ dt=|u_h-w|_{H^1(0,T)}^2.
	\end{split}
	\end{equation}
	Also, thanks to Lax-Milgram Lemma \ref{lax-milgram}, let $z \in H^1_{0,*}(0,T)$ be the unique solution of the following variational problem
	\begin{equation}\label{2 var}
	\begin{split}
	-\int_0^T \partial_t z(t) \partial_t (\overline{\mathcal{H}}_T v)(t) \ dt = -\int_0^T &\partial_t u_h(t) \partial_t(\overline{\mathcal{H}}_T v)(t) \ dt \\
	& + \mu \int_0^T u_h(t)(\overline{\mathcal{H}}_T v)(t)\ dt \quad \forall v \in H^1_{0,*}(0,T)
	\end{split}
	\end{equation}
	Since $w \in H^1_{0,*}(0,T)$ is the solution of \eqref{1 var}, problem \eqref{2 var} is equivalent to
	\begin{equation*}
	-\int_0^T \partial_t [z(t)-(u_h(t)-w(t))] \partial_t(\overline{\mathcal{H}}_T v)(t) \ dt =0 \quad \forall v \in H^1_{0,*}(0,T),
	\end{equation*}
	from which we conclude, by choosing $v=z-(u_h-w) \in H^1_{0,*}(0,T)$, that
	$z(t)=u_h(t)-w(t)$ for all $t \in [0,T]$, since $u_h(0)=w(0)=z(0)=0$. Therefore, from \eqref{auh-w}, we conclude that
	\begin{equation}\label{a=z}
	a(u_h,\overline{\mathcal{H}}_T(u_h-w))=|z|^2_{H^1(0,T)}.
	\end{equation}
	On the other hand, the variational formulation \eqref{2 var} gives
	\begin{equation*}
	\begin{split}
	|z|_{H^1(0,T)} &\overset{\eqref{2 var}}= \frac{a(u_h,\overline{\mathcal{H}}_T z)}{|z|_{H^1(0,T)} } \leq \sup_{0 \neq v \in H^1_{0,*}(0,T)} \frac{a(u_h,\overline{\mathcal{H}}_T v)}{|v|_{H^1(0,T)}} \\
	&\overset{\eqref{2 var}} = \sup_{0 \neq v \in H^1_{0,*}(0,T)} \frac{ \langle \partial_tz,\partial_tv \rangle _{L^2(0,T)}}{|v|_{H^1(0,T)}} \overset{(C-S)}\leq |z|_{H^1(0,T)}.
	\end{split}
	\end{equation*}
	Hence, by using \eqref{stab ode}, there hold the following
	\begin{equation*}
	|z|_{H^1(0,T)}= \sup_{0 \neq v \in H^1_{0,*}(0,T)} \frac{a(u_h,\overline{\mathcal{H}}_T v)}{|v|_{H^1(0,T)}} \overset{\eqref{stab ode}}\geq \frac{2}{2+\sqrt{\mu}T} |u_h|_{H^1(0,T)},
	\end{equation*}
	from which, recalling \eqref{a=z}, we conclude
	\begin{equation}\label{magg a}
	a(u_h,\overline{\mathcal{H}}_T(u_h-w)) \overset{\eqref{a=z}}= |z|^2_{H^1(0,T)} \geq \frac{4}{(2+\sqrt{\mu}T)^2}|u_h|^2_{H^1(0,T)}.
	\end{equation}
	We now discretize the first variational formulation \eqref{1 var}. Let $w_h \in V^h_{0,*}$ be the unique solution of the following variational problem
	\begin{equation}\label{disc dim}
	\int_0^T \partial_t w_h(t) \partial_t v_h(t) dt = -\mu \int_0^T u_h(t) (\overline{\mathcal{H}}_T v_h)(t) \ dt \quad \forall v_h \in V^h_{0,*}.
	\end{equation}
	By using Céa's Lemma \ref{céa} and the error estimate \eqref{err spline} with $r=2$, $p=2$, $q=1$, there hold the following relations
	\begin{equation}\label{w-wh}
	\begin{split}
	|w-&w_h|_{H^1(0,T)}\leq \inf_{0 \neq v_h \in V^h_{0,*}}|w-v_h|_{H^1(0,T)} \quad \text{(Céa)}\\
	& \leq |w-Q^1_2 w|_{H^1(0,T)} \overset{\eqref{err spline}}\leq \frac{h}{\pi}\|\partial_{tt}w\|_{L^2(0,T)} \overset{\eqref{disc dim}}= \mu \frac{h}{\pi}\|u_h\|_{L^2(0,T)}.
	\end{split}
	\end{equation}
	Furthermore, Galerkin orthogonality \eqref{gal ort} is satisfied
	\begin{equation}\label{gal ort dim}
	\int_0^T[\partial_t w(t)-\partial_t w_h(t)] \partial_t v_h(t) \ dt =0 \quad \forall v_h \in V^h_{0,*}.
	\end{equation}
	As a consequence of \eqref{gal ort dim}, we have
	\begin{equation}\label{magg uh}
	\begin{split}
	a(u_h,\overline{\mathcal{H}}_T(w-w_h))&=\int_0^T \partial_t u_h(t)\partial_t[w(t)-w_h(t)] \ dt + \mu \int_0^T u_h(t)[\overline{\mathcal{H}}_T(w-w_h)](t) \ dt \\
	&\overset{\eqref{gal ort dim}}=\mu \int_0^T u_h(t)[\overline{\mathcal{H}}_T(w-w_h)](t) \ dt \\  
	& \leq \mu \|u_h\|_{L^2(0,T)} \|\overline{\mathcal{H}}_T(w-w_h)\|_{L^2(0,T)}.
	\end{split}
	\end{equation}
	As a consequence of Lax-Milgram Lemma \ref{lax-milgram}, let now $\psi \in H^1_{0,*}(0,T)$ be the unique solution of the following variational problem
	\begin{equation}\label{3 var}
	- \int_0^T \partial_t \psi(t) \partial_t [\overline{\mathcal{H}}_Tv](t) \ dt = \int_0^T(\overline{\mathcal{H}}_T(w-w_h))(t)(\overline{\mathcal{H}}_Tv)(t) \ dt \quad \forall v \in H^1_{0,*}(0,T).
	\end{equation}
	In particular, by choosing $v=w-w_h \in H^1_{0,*}(0,T)$ and recalling \eqref{gal ort dim}, we obtain
	\begin{equation*}
	\begin{split}
	\|\overline{\mathcal{H}}_T(w-w_h)\|^2_{L^2(0,T)}&=\int_0^T[\overline{\mathcal{H}}_T(w-w_h)](t)[\overline{\mathcal{H}}_T(w-w_h)](t) \ dt \\
	&\overset{\eqref{3 var}}=-\int_0^T\partial_t \psi(t)\partial_t [\overline{\mathcal{H}}_T(w-w_h)](t) \ dt \\
	&=\int_0^T\partial_t \psi(t)[\partial_t w(t)-\partial_t w_h(t)] \ dt \\
	&\overset{\eqref{gal ort dim}}=\int_0^T \partial_t[\psi(t)-Q^1_2 \psi(t)][\partial_t w(t)-\partial_t w_h(t)] \ dt\\
	& \leq |\psi-Q^1_2 \psi|_{H^1(0,T)}|w-w_h|_{H^1(0,T)}\\
	&\overset{\eqref{err spline},\eqref{w-wh}}\leq \Big(\frac{h}{\pi} \Big)^2\mu\|\partial_{tt} \psi \|_{L^2(0,T)}\|u_h\|_{L^2(0,T)}\\
	&\overset{\eqref{3 var}}= \Big(\frac{h}{\pi} \Big)^2\mu \|\overline{\mathcal{H}}_T(w-w_h)\|_{L^2(0,T)} \|u_h\|_{L^2(0,T)},
	\end{split}
	\end{equation*}
	i.e.,
	\begin{equation*}
	\|\overline{\mathcal{H}}_T(w-w_h)\|_{L^2(0,T)}\leq \frac{h^2}{\pi^2}\mu\|u_h\|_{L^2(0,T)}.
	\end{equation*}
	Therefore, by using \eqref{magg uh} and Poincaré inequality \eqref{Poinc},
	\begin{equation}\label{stim}
     a(u_h,\overline{\mathcal{H}}_T(w-w_h)) \overset{\eqref{magg uh}}\leq \Big(\frac{h}{\pi} \mu \Big)^2  \|u_h\|^2_{L^2(0,T)} \overset{\eqref{Poinc}}\leq \Big(\frac{2 T}{\pi^2} \mu h \Big)^2|u_h|_{H^1(0,T)}^2
  	\end{equation}
  	follows. Hence, as a consequence of \eqref{magg a} and \eqref{stim} we conclude
  	\begin{equation*}
  	\begin{split}
  	a(u_h,\overline{\mathcal{H}}_T(u_h-w_h))&=a(u_h,\overline{\mathcal{H}}_T(u_h-w))+a(u_h,\overline{\mathcal{H}}_T(w-w_h))\\
  	& \overset{\eqref{magg a},\eqref{stim}}\geq \Bigg[ \frac{4}{(2+\sqrt{\mu}T)^2} - \Big(\frac{2 T}{\pi^2} \mu h \Big)^2 \Bigg] |u_h|^2_{H^1(0,T)}\\
  	& \geq \frac{2}{(2+\sqrt{\mu}T)^2} |u_h|^2_{H^1(0,T)},
  	\end{split}
  	\end{equation*}
  	if
  	\begin{equation*}
  	\Big(\frac{2 T}{\pi^2} \mu h \Big)^2 \leq \frac{2}{(2+\sqrt{\mu}T)^2}
  	\end{equation*}
  	is satisfied, i.e.,
  	\begin{equation*}
  	h \leq \frac{\pi^2}{\sqrt{2}(2+\sqrt{\mu}T)\mu T}.
  	\end{equation*}
  	We now consider a lower bound for $|u_h|_{H^1(0,T)}$ dependent of $|u_h-w_h|_{H^1(0,T)}$. Indeed, we have
  	\begin{equation*}
  	\|\partial_t(u_h-w_h)\|_{L^2(0,T)}\leq \|\partial_t u_h\|_{L^2(0,T)}+\|\partial_t w_h\|_{L^2(0,T)},
  	\end{equation*}
  	and, thanks to \eqref{disc dim} and Poincaré inequality \eqref{Poinc},
  	\begin{equation*}
  	\begin{split}
  	\|\partial_t w_h\|^2_{L^2(0,T)}&=-\int_0^T \partial_t w_h(t) \partial_t (\overline{\mathcal{H}}_T w_h)(t) \ dt \overset{\eqref{disc dim}}= -\mu \int_0^T u_h(t)(\overline{\mathcal{H}}_T w_h)(t) \ dt \\
  	&  \leq \mu \|u_h\|_{L^2(0,T)} \|\overline{\mathcal{H}}_T w_h \|_{L^2(0,T)} \overset{\eqref{Poinc}} \leq \frac{4 T^2}{\pi^2}\mu |u_h|_{H^1(0,T)}|w_h|_{H^1(0,T)},
  	\end{split}
  	\end{equation*}
  	i.e.,
  	\begin{equation*}
  	|u_h-w_h|_{H^1(0,T)} \leq \Big(1+\frac{4 T^2}{\pi^2}\mu \Big) |u_h|_{H^1(0,T)}.
  	\end{equation*}
  	Therefore, we obtain the following inequality
  	\begin{equation*}
  	\frac{2 \pi^2}{(2+\sqrt{\mu}T)^2(\pi^2+4\mu T^2)} |u_h|_{H^1(0,T)} \leq \frac{a(u_h,\overline{\mathcal{H}}_T(u_h-w_h))}{|u_h-w_h|_{H^1(0,T)}}.
  	\end{equation*}
\end{proof}

Hence, we obtain the following result.

\begin{theorem}
	Let \eqref{h bound} be satisfied. Then, problem \eqref{iga ode equiv} is well-posed with the stability estimate
	\begin{equation}\label{stab zank}
	|u_h|_{H^1(0,T)} \leq \frac{(2+\sqrt{\mu}T)^2(\pi^2+4\mu T^2)}{2\pi^2}\|f\|_{[H^1_{*,0}(0,T)]'},
	\end{equation}
	where $f \in [H^1_{*,0}(0,T)]'$ and $u_h$ is the unique solution of \eqref{iga ode equiv}. Moreover, a quasi-optimality estimate holds
	\begin{equation}\label{quasi opt zank}
	|u-u_h|_{H^1(0,T)} \leq \Bigg[ 1 + \frac{(2+\sqrt{\mu} T)^2(\pi^2+4 \mu T^2)^2}{2 \pi^4} \Bigg] \inf_{v_h \in V^h_{0,*}} |u-v_h|_{H^1(0,T)},
	\end{equation}
	where $u$ is the unique solution of \eqref{var ode equiv}.
\end{theorem}

\begin{proof}
	The well-posedness with stability estimate \eqref{stab zank} is an immediate consequence of BNB Theorem \ref{BNB}.
	
	In order to prove \eqref{quasi opt zank}, we repeat, with explicit constants, the proof of Proposition \ref{err BNB}. For any $w \in H^1_{0,*}(0,T)$ we define $w_h:= G_h w$ as the Galerkin projection satisfying 
	\begin{equation*}
	a(G_hw,\overline{\mathcal{H}}_T v_h)=a(w,\overline{\mathcal{H}}_T v_h) \quad \forall v_h \in V^h_{0,*},
	\end{equation*}
	which is well defined thanks to the well-posedness of the discrete problem. Hence, by using the stability estimate \eqref{infsup zank} and the continuity of $a(\cdot,\cdot)$ \eqref{cont of a}, there hold
	\begin{equation*}
	|G_h w|_{H^1(0,T)} \leq \frac{(2+\sqrt{\mu}T)^2(\pi^2+4 \mu T^2)^2}{2 \pi^4} |w|_{H^1(0,T)}.
	\end{equation*}
	Indeed,
	\begin{equation*}
	\begin{split}
		\frac{2 \pi^2}{(2+\sqrt{\mu}T)^2(\pi^2+4\mu T^2)} &|G_hw|_{H^1(0,T)}  
		\overset{\eqref{infsup zank}}\leq \sup_{0 \neq v_h \in V^h_{0,*}} \frac{a(G_hw,\overline{\mathcal{H}}_T v_h)}{|v_h|_{H^1(0,T)}}\\
		&=\sup_{0 \neq v_h \in V^h_{0,*}} \frac{a(w,\overline{\mathcal{H}}_T v_h)}{|v_h|_{H^1(0,T)}}\\
		& \overset{\eqref{cont of a}}\leq  \Bigg(1+\frac{4 T^2 \mu }{\pi^2} \Bigg) |w|_{H^1(0,T)}.
	\end{split}
	\end{equation*}
	Since $u_h=G_h u$ and $v_h = G_h v_h$ for all $v_h \in V^h_{0,*}$, we conclude
	\begin{equation*}
	\begin{split}
	|u-u_h|_{H^1(0,T)} &\leq |u-v_h|_{H^1(0,T)}+|G_h(v_h-u)|_{H^1(0,T)} \\
	& \leq \Bigg[ 1 + \frac{(2+\sqrt{\mu} T)^2(\pi^2+4 \mu T^2)^2}{2 \pi^4} \Bigg] |u-v_h|_{H^1(0,T)}.
	\end{split}
	\end{equation*}
\end{proof}

Thus, we are in a position to state a convergence result for the isogeometric solution $u_h$ of the variational formulation \eqref{var ode equiv}.

\begin{cor}
	Let $u \in H^1_{0,*}(0,T)$ and $u_h \in V^h_{0,*}$ be the unique solutions of the variational formulations \eqref{var ode equiv} and \eqref{iga ode equiv}, respectively. Let  $u \in H^3(0,T)$ and \eqref{h bound} be satisfied. Then, there holds true the error estimate
	\begin{equation}\label{ord conv iga zank}
	|u-u_h|_{H^1(0,T)} \leq \frac{1}{\pi^2} \Bigg[ 1 + \frac{(2+\sqrt{\mu} T)^2(\pi^2+4 \mu T^2)^2}{2 \pi^4} \Bigg] h^2 |u|_{H^3(0,T)}.
	\end{equation}
\end{cor}

\begin{proof}
	Estimate \eqref{ord conv iga zank} is a straightforward consequence of quasi-optimality \eqref{quasi opt zank} and of \eqref{err spline} with $r=3,p=2,q=1$.
\end{proof}

\begin{oss}
	Note that the bound on the mesh-size \eqref{h bound} is about $2.015$ times the bound 
	\begin{equation*}
	h \leq \frac{2 \sqrt{3}}{(2+\sqrt{\mu}T)\mu T}
	\end{equation*}
	of (\cite{Steinbach2019}). It's also about $1.81$ times the more accurate bound 
	\begin{equation*}
	h \leq \frac{\sqrt{3}\pi}{\sqrt{2}(2+\mu T)\mu T}
	\end{equation*}	
	of (\cite{Zank2020}). Therefore, in order to get stability and convergence, our mesh-size can be approximately twice the mesh-size of piece-wise linear FEM.
\end{oss}

\subsubsection{Application of Theorem \ref{theo gard} to quadratic IGA with maximal regularity}
Another way to get a bound on the mesh-size, so that, if it is respected, the well-posedness of IGA, stability and convergence (with explicit constants) are guaranteed, is the theory of \textit{Galerkin method applied to G\aa rding-type problems} discussed in Section \ref{sec prel ode}. 

Let now consider problem \eqref{var ode equiv} and its isogeometric discretization \eqref{iga ode equiv}. Actually, these problems lie within the framework of Theorem \ref{theo gard}. Indeed, as a consequence of Rellich-Kondrachov Theorem, the inclusion $H^1_{0,*}(0,T) \subset L^2(0,T)$ is compact. Furthermore, the following result holds.
\begin{lemma}
	Let $b > \frac{\mu T^2}{2}$. The bilinear form defined in \eqref{bil ode} satisfies the G\aa rding inequality:
	\begin{equation}\label{gard ineq iga}
	a(v,\overline{\mathcal{H}}_T v) \geq \Big(1- \frac{\mu T^2}{2b}\Big)|v|_{H^1(0,T)}-\frac{2+b}{2}\mu \|v\|_{L^2(0,T)} \quad \forall v \in H^1_{0,*}(0,T).
	\end{equation}
\end{lemma}
\begin{proof}
	The bilinear form defined in \eqref{bil ode} satisfies the following inequalities
	\begin{equation*}
	\begin{split}
	a(v,\overline{\mathcal{H}}_T v)&= \langle \partial_t v, \partial_t v \rangle _{L^2(0,T)}- \mu \langle v,v \rangle_{L^2(0,T)}+\mu \langle v,v(T) \rangle_{L^2(0,T)} \\
	& \geq |v|^2_{H^1(0,T)}-\mu \|v\|^2_{L^2(0,T)}-\mu \int_0^T |v(t)v(T)| \ dt \\
	& \overset{(C-S)}\geq |v|^2_{H^1(0,T)}-\mu \|v\|^2_{L^2(0,T)} - \mu \|v\|_{L^2(0,T)}\sqrt{T}|v(T)| \\
	& \overset{(F.T.C),(Young)}\geq |v|^2_{H^1(0,T)}-\mu \|v\|^2_{L^2(0,T)} - \frac{\mu b}{2} \|v\|^2_{L^2(0,T)}-\frac{\mu T^2}{2b} |v|^2_{H^1(0,T)}\\
	& = \Big(1- \frac{\mu T^2}{2b}\Big)|v|^2_{H^1(0,T)}-\frac{2+b}{2}\mu \|v\|^2_{L^2(0,T)} ,
	\end{split}
	\end{equation*}
	for all $b >0$, where we used Cauchy Schwarz inequality, Young inequality and the Fundamental Theorem of Calculus. In order to obtain positive coefficients in \eqref{gard ineq iga} we add the constraint $b > \frac{\mu T^2}{2}$.
\end{proof}	
	
Also, Theorem \ref{zank} guarantees that problem \eqref{var ode equiv} is well-posed, then, the only $u_0 \in H^1_{0,*}(0,T)$ such that $a(u_0,\overline{\mathcal{H}}_T v)=0$ for all $v \in H^1_{0,*}(0,T)$ is $u_0=0$. Therefore, we conclude the following result.

\begin{theorem}\label{theo gard iga}
	Let 
	\begin{equation}\label{h bound gard}
	h \leq \frac{\pi^5}{(\pi^2+ 4\mu T^2)[\pi^2+2\mu T^2 (2+\sqrt{\mu} T)]} \sqrt{\frac{2b - \mu T^2}{2b(2+b)\mu}}
	\end{equation}
	be satisfied, with $b > \frac{\mu T^2}{2}$. 
	Then, problem \eqref{iga ode equiv} is well-posed with the stability estimate
	\begin{equation}\label{stab iga gard}
	\|u_h\|_{H^1(0,T)} \leq (2+ \sqrt{\mu} T ) \Bigg[ \frac{3b + \mu T^2\big(\frac{8b}{\pi^2}-\frac{1}{2} \big)}{2b-\mu T^2} \Bigg] \|f\|_{[H^1_{*,0}(0,T)]'},
	\end{equation}
	where $f \in [H^1_{*,0}(0,T)]' $ and $u_h$ is the unique solution of \eqref{iga ode equiv}. Moreover, a quasi-optimality estimate holds
	\begin{equation}\label{opt iga gard}
	|u-u_h|_{H^1(0,T)} \leq \frac{4b}{\pi^2}\frac{ \pi^2+ 4 \mu T^2}{2b-\mu T^2} \inf_{v_h \in V^h_{0,*}} |u-v_h|_{H^1(0,T)},
	\end{equation}
	where $u \in H^1_{0,*}(0,T)$ is the unique solution of \eqref{var ode equiv}.
\end{theorem}

\begin{proof}
	Let $b > \frac{\mu T^2}{2}$ and let $\eta(V^h_{0,*})$ be the parameter defined in \eqref{eta gard} and related to the sequence of isogeometric spaces in \eqref{iga space}. By using the projection operator \eqref{Qpq} with $p=2$, $q=1$, the error estimate \eqref{err spline} with $r=2$, the Poincaré inequality \eqref{Poinc} and the a priori estimate of the abstract problem \eqref{stab ode}, we obtain
	\begin{equation}\label{eta iga}
	\eta(V^h_{0,*}) \leq \frac{h}{\pi} \Bigg[ 1 + \frac{2\mu T^2}{\pi^2}(2+\sqrt{\mu} T) \Bigg].
	\end{equation}
    Let see why \eqref{eta iga} holds. Let $z_g \in H^1_{0,*}(0,T)$ be the solution of the adjoint problem \eqref{adj} with respect to our context, i.e., given $g \in L^2(0,T)$, $z_g$ is the unique element of $H^1_{0,*}(0,T)$ that satisfies
	\begin{equation*}
	a(v,\overline{\mathcal{H}}_T z_g)=(g,v)_{L^2(0,T)} \quad \forall v \in H^1_{0,*}(0,T).
	\end{equation*}
	Therefore, $z:=\overline{\mathcal{H}}_Tz_g \in H^1_{*,0}(0,T)$ is the unique solution of
	\begin{equation*}
	a(v,z)=(g,v)_{L^2(0,T)} \quad \forall v \in H^1_{0,*}(0,T),
	\end{equation*}
	i.e.,
	\begin{equation*}
	-\langle \partial_t v,\partial_t z \rangle_{L^2(0,T)} + \mu \langle v,z \rangle_{L^2(0,T)} = (g,v)_{L^2(0,T)} \quad \forall v \in H^1_{0,*}(0,T).
	\end{equation*}
	As a consequence, the distributional derivative $\partial_{tt} z$ is represented by $g-\mu z \in L^2(0,T)$. Hence, $z \in H^2(0,T)$ and $z_g = \overline{\mathcal{H}}_T^{-1} z = z(0)-z \in H^2(0,T)$ with 
	\begin{equation*}
	\partial_{tt} z_g = \mu \overline{\mathcal{H}}_T z_g - g.
	\end{equation*}
	Note that the adjoint problem has the same stability estimate (w.r.t. the dual norm $\|g\|_{[H^1_{0,*}(0,T)]'}$) of the primal problem, as noted in the second point of Remark \ref{oss gard}. Then, the following relations hold
	\begin{equation*}
	\begin{split}
\inf_{v_h \in V^h_{0,*}} |z_g-v_h|_{H^1(0,T)} &\leq |z_g - Q^1_2 z_g|_{H^1(0,T)} \overset{\eqref{err spline}}\leq \frac{h}{\pi} \|\partial_{tt} z_g\|_{L^2(0,T)}\\
&=\frac{h}{\pi}\|\mu \overline{\mathcal{H}}_T z_g-g\|_{L^2(0,T)}\\
& \leq \frac{h}{\pi} \Big(\|g\|_{L^2(0,T)} + \mu \|\overline{\mathcal{H}}_T z_g\|_{L^2(0,T)} \Big)\\
&\overset{\eqref{Poinc}}\leq \frac{h}{\pi} \Big(\|g\|_{L^2(0,T)} + \frac{2T}{\pi}\mu |z_g|_{H^1(0,T)} \Big)\\
&\overset{\eqref{stab ode}}\leq \frac{h}{\pi} \Big(\|g\|_{L^2(0,T)} + \frac{4T^2}{\pi^2}\mu \frac{2 + \sqrt{\mu} T}{2} \|g\|_{L^2(0,T)} \Big)\\
&=\frac{h}{\pi} \Big[ 1 + \frac{2\mu T^2}{\pi^2}(2+\sqrt{\mu}T) \Big] \|g\|_{L^2(0,T)},
	\end{split}
	\end{equation*}
	which implies estimate \eqref{eta iga}. Therefore, from condition \eqref{cond eta} where \eqref{gard ineq iga} is the G\aa rding inequality of our problem \eqref{iga ode equiv}, we deduce that, if 
	\begin{equation*}
	\frac{h}{\pi} \Bigg[ 1 + \frac{2\mu T^2}{\pi^2}(2+\sqrt{\mu}T) \Bigg] \leq \frac{\pi^2}{\pi^2+4\mu T^2} \sqrt{\frac{2b - \mu T^2}{2b(2+b)\mu}},
	\end{equation*}
i.e.,
\begin{equation*}
h \leq \frac{\pi^5}{(\pi^2+4 \mu T^2)[\pi^2+2\mu T^2 (2+\sqrt{\mu} T)]} \sqrt{\frac{2b - \mu T^2}{2b(2+b)\mu}},
\end{equation*}	
then problem \eqref{iga ode equiv} is well-posed, and conditions \eqref{stab iga gard}, \eqref{opt iga gard} are satisfied. 
\end{proof}

\begin{oss}
	The choice
	\begin{equation}\label{b}
	b=\frac{\mu T^2 + \sqrt{\mu^2 T^4+ 4 \mu T^2}}{2}
	\end{equation}
	maximises $\sqrt{\frac{2b - \mu T^2}{2b(2+b)\mu}}$ in the upper bound of \eqref{h bound gard}.
\end{oss}

As a consequence of Theorem \ref{theo gard iga}, we can state a convergence result for the isogeometric solution $u_h$ of the variational formulation \eqref{var ode equiv}.

\begin{cor}\label{cor gard}
Let $u \in H^1_{0,*}(0,T)$ and $u_h \in V^h_{0,*}$ be the unique solutions of the variational formulations \eqref{var ode equiv} and \eqref{iga ode equiv}, respectively. Let $u \in H^3(0,T)$ and \eqref{h bound gard} be satisfied. Then, there holds true the error estimate
\begin{equation}\label{ord conv iga gard}
|u-u_h|_{H^1(0,T)} \leq \frac{4b}{\pi^4}\frac{ \pi^2+ 4 \mu T^2}{2b-\mu T^2} h^2 |u|_{H^3(0,T)}.
\end{equation}
\end{cor}

\begin{proof}
	Estimate \eqref{ord conv iga gard} is a straightforward consequence of quasi-optimality \eqref{opt iga gard} and of \eqref{err spline} with $r=3,p=2,q=1$.
\end{proof}

\begin{oss}\label{oss grado più alto}
The analysis proposed in this Section, which is based on techniques that exploit the G\aa rding inequality and are alternative to those proposed by O. Steinbach and M. Zank (extended to the IGA case in the previous Section) is useful to understand how the IGA behaves in the case of generic polynomial degree and maximal regularity. Indeed, from the proof of Theorem \ref{theo gard iga} it emerges that the threshold on $h$ \eqref{h bound gard} does not change when the polynomial degree and the regularity of the spline test and trial functions are raised. Only the order of convergence of Corollary \ref{h bound gard} changes: for a generic polynomial degree $p \geq 1$ and an exact solution $u \in H^{p+1}(0,T)$, the order of convergence is $p$. 
On the other hand, how the upper bound \eqref{h bound} behaves in $h$ is not immediately clear from the proof of Theorem \ref{teo stab IGA zank}. This is a consequence of the various auxiliary problems considered, in which the regularity of the corresponding solutions is a key point. 
\end{oss}	

\begin{oss}\label{oss asympt}
	Asymptotically for $\mu \rightarrow \infty$ the optimal value \eqref{b} satisfies $b\simeq\mu T^2$. We then obtain
	\begin{equation*}
	h \leq \overline{h}_2 :=\frac{\pi^5}{(\pi^2+ 4\mu T^2)[\pi^2+2\mu T^2 (2+\sqrt{\mu} T)]} \sqrt{\frac{1}{2\mu(2+\mu T^2)}},
	\end{equation*}
	in place of \eqref{h bound gard} with a general value $b > \frac{\mu T^2}{2}$. 
	
	Let us now recall estimate \eqref{h bound} obtained by extending Theorem $4.7$ of (\cite{Coercive}) to quadratic IGA with maximal regularity, i.e.,
	\begin{equation*}
	h \leq \overline{h}_1 :=\frac{\pi^2}{\sqrt{2}(2+\sqrt{\mu}T)\mu T}.
	\end{equation*}
	Therefore, the techniques of O. Steinbach and M. Zank and those ones using the G\aa rding inequality \eqref{gard} give constraints on $h$ of order
	\begin{gather*}
	\overline{h}_1 \simeq \mathcal{O}(\mu^{-3/2}),\\
	\overline{h}_2 \simeq \mathcal{O}(\mu^{-7/2}),
	\end{gather*}
respectively. Thus, we conclude that, asymptotically, threshold \eqref{h bound gard} is a stronger constraint than \eqref{h bound}. However, asymptotically, stability estimate \eqref{stab iga gard} behaves as $\mathcal{O}(\mu^{3/2})$, whereas, stability estimate \eqref{stab zank} behaves as $\mathcal{O}(\mu^{2})$. Therefore, we conclude that stability estimate \eqref{stab iga gard} has a slower growth than \eqref{stab zank} for $\mu \rightarrow \infty$. 
\end{oss}
	
% Chapter 4

\chapter{Universes with matter fields} % Main chapter title
\label{chapter4}  
\textit{This chapter gives a brief summary of the following articles: \cite{pub4,pub5,pub6}}.\\\\

\section{Scalar fields as coordinates}
\textit{"If people do not believe that mathematics is simple, it is only because they do not realize how complicated life is."} - \textbf{Neumann János}\\\\ 

\textit{This section is based on the publications \cite{pub4,pub5}.}\\\\

Life (the actual physical phenomenon) is indeed complicated, much more complicated than any  model designed in order to describe it. However, usually  simple models are the only ones that can be solved. Also in most physical theories vacuum solutions are the simplest, easiest and first ones to be found. The same is true in GR as most of known solutions of Einstein's field equations are vacuum solutions. It is also the case of CDT, where the first  twenty years of studies  were dominated by pure gravity models  (empty Universes) discussed above. As it was mentioned in Chapters \ref{Chapter1} and \ref{chapter2}, CDT is formulated in  a coordinate-free way, except from the time direction where one has a natural global proper-time coordinate $t$, consistent with the introduced foliation. It would be therefore beneficial to introduce some notion of coordinates, making contact with other gravitational research.\\

The simplest extension of the CDT model is the addition of massless scalar fields. As will be shown, such scalar fields can also play role of "clocks" and "rods", enabling one to define a coordinate system in the triangulated manifold, being an analogue of the harmonic (de Donder) gauge fixing in GR. As we will discuss below, in order to apply this method of defining coordinates in CDT one also has to make a proper choice of the target space of the scalar fields. As an example, take a (smooth) Riemannian manifold $\mathcal{M}$ equipped with a metric tensor $g_{\mu\nu}$ and another Riemannian manifold $\mathcal{N}$ with a trivial flat metric $h_{\alpha\beta}$. A harmonic map $\mathcal{M} \to \mathcal{N}$ can be defined with the help of a four-component scalar field $\phi^\alpha$, with $\alpha = 1, 2, 3, 4$. In case of our setup, if $\mathcal{M}$ has a topology of the four-torus $T^4$, then we also choose $\mathcal{N}$ to have the same toroidal topology, and each component $\phi^\alpha(x)$ is a map $\mathcal{M} \to S^1$, which minimizes the action

\begin{equation}
\label{eq:scalar_action}
S_M[\phi]	= \frac{1}{2} \int \mathrm{d}^{4}x \sqrt{g(x)} \;g^{\mu \nu} (x) \; h_{\alpha \beta}(\phi^\gamma(x)) \;\partial_\mu \phi^\alpha(x) \partial_\nu \phi^\beta(x).
\end{equation}

Due to our   choice of  the trivial metric $h_{\alpha \beta}$ on $\mathcal{N}$, the four-component scalar field can be decomposed into four independent components, later denoted $\phi^x$, $\phi^y$, $\phi^z$, and $\phi^t$. Due to this, it is enough to discuss the case of a single component (let's call it  $\phi$). The Euler-Lagrange equations for the field resulting from eq. (\ref{eq:scalar_action}) give rise to the Laplace equation:

\begin{equation}
\Delta_x \phi (x) =0, \quad 
\Delta_x = \frac{1}{\sqrt{g(x)}} \;\frac{\partial}{\partial x^\mu} 
\Big(\sqrt{g(x)} \,
g^{\mu\nu}(x)\Big) \frac{\partial}{\partial x^\nu},
\quad \phi(x) \in S^1.
\label{eq:laplace}
\end{equation}

In the case when $\mathcal{M}$ is closed, if we chose the target space of the scalar field $\phi$ to be $\mathbb{R}$ then the constant zero-mode of the Laplacian would be the only solution to the equation $\Delta_x \phi(x) = 0$. If instead, as we do, one chooses a nontrivial target space of the field to be $S^1$ (with circumference $\delta$) then one can obtain a nontrivial solution for the scalar field. Technically, the  condition $\phi(x)\in S^1$ can be obtained by considering a scalar field with the target space $\mathbb R$ and identifying 

\begin{equation}
\phi(x) \equiv \phi(x) + n \, \delta,\quad n \in \mathbb{Z}.
\end{equation}
The situation of interest is when we have the toroidal manifold $\mathcal M$ which can be thought of as an elementary cell  periodically repeating in all four directions. In such a case one can define four non-equivalent boundaries of the elementary cell,  i.e. 3-dimensional connected hypersurfaces $H(\alpha), \alpha = \{x,y,z,t\}$. Let us consider  the case when each component of the field $\phi^\alpha \in S^1$ winds around the circle once as we go around any non-contractible loop in  $\mathcal M$ that crosses a boundary in direction $\alpha$. In that case the field $\phi$ is a continuous function  except when one crosses the hypersurface $H(\alpha)$, where a jump of the field with amplitude $\delta$ happens, and the Laplace equation (\ref{eq:laplace}) acquires a nontrivial boundary term leading to a non-trivial solution for the field $\phi$. A corresponding function that is continuous despite the jump, will be a map

\begin{equation}
\phi \to \psi = \frac{\delta}{2\pi}\;e^{2\pi i \phi/\delta},
\label{eq:psi}
\end{equation}
which maps the scalar field $\phi$ to a circle in the complex plane. The interesting point is that, for a given direction $\alpha$, the  map $\psi$ does not depend on the exact choice of the boundary $H(\alpha)$ of the elementary cell.\footnote{Formally it depends only trivially, i.e., a continuous deformation  (a "shift") of the boundary $H(\alpha)$ will only cause a shift of the phase in the complex function $\psi$ by some constant.} \\

In CDT we  consider a discretization of the action (\ref{eq:scalar_action}) and the corresponding Laplace equation (\ref{eq:laplace}), where the field is localized in the center of simplices. We therefore consider a (discretized) Laplacian defined on the dual lattice, i.e., the graph whose  vertices represent the 4-simplices of the original CDT triangulation, and links represent the common interfaces between the 4-simplices in the triangulation. The Laplacian on the dual lattice can be defined via the adjacency matrix $A_{ij}$:

\begin{equation}
A_{ij} =
\begin{cases}
	1 & \textrm{if (the link } i \leftrightarrow j) \in \textrm{dual lattice},\\
	0 & \textrm{otherwise},
\end{cases}
\end{equation}
where $\leftrightarrow$ refers to the adjacency relation of simplex $i$ and $j$. The discrete Laplacian can be defined as:

\begin{equation}
L =  D - A,
\end{equation}
where $A$ is the adjacency matrix and $D$ is a diagonal matrix with $i$-th diagonal element containing the number of neighbors of a simplex labelled $i$. As, in the four-dimensional CDT, each simplex in the  triangulation has exactly 5 neighbors, the dual lattice of any triangulation is a five-valent graph, and therefore
\begin{equation}
    D = 5 \cdot I,
\end{equation}
with $I$ being the identity matrix of size $N_4 \times N_4$, where $N_4$ is the number of all 4-simplices in the triangulation. The discretized form of the scalar field action is then given by:

\begin{equation}
\label{eq:scalar_discrete}
S_M^{CDT}[\{\phi\},\mathcal{T}]	= \frac{1}{2} \sum_{i \leftrightarrow j} (\phi_i - \phi_j)^2 = \sum_{i,j} \phi_i L_{ij} \phi_j \equiv \phi^T L \phi, 
\end{equation}
where $\mathcal{T}$ underlines the impact of the triangulation on the Laplacian matrix $L$. The discrete analog of the Laplace eq. (\ref{eq:laplace}) is then: 

\begin{equation}
    L\phi = 0.
\end{equation}
The above equation has the same issue as before, i.e., if the target space of the field was chosen to be $\mathbb R$ then it would only have a trivial solution  $\phi = const$. Non-trivial solutions can be  found by choosing the field to take values in $S^1$ with circumference $\delta$, which winds around the circle once as one goes around any non-contractible loop in the dual lattice. In order to do  that one  identifies:

\begin{equation}
\phi_i \equiv \phi_i + n \cdot \delta, 
\quad n \in \mathbb{Z}. %\quad \forall i \in {\cal T},
\end{equation}
This can be achieved by adding a jump condition when crossing a boundary  hyper-surface $H(\alpha)$ in direction, $\alpha$. The way of  introducing such boundary hypersurfaces to CDT was proposed in \cite{boundaries}. As already mentioned, the exact position of the (four non-equivalent) boundaries $H(\alpha)$ in the triangulation is not important as it has only a trivial impact on our solutions, thus the boundaries are non-physical. Technically, one can define the "jump" condition by introducing the boundary jump matrix $B_{ij}$:

\begin{equation}
B_{ij} =
\begin{cases}
	+1 & \textrm{if the dual link i} \rightarrow \textrm{j crosses the boundary $H(\alpha)$ in the \textit{positive} direction},\\
	-1 & \textrm{if the dual link i} \rightarrow \textrm{j crosses the boundary $H(\alpha)$ in the \textit{negative} direction},\\
	0 & \textrm{otherwise}
\end{cases}
\end{equation}
and defining
\begin{equation}
V = \frac{1}{2}  \sum_{ij} B_{ij}^2= \frac{1}{2} \sum_i | b_i |, 
\end{equation}
where $b_i = \sum_j B_{ij}$ is the boundary jump vector, and it measures the occasions when a tetrahedral face of a simplex $i$ appears on the boundary. To accommodate to the jump condition we modify the discretized matter action:

\begin{multline}
\label{eq:scalar_w_boundary}
S_M^{CDT}[\{\phi\}, \mathcal{T}]= \frac{1}{2} \sum_{i \leftrightarrow j} (\phi_i - \phi_j - \delta B_{ij})^2 = \sum_{i,j} \phi_i L_{ij} \phi_j - 2 \delta \sum_i \phi_i b_i +\delta^2 V \\ \equiv \phi^T L \phi - 2 \delta \phi^T b + \delta^2 V.
\end{multline}
Now, the Euler-Lagrange equation for the field $\phi$ yields:

\begin{equation}
 L \phi	= \delta \, b,
\end{equation}
so it acquires a non-trivial boundary term: $\delta \ b$.
The classical solution to the scalar field distribution  is formally given by 

\begin{equation}
    \phi^{classical} =\delta \ L^{-1}  b.
\end{equation}
The practical problem is that the Laplacian has zero modes but, fortunately, one can find a solution in the subspace orthogonal to the zero modes. The solution strongly depends on the underlying triangulated geometry and it smoothly interpolates between the boundaries of the (toroidal) elementary cell. In the publication \cite{pub4} we proposed to treat the harmonic map $\phi^{classical}$, or rather the resulting map $\psi^{classical}$, see eq. (\ref{eq:psi}), as a coordinate in the direction $\alpha$. This way one can introduce a coordinate system for every triangulation generated in the MC simulations. The coordinates can be used to visualize the differences between generic triangulations of different CDT phases. It is worth mentioning that the harmonic maps (coordinates) described above have a very good property of smoothly interpolating between the 4-simplices in the geometric  outgrowths, which commonly appear in the CDT triangulations forming fractal structures. Imagine such an outgrowth consisting of many simplices and linked to the rest of the triangulation by only  a few simplices. Due to  the properties of harmonic maps, all simplices in the outgrowth will have almost the same value of the field $\phi^{\alpha}_i$ in all $\alpha$-directions. Therefore the outgrowths should appear as the field condensations in the harmonic maps.


\begin{figure}[h]
    \centering
    \includegraphics[width = 0.6\textwidth]{Figures/maps.pdf}
    \caption{The 4-volume density map projected on two spatial ("$x$" and "$y$") directions measured in phase $C$ ($\kappa_0 = 4.0, \Delta= 0.2, T = 20, \bar{N}_{41} = 720k$). Each point on the plot represents a 4-simplex having the scalar field values (coordinates) $(\phi^x,\phi^y)$. The colors encode the time coordinate $t$ of the original CDT foliation. The dense regions are geometric fractal     outgrowths in the triangulation. The outgrows structure resembles cosmic voids and filaments of the real Universe.}
    \label{fig:density_map}
\end{figure}

A typical map (projected on two spatial directions: "$x$" and "$y$") measured in the semi-classical phase $C$ of the toroidal CDT is presented in Fig. \ref{fig:density_map}. Looking at Fig. \ref{fig:density_map} it becomes apparent that the phase $C$ generic triangulations represent a homogeneous and isotropic geometry on macroscopic scales. However, exactly as it is observed in the real Universe, there are local density fluctuations (geometric outgrows in the case of CDT) showing very non-trivial voids and filaments structure. One should note that in this context this is the emerging feature of pure quantum gravity, as the scalar fields discussed above do not have any impact (back-reaction) on the geometry, and are simply introduced for visualization purposes. Similar maps, obtained for generic triangulations of other CDT phases have completely different shapes, as discussed in publication \cite{pub6}.\\ 

Using the scalar fields as coordinates one can also measure the scaling of 4-volume in a triangulation by picking a (random) center and following a diffusion wave from that center and observing the growth in the volume of the diffusion shell. Looking at the scaling of the volume with radius one can measure the, so-called, Hausdorff dimension, associated with the harmonic coordinates. This was measured for the following fixed lattice volumes $\bar{N}_{41} = \{80k, 160k, 200k, 240k, 300k, $ $360k,400k, 480k, 560k, 600k, 720k\}$. The 4-volume contained in a box (window) of size $\Delta\phi^x\times\Delta\phi^y\times\Delta\phi^z \times \Delta \phi^t$, denoted $\Delta N_{win}$, normalized by the total volume $N_{tot}$ can be used to measure the Hausdorff dimension. It was found that in   phase $C$ one obtains a universal behavior, as presented in Fig. \ref{fig:inc_vol_prof}. The fitted Hausdorff dimension is consistent with $d_H = 4$. 

 
\begin{figure}[h]
    \centering
    \includegraphics[width = 0.8\textwidth]{Figures/4D_varvols_C.pdf}
    \caption{The figure shows the ratio of $\Delta N_{win}$ (4-volume inside the box of size $\Delta\phi^x\times\Delta\phi^y\times\Delta\phi^z \times \Delta \phi^t$) and $N_{tot}$ (total volume) in the function of the normalized size of the box (Radius). This function was measured in measured in phase $C$ ($\kappa_0 = 4.0, \Delta = 0.2, T = 20$). The various thin lines denote measurements for different lattice volumes $\bar N_{41}$, the solid blue line is a fit of the function $a r^b$ to their average.}
    \label{fig:inc_vol_prof}
\end{figure}

\newpage

\section{Dynamical scalar fields}

\textit{"The effort to understand the Universe is one of the very few things that lifts human life a little above the level of farce, and gives it some of the grace of tragedy."} - \textbf{Steven Weinberg}\\\\ 

\textit{This section is based on the publications \cite{pub5,pub6}.}\\\\

So far the back-reaction of the matter field on the purely geometric degrees of freedom was not taken into account. Including back-reaction of quantum (later also called dynamical) scalar fields can lead to nontrivial changes in the geometry. In the results presented below, the scalar fields are massless scalar fields with the (discretized) action (\ref{eq:scalar_w_boundary}), minimally coupled to the geometric (Regge) action (\ref{eq:ation_kappa}). Including such fields in the MC, simulations mean that not only do the field values have to be generated - in the MC simulations the heat bath method \cite{hb1,hb2} was used - but also that the matter action will affect the probability of performing the purely geometric moves. Depending on the parameters of a simulation, either the geometric or the matter part of the action dominates, thus one can expect a phase transition of some sort when moving in the  parameter space, now also including a new coupling constant corresponding to the circumference $\delta$ of the $S^1$ target space (or alternatively the jump amplitude) of the scalar field. \\
\begin{figure}[h]
    \centering
    \includegraphics[width = 0.45\textwidth]{Figures/jump_time.pdf}
    \includegraphics[width = 0.45\textwidth]{Figures/jump_space.pdf}
    \caption{The volume profile in the presence of one field winding around the time direction (left) and three fields winding around the non-equivalent spatial directions (right).}
    \label{fig:field_prof}
\end{figure}

The choice of the $\delta$ value is not the only additional parameter, as one can also choose the number of $\phi$-fields, as well as the number and type (time- or space-like) of  non-equivalent winding directions for the scalar field(s). Adding a field with $\delta = 0$ has already a visible but small effect, as it shifts some characteristics of generic triangulations appearing in the path integral, for example, it lowers the ratio of ${N}_{32}/N_{41}$, and adding more fields the effect is larger. A much stronger effect is observed for large jump magnitude $\delta$. As already discussed, if no scalar fields are added, the measured volume profile of the toroidal CDT model in the semi-classical  phase $C$ is a constant function. This is also the case of CDT coupled to the scalar fields with zero or small jump magnitude $\delta$, but for large $\delta$ one observes a dramatic change in the volume profile, as it is shown in Fig. \ref{fig:field_prof}. In the case with one scalar field winding around the time direction, using a simple minisuperspace-like model presented in Appendix  3 of \cite{pub6}, one can expect to observe a "pinched"  volume profile, turning the constant function into a $cos(t)$ function, as seen in the left panel of Fig. \ref{fig:field_prof}.  
On the other hand, as can be seen on the right panel of Fig. \ref{fig:field_prof}, the jump condition introduced only in the spatial directions will also trigger, for large-enough $\delta$ values,  a  kind of "pinched" volume profile, however, the reason behind it is different. In that case, the fitted volume profile is given by a  $cos^3(t)$ function, which corresponds to the volume profile of the (Euclidean) de Sitter sphere.

\begin{figure}[h]
    \centering
    \includegraphics[width = 0.6\textwidth]{Figures/maps_dyn.pdf}
    \caption{The 4-volume density map projected on two spatial ("$x$" and "$y$") directions measured in phase $C$ ($\kappa_0 = 2.2 \Delta = 0.6, T = 4, \bar{N}_{41} = 160k$) in the presence of 3 scalar fields winding around  non-equivalent spatial directions ($\delta = 1.0$). Each point on the plot represents a 4-simplex having the scalar field values (coordinates) $(\phi^x,\phi^y)$.  The 4-volume is concentrated in the center of the plot, and the low-density region around it shows the "pinching"  effect, leading to the effective  spatial topology change.}
    \label{fig:density_map_din}
\end{figure}

Qualitatively, the same kind of "pinching" happens in the spatial directions, leading to the effective topology change from toroidal to spherical. By the effective topology change we mean a situation where there is still a remnant of the original CDT topology (which by definition cannot change in the MC simulations), but, due to the "pinching", the toroidal part is of  cutoff size, and the dominating geometry has (almost) spherical topology. So effectively, the triangulations start to behave as if the topology of spatial slices was spherical instead of toroidal. It triggers an additional effect, also observed in the spherical CDT, leading to the non-trivial de Sitter-like  volume profile of $\cos^3(t)$, i.e., it causes a "pinching" in the time direction, changing the effective topology to that of $S^4$. Consequently, the toroidal CDT model with  scalar fields winding around spatial directions  behaves effectively as the spherical CDT model. \\ 

Summarizing, the presence of the dynamical scalar fields with a non-trivial  jump  condition (or alternatively a nontrivial target space $S^1$) can trigger a phase transition, which effectively changes the topology of the CDT configurations. Fig. \ref{fig:density_map_din} shows the 4-volume density map (projected to the "$x$" and "$y$" spatial directions) of a generic triangulation in phase $C$ in the presence of 3 scalar fields with large jump magnitude ($\delta = 1.0$) winding around 3 non-equivalent spatial directions.  Most simplices are concentrated in the center of the plot and at the edge of the plot the density becomes much smaller. This is exactly the "pinching" effect, leading to a formation of a single large geometric outgrowth, where almost all 4-volume is concentrated, and therefore changing the effective topology from the toroidal to the spherical one. The geometry looks considerably different than that of the pure gravity model, presented in Fig. \ref{fig:density_map}, where the scalar fields were used only as maps and did not have any back-reaction impact on the underlying manifold.  
% Chapter 5

\chapter{Conclusions} % Main chapter title

\label{Chapter5} % For referencing the chapter elsewhere, use \ref{Chapter3} 

%----------------------------------------------------------------------------------------

% Define some commands to keep the formatting separated from the content 

%----------------------------------------------------------------------------------------

The goal of this thesis is to investigate the first steps towards an unconditionally stable space-time isogeometric method with maximal regularity, using a tensor-product approach, for the wave problem \eqref{eqonde}. 

Inspired by the work (\cite{Steinbach2019}), our starting point is studying the conditioned stability of the conforming quadratic IGA discretization with global $C^1(0,T)$ regularity of the related ordinary differential problem \eqref{eq ode}. In Chapter \ref{Chapter3} we hence obtain two explicit upper bounds on the mesh-size, which, if respected, guarantee stability. The first one \eqref{h bound}, i.e., 
\begin{equation*}
h \leq \frac{\pi^2}{\sqrt{2}(2+\sqrt{\mu}T)\mu T},
\end{equation*}
is an extension to quadratic IGA with maximal regularity of Theorem 4.7 of (\cite{Coercive}), which is a result for the piecewise continuous linear FEM discretization. In particular, our upper bound is about twice as large as the FEM one, and the discrete stability constant of \eqref{infsup zank}, i.e.,
\begin{equation*}
\beta_1(\mu,T):=\frac{2 \pi^2}{(2+\sqrt{\mu}T)^2(\pi^2+4\mu T^2)},
\end{equation*}
depends on the coefficient $\mu >0$ and on the final time $T>0$ of the ODE \eqref{eq ode} with the same order as the FEM one. The second upper bound \eqref{h bound gard}, i.e.,
\begin{equation*}
h \leq \frac{\pi^5}{(\pi^2+ 4\mu T^2)[\pi^2+2\mu T^2 (2+\sqrt{\mu} T)]} \sqrt{\frac{2b - \mu T^2}{2b(2+b)\mu}}
\end{equation*}
where $b > \frac{\mu T^2}{2}$ is an arbitrary fixed real value, is obtained by the theory of \textit{Galerkin method applied to G\aa rding-type problems}. 

The asymptotic case, i.e., $\mu$ that is significantly large, seems to us the most interesting situation for the problem of instability (Remark \ref{pollut effect}) and for the wave equation. Thus, in Remark \ref{oss asympt} we compare the obtained bounds and the corresponding stability constants for $\mu \rightarrow \infty$. It follows that, for ``a very large'' $\mu$, the first bound \eqref{h bound} is a weaker constraint than the second bound \eqref{h bound gard}. Moreover, some numerical results of Chapter \ref{Chapter4} show that the latter, with the optimal choice for $b$ \eqref{b}, can be a stronger constraint than the former even if $\mu$ is not extremely large.

Let us define 
\begin{equation*}
C_1(\mu,T):=\frac{1}{\beta_1(\mu,T)},
\end{equation*}
where $\beta_1(\mu,T)$ is the discrete inf-sup constant of \eqref{infsup zank} that we recall above. In Remark \ref{oss asympt} we note that the stability constant of \eqref{stab iga gard}, i.e.,
\begin{equation*}
C_2(\mu,T):=\Bigg[ \frac{3b + \mu T^2\big(\frac{8b}{\pi^2}-\frac{1}{2} \big)}{2b-\mu T^2} \Bigg] \quad \Bigg( \text{for a fixed} \ b > \frac{\mu T^2}{2} \Bigg),
\end{equation*}
that arises from the second upper bound \eqref{h bound gard} has a slower growth than $C_1(\mu,T)$ for $\mu \rightarrow \infty$. Thus, the theory of \textit{Galerkin method applied to G\aa rding-type problems} is useful for our problem leading to a lower bound of the discrete inf-sup that, asymptotically, is sharper than that obtained by extending the analysis of O. Steinbach and M. Zank (\cite{Coercive}). In Chapter \ref{Chapter4} there are also some numerical results showing that the inf-sup $\beta_2(\mu,T):=\frac{1}{C_2(\mu,T)}$ can be sharper than $\beta_1(\mu,T)$ even if $\mu$ is not significantly large.

In Chapter \ref{Chapter4} we observe that the two upper bounds \eqref{h bound}, \eqref{h bound gard} are not optimal. However, if the mesh-size is uniform, we manage to numerically find a stability constraint \eqref{emp constraint}, i.e., 
\begin{equation*}
h <  \sqrt{\frac{9}{\mu}},
\end{equation*}
which, from the numerical results that we obtain, seems to be sharp. The upper bound \eqref{emp constraint} is of the same order (w.r.t $\mu$) of the sharp upper bound \eqref{stab Linfty} for the stability of FEM discretization, i.e., 
\begin{equation*}
h < \sqrt{\frac{12}{\mu}}.
\end{equation*}
 
Quadratic IGA discretization of maximal regularity appears to be advantageous over piecewise continuous linear FEM. Indeed, the stability upper bounds on the mesh size of the former are very similar to the FEM ones, and the orders of convergence of the IGA method are of one order more than the FEM ones. Moreover, from the numerical tests we see that the error committed by the IGA is, for each mesh-size $h$, strictly smaller than the FEM one.\\ \vspace{0.8cm}

As observed in Remark \ref{oss grado più alto}, if we raise the degree of the splines to $p>2$ while keeping maximal regularity, the constraint on the mesh-size \eqref{h bound gard} and the resulting stability constant do not change. On the other hand, how the upper bound \eqref{h bound} behaves in $h$ is not immediately clear from the proof of Theorem \ref{teo stab IGA zank}. However, we are sure that the orders of convergence of the discrete solution to the exact solution will be $p$ in $|\cdot|_{H^1(0,T)}$ and $p+1$ in $\|\cdot\|_{L^2(0,T)}$, if the exact solution is sufficiently regular. Also, we could expect that the maximal error, and more generally for all values of $h$, is strictly smaller than the error committed by continuous, linear FEM and quadratic $C^1(0,T)$ IGA discretizations. This behaviour of the error would be a consequence of high degree of approximation of B-spline technology (\cite{HUGHES20084104, n-width}) and of the ``good behaviour'' of high-order methods with respect to wave propagation problems (\cite{Babuska2000}). These error considerations are indeed confirmed by our numerical tests in Figures \ref{fig: err grad 3}, \ref{fig: rel err grad 3}, \ref{fig: err grad 4}, \ref{fig: rel err grad 4}.

As in Chapter \ref{Chapter4}, as a numerical example for the Galerkin-Petrov finite element methods  we consider a uniform discretization of the time interval $(0,T)$ with $T = 10$ and a mesh-size $h = T/N$. For $\mu = 1000$ we consider the strong solution $u(t) = \sin^2\Big(\frac{5}{4}\pi t\Big)$
and we compute the integrals appearing at the right-hand side using high-order integration rules.

\begin{figure}
	\hspace{-1.3cm}
	\begin{minipage}[h!]{9cm}
		\centering
		\includegraphics[width=6.5cm]{Figures/err_ode_iga_grado3}
		\caption{A $\log$-$\log$ plot of errors committed by cubic IGA, with maximal regularity, in \\$|\cdot|_{H^1(0,T)}$ seminorm and in \\$\|\cdot\|_{L^2(0,T)}$ norm, with respect to a uniform mesh-size $h$. Also, the best approximation error in\\ $|\cdot|_{H^1(0,T)}$ seminorm and the bound \eqref{emp constraint} are represented. The square of the wave number is $\mu=1000$}
		\label{fig: err grad 3}
	\end{minipage} 
	\hspace{-1.3cm}
	\begin{minipage}[h!]{9cm}
		\centering
		\includegraphics[width=6.5cm]{Figures/rel_err_ode_iga_grado3}
		\caption{A $\log$-$\log$ plot of relative errors committed by cubic IGA, with maximal regularity, in $|\cdot|_{H^1(0,T)}$ seminorm and in \\$\|\cdot\|_{L^2(0,T)}$ norm, with respect to a uniform mesh-size $h$. Also, the best approximation error in\\ $|\cdot|_{H^1(0,T)}$ seminorm and the bound \eqref{emp constraint} are represented. The square of the wave number is $\mu=1000$}
		\label{fig: rel err grad 3}
	\end{minipage}
\end{figure}

\begin{figure}
	\hspace{-1.3cm}
	\begin{minipage}[h!]{9cm}
		\centering
		\includegraphics[width=6.5cm]{Figures/err_ode_iga_grado4}
		\caption{A $\log$-$\log$ plot of errors committed by IGA of fourth degree, with maximal regularity, in $|\cdot|_{H^1(0,T)}$ seminorm and in $\|\cdot\|_{L^2(0,T)}$ norm, with respect to a uniform mesh-size $h$. Also, the best approximation error in $|\cdot|_{H^1(0,T)}$ seminorm and the bound \eqref{emp constraint} are represented. The square of the wave number is $\mu=1000$}
		\label{fig: err grad 4}
	\end{minipage} 
	\hspace{-1.3cm}
	\begin{minipage}[h!]{9cm}
		\centering
		\includegraphics[width=6.5cm]{Figures/rel_err_ode_iga_grado4}
		\caption{A $\log$-$\log$ plot of relative errors committed by IGA of fourth degree, with maximal regularity, in $|\cdot|_{H^1(0,T)}$ seminorm and in $\|\cdot\|_{L^2(0,T)}$ norm, with respect to a uniform mesh-size $h$. Also, the best approximation error in $|\cdot|_{H^1(0,T)}$ seminorm and the bound \eqref{emp constraint} are represented. The square of the wave number is $\mu=1000$}
		\label{fig: rel err grad 4}
	\end{minipage}
\end{figure}

As one can note in Figures \ref{fig: err grad 3}, \ref{fig: rel err grad 3}, \ref{fig: err grad 4}, \ref{fig: rel err grad 4} the constraint \eqref{emp constraint} seems to remain optimal with respect to the error by raising the degree and regularity of the splines.

Although the errors diminish by raising the degree and regularity of the splines, it is of significant importance to find a method that is stable for every degree and regularity, so that we are not forced to work with dense matrices, which cause a high computational cost. 

Finally, let us note that we expect the IGA discretization of degree $p \geq 2$ and regularity $C^{p-1}(0,T)$ to perform better than the piecewise continuous FEM of degree $p$. Indeed, although the IGA matrices have more non-zero entries, the FEM discretization uses more basis functions. Furthermore, we expect that, while approximating solutions of wave propagation problems, the error committed by the IGA method is smaller than the FEM one (\cite{Babuska2000}).\\ \bigskip

Our proposals of stabilizations are all based on non-consistent perturbations.

In order to stabilize the quadratic IGA with maximal regularity, if the mesh-size is uniform, we propose to consider the perturbed bilinear form \eqref{tent 3 iga}, i.e.,
\begin{equation*}
\begin{split}
a_h(u_h,v_h)=-\langle \partial_t u_h, \partial_t v_h \rangle _{L^2(0,T)} + \mu \langle u_h, &v_h \rangle_{L^2(0,T)} \\
&- \delta \mu h^4\langle \partial_t^2 u_h, \partial_t^2 v_h \rangle _{L^2(0,T)},
\end{split}
\end{equation*}
for all $u_h \in V^h_{0,*}$ and $v_h \in V^h_{*,0}$, where $\delta >0$ is a fixed real value. Our numerical results are very promising in the case of $\delta=\frac{1}{100}$. It would therefore be interesting to study the theory that could explain why this stabilization works and then propose an optimal $\delta$. 

For IGA method with generic polynomial degree $p$ and maximal regularity, we propose to consider the following perturbed bilinear form
\begin{equation}\label{ah grado magg iga}
\begin{split}
a_h(u_h,v_h)=-\langle \partial_t u_h, \partial_t v_h \rangle _{L^2(0,T)} + \mu \langle u_h, &v_h \rangle_{L^2(0,T)} \\
&- \delta \mu h^{2p}\langle \partial_t^p u_h, \partial_t^p v_h \rangle _{L^2(0,T)},
\end{split}
\end{equation}
for all $u_h$ and $v_h$, respectively, in the discrete trial and test spaces, where $\delta >0$ is a fixed real value.

If the mesh-size is non-uniform, we suggest considering
\begin{equation}\label{ah iga mesh generica}
\begin{split}
a_h(u_h,v_h)=-\langle \partial_t u_h, \partial_t v_h \rangle _{L^2(0,T)} + \mu \langle u_h, &v_h \rangle_{L^2(0,T)} \\
&- \delta\mu\sum_{l=1}^{N} h_l^{2p} \langle \partial_t^p u_h, \partial_t^p v_h\rangle_{L^2(\tau_l)},
\end{split}
\end{equation}
for all $u_h$ and $v_h$, respectively, in the discrete trial and test spaces, where $\tau_l$, for $l=1, \ldots,N$, are the subintervals of $(0,T)$ given by the Galerkin discretization and $\delta >0$ is a fixed real value. Actually, \eqref{ah grado magg iga} is a subcase of \eqref{ah iga mesh generica}. 

Finally, let us briefly consider the homogeneous Dirichlet problem for the second-order wave equation \eqref{eqonde}, i.e., 
\begin{equation}
\begin{cases}
\partial_{tt}u(x,t)-\Delta_xu(x,t)=g(x,t) \quad (x,t) \in Q:=\Omega \times (0,T)\\
u(x,t)=0 \quad (x,t) \in \partial{\Omega} \times [0,T]\\
u(x,0)=\partial_tu(x,t)_{|t=0}=0 \quad x \in \Omega,
\end{cases}
\end{equation}
where $\Omega \subset \mathbb{R}^d$, with $d=1,2,3$, is an open bounded Lipschitz domain and, for a real value $T>0$, $(0,T)$ is a time interval. In (\cite{Coercive}) the authors introduce a space-time variational formulation of \eqref{eqonde}, where integration by parts is also applied with respect to the time variable, and the classic anisotropic Sobolev spaces with homogeneous initial and boundary conditions are employed. Inspired by (\cite{Steinbach2019, Zank2020}), a possible unconditionally stable space-time IGA method with maximal regularity based on a tensor-product approach could be obtained by considering the perturbed bilinear form
\begin{equation}\label{onde}
\begin{split}
a_h(u_h,v_h)=-\langle \partial_t u_h, \partial_t v_h \rangle _{L^2(Q)} + &\langle \nabla_x u_h, \nabla_x v_h \rangle_{L^2(Q)} \\
&- \delta\sum_{m=1}^{d} \sum_{l=1}^{N_t} h_l^{2p} \langle \partial_t^p \partial_{x_m} u_h, \partial_t^p \partial_{x_m} v_h\rangle_{L^2(\Omega \times \tau_l)},
\end{split}
\end{equation}
for all $u_h$ and $v_h$, respectively, in the discrete trial and test spaces, where $\tau_l$, for $l=1, \ldots,Nt$, are the subintervals of $(0,T)$ given by the Galerkin discretization and $\delta >0$ is a fixed real value. We expect this stabilization to perform well, given the appreciable numerical results of the perturbation \eqref{tent 3 iga}. However, we have not tested \eqref{onde} yet, since we prefer to give priority to a full analysis of the IGA discretization of our model problem \eqref{var ode}, which, given its link to the wave equation, we expect to be significantly useful. 









%----------------------------------------------------------------------------------------
%	ACKNOWLEDGEMENTS
%----------------------------------------------------------------------------------------

\begin{acknowledgements}
\addchaptertocentry{\acknowledgementname} % Add the acknowledgements to the table of contents
First of all I would like to thank for all the support to prof. Jerzy Jurkiewicz, who gave me the possibility to join to the research group and supervised me throughout the years, shared with me many of his thoughts and ideas which definitely influenced the way I see my field of research. Also to dr. Jakub Gizbert-Studnicki for all the discussions during my PhD. and for his patience and the significant amount of time he spent with advising my thesis. Special thanks to dr. Andrzej Görlich also for the discussions and constant technical help with the computers, codes and simulations. I would like to thank also to prof. Jan Ambjorn for all the fruitful discussions throughout our regular CDT meetings, as he shared his deep knowledge with us, my knowledge and understanding also improved. Additionally, I would like to express my gratitude to my wife, dr. Anna Francuz for her presence in my life, and for her support during my studies and her patience and understanding in those times, when I work in late hours instead of being there. Last but not least, I would like to thank to my daughter Eszter Eleonóra, for letting me sleep, sometimes....
\end{acknowledgements}





%----------------------------------------------------------------------------------------
%	THESIS CONTENT - APPENDICES
%----------------------------------------------------------------------------------------

\appendix % Cue to tell LaTeX that the following "chapters" are Appendices


\chapter{Appendix of Chapter 4}
\label{AppendixB}
 % Main appendix title
\section{Depression users posting analysis on CLPSych dataset}
\begin{figure*}[htbp]
\centering
\begin{minipage}{\textwidth}
\includegraphics[width=1\textwidth]{Figures/ttdd_tsne_CLP.png}

\label{fig:ttdd_tsne_CLP}
\end{minipage}
\caption{Document representations distribution using TSNE on CLPsych dataset. (a) Top left: Document vectors learned from LDA. (2)Top Right: Document vectors learned from DocNADE. (3) Bottom Left: Document vectors learned from fdp-DocNADE. (4) Bottom Right: Document vectors learned from fdp-DocNADEa.}
\end{figure*}
 
 \begin{figure*}
\centering
\includegraphics[width=0.85\textwidth]{Figures/wordcloud_topics_CLP.png}
\caption{Top probability words in four topics on CLPcych set with manual defined 'theme'}
\label{fig:word_cloud_CLP}
\end{figure*}


Latent Dirichlet Allocation (LDA) is an classical topic model which is shown in  \figurename{\ref{fig:lda_model}. In \figurename{\ref{fig:lda_model}, a dirichlet distribution $\alpha$ represent a particular document and this topic distribution is $\theta$. A particular topic $Z$ can be selected from $\theta$, then the word distribution $\varphi$ of topic $Z$ will be randomly selected from second dirichlet distribution $\beta$.  $N$ is a generated word from $\varphi$ with topic $Z$. The following procedure is to use Gibbs sampling to maximize the loglikehood of $p(W|Z,\varphi)$. In brief, LDA topic model is a clustering process which concentrates on word frequency and word distribution in corpus. Optimized $\theta$ can be represented as document vectors for classification task and word distribution $\varphi$ can be used to describe document content.
\begin{figure*}[htbp]
\centering
\includegraphics[scale=0.5]{Figures/lda_png.png} 
\caption{Latent Dirichlet Allocation (LDA). Image Courtesy of \cite{blei2003latent}} 
\label{fig:lda_model} 
\end{figure*}
% Appendix A

\chapter{Topological relations between  parameters of a CDT triangulation} % Main appendix title

\label{AppendixA} 


The following list sums up the topological relations valid  for any CDT triangulation. For the definition of the $A,B,C,D$ and $E$ parameters see Chapter \ref{chapter3}. Note, that for simpler notation in the appendix, contrary to the main text, we use a convention of {\it global} numbers which distinguishes between the number of $s_{41}$  and  $s_{14}$ simplices, denoted $N_{41}$ and $N_{14}$, respectively. Similarly, we distinguish between $N_{32}$ and $N_{23}$.

\begin{enumerate}
    \item[$T_{1}$.:] $2A_1 + C_1 + E = 5 \cdot N_{41}$
    \item[$T_{2}$.:] $C_1 + 2B_{1a} + 2B_{2a} + D = 5 \cdot N_{32}$
    \item[$T_{3}$.:] $C_2 + 2B_{2a} + 2B_{2b} + D = 5 \cdot N_{23}$
    \item[$T_{4}$.:] $2A_2 + C_2 + E = 5\cdot N_{14}$
    \item[$T_{5}$.:] $2A_1 + C_1 = 2A_2 + C_2 = 2(N_{41}+N_{14})$
    \item[$T_{6}$.:] $2B_{1b} + D = 3\cdot N_{32}$
    \item[$T_{7}$.:] $2B_{2b} + D = 3\cdot N_{23}$
    \item[$T_{8}$.:] $2B_{1a} + C_1 = 2\cdot N_{32}$
    \item[$T_{9}$.:] $2B_{2a} + C_2 = 2\cdot N_{23}$
    \item[$T_{10}$.:] $(A + B + C + D + E) = N_3 = \frac{5}{2}N_4$ 
\end{enumerate}

A triangulation can be characterized by  the following global parameters, referring to the number of (sub-) simplices of various types,  $N_{10}, N_{20}, N_{11},$ $N_{30}, N_{21},$ $N_{12}, N_{40}, N_{31}, N_{13}, N_{22}, N_{41}, N_{32}, N_{23}, N_{14}, \chi$, where the first number in the subscript denotes the number of vertices in the spatial  slice $t$ and the second one is the number of vertices in $t+1$, and $\chi$ is the Euler characteristics related to the fixed spatial topology. These global numbers can be joined using the seven Dehn-Sommerville relations \cite{nonperturb}:

\begin{itemize}
\item[$DS_{1}$.:] $N_{40} = N_{41} = \frac{1}{2}(N_{41}+ N_{14})$
\item[$DS_{2}$.:] $N_{30} = 2N_{40} = (N_{41}+ N_{14})$
\item[$DS_{3}$.:] $N_4 = \frac{2}{5}(N_{40}+N_{31}+N_{13}+N_{22})$
\item[$DS_{4}$.:] $N_{10}-N_{20}+N_{30}-N_{40} = 0$
\item[$DS_{5}$.:] $N_{22} = \frac{3}{2}(N_{32}+N_{23})$
\item[$DS_{6}$.:] $2N_1 -3N_2 +4N_3 -5N_4 = 0$
\item[$DS_{7}$.:] $N_0 - N_1 + N_2 - N_3 +N_4 = \chi$ 
\end{itemize}
Using the "$T$" relations: 

\begin{equation}
(N_{32} + N_{23}) = \frac{2}{5}B + \frac{2}{5}D + \frac{1}{5}C = \frac{2}{3}B_b + \frac{2}{3}D = B_a + \frac{1}{2}C = \frac{2}{3} N_{22}, 
\end{equation}
and from this it follows, that $D$ can be expressed as:

\begin{equation}
D = \frac{3}{2}B_a - B_b + \frac{3}{4}C.
\end{equation}
Similarly, one can express the other relations for the two 4-dimensional simplices, and using "$DS$" relations one obtains :

\begin{equation}
(N_{41}+N_{14}) = \frac{1}{2}A + \frac{1}{4}C = N_{30} = 2 N_{40}.
\end{equation}
It also follows that:

\begin{equation}
E = \frac{1}{2}A + \frac{1}{4}C.
\end{equation}
Using $DS_3$ one can find the relations fulfilled by the %first type of 
time-like tetrahedra:

\begin{equation}
N_4 = (N_{41}+N_{14})+(N_{32}+N_{23}) = \frac{2}{5}(N_{40} + N_{31} + N_{22} + N_{13}),
\end{equation}
leading to

\begin{equation}
(N_{31}+N_{13}) = 2(N_{41}+N_{14}) + (N_{32}+N_{23}) = A + B_a + C.
\end{equation}
The formula for the spatial links can be expressed with the help of $DS_4$:

\begin{equation}
N_{20} = N_{10} + \frac{1}{2}(N_{41}+N_{14}) = N_{10} + \frac{1}{4}A + \frac{1}{8}C.
\end{equation}
The remaining numbers $N_{11}$ and $(N_{21}+N_{12})$ are calculated in a bit more involved way. Taking $DS_6$ we can express the total number of time-like links as:

\begin{equation}
N_{11} = \frac{3}{2}(N_{30}+N_{21}+N_{12}) -\frac{3}{2}A -\frac{5}{2}B_a -2C -N_0,
\end{equation}
which involves the number of time-like triangles. Using $DS_7$ one can find the following relation:

\begin{equation}
\chi = N_0 - \frac{1}{2}(N_{30}+N_{21}+N_{12})+N_4,
\end{equation}
which leads to the expression for the time-like triangles:

\begin{equation}
(N_{21}+N_{12}) = 2N_0 -2\chi +\frac{1}{2}A + 2B_a +\frac{3}{2}C,
\end{equation}
which now can be used in the previous equation to get the number of the time-like links:

\begin{equation}
N_{11} = 2N_0 -3\chi +\frac{1}{2}B_a +\frac{1}{4}C.
\end{equation}

With the above mentioned relations one can check, that for any CDT triangulation there are 8 independent parameters, which are enough to compute all other global parameters. For example, one can choose the following set of independent parameters

\begin{equation}
Set_R = \{ N_0, \chi, A_1, A_2, B_{1a}, B_{2a},C_1, C_2 \}.
\end{equation}
One can as well use the following  set, including the currently used global numbers $N_0$, $N_{41}$ and $N_{32}$ appearing in the CDT  action:

\begin{equation}
Set_G = \{ N_0, \chi, N_{41}, N_{32}, N_{23}, C_1, C_2, D\}.
\end{equation}

These new parameters can be used not only as order parameters, but also they can be potentially used to extend the CDT action, see eq. (\ref{eq:ation_kappa}), to the following form 

\begin{equation}
    S_{CDT}^{ext} = -(\kappa_0 + 6\Delta) N_0 + \kappa_4 (N_{41} + N_{32}) + \Delta  N_{41} + \kappa_C C + \kappa_D D,
\end{equation}
where $\kappa_C$ and $\kappa_D$ are the new coupling constants related to the $C$ and $D$ parameters, respectively. The physical meaning of these parameters and the related coupling constants is not straightforward and a discussion of it will not be a part of this thesis.






%##########################################################
%##########################################################
%##########################################################
%##########################################################

%\section{Appendix C - Preparation and positioning of Nanodiamonds using Pick-and-Place}
In preparation of transferring SiV hosting NDs to the bullseye antenna, a dispersion of SiV containing NDs was drop-casted onto a fused silica coverslip.
The NDs were fabricated using high-pressure high temperature (HPHT) treatment of the catalyst metals-free hydrocarbon growth system. The NDs were
treated with \ch{HNO3 + HClO4 + H2SO4} and HF to remove sp2 carbon and were
washed and dried afterwards. FIB milled marker structures on the fused silica coverslip serve as a position reference to later locate specific SiV-containing NDs
with an AFM. To find SiV-hosting NDs the sample is examined using a homebuilt confocal microscope, orchestrated based on the open-source software qudi \cite{Binder2017Qudi:Processing}. A 2D galvo scanning mirror, a 4f-system and a 1.35 NA oil objective form the basis of this optical setup. Narrowband filtering (740 ± 13 nm)
around the ZPL of the SiV increases the signal to noise ratio when exciting offresonantly with 532 nm, as the dominant emission originates from SiV centers
within this filter window. After suitable NDs have been located, AFM imaging
of the area of interest is carried out. Triangulation of the ND position is done
with the help of the FIB markers, as well as positions of other fluorescing NDs
in the confocal image. In order to pick and place \cite{Schell2011ADevices} the ND of interest, a platin-coated AFM cantilever is approached with a constant force until the ND attaches to the tip. The pick-up is indicated by either the disappearing ND in a subsequent non-contact AFM scan or by image artifacts such as ’double-tip’ features. Those features also hint towards a successful placement strategy of the ND in the following step. With the ND attached to the cantilever tip, the sample is exchanged for the target sample, here the bullseye antenna. A careful non-contact imaging reveals the target position on the structure. For the antenna structure considered herein the central hole in the structure eases the placement of the ND due to its topology. A contrast in height is useful to detach the ND from
the cantilever. After successful ND placement (compare Fig. \ref{fig:C}b), the sample is again examined in the confocal microscope, with the objective being exchanged to 0.55 NA for the sake of a longer working distance and to leverage the collection efficiency \cite{Waltrich2021High-purityAntenna}. The verification of a successful SiV hosting ND placement is seen in Fig. \ref{fig:C}c.

\label{AppendixC}

%----------------------------------------------------------------------------------------
%	BIBLIOGRAPHY
%----------------------------------------------------------------------------------------


\printbibliography[heading=bibintoc]

%----------------------------------------------------------------------------------------

\end{document}  
