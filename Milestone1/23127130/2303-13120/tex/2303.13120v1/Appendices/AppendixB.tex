% Appendix B

\chapter{Additional Moves}
\label{AppendixB} 

Moves "2", "3", "4" and "5" (and their respective inverses) are the moves that are currently used during the MC computer simulations of CDT. The  new way of visualizing the moves by the "colored dots" graphs, introduced in Chapter \ref{chapter2}, makes it possible to find new moves relatively easily. Therefore in this appendix I propose two new moves. These new moves are not atomic ones but could be expressed with a smaller or larger set of combinations of our atomic moves. It makes sense to propose new moves even if they can be expressed as a sequence of other moves, because implementing new moves may significantly reduce the MC thermalization time and thus speed-up the  numerical simulations. \\


\begin{figure}[h]
    \centering
    \includegraphics[width = 0.8\textwidth]{Figures/move4_alternative.pdf}
    \caption{Alternative Move-4 adds a tetrahedron instead of a vertex to the middle of another tetrahedron.}
    \label{fig:m4p}
\end{figure}
The first proposal is a simple extension of move-4, shown in Fig. \ref{fig:m4p}.
Instead of adding a vertex to the CDT triangulation in the middle of an $s_{41}$ simplex, represented in Fig. \ref{fig:m4p} by a black dot splitting into four "external" black dots, one may propose that the internal structure has four interfaces, which means that it forms another $s_{41}$ simplex, i.e., an additional "internal" black dot. Thus such a move would be replacing a single black dot with five black dots, instead of four. The inverse move would require tracking such $s_{41}$ simplices (black dots) which are only surrounded by other black dots. In general such an extra move could be extended to inserting $N$ black dots, but the larger $N$ the harder it is to track such a structure necessary to perform the inverse move during the simulations. \\


\begin{figure}[h]
    \centering
    \includegraphics[width = 1\textwidth]{Figures/move_new.pdf}
    \caption{The proposed move transforms three blue dots into a bridge of black-blue-black dots.}
    \label{fig:mnew}
\end{figure}

The second proposal of the new move is much harder to be implemented, but potentially much more useful. It is shown in Fig. \ref{fig:mnew}. The move uses a "bridge" structure, which is a set of blue dots ($s_{32}$ simplices) laying along a line connecting two black dots ($s_{41}$ simplices). In general there can be arbitrary many blue dots in the middle (the length of the "bridge" can be arbitrarily long). The simplest version of the move, which involves only one blue dot could be realized by first performing a move-2 on the blue and, say, the left black dot. This would create the three connected blue dots and would place the two black dots next to each other. Then performing a move-5 would create three black dots out of the two. The last step, which is missing from the current set of CDT moves, would be the merging of the two vertices of the spatial link\footnote{In the graphical representation the link would be a closed loop consisting of three black dots.} of coordination number three, in the graphical representation leading to the "annihilation" of the black dots. Such a merging move could be achieved by performing a series of the existing MC moves, but it strongly depends on the details of a triangulation and it could take up even hundreds of them. This move has the potential to create large changes in a triangulation, because it can modify very large structures as well. However, the more massive the move is the less likely that it will be accepted by the MC algorithm. The move embedded in a larger structure is visualized in Fig. \ref{fig:mnew2}. We did not implement this move yet, because our current code does not store the necessary elements to track the required sub-structures. 

\begin{figure}[h]
    \centering
    \includegraphics[width = 0.8\textwidth]{Figures/move_new_embed.pdf}
    \caption{The figure shows the proposed new move embedded in a larger structure. Only the relevant legs are pictured for the sake of readability. On the top a simpler picture of Fig. \ref{fig:mnew} is shown as a hint.}
    \label{fig:mnew2}
\end{figure}

