% Appendix A

\chapter{Topological relations between  parameters of a CDT triangulation} % Main appendix title

\label{AppendixA} 


The following list sums up the topological relations valid  for any CDT triangulation. For the definition of the $A,B,C,D$ and $E$ parameters see Chapter \ref{chapter3}. Note, that for simpler notation in the appendix, contrary to the main text, we use a convention of {\it global} numbers which distinguishes between the number of $s_{41}$  and  $s_{14}$ simplices, denoted $N_{41}$ and $N_{14}$, respectively. Similarly, we distinguish between $N_{32}$ and $N_{23}$.

\begin{enumerate}
    \item[$T_{1}$.:] $2A_1 + C_1 + E = 5 \cdot N_{41}$
    \item[$T_{2}$.:] $C_1 + 2B_{1a} + 2B_{2a} + D = 5 \cdot N_{32}$
    \item[$T_{3}$.:] $C_2 + 2B_{2a} + 2B_{2b} + D = 5 \cdot N_{23}$
    \item[$T_{4}$.:] $2A_2 + C_2 + E = 5\cdot N_{14}$
    \item[$T_{5}$.:] $2A_1 + C_1 = 2A_2 + C_2 = 2(N_{41}+N_{14})$
    \item[$T_{6}$.:] $2B_{1b} + D = 3\cdot N_{32}$
    \item[$T_{7}$.:] $2B_{2b} + D = 3\cdot N_{23}$
    \item[$T_{8}$.:] $2B_{1a} + C_1 = 2\cdot N_{32}$
    \item[$T_{9}$.:] $2B_{2a} + C_2 = 2\cdot N_{23}$
    \item[$T_{10}$.:] $(A + B + C + D + E) = N_3 = \frac{5}{2}N_4$ 
\end{enumerate}

A triangulation can be characterized by  the following global parameters, referring to the number of (sub-) simplices of various types,  $N_{10}, N_{20}, N_{11},$ $N_{30}, N_{21},$ $N_{12}, N_{40}, N_{31}, N_{13}, N_{22}, N_{41}, N_{32}, N_{23}, N_{14}, \chi$, where the first number in the subscript denotes the number of vertices in the spatial  slice $t$ and the second one is the number of vertices in $t+1$, and $\chi$ is the Euler characteristics related to the fixed spatial topology. These global numbers can be joined using the seven Dehn-Sommerville relations \cite{nonperturb}:

\begin{itemize}
\item[$DS_{1}$.:] $N_{40} = N_{41} = \frac{1}{2}(N_{41}+ N_{14})$
\item[$DS_{2}$.:] $N_{30} = 2N_{40} = (N_{41}+ N_{14})$
\item[$DS_{3}$.:] $N_4 = \frac{2}{5}(N_{40}+N_{31}+N_{13}+N_{22})$
\item[$DS_{4}$.:] $N_{10}-N_{20}+N_{30}-N_{40} = 0$
\item[$DS_{5}$.:] $N_{22} = \frac{3}{2}(N_{32}+N_{23})$
\item[$DS_{6}$.:] $2N_1 -3N_2 +4N_3 -5N_4 = 0$
\item[$DS_{7}$.:] $N_0 - N_1 + N_2 - N_3 +N_4 = \chi$ 
\end{itemize}
Using the "$T$" relations: 

\begin{equation}
(N_{32} + N_{23}) = \frac{2}{5}B + \frac{2}{5}D + \frac{1}{5}C = \frac{2}{3}B_b + \frac{2}{3}D = B_a + \frac{1}{2}C = \frac{2}{3} N_{22}, 
\end{equation}
and from this it follows, that $D$ can be expressed as:

\begin{equation}
D = \frac{3}{2}B_a - B_b + \frac{3}{4}C.
\end{equation}
Similarly, one can express the other relations for the two 4-dimensional simplices, and using "$DS$" relations one obtains :

\begin{equation}
(N_{41}+N_{14}) = \frac{1}{2}A + \frac{1}{4}C = N_{30} = 2 N_{40}.
\end{equation}
It also follows that:

\begin{equation}
E = \frac{1}{2}A + \frac{1}{4}C.
\end{equation}
Using $DS_3$ one can find the relations fulfilled by the %first type of 
time-like tetrahedra:

\begin{equation}
N_4 = (N_{41}+N_{14})+(N_{32}+N_{23}) = \frac{2}{5}(N_{40} + N_{31} + N_{22} + N_{13}),
\end{equation}
leading to

\begin{equation}
(N_{31}+N_{13}) = 2(N_{41}+N_{14}) + (N_{32}+N_{23}) = A + B_a + C.
\end{equation}
The formula for the spatial links can be expressed with the help of $DS_4$:

\begin{equation}
N_{20} = N_{10} + \frac{1}{2}(N_{41}+N_{14}) = N_{10} + \frac{1}{4}A + \frac{1}{8}C.
\end{equation}
The remaining numbers $N_{11}$ and $(N_{21}+N_{12})$ are calculated in a bit more involved way. Taking $DS_6$ we can express the total number of time-like links as:

\begin{equation}
N_{11} = \frac{3}{2}(N_{30}+N_{21}+N_{12}) -\frac{3}{2}A -\frac{5}{2}B_a -2C -N_0,
\end{equation}
which involves the number of time-like triangles. Using $DS_7$ one can find the following relation:

\begin{equation}
\chi = N_0 - \frac{1}{2}(N_{30}+N_{21}+N_{12})+N_4,
\end{equation}
which leads to the expression for the time-like triangles:

\begin{equation}
(N_{21}+N_{12}) = 2N_0 -2\chi +\frac{1}{2}A + 2B_a +\frac{3}{2}C,
\end{equation}
which now can be used in the previous equation to get the number of the time-like links:

\begin{equation}
N_{11} = 2N_0 -3\chi +\frac{1}{2}B_a +\frac{1}{4}C.
\end{equation}

With the above mentioned relations one can check, that for any CDT triangulation there are 8 independent parameters, which are enough to compute all other global parameters. For example, one can choose the following set of independent parameters

\begin{equation}
Set_R = \{ N_0, \chi, A_1, A_2, B_{1a}, B_{2a},C_1, C_2 \}.
\end{equation}
One can as well use the following  set, including the currently used global numbers $N_0$, $N_{41}$ and $N_{32}$ appearing in the CDT  action:

\begin{equation}
Set_G = \{ N_0, \chi, N_{41}, N_{32}, N_{23}, C_1, C_2, D\}.
\end{equation}

These new parameters can be used not only as order parameters, but also they can be potentially used to extend the CDT action, see eq. (\ref{eq:ation_kappa}), to the following form 

\begin{equation}
    S_{CDT}^{ext} = -(\kappa_0 + 6\Delta) N_0 + \kappa_4 (N_{41} + N_{32}) + \Delta  N_{41} + \kappa_C C + \kappa_D D,
\end{equation}
where $\kappa_C$ and $\kappa_D$ are the new coupling constants related to the $C$ and $D$ parameters, respectively. The physical meaning of these parameters and the related coupling constants is not straightforward and a discussion of it will not be a part of this thesis.






%##########################################################
%##########################################################
%##########################################################
%##########################################################