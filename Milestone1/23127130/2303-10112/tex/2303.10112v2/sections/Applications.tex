
\section{Applications}\label{sec:app}


Temporal causal discovery has been widely used in many areas, such as scientific endeavors (earth science~\cite{Resources/datasets_surveys/runge2019inferring}, neuroscience~\cite{Applications/neuroscience/Nature_neuroscience_reid2019advancing, Applications/neuroscience/recent_good_summary/jocn_WeichwaldP21, Applications/neuroscience/nature_review/siddiqi2022causal}, bioinformatics~\cite{Applications/Gene_Science_sachs2005causal}), industrial implementations (anomaly detection~\cite{Applications/anomaly/work1_icdm_QiuLSL12}, root cause analysis~\cite{industiral_app_review/vukovic2022causal, Applications/CPSs/kbs_LiuLJS21, Applications/RCA/Assaad23EasyRCA}, business intelligence in online systems~\cite{Applications/interest_social_nx/cikm_ArabzadehFZNB18}, video analysis~\cite{Applications/video/iclr_YiGLK0TT20}). 
Table~\ref{tab:application_overview} summarizes the application areas and corresponding studies.
For scientific research, the learned causal relations should not usually be considered end results but rather starting points and hypotheses for further studies~\cite{Discussion/knowledge/makela2022incorporating}.
As a facilitator, causal discovery can play a supporting role in a multi-stage approach in an industrial setting~\cite{industiral_app_review/vukovic2022causal}.
In the rest part of this section, we will review three areas including earth science, anomaly detection and root cause, to explain these main workflows of incorporating temporal causal discovery into both scientific endeavors and industrial implementations, respectively.




\begin{table}[t]
    \centering
    \caption{Major studies in temporal causal discovery applications.}
    \label{tab:application_overview}
    \small %\footnotesize
    \scalebox{0.95}{
    \begin{tabular}{llp{8cm}}
    \toprule
    Groups & Application areas & Studies \\
    \midrule
    \multirow{10}{*}{Scientific endeavors}   % 此处后期排版的时候需要再调
    & Earth science & Climate change detection and attribution (\eg,~\cite{Applications/climate_attr/kdd_LozanoLNLPHA09}); Quantifying climate interactions (\eg,~\cite{Applications/earthS/climate_runge2014quantifying}); Latent driving force detection (\eg,~\cite{Applications/anomaly/work3_dagm_TrifunovSRERD19, EarthSci/abnorm/causal_intens/DBLP:conf/dagm/ShadaydehDGM19}); Causality validation between temperature and greenhouse gases (\eg,~\cite{Applications/earthS/climate_change_van2015causal}). \\
    & Neuroscience &
    Dynamic causal models for neural connectivity (\eg,~\cite{Applications/neuroscience/work1_neuroimage_PennySMF04, Applications/neuroscience/work1_ploscb_PennySDRFSL10, Applications/neuroscience/work1_neuroimage_JafarianLCFZ20}); Granger causal models for neural connectivity (\eg,~\cite{Applications/neuroscience/work2_bc_KaminskiDTB01, Applications/neuroscience/PNAS/stokes2017study, Applications/neuroscience/PNAS/sheikhattar2018extracting, 10.1371/journal.pcbi.1001110}); Causal inference from noninvasive brain stimulation (\eg,~\cite{Applications/neuroscience/jocn/BergmannH21}). \\

 

    & Bioinformatics & Modeling gene regulatory network (\eg,~\cite{MTS/SB/MMHO_DBN/origin/cibcb/LiN13, MTS/SB/MMHO_DBN/tcbb/LiCZN16, DBLP:journals/ploscb/VernySASI17, Applications/Gene_patil2022learning, DBLP:conf/iclr/Wu0B22}). \\
    \midrule
    \multirow{22}{*}{Industrial implementations}   % 此处后期排版的时候需要再调
    & Anomaly detection & Causal structure as detection reference (\eg,~\cite{Applications/anomaly/work1_icdm_QiuLSL12, Applications/anomaly/work2_icdm_BehzadiHP17, Applications/anomaly/work4_ijdsa_ApteVP21, Applications/anomaly/work5_22}); Detection from imbalanced data (\eg,~\cite{Application/SCGL_use_cikm_HuangXYYWX20}).   \\
    & Root cause analysis & Oscillation propagation tracing in the control loop (\eg,~\cite{landman2014fault, landman2016hybrid, chen2017root, lindner2018diagnosis}); Alarm flood reduction (\eg,~\cite{wang2015data, rodrigo2016causal, wunderlich2017structure}); Industrial knowledge combined analysis (\eg,~\cite{landman2016hybrid, industrial_knowledge/cao2022causal, thambirajah2009cause, 9964900}).  \\
    % & Advertising & XXX \\
    % & Recommendation & XXX \\
    & Business intelligence in online systems & User interest prediction (\eg,~\cite{Applications/interest_social_nx/cikm_ArabzadehFZNB18, Applications/interest_social_nx/asunam_HauffaBG19}); Social media analysis (\eg,~\cite{Applications/twitter/wsdm/ChangWML13, Applications/social_media_financial_data/tsapeli2017non, Applications/socialnx/complexnetworks/KuzmaCC21, Chuzhe_app/nca/ChenCHYX20}; Online advertising (\eg,~\cite{Applications/marketing/www_NuaraSTZ0R19, Applications/marketing/kdd_CausalMTA_YaoGZCB22, MTS/Attention/icdm_InGRA_ChuWMJZY20}); User-item interaction in recommendation (\eg,~\cite{Chuzhe_app/kais/ShangS20}); User activity modeling (\eg,~\cite{Chuzhe_app/cikm/LiGBC17, Chuzhe_app/yao2022high}). \\  


    
    & Video analysis & Video analysis and reasoning (\eg,~\cite{Applications/video/iclr_YiGLK0TT20, Applications/video/nips_Li0AFG20}); Interpretable Gait Recognition (\eg,~\cite{Applications/video/icpr/BalaziaHSP22}). \\


    & Urban data analysis & Trajectory pattern mining (\eg,~\cite{Applications/traj/jcss/ChuWCFH16, Chuzhe_app/yang2021individual}); Traffic flow prediction (\eg,~\cite{Applications/traffic_flow_pre/li2015robust}); Visual urban and causal analytics (\eg,~\cite{Chuzhe_app/tvcg/DengWXBZXCW22}). \\
    & Clinical data analysis & Causal chain discovery (\eg,~\cite{Chuzhe_app/wei2022granger}); Hypothesis testing (\eg,~\cite{Chuzhe_app/pandey2021multimodal}); Stable causal structure learning (\eg,~\cite{Applications/clinic_data/rahmadi2019finding}). \\
    & Signal processing & Blind source separation (\eg,~\cite{Chuzhe_app/crowncom/TestiFG20, Chuzhe_app/tcom/TestiG21}); Compressed sensing (\eg,~\cite{Chuzhe_app/abs-2210-11420}).   \\ 
    & Financial analysis & Causal discovery for financial news (\eg,~\cite{Chuzhe_app/tetereva2018financial, Chuzhe_app/rambaldi2015modeling}). \\
    & Military & Battlefield sequential events analysis (\eg,~\cite{Chuzhe_app/li2022discover}).   \\ 
    & Robotics and dynamic control systems & Identifying causal structure (\eg,~\cite{Applications/dynamic_control_systems/TMLR_corr_abs-2006-03906}); Causal generalization (\eg,~\cite{Applications/dynamic_control_systems/agi_SheikhlarET21}). \\



    \bottomrule
    \end{tabular}}
    \vspace{-4ex}


\end{table}




























\textbf{Earth science and climate change research:} Temporal causal discovery approaches have been widely used in the community of earth science and climate change research \cite{Applications/climate_attr/kdd_LozanoLNLPHA09, Applications/earthS/climate_ebert2012causal, Applications/earthS/climate_runge2014quantifying, Applications/earthS/climate_change_van2015causal, Applications/earthS/climate_attr_hannart2016causal, Resources/datasets_surveys/runge2019inferring, Applications/anomaly/work3_dagm_TrifunovSRERD19}.
Climate is a complex and chaotic system, incorporating spatio-temporal information.
Traditional climate models based on forward simulations have inherent limitation in describing such system due to uncertainties, simplifications, and discrepancies from observed data \cite{Applications/climate_attr/kdd_LozanoLNLPHA09}. % climate的数据/系统特点,现有方法的短板
Whereas, commonly used data centric methods such as lagged cross-correlation and regression analysis, aiming at deriving insights into interaction mechanisms between climate process, may lead to ambiguous conclusions in the field \cite{Applications/earthS/climate_runge2014quantifying}.  
To overcome the aforementioned issues, it's reasonable to meaningfully characterize causal relationships among parameters of interest and make assertions.
Specifically, spatio-temporal Granger modeling via group elastic net is proposed in \cite{Applications/climate_attr/kdd_LozanoLNLPHA09} to conduct climate change detection and attribution, where the extreme-value theory to model and attribute extreme events in climate, such as severe heatwaves and floods. 
In \cite{Applications/earthS/climate_runge2014quantifying}, a graphical Granger model followed by a causal interaction strength measure is proposed to quantify the strength and delay of climate interactions and overcome the possible artifacts from vanilla correlation or regression methods.  % Begin of challenge 2!!
Another challenge is the existence of unobserved confounders, which may either lead to incorrect attribution or perform as a nonnegligible driving factor.  
A line of work \cite{Applications/anomaly/work3_dagm_TrifunovSRERD19, EarthSci/abnorm/causal_intens/DBLP:conf/dagm/ShadaydehDGM19} detect the latent driving force of abnormal event in climate by estimating the causal link intensity between confounded variables.  % Begin of challenge 3!!
Besides, in climate system some parameters of interest show strong coupling, thus impose difficulties for identification of causal orientation. The convergent cross mapping (CCM) technique, which is designed for strong coupling dynamic systems, is used in \cite{Applications/earthS/climate_change_van2015causal} to identify the causality between temperature and greenhouse gases, between which the statistical association is well documented while the causality is different to extract from the observed data. 
A recent overview of time series causal discovery in the earth system is also provided in \cite{Resources/datasets_surveys/runge2019inferring}, where avenues for future work in both method developments and scientific endeavors are depicted.


\textbf{Industrial temporal anomaly detection:} In industrial systems, detecting anomalies in massive temporal data, which is derived from sensors, logs, physical measurements, system settings, etc, is meaningful while challenging.
The anomalies can be roughly categorized into univariate anomaly, which has been extensively studied, and dependency anomaly, which is much more challenging to detect but common in real-world applications.
As the challenges mainly come from high dimensions and complex dependency in data, methods \cite{Applications/anomaly/work1_icdm_QiuLSL12, Applications/anomaly/work2_icdm_BehzadiHP17, Applications/anomaly/work4_ijdsa_ApteVP21, Applications/anomaly/work5_22, Application/SCGL_use_cikm_HuangXYYWX20} based on temporal causal discovery have played a nonnegligible role in the dependency anomaly detection by providing efficient, robust and interpretable results.
Causal discovery can facilitate the detection of the generative mechanisms of an underlying system.
The key idea of this family of work is first to construct causal graphs from multivariate time series, and then detect anomalies according to the extracted causal relations. 
To be specific, in \cite{Applications/anomaly/work1_icdm_QiuLSL12, Applications/anomaly/work2_icdm_BehzadiHP17}, Granger graphical models are built on a reference data set and a testing data set respectively, the distribution differences (such as KL-divergence and Jensen-Shannon divergence) between the two learned models are computed as anomalous measures.  
In \cite{Applications/anomaly/work4_ijdsa_ApteVP21}, the inferred relation based on Granger causality is termed causally anomalous if it violates the domain knowledge or the frequently observed forms.
Recently, a causal perspective is also taken in \cite{Applications/anomaly/work5_22} to detect multivariate time series anomalies and leveraged in AIOps applications. In this work, the computation cost is reduced because instead of modeling joint distribution directly, it models factorized distribution modules from learned causal structures, where each corresponds to a local causal mechanism.    
Besides, as for the imbalanced flight data where the anomalous data points are rare, a time series classification method is proposed in \cite{Application/SCGL_use_cikm_HuangXYYWX20} based on nonlinear Granger causality learning. 




\textbf{Root cause analysis in manufacturing process:} The root cause analysis is a vital task to ensure process safety and productivity in the industrial context, where the manufacturing processes are temporal and complex scenarios usually composed of multiple process units and a large number of feedback control loops. However, the acceptance of powerful ML methods in this field is hindered due to increasing requirements of fairness, accountability, and transparency (a.k.a., FAT principle \cite{FAT_principle/DBLP:journals/chb/ShinP19}), especially in sensitive-use cases \cite{industiral_app_review/vukovic2022causal}. To alleviate this issue, extracting knowledge such as causal relationships is paramount in this field. The last decade has witnessed the proliferation of the causal discovery methods for root cause analysis \cite{landman2014fault, wang2015data, rashidi2018data, Applications/CPSs/kbs_LiuLJS21, industiral_app_review/vukovic2022causal}. For instance, temporal causal discovery approaches such as Granger causality, transfer entropy, and their variants are leveraged to trace the oscillation propagation in the control loop \cite{landman2014fault, landman2016hybrid, chen2017root, lindner2018diagnosis}. The reduction of alarm flood, which has been recognized as a major cause of industrial incidents, is another aspect of industrial root cause analysis. Among three typical nuisance alarms (i.e., repetitive alarms, standing alarms and consequence alarms \cite{ henningsen1995intelligent}), it's challenging to suppress the consequence alarms and to provide a proper on the condition that the abnormality occurs and propagates. To identify all causal relations between alarms is of help \cite{hollender2007intelligent}, and a line of work \cite{wang2015data, rodrigo2016causal, wunderlich2017structure} leverages causal discovery approaches in this task.
Besides, profound industrial knowledge, such as information flow and energy flow, can be combined with causal discovery to eliminate spurious relations \cite{landman2016hybrid, industrial_knowledge/cao2022causal, thambirajah2009cause}.












