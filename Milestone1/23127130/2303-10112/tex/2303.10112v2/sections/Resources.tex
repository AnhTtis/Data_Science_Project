
\section{Performance Evaluation}\label{sec:res}

In this section, we give an overview of the benchmark datasets and evaluation metrics used in temporal causal discovery.

\subsection{Datasets}

We will briefly introduce some of the datasets used in temporal causal discovery, including MTS datasets and event-sequence datasets.



Datasets for MTS causal discovery range from health data to financial data. We discuss some of the commonly used datasets, which are publicly available and with the ground truth of causal graphs.
\begin{itemize}
    % 是模拟数据 % 引用此文: Extensive chaos in the lorenz-96 model
    \item \textbf{Lorenz-96 simulated data}: It's a nonlinear model formulated in~\cite{lorenz1996predictability} to simulate climate dynamics. The continuous dynamics in a $d$-dimensional Lorenz model are given by $    \frac{\partial \mathbf{x}_i^t}{\partial t} = -\mathbf{x}_{i-1}^t (\mathbf{x}_{i-2}^t - \mathbf{x}_{i+1}^t  ) - \mathbf{x}_i^t + F, \ i \leq i \leq d$. The system dynamics become increasingly chaotic for higher values of forcing constant $F$. As a standard benchmark, it's used by~\cite{MTS/Granger/pamiNGC22, MTS/Granger/iclr20_esru, MTS/Granger/iclr21_GVAR_MarcinkevicsV, MTS/Attention/icdm_InGRA_ChuWMJZY20, MTS/Granger/CR_VAE2023}.

    % \item \textbf{H{\'e}non maps}: 
    % 暂时不用
 
    \item \textbf{Linear VAR simulated data}: Time series measurements are generated according to the linear VAR model. In~\cite{MTS/Granger/pamiNGC22, MTS/Granger/iclr20_esru}, it's used to analyze methods' performance when the true underlying dynamics are linear.

    \item \textbf{CMU Human motion capture (CMU MoCap) data}: It's a data set from CMU MoCap database\footnote{\url{http://mocap.cs.cmu.edu/}}, containing data about joint angles, body position.
    Causal discovery methods can be leveraged to extract nonlinear dependencies between different regions of the body~\cite{MTS/Granger/pamiNGC22}.

    \item \textbf{DREAM-3 in Silico Network Inference Challenge}: In DREAM-3 IN Silico Network Challenge~\cite{prill2010towards}, time-series data is simulated using continuous gene expression and regulation dynamics.
    Five gene regulation networks are to be inferred from gene expression level trajectories recorded.
    This dataset has been used to evaluate causal discovery algorithms in~\cite{MTS/Granger/pamiNGC22, MTS/Granger/iclr20_esru}.


    \item \textbf{Blood-oxygenation-level dependent (BOLD) imaging data}: In this dataset\footnote{\url{https://www.fmrib.ox.ac.uk/datasets/netsim/index.html}}~\cite{DBLP:journals/neuroimage/SmithMKWBNRW11}, time-ordered samples of the BOLD signals measure different brain regions of interest in human subjects. It's generated using the dynamic causal modeling functional magnetic resonance imaging (fMRI) forward model. In~\cite{MTS/Granger/iclr20_esru, MTS/Attention/TCDF_NautaBS19}, causal discovery methods are applied to estimate the connections in the human brain based on BOLD imaging data.

    \item \textbf{Simulated financial time series}: The dataset\footnote{\url{http://www.skleinberg.org/data.html}}~\cite{MTS/KleinbergS/kleinberg2013causality} is created using factor model to describe portfolio's return depending on three factors and a portfolio-specific error term. Thus the true relationships are known. It's used by~\cite{MTS/Attention/TCDF_NautaBS19}.
    % \item 
% \begin{equation}

%     \nonumber
% \end{equation}

\end{itemize}

As for event sequences, datasets range from online behavior to electricity.
However, the true information on causal relationships is not accessible under all scenarios.

\begin{itemize}

    \item \textbf{MemeTracker}: 
    It is a dataset\footnote{\url{http://memetracker.org}} that captures online articles' website, publication time, and all the hyperlinks within. This data set originally represents how a meme flow on different websites. The domain of the website and the publication time are considered an event type and its occurring time. And the hyperlinks between different websites can be seen as the ground truth of causal relationships. It's used by~\cite{pmlr-v70-achab17a, NEURIPS2018_aff0a6a4, pmlr-v119-zhang20v}.

    \item \textbf{IPTV viewing records}: 
    This dataset\cite{6717182} records the user's viewing behavior, i.e., what program and when they watch in the IPTV systems. The type of program and the time of watching the program can be deemed as an event type and its occurring time, respectively. It's used by~\cite{pmlr-v48-xuc16, CHEN202222, pmlr-v119-zhang20v}. However, ground-truth causal relationships are not included in this dataset.

    \item \textbf{Power grid failure event data}: 
    This dataset includes abrupt changes in the voltage or current signals within Phasor Measurement Units (PMUs) as well as each PMU's ID. The mission of the causal diagnosis task with this dataset is to infer the causalities within the grid~\cite{NEURIPS2021_15cf7646}. Since the network topology is not given out of privacy concerns, this is a non-ground-truth task.

    \item \textbf{G-7 bonds}: 
    This dataset\cite{https://doi.org/10.1002/jae.2585} includes the daily return volatility of sovereign bonds of countries in the Group of Seven. The goal of dealing with this dataset is to discover the causal network underneath sovereign bonds~\cite{Schindler_Plant_2022}. Expert knowledge from the domain can be deemed as ground truth.
\end{itemize}





\subsection{Evaluation Metrics}


In this part, we will explain different metrics used in the literature.
Given the inferred probability of an edge $p(A_{ij})$ thresholded by $thre \in (0,1)$, the set of ground truth edges in causal graph $E_{GT} = \{(i,j): A_{ij}^* = 1\} $, and the set of ground truth missing edges in causal graph $E_{MS} = \{(i,j): A_{ij}^* = 0\} $, the definition and description of commonly used metrics is provided as follows:
\begin{itemize}
    \item \textbf{True Positive Rate (TPR)}: As a ratio of common edges found in the causal discovery results and the ground truth adjacencies over the total number of ground truth edges, the TPR metric is defined as $TPR = \frac{ | \{ (i,j): p(A_{ij}) \geq thre  \}   \cap     E_{GT} |     }{ | E_{GT} |  } $.
    \item \textbf{False Positive Rate (FPR)}: Similar to that in TPR, FPR refers to the ratio of common edges found in the causal discovery results and the ground truth missing adjacencies over the number of ground truth missing edges, which is defined as $FPR = \frac{ | \{ (i,j): p(A_{ij}) \geq thre  \}   \cap     E_{MS} |     }{ | E_{MS} |  }   $.
    \item \textbf{Area Under the Receiver Operator Curve (AUROC)}: The Receiver Operator Curve (ROC) is defined as the ratio of TPR and FPR given the threshold $thre$ varies between $0$ and $1$. The area under the ROC (AUROC) is then widely used to assess the performance of causal discovery algorithms.
    \item \textbf{Structural Hamming Distance (SHD)}: SHD is a metric describing the number of edge edition that need to be made to turn the discovered graph to its ground truth counterpart, which sums the number of missing edges, extra edges, and incorrect edges.   
\end{itemize}




% Event stream


