\section{Conclusion}\label{sec:conc}

Causal discovery in temporal data is fundamental to understanding the dynamics and estimating the causal effects of interest. 
This article reviews two categories of temporal causal discovery: multivariate time series causal discovery, and event sequence causal discovery. 
Multivariate time series causal discovery can be categorized into four groups, including constraint-based, score-based, FCM-based, and Granger causal model. 
Main ideas and recent advances for each type are reviewed.
For causal discovery in event sequence, we can classify these algorithms into constraint-based, score-based, and Granger causal models, which are in accordance with multivariate time series causal discovery. 
We note that Granger causal models are especially well-developed for event sequence due to a natural match-up between Granger causality and Hawkes processes.
To bridge the gap between abundant temporal causal discovery algorithms with real-world impacts, we introduce several major studies including scientific endeavors and industrial implementations.
We also provide an extensive list of resources, including datasets and metrics, which can be used as a guideline for future research in this field.
Whilst many algorithms are offered with theoretical or empirical guarantees, the quality of the inferred relations is dependent on many issues, including non-stationarity, heterogeneity, unobserved confounders, subsampling and expert knowledge.
We discuss these challenges and practical considerations.
Lastly, we introduce new perspectives of causal discovery, where avenues for future work in amortized modeling, supervised learning, and causal representation learning are depicted.












