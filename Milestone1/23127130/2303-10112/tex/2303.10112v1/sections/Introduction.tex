
\section{Introduction}

Temporal data recording the status changing of complex systems is widely collected by different application domains, such as social networks, bioinformatics, neuroscience and finance, \etc. As one of the most popular data structural, temporal data consists of attribute sequences ordered by time. Owing to the rapid development of sensors and computing devices, research works on temporal data analysis are emerging in recent years. Different approaches have been proposed for different tasks such as classification\cite{ismail2019deep, ratanamahatana2004making}, clustering\cite{aghabozorgi2015time, liao2005clustering}, prediction\cite{weigend2018time}, causal discovery\cite{intro/ts_surveys/bivariate_comparative_edinburgh2021causality, intro/ts_surveys/bivariate_comparative_PRE_krakovska2018comparison}, \etc. 

Among these tasks, causal discovery recognizing the causal relations between many temporal components has become a challenging yet critical task for temporal data analysis. The learned causal structures could be beneficial for explaining the data generation process and guiding the design of data analysis methods. According to whether the data is calibrated, the temporal data for causal discovery can be categorized into two groups, \ie, multivariate time series (MTS) and event sequences. Therefore, existing causal discovery methods can also be divided into two groups respectively. In this survey, we aim to provide a thoughtful overview and summarize the frontiers of temporal data causal discovery.


MTS data, describing the calibrated states of multiple variables changing over time, is a general kind of temporal data in many domains. Discovering causal relations for MTS could be beneficial to the explainability and robustness of data analysis models. However, the definitions of causal relations are not unique, leading to different solutions.
Accordingly, existing works can be grouped into four categories, \ie, constraint-based methods, score-based methods, functional causal model (FCM)-based methods and Granger causality methods. Besides, there also exist some new perspectives such as Takens' causality and differential equations. In this paper, we will specify the main idea and recent advances for each category.


Another task discussed in this survey is the causal discovery in event sequences, which infers causal relationships within irregularly and asynchronously observed time series. Specifically, it takes a sequence of different events as the input and outputs a causal graph representing the causal interactions between different events. This task is of great importance since most real-world events cannot emerge within a fixed time interval. In accordance with the MTS task, we classify the corresponding methods into three main categories: constraint-based, score-based, and Granger causality-based methods. Among these three categories, Granger causality-based methods, especially Granger causality-based Hawkes process models, are well-developed since a natural match-up exists between Granger causality and Hawkes processes. We will further describe these approaches in detail within this review.



Recently, many surveys \cite{intro/nonts_surveys/glymour2019review, intro/surveys_guoruocheng/csur/GuoCLH020, intro/nonts_surveys/Dya21, intro/nonts_surveys/XJTU22, intro/ts_surveys/Moraffah21, intro/ts_surveys/AliGranger21, intro/ts_surveys/AssaadDG22, intro/nonts_surveys/BN21, intro/DL_surveys/corr/abs-2211-03374, intro/surveys/CSL/heinze2018causal} have been published to summarize the progress of casual discovery. 
We compared the representative reviews and their highlight points in Table \ref{tab:surveys_overview}. As shown, these surveys fall into two lines. Research works in the first line \cite{intro/nonts_surveys/glymour2019review, intro/nonts_surveys/Dya21, intro/surveys_guoruocheng/csur/GuoCLH020, intro/nonts_surveys/XJTU22} discuss the general casual discovery problem in different perspectives. For example,
\cite{intro/nonts_surveys/glymour2019review} provide a brief review of the computational causal discovery methods. % including constraint based, score based, and functional causal model based methods,  (这句本来是在as well as 之前,但似乎没必要,所以不加了)
\cite{intro/nonts_surveys/Dya21} focus on the flurry developments of continuous optimization approaches. 
% instead of earlier work that covers combinatorial approaches to causal discovery.
To handle big data, both causal inference and causal discovery methods based on machine learning are introduced in \cite{intro/surveys_guoruocheng/csur/GuoCLH020}. 
Moreover, deep learning causal discovery methods are reviewed in different variable paradigms \cite{intro/nonts_surveys/XJTU22}, where the causal relations in data are discussed from a broader perspective. In these papers, temporal data was taken as one special application and many data-specified methods are not included.
The surveys in the second line focus on temporal data casual discovery. As illustrated in Table \ref{tab:surveys_overview}, causal discovery methods for bivariate time series are reviewed in \cite{intro/ts_surveys/bivariate_comparative_edinburgh2021causality, intro/ts_surveys/bivariate_comparative_PRE_krakovska2018comparison}. 
The approaches for causal inference in time series are recently reviewed in \cite{intro/ts_surveys/Moraffah21, intro/ts_surveys/AliGranger21}.
The recent work \cite{intro/ts_surveys/AssaadDG22} discusses and comparatively evaluates the existing solutions of time series causal discovery. Nevertheless, causal discovery methods for event sequences are ignored in these reviews.  In this paper, we not only provide a thoughtful overview of causal discovery methods of the two kinds of temporal data but also give an analysis of the connections and differences between them.


\renewcommand{\thefootnote}{\fnsymbol{footnote}}

\begin{table}[h]
    \tiny %\scriptsize %\footnotesize  %\small
    \label{tab:surveys_overview}
    \caption{Highlights of existing reviews on causal discovery.}
    \centering
    \begin{threeparttable} % add @ 20230104
    \resizebox{\textwidth}{14mm}{ %14mm is ok
        \begin{tabular}{c|ccccc|c|l}
           \toprule
           \multirow{2}{*}{Reviews} & \multicolumn{5}{c|}{Multivariate Time-series} & \multirow{2}{*}{Event Sequence} & \multirow{2}{*}{Highlights} \\
            & Constrain-based & Score-based & FCM-based & Granger & Deep Learning & & \\
    
        %    \multirow{2}{*}{Reviews} &
        %    \multicolumn{5}{c|}{Causal Discovery in MTS} &
        %    \multirow{2}{*}{Causal Discovery in Event Stream}
        %    \multirow{2}{c}{Reviews} & \multicolumn{5}{c}{Multivariate Time-series} &\multirow{2}{c}{Event Stream} &\multirow{2}{c}{Highlights}\\
        % %    &Constrain-based & Score-based & FCM-based & Granger & Deep Learning & \multicolumn{2}{c|}{} \\
        %     &Constrain-based & Score-based & FCM-based & Granger & Deep Learning  & &  \\
           \midrule
            % Review of Causal Discovery Methods Based on Graphical Models
            \cite{intro/nonts_surveys/glymour2019review} & No\footnotemark[1] & No\footnotemark[1] & No\footnotemark[1] & Yes & No & No & An overview for causal discovery methods with practical issues and insightful guidelines \\  % Computational causal discovery methods in the last three decades with practical issues and insightful guidelines
            % A Survey of Learning Causality with Data: Problems and Method 
            \cite{intro/surveys_guoruocheng/csur/GuoCLH020} & No\footnotemark[1] & No\footnotemark[1] & No\footnotemark[1] & No & No\footnotemark[1] & No & Causal discovery methods dealing with big data (high-dimensional, mixed data) are reviewed   \\   %Methods based on machine learning to deal with big data
            % D’ya like DAGs? A Survey on Structure Learning and Causal Discovery 
            \cite{intro/nonts_surveys/Dya21} & No\footnotemark[1]& No\footnotemark[1] & No\footnotemark[1] & Yes & No\footnotemark[1] & No & A more extensive coverage of continuous optimization approaches compared to other surveys \\  % claim其他survey多着眼于combinatoric approaches % A more extensive coverage of continuous optimization approaches
            % A Review and Roadmap of Deep Learning Causal Discovery in Different Variable Paradigms 
            \cite{intro/nonts_surveys/XJTU22} & No\footnotemark[1] & No\footnotemark[1] & No\footnotemark[1] & No & No\footnotemark[1] & No & A wider concept of deep learning causal discovery methods is introduced \\  % An exploration of deep learning causal discovery methods from the perspective of variable paradigms
            % Causal Inference for Time series Analysis: Problems, Methods and Evaluation
            \cline{1-8}
             \cite{intro/ts_surveys/Moraffah21} & Yes & No & Yes & Yes & No & No & The first survey covers the current progress to analyze time series from a causal perspective  \\ %  Methods, metrics and datasets for time series with in-depeth insights
            % Granger Causality: A Review and Recent Advances 
            \cite{intro/ts_surveys/AliGranger21} & No & No & No & Yes & Yes & No\footnotemark[2] & Recent advances including network-form and more general notions of Granger causality  \\
            % Survey and evaluation of causal discovery methods for time series 
            \cite{intro/ts_surveys/AssaadDG22} & Yes & Yes& Yes& Yes& No& No & A recent and comprehensive review for causal discovery in time series with comparative evaluations \\
            Ours & Yes & Yes & Yes& Yes& Yes& Yes&  A systematic review of causal discovery in both MTS and event sequence, with new perspectives \\
           \bottomrule
        \end{tabular}
    }
    % \begin{tablenotes} %\tnote{}
    %     \tiny %\footnotesize
    %     \item[*] Entries correspond to methods reviewed which are mainly for non-temporal settings.
    %     \item[**] Mainly about mining Granger causal interactions in point process. \textcolor{red}{(Mainly about the Hawkes process? To edit?)}
    %   \end{tablenotes}

    \end{threeparttable}  % add @ 20230104
\end{table}

Next, we first introduce the background and preliminary of the casual discovery problem in Section \ref{sec:prem}. The recent progress of causal discovery in MTS and event sequences are specified in Section \ref{sec:mts} and Section~\ref{sec:event} respectively. After that, we provide an overview of the applications of temporal data causal discovery in Section \ref{sec:app} and summarize the available resources in Section \ref{sec:res}. At last, we discuss the limitations and new perspectives of recent temporal data causal discovery methods in Section \ref{sec:discuss}.






\footnotetext[1]{Entries correspond to methods reviewed which are mainly for non-temporal settings.}
\footnotetext[2]{Mainly about causalities related to the Hawkes process. }

\renewcommand{\thefootnote}{\arabic{footnote}}


