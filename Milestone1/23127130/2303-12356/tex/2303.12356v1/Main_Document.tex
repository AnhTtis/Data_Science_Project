
\documentclass{jfm}
\usepackage{color}
\usepackage{graphicx}
%\usepackage{epstopdf,epsfig}
%\usepackage{bm}
\usepackage{newtxtext}
\usepackage{newtxmath}
\usepackage{natbib}
\usepackage{hyperref}
\hypersetup{
    colorlinks = true,
    urlcolor   = red,
    linkcolor = blue,
    citecolor = blue,
    filecolor = black,
}
\linespread{1.25}
\newtheorem{lemma}{Lemma}
\newtheorem{corollary}{Corollary}
\newcommand{\RomanNumeralCaps}[1]


% {\MakeUppercase{\romannumeral #1}}

\title{Plasma kinetics: Discrete Boltzmann modelling and Richtmyer-Meshkov instability}

\author{Jiahui Song\aff{1,2},
  Aiguo Xu\aff{2,3,4}
  \corresp{\email{Xu\_Aiguo@iapcm.ac.cn}},
  Long Miao\aff{1}
  \corresp{\email{miaolong@bit.edu.cn}},
  Feng Chen\aff{5},
  Zhipeng Liu\aff{6},
  Lifeng Wang\aff{2,3},
  Ningfei Wang\aff{1},
 \and Xiao Hou\aff{1}}

\affiliation{\aff{1}School of Aerospace Engineering, Beijing Institute of Technology, Beijing, 100081, PR China
\aff{2}Laboratory of Computational Physics, Institute of Applied Physics and Computational Mathematics, Beijing 100088, PR China
\aff{3}HEDPS, Center for Applied Physics and Technology, and College of Engineering, Peking University, Beijing 100871, PR China
\aff{4}State Key Laboratory of Explosion Science and Technology, Beijing Institute of Technology, Beijing 100081, PR China
\aff{5}School of Aeronautics, Shandong Jiaotong University, Jinan 250357, PR China
\aff{6}Department of Physics, School of Science, Tianjin Chengjian University, Tianjin, 300384, PR China}

\begin{document}
\maketitle

\begin{abstract}
A discrete Boltzmann model (DBM) for plasma kinetics is proposed. The DBM contains two physical functions. The first is to capture the main features aiming to investigate and the second is to present schemes for checking thermodynamic non-equilibrium (TNE) state and describing TNE effects. For the first function, mathematically, the model is composed of a discrete Boltzmann equation coupled by a magnetic induction equation. Physically, the model is equivalent to a hydrodynamic model plus a coarse-grained model for the most relevant TNE behaviors including the entropy production rate. The first function is verified by recovering hydrodynamic non-equilibrium (HNE) behaviors of a number of typical benchmark problems. Extracting and analyzing the most relevant TNE effects in Orszag-Tang problem are practical applications of the second function. As a further application, the Richtmyer-Meshkov instability with interface inverse and re-shock process is numerically studied. It is found that, in the case without magnetic field, the non-organized momentum flux shows the most pronounced effects near shock front, while the non-organized energy flux shows the most pronounced behaviors near perturbed interface. The influence of magnetic field on TNE effects shows stages: before the interface inverse, the TNE strength is enhanced by reducing the interface inverse speed; while after the interface inverse, the TNE strength is significantly reduced. Both the global averaged TNE strength and entropy production rate contributed by non-organized energy flux can be used as physical criteria to identify  whether or not the magnetic field is sufficient to prevent the interface inverse. 
\end{abstract}

\begin{keywords}
plasma kinetic; magnetohydrodynamic; Richtmyer-Meshkov instability
\end{keywords}

{\bf MSC Codes }  {\it(Optional)} Please enter your MSC Codes here
%%%%%%%%%%%%%%%%%%%%%
\section{Introduction}
\label{Introduction}
As the fourth state of matter, plasma exists extensively in natural and various industrial fields such as initial confinement fusion (ICF) \citep{betti2016Inertial,abu2022lawson}.  
In ICF, the target pellet is driven by strong laser or x ray, which ionizes the shell ablator material and forms strong shock waves to compress the fuel centrally to a high temperature and density of the ignition state.
When the imposing shock wave passes through a perturbed interface, the perturbation amplitude increases with time, resulting in the mix of ablator material into the fuel plasmas and even causing ignition failure \citep{brouillette2002richtmyer}.
This phenomenon was first investigated theoretically by \citet{Richtmyer1960} through linear stability analysis, then qualitatively verified by \citet{meshkov1969instability} through shock tube experiments.
Now, it is generally referred to as the Richtmyer-Meshkov instability (RMI).
As a physical phenomenon that is bound to occur when certain conditions are met, the RMI also plays important roles in astrophysics \citep{arnett2000role,sano2021laser}, shock wave physics, combustion \citep{khokhlov1999numerical} and other kinds of systems \citep{zhou20171,zhou20172,zhou2021rayleigh}. 

Due to the importance and complexity of RMI, extensive theoretical, experimental and numerical studies have been carried out \citep{zhai2011evolution,xu2016complex,lei2017experimental,zhou20171,zhou20172,wang2017theoretical,zhai2018review}.
Among them, plasma RMI is one of the important research areas.
Previous theoretical and numerical studies for plasma RMI are mainly based on two kinds of models. 
The first kinds are various magnetohydrodynamics (MHD) models which simultaneously couple Euler/Navier-Stokes (NS) equations with Maxwell equations. 
Based on MHD models, the effects of applied magnetic fields and self-generated electromagnetic fields on the development of plasma RMI have been extensively studied \citep{samtaney2003,wheatley2005,wheatley2005stability,cao2008effects,qiu2008effects,wheatley2009richtmyer,Sano2012,
sano2013,wheatley2014transverse,mostert2015,mostert2017magnetohydrodynamic,bond2017richtmyer,zhang2020numerical,
qin2021richtmyer,li2022linear,bakhsh2022linear,zhang2023suppression,tapinou2023effect}.
It should be noted that, the MHD models based on continuum hypothesis are only valid when Knudsen number $Kn$ (defined as the ratio of molecular mean free path $l$ to a characteristic length scale $L$) is sufficient small. 
However, the local $Kn$ near the imposing plasma shock \citep{vidal1993ion,liu2022discrete,bond2017} and perturbed interface is generally so large that it challenges the validity of the continuum assumption \citep{mcmullen2022}, resulting in inaccurate predictions of physical quantities \citep{gan2013lattice,zhang2019entropy,zhang2019discrete,qiu2021,gan_xu2022}.
Besides, a considerable part of the MHD models neglect the particle collision, while \citet{robey2004effects,rinderknecht2018,2020Cai-kinetic-effects,Yao2020Kinetic,cai2021hybrid,2021Shan-kinetic-effects} point out that the kinetic effects caused by particle collisions have the potential to impact the ICF.
In order to address the above limitations, the second kinds of models based on the non-equilibrium statistical physics are promising, which are often referred to as the kinetic methods. 
Boltzmann equation is one of the most fundamental and widely used kinetic equations. 
Based on Boltzmann equation, kinetic methods which consider particle collision have been rapidly developed, such as Direct simulation Monte Carlo (DSMC) \citep{bird1998recent}, 
gas-kinetic unified algorithms (GKUA) \citep{li2015rarefied}, 
unfined gas-kinetic scheme (UGKS) \citep{xu2010unified,liu2017unified},
lattice Boltzmann method (LBM) \citep{chen1991lattice,wagner1998breakdown,succi2001lattice,sofonea2003viscosity,dellar2002lattice,sofonea2004finite,
pattison2008progress,li2016lattice,li2012additional,ledesma2014lattice,chen2018highly,wang2020simplified,de2021double,
de2021one,huang2021transition,czelusniak2022shaping}, particle-in-cell (PIC) \citep{ji2013closure,asahina2017validation}, 
hybrid fluid-PIC \citep{cai2021hybrid} and Vlasov-Fokker-Planck (VFP) method \citep{larroche2016ion,keenan2017deciphering}, etc. Some of these methods have provided a promising solution for studying the RMI \citep{meng2019modeling,kumar2020viscous,liu2020contribution,yan2021ion}.

When considering particle collisions, the temporal and spatial scales in plasma RMI system are more abundant, forming complex hydrodynamic non-equilibrium (HNE) and thermodynamic non-equilibrium (TNE) behaviors. 
The HNE refers to the non-equilibrium that could be described by hydrodynamic theory. 
The TNE refers to the non-equilibrium described by kinetic theory and due to that the distribution function $f$ deviates from its corresponding equilibrium distribution function $f^{eq}$. 
It is clear that the HNE is only a small portion of the TNE \citep{xu2022complex}. 
In fact, the insufficient particle collision frequency causes the system to deviate from its local thermodynamic equilibrium state, leading to complex non-equilibrium behaviors  such as diffusion, viscosity and heat conduction. 
Thus, the accurate extraction and analysis of TNE is of substantial importance for evaluating the kinetic effects.

From the point view of continuum, the $Kn$ can be regarded as the rescaled averaged particle spacing, which characterizes the degree of non-continuum (or discreteness). 
From the point view of TNE, the $Kn$ can be regarded as the rescaled thermodynamic relaxation time, which characterizes the degree of TNE of the system.
Thus, for the shock wave and perturbed interface where $Kn$ is high in plasma RMI, the system may deviate significantly from its local thermodynamic equilibrium state. 
When the kinetic effects are strong, without considering TNE, the physical quantities such as density, temperature, pressure, heat flux and viscous stress will have significantly deviation \citep{lin2019discrete,chen2020morphological,gan2018discrete}, and the heat flux and viscous stress may even be wrong in direction \citep{zhang2019discrete}.
\emph{Even for the case where the $Kn$ is small enough and consequently the NS description is reasonable, the NS shows deficiency or inconvenice for investigating some TNE behaviors, such as how the various TNE mechanisms influence the entropy production rates.}
Nowadays, TNE is attracting more attention with time \citep{chen2017,qiu2020,qiu2021,bao2022}, but is still far from being fully understood. 

In order to extract and analyze TNE, the deviation between $f$ and $f^{eq}$ needs to be quantified. 
According to Chapman-Enskog (CE) multi-scale analysis, the macroscopic hydrodynamic equations can be regarded as the conservation moments equation of Boltzmann equation near the continuum and thermodynamic equilibrium state.
Since the current MHD models and mesoscopic kinetic methods mentioned above only consider the conservation moments of $f$, which corresponding to mass, momentum and energy conservation, many TNE information which are helpful for physical understanding the underlying fluid kinetic are not available through the above MHD and kinetic methods. 
In recent years, Xu's group  \citep{Xu2018-Chap2,lai2016nonequilibrium,chen2016viscosity,lin2017discrete,chen2018collaboration,ye2020knudsen,
chen2020morphological,lin2021multiple,xu2021progress,xu2021Progressofmesoscale,xu2021modeling,chen2022discrete,chen2022effects,
li2022influence} has developed a new physical modelling and analysis method, that is the discrete Boltzmann method (DBM). 
Besides being able to recover the corresponding hydrodynamic model described by the Navier-Stokes, Burnett or super-Burnett equations, the core of the second function of DBM is to use the physical quantities defined by non-conserved moments of $(f-f^{eq})$ to extract TNE effects. 
The non-conserved moments here refer to higher-order kinetic moments other than mass, momentum and energy moments. 
The starting point of DBM and very basic understanding are that with increasing the non-continuity and TNE, the complexity of system behaviors increases sharply, and more physical quantities are needed to describe the state and behaviors. 
According to the CE theory, \emph{with the increase of $Kn$, the non-significant decrease of the system state and behavior description function requires the introduction of more non-conserved moments} \citep{Xu2018-Chap2,xu2021progress,xu2021Progressofmesoscale,xu2021modeling}.
Until now, nearly all existing DBM methods are for neutral fluids \citep{lai2016nonequilibrium,chen2016viscosity,lin2017discrete,chen2018collaboration,ye2020knudsen,
chen2020morphological,lin2021multiple,chen2022effects,li2022influence,chen2022discrete,Li_2022}, while the temporal and spatial scales in the plasma RMI are more abundant, forming complex HNE and TNE behaviors.

In this paper, the DBM coupling the effects of electromagnetic fields is constructed and further applied to study the RMI in plasma system. 
The paper is arranged as follows. 
The state of art for plasma kinetics is reviewed in Section I, where two problems, (i) how to simulate and (ii) how to analyze the kinetic behaviors, are addressed. 
Specifically, the two problems are as follows: (i) how to simulate the main kinetic features aiming to investigate, and (ii) how to check the TNE state, describe the corresponding effect and extract more information having potential applications. 
The framework for constructing a singe-distribution-function (SDF) DBM is formulated in Section II. 
The SDF DBM corresponds to the single-fluid MHD model where only one set of hydrodynamic quantities are used. 
The DBM is beyond the MHD model from the sense that it has more physical functions than the latter.   
In Section III, some HNE behaviors of a number of typical benchmark problems are used to validate the first function of the DBM, and then the TNE behaviors of Orsazg-Tang vortex problem are studied.
In Section IV, the HNE and TNE behaviors of RMI with and without initial applied magnetic field are investigated. 
At last, the conclusion and discussions are given in Section V.
%%%%%%%%%%%%%%%%%%%%%%%%%%%%%%%%
\section{Physical models}
\label{Physical models}
In this section, the differences in research ideas between DBM and LBM are first discussed. 
Then, three main steps for constructing DBM are presented in detail, including two coarse-grained physical modelling steps and a step to extract and measure non-equilibrium effects. 
Furthermore, we introduce a model describing the evolution of the electromagnetic field and explain in detail how this model and the DBM are coupled. 
\subsection {Discrete Boltzmann Model}
LBM is one of the fastest-growing and most widely-used kinetic methods in recent years \citep{succi2001lattice}. 
The LBM studies have two different branches. 
The first branch aims to work as a novel discrete format for numerical solving hydrodynamic equations or other kinds of partial differential equations (PDE). 
The second aims to work as a new method for constructing kinetic models of complex flows\citep{succi2001lattice,xu2022complex,gan_xu2022}.
The vast majority of LBM methods in the current literature belong to the first branch.
As a numerical solution method, the first branch of LBM should be loyal to the corresponding original model. 
The numerical accuracy and efficiency are its main concerns. 
At present, a variety of  LBM models for solving magnetohydrodynamic equations have been developed and used \citep{dellar2002lattice,pattison2008progress,de2021one,de2021double}. 

DBM is a modelling and analysis method developed from the second branch of LBM working as a physical model construction method.
DBM aims at the two problems. 
The first is how to describe, and the second is how to analyse the flow behaviors. 
It breaks through the continuity and near-equilibrium assumptions of traditional macroscopic modelling methods, no longer uses the lattice gas image of standard LBM, and adds methods based on phase space for checking, exhibiting, describing and analyzing the non-equilibrium state and effect. 
A DBM is equivalent to a hydrodynamic model plus a coarse-grained model for the most relevant TNE behaviors. Besides the hydrodynamic model described by the Euler, NS, Burnett equations, etc., \emph{
the appearance of second function of DBM, the coarse-grained model for describing the most relevant TNE effects, is an inevitable requirement for the description and control of non-equilibrium and discrete behaviors.}
As the degree of non-continuity and non-equilibrium increases, more physical quantities are used to describe the state and behavior of the system, which is a typical feature of DBM that differs from traditional modelling methods. 
From the perspective of kinetic macro modelling (KMM), it is a requirement of obtaining the more accurate constitutive relations; 
From the perspective of kinetics theory, it is a requirement of obtaining the more accurate distribution function. 
Different perspectives lead to the same destination \citep{xu2022complex}.
According to the research needs, DBM grasps the main contradiction, selects a perspective, and studies a set of kinetic properties of the system. 
Therefore, it requires that the calculation results of the kinetic moments describing this set of kinetic properties cannot be changed due to model simplification. Then, with time, more schemes are introduced to check the TNE state, extract the TNE information and describe corresponding TNE effects \citep{xu2012lattice,xu2015multiple,xu2021modeling,xu2021phase}.

From Boltzmann equation to DBM, for bulk flow, three basic steps are needed: (i) Modification and simplification of the Boltzmann equation, (ii) discretization of the particle velocity space, (iii) extraction and measurement of non-equilibrium effects. 
Among them, the first two steps are the coarse-grained physical modelling process, which aim is to build a simple physical model with sufficient physical functions to meet the problem research. 
The third step is the core of the second function of DBM, and its purpose is to quickly and effectively extract non-equilibrium effects with the help of non-conserved kinetic moments. 
In addition, it should be emphasized that the above three steps are aimed at the fluid far away from the wall. 
For near-wall fluids where the boundary effects needs to be considered, appropriate kinetic boundary conditions must be introduced \citep{zhang2018discrete,zhang2022non}.

\subsubsection {Modification and simplification of the Boltzmann equation}
As one of the most basic model equations in non-equilibrium statistical physics, the origin Boltzmann equation has the ability to describe the whole flow regimes (from continuous flow to free molecular flow) and different extent of non-equilibrium effects (from quasi-equilibrium to non-equilibrium flows). 
However, the high-dimensional integral collision term on the right hand of Boltzmann equation is too complicated to solve, which brings inconvenience to practical research. 
In order to simplify the collision term, a convenient and effective way is to introduce a equilibrium distribution function $f^{eq}$ and write the Boltzmann collision term into the form of a linearized collision operator as follows,
\begin{equation}
\frac{{\partial f}}{{\partial t}} + {\boldsymbol{v}} \bcdot \frac{{\partial f}}{{\partial {\boldsymbol{r}}}} + {\boldsymbol{a}} \bcdot \frac{{\partial f}}{{\partial {\boldsymbol{v}}}} =  - \frac{1}{\tau }\left( {f - {f^{eq}}} \right),
\label{eq:1}
\end{equation}
where $f$ and $f^{eq}$ are the distribution function and equilibrium distribution function, respectively. 
The variables $\boldsymbol{r}$, $\boldsymbol{v}$, $\boldsymbol{a}$, $t$, $\tau$ are the particle space coordinate, velocity, acceleration caused by external force, time and relaxation time, respectively. 
This simplification method is based on the following physical basis: Assuming that the details and effects of a specific collision have no obvious influence on the main characteristics of the collision term when describing a fluid system, so we can obtain the main characteristics of the collision term by statistically averaging the effects of all collisions. 
Here, the average effect of particle collisions is to make $f$ gradually evolves towards $f^{eq}$, and the speed of this process is controlled by relaxation time $\tau$.
The above form was first proposed by Bhatnagar, Gross, and Krook, so it is also known as Boltzmann-BGK model \citep{bhatnagar1954model}.

At present, BGK-like models such as the BGK model, ES-BGK model, Shakov model, and Rykov model have been widely used in various kinetic methods. 
It should be emphasized that the original BGK model only preserves the mass/momentum/energy conservation and the $H$-theorem, which means the original BGK model only suits for the system that is always in local quasi-equilibrium state. 
However, as $Kn$ increases, small structures and fast modes appear, and real system tends to deviate significantly from the equilibrium state. 
It is clear that the BGK in various kinetic methods is surely not the original BGK.
It is actually a simplified model combining the kinetic theory and mean field theory \citep{xu2022complex,gan_xu2022}.
Here, the BGK model is adopted, and the form of $f^{eq}$ is as follows,
\begin{equation}
{f^{eq}}\left( {\rho ,{\boldsymbol{u}},T} \right)=\rho {\left( {\frac{1}{{2\upi RT}}} \right)^{D/2}}{\left( {\frac{1}{{2\upi nRT}}} \right)^{1/2}}
\exp \left[ { - \frac{{{{\left( {{\boldsymbol{v}} - {\boldsymbol{u}}} \right)}^2}}}{{2RT}} - \frac{{{\eta ^2}}}{{2nRT}}} \right],
\label{eq:2}
\end{equation}
where $\rho$, $\boldsymbol{u}$, $T$, $R$ are density, bulk velocity, temperature and gas constant, respectively. $D$ is the number of space dimension, and $n$ is the number of extra degrees of freedom, with which the specific heat ratio is $\gamma=(D+n+2)/(D+n)$. $\eta$ is a free parameter that describes the energy of extra degrees of freedom including molecular rotation and vibration inside the molecules.

According to the theory of mesoscopic non-equilibrium statistical physics, all the information of system during evolution is contained in the distribution function $f$, and the corresponding macroscopic physical quantities can be obtained by taking the kinetic moments of $f$. 
Here, two sets of kinetic moments are defined as follows:
\begin{equation}
{\boldsymbol{M}}_{m,n}\left( f \right) = \int_{ - \infty }^\infty  {{{\left( {\frac{1}{2}} \right)}^{1 - {\delta _{m,n}}}}\left( f \right)\underbrace {{{\boldsymbol{v}}}{{\boldsymbol{v}}} \cdots {{\boldsymbol{v}}}}_n{{\left( {{{\boldsymbol{v}}} \cdot {{\boldsymbol{v}}}} \right)}^{\left( {m - n} \right)/2}}d{\boldsymbol{v}}},
\label{eq:3}
\end{equation}
\begin{equation}
{\boldsymbol{M}}_{m,n}^*\left( f \right) = \int_{ - \infty }^\infty  {{{\left( {\frac{1}{2}} \right)}^{1 - {\delta _{m,n}}}}\left( f \right)\underbrace {{{\boldsymbol{v}}^*}{{\boldsymbol{v}}^*} \cdots {{\boldsymbol{v}}^*}}_n{{\left( {{{\boldsymbol{v}}^*} \cdot {{\boldsymbol{v}}^*}} \right)}^{\left( {m - n} \right)/2}}d{\boldsymbol{v}}},
\label{eq:4}
\end{equation}
where $\delta_{m,n}$ is the Kronecker delta function. 
The subscript $m,n$ represents the $m$th-order tensors contracted to $n$th-order ones. 
When $m=n$,  ${\boldsymbol{M}}_{m,n}$ and ${\boldsymbol{M}}_{m,n}^*$ are referred to as ${\boldsymbol{M}}_{m}$ and ${\boldsymbol{M}}_{m}^*$. 
${\boldsymbol{v}}^* = {{\boldsymbol{v}}} - {\boldsymbol{u}}$ represents the thermal fluctuation velocity of particles relative to bulk velocity $\boldsymbol{u}$.
Through constructing kinetic moments ${{\boldsymbol{M}}_0}$, ${{\boldsymbol{M}}_1}$, and ${{\boldsymbol{M}}_{2,0}}$ on both sides of Eq~(\ref{eq:1}), the generalized hydrodynamic equations\footnote{where the viscous stress and heat conduction may contain not only the contribution of $f^{(1)}$  , among which the NS viscous stress and NS heat conduction are the simplest cases} could be obtained as follows,
\begin{equation}
\frac{{\partial \rho }}{{\partial t}} + \nabla  \bcdot \left( {\rho {\boldsymbol{u}}} \right) = 0,
\label{eq:5}
\end{equation}
\begin{equation}
\frac{{\partial \rho {\boldsymbol{u}}}}{{\partial t}} + \nabla  \bcdot \left( {\rho {\boldsymbol{uu}} + {{\boldsymbol{\Delta }}_2}} \right) = 0,
\label{eq:6}
\end{equation}
\begin{equation}
\frac{{\partial {E_T}}}{{\partial t}} + \nabla  \bcdot \left[ {{E_T}{\boldsymbol{u}} + {{\boldsymbol{\Delta }}_{3,1}}} \right] = 0,
\label{eq:7}
\end{equation}
where ${E_T} = \rho e + \rho {u^2}/2$ is the total energy, and $e$ is the energy density. ${{\boldsymbol{\Delta }}_{2}}$ and ${{\boldsymbol{\Delta }}_{3,1}}$ are two non-equilibrium quantities. For the convenience of discussion, we first give two kinds of definitions of non-equilibrium quantities, as follows,
\begin{equation}
{\boldsymbol{\Delta }}_{m,n} = {\boldsymbol{M}}_{m,n}\left( {f - {f^{eq}}} \right) = \int_{ - \infty }^\infty  {{{\left( {\frac{1}{2}} \right)}^{1 - {\delta _{m,n}}}}\left( {{f^{(1)}} + {f^{(2)}} +  \cdots } \right)\underbrace {{{\boldsymbol{v}}}{{\boldsymbol{v}}^*} \cdots {{\boldsymbol{v}}}}_n{{\left( {{{\boldsymbol{v}}} \cdot {{\boldsymbol{v}}}} \right)}^{\left( {m - n} \right)/2}}d{\boldsymbol{v}}},
\label{eq:8}
\end{equation}
\begin{equation}
{\boldsymbol{\Delta }}_{m,n}^* = {\boldsymbol{M}}_{m,n}^*\left( {f - {f^{eq}}} \right) = \int_{ - \infty }^\infty  {{{\left( {\frac{1}{2}} \right)}^{1 - {\delta _{m,n}}}}\left( {{f^{(1)}} + {f^{(2)}} +  \cdots } \right)\underbrace {{{\boldsymbol{v}}^*}{{\boldsymbol{v}}^*} \cdots {{\boldsymbol{v}}^*}}_n{{\left( {{{\boldsymbol{v}}^*} \cdot {{\boldsymbol{v}}^*}} \right)}^{\left( {m - n} \right)/2}}d{\boldsymbol{v}}}.
\label{eq:9}
\end{equation}

When $m=n$, the ${\boldsymbol{\Delta }}_{m,n}$ and ${\boldsymbol{\Delta }}_{m,n}^*$ are also referred to as ${\boldsymbol{\Delta }}_{m}$ and ${\boldsymbol{\Delta }}_{m}^*$. 
For the convenience of discussion, ${\boldsymbol{\Delta }}_{m,n}$ is called the non-central moment, and the non-equilibrium information it describes is referred to as thermo-hydrodynamic non-equilibrium (THNE). ${\boldsymbol{\Delta }}_{m,n}^*$ is called the central moment, and the non-equilibrium information it describes is referred to as TNE. Parts of the relationship between THNE and TNE are given in reference \citep{gan_xu2022} as follows,
\begin{equation}
{{\boldsymbol{\Delta }}_2} = {\boldsymbol{\Delta }}_2^*,
\label{eq:10}
\end{equation}
\begin{equation}
{{\boldsymbol{\Delta }}_{3,1}} = {\boldsymbol{\Delta }}_{3,1}^* + {\boldsymbol{\Delta }}_2^* \bcdot {\boldsymbol{u}}.
\label{eq:11}
\end{equation}

Here, ${\boldsymbol{\Delta }}_2^*$ is defined as non-organized momentum flux (NOMF) , and ${\boldsymbol{\Delta }}_{3,1}^*$ is defined as non-organized energy flux (NOEF). 
Compared with the constitutive of viscous stress and heat flux in NS, Burnett and Super-Burnett equations, ${\boldsymbol{\Delta }}_2^*$ and ${\boldsymbol{\Delta }}_{3,1}^*$ contain the most complete constitutive information. 
Thus, ${\boldsymbol{\Delta }}_2^*$ and ${\boldsymbol{\Delta }}_{3,1}^*$ are also referred to as generalized viscous stress and heat flux.
However, the specific forms of ${\boldsymbol{\Delta }}_2^*$ and ${\boldsymbol{\Delta }}_{3,1}^*$ are basically unknown. 
To get the specific forms of ${\boldsymbol{\Delta }}_2^*$ and ${\boldsymbol{\Delta }}_{3,1}^*$, we need to determine the order of $f$ to be retained (or the number of kinetic moments to be preserved) through CE multi-scale analysis. 
The form of CE multi-scale analysis is as follows,
\begin{equation}
f = {f^{(0)}} + Kn{f^{(1)}} + K{n^2}{f^{(2)}} + K{n^3}{f^{(3)}} +  \cdots ,
\label{12}
\end{equation}
\begin{equation}
\frac{\partial }{{\partial t}} = Kn\frac{\partial }{{\partial {t_1}}} + K{n^2}\frac{\partial }{{\partial {t_2}}} +  \cdots ,
\label{13}
\end{equation}
\begin{equation}
\frac{\partial }{{\partial {\boldsymbol{r}}}} = Kn\frac{\partial }{{\partial {{\boldsymbol{r}}_1}}},
\label{14}
\end{equation}
where ${f^{(0)}}={f^{eq}}$. 
Here, the zero-order expansions of the time and space derivatives $\partial /\partial {t_0}$ and $\partial /\partial {\boldsymbol{r}_0}$ are neglected, for the subscript $0$ represents the scale of system and the internal change of system cannot be observed through system scale. 
Besides, the time is expanded to the second-order and the space is expanded to the first-order. 
Through this way, the derived equations are exactly the hydrodynamic equations currently used such as NS and Burnett, etc.
Theoretically, it is also optional to expand time to the first-order and expand space to the second-order. 
As long as the derivation process is correct, the obtained hydrodynamic equations are also correct.
Substituting Eqs.~(\ref{12})-(\ref{14}) into Eq.~(\ref{eq:1}),  and retaining the same order terms of $Kn$ numbers, we get,
\begin{equation}
K{n^1}:\frac{{\partial {f^{(0)}}}}{{\partial {t_1}}} + \frac{{\partial \left( {{f^{(0)}}{\boldsymbol{v}}} \right)}}{{\partial {{\boldsymbol{r}}_1}}} =  - \frac{1}{\tau }{f^{(1)}},
\label{15}
\end{equation}
\begin{equation}
K{n^2}:\frac{{\partial {f^{(0)}}}}{{\partial {t_2}}} + \frac{{\partial {f^{(1)}}}}{{\partial {t_1}}} + \frac{{\partial \left( {{f^{(1)}}{\boldsymbol{v}}} \right)}}{{\partial {{\boldsymbol{r}}_1}}} =  - \frac{1}{\tau }{f^{(2)}}.
\label{16}
\end{equation}

By taking the ${{\boldsymbol{M}}_0}$, ${{\boldsymbol{M}}_1}$, and ${{\boldsymbol{M}}_{2,0}}$ moments simultaneously on both sides of Eq.~(\ref{15}) and use the relation ${f^{(1)}} =  - \tau \left[ {\partial {f^{(0)}}/\partial {t_1} + \partial ({f^{(0)}}{\bf{v}})/\partial {{\bf{r}}_1}} \right]$, the viscous stress and heat flux in NS level can be deduced after performing some mathematical transformations, as follows \citep{gan_xu2022},
\begin{equation}
{\boldsymbol{\Delta }}_{2, NS}^{*}={\boldsymbol{\Delta }}_2^{*(1)} = \int_{ - \infty }^\infty  {{f^{(1)}}{{\boldsymbol{v}}^*}{{\boldsymbol{v}}^*}d{\boldsymbol{v}}}  =  - \mu \left[ {\left( {\nabla {\boldsymbol{u}}} \right) + {{\left( {\nabla {\boldsymbol{u}}} \right)}^{\rm{T}}}} \right],
\label{17}
\end{equation}
\begin{equation}
{\boldsymbol{\Delta }}_{3,1, NS}^{*}={\boldsymbol{\Delta }}_{3,1}^{*(1)} = \int_{ - \infty }^\infty  {\frac{1}{2}{f^{(1)}}{{\boldsymbol{v}}^*} \bcdot {{\boldsymbol{v}}^*}{{\boldsymbol{v}}^*}d{\boldsymbol{v}}}  =  - \kappa \nabla T,
\label{18}
\end{equation}
where $\mu = \tau \rho RT$ is the viscosity coefficient and $\kappa = 2\tau \rho RT$ is the heat conductivity. 
Similarly, the viscous stress ${\boldsymbol{\Delta }}_2^{*(2)}$ and heat flux ${\boldsymbol{\Delta }}_{3,1}^{*(2)}$ contributed by $f^{(2)}$ can be deduced from Eq.~(\ref{16}), and the viscous stress and heat flux in Burnett level are ${\boldsymbol{\Delta }}_{2, Burnett}^{*}={\boldsymbol{\Delta }}_2^{*(1)}+{\boldsymbol{\Delta }}_2^{*(2)}$ and 
${\boldsymbol{\Delta }}_{3,1, Burnett}^{*}={\boldsymbol{\Delta }}_{3,1}^{*(1)}+{\boldsymbol{\Delta }}_{3,1}^{*(2)}$, respectively.
It should be noted that, CE gives the dependence of  $(n+1)$-th order distribution function $f^{(n+1)}$ on the ${n}$-th order distribution function $f^{(n)}$, and finally gives the dependence of $f^{(n+1)}$ on $f^{(0)}$, where $f^{(0)}$ is the only one whose kinetic moments are known and can be relied upon in the modelling process. In the derivation of the first-order constitutive relations in NS, the highest-order kinetic moment used is ${\boldsymbol{M}}_{4,2}$. As for the derivation of the second-order constitutive relations in Burnett, the highest-order kinetic moment used is ${\boldsymbol{M}}_{5,3}$.

There exist two kinds of methods to obtain the macroscopic hydrodynamic equations. The first is the traditional macroscopic direct modelling method which is based on the continuum assumption and near equilibrium approximation. The second is the KMM method, that is, starting from kinetic equations, to derive the corresponding macroscopic hydrodynamic equations via CE multi-scale analysis. 
Due to different research ideas, KMM is divided into two categories. The first category follows the traditional modelling idea which concerns only the evolution equations of the conserved kinetic moments, that are, the density, momentum and energy. The second category realizes the insufficiency of the first category in capturing the system behaviors for the high $Kn$ cases and consequently derives also the evolution equations of the most relevant non-conserved kinetic moments \citep{chen2017}. 
For convenience of description, we refer the model equations derived from the second category of KMM to as extended hydrodynamic equations (EHE)\footnote{where the model equations include not only the evolution equations of density, momentum and energy, but also those of the most relevant non-conserved kinetic moments}.
Currently, most of the KMM studies belong to the first category. 

In contrast, the DBM is a kinetic direct modelling method (KDM). Here `direct' means without needing to know the specific EHE. Because the current DBM is still working in the case where the CE theory is valid, the CE theory is the mathematical guarantee for rationality and effectiveness of DBM.
DBM is responsible for the following two aspects: (i) According to the discrete, non-equilibrium degree (described by Knudsen number), determine the system behavior needs to be grasped from what aspects, so as to determine which kinetic moments must be preserved in the process of model simplification, (ii) based on the non-conserved moments of $(f-f^{eq})$, present as many as possible schemes for the detection, description, presentation and analysis of the TNE state and effect.
For the first aspect, from the perspective of KMM, DBM determines the kinetic moments of $f^{eq}$ that need to be preserved according to the requirements to obtain the more accurate constitutive expressions; from the perspective of kinetic theory, DBM determines the kinetic moments of $f^{eq}$ that need to be preserved according to the requirements to obtain the more accurate distribution function expressions \citep{xu2022complex}.
Obviously, the second category of KMM is closest to DBM in physical function. 

In addition to the equations for the conservation of mass, momentum, and energy, the extended hydrodynamic equations derived through KMM also includes equations describing the evolution of non-equilibrium quantities such as viscous stress and heat flux. 
For example, by integrating ${\boldsymbol{v}}^* {\boldsymbol{v}}^*$ and $\frac{1}{2}{{\boldsymbol{v}}^*} \bcdot {{\boldsymbol{v}}^*}{{\boldsymbol{v}}^*}$ on both sides of Eq.~(\ref{eq:1}) and ignore the external force term, we obtain the following equation \citep{gan_xu2022},
\begin{equation}
\frac{{\partial {\boldsymbol{\Delta }}_2^*}}{{\partial t}} + \frac{{\partial {\bf{M}}_2^*\left( {{f^{(0)}}} \right)}}{{\partial t}} + \nabla  \bcdot \left[ {{\boldsymbol{M}}_3^*\left( {{f^{(0)}}} \right) + {\boldsymbol{M}}_2^*\left( {{f^{(0)}}} \right){\boldsymbol{u}} + {\boldsymbol{\Delta }}_3^* + {\boldsymbol{\Delta }}_2^*{\boldsymbol{u}}} \right] =  - \frac{1}{\tau }{\boldsymbol{\Delta }}_2^*,
\label{19}
\end{equation}
\begin{equation}
\frac{{\partial {\boldsymbol{\Delta }}_{3,1}^*}}{{\partial t}} + \frac{{\partial {\boldsymbol{M}}_{3,1}^*\left( {{f^{(0)}}} \right)}}{{\partial t}} + \nabla  \bcdot \left[ {{\boldsymbol{M}}_{4,2}^*\left( {{f^{(0)}}} \right) + {\boldsymbol{M}}_{3,1}^*\left( {{f^{(0)}}} \right){\boldsymbol{u}} + {\bf{\Delta }}_{4,2}^* + {\boldsymbol{\Delta }}_{3,1}^*{\boldsymbol{u}}} \right] =  - \frac{1}{\tau }{\boldsymbol{\Delta }}_{3,1}^*,
\label{20}
\end{equation}
where Eqs.~(\ref{19}) and (\ref{20}) describe the evolution of viscous stress and heat flux, respectively. 
Therefore, the obtaining of higher-order of TNE quantities such as ${\boldsymbol{\Delta }}_3^{*}$ and ${\boldsymbol{\Delta }}_{4,2}^{*}$ could help us better understand the evolution of constitutive relationships.

In practical applications, if a same-order kinetic moment of $f$ needs to be accurately known, it may be necessary to add the appropriate kinetic moment(s) of $f^{(0)}$ to be preserved according to the dependency given by CE. 
For example, in DBM considering up to the second-order TNE effect, if we need further to accurately know $\boldsymbol{\Delta}_{5,3}^{*}$ or $\boldsymbol{M}_{5,3}^{*}(f)$, then according to Eq.~(\ref{15}), we need only to add $\boldsymbol{M}_{6,4}^{*}(f^{(0)})$ to the list of kinetic moments to be preserved.
%%%%%%%%%%%%%%%%%%%%%%%%%%%%%%%%%%%
\subsubsection {Discretization of the particle velocity space}
The distribution function $f$ in Boltzmann equation is defined in the phase space of particle position, particle velocity and time. In addition to discretizing space and time, it is also necessary to simultaneously discretize particle velocity space. Before discretization, we first assume that the non-equilibrium effects caused by the external force is sufficiently weak, so the $f$ in the external force term can be replaced by $f^{eq}$ and Eq.~(\ref{eq:1}) becomes,
\begin{equation}
\frac{{\partial f}}{{\partial t}} + {\boldsymbol{v}} \bcdot \frac{{\partial f}}{{\partial {\boldsymbol{r}}}} + {\boldsymbol{a}} \bcdot \frac{{\left( {{\boldsymbol{u}} - {\boldsymbol{v}}} \right)}}{{RT}}{f^{eq}} =  - \frac{1}{\tau }\left( {f - {f^{eq}}} \right).
\label{eq:21}
\end{equation}

After further discretizing the particle velocity space, Eq.~(\ref{eq:21}) becomes,
\begin{equation}
\frac{{\partial {f_i}}}{{\partial t}} + {{\boldsymbol{v}}_i} \bcdot \frac{{\partial {f_i}}}{{\partial {\boldsymbol{r}}}} + {\boldsymbol{a}} \bcdot \frac{{\left( {{\boldsymbol{u}}-{{\boldsymbol{v}}_i}} \right)}}{{RT}}f_i^{eq} =  - \frac{1}{\tau }\left( {{f_i} - f_i^{eq}} \right),
\label{eq:22}
\end{equation}
where $f_i$ and $f_i^{eq}$ are the discrete distribution function (DDF) and discrete equilibrium distribution function (DEDF), respectively. $i = 1, \cdots N$, $N$ is the number of discrete velocities. Unlike space and time discretization, the particle velocity can take any value range from $\left( { - \infty , + \infty } \right)$, which means the $f_i$ and $f_i^{eq}$ are not the exact distribution functions and do not have specific physical meaning. 
Here, what DBM gives are the most necessary physical constraints on the DDF and DEDF, that is, the kinetic moments preserved must keep their value unchanged before and after discretization.
In other words, the integral form of kinetic moments preserved is equal to the summation form, as follows,
\begin{equation}
\int {f{\boldsymbol{\Psi }}\left( {\boldsymbol{v}} \right)} d{\boldsymbol{v}} = \sum\limits_i {{f_i}{\boldsymbol{\Psi }}\left( {{{\boldsymbol{v}}_i}} \right)},
\label{eq:23}
\end{equation}
where ${\boldsymbol{\Psi }}\left( {\boldsymbol{v}} \right) = \left[ {1,{\boldsymbol{v}},{\boldsymbol{vv}}, \cdots } \right]$ represents the kinetic moments preserved. According to CE multi-scale analysis, $f$ can be finally expressed by $f^{eq}$ and Eq.~(\ref{eq:23}) is equal to,
\begin{equation}
\int {{f^{eq}}{\boldsymbol{\Psi }}'\left( {\boldsymbol{v}} \right)} d{\boldsymbol{v}} = \sum\limits_i{f_i^{eq}{\boldsymbol{\Psi }}'\left( {{{\boldsymbol{v}}_i}} \right)},
\label{eq:24}
\end{equation}
where ${\boldsymbol{\Psi }}'\left( {\boldsymbol{v}} \right)$ represents the higher order kinetic moments. 

The depth of TNE that DBM could describe varies according to the specific choice of kinetic moments. 
For example, the DBM considering the zero-order TNE effects requires five kinetic moments of $f^{eq}$ (${{\boldsymbol{M}}_0}$, ${{\boldsymbol{M}}_1}$, ${{\boldsymbol{M}}_{2,0}}$, ${{\boldsymbol{M}}_2}$, ${{\boldsymbol{M}}_{3,1}}$). 
The DBM considering up to the first-order TNE effects requires seven kinetic moments of $f^{eq}$ (${{\boldsymbol{M}}_0}$, ${{\boldsymbol{M}}_1}$, ${{\boldsymbol{M}}_{2,0}}$, ${{\boldsymbol{M}}_2}$, ${{\boldsymbol{M}}_{3,1}}$, ${{\boldsymbol{M}}_3}$, ${{\boldsymbol{M}}_{4,2}})$, and DBM considering up to the second-order TNE effects requires nine kinetic moments of $f^{eq}$ (${{\boldsymbol{M}}_0}$, ${{\boldsymbol{M}}_1}$, ${{\boldsymbol{M}}_{2,0}}$, ${{\boldsymbol{M}}_2}$, ${{\boldsymbol{M}}_{3,1}}$, ${{\boldsymbol{M}}_3}$, ${{\boldsymbol{M}}_{4,2}}$, ${{\boldsymbol{M}}_{4}}$, ${{\boldsymbol{M}}_{5,3}}$). The details about the above kinetic moments are listed in Appendix~\ref{appA}.

In this paper, as the first application of the new DBM model for magnetohydrodynamic flow, the first-order TNE effects are considered and analyzed, and the seven kinetic moments to be preserved could be written in matrix form as,
\begin{equation}
{\mathsfbi{C}} \cdot {{\mathsfbi{{f}}^{eq}}} = {{\mathsfbi{{\hat f}}^{eq}}},
\label{eq:25}
\end{equation}
where ${\mathsfbi{C}}$, ${{\mathsfbi{{f}}^{eq}}}$, ${{\mathsfbi{{\hat f}}^{eq}}}$ are the discrete velocity polynomial, discrete equilibrium distribution function and macroscopic quantities in matrix form, respectively. 
In order to determine the specific value of ${{\boldsymbol{f}}^{eq}}$, the discrete velocity model (DVM) needs to be constructed. 
The selection or construction of DVM is highly flexible, which is based on the comprehensive consideration of physical symmetry, stability and computational efficiency, etc. 


\subsubsection{Extraction and measurement of non-equilibrium effects}
The most significant function of DBM beyond traditional hydrodynamic modelling is to extract and measure the non-equilibrium effects based on the non-conserved kinetic moments of $( f - {f^{eq}})$. At present, no matter the macroscopic or mesoscopic kinetic models, what are mainly concerned and used to describe the system are the physical quantities defined in hydrodynamic equations, such as density, temperature, pressure, viscosity and heat conduction. However, such physical quantities are insufficient to describe the system behaviors when the degree of non-continuity/non-equilibrium further increases.

By further investigating the kinetic moments, we find that for the first three conserved moments, the results obtained by integrating ${f^{eq}}$ and $f$ are the same. However, for the non-conserved moments, the results show difference, which constitutes the most basic description of the current TNE state. Based on non-conserved kinetic moments, more TNE information can be investigated such as the distribution of momentum and internal energy in different degrees of freedom, etc. Two kinds of TNE quantities have been defined by Eqs.~(\ref{eq:8})-(\ref{eq:9}).
In this paper, the TNE quantities defined by four non-conserved kinetic moments are shown as follows,
\begin{equation}
{\boldsymbol{\Delta}} _2^* = \sum\limits_i {\left( {{f_i} - f_i^{eq}} \right){\boldsymbol{v}}_i^*{\boldsymbol{v}}_i^*},
\label{eq:28}
\end{equation}
\begin{equation}
{\boldsymbol{\Delta}} _3^* = \sum\limits_i {\left( {{f_i} - f_i^{eq}} \right){\boldsymbol{v}}_i^*{\boldsymbol{v}}_i^*} {\boldsymbol{v}}_i^*,
\label{eq:29}
\end{equation}
\begin{equation}
{\boldsymbol{\Delta}} _{3,1}^* = \frac{1}{2}\sum\limits_i {\left( {{f_i} - f_i^{eq}} \right)\left( {{\boldsymbol{v}}_i^* \cdot {\boldsymbol{v}}_i^* + {\boldsymbol{\eta}} _i^2} \right){\boldsymbol{v}}_i^*},
\label{eq:30}
\end{equation}
\begin{equation}
{\boldsymbol{\Delta}} _{4,2}^* = \frac{1}{2}\sum\limits_i {\left( {{f_i} - f_i^{eq}} \right)\left( {{\boldsymbol{v}}_i^* \cdot {\boldsymbol{v}}_i^* + {\boldsymbol{\eta}} _i^2} \right){\boldsymbol{v}}_i^*} {\boldsymbol{v}}_i^*.
\label{eq:31}
\end{equation}

Here, the ${\boldsymbol{\Delta}} _3^*$ can be regarded as the flux of NOMF, and the ${\boldsymbol{\Delta}} _{4,2}^*$ can be regarded as the flux of NOEF, respectively. 
The first role of ${\boldsymbol{\Delta}} _{3}^*$ and ${\boldsymbol{\Delta}} _{4,2}^*$ is to help  understanding the evolution of stress and heat flux from a more fundamental level, and the second role is to help assessing the necessity of introducing second-order TNE effects in the constitutive relations of KMM \citep{xu2022complex,zhang2022discrete}. 

It is obvious that \emph{roughly equivalent physical function to DBM are the extended hydrodynamic equations obtained by KMM.} In addition to the conservation equations of mass, momentum and energy, the extended hydrodynamic equations also contain the evolution equations describing some higher order kinetic moments such as ${{\boldsymbol{M}}_3^{*}}$ and ${{\boldsymbol{M}}_{4,2}^{*}}$.
The first function of ${\boldsymbol{\Delta}} _{n+1}^*$ is to help  understanding the evolution of ${\boldsymbol{\Delta}} _{n}^*$ from a more fundamental level, and the second function is to help assessing the necessity of introducing the $(n+1)$-th order TNE effects in the constitutive relations of KMM \citep{xu2022complex,zhang2022discrete}.

With the help of the phase space opened by each subcomponent of ${{\boldsymbol{\Delta }}_{m,n}^*}$, the following non-equilibrium strength can be defined,
\begin{equation}
D_{m,n} = \left| {{\boldsymbol{\Delta }}_{m,n}^*} \right|,
\label{eq:32}
\end{equation}
where the square of $\left| {{\boldsymbol{\Delta }}_{m,n}^*} \right|$ is equal to the sum of the squares of the individual components of $ {{\boldsymbol{\Delta }}_{m,n}^*} $. To give a rough description of the total TNE strength, the function in \citet{chen2016viscosity} is also adopted, which is defined as,
\begin{equation}
D_{T} = \sqrt {{{\left| {\boldsymbol{\Delta } _2^*} \right|}^2} + {{\left| {\boldsymbol{\Delta } _{3,1}^*} \right|}^2} + {{\left| {\boldsymbol{\Delta } _3^*} \right|}^2} + {{\left| {\boldsymbol{\Delta } _{4,2}^*} \right|}^2}}.
\label{eq:33}
\end{equation}

Further, the global averaged TNE strength ${{\bar D}_{m,n}}$ and ${{\bar D}_{T}}$ are defined as
\begin{equation}
{{\bar D}_{m,n}} = \frac{1}{{{N_x} \times {N_y}}}\sum {{D_{m,n}}},
\label{eq:34}
\end{equation}
\begin{equation}
{{\bar D}_{T}} = \frac{1}{{{N_x} \times {N_y}}}\sum {{D_{T}}},
\label{eq:35}
\end{equation}
where $N_x$ and $N_y$ are the grid numbers in $x$ and $y$ direction, respectively.
Since any definition of TNE intensity depends on the research perspective, it is helpful to introduce TNE intensity vectors (each component is a non-equilibrium intensity of one perspective) to cross-locate the TNE intensity from multiple perspectives \citep{xu2021modeling,zhang2022discrete}.
Intuitive geometric correspondence is very helpful for grasping complex behaviors.
Therefore, we can further open a phase space based on the components of the TNE intensity vector. Then the TNE intensity vector obtains an intuitive geometric correspondence.

\emph{Entropy generation rate is an important concern in many fields related to compression science, such as shock wave physics, ICF and aerospace.} Conveniently, according to the non-equilibrium quantities defined before, the entropy production rate caused dissipation effects could be measured. The total entropy production rate is defined as \citep{zhang2016kinetic,zhang2019entropy},
\begin{equation}
\frac{{d{S_b}}}{{dt}} = \int {\left( {{\boldsymbol{\Delta }}_{3,1}^* \cdot \nabla \frac{1}{T} - \frac{1}{T}{\boldsymbol{\Delta }}_2^*:\nabla {\boldsymbol{u}}} \right)d{\boldsymbol{r}}},
\label{eq:36}
\end{equation}
where the entropy production rate can be divided by two parts, that is, the part denoted by NOMF and NOEF, respectively. The two parts are defined as,
\begin{equation}
{{\dot S}_{NOMF}} = \int { - \frac{1}{T}{\boldsymbol{\Delta }}_2^*:\nabla {\boldsymbol{u}}d{\boldsymbol{r}}},
\label{eq:37}
\end{equation}
\begin{equation}
{{\dot S}_{NOEF}} = \int {{\boldsymbol{\Delta }}_{3,1}^* \cdot \nabla \frac{1}{T}d{\boldsymbol{r}}}.
\label{eq:38}
\end{equation}

\subsection{Equations of electromagnetic field}
The evolution of electromagnetic field is described by Maxwell equations,
\begin{equation}
\nabla  \cdot {\boldsymbol{E}} = \frac{\rho }{{{\varepsilon _0}}},
\label{eq:39}
\end{equation}
\begin{equation}
\nabla  \times {\boldsymbol{E}} =  - \frac{{\partial {\boldsymbol{B}}}}{{\partial t}},
\label{eq:40}
\end{equation}
\begin{equation}
\nabla  \cdot {\boldsymbol{B}} = 0,
\label{eq:41}
\end{equation}
\begin{equation}
\nabla  \times {\boldsymbol{B}} = {\mu _0}{\boldsymbol{j}} + \frac{1}{{{c^2}}}\frac{{\partial {\boldsymbol{B}}}}{{\partial t}}.
\label{eq:42}
\end{equation}

In this work, several assumptions are introduced to simplify the Maxwell equation as:
(i) The Larmor radius of ion and electron are sufficiently small compared to the characteristic scale of the system, so the charge separation is ignored and the Poisson equation, that is, Eq.~(\ref{eq:39}) degenerate,
(ii) the displacement current is sufficiently small compared to the conduction current, so the second term on the right-hand side of Eq.~(\ref{eq:42}) is ignored,
(iii) the fluid is perfectly conductive and the electric conductivity is infinite, so the generalized Ohm's law is simplified as ${\boldsymbol{E}} =  - {\boldsymbol{u}} \times {\boldsymbol{B}}$.

Physically, the divergence-free constraint Eq.~(\ref{eq:41}) is always preserved if it is initially satisfied. However, the errors caused by long time numerical calculation can break this limit. To preserve the divergence-free constraint, the magnetic field $\boldsymbol{B}$ is represented by the magnetic potential $\boldsymbol{A}$ as ${\boldsymbol{B}} = \nabla  \times {\boldsymbol{A}}$. For two-dimensional simulation, $\boldsymbol{A}$ contains only one component $A_z$. The evolution of ${\boldsymbol{A}}$ can be represented by the following equations,
\begin{equation}
\frac{{\partial {A_z}}}{{\partial t}} =  - {\boldsymbol{u}} \cdot \nabla {A_z} =  - {u_x}\frac{{\partial {A_z}}}{{\partial x}} - {u_y}\frac{{\partial {A_z}}}{{\partial y}}.
\label{eq:43}
\end{equation}

By solving Eq.~(\ref{eq:43}), the divergence-free constraint will automatically hold during the numerical calculation. In order to construct MHD model, Eq.~(\ref{eq:22}) and Eq.~(\ref{eq:43}) needs to be combined. The method is to introduce the Lorentz force in the external force term (the third term on the left-hand side of Eq.~(\ref{eq:22})), and the acceleration $\boldsymbol{a}$ is rewritten as ${\boldsymbol{a}} = {\boldsymbol{j}} \times {\boldsymbol{B}}/\rho$, where ${\boldsymbol{j}} \times {\boldsymbol{B}}$ represent the Lorentz force as,
\begin{equation}
{\boldsymbol{j}} \times {\boldsymbol{B}} = \frac{{{\boldsymbol{B}} \cdot \nabla {\boldsymbol{B}}}}{{{\mu _0}}} - \nabla \left( {\frac{{{B^2}}}{{2{\mu _0}}}} \right),
\label{eq:44}
\end{equation}
where the first term is called the magnetic tension, while the second term is called the magnetic pressure, with which the total pressure is expressed as ${p^{*}} = p + {B^2}/\left( {2{\mu _0}} \right)$. 
Through CE analysis, it is easy to find that the first function of DBM is to recover the magnetohydrodynamic model described by 
the following equations,
\begin{equation}
\frac{{\partial \rho }}{{\partial t}} + \nabla  \bcdot \left( {\rho {\boldsymbol{u}}} \right) = 0,
\label{eq:45}
\end{equation}
\begin{equation}
\frac{{\partial \left(\rho \boldsymbol{u}\right) }}{{\partial t}}+ \nabla  \bcdot \left( {\rho {\boldsymbol{uu}} + p{\boldsymbol{I}}} \right) =  - \nabla  \bcdot {\boldsymbol{P}}' + {\boldsymbol{j}} \times {\boldsymbol{B}},
\label{eq:46}
\end{equation}
\begin{equation}
\frac{{\partial {E_T} }}{{\partial t}} + \nabla  \bcdot \left[ {\left( {{E_T} + p} \right){\boldsymbol{u}}} \right] = \nabla  \bcdot \left( {\kappa \nabla T + {\boldsymbol{P}}' \bcdot {\boldsymbol{u}}} \right) + \left( {{\boldsymbol{j}} \times {\boldsymbol{B}}} \right) \bcdot {\boldsymbol{u}}.
\label{eq:47}
\end{equation}
where ${\boldsymbol{P}}'$ is the viscous stress and $\kappa$ is the heat conductivity.

In summary, the equations used for simulation are Eqs.~(\ref{eq:22}),~(\ref{eq:43}) and~(\ref{eq:44}), and the schemes used for analyze the TNE are given by equations from (\ref{eq:28}) to (\ref{eq:38}).
\emph{As a model construction and TNE analysis method, what DBM presents are the basic constraints on the discrete formats.} The DBM itself does not give specific discrete formats. The specific discrete formats should be chosen according to the specific problem to be simulated.     
%%%%%%%%%%%%%%%%%%%%%
\section{Numerical Validation and Investigation}
After the method formulation, a number of typical benchmark problems such as sod shock tube, thermal Couette flow and the compressible Orszag-Tang (OT) vortex problem are simulated, where the hydrodynamic behaviors are used to validate the plasma DBM. It should be noted that, all physical quantities used in the calculations are nondimensionalized, as shown in Appendix~\ref{appB}.

In this paper, the first-order forward Euler finite difference scheme and the second-order non-oscillatory nonfree dissipative (NND) scheme are used to discrete spatial and temporary derivatives in Eq.~(\ref{eq:22}) and Eq.~(\ref{eq:43}), respectively. The second-order central difference scheme is used to discretize the space derivative in Eq.~(\ref{eq:44}).
Two two-dimensional DVMs are adopted in this paper.  
The DVM used to simulate the benchmark problems has $25$ discrete velocities as shown in figure~\ref{fig:1}(a), and the DVM used to simulate the RMI has $16$ discrete velocities as shown in figure~\ref{fig:1}(b).

The mathematical expressions of these two kinds of DVM are as follows,
\begin{figure}
  \centerline{\includegraphics[width=11cm]{Fig1}}% Images in 100% size
  \caption{Schematics of two kinds of discrete velocity models used in the present paper. (a)D2V25, (b)D2V16. The numbers in the figures represent the index of discrete velocities.}
\label{fig:1}
\end{figure}
\begin{eqnarray}
\renewcommand{\arraystretch}{1.5}
\text{D2V25}:({v_{ix}},{v_{iy}}) = \left\{ \begin{array}{l}
0,\qquad \qquad \qquad \qquad \qquad \qquad \; \! \quad i = 1\\
c[\cos \frac{{(i - 2)\pi }}{2},\sin \frac{{(i - 2)\pi }}{2}],{\rm{\qquad     \quad   \quad   \!  }}i = 2 - 5,\\
2c[\cos \frac{{(2i - 3)\pi }}{4},\sin \frac{{(2i - 3)\pi }}{4}],{\rm{   \qquad \quad \! \!    }}i = 6 - 9,\\
3c[\cos \frac{{(i - 10)\pi }}{2},\sin \frac{{(i - 10)\pi }}{2}],{\rm{   \qquad  \quad  \! \!    }}i = 10 - 13,\\
4c[\cos \frac{{(2i - 11)\pi }}{4},\sin \frac{{(2i - 11)\pi }}{4}],{\rm{  \quad  \quad  \!  }}i = 14 - 17,\\
5c[\cos \frac{{(i - 18)\pi }}{2},\sin \frac{{(i - 18)\pi }}{2}],{\rm{  \quad \quad  \;  \;  }}i = 18 - 21,\\
6c[\cos \frac{{(2i - 19)\pi }}{4},\sin \frac{{(2i - 19)\pi }}{4}],{\rm{ \quad \quad \! }}i = 22 - 25,
\end{array} \right.
\label{eq:27}
\end{eqnarray}
\begin{eqnarray}
\renewcommand{\arraystretch}{1.5}
\text{D2V16}:\left( {{v_{ix}},{v_{iy}}} \right) = \left\{ \begin{array}{l}
cyc:c\left( { \pm 1,0} \right){\rm{,  \quad   \quad  \;    1}} \le i \le 4,\\
c\left( { \pm 1, \pm 1} \right){\rm{,   \quad   \quad  \quad  \quad     1}} \le i \le 8,\\
cyc:2c\left( { \pm 1,0} \right){\rm{,   \quad \quad  9}} \le i \le 12,\\
2c\left( { \pm 1, \pm 1} \right){\rm{,  \quad \quad \quad \;  \;    13}} \le i \le 16,
\end{array} \right.
\label{eq:26}
\end{eqnarray}
where `cyc' represents the indicates cyclic permutation. $c$ is a free parameter to optimize the properties of the DVM.

In order to save space, here we only show the results of OT vortex problem. This problem is first proposed by Orszag and Tang in 1979 \citep{orszag1979small}. Due to the complex vortex and shock wave structures generated during evolution, this problem has been widely used to demonstrate the validity of new models. Here, the initial configuration and conditions are identical to \citet{jiang1999}, as follows,
\begin{eqnarray}
\begin{array}{l}
\rho \left( {x,y,0} \right) = {\gamma ^2},  \qquad \qquad \qquad{\rm{          }}{v_x}\left( {x,y,0} \right) =  - \sin y,{\rm{   }}\\
{v_y}\left( {x,y,0} \right) = \sin x,            \qquad \qquad \quad{\rm{       }}p\left( {x,y,0} \right) = \gamma ,\\
{B_x}\left( {x,y,0} \right) =  - \sin y,        \qquad \qquad{\rm{   }}{B_y}\left( {x,y,0} \right) = \sin 2x,
\end{array}
\label{eq:48}
\end{eqnarray}
where $\gamma $ is equal to $5/3$. The computational domain is $\left[ {0,2\pi } \right] \times \left[ {0,2\pi } \right]$, which has been divided into $N_{x} \times N_{y}= 400 \times 400$ mesh-cells. The D2V25 model in figure~\ref{fig:1} is used to discretize the velocity space, where $c = 0.6$, ${\eta _0} = 0.4$ for $i=2,\cdots,5$, ${\eta _0} = 0.8$ for $i=6,\cdots,9$ and ${\eta _0} = 0$ for others. The time step is $\Delta t = 5 \times {10^{ - 4}}$, the space step is $\Delta x = \Delta y = 5 \times {10^{ - 3}}\pi $, the relaxation time is $\tau  = 1 \times {10^{ - 3}}$, and the number of extra degree of freedom is $n = 1$. Moreover, the periodic boundaries are applied in both $x$ and $y$ direction.
\begin{figure}
\centerline{\includegraphics[width=11cm]{Fig2}}% Images in 100% size
\caption{DBM results of Orszag-Tang vortex problem. (a) Pressure contour at time $t=0.5$ of DBM results. (b) Pressure contour at time $t=2$ of DBM results. (c) Pressure contour at time $t=3$ of DBM results. (d) Pressure distributions along $y=0.625\pi$ at $t=3$. The solid line represent the DBM result while the red dots represent the result of \citet{jiang1999}.}
\label{fig:2}
\end{figure}

Figure~\ref{fig:2} shows the results calculated by DBM. From figure~\ref{fig:2}(c), it is observed that the system is very complex, forming a variety of vortex and shock wave structures, especially in the center of the flow field. 
By comparing the DBM results in figures~\ref{fig:2}(a)-(c)  with the MHD results in figures (12)-(14)  given in  \citet{jiang1999}
, it is found that the two sets of results are in good agreement. For quantitative comparison, the pressure distribution along $y=0.625\pi$ 
at time $t=3$
is extracted and plotted together with the result of \citet{jiang1999} in figure~\ref{fig:2}(d). It can be seen that the DBM results from $x=0.2$ to $x=0.4$ is slightly lower, while the rest of results is in good agreement with those of \citet{jiang1999}, which proves the validity of the new magnetohydrodynamic DBM. 
In fact, the kinetics behavior of OT turbulence evolution is far from well studied.
In addition to the comparison of the above HNE characteristics, the TNE characteristics are further extracted and plotted in figure~\ref{fig:3}.

Figure~\ref{fig:3}(a) shows the contour map of total TNE strength ${D_{T}}$  at $t=3$. 
It is found that the TNE effects are most pronounced in the shock front where high physical gradients exist. For the rest of the flows field, the TNE strength is weak, indicating that the system is close to the local thermodynamic equilibrium state.
figure~\ref{fig:3}(b) shows the evolution of four kinds of entropy production rates with time. It can be seen that, both the four kinds of entropy production rates show two stages effect: before $t=1.7$, the entropy production rates keep increasing with time; After $t=1.7$, the entropy production rates reach the peak value and then keep decreasing with time. 
In fact, there is a positive correlation between the entropy production rate and the physical quantity gradient. Before $t=1.7$, the evolution of flow field is in dominated and the velocity and temperature gradients keep increasing, which leads to the increase of entropy production rates. After $t=1.7$, the dissipative effects are in dominated, resulting in the reduction of gradients and entropy production rates. Besides, in the early stage, it is found that the ${{\dot S}_{NOEF}}$ increases from $0$, while the ${{\dot S}_{NOMF}}$ increases from about $0.1$. This is because the initial temperature field is uniform and there exists no temperature gradient. For the initial velocity field, the initial velocity gradient induces a momentum exchange and leads to an initial entropy production rate ${{\dot S}_{NOMF}}$. Besides, as the flow field develops, the entropy production rate ${{\dot S}_{NOEF}}$ exceeds ${{\dot S}_{NOMF}}$ after $t=1$ and is always larger than ${{\dot S}_{NOMF}}$ in the subsequent evolution, indicating that the entropy production caused by NOEF is in dominant in the late stage of OT vortex problem.
\begin{figure}
  \centerline{\includegraphics[width=13.5cm]{Fig3}}% Images in 100% size
  \caption{TNE effects of Orszag-Tang vortex problem. (a) Contour of total TNE strength $D_{T}$ at $t=3$, (b) evolution of three kinds of entropy production rate with time. The red line with square symbol corresponding to ${{\dot S}_{NOEF}}$. The green line with delta symbol corresponding to ${{\dot S}_{NOMF}}$. The blue line with circle symbol corresponding to the total entropy production rate. The gray line with diamond symbol corresponding to the difference between ${{\dot S}_{NOEF}}$ and ${{\dot S}_{NOMF}}$.}
\label{fig:3}
\end{figure}

\emph{From the perspective of compression science, generally, the increase of entropy production rate predicts the increase of compression difficulty. }In the process of OT evolution, the difficulty of compression is divided into stages: in the first stage, the difficulty of compression increases with time; in the second stage, the difficulty of compression decreases with time.
%%%%%%%%%%%%%%%%%%%%%%%
\section{Richtmyer-Meshkov instability}
\label{Richtmyer Meshkov instability}
When the initial shock wave hits the perturbed interface, the non-collinear pressure and density gradient will induce vorticity at the perturbed interface and change the fluid velocity near the interface, resulting in the development of perturbation. In this section, the RMI induced by a shock wave passing through a heavy/light density interface is simulated by using the magnetohydrodynamic DBM. 
The effects of different initial magnetic field ${B_0}$ on the evolution of RMI and the subsequent re-shock process, both the induced HNE and TNE effects, are carefully investigated. 
 
This section consists of three subsections. In the first subsection, the initial configuration of RMI is given. In the next subsection, the HNE and TNE effects of RMI without magnetic field are investigated. In the last subsection, the HNE and TNE effects of RMI with different initial magnetic field are carefully investigated. Meanwhile, the entropy production rates under different  initial magnetic field are calculated and analyzed. To ensure the resolution, a grid convergence analysis is performed and analyzed in  Appendix~\ref{appC}.
\subsection{Flow field}
\begin{figure}
  \centerline{\includegraphics[width=13cm]{Fig4}}% Images in 100% size
  \caption{Schematic of initial configuration of Richtmyer-Meshkov instability.}
\label{fig:4}
\end{figure}

Figure~\ref{fig:4} shows the initial flow field of RMI. The length and height of the two-dimension computation domain are 20 and 80, respectively. The computation domain is divided into $N_x \times N_y = 200 \times 800$ mesh-cells. Initially, a sinusoidal perturbation interface is located at $y = 3N_{y}/4$, with wavelength $\lambda  = d$, where $d=20$ is equal to the length of computation domain. The amplitude is set to be ${y_0} = 0.1d$, which corresponds to a small perturbation case. The initial shock wave is located at $y = 7N_{y}/8$, with the physical quantities on the left and right sides of the shock wave connected by the Rankine–Hugoniot conditions. Thus, the shock wave and perturbed interface separate the domain into three regions as ${S_1}$, ${S_2}$, and ${S_3}$. ${S_1}$ is the area that has been compressed by the passed shock wave. ${S_2}$ is the region with high density and ${S_3}$ is the region with light density. Furthermore, the periodic boundary conditions are applied to the boundaries in $x$ direction, and the free inflow and solid boundary condition are applied to the boundaries at the top and bottom of $y$ direction, respectively. 
\subsection{RMI without magnetic field}
In this section, the HNE and TNE effects of RMI without magnetic field are investigated. The initial hydrodynamic quantities are as follows,
\begin{eqnarray}
\left\{ \begin{array}{l}
{\left( {\rho ,{u_x},{u_y},p} \right)_{S1}} = \left( {3.1304,0,-0.28005,2.1187} \right)\\
{\left( {\rho ,{u_x},{u_y},p} \right)_{S2}} = \left( {2.3333,0,0,1.4} \right)\\
{\left( {\rho ,{u_x},{u_y},p} \right)_{S3}} = \left( {1,0,0,1.4} \right)
\end{array} \right.
\label{eq:49}.
\end{eqnarray}
With the above initial quantities, a shock wave with $\rm{Ma}= 1.2$  is formed to impact the sinusoidal perturbation interface. The Atwood number is $\rm{At}=0.4$. The D2V16 model in figure~\ref{fig:1} is adopted to discrete the particle velocity space, and the parameters are $c=1$, ${\eta _0} = 5$ for $i=1,\cdots,4$ and ${\eta _0} = 0$ for others. The other parameters of DBM are as follows: space step $\Delta x = \Delta y = 1 \times {10^{ - 1}}$, time step $\Delta t = 1 \times {10^{ - 3}}$, relaxation time $\tau = 1 \times {10^{ - 3}}$ and the number of extra degrees of freedom $n=3$, i.e., the specific heat ratio $\gamma=1.4$.

\subsubsection{HNE characteristics}
Figure~\ref{fig:5} shows the density evolution in RMI via snapshots at times $t = 0$,$10$,$15$,$40$,$70$,$100$,$250$. At $t=10$, the shock wave hits the interface and interacts with the interface. At $t=15$, the shock wave passes through the interface, forming a reflected rarefaction wave propagating upward to top boundary and a transmission shock wave propagating downward to the bottom boundary. Meanwhile, the perturbation amplitude gradually decreases to zero with the motion of interface. Then, the interface reverses and the perturbation amplitude grows again, accompanied by the reversal of the initial perturbation interface peak and valley. At $t=40$, the transmission shock wave moves to the bottom solid wall and reflects. At $t=70$, the  reflected transmission shock wave passes through the interface again, forming a reflected rarefaction wave to light fluids and a transmission shock wave to heavy fluids. Under the action of the secondary shock (reflected transmission shock wave), the speed of interface development is greatly accelerated and the speed difference on both sides of the interface increases rapidly. Then, the Kelvin-Helmholtz instability (KHI) appears at the head of the spike, eventually forming a mushroom-like structure. During the interface evolution, the light and heavy fluids continuously mix with each other.
\begin{figure}
  \centerline{\includegraphics[width=14cm]{Fig5}}% Images in 100% size
  \caption{Density contours in the evolution of RMI without initial applied magnetic field at different times: (a)$t=0$, (b)$t=10$, (c)$t=15$, (d)$t=40$, (e)$t=70$, (f)$t=100$, (g)$t=250$.}
\label{fig:5}
\end{figure}

\subsubsection{TNE effects}
Figure~\ref{fig:7}(a) shows the global averaged TNE strength ${\bar D_{T}}$, NOMF ${\bar D_2}$, NOEF ${\bar D_{3,1}}$, the flux of NOMF ${ \bar D_{3}}$ and the flux of NOEF ${\bar D_{4,2}}$ from $t=0$ to $t=500$. It can be found that the strength of ${\bar D_2}$ is much lower than other TNE effects, and the trend of ${\bar D_{T}}$ is basically the same as that of ${\bar D_{3,1}}$, which means that the TNE effects of the whole system caused by viscosity are weaker than the TNE effects caused by heat conduction. The TNE strength ${\bar D_{T}}$ of the system basically increases with time before about $t=320$, while decreases with time after $t=320$. In fact, there exists two competition mechanisms. Before $t=320$, the evolution of RMI is located at linear and weak nonlinear development stage. With the development of the interface, the physical quantities gradient near the interface increases rapidly, and the non-equilibrium region augments, which cause the increasing of ${\bar D_{T}}$. After  $t=320$, the evolution of RMI is located at the late stage of nonlinear development. At this time, the dissipation effects such as viscosity and heat conduction are dominant, which increase the degree of mixing near the interface and reduce the gradient of physical quantities, causing the decrease of ${\bar D_{T}}$. Besides, from the red line $({\bar D_{T}})$, three key time points are marked, which are shown in the enlarged picture figure~\ref{fig:7}(b).


Figure~\ref{fig:7}(b) shows the the global averaged TNE strength ${\bar D_{T}}$, NOMF ${\bar D_2}$, NOEF ${\bar D_{3,1}}$ and the flux of NOEF ${\bar D_{4,2}}$ from $t=0$ to $t=100$. It is found that, before time point 1, TNE effects decrease very slowly with time due to dissipation effects. At time point 1, the shock wave passes through the interface, causing the distribution function near the interface to deviate greatly from the local thermodynamic equilibrium state. After time point 1, the dissipative effects tend to reduce TNE effects, while the high $Kn$ makes it difficult to recover to the local thermodynamic equilibrium state, causing the increase of all TNE effects.
At time point 2, the transmission shock wave hits the boundary and reflects. In this process, the gradients of macroscopic physical quantities change, resulting in the fluctuation of TNE quantities.
At time point 3, the reflected transmission shock wave hits the interface again. Since the direction of the shock wave is opposite to that of the first time, not all TNE quantities increase. The flux of NOEF, i.e., ${\bar D_{4,2}}$ decreases, while the rest of the TNE quantities increase. Because the interface has reversed when the shock wave hits the interface for the second time, the vorticity generated is further deposited at the interface after time point 3, which leads to the accelerated development of the interface, the expansion of the non-equilibrium area and the continuous increase of all TNE quantities.
\begin{figure}
  \centerline{\includegraphics[width=14cm]{Fig6}}% Images in 100% size
  \caption{Contours of global averaged TNE quantities ${\bar D_{T}}$, ${\bar D_{2}}$, ${\bar D_{3}}$, ${\bar D_{3,1}}$ and ${\bar D_{4,2}}. $ (a) The time ranges from $t=0$ to $t=500$, (b) the time ranges from $t=0$ to $t=100$.}
\label{fig:7}
\end{figure}

\begin{figure}
  \centerline{\includegraphics[width=13cm]{Fig7}}% Images in 100% size
  \caption{Contours of different TNE quantities of RMI at $t=100$. (a)$D_2$, (b)$D_3$, (c)$D_{3,1}$, (d)$D_{4,2}$}
\label{fig:8}
\end{figure}
Figure~\ref{fig:8} shows the contours of TNE quantities $D_2$, $D_3$, $D_{3,1}$, $D_{4,2}$ at $t=100$, at which time the reflected transmission shock wave has passed the perturbed interface. From figure~\ref{fig:8}(a), it is found that $D_2$ mainly distributed at shock wavefront and the perturbed interface, and the intensity at the shock wavefront is much higher than that at the perturbed interface. For the rest of the flow field, the $D_2$ is almost zero. Therefore, $D_2$ can be used to capture the location of shock wave during RMI. Due to the strong momentum transport and shear effects caused by the large density and velocity gradient, the $D_2$ is most remarkable at shock wavefront. Figure~\ref{fig:8}(c) shows the distribution of $D_{3,1}$. It reaches the maximum value at perturbed interface. Thus, $D_{3,1}$ can be used to capture the evolution of interface and the amplitude. In this case, the temperature gradient near perturbed interface is bigger than that near shock wavefront, causing strong energy transport. Thus, the $D_{3,1}$ is most remarkable at perturbed interface. Though $D_2$ and  $D_{3,1}$ can be used to identify the shock wavefront and perturbed interface respectively, they cannot be used to identify both of them. Compared with $D_2$ and  $D_{3,1}$, $D_3$ and $D_{4,2}$ are more suitable. $D_3$ provides the most distinguishable interfaces, which is also proved by \citet{zhang2019discrete}. During the evolution of RMI, both of these four TNE quantities can be used as the supplement of the macroscopic quantities contours of flow field.

In order to investigate the evolution of non-equilibrium effects with time for a more detailed analysis, the axes $x=N_{x}/2$ is selected and observed to see how the distribution of TNE quantities on the axes evolve over time. Figure~\ref{fig:9}(a) shows the distribution of the component $\Delta_{2yy}^{*}$ of NOMF $\boldsymbol{\Delta}_2^{*}$. It can be seen from the red line that $\Delta _{2yy}^{*}$ mainly distributes in three areas marked with "1", "2", "3" from right to left, which exactly corresponding to rarefaction, material interface and transmission shock wave. Besides, the strength of $\Delta _{2yy}^{*}$ near material interface is stronger than rarefaction but weaker than transmission shock wave. 
With time evolution, the intensity of $\Delta _{2yy}^{*}$ near interface gradually decreases, but the intensity of $\Delta _{2yy}^{*}$  near transmission shock wave gradually increases.
Figure~\ref{fig:9}(b) shows the distribution of the component $\Delta_{3,1y}^{*}$ of NOEF $\boldsymbol{\Delta}_{3,1}^{*}$. It can be seen from the red line that the strength of $\Delta_{3,1y}^{*}$ near material interface is much stronger than that near transmission shock wave. After the reflection of transmission shock wave, the direction of $\Delta_{3,1y}^{*}$ reverses, which is different from $\Delta_{2yy}^{*}$. 
By further observing the distribution of $\Delta_{3,1y}^{*}$ near material interface, it is found that the intensity of  $\Delta_{3,1y}^{*}$ first decreases then increases. 
The decreasing of $\Delta_{3,1y}^{*}$ is mainly due to the increase of mixing area near interface, where the gradients of physical quantities decrease.
The increasing of $\Delta_{3,1y}^{*}$ is due to the passing of secondary shock, where the energy is transported from shock to material interface.
Besides, there exists double-peak structure near material interface.
In this case, the $\Delta_{3,1y}^{*}$ first gets its maximum value at the right peak then the left one. The shift of the maximum peak depends on the direction of shock wave.
\begin{figure}
  \centerline{\includegraphics[width=14cm]{Fig8}}% Images in 100% size
  \caption{Distribution of  different TNE quantities along $N_{x}/2$ at $t=15$, $t=40$, $t=70$ and $t=100$. (a)$\Delta_{2yy}^{*}$, (b)$\Delta_{3,1y}^{*}$.}
\label{fig:9}
\end{figure}

\subsection{RMI with magnetic field}
The magnetic field can suppress the evolution of RMI through transporting the barochorically generated vorticity away from the interface \citep{samtaney2003,sano2013}.
In this section, the effects of magnetic fields on the evolution of RMI, including HNE and TNE effects, are analyzed. The initial magnetic field is set on the $y$ direction, and the parameters setting in different cases are shown in Table \ref{tab:1}.
\begin{table}
  \begin{center}
\def~{\hphantom{0}}
  \begin{tabular}{lccc}
      Cases & $B_{0}$ & Cases & $B_{0}$  \\[3pt]
       Case I   & 0.01 & Case VI   & 0.10\\
       Case II  & 0.02 & Case VII  & 0.15\\
       Case III & 0.03 & Case VIII & 0.20\\
       Case IV  & 0.04 & Case IX   & 0.25\\
       Case V   & 0.05 & Case X    & 0.30\\
  \end{tabular}
  \caption{Settings for RMI with different initial applied magnetic fields.}
  \label{tab:1}
  \end{center}
\end{table}

\subsubsection{HNE characteristics}
Figure~\ref{fig:10} shows the density contours of different initial magnetic fields range from $0.01$ to $0.30$ at $t=250$. It is found that the evolution of interface is gradually suppressed with the increase of magnetic fields. 
For the case of $B_0=0.01$, the Kelvin-Helmholtz instability (KHI) is still developed and observed, forming the "mushroom-like" structure.
For the case of $B_0=0.03$ and $B_0=0.05$, the KHI degenerates and the amplitude of interface decreases.
From figures~\ref{fig:10} (a)-(e), it can be observed that the interface can still reverse.
However, from figure~\ref{fig:10} (f), it is observed that the interface no longer reverses for the case $B_0=0.30$, indicating that there exists a critical magnetic field $B_{0,C}$ under which the amplitude of interface can be suppressed to $0$. 
\begin{figure}
  \centerline{\includegraphics[width=14cm]{Fig9}}% Images in 100% size
  \caption{Density contours with different applied magnetic fields at $t=250$. (a)$B_{0}=0.01$, (b)$B_{0}=0.03$, (c)$B_{0}=0.05$, (d)$B_{0}=0.10$, (e)$B_{0}=0.20$, (f)$B_{0}=0.30$}
\label{fig:10}
\end{figure}
\begin{figure}
  \centerline{\includegraphics[width=7.5cm]{Fig10}}% Images in 100% size
  \caption{Evolution of amplitude of RMI with different initial applied magnetic fields from $t=0$ to $t=200$.}
\label{fig:12}
\end{figure}

Figure~\ref{fig:12} further shows the evolution of interface amplitude with time under initial applied different magnetic fields. 
Obviously, the magnetic fields delay the time interface inverse.
For the cases ($B_0 \le 0.05$), the magnetic fields have little effect on the flow field before inverse but can significantly inhibit the development of interface amplitude after inverse.
With the increase of the magnetic fields, the influence of the secondary shock on the perturbation amplitude is weakened. 
This is mainly because the magnetic field inhibits the development of the interface, which reduces the perturbation amplitude when the secondary shock passing through interface. Thus, the induced vorticity reduces.
With the increase of magnetic fields, a critical situation appears where the perturbation is nearly inhibited and the amplitude no longer increases. 
As the magnetic field further increase, the interface no longer reverse, and the vorticity induced by secondary shock will prevent the development of interface.
When magnetic field exceeds the critical magnetic field, the stronger the magnetic field, the closer the interface shape is to the initial state. 

\subsubsection{TNE effects}
\begin{figure}
  \centerline{\includegraphics[width=14cm]{Fig11}}% Images in 100% size
  \caption{Evolution of global averaged TNE effects with different initial applied magnetic fields from $t=0$ to $t=500$.}
\label{fig:13}
\end{figure}
Figure~\ref{fig:13}(a) shows the global averaged TNE strength $\bar D_T$ with different initial magnetic fields range from $0.00$ to $0.05$.
It is observed that, before the secondary shock, the increase of magnetic fields shows little effect to $\bar D_T$.
After secondary shock, $\bar D_T$ quickly increases with the evolution of interface, and the magnetic fields show greatly effects for suppressing $\bar D_T$. 
However, it is found that $\bar D_T$ for the case of $B_0=0.01$ and $B_0=0.02$ are greater than that of $B_0=0.00$ after time marked with circle.
This is because the magnetic fields suppress the shear effects around interface, thus delay the time KHI arises and decrease the strength of KHI.
The emergence and development of KHI can enhance mixing and reduce the gradients of physical quantities, which is helpful for reducing $\bar D_T$.
The inflection point of TNE intensity can be regarded as the criterion to judge whether KHI is fully developed. Before the inflection point, the development of perturbation amplitude dominates and the TNE is enhanced; after the inflection point, the fully development of KHI results in the enhancement of dissipation effect and the decrease of TNE intensity.

Figure~\ref{fig:13}(b) shows the global averaged TNE strength $\bar D_T$ with different initial magnetic fields range from $0.10$ to $0.30$.
It is found that the influence of magnetic field for TNE could be divided into two stages. 
Before the inverse of interface, as the strength of magnetic field increases, the TNE strength slightly increases. 
After the inverse of interface, the TNE strength decreases significantly with the increase of magnetic field strength. 
In fact, the magnetic field consistently inhibits the interface development during the RMI evolution. 
Before the interface inverse, the magnetic field inhibits the interface inverse, indirectly enhancing the physical quantity gradient near the interface, resulting in the enhancement of TNE intensity. 
After the interface inverse, the magnetic field inhibits the shear effect near the interface, as has been explained before.
From figure~\ref{fig:13}(b), it can also be found that after secondary, $\bar D_T$ keeps decreasing with the increase of magnetic fields, and there exists a critical magnetic field, above which $\bar D_T$ no longer decreases.
Thus, the magnetic field intensity $B_{0,C}$, corresponding to the minimum value $\bar D_{T,min}$ of global averaged TNE intensity just after the reshock stage, can be used as the critical magnetic field intensity to inhibit the development of RMI interface.
%%%%%%%%%%%%%%%%%%%%%%%%%%%%%%%
\subsubsection{Entropy production}
\begin{figure}
  \centerline{\includegraphics[width=14cm]{Fig12}}% Images in 100% size
  \caption{Evolution of entropy production rates from $t=0$ to $t=500$ with magnetic fields range from $0.01$ to $0.05$.}
\label{fig:14}
\end{figure}

Figure~\ref{fig:14} shows the two parts of entropy production rate ${{\dot S}_{NOMF}}$ and ${{\dot S}_{NOEF}}$ of the global system with magnetic fields range from $0.01$ to $0.05$. From figure~\ref{fig:14}(a) we can find that the entropy production rate ${{\dot S}_{NOMF}}$ shows stages. 
Before the shock wave contact the perturbed interface, the rate sharply decreased. Then, after the shock wave contacts the solid wall and passed through interface again, the rate appears two jumps. 
After re-shock, the rate first decreases and then increases to the maximum. 
With the increase of magnetic fields, the values of ${{\dot S}_{NOMF}}$ are basically the same.
From figure~\ref{fig:14}(b), it is observed that when the shock first contacts the interface, the entropy production rate ${{\dot S}_{NOEF}}$ suddenly increases, as shown by point ‘1', which is different from ${{\dot S}_{NOMF}}$. It means that the shock wave enhances physical quantity gradients near interface. 
When the transmitted shock wave passes interface again, as shown by point ‘2', the rate increases greatly, which is also observed from ${{\dot S}_{NOMF}}$.
After re-shock, the rate first gradually increases and then decreases, which can be divided into two stages.
When the initial magnetic field strength gradually increases, the  ${{\dot S}_{NOEF}}$ also shows no difference before re-shock, but is greatly reduced after re-shock.
In general, the ${{\dot S}_{NOEF}}$ contributes more to entropy increase than ${{\dot S}_{NOMF}}$, but the strength of ${{\dot S}_{NOEF}}$ could be greatly inhibited by adding magnetic field.

\begin{figure}
  \centerline{\includegraphics[width=14cm]{Fig13}}% Images in 100% size
  \caption{Evolution of entropy production rates from $t=0$ to $t=500$ with magnetic fields range from $0.10$ to $0.30$.}
\label{fig:15}
\end{figure}
Figure~\ref{fig:15} shows the two parts of entropy production rate ${{\dot S}_{NOMF}}$ and ${{\dot S}_{NOEF}}$ of the global system with magnetic fields range from $0.10$ to $0.30$. 
From figure~\ref{fig:14}(a) and figure~\ref{fig:15}(a), it can observed that the fluctuations in the ${{\dot S}_{NOMF}}$ curve have been smoothed due to the increase of magnetic fields. 
Different from ${{\dot S}_{NOMF}}$, the evolution of ${{\dot S}_{NOEF}}$ is strongly affected by magnetic fields. With the increase of magnetic fields, the evolution of ${{\dot S}_{NOEF}}$ after re-shock is significantly inhibited. 
It can be observed that there exists a critical magnetic field, under which the ${{\dot S}_{NOEF}}$ no longer decreases.
\section{Conclusion and discussion}
When the particle collision frequency is sufficiently high, the kinetic behaviors can be described by the reduced hydrodynamic equations. When the particle collision frequency is negligibly small, the kinetic behaviors can be described by the reduced Vlasov equation where the collision effect is completely ignored. The situation between the two extreme cases is generally difficult to treat with and consequently poorly understood, which is responsible for the fact that the kinetic effects caused by particle collision in ICF are still far from clear understanding although they have potential impact on ICF ignition. 

To attack the problem, the preferred method/model needs to have simultaneously two physical functions. First, it can present numerical data which can recover the HNE and TNE behaviors we aim to investigate. Second, it can present methods which can be used to check the non-equilibrium state, describe and analyze the resulting effects. In the DBM, for the first physical function, the model equations are composed of a discrete Boltzmann equation coupled by a magnetic induction equation. For the second physical function, DBM intrinsically includes some schemes for studying the TNE behaviors. The most fundamental one is to use the non-conserved kinetic moments of $(f-f^{eq})$ to check the TNE state and describe the TNE behaviors. The phase space description method based on the non-conserved kinetic moments of $(f-f^{eq})$ presents an intuitive geometrical correspondence for the complicated TNE state and behaviors, which is important for the deep investigation and clear understanding. 
The first function is verified by recovering HNE behaviors of a number of typical benchmark problems including sod shock tube, thermal couette flow and OT vortex problem. Besides, the most relevant TNE behaviors of OT vortex problem are investigated for the first time.  
As a further application, the kinetic study on RMI system with initial horizontal magnetic field is performed.  Physical findings are as below:
(i) Generally, the entropy production rate is in positive correlation to the difficulty of compression. During the evolution of OT system, the stage behavior of entropy production rate, first increases then decreases with time, shows the same stage behavior of compression difficulty. 
(ii) In the case without external magnetic field, the NOMF gets its maximum value near the shock front, while the NOEF gets its maximum value near the perturbed interface.  During the perturbed interface inverse process of RMI, the NOEF along the central axis shows two-peak values near perturbed interface, and the impact of shock wave significantly enhances the NOEF.
(iii) In the case with external magnetic field, the magnetic field shows inhibitory effects on the evolution of RMI. Specifically, the magnetic field has a pronounced inhibitory effect on the nonlinear stage, especially on the generation of KHI. Besides, there exists a critical magnetic field under which the amplitude of interface can be suppressed to $0$.
(iv) Before the interface inverse, the magnetic field indirectly enhances the TNE intensity by suppressing the interface inverse. After the interface inverse, the magnetic field significantly suppresses the TNE intensity by inhibiting the further development of the perturbed interface. 
(v) In terms of entropy production rate, the magnetic field has pronounced inhibitory effect on the entropy production rate caused by heat conduction, but has a weak effect on the entropy production rate caused by viscosity. 

Potential applications of above physical findings are as follows: 
(i) Based on features of NOMF and NOEF, shock front and perturbed interface during RMI evolution can be physically captured in the numerical experiments.
(ii) The magnetic field intensity $B_{0,C}$, corresponding to the minimum value $\bar D_{T,min}$ of global averaged TNE intensity just after the re-shock stage, can be used as the critical magnetic field intensity to inhibit the development of RMI interface.
(iii) From the perspective of entropy production rate, the magnetic field intensity $B_{0,C}$, corresponding to the minimum value of entropy production rate caused by heat conduction just after the re-shock stage, plays the same role.
The future work includes two-fluid DBM where the ion and electrons are considered separately and Prantl number adjustable plasma DBM, etc.

\backsection[Acknowledgements]{The authors sincerely thank Lingxiao Li for his helpful suggestion on choosing discrete format. The authers thank Yudong Zhang, Chuandong Lin, Ge Zhang, Dejia Zhang, Yiming Shan, Jie Chen, Hanwei Li and Yingqi Jia for the helpful discussion on DBM. The authors thank Song Bai, Fuwen Liang, Feng Tian, Zihao He, Mingqing Nie, Zhengxi Zhu for the helpful discussion on results analysis.}

\backsection[Funding]{This work was supported by the National Natural Science Foundation of China (grant numbers 52202460, 12172061 and 11975053), the National Key R D Program of China (grant numbers 2020YFC2201100, 2021YFC2202804, 2022YFB3403504), the Natural Science Foundation of Shandong Province (grant numbers ZR2020MA061), Shandong Province Higher Educational Youth Innovation Science and Technology Program (grant numbers 2019KJJ009), the opening project of State Key Laboratory of Explosion Science and Technology (Beijing Institute of Technology) (grant numbers KFJJ2023-02M), Foundation of Laboratory for Shock Wave and Detonation Physics.}

\backsection[Declaration of interests]{The authors report no conflict of interest.}

\backsection[Author ORCIDs]{Jiahui Song, https://orcid.org/0000-0002-4996-6178; Aiguo Xu, https://orcid.org/0000-0002-6179-5973; Long Miao, https://orcid.org/0000-0001-5923-8740; Feng Chen, https://orcid.org/0000-0003-0094-2279;  }

\appendix
%%%%%%%%%%%%%%%%%%%%%%%%%%%%%%%%%%%%%%%%%%%%%
\section{Kinetic moments used to characterize the first-order and second-order TNE effects}
\label{appA}
Here, the kinetic moments used to characterize the first-order and second-order TNE effects are given as below,
\begin{equation}
M_0^{eq} = \sum\limits_i {f_i^{eq}}  = \rho,
\label{eq:A1}
\end{equation}

\begin{equation}
M_{1,x}^{eq} = \sum\limits_i {f_i^{eq}{v_{ix}}}  = \rho {u_x},
\label{eq:A2}
\end{equation}

\begin{equation}
M_{1,y}^{eq} = \sum\limits_i {f_i^{eq}{v_{iy}}}  = \rho {u_y},
\label{eq:A3}
\end{equation}

\begin{equation}
M_{2,0}^{eq} = \sum\limits_i {f_i^{eq}\left( {v_{ix}^2 + v_{iy}^2 + \eta _i^2} \right)}  = \rho \left[ {\left( {n + 2} \right)RT + u_x^2 + u_y^2} \right],
\label{eq:A4}
\end{equation}

\begin{equation}
M_{2,xy}^{eq} = \sum\limits_i {f_i^{eq}{v_{ix}}{v_{iy}}}  = \rho {u_x}{u_y},
\label{eq:A5}
\end{equation}

\begin{equation}
M_{2,xx}^{eq} = \sum\limits_i {f_i^{eq}{v_{ix}}{v_{ix}}}  = \rho \left( {RT + u_x^2} \right),
\label{eq:A6}
\end{equation}

\begin{equation}
M_{2,yy}^{eq} = \sum\limits_i {f_i^{eq}{v_{iy}}{v_{iy}}}  = \rho \left( {RT + u_y^2} \right),
\label{eq:A7}
\end{equation}

\begin{equation}
M_{3,1,x}^{eq} = \sum\limits_i {f_i^{eq}\left( {v_{ix}^2 + v_{iy}^2 + \eta _i^2} \right){v_{ix}}}  = \rho {u_x}\left[ {\left( { n + 4} \right)RT + u_x^2 + u_y^2} \right],
\label{eq:A8}
\end{equation}

\begin{equation}
M_{3,1,y}^{eq} = \sum\limits_i {f_i^{eq}\left( {v_{ix}^2 + v_{iy}^2 + \eta _i^2} \right){v_{iy}}}  = \rho {u_y}\left[ {\left( { n + 4} \right)RT + u_x^2 + u_y^2} \right],
\label{eq:A9}
\end{equation}

\begin{equation}
M_{3,xxx}^{eq} = \sum\limits_i {f_i^{eq}v_{ix}^3}  = \rho {u_x}\left( {3RT + u_x^2} \right),
\label{eq:A10}
\end{equation}

\begin{equation}
M_{3,yyy}^{eq} = \sum\limits_i {f_i^{eq}v_{iy}^3}  = \rho {u_y}\left( {3RT + u_y^2} \right),
\label{eq:A11}
\end{equation}

\begin{equation}
M_{3,xxy}^{eq} = \sum\limits_i {f_i^{eq}v_{ix}^2{v_{iy}}}  = \rho {u_y}\left( {RT + u_x^2} \right),
\label{eq:A12}
\end{equation}

\begin{equation}
M_{3,xyy}^{eq} = \sum\limits_i {f_i^{eq}{v_{ix}}v_{iy}^2}  = \rho {u_x}\left( {RT + u_y^2} \right),
\label{eq:A13}
\end{equation}

\begin{eqnarray}
M_{4,2,xx}^{eq} & = &\sum\limits_i {f_i^{eq}\left( {v_{ix}^2 + v_{iy}^2 + \eta _i^2} \right)v_{ix}^2}  \nonumber\\
& = & \rho \left[(n+4) R^2 T^2+R T \left((n+7) u_x^2+u_y^2\right)+u_x^2 \left(u_x^2+u_y^2\right)\right],
\label{eq:A14}
\end{eqnarray}

\begin{eqnarray}
M_{4,2,yy}^{eq} & = & \sum\limits_i {f_i^{eq}\left( {v_{ix}^2 + v_{iy}^2 + \eta _i^2} \right)v_{iy}^2}  \nonumber\\
& = & \rho \left[(n+4) R^2 T^2+R T \left((n+7) u_y^2+u_x^2\right)+u_y^2 \left(u_x^2+u_y^2\right)\right],
\label{eq:A15}
\end{eqnarray}

\begin{equation}
M_{4,2,xy}^{eq} = \sum\limits_i {f_i^{eq}\left( {v_{ix}^2 + v_{iy}^2 + \eta _i^2} \right){v_{ix}}} {v_{iy}} = \rho {u_x}{u_y}\left[ {\left( { n + 6} \right)RT + u_x^2 + u_y^2} \right],
\label{eq:A16}
\end{equation}

\begin{equation}
M_{4,xxxx}^{eq} = \sum\limits_i {f_i^{eq}v_{ix}^4}  = \rho \left(3 R^2 T^2+6 R T u_x^2  + u_x^4\right),
\label{eq:A17}
\end{equation}

\begin{equation}
M_{4,yyyy}^{eq} = \sum\limits_i {f_i^{eq}v_{iy}^4}  = \rho \left(3 R^2 T^2+6 R T u_y^2  + u_y^4\right),
\label{eq:A18}
\end{equation}

\begin{equation}
M_{4,xxxy}^{eq} = \sum\limits_i {f_i^{eq}v_{ix}^3v_{iy}}  = \rho u_x u_y \left(3 R T+u_x^2\right),
\label{eq:A19}
\end{equation}

\begin{equation}
M_{4,xyyy}^{eq} = \sum\limits_i {f_i^{eq}v_{ix}v_{iy}^3}  = \rho u_x  u_y \left(3 R T+u_y^2\right),
\label{eq:A20}
\end{equation}

\begin{equation}
M_{4,xxyy}^{eq} = \sum\limits_i {f_i^{eq}v_{ix}^2v_{iy}^2}  = \rho  \left(R T+u_x^2\right) \left(R T+u_y^2\right),
\label{eq:A21}
\end{equation}


\begin{eqnarray}
M_{5,3,xxx}^{eq} & = &\sum\limits_i {f_i^{eq}\left( {v_{ix}^2 + v_{iy}^2 + \eta _i^2} \right)v_{ix}^3} \nonumber\\
  & = & \rho u_x \left[3 (n+6) R^2 T^2+R T \left((n+11) u_x^2+3 u_y^2\right)+u_x^2 \left(u_x^2+u_y^2\right)\right],
 \label{eq:A22}
\end{eqnarray}

\begin{eqnarray}
M_{5,3,yyy}^{eq} & = & \sum\limits_i {f_i^{eq}\left( {v_{ix}^2 + v_{iy}^2 + \eta _i^2} \right) v_{iy}^3} \nonumber\\ 
& = & \rho u_y \left[3 (n+6) R^2 T^2+R T \left((n+11) u_y^2+3 u_x^2\right)+u_y^2 \left(u_x^2+u_y^2\right)\right],
\label{eq:A23}
\end{eqnarray}

\begin{eqnarray}
M_{5,3,xxy}^{eq} & = & \sum\limits_i {f_i^{eq}\left( {v_{ix}^2 + v_{iy}^2 + \eta _i^2} \right)v_{ix}^2 v_{iy}} \nonumber\\ 
& = & \rho u_y \left[(n+6) R^2 T^2+R T \left((n+9) u_x^2+u_y^2\right)+u_x^2 \left(u_x^2+u_y^2\right)\right],
 \label{eq:A24}
\end{eqnarray}

\begin{eqnarray}
M_{5,3,xyy}^{eq} & = &\sum\limits_i {f_i^{eq}\left( {v_{ix}^2 + v_{iy}^2 + \eta _i^2} \right)v_{ix} v_{iy}^2} \nonumber\\ 
& = & \rho u_x \left[(n+6) R^2 T^2+R T \left((n+9) u_y^2+u_x^2\right)+u_y^2 \left(u_x^2+u_y^2\right)\right],
 \label{eq:A24}
\end{eqnarray}

where $p=\rho RT$ obeys the ideal gas law. 
%%%%%%%%%%%%%%%%%%%%%%%%%%%%%%%%%%%%%%%%%%%%%
\section{Non-dimensional quantities}
\label{appB}
All of the physical quantities used in this paper are dimensionless according to the following equations,
\begin{equation}
\hat \rho  = \frac{\rho }{{{\rho _0 }}}, \hat T = \frac{T}{{{T_0 }}}, \hat P = \frac{P}{{{\rho _0}{R }T_0}}, \left( {\hat t,\hat \tau } \right) = \frac{{\left( {t,\tau } \right)}}{{\sqrt {R{T_0 }} }}, {\hat r_\alpha } = \frac{{{r_\alpha }}}{{{L_0 }}},
\label{eq:B1}
\end{equation}
\begin{equation}
\left( {{{\hat v}_\alpha },{{\hat c}_\alpha },{{\hat u}_\alpha }} \right) = \frac{{\left( {{v_\alpha },{c_\alpha },{u_\alpha }} \right)}}{{\left( {{L_0 }/{t_0 }} \right)}}, \hat B = \frac{{B{L_0 }}}{{{t_0 }}}\sqrt {{\mu _0}{\rho _0 }},
\label{eq:B2}
\end{equation}
\begin{equation}
\hat \kappa  = \frac{\kappa }{{{\rho _0 }R{L_0 }\sqrt {R{T_0 }} }}, \hat \mu  = \frac{\mu }{{{\rho _0 }{L_0 }\sqrt {R{T_0 }} }}, 
\label{eq:B3}
\end{equation}
where subscript `$0$' represents the reference physical quantities. 
%%%%%%%%%%%%%%%%%%%%%%%%%%%%%%%%%%%%%%%%%%%%%
\section{Grid convergence test}
\label{appC}
In order to verify the effectiveness of numerical resolution, a grid convergence test is performed. Four kinds of grid numbers are selected as $N_x \times N_y = 50 \times 200$, $N_x \times N_y = 100 \times 400$, $N_x \times N_y = 200 \times 800$ and $N_x \times N_y = 400 \times 1600$. The corresponding space steps are $\Delta x = \Delta y = 4 \times {10^{ - 1}}$, $\Delta x = \Delta y = 2 \times {10^{ - 1}}$, $\Delta x = \Delta y = 1 \times {10^{ - 1}}$, $\Delta x = \Delta y = 5 \times {10^{ - 2}}$, respectively. The other calculation parameter settings of DBM are: time step $\Delta t = 1 \times {10^{ - 1}}$, relaxation time $\tau = 1 \times {10^{ - 3}}$ and the number of extra degrees of freedom $n=3$, i.e., the specific heat ratio $\gamma=1.4$. Figure~\ref{fig:appC} shows the distribution of density and temperature along $y=20$. It can be seen that the results of $200 \times 800$ and $400 \times 1600$ meshes are quite identical. By comprehensively considering numerical resolution and computational cost, the $200 \times 800$ meshed are selected for calculation and analysis in this paper.
\begin{figure}
  \centerline{\includegraphics[width=14cm]{appC}}% Images in 100% size
  \caption{Distribution of physical quantities along $x=10$ of four kinds of grid at $t=100$. (a)Density, (b)Temperature.}
\label{fig:appC}
\end{figure}

\begin{thebibliography}{108}
\expandafter\ifx\csname natexlab\endcsname\relax\def\natexlab#1{#1}\fi
\def\au#1{#1} \def\ed#1{#1} \def\yr#1{#1}\def\at#1{#1}\def\jt#1{\textit{#1}}
  \def\bt#1{#1}\def\bvol#1{\textbf{#1}} \def\vol#1{#1} \def\pg#1{#1}
  \def\publ#1{#1}\def\arxiv#1{#1}\def\org#1{#1}\def\st#1{\textit{#1}}

\bibitem[Abu-Shawareb {\em et~al.\/}(2022)Abu-Shawareb, Acree, Adams, Adams,
  Addis, Aden, Adrian, Afeyan, Aggleton, Aghaian {\em et~al.\/}]{abu2022lawson}
{\sc \au{Abu-Shawareb, H}, \au{Acree, R}, \au{Adams, P}, \au{Adams, J},
  \au{Addis, B}, \au{Aden, R}, \au{Adrian, P}, \au{Afeyan, BB}, \au{Aggleton,
  M}, \au{Aghaian, L} \& \au{others}} \yr{2022}  \at{Lawson criterion for
  ignition exceeded in an inertial fusion experiment}.  \jt{Physical Review
  Letters}  \bvol{129}~(7),  \pg{075001}.

\bibitem[Arnett(2000)]{arnett2000role}
{\sc \au{Arnett, David}} \yr{2000}  \at{The role of mixing in astrophysics}.
  \jt{The Astrophysical Journal Supplement Series}  \bvol{127}~(2),  \pg{213}.

\bibitem[Asahina {\em et~al.\/}(2017)Asahina, Nagatomo, Sunahara, Johzaki,
  Hata, Mima \& Sentoku]{asahina2017validation}
{\sc \au{Asahina, T}, \au{Nagatomo, H}, \au{Sunahara, A}, \au{Johzaki, T},
  \au{Hata, M}, \au{Mima, K} \& \au{Sentoku, Y}} \yr{2017}  \at{Validation of
  thermal conductivity in magnetized plasmas using particle-in-cell
  simulations}.  \jt{Physics of Plasmas}  \bvol{24}~(4),  \pg{042117}.

\bibitem[Bakhsh(2022)]{bakhsh2022linear}
{\sc \au{Bakhsh, Abeer}} \yr{2022}  \at{Linear analysis of magnetohydrodynamic
  richtmyer--meshkov instability in cylindrical geometry for double interfaces
  in the presence of an azimuthal magnetic field}.  \jt{Physics of Fluids}
  \bvol{34}~(11),  \pg{114120}.

\bibitem[Bao {\em et~al.\/}(2022)Bao, Qiu, Zhou, Zhou, Weng, Lin \&
  You]{bao2022}
{\sc \au{Bao, Yue}, \au{Qiu, Ruofan}, \au{Zhou, Kang}, \au{Zhou, Tao},
  \au{Weng, Yuxin}, \au{Lin, Kai} \& \au{You, Yancheng}} \at{ \yr{2022} }
  \jt{Physics of Fluids}  \bvol{34},  \pg{046109}.

\bibitem[Betti \& Hurricane(2016)]{betti2016Inertial}
{\sc \au{Betti, R} \& \au{Hurricane, OA}} \yr{2016}  \at{Inertial-confinement
  fusion with lasers}.  \jt{Nature Physics}  \bvol{12}~(5),  \pg{435--448}.

\bibitem[Bhatnagar {\em et~al.\/}(1954)Bhatnagar, Gross \&
  Krook]{bhatnagar1954model}
{\sc \au{Bhatnagar, Prabhu~Lal}, \au{Gross, Eugene~P} \& \au{Krook, Max}} \at{
  \yr{1954} } \jt{Physical Review}  \bvol{94},  \pg{511}.

\bibitem[Bird(1998)]{bird1998recent}
{\sc \au{Bird, GA}} \at{ \yr{1998} } \jt{Computers \& Mathematics with
  Applications}  \bvol{35},  \pg{1--14}.

\bibitem[Bond {\em et~al.\/}(2017{\natexlab{{\em a\/}}})Bond, Wheatley,
  Samtaney \& Pullin]{bond2017}
{\sc \au{Bond, D}, \au{Wheatley, V}, \au{Samtaney, Ravi} \& \au{Pullin, DI}}
  \at{ \yr{2017{\natexlab{{\em a\/}}}} } \jt{Journal of Fluid Mechanics}
  \bvol{833},  \pg{332--363}.

\bibitem[Bond {\em et~al.\/}(2017{\natexlab{{\em b\/}}})Bond, Wheatley,
  Samtaney \& Pullin]{bond2017richtmyer}
{\sc \au{Bond, D}, \au{Wheatley, V}, \au{Samtaney, Ravi} \& \au{Pullin, DI}}
  \yr{2017{\natexlab{{\em b\/}}}}  \at{Richtmyer--meshkov instability of a
  thermal interface in a two-fluid plasma}.  \jt{Journal of Fluid Mechanics}
  \bvol{833},  \pg{332--363}.

\bibitem[Brouillette(2002)]{brouillette2002richtmyer}
{\sc \au{Brouillette, Martin}} \yr{2002}  \at{The richtmyer-meshkov
  instability}.  \jt{Annual Review of Fluid Mechanics}  \bvol{34}~(1),
  \pg{445--468}.

\bibitem[Cai {\em et~al.\/}(2021)Cai, Yan, Yao \& Zhu]{cai2021hybrid}
{\sc \au{Cai, Hong-bo}, \au{Yan, Xin-xin}, \au{Yao, Pei-lin} \& \au{Zhu,
  Shao-ping}} \yr{2021}  \at{Hybrid fluid--particle modeling of shock-driven
  hydrodynamic instabilities in a plasma}.  \jt{Matter and Radiation at
  Extremes}  \bvol{6}~(3),  \pg{035901}.

\bibitem[Cai {\em et~al.\/}(2020)Cai, Zhang, Du, Yan, Shan, Hao, Li, Zhang,
  Gong, Yang, Zou, Zhu \& He]{2020Cai-kinetic-effects}
{\sc \au{Cai, H.~B.}, \au{Zhang, W.~S.}, \au{Du, B.}, \au{Yan, X.~X.},
  \au{Shan, L.~Q.}, \au{Hao, L.}, \au{Li, Z.~C.}, \au{Zhang, F.}, \au{Gong,
  T.}, \au{Yang, D.}, \au{Zou, S.~Y.}, \au{Zhu, S.~P.} \& \au{He, X.~T.}} \at{
  \yr{2020} } \jt{High Power Laser and Particle Beams}  \bvol{32},
  \pg{092007}.

\bibitem[Cao {\em et~al.\/}(2005)Cao, Wu, Ren \& Li]{cao2008effects}
{\sc \au{Cao, Jintao}, \au{Wu, Zhengwei}, \au{Ren, Haijun} \& \au{Li, Ding}}
  \at{ \yr{2005} } \jt{Physics of Plasmas}  \bvol{15},  \pg{042102}.

\bibitem[Chen {\em et~al.\/}(2018{\natexlab{{\em a\/}}})Chen, Xu \&
  Zhang]{chen2018collaboration}
{\sc \au{Chen, Feng}, \au{Xu, Aiguo} \& \au{Zhang, Guangcai}} \at{
  \yr{2018{\natexlab{{\em a\/}}}} } \jt{Physics of Fluids}  \bvol{30},
  \pg{102105}.

\bibitem[Chen {\em et~al.\/}(2022{\natexlab{{\em a\/}}})Chen, Xu, Zhang, Gan,
  Liu \& Wang]{chen2022effects}
{\sc \au{Chen, Feng}, \au{Xu, Aiguo}, \au{Zhang, Yudong}, \au{Gan, Yanbiao},
  \au{Liu, Bingbing} \& \au{Wang, Shuang}} \at{ \yr{2022{\natexlab{{\em a\/}}}}
  } \jt{Frontiers of Physics}  \bvol{17},  \pg{1--13}.

\bibitem[Chen {\em et~al.\/}(2020)Chen, Xu, Zhang \&
  Zeng]{chen2020morphological}
{\sc \au{Chen, Feng}, \au{Xu, Aiguo}, \au{Zhang, Yudong} \& \au{Zeng, Qingkai}}
  \at{ \yr{2020} } \jt{Physics of Fluids}  \bvol{32},  \pg{104111}.

\bibitem[Chen {\em et~al.\/}(2016)Chen, Xu \& Zhang]{chen2016viscosity}
{\sc \au{Chen, Feng}, \au{Xu, Ai-Guo} \& \au{Zhang, Guang-Cai}} \at{ \yr{2016}
  } \jt{Frontiers of Physics}  \bvol{11},  \pg{1--14}.

\bibitem[Chen {\em et~al.\/}(2022{\natexlab{{\em b\/}}})Chen, Xu, Chen, Zhang
  \& Chen]{chen2022discrete}
{\sc \au{Chen, Jie}, \au{Xu, Aiguo}, \au{Chen, Dawei}, \au{Zhang, Yudong} \&
  \au{Chen, Zhihua}} \at{ \yr{2022{\natexlab{{\em b\/}}}} } \jt{Physical Review
  E}  \bvol{106},  \pg{015102}.

\bibitem[Chen {\em et~al.\/}(1991)Chen, Chen, Martnez \&
  Matthaeus]{chen1991lattice}
{\sc \au{Chen, Shiyi}, \au{Chen, Hudong}, \au{Martnez, Daniel} \&
  \au{Matthaeus, William}} \at{ \yr{1991} } \jt{Physical Review Letters}
  \bvol{67},  \pg{3776}.

\bibitem[Chen \& Zhao(2017)]{chen2017}
{\sc \au{Chen, W.~F.} \& \au{Zhao, W.W.}} \yr{2017} {\em Moment equations and
  numerical methods for rarefied gas flows (in Chinese)\/}, 1st edn.
  \publ{Science Press, Beijing}.

\bibitem[Chen {\em et~al.\/}(2018{\natexlab{{\em b\/}}})Chen, Shu \&
  Tan]{chen2018highly}
{\sc \au{Chen, Z}, \au{Shu, C} \& \au{Tan, D}} \at{ \yr{2018{\natexlab{{\em
  b\/}}}} } \jt{Physics of Fluids}  \bvol{30},  \pg{103605}.

\bibitem[Czelusniak {\em et~al.\/}(2022)Czelusniak, Mapelli, Wagner \&
  Cabezas-G{\'o}mez]{czelusniak2022shaping}
{\sc \au{Czelusniak, Luiz~Eduardo}, \au{Mapelli, Vin{\'\i}cius~Pessoa},
  \au{Wagner, Alexander~J} \& \au{Cabezas-G{\'o}mez, Luben}} \yr{2022}
  \at{Shaping the equation of state to improve numerical accuracy and stability
  of the pseudopotential lattice boltzmann method}.  \jt{Physical Review E}
  \bvol{105}~(1),  \pg{015303}.

\bibitem[De~Rosis {\em et~al.\/}(2021{\natexlab{{\em a\/}}})De~Rosis, Al-Adham,
  Al-Ali \& Meng]{de2021double}
{\sc \au{De~Rosis, Alessandro}, \au{Al-Adham, Joanne}, \au{Al-Ali, Hamda} \&
  \au{Meng, Ran}} \at{ \yr{2021{\natexlab{{\em a\/}}}} } \jt{Physics of Fluids}
   \bvol{33},  \pg{035143}.

\bibitem[De~Rosis {\em et~al.\/}(2021{\natexlab{{\em b\/}}})De~Rosis, Liu \&
  Revell]{de2021one}
{\sc \au{De~Rosis, Alessandro}, \au{Liu, Ruizhi} \& \au{Revell, Alistair}} \at{
  \yr{2021{\natexlab{{\em b\/}}}} } \jt{Physics of Fluids}  \bvol{8},
  \pg{085114}.

\bibitem[Dellar(2002)]{dellar2002lattice}
{\sc \au{Dellar, Paul~J}} \at{ \yr{2002} } \jt{Journal of Computational
  Physics}  \bvol{179},  \pg{95--126}.

\bibitem[Gan {\em et~al.\/}(2022)Gan, Xu, Lai, Li, Sun \& Succi]{gan_xu2022}
{\sc \au{Gan, Yanbiao}, \au{Xu, Aiguo}, \au{Lai, Huilin}, \au{Li, Wei},
  \au{Sun, Guanglan} \& \au{Succi, Sauro}} \yr{2022}  \at{Discrete boltzmann
  multi-scale modelling of non-equilibrium multiphase flows}.  \jt{Journal of
  Fluid Mechanics}  \bvol{951},  \pg{A8}.

\bibitem[Gan {\em et~al.\/}(2013)Gan, Xu, Zhang \& Yang]{gan2013lattice}
{\sc \au{Gan, Yanbiao}, \au{Xu, Aiguo}, \au{Zhang, Guangcai} \& \au{Yang,
  Yang}} \yr{2013}  \at{Lattice bgk kinetic model for high-speed compressible
  flows: Hydrodynamic and nonequilibrium behaviors}.  \jt{EPL (Europhysics
  Letters)}  \bvol{103}~(2),  \pg{24003}.

\bibitem[Gan {\em et~al.\/}(2018)Gan, Xu, Zhang, Zhang \&
  Succi]{gan2018discrete}
{\sc \au{Gan, Yanbiao}, \au{Xu, Aiguo}, \au{Zhang, Guangcai}, \au{Zhang,
  Yudong} \& \au{Succi, Sauro}} \yr{2018}  \at{Discrete boltzmann trans-scale
  modeling of high-speed compressible flows}.  \jt{Physical Review E}
  \bvol{97}~(5),  \pg{053312}.

\bibitem[Huang {\em et~al.\/}(2021)Huang, Tian, Young \&
  Lai]{huang2021transition}
{\sc \au{Huang, Qiuxiang}, \au{Tian, Fang-Bao}, \au{Young, John} \& \au{Lai,
  Joseph~CS}} \yr{2021}  \at{Transition to chaos in a two-sided collapsible
  channel flow}.  \jt{Journal of Fluid Mechanics}  \bvol{926},  \pg{A15}.

\bibitem[Ji \& Held(2013)]{ji2013closure}
{\sc \au{Ji, Jeong-Young} \& \au{Held, Eric~D}} \yr{2013}  \at{Closure and
  transport theory for high-collisionality electron-ion plasmas}.  \jt{Physics
  of Plasmas}  \bvol{20}~(4),  \pg{042114}.

\bibitem[Jiang \& Wu(1999)]{jiang1999}
{\sc \au{Jiang, Guang-Shan} \& \au{Wu, Cheng-chin}} \at{ \yr{1999} }
  \jt{Journal of Computational Physics}  \bvol{150},  \pg{561--594}.

\bibitem[Keenan {\em et~al.\/}(2017)Keenan, Simakov, Chac{\'o}n \&
  Taitano]{keenan2017deciphering}
{\sc \au{Keenan, Brett~D}, \au{Simakov, Andrei~N}, \au{Chac{\'o}n, Luis} \&
  \au{Taitano, William~T}} \at{ \yr{2017} } \jt{Physical Review E}  \bvol{96},
  \pg{053203}.

\bibitem[Khokhlov {\em et~al.\/}(1999)Khokhlov, Oran \&
  Thomas]{khokhlov1999numerical}
{\sc \au{Khokhlov, AM}, \au{Oran, ES} \& \au{Thomas, GO}} \yr{1999}
  \at{Numerical simulation of deflagration-to-detonation transition: the role
  of shock--flame interactions in turbulent flames}.  \jt{Combustion and flame}
   \bvol{117}~(1-2),  \pg{323--339}.

\bibitem[Kumar \& Maheshwari(2020)]{kumar2020viscous}
{\sc \au{Kumar, GN~Sashi} \& \au{Maheshwari, NK}} \yr{2020}  \at{Viscous
  multi-species lattice boltzmann solver for simulating shock-wave structure}.
  \jt{Computers \& Fluids}  \bvol{203},  \pg{104539}.

\bibitem[Lai {\em et~al.\/}(2016)Lai, Xu, Zhang, Gan, Ying \&
  Succi]{lai2016nonequilibrium}
{\sc \au{Lai, Huilin}, \au{Xu, Aiguo}, \au{Zhang, Guangcai}, \au{Gan, Yanbiao},
  \au{Ying, Yangjun} \& \au{Succi, Sauro}} \at{ \yr{2016} } \jt{Physical Review
  E}  \bvol{94},  \pg{023106}.

\bibitem[Larroche {\em et~al.\/}(2016)Larroche, Rinderknecht, Rosenberg,
  Hoffman, Atzeni, Petrasso, Amendt \& S{\'e}guin]{larroche2016ion}
{\sc \au{Larroche, O}, \au{Rinderknecht, HG}, \au{Rosenberg, MJ}, \au{Hoffman,
  NM}, \au{Atzeni, S}, \au{Petrasso, RD}, \au{Amendt, PA} \& \au{S{\'e}guin,
  FH}} \yr{2016}  \at{Ion-kinetic simulations of d-3he gas-filled inertial
  confinement fusion target implosions with moderate to large knudsen number}.
  \jt{Physics of Plasmas}  \bvol{23}~(1),  \pg{012701}.

\bibitem[Ledesma-Aguilar {\em et~al.\/}(2014)Ledesma-Aguilar, Vella \&
  Yeomans]{ledesma2014lattice}
{\sc \au{Ledesma-Aguilar, Rodrigo}, \au{Vella, Dominic} \& \au{Yeomans,
  Julia~M}} \yr{2014}  \at{Lattice-boltzmann simulations of droplet
  evaporation}.  \jt{Soft Matter}  \bvol{10}~(41),  \pg{8267--8275}.

\bibitem[Lei {\em et~al.\/}(2017)Lei, Ding, Si, Zhai \&
  Luo]{lei2017experimental}
{\sc \au{Lei, Fan}, \au{Ding, Juchun}, \au{Si, Ting}, \au{Zhai, Zhigang} \&
  \au{Luo, Xisheng}} \yr{2017}  \at{Experimental study on a sinusoidal air/sf
  interface accelerated by a cylindrically converging shock}.  \jt{Journal of
  Fluid Mechanics}  \bvol{826},  \pg{819--829}.

\bibitem[Li {\em et~al.\/}(2022{\natexlab{{\em a\/}}})Li, Xu, Zhang \&
  Shan]{Li_2022}
{\sc \au{Li, Hanwei}, \au{Xu, Aiguo}, \au{Zhang, Ge} \& \au{Shan, Yiming}}
  \yr{2022{\natexlab{{\em a\/}}}}  \at{Rayleigh–taylor instability under
  multi-mode perturbation: Discrete boltzmann modeling with tracers}.
  \jt{Communications in Theoretical Physics}  \bvol{74}~(11),  \pg{115601}.

\bibitem[Li {\em et~al.\/}(2012)Li, Luo, Gao \& He]{li2012additional}
{\sc \au{Li, Qing}, \au{Luo, KH}, \au{Gao, YJ} \& \au{He, YL}} \at{ \yr{2012} }
  \jt{Physical Review E}  \bvol{85},  \pg{026704}.

\bibitem[Li {\em et~al.\/}(2016)Li, Luo, Kang, He, Chen \& Liu]{li2016lattice}
{\sc \au{Li, Qing}, \au{Luo, Kai~Hong}, \au{Kang, QJ}, \au{He, YL}, \au{Chen,
  Q} \& \au{Liu, Q}} \yr{2016}  \at{Lattice boltzmann methods for multiphase
  flow and phase-change heat transfer}.  \jt{Progress in Energy and Combustion
  Science}  \bvol{52},  \pg{62--105}.

\bibitem[Li {\em et~al.\/}(2022{\natexlab{{\em b\/}}})Li, Bakhsh \&
  Samtaney]{li2022linear}
{\sc \au{Li, Yuan}, \au{Bakhsh, Abeer} \& \au{Samtaney, Ravi}}
  \yr{2022{\natexlab{{\em b\/}}}}  \at{Linear stability of an impulsively
  accelerated density interface in an ideal two-fluid plasma}.  \jt{Physics of
  Fluids}  \bvol{34}~(3),  \pg{036103}.

\bibitem[Li {\em et~al.\/}(2022{\natexlab{{\em c\/}}})Li, Lai, Lin \&
  Li]{li2022influence}
{\sc \au{Li, Yaofeng}, \au{Lai, Huilin}, \au{Lin, Chuandong} \& \au{Li, Demei}}
  \yr{2022{\natexlab{{\em c\/}}}}  \at{Influence of the tangential velocity on
  the compressible kelvin-helmholtz instability with nonequilibrium effects}.
  \jt{Frontiers of Physics}  \bvol{17}~(6),  \pg{1--17}.

\bibitem[Li {\em et~al.\/}(2015)Li, Peng, Zhang \& Yang]{li2015rarefied}
{\sc \au{Li, Zhi-Hui}, \au{Peng, Ao-Ping}, \au{Zhang, Han-Xin} \& \au{Yang,
  Jaw-Yen}} \at{ \yr{2015} } \jt{Progress in Aerospace Sciences}  \bvol{74},
  \pg{81--113}.

\bibitem[Lin \& Luo(2019)]{lin2019discrete}
{\sc \au{Lin, Chuandong} \& \au{Luo, Kai~H}} \yr{2019}  \at{Discrete boltzmann
  modeling of unsteady reactive flows with nonequilibrium effects}.
  \jt{Physical Review E}  \bvol{99}~(1),  \pg{012142}.

\bibitem[Lin {\em et~al.\/}(2021)Lin, Luo, Xu, Gan \& Lai]{lin2021multiple}
{\sc \au{Lin, Chuandong}, \au{Luo, Kai~H}, \au{Xu, Aiguo}, \au{Gan, Yanbiao} \&
  \au{Lai, Huilin}} \at{ \yr{2021} } \jt{Physical Review E}  \bvol{103},
  \pg{013305}.

\bibitem[Lin {\em et~al.\/}(2017)Lin, Xu, Zhang, Luo \& Li]{lin2017discrete}
{\sc \au{Lin, Chuandong}, \au{Xu, Aiguo}, \au{Zhang, Guangcai}, \au{Luo,
  Kai~Hong} \& \au{Li, Yingjun}} \at{ \yr{2017} } \jt{Physical Review E}
  \bvol{96},  \pg{053305}.

\bibitem[Liu \& Xu(2017)]{liu2017unified}
{\sc \au{Liu, Chang} \& \au{Xu, Kun}} \yr{2017}  \at{A unified gas kinetic
  scheme for continuum and rarefied flows v: multiscale and multi-component
  plasma transport}.  \jt{Communications in Computational Physics}
  \bvol{22}~(5),  \pg{1175--1223}.

\bibitem[Liu {\em et~al.\/}(2020)Liu, Yu, Chen, Zhang, Xu \&
  Liu]{liu2020contribution}
{\sc \au{Liu, Hao-Chen}, \au{Yu, Bin}, \au{Chen, Hao}, \au{Zhang, Bin}, \au{Xu,
  Hui} \& \au{Liu, Hong}} \yr{2020}  \at{Contribution of viscosity to the
  circulation deposition in the richtmyer--meshkov instability}.  \jt{Journal
  of Fluid Mechanics}  \bvol{895},  \pg{A10}.

\bibitem[Liu {\em et~al.\/}(2022)Liu, Song, Xu, Zhang \& Xie]{liu2022discrete}
{\sc \au{Liu, Zhipeng}, \au{Song, Jiahui}, \au{Xu, Aiguo}, \au{Zhang, Yudong}
  \& \au{Xie, Kan}} \at{ \yr{2022} } \jt{Proceedings of the Institution of
  Mechanical Engineers, Part C: Journal of Mechanical Engineering Science}
  \pg{p. 09544062221075943}.

\bibitem[McMullen {\em et~al.\/}(2022)McMullen, Krygier, Torczynski \&
  Gallis]{mcmullen2022}
{\sc \au{McMullen, Ryan~M}, \au{Krygier, Michael~C}, \au{Torczynski, John~R} \&
  \au{Gallis, Michael~A}} \at{ \yr{2022} } \jt{Physical Review Letters}
  \bvol{128},  \pg{114501}.

\bibitem[Meng {\em et~al.\/}(2019)Meng, Zeng, Tian, Zhou \&
  Shen]{meng2019modeling}
{\sc \au{Meng, Baoqing}, \au{Zeng, Junsheng}, \au{Tian, Baolin}, \au{Zhou, Rui}
  \& \au{Shen, Weidong}} \yr{2019}  \at{Modeling and simulation of a
  single-mode multiphase richtmyer--meshkov instability with a large stokes
  number}.  \jt{AIP Advances}  \bvol{9}~(12),  \pg{125311}.

\bibitem[Meshkov(1969)]{meshkov1969instability}
{\sc \au{Meshkov, \_E~E}} \yr{1969}  \at{Instability of the interface of two
  gases accelerated by a shock wave}.  \jt{Fluid Dynamics}  \bvol{4}~(5),
  \pg{101--104}.

\bibitem[Mostert {\em et~al.\/}(2017)Mostert, Pullin, Wheatley \&
  Samtaney]{mostert2017magnetohydrodynamic}
{\sc \au{Mostert, Wouter}, \au{Pullin, Dale~I}, \au{Wheatley, Vincent} \&
  \au{Samtaney, Ravi}} \yr{2017}  \at{Magnetohydrodynamic implosion symmetry
  and suppression of richtmyer-meshkov instability in an octahedrally symmetric
  field}.  \jt{Physical Review Fluids}  \bvol{2}~(1),  \pg{013701}.

\bibitem[Mostert {\em et~al.\/}(2015)Mostert, Wheatley, Samtaney \&
  Pullin]{mostert2015}
{\sc \au{Mostert, Wouter}, \au{Wheatley, Vincent}, \au{Samtaney, Ravi} \&
  \au{Pullin, Dale~I}} \at{ \yr{2015} } \jt{Physics of Fluids}  \bvol{27},
  \pg{104102}.

\bibitem[Orszag \& Tang(1979)]{orszag1979small}
{\sc \au{Orszag, Steven~A} \& \au{Tang, Cha-Mei}} \yr{1979}  \at{Small-scale
  structure of two-dimensional magnetohydrodynamic turbulence}.  \jt{Journal of
  Fluid Mechanics}  \bvol{90}~(1),  \pg{129--143}.

\bibitem[Pattison {\em et~al.\/}(2008)Pattison, Premnath, Morley \&
  Abdou]{pattison2008progress}
{\sc \au{Pattison, MJ}, \au{Premnath, KN}, \au{Morley, NB} \& \au{Abdou, MA}}
  \at{ \yr{2008} } \jt{Fusion Engineering and Design}  \bvol{83},
  \pg{557--572}.

\bibitem[Qin \& Dong(2021)]{qin2021richtmyer}
{\sc \au{Qin, Jianhua} \& \au{Dong, Guodan}} \yr{2021}  \at{The
  richtmyer--meshkov instability of concave circular arc density interfaces in
  hydrodynamics and magnetohydrodynamics}.  \jt{Physics of Fluids}
  \bvol{33}~(3),  \pg{034122}.

\bibitem[Qiu {\em et~al.\/}(2020)Qiu, Bao, Zhou, Che, Chen \& You]{qiu2020}
{\sc \au{Qiu, Ruofan}, \au{Bao, Yue}, \au{Zhou, Tao}, \au{Che, Huanhuan},
  \au{Chen, Rongqian} \& \au{You, Yancheng}} \at{ \yr{2020} } \jt{Physics of
  Fluids}  \bvol{32},  \pg{106106}.

\bibitem[Qiu {\em et~al.\/}(2021)Qiu, Zhou, Bao, Zhou, Che \& You]{qiu2021}
{\sc \au{Qiu, Ruofan}, \au{Zhou, Tao}, \au{Bao, Yue}, \au{Zhou, Kang}, \au{Che,
  Huanhuan} \& \au{You, Yancheng}} \at{ \yr{2021} } \jt{Physical Review E}
  \bvol{103},  \pg{053113}.

\bibitem[Qiu {\em et~al.\/}(2008)Qiu, Wu, Cao \& Li]{qiu2008effects}
{\sc \au{Qiu, Zhiyong}, \au{Wu, Zhengwei}, \au{Cao, Jintao} \& \au{Li, Ding}}
  \at{ \yr{2008} } \jt{Physics of Plasmas}  \bvol{15},  \pg{042305}.

\bibitem[Richtmyer;(1960)]{Richtmyer1960}
{\sc \au{Richtmyer;, Robert~D.}} \yr{1960}  \at{Taylor instability in shock
  acceleration of compressible fluids}.  \jt{Communications on Pure and Applied
  Mathematics}  \pg{pp. 297--319}.

\bibitem[Rinderknecht {\em et~al.\/}(2018)Rinderknecht, Amendt, Wilks \&
  Collins]{rinderknecht2018}
{\sc \au{Rinderknecht, Hans~G}, \au{Amendt, PA}, \au{Wilks, SC} \& \au{Collins,
  G}} \at{ \yr{2018} } \jt{Plasma Physics and Controlled Fusion}  \bvol{60},
  \pg{064001}.

\bibitem[Robey(2004)]{robey2004effects}
{\sc \au{Robey, HF}} \yr{2004}  \at{Effects of viscosity and mass diffusion in
  hydrodynamically unstable plasma flows}.  \jt{Physics of plasmas}
  \bvol{11}~(8),  \pg{4123--4133}.

\bibitem[Samtaney(2003)]{samtaney2003}
{\sc \au{Samtaney, Ravi}} \at{ \yr{2003} } \jt{Physics of Fluids}  \bvol{15},
  \pg{L53--L56}.

\bibitem[Sano {\em et~al.\/}(2013)Sano, Inoue \& Nishihara]{sano2013}
{\sc \au{Sano, Takayoshi}, \au{Inoue, Tsuyoshi} \& \au{Nishihara, Katsunobu}}
  \at{ \yr{2013} } \jt{Physical review letters}  \bvol{111},  \pg{205001}.

\bibitem[Sano {\em et~al.\/}(2012)Sano, Nishihara, Matsuoka \& Inoue]{Sano2012}
{\sc \au{Sano, Takayoshi}, \au{Nishihara, Katsunobu}, \au{Matsuoka, Chihiro} \&
  \au{Inoue, Tsuyoshi}} \at{ \yr{2012} } \jt{The Astrophysical Journal}
  \bvol{758},  \pg{126}.

\bibitem[Sano {\em et~al.\/}(2021)Sano, Tamatani, Matsuo, Law, Morita,
  Egashira, Ota, Kumar, Shimogawara, Hara {\em et~al.\/}]{sano2021laser}
{\sc \au{Sano, Takayoshi}, \au{Tamatani, Shohei}, \au{Matsuo, Kazuki}, \au{Law,
  King Fai~Farley}, \au{Morita, Taichi}, \au{Egashira, Shunsuke}, \au{Ota,
  Masato}, \au{Kumar, Rajesh}, \au{Shimogawara, Hiroshi}, \au{Hara, Yukiko} \&
  \au{others}} \yr{2021}  \at{Laser astrophysics experiment on the
  amplification of magnetic fields by shock-induced interfacial instabilities}.
   \jt{Physical Review E}  \bvol{104}~(3),  \pg{035206}.

\bibitem[Shan {\em et~al.\/}(2021)Shan, Wu, Yuan, Wang, Cai, Tian, Zhang,
  Zhang, Deng, Zhang, Teng, Bi, Yang, Yang, Zhou, Gu, Zhang \&
  Zhu]{2021Shan-kinetic-effects}
{\sc \au{Shan, L.~Q.}, \au{Wu, F.~J.}, \au{Yuan, Z.~Q.}, \au{Wang, W.~W.},
  \au{Cai, H.~B.}, \au{Tian, C.}, \au{Zhang, F.}, \au{Zhang, T.~K.}, \au{Deng,
  Z.~G.}, \au{Zhang, W.~S.}, \au{Teng, J.}, \au{Bi, B.}, \au{Yang, S.~Q.},
  \au{Yang, D.}, \au{Zhou, W.~M.}, \au{Gu, Y.~Q.}, \au{Zhang, B.~H.} \&
  \au{Zhu, S.~P.}} \at{ \yr{2021} } \jt{High Power Laser and Particle Beams}
  \bvol{33},  \pg{012004}.

\bibitem[Sofonea {\em et~al.\/}(2004)Sofonea, Lamura, Gonnella \&
  Cristea]{sofonea2004finite}
{\sc \au{Sofonea, V}, \au{Lamura, A}, \au{Gonnella, G} \& \au{Cristea, A}}
  \yr{2004}  \at{Finite-difference lattice boltzmann model with flux limiters
  for liquid-vapor systems}.  \jt{Physical Review E}  \bvol{70}~(4),
  \pg{046702}.

\bibitem[Sofonea \& Sekerka(2003)]{sofonea2003viscosity}
{\sc \au{Sofonea, Victor} \& \au{Sekerka, Robert~F}} \yr{2003}  \at{Viscosity
  of finite difference lattice boltzmann models}.  \jt{Journal of Computational
  Physics}  \bvol{184}~(2),  \pg{422--434}.

\bibitem[Succi(2001)]{succi2001lattice}
{\sc \au{Succi, Sauro}} \yr{2001} {\em The lattice Boltzmann equation: for
  fluid dynamics and beyond\/}.  \publ{Oxford university press}.

\bibitem[Tapinou {\em et~al.\/}(2023)Tapinou, Wheatley, Bond \&
  Jahn]{tapinou2023effect}
{\sc \au{Tapinou, KC}, \au{Wheatley, Vincent}, \au{Bond, D} \& \au{Jahn, I}}
  \yr{2023}  \at{The effect of collisions on the multi-fluid plasma
  richtmyer--meshkov instability}.  \jt{Physics of Plasmas}  \bvol{30}~(2),
  \pg{022707}.

\bibitem[Vidal {\em et~al.\/}(1993)Vidal, Matte, Casanova \&
  Larroche]{vidal1993ion}
{\sc \au{Vidal, F}, \au{Matte, JP}, \au{Casanova, M} \& \au{Larroche, O}} \at{
  \yr{1993} } \jt{Physics of Fluids B: Plasma Physics}  \bvol{5},
  \pg{3182--3190}.

\bibitem[Wagner \& Yeomans(1998)]{wagner1998breakdown}
{\sc \au{Wagner, Alexander~J} \& \au{Yeomans, JM}} \yr{1998}  \at{Breakdown of
  scale invariance in the coarsening of phase-separating binary fluids}.
  \jt{Physical Review Letters}  \bvol{80}~(7),  \pg{1429}.

\bibitem[Wang {\em et~al.\/}(2017)Wang, Ye, He, Wu, Fan, Xue, Guo, Miao, Yuan,
  Dong {\em et~al.\/}]{wang2017theoretical}
{\sc \au{Wang, LiFeng}, \au{Ye, WenHua}, \au{He, XianTu}, \au{Wu, JunFeng},
  \au{Fan, ZhengFeng}, \au{Xue, Chuang}, \au{Guo, HongYu}, \au{Miao, WenYong},
  \au{Yuan, YongTeng}, \au{Dong, JiaQin} \& \au{others}} \yr{2017}
  \at{Theoretical and simulation research of hydrodynamic instabilities in
  inertial-confinement fusion implosions}.  \jt{Science China Physics,
  Mechanics \& Astronomy}  \bvol{60},  \pg{1--35}.

\bibitem[Wang {\em et~al.\/}(2020)Wang, Zhong, Cao, Zhuo \&
  Liu]{wang2020simplified}
{\sc \au{Wang, Yong}, \au{Zhong, Chengwen}, \au{Cao, Jun}, \au{Zhuo, Congshan}
  \& \au{Liu, Sha}} \at{ \yr{2020} } \jt{Computers \& Mathematics with
  Applications}  \bvol{79},  \pg{1590--1618}.

\bibitem[Wheatley {\em et~al.\/}(2005{\natexlab{{\em a\/}}})Wheatley, Pullin \&
  Samtaney]{wheatley2005}
{\sc \au{Wheatley, V}, \au{Pullin, DI} \& \au{Samtaney, R}} \at{
  \yr{2005{\natexlab{{\em a\/}}}} } \jt{Journal of Fluid Mechanics}
  \bvol{522},  \pg{179--214}.

\bibitem[Wheatley {\em et~al.\/}(2005{\natexlab{{\em b\/}}})Wheatley, Pullin \&
  Samtaney]{wheatley2005stability}
{\sc \au{Wheatley, Vincent}, \au{Pullin, DI} \& \au{Samtaney, R}} \at{
  \yr{2005{\natexlab{{\em b\/}}}} } \jt{Physical review letters}  \bvol{95},
  \pg{125002}.

\bibitem[Wheatley {\em et~al.\/}(2009)Wheatley, Samtaney \&
  Pullin]{wheatley2009richtmyer}
{\sc \au{Wheatley, Vincent}, \au{Samtaney, R} \& \au{Pullin, DI}} \at{
  \yr{2009} } \jt{Physics of Fluids}  \bvol{21},  \pg{082102}.

\bibitem[Wheatley {\em et~al.\/}(2014)Wheatley, Samtaney, Pullin \&
  Gehre]{wheatley2014transverse}
{\sc \au{Wheatley, V}, \au{Samtaney, Ravi}, \au{Pullin, DI} \& \au{Gehre, RM}}
  \at{ \yr{2014} } \jt{Physics of Fluids}  \bvol{1},  \pg{016102}.

\bibitem[Xu {\em et~al.\/}(2021{\natexlab{{\em a\/}}})Xu, Chen, Song, Chen \&
  Chen]{xu2021progress}
{\sc \au{Xu, Aiguo}, \au{Chen, Jie}, \au{Song, Jiahui}, \au{Chen, Dawei} \&
  \au{Chen, Zhihua}} \yr{2021{\natexlab{{\em a\/}}}}  \at{Progress of discrete
  boltzmann study on multiphase complex flows}.  \jt{Acta Aerodyn.Sin}
  \bvol{39}~(3),  \pg{138--169}.

\bibitem[Xu {\em et~al.\/}(2021{\natexlab{{\em b\/}}})Xu, Jiahui, Feng, Kan \&
  Yangjun]{xu2021phase}
{\sc \au{Xu, Aiguo}, \au{Jiahui, Song}, \au{Feng, Chen}, \au{Kan, Xie} \&
  \au{Yangjun, Ying}} \at{ \yr{2021{\natexlab{{\em b\/}}}} } \jt{Chinese
  Journal of Computational Physics}  \bvol{38},  \pg{631}.

\bibitem[Xu {\em et~al.\/}(2015)Xu, Lin, Zhang \& Li]{xu2015multiple}
{\sc \au{Xu, Aiguo}, \au{Lin, Chuandong}, \au{Zhang, Guangcai} \& \au{Li,
  Yingjun}} \at{ \yr{2015} } \jt{Physical Review E}  \bvol{91},  \pg{043306}.

\bibitem[Xu {\em et~al.\/}(2021{\natexlab{{\em c\/}}})Xu, Shan, Chen, Gan \&
  Lin]{xu2021Progressofmesoscale}
{\sc \au{Xu, Aiguo}, \au{Shan, Yi~Ming}, \au{Chen, Feng}, \au{Gan, Yanbiao} \&
  \au{Lin, Chuandong}} \yr{2021{\natexlab{{\em c\/}}}}  \at{Progress of
  mesoscale modeling and investigation of combustion multiphase flow}.
  \jt{Acta Aeronauticaet Astronautica Sinica}  \bvol{42}~(12),  \pg{46--62}.

\bibitem[Xu {\em et~al.\/}(2021{\natexlab{{\em d\/}}})Xu, Song, Chen, Xie \&
  Ying]{xu2021modeling}
{\sc \au{Xu, Aiguo}, \au{Song, Jiahui}, \au{Chen, Feng}, \au{Xie, Kan} \&
  \au{Ying, Yangjun}} \yr{2021{\natexlab{{\em d\/}}}}  \at{Modeling and
  analysis methods for complex field sbased on phase space}.  \jt{Chinese
  Journal of Computational Physics}  \bvol{38}~(6),  \pg{631--660}.

\bibitem[Xu {\em et~al.\/}(2012)Xu, Zhang, Gan, Chen \& Yu]{xu2012lattice}
{\sc \au{Xu, Aiguo}, \au{Zhang, Guangcai}, \au{Gan, Yanbiao}, \au{Chen, Feng}
  \& \au{Yu, Xijun}} \at{ \yr{2012} } \jt{Frontiers of Physics}  \bvol{7},
  \pg{582--600}.

\bibitem[Xu {\em et~al.\/}(2016)Xu, Zhang, Ying \& Wang]{xu2016complex}
{\sc \au{Xu, AiGuo}, \au{Zhang, GuangCai}, \au{Ying, YangJun} \& \au{Wang,
  Cheng}} \yr{2016}  \at{Complex fields in heterogeneous materials under shock:
  Modeling, simulation and analysis}.  \jt{Science China Physics, Mechanics \&
  Astronomy}  \bvol{59},  \pg{1--49}.

\bibitem[Xu {\em et~al.\/}(2018)Xu, Zhang \& Zhang]{Xu2018-Chap2}
{\sc \au{Xu, A.}, \au{Zhang, G.} \& \au{Zhang, Y.}} \yr{2018}  \at{{Discrete}
  {Boltzmann} {Modeling} of {Compressible} {Flows}}.  \bt{In {\em Kinetic
  Theory\/} (ed. \ed{G.~Kyzas \& A.~Mitropoulos})}, chap.~02.  \publ{Rijeka:
  InTech}.

\bibitem[Xu \& Zhang(2022)]{xu2022complex}
{\sc \au{Xu, Aiguo} \& \au{Zhang, Yudong}} \yr{2022} {\em Complex Media
  Kinetics\/}, 1st edn.  \publ{Science Press, Beijing}.

\bibitem[Xu \& Huang(2010)]{xu2010unified}
{\sc \au{Xu, Kun} \& \au{Huang, Juan-Chen}} \yr{2010}  \at{A unified
  gas-kinetic scheme for continuum and rarefied flows}.  \jt{Journal of
  Computational Physics}  \bvol{229}~(20),  \pg{7747--7764}.

\bibitem[Yan {\em et~al.\/}(2021)Yan, Cai, Yao, Huang, Zhang, Zhang, Du, Zhu \&
  He]{yan2021ion}
{\sc \au{Yan, XX}, \au{Cai, HB}, \au{Yao, PL}, \au{Huang, HX}, \au{Zhang, EH},
  \au{Zhang, WS}, \au{Du, B}, \au{Zhu, SP} \& \au{He, XT}} \yr{2021}  \at{Ion
  kinetic effects on the evolution of richtmyer--meshkov instability and
  interfacial mix}.  \jt{New Journal of Physics}  \bvol{23}~(5),  \pg{053010}.

\bibitem[Yao {\em et~al.\/}(2020)Yao, Cai, Yan, Zhang, Du, Tian, Zhang, Wang \&
  Zhu]{Yao2020Kinetic}
{\sc \au{Yao, Pei~Lin}, \au{Cai, Hong~Bo}, \au{Yan, Xin~Xin}, \au{Zhang,
  Wen~Shuai}, \au{Du, Bao}, \au{Tian, Jian~Min}, \au{Zhang, En~Hao}, \au{Wang,
  Xue~Wu} \& \au{Zhu, Shao~Ping}} \at{ \yr{2020} } \jt{Matter and Radiation at
  Extremes}  \bvol{5},  \pg{054403}.

\bibitem[Ye {\em et~al.\/}(2020)Ye, Lai, Li, Gan, Lin, Chen \&
  Xu]{ye2020knudsen}
{\sc \au{Ye, Haiyan}, \au{Lai, Huilin}, \au{Li, Demei}, \au{Gan, Yanbiao},
  \au{Lin, Chuandong}, \au{Chen, Lu} \& \au{Xu, Aiguo}} \at{ \yr{2020} }
  \jt{Entropy}  \bvol{22},  \pg{500}.

\bibitem[Zhai {\em et~al.\/}(2011)Zhai, Si, Luo \& Yang]{zhai2011evolution}
{\sc \au{Zhai, Zhigang}, \au{Si, Ting}, \au{Luo, Xisheng} \& \au{Yang, Jiming}}
  \yr{2011}  \at{On the evolution of spherical gas interfaces accelerated by a
  planar shock wave}.  \jt{Physics of Fluids}  \bvol{23}~(8),  \pg{084104}.

\bibitem[Zhai {\em et~al.\/}(2018)Zhai, Zou, Wu \& Luo]{zhai2018review}
{\sc \au{Zhai, Zhigang}, \au{Zou, Liyong}, \au{Wu, Qiang} \& \au{Luo, Xisheng}}
  \yr{2018}  \at{Review of experimental richtmyer--meshkov instability in shock
  tube: from simple to complex}.  \jt{Proceedings of the Institution of
  Mechanical Engineers, Part C: Journal of Mechanical Engineering Science}
  \bvol{232}~(16),  \pg{2830--2849}.

\bibitem[Zhang {\em et~al.\/}(2022{\natexlab{{\em a\/}}})Zhang, Xu, Zhang, Gan
  \& Li]{zhang2022discrete}
{\sc \au{Zhang, Dejia}, \au{Xu, Aiguo}, \au{Zhang, Yudong}, \au{Gan, Yanbiao}
  \& \au{Li, Yingjun}} \yr{2022{\natexlab{{\em a\/}}}}  \at{Discrete boltzmann
  modeling of high-speed compressible flows with various depths of
  non-equilibrium}.  \jt{Physics of Fluids}  \bvol{34}~(8),  \pg{086104}.

\bibitem[Zhang {\em et~al.\/}(2020)Zhang, Zheng, Aubry, Wu \&
  Chen]{zhang2020numerical}
{\sc \au{Zhang, Huan-Hao}, \au{Zheng, Chun}, \au{Aubry, Nadine}, \au{Wu,
  Wei-Tao} \& \au{Chen, Zhi-Hua}} \yr{2020}  \at{Numerical analysis of
  richtmyer--meshkov instability of circular density interface in presence of
  transverse magnetic field}.  \jt{Physics of Fluids}  \bvol{32}~(11),
  \pg{116104}.

\bibitem[Zhang {\em et~al.\/}(2023)Zhang, Zhang, Chen \&
  Zheng]{zhang2023suppression}
{\sc \au{Zhang, Sheng-Bo}, \au{Zhang, Huan-Hao}, \au{Chen, Zhi-Hua} \&
  \au{Zheng, Chun}} \yr{2023}  \at{Suppression mechanism of richtmyer--meshkov
  instability by transverse magnetic field with different strengths}.
  \jt{Physics of Plasmas}  \bvol{30}~(2),  \pg{022107}.

\bibitem[Zhang {\em et~al.\/}(2022{\natexlab{{\em b\/}}})Zhang, Xu, Chen, Lin
  \& Wei]{zhang2022non}
{\sc \au{Zhang, Yudong}, \au{Xu, Aiguo}, \au{Chen, Feng}, \au{Lin, Chuandong}
  \& \au{Wei, Zon-Han}} \at{ \yr{2022{\natexlab{{\em b\/}}}} } \jt{AIP
  Advances}  \bvol{12},  \pg{035347}.

\bibitem[Zhang {\em et~al.\/}(2019{\natexlab{{\em a\/}}})Zhang, Xu, Zhang, Chen
  \& Wang]{zhang2019discrete}
{\sc \au{Zhang, Yudong}, \au{Xu, Aiguo}, \au{Zhang, Guangcai}, \au{Chen,
  Zhihua} \& \au{Wang, Pei}} \yr{2019{\natexlab{{\em a\/}}}}  \at{Discrete
  boltzmann method for non-equilibrium flows: Based on shakhov model}.
  \jt{Computer Physics Communications}  \bvol{238},  \pg{50--65}.

\bibitem[Zhang {\em et~al.\/}(2019{\natexlab{{\em b\/}}})Zhang, Xu, Zhang, Gan,
  Chen \& Succi]{zhang2019entropy}
{\sc \au{Zhang, Yudong}, \au{Xu, Aiguo}, \au{Zhang, Guangcai}, \au{Gan,
  Yanbiao}, \au{Chen, Zhihua} \& \au{Succi, Sauro}} \at{
  \yr{2019{\natexlab{{\em b\/}}}} } \jt{Soft matter}  \bvol{15},
  \pg{2245--2259}.

\bibitem[Zhang {\em et~al.\/}(2016)Zhang, Xu, Zhang, Zhu \&
  Lin]{zhang2016kinetic}
{\sc \au{Zhang, Yudong}, \au{Xu, Aiguo}, \au{Zhang, Guangcai}, \au{Zhu,
  Chengmin} \& \au{Lin, Chuandong}} \at{ \yr{2016} } \jt{Combustion and Flame}
  \bvol{173},  \pg{483--492}.

\bibitem[Zhang {\em et~al.\/}(2018)Zhang, Xu, Zhang \& Chen]{zhang2018discrete}
{\sc \au{Zhang, Yu-Dong}, \au{Xu, Ai-Guo}, \au{Zhang, Guang-Cai} \& \au{Chen,
  Zhi-Hua}} \at{ \yr{2018} } \jt{Communications in Theoretical Physics}
  \bvol{69},  \pg{77}.

\bibitem[Zhou(2017{\natexlab{{\em a\/}}})]{zhou20171}
{\sc \au{Zhou, Ye}} \at{ \yr{2017{\natexlab{{\em a\/}}}} } \jt{Physics Reports}
   \bvol{720-722},  \pg{1--136}.

\bibitem[Zhou(2017{\natexlab{{\em b\/}}})]{zhou20172}
{\sc \au{Zhou, Ye}} \at{ \yr{2017{\natexlab{{\em b\/}}}} } \jt{Physics Reports}
   \bvol{723},  \pg{1--160}.

\bibitem[Zhou {\em et~al.\/}(2021)Zhou, Williams, Ramaprabhu, Groom, Thornber,
  Hillier, Mostert, Rollin, Balachandar, Powell {\em
  et~al.\/}]{zhou2021rayleigh}
{\sc \au{Zhou, Ye}, \au{Williams, Robin~JR}, \au{Ramaprabhu, Praveen},
  \au{Groom, Michael}, \au{Thornber, Ben}, \au{Hillier, Andrew}, \au{Mostert,
  Wouter}, \au{Rollin, Bertrand}, \au{Balachandar, S}, \au{Powell, Phillip~D}
  \& \au{others}} \yr{2021}  \at{Rayleigh--taylor and richtmyer--meshkov
  instabilities: A journey through scales}.  \jt{Physica D: Nonlinear
  Phenomena}  \bvol{423},  \pg{132838}.

\end{thebibliography}

%\bibliographystyle{jfm}
%\bibliography{jfm}
%Use of the above commands will create a bibliography using the .bib file. Shown below is a bibliography built from individual items.

\end{document}
