%                                                                 aa.dem
% AA vers. 9.1, LaTeX class for Astronomy & Astrophysics
% demonstration file
%                                                       (c) EDP Sciences
%-----------------------------------------------------------------------
%
%\documentclass[referee]{aa} % for a referee version
%\documentclass[onecolumn]{aa} % for a paper on 1 column  
%\documentclass[longauth]{aa} % for the long lists of affiliations 
%\documentclass[letter]{aa} % for the letters 
%\documentclass[bibyear]{aa} % if the references are not structured 
%                              according to the author-year natbib style

\documentclass{aa} 

%
\usepackage{longtable}
\usepackage{lipsum} 
\usepackage{graphicx}
%%%%%%%%%%%%%%%%%%%%%%%%%%%%%%%%%%%%%%%%
\usepackage{txfonts}
\usepackage{multirow}
\usepackage{hyperref}
\usepackage{comment}
\usepackage{microtype}

%%%%%%%%%%%%%%%%%%%%%%%%%%%%%%%%%%%%%%%%
%\usepackage[options]{hyperref}
% To add links in your PDF file, use the package "hyperref"
% with options according to your LaTeX or PDFLaTeX drivers.
%

\makeatletter
\renewcommand*\aa@pageof{, page \thepage{} of \pageref*{LastPage}}

\begin{document} 


     
   \title{XRBcats: Galactic High Mass X-ray Binary Catalogue\thanks{A web-version is publicly accessible at \url{http://astro.uni-tuebingen.de/~xrbcat/} and at CDS via anonymous ftp to cdsarc.u-strasbg.fr (130.79.128.5) or via \url{http://cdsarc.u-strasbg.fr/viz-bin/cat/J/A+A/vol/pag}}}


   \author{  M. Neumann\inst{1}\thanks{E-mail: marvin.neumann@astro.uni-tuebingen.de}, 
        A. Avakyan\inst{1},
        V. Doroshenko\inst{1}
        \and 
        A. Santangelo\inst{1}.
          }
          
   \institute{Universit{\"a}t T{\"u}bingen, Institut f{\"u}r Astronomie und Astrophysik T{\"u}bingen, Sand 1, T{\"u}bingen, Germany}


   \date{Received XXX; accepted YYY}

% \abstract{}{}{}{}{} 
% 5 {} token are mandatory
 
  \abstract
  % context heading (optional)
  % {} leave it empty if necessary  
   {We present a new catalogue of the high-mass X-ray binaries (HMXBs) in the Galaxy improving upon the most recent such catalogue. We include new HMXBs discovered since aforementioned publication and revise the classification for several objects previously considered HMXBs or candidates. The catalogue includes both basic information such as source names, coordinates, types, and more detailed data such as distance and X-ray luminosity estimates, binary system parameters and other characteristic properties of 169 HMXBs, together with appropriate references to the literature. Finding charts in several bands from infra-red to hard X-rays are also included for each object.}
  % aims heading (mandatory)
   {The aim of this catalogue is to provide the reader a list of all currently known Galactic HMXBs with some basic information on both compact objects and non-degenerate counterpart properties (where available). We also include objects tentatively classified as HXMBs in the literature and give a brief motivation for the classifcation in each relevant case.}
  % methods heading (mandatory)
   {The catalogue is compiled based on a search of known HMXBs and candidates in all commonly available databases and literature published before 31 October 2022. Relevant properties in the optical and other bands were collected for all objects either from the literature or using the data provided by large-scale surveys. In the later case, the counterparts in each individual survey were found by cross-correlating positions of identified HMXBs with relevant databases.}
   {An up-to date catalogue of Galactic HMXBs is presented to facilitate research in this area. An attempt was made to collect a larger set of relevant HMXB properties in a more uniform way compared to previously published works.}
  % conclusions heading (optional), leave it empty if necessary 
   {}

   \keywords{catalogues -- binaries: close --
                stars: early-type  -- X-rays:
                binaries
               }
   \maketitle
%
%-------------------------------------------------------------------

\section{Introduction}
X-ray binaries (XRBs) are still at the focus of observational X-ray astronomy almost 60 years after their first discovery.
These systems consist of a compact object, which can be either a neutron star (NS), a black hole (BH), or a white dwarf (WD), and a non-degenerate companion feeding the accretion which powers the observed X-ray emission. Depending on the mass of the optical companion, XRBs can subdivided into two large groups: systems with low-mass optical companion with $M_{\mathrm{opt}}\lesssim1M_\odot$ are classified as Low-mass X-ray binaries (LMXBs, see review in \citealt{2011hea..book.....L}), whereas those with $M_{\mathrm{opt}}\gtrsim 5 M_\odot$ High-mass X-ray binaries (HMXBs, see review in \citealt{2011hea..book.....L}). In both cases, accretion onto either NS or BH is usually assumed, while systems in which the compact object is a WD are commonly considered separately and most often referred to as cataclysmic variables (CVs) rather than XRBs due to both physical and historical reasons. The gap between LMXBs and HMXBs is populated by a handful of sources (e.g. Her~X-1 with an optical companion mass of $M_{\mathrm{opt}}\sim2M_\odot$ \citep{1972ApJ...174L.143T}), those systems are often refereed as Intermediate-mass X-ray binaries (IMXBs, \citealt{2003ApJ...597.1036P}). All of these binaries host, however, a compact object and thus represent endpoints of stellar evolution. As such, understanding the properties of their population is key for understanding the evolution of the Galaxy as a whole. 
Due to their short characteristic lifetime of $\sim 2\times 10^7 \mathrm{years}$, HMXBs can be considered young systems in comparison to LMXBs ($\sim 10^{10} \mathrm{years}$) and thus are a good tracer for star forming activity \citep{2003MNRAS.339..793G}. In the case of the Milky Way they are concentrated around the Galactic plane and trace spiral arms. Similarly to other celestial sources, there are still, however, large uncertainties in the spatial and luminosity distributions, numbers, physics, and evolutionary scenarios for XRBs. Part of the problem here is of course that we only observe a fraction of their overall population.
The number of known objects is, however, ever increasing, so it is important to keep the census of known XRBs and their properties up to date, especially in the era of new large-scale surveys like eRosita and \textit{Gaia}, which provide a wealth of new observational information for the entire sky. The goal of this paper is to provide such an update for HMXBs. In parallel, similar efforts for LMXBs and IMXBs is undertaken independently (\citealt{Artur22}, in preparation).

As already mentioned, HMXBs constitute a relatively broad class of objects defined based on the mass of the companion donor star. Depending on the type of the latter and the accretion mechanism they can be sub-divided in several sub-classes. Systems accreting from stellar winds of the optical companion with mass loss rates of up to $\Dot{M}\sim 10^{-6..-5} \mathrm{M_\odot \  yr^{-1}}$ \citep{2000ARA&A..38..613K} are usually persistent sources of X-rays and referred to as Supergiant X-ray binaries (SGXBs, \citealt{1986MNRAS.220.1047C}). These also include highly variable Supergiant Fast X-ray Transients (SFXTs, \citealt{2006ESASP.604..165N}), which are non-accreting most of the time but show bright flares on timescales from minutes to days with peak fluxes comparable to other SGXBs. Depending on the mass ratio of the components, accretion can be quasi-symmetric, or focused by gravity via the L1 point like, for instance, in Vela~X-1 \citep{2021A&A...652A..95K}. Ultimately, Roche-lobe overflow (RLO) can also occur. As an example one could quote the Galactic micro-quasar SS~433 and the first X-ray pulsar ever discovered, Cen~X-3. With mass loss rates of $\Dot{M}\sim 10^{-3} \mathrm{M_\odot \  yr^{-1}}$ (for a fully developed RLO)
those systems have a rather short lifetime of the order of the thermal timescale of the optical companion and are thus rare. However \citet{1978A&A....62..317S} showed that an atmospheric RLO is possible and could sustain thousands of years with mass-loss rates below $\Dot{M}\sim 10^{-8} \mathrm{M_\odot \  yr^{-1}}$, and indeed more HMXBs of this type are known outside of our own Galaxy. The biggest fraction of the known HMXB population is, however, represented by the so called Be X-ray binaries (BeXBs, \citealt{2011Ap&SS.332....1R}). In those systems the optical companion is either a early-type B (earlier than B3) or  a  late-type O (later than O8) star with a decretion disc around its equator. This disk is formed by the material released from the equator of the optical companion due to its high rotation velocity \citep{2020MNRAS.493.2528B}. The decretion disc size is known to be variable and can be traced by the characteristic emission lines in the optical spectra of Be star. Passage of the compact object (in the vast majority of cases a NS) through the disc leads to enhanced accretion, especially if the disk itself is in an expanding state. From an observational point of view, this leads to the appearance of two different kinds of outbursts observed from and characteristic for BeXB systems. Type I outbursts are periodic and regular occurring events with typical peak luminosity below $10^{37} \mathrm{erg\  s^{-1}}$, which normally coincide with the periastron passage of the compact object. The Type II outbursts are more irregular and less frequent events, and are apparently associated with the expansion of circumbinary disk. The peak luminosity can reach up to $10^{39} \mathrm{erg\  s^{-1}}$ \citep{2020MNRAS.491.1857D}, thereby exceeding the Eddington luminosity of the neutron star. Outbursts of this type are usually referred to as giant outbursts and  do not show any preferred orbital phase. They also usually last longer. The important point, however, is that all BeXBs are transients with relatively low duty cycles and mostly observable only during outbursts. They continue, therefore, to be  discovered at a steady rate, and contribute to the increase of the number of known HMXBs.

The most recent catalogue of the Galactic HMXBs was published 16 years ago by \citet{2006A&A...455.1165L}, and since then many new objects were discovered, some HMXB candidates lost their status. What is, perhaps, even more important, a wealth of additional observational data were collected with new facilities such as the X-ray Multi-Mirror Mission (\textit{XMM-Newton}) \citep{2001A&A...365L...1J}, the International Gamma-Ray Astrophysics Laboratory (\textit{INTEGRAL}) \citep{2003A&A...411L...1W} or \textit{Gaia} \citep{2016A&A...595A...1G}, to name a few. Here we present an updated catalogue of the Galactic HMXBs including this multi-wavelength information as well as the new sources discovered since the publication by \citet{2006A&A...455.1165L}.
The current catalogue only includes Galactic HMXBs and does not include objects from the Large and Small Magellanic Clouds as for those objects optical companions are often either not identified or poorly characterised. On the other hand, for the Galactic HMXBs effort has been made to collect all relevant and up-to date multi-wavelength information including distances, optical magnitudes, variability information in soft and hard X-rays, information on detection in radio band, and more, so that the catalogue can be useful in identification of new HMXBs in the ongoing and planned X-ray surveys, population studies and other investigations.

\section{Definition of the sample and data sources}\label{sec:Definition}
As a first step, we compiled a sample of 169 known (123) and candidate (46) HMXBs by searching the \textit{SIMBAD}\footnote{\url{http://simbad.u-strasbg.fr/simbad/}} and \textit{VizieR}\footnote{\url{https://vizier.cds.unistra.fr/viz-bin/VizieR}}, archives hosted by the Centre de Donn\'{e}es astronomiques de Strasbourg (CDS) databases and the literature. The largest fraction of this sample (105 objects) originates from the already mentioned \citet{2006A&A...455.1165L} catalogue. We also systematically searched the literature  (including Astronomoner's Telegrams\footnote{\url{https://astronomerstelegram.org/}}) for the reports of new HMXB discoveries. A large fraction of new HMXBs reported in the literature was discovered between 2006 and 2020 by the \textit{INTEGRAL} mission and the Neil Gehrels Swift Observatory (\textit{Swift}) \citep{2004ApJ...611.1005G}  (29 objects) as summarised in \citet{2019NewAR..8601546K}, which can, therefore, also be considered as one of the main sources used in this work.  Finally, we also considered all objects classified as HXMB or candidate in \textit{SIMBAD} and \textit{VizieR} databases. Considering that the majority of such objects which are not present in an already mentioned database or literature are actually extra-galactic, i.e. are located in nearby galaxies, we restricted the search to the galactic latitude between $\pm 15^{\circ}$ (HMXBs are strongly clustered towards Galactic plane, \citet{2002MmSAI..73.1053G}) in the latter case. 
We emphasize that although the effort was made to find all possible objects, it is still possible that we could have missed some sources, so we urge the reader to report such omissions to the corresponding author. We note also that some of the objects considered in the catalogue by \citet{2006A&A...455.1165L} as HMXBs or candidates are not considered as such anymore. In Section \ref{sec:Exception} we discuss these cases separately. 

\subsection{HMXB sample and classification}
To reflect the huge variety of HMXB properties, we include information on the origin of the donor star and compact object, as well as variability and other relevant information which is incorporated in a set of flags defined to divide HMXBs into several sub-classes mentioned above. In general, those flags are based on the classification introduced by \citet{2006A&A...455.1165L}, but supplemented with additional flags based on the information provided by \citet{2019NewAR..8601546K}, \citet{2019A&A...622A..61S}, and \citet{2021MNRAS.507.3899V}. As a result, we converged on a set of flags listed below. The frequency of occurrence of individual flags is provided below in brackets and also presented in a graphical form in Fig.~\ref{fig:hist} to give a broad overview for the sample of known HMXBs. Note that the flags below are not meant as a basis for classification but rather to reflect properties of individual systems each of which might have one or several flags: 

 \begin{enumerate} \label{em: Flags}
    \item BH: black hole candidate (6).
    \item  EB: eclipsing or partially eclipsing binary system (9).
    \item  MQ: micro-quasar (4).
    \item RS: radio emitting HMXB(23). 
    \item XB: X-ray burst source (3).
    \item XP: X-ray pulsar (63).
    \item  XT: transient X-ray source (67).
    \item  US: ultra-soft X-ray spectrum (1).
    \item HT: hard transient (9).
    \item Gcas: Gamma Cassiopeiae like source (9).
    \item QPO:  Quasi-periodic oscillation (8).
    \item SG: Supergiant optical companion (42).
    \item BE: Be star companion (72).
    \item CL: Cyclotron Resonance Scattering Feature in X-ray spectrum (37).
    \item GP: High-mass Gamma-ray Binary (8).
 \end{enumerate}
 
 \subsection{X-ray properties}
 Considering that the prime feature of XRBs in general is their X-ray emission, we attempted to compile relevant properties of the HMXBs in the catalogue in the X-ray band. This includes fluxes in the soft X-ray band (\textit{XMM-Newton}, Chandra X-Ray Observatory (\textit{Chandra}) \citealt{2000SPIE.4012....2W} and Swift X-Ray Telescope (\textit{Swift}/XRT, \citealt{2005SSRv..120..165B}), and hard X-ray bands (\textit{INTEGRAL}  and Swift Burst Alert Telescope (\textit{Swift}/BAT, \citealt{2005SSRv..120..143B}), X-ray positions (in order to assess reliability of optical counterpart identification or identify plausible counterparts as described below). In the soft X-rays, we report flux in the 0.2-12\,keV energy band (corresponding to EPIC\_8 band in the \textit{XMM-Newton} catalogues\footnote{\url{https://heasarc.gsfc.nasa.gov/W3Browse/xmm-newton/xmmssc.html}}  \citealt{2020A&A...641A.136W}) and similar energy bands of \textit{Chandra}\footnote{\url{https://vizier.cds.unistra.fr/viz-bin/VizieR-3?-source=IX/57/csc2master}} \citep{2010ApJS..189...37E} and \textit{Swift}/XRT\footnote{\url{https://vizier.cds.unistra.fr/viz-bin/VizieR-3?-source=IX/58/2sxps}} \citep{2020ApJS..247...54E}, i.e. the $0.3-10 $\,keV  for \textit{Swift}/XRT, and either the broad ACIS band ($0.5-7.0$\,keV) or the wide HRC band ($0.1-10.0 \mathrm{keV}$) for \textit{Chandra}. In the hard X-rays we used an energy range of $14-145$\, for \textit{Swift}/BAT\footnote{\url{https://heasarc.gsfc.nasa.gov/W3Browse/swift/swbat105m.html}} \citep{2018ApJS..235....4O} and an energy range of $17-60$\,keV for \textit{INTEGRAL}\footnote{\url{https://vizier.cds.unistra.fr/viz-bin/VizieR-3?-source=J/A\%2bA/545/A27}} \citep{2012A&A...545A..27K}. Note that these \textit{INTEGRAL}/BAT bands are largely equivalent as most HMXBs exhibit high energy cutoff at $\sim15-20$\,keV in their spectra \citep{1989PASJ...41....1N,2022arXiv220414185M}, and thus flux above 60\,keV is negligible.

\subsection{Optical position and Astrometry}
 Most of the HMXBs in the sample already have identified optical counterparts (which is almost an essential pre-requisite for HXMB classification). However, optical positions reported in \citet{2006A&A...455.1165L} are often fairly dated and thus not really accurate by modern standards. We made an effort to update those to current astrometry provided by \textit{Gaia} mission. To do this, we first checked whether the \textit{SIMBAD} database already contained identified \textit{Gaia}~DR3\footnote{\url{https://vizier.cds.unistra.fr/viz-bin/VizieR-3?-source=I/355}} \citep{2022arXiv220800211G} or Two Micron All Sky Survey\footnote{\url{https://vizier.cds.unistra.fr/viz-bin/VizieR?-source=II/246}} (2MASS, \citealt{2003tmc..book.....C}) counterparts. If this was not the case, we matched literature positions which includes V~band magnitudes to \textit{Gaia}~DR3 using a large search radius of 10$^{\prime\prime}$. 
 The \textit{Gaia} counterpart that matched the literature position and magnitudes within the uncertainties was then selected.
 In most cases this corresponded to an object closest to the literature position with V~band magnitude within 2\,mag from G~magnitude reported by \textit{Gaia}. We note that most of the HMXBs are variable to some extent also in the optical band so the exact comparison would be unfeasible, so the threshold of 2~magnitudes was determined empirically by inspecting the observed difference between literature and \textit{Gaia} magnitudes for sources where the \textit{Gaia} counterpart was already unambiguously identified.
 
For all HMXBs with a \textit{Gaia} counterpart, we provide distance information from \citet{2022arXiv220800211G}, \citet{2021AJ....161..147B}\footnote{\url{https://vizier.cds.unistra.fr/viz-bin/VizieR-3?-source=I/352}} and \citet{2022A&A...658A..91A}\footnote{\url{https://vizier.cds.unistra.fr/viz-bin/VizieR-3?-source=I/354/starhorse2021}}. In addition, for objects without an identified \textit{Gaia} counterpart but available distance estimates in the literature, we  give those estimates and their corresponding references. If multiple distance estimates were available, the mean distance was calculated by using the arithmetic mean on all available distance estimations linked to the given source. The range of possible distances taking the lowest and highest estimations of all distance estimates for a given system (both from the literature and \textit{Gaia}) is also reported.

 \subsection{Additional information}
 We also attempted to include some extra information related to properties of the optical counterpart and the compact objects.

 
 For instance, besides the G-band magnitudes of \textit{Gaia} published in \citet{2022arXiv220800211G}, magnitudes in the J-, H- and K-Band provided by \citet{2003tmc..book.....C} as well as magnitudes in the W1- and W2-band of the Wide-field Infrared Survey Explorer mission \citep{2010AJ....140.1868W} in CatWISE2020\footnote{\url{https://vizier.cds.unistra.fr/viz-bin/VizieR?-source=II/365}} \citep{2021ApJS..253....8M} and available V-band magnitudes, were included. 
 In addition, any luminosity estimates of the optical companion from \citet{2022arXiv220800211G} were added. 
 The spectral type of the counterpart is also of interest, so we attempted to include this information using the following sources: \citet{2019NewAR..8601546K}, \citet{2019A&A...622A..61S}, \citet{2006A&A...455.1165L}, Mauro Orlandini's website\footnote{\url{http://www.iasfbo.inaf.it/~mauro/pulsar_list.html}},
 and \textit{SIMBAD}. 

 Due to the close connection between spectral type and effective stellar temperature, estimations on the temperatures provided by \citet{2022A&A...658A..91A} and \citet{2022arXiv220800211G} were included. The catalogue includes only the  mean stellar effective temperature.

 
 One of the key properties for NS systems is the magnetic field strength of the compact object, which can be measured using observed energies of the so-called Cyclotron Resonance Scattering Features (CRSF or Cyclotron lines). We included, therefore, literature values for the observed CRSF energies where such feature was actually claimed to be detected. This includes the fundamental line as well as harmonics. The line energies and corresponding references were sourced from the recent review by \citet{2019A&A...622A..61S} and X-ray pulsar properties database by Mauro Orlandini\footnotemark[14], which appears to cover all X-ray pulsars where detection of a CRSF was ever claimed in the literature. 

\section{Catalogue content and quality assurance}
\subsection{Description of the fields}
The catalogue contains a total of 169 HMXBs and candidates, sorted by increasing right ascension (second column). The table consist of 61 columns, listing various parameter and references for a given HMXB. The first column contains the source name, in this case it is also the most common name in the literature (determined by number of mentions in NASA ADS system). The following seven columns are dedicated to the coordinates of the system, the second and third columns displaying the Right ascension (RA) and Declination (DEC) in degree, followed by statistical uncertainties of the coordinates in arcsec. The Coord\_Ref column provides the reference of the catalogue or literature, which were used to extract the coordinates and uncertainties (e.g. \citet{2022arXiv220800211G} for \textit{Gaia} DR3), as a NASA ADS bibcode. The sixth column indicates if the optical counterpart is solidly identified in the literature (0), if its a tentative optical counterpart identified as such in the literature (1) or if its not identified yet (2). In the seventh and eighth columns the galactic longitude (GLON) and galactic latitude (GLAT) in degrees can be found. The following column displays the different X-ray flags related to HMXB classification, which were discussed in Sect. \ref{sec:Definition}. Columns 10 and 11 provide the orbital period in days and the spin period in seconds for pulsars respectively. For NS HMXB with CRSFs reported in the literature line energies are listed in column 12 in units of keV, regardless if its the fundamental line or any harmonics. In the Alt\_name column, the second most used identifier in the literature was selected as an alternative name for the HMXB. Column 14 displays the spectral type of the optical companion. If the identification of the \textit{Gaia} counterpart was possible, the \textit{Gaia}-DR3-ID can be found in column 15. Column 16 and 17 show the magnitude in the G-band as well as its uncertainty followed by the V-band magnitude in Column 18 and its uncertainty in Column 19. In Columns 20-29, the JHK-magnitudes as well as W1- and W2-magnitudes with their respective uncertainties are shown. The neutral hydrogen equivalent column density ($N_{\rm H}$) in the direction of the source available in \citet{2020ApJS..247...54E} assuming a power-law spectrum
are listed in column 30 in units of $10^{21} \mathrm{cm^{-2}}$. The following six columns contain X-ray flux information: the first two are for the minimal and maximal fluxes in the soft X-ray band, the third displays the ratio between maximal and minimal soft X-ray flux, and the last three for the hard X-ray band respectively. The minimal and maximal fluxes are reported in units of $10^{-12}\ \mathrm{erg\  cm^{-2}\ s^{-1}}$. Columns 37-39 contain mass estimates of the compact object. In particular, mean mass, as well as upper and lower boundaries reported in the literature (in $\mathrm{M}_\odot$).
Columns 40-42 contain distance estimates (in parsecs) to a given source, including the mean distance and the upper and lower limits given in the literature. Stellar effective temperature and the luminosity of the optical counterpart can be found in column 43 and 44 in which $T_{eff}$ is in Kelvin and the luminosity of the companion in $L_\odot$. All \textit{SIMBAD} identifiers associated with the HMXB are quoted in column 45. The next 7 columns are dedicated to the references of CRSF, pulsation period, orbital parameter and spectral type, distance-estimation (if literature-values were used), mass-estimation as well as miscellaneous References. Column 53 is dedicated to the comments e.g. if the source is considered a HMXB candidate or similar. The last 8 columns contain the Source name in the diffrent catalogues of 2MASS, CatWISE, \textit{ROSAT}, \textit{XMM-Newton}, \textit{Swift}/XRT, \textit{Swift}/BAT, and \textit{INTEGRAL}.

\subsection{Finding charts and problematic cases}\label{sec:Exception}
For all sources we provide finding charts which consist of up to 6 different images ranging from near infra-red to hard X-rays as part of the catalogue. For the majority of the finding charts, we used the Hierarchical progressive surveys (HiPS, \citealt{2015A&A...578A.114F}); only in the case of \textit{Swift}/XRT did we use the SkyView Query. Both HiPS and SkyView offers the possibility to query their data automatically with Python. In case of HiPS, we used astroquery.hips2fits\footnote{\url{https://astroquery.readthedocs.io/en/latest/hips2fits/hips2fits.html}} package, and to access SkyView we used the astroquery.skyview\footnote{\url{https://astroquery.readthedocs.io/en/latest/skyview/skyview.html}} package. In the following, we will mention the used surveys and with their corresponding links as footnotes. The finding charts consist of up to 6 different surveys which we mention below, with their corresponding links as footnotes. 
In the top row we included, the image of the Visible and Infrared Survey Telescope for Astronomy (VISTA) Variables in the the Via Lactea (VVV\footnote{\url{http://alasky.cds.unistra.fr/VISTA/VVV_DR4/VISTA-VVV-DR4-J/}}) DR4 catalogue \citep{2010NewA...15..433M} or 2MASS\footnote{\url{http://alasky.cds.unistra.fr/2MASS/J/}} (in order of preference, left), an image of \textit{unWISE}\footnote{\url{http://alasky.cds.unistra.fr/unWISE/W1/}} \citep{2019ApJS..240...30S} in the middle and the RGB-image of either \textit{Chandra}\footnote{\url{https://cdaftp.cfa.harvard.edu/cxc-hips/}}, \textit{XMM-Newton}\footnote{\url{http://skies.esac.esa.int/XMM-Newton/EPIC-RGB/}} or the Roentgensatellit (\textit{ROSAT}\footnote{\url{http://alasky.cds.unistra.fr/RASS/}}, \citealt{1982AdSpR...2d.241T, ROSAT, ROSAT99}) (in order of preference, right). In the bottom row, there is the soft X-ray image of \textit{Swift}/XRT (SwiftXRTInt in astroquery.skyview) at the left corner, for hard X-rays \textit{Swift}/BAT\footnote{\url{http://cade.irap.omp.eu/documents/Ancillary/4Aladin/BAT_14_20/}} images in the middle and \textit{INTEGRAL}\footnote{\url{http://cade.irap.omp.eu/documents/Ancillary/4Aladin/INTEGRAL_17_60/}} images in the right corner. A 1 arcmin field of view was used for creation of the VVV, 2MASS, \textit{unWISE}, \textit{XMM-Newton}, and \textit{Chandra} images. In case of \textit{Swift}/XRT and \textit{ROSAT}, we used a field of view of 5 arcmin and 15 arcmin, respectively. The field of view in case \textit{Swift}/BAT and \textit{INTEGRAL}  was chosen to be $10^\circ$. In each case, the size of the region was chosen considering field of view and angular resolution of a given instrument. Every panel also shows coordinates and uncertainties of all detected sources within respective regions (position uncertainties are represented by error circles). A red cross indicates the position which is mentioned in \textit{SIMBAD} for the Source, and a dodger-blue diamond for the coordinates which are described in the literature. The position of the source which is used in this catalogue is indicated with an orange star. In case of the  soft X-ray instruments, the position of the observations is indicated by the error circles to prevent overcrowding, a golden circle indicates the \textit{Chandra} position, a red circle \textit{XMM-Newton}, a green circle \textit{Swift}/XRT, and a navy-blue circle \textit{ROSAT}. \textit{Swift}/BAT and \textit{INTEGRAL} are indicated as a deep-pink pentagon and lime-green triangle, respectively. In the optical band, we indicate \textit{CatWISE} with an orange X, 2MASS with a cyan +, and \textit{Gaia}~DR3 data with purple square.
As an example, Figure \ref{fig:finding} shows the finding chart of GRO~J1008$-$57.
\begin{figure*}[t]   
    \begin{center}
    \includegraphics[width=\textwidth]{GRO_J1008-57.pdf}
    \caption{Finding Charts of GRO~J1008$-$57. The finding Charts are overlaid by markers and error circles to indicate observation of different instruments. A cyan cross indicates the position of the 2MASS-observation, an orange X the CatWISE-position, and a purple square the observation in \textit{Gaia} DR3. The soft X-ray observations are only using the error circles to prevent overcrowding, the yellow circle is for \textit{Chandra}, and the green one for \textit{Swift}/XRT. A deep-pink pentagram together with an deep-pink error circle is used for \textit{Swift}/BAT, and the lime-green triangle together with the corresponding circle indicates the INTEGRAL-observation. The red cross and the blue diamond are indicating the position reported by SIMBAD and in the Literature, respectively. An orange star indicates the position of the HMXB, which is used in this catalogue.   
    } 
    \label{fig:finding}
    \end{center}
    \end{figure*}
\newline
\newline
Based on the visual inspection of the finding charts and literature research, several problematic cases have been identified. In the following, the problem cases are listed in their respective category in ascending order of their right ascension coordinates. 
Several sources initially assumed to be HMXBs were removed from the sample:

\begin{itemize}
    \item  1E~1048.1-5937 is listed as a HMXB in \textit{SIMBAD}, however \citet{2009ApJ...702..614D} concluded that this is actually a magnetar. The source is also listed as magnetar in the McGill magnetar database\footnote{\url{https://www.physics.mcgill.ca/~pulsar/magnetar/main.html}} \citet{2014ApJS..212....6O}. 
    \item IGR~J18151-1052 was suggested to be a HMXB \citet{2009ATel.2193....1B}, however, this was shown not to be the case by \citet{2012AstL...38....1L}. The object might be associated with magnetar candidate PSR~J1845-0258 or be a CV, but in any case HMXB identification is unlikely, so we decided not to include this source.
    \item Another magnetar 1E~2259+58.6 \citep{2002ApJ...567.1067G} was excluded using the same logic as 1E~1048.1-5937.
 \end{itemize}
Several objects do actually have tentative or known optical counterparts in the literature but could not be found automatically through the procedure outlined above. For these objects, the literature was searched manually and known counterparts and their coordinates were added to the catalogue:

\begin{itemize}
    \item Not much is known about RX~J0148.9+6121 which was only mentioned once by \citet{2014RAA....14..673W} to be a HMXB. We, therefore, consider it as a candidate rather than confirmed HXMB.
    \item \citet{2022arXiv220603904F} compiled a list of galactic HMXBs which contains a NS as the compact object and made an effort to identify optical counterparts for those in Gaia DR3. They only list objects with unambigously identified counterparts both in \textit{Gaia} DR3 and 2MASS catalogues within $0.5"$ from each other. In case of PSR~J0635+0533, they identified \textit{Gaia}~DR3~3131755947406031104 as the counterpart.
    \item For 1H~0749-600, \citet{2001A&A...377..148T} mentioned that the proposed optical counterpart HD~65663 is not the real counterpart but rather a single Be-star spatially coincident with an X-ray transient by chance. \citet{2001A&A...377..148T} considered the possibility that the systems is not an X-ray binary at all. We therefore mark this association as tentative.
    \item For 1FGL~J1018.6-5856 the same procedure was done as for PSR~J0635+0533. Here \citet{2022arXiv220603904F} identified \textit{Gaia}~DR3~5255509901121774976 as the \textit{Gaia} counterpart for 1FGL~J1018.6-5856.
    \item Based on the coordinates of the soft X-ray counterpart of 1ES~1210-64.6 \citep{2007ATel.1253....1R}, \citet{2009A&A...495..121Mx} were capable to identify the optical counterpart (\textit{Gaia}~DR3~6053076566300433920) This conclusion was confirmed by follow-up optical spectroscopy.
    \item A soft X-ray counterpart for IGR~J12341-6143 was discovered quite recently by \citet{2020ATel14039....1S} with \textit{Swift}/XRT which also led to the identification of an tentative optical counterpart (\textit{Gaia}~DR2~6054778507172454912).
    \item We suggest to consider 1A~1238-59 as just an HMXB candidate due to an error radius of $30"$ \citep{1978Natur.273..364D} and no detected optical counterpart.
    \item We decided to mark IGR~J14059-6116 as a HMXB candidate, due to a missing optical counterpart. \citet{2019ApJ...884...93C} suggested that IGR~J14059-6116 has a possible association with the HMXB 4FGL~J1405.1-6119, therefore we also noted that IGR~J14059-6116 is probably the same source as 4FGL~J1405.1-6119.
    \item In 2010 \citet{2010ATel.2962....1K} could locate an uncatalogued X-ray source inside the MAXI error circle of MAXI~J1409-619 using \textit{Swift}/XRT. Inside the \textit{Swift}/XRT error circle of this source, a catalogued IR-Source (2MASS~J14080271-6159020) was found which is considered as a tentative optical counterpart.
    \item \textit{Gaia}~DR3~5878377736381364608 is mentioned as the unambiguous optical counterpart of IGR~J14331-6112 by \citet{2022arXiv220603904F}.
    \item Since its discovery Cir~X-1 was often referred to as a LMXB, until \citet{2013ApJ...779..171H} could determine the age of the system to be about $4500$ yr. In addition, according to \citet{2007MNRAS.374..999J} system contains A0 to B5 type supergiant companion. However, Cir~X-1 also has shown type I X-ray bursts~\citep{1986MNRAS.221P..27T}, which indicates that source is a LMXB. Possible LMXB origin nature is also supported by the fact that companion star itself in Cir~X-1 still cannot be unambiguously detected at optical wavelengths.
    \citet{2016MNRAS.456..347J} concluded that the donor can be: 1) not have evolved low mass star or 2) a giant star, that it is still recovering from the impact of the supernova blast, that happened less than 5000 yr ago. Taking into account all this obscurity around Cir~X-1's nature we decided to add it to both LMXB and HMXB catalogues. 
    \item In March 2012, \textit{Swift}/BAT detected an increased flux from XTE~J1543-568, which led to a more accurate determination of the source position. This allowed \citet{2012ATel.4008....1K} to identify a tentative counterpart (2MASS~J15440515-5645425) within the error box provided by \textit{Swift}/XRT.
    \item With the help of \textit{Swift}/XRT and \textit{Chandra} Observations, \citet{2016MNRAS.462.3823L} were capable to identify a likely infrared counterpart for 2S~1553-542 at  RA(J2000)=15h 57m 48.28s DEC(J2000)=-54$^\circ$24' 53.5".
    \item Inside the error circle around the \textit{Swift}/XRT position of IGR~J16374-5043, \citet{2020MNRAS.491.4543S} found only one infrared source. This source is not listed in the 2MASS catalogue but is in the catalogue of \textit{Gaia} (\textit{Gaia}~DR3~5940285090075838848) and it is likely the optical counterpart of IGR~J16374-5043.
    \item  \citet{2010MNRAS.408.1866R} used \textit{Chandra} for localisation of galactic X-ray sources and did follow-up observation in optical and near-infrared. One of the observed sources was XTE~J1716-389. The refined \textit{Chandra} position allowed \citet{2010MNRAS.408.1866R} to identify the only possible optical counterpart, an infrared source 2MASS~J17155645-3851537 which is thus considered the true XTE~J1716-389 counterpart.
    \item IGR~J17375-3022 was a poorly studied source until \citet{2020MNRAS.491.4543S} investigated it in the hard and soft X-ray as well as in the infrared. With an enhanced source position with \textit{Swift}/XRT, they could find only one detected NIR-object in VVV-survey within the \textit{Swift}/XRT error circle. Therefore, \citet{2020MNRAS.491.4543S} proposed that VVV~J173733.74-302314.5 is the best counterpart candidate for IGR~J17375-3022.
    \item RX~J1739.4-2942 originally identified as LMXB, could also be a Be/HMXB \citet{2016ATel.8704....1B}. Therefore we included RX~J1739.4-2942 in the catalogue as a HMXB-Candidate  [Number 67 in \citet{2006A&A...455.1165L}].
    \item 
    \citet{2006A&A...453..133W} proposed two possible NIR-Counterparts for  IGR~J18029-2016, 2MASS~J18024194-2017172 was found in the 2MASS catalogue whereas 2MASS~J180242.0-201720.2 was found in the Second Guide Star Catalogue. The first possible Counterpart 2MASS~J18024194-2017172, has been confirmed to be the counterpart of the HMXB later by \citet{2008A&A...482..113M}.
    
    \item With the refined position of IGR~J18179-1621 reported by \citet{2012MNRAS.426L..16L}, \citet{2012ApJ...757..143N} were able to observe the object with \textit{Chandra} and identify a possible optical counterpart. 2MASS~J18175218-1621316 was identified as a possible counterpart, due to being the only NIR-sources in the 2MASS-catalogue in an $1"$ radius around the \textit{Chandra} position.
    \item Quite recently \citet{2022ApJ...927..139O} identified a bright IR-counterpart for IGR~J18219-1347 close to the \textit{Chandra} localization at RA(J2000)=18h 21m 54.821s DEC(J2000)=-13$^\circ$47' 26.703", which appeared to be combination of two point sources (Star A and Star B). They concluded that Star A is the Counterpart of IGR~J18219-1347, due to the fact that is consistent with the SED of a Be star, therefore they classified the system as a BeXRB.
    \item \citet{2022arXiv220702114F} queried confirmed HMXBs of their previous work \citep{2022arXiv220603904F} in \textit{Gaia} DR3. For AX~J1841.0-0536 the source \textit{Gaia}~DR3~4256500538116700160 was therefore identified as an optical counterpart.
    \item For \textit{Ginga}~1839-04 no optical counterpart is identified yet. \citet{2006A&A...455.1165L} included this source in the catalogue due to a possible pulsation of $\sim 81 \mathrm{s}$ \citep{1990Natur.343..148K} detected by \textit{Ginga} during an outburst in 1989. Since then no more X-ray detections have been reported \citep{2009ApJ...697.1194S}. Therefore \textit{Ginga}~1839-04 is marked as a candidate HMXB in this catalogue.
    \item \citet{2012ApJ...753....3B} observed five \textit{INTEGRAL} Sources towards the Scutum Arm, one of those sources is IGR~J18482+0049. With the refined position, they found only one object in the 2MASS-catalogue which was consistent with the \textit{XMM-Newton} position. Based on this observation, IGR~J18482+0049 has a possible association with 2MASS~J18481540+0047332 to date.
    \item In the case of \textit{Ginga}~1855-02, the positional error is of the size of $10'$ \citep{1990Natur.343..148K}. Like \textit{Ginga}~1839-04, \textit{Ginga}~1855-02 is a poorly studied source which does not have an identified optical counterpart, therefore it is also marked as a candidate HMXB based on its transient behaviour.
    \item After an outburst of XTE~J1859+083 detected by MAXI \citet{2015ATel.7034....1N}, follow-up \textit{Swift}/XRT observations by \citet{2015ATel.7067....1L} allowed to improve position of the HMXB. No UVOT counterpart was detected, but with the enhanced position \citet{2015ATel.7067....1L}, were able to find a possible counterpart in the USNO-B1.0 (USNO-B1.0~0982-0467424) and 2MASS catalogues (2MASS~J18590163+0814444).

    \item \citet{2011ATel.3326....1B} identified 2MASS~J19145680+1036387 as the most likely counterpart of 1E~1912.5+1031, based on the position inside the error circle of their estimated \textit{Swift}/XRT position for 1E~1912.5+1031.
    
    \item For AX~J1949.8+2534 \citet{2017MNRAS.469.3901S} proposed 2MASS~J19495543+2533599 to be the optical counterpart, which was later confirmed by \citet{2019ApJ...878...15H}. 
    \item W63~X-1 is an X-ray binary within the Supernova Remnant W63 with a pulsation period of 36 sec \citep{2004HEAD....8.1730R}. Based on the pulsation period and the spectrum, \citet{2004HEAD....8.1730R} concluded that the companion should be either a isolated neutron star, HMXB, or LMXB. They found an optical counterpart with $H\alpha$-excess emission typical for Be companion and thus classified the source as a HMXB.
 \end{itemize}
 
Several objects present in \citet{2006A&A...455.1165L} were removed from the current catalogue as they are not classified as HMXBs anymore:

\begin{itemize}
\item 1WGA~J0648.0-4419, previously listed as source \#18 in \citet{2006A&A...455.1165L}, was removed due to the fact that the optical star in the system is a hot sub-dwarf \citep{1963PASP...75..365J}.
\item IGR~J12349-6434 [Number 35 in \citet{2006A&A...455.1165L}] was initially classified as a new source, nowadays  it is associated, however, with RT~Cru \citep{2005ATel..591....1T}. RT~Cru is a Symbiotic star containing a WD \citep{2007ApJ...671..741L} and therefore it was excluded from the catalogue.
\item 1A~1246-588, which was number 38 in \citet{2006A&A...455.1165L}, is a LMXB \citet{2006A&A...446L..17B}.
\item With the de-reddened magnitudes of their observations in the JHK-bands and the assumption of a black-body model, \citet{2009MNRAS.394.1597K} estimated the distance of SAX~J1452.8-5949 [Number 46 in \citet{2006A&A...455.1165L}]. This ruled out the possibility of a HMXB, due to the fact that the system would be an extra-galactic source. They concluded that the binary system must have a low-mass companion and therefore is either a LMXB or a Intermediate Polar (accreting magnetized white dwarf, IP).
\item \citet{2008A&A...484..783C} classified IGR~J16358-4726 [Number 55 in \citet{2006A&A...455.1165L}] as an HMXB, however, this classification was revoked by \citet{2010A&A...516A..94N} and now the source is considered to be a Symbiotic X-ray binary.
\item Based on the NIR observation, \citet{2010MNRAS.402.2388K} estimated that the companions of AX~J1700.1-4157\footnotemark[27] [Number 63 in \citet{2006A&A...455.1165L}]  should be a low-mass star. In combination with a detected Fe emission line, \citet{2010MNRAS.402.2388K} concluded that AX~J1700.1-4157 is most likely a IP, therefore AX~J1700.1-4157 was removed from this catalogue.
\item IGR~J17091-3624 [Number 64 in \citet{2006A&A...455.1165L}], is also classified as LMXB \citep{2016ATel.8761....1G} today.
\item AX~J1740.1-2847 [Number 68 in \citet{2006A&A...455.1165L}] is a similar case as AX~J1700.1-4157, \citet{2010MNRAS.402.2388K} also estimated that the companion should be a low-mass star and due to a detected Fe emission line they concluded that AX~J1740.1-2847 is most likely a IP as well, hence it was removed from the catalogue. 
\item 4U~1807-10 [Number 75 in \citet{2006A&A...455.1165L}] was detected by \textit{UHURU} \citep{1978ApJS...38..357F} with a large error of 1.3$^\circ$.
    Until now, it is not certain if 4U~1807-10 is a HMXB or a LMXB. However, it is more likely that 4U~1807-10 is a LMXB, as the system shows type I X-ray bursts \citep{2017AstL...43..781C}. Therefore, we have excluded it from this catalogue.
        
\item \citet{2014ApJ...784....2M} determined a mass of 2.9 $M_\odot$ for the optical companion of SAX~J1819.3-2525\footnotemark[27] [Number 77 in \citet{2006A&A...455.1165L}], which classifies this system as IMXB, therefore it was excluded from the catalogue.
\item AX~J1838.0-0655\footnotemark[27] [Number 82 in \citet{2006A&A...455.1165L}] is nowadays classified as a supernova remnant. \citet{2005ApJ...630L.157M} does not exclude the possibility of AX~J1838.0-0655 to be a X-ray binary, but mentioned that this scenario is unlikely due to the lack of strong X-ray and $\gamma$-ray variability as well as the extension of the source to TeV energies. Therefore AX~J1838.0-0655 was removed from the catalogue.
\item \citet{2008AstL...34..753K} proposed that XTE~J1901+014 [Number 94 in \citet{2006A&A...455.1165L}] could be the first low-mass Fast X-ray transient. To date, the nature of the system is not completely clear \citep{2019ATel13328....1S}, therefore we decided to exclude this source from the current version of this catalogue.
\item XTE~J1906+09 [Number 96 in \citet{2006A&A...455.1165L}] was not excluded, but the name changed to XTE~J1906+090 to match that commonly used in the literature.
\footnotetext[27]{can be found in the recently published catalogue of high-mass X-ray binaries in the Galaxy by \citet{2023arXiv230202656F}}
\end{itemize}
A handful of sources which are included in the recently published catalogue of high-mass X-ray binaries by \citet{2023arXiv230202656F} are missing in our catalogue.Those are the already mentioned Sources AX~J1700.1-4157, SAX~J1819.3-2525 and AX~J1838.0-0655, as well as GRS~1758-258. In the case of GRS~1758-258, there is a possibility that the system is a intermediate-mass X-ray binary as stated by \citet{2016A&A...596A..46M}, hence this source can be found in \citet{Artur22} instead of this catalogue.

\subsection{Summary of other differences with respect to the \texorpdfstring{\citet{2006A&A...455.1165L}}{Liu et al. (2006)} catalogue}
Major changes with respect to \citet{2006A&A...455.1165L}, in addition to the problematic cases, are listed below:
\begin{itemize}
    \item  We homogenised the energy ranges for reported X-ray fluxes. \citet{2006A&A...455.1165L} used an energy range between $2-10$\,keV for most sources, but in some cases the actual energy range was different and it was not trivial to identify such cases. We now quote soft and hard X-ray fluxes in well defined energy ranges separately.
    \item \citet{2006A&A...455.1165L} for most cases only reported the maximal value of X-ray flux in units of Jy, now we report flux ranges in two energy bands in more commonly used cgs units which makes more sense for X-ray sources.
    \item We increased the number of flags used to characterize source properties from 6 in \citet{2006A&A...455.1165L} to a total of 15 in our case to better reflect various features reported in the literature.
    \item While \citet{2006A&A...455.1165L} does flag sources where a cyclotron line was detected, no information on its energy is available in most cases, whereas we report both observed energies and corresponding references. The number of objects where a line was detected also increased substantially due to availability of new observations and publications.
    \item Effective stellar temperature and estimated optical luminosity, as well as the hydrogen column density are three completely new fields introduced in this version of the catalogue.
    \item Finally, we include up to 6 finding charts as part of the catalogue (from near-infrared images to hard X-ray bands), instead of a reference for a finding chart.
\end{itemize}
   
\begin{figure}[t]   
    \begin{center}
    \includegraphics[width=\columnwidth]{Types_hist.pdf}
    \caption{HMXBs population of each type in the Galaxy.} 
    \label{fig:hist}
    \end{center}
    \end{figure}









\begin{acknowledgements}

This research has made use of the SIMBAD data base and VizieR catalogue access tool operated at CDS, Strasbourg, France, and NASA’s Astrophysics Data System (ADS).
This research made use of hips2fits,\footnote{https://alasky.u-strasbg.fr/hips-image-services/hips2fits} a service provided by CDS.
AA thanks Deutsche Forschungsgemeinschaft (DFG) within the eROSTEP research unit under DFG project number 414059771 for support (DO 2307/2-1).
This work has made use of data from the European Space Agency (ESA) mission
{\it Gaia} (\url{https://www.cosmos.esa.int/gaia}), processed by the {\it Gaia}
Data Processing and Analysis Consortium (DPAC,
\url{https://www.cosmos.esa.int/web/gaia/dpac/consortium}). 
Funding for the DPAC
has been provided by national institutions, in particular the institutions
participating in the {\it Gaia} Multilateral Agreement. We acknowledge the public data from  \textit{XMM-Newton}, \textit{Chandra}, \textit{Swift} and \textit{INTEGRAL}.


\end{acknowledgements}
\bibliographystyle{aa}
\bibliography{bibtex.bib}

\newpage
\appendix

\onecolumn
\section{Catalogue format}
%\small

\begin{longtable}{p{0.2cm}p{2cm}p{2.2cm}p{12.8cm}}
\caption{\label{tab:cols} Definition of columns in the HMXB catalogue. In total the catalogue contains \textbf{61} columns.}\\


\hline 
№ & Column name & Unit & Description \\
\hline
\endfirsthead
\multicolumn{4}{c}%
{{\bfseries \tablename\ \thetable{} -- continued}} \\
\hline\hline
№ & Column name & Unit & Description \\ 
\hline
\endhead

\hline
\endlastfoot
    1 & 'Name' &  &  Object name, which is the most common name in the literature
    \\

    2 & 'RAdeg' &  deg  & Right Ascension in degrees (ICRS). 
     \\
  
    3 & 'DEdeg' &  deg &  Declination in degrees (ICRS).
    \\
  
    4 &  'PosErr' &  arcsec & Positional error in arcseconds. 
    \\
  
    5 &  'Coord\_Ref' &  & Reference of catalogue or literature, which was used to extract the coordinates and uncertainties \\ 
    
    6 & 'ID\_Flag' & & Robust identification of the optical counterpart: \\
    %& & &\\
    & & & 0 --- solidly identified optical counterpart in the literature \\
    & & & 1 --- tentative optical counterpart identification in the literature\\
    & & & 2 --- no identified optical counterpart\\

    7 &  'GLON' & deg & Galactic longitude in degrees. 
    \\

    8 & 'GLAT' & deg &  Galactic latitude in degrees. 
    \\

    9 & 'Xray\_Type' & &  List of the X-ray types assigned to the object. For more information see section above~\S~\ref{em: Flags}.
    \\

    10 &  'Porb' & day &  Orbital period of the binary system in days if determined.
         \\
    11 &  'Ppulse' & s &  Pulsation period (spin) of the binary (with NS) in seconds if determined.
    \\
    12 & 'CRSF' & keV & Cyclotron line energies in keV
    \\

    13 &  'Alt\_Name' & & second most used name in the literature.
    \\
    14 &  'SpType' & &  Spectral type of the optical counterpart.
    \\
  
    15 &  'Gaia\_DR3\_ID' & &  Source Name in the Gaia DR3 catalogue.
    \\
    16 &  'Gmag' & mag &  Optical magnitude in G band according to a Gaia's catalogue. 
    \\     
    17 &  'e\_Gmag' & mag &  Corresponding  magnitude's error in G band according to Gaia's catalogue. 
    \\     
    18 & 'Vmag' & mag &  Optical magnitude in V band according to SIMBAD database. 
    \\ 
    19 & 'e\_Vmag' & mag &  Corresponding  magnitude's error in V band according to SIMBAD database. 
    \\   
    20 & 'Jmag' & mag &  IR magnitude in J band according to 2MASS \citep{2003tmc..book.....C}.
    \\     
    21 & 'e\_Jmag' & mag &  Corresponding  magnitude's error in J band according to 2MASS \citep{2003tmc..book.....C}. 
    \\
    22 &  'Hmag' & mag&  IR magnitude in H band according to 2MASS \citep{2003tmc..book.....C}.
    \\
    23 & 'e\_Hmag' & mag &  Corresponding  magnitude's error in H band according to 2MASS \citep{2003tmc..book.....C}.  
    \\
    24 & 'Kmag' & mag &  IR magnitude in K band according to 2MASS \citep{2003tmc..book.....C}.
    \\
    25 & 'e\_Kmag' & mag &  Corresponding  magnitude's error in K band according to 2MASS \citep{2003tmc..book.....C}.
    \\
    26 & 'W1mag' & mag &  IR magnitude in W1 \textit{CatWISE}2020 band \citep{2021ApJS..253....8M}.  
    \\ 
    27 & 'e\_W1mag' & mag &  Corresponding magnitude's error in W1 \textit{CatWISE}2020 band \citep{2021ApJS..253....8M}.
    \\ 
    28 & 'W2mag' & mag &  IR magnitude in W2 \textit{CatWISE}2020 band \citep{2021ApJS..253....8M}. \\ 
    
    29 & 'e\_W2mag' & mag &  Corresponding magnitude's error in W2 \textit{CatWISE}2020 band \citep{2021ApJS..253....8M}.
    \\   
    30 & 'N\_H' & $10^{21}\mathrm{cm^{-2}}$ &  Neutral hydrogen column density, assuming a power-law spectrum \citep{2020ApJS..247...54E}. 
    \\   
    31 & 'min\_soft\_flux' & $10^{-12}\mathrm{erg \ cm^{-2} s^{-1}}$  &  Minimum flux from a source in the soft X-ray range. By 'soft' here  \\ & & &  and after is meant the collective  range of  Swift/XRT, XMM-Newton and Chandra observatories 
    \\
    & & & {\citep[][respectively]{2020ApJS..247...54E,2020A&A...641A.136W,2010ApJS..189...37E}} 
    
    \\   
    32 & 'max\_soft\_flux'& $10^{-12}\mathrm{erg \ cm^{-2} s^{-1}}$ &  Maximum flux from a source in the soft X-ray range.
    \\
    33 & 'soft\_Xray\_var& & Ratio of maximal soft X-ray flux and minimal soft X-ray flux \\
    34& 'min\_hard\_flux' & $10^{-12}\mathrm{erg \ cm^{-2} s^{-1}}$ &  Minimum flux from a source in the hard X-ray range. By 'hard' here  \\ & & &  and after is meant the collective range of Swift/BAT and INTEGRAL observatories 
    \\ & & & ~{\citep[][respectively]{2018ApJS..235....4O, 2012A&A...545A..27K}}. 
    \\   
    35 & 'max\_hard\_flux' & $10^{-12}\mathrm{erg \ cm^{-2} s^{-1}}$ &  Maximum flux from a source in the hard X-ray range.\\
    36 & 'hard\_Xray\_var' & & Ratio of maximal hard X-ray flux and minimal hard X-ray flux\\
    37 & 'Mean\_Mass' & $\mathrm{M}_\odot $ & Mean mass of the compact object. \\
    38 & 'Low\_Mass' & $\mathrm{M}_\odot$ & Lower limit of the compact object mass.\\
    39 & 'High\_Mass' & $\mathrm{M}_\odot$ & Upper limit of the compact object mass.  \\
     
    40 & 'Mean\_Dist' & pc &  Mean (between the two, minimum and maximum) estimate of the distance to the binary system \\ & & & according to {\citet{2022arXiv220800211G,2021AJ....161..147B,2022A&A...658A..91A}} or taken from literature. In every literature case only one source of information was taken.
    \\
    41 & 'Low\_Dist' & pc &  The lowest (minimum) estimate of the distance to the binary system according to {\citet{2022arXiv220800211G,2021AJ....161..147B,2022A&A...658A..91A}} or taken from literature. In every literature  case only one source of information was taken.
    \\     
    42 & 'High\_Dist' & pc & The highest (maximum) estimate of the distance to the binary system according to {\citet{2022arXiv220800211G,2021AJ....161..147B,2022A&A...658A..91A}} or taken from literature. In every literature case only one source of information was taken. \\
    
    43 & 'Teff' & K & Effective stellar temperature provided by {\citet{2022arXiv220800211G,2022A&A...658A..91A}}.  \\
    44 & 'Lopt' & $\mathrm{L}_\odot$ & Luminosity estimate of the optical companion provided by \citet{2022arXiv220800211G}.\\
    45 & 'IDS' & &  List of all identifiers, that we found in SIMBAD and in the catalogues and articles. \\
    46& 'CRSF\_Ref' & & References to the articles, which providing the corresponding values for Cyclotron line energies. \\
    47 & 'Spin\_Ref'& & References to the articles, which providing the corresponding value for Pulsation Periods.  \\
    48 & 'Orb\_Ref'& & References to the articles, which providing the corresponding value for Orbital parameters. \\
    49 & 'Spectral\_Ref' & &References to the articles, which providing the spectral type of the optical star. \\
    50 & 'Dist\_Ref' & & Reference to the article, which provides the distance estimation, if literature values were used.  \\
    51 & 'Mass\_Ref' & & Reference to the article, which provides the mass estimation of the compact object. \\
    52 & 'misc\_Ref' & & miscellaneous references regarding the object.\\
    53 & 'comments' & & Comments regarding the object.  \\
    54 & '\_2MASS\_ID' & & Source name in the 2MASS All-Sky Catalog of Point Sources\\
    55 & 'CatWISE\_ID' & & Source name in the CatWISE2020 catalog\\
    56 & 'ROSAT\_ID' & & Source name in the Second ROSAT all-sky survey (2RXS) source catalog\\
    57 & 'XMM\_ID & & Source name in the XMM-Newton Serendipitous Source Catalog\\
    58 & 'Chandra\_ID' & & Source name in the Chandra Source Catalog (CSC) Release 2.0\\
    59 & 'XRT\_ID' & & Source name in the 2SXPS Swift X-ray telescope point source catalog\\
    60 & 'BAT\_ID' & & Source name in the Swift-BAT 105-Month All-Sky Hard X-Ray Survey catalog\\
    61 & 'INTEGRAL\_ID & & Source name in the INTEGRAL/IBIS 9-year Galactic Hard X-Ray Survey catalog\\
\end{longtable}
\normalsize
\twocolumn


% WARNING
%-------------------------------------------------------------------
% Please note that we have included the references to the file aa.dem in
% order to compile it, but we ask you to:
%
% - use BibTeX with the regular commands:
%   \bibliographystyle{aa} % style aa.bst
%   \bibliography{Yourfile} % your references Yourfile.bib
%
% - join the .bib files when you upload your source files
%-------------------------------------------------------------------


\end{document}
