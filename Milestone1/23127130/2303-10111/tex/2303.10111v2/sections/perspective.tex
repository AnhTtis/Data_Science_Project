\section{Perspective and summary}\label{SEC:CONCLUSION}

\subsection{Multiscale method for light-matter interaction}
Though there have been extensive developments in QED methods, most of the current implementations are based on the dipole approximation. However, the dipole approximation can fail in many cases. The nanoplasmonic cavities allow for light localization into deeply subwavelength dimensions, leading to effective mode volumes as small as a few nm$^3$ or even \AA$^3$ (picocavities)~\cite{Jeremy2022nl}. Consequently, the size of molecules becomes comparable to the cavity volumes, and the widely used dipole approximation breaks down. Moreover, the cavity confines the photon field in a certain direction when coupling molecules with many modes inside a cavity (either nanophotonic or nanoplasmonic ones). For a given probing angle $\theta$ (relative to the normal vector of the cavity), the photon energy has a certain dispersion function. Under this situation, the dipole approximation along the in-plane direction of the cavity no longer holds as well~\cite{Tichauer2021JCP, Li_MolPolRev_ARPC2021}.
Hence, a general method that can compute the multiple spatial-dependent cavity eigenmodes $\lambda_{n\bk}(\br)$ and light-matter couplings (from Maxwell's equations) is required. 

As argued above, the EM field within the cavity is, in principle, not homogeneous, and spatial dependence matters in light-matter interactions. The spatial distribution of the EM field can be solved via standard computational electromagnetic methods, such as Finite-difference time-domain (FDTD)~\cite{taflove2005computational}, for Maxwell's equations. Consequently, the light-matter coupling strength can stem from a delicate balance between the spatial dependence of the electronic wavefunctions and the photonic fields. Therefore, only a quantum model that fully incorporates the inhomogeneities of the exciton transition charge density can quantitatively describe this interplay. Using a fully first-principles methodology to describe the quantum chemistry of molecules placed inside the cavity, we can reveal the limitations of the point-dipole approach to address the exciton dynamics in weak and strong coupling regimes.
Besides, current methods usually don't consider the feedback of molecular systems on the cavities. In many situations, modeling of molecular systems only is insufficient as the molecular response can significantly affect the EM distributions, especially in nanoplasmonic cavities. Even though the \textit{ab initio} QED methods we previously developed take into account the interplay between the electronic and photonic DOFs, the molecular response on the EM environment is not considered. 

In principle, we could solve the Schr\"odinger and Maxwell equations simultaneously in order to obtain access to the radiated fields and, with it to the self-consistent evolution of light and matter, i.e., the cavity photon modes can be affected by the modular dipoles due to the Ampere's Law (Helmholtz's equation):
\begin{align}
    &\left[\nabla\times\frac{1}{\mu_r(\br\omega)}\nabla\times -\omega^2\mu_0\epsilon_0\epsilon(\br\omega)\right]
    \bE(\br)=-e\bJ(\br), 
    \text{ or }\\
    &  \left(\nabla^2-\frac{1}{c^2}\frac{\partial^2}{\partial t^2}\right)\bA=-\mu_0 \bJ.
\end{align}
Where $\bJ$ is the paramagnetic current density due to the molecular dipoles. According to Maxwell's equations, each current induces an electromagnetic field for which its precise spatial and polarization structure depends on the electromagnetic environment—oscillating charges emit light. 
Hence, a fully self-consistent QED method should consider the feedback of molecular dipoles on the cavity properties. The flowchart is demonstrated in Figure~\ref{fig:flowchart}.


\begin{figure}[htb]
    \vspace{0.0 em}
    \centering
    \includegraphics[width=1.0\linewidth]{figures/flowchart.pdf}
    \vspace{-1.5em}
    \caption{Self-consistent loop between two sets of equations. The current density from the QED solvers is used to update EM fields.
    The self-consistency between Maxwell and Schr\"odinger equation can be bypassed (in the linear-response regime) by using the pre-calculated Dyadic Green's functions.
    }
    \vspace{0 em}
    \label{fig:flowchart}
\end{figure}

Alternatively, the generated field can be expressed with the help of the dyadic Green's tensor,
\begin{equation}\label{eq:efildfromg}
    \bE(\br,\omega) = i\mu_0\omega\int d\br' G(\br,\br',\omega) [-e\bJ(\br',\omega)],
\end{equation}
where Green's function is the formal solution of Helmholtz's equation
\begin{equation}\label{eq:greenf}
    \left[\nabla\times\frac{1}{\mu_r(\br\omega)}\nabla\times -\omega^2\mu_0\epsilon_0\epsilon(\br\omega)\right]
    G(\br,\br',\omega)=\delta(\br,\br'),
\end{equation}
with linear media $\epsilon(\br)$. The Green's function in Eq.~\ref{eq:greenf} can be solved from the Frequency domain FDTD methods. 

Utilizing Dyadic Green's functions offers a powerful approach to simplify the description of certain components of a matter system while focusing on the computation of electronic structure only. Such a concept has been proposed in the past in a semiclassical treatment of light-matter interaction~\cite{Christian2022PRL,schafer_makingQEDFTFunc_PNAS2021}. Introducing Green's function is particularly advantageous in multicomponent systems that span different length scales, such as a microscopic molecule (which is described by the current density $\bJ$) and macroscopic solvent (represented by a parameterized dielectric function $\epsilon(\br)$). The EM environment embedded via Dyadic Green's function can be obtained numerically using many standard Maxwell solvers with certain boundary conditions, such as FDTD~\cite{taflove2005computational} and method of moments~\cite{peterson1998computational}, providing the photon mode structures that are coupled with QED electronic structure solvers. By computing $\mathbf{G}$ beforehand, the self-consistency is embedded solely via the current density, eliminating the need to treat QED electronic structures and Maxwell's equation simultaneously. The bypassing of the self-consistency between Maxwell-Schr\"odinger equations is a significant strength of Dyadic Green's function approach. This makes the approach more accessible and easier to implement in various electronic structure solvers. But, \textit{it should be noted that introducing linear media restricts the evaluation in the linear response regime.} 

Moreover, the multiscale methods described above can have a significant influence other than polariton chemistry. It could affect the study of light-matter interactions in chemistry, physics, materials science, and energy science at large, such as plasmon chemistry~\cite{Zhan:2023tc}, quantum plasmonics~\cite{Tame2013np}, quantum information transduction~\cite{Gaita-Arino:2019ti}, and photonic-microelectronics integration and polaritonic devices~\cite{Sanvitto:2016td}. The integration with open quantum system method, particularly time-dependent quantum transport theories~\cite{Zheng2007prb, zhang2013first,zhang2013dissipative}, will also open the door to simulating light-driven quantum transport phenomena~\cite{chen2018stark, Boolakee:2022tm} in either weak or strong coupling regimes.


%==============================================
\subsection{Summary}
Controlling chemistry or molecular properties has been a long-standing Holy Grail for many decades. Only recently, new possibilities have emerged in the context of light-matter interactions via either plasmon- or polariton-mediated chemistry, which holds the promise of providing fundamentally new strategies to control chemical reactions that are completely distinct from traditional ones (such as electrochemistry, thermochemistry, and photochemistry). Consequently, the application of light-matter interactions in manipulating chemistry has attracted increasing experimental and theoretical attention. However, the inherent multiscale nature of light-matter interaction problems poses significant challenges for both experimental and theoretical investigations of the underlying processes. Besides, multiscale processes at different length and time scales make it difficult to optimize the performance of light-matter interaction-mediated chemistry, and precise modeling of these processes is essential for obtaining a comprehensive understanding of the fundamental mechanisms or current experiments. Hence, the development of multiscale theoretical and computational models that can precisely describe the dynamical processes in light-matter interaction-mediated chemistry across different time and length scales, in conjunction with massively parallel algorithms for large-scale simulations, is essential for obtaining comprehensive insights into the fundamental mechanisms of current experiments. These simulations allow for the modeling of complex, multiscale processes that would be otherwise difficult or impossible to observe experimentally alone. Such development will ultimately facilitate the discovery of new reaction pathways enhanced by light-matter interactions. 
