\section{Polariton chemistry}\label{SEC:POLARITON}

Since there are several recent experimental and theoretical reviews on polariton chemistry~\cite{mandal_ChemRev2022, foley:2023arxiv, Herrera:2020jcp, Simpkins:2023tm, Dunkelberger:2022uo, Ebbesen2016ACR},  only a brief introduction to the recent theoretical development of polariton chemistry will be reviewed in this perspective. Instead, we will focus on our perspective on the theoretical challenges and ongoing and future development of quantum many-body and multiscale methods toward solving the mysteries of polariton chemistry and beyond.


\subsection{Introduction and background}

As introduced in Sec.~\ref{sec:basics}, in the strong light-matter coupling limit (strong enough to compete with individual DOF's dissipation rate), the formed quasiparticles (polaritons) can alter the potential energy landscape and consequently manipulate the chemical processes. 
%When the coupling of the light and molecular DOFs is strong compared to the dissipation of the excited state population (\textit{e.g.}, cavity leakage or spontaneous/stimulated electronic emission), the system is said to be strongly coupled. 
Generally, the strong coupling regime is characterized by the coherent energy exchange between the photon field and the electronic emitter (\textit{i.e.} Rabi oscillations). In this regime, the electronic and photon subsystems can no longer be treated separately, making the accurate simulation of polariton dynamics very challenging. In this section, we explore various theoretical approaches in modeling the light-matter interactions present in optical cavities, such as Fabry–P\'{e}rot, where the coherent exchange in energy between all DOFs directly affects the resulting chemistry. Here, the electronic and photonic DOFs must be treated quantum mechanically on equal footing.

Recent experiments have shown a propensity to change chemistry via the coupling of quantized radiation and various molecular DOFs, namely electronic and vibrational strong coupling. The Fabry–P\'{e}rot-like cavities offer an extremely tunable cavity frequency (via the effective length between the cavity mirrors) while exhibiting widely varying coupling strengths highly dependent on the experimental setup~\cite{Hutchison2012ACIE, Hutchison2012ACIE, Mauro2021PRB, George2016PRL, Vergauwe2019ACIE, Lather2019ACIE, Sau2021ACIE, Hirai2021CS, Takele2021JPCC, Thomas2019S, Lather2022CS}. There are many open questions regarding these experiments, such as the collective effects (\textit{i.e.}, many-molecule or many-mode effects), which are present in many of the recent works due to the complexity of performing single-molecule experiments.

\subsection{Theoretical and Computational Challenges} 

There are conceptual and technical challenges in performing simulations of polaritonic systems. Like the situation in plasmon chemistry, the challenges in understanding polariton chemistry still lie in the multiple coherent and dissipative processes across various time/length scales.
\begin{enumerate}
 \item The first theoretical hurdle is that of the Hamiltonian itself. For the Pauli-Fierz Hamiltonian (Eq.~\ref{EQ:H_PF}), the light-matter interaction requires explicit knowledge of the molecular dipole operator $\hat{D}$ in the working basis. It turns out that a significant simplification can be made if one neglects all the contributions from the dipole matrix except for the ground-to-excited transition dipole matrix elements. There are two primary reasons for making this approximation: (I) The entirety of this quantity is not usually printed by default when executing standard electronic structure software for electronically excited states; however, the usual information given from these calculations is the ground-to-excited transition dipole moment, $D_{0J}$, which, for example in linear response calculations (\textit{e.g.}, TD-HF or TD-DFT), is a trivial quantity to achieve since the result of such schemes is the ground-to-excited transition density.\cite{tretiak2002CR} (II) Historically, the field of quantum optics was narrowly focused on the light-matter interaction between single atoms and the quantized cavity field. In such cases, the diagonal elements of the dipole operator are zero by construction, and the electronic energy differences between excited states were large, leading to neglecting the excited-to-excited dipole coupling. With these approximations, one would find that a reduced Hamiltonian can be achieved, commonly referred to as the Jaynes-Cumming (JC) Hamiltonian.
 \item Generally, the strong coupling relies on a larger number of molecules collectively coupled to a cavity, introducing significant challenges to computing the hybridized polaritonic states of many ($10^6$ or even larger) molecules in the cavity.
 \item While plasmonic cavities are able to achieve strong coupling in a few (or even single) molecule limits, these cavities are highly heterogeneous and dispersive. Consequently, multiple models, dispersion of the cavities, coupling beyond the dipole approximation, and the inherent strong dissipation should be taken into account, introducing different computational complexity. Besides, the molecules inside the nanoplasmonic cavities can also interact with the ones outside the cavity. For example, a recent 2DES investigation of J-aggregates in the plasmonic cavity reveals rich photophysical dynamics between the polaritonic states inside the cavity and uncoupled exciton states outside the cavity~\cite{timmer_plasmon_2023}, which indicates that the molecules outside the cavity cannot be trivially ignored.
 \item The damping of the polariton state is usually ignored or introduced by an empirical lifetime of cavity photon (cavity leakage). In principle, cavity leakage depends on the photonic density of states, which can be computed from the first principles. Besides, the light-matter coupling could affect the photonic DOF and, in return, the photon lifetime. Hence, one open question is to what extent the strong coupling changes the photon lifetime, which requires a multiscale method to address the feedback of molecular systems on the cavities and first-principles calculations of cavity leakage.
 \item Polariton-mediated phenomena are fundamentally kinetic procedures involving the dynamic interplay between electronic, nuclear, and photonic DOFs at various time and length scales. Indeed, the accurate simulation of the quantum dynamics of a molecule itself is already a challenging task~\cite{Curchod2018CR, Crespo-Otero:2018wa, Nelson2020ChemRev}. The inclusion of photonic DOFs in polariton dynamics adds further complexity. To accurately simulate polariton dynamics, efficient and precise calculations of the gradients and derivative coupling of polariton states are essential. Despite the recent advancements in accurately computing gradients of polariton states~\cite{mandal_ChemRev2022}, more efforts are needed for large-scale ab initio simulations of polariton dynamics of many molecules.
 \item Polariton-mediated phenomena are fundamentally kinetic procedures involving the dynamic interplay between electronic, nuclear, and photonic DOFs at various time and length scales. Indeed, the accurate simulation of the quantum dynamics of a molecule itself is already a challenging task. The inclusion of photonic DOFs in polariton dynamics adds further complexity. To accurately simulate polariton dynamics, efficient and precise calculations of polariton states' gradients and derivative coupling are essential. Despite the recent advancements in accurately computing gradients of polariton states~\cite{mandal_ChemRev2022}, more efforts are needed for large-scale ab initio simulations of polariton dynamics in multiple molecules.
\end{enumerate}
Hence, simulating polariton chemistry is non-trivial, and state-of-the-art theoretical models and efficient numerical methods are crucial for understanding the polariton dynamics of many molecules.


\subsection{Common Approximations Toward Historic Quantum Optics Models}

The PF Hamiltonian (as written in Eq.~\ref{EQ:H_PF}) is derived by applying the {\bf dipole approximation}, which assumes the wavelength of the cavity fields is substantially larger than the matter system so that the spatial dependence of the transverse fields is neglected. In addition, many of the recent works on the simulation of \textit{ab initio} polaritons have relied on approximate versions of the PF Hamiltonian that stem from historical applications in the quantum optics community.\cite{yang2021JCP,zhang2019jcp,Tichauer2021JCP,Groenhof2018JPCL,Luk2017JCTC,Groenhof2019JPCL,Tichauer2022JPCL} In these heavily approximated Hamiltonians, usually a large truncation of the electronic and photonic subspaces is also performed such that only the ground and a single electronic excited state, $|g\rangle$ and $|e\rangle$ are included while only including the vacuum and singly excited Fock states, $|0\rangle$ and $|1\rangle$ in a single-mode cavity $\alpha = 0$. The total basis for this simple model is then confined to $\{|g,0\rangle,|g,1\rangle,|e,0\rangle,|e,1\rangle\}$. Note here that another approximation is that there is no permanent dipole in the ground $|g\rangle$ or excited $|e\rangle$ electronic states, \textit{i.e.}, $\bD_{gg} = \bD_{ee} = 0$. The most commonly used Hamiltonian for modeling \textit{ab initio} polaritons is the {\bf Jaynes-Cummings (JC)} Hamiltonian $\hH_\mathrm{JC}$, which can be written as,
\begin{equation}\label{EQ:H_JC}
    \hH_\mathrm{JC} = \hH_m + \hH_p + \sqrt{\frac{\omega_\mathrm{c}}{2}} \blambda \cdot \bD_{ge} (\hat{\sigma}\ha^\dag + \hat{\sigma}^\dag\ha),
\end{equation}
where $\hat{\sigma}^\dag$ ($\hat{\sigma}$) is the creation (annihilation) operator for the molecule excitation between the ground $g$ and excited $e$ states. And the {\bf Tavis-Cummings (TC)}~\cite{Tavis1968PR} model is the many molecules generalization of the JC model. 4) Besides, the feedback of molecular systems on the cavity modes (via the current density) is usually ignored. Here, two approximations have been made to derive JC/TC models: (I) the {\bf rotating wave approximation (RWA)} -- which is to say, neglecting the highly oscillatory $\hat{\sigma}^\dag\ha^\dag_\alpha$ and $\hat{\sigma}\ha_\alpha$ terms in the light-matter interaction -- and (II) neglecting the DSE $\hH_\mathrm{DSE}$. This Hamiltonian is valid at ultra-low coupling strengths where the splitting between the one-photon-dressed ground state $|g,1\rangle$ and the excited state with zero photons $|e,0\rangle$ exhibit linear splitting (\textit{e.g.} Rabi splitting) with an increase in the coupling strength $\blambda$. The other two basis states $|g,0\rangle$ and $|e,1\rangle$ are completely decoupled from the interaction.

Two other common approximations to the PF Hamiltonian are the {\bf Rabi model} $\hH_\mathrm{Rabi}$ after explicitly dropping the dipole self-energy (DSE) term and the RWA $\hH_\mathrm{RWA}$. These two approximate Hamiltonians can be written as,
\begin{align}\label{EQ:H_RABI__H_RWA}
    \hH_\mathrm{Rabi} &= \hH_\mathrm{M} + \hH_\mathrm{p} + \sqrt{\frac{\omega}{2}} \blambda \cdot \hat{\bD}(\hat{\sigma} + \hat{\sigma}^\dag) (\ha^\dag + \ha),\\
    \hH_\mathrm{RWA} &= \hH_\mathrm{M} + \hH_\mathrm{p} + \sqrt{\frac{\omega}{2}} \blambda \cdot \bD_{ge} (\hat{\sigma}_{ge}\ha^\dag + \hat{\sigma}_{ge}^\dag\ha) + \hH_\mathrm{DSE},
\end{align}
where the Rabi Hamiltonian $\hH_\mathrm{Rabi} = \hH_\mathrm{PF} - \hH_\mathrm{DSE}$ and the RWA Hamiltonian applies the RWA to the PF Hamiltonian. Here, $\hat{\sigma}_{ge} = |g\rangle\langle e|$ is the annihilation operator of the two-level electronic system. A more in-depth discussion on these Hamiltonians and the effects of the dipole self-energy term can be found in Refs.~\citenum{mandal_ChemRev2022},~\citenum{yang2021JCP},~\citenum{Rokaj2018JPB}, and~\citenum{Mandal_QED_eT_JPCB2020}.

%Note that in a two-state electronic system with only $g$ and $e$ states with no permanent dipole -- \textit{e.g.}, historically, a single-atom system -- then the DSE term only provides a uniform energy shift to the overall energy, similar to the zero-point energy of the photon Hamiltonian. In this sense, neglecting this term historically was well-motivated, but in systems with permanent dipoles in the ground or excited states (\textit{e.g.}, Li-F) or a system with many electronic states (\textit{e.g.}, molecules or solid-state materials), then the DSE term provides non-trivial physics and must be included. More recently, a many-level, many-mode generalization of these approximated Hamiltonians has been used, which retains the main approximations in each but is now not identical to the historical definition of each.\cite{yang2021JCP}

Since the correct PF Hamiltonian in the dipole gauge has been known, the question remains of why the community returns to the approximated Hamiltonians for \textit{ab initio} as well as model calculations. There are many subtleties to using the full PF Hamiltonian. From the electronic structure perspective, the many-level dipole operator needs to be computed as well as its square. The DSE term provides a very complicated description of the system for large coupling strengths since the dipole matrix, $\hat{\bD}$, in realistic molecules is far from sparse with its square leading to further complications.\cite{mandal_ChemRev2022,weight_abQED_JPCL2023} This allows for strong coupling between arbitrary states that is not trivial to know \textit{a priori} based on chemical or physical intuition. Often, these Hamiltonians are parameterized based on cavity-free electronic structure calculations to obtain the energies and dipoles of the electronic adiabatic states (see more details in Sec.~\ref{SEC:Direct_DIAG}). In this case, the number of included electronic (as well as photonic) basis states should be treated as a convergence parameter. In this case, the DSE causes mixing between far-separated-in-energy electronic states, which leads to slow convergence in the basis set size for the matter DOFs,\cite{yang2021JCP} and in general poor results compared to benchmarks, especially at large light-matter coupling strengths when using a small electronic basis.\cite{Flick2020JCP,wang_dissipative_2021JCP, mctague_nhCIS_JCP2022} This is precisely why many authors are developing self-consistent formulations to construct the low-energy polaritonic states without the need to calculate all the high-energy electronic states (see Sec.~\ref{SEC:Direct_DIAG}).\cite{Flick2020JCP,Vu_enhanced_JPCA2022,mctague_nhCIS_JCP2022,haugland2020PRX,weight_abQED_JPCL2023,Haugland_intermolecular_JCP2021,deprince_IP_EA_2021JCP,liebenthal_EOMCC_2022JCP}. Further when considering the quantum dynamical propagation of polaritons in \textit{ab initio} systems, additional nuclear gradients are required compared to cavity-free simulations where only the gradients of the adiabatic state energies are required to propagation the quantum dynamics, namely the nuclear gradients on the adiabatic dipole matrix and its square (\textit{i.e.}, $\nabla_{R} \hat{\bD}$ and $\nabla_{R} \hat{\bD}^2$).\cite{zhang2019jcp}


\subsection{Theoretical modeling of polariton chemistry}

\subsubsection{Cavity Born-Oppenheimer approximation}
Within the Born-Oppenheimer (BO) approximation,\cite{Flick2017JCTC} the total electronic-photonic-nuclear can be factorized as,
\begin{equation}
    \Phi(\br,\bR,{\bf q}_\alpha)=\chi(\bR)\Psi(\br,{\bf q}_\alpha;\bR),
\end{equation}
where $\chi(\bR)$ and $\Psi(\br,{\bf q}_\alpha;\bR)$ are the nuclear and polaritonic (\textit{i.e.,} electronic and photonic) wavefunctions, respectively. Note here that the polaritonic wavefunction is parameterized by the nuclear positions, exactly like the case without photonic DOFs outside the cavity. Further, one can invoke the usual Born-Huang-like expansion over the Born-Oppenheimer factorization as,
\begin{equation}
    \Phi(\br,\bR,{\bf q}_\alpha)=\sum_{\mu}\chi_a(\bR)\psi_\mu(\br,{\bf q}_\alpha;\bR),
\end{equation}
where $\psi_\mu(\br,{\bf q}_\alpha;\bR)$ are the BO wavefunctions analogous to those as outputted in standard electronic structure packages for the ground and excited adiabatic states. In this basis, which we will call the adiabatic polaritonic basis to draw a direct connection to the bare electronic case, we will discuss various ways to calculate such polaritonic wavefunctions $\psi_\mu(\br,{\bf q}_\alpha;\bR)$ from \textit{ab initio} calculations in a variety of approaches and levels of approximation.

In the following three sections, we will explore ways to obtain rigorous nuclear-position-parameterized wavefunctions for the entangled adiabatic electron-photon states $\psi_\mu(\br,{\bf q}_\alpha;\bR)$. First, a brief description of a direct diagonalization approach (Sec.~\ref{SEC:Direct_DIAG}) with Hamiltonians parameterized with information from standard electronic structure, while the following sections (Secs.~\ref{SEC:scQED_SP} and~\ref{SEC:scQED_CCSD}) will focus on the self-consistent approach toward re-developing the standard many-body schemes in electronic structure theory for the QED Hamiltonian (\textit{e.g.}, QED-HF, QED-DFT, etc.).



\subsubsection{Direct Diagonalization}\label{SEC:Direct_DIAG}

%Historically, the application of QED theory was focused on atomic physics. Here, single atoms were subjected to a cavity photon yielding Jaynes-Cummings-like linear splitting (see Eq.~\ref{EQ:H_JC}) since the atomic dipole matrix was approximated as a single matrix element between the ground and excited state. In this approximation, the QED Hamiltonian was ``parameterized'' by the energies of two electronic adiabatic states, $E_g, E_e$, and transition dipole element $\boldsymbol{\mu}_{eg}$ connecting the ground and excited state. For \textit{ab initio} systems, this single dipole matrix element can be approximated by a Taylor-Series expansion around the Frank-Condon point.\cite{Tichauer2021JCP,Tichauer2022JPCL,Luk2017JCTC} 
%In this way, one can directly diagonalize this 4x4 Hamiltonian in the $\{|g,0\rangle,|g,1\rangle,|e,0\rangle,|e,1\rangle\}$ using standard, exact diagonalization techniques. 
%The left basis $\{|g\rangle,|e\rangle\}$ states are the electronic adiabatic states that are eigenfunctions of the electronic Hamiltonian $\hH_\mathrm{el} = \hat{T}_e + \hat{V} = \hH_\mathrm{M} - \hat{T}_R$, where $\hat{V}$ includes all Coulomb interactions between nuclei and electrons. The right basis states $\{|0\rangle,|1\rangle\}$ are the vacuum and single-photon Fock states, which are eigenstates of the photonic Hamiltonian $\hH_\mathrm{p} = \hbar \omega_\alpha(\ha_\alpha^\dag\ha_\alpha + \frac{1}{2})$. 
%In this minimal basis and for a single cavity mode ($\alpha = 0$), the JC Hamiltonian can be written as,
%\begin{equation}
%    (\hH_\mathrm{JC})_{IJ,nm} \dot{=} 
%    \begin{bmatrix}
%    E_g & 0 & 0 & 0 \\
%    0 & E_e & \sqrt{\frac{\omega_0}{2}} \blambda_0 \cdot \bD_{eg} & 0\\
%    0 & \sqrt{\frac{\omega_0}{2}} \blambda_0 \cdot \bD_{eg} & E_g + \hbar \omega_0 & 0\\
%    0 & 0 & 0 & E_e + \hbar \omega_0
%    \end{bmatrix},
%\end{equation}
%where the lowest and highest-energy basis states $\{|g,0\rangle,|e,1\rangle\}$ are completely decoupled from the other basis states as well as each other. 

%In this sense, the polaritonic ground state $|\Psi_{\mu = 0}\rangle = |g,0\rangle$ and the fourth polaritonic state $|\Psi_{\mu = 4}\rangle = |e,1\rangle$, which is valid for ultra-low light-matter coupling strengths where the linear splitting is valid at the resonance condition. For non-linear effects, such as those including permanent dipoles, off-resonant excitations, or simply the exploration of the ground polaritonic state. To make this leap, one needs to make two major changes: (I) invoke the more rigorous Pauli-Fierz (PF) Hamiltonian $\hH_\mathrm{PF}$ (see Eq.~\ref{EQ:H_PF}) and (II) treat the number of basis electronic and photonic states convergence parameters.

The polaritonic state can be readily computed via exact diagonalization of polariton Hamiltonian (either JC, TC, Rabi, RWA, or rigorous PF) in a certain basis set. For instance, 
the matrix elements of the single-mode PF Hamiltonian in the widely used adiabatic-Fock basis 
%-- but now extended to include, in principle, an infinite number of electronic $\{I,J,K\}$ and photonic $\{n,m\}$ basis states -- 
can be written as,
\begin{align}
    (\hH_\mathrm{PF})_{IJ,nm} &= \bigg[ E_{I} + \omega_\alpha(n + \frac{1}{2})\bigg]\delta_{IJ}\delta_{nm} \\
    &+ \sqrt{\frac{\omega_\alpha}{2}} \blambda \cdot \bD_{IJ} (\sqrt{m+1}\delta_{n,m+1} + \sqrt{m}\delta_{n,m-1})\nonumber  \\
    &+ \frac{1}{2} \sum_{K}^{\mathcal{N}_\mathrm{el}} (\blambda \cdot \bD_{IK}) (\blambda \cdot \bD_{KJ}) \delta_{nm}\nonumber,
\end{align}
where $\mathcal{N}_\mathrm{el}$ is the number of adiabatic electronic states included in the basis. It is important to note that in this basis, the PF Hamiltonian is extremely sparse since the coupling elements only connect adjacent Fock states via the molecular dipole matrix since the matrix elements of the photonic coordinate $\hat{q}$ are that of the harmonic oscillator (\textit{i.e.}, only have nonzero super- and sub-diagonal elements). Further, the DSE contributions, in general, connect all of the electronic states of the system (with the same photon number) and is the most non-trivial aspect of this Hamiltonian and will vary strongly between molecular systems. 

Except for the adiabatic-Fock basis, other options for expanding the JC and PF Hamiltonians exist, such as the coherent states\cite{philbin2014AmerJourPhys,haugland2020PRX,riso_molecularQED_NatComm2022} or polarized Fock states.\cite{Mandal2019JPCL} In both of these bases, the photonic states are chosen such that the molecular dipole parameterizes the photonic state, thereby, in principle, reducing the convergence of the photonic basis. Further, this ``direct diagonalization'' approach to solving the PF Hamiltonian is dependent on this basis convergence, and the electronic basis is much more rigid since the adiabatic basis is ubiquitously used for its convenience. However, the convergence of this basis has yet to be thoroughly tested for a wide range of systems in solving the PF Hamiltonian, but it is expected to converge slowly due to the contributions from the DSE term.\cite{weight_abQED_JPCL2023,yang2021JCP} This evidences the need to move to a more rigorous self-consistent solution for the polaritonic adiabatic states as is done for the electronic adiabatic states themselves.

\subsubsection{Self-consistent Polaritonic Single-particle Approaches}\label{SEC:scQED_SP}

{\bf scQED Hartree-Fock and Density Functional Theories}\newline
The mean-field approach to the QED Hamiltonian can be first cast in an identical way as the standard Hartree-Fock procedure in a larger Hilbert space, including the photonic states~\cite{riso_molecularQED_NatComm2022,riso_QEDionization_JCP2022,deprince_IP_EA_2021JCP,liebenthal_EOMCC_2022JCP,mctague_nhCIS_JCP2022}. A useful basis, referred to as coherent states (CS), can be performed such that the light-matter interaction part of the PF Hamiltonian can be shifted away. 

For a given bare electronic ground state wavefunction ($|\mathrm{HF}\rangle$), the photonic Hamiltonian can be obtained by integrating out the electronic DOF,
\begin{align}\label{EQ:PF_HF_partialDIAG}
\hat{H}_p \equiv \langle \mathrm{HF}&| \hat{H}_\mathrm{PF} |\mathrm{HF}\rangle = E_\mathrm{HF} +  \omega_\alpha \big(\hat{a}^\dag_\alpha \hat{a}_\alpha+\frac{1}{2}\big)\\
&+\sqrt{\frac{\omega_\alpha}{2}} \langle \blambda_\alpha \cdot \hat{\bD} \rangle_\mathrm{HF} \cdot (\hat{a}^\dag_\alpha + \hat{a}_\alpha)+ \frac{1}{2} \langle (\blambda_\alpha \cdot \hat{\bD})^2\rangle_\mathrm{HF},\nonumber
\end{align}
where $\langle \cdots \rangle_\mathrm{HF} = \langle \mathrm{HF} | \cdots | \mathrm{HF} \rangle$ is the HF ground state expectation value of the electronic subsystem. 
The photonic Hamiltonian can be trivially diagonalized 
by introducing the CS transformation $\hat{U}({\bf z})\hat{H}_p\hat{U}^{\dagger}({\bf z})$,
\begin{equation}\label{EQ:CS_ROTATION}
    \hat{U}({z_\alpha}) = e^{z_\alpha\hat{a}_\alpha^\dag - z_\alpha^*\hat{a}_\alpha},
\end{equation}
where ${\bf z} = \{z_\alpha\}$ is a vector of complex numbers specific to each cavity mode,
\begin{equation}
    {\bf z} \rightarrow -\frac{\langle \blambda_\alpha \cdot \hat{\bD}\rangle_\mathrm{HF}}{\sqrt{2\omega_\alpha}}.
\end{equation}
Thus, considering a QED-HF ansatz as $\ket{HF}\otimes\hat{U}({\bf z})\ket{0}$, the QED-HF energy and corresponding Fock matrix can be derived. 
Compared to the HF theory for bare electrons, the QED-HF method introduces DSE-mediated one-body and two-body integrals~\cite{foley:2023arxiv}.


Similarly to the scQED-HF approach, the self-consistent QED density functional theory (scQED-DFT) is composed in a similar way, where the main elements of DFT remain, such as the exchange-correlation functional of the density. 
%However, there are additional photonic DOF that influence the electronic subsystem. As for the scQED-HF case, the coherent state basis removes the light-matter coupling term while leaving the DSE term. In this sense, the energy of the Kohn-Sham energy can be written as,
%\begin{equation}
%    E_\mathrm{QED-DFT} = E_\mathrm{DFT} + \frac{1}{2}\sum_\alpha \langle(\blambda_\alpha \cdot \hat{\bD})^2\rangle,
%\end{equation}
%which is then minimized with respect to the Kohn-Sham spin-orbitals. 
There are many ways to set up the scQED-DFT problem, such as employing a novel exchange-correlation functional to account for light-matter correlation effects\cite{pellegrini_OEP_PRL2015,ruggenthaler_QEDFT_2014,ruggenthaler_QEDSpectra_NatRevChem2018,flick_KSQED_PNAS2015} or working with electron-only exchange-correlation functionals using the coherent state basis for the photonic DOFs.\cite{Vu_enhanced_JPCA2022,yang2021JCP,haugland2020PRX} In any case, the resulting ground state is uncorrelated by the nature of the DFT formalism.

{\bf Excited States scQED-TD-(HF,DFT)}

The time-dependent analogues to the aforementioned single-particle approaches are powerful tools to probe non-equilibrium densities that give rise to electronic excited states. One of the most popular approaches is one of linear response (LR), resulting in LR-TD-HF and LR-TD-DFT for bare molecular systems in the random phase approximation (RPA). Although, it should be noted that the real-time propagation of the single-particle density matrix -- leading to the real-time TD-HF and real-time TD-DFT approaches -- is, in principle, a more robust approach but one that is usually more costly than that of linear response. Such schemes have already been developed for the simulation of molecular polaritons using classical photon DOFs.\cite{Li_QEDNEO_JCTC2022,li_QEDNEO_2023}  Here, we will focus our attention on the LR formalism, specifically using a Casida-like approach to writing the random phase approximation (RPA), originally formulated by Flick and co-workers\cite{Flick2020JCP} using the QEDFT (or scQED-DFT in the notation of this0 work) method in the language of Casida and further used by the groups of Shao\cite{yang2021JCP} and DePrince.\cite{Vu_enhanced_JPCA2022} It should be noted that other formulations of CIS-like excited states can be found in the community, such as the non-Hermitian CIS aimed at simulating cavity loss via a complex photon frequency\cite{mctague_nhCIS_JCP2022}.

As per usual, and following the notation of Ref.~\citenum{Vu_enhanced_JPCA2022}, the LR-TD-HF and LR-TD-HF eigenvalue equations using the Casida formalism can be written to include the QED components that satisfy the Pauli-Fierz Hamiltonian (Eq.~\ref{EQ:H_PF}).

An important distinction between various implementations of the QED-TD-DFT approaches in the community is whether the single-particle orbitals used in the formulation are ``relaxed'' in the presence of the cavity or are simply the bare electronic single-particle states. For example, in Ref.~\citenum{yang2021JCP}, the orbitals are not relaxed while in the Refs.~\citenum{Flick2020JCP} and~\citenum{Vu_enhanced_JPCA2022} the orbitals are relaxed. While it is clear that using a relaxed reference state for the basis of the RPA equations would provide a more rigorous result, it is not clear whether identical results can be obtained in the infinite basis limit of both approaches, \textit{i.e.}, including more single-particle states in the CIS-like expansion in excited Slater determinants. Since the RPA equations are iteratively solved, the expansion coefficients of the excited Slater determinants may result in the same excited state observables in the infinite basis limit, while for a finite basis, it may not. 

\subsubsection{Self-consistent Polaritonic Coupled Cluster Approaches (CC,EOM-CC)}\label{SEC:scQED_CCSD}

Despite computational efficiency, DFT or mean-field HF methods usually underestimate the correlations. In particular, the mean-field method cannot describe the electron-photon and photon-mediated electron-electron correlations, and the exchange-correlation function for electron-photon interaction is unknown. Consequently, the QED counterpart of coupled-cluster theory (QED-CC) is proposed.\cite{deprince_IP_EA_2021JCP,Haugland_intermolecular_JCP2021,liebenthal_EOMCC_2022JCP,mordovina_QEDCC_PRR2020,pavosevic_ClickChem_Arxiv2022,pavosevic_PTQED_JACS2022,fregoni_QEDCCPlasmon_NanoLett2021} Similar to the conventional CC theory, QED-CC employs an exponential wavefunction Ansatz to derive the ground state, 
\begin{equation}\label{eq_ansatz}
    \ket{\Psi_{CC}}=e^{\hT}\ket{\Phi_0},
\end{equation}
where $\ket{\Phi_0}$ is the reference wave function, which is usually chosen to be the tensor product of  Hartree-Fock (HF) determinant and photon vacuum state, i.e., $\ket{\Phi_0}=\ket{\varphi_0}\otimes\ket{0}$. $\hT$ is defined in excitation configurations. Within QED-CC theory, the generalized excitation operator includes three components, $\hT=\hT_e+\hT_p+\hT_{ep}$, including electronic ($\hT_e$), photonic ($\hT_p$), and coupled electronic-photonic ($\hT_{ep}$) excitations,
\begin{align}
T_e=&\sum_{ia} t^a_i \ha^\dag_a\ha_i + \sum_{ijab} t^{ab}_{ij} \ha^\dag_a\ha^\dag_b\ha_j\ha_i +\cdots\equiv \sum^{N_e}_{\mu} t_{\mu} \htau_\mu, 
\\ 
\quad \hT_{p}= & \sum_\alpha \gamma_\alpha \hb^\dag_\alpha +
\frac{1}{2}\sum_{\alpha\beta} \gamma_{\alpha\beta} \hb^\dag_\alpha\hb^\dag_\beta + \cdots \equiv \sum^{N_p}_n \gamma_n \hat{B}_n,
\\
\quad \hT_{ep}=&\sum_{ia,\alpha} t^a_i \ha^\dag_a\ha_i \hb^\dag_\alpha + \frac{1}{2}\sum_{ijab,\alpha\beta} t^{ab}_{ij} \ha^\dag_a\ha^\dag_b\ha_j\ha_i  \hb^\dag_\alpha\hb^\dag_\beta
+ \cdots \equiv 
\sum_{\mu,n} \chi_{\mu,n}\htau_\mu \hat{B}_n.
\end{align}
where $N_e$ and $N_f$ are the numbers of electrons and photon Fock states, respectively. $\htau_\mu=\prod^\mu_k \hE^{a_k}_{i_k}$ and $\hE^a_i=\ha^\dag_a \ha_i + h.c.$ are the $\mu$-body and single-body excitation operators, respectively. $\hat{B}_n=\prod^n_{\alpha} b^\dag_\alpha$ is the $n$-body photonic excitation. 
The parameters ($t_\mu, \gamma_n, \chi_{\mu,n}$) are the cluster amplitudes. By projecting CC wavefunction into a set of orthogonal excited configurations ($\{\ket{\mu}\}$),
\begin{equation}
    \ket{\mu}=\hat{\mu}\ket{\Phi_0},
\end{equation}
where $\mu\in\{\htau_\mu, \hat{B}_n, \htau_\mu \hat{B}_n\}$ is the cluster operator,
The cluster amplitudes can be determined from the projected equation by following the standard CC procedure,
\begin{equation}
    \Omega_\mu\equiv\bra{\mu}e^{-\hT}\hH e^{\hT}\ket{\Phi_0}=0
\end{equation}
Though the scQED-CC theory leads to the FCI solution if no truncation on the excitation operator $\hT$ is applied, the excitation operator is usually truncated at the doubles (CCSD) level in order to trade-off between accuracy and computational efficiency. Some tests have been performed on the level of truncation in the photonic excitations (up to 10) in the CC operator for model systems.\cite{mordovina_QEDCC_PRR2020} However, the scaling of scQED-CCSD is $\mathcal{O}(N_{el}^6 N^{M_p}_{p})$ in general, where $M_p$ is the number of photon modes. Such a scaling makes it a bit expensive to rigorously test in extended molecular systems that include many electrons and/or when many photon modes and Fock states are used in the calculations. To further reduce the computational cost, the coherence state~\cite{Philbin:2014jcp} may be used to reduce the photon basis in the scQED-CC calculations.

\begin{figure}[t!]
    \centering
    \includegraphics[width=0.4\textwidth]{figures/eomccsd_lih.jpg}
    \caption{Polariton states of LiH as a function of Li-H bond length. (a) CCSD (black circles) and scQED-CCSD (solid red curve) ground states at zero light-matter coupling strength $\lambda = 0$ a.u. (b) EOM-CCSD (black circles) triplet $|T_1\rangle$ and singlet $|S_1\rangle$ states as well as the low-lying states ($|T_1\rangle$, $|\omega\rangle$, and $|S_1|rangle$) calculated at scQED-EOM-CCSD with light-matter coupling strength $\lambda = 0$ a.u. The plot clearly shows the photon line ($|\omega\rangle$) and perfect overlap of the $S_1$ states. (c) Upper ($|P_U\rangle$) and lower ($|P_L\rangle$) polariton states with $\lambda = 0.088$ a.u. with the triplet state $|T_1\rangle$ uncoupled.}
    \label{fig:lih_r}
\end{figure}

It should be noted that CC methods exhibit size extensivity even in their truncated forms, meaning that the sum of the energies of non-interacting subsystems is equal to the total energy. This is in contrast to CI approaches, where errors in extensivity can become increasingly large with an increasing number of subsystems. In addition, CC theory has the advantage of size extensivity in that the excitation energies of each subsystem do not vary with the size of the total system when the subsystems are non-interacting, making it an ideal candidate for computing the polaritonic eigenstates of many (non-interacting) molecules. 

%\subsubsection*{Excited States}
In addition, the excited polariton states can be computed with the corresponding EOM formalism, which parameterizes a neutral or charged excitation by applying an excitation operator to the CC ground state~\cite{jcp0033132}
\begin{equation}
    \ket{\hR}=\hR\ket{\Phi_{cc}}=\hR e^{\hT}\ket{\Phi_0,0}.
\end{equation}
Where $\hR$ also include the electronic ($\hR_e$), photonic ($\hR_p$), and coupled electronic-photonic ($\hR_{ep}$) excitations.
Because the excitation operator, $\hR$, commutes with the excitation operators in $\hT$, solving this eigenvalue problem is equivalent to finding a right eigenvector of the similarity transformed Hamiltonian,
\begin{equation}
    \bra{\mu}\bar{H} \hR^n \ket{\Phi_0,0}= E_n \hR^n_\mu.
\end{equation}
Here, $E_n$ is the energy of the nth excited state, and $\mu$ indexes an element of the excitation operator $\hR$. The excitation operator, $\hR$, can be chosen to access charged or neutral excitations. 


We recently implemented such QED-CCSD/EOM-CCSD method to compute the polariton states. With the diagrammatic technique, an auto code generator and optimizer are developed to generate the QED-CCSD/EOM-CCSD equations to arbitrary photon order. Fig.~\ref{fig:lih_r} is an example of the Li-H bond dissociation curves in a single-mode optical cavity. Fig.~\ref{fig:lih_r}a shows the ground state Born-Oppenheimer potential energy surface for the bare electronic system (black circles) and for the polaritonic system (red curve) at zero light-matter coupling strength ($\lambda = 0$). The excited states inside (red curves) and outside (black circles) of the cavity are shown at zero coupling strength ($\lambda = 0$), where two multiplicities (singlet $|S_1\rangle \equiv |S_1,0\rangle$ and triplet $|T_1\rangle \equiv |T_1,0\rangle$) of electronic states are shown as well as a single cavity state ($|\omega\rangle \equiv |S_0,1\rangle$). Here, we have used the notation $|S_J\rangle\otimes|n\rangle = |S_J,n\rangle$ where the left ket in the product signifies the electronic DOFs while the right signifies the photonic Fock state label (\textit{i.e.}, the number of photons in the photon-dressed electronic state). At finite light-matter coupling of $\lambda = 0.088$ a.u., the cavity state with one photon $|\omega\rangle$ couples strongly with the singlet excited electronic state with zero photons $|S_1\rangle$ while the excited electronic triplet state $|T_1\rangle$ is negligibly affected. The so-called Rabi splitting appears at the degenerate point between the singlet electronic state and the cavity state, $R_\mathrm{Li-H} \approx 3.75$, forming the upper ($|P_U\rangle$) and lower ($|P_L\rangle$) polaritonic states with mixed electronic and photonic character. 

The computational efficiency of many-body electronic (or polaritonic) structure codes is nearly as important as the method itself. We have implemented various backend options for our code (presented in Fig.~\ref{fig:lih_r}). The efficiency of Numpy \textit{einsum} (CPU-accessible) and Torch \textit{einsum} (GPU-accessible) functions are shown in Fig.~\ref{fig:modularcode} on a vertical log scale as a function of the number of orbitals (\textit{i.e.}, electrons) included in the calculation. The GPU hardware allows the QED-CC code to consistently operate with an order of magnitude less wall time than the CPU version. These results indicate that the conversion toward GPU hardware is required for exploring the frontier science of molecular polaritons.

\begin{figure}[t!]
  \vspace{-0.0 em}
    \centering
    \includegraphics[width=0.99\linewidth]{figures/backends.png}
    \caption{\small  Example of running calculations with our modular QED-CC code on different architectures (CPU vs. GPU on a desktop) by simply setting the backend (see the inset code). $10\times-20\times$ speed up is observed on Torch GPU.
    }
    \vspace{0 em}
    \label{fig:modularcode}
\end{figure}

