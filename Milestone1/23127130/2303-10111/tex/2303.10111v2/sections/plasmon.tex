\section{Plasmon chemistry} \label{SEC:PLASMON}

Although confined plasmonic modes result in localized and enhanced fields, plasmons usually suffer from strong dissipation, making it hard to enter the strong coupling regime in plasmonic systems. Nevertheless, even without strong coupling, plasmonics provides a unique setting for manipulating light via the confinement of EM (below the diffraction limit). Such extreme concentration of EM field~\cite{Schuller2010NatMat} has led to a wide range of applications, such as plasmon-enhanced molecular spectroscopy~\cite{Stiles2008annurev, Jiang2003jpcb, Brus2008acr, Nie1997sci, Zhan:2018tv}, photovoltaics~\cite{NatMater2010,zhang2016fundamental}, nanophotonic lasers and amplifiers~\cite{Ma2013laser, Berini2012natphotonics, Guan2021am, Guan2022cr}, quantum information~\cite{Flamini_2018}, and many others~\cite{Cushing016jpcl, Mirkovic2018cr, Li2015natphotonics}. 

Following the ultrafast plasmon excitation, nonradiative plasmon decay leads to the formation of energetic electron–hole pairs (namely hot-carriers)~\cite{Zhang2021JPCA, Clavero2014,bernardi2015,acsnano5b06199,nn502445f}, which are highly nonthermal and can have considerably higher energies than those rising from thermal equilibrium. The hot electrons (HEs) (and their corresponding holes) redistribute their energies quickly as a result of electron–electron scattering~\cite{Zhang2021JPCA, Clavero2014}, reaching a quasi-thermal equilibrium but with an effective high temperature. Further cooling of the hot electrons takes place via energy dissipation into the phonon modes of the nanoparticle, and the energy is ultimately dissipated to the surroundings via thermal conduction. Nevertheless, investigations in the last two decades have found that chemical reactions can be stimulated by localized EM fields (or photons), and electronic and/or thermal energies (that result from plasmon decay) via various pathways (more details in Sec.~\ref{sec:plasmonchemdetails}). This leads to an emerging field of plasmon chemistry that designs nanostructure-based surface plasmons as mediators to redistribute and convert photon energy in various time, space, and energy scales to drive chemical reactions~\cite{Zhan:2018tv, Zhan:2023tc, Yuan:2022tb, Brongersma:2015vd, Zhou:2018tu, wu2023chemsci, zhang2018plasmonic, wu2020mechanistic, Kazuma2019angew, Kazuma2018science, ZhangYuchao2018chemrev, Tesema2019jpcc, Adleman2009nl, Boergter2016natcomm, zhang2019atomistic}.
One extraordinary feature of Plasmon chemistry in the realm of chemical reaction mechanisms is its blend of facets from thermochemistry, photochemistry, and photocatalysis~\cite{Zhan:2018tv}. Unlike traditional reactions that typically focus on singular mechanisms, plasmon chemistry showcases the interaction between different mechanisms in complex electronic and optical environments. At the same time, the mixture of different mechanisms makes it very challenging to comprehend the underlying principles of plasmon chemistry, making it considerably more complicated than transition chemistry. This unique interplay often results in a diverse distribution of reactive zones on substrates~\cite{Zhan:2018tv}. To fully understand plasmon chemistry's nuances, an interdisciplinary approach is essential, which should consider multi-scale processes, the current state of the field, and the need for advanced experimental methods.

\subsection{Experimental Endeavours in Plasmon-Mediated Chemical Reactions: The Need for Theoretical Insights}

Plasmon chemistry stands at the intersection of innovation and discovery for boosting chemical transformation. However, despite the spreading interest, inherent complexities make it still very challenging for efficiency optimization. By leveraging knowledge from established domains such as plasmon-enhanced spectroscopy, many experimental strategies to bolster plasmon chemistry efficiency can be extrapolated. In this section, we discuss several typical experimental investigations into plasmon chemistry across diverse applications and illustrate the essential role of theoretical/numerical modeling in understanding the underlying mechanisms for further improvement. For a more expansive disquisition on the plasmon-mediated chemical reactions for various applications, enthusiasts are redirected to several recent reviews~\citenum{Gelle:2020uh,Zhan:2019wp, ZhangYuchao2018chemrev, Li:2023ux}.

\begin{figure*}[!htb]
    \centering
    \includegraphics[width=0.95\textwidth]{figures/experimental_examples.png}
    \caption{
    Experimental examples: 
    a) Structure and mechanism of operation of the autonomous plasmonic solar water splitter. Adapted with permission from Ref.~\citenum{Mubeen:2013tk}. Copyright 2013 Springer Nature Limited.
    b) Light-driven CO$_2$ reduction with single atomic site antenna-reactor plasmonic photocatalysts. Adapted with permission from Ref.~\citenum{Zhou:2020tn}. Copyright 2020 the author(s), under exclusive license to Springer Nature Limited.
    c) Schematic diagram of the proposed mechanism for the oxidation of PTAP to DMAB. Hot electrons generated from the plasmon transfer to the adsorbed O$_2$ molecules, generating $^3$O$_2$ that participated in the PATP oxidation to DMAB. Adapted with permission from Ref.~\citenum{Wang:2015uk}. Copyright 2015 John Wiley and Sons.
    d) Schematic of the three-electrode electrochemical system for distinguishing thermal from hot electron effects by utilizing photoelectrochemical characterization under the chopped light. Adapted from Ref.~\citenum{Zhan:2019wc} under Creative Commons CC BY.
    }
    \label{fig:exp}
\end{figure*}

\textbf{Artificial photosynthesis}
One of the most enticing applications of plasmon chemistry echoes the wonders of photosynthesis~\cite{Mirkovic2018cr, Mubeen:2013tk, Xiao:2017uw, Yu:2019ut}. The goal is to convert solar energy into useful chemicals, a task intricate due to its kinetic and thermodynamic demands. However, in advancements in the realm of solar conversion, a groundbreaking study has showcased the potential 
 of autonomous plasmonic catalysts capable of water splitting under visible light~\cite{Mubeen:2013tk}, as shown in Figure~\ref{fig:exp}a, present promising trajectories for renewable energy solutions.
This paradigm shift focuses on deriving charge carriers from these plasmonic structures.
The research highlights the introduction of a solar water-splitting device engineered from a gold nanorod array. Unlike conventional methods, this device leverages the power of ``hot electrons" initiated by stimulating surface plasmons within the gold nanostructure. A standout feature is each nanorod's autonomous operation, eliminating the need for external wiring and its impressive capability to produce H2 molecules under sunlight. The device's resilience and long-term stability further cement its potential.
While the initial solar-to-hydrogen efficiency aligns with that of early semiconductor-based water splitters, it remains on the lower side for real-world applications. Nevertheless, the research suggests prospective structural enhancements, emphasizing the modification of the nanorod configuration, that could potentially escalate its efficiency. An important note for consideration, despite the current efficiency metrics, is the device's remarkable longevity, which overtakes some of the top-performing semiconductor-based alternatives. This study marks a pivotal step towards the innovative utilization of plasmonic devices in solar conversion.


\textbf{CO$_2$ Reduction.} 
Given the increasing global CO$_2$ levels, plasmon chemistry offers a promising solution. Plasmonic structures have shown potential in boosting the photo-reduction of CO$_2$~\cite{Singh:2023wt, Zhou:2020tn, Li:2023ux, Dhiman2023chemsci}, which is crucial for environmental cleanup and advancing green chemistry. In a new development, researchers have presented a plasmonic photocatalyst that blends a copper (Cu) nanoparticle ``antenna" with individual atomic ruthenium (Ru) sites (Figure~\ref{fig:exp}b). This design allows methane dry reforming to happen at room temperature, using the energy from light. Unlike traditional thermocatalytic methods, this photocatalyst works effectively with light at normal conditions and has notable stability and selectivity in its actions. This difference from standard thermally-driven reactions is due to the creation of hot carriers, which improve the rate of carbon-hydrogen activation on Ru sites and speed up hydrogen release. It should also be noted that, different from conventional plasmonic structure, the design of antenna–reactor systems utilize plasmonic metal (the antenna) to collect and concentrate visible light energy and transfer that energy to a catalytic metal (that is, the reactor) to drive a chemical reaction~\cite{Aslam:2017tw, Zhang:2016vx, Swearer:2016pnas}. Because the reactors were not in direct contact with the plasmonic nanoparticles, it was argued that the energy transfer resulting in increased reaction rates could only take place via a field effect where the field from the plasmonic metal was felt by the catalytic reactors resulting in the excitation of charge carriers in reactors.

\textbf{Organic transformation.}  Beyond the catalysis of inorganic reactions, as previously addressed, plasmonic NPs have showcased commendable performance in catalyzing organic transformations. These hold potential applications, spanning from the synthesis of basic commodity molecules to intricate pharmaceutical compounds. This approach boasts benefits such as superior selectivity, accelerated reaction rates, and comparatively lenient reaction conditions. An illustrative instance is the oxidation of p-aminothiophenol (PATP) or p-nitrothiophenol (PNTP) to dimercaptoazobenzene (DMAB). This reaction has been the focal point of myriad studies in recent years~\cite{Wang:2015uk, Silva:2016tp, Huang:2010uf, Zhan:2019wp, Huang:2014ux}. The hypothesized microscopic mechanism posits that plasmon-excited electrons transfer from plasmonic nanostructures to O$_2$ molecules, resulting in the emergence of activated oxygen species, which, in turn, catalyze the formation of DMAB, as shown in Figure~\ref{fig:exp}(c). 
Such a mechanism found experimental validation through experiments toggling specific conditions. The obvious evidence of PATP's transformation into DMAB, when air permeates the PATP and plasmonic nanostructured system, is corroborated by the ascendant Raman modes of DMAB over time. In contrast, in the absence of O$_2$ or electron transfer (attained by coating the plasmonic NPs with an insulator like SiO$_2$), merely the Raman signals of PATP are~\cite{Huang:2014ux}. In addition, it was found that the plasmon-induced hot holes can trigger the oxidation of PATP, even in the absence of oxygen~\cite{Zhao:2014vj}. Notably, despite the absence of a tangible PATP transformation on gold nanoparticles due to inefficient charge transfer from the plasmonic nanostructure directly to the adsorbates, PATP oxidation has been witnessed either in the presence or absence of O$_2$, specifically with Au nanostructures fortified with a TiO$_2$ shell. These reactions outperform their counterparts in sheer gold nanoparticle systems in terms of efficiency~\cite{Zhan:2019wp}. Fascinatingly, it was unveiled that achieving the selective oxidation of PATP to either DMAB or PNTP is possible, depending on whether the plasmonic NPs are reinforced by TiO$_2$ and the presence of UV-illumination~\cite{Wang:2015uk}. Under UV exposure on the bare plasmonic NPs, PATP's oxidation to DMAB is observed. In contrast, there is no catalytic effect with the TiO$_2$ support. However, after applying UV light exposure, a transition to PNTP is detected with the TiO$_2$ support. For an expansive disquisition on the plasmon-mediated organic transformation, enthusiasts are redirected to a contemporary review in Ref.~\citenum{Gelle:2020uh}.

\textbf{Experimental Limitations.} Though there are many demonstrated applications of plasmon chemistry as a promising approach to accelerate or manipulate chemical processes, the underlying mechanisms are still unclear. 
The advances in experimental techniques have played a pivotal role in elucidating plasmon chemistry mechanisms. Techniques, such as transient absorption spectroscopy, offer insights into the domain of plasmonic hot-carriers~\cite{Reddy:2020sci}. Nonetheless, the vast expanse of these reactions remains to be traversed for a holistic comprehension. 
Traversing the landscape of plasmon chemistry requires a comprehensive understanding of multiple coherent and dissipative processes, including the enhancement of electromagnetic near-fields, local heating effects, and charge-carrier excitation/transfer. These processes occur across varied scales, from femtoseconds to nanoseconds. Since all these effects can drive chemical processes simultaneously, the individual contributions of these elements to overarching reactions are challenging to discern. On the other hand, decoupling individual contributions is vital for designing better plasmon catalysts that synchronize the orchestration of these effects to improve efficiency. In particular, distinguishing the thermal effect from the hot electron effect has become an important topic and is the subject of many debates~\cite{}. Despite the complexity of distinguishing these effects, there are growing efforts to solve these key questions via various designs~\cite{Zhan:2019wc, Yu:2018vh, Zhou:2018aa, Zhang:2018ug, Ou:2020vk, Baffou:2020ts}. One of these efforts utilized the unique photoelectrochemical behavior of a plasmonic Au nanoelectrode array as shown in Figrue~\ref{fig:exp}d~\cite{Zhan:2019wc}. The plasmonic photocurrent can be intensified at negative and positive potentials, with its direction determined by the selected potential. This allows the electrode to favor either reduction or oxidation reactions. The photocurrent is composed of a fast photoelectronic component and a slower photothermal component. The photoelectronic current aligns with the plasmon absorption spectrum and intensifies with light intensity, whereas the photothermal current displays a linear relationship only within specific light intensities. Consequently, this proposed design can differentiate between the photoelectronic and photothermal effects. However, the photoelectronic response measured in Ref~\ref{Zhan:2019wc} happens at a time scale several orders of magnitude larger than the hot-electron lifetime and thermal conduction time scales. More advanced time-resolved techniques are required to provide more comprehensive evidence.

Overall, the distinctiveness of plasmon chemistry is rooted in the interplay between molecules, incident photons, and plasmonic nanostructures. Though many recent breakthroughs have been made to demonstrate plasmon chemistry's application in various areas, further improvements require a deep microscopic understanding of the synergy of multiple effects at the electron level. Such understanding will play an indispensable role in the strategic design and fabrication of plasmonic nanostructures, optimization of surface or interface intermediaries, and their harmonized integration. To elucidate the intricate mechanism that governs plasmon chemistry, comprehensive investigations encompassing various time, spatial, and energy scales are essential, which cannot be achieved by experiments alone. It's crucial that advanced experimental methods, boasting high space-time resolutions, are synergized with microscopic theoretical modeling. 
Specifically, theoretical models capable of addressing all potential mechanisms on equal footing provide invaluable perspectives into quandaries that remain elusive to experimental endeavors alone. Intuitively, the convenience of modeling allows for the toggling of individual effects to scrutinize their ramifications on holistic reactions. The real theoretical challenge, however, lies in considering these multifaceted effects equitably, spanning varied time and length scales. In the following sections, we will review our theoretical efforts toward simulating all conceivable mechanisms on an equal footing, with an aspiration to understand plasmon chemistry at the electron level.


\subsection{Ab initio method for quantum plasmonics}
In most scenarios, the nanoplasmonic properties are obtained by solving Maxwell's equations with proper dielectric function and boundary conditions. However, Maxwell's equations fail to describe the quantum effects (such as the tunneling in charge transfer mode and quantum confinement~\cite{Marinica2012nl}), and quantum theory for plasmon (quantum plasmonics) is required in nanoscale. 

From a quantum mechanical point of view, a plasmon is nothing but a specific collective excitation. Hence, the quantum chemistry method for excited states can be used to predict the optical properties of plasmonic nanostructures~\cite{Varas:2016tf}. The many-body perturbation theory, time-dependent density functional theory (TDDFT), or time-dependent density functional tight-binding (TDDFTB) are methods of choice to compute the plasmon excitations~\cite{Perera:2020jpcc, Rossi:2017jctc, Asadi:2020jpcc, Pandeya:2021jpca, Alkan:2018ug, Alkan:2021vq, DellaSala:2022vo, DAgostino:2018tg, You:2019vn}. Without losing generality, here we use TDDFT theory as an example to demonstrate the computation of plasmon excitations. 

TDDFT is the formal extension of the Hohenberg-Kohn-Sham density functional theory (DFT). Within the DFT, The Hohenberg-Kohn (HK) theorem~\cite{HK1964} states that the ground-state electron density unambiguously defines the many-electron ground state for an $N-$electron system under the influence of an external potential. And there exists an energy functional that guarantees the variational principle can reach the ground state energy, though the exact functional is an unsolved problem,
\begin{equation}
    E[\rho]=T+V_H(\br)+E_{xc}[\rho]+\int V(\br)\rho(\br)d\br.
\end{equation}
Where $V_H[\rho]=\int \frac{\rho(\br')}{|\br-\br' |}d\br'$ is the Hartree potential, $ E_{xc}$ is the exchange-correlation functional, which contributes $<10\%$ to the total energy, but describes the most critical correlation effects~\cite{Parr1995annuphys}. But DFT is a ground-state theory. Light-matter interactions usually result in many elementary excitations within the matter. Among various many-body methods, TDDFT (the formal extension of DFT theory) is the preferred tool to evaluate the excited states and optical properties in extended molecular or condensed matter systems. 

In the linear response regime, the induced electron density due to the light-matter interaction $\delta V(\br)$ is given by 
\begin{equation}\label{eq:rho1}
    \delta\rho(\br) = \int \chi(\br,\br')\delta V_{ext}(\br')d\br'.
\end{equation}
Where $\chi(\br,\br')$ is the polarizability describing the density response of the many-body ground state with respect to an external perturbation. In addition, according to Runge-Gross theorem~\cite{RG1984}, $\delta\rho(\br)$ is also the induced density of the KS system, but due to a perturbation
\begin{equation}\label{eq:rho2}
    \delta\rho(\br)=\int\chi_{KS}(\br,\br')\delta V_{KS}(\br')d\br'.
\end{equation}
$\chi_{KS}$ here is the linear response of the KS electrons, which can be trivially evaluated from the KS orbitals and energies.   

\begin{equation}\label{eq:vks}
    \delta V_{KS}(\br) = \delta V_{ext}(\br) + V_H[\delta\rho(\br)] + \delta V_{xc}(\br).
\end{equation}
The induced XC potential is given by $\delta V_{xc}(\br)=\int f_{xc}(\br,\br')\delta\rho(\br')d\br'$, where $f_{xc}$ is the dynamical XC kernel $\frac{\delta V_{xc}(\br)}{\delta n(\br')}$~\cite{Casida:2012ti},
\begin{equation}\label{eq:tddftkernel}
f_{xc}(\br,\br')=\frac{\delta V_{xc}(\br)}{\delta\rho(\br')},
\end{equation}
Hence, a linear equation for the induced density can be derived from Equations~\ref{eq:rho1}-\ref{eq:tddftkernel},
\begin{equation}\label{eq:LRTDDFT}
    [1-\chi_{KS}(\omega) f_{hxc}(\omega)]\chi(\omega)\delta V_{ext}(\omega)=\chi_{KS}(\omega)\delta V_{ext}(\omega).
\end{equation}
where $f_{hxc}(\br,\br')=\frac{1}{|\br-\br'|}+f_{XC}(\br,\br')$. Casting Equation~\ref{eq:LRTDDFT} into the matrix of coupled KS single excitations, it is possible to calculate the excitation energies $\omega_s$ of the system (the poles of the response function) and transition densities (and oscillator strengths). Consequently, the response function $\chi(\omega)$ can be rewritten as
\begin{equation}\label{eq:chi}
\chi(\br,\br',\omega)=2\sum_s\rho_s(\br)\rho_s(\br')\zeta_s(\omega),
\end{equation}
where $\zeta_s(\omega)=\frac{1}{\omega-\omega_s+i\eta}-\frac{1}{\omega+\omega_s-i\eta}$ and $\eta$ represents a positive inifinitesimal. The factor of 2 comes from summation over spin indices. The excitation energies in the system are denoted as $\omega_s$, which are calculated from the Casida method~\cite{Casida:2012ti}. 
$\rho_s(\br)$ is the transition density, which can be expanded in terms of electronic transitions between occupied state $i$ to unoccupied states $a$~\cite{PRB73205334}, 
\begin{equation}
\rho_s(\br)=\sum_{ia}X^s_{ia}\psi_i(\br)\psi_a(\br)\left(\frac{\epsilon_a-\epsilon_i}{\omega_s}\right)^{1/2}, 
\end{equation}
where $\epsilon_i$ are the KS eigenvalues, and the corresponding molecular orbitals are $\psi_i$. And $X^s_{ia}$ are the Casida transition coefficient from the occupied $i$ state to the unoccupied $a$ state of the $s^{th}$ excitation. By examining the nature of $X^s_{ia}$, it is possible to distinguish between normal excitation and plasmonic excitation. Plasmonic excitations generally are characterized by collective transitions from occupied to virtual orbitals~\cite{Castellanos, wu2022jcp_relaxation}.

\subsection{Hot electron generation and relaxation}
Once the plasmon excitation is obtained, the electron-plasmon interaction can be described by the Hamiltonian 
\begin{equation}
\hH_{int}=\frac{e}{2m_e}\int d\br \hat{\Psi}^\dag V_{\textit{eff}}(\br) \hat{\Psi},    
\end{equation}
where $V_{\textit{eff}}(\br)$ is the effective potential induced by the excitation, which can be calculated from the transition density $\rho_s(\br)$ by following Lundquvist's approach~\cite{Castellanos, PRB73205334}. The polarizability of the NP upon external excitation is given by Equation~\ref{eq:chi}. And the induced potential is given by~\cite{PRB73205334}, 
\begin{equation}
    V_{\textit{eff}}(\br)=\epsilon^{-1}(\br,\br')\delta V_{ext}(\br'),
\end{equation}
where the dielectric function $\epsilon^{-1}(\br,\br')$ is given by
\begin{equation}
\epsilon^{-1}(\br,\br')=\delta(\br,\br')+\int d\br_1 f_{hxc}(\br,\br_1)\chi(\br_1,\br',\omega).
\end{equation}
Thus, within the second quantization, electron-excitation Hamiltonian $\hH_{eps}$ reads
\begin{equation}\label{eq:coupling}
\hH_{eps}=\sum_{ij}\left[\MM^s_{ij} \hc^\dag_{i}\hc_{j} \hb_s+\text{h.c.}\right],
\end{equation}
where $\MM^s_{ij}$ is the electron-excitation coupling strength, describing the scattering of quasiparticles from state $i$ into state $j$ via the emission or absorption of excitation in the state $s$. These elements are given by (see SI for details)
\begin{equation}\label{eq:eplasmoncoup}
\MM^s_{ij}=  \zeta_s(\omega)\bE\cdot \bbf^s\langle\psi_i(\br_1)|V^s_H(\br_1)|\psi_j(\br_1)\rangle
%\int d\br \psi_i(\br)\psi_j(\br)V^s_H(\br) 
\end{equation}
where $\bbf^s=\sqrt{\frac{2m_e\omega_s}{2\hbar^2}}\sum_{ia}X^s_{ia}\left(\frac{\epsilon_a-\epsilon_i}{\omega_s}\right)^{1/2}\bd_{ia}$ is the oscillator vector and $|\bbf^s|^2=\frac{2m_e}{3\hbar^2}\omega_s |\langle\Psi_0|\br|\Psi_s\rangle|^2$ is the oscillator strength.
$V^s_H(\br)=\int d\br'\frac{\rho_s(\br')}{|\br-\br'|}$ is the Hartree potential induced by the transition density.
Detailed derivation can be found in the SI. Above equation actually includes both plasmon excitation ($\zeta_s(\omega)\bE\cdot \bbf^s$), screening effect and hot electron-hole pairs generation ($\langle\psi_i(\br_1)|V^s_H(\br_1)|\psi_j(\br_1)\rangle $). Neglecting the screening effect would result in a much larger electron-plasmon coupling matrix and shorter HC lifetime distribution~\cite{acsph7b00881}. In contrast, our model connects the widely used semiempirical model and the recently developed quantum model with the plasmon excitation and HC generation treated on equal footing. Equation~\ref{eq:coupling} is the general formalism that describes the coupling between electrons and photoexcitation. If the external field matches the plasmon energy, plasmon resonance will be excited, and Equation~\ref{eq:coupling} reduces to the electron-plasmon coupling. 

{\bf HC generation.}
The electron-plasmon coupling describes the HC generation following plasmon decay. After the electron-plasmon coupling matrix is obtained, the HC generation can be readily calculated from the Fermi golden rule~\cite{nn502445f,Zhang2021JPCA},
\begin{equation}\label{eq:hcgen}
\Gamma^{ex}_{i\rightarrow a}=\frac{4}{\hbar}\sum_s|\MM^s_{ia}|^2 \frac{\gamma_{ex}}{(\epsilon_i-\epsilon_a+\omega_s)^2+\gamma^2_{ex}}\delta(\omega-\omega_s).
\end{equation}
$\gamma_{ex}$ refers to the linewidth of the excitation. $\omega$ is the excitation energy. $\delta(\omega-\omega_s)$ describes the generalized photoexcitation. Hence, Equation~\ref{eq:hcgen} describes the photoexcitation and HC generation. When the energy of the external field $\omega$ matches the plasmon energy, Equation~\ref{eq:hcgen} describes the HC generation from plasmon decay. Otherwise, it describes the HC generation from regular excitation.


{\bf HC relaxation.}
After generation from plasmon decay, the HCs will undergo a relaxation process mainly due to the electron-electron and electron-phonon scatterings. The dynamics of the HCs are investigated by propagating the density matrix $\rho(t)$. The equation of motion (EOM) of $\rho$ is subject to the following quantum Liouville Von-Neumann equation~\cite{breuer2002theory},
\begin{equation}\label{mastereom}
i\partial_t\rho(t)=i\partial_t\rho(t)|_{\text{coh}}+\ML_{ee}[t]+\ML_{ph}[t]+\ML_{s}[t].
\end{equation}
The above equation's first part on the right-hand side (RHS) describes the coherent evolution of the density matrix between different states. The second and third parts on the RHS of Equation~\ref{mastereom} represent the dissipations induced by the electron-electron and electron-phonon scattering effects, respectively. Finally, the last term on the RHS of Equation~\ref{mastereom} describes the extraction of HCs. In general, in the presence of electron-electron and electron-phonon interaction, the dissipation can be described by a Liouville equation by employing the many-body perturbation theory (MBPT)~\cite{haug1998}, $\ML(t)=\mq(t)-\mq^\dag(t)$. Where the dissipation matrix $\mq_t$ can be written in terms of Green's functions $G$ and self-energies $\Sigma$~\cite{haug1998,Zhang2021JPCA}, i.e., 
\begin{equation}
\mq(t)=i\int^t dt'_{-\infty}\left[\Sigma^<(t,t')G^>(t',t) - \Sigma^>(t,t)G^<(t',t)\right].
\end{equation}
Hence, the key point is to develop approximations to the self-energies $\Sigma^{<,>}(t,t')$ and efficient numerical methods to compute the $\mq(t)$. The detailed formalisms of $\mq(t)$ for electron-electron and electron-phonon scatterings can be found in Ref.~\cite{Zhang2021JPCA}. But it should be noted that the approximation used for electron-electron scattering should conserve both particles and energy. 


\subsection{Plasmon-mediated chemical reactivities and theoretical challenges}\label{sec:plasmonchemdetails}

\begin{figure}[!htb]
    \centering
    \includegraphics[width=0.5\textwidth]{figures/plasmonchem.png}
    \caption{Proposed mechanisms of plasmon-mediated reactivities. 1) Enhanced intramolecular excitation, 2) direct charge transfer, 3) indirect charge transfer, and 4) local heating. 
    }
    \label{fig:plasmonchem}
\end{figure}

However, the computations of plasmon excitation and distribution of hot electrons (from plasmon decay) are insufficient to give insights into plasmon-mediated chemistry because plasmon-induced photocatalysis is a complex dynamical process that involves multiple interactions and mechanisms. To date, four major microscopic mechanisms have been proposed to explain how plasmonics facilitates various chemical reactions through the concentration of light and hot-carrier dynamics (Figure.~\ref{fig:plasmonchem}). The first mechanism involves the extreme concentration of light, which significantly promotes the excitation of electrons within the adsorbed molecule. Such an excitation takes place within the adsorbate only but is significantly enhanced by plasmonics, which is referred to as the enhanced intramolecular excitation mechanism~\cite{Tesema2019jpcc, Kazuma2018science, Tesema2017jpcc}. The second mechanism involves the transfer of plasmonic hot carriers. Due to the hybridization of the adsorbate–nanoparticle system, the generated hot electrons (or holes) can transfer to the absorbed molecules. Such transfers result in the placement of an adsorbate–nanoparticle system onto a manifold of excited potential energy surfaces (PESs), where the adsorbed molecule experiences strong forces that activate its internal vibrational excitations and, ultimately, chemical transformation. This mechanism is attributed to an indirect HE transfer from metal nanoparticles to the adsorbed molecule~\cite{ZhangYuchao2018chemrev, Kazuma2019angew}. Third, due to the hybridization, direct excitation of electrons from metal states near the Fermi level to the unoccupied molecular orbital (LUMO) of the adsorbed molecule can occur when the plasmon frequency is resonant with the excitation energy between the metallic occupied orbitals and hydrated metal-molecular orbitals. Such a reaction pathway is referred to as the direct charge transfer (CT) mechanism. Such excitation circumvents the thermalization of HEs~\cite{Boergter2016natcomm, Boerigter2016acsnano, Longrun2014jacs, Long2012jacs}, but requires matching between the plasmon energy and the energy gap between the metal states and the molecular LUMO. Finally, local heating resulting from hot-carrier relaxation can thermally activate a reaction~\cite{Bora2016scirep, Adleman2009nl, Sivan2019farady}. 

These mechanisms share some common features that render their clear distinction very challenging. Indeed, different mechanisms have been proposed even for the same chemical reaction, leaving a very confusing situation. For example, The chemical decomposition of methylene blue molecules on the surface of plasmonic NPs was reported by different research groups~\cite{Fusco:2022jmcc, Boergter2016natcomm, Boerigter2016acsnano, Chen:2012ur, Tesema2017jpcc}, direct hot electron transfer mechanism was proposed in Ref.~\citenum{Boergter2016natcomm, Boerigter2016acsnano} and indirect hot electron transfer is proposed in Ref.~\citenum{Chen:2012ur}. The same situation also applies to the plasmon-mediated reduction of PNTP, where both the nonthermal and local heating effects were proposed~\cite{Golubev:2018vr, Keller:2018tl}.
Hence, in order to obtain a comprehensive understanding of plasmon-mediated chemistry, it's essential for the theoretical methods to fulfill the following requirements:
\begin{enumerate}
    \item Efficient computation of a dense manifold of excited states because plasmonic excitations in metallic nanostructures are usually not low-lying states. Even for a small nanoparticle (2~nm diameter, for example), thousands of excited states are to be computed to reach the plasmon excitations~\cite{wu2023chemsci}.
    
    \item The ability to capture the nonadiabatic transition between excited states since the indirect transfer mechanism involves the transition between the hot electron state (excitations localized within the plasmonic system only) and the charge transfer state~\cite{wu2020mechanistic}.
    
    \item Inclusion of electron-vibrational couplings that lead to the hot electron relaxation and local heating.

    \item Inclusion of electron-electron scattering that leads to the redistribution of hot electrons~\cite{Brown2017prl, acsnano5b06199}.
\end{enumerate}
In particular, recent debates on the thermal impact underpin the necessity of atomistic and dynamical insights via nonadiabatic simulations of the HE generation, transfer, and relaxation processes on equal footing.  

\subsection{Nonadiabatic simulation of plasmon-mediated chemical reactivities}

{\bf Ehrenfest dynamics and Surface Hopping.}
With the Born-Oppenhermier approximation, the electronic wave function $\Theta(\mathbf{r},\mathbf{R})$ is expanded on the basis of adiabatic BO states, which depend on the electronic coordinates $\mathbf{r}$ and the nuclear coordinates $\mathbf{R}(t)$ according to
\begin{equation}
    \Theta(\textbf{r},\textbf{R}) = \sum_{n=1}^{N_{st}} c_n(t)\ket{\phi_{n}(\textbf{r},\textbf{R}(t))}. \label{eq:1}
\end{equation}
Here $N_{st}$ is the total number of adiabatic electronic states, $\ket{\phi_{n}(\textbf{r},\textbf{R}(t))}$ is the adiabatic electronic wavefunction of state n and $c_n(t)$ are the time-dependent complex expansion coefficients. $\textbf{R}(t) = \left\{\textbf{R}_{A}(t)\right\}_{A=1}^{A=N_{A}}$ ($N_A$ is the total number of atoms in the system) are the nuclear trajectories which are obtained by solving the classical Newton's equations of motion (EOMs) with the mixed quantum-classical framework~\cite{Nelson2020ChemRev},
\begin{equation}
    M_{A}\frac{d^2\textbf{R}_{A}}{dt^2} = -\nabla_{\textbf{R}_{A}}E(\textbf{R}), \label{eq:2}
\end{equation}
where $M_A$ is the mass of $A^{th}$ atom. $E(\textbf{R})$ is the potential energy surface (PES), which can be an averaged one $E(\bR)=\sum_n C_n(t) E_n(\bR)$ within the mean-field Ehrenfest dynamics or a single PES of a certain state $E_k(\bR)$ within the Surface hopping (SH) framework~\cite{tully1990molecular, Tully2012jcp, Nelson2020ChemRev}. 

By substituting Eq.~\ref{eq:1} into the time-dependent Schr\" {o}dinger equation and keeping only the first-order nonadiabatic coupling terms,
a set of EOMs for the coefficients $c_n(t)$ along a given classical trajectory can be obtained~\cite{Nelson2020ChemRev,curchod2013trajectory}
\begin{equation}
    i\hbar\frac{\partial c_{n}(t)}{\partial t} = c_{n}(t)E_{n}(\textbf{R}) - i\hbar\sum_{m}c_{m}(t)\dot{\textbf{R}}\cdot \textbf{d}_{nm}. \label{eq:3}
\end{equation}
Here the orthogonal condition of adiabatic states $\langle \Psi_n |\Psi_m \rangle = \delta_{nm}$ is used. $\mathbf{d}_{nm} = \langle \Psi_n|\nabla_{\textbf{R}}|\Psi_{m} \rangle$ is the nonadiabatic derivative coupling term (or nonadiabatic coupling vector, NACR).
A key variable in Eq.~\ref{eq:3} is the time-derivative nonadiabatic coupling scalar (NACT) between two adiabatic states
\begin{equation}\label{eq:4}
    \dot{\textbf{R}}\cdot \textbf{d}_{nm} = \langle \Psi_{n}|\frac{\partial}{\partial t}|\Psi_{m}\rangle, 
\end{equation}
which is responsible for the nonadiabatic transitions between different adiabatic states and can be easily calculated with many ab initio methods for excited states.

With the SH algorithm, the state that governs the dynamics of nuclei is determined by a stochastic process. The time-dependent coefficients $c_{n}(t)$ obtained in Eq.~\ref{eq:3} are used to calculate the hopping probabilities between different electronic excited states within the framework of the FSSH algorithm. The hopping probabilities between excited states $n$ and $m$ are given by\cite{malone2020nexmd}
\begin{equation}\label{eq:7}
    g_{n \rightarrow m}(\textbf{R},t) = \frac{\int_{t}^{t+N_{q}\delta t}dt\  b_{mn}(\textbf{R},t)}{a_{nn}(t)},
\end{equation}
where $N_q=\frac{\Delta t}{\delta t}$, with $\Delta t$ and $\delta t$ correspond to the time steps for evolving motions of nuclei and electrons in Eq.~\ref{eq:2} and Eq.~\ref{eq:3}, respectively. The chosen value of $\delta t$ must be small enough to resolve strongly localized peaks in NACT in order to avoid underestimation of transition probabilities. This is particularly important when crossings between adiabatic states are encountered in the trajectory, in which $\delta t$ will be further refined. $a_{nn}(t) = c_n(t)c_n^*(t)$ defines the time-dependent density matrix elements, and $b_{mn}(\textbf{R},t) = -2Re(a_{nm}^*\ \dot{\textbf{R}}\cdot \textbf{d}_{nm})$. Note that $g_{n\rightarrow m} = -g_{m \rightarrow n}$ and $g_{n\rightarrow n} = 0$ since $\textbf{d}_{nm}$ are antisymmetric. Hopping between adiabatic states is determined stochastically by comparing $g_{n\rightarrow m}$ to a random number $\xi (\xi \in (0,1))$. A hop from state $n$ to state $m$ is performed if
\begin{equation}
  \sum_{l=1}^{m-1} g_{n \rightarrow l} < \xi \leq \sum_{l=1}^{m} g_{n \rightarrow l},
    \label{eq:8}
\end{equation}
where states are assumed to be ordered with increasing transition energy. On the other hand, the system remains in state n when $\sum_{l=1}^{N_{st}} g_{n \rightarrow l} < \xi < 1$. If $g_{n\rightarrow m} < 0$, the hop is unphysical, and the probability is set to zero. Finally, if a hop to a higher energy state is predicted, there must be sufficient nuclear kinetic energy along the direction of NACR. Otherwise, the hop is rejected. After a successful hop, the total electron-nuclear energy is conserved by rescaling the nuclear velocity in the direction of the NACR according to the procedure described in reference~\cite{Tully2012jcp, Fabiano2008chemphys}. In addition, during the dynamics, we monitor the relative phase of the ground to excited state transitions and maintain the same phase (sign) to avoid a sudden sign change in the NACT. This is done by enforcing the sign of the largest component of the Casida eigenvectors to the same along the trajectory.

Despite broad popularity in the community, either Ehrenfest dynamics or the surface hopping (SH) approach have well-known limitations such as generating artificial electronic coherence or giving incorrect long-time population~\cite{Tully2012jcp}. To address these challenges and provide accurate dynamics, a multiconfigurational Ehrenfest (MCE) dynamics approach~\cite{makhov_MCEh_ChemPhys2017} and ab initio multiple cloning (AIMC)~\cite{makhov_AIMS_JCP2014,makhov_AIMS_PCCP2015,makhov_AIMS_FaraDisc2016,song2021ab} are developed accordingly. 

{\bf Multiconfigurational Ehrenfest (MCE).} MCE generalizes EHR formalism by representing the wave function as a linear combination of Ehrenfest configurations. Each MCE configuration moves along its own Ehrenfest (mean-field) trajectory. Within the MCE formalism, the molecular wavefunction $\ket{\Psi}$ is expressed in the trajectory-guided Gaussian basis functions (TBF) representation ($\ket{\psi_n}$),
\begin{equation}\label{eq_mcewf}
    \ket{\Psi(t)}=\sum c_n \ket{\psi_n(t)}.
\end{equation}
And each configuration (or TBF) is described by the product of nuclear and electronic parts,
\begin{equation}
    \ket{\psi_n(t)} = \ket{\chi_n(\bR,t)} \sum_I a^{n}_I\ket{\phi^n_I(\br,\bR(t))}.
\end{equation}
Where $\ket{\phi^n_I(\br,\bR(t))}$ is the adiabatic state of configuration $n$. $\ket{\chi_n}$ are Gaussian nuclear basis functions
\begin{equation}
    \ket{\chi_n}=\left(\frac{2\alpha}{\pi}\right)^{N_d/4}e^{\left\{-\alpha(R-\bar{R})+\frac{i}{\hbar}P(R-\bar{R})+\frac{i}{\hbar}\gamma_n(t)\right\}}.
\end{equation}

The couplings between TBFs in the MCE approach are described by the EOM of $c_n(t)$, which can be readily obtained by substituting Eq.~\ref{eq_mcewf} into the Schr\ "odinger equation:
\begin{equation}
    i\hbar\sum_n S_{mn} \dot{c}_n = 
    \sum_n \left[H_{mn} - i\hbar\bra{\psi_m}\frac{d\psi_m}{dt}\rangle \right]c_n.
\end{equation}
where 
\begin{equation}\label{eq_hmn}
    H_{mn}=\sum_{I,J}(a^m_I)^*a^n_J
\bra{\chi_m\phi^m_I}T+V\ket{\chi_n\phi^n_J},
\end{equation}
and the overlap $S_{mn}$ is
\begin{equation}
    S_{mn}=\bra{\psi_m}\psi_n\rangle
    =\langle{\chi_m}\ket{\chi_n}\sum_{I,J}(a^m_I)^*a^n_J
    \langle{\phi^m_I}\ket{\phi^n_J}.
\end{equation}
The nuclear part of Eq.~\ref{eq_hmn} can be obtained analytically,
\begin{equation}
     \bra{\chi_m\phi^m_I}V\ket{\chi_n\phi^n_J}=
     \bra{\chi_m}-\frac{\hbar^2}{2}\nabla_{\bR}M^{-1}\nabla_{\bR}
     \ket{\chi_n}\langle\phi^m_I\ket{\phi^n_J}.
\end{equation}
While the electronic part (or the potential energy matrix elements) are approximated by~\cite{song2021ab}
\begin{align}
& \bra{\chi_m\phi^m_I}V\ket{\chi_n\phi^n_J}=\frac{1}{2}
 \bra{\phi^m_I}\phi^n_J\rangle\bra{\chi_m}\chi_n\rangle\times \nonumber\\
 &\Big\{ (V^m_I+V^n_J)+\frac{i}{4\alpha\hbar}(\bP_n-\hP_m)\cdot(\nabla_{\bR}V^m_I+\nabla_{\bR}V^n_J)\nonumber\\
  &-\frac{1}{2}(\bR_m-\bR_n)\cdot(\nabla_{\bR}V^m_I-\nabla_{\bR}V^n_J)\Big\}.
\end{align}
As shown above, within the MCE schemes, the electronic states are now different for different configurations. The overlaps between TBFs have to be calculated and taken into account. 


{\bf Ab initio multiple cloning (AIMC).} The AIMC method combines the best features of ab initio Multiple Spawning (AIMS)~\cite{bennun_AIMS_JPCA2000} and Multiconfigurational Ehrenfest (MCE) methods. Similar to the MCE method, the individual trajectory basis functions (TBFs) of AIMC follow Ehrenfest equations of motion. However, the basis set is expanded in a similar manner to AIMS when these TBFs become sufficiently mixed. Consequently, AIMC avoids prolonged evolution on the mean-field potential energy surface (PES).

Within the MCE formalism, the Ehrenfest basis set is guided by an average potential, which is accurate for dynamical processes where the coupling between states persists in time between nearly parallel PES. But the mean-field treatment can be unphysical when the PES of two or more populated electronic states become different in shape, which leads to wave packet branching after leaving the nonadiabatic coupling region. To deal with these cases, the AIMC algorithm is applied to expand the original basis set of TBFs by "cloning" one TBF into two copies in a way that does not alter the original wave function~\cite{makhov_AIMS_FaraDisc2016,makhov_AIMS_JCP2014,makhov_AIMS_PCCP2015} This is done by creating one of the clones $\ket{\psi_{n_1}}$ in a pure state and the other clone $\ket{\psi_{n_2}}$, which includes contributions from all other electronic states. The corresponding MCE amplitudes $\{c_{n_1}, c_{n_2}\}$ are adjusted to conserve the original wavefnction~\cite{song2021ab, makhov_AIMS_FaraDisc2016}. 

It should be noted the MQC method can be complemented with any electronic structure solvers as long as the gradients and NACs are available. Practical applications need to balance accuracy and numerical efficiency. 

\begin{figure}[!htb]
    \centering
    \includegraphics[width=0.5\textwidth]{figures/jellium.png}
    \caption{(a) Schematic diagram of H$_2$ molecule adsorbed on the NP surface. b)-c) Spatial distribution of the bonding and antibonding orbitals of H$_2$ on Jellium NP. d) Local DOS of bonding and antibonding orbitals of H$_2$ modules as a function of H-H bond length. e) Depending on the symmetry, the H$_2$ dissociation can be suppressed or restored in the plasmonic dimer. 
    Figures are adapted with permission from Ref.~\citenum{zhang2018plasmonic}. Copyright 2018 American Chemical Society. 
    }
    \label{fig:jellium}
\end{figure}

\subsection{Simulation of plasmon-mediated phenomena}
\subsubsection{Jellium model-based NAMD simulation} 
Compared to the costly atomistic ab initio calculations for plasmonic nanoparticles, the optical absorption of simple sp-metal nanoparticles described by the Jellium model can be easily calculated and analyzed~\cite{Besterio2017acsphotonics, Chang2019acs, Varas:2016tf, Yan:2016uz, nn502445f, zhang2018plasmonic}. In general, good agreement with experimental optical properties can be achieved by tuning the Jellium radius. Using the Jellium model, we investigated the generation and relaxation of plasmonic hot carriers~\cite{Zhang2021JPCA}, as well as the atomic-scale mechanism of plasmonic hot-carrier-mediated chemical processes such as H$_2$ dissociation~\cite{zhang2018plasmonic}. 
Our numerical simulations showed that after photoexcitation, hot carriers transfer to the antibonding state of the H$_2$ molecule from the nanoparticle, leading to a repulsive-potential-energy surface and H$_2$ dissociation (Figure~\ref{fig: Jellium}(b-d)). This process occurs when the molecule is close to a single nanoparticle. However, in a plasmonic dimer, dissociation can be inhibited due to sequential charge transfer that effectively reduces the occupation of the antibonding state, as shown in Figure~\ref{fig:jellium}(e). When the molecule is asymmetrically positioned in the gap, the symmetry is broken, and dissociation is restored by significantly suppressing additional charge transfer. Thus, these models illustrate the potential for structurally adjustable photochemistry through plasmonic hot carriers.

\subsubsection{TDDFT calculations.} 
However, the Jellium model oversimplifies the electronic structures of the plasmonic nanostructures, and it lacks atomistic details. Insights from PESs built on the ab initio atomistic model are essential for a more in-depth understanding of plasmon chemistry. Thus, we employed the linear response TDDFT (LR-TDDFT) calculations within the Casida formalism to compute the adiabatic PESs of the H$_2$ molecule adsorbed on an Au6 cluster (H2@Au6)~\cite{wu2020mechanistic}, in order to explore key pathways in LSPR-promoted chemical reactions. Despite the model system being too small to support plasmonic mode and thus cannot describe the dephasing of plasmons that produce hot electrons, the key point of using the model system is to capture key aspects of the later stages of plasmon-facilitated photocatalysis, thus providing mechanistic insights.

Our findings based on DFT calculations indicate that in the ground state, the Au6 cluster supports the adsorption of the H$_2$ molecule at the tip site. We further use LR-TDDFT calculations within the Casida formalism to determine the adiabatic excited states and corresponding oscillator strengths. We calculate two-dimensional PESs in the desorption and dissociation reaction coordinates and observe that the adiabatic excited PESs bear similarity to the ground state PES in the Franck-Condon region. Therefore, the H$_2$ adsorbate is comparatively stable, as corroborated by the relatively low dissociation and desorption probabilities from our quantum dynamics simulations. By developing the orbital wave function overlapping (OWO) diabaization scheme, we are able to divide the dense manifold of the excited state into two groups: 1) one group is dominated by electronic excitations confined to the Au$_6$ cluster, which can be likened to HE states in metal nanoclusters; 2) the other group of excited states has the antibonding $\sigma^*$ characteristics of the H2 adsorbate due to the hybridization between H$_2$ and Au$_6$ cluster, which are denoted as CT states. The crossings among the HE and CT states provide pathways leading from the excited HE states to CT states via nonadiabatic transitions. Quantum dynamics simulations on the diabatic PESs demonstrate that the CT diabatic states are able to drive H$_2$ dissociation efficiently and thus are responsible for the experimentally observed HD formation on Au nanoparticles. Our results nevertheless give a clear physical picture of photoinduced H$_2$ dissociation on Au clusters. 

\begin{figure}[!htb]
    \centering
    \vspace{-10pt}
    \includegraphics[width=0.5\textwidth]{figures/plasmonchem_pathways.png}
    \caption{
    Schematic diagram of competition among different pathways in plasmon-mediated chemical reactions on a dense manifold of excited states. The excited states can be divided into HE and CT states. The CT states are responsible for the chemical reaction, which can be triggered by the nonadiabatic transition between HE and CT states. Such nonadiabatic transitions have to compete with hot carrier relaxations that lead to local heating.
    }
    \label{fig:pathway}
\end{figure}

The presence of HE-CT crossings is not unique to plasmonic catalysis. Such features have also been found in other nonplasmonic catalysis, such as those on the surface of semiconductors~\cite{akimov_theoretical_ChemRev2013}. However, plasmonic materials are unique in generating a high concentration of HE due to their larger absorption cross sections. The combination of a high concentration of HE and HE-CT crossings is thus the distinctive feature of plasmonic catalysis. 


\subsubsection{TDDFTB-based NAMD simulations}
As shown in Figure~\ref{fig:pathway}, plasmon-mediated chemistry involves a dense manifold of excited states, nonadiabatic transitions between HE and CT states, and their competition with various dissipation channels. Such complicated processes require nonadiabatic simulations that treat the HE generation, relaxation, and HE-CT transitions treated on equal footing. To this end, larger clusters should be used in modeling plasmonic catalysis, and more efficient semiempirical methods, such as those based on time-dependent density functional tight binding (TDDFTB)~\cite{trani_TDDFTB_JCTC2011}, might be needed for the efficient electronic structure calculations. To this aim, We recently developed an efficient NAMD method by combining the TSH algorithm and LR-TDDFTB method.~\cite{niehaus2001tight, malone2020nexmd, song2020first, wu2022nonadiabatic} LR-TDDFTB is the tight-binding extension of TDDFT,~\cite{hourahine2020dftb+, niehaus2001tight} which has been successfully applied to investigate the optical properties of plasmonic NPs.~\cite{douglas2019plasmon, berdakin2019interplay} It enables calculations of plasmonic excitation along with several hundred excitation states per time step. The relaxation processes of plasmon induced by electron-phonon interactions are treated by the TSH algorithm. With the NAMD-DFTB method, we demonstrate the plasmon relaxation of an Au$_{20}$ cluster. Our simulations show that Au$_{20}$ can support plasmon-like excitation, which includes the superposition of multiple single-particle excitation components. 

The numerically efficient LR-TDDFTB method allows us to address a dense manifold of excited states to ensure the inclusion of plasmon excitation. Starting from the photoexcited plasmon states in Au$_{20}$ cluster, we find that the time constant for relaxation from plasmon excited states to the lowest excited states is about 2.7~ps, mainly resulting from a step-wise decay process caused by low-frequency phonons of the Au$_{20}$ cluster. Furthermore, our simulations show that the lifetime of the phonon-induced plasmon dephasing is $\sim$10.4~fs, and such a swift process can be attributed to the strong nonadiabatic effect in small clusters. Our simulations demonstrate a detailed description of the dynamic processes in nanoclusters, including plasmon excitation, hot carrier generation from the plasmon excitation dephasing, and the subsequent phonon-induced relaxation process.

The NEXMD-DFTB method was employed to investigate the plasmon-induced bond activation of CO adsorbed on Au${20}$ cluster~\cite{wu2023chemsci}. The simulations provide a comprehensive and accurate depiction of the multiple dynamic processes involved in plasmon-mediated chemistry, including plasmon excitation, HEs relaxation, direct/indirect HE transfer, and the activation of CO vibration mode induced by HE transfers. These simulations reveal the critical role of charge transfer (CT) states in plasmon-induced CO activation. The CT states excite both direct and indirect HE transfer, which leads to the activation of CO stretching mode, an essential component of plasmon energy relaxation. Our simulations demonstrate the high efficiency of CO vibrational mode activation, achieving a success rate of approximately 40\%. Notably, the HE transfer occurs at a faster rate than the conventional scattering process of the Au${20}$, completing within approximately 100 fs, while the energy relaxation takes place over a timescale of approximately 1 ps. Furthermore, our direct atomistic simulations provide detailed insights into the potential energy evolution during the plasmon-mediated chemical transformation, thereby enabling a comprehensive understanding of the energy relaxation and HE transfers during the reaction, as well as elucidating the reaction pathways.

