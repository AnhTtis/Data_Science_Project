\section{Introduction}
In 2007, a report compiled for the Office of Science of the Department of Energy (DOE) identified five grand challenges in basic energy science~\cite{ratner2011physicstoday}, including 1) controlling material processes at the level of electrons, 2) designing and perfecting atom- and energy-efficient syntheses of new forms of matter with tailored properties, 3) understanding and controlling the remarkable properties of matter that emerge from complex correlations of atomic and electronic constituents, 4) mastering energy and information on the nanoscale to create new technologies with capabilities rivaling those of living things, and 5) characterizing and controlling matter away—especially far away—from equilibrium. One of the emerging techniques to address (some of) these challenges is light-matter interaction, which can be used to monitor, manipulate, and design materials' properties through multiple interactions between electrons, photons, and phonons~\cite{Anton2019NatRevPhys,Revera2020NatRevPhys}.

In conventional chemistry, chemists modify the functionalities of molecules based on different functional groups. For centuries, such a vision has been widely used in chemical synthesis to modify the chemical and physical properties of molecules or molecular materials. However, more is different~\cite{anderson1972more}, and chemistry in a complex electromagnetic (EM) environment has emergent properties due to multiple couplings that can be used to control chemistry. Light-matter interactions have been instrumental in many branches of physics, chemistry, materials, and energy science~\cite{novotny2006principles,Revera2020NatRevPhys,weiner2012light}. In most of the previous applications, light-matter interaction is within the weak coupling scenario and can usually be treated at the lowest order in quantum electrodynamics via many-body perturbation theory~\cite{mahan2000many}. Such treatments are widely used in different types of spectroscopy techniques~\cite{Stiles2008annurev,Mukamel2000annurev}, optoelectronics~\cite{zhang2014quantum,yam2015multiscale,meng2015multiscale,meng2017multiscale, wang2015quantum}, quantum sensing~\cite{Degen2017rmp}, quantum information~\cite{Flamini_2018}, light harvesting~\cite{zhou2018interlayer,Mirkovic2018cr}, and beyond.

Alternatively, light-matter interactions open multiple new avenues for manipulating matter through novel emerging elementary excitations~\cite{Forndaz:2019rmp}, including plasmons and polaritons. Plasmons (either surface plasmons or localized surface plasmons) are the collective oscillations of conduction electrons in nanostructures, which can be excited when the frequency of external light matches the plasmon resonant energies. Plasmon excitation results in significantly amplified absorption and scattering cross-sections. Moreover, plasmon excitations can overcome the diffraction limit and concentrate the incident light into a highly localized volume, enhancing the electromagnetic (EM) field (or photons) by several orders of magnitude in the near field. The subsequent plasmon decay process generates hot electrons or heat through electron-electron and electron-phonon scatterings at different length and time scales~\cite{Brongersma:2015vd}. Nevertheless, the resulting locally enhanced EM field, hot electrons, and heat can stimulate chemical reactions through various mechanisms~\cite{Zhan:2023tc, Kazuma2019angew}. In fact, plasmon-mediated chemical reactions (namely plasmon chemistry) have become a promising strategy to drive chemical processes over the past decade~\cite{Zhan:2023tc}.

On the other hand, when molecules are collectively and resonantly coupled with plasmon excitation, a new quasiparticle (namely polaritons) can be formed in the strong coupling regime. The term "strong" coupling is relative, meaning that the light-matter coupling is large enough to compete with or overcome dissipation or dephasing (i.e., when the coherent energy exchange between a confined light mode and quantum matter is faster than the decay and decoherence time scales of each part). Such strong coupling can be achieved when either a plasmonic mode is coupled with a few molecules~\cite{Jeremy2022nl} or many molecules are collectively coupled to a single cavity mode~\cite{Ribeiro2018CS}. In the strong coupling regime, photons and electronic/excitonic excitations in matter become equally important and are strongly coupled on an equal quantized footing. As a consequence, individual "free" particles no longer exist. Instead, the fundamental excitations of the light-matter interacting system are polaritons, which are hybrid light-matter excitations (superpositions of quantized light and matter) (Fig 1a)\cite{Revera2020NatRevPhys} and possess both light and matter characteristics/topologies. Experiments have shown that matter properties can be modified with the formed polaritons, resulting in different photophysics and photochemistry\cite{Ebbesen2016ACR}. Since photon energies and light-matter coupling strengths are relatively tunable through cavity control, light-matter interaction in the strong coupling regime provides a fundamentally new way to manipulate matter properties for various desired applications, including lasing~\cite{Kena-Cohen:2010ue,Kang:ws}, electronics~\cite{Orgiu:2015us}, long-range energy transfer~\cite{Zhong:2017vq, Georgiou2021angwew, Want2021natcomm, Coles:2014wm}, Bose-Einstein condensates~\cite{Dusel:2020wt, Zasedatelev:2019um, Kavokin:2022tn}, and chemical reactions~\cite{pavosevic_ClickChem_Arxiv2022, pavosevic_PTQED_JACS2022, pavosevic_PTQED_JACS2022, schafer_shining_NatComm2022, schafer_EmbeddingRadReaction_JPCL2022, Cave1997JCP, MartinezMartinez2017AP, Yang2021JPCL, Climent2020PCCP, Wang2022JPCL, Imperatore2021JCP, Galego2016NC, CamposGonzalezAngulo2019NC, Philbin2022JPCC, Efrima1974CPL, Phuc2020SP, Davidsson2020JCP, Galego2017PRL, Mauro2021PRB, Vurgaftman2020JPCL, Hiura2021C, Hiura2019-rate}. 
Nevertheless, in polariton chemistry, light and matter cannot be treated as separate entities, as the strong coupling between them dresses each of them. Consequently, previous quantum chemistry methods for treating the electronic structure of matter become invalid. Brand new theoretical and modeling capabilities for describing polariton chemistry are required.

\begin{figure}[!htb]
    \centering
    \includegraphics[width=0.4\textwidth]{figures/light_matter2.pdf}
    \caption{\label{fig:lm1}
    Light-matter interaction is a fundamentally multiscale/multiphysics problem. It involves multiple interactions among electrons, photons, and phonons at different time and length scales. As a result, multiple quasiparticles may be excited due to the light-matter interactions, including exciton (due to electron-hole interaction), (excitonic/vibrational) polariton (due to coupling between photon and exciton/phonon), polaron (due to electron-phonon coupling), and polaron-polaritons. In addition, all quantum systems are fundamentally open systems, leading to dissipation/dephasing due to the bath.
    }
\end{figure}

Despite the attractive applications of strong light-matter interaction, there are many open questions and fundamental problems about the mechanisms of physics and chemistry mediated by the strong light-matter interaction. The experimental progress should be complemented by theoretical advances. Progress in understanding the coupling between photons and elementary quasiparticles (plasmons, phonons, and excitons) in materials requires a generalized treatment of photons as one of the core degrees of freedom (DOFs) in light-matter interaction. Unfortunately, the strong light-matter interaction is fundamentally a multiscale and multiphysics problem that involves multiple interactions between many DOFs and their interplay with environments across different spatial and time scales. The traditional perturbation methods in the weak coupling regime are not applicable to the strong coupling regime, making it urgent to develop new theoretical and modeling techniques, especially multiscale methods, to understand and ultimately predict strong light-matter interaction-mediated physics and chemistry.

This perspective reviews recent theoretical efforts toward understanding the underlying mechanisms of plasmon and polariton chemistry due to the complex light-matter interactions and discusses our thoughts on future work. The article is structured as follows: Section~\ref{sec:basics} introduces the framework and basic mathematical structure of molecular quantum electrodynamics theory (QED), Section~\ref{SEC:PLASMON} explores the theoretical development in simulating plasmonic cavities and their strong interactions with molecules, Section~\ref{SEC:POLARITON} surveys recent progress in polariton chemistry in Fabry-Pérot-like cavities, and Section~\ref{SEC:CONCLUSION} concludes the discussion and provides a final perspective on future work in all realms of strong light-matter interaction.