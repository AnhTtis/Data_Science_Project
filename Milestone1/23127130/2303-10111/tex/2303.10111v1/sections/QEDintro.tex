%%%%%%%%%%%%%%%%%%%%%%%%%%%%%%%%%%%%%%%%%%%%%%%%%%%%%%%%%%%%%%%%%%%
\section{Brief introduction to light-matter interactions and molecular QED theory}\label{sec:basics}

This section briefly introduces the quantum theories of light-matter interactions in the nonrelativistic limit~\cite{craig1998molecular}. All light-matter interactions arise from the interplay of matter DOFs (degrees of freedom) (electrons or spins, nuclei) and an EM environment. Hence, a full ab initio theory for light-matter interactions fundamentally requires the electronic structure theory of matter and principles of electrodynamics. Quantum electrodynamics (QED) is the indispensable and most precise theory for describing the interactions of charged particles and the dynamics of the EM field in mutual interaction.

\subsection{Molecular quantum electrodynamics theory}
The minimally coupled Coulomb Hamiltonian governs the nonrelativistic dynamics of matter in an EM environment, 
\begin{equation}\label{EQ:QED_HAM_GENERAL}
    \hH=\sum^{N_e+N_n}_i \frac{1}{2m_i}\left[\hP_i - z_i \bA(\br_it)/c\right]^2 + \hat{V} + \hH_{EM},
\end{equation}
with the Coulomb gauge $\nabla\cdot\bA=0$. $\hP_i= -i\hbar\nabla_i$ and $z_i$ are the momentum operator and charge of particle $i$, respectively. $N_{e}$ and $N_n$ are the number of electrons and nuclei in the systems.
$\hat{V}=\hat{V}_{ee}+ \hat{V}_{nn} + \hat{V}_{en}+V_{ext}$ contains all of the Coulomb interactions ($\frac{1}{8\pi\epsilon_0}\frac{q_i q_j}{|\br_i-\br_j|}$) between the electronic and nuclear DOFs and another external potential. The Hamiltonian that describes the EM fields is
\begin{equation}\label{EQ:H_PH}
    \hH_{EM} = \frac{\epsilon_0}{2}\int d\br\left[
    \hat{\bE}^2(\br) + c^2\hat{\bB}^2(\br)\right].
\end{equation}
It obeys Maxwell's equations of motion and couples with the Schr\"odinger equation self-consistent through the vector potential $\bA(\br_i)$. 
With multiple modes available, each field (including the vector potential $\hat{\bA}$) can be expressed as a sum over all possible radiation modes~\cite{Archambault2010prb,andrew1989irpc}
\begin{align}
    \hat{\bA}(\br,t)=&\sum_\alpha \bu_\alpha \Big(\hat{A}_\alpha e^{i(\bK\cdot\br-\omega_\alpha t)} + \hat{A}^\dag_\alpha e^{i(\bK\cdot\br+\omega_\alpha t)}\Big)
    \\
    \hat{\bE}(\br,t)=&i\sum_\alpha \omega_\alpha \bu_\alpha \Big(\hat{A}_\alpha e^{i(\bK\cdot\br-\omega_\alpha t)} + \hat{A}^\dag_\alpha e^{i(\bK\cdot\br+\omega_\alpha t)}\Big)
    \\
    \hat{\bB}(\br,t)=&i\sum_\alpha (\nabla\times \bu_\alpha) \Big(\hat{A}_\alpha e^{i(\bK\cdot\br-\omega_\alpha t)} + \hat{A}^\dag_\alpha e^{i(\bK\cdot\br+\omega_\alpha t)}\Big)
\end{align}
where $\nabla\times \bu=\bK\times\bu$, $\alpha \equiv \lambda\bK$ denotes the photon mode with momentum $\bK$ and polarization $\lambda \in \{-1,1\}$, $\bu$ is the unit vector denoting the direction of the vector potential, and $\hat{A}_\alpha$ ($\hat{A}_\alpha^\dag$) is the mode decomposition coefficients that create (annihilate) the $\alpha^\mathrm{th}$ radiation mode. Note that $\hat{\bE}=-\frac{1}{c}\partial_t \hat{\bA}(\br)$ and $\hat{\bB}=\frac{1}{c}\nabla\times \hat{\bA}(\br)$ in the Coulomb gauge (\textit{i.e.}, $\nabla\cdot\hat{\bA}=0$). The photon modes in a given nanophotonic/nanoplasmonic structure can be readily calculated via standard mode decomposition techniques~\cite{SnyderLove:1983}. 


In the second quantization, the photonic Hamiltonian can then be written as,
\[
\hH_p=\sum_\alpha \frac{\hbar\omega}{2}\left[\ha_\alpha\ha^\dag_\alpha+h.c.\right]=\sum_\alpha\hbar\omega(\ha^\dag_\alpha\ha_\alpha+1/2).
\]
with the equivalence,
\begin{align}
    A_\alpha \equiv \sqrt{\frac{\hbar}{2\epsilon_0\omega V}} \ha_\alpha,
    \ \ \ \ \ \
    A^*_\alpha \equiv \sqrt{\frac{\hbar}{2\epsilon_0\omega V}} \ha^\dag_\alpha,
\end{align}
Thus, the fields are quantized by the isomorphic association of a quantum mechanical harmonic oscillator to each radiation mode $\alpha$. Introducing the canonical position and momentum operators as, $\hq_\alpha=\sqrt{\frac{\hbar}{2\omega_\alpha}}(\ha^\dag_\alpha+\ha_\alpha)$, $\hp_\alpha=i\sqrt{\frac{\hbar\omega_\alpha}{2}}(\ha^\dag_\alpha-\ha_\alpha)$. The photonic Hamiltonian can now be rewritten as,
\begin{equation}
    \hH_p = \frac{1}{2}\sum_\alpha (\hp^2_\alpha +\omega^2\hq^2_\alpha). 
\end{equation}
Then the full light-matter Hamiltonian in Coulomb gauge can now be fully constructed and written as,
\begin{equation}\label{EQ:H_QED_MinCoup}
    \hH_c = \sum^{N_e+N_n}_i \frac{1}{2m_i}\left(\hP_i-z_i\hat{\bA}(\br_i)\right)^2
    +\hat{V} + \hH_p,
\end{equation}
which is often referred to as the minimal coupling Hamiltonian.

We now aim to apply a unitary transformation on Eq.~\ref{EQ:H_QED_MinCoup} to achieve an expression where the momenta of the molecular DOFs ($\hp_i$) are decoupled from the vector potential ($\hat{\bA}(\br_i)$). In other words, we will shift the light-matter coupling from momentum fluctuations into displacement fluctuations~\cite {}. This transformation, referred to as the PZW transformation, can be written as $\hat{U}_\mathrm{PZW} = exp[ -\frac{i}{\hbar}\hat{\bD}\cdot\hat{\bA} ]$ where $\hat{\bD}=\sum_i^{N_n} z_i\hat{\bR}_i -\sum_i^{N_e} e\hat{\br}_i$ is the molecular dipole moment. The PZW transformation $\hU(\bd)\hH_c\hU^\dag(\bd)$ is nothing but a reduction in matter momentum such that $\hp_i-z_i\bA\rightarrow \hp_i$ and a boost in photonic momentum by $\hp_\alpha \rightarrow\hp_\alpha+\sqrt{2\omega/\hbar}\bd\cdot\bA_0$. Applying PZW transformation results in the QED Hamiltonian, referred to as the dipole gauge Hamiltonian,
\begin{equation}
    \hH=\hT + \hV + \hH_{ep}
\end{equation}
where 
\begin{align}\label{EQ:H_PH_SHIFTED}
\hH_{ep}=\hat{U}_\mathrm{PZW}^\dag\hH_p\hat{U}_\mathrm{PZW}&=\frac{1}{2}\sum_\alpha \left[(\hp_\alpha+\blambda_\alpha\cdot\bD)^2 + \omega^2_\alpha(\hq_\alpha)^2\right] \\
    &=\frac{1}{2}\sum_\alpha \left[\hp^2_\alpha + \omega^2_\alpha(\hq_\alpha-\frac{\blambda_\alpha}{\omega_\alpha}\cdot\bD)^2\right].\nonumber
\end{align}
The second line is reached via canonical transformation between coordinate and momentum operators, $\hp \rightarrow -\omega \hq,   \hq\rightarrow 1/\omega \hp$.
Here, $\blambda_\alpha=\sqrt{\frac{1}{\epsilon V}}\bu_\alpha$ and $g_\alpha\equiv\blambda_\alpha\cdot\bD$ defines the light-matter coupling strength, which depends on the volume of quantized photon $V$ in radiation mode $\alpha$. The total light-matter Hamiltonian can now be \textit{re-partitioned} after the unitary transformation as,
\begin{align}\label{EQ:H_PF}
    \hH_\mathrm{PF} &= \hH_\mathrm{M} + \hH_\mathrm{p} + \hH_\mathrm{ep} + \hH_\mathrm{DSE} \\ &= \hH_\mathrm{M} + \sum_{\alpha} \Big[ \omega_\alpha (\ha^\dag_\alpha\ha_\alpha+\frac{1}{2}) + \sqrt{\frac{\omega_\alpha}{2}} \blambda_\alpha \cdot \hat{\bD} (\ha^\dag_\alpha + \ha_\alpha) + \frac{1}{2} (\blambda_\alpha \cdot \hat{\bD})^2 \Big].
\end{align} 
Here, $\hat{H}_\mathrm{M} = \hat{T}_\mathrm{n} + \hat{T}_\mathrm{e} + \hat{V}$ is the bare molecular Hamiltonian which includes all Coulomb interactions $\hat{V}$ between electrons and nuclei as well as the kinetic energy operators of both, $\hat{T}_\mathrm{e}$ and $\hat{T}_\mathrm{n}$, respectively. This Hamiltonian is often referred to as the Pauli-Fierz (PF) Hamiltonian.

Now it is clear that the light-matter coupling strength is determined by the two quantities: the molecular dipole strength and the quantized cavity volume. Hence, there are two general strategies to enter the strong coupling regime: (I) reduce the cavity volume and (II) increase the number of molecules (increasing the total dipole moment). Therefore, there are two major experimental nanocavity designs for strong light-matter coupling. The first is the nanophotonic cavity that leverages a large number of molecules to achieve strong coupling. The other is the nanoplasmonic cavity that leverages a locally enhanced electric field confined in a small volume to enhance the coupling. This local field is effectively confined to the nearby surrounding of a spherical nanoparticle residing on a lattice of nanoparticles on the scale of ~nm$^3$ or even ~\AA$^3$ (namely picocavities~\cite{Chikkaraddy2016nature,Jeremy2022nl,Benz2016science}). 

%Schematic diagrams of the two cavities are shown in Fig.~\ref{fig:cavities}.

\iffalse
\begin{figure}[!htb]
    \centering
    \includegraphics[width=0.5\textwidth]{figures/plasmonic.png} %To be replaced. 
    \caption{Schematic diagrams of two types of cavities, nanophotonic and nanoplasmonic ones. Nanoplasmonic cavities could achieve strong coupling at the single molecule level at the cost of strong dissipation. The dissipation can be suppressed by designing a surface-lattice plasmon mode. {\color{red}(To be replaced!!!)}}
    \label{fig:cavities}
\end{figure}
\fi 

The derivaiton of Eq.~\ref{EQ:H_PH_SHIFTED} assumes dipole approximation. However, the dipole approximation may fail in the ultra-confined nanoplasmonic cavities where the EM fields are confined within nanometric volumes~\cite{Neuman2018nl,Wang:2021jpcc}. Consequently, the size of molecules becomes comparable to the cavity volumes, and the widely used dipole approximation breaks down. Position-dependent coupling strength that requires the spatial distribution of excitonic and photonic quantum states is found to be a key aspect in determining the dynamics in ultrasmall cavities both in the weak and strong coupling regimes~\cite{Neuman2018nl}. 

\subsection{Weak and strong coupling}
The light-matter interactions in nanoplasmonic environments can be split into weak and strong coupling regimes. The weak-coupling regime is associated with the Purcell enhancement of spontaneous emission. This effect has been found to be particularly strong when the molecule is placed next to a metallic surface or nanostructure~\cite{Kelly:2002jpcb}. In such a regime, the plasmon has been found to be able to tune photophysics and photochemistry (i.e., plasmon chemistry) via various possible pathways. 

The other regime is the strong coupling regime, where the light-matter interaction cannot be treated perturbatively. The strong coupling is characterized by a reversible (coherent) exchange of energy (known as Rabi oscillation) between the matter and the cavity photon, as the coupling is strong enough to compete with the dissipation. In this regime, 
the formation of polariton requires theoretical methods that treat the matter and photonic DOFs on equal footing and describe the multiple interactions between photons, electrons, and nuclei on different length and time scales (polariton chemistry). 