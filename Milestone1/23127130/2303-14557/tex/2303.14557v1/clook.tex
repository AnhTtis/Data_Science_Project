\documentclass[sigconf]{acmart}

%% Fonts used in the template cannot be substituted; margin 
%% adjustments are not allowed.
%%
%% \BibTeX command to typeset BibTeX logo in the docs
\AtBeginDocument{%
  \providecommand\BibTeX{{%
    \normalfont B\kern-0.5em{\scshape i\kern-0.25em b}\kern-0.8em\TeX}}}

%% Rights management information.  This information is sent to you
%% when you complete the rights form.  These commands have SAMPLE
%% values in them; it is your responsibility as an author to replace
%% the commands and values with those provided to you when you
%% complete the rights form.
\setcopyright{acmcopyright}
\copyrightyear{2023}
\acmYear{2023}
\acmDOI{XXXXXXX.XXXXXXX}

%% These commands are for a PROCEEDINGS abstract or paper.
\acmConference[CHI '23 HCI Across Borders (HCIxB) Workshop Papers]{CHI Conference on Human Factors in Computing Systems}{April 23--28,
  2023}{Hamburg, Germany}


%%
%% end of the preamble, start of the body of the document source.
\begin{document}

%%
%% The "title" command has an optional parameter,
%% allowing the author to define a "short title" to be used in page headers.

\newcommand{\cl}{Clo(o)k}
\title{\cl{}: A Clock That Looks}
% \title{What is time?}

%%
%% The "author" command and its associated commands are used to define
%% the authors and their affiliations.
%% Of note is the shared affiliation of the first two authors, and the
%% "authornote" and "authornotemark" commands
%% used to denote shared contribution to the research.

% \author{Zhuoyue Lyu}
% \affiliation{%
%   \institution{Harvard University}
%   % \streetaddress{13 Appian Way}
%   \city{Cambridge}
%   \state{MA}
%   \country{USA}}
% \email{zhuoyue_lyu@gse.harvard.edu}


% 最终提交的work再写Harvard好了...?
\author{Zhuoyue Lyu}
\affiliation{%
  \institution{Massachusetts Institute of Technology (MIT)}
  \city{Cambridge}
  \state{MA}
  \country{USA}}
\email{zhuoyuel@mit.edu}



%%
%% The abstract is a short summary of the work to be presented in the
%% article.
\begin{abstract}
What if a clock could do more than just tell time - what if it could actually see? This paper delves into the conceptualization, design, and construction of a timepiece with visual perception capabilities, featuring three applications that expand the possibilities of human-time interaction. Insights from an Open House showcase are also shared, highlighting the unique user experiences of this device.
\end{abstract}

%%
%% The code below is generated by the tool at http://dl.acm.org/ccs.cfm.
%% Please copy and paste the code instead of the example below.
%%
\begin{CCSXML}
<ccs2012>
   <concept>
       <concept_id>10003120.10003121.10003125</concept_id>
       <concept_desc>Human-centered computing~Interaction devices</concept_desc>
       <concept_significance>500</concept_significance>
       </concept>
 </ccs2012>
\end{CCSXML}

\ccsdesc[500]{Human-centered computing~Interaction devices}
% \vspace{-1cm}
%%
%% Keywords. The author(s) should pick words that accurately describe
%% the work being presented. Separate the keywords with commas.
\keywords{clock, time, tangible, telepresence}

%% A "teaser" image appears between the author and affiliation
%% information and the body of the document, and typically spans the
%% page.
\begin{teaserfigure}
  \centering  
  \includegraphics[width=0.81\textwidth]{pic/clook-1.jpg}
  \caption{The \cl{} system. \href{https://zhuoyuelyu.com/clook}{\textbf{zhuoyuelyu.com/clook}}}
  \Description{}
  \label{fig:teaser}
  \vspace{0.2cm}
\end{teaserfigure}

% \received{20 February 2007}
% \received[revised]{12 March 2009}
% \received[accepted]{5 June 2009}

%%
%% This command processes the author and affiliation and title
%% information and builds the first part of the formatted document.
\maketitle


\section{Introduction}
Time is a fascinating concept, encompassing both objective and subjective dimensions. On one hand, it can be precisely measured and represented through the steady, periodic movement of a clock's hands. On the other hand, time can feel fluid and elusive, easily slipping away unnoticed as we become engrossed in our tasks. 

Furthermore, time is a universal experience that connects us across the world. When it is 3 AM in Boston, it is also 3 PM in Beijing, and this shared knowledge allows us to communicate across vast time zones without social faux pas.

Disney movies like \textit{Snow White and the Seven Dwarfs}~\cite{snow-white-and-the-seven-dwarfs} and \textit{Beauty and the Beast}~\cite{beauty-and-the-beast} bring ordinary objects to life, imbuing them with human-like emotions and behavior. What if a clock could have its own distinct character too? What if it had the ability to see and interact with humans in its unique way? This thought-provoking concept opens up intriguing possibilities for human-clock interactions that are both playful and creative.

Motivated by the desire to explore the intricacies of human-time interaction~\cite{mcgrath1986time}, I designed and built the Clo(o)k - a clock that looks.

% The system is flexible: you can print your frames and customize movements. 

\section{Related Work}
Previous research has explored the practical use of clock, such as visualizing upcoming event~\cite{10.1145/506443.506505}, tracking sleep patterns~\cite{10.1145/1517664.1517672}, and supporting planning and reflection~\cite{10.1145/3334480.3382830,10.1145/3357236.3395439}. However, this work takes a more playful approach to clock design, aiming to create an interactive experience that encourages users to reflect on the concept of time. 

In a similar vein, telepresence technology has used various sensory modalities, including sound~\cite{10.1145/1935701.1935705}, touch~\cite{10.1145/1120212.1120435,10.1145/2642918.2647377}, and sight~\cite{10.1145/142750.142977} to enable people to feel the presence of others even when they are far apart. This work builds upon these ideas by exploring how time can be integrated into such interactions, creating new possibilities for connecting with others across time and space.



\section{Interactions}
This section showcases three interactive applications using the Clo(o)k, as depicted in Figures~\ref{fig:1}, \ref{fig:2}, and \ref{fig:3}.

\begin{figure}[htb!]
    \centering
    \includegraphics[width=\linewidth]{pic/interaction-1.jpeg}
    \vspace{-0.6cm}
    \caption{The Clo(o)k operates like a normal clock when the user is looking at it, but speeds up when the user's attention shifts away.}
    \label{fig:1}
    % \vspace{-0.4cm}
\end{figure}


\begin{figure}[htb!]
    \centering
    \includegraphics[width=0.6\linewidth]{pic/interaction-2.jpeg}
    \vspace{-0.15cm}
    \caption{If the Clo(o)k notices that the user is engaged in conversation with someone, it stops moving to alleviate any concerns about time and allow for uninterrupted communication.}
   \label{fig:2}
    % \vspace{-0.4cm}
\end{figure}



\begin{figure}[htb!]
    \centering
    \includegraphics[width=\linewidth]{pic/interaction-3.jpeg}
    \vspace{-0.6cm}
    \caption{In the scenario where the user's loved one resides in a different time zone, if both parties look at their respective Clo(o)ks simultaneously, they will be able to see each other's current time.}
    \label{fig:3}
\end{figure}




\section{\cl{} System}
This section details the design and construction process of the \cl{}. The relevant files are available for download at \href{https://zhuoyuelyu.com/clook}{\textbf{zhuoyuelyu.com/clook}}.

% 这里全用过去时也不对,全现在时似乎也不对...咋搞嘞
\subsection{Design}
To achieve a simple and clean design for \cl{}, I used basic shapes, such as circles and triangles, to represent the clock face and base. The camera is positioned at the center of the clock face to maintain the appearance of a standard clock. This subtle placement allows users to initially perceive it as a normal clock, only to be pleasantly surprised by its intriguing behavior once they realize the central element is actually a camera. However, this design choice meant that the traditional clock mechanism, where all hands are attached to the centerpiece, could not be used. To overcome this, I utilized the hollow clock mechanism~\cite{hallow-clock} - the minute and hour hands are two rings driven by two sets of gears hidden in the base (Figure~\ref{fig:gear}). 

% \begin{figure}[htb!]
%     \centering
    
%     \caption{The hardware of the \cl{}}
%    \label{fig:2}
% \end{figure}


\begin{figure}[!htb]
    \includegraphics[width=0.95\linewidth]{pic/clook-9.jpg}
    \caption{The rings and the base fit perfectly.}
\end{figure}
\begin{figure}[!htb]
   \includegraphics[width=0.95\linewidth]{pic/clook-5.jpg}
    \caption{These gears drive the rings.}
     \label{fig:gear}
\end{figure}


\subsection{Fabrication}
The 3D design was created using Fusion 360~\cite{fusion-360} and Blender~\cite{blender}, and then exported as STL files to be printed. The rings and gears were printed with PLA using Prusa~\cite{prusa}, while the hands were spray-painted red. The base (Figure~\ref{fig:base}) was printed using Fuse 1~\cite{fuse1} with powder for better quality. 

\begin{figure}[!htb]
    \includegraphics[width=\linewidth]{pic/clook-3.jpeg}
    \caption{The back cover is screwed to hide the components.}
\end{figure}
\begin{figure}[!htb]
   \includegraphics[width=\linewidth]{pic/clook-2.jpg}
    \caption{The base that holds everything.}
     \label{fig:base}
\end{figure}





\subsection{Hardware}
% TODO: 讲到的每个hardware最好都在图片里面指出来是哪个否则外人不知道鸭...
\cl{} utilizes an ESP32-CAM~\cite{esp32-cam} with a modified lens to capture the video, control one of the stepper motors, and wirelessly communicate with the computer and other clocks. A D11C~\cite{d11c} board was designed and milled to extend the pins of the ESP32-CAM so it can receive the serial signal from the ESP32-CAM to control another stepper motor. Two 28BYJ-48 stepper motors~\cite{stepper} with driver boards are used to control the rings. An FTDI Converter~\cite{ftdi} is used to program ESP32-CAM and provide the power (Figure~\ref{fig:10}).

\subsection{Software}
% TODO 讲一下怎么检测三个interaction的,或者写在interaction那里?就是靠几个人头这样...
Both ESP32-CAM and D11C were programmed using Arduino~\cite{arduino}. Although there is a reduced version of OpenCV~\cite{opencv-esp32} for ESP32-CAM, for more flexibility and convenience,  a separate laptop running Python script using OpenCV~\cite{opencv} is used for facial detection, which gets the video stream from ESP32-CAM wirelessly. ESP32-CAM updates the movements of the stepper motors based on the information (number of faces) it receives from Python through the USB serial connection.

\begin{figure}[!htb]
    \includegraphics[width=\linewidth]{pic/clook-6.jpeg}
    % \vspace{-1.2cm}
    \caption{The camera sits at the center of the clock face.}
\end{figure}
\begin{figure}[!htb]
   \includegraphics[width=\linewidth]{pic/clook-7.jpg}
    \caption{The electronics of the \cl{}.}
    \label{fig:10}
\end{figure}



% \begin{figure}[htb!]
%     \centering
%     \includegraphics[width=0.6\linewidth]{pic/clook-7.jpg}
%     \caption{The hardware of the \cl{}}
%    \label{fig:2}
% \end{figure}





\section{Feedback and Limitations}
\cl{} was showcased at an Open House event at the MIT Media Lab, attended by over 100 participants. Below are some of the feedback and comments gathered during the event.
\begin{itemize}
\item Attendees consistently smiled upon hearing about the user interactions with the clock, finding the exploration of time's dual nature – both subjective and objective – particularly intriguing.
\item The minimalist design of the clock caught the attention of many participants, and they were pleasantly surprised to learn that the rings were detachable from the base.
\item Several international students, who have loved ones living in different time zones, found the third interaction to be incredibly useful. They appreciated the ability to feel a sense of presence from their distant family and friends.
\end{itemize}






However, there are still areas for improvement that can be addressed:

\begin{itemize}
\item The poor lighting conditions affected the system's ability to accurately identify faces.
\item The serial connections between ESP32-CAM and D11C were slightly laggy, resulting in a noticeable delay between the minute hand and hour hand.
\end{itemize}

\begin{figure}[!htb]
    \includegraphics[width=.95\linewidth]{pic/clook-21-blur.jpg}
    % \caption{The camera sits nicely at the center of the clock face.}
   \includegraphics[width=.95\linewidth]{pic/clook-22-blur.jpg}
    % \vspace{-1.2cm}
\caption{Pictures from the demo of \cl{} at the Open House event for MAS.863 class. (MIT Media Lab, Dec. 20, 2022)}
% \vspace{-0.5cm}
\end{figure}

% One participant suggested since the rings are detachable, there could be more rings, where different rings represent different things or people. They said the ring itself could also be combined with different LED lighting to provide richer information.

\section{Future Work}
\cl{} is a versatile system that offers a multitude of potential interactions. The three interactions presented in this paper are just examples of what can be achieved. The detachable rings provide the opportunity to explore different ring styles and the novel interactions they can generate. For instance, users can customize their rings to suit their preferences (as shown in Figure~\ref{fig:ring} on the left), or turn them into storytelling devices (as demonstrated in Figure~\ref{fig:ring} on the right). As such, in addition to addressing the current limitations of the system and performing formal evaluations, further research will be pursued conducted to explore the vast possibilities for interactions that \cl{} can offer.

\begin{figure}[htb!]
    \centering
    \includegraphics[width=\linewidth]{pic/clook-30.jpg}
    % \vspace{-0.2cm}
    \caption{Customized ring designs.}
   \label{fig:ring}
    % \vspace{-0.5cm}
\end{figure}

% interesting 

% TODO...video of detaching the frames?
\section{Conclusion}
This paper presents the design and build of \cl{}, a system that supports playful interactions with time. This work proposes an innovative approach to employing a tangible, everyday object for stimulating reflections on intangible concepts such as time. It is hoped that this research will inspire further exploration and spark new ideas within the CHI community.

%%
%% The acknowledgments section is defined using the "acks" environment
%% (and NOT an unnumbered section). This ensures the proper
%% identification of the section in the article metadata, and the
%% consistent spelling of the heading.
\begin{acks}
Many thanks to Dr. Marcelo Coelho, Bill McKenna, Dominic Lim Co, and students of MIT 4.031 for their feedback on the design and fabrication of the \cl{}. Thank Prof. Neil Gershenfeld, TAs, and students of MAS.863, for guidance and support on the electronics. Thank Prof. Hiroshi Ishii and his Tangible Interfaces class for the inspiration on the interaction design. Finally, thank all who stopped by my booth in the Open House of MAS.863 for insights and comments.
\end{acks}

%%
%% The next two lines define the bibliography style to be used, and
%% the bibliography file.
% \newpage
\bibliographystyle{ACM-Reference-Format}
\bibliography{clook}

\end{document}
