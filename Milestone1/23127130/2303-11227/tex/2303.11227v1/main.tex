\documentclass{amsart}
\usepackage[foot]{amsaddr}
% Language setting
% Replace `english' with e.g. `spanish' to change the document language
\usepackage[english]{babel}
\usepackage{amsthm}

\usepackage{url}
\usepackage{breakurl}
\usepackage[breaklinks]{hyperref}
 
 
\newcommand{\introthmname}{}
\newtheorem{introthminn}{\introthmname}
\newenvironment{introthm}[1]
{\renewcommand{\introthmname}{#1}\begin{introthminn}}
	{\end{introthminn}}


% Set page size and margins
% Replace `letterpaper' with`a4paper' for UK/EU standard size
%\usepackage[letterpaper,top=2cm,bottom=2cm,left=3cm,right=3cm,marginparwidth=1.75cm]{geometry}
\usepackage{amssymb}
\usepackage{mathtools}
\usepackage{thmtools}
\usepackage{xfrac}
\usepackage{thm-restate}
%\usepackage{authblk}
\usepackage{enumerate}
\usepackage{hyperref}
\usepackage{xfrac}
\usepackage{cleveref}
\usepackage{verbatim}
% Useful packages
\usepackage{amsmath}
\usepackage{amssymb}
\usepackage{mathabx}
\usepackage{graphicx}
\usepackage{amsfonts}
\usepackage{pb-diagram}
\usepackage{tikz-cd}
\usepackage{graphicx}
\usepackage{xcolor}
\usepackage{adjustbox}
%\usepackage{comment}
\usepackage{mathtools}
\usepackage{todonotes}
\newtheorem{theorem}{Theorem}
%\numberwithin{theorem}{section}
\usepackage{mathrsfs}
\usepackage{bm}


\newcommand{\transv}{\mathrel{\text{\tpitchfork}}}
\makeatletter
\newcommand{\tpitchfork}{%
  \vbox{
    \baselineskip\z@skip
    \lineskip-.52ex
    \lineskiplimit\maxdimen
    \m@th
    \ialign{##\crcr\hidewidth\smash{$-$}\hidewidth\crcr$\pitchfork$\crcr}
  }%
}


%\newtheorem{lemma}[thm]{Lemma}
\newtheorem{definition}[theorem]{Definition}




\newtheorem{corollary}[theorem]{Corollary}
\newtheorem{lemma}[theorem]{Lemma}
\newtheorem{remark}[theorem]{Remark}
\newtheorem{observation}[theorem]{Observation}
\newtheorem{fact}[theorem]{Fact}

\newtheorem{proposition}[theorem]{Proposition}
\newtheorem{problem}[theorem]{Problem}
%\newtheorem{example}[thm]{Example}

\newtheorem{question}{Question}
\newtheorem{conjecture}{Conjecture}

\newtheorem{claim}[theorem]{Claim}
\newtheorem*{notation}{Notation}
\newcommand{\jet}[0]{\mathsf{J}^{1}}

%\renewcommand{\thethm}{\arabic{theorem}}
\newcommand{\loc}[1]{\big|_{#1}} 

 


%\usepackage[colorlinks=true, allcolors=blue]{hyperref}
\newcommand{\lu}[1]{{\color{magenta}#1}}
\newcommand*{\greysquare}{\textcolor{gray}{\blacksquare}}
\newenvironment{subproof}[1][\proofname]{%
\renewcommand{\qedsymbol}{$\greysquare$}%
\begin{proof}[#1]%
	}{%
\end{proof}%
}
%\renewcommand{\emph}[1]{\textbf{#1}}

\newenvironment{subsubproof}[1][\proofname]{%
\renewcommand{\qedsymbol}{$\blacksquare$}%
\begin{proof}[#1]%
}{%
\end{proof}%
}

\newcommand{\pri}[1]{#1 \times I} 
\newcommand{\es}[0]{supp^{e}}
\renewcommand{\L}[0]{\mathcal{L}}
\newcommand{\U}[0]{\mathbb{U}}
%\bibliographystyle{alpha}
%\bibliography{sample}
\newcommand{\ja}[1]{{\color{magenta}#1}}
\newcommand{\p}[0]{\mathcal{P}}
\newcommand{\nin}[0]{\notin}
\newcommand{\nb}[2]{N_{#2}(#1)}
\newcommand{\nn}[0]{\mathsf{N}}
\newcommand{\N}[0]{\mathbb{N}}
\newcommand{\Q}[0]{\mathcal{Q}}
%\newcommand{\G}[0]{\H_{\partial S}(S)}
\newcommand{\HQ}[0]{\H_{c0}(X,\Q)}

\newcommand{\MCG}[0]{PMod}

\renewcommand{\U}[0]{\mathcal{U}}
\newcommand{\V}[0]{\mathcal{V}}
\newcommand{\W}[0]{\mathcal{W}}

\title[Minimality of the compact-open topology]{The compact-open topology on the diffeomorphism or homeomorphism group of a smooth manifold without boundary is minimal in almost all dimensions}
%{The compact-open topology on the homeomorphism group of a surface without boundary is minimal}

\renewcommand{\emph}[1]{\textbf{#1}}
\newcommand{\R}[0]{\mathbb{R}}
        
\renewcommand{\d}[1]{\D_{co}[#1]}   
\newcommand{\hh}[0]{\mathfrak{H}}
\renewcommand{\gg}[0]{\mathfrak{G}}
\newcommand{\PS}[2]{#1_{\rl{#2}}}   
\renewcommand{\SS}[2]{#1_{#2}} 
\renewcommand{\emph}[1]{#1} 

\newcommand{\Sp}[0]{\mathsf{Sp}}
\newcommand{\Tsp}[0]{\mathsf{Tsp}}

\author{J. de la Nuez Gonz\'alez }
\email{jnuezgonzalez@gmail.com}
\address{Korea Institute for Advanced Study (KIAS)}
\date{\today}
\thanks{Work supported by KIAS individual grant MG084001.}
%\makeindex
\begin{document}
\begin{abstract}
We show that for any connected smooth manifold $M$ of dimension different from $3$ the restriction of the compact-open topology to the diffeomorphism group of $M$ is minimal, i.e. the group does not admit a strictly coarser Hausdorff group topology. This implies the minimality of the compact-open topology on the homeomorphism group of $M$ in all dimensions different from $3$ and $4$. In those cases for which in addition to all of this automatic continuity is known to hold, such as when $M$ is closed, one can conclude that the compact-open topology is the unique separable Hausdorff group topology on the homeomorphism group.
%For a surface with boundary we show that Chang and Gartside's counterexample to the minimality of the compact-open topology is minimal.
\end{abstract}

\maketitle
%\printindex
%\tableofcontents

\newcommand{\rl}[1]{|#1|} % set of points
\renewcommand{\H}[0]{\mathcal{H}}
\newcommand{\G}[0]{\mathcal{G}}
\newcommand{\T}[0]{\mathfrak{t}}
\newcommand{\tc}[0]{\T_{co}}
\newcommand{\nd}[0]{\mathcal{N}_{\T}(1)}
\newcommand{\cs}[0]{\mathcal{CS}}
\numberwithin{fact}{section}
\numberwithin{definition}{section}
\numberwithin{remark}{section}
\newcommand{\norm}[1]{\lVert #1 \rVert_{2}} 
\newcommand{\I}[0]{I}
\newcommand{\J}[0]{\hat{I}} 


\numberwithin{theorem}{section}
\numberwithin{lemma}{section}
\numberwithin{corollary}{section}
\numberwithin{observation}{section}
\numberwithin{claim}{section}

\setcounter{section}{-1}
\section{Introduction}
  
  We begin by recalling the following definition:
  \begin{definition}
  Let $G$ be a group and $\T$ a Hausdorff group topology on $G$. We say that $\T$ is minimal if $G$ admits no Hausdorff group topology strictly coarser than $\T$.
  \end{definition}
  This abstracts a key feature of compact topological groups among Hausdorff topological groups and has received substantial attention in the literature. For the broader context and further questions we refer the reader to the comprehensive survey \cite{dikranjan2014minimality}.
  
  Given a topological space $X$, the compact-open topology on the homeomorphism group of $\H(X)$ of $X$ is the topology generated by all subsets of the form $[K,U]=\{f\in G\,|\,f(K)\subseteq U\}$, for $K\subseteq X$ compact and $U\subseteq X$ open in $X$. The following classical result is due to Arens \cite{arens1946topologies}. %, see also \cite{dijkstra2005homeomorphism}
  \begin{fact}
  Let $X$ be a Hausdorff, locally compact and locally connected topological space. Then the compact-open topology is a group topology on $\H(X)$.
  \end{fact}
  
  \newcommand{\D}[0]{\mathcal{D}}
  %It is for homeomorphism groups of topological manifolds that the question of minimality of the compact-open topology makes the most sense. However, their study in higher dimensions presents several technical challenges that have led us here to be less ambitious and restrict ourselves to the smooth category.
  
  Given a differentiable manifold $M$, we denote by $\D_{c0}(M)$ the subgroup of its diffeomorphism group $\D(M)$ consisting of all those $h\in\D(M)$ diffeotopic to the identity through a compactly supported diffeotopy (see Section \ref{s: preliminaries}). In the category of topological manifolds we define the subgroup $\H_{c0}(M)$ of the homeomorphism group $\H(M)$ analogously with isotopy in place of diffeotopy.  
 
  Our goal is to present a rather elementary proof of the following result:
  \begin{introthm}{Theorem}
  \label{t: main}Let $M$ be a smooth connected manifold without boundary and $\gg$ some subgroup of $\D(M)$ that contains $\D_{c0}(M)$ and any element supported in the interior of a smoothly embedded closed ball. If $dim(M)\neq 3$, then the restriction of the compact-open topology to $\gg$ is minimal. 
  \end{introthm}
  
  \begin{remark}
  	Notice that whenever $\D_{0}(\bar{B}^{m},\partial B^{m})=\D(B^{m},\partial B^{m})$, $m=dim(M)$, then the condition $\D_{c0}(M)\leq\gg$ automatically implies the second. Short summaries touching upon the issue of when this is known to hold can be found in \cite{hatcher50} and \cite{kupers2019diffeomorphisms}.
  \end{remark}
   \begin{remark}
   	The dimension $1$ case follows from the same argument that applies to the group of homeomorphisms (see \cite{dikranjan2014minimality}).       
   \end{remark}
  
  For the following see \cite{dikranjan2014minimality}, Theorem 3.1 or \cite{stephenson1971minimal}.
  \begin{fact}
  	\label{f: density and minimality} Let $(G,\T)$ be a topological group and $H$ a dense subgroup of $G$. Then the restriction of $\T$ to $H$ is minimal if and only if $\T$ is a minimal group topology on $G$ and $H$ intersects all closed normal subgroups of $G$. 
  \end{fact}
    
  All manifolds in dimension $\leq 3$ admit a smooth structure and their homeomorphisms can be uniformly approximated by diffeomorphisms (see \cite{munkres1960obstructions}). In higher dimension we have the following elegant characterization due to S. M\"uller, which builds on \cite{munkres1960obstructions}, \cite{connell1963approximating} and \cite{bing1963stable}:
  % (\footnote{Which we do not need in full generality but cannot help but quote.}):   
  \begin{fact}
	  \label{f: mullers approximation}[\cite{muller2014uniform}, Thm 4] Let $M$ be a smooth connected manifold of dimension $n\geq 5$, possibly non-compact and with non-empty boundary, and let $\phi$ be a homeomorphism of $M$ that is compactly supported in $M\setminus \partial M$. Then $\phi$ can be approximated uniformly by diffeomorphisms if and only if it is isotopic to a diffeomorphism. 	
  \end{fact}
 
  %The latter also applies to diffeomorphisms of $\R^{n}$, $n\geq 5$, $b$  We conclude:
  \begin{corollary}
  	\label{c: main} The compact-open topology on the group $\H(M)$ is minimal for any connected manifold $M$ which has no boundary, admits a smooth structure and satisfies $dim(M)\nin\{3,4\}$.
  	\begin{comment}
	  	\begin{itemize}
	  		\item a $2$-dimensional manifold 
	  		\item $\R^{n}$ for $n\geq 5$ 
	  	\end{itemize} 
  	\end{comment}
    %In the first case the result also applies to any group $\gg$ with $\H_{c0}(M)\leq \gg\leq\H(M)$. 
  \end{corollary} 
   \begin{proof}
	   Let $\gg$ be the closure of $\D(M)$ in $\H(M)$. 	   
	   By Fact \ref{f: density and minimality} applied to $\gg$ we conclude that any Hausdorff group topology $\T$ on $\H(M)$ coarser than $\tc$ must agree with $\tc$ on $\gg$. However, by \ref{f: mullers approximation} the group $\gg$ must contain every element isotopic to the identity. By Observation \ref{o: intrusive homeos} all the above is incompatible with $\T$ being strictly coarser than $\tc$.    
   \end{proof}
  \begin{comment}
	  By Alexander's trick \cite{alexander1923deformation} this applies, in particular, to every element the closure of whose support lies in an embedded open ball, so that $\gg$ intersects every normal subgroup of $\H(M)$  by \ref{f: whittaker}. 
	\end{comment}
  
    
  It was first shown by C. Rosendal in \cite{rosendal2008automatic} that any group homomorphism from the homeomorphism group of a compact surface to a separable topological group is automatically continuous. This was later generalized by K. Mann to homeomorphism groups of compact manifolds in arbitrary dimension \cite{mann2016automatic} and later in \cite{mann2020automatic} to certain groups of homeomorphisms of non-compact manifolds. Combining these results with Theorem \ref{t: main} one obtains:
  
  \begin{corollary}
  \label{c: uniqueness}Let $M$ be as in Corollary \ref{c: main} and assume furthermore that $M$ is of the form 
  $M'\setminus F$, where $M'$ is either closed or the interior of a compact manifold with boundary and $F\subseteq M\setminus \partial M$ is either:
  \begin{itemize}
  	\item empty if $dim(M)\neq 2$ 
  	\item a finite set or the union of a finite set and a cantor set if $dim(M)=2$ 
  \end{itemize}
  Then the compact-open topology is the only separable Hausdorff group topology on the group $\H(M)$.
  \end{corollary}
  
  Note that the compact-open topology was already known to be the unique complete separable group topology on the homeomorphism group of a compact manifold (see \cite{kallman1986uniqueness} and \cite{mann2016automatic}).
  The result in \cite{mann2020automatic} is stated in terms of the group of homeomorphisms of a manifold $M$ preserving a set $F$ as above. However, as discussed there, in dimension $2$ the restriction map is an isomorphism of topological groups between the former and $\H(M\setminus F)$ (\footnote{That the two topologies are the same can be seen by applying the criterion of equality between $\tc$ and $(\tc)_{\restriction F}$ in Theorem $5$ from \cite{chang2017minimum} to the set-wise stabilizer of $F$ in $\H_{0}(M)$. }).
    \\
    
  \paragraph{\bf{Outline of the proof}}
      
        The proof of \ref{t: main} is structured as follows. Fix $M$ of dimension $m=2$ or $m\geq 4$ and $\gg\leq\D(M)$ as in the statement of the theorem as well as some Hausdorff group topology $\T$ on $\gg$ strictly coarser than the restriction of the compact-open topology.
               
        Section \ref{s: preliminaries} establishes some notation and recalls some basic facts many readers will be familiar with. In Section \ref{s: neighbourhoods are rich} we see that if $\T$ is a group topology on $\G$ strictly coarser than the compact-open topology, then $\V\in\mathcal{N}_{\T}(1)$ contains many fix-point stabilizers of embedded codimension $1$ simplicial complexes. In Section \ref{s: compression moves} we use this to conclude, roughly speaking, that given an embedded arc $\alpha$ in $M$ one can essentially find elements acting arbitrarily on $\alpha$ in any given $\V\in\nd$. 
        
        Section \ref{s: Morse} proves that any isotopy of an $(m-1)$-dimensional manifold into an $m$-dimensional manifold $M$ it transverse with respect to any given codimension $1$ submanifold $L$ of $M$, except at a discrete set of points, where well-understood transitions occur, a subject on which we further elaborate in \ref{s: isotopy operations}. This allows a way decompose elements of $\D_{c0}(M)$ as products of elements in the stabilizers of given submanifolds and elements supported in arbitrary neighbourhoods of embedded balls of proper codimension. Finally, in \ref{conclusion} we show all of the above to be in contradiction with the assumption that $\T$ is Hausdorff.´
             
\section{Preliminaries}
  \label{s: preliminaries}
  
  \subsection*{Generalities}
  
  Let $M$ be a connected topological manifold. If we fix a metric $d$ on $M$ compatible with its topology, then a base of neighbourhoods of the identity for the compact-open topology on $\H(M)$ is given by the collection of sets
  $$
  \V_{K,\epsilon}=\{g\in G\,|\,\forall p\in K\,\,d(p,g\cdot p),d(p,g^{-1}\cdot p)<\epsilon\}
  $$
  where $K$ ranges over all compact subsets of $M$ and $\epsilon$ over all positive reals. Sometimes we might only be interested in the set $\V_{\epsilon}:=\V_{M,\epsilon}$. For any subset $A\subseteq M$ and $\epsilon>0$ we let $\nn_{\epsilon}(A):=\{p\in M\,|\,d(p,A)<\epsilon\}$.
  When working with $\gg\leq\H(M)$ we will use the same notation to denote the restriction of these to $\gg$.
  
  From now on all the manifolds considered will be assumed to be smooth and the distance function to be induced from some complete riemannian metric on the manifold.  
%  We may and will assume that for any $p$ there is $\delta_{p}$ such that the ball $B(p,\delta_{p})$ is connected.
  Given an open set $U\subseteq M$ we write $\gg[U]$ for the subgroup of $\gg$ consisting of the elements for which
  $\overline{supp(g)}:=\{p\in M\,|\,p\neq g\cdot p\}\subseteq U$. 
  
  We will often use the fact that a smooth manifold $M$ always can always be written as a union of an ascending chain of compact submanifolds $M_{0}\subseteq\mathring{M}_{1}\subseteq M_{1} \dots$. \footnote{See, for instance, the remark after 4.3.2 in \cite{wall2016differential}.}
  % and note that the function $L_{P}$ will have a measure zero set of critical values even if it is not degenerate}. 
  
  Working in a group we use the notation $g^{h}:=h^{-1}gh$ when graphically sound.
  
  \subsection*{The Whitney topology and the $\mathcal{D}$-topology}
  \newcommand{\wt}[0]{\mathsf{W}}
  \newcommand{\swt}[0]{\mathsf{W}^{s}}
  Given smooth manifolds $M,N$, we denote the strong Whitney topology (or simply the Whitney topology) 
  on $C^{\infty}(M,N)$ by $\wt(M,N)$. Recall that this is the union, for $R\to\infty$ of the restrictions to 
  $C^{\infty}(M,N)$ of the topologies $\wt^{r}(M,N)$ on $C^{r}(M,N)$ of locally uniform convergence in all derivatives of order at most $r$.  

  A useful refinement of the Whitney topology on $C^{\infty}(M,N)$ is the so called $\mathcal{D}$-topology, or $\D(M,N)$ (see \cite{michor2011manifolds} p.37). 
  \begin{definition}
  	\label{d: D-topology}Fix any sequence $(K_{n})_{n\in N}$ of compact submanifolds of $M$ with $K_{0}=\emptyset$, $K_{n}\subseteq\mathring{K}_{n+1}$ for all $n\geq 0$ and $\bigcup_{n\geq 0}K_{n}=M$  (we only require $\partial K_{n}$ to be submanifold in a weak sense, i.e. we impose no condition on $\partial K_{n}$). 
  	For any $n\geq 0$ let $\rho_{n}:C^{\infty}(M,N)\to C^{\infty}(M\setminus\mathring{K}_{n},N)$ be the obvious restriction map
  	 
  	Then a basis of neighbourhoods for $\mathcal{D}(M,N)$ is given by all the intersections of the form $\bigcap_{n\geq 0}\rho_{n}^{-1}(U_{n})$, where $U_{n}\in\wt(M\setminus \mathring{K}_{n},N)$.  
  \end{definition}
  
  
  
  We collect some basic facts below, where (\ref{whitney restriction}) and (\ref{whitney infinity}) follow easily from the definition and (\ref{d infinity}) from (\ref{whitney infinity}) and we refer the reader to Chapter $2$ in \cite{hirsch2012differential}, and Chapter $4$ in \cite{michor2011manifolds} for more details. 

  \begin{fact}
  	\label{f: properties Whitney}
  	The following holds: 
  	\begin{enumerate}
  		\item \label{whitney baire}$C^{\infty}(M,N)$ endowed with either $\wt(M,N)$ or $\mathcal{D}(M,N)$ is a Baire space 
  		(\cite{michor2011manifolds}, pp. 34, 38)
  		\item \label{whitney rank}The following collections of maps are open in $\wt(M,N)$:
  		\begin{itemize}
  			\item the collection of maps whose differential is uniformly of rank $\geq r$,
  			\item the collection of embeddings of $M$ in $N$ (\cite{hirsch2012differential}, Ch2, Thm 1.4),
  			\item the collection of proper maps. (\cite{hirsch2012differential}, Ch2, Thm 1.5)
  		\end{itemize}
  		\item \label{whitney restriction} For any closed submanifold $V\subseteq M$ the restriction map $\rho:C^{\infty}(M,N)\to C^{\infty}(V,N)$ is continuous and open and thus the image or the preimage of a comeager set by $\rho$ is still comeager.
  		%\item \label{whitney composition}\lu{composition}
  		\item \label{whitney infinity} Let $L\subseteq\mathring{M}$ be a codimension $0$ submanifold of $M$, $f\in C^{r}(M,N)$ and $\W$ some neighbourhood of $f$ in $\wt^{r}(M,N)$, $r\geq 0$. Then there exists some neighbourhood $\V$ of $f_{\restriction \mathring{L}}$ in $\wt(\mathring{L},N)$ such that for any $h\in\V$ we have $f^{*}:=f_{\restriction M\setminus\mathring{L}}\cup h\in \V \subseteq C^{r}(M,N)$
  		\item \label{d infinity} Let $L\subseteq\mathring{M}$ be a codimension $0$ submanifold of $M$, $f\in C^{\infty}(M,N)$ and $\W$ some neighbourhood of $f$ in $\wt(M,N)$. Then there exists some neighbourhood $\V$ of $f_{\restriction \mathring{L}}$ in $\mathcal{D}(\mathring{L},N)$ such that for any $h\in\V$ we have $f^{*}:=f_{\restriction M\setminus\mathring{L}}\cup h\in C^{\infty}(M,N)$, $f^{*}\in\V$.
  	\end{enumerate}  	 
  \end{fact}

  
 
  
  
  \subsection*{Isotopies}
  
 
  We denote by $I$ the interval $I=[-1,1]$. 
  Given manifolds $M,N$ an isotopy of $M$ in $N$ is a smooth map
  $H: M\times I\to M$ such that for any $t\in I$ the map $H_{t}:=H(-,t):M\to N$ is an embedding. 
  Given $g ,g'\in\D(M)$ a diffeotopy from $g$ to $g'$ is a smooth map 
  is an isotopy of $M$ in itself such that $G_{t}\in\D(M)$ at every point in time. We will also say that $G$ is a diffeotopy of $g$. If we replace $I$ by some other interval we speak of a generalized isotopy.
  We will refer to $I$ as the time component of $M\times I$ and denote by the associated partial differential $\partial_{t}$ 
 
  For convenience, we will sometimes use the notation $(G_{t})_{t\in I}$ to exhibit certain diffeotopies (isotopies), keeping in mind the condition of global differentiability.  
 
  %\footnote{Some authors use the terms "isotopy" and "ambient isotopy" or "diffeotopy" and "ambient diffeotopy" to denote what we refer to as "isotopy" and "diffeotopy".}
  By the support of an isotopy $H$ of $M$ in $N$ we mean the set $supp(H):=\{p\in M\,|\,\exists t\in I\,\,H_{t}(p)\neq H_{-1}(p)\}$. It is easy to check that $supp(H)$ is an open set and that
  in case $H$ is a diffeotopy of $Id_{M}$, then $H(supp(H)\times I)=supp(H)$. 
  We say that an isotopy $H$ of $M$ in $N$ is compactly supported if there is some compact set $K\subseteq M$ such that 
  $supp(M)\subseteq K$. 
  
  \begin{comment}
  	$G(p,t)$ for all $t$ and $p\in M\setminus K$. And we say that an isotopy of $M$ in $N$ is  a compact set if 
  $F(p,t)$ for all $t\in I$ and all $p\in N\setminus K$ for some fixed compact $K$.  
  \end{comment}
   
  We will often use the following: 
  \begin{fact}
  	\label{c: horizontal composition isotopies} If $G$ is a compactly supported diffeotopy, so is $(G_{t}^{-1})_{t\in I}$.
  	If $G^{1}$ and $G^{2}$ are compactly supported isotopies of $M_{1}$ in $M_{2}$ and of $M_{2}$ in $M_{3}$ respectively, then $(G^{2}_{t}\circ G^{1}_{t})_{t\in I}$. 
  \end{fact}
%  As a consequence the set $\D_{c0}(M)\leq\D(M)$, where $\D_{c0}(M)$ is the set of all diffeomorphisms of $M$ isotopic to the identity through a compactly supported diffeotopy.   
  By a (time) reparametrization of an isotopy $G$  we mean an isotopy of the form $G\circ (Id\times\lambda)$, where $\lambda:I\to I$ is an order-preserving smooth bijection. 
  \begin{fact} [see 2.4 in \cite{wall2016differential}]
	  \label{f: concatenation}Given isotopies $G^{1}$ and $G^{2}$
	  of $M$ in $N$ with $G^{1}_{1}=G^{2}_{1}$ up to replacing $G^{i}$ by reparametrizations the map
	  $G:M\times I\to N$ given by $G(p,t)=G^{1}(p,2t+1)$ if $t\leq 0$ and $G(p,t)=G^{1}(p,2t-1)$ if $t\geq 0$ is an isotopy. 
  \end{fact}
  We will refer to $G$ as a concatenation of $G^{1}$ and $G^{2}$, and indicate it by $G^{1}*G^{2}$ (ignoring non-uniqueness). 
 
  We say that an isotopy $F$ of a manifold $M$ in $N$ is covered by a diffeotopy $G$ of $id_{N}$ if $F_{t}=G_{t}\circ F_{-1}$ for all $t\in I$. The following refinement of Thm 2.4.2 in \cite{wall2016differential} is probably well-known. For the sake of completeness we point out the observations needed to extract the additional clauses from the proof there.
  \begin{lemma}
  	\label{l: isotopy extension}Let $V,M$ be smooth manifolds with $\partial M=\emptyset$ and $H$ a compactly supported isotopy of $V$ in $M$. Then $H$ is covered by some compactly supported diffeotopy $G$ of $Id_{M}$.
  	Moreover, if $$\epsilon=\sup\{\,\norm{DH\loc{(p,t)}(\partial_{t})}\,\,|\,\,(p,t)\in V\times I\,\},$$ 
  	then for any $\delta>0$ we can assume:
  	\begin{enumerate}
  		 \item \label{support}$supp(G)\subseteq \nn_{\delta}(H(supp(H)\times I))$
  		 \item  \label{slow speed} $G_{t}\in\V_{(1+t)(\epsilon+\delta)}$ . 
  	\end{enumerate}
    Moreover, if $\partial V=\emptyset$ and $V_{1}\cap V_{2}$ for two embedded manifolds $V_{1},V_{2}\subseteq M$ in general position (defined below), and $H_{t}(V)=V$ for all $t\in I$ (\footnote{Equivalently, $H=\iota\circ G_{V}$ for some compactly supported diffeotopy of $Id_{V}$.}) then we can assume that $G_{t}$ preserves $V_{1}$ and $V_{2}$ for all $t\in I$. 
  \end{lemma} 
  \begin{proof}
    Item (\ref{support}) is immediate by restricting the ambient manifold.

  	Using Whitney extension theorem (\cite{whitney1992analytic}) we might as well assume that $H$ is defined on $M\times\R$, as in the proof of 2.4.2 in in \cite{wall2016differential}, which involves the construction of a vector field $\xi$ on $M\times\R$ with constant time ($\R$) component and which extends the vector field $\chi$, given by $\xi(K(p,t))=DK\loc{(p,t)}(\partial_{t})$, $K:=(H,Id)$ on the submanifold  $V^{*}:=K(V\times \R)\subseteq M\times\R$. 
    
    Since the covering diffeotopy is obtained from integrating $\xi$, so in order to prove the moreover part it suffices to show that we can take the norm of the $TM$-component of 
    $\xi^{M}$ to be bounded by $(\epsilon+\delta)$. One starts with a locally finite covering $\{U_{\lambda}\}_{\lambda\in\Lambda}$ of $V^{*}$ in $M\times\R$ and charts $\{\phi_{\lambda}\}_{\lambda\in\Lambda}$ with $\phi_{\lambda}(U_{\lambda})=B^{n}\times(-1,1)$ and $\phi_{\lambda}(K(V\times\R)\cap U)=B^{n}\times\{0\}$. Notice that without loss of generality $\bar{U}_{\lambda}$ is compact.
    
    \begin{comment}
	    Notice that by the inverse function theorem the charts $\phi_{\lambda}$ above can be taken to be of the form
	    $(\phi,\pi_{\R})\circ K^{-1}_{\restriction U_{\lambda}}$, where $\phi:W_{\lambda}\to\R^{dim(M)}$ is some chart
	    that maps $W\cap V_{\lambda}$ to a set of the form $im(\phi')\cap \{y_{k+1},\dots y_{m}=0\}$  
	    or $\{y_{k}\geq 0,y_{k+1},\dots y_{m}=0\}$.
    \end{comment}
    
    One then uses the charts to define a local extension $\xi^{M}_{\lambda}$ on $U_{\lambda}$ of the $TM$ component $\chi^{M}$ of $\chi$. The nature of the construction implies  
    that $\xi^{M}_{\lambda}$ is $0$ whenever $\chi$ is $0$ on $U_{\lambda}\cap V^{*}$. Local finiteness and the assumption that $\chi^{M}$ is compactly supported implies this is the case for all but finitely many values of $\lambda$ and compact support follows.
    Setting $U_{0}=M\times\R\setminus V^{*}$ one considers a partition of unity $\{\Psi_{\lambda}\}_{\lambda\in\Lambda\cup\{0\}}$ and finally lets $\xi^{M}=\sum_{\lambda\in\Lambda\cup\{0\}}\Psi_{\lambda}\xi_{\lambda}$, where
    $\xi_{\lambda,0}^{M}=0$. 
    
    By multiplying $\Psi_{\lambda}$ by a suitable function depending only on the last coordinate 
    %of the image by $\phi_{\lambda}$ 
    one can easily construct a sequence of partitions of unity: 
    $\mathcal{P}_{n}:=\{\Psi_{\lambda,n}\}_{\lambda\in\Lambda\cup\{0\}}$, $n\in\N$ such that
    $supp(\Psi_{\lambda,n})\subseteq supp(\Psi_{\lambda})\cap\phi^{-1}(B^{n}\times(-\frac{1}{n},\frac{1}{n}))$ for $\lambda\in\Lambda$.   
    For $n$ large enough the diffeotopy obtain by integrating the $\xi_{n}$ constructed from $\mathcal{P}_{n}$ must satisfy (\ref{slow speed}) after restricting the time interval to $I$. 
    Otherwise for some $\delta>0$, some $\lambda\in\Lambda$ and all $n$ 
    there is $p_{n}\in supp(\Psi_{\lambda})$ with $\norm{\xi_{\lambda}(p_{n})}>\epsilon+\delta$
    then after extracting a subsequence we obtain a contradiction.

    The last claim it suffices to guarantee $\xi^{M}(p,t)\in T_{p}V_{i}\subseteq T_{(p,t)}M\times I$.  
    This follows from the fact that $V$ is covered by charts $\{\phi_{\lambda}'\}_{\lambda\in\Lambda}$ mapping each $V_{i}$ to a linear subspace (Lem III, 3.1 in \cite{golubitsky2012stable}) and that by the inverse function theorem we may assume the charts $\phi_{\lambda}$ above to be of the form $(\phi'_{\lambda},\pi_{\R})\circ (K^{-1})_{\restriction U_{\lambda}}$, where $\pi_{\R}:M \times\R\to \R$ is the natural projection and $\phi'_{\lambda}$ a chart on a neighbourhood of $V$.   
  \end{proof}
   \begin{comment}
	   \begin{remark}
	   	\label{r: isotopy extension with boundary} Note that the result above works even if $V$ is a manifold with boundary, since local extensions can be found in a chart mapping the intersection with $V$ to a set of the form $\{y_{k+1},\dots y_{m}=0\}$ just as well as in case it maps it to a set of of the form $\{y_{k}\geq 0,y_{k+1},\dots y_{m}=0\}$. 
	   \end{remark}
   \end{comment}
   
  The following can be directly extracted from the proof of Prop 4.4.4 in \cite{wall2016differential}.
  \begin{fact}
    \label{f: close implies isotopic}Let $f:M\to N$ an embedding, where $M$ is compact, possibly with boundary. Then there is some neighbourhood $\mathcal{W}$  of $f$ in the Whitney topology such that for any $f'\in\mathcal{W}$ there is an isotopy $F$ from $f'$ to $f$. Moreover, for $p\in M$ the curve $t\mapsto F_{t}(p)$ is a geodesic of constant speed $\frac{1}{2}d(f(p),f'(p))$. 
  \end{fact}
  Note that this trivially extends to pairs of closed embeddings on not necessarily compact manifolds provided they agree on the complement of a compact set.
 
  \begin{comment}
	  The following 
	  \begin{lemma}
	  	\label{l: patching isotopies}Let $H$ be a proper isotopy of $N$ in $M$, $\partial M=\emptyset$ such that $(H_{t})_{\restriction N\setminus C}=(H_{-1})_{\restriction N\setminus C}$ on some compact $C\subseteq M$. Then for any neighbourhood $\mathcal{W}$ of $H$ in $\wt(N\times I,M)$ there exists some neighbourhood $\mathcal{W}'$ of $H$ in $\wt(N\times I,M)$ such that for all $H'$ in $\mathcal{W}'$ one can find $H''$ in $\mathcal{W}$ with $H''_{\restriction C\times I}=H_{\restriction C\times I}$ and $H''_{\restriction (N\setminus C')\times I}=H'_{\restriction (N\setminus C')\times I}$ for some compact set $C'$. 
	  \end{lemma}
	   \begin{proof}
	   	 We may assume $C$ is a submanifold $N_{0}$ and pick compact submanifolds $N_{1},N_{2},N_{3}\subseteq N$ so that 
	   	 $N_{i}\subseteq\mathring{N}_{i+1}$, $i=0,1,2$. Write $F^{i}:=N_{i+1}-N_{i}$.  
	     Let $\mathcal{W}^{0}$ be some neighbourhood of $H$ such that for all $H^{0}\in\mathcal{W}^{0}$ 
	     we have $H^{0}(N_{0}\times I)\cap H_{-1}(N\setminus N_{1}))=\emptyset$. By item \ref{} of Fact \ref{f: properties Whitney} there exists some neighbourhood $\V$ of the trivial diffeotopy $G_{t}=Id_{N}$ for all in $\wt(M\times I\to M)$ such that any diffeotopy $G\in\V$ satisfies $(G_{t}\circ H_{t})_{t\in I}\in\mathcal{W}$ 
	     
	     \todo{change conclusion} 
	    \end{proof}
  \end{comment}
    %  The following follows from Theorem 5.1 in \cite{edwards1971deformations} (see the note after the statement of said theorem):
  %  \begin{fact}
  %  \label{f: local contractibility}Suppose that $Y$ is a compact manifold. Then  $\H_{0}(Y,\partial Y)$ is open with respect to the restriction of the compact-open topology to $\H(Y,\partial Y)$.
  %  \end{fact}
  
  %  For a finite $\Q\subseteq int(Y)$ write $Y_{\Q}:=Y\setminus\Q$. We can identify $\H_{0}(Y,\Q\cup\partial Y)$ with $\H_{0}(Y_{\Q},\partial Y)$ ($\partial Y_{\Q}=\partial Y$). \\
  
  \subsection*{Embedded balls}
        
        \newcommand{\cmp}[2]{\mathcal{C}_{#1}(#2)} 
        For $R>0$ and some integer $k>0$ we write: 
        $$
        B^{k}(R)=\{\underline{x}\in\R^{k}\,|\,\norm{x}<R\} \quad \bar{B}^{k}=\{\underline{x}\in\R^{k}\,|\,\norm{x}\leq R\}.
        $$
        We abbreviate $B^{k}(1)$ by $B^{k}$.
        %and $B^{k}(R)\setminus\bar{B}(k,r)$ by $B^{k}(r,R)$ for $0<r<R$ \todo{check closure}. 
        By a $k$-ball $D$ in a manifold without boundary $M$ (alt. an embedded $k$-ball) we intend a smooth embedding (or its image, by abuse of notation) of $\bar{B}^{k}$ in $M$. By the interior $\mathring{D}$ of $D$ we mean its restriction to $B^{k}$.
        By a $k$-cylinder $D$ in $M$, $k\geq 0$, we mean the smooth embedding of $\bar{B}^{k}\times I$,
        %(\footnote{Which makes sense even if $\bar{B}^{k}\times I$ is only a manifold with corners.} 
        in $M$ and of $I=\{*\}\times I$ if $k=0$. We define $\mathring{D}$ similarly. 
        
        Applying \cite{wall2016differential}, Thm 2.5.6 iteratively we get:
        \begin{fact}
        \label{f: transitive on disks} For any two families of disjoint $k$-balls  $\{D_{i}\}_{i=1}^{k},\{D'_{i}\}_{i=1}^{k}$ in some connected manifold $M$ and any collection of diffeomorphisms $h_{i}:D_{i}\cong D'_{i}$, which we assume to be orientation-preserving if $M$ is orientable, there exists $h\in \D_{c0}(M)$ extending all $h_{i}$ simultaneously.
        \end{fact}
        %Using the corner smoothing argument in 2.6.2 of \cite{wall2016differential} one can also show.

             
      \subsection*{Embedded simplicial complexes} 
      By an embedded simplicial complex $\Delta$ in $M$ we mean an embedding of the geometric realization of a simplicial complex into $M$ which is smooth on every simplex with respect to its affine structure, 
      although for the most part we will think of it merely as a family $\{\Delta^{k}\}_{k\geq 0}$ of subsets of $M$, where $\Delta^{k}$ stands for the collection of all $k$-simplices of $\Delta$. We also write $\Delta^{(k)}$ for the $k$-skeleton of $\Delta$ or $\bigcup_{l\geq k}\Delta^{l}$.  
       We write $\rl{\Delta}$ for the collection of points in the image of $\Delta$
      By Whitney's extension theorem \cite{whitney1992analytic} and the inverse function theorem one can always assume 
      each simplex $\sigma\in\Delta^{K}$ can be seen as the image by some smooth embedding $\phi:\bar{B}^{k}\to M$ 
      of some affine simplex $\tau\subseteq B^{k}$. 
            
       If $\Delta$ is a $(m-1)$-dimensional embedded simplicial complex in an $m$-manifold $M$ and $\gg$ some subgroup of its diffeomorphism group, we denote by $\PS{\gg}{\Delta}$ the subgroup of $\gg$ consisting of all those elements 
       $g\in\gg$ which are the identity on $\rl{\Delta}$ and some neighbourhood of $\rl{\Delta^{(m-2)}}$ in 
       $M$ and by $\SS{\gg}{\Delta}$ the set of all elements for which there exists some diffeotopy 
       $G: M\times I\to M$ from $Id_{M}$ to $G_{1}=g$ such that for all $t\in I$:  
        \begin{itemize}
        \item $G_{t}$ preserves each simplex of $\Delta$ set-wise 
        \item $G_{t}$ is the identity on some neighbourhood of $\rl{\Delta^{(m-2)}}$ in $M$ 
        \end{itemize}
        
     \subsection*{Triangulations and spines}
      When the embedding is homeomorphism, we will refer to the embedded simplicial complex as a triangulation of $M$. An account of the fundamentals of the theory of triangulations of smooth manifolds can be found in \cite{munkres2016elementary} (see also e.g. \cite{cairns1961simple} for existence). 
      \begin{fact}
      	\label{f: existence of triangulations}Every smooth manifold admits a triangulation, which can be taken to be finite in case the manifold is compact and for any given $\epsilon>0$ its simplices can be assumed to have diameter uniformly bounded by $\epsilon$. 
      \end{fact}   
     
     \newcommand{\tr}[0]{\Upsilon}
     
     Given a triangulation $\tr$ of an $m$-manifold with boundary we let $\mathcal{G}(\tr)$ be the graph which has $\tr^{m}$ as a set of vertices and an edge between $\sigma$ and $\sigma'$ whenever $\sigma$ and $\sigma'$ share an $(m-1)$-dimensional face.  
     
     \begin{definition}
	     \label{d: retracting forest} By a retracting forest for $\tr$ we mean a finite collection $\mathcal{F}=\{(\Gamma_{i},\tau_{i})\}_{i=1}^{r}$,
	     where $\{\Gamma_{i}\}_{i=1}^{r}$ is a collection of disjoint subforests of $\mathcal{G}(\tr)$ and $\tau_{i}$ some $(m-1)$-dimensional lying on $\partial M$ and which is a face of one of the simplices in $V(\Gamma_{i})$. 
	     We recycle notation, letting $\rl{\Gamma_{i}}:=\bigcup_{\sigma\in V(\Gamma_{i})}\sigma$ and $\rl{\mathcal{F}}:=\bigcup_{i=1}^{r}\rl{\Gamma_{i}}$.
	     
	     We let the $\mathcal{F}$-spine $\Sp(\mathcal{F})$ be the $(m-1)$-dimensional simplicial subcomplex of $\tr$ consisting all simplices of dimension at most $m-2$ contained in some $\sigma\in\bigcup_{i=1}^{r}V(\Gamma_{i})$ and all those $(m-1)$-simplices contained in simplices from $\mathcal{F}$ which are not in $\{\tau_{i}\}_{i=1}^{r}$ and nor dual to an edge in one of the $\Gamma_{i}$. 
	     
	     We define the thick $\mathcal{F}$-spine or $\Tsp(\mathcal{F})$ as the union of $\Sp(\mathcal{F})$ and the subcomplex spanned by all the $m$-dimensional simplices not in $\mathcal{F}$.
     \end{definition}
     
     From the proof of Theorem 2.1 in \cite{whitehead1961immersion} one can extract the following:
     \begin{fact}
     	\label{f: spine} Let $N$ be a smooth manifold with non-empty boundary, $\tr$ some triangulation of $N$ and
     	$\mathcal{F}=\{\Gamma_{i}\}_{i=1}^{r}$ a retracting forest for $\tr$ and $W$ be a neighbourhood of 
     	$\rl{\Tsp(\mathcal{F})}$. 
     	Then there exists some open set $V$, $\rl{\Tsp(\mathcal{F})}\subseteq V\subseteq W$ and some isotopy $H$ of $N$ in itself supported on $N\setminus V$ such that $H_{1}(N)\subseteq W$. 
     \end{fact}
     
     \subsection*{Tubular neighbourhoods} 
     
     By an (open) tubular neighbourhood of a submanifold $N\subseteq M$, $dim(M)=m$, $dim(N)=n$ we mean a bundle
     $\pi:E\to N$ with fiber $\bar{B}^{n-m}$ ($B^{n-m}$) together with an embedding $\xi: E\to M$ whose restriction to the zero section $E_{0}$ (\footnote{That is, the image of the map that chooses the center of the ball on each fiber.}) is a homeomorphism onto $N$. We will denote it simply by $(E,\xi)$. An (open) tubular neighbourhood is known to exist for any submanifold $N$. We will write $E_{p}=\pi^{-1}(p)$ for the fiber over $p$ and refer to $\xi(E_{p})$ as a fiber image.
     %We may also refer to $im(\xi)$ as a tubular neighbourhood.
                    
     The following refinement of a particular case of Lem 2.5.2 and Cor 2.5.3 in \cite{wall2016differential} is probably well known, but we sketch the proof for the sake of completeness.
     \begin{lemma}
     	\label{l: tubular neighbourhoods} Let $N\subseteq M$ be codimension $1$ manifold without boundary and $(E,\xi)$,$(E,\xi')$ two open tubular neighbourhoods of $N$ in $M$ with the same domain such that $\xi$ and $\xi'$ agree on $E\setminus\pi^{-1}(K)$ for some compact $K\subseteq N$.
     	Then there exists some isotopy between $\xi$ and $\xi'$ constant on $\pi^{-1}(N\setminus K)\cup E_{0}$. 
     \end{lemma}
     \begin{proof}   	
     A standard partition of unity argument yields a fiber-preserving map $\lambda:E\to E$ which satisfies $\xi'\circ\lambda\subseteq im(\xi)$, restricts to multiplication by a scalar on each fiber and is the identity $\pi^{-1}(N\setminus K)$. The map $\iota:=\xi^{-1}\circ\xi'\circ\lambda$ is also the identity on $\pi^{-1}(N\setminus K)$. As in the proof of 2.5.2 in \cite{wall2016differential} 
     	if we let $H_{t}(p)=\frac{2}{t+1}\cdot\iota(\frac{t+1}{2}\cdot p)$ for $p\in E$ and $t\in I$, then this extends to an isotopy
     	$(H_{t})_{t\in I}$ of $E$ in itself so that
     	$H_{-1}:E\to E$ is multiplication by some positive scalar.
     	There are obvious fiber-preserving isotopies $H'$ from      	
     	$Id_{E}$ to $H_{0}$ and $H''$ from $\lambda$ to $Id_{e}$ constant on $\pi^{-1}(N\setminus K)$ and $(\xi\circ (H'*H))*(\xi'\circ H'')$ satisfies the required conditions.  	
     \end{proof}
     \begin{corollary}
     	 \label{c: supported away} Let $N\subseteq M$ be a codimension $1$ manifold and $g\in\D_{0}(M)$ a map which is the identity on $N$ and on some neighbourhood of $N\setminus K$ for some compact $K\subseteq N$. Then for any 
     	 $\epsilon>0$ there exists some $h\in\V_{\epsilon}$ diffeotopic to the identity by an isotopy supported on 
     	 $\nn_{\epsilon}(K)\setminus N$ and such that $hg$ is the identity on some neighbourhood of $N$.  
     \end{corollary}
     \begin{proof}
        Let $(E,\xi)$ be some open tubular neighbourhood of $N$. By Lemma \ref{l: tubular neighbourhoods} there exists some isotopy $H$ from $E$ to $(E,g\circ\xi)$ which is constant on $E_{0}$ and some neighbourhood of 
        $\pi^{-1}(N\setminus K)$. By compactness there exists some closed subbundle $E'$ of $E$ such that the velocity $t\mapsto H_{t}(p)$ is uniformly bounded by $\frac{\epsilon}{4}$ for $p\in E'$.
        %as well as $H(supp(H)\times I)\subseteq \nn_{\frac{\epsilon}{2}}(K)$.
        
        Let $H'$ the restriction to $E'$. It is easy to find some open subset $W\subseteq N$ 
        such that $H'(supp(H')\times I)\subseteq W$ and $\xi(E')$ is a closed submanifold of $W$.  
        If we let $G$ be the diffeotopy $G$ of $Id_{N}$ given by applying Lemma \ref{l: isotopy extension} to the restriction of $H'$ to $W\cap\xi(E')\subseteq W$ and then extending by the identity, then we may assume $supp(G)\subseteq \nn_{\epsilon}(K)$ and $G_{1}\in\V_{\epsilon}$ and we may take $h=G_{1}^{-1}$. 
     \end{proof}
      
     \subsection*{Transversality}
     We establish some notation, and recall some basic facts. More details can be found in Chapter 4
     of \cite{wall2016differential}.  
     Given a smooth map $f\in C^{\infty}(M,N)$ and a submanifold $L\subseteq N$ we say that $f$ is transverse to $L$ if and only if we have $im(Df\loc{p})+T_{q}L=T_{q}M$ for any point $p\in M$ such that 
     $q:=f(p)\in K$, written as $f\transv K$. Recall that if $f\transv K$ then $f^{-1}(K)$ is a submanifold of $M$ with the same codimension as that of $K$ in $N$.  
     If $N_{1},N_{2}$ are submanifolds, we say that they are in general position if $\iota\transv N_{2}$ for some embedding whose image is $N_{1}$ (\footnote{Note that this relation is actually symmetric.}). Given an embedded simplicial complex $\Delta$ we write $N\transv\Delta$ if each of the simplices of $\Delta$ can be extended to an embedded manifold in general position with respect to $N$.
        
     Given manifolds $N$ and $M$ of dimension $n$ and $m$ respectively the space of $1$-jets from $N$ to $M$, $\jet(N,M)$ is the set of triples $(p,q,\lambda)$, where $p\in M$, $q\in N$ and $\lambda$ belongs to the set of linear maps $\mathcal{L}(T_{p}M,T_{q}N)$. 
     
     \begin{fact}
	     \label{f: jet coordinates} One can give $\jet(M,N)$ the structure of a smooth manifold and in fact a vector bundle over $M\times N$ (with the obvious projection). One does this by constructing, for each pair of local charts $\phi=\pmb{x}$ on $U\subseteq M$ and $\psi=\pmb{y}$ on $V\subseteq N$ 
	     a chart $\Psi(\phi,\psi)$ on the set of points of $\jet(M,N)$ over $U\times V$ with coordinates: 
	     $$\{x_{i},y_{j},\zeta_{i,j}\,|\,1\leq i\leq m, 1\leq j\leq n\},$$ where $\zeta_{i,j}((p,q,\lambda))$
	     is the $\partial_{y_{j}}$-component of the vector $\lambda(\partial_{x_{i}})$ with respect to the base
	     $\{\partial_{y_{i}}\}_{i=1}^{m}$.
     \end{fact}
     
     
     Any $f\in C^{\infty}(M,N)$ induces a map $j^{1}f:M\to N$ given by 
     $p\mapsto (p,f(p),Df\loc{p})$.  
     \begin{fact}
     	\label{f: local jets} The map $j^{1}f$ is smooth and given local coordinates $\pmb{x}$ around $p\in M$ and $\pmb{y}$ around $q=f(p)\in N$ we have:
     	$$
     	Dj^{1}f\loc{\underline{x}}(\partial_{x_{i}})=\sum_{j=1}^{n}\frac{\partial f_{j}}{\partial x_{i}}(\underline{x})\partial_{y_{j}}+\sum_{j=1}^{n}\sum_{l=1}^{n}\frac{\partial^{2}f_{j}}{\partial x_{i}\partial x_{l}}(\underline{x})\partial_{\zeta_{l,j}} 
     	$$ 
     	where $f_{j}$ is the $j$-th component of $f$.
     \end{fact}
     
     We will need the following refinement of the first order approximation to Thom's transversality theorem (\footnote{See Thm 4.5.6 in \cite{wall2016differential} or Ch3, Thm 2.5 in \cite{hirsch2012differential} for the most common form.}): 
      \begin{fact}[\cite{michor2011manifolds}, Thm 6.8, p.55]
      	\label{f: Thom}Let $M,N$ be smooth manifolds and $W\subseteq \jet(M,N)$ a submanifold. Then the set of $f\in C^{\infty}(M,N)$ such that $j^{1}f\transv W$ (in particular the set of $f$ such that $f\transv W$ for a submanifold $W\subseteq N$) is comeager and thus dense in the $\mathcal{D}$-topology on $C^{\infty}(M,N)$.
      \end{fact}
      
      From this and items (\ref{whitney baire}) and (\ref{d infinity}) in Fact \ref{f: properties Whitney} one easily obtains the following:    
      \begin{corollary}
      	\label{c: refined transversality} Let $M,N$ be manifolds, $L\subseteq\mathring{M}$ a codimension $0$ submanifold and 
      	$W\subseteq\jet(M,N)$ a submanifold and $f\in C^{\infty}(M,N)$ be such that
      	 $(j^{1}f)(M\setminus\mathring{L})\cap W=\emptyset$. Then for any neighbourhood $\V$ of $f$ in $\wt(M,N)$ there exists $f'\in\V$ such that $j^{1}f\transv V$ and $f'_{\restriction M\setminus\mathring{L}}=f$. 
      \end{corollary}
            
     \begin{lemma}
      \label{l: wlog transverse} Let $K_{0}\subseteq M$ be a compact submanifold and $\{K_{\lambda}\}_{\lambda\in\Lambda}$, $\lambda\in\Lambda$ a countable collection of submanifolds of $K_{0}$ ($0\in\Lambda$). Then for any finite embedded simplicial complex $\Delta\subseteq N$ embedded simplicial complex and $\epsilon>0$ there exists some $g\in\D_{c0}\cap\V_{\epsilon}$ 
      such that $\Delta\transv g\cdot K_{\lambda}$ (eq. $g^{-1}\cdot\Delta\transv K_{\lambda}$) for all $\lambda\in\Lambda$.
     \end{lemma}
     \begin{proof}
     	By Fact \ref{f: properties Whitney} the collection of $f\in C^{\infty}(K_{0},N)$ such that 
     	$f_{\restriction K_{\Lambda}}$ is an embedding transverse to $\Delta$ for all $\lambda$ is comeager and thus locally dense in $C^{\infty}$. By \ref{f: close implies isotopic} and \ref{l: isotopy extension} any such $f$ sufficiently close to the inclusion extends to $g\in\D_{c0}(M)\cap\V_{\epsilon}$. 
     \end{proof}
         
     \subsection*{Stability}    
       
       A basic account of the different notions of stability for smooth manifolds and the equivalences between them can be found in Chapter $\text{V}$ of \cite{golubitsky2012stable}. % and some sufficient conditions for stability are given in Chapter $\text{III}$.  
        We recall the concept of stability under deformations, discussed in V, 2.1 in \cite{golubitsky2012stable} in a somewhat different language. 
       
       \begin{definition}
	       \label{d: stable under deformations}Let $F\in C^{\infty}(M\times J,N)$, $J$ an open interval. We say that $F$ is locally trivial around $s\in J$ if there are $\epsilon>0$ and generalized diffeotopies 
	       $G^{M}: M\times[-\epsilon,\epsilon]\to M$, $G^{M}_{0}=Id_{M}$ and $G^{N}: N\times[-\epsilon,\epsilon]\to N$, $G^{N}_{0}=id_{N}$ 
	       such that $F_{s+t}=G^{M}_{t}\circ F_{s}\circ G^{N}_{t}$ for all $t\in[-\epsilon,\epsilon]$.  
         We say that $f\in C^{\infty}(M,N)$ is stable under deformations if and only if any $F\in C^{\infty}(M\times J,N)$ as above with $f=F_{s}:=F(-,s)$ is locally trivial around $s$.
         % \footnote{By reparametrizing one easily sees this to be equivalent to the definition in \cite{golubitsky2012stable}, which only involves trivialiy at $0$.} 
       \end{definition}
       
       \begin{remark}
       	\label{r: different basepoint}Note that in the definition of local triviality $t\in[-\epsilon,\epsilon]$ one can write $F_{t}=G^{N}_{t-s}(G^{N}_{-\epsilon})^{-1}\circ F_{s-\epsilon}\circ(G^{M}_{-\epsilon})^{-1}\circ G^{M}_{t-s}$.
       \end{remark}
                    
       \begin{lemma}
       	\label{l: deformation stability} Let $F$ be a compactly supported isotopy of a manifold $M$ in a manifold $N$ such that $F_{t}$ is stable under deformations for all $t$. Then, up to replacing $F$ by a time reparametrization, we can assume there exist compactly supported diffeotopies $G^{N}$ of $Id_{N}$ and $G^{M}$ of $Id_{M}$ such that $F_{t}=G^{N}_{t}\circ F_{-1}\circ G^{M}_{t}$ for all $t\in I$.       	 
        \end{lemma}
       \begin{proof}
       	By inspecting the proof of Prop V, 4.3 in \cite{golubitsky2012stable} one can see that it also applies when $M$ is non-compact case provided $F$ is proper and compactly supported (\footnote{The vector field which solves the local problem can be taken to be zero on charts on which the datum is also zero.}) and that in that case the diffeotopies in Definition \ref{d: stable under deformations} can be assumed to be compactly supported.    	
       	After reparametrizing we might assume that $F$ is defined on $M\times J$ for some open interval $J$ containing $I$. 
       	Compactness and Remark \ref{r: different basepoint} yields $-1=t_{0}<t_{1}\dots <t_{k}=1$ 
       	and for $1\leq i\leq k-1$ compactly supported diffeotopies $(G^{N,i})_{t\in[0,t_{i+1}-t_{i}]}$ from $Id_{N}$ to $g_{i}\in\D_{c0}(M)$ and $(G^{M,i})_{t\in[0,t_{i+1}-t_{i}]}$ from $Id_{M}$ to $h_{i}\in\D_{c0}(N)$ such that 
       	$H_{t_{i}+s}=G^{N,i}_{s-t_{i}}\circ H_{t_{i}}\circ G^{M,i}_{s-t_{i}}$ for $s\in [0,t_{i+1}-t_{i}]$. 
       	For $s\in [t_{i},t_{i+1}]$ consider the (well-defined) diffeomorphisms $G^{M}_{s}:=G^{M,i}_{s-t_{i}}g_{i-1}\cdots g_{0}$ and $G^{N}_{s}:=h_{0}\cdots h_{i-1}G^{N,i}_{s-t_{i}}$. By \ref{f: concatenation} both $(G^{N}_{s})_{s\in I}$ and $(G^{M}_{s})_{s\in I}$ will become diffeotopies after reparametrizing in parallel.
       \end{proof}

       \begin{comment}
	       We say that an smooth embedding $f:M\to N$ has normal crossings if for any distinct points 
	       $p_{1},\dots p_{r}\in M$ that are mapped to the same point $q\in N$ 
	       we have $codim(\bigcap_{i=1}^{r} im(D_{p_{i}}))=\sum_{i\in F}codim(im(D_{p_{i}}))$
	       (for the equivalence with III,3.1 in \cite{golubitsky2012stable}, see 
       \end{comment}
       
       \begin{comment}
	       \begin{fact}
	       	\label{f: normal crossings} Let $f:M\to N$ be a proper embedding and assume that $|f^{-1}(q)|\leq 2$ for all $q\in N$ and that $im(Df_{p_{1}})+im(Df_{p_{2}})=T_{q}N$ whenever $p_{1},p_{2}$ are distinct and $f(p_{1})=f(p_{2})=q$.
	       	Then $f$ is stable under deformations. 
	       \end{fact}
	       \begin{proof}
	       	  It follows from theorems III,3.11 and V,4.3 and in \cite{golubitsky2012stable} that proper immersions with normal crossings are stable under deformations. This applies here (compare Definition III,3.1 in \cite{golubitsky2012stable}) with Lem 4.6.5 in \cite{wall2016differential}).
	       	  \todo{merge with the result in the other section} 
	       \end{proof}
       \end{comment}
     \begin{lemma}
   		\label{l: stability application}Let $M$ an $m$-manifold without boundary, $L$ a closed codimension $1$ submanifold of $M$, $N$ be an $(m-1)$ manifold and $H$ a proper isotopy of $N$ into $M$ with compact support such that 
   		$H_{t}\transv L$ for all $t\in I$. Then, up to replacing $H$ by a time reparametrization, there are compactly supported diffeotopies
   		$G^{M}:M\times I\to M$ and $G^{N}:N \times I\to N$ such that 
   		\begin{itemize}
   			\item $G^{M}_{t}(L)=L$ for all $t\in I$ 
   			\item $G^{M}_{t}\circ H_{t}\circ G^{N}_{t}=H_{-1}$ for all $t\in I$  
   		\end{itemize}
   	  so that in particular $H_{t}^{-1}(L)=G^{N}_{t}(H_{-1}^{-1}(L))$ for all $t\in I$. 
   	\end{lemma}
   	\begin{proof}
   		Let $\iota$ denote the inclusion of $L$ in $M$.  
      Consider the manifold $\hat{N}:=N \coprod L$ and the isotopy $\hat{H}: \hat{N}\times I\to M$ such that 
      $(\hat{H}_{t})_{\restriction L}=\iota$ and $(\hat{H}_{t})_{\restriction N}=H_{t}$.     
      It follows from theorems III,3.11 and V,4.3 and in \cite{golubitsky2012stable} that proper immersions with normal crossings are stable under deformations. This applies to $\hat{H}_{t}$ for all $t\in I$ (compare Definition III,3.1 in \cite{golubitsky2012stable}) with Lem 4.6.5 in \cite{wall2016differential}).
      
      
      By Lemma \ref{l: deformation stability} we may assume the existence of compactly supported diffeotopies $G^{M}:M \times I\to M$ and $G^{\hat{N}}:\hat{N}\times I\to \hat{N}$ such that 
      $G^{\hat{N}}_{-1}=Id_{\hat{N}}$, $G^{\hat{M}}_{-1}=Id_{M}$ and $G^{M}_{t}\circ \hat{H}_{t}\circ G^{\hat{N}}_{t}=\hat{H}_{-1}$ for all $t\in I$. 
      The equality above restricts to the equality $(G^{M}_{t}\circ\iota\circ (G^{\hat{N}}_{t})_{\restriction L}=\iota$ on $L$ for all $t\in I$, which implies that $G^{M}_{t}$ must preserve the image of $\iota$ set-wise.       
      We can now let $G^{N}=G^{\hat{N}}_{\restriction N\times I}$. The final claim is clear from the equality:
      $H_{-1}=G^{M}_{t}\circ H_{t}\circ G^{N}_{t}$ and the fact that $G^{M}_{t}$ preserves $\iota(L)\subseteq M$ for all $t\in I$.   
   	\end{proof}
    
	     \subsection*{The Thom-Boardman stratification}
	     
	     We will recall the bare minimum needed here. A short introduction can be found in chapter VI of \cite{golubitsky2012stable} and a more detailed one in 
	     \cite{arnold2014singularities}.
	     % Ch VI of \cite{golubitsky2012stable}. 
	     Let $M,N$ me manifolds without boundary and $f\in C^{\infty}(M,N)$. 
	     Let $k=\min(dim(M),dim(N))$. For any integer $r\geq 0$ one can consider the collection 
	     $$S_{r}(f)=\{p\in M\,|\,rk(Df\loc{p})=k-r\}\subseteq M.$$
	     If $S_{r}(f)$ is a submanifold of $M$, one may consider for any $s\geq 0$ the collection 
	     $S_{r,s}(f)=S_{s}(f_{\restriction S_{r}(f)})$ and, if $S_{r.s}(f)$ happens to be a submanifold of $S_{r}(f)$, continue the process further. A result of Boardman \cite{boardman1967singularities} states that generically this process can be continued indefinitely.  
	    \begin{fact}
	    	\label{f: boardman}Let $S_{\varnothing}(f)=M$ and $\mathcal{I}$ the collection of all sequences of positive integers including the empty sequence $\varnothing$. There exists some comeager (and hence dense) set $\mathcal{B}\subseteq C^{\infty}(M,N)$ such that 
	    	for any $f\in\mathcal{B}$ and $\underline{i}\in\mathcal{I}$ and $j>0$ we have that $S_{\underline{i}, j}(f)$ is a proper submanifold of $S_{\underline{i}}(f)$, so that $S_{\underline{i}, 0}(f)=S_{\underline{i}}(f)\setminus\bigcup_{j>0}S_{\underline{i}, j}(f)$ is a non-empty open subset of $S_{\underline{i}}(f)$.% (\footnote{In fact the codimension of $S_{\underline{i}}(f)$ is a function of $\underline{i}$, but we shall make no use of this.}).  
	    \end{fact}
      \begin{remark}
      	\label{r: boardman closure} Although for $f\in\mathcal{B}$ the set $S_{\underline{i}}(f)$ is not necessarily closed, 
      	lower semicontinuity of the rank of the differential of a smooth function ensures that 
      	$\bigcup_{l\geq k}S_{\underline{i},l}(f)$ is always closed in $S_{\underline{i}}(f)$.  
      \end{remark}
    
	     We are interested in a very specific case:
	     \begin{corollary}
	     	\label{c: projecting simplicial complexes}Let $f$ be an embedding of a compact manifold $L'$ into a manifold of the form $M\times (0.1)$ and $\pi_{M},\pi_{(0,1)}$ the factor projections and let $L:=L'\setminus\partial L$. Then for any $\epsilon>0$ there is some $g\in\mathcal{V}_{\epsilon}\cap\D_{co}(M\times(0,1))$ such that if we let $h=\pi_{1}\circ g\circ f$, then there exists some sequence $L=L_{0}\supsetneq L_{1}\supsetneq\dots L_{n}\supsetneq L_{n+1}=\emptyset$ where
	     	$L_{i+1}$ is a closed sub-manifold of $L_{i}$ of positive codimension and $h_{\restriction L_{i}\setminus L_{i+1}}$ an immersion.     
	     \end{corollary}
	     \begin{proof}
	        By Fact \ref{f: properties Whitney} there is a neighbourhood $\mathcal{W}$ of $\pi\circ f$ in $\wt^{\infty}(L,M)$ such that for all $h\in\mathcal{W}$ the differential of $h$ has rank at least $dim(L)-1$ at all points and the map $(h,\pi_{2}\circ f):L\to M$ is an embedding. By Facts \ref{f: close implies isotopic} and \ref{l: isotopy extension} we may assume that for any $h\in\mathcal{W}$ the map $(h,\pi_{2}\circ f)$ is of the form 
	        $g\circ f$ for some $g\in\D_{c0}(M)$ and given $\mathcal{V}$ as in the statement we can always find $g\in\mathcal{V}$ provided we choose $\mathcal{W}$ to be small enough. If $\mathcal{B}$ is as in \ref{f: boardman}, then for $h\in\mathcal{W}\cap\mathcal{B}$ we have $S_{\underline{i}}(h)\neq\emptyset$ only if no entry of $\underline{i}$ is greater than one. If we let $L_{k}:=S_{\underline{1}_{k}}$, where $\underline{1}_{k}$ stands for the sequence of $k$ ones, then this implies that the restriction of $h$ to $L_{k}\setminus L_{k+1}$ is an immersion.
	        By Remark \ref{r: boardman closure} $L_{k+1}$ is closed in $L_{k}$.
	     \end{proof}
	      
     

\section{Neighbourhoods of the identity contain fix-point stabilizers of finite embedded simplicial complexes}
  
  \newcommand{\kk}[0]{\mathfrak{H}} 
  \newcommand{\subg}[1]{\langle #1 \rangle}
  \label{s: neighbourhoods are rich}
  For the rest of the paper, with the exceptions of the more general Sections \ref{s: Morse} and \ref{s: isotopy operations}, we fix some $m$-dimensional manifold $M$,
  write $\kk:=\D_{c0}(M)$, some group $\gg$ with $\kk\leq\gg\leq\D(M)$ and some group topology $\T$ on $\gg$ coarser than $\tc$.
  
  We begin with the following observation:
  \begin{observation}
  \label{o: generation compact-open topology} If $\T$ is strictly coarser than $\tc$, then there do not exist $m$-balls $D,D'$ in $M$ with $D\subseteq D'$
  and $\mathcal{V}\in\nd$ such that $g\cdot D\subseteq D'$ for all $g\in \mathcal{V}$.
  \end{observation}
  \begin{proof}
  Indeed, given such $D,D'$, any $p\in M$ and any $\epsilon>0$ small enough applying Fact \ref{f: transitive on disks} two times yields some $h\in\kk$ such that  $p\in h\cdot \mathring{D}\subseteq h\cdot D'\subseteq \nn_{\epsilon}(p)$. Then $g\cdot D_{0}\subseteq \nn_{\epsilon}(p)$ for any $g\in\V^{h^{-1}}$, where $D_{0}=h\cdot D$. It follows easily by compactness that the conjugates of $\V$ by the action of $H$ generate a system of neighbourhoods of $\tc$ at the identity.
  %   while it is immediate from compactness that for any compact set $K$ and open $U$ containing $U$ there are finitely many $p_{1},\dots p_{k}$ and $\epsilon_{1},\dots \epsilon_{k}$ such that $\bigcap_{i=1}^{k}[\bar{B}(p_{i},\epsilon_{i}),B(p_{i},2\epsilon_{i})]\subseteq[K,U]$.
  \end{proof}
   
  \begin{lemma}
  \label{l: mixing disks}Assume that $\T$ is strictly coarser than $\tc$. Let $D,E_{1},\dots E_{k}$ be $m$-balls in $M$ with $E_{j}\nsubseteq D$ for $1\leq j\leq k$. Let also $K\subseteq M$ be a compact subset and $\epsilon$ a positive real. Then there exists $h_{1},\dots h_{d}\in \kk$ fixing $D$ such that for any $1\leq j_{0},j_{1},\dots j_{d}\leq k$ and any connected component $C$ of the complement of $\bigcup_{l=1}^{d}h_{l}\cdot E_{j_{l}}$ either:
  \begin{itemize}
  \item $K\cap C=\emptyset$
  \item $diam(C)<\epsilon$
  \item $C\subseteq \nn_{\epsilon}(D)$
  \end{itemize}
  \end{lemma}
  \begin{proof}
  We may assume that the $E_{j}$ are mutually disjoint and disjoint from $D$. 
  There exists some $m$-ball $D'$ in $M$ with $D\subseteq\mathring{D}'\subseteq D'\subseteq\nn_{\epsilon}(D)$, as well as some compact submanifold $N\subseteq M$ such that $D'\subseteq K\subseteq N$. Let $N'=N\setminus\mathring{D}'$. 
  
  Let $\delta=\min\{\epsilon,d(D,M\setminus \mathring{D}')\}$. 
  By compactness, there exist finite collections of $m$-balls $\{C_{i}\}_{i=1}^{r},\{C_{i}'\}_{i=1}^{r}$ in $M$ 
  such that:
  \begin{itemize}
  	\item $C_{i}\subseteq\mathring{C}'_{i}\subseteq M\setminus \mathring{D}$ 
  	\item $N'\subseteq\bigcup_{i=1}^{r}C_{i}$
  	\item $diam(C'_{i})<\delta$ 
  \end{itemize}
  For $1\leq i\leq r$ choose two families of disjoint $m$-balls $\{D_{i,j}\}_{1\leq j\leq k},\{D_{i,j}'\}_{1\leq j\leq k}$ in  $U_{i} := \mathring{C}'_{i}\setminus C_{i}$ such that
  for any $1\leq j_{1},j_{2}\leq r$ the set $D_{i,j_{1}}\cup D'_{i,j_{2}}$ separates $C_{i}$ from $M\setminus\mathring{C}'_{i}$. This can be done, for instance, by taking a diffeomorphism $\phi: \bar{U}_{i}\to S^{m-1}\times I$, fixing some vector $v_{0}\in S^{1}$, considering the disks 
  $$K_{j}^{\nu}:=\{(v,f_{\nu,j}(v))\,|\,\norm{v+\nu v_{0}}>\delta\} \quad f_{\nu,j}(v)=-\frac{\nu\delta j}{3k}+\frac{\norm{v+\nu v_{0}}}{2}$$ 
  for $1\leq j\leq k$, $\nu\in\{1,-1\}$ and some small $\delta>0$ and then letting $D_{i,j}$ and $D'_{i,j}$ be given by suitable thickenings of  $K_{j}^{1}$ and $K_{j}^{-1}$.
   
  By Fact \ref{f: transitive on disks} there are $h_{i},h'_{i}\in\kk$
  such that $h_{i}\cdot D_{j}=D_{i,j}$, $h'_{i}\cdot D_{j}=D'_{i,j}$. We claim that the set 
  $\{h_{i},h'_{i}\}_{1\leq i\leq r}$ satisfies the desired properties ($d=2r$).  
  Indeed, fix some choice of $j_{i},j'_{i}\in\{1,\dots k\}$, $1\leq i\leq d$ and let $\{\mathcal{C}_{\lambda}\}_{\lambda\in\Lambda}$ be the collection of connected components of $X:=M\setminus\bigcup_{i=1}^{r}(D_{i,j_{i}}\cup D'_{i,j'_{i}})$.
  If $p\in X\cap N'$, then $p\in \mathcal{C}_{\lambda(p)}$, where    
  $diam(\mathcal{C}_{\lambda(p)})<\delta$. If we write $Y :=\bigcup_{p\in N'}\mathcal{C}_{p}$, then any connected component of $ X\setminus Y$ is contained in some connected component of the superset $Z:=\mathring{D}\cup (M\setminus N)$ of $Y\setminus X$, of which one of the latter is contained in $D'$ and the other is disjoint from $K$.
  \end{proof}
   
  \begin{remark}
  	We believe one may take $d=m$ in the lemma to be a function of $m=dim(M)$ only, but we found the technicalities involved to be not worth the price. 
  	% It does not imply (but could be made to rather easily) that there is at most one connected component in the complement intersecting $K$ and having diameter greater than $D$. 
  \end{remark}
  
  Arguing somewhat informally we can also make the following observation:
  \begin{observation}
  	\label{o: intrusive homeos}Let $M$ be a topological manifold and $\T$ some group topology on $\H(M)$ strictly coarser than the compact-open topology. Then for any $0<\epsilon<diam(M)$ and compact $K\subseteq M$, $\mathring{K}\neq\emptyset$ there is no $\V\in\nd$ such that $\H_{c0}(M)\cap\V\subseteq\V_{K,\epsilon}$ . 
  \end{observation} 
  \begin{proof}
  	One starts by observing that the entire argument of Lemma \ref{l: mixing disks} could have been carried in $\H(M)$ 
  	using replacing Fact \ref{f: transitive on disks} with the equivalent statement, which is as a consequence of the annulus theorem (\cite{moise2013geometric},\cite{kirby2010stable},\cite{Quinn1982}). Here we must require the embedded balls to be collared. If $\T$ is a group topology on $\H(M)$ strictly coarser than $\tc$, then for any $\V\in\nd$ 
  	consider $\V_{0}=\V_{0}^{-1}\in\nd$ with $\V_{0}^{3}\subseteq\V$ and some closed collared ball $D$ in $M$ of diameter
  	less than $\epsilon$, where $\V_{\epsilon}\subseteq\V_{0}$. For $K$ and $\epsilon$ as above the analogue of \ref{l: mixing disks} applied to $\V_{0}$ and $D$ easily implies the existence of some collared ball $D$ whose interior $U$ 
  	satisfies $diam(U)>\epsilon$, $U\cap K\in\emptyset$ and such that all elements supported on $U$ are contained in $\V$.
  	In particular, there is one such element does not belong to $\V_{K,\epsilon}$. On the other hand, it must belong to 
  	$\H_{c0}(M)$ by Alexander's trick \cite{alexander1923deformation}.
  \end{proof}
  

 
  \begin{comment}
	  \begin{corollary}
	  \label{c: mixing disks}Let $D,E_{1},\dots E_{k}$ be $m$-balls in $M$.
	  Assume that $E_{j}\nsubseteq D$ for $1\leq j\leq k$ and there exists $\V\in\nd$ such that for all $g\in\V$ there is $1\leq j\leq k$ with $g\cdot E_{j}\cap D=\emptyset$.
	  Then for any $\epsilon>0$ and any compact set $K\subseteq M$ there is $\V_{D}^{K,\epsilon}\in\nd$ such that for any $g\in\V_{D}^{K,\epsilon}$ either:
	  \begin{itemize}
	  \item $g\cdot D\subseteq \nn_{\epsilon}(D)$
	  \item $diam(g\cdot D)<\epsilon$
	  \item $g\cdot D\cap K=\emptyset$
	  \end{itemize}
	  \end{corollary}
	  \begin{proof}
	  Simply let $\mathcal{V}_{D}^{K,\epsilon}=\mathcal{V}^{h_{1}^{-1}}\cap\mathcal{V}^{h_{d}^{-1}}$, where $h_{1},\dots h_{d}$ are as given by Lemma \ref{l: mixing disks} applied to $D,E_{1},\dots E_{k}$, $K$ and $\epsilon$. 
	
	  \end{proof}
	 
  \end{comment}
  
  \begin{lemma}
  \label{l: spread all over}Assume that $\tc$ is strictly coarser than $\tc$. Let $D,E_{1},\dots E_{k}$ be $m$-balls in $M$ and $\V\in\nd$. Then for any $\V\in\nd$ there exists $g\in\V$ such that $g\cdot D\cap E_{i}\neq\emptyset$ for all $1\leq i\leq k$.
  \end{lemma}
  \begin{proof}
  Suppose that $\V\in\nd$ fails to satisfy the property. Up to making $D$ smaller, we may assume that $E_{i}\nsubseteq D$ for all $1\leq i\leq k$. 
  Pick any arbitrary $g^{*}\in\kk\setminus\{1\}$ with $supp(g^{*})\subseteq D$ 
  and let $\V_{0}=\V_{0}^{-1}\in\nd$ be such that $g^{*}\nin\V_{0}^{3}$. Choose some compact set $K\subset M$ and $\epsilon>0$ such that $\V_{K,\epsilon}\subseteq\V_{0}$ and $\nn_{\epsilon}(D)$ is contained in a ball $D'$.
  
  Let $h_{1},\dots h_{d}$ be the elements resulting from applying Lemma \ref{l: mixing disks} to the balls $D,E_{1}\dots E_{k}$, $K$ and $\epsilon$ and let $\V_{1}=\bigcap_{i=1}^{d}\V^{h_{i}^{-1}}$. Consider the intersection $\V_{2}:=\V_{0}\cap\V_{1}$.
  %, where $\V_{D}^{K,\epsilon}$ is given by Corollary \ref{c: mixing disks} applied to $\V'$ and $D,h\cdot E_{i}$.
  Our assumption on $\V$ and the conclusion of Lemma \ref{l: mixing disks} implies that for any given $g\in\V_{1}$ at least one of the following possibilities holds:
  \begin{itemize}
  \item $g\cdot D\subseteq D'$
  \item $diam(g\cdot D)<\epsilon$
  \item $g\cdot D\cap K=\emptyset$
  \end{itemize}
  If $g\in\V_{2}\subseteq\V_{1}$ satisfies the second or third possibility, then $(g^{*})^{g^{-1}}\in\V_{K,\epsilon}\subseteq
  \V_{0}$ so that $g^{*}\in\V_{0}^{3}$, contradicting the choice of $\V_{0}$. Hence the first alternative must always hold, contrary to the assumption that $\T\subsetneq\tc$, by Observation \ref{o: generation compact-open topology}.
  \end{proof}
  
  \begin{definition}
  	We will refer to any group topology $\T$ coarser than the compact open topology which satisfies the conclusion of Lemma \ref{l: spread all over} as intrusive. 
  \end{definition}
  
  \begin{lemma}
  \label{l: nerve} Assume that $\T$ is intrusive. Then for any $\V\in\nd$ and any compact set $L\subseteq M$ there exists some finite embedded simplicial complex $\Delta$ such that $\PS{\gg}{\Delta}\subseteq\V$ and $L\cup\rl{\Delta}$ is contained in the closure of one of the connected component $U_{0}$ of $M\setminus\rl{\Delta}$. 
  %  \V_{\Gamma,\epsilon}
  \end{lemma}
  \begin{proof}
  \newcommand{\trr}[0]{\tr_{0}}	
  
  Take $\V_{0}=\V_{0}^{-1}\in\nd$ with $\V_{0}^{7}\subseteq \V$. 
  Let the compact set $K$ and $\epsilon>0$ be such that $\V_{K,3\epsilon}\subseteq\V_{0}$. Now, pick some $m$-ball $D$ in $M$ with $diam(D)<\epsilon$, some compact $m$-submanifold $N\subseteq M$ containing $K\cup L$ and some triangulation $\tr$ of $N$ in which each simplex has diameter less than $\epsilon$. 
  
  By  %Lemma \ref{l: spread all over} there is
  the assumption that $\T$ is intrusive there is
  $g\in\V_{0}$ such that $g\cdot D\cap\mathring{\sigma}\neq\emptyset$ for each simplex $\sigma\in\tr$. We may also assume $g\cdot\partial D\transv\partial N$ by \ref{l: wlog transverse}.   
  Let $\sigma_{1},\dots\sigma_{r}$ be an enumeration of the set   $$\{\sigma\in\tr^{m}\,|\,\sigma\cap D\neq\emptyset,\,\sigma\nsubseteq g\cdot D\}$$
  and for $1\leq i\leq r$ choose $p_{i}\in (g\cdot \partial D)\cap\mathring{\sigma_{i}}$.
    
  Let $N\subseteq M$ be a compact submanifold such that $g\cdot D\subseteq int(N)$. 
  Pick $\delta>0$ smaller than the minimum of 
  $$\{\epsilon\}\cup\{d(\sigma,\sigma')\,|\,\sigma,\sigma'\in\trr^{m},\,\sigma\cap\sigma'=\emptyset\}\cup\{d(p_{i},\partial\sigma_{i})\}_{i=1}^{r}$$ 
  and let $\trr$ be a triangulation of $N':=N\setminus g\cdot \mathring{D}$ in which every simplex has diameter less than $\delta$. By perturbing $\trr$ slightly (applying a small element of $\D(N)$) we may also assume that for any $\sigma$ in $\trr^{m}$ and $1\leq i\leq r$ we have $\sigma\cap\sigma_{i}\neq\emptyset$ if and only if $\mathring{\sigma}\cap\mathring{\sigma_{i}}\neq\emptyset$. 
   
  Let $\trr$ be some finite triangulation of the manifold with boundary $N'\cdot D$ in which every simplex has diameter less than $\delta$. 
  
  \begin{lemma}
     \label{l: small retracting forest}There is some retracting forest $\mathcal{F}=\{(\Gamma_{i},\tau_{i})\}_{i=1}^{r}$ for $\trr$ (see Definition \ref{d: retracting forest}) with the following properties:
  \begin{itemize}
  	\item $diam(\bigcup_{\sigma\in V(\Gamma_{i})}\sigma)<3\epsilon$ for $1\leq i\leq r$  
  	\item $N\subseteq g\cdot D\cup\bigcup_{i=1}^{r}\bigcup_{\sigma\in V(\Gamma_{i})}\sigma$ 
  \end{itemize}  	
  \end{lemma}
\begin{subproof}
	
   To begin with we choose some $\tau_{i}\in\trr^{m-1}$ contained in $g\cdot \partial D\subseteq\partial N'$ containing $p_{i}$ 
   and let $\hat{\tau}_{i}$ be the top-dimensional simplex containing it. 
   For $1\leq i\leq r$ let $\mathscr{V}_{i}$ be the collection of vertices of $\mathcal{G}(\trr)$ intersecting
   $\sigma_{i}$. 
   
   We start by iteratively constructing sets of vertices $\mathscr{D}_{i}\subseteq\mathscr{V}_{i}$, $1\leq i\leq r$ as follows. Given $\mathscr{D}_{1},\dots\mathscr{D}_{i-1}$ we take as 
   $\mathscr{D}_{i}$ the set of vertices in the connected component of $\mathscr{V}_{i}\setminus\bigcup_{l=1}^{i-1}\mathscr{D}_{l}$ that contain $\hat{\tau}_{i}$ (non-empty by the choice of $\delta$). 

   Let $\mathscr{W}_{i}=\mathscr{V}_{i}\setminus\bigcup_{j=1}^{i-1}\mathscr{V}_{j}$.
   If $\mathcal{C}$ is the set of vertices in one of the connected components of  $\mathscr{W}_{i}\setminus\bigcup_{j=1}^{i-1}\mathscr{D}_{j}$, then there is an edge between some $\sigma\in\mathscr{C}$ and some $\sigma'\in\mathscr{D}_{j}$ where $j<i$ and $\sigma_{i}\cap\sigma_{j}\neq\emptyset$. For there must be a path in the graph spanned by $\mathscr{V}_{i}$ connecting 
   $\mathscr{C}$ to $\hat{\tau}_{i}$ (\footnote{This can be seen as a consequence of the fact that, by transversality, there is an arc in $\sigma_{i}$ connecting some simplex in $\mathcal{C}$ with $p_{i}$ and avoiding $\trr^{m-2}$.}) and by construction such path must 
   contain some $\sigma\in\mathscr{D}_{j}\cap\mathscr{V}_{i}$ for $j<i$, while the choice of $\delta$ guarantees that $\sigma_{i}\cap\sigma_{j}\neq\emptyset$. 
   We assign to $\mathcal{C}$ one of such $j$, $\lambda(\mathcal{C})$ for all choices of $i$ and $\mathcal{C}$.
   The set $\mathscr{D}'_{j}=\mathscr{D}_{j}\cup\bigcup_{\lambda(\mathcal{C})=j}\mathcal{C}$ spans a connected graph from which we can extract a maximal subtree $\Gamma_{i}$.
   
   Clearly $\rl{\tr}\subseteq\bigcup_{i=1}^{r}\rl{\Gamma_{i}}$ and since for $\mathcal{C}$ as above we have $\mathscr{C}\subseteq\nn_{\delta}(\sigma_{i})$ and $diam(\sigma_{i})<\epsilon$, necessarily $diam(\bigcup_{\sigma\in\mathscr{D}'_{j}}\sigma)<2\epsilon+\delta<3\epsilon$. 
\end{subproof}
   
   Let $\Delta=\Sp(\mathcal{F})$ where $\mathcal{F}$ is given by the previous Lemma. 
  \begin{lemma}
  \label{l: out of neighbour}For any neighbourhood $W$ of $\rl{\Delta}$ there exists some $h\in\V_{3\epsilon}\cap\kk$ such that $h\cdot (U_{0}\setminus W)\subseteq g\cdot D$.
  \end{lemma}
  \begin{subproof}
    By Fact \ref{f: spine} there is some isotopy $H$ of $N'$ in itself with $H_{-1}=Id_{N'}$ and
    supported on the complement some neighbourhood $V$ of $\rl{\Tsp(\mathcal{F})}$, $V\subseteq W$
    and such that $H_{1}(\rl{\mathcal{F}})\subseteq W$. 
    Since $diam(|\Gamma_{i}|)<3\epsilon$, the covering isotopy 
    $G$ of $id_{M}$ provided by Lemma \ref{l: isotopy extension} can also be taken to be supported on a disconnected union of open sets of diameter less than $3\delta$, from which it follows that $h:=G_{1}^{-1}\in\V_{3\delta}\subseteq\V_{0}$. 
    And we are done, since $h\cdot (N'\setminus g\cdot D)=h\cdot\rl{\mathcal{F}}\subseteq W$.
  \end{subproof}
 
  To check that $\PS{\gg}{\Delta}\subseteq\V$
  pick an arbitrary $h\in\PS{\gg}{\Delta}$. Corollary \ref{c: supported away} provides some $g_{0}\in \V_{\epsilon}\cap \kk\subseteq
  \V_{0}$ such that $g_{0}h$ is the identity in some neighbourhood $W$ of $\Delta$, while Lemma \ref{l: out of neighbour}  yields $g_{1}\in\V_{0}$ such that
  $g_{1}\cdot (U_{0}\setminus W)\subseteq g\cdot D$.
  We can now write $(g_{0}h)^{g_{1}^{-1}}=h_{0}h_{1}$,
 where $h_{0}$ is supported on $g\cdot D$ and $h_{1}$ is supported on $M\setminus\mathring{U}_{0}$.

 Thus $h_{0}\in\gg$, by our assumption on $\gg$, and thus $h_{1}\in\gg$.
 This also implies that $h_{0}^{g}$ is supported on $D$ and thus $h_{0}^{g}\in\V_{\epsilon}\subseteq\V_{0}$, as $diam(D)<\epsilon$. On the other hand, $h_{1}\in\V_{0}$ since $supp(g)\cap K=\emptyset$. And we are done, since
 $
 h=g_{0}^{-1}h_{0}^{g_{1}}h_{1}^{g_{1}}\in\V_{0}^{7}\subseteq\V.
 $
  
  \end{proof}
  
  \begin{corollary}
  \label{c: control on vertices}If $\T$ is intrusive, then for any $\V\in\nd$, any open $U\subseteq M$ 
  and any finite collection $(K_{\lambda})_{\lambda\in\Lambda}$ of embedded manifolds, $K_{\lambda}\subseteq K_{0}$, $0\in\Lambda$ there exists some embedded simplex
  $\Delta$ such that:
  \begin{itemize}
  	\item $\PS{\gg}{\Delta}\subseteq\V$
  	\item $\Delta\transv K_{\lambda}$ for all $\lambda\in\Lambda$ 
  	\item $\Delta^{(0)}\subseteq U$
  \end{itemize}
  \end{corollary}
  \begin{proof}
  On the one hand, for any finite set $\mathcal{F}$ of points, any $m$-ball $D$ in $M$ and any
  $\W\in\nd$ there is, by Lemma \ref{l: nerve} some finite embedded simplicial complex $\Delta$ such that $\PS{\gg}{\Delta}\subseteq \W$ and $\mathcal{F}\cup B\subseteq\bar{U}_{0}$ for some connected component $U_{0}$ of $M\setminus\rl{\Delta}$. Since $\T\subset\tc$, there exists some $g_{0}\in\W$ such that
  $g_{0}\cdot\mathcal{F}\subseteq U_{0}$ and then some $g_{1}\in\PS{\kk}{\Delta}\subseteq\W$ such that $g_{1}\cdot(g_{0}\cdot\mathcal{F})\subseteq B$ by Fact \ref{f: transitive on disks}. 
  
  On the other hand, for any collection $(K_{\lambda})_{\lambda\in\Lambda}$ as above, any finite embedded simplicial complex $\Delta$ and any finite collection of points $\mathcal{F}$ contained in some open set $U$ there exists some $\delta>0$ such that any $g\cdot\mathcal{F}\subseteq U$ for any $g\in\V_{\delta}$. And by Lemma \ref{l: wlog transverse} for any $\Delta$ and any family $(K_{\lambda})_{\lambda\in\Lambda}$ as in the statement and any $\delta>0$ there is $g\in\V_{\delta}$ such that $K_{\lambda}\transv g\cdot\Delta$ for all $\lambda\in\Lambda$. The result easily follows from all of the above and Lemma \ref{l: nerve} by the continuity of multiplication.
  
  \begin{comment}
	  Let $\V_{0}=\V_{0}^{-1}\in\nd$ be such that $\V_{0}^{5}\subseteq\V$ and let $\Delta$ the embedded simplicial complex obtained from applying Lemma \ref{l: nerve} to $\V_{0}$. 
	  Pick some small ball $B\subseteq U$ and some $g\in\V_{0}$ such that $g\cdot\Delta^{(0)}\subseteq B$. 
	  Let also $g'\in\V_{\delta}$ such that $g'g\cdot\Delta\transv K_{\lambda}$ for all $\lambda$, as per Lemma \ref{l: wlog transverse}. For $\delta>0$ small enough we have $g'\in\V_{0}$ and $g'g\cdot\Delta^{(0)}\subseteq B$
	  and so
	   $\PS{\gg}{g'g\cdot\Delta}\subseteq\V_{0}^{(g'g)^{-1}}\subseteq\V_{0}^{5}\subseteq\V$, as needed.
  \end{comment}
  \end{proof}
  
  \section{Rotating cubes and the density of arc compressions} 
  \label{s: compression moves}
  
   \begin{definition}
    	\label{d: compressions}Given a $(k+1)$-ball (resp. $k$-cylinder) $D$ in $M$, $k+1<m$ and $\eta>0$ we write
    	$\cmp{D}{\eta}$ for the group $\gg[\nn_{\eta}(D)]$.
    \end{definition}
        
    \begin{definition}
    	\label{d: dense compressions}Let $\T$ be a group topology on $\gg$. We say that $\T$ is compressive if 
    	for any $k$-cylinder (equivalently, any $k+1$-disk) $C$ in $M$, $k\in \{0,\dots n-2\}$ and any $\V\in\nd$ there is some $\eta>0$ such that $\cmp{\eta}{C}\subseteq\V$.       	
     	If the former is true for $k=0$ we say that arc compressions are dense in $\T$.   
    \end{definition}
    
    Continuity of multiplication easily implies:
    \begin{observation}
    \label{o: intrusiveness} If arc compressions are dense in $\T$ (with the usual assumptions), then $\T$ is intrusive. 
    \end{observation}
         
  Let $Q :=I^{m}$. There is an obvious cube CW-structure on $Q$ (which can be refined to a simplicial structure) in which all lower-dimensional cells, which we will refer to as faces of $Q$  are cubes themselves.
  The argument below is rather standard, but we include it for the sake of completeness.
  \begin{lemma}
  	\label{l: intersection point} Let $f_{i}:Q\to I,\,\,1\leq i\leq m$ be functions such that
  	$$f_{i}(I^{i-1}\times\{\epsilon\}\times I^{m-i})=\{\epsilon\}$$ for $\epsilon\in\{-1,1\}$. 
  	Then there exists some $\underline{x}\in Q$ such that $f_{i}(\underline{x})=0$ for all $i$.
  \end{lemma}
  \begin{proof}
  	Suppose not. Consider the function $f :=(f_{i})_{i=1}^{m}:Q\to Q$.  By assumption, $f$ is the identity on $Q^{(0)}$.
  	There is an obvious homotopy from $f_{\restriction Q^{(1)}}$ to $Id_{Q^{(1)}}$ which is the identity on $Q^{(0)}$. 
  	By the homotopy extension property, this extends to a homotopy of $f:Q\to Q\setminus\{\underline{0}\}$ that preserves the image of each proper face of $Q$ at every point in time. Iterating this process allows us to modify $f:Q\to Q\setminus\{\underline{0}\}$ within its homotopy class so that $f_{\restriction Q^{(m-1)}}=Id_{Q^{(m-1)}}$. 
  	Composing with a retraction of $Q\setminus \{\underline{0}\}$ onto $Q^{(m-1)}$ we end up with a witness of the triviality of the generator of the homology group $\mathit{H}_{m-1}(S^{m-1})$. A contradiction.       	
  \end{proof}
 
  \begin{comment}
    The following is an easy consequence of the fact that the limit of a convergent sequence whose distance to a closed set tends to $0$ must lie in the compact set itself. 
	  \begin{observation}
	  	\label{o: intersection of of simplices}Let $\Delta\subseteq M$ a finite embedded simplicial complex and $\delta>0$. Then there exists some $\epsilon>0$ such that 
	  	whenever $\{\sigma_{k}\}_{k=1}^{r}$ are simplices of $\Delta$
	  	and $p$ satisfies $d(p,\sigma_{k})<\epsilon$ for all $1\leq k\leq r$, then $d(p,q)<\delta$ for some $q\in\bigcap_{k=1}^{r}\sigma_{k}$.
	  \end{observation}
  \end{comment}
  
    %\begin{definition}
	  %\label{d: smoothed cube}
	  It is clearly possible to endow $Q$ with a smooth structure making it diffeomorphic to $\bar{B}^{m}$ in such a way that 
	  there exists an element of the diffeomorphism group witnessing any homeomorphism of $Q$ given by coordinate permutation. By a smoothed cube we mean a diffeomorphic embedding $\phi:Q\to M$ with respect to this structure. Its faces will be the images of the faces of $Q$. In particular, we will write $F_{i}^{\pm 1}(\phi):=\phi(I^{i-1}\times\{\pm 1\}\times I^{m-i})$. %Since $\phi$ will be clear from the context for the most part we will simply write $F_{i}^{\pm 1}$. 
  %\end{definition}
  
 
  \begin{lemma}
   \label{l: forcing vertices} Let $\phi:Q\cong D\subseteq M$ be a smoothed $m$-cube in $M$. 
  Suppose that $\Delta$ is a finite embedded $(m-1)$-dimensional simplicial complex in general position with respect to each of the faces of $Q$ (i.e. in general position with some extension of each of them to a ball of the same dimension in $\partial D$ ). Then writing $F_{i}^{\nu}:=F_{i}^{\nu}(\phi)$ at least one of the following holds:
  \begin{itemize}
  \item $D\cap\Delta^{(0)}\neq\emptyset$
  \item there exists some $1\leq i\leq n$ and some component $W$ of  $D\setminus\bigcup\Delta$ with 
  $$F_{i}^{1}\cap W\neq\emptyset, \quad F_{i}^{-1}\cap W\neq\emptyset.$$
  \end{itemize}	
  \end{lemma}
  \begin{proof}	
  	Assume that the second alternative does not hold. Let 
  	$U_{i}$ be the interior in $D$ of the closure of all the connected components of 
  	$D\setminus\Delta$ that contain points of $F_{i}^{-1}$. 
  	The transversality condition implies on its own that $F^{-1}_{i}\subseteq U_{i}$
  	and combined with the assumption also implies that $\bar{U}_{i}\cap F^{1}_{i}=\emptyset$ for all $i$. 
  	Clearly, the frontier $B_{i}$ of $U_{i}$ in $D$ is contained in $\rl{\Delta}$.  
  	
  	On the one hand, using the strong version of Urysohn's lemma (\cite{munkrestopology}, Ex 33.5) 
  	to each component of $D\setminus B_{i}$ one obtains continuous functions $f_{i}:D\to I$ such that $f_{i}(F_{i}^{\lambda})=\{\lambda\}$ for $\lambda\in\{1,-1\}$ and $f_{i}^{-1}(0)=B_{i}$.   	
  	Lemma \ref{l: intersection point} applied to $(f_{i})_{i=1}^{m}$ implies $\bigcap_{i=1}^{m}B_{i}\neq\emptyset$.
  	Notice that $\partial D\cap\bigcap_{i=1}^{m}B_{i}=\emptyset$. 
  	
  	On the other we have the following:
  	\begin{observation}
  		\label{l: nice boundary} For each $\sigma\in\Delta$ and each connected component $\mathcal{C}$ of $\sigma\cap \mathring{D}$
  		either $\mathcal{C}\cap B_{i}=\emptyset$ or $\mathcal{C}\subseteq B_{i}$ and thus $\bar{\mathcal{C}}\subseteq B_{i}$. 
  	\end{observation}  	
    \begin{proof}
    	Follows from the fact that any point $p\in\mathring{D}\cap\sigma$ admits a neighbourhood $U$ in $\mathring{D}$ such that $U\setminus\rl{\Delta}$ consists of finitely many components $\mathcal{C}_{1},\dots \mathcal{C}_{l}$ with $\sigma\cap U\subseteq\bar{\mathcal{C}_{l}}$, $1\leq l\leq k$. 
    \end{proof}
    
  	Since $\bigcap_{i=1}^{m}B_{i}\neq\emptyset$, there must be some connected component $\mathcal{C}$ of 
  	$\mathring{D}\cap\sigma$ for some positive-dimensional simplex $\sigma$ of $\Delta$ for which $\mathcal{C}\subseteq\bigcap_{i=1}^{m}B_{i}$. 
  	If $\mathcal{C}$ does not contain any of the vertices of $\sigma$, then necessarily $\bar{\mathcal{C}}\cap\partial D\neq\emptyset$ (a path in $\sigma$ from $\mathcal{C}$ to a vertex of $\sigma$ will be contained in $\mathcal{C}$ until eventually hitting $\partial D$) and thus 
  	$\bar{\mathcal{C}}\cap(F^{1}_{i}\cup F^{-1}_{i})$ for some $1\leq i\leq n$, contradicting that $\bar{\mathcal{C}}\subseteq B_{i}$.
  \end{proof}
  
  Given a smooth arc $\alpha$ and an open set $U$ containing $\alpha$, let $\mathcal{A}_{U}(\alpha)$ be the collection of arcs $\beta$ such that $\alpha(s)=\beta(s)$ for $s\in\{-1,1\}$.
  
  
  \begin{lemma}
  \label{l: many point pushing maps}Let $\alpha$ be an arc and $U$ an open set containing $\alpha$. Then for any $\V\in\nd$ there exists some (smooth) arc $\beta\in\mathcal{A}_{U}(\alpha)$ and some neighbourhood $V$ of $\beta$ such that $\gg[V]\subseteq\V$. 
  \end{lemma}
  \begin{proof}
  We first claim that for every smoothed $m$-cube $\phi:Q\to D$ and every $\V\in\nd$ 
  there exists some finite embedded simplicial complex $\Delta\subseteq M$ such that:
  \begin{itemize}
  	\item  $\partial D\transv\Delta$ 
  	\item  $\PS{\gg}{\Delta}\subseteq\V$,
  	\item  some connected component of $D\setminus\Delta$ intersects both $F^{1}_{1}$ and $F^{-1}_{1}$. 
  \end{itemize}
  
  Otherwise there exists some $\V\in\nd$ such that for all embedded simplices $\Delta$ one of the three conditions above is satisfied. By Fact \ref{f: transitive on disks} for $i\geq 2$ there exists $g_{i}\in\G$ which 
  preserves $D$ set-wise and maps $F_{i}^{\nu}:=F_{i}^{\nu}(\phi)$ onto $F^{\nu}_{1}$ for $\nu\in\{-1,1\}$.   
  
  If we let $\V^{*}:=\V\cap\bigcap_{i=2}^{m}\V_{0}^{g_{i}}$, then for all finite embedded simplicial complexes $\Delta$ that are transverse to $\partial D$ and satisfy $\G[\Delta]\subseteq\V^{*}$ 
  necessarily $F_{i}^{1}$ and $F^{-1}_{i}$ are separated by $\Delta\cap D$ for all $i$. So by Lemma \ref{l: forcing vertices} we always have $\Delta^{0}\cap D\neq\emptyset$, contradicting Corollary \ref{c: control on vertices}.   
  
  To conclude, let $\alpha,U$ and $\V$ be as in the hypotheses. Pick $\V_{0}=\V_{0}^{-1}\in\nd$ with $\V_{0}^{3}\subseteq\V$ and let $\epsilon>0$ be such that $\V_{\epsilon}\subseteq\V_{0}$. It is clear one can find some smoothed $m$-cube 
  $D$ and embedded balls $C_{-1},C_{1}$ of diameter less than $\epsilon$ 
  such that:
  \begin{itemize}
  	\item $\alpha\subseteq D\subseteq U$,
  	\item $\alpha(\nu)\in F_{1}^{\nu}$ for $\nu\in\{-1,1\}$, 
  	\item $F^{\nu}_{1}\subseteq C_{\nu}\subseteq U$ for $\nu\in\{-1,1\}$,
  \end{itemize} 
  where $D\subseteq U$, $diam(F_{1}^{1}),diam(F_{1}^{-1})<\epsilon$ and $\alpha\subseteq D$ 
  goes from a point in $F^{-1}_{1}$ to a point in $F^{1}_{1}$. The claim provides some $\beta$ in $D$ from 
  $F^{-1}_{1}$ to $F^{1}_{1}$ and some neighbourhood $V$ of $\beta$ such that $\gg[V]\subseteq\V_{0}$. Conjugating by a suitable element supported in $C_{-1}\cup C_{1}$ yields the result.  
  \end{proof}
  
  
  \begin{corollary}
  	\label{c: 0-slices}If $m\neq 3$, then arc compressions are dense in any Hausdorff group topology $\T\subsetneq\tc$ on $\G$.
  \end{corollary}
  \begin{proof}
  	 Let $\V\in\nd$ and choose $\V_{0}=\V_{0}^{-1}\in\nd$ such that $\V_{0}^{7}\subseteq\V$. 
  	 Let $\alpha$ be an arc in $M$. By Lemma \ref{c: control on vertices} there exists some finite embedded $(m-1)$-dimensional simplicial complex
  	 $\Delta$ with $\alpha\transv \Delta$, does not contain its endpoints and satisfies  
  	 $\PS{\gg}{\Delta}\subseteq\V_{0}$. 
  	 The intersection of $\alpha$ with $\Delta$ consists of finitely many points $p_{1},\dots p_{r}$, appearing on that order on $\alpha$, where $p(t_{i})=t_{i}$. Extend this notation by letting $t_{0}=-1$ and $t_{r+1}=1$. 	   	 
  	 Condition $\alpha\transv \Delta$ and the inverse function theorem implies the existence of a small neighbourhood $U$ of $\alpha$ for which a diffeomorphism $\phi: U\cong (0,1)\times B^{m-1}$ mapping $U\cap|\Delta^{(m-2)}|=\emptyset$ to a union of finitely many sets of the form $\{t\}\times B^{m-1}$ and $\alpha$ into the set $(0,1)\times\{\underline{0}\}$. 
 	 
  	 We claim that there is an arc $\beta\in\mathcal{A}_{U}(\alpha)$ intersecting $\Delta$ only at the points 
  	 $p_{1},\dots p_{r}$ in that order and some element $h\in\V_{0}$ mapping an initial subarc of 
  	 $\beta$ in $U\setminus\rl{\Delta}$ onto $\beta$. 	   
  	 Take $\V_{1}=\V_{1}^{-1}\in\nd$ such that $\V_{1}^{2r}\subseteq\V_{0}$ and let $\delta>0$ be such that $\V_{\delta}\subseteq\V_{1}$. For each $1\leq i\leq r$ let $q_{i}=\alpha(s_{i}),q'_{i}=\alpha(s'_{i})$, where 
  	 $s_{i}\in(t_{i-1},t_{i})$ and $s'_{i}\in(t_{i},t_{i+1})$ and $q_{i},p_{i},q'_{i}$ are in the interior of an $m$-ball of diameter less than $\delta$. 
  	 Using Lemma \ref{l: many point pushing maps} there exists arcs in $U\setminus\rl{\Delta}$ 
  	 $\gamma_{0}$ from $p_{0}$ to $q_{1}$, $\gamma_{i}$ from $q'_{i}$ to $q_{i+1}$ for $1\leq i\leq r$ and 
  	 $\gamma_{r}$ from $q'_{r}$ to $p_{r+1}$ and neighbourhoods $W_{i}$ of $\gamma_{i}$ such that
  	 $\kk[W_{i}]\subseteq\V_{1}$.  
  	 Using Fact \ref{f: transitive on disks} it is easy to find $h_{0}\in \kk[W_{i}]$ mapping an initial subarc of $\gamma_{0}$ onto $\gamma'_{0}$, then $h'_{0}\in\V_{\delta}$ mapping $\gamma_{0}$ onto 
  	 some arc from $p_{0}$ to $q_{1}$ properly containing a sub-arc of $\alpha$ around $p_{0}$ outside of $\bigcup_{j=1}^{r}W_{i}$, then $h_{1}\in\kk[W_{1}]$ and so on.
  	 The product $h=h_{r+1}h'_{r}h_{r}\cdots h'_{0}h_{0}$ will map an initial subsegment of $\beta$ onto $\beta$ for $\beta$ the suitable concatenation of the previous arcs.
  	 
  	 To conclude, we use the $dim(M)\neq 3$ and the well known fact that there are no knots in dimension $4$ (see \cite{rolfsen2003knots}) to find some $g\in\PS{\gg}{\Delta}$ fixing the endpoints of $\alpha$ such that $\beta=g\cdot\alpha$. Then $h^{g}$ maps some initial subarc of $\alpha$ which does not intersect 
  	 $\Delta$ onto $\alpha$. Then 
  	 $$
  	 (h^{g}\cdot\Delta)\cap\alpha=(g^{-1}h\cdot\Delta)\cap\alpha=g^{-1}h\cdot(\Delta\cap(g\cdot \alpha))=g^{-1}h\cdot(\Delta\cap h^{-1}\cdot\beta)=\emptyset
  	 $$
  	 and we are done, since then for some neighbourhood $W$ of $\alpha$ we have 
  	 $$\kk[W]\subseteq\PS{\gg}{\Delta}^{(h^{g})^{-1}}\subseteq\V_{0}^{7}\subseteq\V.$$
  	 
  \end{proof}



  \section{A Morse-like lemma for isotopies and embedded hypersurfaces}
  \label{s: Morse}
  The following two sections present general results in which $\T$ and $\gg$ play no role. 
  ´  \begin{definition}
  	By a nice $m$-triple $(N,M,L)$ we mean a triple of smooth manifolds of dimensions $m-1$, $m$ and $m-1$ respectively 
  	where $L\subseteq M$ is a closed submanifold. 
  \end{definition}
  
  \begin{definition}
  	\label{d: compatible isotopy}Let $(N,M,L)$ be a nice $m$-triple and $H: N\times I\to M$ a smooth isotopy from a smooth $(m-1)$-dimensional submanifold $N$ in $M$. We say that $H$ is $L$-compatible if it is proper, has compact support and there are $-1<t_{0}<t_{1}<\dots t_{r}<1$ and points $p_{0},p_{1},\dots p_{k}\in N$ (the latter not necessarily distinct) such that:
  	\begin{itemize}
  		%\item \label{almost always transverse}$H_{t}\transv L$ for any $t\in I\setminus\{t_{l}\}_{l=0}^{r}$,
  		\item \label{kissing points}$H(p,t)\in L$ and $T_{H(p,t)}L\subseteq im(DH\loc{(p,t)})$ if and only if $(p,t)\in\{(p_{i},t_{i})\}_{i=0}^{r}$ and so, in particular, $H_{t}\transv L$ if $t\in I\setminus\{t_{i}\}_{i=0}^{r}$ 
  		\item \label{local isotopy model} for any $0\leq i\leq m$ there is some $\delta_{i}>0$ and a system of local coordinates $\phi_{i}=\pmb{x}:U_{i}\to\R^{m-1}$ around $p_{i}$ and 
  		$\psi_{i}:=\pmb{y}:V_{i}\to\R^{m}$ around $H(p_{i},t_{i})\in M$ with respect to which $H$ can be written as: 
  		$$
  		(\underline{x},t)\mapsto (\underline{x},t-t_{i}+\sum_{i=0}^{r}\epsilon_{i}x_{i}^{2})
  		$$ 
  		for $t\in (t-\delta_{i},t+\delta_{i})$ and some constants $\epsilon_{i}\in\{\pm 1\}$. 
  	\end{itemize}
    In the situation above we refer to each $t_{i}$ as a \emph{tangency time} and to  $(p_{i},t_{i})$ as a \emph{tangency point} of $H$ and we let 
    the \emph{dimension} of the tangency point be the number of values of $i$ such that $\epsilon_{i}=1$. It is easy to see that this does not depend on the chart.     
    By the \emph{signature} of $H$ we mean the sequence of indices of the tangency points in $H$, arranged in temporal succession.
  \end{definition}
  
  \begin{observation}
  	\label{o: tangency not perturbed}For any $L$-compatible isotopy, by restriction and reparametrizion of the charts one may always assume $\phi_{i}$ to be a homeomorphism between $U_{i}$ and  $B^{m-1}$ and $\psi_{i}$ a homeomorphism between $V_{i}$ and $B^{m-1}\times(-1-\delta_{i},1+\delta_{i})$ in the definition above. Using the fact that each $H_{t}$ is a proper embedding for all $t$ a standard argument by extraction of convergent subsequences also shows one can assume  
  	$H^{-1}(V_{i})=U_{i}\times(t_{i}-\delta_{i},t_{i}+\delta_{i})$.  
  \end{observation}
  
  \begin{lemma}
  	\label{l: morse for isotopies}Let $(N,M,L)$ be a nice $m$-triple and $H: N\times I\to M$ a smooth proper isotopy with compact support of $N$ in $M$ 
  	such that for some compact set $C\subseteq N$ we have $H(p,t)\notin L$ for any $(p,t)\in(N\setminus C)\times I$.
  	Then for any neighbourhood $\mathcal{W}$ of $H$ in $\wt(N\times I,M)$ there exists some $L$-compatible isotopy $H'\in \mathcal{W}$.
  \end{lemma}
  \begin{proof}	
  	\newcommand{\pl}[0]{P_{L}}
  	\newcommand{\jj}[0]{J}
  	Using Whitney extension theorem we may assume $H$ is defined on $N\times\jj$ where $\jj$ is an open interval containing $I$ (\footnote{It is possible that the extension $\bar{H}$ given by the theorem does not satisfy $H^{-1}(L)\subseteq (N\setminus C)\times J$ for some compact set on the nose, but using the fact that $\bar{H}^{-1}(L)$ is closed one can always find a neighbourhood $U$ of $N\times I$ and a diffeomorphism $\theta:U\to N\times J'$, $\theta_{N\times I}=Id_{N\times I}$ such that the property holds for $\bar{H}\circ\theta$.}). 
  	Consider the subset of $\jet(N\times J,M)$ given by
  	$$
  	\pl:=\{\,((p,t),q,\lambda)\,|\,q\in L,\,\,\lambda(T_{(p,t)}(N\times\{t\}))\subseteq T_{q}L\,\}
  	$$ 
  	
  	\begin{observation}
  		\label{o: subspace of jets}
  		Let $((p_{0},t_{0}),q_{0})\in(N\times J)\times M$, $\psi=\pmb{y}:V\to\R^{m}$ and be a chart around $q_{0}$ such that 
  		$$\phi(V\cap L)=\phi(V)\cap\{y_{m}=0\}$$
  		and $\phi=\pmb{x}:U\to\R^{m-1}$ a chart around $p_{0}$.
  		
  		Let $\jet(\phi\times Id,\psi)$ be the chart on 
  		$\pi_{\jet}^{-1}(U\times J\times V)$ with coordinates:
  		$$(x_{1},\dots x_{m-1},t,y_{1},\dots y_{m},\zeta_{i,j})_{1\leq i,j\leq n},$$
  		as described in \ref{f: jet coordinates}, where $t$ plays the role of $x_{m}$ in the definition of $\zeta_{m,i}$.
  		 
  		Then, with respect to this chart the set $\pl\cap\pi_{\jet}^{-1}(U\times V)$ is described by the linear equations $$y_{m}=0, \quad \zeta_{i,n}=0, \quad 1\leq i\leq m-1.$$
  		 In particular, $\pl$ is a submanifold of $\jet(N\times J,M)$ of codimension $m$.
  	\end{observation}

    	
  	By Corollary \ref{f: Thom} for any neighbourhood $\mathcal{V}$ of $H$ in the Whitney topology there exists some $H'\in\mathcal{V}$ such that $j^{1}H'\transv \pl$ and we may assume that $H=H'$ outside $L\times J$ for some compact set $L\subseteq N$. Now $\mathcal{F}:=(j^{1}H')^{-1}(\pl)$ is a codimension $m$ submanifold of $N\times J$, i.e. a discrete set of points in $N\times J$.
  	%Our assumption implies that for all $H'$ in some neighbourhood of $H$ in the Whitney topology those points might be assumed to lie in some comapct subset of 
  	%$N\times J$, so that there are only finitely many of them.
  	
     The result now boils down to unpacking the meaning of the transversality condition at a point 
     $(p_{0},t_{0})\in \mathcal{F}$ as is done for Morse functions. Let $q_{0}=H'(p_{0},t_{0})$ and 
     $r_{0}=j^{1}H'(p_{0},t_{0})$.
     Let $J_{0}$ be some interval around $t_{0}$
     $\phi=\mathbf{\underline{x}}: U\to\R^{m-1}$ be a chart around $p_{0}$ mapping $p_{0}$ to $\underline{0}$
     and $\psi=\pmb{y}:V\to\R^{m}$ a chart around $q_{0}$ mapping $q_{0}$ to $\underline{0}$ and 
     with respect to which the set $L$ is given by $\{y_{m}=0\}$ and assume that $H'(U\times J_{0})\subseteq V$.  
     Let
     $$F:=\psi\circ H'\circ(\phi^{-1}\times Id_{J_{0}}):U'\times J_{0}\to V',$$
     where $U'=\phi(U)$, $V'=\psi(V)$. 
     
     %\lu{Recall the elementary fact that a matrix } 
     By Observation \ref{o: subspace of jets} and the expression for the differential of $h:=j^{1}H'$ given by Fact \ref{f: local jets} the condition $h\transv_{r}\pl$ at $(p_{0},t_{0})$ translates into the following two conditions on the $m$-th component $F^{m}$ of $F$ at $(\underline{0},t_{0})$:
     \begin{enumerate}
     	\item \label{first differential}$\frac{\partial F^{m}}{\partial t}\big|_{(\underline{0},t_{0})}\neq 0$ and
     	%, since $\partial_{t}$ is the only element of the base of $T_{p_{0}}N$ whose image can have non-trivial $\partial_{y_{m}}$ component.
     	\item \label{second differential}the matrix $\left(\frac{\partial^{2}F^{m}}{\partial x_{i}\partial x_{j}}\big|_{(\underline{0},t_{0})}\right)_{\substack{1\leq i\leq n\\ 1\leq j\leq n}}$ is non-singular (by \ref{first differential} and the fact that $\frac{\partial F^{m}}{\partial_{x_{i}}}\loc{(\underline{0},t_{0})}=0$ for $1\leq i\leq m-1$)
     	%
     	%since if we let $V\subseteq T_{\underline{0}}\R^{m}$ be the space of vectors whose coefficients with respect to each of the elements  
     	% $\{\partial_{t},\partial_{\zeta_{i,m}}\}_{i=1}^{m-1}$ is zero, then      	
     	% by \ref{first differential} only vectors in the subspace $\subg{\partial_{x_{i}}}_{i=1}^{m-1}\subseteq T_{(\underline{0},t_{0})}\R^{m}$ can be mapped by $D(j^{1}F)_{\underline{0}}$ to a vector whose image in $T_{\underline{0}}\R^{M}/V$ is in the space generated by the images of $\{\partial_{\zeta_{i,m}}\}_{i=1}^{m-1}$.
     \end{enumerate}
    
     \begin{lemma}
	     Up to restricting the charts $\phi$,$\psi$ and postcomposing $\psi$ with some local homeomorphism $\Phi^{*}$ fixing $\{y_{m}=0\}$ in a neighbourhood of $\underline{0}$ the local expression $F$ of $H'$ can be assumed to be of the form $F(\underline{x},t)=F(\underline{x},0)+(\underline{0},t-t_{0})$ around $(p,t_{0})$.	     
     \end{lemma}
      \begin{subproof}
     Notice first that (\ref{first differential}) and (\ref{second differential}) above imply that 
     $F$ is a diffeomorphism from a neighbourhood of $(\underline{0},t_{0})$ in $U'\times J$ to a neighbourhood 
     of $\underline{0}$ in $V'$.%, with $\frac{\partial F_{m}}{\partial t}$ uniformly positive. 
     Thus the formula $\xi(\underline{y}):=\frac{\partial F}{\partial t}\loc{F^{-1}(\underline{y})}$ defines a non-zero vector field locally around $\underline{0}\in\R^{m}$. Integrating yields some smooth map
     $\Psi: V'_{0}\times (-\epsilon,\epsilon)\to \R^{m}$, where $V'_{0}$ is some neighbourhood of $\underline{0}$ and
     $\epsilon>0$ with $(t_{0}-\epsilon,t_{0}+\epsilon)\subseteq J$.
     
     Consider the map 
     $$
     (y_{1},\dots y_{m-1},t)\mapsto \Phi(y_{1},\dots y_{m-1},t):=\Psi(y_{1},\dots y_{m-1},0,t)\in\R^{m}
     $$ 
     defined on some appropriate neighbourhood of $\underline{0}\in\R^{m}$. Notice that $D\Phi\big|_{\underline{0}}$ is non-singular, since if we write $\pi$ and $\pi'$ for the projections onto the first $m-1$ and the last coordinate respectively, then we have:
     \begin{align*}
     D\Phi\big|_{\underline{0}}(\partial_{y_{i}})&=DF_{t_{0}}\big|_{\underline{0}}(\partial_{x_{i}}),\quad 1\leq i\leq m-1 \\
     D(\pi'\circ\Phi)\big|_{\underline{0}}(\partial_{t})&=d\pi'(\xi(\underline{0}))=d\pi'(\frac{\partial F}{\partial t}(\underline{0},t_{0}))=\frac{\partial F_{m}}{\partial t}(\underline{0},t_{0})\neq 0
     %\frac{\partial(\pi_{m}\circ\Phi)}{\partial t}(\underline{0})
     \end{align*}  
     and $F_{t_{0}}$ is an immersion with $D(\pi'\circ F_{t_{0}})\loc{\underline{0}}(\partial_{y_{l}})=0$ for $1\leq l\leq m-1$. Let $\Phi^{*}$ be a local inverse for $\Phi$ in some neighbourhood of $\underline{0}$. Since $\Phi$ fixes the level set $\{y_{m}=0\}$ so does $\Phi^{*}$. 
      
      For  $(\underline{x},t)$ sufficiently close to $(\underline{0},t_{0})$
      some $(\underline{x}',t')$ with $\Psi(\underline{x}',0,t')=F(\underline{x},t)$ exists and 
      \begin{align*}
      	D\Phi\big|_{(\underline{x'},t')}(\partial_{t})=D\Psi\big|_{(\underline{x}',0,t')}(\partial_{t})=
	      D\Psi\big|_{(\underline{x}',0,t')}(\partial_{t})=\xi(\Psi(\underline{x}',0,t'))=DF\big|_{(\underline{x},t)}(\partial_{t})
      \end{align*}
      This implies that $D(\Phi^{*}\circ F)(\partial_{t})=\partial_{t}$ locally around $(\underline{0},t_{0})$, as needed\footnote{here $t$ plays the role of $y_{m}$ in the new local parametrization}.       
     \end{subproof}
     Since the transformation of the Lemma does not affect the validity of conditions (\ref{first differential}),(\ref{second differential}) nor depends on the chart $\phi$ Morse Lemma (\cite{wall2016differential}, Proposition 4.8.2) applies directly to $f$ and we may further assume that
     $f(\underline{x})=\sum_{i=1}^{m}\epsilon_{i}x_{i}^{2}$ for $\underline{x}\in U'$ and constants $\epsilon_{i}\in\{\pm 1\}$, resulting in the desired expression for $F$.
     
     All is left to do is to take care of a couple of minor points. 
     %The first thing is to ensure is that the time coordinates of the points in $\mathcal{F}:=(j^{1}H')^{-1}(\pl)$ are all different and different from $\{-1,1\}$. 
    \begin{lemma}
  		We may assume that $\mathcal{F}\cap N\times\{-1,1\}=\emptyset$ and that $\#(N\times\{t\}\cap\mathcal{F})\leq 1$  for any $t\in(-1,1)$. 
  	\end{lemma}
    \begin{subproof}
    	Given $(p,t)\in \mathcal{F}$ choose $m$-balls $D,D'$ in $M\times J$ with $(p,t)\in D\subseteq\mathring{D}'$ and $\{(p,t)\}=\mathcal{F}\cap D'$. By integrating a suitable vector field supported on $D'$ 
    	we can find a diffeotopy of $Id_{N\times J}$ supported on $D'$ with the property that 
    	$G_{s}$ acts as a translation 
    	$(p,t)\mapsto(p,t+s)$ on $D$. Since $G_{s}$ converges to the identity in the Whitney topology, as $s$ converges to $0$,
    	it follows that $(j^{1}(H'\circ G_{s}))^{-1}(\pl)\cap (D'\setminus\mathring{D})=\emptyset$ for $s$ small enough. 
    	Replacing $H'$ by $H'':=H'\circ G_{s}$ for small enough $s$ thus preserves the conclusion of the first part, and modifies the time component of some point of $\mathcal{F}$ any amount in some open interval and one can also ensure that $H''$ remains in some arbitrary neighbourhood of $H'$ in the Whitney topology. The result easily follows.
    \end{subproof}
    This concludes the proof.    \end{proof}
   
   \section{More on compatible isotopies} 
  \label{s: isotopy operations}
  
  Given a smooth manifold $M$ and a closed set $F\subseteq M$ we write 
  %$\D(N,F)$ for the subgroup of all diffeomorphisms
  $\D_{c0}^{F}(M)$ for the subgroup of all those diffeomorphisms of $M$ diffeotopic to the identity by a compactly-supported isotopy fixing $F$ at any point in time. Given a submanifold $L\subseteq M$ we write $\D_{c0}(M)^{\{L\}}$ for the collection of all
  $g\in\D(M)$ that are diffeotopic to the identity through a compactly supported diffeotopy that preserves $L$ set-wise. We also write $\D_{c0}^{F,\{L_{1},L_{2}\}}(M):=\D_{c0}^{F}\cap\D_{c0}^{\{L_{1}\}}\cap\D_{c0}^{\{L_{2}\}}$ and so on.
  
  
  \begin{observation}
  	\label{l: niceness reversal} Let $(N,M,L)$ be a nice $m$-triple and $H$ an $L$-compatible isotopy of 
  	$N$ in $M$. Then the isotopy $H^{op}$, $H^{op}(p,t)=H(p,-t)$ is itself $L$-compatible, with every tangency point of $H$ of dimension $d$ of $H$ corresponding to an tangency point of dimension $m-d-1$ of $H^{op}$.
  \end{observation}
  
  \begin{definition}
  	Let $(N,M,L)$ be a nice $m$-triple and $H$ an $L$-compatible isotopy of $N$ in $M$. We say that $H$ is \emph{non-destructive} if all the tangency points of $M$ have dimension at most $m-2$. We say that $H$ is \emph{purely destructive} if all the tangency points of $M$ are of dimension $m-1$. 
  \end{definition}


   \begin{definition}
   	Let $\iota:\bar{B}^{d}\times I\cong C\subseteq M$ be a $d$-cylinder and $L,N\subseteq M$ submanifolds in general position. We say that $\iota$ (or simply $C$) bridges $N$ and $L$ if it is transverse to $N$ and $L$ and up to precomposing $\iota$ by some diffeomorphism of $\bar{B}^{d}\times I$ 
   	%(\footnote{In the obvious, generalized sense. $\bar{B}^{d}\times I$ is a manifold with corners.}) 
   	fixing its boundary we can assume that 
   	$\iota^{-1}(L)=\bar{B}^{m-1}\times\{0\}$ and $\iota^{-1}(L)$ is the graph of the function 
   	$\underline{x}\mapsto \norm{\underline{x}}^{2}-\frac{1}{2}$. 
   	
   	%More often than not we will blurry the distinction between $\iota$ and its image $C$.
   	In the situation above $C\cap N$ and $C\cap L$ are embedded closed $d$-balls in $N$ and $L$ respectively 
   	which we will refer to as the $N$-hemisphere and $L$-hemisphere of $C$. We will refer to the topological (closed) $(d+1)$-ball in $C$ bounded by the two hemispheres as the core of $C$ (with respect to $L$ and $N$, but this will be always be clear from context). We will refer to the intersection of the interior of the core with the image of $\underline{0}\times I$ as the axis of $C$ (which coincides with $C$ if $d=0$).   	
   	The intersection of the two hemispheres is a $(d-1)$-embedded sphere in $N\cap L$ which we will refer to as the equator of $C$.    	
   \end{definition}
   
   \begin{comment}
	   \begin{observation}
	   	 \label{retraction core} In the situation above for any neighbourhood $U$ of the core of $C$ there exists some $g\in\D_{c0}^{L,N}(M)$ such that $g\cdot C\subseteq U$.   \marginpar{maybe drop this}
	   \end{observation}
   \end{comment}

  \begin{lemma}
  	 \label{l: non-destructive isotopies} Let $(L,M,N)$ be a nice $m$-triple and $H: N\times I\to M$ an $L$-compatible isotopy with a single tangency point $(p_{0},t_{0})$ of dimension $d$. Then for any $-1\leq t^{*}<t_{0}$ sufficiently close to $t_{0}$ there is an embedded closed $d$-cylinder $D\subseteq M$ and for any $\eta>0$:
  	 \begin{enumerate}[(i)]
  	 	\item \label{tubular}some tubular neighbourhood $(E,\xi^{\eta})$ of $\mathring{D}$ in $\nn_{\eta}(D)$   
  	 	\item \label{diffeotopy}a diffeotopy $\chi^{\eta}$ of $Id_{M}$ supported on $\xi^{\eta} (E)\subseteq\nn_{\eta}(D)$ 
 	 		\item \label{connecting isotopy}some compactly supported isotopy from $h_{\eta}\circ H_{t^{*}}$ to $H_{1}$ with no $L$-tangency point,
 	 		where $h_{\eta}=\chi^{\eta}_{1}$ 	 		 
  	 \end{enumerate}
  	  such that $\chi^{\eta}$ and the isotopy $(\xi^{\eta}_{t}\circ H_{t^{*}})_{t\in I}$ are \emph{compatible}, by which we mean:
  	 \begin{enumerate}[(a)]
  	 		 \item \label{first property}the isotopy $(\chi^{\eta}_{t}\circ H_{t^{*}})_{t\in I}$ has a single $L$-tangency point of the same dimension as the one in $H$ and the same is true if we replace $L$ with any of the $\xi^{\eta}$-fiber images through a point in the axis of $D$ 
         \item \label{second property}for any $t\in I$ and $p\in H_{t^{*}}^{-1}(\xi^{\eta}(E))\subseteq N$
               the image of the map $D(\chi_{t}^{\eta}\circ H_{t^{*}})\loc{p}$ is tangent to the fiber image through $\chi_{t}^{\eta}\circ H_{t^{*}}(p)$ only if $H_{t_{*}}^{-1}(p)$ is in the axis of $D$ 
      \end{enumerate}   	 
  \end{lemma}
   \begin{proof}
   	Let $(p_{0},t_{0})$ be the unique tangency point of $H$.
   	Pick $\delta=\delta_{0}>0$ and charts $\phi:U\to B^{m-1}$ and $\psi:V\to B^{m-1}\times(-1-\delta,1+\delta)$ as in Observation \ref{o: tangency not perturbed}. We may assume that the local expression for 
   	$H_{t}$ is of the form $$\underline{x}\mapsto (\underline{x},f_{t}(\underline{x})):=(\underline{x},t-t_{0}+\sum_{i=1}^{d}x_{i}^{2}-\sum_{i=d+1}^{m-1}x_{i}^{2}).$$
   	Given any $\underline{x}\in B^{m-1}$ write from now on $x^{+}=(x_{i})_{i=1}^{d}$, $x^{-}=(x_{i})_{i=d+1}^{m-1}$.
    Fix any $\delta'\in(0,\min\{\delta,\frac{1}{4}\})$ and let $t^{*}:=t_{0}-\delta'$.  	
   	Let $\rho:\R\mapsto\R$ some smooth function which is symmetric, supported on $I$, non-increasing 
   	on $\R_{\geq 0}$ and agrees with some function of the form $g(s)=D-Cs^{2}$ in some neighbourhood of $0$.
   	
   	For $0<\mu\leq 1$ let $\rho_{\mu}(s)=\rho(\frac{1}{\mu} s)$ and write also $\rho^{l}_{\mu}(\underline{z})=\rho_{\mu}(\norm{\underline{z}})$ for $\underline{z}\in\R^{l}$. 
   	Pick some smooth function $\sigma:\R_{\geq 0}\to\R_{>0}$ supported on  $(-\frac{1}{2},\frac{1}{2})$ such that $\sigma(s^{2})+s^{2}-\delta'\in (0,\frac{1}{2})$ for all $t\in [-\frac{1}{2},\frac{1}{2}]$ and $s\mapsto s^{2}-\delta'+\sigma(s^{2})$ is strictly convex, with positive second derivative on $(-1,1)$. This can be done, for instance, by postcomposing the function $g$ given by $s\mapsto s^{2}-\delta$ with a suitable reparametrization of its codomain, taking the difference with $g$ and then using that any symmetric smooth real function can be written as $f(s^{2})$ for some smooth function $f$ (p.108 of \cite{golubitsky2012stable}). 
   	
   	Given $0<\alpha<1$ we now define $\theta_{\alpha}:B^{m-1}\to\R$ by the formula 
   	$$
   	\theta_{\alpha}(\underline{x})=\norm{\underline{x}^{+}}^{2}-\norm{\underline{x}^{-}}^{2}+\rho_{\frac{1}{2}\alpha}^{m-d-1}(\norm{\underline{x}^{-}})\cdot \sigma(\norm{\underline{x}^{+}}^{2})-\delta'. 
   	$$
   	
   	The following follows from the properties of $\rho$ and $\sigma$ specified above and a simple calculation:
   	\begin{observation}
   	\label{o: derivatives}The derivative $\frac{\partial\theta_{\alpha}}{\partial x_{i}}(\underline{x})$ equals $0$ if and only if $x_{i}=0$ and has sign 
   	$\epsilon_{i}\frac{x_{i}}{|x_{i}|}$ otherwise, where $\epsilon_{i}=1$ if $i\leq d$ and $-1$ if $i>d$.
   	The second order derivative $\frac{\partial^{2}\theta_{\alpha}}{\partial x_{i}\partial x_{j}}(\underline{0})$ is
   	$0$ if $i\neq j$, $1$ if $i=j\leq d$ and $-1$ otherwise.
   	\end{observation}
   	 
   	\begin{lemma}
   		 \label{l: Gamma} For any $\lambda\in[-1,1]$ let $L_{\lambda}$ be the submanifold $\{y_{m}=\lambda\}$ of $B^{m-1}\times(-1-\delta,1+\delta)$. The following hold:
   		 \begin{enumerate}
   		 	\item  \label{first bullet} There exists an diffeotopy $G^{\alpha}$ of $B^{m-1}\times(-1-\delta,1+\delta)$ supported on 
   		$B^{d}(\frac{1}{2})\times B^{m-d-1}(\alpha)\times I$ such that:
   		\begin{enumerate}
   			\item $f'=G_{1}\circ f_{t^{*}}$
   			\item \label{tangency condition}the isotopy $(\underline{x},t)\mapsto G_{t}((\underline{x},f(\underline{x})))$ is 
   			$L_{\lambda}$-compatible, with a unique tangency point of type $d$ at $\underline{x}=0$ in case $\lambda\in(-\delta',0)$ 
   		 \end{enumerate}
   		 \item \label{second bullet} For any $t'\in (t_{0},t_{0}+\delta)$ there is an isotopy $F^{'}$ 
   		 %$F^{*}:B^{m-1} \times I\to B^{m-1}\times(t-\delta,t+\delta)$
   		  from $f'$ to $f_{t'}$ such that $F^{'}_{s}$ is transverse to $\{y_{m}=0\}$ for all $s\in I$.   
   		  \end{enumerate}
   	\end{lemma}
   	\begin{subproof} 
   		Notice that by Observation \ref{o: derivatives} for any $1\leq i\leq m-1$ and $t\in(t-\delta,t+\delta)$ the partial derivatives
   		$\frac{\partial f_{t}}{\partial x_{i}}$ and $\frac{\partial \theta_{\alpha}}{\partial x_{i}}$ have the same set of zeroes and 
   		the same sign elsewhere and the same is true for $\frac{\partial^{2}f_{t}}{\partial x_{i}\partial x_{j}}(\underline{0})$ and 
   		$\frac{\partial^{2}\theta_{\alpha}}{\partial x_{i}\partial x_{j}}(\underline{0})$ for $1\leq i, j\leq d-1$. Hence the same applies to any convex combination of $f_{t}$ and $\theta_{\alpha}$.
   		In particular $\underline{0}$ is the only critical point of any such combination.  
   		
   		Now it suffices to consider the isotopies of $B^{m-1}$ in $B^{m-1}\times(-1-\delta,1+\delta)$ given by:
   		\begin{align*}
	   		F^{*}(\underline{x},s)=(\underline{x},\frac{1}{2}((1+s)\theta_{\alpha}(\underline{x})+(1-s)f_{t^{*}}(\underline{x})))\\
        F^{'}(\underline{x},s)=(\underline{x},\frac{1}{2}((1-s)\theta_{\alpha}(\underline{x})+(1+s)f_{t'}(\underline{x}))) 		
   		\end{align*}
   	  Lemma \ref{l: isotopy extension} implies $F'$ is covered  a diffeotopy of $Id_{B^{m-1}\times(-1-\delta,1+\delta)}$ with the right support. The remarks above then imply that $F^{*}$ witnesses (\ref{first bullet})
   	  and also that $F^{'}$ witnesses (\ref{second bullet}), the unique singular value of the second component remaining positive at all times in that case.
   	\end{subproof}
   	If we let $D$ be the $d$-cylinder $\psi^{-1}(\bar{B}^{d}(\frac{1}{2})\times\{\underline{0}\}\times I)$, then for $\alpha$ small enough $B^{d}(\frac{1}{2})\times B^{m-d-1}(\alpha)\times I\subseteq N_{\eta}(D)$. 
   	And if we let $\chi^{\eta}$ be the extension by the identity of the pullback by the chart $\psi$ 
   	of the diffeotopy $G^{\alpha}$ given by Lemma \ref{l: Gamma}, then item (\ref{tangency condition}) with $\lambda=0$ ensures that $\chi^{\eta}:=G^{\alpha}\circ H_{t^{*}}$ has the right $L$-signature and also that the neighbourhood  
   	$\psi^{-1}(\bar{B}^{d}(\frac{1}{2})\times B^{m-d-1}(\alpha)\times (-1,1))$ with the obvious tubular structure satisfies the required properties. 
   	The existence of the isotopy (\ref{connecting isotopy}) is immediate from that of $F'$ by Fact \ref{f: concatenation}.
%   	The concatenation of the extension by the identity of the isotopy $F'$ in the Lemma, concatenated with the restriction of $F$ to the interval $[t',1]$ has no tangency points with $L$ and connects $h_{\eta}\circ F_{t^{*}}$ to $F_{1}$.
   \end{proof}
   
     We record the following straightforward algebraic consequence of Lemma \ref{l: non-destructive isotopies} for later use.
  \begin{corollary}
  	 \label{c: non-destructive isotopies} Let $(N,M,L)$ be a nice $m$-triple and $H$ an $L$-compatible non-destructive isotopy of $N$ in $M$ with $r$ tangency points. Let also $\mathscr{X}\subseteq\D_{c0}(M)$ be a set which for every
  	 $k$-cylinder $D$ in $M$, $k\leq m-2$ contains $\cmp{\eta}{D}$ for some $\eta>0$.
  	 Then there are  
  	 $h_{0},\dots h_{m}\in\mathscr{X}$ as well as elements $g_{L}\in \D_{c0}^{\{L\}}(M)$, $g_{N}\in \D_{c0}(N)$ such that $H_{1}=g_{L}h_{m-1}\cdots h_{0}\circ H_{-1}\circ g_{N}$.        
  \end{corollary}
  \begin{proof}
	  \begin{comment}
		  An easy induction on $r$, where $r=0$ is just Lemma \ref{l: stability application}. 	   	  
		  If $r>0$, let $t_{0},\dots t_{r-1}$ be the tangency times of $H$.
		  As noticed in the proof of Lemma \ref{l: commutation move}, for any $g\in\D(M)$ the $\mathfrak{X}^{g}$ satisfies the same property as $\mathfrak{X}$. Hence, applying Lemma \ref{l: non-destructive isotopies} to the restriction of $H$ to $(t_{r-1},1]$ yields $t^{*}\in(t_{r-1},1]$ such that for any given $g$ 
		  we can write $H_{1}=g_{L}h\circ H_{t^{*}}\circ g_{N}$, where $g_{L}\in\D^{\{L\}}_{c0}$ and $g_{N}\in\D_{c0}(N)$ and $h^{g}\in\mathfrak{X}$. By the induction hypothesis we can write $H_{t^{*}}=g'_{L}h_{r-1}\dots h_{0}H_{-1}g'_{N}$ where $h_{l}\in\mathscr{X}$ for $0\leq l\leq r-1$, $g'_{L}\in\D_{c0}^{\{L\}}(M)$ and $g'_{N}\in\D_{c0}(N)$. 
		  Then $$H_{1}=g''_{L}g'_{L}h_{\eta}^{g'_{L}}h_{r-1}\cdots h_{0}\circ H_{-1}\circ g_{N}'g_{N}''$$
		  and since we could have taken $g=g'_{L}$, we are done.
	  \end{comment}
	   An easy induction on $r$, where $r=0$ is just Lemma \ref{l: stability application}. 	   	  
	  If $r>0$, let $t_{0},\dots t_{r-1}$ be the tangency times of $H$.
	  As noticed before, for any $g\in\D(M)$ continuity of $g$ implies $\mathfrak{X}^{g^{-1}}$ satisfies the same property as $\mathfrak{X}$. Hence, applying Lemma \ref{l: non-destructive isotopies} to the restriction of $H$ to $(t_{r-1},1]$ yields $t^{*}\in(t_{r-1},1]$ such that for any given fixed $g$ 
	  we can write $H_{1}=g_{L}h\circ H_{t^{*}}\circ g_{N}$, where $g_{L}\in\D^{\{L\}}_{c0}$ and $g_{N}\in\D_{c0}(N)$ and $h^{g}\in\mathfrak{X}$. The induction hypothesis yields a similar expression $H_{t^{*}}=g'_{L}h_{r-1}\dots h_{0}H_{-1}g'_{N}$ 
	  %be the expression reswhere $h_{l}\in\mathscr{X}$ for $0\leq l\leq r-1$, $g'_{L}\in\D_{c0}^{\{L\}}(M)$ and $g'_{N}\in\D_{c0}(N)$. 
	  so that $H_{1}=g''_{L}g'_{L}h_{\eta}^{g'_{L}}h_{r-1}\cdots h_{0}\circ H_{-1}\circ g_{N}'g_{N}''$
	  and then we are done, since we could have taken $g=g'_{L}$.  
  \end{proof}
   
    The following is an easy but crucial observation: 
   \begin{lemma}
   	\label{l: conjugation}Let $(N,M,L)$ be a nice $m$-triple and $f: N\to M$ a proper embedding with $f\transv L$. Suppose that we are given a diffeotopy $G$ of $Id_{M}$ such that the isotopy $H:=(G_{t}\circ f)_{t\in I}$ is $L$-compatible and has signature $\sigma$. Let also $G'$ be a diffeotopy of $Id_{M}$ preserving $L$ and $f(N)$ set-wise at all times and let $g=G_{1}'$. Then the isotopy $H' :=(gG_{t}g^{-1}\circ f)_{t\in I}$ of $N$ in $M$ is also $L$-compatible, has the same signature as $H$ and there exists an isotopy $H''$ from $H'_{1}$ to $H_{1}$ with no $L$-tangency points. 
   \end{lemma}
   \begin{proof}
   	 Notice that $g^{-1}\circ f=f\circ\hat{g}^{-1}$ for some $\hat{g}\in\D(N)$. 
   	 If $(p,t)\in N\times I$, clearly $gG_{t}g^{-1}\circ f(p)=gG_{t}\circ f\circ ^{-1}(p)$ is in $L=g(L)$ if and only if $G_{t}\circ f(p)\in L$ and for $(p,t)\in N\times (-1,1)$ for which this holds the same argument at the level of tangent spaces yields $im(D(G_{t}\circ f)\loc{p})=T_{H_{t}(p)}L$ if and only if $im(DH'_{t}\loc{\hat{g}^{-1}p})=T_{H'_{t}(\hat{g}^{-1}(p))}L$, in which case precomposing the charts in Definition \ref{d: compatible isotopy} with $\hat{g}$ and $g^{-1}$ results in local expressions of the exact same form.
   	 \begin{comment}
	   	 Suppose we are given $(p,t)\in N\times (-1,1)$ such that $H'_{t}(p)\in L$. 
	   	 Since $g^{-1}$ preserves $TL$ and $g^{-1}\circ f=f\circ\hat{g}^{-1}$ for some diffeomorphism $\hat{g}$ of $N$, 
	   	 composing with $Dg^{-1}$ on the left we see that condition
	   	 $$im(DH'_{t}\big|_{p})=im(D(g\circ G_{t}\circ g^{-1}\circ f)\big|_{p})\in TL\subseteq TM$$
	 	   takes place precisely when $DH_{t}\big|_{p'}(T_{p'})=T_{q'}L$, where $p'=\hat{g}^{-1}\cdot p$ and $q'=H_{t}(p')$
	 	   and if $\phi:N\supset U\to\R^{m-1}$ and $\psi:M\supset V\to\R^{m}$ are charts as in Definition \ref{d: compatible isotopy} it is clear that the charts $\psi\circ g^{-1}$ and $\phi\circ\hat{g}$ will result for an expression for $H'$  around $(p',t)$ and $q'$ of the exact same form.  	 
   	 \end{comment}
   	 Finally, we can consider the isotopy $(H''_{t})_{t\in I}:=(G'_{-t}G_{1}(G'_{-t})^{-1}\circ f)_{t\in I}$ from 
   	 $H'_{1}=gG_{1}g^{-1}\circ f$ to $G_{1}$ and deduce from $G_{1}\circ f\transv L$ and a similar argument as above that it has no $L$-tangency points.
   \end{proof}

   \begin{lemma}
   	\label{l: commutation move} Let $(N,M,L)$ be a nice $m$-triple and $H$ an $L$-compatible isotopy of $N$ in $M$ with signature $(d_{0},\dots d_{m})$ and assume that the signature of $H$ contains some subword $\omega$ of either of the following two forms:
   	\begin{enumerate}
   		[(a)]
   		\item \label{big step} $(d,d')$ where $d\geq d'+2$ 
   		%\item \label{small step}$(\underbrace{m-1,\dots m-1}_{\text{$n>0$ times}},m-2)$
   		\item \label{small step}$(m-1,\dots m-1,m-2)$
   	\end{enumerate}   	
   	Then there exist $H'$ such that:
   	\begin{enumerate}[(i)]
   		\item $H'_{\nu}=H_{\nu}$ for $\nu\in\{-1,1\}$. 
   		\item The signature of $H'$ is the same as that of $H$ except that the the subword $\omega$ has been replaced by:
       \begin{itemize}
       	\item $(d',d)$ in case (\ref{big step}),
       	\item some word of the form $(f_{1},\dots f_{q},m-2,m-1,\dots m-1)$, where $f_{l}\in\{0,m-2\}$ for all $1\leq l\leq q$
       	in case (\ref{small step}).
       \end{itemize}   		
   	\end{enumerate}  
   \end{lemma}
   \begin{proof}
   	Using \ref{f: concatenation} one may assume the signature of $H$ is precisely $\omega$.
   	For convenience, we will write $\omega=(d_{-r},\dots d_{0},d_{1})$ and $(t_{-r},\dots t_{0},t_{1})$ for the sequence of tangency times.
   	
   	By Lemma \ref{l: non-destructive isotopies} there exists some $t^{*}_{1}\in (t_{0},t_{1})$ and some  $d_{1}$-cylinder $D_{1}$ bridging $H_{t^{*}_{1}}$ and $L$ such that for all $\eta>0$ we may modify the restriction of $H$ to $(t^{*}_{1},1)$ without changing its signature in such a way that for some $s_{1}>t_{1}>t^{*}_{1}$ (the new tangency time $t_{1}$) the restriction of $H$ to $[t^{*}_{1},s_{1}]$ is of the form 
   	$(\chi^{1}_{t}\circ H_{t_{1}^{*}})_{t\in I}$ for some diffeotopy $(\chi^{1}_{t})_{t\in I}$ of $Id_{M}$.
   	
   	\subsection*{Case (\ref{big step})}  Observation \ref{l: niceness reversal} allows us to apply Lemma \ref{l: non-destructive isotopies} to $(H_{-t})_{t\in I}$, which yields some $t^{*}_{0}\in(t_{0}.t^{*}_{1})$ and some $(m-d_{0}-1)$-cylinder $D_{0}$ in $M$ bridging $H_{t^{*}_{0}}(N)$ and $L$ such that for all $\eta>0$ one can modify the restriction of $H$ to $(-1,t^{*}_{0})$ without altering is signature so that for some $s_{0}<t_{0}$ the restriction of $H$ to $(s_{0},t^{*}_{0})$ is a time-reparametrization of 
   	%$(\chi^{0}_{-1}\circ H_{s_{-1}})_{t\in I}$ 
   	$(\chi^{0}_{-t}\circ H_{t^{*}_{0}})_{t\in I}$
   	for some diffeotopy $(\chi^{0}_{t})_{t\in I}$ of $Id_{M}$ supported on $\nn_{\eta}(D_{0})$. 
   	%$\eta$-slice $h_{0}$ over some \lu{complete} $D_{0}$ exists with $H_{t^{*}_{0}}=h_{0}\circ f$ for some embedding $f$ (depending on $\eta$)
   	%isotopic to $H_{-1}$ through an isotopy without $L$-tangency points  	
   	%Write $f^{*}=H_{t^{*}}$, $N^{*}=f^{*}(N)$ and $f^{'}=H_{t^{*}}$, $N^{'}=f^{'}(N)$.

   	By Lemma \ref{l: stability application} the restriction of $H$ to the interval $[t^{*}_{0},t^{*}_{1}]$ 
   	is a time reparametrization of $(G_{t}^{L}\circ H_{t^{*}_{0}}\circ G^{N}_{t})_{t\in I}$ for compactly supported 
   	diffeotopies $G^{L}$ of $Id_{M}$ preserving $L$ and $G^{N}$ of $Id_{N}$.   	
   	Let $g^{L}=G^{L}_{1}$ and $g^{N}=G^{N}_{1}$. We can replace the 
   	restriction of $H$ to $[t_{0}^{*},s_{1}]$ with the concatenation of the following isotopies:
   	
   	\begin{itemize}
   	   \item 	$((g^{L})^{-1}\chi^{1}_{t}g^{L}\circ H_{t_{0}^{*}})_{t\in I}$,
   	which by the proof of Lemma \ref{l: conjugation} is $L$-compatible, with the same signature as $(\chi^{1}_{t}\circ H_{t^{*}_{1}})_{t\in I}$ and
   	  \item $((G^{L}_{-t})^{-1}\circ\chi^{1}_{1}\circ g^{L}\circ H_{t_{0}^{*}}\circ G^{N}_{t})_{t\in I}$, which has no tangency points.
   	\end{itemize} 
   	  Since we can choose $\chi^{1}$ so that $((g^{L})^{-1}\chi^{1}_{t}g^{L})_{t\in I}$ is supported on any given neighbourhood of the cylinder $(g^{L})^{-1}(D_{1})$, we might as well assume that $t^{*}_{0}=t^{*}_{1}:=t^{*}$. Write $f^{*}=H_{t^{*}}$, $N^{*}=f^{*}(N)$, $f'=H_{s_{1}}$, $N'=f'(N)$. 
   
   	Lemma \ref{l: conjugation} allows us to replace the ball $D_{1}$ with its image $D':=g\cdot D_{1}$ by any 
   	$g\in\D_{c0}^{\{L,N^{*}\}}(M)$, which we will use repeatedly. Let $C_{i}$ be the equator of $D_{i}$. Since $C_{1},C_{2}$ are embedded spheres of dimension 
   	$m-d_{0}-2$ and $d_{1}-1$ respectively in the $(m-2)$-manifold $P:=L\cap N^{*}$ and by Corollary \ref{c: refined transversality} and Fact \ref{f: close implies isotopic} there exists $\phi\in\D_{c0}(P)$ such that $\phi(C_{1})\cap C_{2}=\emptyset$. 
   	By the last claim of Lemma \ref{l: isotopy extension} the map $\phi$ extends to an element of $\D_{c0}^{\{N^{*},L\}}(M)$. We can thus assume that $C_{1}\cap C_{2}=\emptyset$.
   	
   	For $Z\in\{L,N^{*}\}$ let $E^{Z}_{i}$ be the $Z$-hemisphere of $E_{i}$.
   	The sum of the dimensions of $E^{Z}_{0}$ and $E^{Z}_{1}$ is $m-d_{0}-1+d_{1}<m-1$.
   	Since the boundaries of $E^{Z}_{0}$ and $E^{Z}_{1}$ are disjoint, we can find some $\phi_{Z}\in\D_{c0}^{P}(Z)$ such that 
   	$\phi_{Z}(g_{0}(E^{Z}_{1}))\cap E^{Z}_{0}=\emptyset$ for $Z\in\{L,N^{*}\}$. By Lemma \ref{l: isotopy extension} we can extend 
   	$\phi_{Z}$ to some element $g_{Z}\in\D_{c0}^{\{Z\},Z'}(M)$ where $\{Z,Z'\}=\{N^{*},L\}$. After applying $g_{L}g_{N^{*}}$ we can assume that
   	$E^{Z}_{0}\cap E^{Z}_{1}=\emptyset$ for $Z\in\{L,N^{*}\}$.  
   	
    Since the sum of the dimensions of $D_{0},D_{1}$ is at most 
    $m-d_{0}+(d_{1}+1)<n$, so applying some element of $\D_{c0}^{L\cup N^{*}}(M)$ we may assume $D_{0}\cap D_{1}=\emptyset$ (by first making the cores disjoint and then shrinking $D_{1}$ to a suitable neighbourhood of its core) and thus also that supports of $(\chi^{i}_{t})_{t\in I},\,i=0,1$ are disjoint. We can then replace the concatenation of the isotopies $(\chi^{0}_{-t}\circ f^{*})_{t\in I}$ 
    and $(\chi^{1}_{t}\circ f^{*})_{t\in I}$ with that of the isotopies 
    $(\chi^{1}_{t}\circ f')_{t\in I}$  
    and $(\chi^{0}_{-t}\chi^{1}_{1}\circ f^{'})_{t\in I}=(\chi^{1}_{1}\chi^{0}_{-t}\circ f^{'})_{t\in I}$ and we are done.    
    \subsection*{Case (\ref{small step})} Here the final step above fails, since the sum of the dimensions of $D_{0}$ and $D_{1}$ is exactly $n$, so we may only assume that $D_{0}$ and $D_{1}$ intersect transversely in finitely many points in the interior of their respective cores. This will turn out to be good enough, but we will need to proceed with care, taking care of all the $L$-tangencies of dimension $(m-1)$ in the signature at the same time and not just the one associated to $D_{0}$. 
      	
   	We begin by fixing some open neighbourhood $U$ of $M$ admitting some chart $\phi=\pmb{y}:U\cong (-3,3)^{m}$ with respect to which $L$ is given by the set $\{y_{m}=0\}$ and such that $N^{*}\cap \bar{T}=\emptyset$, where 
   	$T :=\phi^{-1}(T')$, $T'=(-2,2)^{m-2}\times(-3,3)\times(-2,2)$. We also consider the sets $Q'=(-2,2)^{m}$
   	and $Q:=\phi^{-1}(Q')$, $W':=(-2,2)^{m-1}\times(-3,3)$ and $W=\phi^{-1}(W')$ (\footnote{It may be useful to think of the $(m-1)$-th factor as "thickness" and two the $m$-th factor as "height". }).  
   	
   	We choose some curve
   	$$
	   	\gamma:(-3,3)\to Q',\quad\gamma(s)=(\underline{0},\theta(s)))
   	$$
   	 where $\theta=(\theta_{1},\theta_{2}):(-3,3)\to (-3,3)\times (0,2)$ is a regular curve 
   	 satisfying the following conditions:
   	\begin{itemize}
   		\item $\theta_{1}$ is non-decreasing with $\displaystyle\lim_{t\to 2\nu}\theta_{1}(t)=3\nu$ for $\nu\in\{-1,1\}$ 
   		\item $\theta_{2}$ is constant in some neighbourhood of $\{-3,3\}$ 
   		\item $\theta(-s)=(-\theta_{1}(s),\theta_{2}(s))$ for all $s$ 
   		\item $(\theta_{2})'(s)>0$ for $s\in (0,3)$ 
   		\item $\theta_{1}'(s)=0$ for $s\in(1,2)$   
   		\item $(\theta_{2})''(s),(\theta_{1})'(s)>0$ for $s\in [0,1)$  
   	\end{itemize}
    and write $\alpha:=\theta_{2}(1)$ and $\beta:=\theta_{2}(2)$.  
   	Then there is some smooth embedding 
   	$$\xi:V:=(-3,3) \times B^{m-1}\to (-2,2)^{m-1}\times(0,2)$$ 
   	% \phi^{-1}
   	of the form $\xi(s,v)=\gamma(s)+\epsilon\mu_{s}(v)$, where $\epsilon>0$ and
   	$\mu_{s}$ is an isometry between $\R^{m-1}$ and the space of vectors orthogonal to $\gamma'(s)$ 
   	which restricts to the identity on the space generated by the first $m-2$ coordinates. We may also assume that
   	for some $\delta_{1}\in(\alpha,\beta)$ we have
   	$\xi(s,v)\subseteq (-2,2)^{m-1}\times(\delta_{1},2)$ for $s\notin (-2,2)$, $v\in B^{m-1}$. 
         
    By a nice annulus we mean a subsurface of the form $(-3,3)\times\Sigma\subseteq V$, where $\Sigma\subseteq B^{m-2}$ 
    is a homotethical copy of the standard sphere $S^{m-2}=\partial\bar{B}^{m-2}$. By a $\xi$-compatible annulus in $Q'$ we will mean the image by $\xi$ of a nice annulus. 
    
    We leave the following claim to the reader, where the assertion about the index of the Hessian can be proven by directly exhibiting subspaces of the right dimension on which the Hessian is positive and negative definite.
    \begin{claim}
    	\label{c: big hessian claim} It is possible to choose $\xi$ and $\epsilon$ in such a way that for any $\xi$-compatible annulus $\Sigma$ in $Q'$ there are exactly two values of $\lambda$ in $(0,2)$ for which the plane $\{y_{m}=\lambda\}$ is tangent to $\Sigma$, with a single tangency point $p_{i}\in\Sigma$ for each $\lambda_{i}$. Locally around $p_{i}$ the surface $\Sigma$ coincides with the graph of some function $h$ with a unique critical point and non-singular hessian whose index is $0$ for $i=0$ and $m-2$ for $i=1$.     
    \end{claim}
    
    For any $1$-ball ($0$-cylinder) $D$ bridging $L$ and some embedded copy $N'$ of $N$ and $\iota:(-1,1) \times B^{m-1}\to M$ some trivial tubular neighbourhood of the interior $D'$ of the core of $D$  ($\iota((-1,1)\times\{\underline{0}\})=D'$) we say that $\iota$ is is adapted to $\xi$ if 
    there are finitely many disjoint intervals $\{J_{l}\}_{l=1}^{n}$ such that $\bar{J}_{l}\subseteq(-1,1)$ and 
    the following holds:
    \begin{itemize}
    	\item $\iota(\{t\}\times B^{m-1})\cap Q=\emptyset$ for $t\notin \bigcup_{l=1}^{n}J_{l}$ 
      \item for $t\in J_{l}$, $1\leq l\leq n$ we have $\iota(\{t\}\times B^{m-1})\subseteq\xi(\{s\}\times B^{m-1})$ for some $s\in(-3,3)$
      \item the image of the arc $\iota(J_{l})$ in $Q'$ is of the form $\xi((-3,3)\times\{v\})$ for some $v\in B^{m-1}$ 
    \end{itemize}
    
    Back to our situation at time $t^{*}=t^{*}_{1}$ we have a disk $D_{1}$ bridging 
    $N^{*}=f^{*}(N)$ and $L$ and freedom to choose a diffeotopy $\chi^{1}$ supported close to $D_{1}$ 
    so that the restriction of $H$ to $[t^{*},1]$ starts with an isotopy $(\chi^{1}_{t}\circ f^{*})_{t\in I}$ containing the $(m-2)$ $L$-tangency point. As in case (\ref{big step}) our goal is to ensure that the restriction of $H$ to $[-1,t^{*}]$, read backwards, is arranged nicely in relation to $D_{1}$. 
    
    %\newcommand{\nii}[1]{N_{-#1}} 
    %\newcommand{\fii}[1]{f_{-#i}}     
    \begin{lemma}
    	\label{l: big arranging pushes sublemma} After replacing the restriction of $H$ to $[-1,t^{*}]$ by an isotopy with the same endpoints (\footnote{In fact, the two isotopies are connected by a $2$-cell.}) we may assume said restriction is given by a concatenation $H^{-r-1}*H^{-r}\dots H^{0}$, where $H^{-r-1}$ has no $L$-tangency points and $H^{-i}$ has a single $L$-tangency point of dimension $m-1$ (so that $(H_{-t}^{-i})_{t}$ has a tangency point of dimension $0$).     	
    	Moreover, if we let 
    	$f_{-i}:=H^{-i}_{1}$ and $N^{-i}:=f_{-i}(N)$ (so that $f_{0}=f^{*}$), then for $0\leq i\leq r$ there exists some 
    	$1$-ball $D_{-i}$ in $M$ bridging $N^{-i}$ and $L$, some tubular neighbourhood $\xi^{-i}:(-1,1) \times B^{m-1}$ around $\mathring{D}_{-i}$ and some diffeotopy $\chi^{-i}$ of $Id_{M}$ supported on $im(\xi^{-i})$, then the following holds for $0\leq i\leq r$:
    	\begin{enumerate} [(i)]
    		\item \label{isotopy} $H^{-i}_{t}=\chi^{-i}_{-t}\circ f_{-i}$ for all $t\in I$
    		\item \label{compatible} $\xi^{-i}$ and $(\chi^{-i}_{-t}\circ f_{-i})_{t \in I}$ are compatible in the sense of Lemma \ref{l: non-destructive isotopies}
    		\item \label{adapted} $\xi^{-i}$ is adapted to $\xi$  
    		\item \label{nice annuli} the intersection $N^{-i}\cap W$ is a disjoint union of $N^{*}\cap W$ and some  
    	 $\mathcal{A}\subset Q$ such that $\phi(\mathcal{A})$ is a finite union of $\xi$-compatible annuli 
    	\end{enumerate}
    \end{lemma}
    \begin{subproof}
   	   %This follows from repeatedly applying Lemmas \ref{l: non-destructive isotopies} and \ref{l: conjugation}. 
   	   We construct $H^{0},H^{-1},\dots$ in that order so that at stage $k$  
   	   one may assume the restriction of $H$ to the interval $[-1,t^{*}]$ is of the form $H'*H^{-k+1}*H^{-k+2}\dots H^{0}$, 
   	   where $H^{j}$, $j\leq k$ satisfy the assumptions of the conclusion and $H'$ is $L$-compatible, with signature consisting of exactly $r-k$ copies of $m-1$, with $f_{-l}$, $N^{-l}$, etc. defined accordingly and, in particular 
   	   $f_{-k}:=H^{-k+1}_{-1}$, $N^{-k}:=f_{-k}(N)$.  
   	    
   	   In this situation Lemma \ref{l: non-destructive isotopies} applied in reverse and the same argument used in the proof of Corollary \ref{c: non-destructive isotopies} and at the beginning of the parent proof
   	   provides some $1$-ball $D_{-k}$ bridging $N^{-k}$ and $L$, some 
   	   tubular neighbourhood $\xi^{-k}$ of the core $D'_{-k}$ of $D_{-k}$ and a diffeotopy 
   	   $\chi^{-k}$ compatible with $\xi^{-k}$ such that $(\chi^{-k}_{t}\circ f_{-k})_{t\in I}$ has a single $L$-tangency point
   	   of dimension $0$ and there is some $L$-compatible isotopy $H''$ from $H_{-1}$ to $\chi_{1}^{-k}\circ f_{-k}$ with $r-k-1$ tangency points of dimension $m-1$ (dimension $0$ in reverse). 
   	   
   	   Part (\ref{nice annuli}) of our inductive assumption is that the intersection of $N^{-k}$ with $W$ is the disjoint union of $N^{*}\cap W$ and the preimage by $\phi$ of some finite collection of $\xi$-compatible annuli.
   	    \begin{comment}
	   	    This allows us to find an element of $\D_{c0}^{N^{-k}\cap L\{N^{-k}\}}(M)$ and thus of $\D_{c0}^{\{N^{-k},L\}}(M)$, by Lemma \ref{l: isotopy extension}, moving the intersection $D_{-k}\cap N^{-k}$ away from $\bar{W}$. Lemma \ref{l: conjugation} allows us to assume that actually $D_{-k}\cap N^{-k}\cap\bar{W}=\emptyset$.  
   	    \end{comment}
   	   Corollary \ref{c: refined transversality} and Lemma \ref{l: conjugation} again
   	   %(working this time in $\D_{c0}^{L\cup N^{-k}}(M)$) 
   	   allow us to assume that $D_{1}$ and $D_{-k}$ intersect transversely in finitely many points the interior of their respective cores and that these are all contained in $Q\cap D_{1}$.
   	   
   	   Using Lemma \ref{l: conjugation} we can assume furthermore that the intersection $D_{-k}\cap T$ 
   	   is in fact contained in $Q$ and consists of finitely many arcs, whose closure intersects the sets
   	   $\phi^{-1}(\{y_{m-1}=-2\})$ and $\phi^{-1}(\{y_{m-1}=2\})$ each at a single point, and each of which crosses $D_{0}$ 
   	   in a single point. 
   	   
   	   By virtue of part (\ref{nice annuli}) of the inductive hypothesis, we can assume the image of 
   	   each of those arcs in $Q'$ is of the form $\xi((-3,3)\times\{v\})$ for some $v\in B^{m-1}$. 
   	   By Lemma \ref{l: conjugation} again and the equivalence of tubular neighbourhoods (Lemma \ref{l: tubular neighbourhoods}) we can assume that in fact $\xi^{-k}$ is adapted to $\xi$ (\ref{adapted}), and it follows from the description 
   	   of $\xi^{\eta}_{1}\circ H_{t^{*}}$ in Lemma \ref{l: non-destructive isotopies} that after performing some isotopy  acting like scalar multiplication on each individual fiber we may assume that $N^{-k-1}\cap supp(\chi^{k})$ intersects $W$ in a finite union of preimages of $\xi$-compatible annuli.
   	   This is covered by a diffeotopy fixing $L\cup N_{-k}$ so that another application of \ref{l: conjugation} ensures condition (\ref{nice annuli}) for the new isotopy $H_{-k}:=(\chi^{-k}_{-t}\circ f_{-k})_{t\in I}$.
    \end{subproof}
       	
   	It is easy to use Lemma \ref{l: isotopy extension} to find some diffeotopy $G^{V}$ of $Id_{M}$ supported on
   	$T$ such that if we let $\hat{G}_{t}:=\phi\circ G^{V}_{t}\circ\phi^{-1}$ we have for some $\delta_{0}\in(\alpha,\delta_{1})$:
   	%$\phi^{-1}((-1,1)^{m-1}\times(-2,2)^{2})
   	\begin{itemize}  
   		\item $\hat{G}^{V}_{t}(\{y_{m}>\delta_{1}\})\subseteq\{y_{m}>\delta_{1}\}$ for all $t\in I$ 
   	 	\item $\hat{G}^{V}_{t}(\underline{y})=(y_{1},\dots y_{m-1},\lambda_{t}(y_{m}))$
   	        for any $t\in I$ and $\underline{y}\in (-2,2)^{m-1}\times(0,\beta)$, where $\lambda_{t}(y_{m})$ is a diffeomorphism of $(-2,2)$ restricting on $(0,\delta_{0})$ to the translation by the vector $-\frac{\delta_{0}(s+1)}{2}$.  	  
   	 \end{itemize}
    It follows from Claim \ref{c: big hessian claim} and item (\ref{nice annuli}) in Lemma \ref{l: big arranging pushes sublemma} that the isotopy $$F^{1}:=(G^{V}_{t}\circ f_{-(r+1)})_{t\in I}$$ is $L$-compatible 
   	and its signature contains only entries of type $0$ and $m-2$. Let $g_{V}:=G^{V}_{1}$.    
    For $0\leq l\leq r$ let now $h_{l}=\chi^{-l}_{1}$ and write $\bar{h}_{l}=h_{l}h_{l-1}\cdots h_{0}$.
    Consider the compactly supported diffeotopies $\bar{\chi}^{-l}=(\chi^{-l}_{t}\bar{h}_{l-1})$, $0\leq l\leq r$ and $G^{-}:=\bar{\chi}^{-r}*\bar{\chi}^{-r+1}*\dots \bar{\chi}^{0}$. Note that $G^{-}_{1}=Id_{M}$.
      	
   	By modifying $\chi^{1}$ using Lemma \ref{l: conjugation} if needed (\footnote{As already used before, there are elements of $\D(M)$ contracting $D_{1}$ to an arbitrary small neighbourhood of its core, while fixing the core, $N^{*}$ and $L$ fixed.}) we may assume that
   	\begin{equation}
   	\label{disjoint supports} \tag{$\dagger$} supp(g_{V}\chi^{-l}_{t}g_{V}^{-1})\cap supp(\chi^{1})=\emptyset\quad \text{for all $0\leq l\leq r$}. 
   	\end{equation}
   	for all $t\in I$. This can be done without modifying $\chi^{-l}$, as we are applying the lemma in the forward time direction. In particular, the isotopy $$F^{2}:=(\chi^{1}_{t}\circ F^{1}_{1})_{t\in I}$$ agrees 
   	with $(\chi_{t}^{1}\circ f^{*})_{t\in I}$ on its support and hence is $L$-compatible with signature $(m-2)$.
    Finally, consider the isotopy 
   	$$F^{3}:=(g_{V}G^{-}_{t}g_{V}^{-1}\chi^{1}_{1}\circ f^{*})_{t\in I}$$
   	and notice:
   	\begin{align*}
	   	F^{3}_{-1}=&g_{V}G^{-}_{-1}g_{V}^{-1}\chi^{1}_{1}\circ f^{*}\\
	   	=&\chi^{1}_{1}(g_{V}G^{-}_{-1}g_{V}^{-1}) \circ f^{*}=\chi^{1}_{1}g_{V}\circ f^{-r-1}=F^{2}_{1},
   	\end{align*}   
    where we use (\ref{disjoint supports}) for the first equality and $N^{*}\cap supp(G^{V})=\emptyset$ for the second.
   	The properties of $\xi$, the description of $G^{V}$ and item (\ref{adapted}) of Lemma \ref{l: big arranging pushes sublemma} imply that for any point $q\in G^{V}_{1}(D_{-l})\cap L$ the  
   	the fiber image through $q$ of the tubular neighbourhood $G^{V}_{1}\circ\xi^{-l}$ is contained in $L$. 
   	
   	This, together with properties (\ref{first property}) and (\ref{second property}) in Lemma \ref{l: non-destructive isotopies} implies that $F^{3}$ is $L$-compatible and its signature contains only the dimension $m-1$. For each $0\leq l\leq r$ there is one entry corresponding to the original tangency point of $(\chi^{-l}_{-t}\circ f^{-l})_{t\in I}$ and  	 
   	two more for each point in $D_{-l}\cap D_{1}$. Since we can concatenate $F^{1},F^{2}$ and $F^{3}$ and $F^{3}_{1}=\chi^{1}_{1}\circ f^{*}$ we are done.   	  
   \end{proof}
   
   An easy induction on signature length now shows:
   \begin{corollary}
   	\label{c: compatibility and decompositions}Given a nice triple $(N,M,L)$ and an $L$-compatible isotopy $H$ of $N$ in $M$ there are $L$-compatible isotopies $H'$, $H''$ of $N$ in $M$ such that:
   	\begin{itemize}
   		\item $H_{-1}=H'_{-1}$, $H'_{1}=H''_{-1}$, $H''_{1}=H_{1}$ 
   		\item $H'$ is non-destructive and $H''$ is purely destructive.
   	\end{itemize}
   \end{corollary}
   


   
   
\section{Completing the proof of Theorem \ref{t: main}}
  \label{conclusion}  
        We continue to make no general assumptions on $\T$ other than it is coarser than the restriction of the compact-open topology.           
        We start with a lemma allowing us to ignore the lower-dimensional skeleton of embedded simplicial complexes.
        \begin{lemma}
        	\label{l: cleaning}Let $C\subseteq M$ be an $m$-ball and $\Delta\subseteq M$ a finite embedded simplicial complex of dimension $(m-1)$. If $\T$ is compressive, then for any $\mathcal{V}\in\nd$ there exists some $g\in\mathcal{V}$ such that $g\cdot\rl{\Delta^{m-2}}\cap C=\emptyset$. 
        \end{lemma}
	        \begin{proof}
	          Pick some $m$-ball $D\subseteq \mathring{C}$ such that 
	        	$D\cap\Delta=\emptyset$ and some $m$-ball $C'$ such that $C\subseteq U$, where $U :=\mathring{C}'$.
	        	Then there exists:     
	        	$
	        	\phi:U\setminus D\cong S^{m-1}\times(0,\infty)  
	        	$
	          mapping $C\setminus D$ onto $S^{m-1}\times(0,1]$.   
	          We may assume every simplex in $\Delta^{(m-2)}$ is the face of some $(m-2)$-simplex and 
	        	after subdividing $\Delta$ we may assume that any simplex in 
	        	$\Delta^{m-2}$ is either contained in $U$ or in $M\setminus C$.  
	        
	         	
	         	We will use the term $U$-lift to refer to the 
	         	extension by the identity of some element $h\in\D_{c0}(U)$ such that for all $p\in S^{m-1}\times(0,\infty)$ the push-forward $\hat{h}$ of $h$ to $S^{m-1}\times(0,\infty)$ satisfies: 
	          $$\pi_{(0,\infty)}(\hat{h}(p))\geq\pi_{(0,\infty)}(p), \quad \pi_{S^{m-1}}(\hat{h}(p))=\pi_{S_{m-1}}(p).$$
	          
	          
	          Let $\{\sigma_{k}\}_{k=1}^{r}$ be an enumeration of the simplices in $\Delta^{m-2}$ which intersect $C$ and 
	        	and for every $1\leq i\leq k$ let $\iota_{k}:B^{m-2}\to U$ be some smooth embedding 
	        	such that $\sigma_{k}=\iota_{k}(\tau_{k})$ for some affine simplex $\tau_{k}\subseteq B^{m-2}$. Pick $\V_{0}\in\nd$ such that $\V_{0}^{2r}\subseteq\V$.  
	          
	          Let $O$ be an open set of the form 
	        	$O:=(M\setminus U)\cup \phi^{-1}(S^{m-1}\times (\lambda,\infty))$, $\lambda>1$,
	        	such that any simplex in $\Delta^{m-2}\setminus\Theta$ is contained in $O$. 
	        	
	        	One can easily find a sequence $O_{0}=O,O_{1},\dots O_{r}$ of sets of the same form (in particular, $O_{i}\cap C=\emptyset$) and  
	        	$\W=\V_{\eta}$ for some $\eta>0$ such that 
	        	$g\cdot O_{i}\subseteq O_{i+1}$ for all $g\in\W$.
	          The following is a straightforward consequence of the compressiveness of $\T$.
	          \begin{observation}
	          	\label{o: narrow lifts} Assume that $F\subseteq U$ is a compact set such that $(\pi_{S^{m-1}}\circ\phi)(F)\subseteq S^{m-1}$ is a $d$-disk in $S^{m-1}$, $d<m-1$.
	          	Then there exists some some $U$-lift $g$ such that $g\cdot F\subseteq O$. 
	          \end{observation}
	 
	          We proceed to choose for $1\leq k\leq r$ some $g'_{k}\in\V_{0}\cap \W$ and then $g_{k}\in\V_{0}$ such that if we let $\bar{g}_{0}=1$, $\bar{g}_{k}=g_{k}g'_{k}\cdots g_{1}g'_{1}$, then:
	          \begin{enumerate}
	          	\item \label{boardman condition} If we let $\iota'_{k}:= g'_{k}\bar{g}_{k-1}\circ\iota_{k}:B^{m-2}\to S^{m-1}$, then there exists a filtration
	          	$L_{0}=B^{m-2}\supsetneq L_{1}\supsetneq\dots L_{n}\supsetneq L_{n+1}=\emptyset$, 
	            where $L_{l+1}$ is a closed submanifold of $L_{l}$ and $\pi_{S^{m-2}}\circ\phi\circ\iota'_{k}$ restricts to an immersion on $L_{l}\setminus L_{l+1}$.
	            \item \label{lifted enough} $g_{k}$ is a $U$-lift for which $\bar{g}_{k}(\sigma_{k})\subseteq O$.
	          \end{enumerate}
	          
	          Notice that then $g=\bar{g}_{r}\in\V_{0}^{2r}\subseteq\V$ will satisfy $g\cdot\rl{\Delta^{(m-2)}}\cap C=\emptyset$, since the fact that the $g_{l}$ are $U$-lifts implies $g_{r}g'_{r}\dots g_{k}g'_{k}\cdot O_{0}\subseteq\subseteq O_{r-k+1}$.
	          
	          The element $g'_{k}$ is provided to us by Corollary \ref{c: projecting simplicial complexes}. To construct $g_{k}$, choose in advance $\V_{l}=\V_{l}^{-1}\in\nd$ such that $\V_{l}^{2}\subseteq\V_{l-1}$ for $l\geq 1$ and $\V_{l,j}=\V_{l,j}^{-1}\in\nd$, $j\geq 2$ such that 
	          $\V_{l,j}^{j}\subseteq\V_{l}$. 
	          We will successively construct $U$-lifts $h_{l}\in\V_{n-l+1}$ for $0\leq l\leq n$ so that 
	          if we let $\bar{h}_{l}=h_{l}\cdots h_{0}$, then 
	          $(\bar{h}_{l}\circ\iota_{k}')(V_{l})\subseteq O$ for some neighbourhood $V_{l}$ of $\tau_{k}\cap L_{n-l}$. The element $g_{k}:=h_{n}\cdots h_{0}\in\V_{0}$ will then satisfy (\ref{lifted enough}), using again the fact that $U$-lifts map $O$ into itself. 
	          
	          Assume that we have already constructed $h_{l'}$ for $0\leq l'<l$ and let $V_{-1}=\emptyset$. Since $(L_{n-l}\cap\tau_{k})\setminus V_{l-1}$ is compact, it can be covered by a collection of $\{D_{l,s}\}_{1\leq s\leq r_{l}}$ of finitely many $dim(L_{n-l})$-balls in $L_{n-l}$ (\footnote{As a matter of fact $dim(L_{l})=m-2(l+1)$, see \cite{arnold2014singularities}.}). We now choose lifts $h_{l,s}$ successively for $1\leq s\leq r_{l}$.
	          Given $s$, the map 
	          $$\pi_{S^{m-1}}\circ\phi\circ h_{l,s-1}\cdots h_{l,0}\bar{h}_{l-1}\circ\iota'_{\restriction D_{l,s}}=\pi_{S^{m-1}}\circ\phi\circ\iota'_{\restriction D_{l,s}}:D_{l.s}\to S^{m-1}$$
	          is a smooth embedding (using that all the diffeomorphisms in the expression are lifts). Observation \ref{o: narrow lifts} provides some $h_{l,s}\in\V_{m-l+1,r_{l}}$ such that  
	          $(h_{l,s}\cdots h_{l,0}\bar{h}_{l-1}\circ\iota'_{k})(D_{l,s})\subseteq O$. 
	          If we write $h_{l}:=h_{l,r_{l}}\cdots h_{l,0}$, then using again that all elements involved are lifts we have
	          $\bar{h}_{l}\cdot (L_{n-l}\cap\tau_{k})\subseteq O$ and the existence of $V_{l}$ follows by continuity. 
	        \end{proof}
    
    
    We also observe that the difference between $\PS{\gg}{\Delta}$ and $\SS{\gg}{\Delta}$ is irrelevant. 
    \begin{lemma}
      \label{l: different stabilizers}For any finite $(m-1)$-dimensional simplex	$\Delta$ in $M$ and any $\epsilon>0$ we have that 
      $\SS{\gg}{\Delta}\subseteq\V_{\epsilon}\PS{\gg}{\Delta}\V_{\epsilon}\PS{\gg}{\Delta}$. In particular, 
       $\PS{\gg}{\Delta}$ can be replaced by $\SS{\gg}{\Delta}$ in the conclusion of in Corollary \ref{c: control on vertices}. 
    \end{lemma}
    \begin{proof}
	    Fix $g\in\SS{\gg}{\Delta}$. By assumption there is a diffeotopy $G$ of $Id_{M}$ preserving each of the simplices 
	    of $\Delta$ set-wise at all times and whose support is contained in $M\setminus U$ for some neighbourhood 
	    $U$ of $\rl{\Delta^{(m-2)}}$ in $M$.	    
	    Let $\{\sigma_{i}\}_{i=1}^{r}$ be an enumeration of $\Delta^{(k-1})$. 
	    Choose some collections of disjoint embedded $(m-1)$ balls 
	    $\{D_{i}^{l}\}_{i=1}^{r}$, $l=1,\dots 3$ such that $D^{l}_{i}\subseteq \mathring{D}^{l+1}_{i}\subseteq D^{l+1}_{i}\subseteq\sigma_{i}\setminus\partial\sigma_{i}$ 
	    and if we let $N^{l}:=\bigcup_{i=1}^{r}\mathring{D}_{i}^{l}$, then 
	    $\rl{\Delta}\subseteq N^{1}\cup U$ and $\partial N^{1}\subseteq U$. 
	    
	    The existence of tubular neighbourhoods yields an embedding $\xi:E:=N^{3}\times (-1,4)\to N^{3}\to M$
	    mapping $N^{3}\times\{0\}$ onto $N$ such that: 
	    \begin{enumerate}
	    	\item $\xi(\pi^{-1}(N^{2}))\cap\Delta=\xi(N^{2}\times\{0\})$ 
	    	\item \label{fibers}for any $p\in N^{2}$ we have $diam(\{p\}\times(-1,4))<\epsilon$ 
	    \end{enumerate}  
 
      Let $f:N^{2}\to[0,\frac{1}{2})$ some smooth map such that $f_{\restriction N^{1}}=2$, $f_{\restriction V}=0$ in some neighbourhood $V$ of $\partial N^{2}$. Applying \ref{l: isotopy extension} to $\{0\}\subseteq (-1,1)$ 
      one easily constructs some fiber-preserving $\hat{h}_{1}\in\D_{c0}(N^{2}\times(-1,4))$ mapping 
      $N^{2}\times\{0\}$ to the graph of $f$. Consider now the element $\hat{h}_{2}\in\D_{c0}(N^{2}\times(-1,4))$ given by:
      $\hat{h}_{2}(p,s)= G(p,\rho(s))$, where $\rho:(-1,4)\to [-1,1]$ is any smooth function with $\rho(2)=1$ and $\rho(s)=-1$  outside of $(1,3)$. Letting $h_{1},h_{2}$ be the trivial extensions of $\hat{h}_{1},\hat{h}_{2}$ to $M$ we have $h_{1}\in\V_{\epsilon}\cap\hh$ (by (\ref{fibers})) $h_{2}\in\PS{\gg}{\Delta}$ (since $\overline{supp(h)}\cap\rl{\Delta}=\emptyset$) and $g^{-1}h_{2}^{h_{1}}\in\PS{\gg}{\Delta}$, which proves the first claim. The second one then follows follows immediately by the continuity of multiplication.
    \end{proof}

  \begin{theorem}
  	\label{t: slice induction}Assume that $\T$ is a group topology in which arc compressions are dense. Then for any $m$-ball $D$ in $M$ and any $g\in\D_{c0}(\mathring{D})\subseteq\D_{c0}(M)$ we have $g\in\V$ for all $\V\in\nd$. 
  \end{theorem}
  \begin{proof}
  	 Note that Corollary \ref{c: control on vertices} applies (by Observation \ref{o: intrusiveness}). We will use this repeatedly. 
  	 We prove the result by induction on $m$, starting from the tautological case $m=1$. 
     Assume now the result is true for any $k<m$. We start with the following sublemma.    
     \begin{lemma}
	     \label{l: inductive slices}Under the assumptions of the main Lemma compressions are dense in $\T$.
     \end{lemma}
   \begin{subproof}
%	     We start by noticing that any topology $\T$ in which arc compressions are dense has to be strictly coarser than the compact-open topology.
	     Pick any $(m-1)$-ball $D$ in $M$ and $\V\in\nd$ and let $\W=\W^{-1}\in\nd$ satisfy $\W^{3}\subseteq\V$. Using Whitney extension theorem and the existence of tubular neighbourhoods one can easily construct an embedding $\phi:S^{m-1}\to M$ extending $D$. Consider the collection $\mathcal{N}^{*}:=\{\V^{*}\}_{\V\in\nd}$ of subsets of $\D(S^{m-1})$ given by: 
	     $$
	      \V^{*}:=\{\,g\in\D_{c0}(S^{m-1})\,\,|\,\,\exists\hat{g}\in\V\cap\D_{c0}^{\{\phi(S^{m-1})\}}(M)\,\,\,\,\phi^{-1}\circ\hat{g}\circ\phi=g\,\}
	     $$
	     We claim that $\mathcal{N}^{*}$ is the system of neighbourhood of the identity of a group topology $\T^{*}$ on $\D_{c0}(S^{m-1})$ coarser than the compact-open topology and in which arc compressions are dense.
	     That $\mathcal{N}^{*}$ generates a filter follows from 
	     the containment $(\V_{0}\cap\V_{1})^{*}\subseteq\V_{0}^{*}\cap\V_{1}^{*}$, 
	     while continuity of the inverse map at the identity follows from equality
	      $(\V^{-1})^{*}=(\V^{*})^{-1}$ and continuity in $\T$, that of multiplication from the containment  $\V_{0}^{*}\V_{1}^{*}\subseteq(\V_{0}\V_{1})^{*}$ and continuity in $\T$. Invariance of $\V^{*}$ under conjugation follows from that of $\V$ and the fact that any element of 
	     $\D_{c0}(S^{m-1})$ extends via $\phi$ to an element of $\D_{c0}(M)$, by Lemma \ref{l: isotopy extension}.
	     That $\T^{*}$ contains the compact-open topology on $\D(S^{m-1})$ follows from Fact \ref{f: close implies isotopic} and Lemma \ref{l: isotopy extension}. Density of arc compressions in $\T$ straightforwardly implies the same property for $\T^{*}$.
	     
	     Now, given $\V_{0}=\V_{0}^{-1}\in\nd$ with $\V_{0}^{3}\subseteq\V_{0}$ by Corollary \ref{c: control on vertices} we can find some finite embedded simplicial complex $\Delta$ in $M$ such that $\PS{\gg}{\Delta}\subseteq\V_{0}$ and $\phi(S^{m-1})\transv \Delta$. 
	     The last condition implies $\rl{\Delta}\cap D$ is nowhere dense in $D$, for  any embedded $(m-1)$-ball $D'$ in $S^{m-1}$ with $\phi^{-1}(D)\subseteq\mathring{D}'$ there is $\hat{h}\in\D_{c0}(D')\subseteq\D_{c0}(S^{m-1})$, such that $\hat{h}\cdot\phi^{-1}(\Delta)\cap\phi^{-1}(D)=\emptyset$. 
	     The induction hypothesis implies $\hat{h}\in\W^{*}$, which yields an extension $h\in\W^{*}$ 
	     for which  $h\cdot\rl{\Delta}\cap D=\emptyset$ and then for some $\eta>0$ we have:
	     $$
	      \cmp{\eta}{D}\subseteq\SS{\gg}{h\cdot\Delta}\subseteq\SS{\gg}{\Delta}^{h^{-1}}\subseteq\W^{3}\subseteq\V.
	     $$
   \end{subproof}
      
     We will now derive the conclusion of the Theorem from the hypothesis that $\T$ is compressive.
     Let $D,D'$ be $m$-balls, $D\subseteq\mathring{D'}$ in $M$ and $g\in \D_{c0}(\mathring{D})\subseteq\D_{c0}(M)$.
          
     Let $\V\in\nd$ and choose $\V_{0}=\V_{0}^{-1}\in\nd$ with 
     $\V_{0}^{15}\subseteq\V$. Let $\Delta_{0}$ be a finite embedded simplicial complex with 
     $\SS{\gg}{\Delta}\subseteq\V_{0}$ as given by Lemmas \ref{l: nerve} and \ref{l: different stabilizers}. By Lemma \ref{l: cleaning} there is $h_{0}\in\V_{0}$ such that $h_{0}\cdot\Delta_{0}^{(m-2)}\cap D'=\emptyset$.
     By Lemma \ref{l: wlog transverse} for any $\delta>0$ there is $h_{1}\in\V_{\delta}$ such that $\Delta:=h_{1}h_{0}\cdot\Delta_{0}$ satisfies the previous property and, in addition, $\partial D'\transv \Delta$. Using Lemma \ref{l: wlog transverse} and the existence of tubular neighbourhoods as in the proof of Lemma \ref{l: different stabilizers} one finds $h_{2}\in\V_{\delta}$ such that
     $\Theta:=h_{2}\cdot\Delta$ is still transverse to $\partial D'$ and satisfies $\rl{\Theta}\cap\rl{\Delta}\cap D'=\emptyset$. If $\delta$ is small enough, then $\V_{\delta}\subseteq\V_{0}$ and thus
     $\SS{\gg}{\Delta}\subseteq\V_{0}^{5}$, $\SS{\gg}{\Theta}\subseteq\V_{0}^{7}$. 
     
     If we let $U :=\mathring{D}$, then $L :=\Delta\cap U$ and $N:=\Theta\cap U$ are disjoint closed submanifolds of $U$. By assumption there is some compactly supported diffeotopy $G$ of $Id_{U}$ such that $\psi:=g_{\restriction U}=G_{1}$
     Consider the isotopy $H$ of $N$ in $U$ given by $(G_{t}\circ\iota_{N})_{t\in I}$, where $\iota_{N}: N\to U$ is the inclusion. 
     
     By Lemma \ref{l: morse for isotopies} we can find an $L$-compatible isotopy $H^{*}$ arbitrarily close to $H$  
     in $\wt(N\times I,U)$ and agreeing with $H$ outside some compact set. Then on the one hand, by Lemma \ref{l: isotopy extension} and Fact \ref{f: close implies isotopic} for $H^{*}$ close enough to $H$ we have 
     $H_{1}=\psi_{3}\psi\circ\iota_{N}$ for some $\psi_{3}\in\D_{c0}(U)$ that extends by the identity to $g_{3}\in\V_{0}$. On the other hand, Fact \ref{f: close implies isotopic} implies that if $H^{*}$ is close enough to $H$, then $H^{*}_{-1}$ can be joined to $\iota_{N}$ by a compactly supported isotopy the image of whose is disjoint from $L$, so that we may assume $H^{*}_{-1}=\iota_{N}$.
     
     By Corollary \ref{c: non-destructive isotopies} we may also assume $H^{*}$ is the concatenation of $L$-compatible $H'$ and $H''$, where $H'$ is non-destructive and $H^{''}$ is purely destructive. Let $r',r''$ be the number of $L$-tangency points of $H'$ and $H''$ respectively and choose $\V_{1}=\V_{1}^{-1}\in\nd$ such that $\V_{1}^{\min\{r',r''\}}\subseteq\V_{0}$. 
      
     Lemma \ref{l: inductive slices} allows us to apply Corollary \ref{c: compatibility and decompositions} separately
     with $\mathscr{X}=\{g_{\restriction U}\,|\,g\in\V_{1}\}\cap\D_{c0}(U)$ to the two non-destructive isotopies $H'$ and $(H''_{-t})_{t\in I}$. After performing some straightforward algebraic manipulations on the result we get: 
     \begin{equation}
     	\label{dagdag}\tag{$\dagger\dagger$}\psi_{3}\psi\circ\iota_{N}=(\phi_{r''}''\cdots \phi''_{1})^{-1}\psi_{L}\phi_{r'}'\cdots \phi'_{1}\circ\iota_{N}\circ\psi_{N}
     \end{equation}
     where $\psi_{L}\in\D_{c0}^{\{L\}}(U)$, $\psi_{N}\in\D_{c0}(N)$ and $\phi'_{i},\phi'_{j}\in\mathcal{X}$. 
     Extending by the identity we have inclusions $\D_{c0}^{\{L\}}(U)\subseteq\SS{\kk}{\Delta}$
     and $\D_{c0}^{N}(U)\subseteq\PS{\kk}{\Theta}$, while by Lemma \ref{l: isotopy extension} for any $\psi\in\D_{c0}(N)$ we have $\hat{\psi}\circ\iota_{N}=\iota_{N}\circ\psi$ for some $\hat{\psi}\in\D_{c0}^{\{N\}}(U)$. 
     We thus deduce from (\ref{dagdag}) that 
     $$
     g\in g_{3}^{-1}\V_{1}^{r''}\SS{\kk}{\Delta}\V_{1}^{r'}\SS{\gg}{\Theta}\in\V_{0}^{15}\subseteq\V.
     $$   
     %, $\psi_{L}$ extends to $g_{L}\in\SS{\kk}{\Delta}$. 
   \end{proof}
     
  \begin{proof}
  [\textit{Proof} of Theorem \ref{t: main}] Immediate consequence of Lemma \ref{l: spread all over}, Theorem \ref{t: slice induction} and Corollary \ref{c: 0-slices}. 
  \end{proof}
  
  \section{Questions and remarks}
  
  \subsection*{Dimension 2} In the dimension $2$ case Lemma \ref{l: commutation move} is not needed on account of the fact that given two systems of curves (arcs) in general position one of which can be isotoped to be disjoint from the other we can always assume the isotopy to be purely destructive (decomposable as a sequence of bigon removals). This was used on a previous iteration of this work which dealt only with the dimension $2$ case. 
  
  \subsection*{Boundary components}
  If $\partial M\neq\empty$ and $dim(M)>1$, then there the compact-open topology on the homeomorphism or group is not minimal. An witness of this fact, provided in \cite{chang2017minimum} (for compact $M$, but the proof works in the non-compact case as well) can be given as follows. Let $\rho:\H(M)\to \H(int(M))$ be the restriction homomorphism, where $int(M)=M\setminus\partial M$ we let $\T_{co}^{\partial}$ denote the preimage by $\rho$ of the compact-open topology on $\H(int(M))$ ($(\tc)_{\restriction \partial M}$ in the notation of \cite{chang2017minimum}). Then $\T_{co}^{\partial}$ is a Hausdorff group topology strictly coarser than $\tc$. For a smooth $M$ it is easy to see that the inclusion remains proper if we restrict to the group of diffeomorphisms. On the other hand, since $\D_{c0}(int(M))\leq im(\rho)$, Theorem \ref{t: main} applies and thus we get:
  \begin{corollary}
  For any smooth manifold with boundary $M$ of dimension $m\neq 3$ the restriction of $\T_{co}^{\partial}$ to $\D(M)$ is minimal. For any surface with boundary $\T_{co}^{\partial}$ is minimal. 
  \end{corollary}
  
  As a matter of fact, given some subset $\mathcal{X}$ of the connected components of $\partial M$, we could have defined a topology $\tc^{\mathcal{X}}$ using the restriction map with respect to the inclusion $M\setminus\bigcup_{C\in\mathcal{X}}C\subset M$. We are inclined to believe the proof of Theorem \ref{t: main} could be generalized without much complication to show this is all there is:
  \begin{conjecture}
  	Let $M$ be a connected smooth manifold with boundary of dimension $m\neq 3$. Then any group topology on $\D(M)$ coarser than the compact-open topology is of the form $\tc^{\mathcal{X}}$ for some $\mathcal{X}\subseteq\pi_{0}(\partial M)$.
  \end{conjecture}
  
  \subsection*{Further questions}
   The elephant in the room is, of course: 
   \begin{question}
   	Suppose that $M$ is a $3$-manifold is the restriction to $\D(M)$ of the compact-open topology minimal?   
   \end{question}
    We believe elementary techniques should be sufficient to decide the question either way, with knots being involved in case of a negative answer. 
  
   We are cautiously positive about the following:
   \begin{question}
   	Does our result extend to the group of $PL$-homeomorphisms of a $PL$-manifold?  
   \end{question}
   
     
  
  \bibliographystyle{plain}
  \bibliography{bibliography}
  
\end{document}