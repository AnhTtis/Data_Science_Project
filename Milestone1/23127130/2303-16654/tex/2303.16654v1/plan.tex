
Abhish
- [x] PO-UCT Section: Update the notation and clean text 
- [x] Experimental Setup section: make the list look and "feel" more like Raihan's version (see PDF on slack).
- [x] Results "describing environments" section: make the list look and "feel" more like Raihan's version (see PDF on slack). Make sure that you describe the environments.
- [x] Related Work: finish it.
- [ ] Mention (in a couple places) that you're using images as input to the learning.
- [ ] You need to add more citations for the topologically-constrained actions.

Updated:
- [x] Make Overview figure.
- [ ] Make Maze environment result figure showing optimistic planner (1 robot, 2 robots), MR-LSP planner (1 robot, 2 robots).
- [ ] Make Office environment result figure showing optimistic planner, MR-LSP planner (1 robot, 2 robots).
- [ ] Finalize Result (By: Sunday evening)
- [ ] You really need to talk more about the neural network. It is not sufficient to just reference the LSP paper without giving any detail. Our paper should be as self-contained as possible. PO-UCT isn't your contribution either, but it *must* be there. The same is true here: what does the CNN look like...? You can have a short paragraph/section at the end of the LSP section briefly describing how data is collected and used for training and then another short section *very briefly* describing the CNN (and then reference Chris Bradley's ICRA 21 paper for more details).
- [x] You need some schematics. A high-level figure of the approach will help, but you will also need a diagram showing the state transition model, etc. Those need to be in the paper by this afternoon [9/11/22].



GJSTEIN:
- [X] Review PO-UCT Section
- [ ] Review experiment section
- [X] Look at new subgoal property estimation section
- [X] Review related works