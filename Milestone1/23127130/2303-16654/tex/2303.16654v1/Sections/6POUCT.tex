

\section{Computing MR-LSP Expected Cost Via Sample-Based Tree Search}\label{sec:pouct}
% Talk about how PO-UCT is implemented. Talk about history, and rollout cost.
Despite our action abstraction to simplify planning, the space of collective actions can become large in practice, making calculating cost via Eq.~\eqref{eq:MR-LSP}  computationally intensive.
Instead, we rely on sample-based high-level planning and use a Partially Observable UCT (PO-UCT) \cite{silver2010monte}---a variant of Monte-Carlo tree search. PO-UCT is an anytime planning algorithm and allows us to approximate the expected cost without the need to exhaustively simulate all states.

During expansion of the planning search tree in PO-UCT, we maintain a \emph{rollout history} associated with each node: $\mathcal{H}_{b_t} = [[a_0,n_0,Q_0],\cdots,[a_k,n_k,Q_k]]$, that retains each node's history of (i) executed high-level actions $a_t$ and their outcomes, (ii) the number of times the node has been visited, and (iii) the accumulated expected cost $Q_t$ up to that node.
By simulating a collective action $a_t$ from the abstract state $b_t$, finding the subgoal $\sigma'$ via Eq.~\eqref{eq:frontiers-and-cost}, we expand the tree stochastically, where the outcome of a particular action is sampled from a Bernoulli distribution parametrized by the estimated $P_S(\sigma')$.
After the action outcome is sampled, the belief transitions to 
a node either corresponding to a success ($b_S$) or failure state ($b_F$), as defined in Eq.~\eqref{eq:updated-beliefs}.

Rollouts proceed similar to other Monte-Carlo Tree Search approaches.
The cost of each node corresponding to belief state ($b_S$ or $b_F$) is the sum of cost accrued to reach the current belief state from the initial belief state ($b_t$) and a search heuristic corresponding to the lower bound cost for the team to reach the goal if unseen space were assumed to be unoccupied.
After each node is visited, its count is incremented, which is used to control the rate of exploration during traversal. In all our \replaced{}{MR-LSP} experiments, we use 15,000 samples at each planning step.
