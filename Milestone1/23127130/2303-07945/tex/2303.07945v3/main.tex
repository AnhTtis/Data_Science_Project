\documentclass[10pt,twocolumn,letterpaper]{article}


\usepackage{iccv}
\usepackage{times}
\usepackage{epsfig}
\usepackage{graphicx}
\usepackage{amsmath}
\usepackage{amssymb}
\usepackage{kotex}
\usepackage{xcolor, soul}
\newcommand{\hlc}[2][yellow]{ {\sethlcolor{#1} \hl{#2}} }
\usepackage{caption}
\DeclareCaptionType{Figure}
% \newcommand\blfootnote[1]{
%   \let\thefootnote\relax
%   \footnotetext{#1}
%   \let\thefootnote\svthefootnote
% }

\setcounter{figure}{1}

% Include other packages here, before hyperref.
\usepackage{authblk}
% If you comment hyperref and then uncomment it, you should delete
% egpaper.aux before re-running latex.  (Or just hit 'q' on the first latex
% run, let it finish, and you should be clear).
\usepackage[breaklinks=true,bookmarks=false]{hyperref}

\iccvfinalcopy % *** Uncomment this line for the final submission

\def\iccvPaperID{12168} % *** Enter the ICCV Paper ID here
\def\httilde{\mbox{\tt\raisebox{-.5ex}{\symbol{126}}}}

% Pages are numbered in submission mode, and unnumbered in camera-ready
\ificcvfinal\pagestyle{empty}\fi

\begin{document}
%%%%%%%%% TITLE
\title{Edit-A-Video: Single Video Editing with Object-Aware Consistency}

\author{Chaehun Shin$^{*1}$ ~~~~~~~ Heeseung Kim$^{*1}$ ~~~~~~~ Che Hyun Lee$^1$  ~~~~~~~ Sang-gil Lee$^1$ ~~~~~~~ Sungroh Yoon$^{\dag1, 2}$\\
$^1$ Data Science and AI Lab, ECE, Seoul National University, Seoul 08826, Korea\\
$^2$ Interdisciplinary Program in AI, Seoul National University, Seoul 08826, Korea\\
{\tt\small \{chaehuny, gmltmd789, saga1214, tkdrlf9202, sryoon\}@snu.ac.kr}\\
\tt\small\href{https://edit-a-video.github.io/}{\color{magenta}{\textbf{https://edit-a-video.github.io/}}}
}
\twocolumn[
{\renewcommand\twocolumn[1][]{#1}
\vspace{-2.5em}
\maketitle
% to be changed
% \begin{figure*}
\vspace{-2.5em}
\begin{center}
\makebox[0.12\textwidth]{\colorbox{pink}{\textbf{Training video}} A man is doing a pushup.}\\
\includegraphics[width=0.12\textwidth]{figures/teaser/training/frame_1.pdf}
\includegraphics[width=0.12\textwidth]{figures/teaser/training/frame_2.pdf}
\includegraphics[width=0.12\textwidth]{figures/teaser/training/frame_3.pdf}
\includegraphics[width=0.12\textwidth]{figures/teaser/training/frame_4.pdf}
\includegraphics[width=0.12\textwidth]{figures/teaser/training/frame_5.pdf}
\includegraphics[width=0.12\textwidth]{figures/teaser/training/frame_6.pdf}
\includegraphics[width=0.12\textwidth]{figures/teaser/training/frame_7.pdf}
\includegraphics[width=0.12\textwidth]{figures/teaser/training/frame_8.pdf}

\makebox[0.12\textwidth]{A \textcolor{blue}{\textbf{Iron Man}} is doing a pushup.}\\
\includegraphics[width=0.12\textwidth]{figures/teaser/hulk/frame_1.pdf}
\includegraphics[width=0.12\textwidth]{figures/teaser/hulk/frame_2.pdf}
\includegraphics[width=0.12\textwidth]{figures/teaser/hulk/frame_3.pdf}
\includegraphics[width=0.12\textwidth]{figures/teaser/hulk/frame_4.pdf}
\includegraphics[width=0.12\textwidth]{figures/teaser/hulk/frame_5.pdf}
\includegraphics[width=0.12\textwidth]{figures/teaser/hulk/frame_6.pdf}
\includegraphics[width=0.12\textwidth]{figures/teaser/hulk/frame_7.pdf}
\includegraphics[width=0.12\textwidth]{figures/teaser/hulk/frame_8.pdf}

\makebox[0.12\textwidth]{A \textcolor{blue}{\textbf{digital illustration}} that a man is doing a pushup.}\\
\includegraphics[width=0.12\textwidth]{figures/teaser/illu/frame_1.pdf}
\includegraphics[width=0.12\textwidth]{figures/teaser/illu/frame_2.pdf}
\includegraphics[width=0.12\textwidth]{figures/teaser/illu/frame_3.pdf}
\includegraphics[width=0.12\textwidth]{figures/teaser/illu/frame_4.pdf}
\includegraphics[width=0.12\textwidth]{figures/teaser/illu/frame_5.pdf}
\includegraphics[width=0.12\textwidth]{figures/teaser/illu/frame_6.pdf}
\includegraphics[width=0.12\textwidth]{figures/teaser/illu/frame_7.pdf}
\includegraphics[width=0.12\textwidth]{figures/teaser/illu/frame_8.pdf}

\makebox[0.12\textwidth]{A \textcolor{blue}{\textbf{gorilla}} is doing a pushup, \textcolor{blue}{\textbf{cartoon style}}.}\\
\includegraphics[width=0.12\textwidth]{figures/teaser/gorilla_cartoon/frame_1.pdf}
\includegraphics[width=0.12\textwidth]{figures/teaser/gorilla_cartoon/frame_2.pdf}
\includegraphics[width=0.12\textwidth]{figures/teaser/gorilla_cartoon/frame_3.pdf}
\includegraphics[width=0.12\textwidth]{figures/teaser/gorilla_cartoon/frame_4.pdf}
\includegraphics[width=0.12\textwidth]{figures/teaser/gorilla_cartoon/frame_5.pdf}
\includegraphics[width=0.12\textwidth]{figures/teaser/gorilla_cartoon/frame_6.pdf}
\includegraphics[width=0.12\textwidth]{figures/teaser/gorilla_cartoon/frame_7.pdf}
\includegraphics[width=0.12\textwidth]{figures/teaser/gorilla_cartoon/frame_8.pdf}
\captionof{Figure}{ Edit-A-Video performs text conditioned video editing from a single $<$text, video$>$ pair and a text-to-image model.}
\label{fig:teaser}
% \vspace{0.8em}
\end{center}
}
% \end{figure*}
]
% Remove page # from the first page of camera-ready.
\ificcvfinal\thispagestyle{empty}\fi


\vspace{0.8em}
%%%%%%%%% ABSTRACT
{\let\thefootnote\relax\footnotetext{$*$ Equal Contribution. $\dag$ Corresponding Author.}}

\begin{abstract}
    Despite the fact that text-to-video (TTV) model has recently achieved remarkable success, there have been few approaches on TTV for its extension to video editing. 
    Motivated by approaches on TTV models adapting from diffusion-based text-to-image (TTI) models, we suggest the video editing framework given only a pretrained TTI model and a single $<$text, video$>$ pair, which we term \textbf{Edit-A-Video}.
    The framework consists of two stages: (1) inflating the 2D model into the 3D model by appending temporal modules and tuning on the source video (2) inverting the source video into the noise and editing with target text prompt and attention map injection.
    Each stage enables the temporal modeling and preservation of semantic attributes of the source video.
    One of the key challenges for video editing include a background inconsistency problem, where the regions not included for the edit suffer from undesirable and inconsistent temporal alterations.
    To mitigate this issue, we also introduce a novel mask blending method, termed as sparse-causal blending (SC Blending). 
    We improve previous mask blending methods to reflect the temporal consistency so that the area where the editing is applied exhibits smooth transition while also achieving spatio-temporal consistency of the unedited regions.
    We present extensive experimental results over various types of text and videos, and demonstrate the superiority of the proposed method compared to baselines in terms of background consistency, text alignment, and video editing quality.    
\end{abstract}

\section{Introduction}
Recently, generative models have made remarkable progress across various domains. 
Diffusion models \cite{thermodynamics, ddpm, scoresde}, in particular, have shown state-of-the-art generation performance across multiple domains, including image \cite{iddpm, dhariwal2021diffusion} and video \cite{ho2022imagen, ho2022video, singer2022make}. 
The development of large-scale image-text datasets and improvements in multi-modal representation learning, such as CLIP \cite{radford2021learning}, have contributed to the significant progress made by Text-to-Image (TTI) models, where the goal is to generate images based on textual prompts. 
TTI models have gained significant attention in the scientific research community and industrial applications alike because of their controllability corresponding to the conditional information represented as natural text.

In addition to generating images from text prompts, several pioneering works have used a pretrained TTI model for extended applications, such as adding specific concepts to the model or editing images. For example, DreamBooth \cite{ruiz2022dreambooth}, Textual Inversion \cite{gal2022image}, and Custom Diffusion \cite{kumari2022multi} fine-tune either the text embedding or the model weight itself to map specific concepts from a set of images to new text tokens. Using these personalized tokens, users can generate images with unique concepts based on user preference. In contrast, Prompt-to-Prompt image editing \cite{hertz2022prompt}, Null-text Inversion \cite{mokady2022null}, Imagic \cite{kawar2022imagic}, DiffEDIT \cite{couairon2022diffedit}, and Plug-and-Play diffusion features \cite{tumanyan2022plug} propose prompt-based image editing. 
These approaches offer a convenient and intuitive interface for editing images.


Inspired by the success of the diffusion-based TTI models, several pioneering works have extended its output modality to videos, i.e., \textit{text-to-video} (TTV). 
TTV models can be trained from scratch \cite{ho2022imagen} or fine-tuned from TTI models \cite{singer2022make, wu2022tune}, with additional modules for capturing spatio-temporal relationships. 
Moreover, several studies have focused on video editing \cite{bar2022text2live, loeschcke2022text, molad2023dreamix}. In general, such approaches edit the video based on the spatio-temporal model trained on a large-scale video dataset.

Our key motivation starts from the fact that previous works on TTV are based on extending the pretrained TTI models efficiently. Inspired by Tune-A-Video \cite{wu2022tune}, which proposes \textit{one-shot video generation}, we leverage the pretrained TTI model to enable \textit{one-shot video editing} functionality based on the user-defined text prompt without the need for a large amount of paired text-video data. To the best of our knowledge, our work, which we call \textit{Edit-A-Video}, is the first to propose a text-guided semantic video editing framework using a single $<$text, video$>$ pair.
 
Edit-A-Video is a two-stage framework (see Fig. \ref{fig1} for illustration). In the first stage, a pretrained 2D TTI model is inflated to the 3D TTV model and fine-tuned using a single video to handle spatio-temporal modeling from a given source video. 
In the second stage, the source video is edited to match the target text description by inverting the source video to the Gaussian noise and injecting attention maps from the source to the target along the denoising process.

A key challenge towards diffusion-based video editing is an accurate handling of the target object and the background. We observe that the edited video suffers from a \textit{background inconsistency problem}, where the edited video contains abrupt temporal changes in the background, which significantly degrades the quality. To tackle this issue, we propose a novel masking method tailored to the task, called \textit{sparse-causal blending (SC Blending)}. SC Blending extends a spatial local blending masking technique originally proposed for a still image \cite{hertz2022prompt} and automatically generates a sharp, spatio-temporally consistent blending mask that closely approximates the region to be edited in the source video, based on the object to edit in the text prompt.


Together with the proposed SC Blending method, Edit-A-Video can effectively generate a realistic video that matches the target text description and captures the dynamic actions of the source video ensuring a smooth transition, while also maintaining the spatio-temporal consistency of the background.
We carry out extensive experiments on one-shot editing over various videos and prompts, and demonstrate that Edit-A-Video is the most favorable method towards high quality video editing compared to baselines through subjective human preference study along with qualitative analysis.
We further conduct an in-depth analysis by comparing to baselines based on automatic evaluation of numerical metrics, which shows the improvement of consistency and text alignment.
We additionally perform an ablation study on the proposed SC Blending, and show that it improves the frame consistency in the edited results.

In summary, our key contributions are as follows:
\begin{itemize}
    \item This study presents Edit-A-Video, which is the first to propose a one-shot video editing framework that leverages pretrained text-to-image models.
    \item Along with the extension of image editing techniques to video, we propose a novel blending method called sparse-causal blending (SC Blending) tailored to video editing, which alleviates the background inconsistency problem. 
    \item We extensively validate our framework with comparisons to baselines and ablation studies, and analyze the effect of various attention modules for spatio-temporal consistent modeling for video editing.
\end{itemize}


\section{Background}
\subsection{Denoising Diffusion Probabilistic Models (DDPM)}
Denoising Diffusion Probabilistic Models (DDPM) \cite{thermodynamics, ddpm} is a type of probabilistic generative model that consists of a forward process that gradually transforms data into $z\sim N(0,I)$ and a reverse process that is the opposite trajectory. Following notation in \cite{ddpm}, the forward process of diffusion is defined as follows:
\begin{equation}
  \begin{aligned}
    \label{forward diffusion}
    q(x_t|x_{t-1})&:=N(x_t;\sqrt{1-\beta_t}x_{t-1}, \beta_tI),\\
    x_t&=\sqrt{\bar{\alpha}_t}x_0 + \sqrt{1 - \bar{\alpha}_t}\epsilon_t,
  \end{aligned}
\end{equation}
where $z_t\sim N(0,I)$, $\beta_t$ is a user-defined noise schedule, and $\bar{\alpha}_t:=\prod_{i=1}^{t}(1-\beta_i)$. Using Eq. \ref{forward diffusion}, we can obtain the posterior $q(x_{t-1}|x_t,x_0)=N(x_{t-1};\frac{1}{\sqrt{\bar{\alpha}_t}}(x_t-\frac{\beta_t}{\sqrt{1-\bar{\alpha}_t}}\epsilon_t),\frac{1-\bar{\alpha}_{t-1}}{1-\bar{\alpha}_{t}}\beta_t)$.

For sampling data from noise, we need to define a reverse process that gradually transforms noise into data for sampling. Since $q(x_{t-1}|x_t)$ is intractable, \cite{ddpm} approximates this term to the transition network $p_\theta(x_{t-1}|x_t)$, which is defined in a Gaussian form as below.
\begin{equation}
    \begin{aligned}
    \label{reverse diffusion}
    p_\theta(x_{t-1}|x_t)&:=N(x_{t-1};\mu_\theta(x_t,t), \Sigma_\theta(x_t,t)), \\   
    \mu_\theta(x_t,t)&=\frac{1}{\sqrt{\bar{\alpha}_t}}(x_t-\frac{\beta_t}{\sqrt{1-\bar{\alpha}_t}}\epsilon_\theta(x_t,t)),
    \end{aligned}
\end{equation}
where $\Sigma_\theta(x_t,t)$ can be a fixed value \cite{ddpm} or trainable parameters \cite{iddpm}. On the contrary, DDIM \cite{ddim} proposes to set $\Sigma_\theta(x_t,t)$ to 0 for a subset of the reverse diffusion process, which enables a deterministic and fast sampling. In this paper, we choose not to train and use a fixed value $\Sigma(x_t,t)$, same as \cite{ddim}. According to \cite{ddim}, $\epsilon_\theta(x_t,t)$, can be trained by minimizing $\mathop{\mathbb{E}}_{t,x_0,\epsilon}[||\epsilon-\epsilon_\theta(x_t,t)||^2]$.

We can sample the data by applying the fixed $\Sigma(x_t,t)$ and the estimated $\epsilon_\theta(x_t,t)$ into Equation 2. For editing, only the editing target needs to be changed and the other parts should not be affected, which can be benefited by opting for the sampling method with reduced stochasticity. Therefore, for image and video editing, deterministic sampling is preferred over stochastic sampling in the existing literature. We also apply the deterministic sampling scheme following \cite{ddim}.

\subsection{Tune-A-Video}
Compared to text-to-image generation, text-to-video generation~\cite{singer2022make,ho2022imagen,esser2023structure} is a data-hungry task since the training of TTV model requires large $<$text, video$>$ paired dataset.
Recently, Tune-A-Video \cite{wu2022tune} successfully trained a TTV model with a single video using a pretrained TTI model, while maintaining consistency between frames. To generate a temporally-coherent video, Tune-A-Video inflated 3x3 2D convolution layers from TTI models to 1x3x3 3D convolution layers, and appended additional temporal attention (TA) modules between frames. Furthermore, Tune-A-Video proposed a novel attention layer named as sparse-causal attention (SCA), where SCA limits the model to attend only the first and previous frames instead of every other frame. SCA reduces the computational cost from the attention which was initially quadratic to the number of frames in video to linear, while increasing the temporal consistency in one-shot video generation. By tuning three types of attention layers (TA, SCA, and cross-attention between text and video) from the inflated model, a temporally-coherent video can be generated by leveraging rich prior knowledge from pretrained TTI models which also can capture the action from video source well.

While Tune-A-Video can \textit{generate} a different video using a modified text prompt, the presented \textit{editing} performance of the model is restricted to the main objects in the text prompt. For example, background images in the frames generated by Tune-A-Video, which are generally not the main target for the edit, show significantly worse spatio-temporal consistency compared to the source video. In contrast, our work, Edit-A-Video, can edit a video with given text prompts while keeping the attributes unrelated to the modified text unchanged based on the methods from image editing.
% 수식?

\begin{figure*}[t]
    \centering
    \includegraphics[width=\linewidth]{figures/arch.pdf}
    \caption{\textbf{Overall editing procedure of Edit-A-Video} In stage 1, Edit-A-Video inflates the 2D UNet into 3D model by appending the temporal module and evolving the 2D conv into 3D conv and self-attention into sparse-causal attention. Then, the source video is inverted to a specific Gaussian noise by DDIM inversion, and null-text embedding is optimized so that the source video can be reconstructed along with the null-text guidance in stage 2(a). Finally, in stage 2(b), the edited video is generated from the inverted noise and target text through attention map injection, SC Blending, and optimized null-text embedding guidance.} 
    \label{fig1}
    \vskip -0.1in
\end{figure*}

\subsection{ Real Image Editing with Null-Text Inversion} \label{null-text}

Prompt-based image editing is a task to edit an image with text prompts only, without explicit image masks indicating the image area to modify. While TTI models can produce high-quality images from the given text prompts, the model tends to generate completely different images when the new target text prompt is provided for editing. Prompt-to-Prompt (PTP) \cite{hertz2022prompt} proposed a novel framework for image editing based on an observation that cross-attention maps between the source text prompt and the image reflect the spatial layout of the source image. PTP injects the attention maps of the source to target for editing without compromising the internal structure of the source image. Furthermore, PTP approximates the desired image mask for editing using a cross-attention map between the text prompt and the image. The source image is then \textit{blended} using the approximated image mask to preserve the unedited region of the source image.


Although PTP is a strong framework for prompt-based image editing, the application is limited to synthetic images. 
DDIM inversion is one of the methods to invert the real image to a latent vector. 
However, DDIM trajectory is severely distorted and the reconstruction of source image often results in poor quality when a classifier-free guidance~\cite{pmlr-v162-nichol22a} is applied for text guiding. 
Null-Text Inversion (NTI) \cite{mokady2022null} proposes a novel inversion method where the null-text embedding used for classifier-free guidance is additionally trained for inversion while using the initial DDIM trajectory as a pivot. NTI results in the near-perfect reconstruction of the image even under text guidance. Together with PTP framework, NTI shows that prompt-based image editing can also be effective for a real image.

Despite the success of prompt-based image editing, the image editing methods cannot be directly applied to inflated 3D models for video editing.
Compared to TTI models, TTV models need an additional method to maintain the temporal consistency between frames.
We hence append sparse-causal attention (SCA) and temporal attention (TA) on top of NTI and PTP framework and propose a novel blending method called sparse-causal blending (SC Blending) for successful video editing.

\section{Method}
In this section, we introduce Edit-A-Video, a framework designed to consistently edit a given video into a desired object or style using a diffusion-based TTI model.
In Sec. \ref{framework}, we formulate our 2 stage editing procedure, extending TTI model to learn the temporal relationship and successfully editing the contents of the source video according to the target prompt through the attention map injection.
To achieve more consistent editing frame by frame, we propose a sparse-causal blending (SC Blending) method in Sec. \ref{sc}. 
Finally, we discuss the role and effect of three types of attention modules in our methodology in Sec. \ref{hparams}.


\subsection{Framework} \label{framework}

\begin{figure*}[t]
\begin{center}
\makebox[0.12\textwidth]{A man is on the surfing $\rightarrow$ A \textcolor{blue}{\textbf{wooden man sculpture}} is surfing}\\

\includegraphics[width=0.12\textwidth]{figures/ablation/surfing_vis/step_1.pdf}
\includegraphics[width=0.12\textwidth]{figures/ablation/surfing_vis/step_2.pdf}
\includegraphics[width=0.12\textwidth]{figures/ablation/surfing_vis/step_3.pdf}
\includegraphics[width=0.12\textwidth]{figures/ablation/surfing_vis/step_4.pdf}
\includegraphics[width=0.12\textwidth]{figures/ablation/surfing_vis/step_5.pdf}
\includegraphics[width=0.12\textwidth]{figures/ablation/surfing_vis/step_6.pdf}
\includegraphics[width=0.12\textwidth]{figures/ablation/surfing_vis/step_7.pdf}
\includegraphics[width=0.12\textwidth]{figures/ablation/surfing_vis/step_8.pdf}

\includegraphics[width=0.12\textwidth]{figures/ablation/surfing_result/step_1.pdf}
\includegraphics[width=0.12\textwidth]{figures/ablation/surfing_result/step_2.pdf}
\includegraphics[width=0.12\textwidth]{figures/ablation/surfing_result/step_3.pdf}
\includegraphics[width=0.12\textwidth]{figures/ablation/surfing_result/step_4.pdf}
\includegraphics[width=0.12\textwidth]{figures/ablation/surfing_result/step_5.pdf}
\includegraphics[width=0.12\textwidth]{figures/ablation/surfing_result/step_6.pdf}
\includegraphics[width=0.12\textwidth]{figures/ablation/surfing_result/step_7.pdf}
\includegraphics[width=0.12\textwidth]{figures/ablation/surfing_result/step_8.pdf}

\caption{\textbf{Background inconsistency problem} The spatial local blending mask originally proposed in \cite{hertz2022prompt} cannot consider the temporal consistency, resulting in temporally variant background after editing, which can be identified in the waves close to the wooden sculpture.}
\label{fig:bg}
\end{center}
\vspace{-1.3em}
\end{figure*}

\begin{figure}[t]
    \centering
    \includegraphics[width=\linewidth]{figures/scattn.pdf}
    \caption{\textbf{SC Blending mask computation} The first and previous frames interact with the corresponding sparse-causal attention map and yield the new temporal consistent blending mask.}
    \label{fig2}
    \vskip -0.1in
\end{figure} 

As shown in Fig. \ref{fig1}, Edit-A-Video goes through the two-step process to edit the given video to a target prompt. 
In the first stage, we inflate the TTI model to TTV model using the method of Tune-A-Video \cite{wu2022tune}.
Unlike the TTI model, which has two types of attention (self-attention in images and cross-attention between text and image), our TTV framework requires three types of attention: cross-attention, temporal attention, and sparse-causal attention. 
We only train these attention modules in the inflated 3D TTV model using a single video.
This ensures frame-by-frame consistency during video editing.

In the second stage, to edit only the target object while preserving the source video's contents as much as possible, we extract the deterministic latent variable ${z_{T}}$ corresponding to the source video using the DDIM method. 
Next, we train a null-text embedding to reconstruct the source video even when classifier-free guidance is applied for sampling, as described in \ref{null-text}.
Starting from the latent variable ${z_{T}}$ of the source video, we edit the video by injecting three types of attention maps corresponding to the source text prompt and video to target, extending the editing methods used in the image domain \cite{hertz2022prompt, mokady2022null}.
Since the inflated model has newly added attentions previously absent in the TTI model, we analyze the role and effect of each attention module and describe in Sec. \ref{hparams}.

However, when we edit the video as described above, we observe that unwanted region is edited inconsistently along the frames, which results in undesirable and abrupt artifacts.
We call this issue as the \textit{background inconsistency problem}, which occurs due to the inherent temporal property of video as opposed to still image. 
To mitigate this issue, we analyze the cause and propose \textit{a sparse-causal blending}, which we call \textit{SC Blending} for short, in the following section.


\subsection{Sparse-Causal Blending} \label{sc}
When Edit-A-Video edits an object, only the specific region that correlates to the target object should be modified, while the rest of the video should be preserved.
In previous work of image editing, PTP \cite{hertz2022prompt} proposed the local blending method, which approximates a mask of the object region from the cross-attention map and performs editing only in the masked region.
However, since the cross-attention of the inflated 3D model is computed frame-by-frame, the local blending mask from the cross-attention map considers only the spatial dimension and lacks modeling of temporal dependency.
This means that the sequence of blending masks does not show a smooth transition, and color-variant artifacts may appear when the same region is included in blending mask in one frame but not in another.
This background inconsistency problem is demonstrated clearly in Fig. \ref{fig:bg}, where the background region close to the target object of editing is severely distorted, and the frames are highly inconsistent across the temporal axis.

To address this problem, we propose a novel blending method called sparse-causal blending (SC Blending), which acquires a spatio-temporally consistent blending mask.
Motivated by the SC attention of Tune-A-Video \cite{wu2022tune} for efficient temporal modeling, we leverage the SC attention map in our 3D inflated model as a proxy for the interaction between the current frame mask and the first and previous frame mask.
Intuitively, the SC attention map of the current frame with the first frame ensures that the target object is maintained in every mask, while that with the previous frame enforces a smooth transition in the blending mask sequence.
With SC Blending, it is possible to achieve a sharp blending mask that accurately models the target object in the frame and a smooth transition in the blending masks, which mitigates the background inconsistency problem.

Following \cite{hertz2022prompt}, we first acquire the cross-attention map $m \in {\rm I\!R^{F \times H \times W}}$ according to the original word and new word as an initial blending mask, where the $F,H,W$ are the temporal and spatial dimension of the feature.
Then, we normalize the attention map $m$ frame-wise to balance the scale between each frame and flattening:
\begin{equation}
    \tilde{m} = Flatten(m/ \sum_{H,W} (m)), \text{where } \tilde{m} \in {\rm I\!R^{FxHW}}.
\end{equation}
% where $\tilde{m} \in {\rm I\!R^{FxHW}}$.

Afterward, we follow the computation of the sparse-causal attention \cite{wu2022tune} to encourage the interaction between the frame-wise masks from a weighted average of the first and previous mask $[\tilde{m}_{1}; \tilde{m}_{f-1}] \in {\rm I\!R^{(HW\times 2)}}$ with SC Attention map $SCA_{f} \in {\rm I\!R^{(HW\times2)\times(HW)}}$ of the current frame, which is shown in Fig.~\ref{fig2}.
Then, we binarize it by applying thresholding:
\begin{equation}
    \alpha_{f} = B(\hat{m}_{f}, \tau), \hat{m}_{f} = [\tilde{m}_{1}; \tilde{m}_{f-1}] \times SCA_{f},
\end{equation}
where the $[;]$ denotes concatenation, $f$ is the index of current frame, and $B$ is the binarizing function with threshold $\tau$.
We use this binary blending mask to perform local editing, similar to inpainting, where we preserve the background outside the mask and only generate the contents inside the mask:
\begin{equation}
    \hat{z}_{t} = \bar{z}_{t} \odot (1 - \alpha) + z_{t}^{*} \odot \alpha,
\end{equation}
where $\bar{z}_{t}$ is the reconstruction of the source video, $z_{t}^{*}$ is the edited video, and $\odot$ is element-wise multiplication.


\subsection{Hyperparameters for Editing}  \label{hparams}
The core idea of Prompt-to-Prompt (PTP) to preserve the spatial layout of input is by injecting 2D cross-attention maps and self-attention maps from pretrained TTI models.
The duration of the injection process, a portion of timestep over the entire sampling step until which the attention map is injected, is a key factor in controlling the reflection ratio of the target prompt. 
Edit-A-Video, on the other hand, uses sparse-causal attention and temporal attention instead of self-attention.
As a result, the effect of the duration of attention injection may differ from that of PTP.
In this section, we describe the effect of injection for each type of attention, and the corresponding samples are visualized in Fig.~\ref{fig:supp_attention}.


\textbf{Cross-Attention}
The cross-attention layer performs an attention operation between text tokens and frames, taking into account the spatial layout of each text token in the frames. There is a trade-off depending on the duration of the injection phase. If injection occurs only at the beginning of generation, the generated frames are strongly conditioned on the target text prompt, but hard to maintain the spatial layout. 
On the other hand, if injection occurs throughout the entire generation process, 
the spatial layout of the source video is well preserved, yet it does not include the concepts from the text prompt. 
Additionally, we found that the duration of the injection phase affects the consistency between frames along the temporal axis. 
The longer the injection occurs during the generation process, the better the consistency is maintained in the generated video.

\textbf{Temporal Attention}
The temporal attention (TA) layer is an additional attention module computed along the time axis of a video to model the temporal relationship between frames, yet it does not model spatial dependency. 
We found that TA maps are distributed uniformly and that the duration of injection does not have a significant impact.

\textbf{Sparse-Causal Attention}
The sparse-causal attention (SCA) layer is an attention method designed for video, where the model attends only to the first and previous frames. 
SCA replaces the self-attention layer from the pretrained TTI models, where the model only requires the dependencies between pixels in a single image. 
In addition to temporal attention, SCA can improve temporal consistency by attending to other frames while maintaining efficiency by only visiting two frames. As the duration of SCA injection gets longer, we found that SCA helps to generate a video that effectively captures the dynamic actions of objects from text prompts.

\section{Experiments}

\begin{figure*}
\vspace{0.8em}
\begin{center}
\makebox[0.12\textwidth]{\colorbox{pink}{\textbf{Training video}} A woman is on the swing.}\\

\includegraphics[width=0.12\textwidth]{figures/swing/training/00000.pdf}
\includegraphics[width=0.12\textwidth]{figures/swing/training/00001.pdf}
\includegraphics[width=0.12\textwidth]{figures/swing/training/00002.pdf}
\includegraphics[width=0.12\textwidth]{figures/swing/training/00003.pdf}
\includegraphics[width=0.12\textwidth]{figures/swing/training/00004.pdf}
\includegraphics[width=0.12\textwidth]{figures/swing/training/00005.pdf}
\includegraphics[width=0.12\textwidth]{figures/swing/training/00006.pdf}
\includegraphics[width=0.12\textwidth]{figures/swing/training/00007.pdf}


\makebox[0.12\textwidth]{\colorbox{yellow}{\textbf{Edit-A-Video (Ours)}} A \textcolor{blue}{\textbf{watercolor}} painting that a woman is on the swing.}\\

\includegraphics[width=0.12\textwidth]{figures/swing/watercolor/00000.pdf}
\includegraphics[width=0.12\textwidth]{figures/swing/watercolor/00001.pdf}
\includegraphics[width=0.12\textwidth]{figures/swing/watercolor/00002.pdf}
\includegraphics[width=0.12\textwidth]{figures/swing/watercolor/00003.pdf}
\includegraphics[width=0.12\textwidth]{figures/swing/watercolor/00004.pdf}
\includegraphics[width=0.12\textwidth]{figures/swing/watercolor/00005.pdf}
\includegraphics[width=0.12\textwidth]{figures/swing/watercolor/00006.pdf}
\includegraphics[width=0.12\textwidth]{figures/swing/watercolor/00007.pdf}

\makebox[0.12\textwidth]{\colorbox{yellow}{\textbf{Edit-A-Video (Ours)}} A woman is on the swing, \textcolor{blue}{\textbf{Van Gogh}} style.}\\

\includegraphics[width=0.12\textwidth]{figures/swing/vangogh/00000.pdf}
\includegraphics[width=0.12\textwidth]{figures/swing/vangogh/00001.pdf}
\includegraphics[width=0.12\textwidth]{figures/swing/vangogh/00002.pdf}
\includegraphics[width=0.12\textwidth]{figures/swing/vangogh/00003.pdf}
\includegraphics[width=0.12\textwidth]{figures/swing/vangogh/00004.pdf}
\includegraphics[width=0.12\textwidth]{figures/swing/vangogh/00005.pdf}
\includegraphics[width=0.12\textwidth]{figures/swing/vangogh/00006.pdf}
\includegraphics[width=0.12\textwidth]{figures/swing/vangogh/00007.pdf}

\makebox[0.12\textwidth]{\colorbox{yellow}{\textbf{Edit-A-Video (Ours)}} A \textcolor{blue}{\textbf{man}} is on the swing}\\

\includegraphics[width=0.12\textwidth]{figures/swing/man/00000.pdf}
\includegraphics[width=0.12\textwidth]{figures/swing/man/00001.pdf}
\includegraphics[width=0.12\textwidth]{figures/swing/man/00002.pdf}
\includegraphics[width=0.12\textwidth]{figures/swing/man/00003.pdf}
\includegraphics[width=0.12\textwidth]{figures/swing/man/00004.pdf}
\includegraphics[width=0.12\textwidth]{figures/swing/man/00005.pdf}
\includegraphics[width=0.12\textwidth]{figures/swing/man/00006.pdf}
\includegraphics[width=0.12\textwidth]{figures/swing/man/00007.pdf}


\makebox[0.12\textwidth]{\colorbox{yellow}{\textbf{Edit-A-Video (Ours)}} A \textcolor{blue}{\textbf{Iron Man}} is on the swing}\\

\includegraphics[width=0.12\textwidth]{figures/swing/iron_man/00000.pdf}
\includegraphics[width=0.12\textwidth]{figures/swing/iron_man/00001.pdf}
\includegraphics[width=0.12\textwidth]{figures/swing/iron_man/00002.pdf}
\includegraphics[width=0.12\textwidth]{figures/swing/iron_man/00003.pdf}
\includegraphics[width=0.12\textwidth]{figures/swing/iron_man/00004.pdf}
\includegraphics[width=0.12\textwidth]{figures/swing/iron_man/00005.pdf}
\includegraphics[width=0.12\textwidth]{figures/swing/iron_man/00006.pdf}
\includegraphics[width=0.12\textwidth]{figures/swing/iron_man/00007.pdf}

\makebox[0.12\textwidth]{\colorbox{green}{\textbf{Frame-wise Editing}} A \textcolor{blue}{\textbf{Iron Man}} is on the swing}\\

\includegraphics[width=0.12\textwidth]{figures/swing/framewise_ironman/00000.pdf}
\includegraphics[width=0.12\textwidth]{figures/swing/framewise_ironman/00001.pdf}
\includegraphics[width=0.12\textwidth]{figures/swing/framewise_ironman/00002.pdf}
\includegraphics[width=0.12\textwidth]{figures/swing/framewise_ironman/00003.pdf}
\includegraphics[width=0.12\textwidth]{figures/swing/framewise_ironman/00004.pdf}
\includegraphics[width=0.12\textwidth]{figures/swing/framewise_ironman/00005.pdf}
\includegraphics[width=0.12\textwidth]{figures/swing/framewise_ironman/00006.pdf}
\includegraphics[width=0.12\textwidth]{figures/swing/framewise_ironman/00007.pdf}

\makebox[0.12\textwidth]{\colorbox{green}{\textbf{Tune-A-Video}} A \textcolor{blue}{\textbf{Iron Man}} is on the swing}\\

\includegraphics[width=0.12\textwidth]{figures/swing/tuneavideo_ironman/00000.pdf}
\includegraphics[width=0.12\textwidth]{figures/swing/tuneavideo_ironman/00001.pdf}
\includegraphics[width=0.12\textwidth]{figures/swing/tuneavideo_ironman/00002.pdf}
\includegraphics[width=0.12\textwidth]{figures/swing/tuneavideo_ironman/00003.pdf}
\includegraphics[width=0.12\textwidth]{figures/swing/tuneavideo_ironman/00004.pdf}
\includegraphics[width=0.12\textwidth]{figures/swing/tuneavideo_ironman/00005.pdf}
\includegraphics[width=0.12\textwidth]{figures/swing/tuneavideo_ironman/00006.pdf}
\includegraphics[width=0.12\textwidth]{figures/swing/tuneavideo_ironman/00007.pdf}

\makebox[0.12\textwidth]{\colorbox{green}{\textbf{SDEdit}} A \textcolor{blue}{\textbf{Iron Man}} is on the swing}\\

\includegraphics[width=0.12\textwidth]{figures/swing/sdedit_ironman/00000.pdf}
\includegraphics[width=0.12\textwidth]{figures/swing/sdedit_ironman/00001.pdf}
\includegraphics[width=0.12\textwidth]{figures/swing/sdedit_ironman/00002.pdf}
\includegraphics[width=0.12\textwidth]{figures/swing/sdedit_ironman/00003.pdf}
\includegraphics[width=0.12\textwidth]{figures/swing/sdedit_ironman/00004.pdf}
\includegraphics[width=0.12\textwidth]{figures/swing/sdedit_ironman/00005.pdf}
\includegraphics[width=0.12\textwidth]{figures/swing/sdedit_ironman/00006.pdf}
\includegraphics[width=0.12\textwidth]{figures/swing/sdedit_ironman/00007.pdf}

\caption{\textbf{Qualitative results} Edit-A-Video outperforms in editing compared to other baselines.}
\label{fig:main_result}
\end{center}
\end{figure*}

\begin{table*}[t]
\small{
\begin{center}
\begin{tabular}{|c|c|c|c|c|c|}
\hline
Method & Human Preference ($\uparrow$) & Frame Consistency ($\uparrow$) & Prompt Consistency ($\uparrow$) & LPIPS ($\downarrow$) & PSNR ($\uparrow$) \\
\hline\hline
Frame-wise Editing & $2.996 \pm 0.136$ & $91.1731$ & $30.5025$ & $0.2902$ & $21.0335$  \\
Tune-A-Video & $3.220 \pm 0.155$ & $94.5269$ & $32.1523$ & $0.6947$ & $9.4568$ \\
SDEdit & $3.322 \pm 0.134$ & $96.4088$ & $28.0663$ & $0.1960$ & $22.9547$ \\
Edit-A-Video (Ours) & $3.640 \pm 0.125$ & $96.1669$ & $30.2688$ & $0.2466$ & $20.1140$ \\
\hline
\end{tabular}
\end{center}
\vspace{-0.5em}
\caption{\textbf{Qualitative comparisons to baselines} We measure the human preference score and automatic metric scores for the comparisons to baselines. Edit-A-Video is the most favorable editing in the subjective score with statistical significance (p-value $<$ $0.05$ from the Wilcoxon signed-rank test) and exhibits a better trade-off between the consistency and text-alignment than other baselines.}
\vspace{-1.0em}
\label{tab:comparisons}
}
\end{table*}

\subsection{Implementation Details}
We implement our method based on the stable-diffusion-v1-4\footnote{Stable Diffusion: \href{https://github.com/CompVis/stable-diffusion}{https://github.com/CompVis/stable-diffusion}}, publicly available TTI model~\cite{rombach2022high}.
We finetune only the latent diffusion model in TTI model on $8$ frame $512\times512$ video for $300$ steps in temporal modeling and $500$ steps in inversion, while fixing the autoencoder to encode each frame independently.
At inversion and sampling, we use $50$ step DDIM sampler and set the classifier-free guidance scale to $7.5$.
From the analysis introduced in Sec.~\ref{hparams}, we use cross-attention injection duration as $0.2$, sparse-causal attention injection duration as $0.5$, and temporal attention injection duration as $0.8$ and SC blending threshold ($\tau$) as $0.25$.

\begin{table}[t]
\small{
\begin{center}
\begin{tabular}{|c|c|c|}
\hline
Method & w/o SC-Bld & w/ SC-Bld \\
\hline\hline
Human Preference ($\uparrow$) & $3.538 \pm 0.121$ & $3.640 \pm 0.125$ \\
Temporal Consistency ($\uparrow$) & $3.677 \pm 0.104$ & $3.745 \pm 0.107$ \\
LPIPS ($\downarrow$) & $0.2572$ & $0.2466$ \\
PSNR ($\uparrow$) & $19.8760$ & $20.1140$ \\
\hline
\end{tabular}
\end{center}
\vspace{-0.5em}
\caption{\textbf{SC Blending ablation study} SC Blending improves the temporal consistency, which is supported by the human preference with statistical significance (p-value $<$ $0.05$ from the Wilcoxon signed-rank test), background preservation score, and metric scores.
}
\label{tab:ablation}
}
\vspace{-0.5em}
\end{table}

For the quantitative evaluation, we edit a total $100$ $<$text, video$>$ pairs (four captions for each of $25$ videos), and compare our model to baselines by human preference study and automatic evaluation metrics.
In the human preference study, we ask $51$ users to grade the overall quality score of the edited video on a scale of $1-5$ considering three aspects: background preservation, text-alignment, and video realism.
We include detailed explanations in Sec.~\ref{sec:human_preference}.

For the detailed analysis, we evaluate all models with four automatic metrics, two for consistency and two for background preservation.
We measure the frame consistency which quantifies the temporal consistency as an average of the cosine similarity of CLIP image embeddings between all pairs of consecutive frames, and the prompt consistency which estimates how much the editing reflects the target editing by averaging the cosine similarity of CLIP text embedding and CLIP image embeddings of all frames.
We further measure the distance between the source video and the target video for the background preservation by LPIPS~\cite{zhang2018unreasonable} and PSNR following the previous works~\cite{esser2023structure, kawar2022imagic,mokady2022null,hertz2022prompt}.


\subsection{Baseline Comparisons}

We compare our method with three baselines quantitatively and qualitatively:
(1)  \textit{Frame-wise editing}: applying the null-text inversion and Prompt-to-Prompt on each frame individually with pretrained TTI model.
(2) \textit{Tune-A-Video}: generating the video from target prompt after tuning the inflated 3D model with source text-video pair.
(3) \textit{SDEdit}: Based on Tune-A-Video, editing the video with another method, SDEdit~\cite{meng2021sdedit} which injects the noise to video until intermediate timestep $t_{0}=25$ among $50$ steps and denoises from it.

\textbf{Quantitative Results}
We perform the user evaluation to let the participants grade the scores on the edited videos. 
Owing to the proposed techniques, Edit-A-Video achieves superior performance compared to baselines, as shown in table~\ref{tab:comparisons}.


We further measure two types of automatic evaluation metrics to support the user study, which are also included in table~\ref{tab:comparisons}.
In Frame-wise Editing, it shows decent prompt consistency, yet very low frame consistency since there are no dependencies between frames, causing inferior temporal consistency.
Inferior temporal consistency makes the editing result not realistic, so that raters grade the lowest score on Frame-wise Editing.
On the other hand, Tune-A-Video shows better frame consistency than Frame-wise Editing due to the temporal modeling.
However, Tune-A-Video cannot preserve the property of the source video, which is shown in low LPIPS and PSNR, as it generates the new video from the target text, not editing the source video.
This is not desirable because a video editing method should preserve the background or spatial layout of the source video.
This necessitates the conditioning or inversion of the source video.
Through the conditioning on the source video in the generation, SDEdit shows editing which preserves the source video.
However, SDEdit shows the lowest text-alignment as in low prompt consistency.
Owing to both temporal tuning and inversion for editing, Edit-A-Video preserves the source video sufficiently and shows accurate and temporally consistent editing results, the most preferred editing method in the user study.



\textbf{Qualitative Results}
Fig.~\ref{fig:main_result} presents the qualitative results of ours and baselines.
The samples of Frame-wise editing lack temporal consistency between consecutive frames, and Tune-A-Video fails to preserve the unedited details as it performs the text-to-video generation.
Since we use SDEdit with intermediate time $t_{0}$ as 0.5, it is insufficient to change the source video large and cannot reflect the target text accurately.
Compared to baselines, Edit-A-Video demonstrates outperforming results as it shows temporal consistent editing and well preserved background or contents.
More examples are included in Fig. \ref{fig:supp_comparison1}, \ref{fig:supp_comparison2}.

\begin{figure}[t]
% \vspace{-1.8em}
\begin{center}
\makebox[0.12\textwidth]{\colorbox{green}{\textbf{Edit-A-Video w/o SC-Bld}} A \textcolor{blue}{\textbf{Spider Man}} is skiing }\\
\includegraphics[width=0.10\textwidth]{figures/ablation/spiderman_wo_scbld_vis/time_1.pdf}
\includegraphics[width=0.10\textwidth]{figures/ablation/spiderman_wo_scbld_vis/time_3.pdf}
\includegraphics[width=0.10\textwidth]{figures/ablation/spiderman_wo_scbld_vis/time_5.pdf}
\includegraphics[width=0.10\textwidth]{figures/ablation/spiderman_wo_scbld_vis/time_7.pdf}

\includegraphics[width=0.10\textwidth]{figures/ablation/spiderman_wo_scbld_result/step_1.pdf}
\includegraphics[width=0.10\textwidth]{figures/ablation/spiderman_wo_scbld_result/step_3.pdf}
\includegraphics[width=0.10\textwidth]{figures/ablation/spiderman_wo_scbld_result/step_5.pdf}
\includegraphics[width=0.10\textwidth]{figures/ablation/spiderman_wo_scbld_result/step_7.pdf}

\makebox[0.12\textwidth]{\colorbox{yellow}{\textbf{Edit-A-Video w/  SC-Bld}} A \textcolor{blue}{\textbf{Spider Man}} is skiing }\\
\includegraphics[width=0.10\textwidth]{figures/ablation/spiderman_w_scbld_vis/time_1.pdf}
\includegraphics[width=0.10\textwidth]{figures/ablation/spiderman_w_scbld_vis/time_3.pdf}
\includegraphics[width=0.10\textwidth]{figures/ablation/spiderman_w_scbld_vis/time_5.pdf}
\includegraphics[width=0.10\textwidth]{figures/ablation/spiderman_w_scbld_vis/time_7.pdf}

\includegraphics[width=0.10\textwidth]{figures/ablation/spiderman_w_scbld_result/step_1.pdf}
\includegraphics[width=0.10\textwidth]{figures/ablation/spiderman_w_scbld_result/step_3.pdf}
\includegraphics[width=0.10\textwidth]{figures/ablation/spiderman_w_scbld_result/step_5.pdf}
\includegraphics[width=0.10\textwidth]{figures/ablation/spiderman_w_scbld_result/step_7.pdf}

\caption{\textbf{Qualitative ablation study} We visualize the improved blending mask qualitatively, which shows outstanding improvement in content preservation outside the mask.}
\label{fig:ablation}
\end{center}
\vspace{-1.5em}
\end{figure}
\subsection{Ablation}

In this section, we conduct an analysis on the effect of the proposed SC Blending mask through an ablation study.
As shown in Fig.~\ref{fig:ablation}, original blending mask is very broadly captured and temporally varying, which results in inconsistent color or flickering effect after editing.
However, SC Blending method makes the sharp mask concentrated on the specific target and brings improved detail preservation and temporal consistency.
This is also proven through the metric evaluation, where we perform two human preference studies (overall score and background preservation) and two automatic metrics (LPIPS and PSNR).
Raters grade higher scores on our framework with SC Blending, and LPIPS and PSNR improved, as shown in table~\ref{tab:ablation}.




\section{Conclusion}

We propose Edit-A-Video, the video editing framework only given the single $<$Text, Video$>$ pair and pretrained text-to-image (TTI) model.
Edit-A-Video inflates the TTI model and tunes the model on a given source video for the temporal modeling, and enables editing the video with target prompt by the inversion and attention map injection.
We also suggest the sparse-causal blending method which ensures content preservation and temporal coherence by making use of the temporal modeling capability in the model.
Our framework achieves superior performance to the baselines in various aspects.
We anticipate that this method will provide a simple and intuitive video editing method.


{\small
\bibliographystyle{ieee_fullname}
\bibliography{main}
}


\clearpage
\newpage

\twocolumn[
\centering
\Large
\textbf{Edit-A-Video: Single Video Editing with Object-Aware Consistency} \\
\vspace{0.5em}-- Supplementary Materials --\\
\vspace{1.0em}
]
\setcounter{section}{0}
\renewcommand\thesection{\Alph{section}}
% \setcounter{table}{0}
% \renewcommand{\thetable}{\Alph{table}}
% \setcounter{figure}{0}
% \renewcommand{\thefigure}{\Alph{figure}}
% \setcounter{equation}{0}
% \renewcommand{\theequation}{\Alph{equation}}

\section{Human Preference Study}
\label{sec:human_preference}
We conduct two distinct human preference studies. The first one involves evaluating the overall video editing quality of our model and baselines. We collect feedback from 51 participants who are asked to rate scores on a 5-point scale from 1-5 by taking into account three key factors as follows:

\begin{itemize}
    \item \textbf{Background Preservation} Edited video preserves unedited details of the original video.
    \item \textbf{Text-Alignment} Edited video matches the target textual edit description provided.
    \item \textbf{Video Realism} The overall visual quality and smoothness of the edited video.
\end{itemize}

In the second study, we measure the effectiveness of our proposed SC blending technique in terms of the \textit{background inconsistency problem}. We check this by assessing the degree of preservation and consistency of the background in the presence or absence of the SC blending technique. Once again, we ask 51 participants to provide feedback, with scores related to background preservation.

\section{Additional Samples}
\label{sec:add_samples}
We show several additional samples in this section. First, various examples of our model are in Fig. \ref{fig:supp_qual1}, \ref{fig:supp_qual2}, \ref{fig:supp_qual3}, \ref{fig:supp_qual4}. We also compare the generated samples of our model and several baselines in Fig. \ref{fig:supp_comparison1}, \ref{fig:supp_comparison2}. 
In addition, samples according to the adjustment of various hyperparameters of attention injection are shown in Fig. \ref{fig:supp_attention}. Finally, we summarize the cases where our model fails in Fig.~\ref{fig:failure_case}.
Some failure cases are caused when the edited region is very narrow due to the small attention map for the object or when the action of the source video cannot be natural in the target video.

\begin{figure*}
\vspace{0.8em}
\begin{center}
\makebox[0.12\textwidth]{\colorbox{pink}{\textbf{Training video}} A man is dribbling a basketball}\\
\includegraphics[width=0.12\textwidth]{figures/basketball/src/00000.pdf}
\includegraphics[width=0.12\textwidth]{figures/basketball/src/00001.pdf}
\includegraphics[width=0.12\textwidth]{figures/basketball/src/00002.pdf}
\includegraphics[width=0.12\textwidth]{figures/basketball/src/00003.pdf}
\includegraphics[width=0.12\textwidth]{figures/basketball/src/00004.pdf}
\includegraphics[width=0.12\textwidth]{figures/basketball/src/00005.pdf}
\includegraphics[width=0.12\textwidth]{figures/basketball/src/00006.pdf}
\includegraphics[width=0.12\textwidth]{figures/basketball/src/00007.pdf}

\makebox[0.12\textwidth]{A \textcolor{blue}{\textbf{Lionel Messi}} is dribbling a basketball.}\\
\includegraphics[width=0.12\textwidth]{figures/basketball/messi/frame_0.pdf}
\includegraphics[width=0.12\textwidth]{figures/basketball/messi/frame_1.pdf}
\includegraphics[width=0.12\textwidth]{figures/basketball/messi/frame_2.pdf}
\includegraphics[width=0.12\textwidth]{figures/basketball/messi/frame_3.pdf}
\includegraphics[width=0.12\textwidth]{figures/basketball/messi/frame_4.pdf}
\includegraphics[width=0.12\textwidth]{figures/basketball/messi/frame_5.pdf}
\includegraphics[width=0.12\textwidth]{figures/basketball/messi/frame_6.pdf}
\includegraphics[width=0.12\textwidth]{figures/basketball/messi/frame_7.pdf}

\makebox[0.12\textwidth]{A \textcolor{blue}{\textbf{racoon}} is dribbling a basketball.}\\
\includegraphics[width=0.12\textwidth]{figures/basketball/racoon/frame_0.pdf}
\includegraphics[width=0.12\textwidth]{figures/basketball/racoon/frame_1.pdf}
\includegraphics[width=0.12\textwidth]{figures/basketball/racoon/frame_2.pdf}
\includegraphics[width=0.12\textwidth]{figures/basketball/racoon/frame_3.pdf}
\includegraphics[width=0.12\textwidth]{figures/basketball/racoon/frame_4.pdf}
\includegraphics[width=0.12\textwidth]{figures/basketball/racoon/frame_5.pdf}
\includegraphics[width=0.12\textwidth]{figures/basketball/racoon/frame_6.pdf}
\includegraphics[width=0.12\textwidth]{figures/basketball/racoon/frame_7.pdf}

\makebox[0.12\textwidth]{A \textcolor{blue}{\textbf{oil painting}} that a man is dribbling a basketball.}\\
\includegraphics[width=0.12\textwidth]{figures/basketball/oil/frame_0.pdf}
\includegraphics[width=0.12\textwidth]{figures/basketball/oil/frame_1.pdf}
\includegraphics[width=0.12\textwidth]{figures/basketball/oil/frame_2.pdf}
\includegraphics[width=0.12\textwidth]{figures/basketball/oil/frame_3.pdf}
\includegraphics[width=0.12\textwidth]{figures/basketball/oil/frame_4.pdf}
\includegraphics[width=0.12\textwidth]{figures/basketball/oil/frame_5.pdf}
\includegraphics[width=0.12\textwidth]{figures/basketball/oil/frame_6.pdf}
\includegraphics[width=0.12\textwidth]{figures/basketball/oil/frame_7.pdf}


\makebox[0.12\textwidth]{A man is dribbling a basketball, \textcolor{blue}{\textbf{Renoir style}}.}\\
\includegraphics[width=0.12\textwidth]{figures/basketball/renoir/frame_0.pdf}
\includegraphics[width=0.12\textwidth]{figures/basketball/renoir/frame_1.pdf}
\includegraphics[width=0.12\textwidth]{figures/basketball/renoir/frame_2.pdf}
\includegraphics[width=0.12\textwidth]{figures/basketball/renoir/frame_3.pdf}
\includegraphics[width=0.12\textwidth]{figures/basketball/renoir/frame_4.pdf}
\includegraphics[width=0.12\textwidth]{figures/basketball/renoir/frame_5.pdf}
\includegraphics[width=0.12\textwidth]{figures/basketball/renoir/frame_6.pdf}
\includegraphics[width=0.12\textwidth]{figures/basketball/renoir/frame_7.pdf}

\caption{\textbf{Qualitative results} Additional selected samples for our model.}
\label{fig:supp_qual1}
\end{center}
\end{figure*}

\begin{figure*}
\vspace{0.8em}
\begin{center}
\makebox[0.12\textwidth]{\colorbox{pink}{\textbf{Training video}} A man is running}\\
\includegraphics[width=0.12\textwidth]{figures/running/src/00000.pdf}
\includegraphics[width=0.12\textwidth]{figures/running/src/00001.pdf}
\includegraphics[width=0.12\textwidth]{figures/running/src/00002.pdf}
\includegraphics[width=0.12\textwidth]{figures/running/src/00003.pdf}
\includegraphics[width=0.12\textwidth]{figures/running/src/00004.pdf}
\includegraphics[width=0.12\textwidth]{figures/running/src/00005.pdf}
\includegraphics[width=0.12\textwidth]{figures/running/src/00006.pdf}
\includegraphics[width=0.12\textwidth]{figures/running/src/00007.pdf}

\makebox[0.12\textwidth]{A \textcolor{blue}{\textbf{zombie}} is running.}\\
\includegraphics[width=0.12\textwidth]{figures/running/zombie/frame_0.pdf}
\includegraphics[width=0.12\textwidth]{figures/running/zombie/frame_1.pdf}
\includegraphics[width=0.12\textwidth]{figures/running/zombie/frame_2.pdf}
\includegraphics[width=0.12\textwidth]{figures/running/zombie/frame_3.pdf}
\includegraphics[width=0.12\textwidth]{figures/running/zombie/frame_4.pdf}
\includegraphics[width=0.12\textwidth]{figures/running/zombie/frame_5.pdf}
\includegraphics[width=0.12\textwidth]{figures/running/zombie/frame_6.pdf}
\includegraphics[width=0.12\textwidth]{figures/running/zombie/frame_7.pdf}

\makebox[0.12\textwidth]{A \textcolor{blue}{\textbf{werewolf}} is running.}\\
\includegraphics[width=0.12\textwidth]{figures/running/werewolf/frame_0.pdf}
\includegraphics[width=0.12\textwidth]{figures/running/werewolf/frame_1.pdf}
\includegraphics[width=0.12\textwidth]{figures/running/werewolf/frame_2.pdf}
\includegraphics[width=0.12\textwidth]{figures/running/werewolf/frame_3.pdf}
\includegraphics[width=0.12\textwidth]{figures/running/werewolf/frame_4.pdf}
\includegraphics[width=0.12\textwidth]{figures/running/werewolf/frame_5.pdf}
\includegraphics[width=0.12\textwidth]{figures/running/werewolf/frame_6.pdf}
\includegraphics[width=0.12\textwidth]{figures/running/werewolf/frame_7.pdf}

\makebox[0.12\textwidth]{A \textcolor{blue}{\textbf{pencil sketch}} that a man is running.}\\
\includegraphics[width=0.12\textwidth]{figures/running/sketch/frame_0.pdf}
\includegraphics[width=0.12\textwidth]{figures/running/sketch/frame_1.pdf}
\includegraphics[width=0.12\textwidth]{figures/running/sketch/frame_2.pdf}
\includegraphics[width=0.12\textwidth]{figures/running/sketch/frame_3.pdf}
\includegraphics[width=0.12\textwidth]{figures/running/sketch/frame_4.pdf}
\includegraphics[width=0.12\textwidth]{figures/running/sketch/frame_5.pdf}
\includegraphics[width=0.12\textwidth]{figures/running/sketch/frame_6.pdf}
\includegraphics[width=0.12\textwidth]{figures/running/sketch/frame_7.pdf}


\makebox[0.12\textwidth]{A man is running, \textcolor{blue}{\textbf{pixar style}}.}\\
\includegraphics[width=0.12\textwidth]{figures/running/pixar/frame_0.pdf}
\includegraphics[width=0.12\textwidth]{figures/running/pixar/frame_1.pdf}
\includegraphics[width=0.12\textwidth]{figures/running/pixar/frame_2.pdf}
\includegraphics[width=0.12\textwidth]{figures/running/pixar/frame_3.pdf}
\includegraphics[width=0.12\textwidth]{figures/running/pixar/frame_4.pdf}
\includegraphics[width=0.12\textwidth]{figures/running/pixar/frame_5.pdf}
\includegraphics[width=0.12\textwidth]{figures/running/pixar/frame_6.pdf}
\includegraphics[width=0.12\textwidth]{figures/running/pixar/frame_7.pdf}

% \label{fig:main_result}
\caption{\textbf{Qualitative results} Additional selected samples for our model.}
\label{fig:supp_qual2}
\end{center}
\end{figure*}

\begin{figure*}
\vspace{0.8em}
\begin{center}
\makebox[0.12\textwidth]{\colorbox{pink}{\textbf{Training video}} A man is playing a guitar}\\
\includegraphics[width=0.12\textwidth]{figures/guitar/src/00000.pdf}
\includegraphics[width=0.12\textwidth]{figures/guitar/src/00001.pdf}
\includegraphics[width=0.12\textwidth]{figures/guitar/src/00002.pdf}
\includegraphics[width=0.12\textwidth]{figures/guitar/src/00003.pdf}
\includegraphics[width=0.12\textwidth]{figures/guitar/src/00004.pdf}
\includegraphics[width=0.12\textwidth]{figures/guitar/src/00005.pdf}
\includegraphics[width=0.12\textwidth]{figures/guitar/src/00006.pdf}
\includegraphics[width=0.12\textwidth]{figures/guitar/src/00007.pdf}

\makebox[0.12\textwidth]{A \textcolor{blue}{\textbf{bear}} is playing a guitar.}\\
\includegraphics[width=0.12\textwidth]{figures/guitar/bear/frame_0.pdf}
\includegraphics[width=0.12\textwidth]{figures/guitar/bear/frame_1.pdf}
\includegraphics[width=0.12\textwidth]{figures/guitar/bear/frame_2.pdf}
\includegraphics[width=0.12\textwidth]{figures/guitar/bear/frame_3.pdf}
\includegraphics[width=0.12\textwidth]{figures/guitar/bear/frame_4.pdf}
\includegraphics[width=0.12\textwidth]{figures/guitar/bear/frame_5.pdf}
\includegraphics[width=0.12\textwidth]{figures/guitar/bear/frame_6.pdf}
\includegraphics[width=0.12\textwidth]{figures/guitar/bear/frame_7.pdf}

\makebox[0.12\textwidth]{A \textcolor{blue}{\textbf{monkey}} is playing a guitar.}\\
\includegraphics[width=0.12\textwidth]{figures/guitar/monkey/frame_0.pdf}
\includegraphics[width=0.12\textwidth]{figures/guitar/monkey/frame_1.pdf}
\includegraphics[width=0.12\textwidth]{figures/guitar/monkey/frame_2.pdf}
\includegraphics[width=0.12\textwidth]{figures/guitar/monkey/frame_3.pdf}
\includegraphics[width=0.12\textwidth]{figures/guitar/monkey/frame_4.pdf}
\includegraphics[width=0.12\textwidth]{figures/guitar/monkey/frame_5.pdf}
\includegraphics[width=0.12\textwidth]{figures/guitar/monkey/frame_6.pdf}
\includegraphics[width=0.12\textwidth]{figures/guitar/monkey/frame_7.pdf}

\makebox[0.12\textwidth]{A man is running, \textcolor{blue}{\textbf{cartoon style}}.}\\
\includegraphics[width=0.12\textwidth]{figures/guitar/cartoon/frame_0.pdf}
\includegraphics[width=0.12\textwidth]{figures/guitar/cartoon/frame_1.pdf}
\includegraphics[width=0.12\textwidth]{figures/guitar/cartoon/frame_2.pdf}
\includegraphics[width=0.12\textwidth]{figures/guitar/cartoon/frame_3.pdf}
\includegraphics[width=0.12\textwidth]{figures/guitar/cartoon/frame_4.pdf}
\includegraphics[width=0.12\textwidth]{figures/guitar/cartoon/frame_5.pdf}
\includegraphics[width=0.12\textwidth]{figures/guitar/cartoon/frame_6.pdf}
\includegraphics[width=0.12\textwidth]{figures/guitar/cartoon/frame_7.pdf}


\makebox[0.12\textwidth]{A man is running, \textcolor{blue}{\textbf{Matisse style}}.}\\
\includegraphics[width=0.12\textwidth]{figures/guitar/matisse/frame_0.pdf}
\includegraphics[width=0.12\textwidth]{figures/guitar/matisse/frame_1.pdf}
\includegraphics[width=0.12\textwidth]{figures/guitar/matisse/frame_2.pdf}
\includegraphics[width=0.12\textwidth]{figures/guitar/matisse/frame_3.pdf}
\includegraphics[width=0.12\textwidth]{figures/guitar/matisse/frame_4.pdf}
\includegraphics[width=0.12\textwidth]{figures/guitar/matisse/frame_5.pdf}
\includegraphics[width=0.12\textwidth]{figures/guitar/matisse/frame_6.pdf}
\includegraphics[width=0.12\textwidth]{figures/guitar/matisse/frame_7.pdf}

% \label{fig:main_result}
\caption{\textbf{Qualitative results} Additional selected samples for our model.}
\label{fig:supp_qual3}
\end{center}
\end{figure*}

\begin{figure*}
\vspace{0.8em}
\begin{center}
\makebox[0.12\textwidth]{\colorbox{pink}{\textbf{Training video}} A cow is walking}\\
\includegraphics[width=0.12\textwidth]{figures/cow/src/00000.pdf}
\includegraphics[width=0.12\textwidth]{figures/cow/src/00001.pdf}
\includegraphics[width=0.12\textwidth]{figures/cow/src/00002.pdf}
\includegraphics[width=0.12\textwidth]{figures/cow/src/00003.pdf}
\includegraphics[width=0.12\textwidth]{figures/cow/src/00004.pdf}
\includegraphics[width=0.12\textwidth]{figures/cow/src/00005.pdf}
\includegraphics[width=0.12\textwidth]{figures/cow/src/00006.pdf}
\includegraphics[width=0.12\textwidth]{figures/cow/src/00007.pdf}

\makebox[0.12\textwidth]{A \textcolor{blue}{\textbf{zebra}} is walking.}\\
\includegraphics[width=0.12\textwidth]{figures/cow/zebra/frame_0.pdf}
\includegraphics[width=0.12\textwidth]{figures/cow/zebra/frame_1.pdf}
\includegraphics[width=0.12\textwidth]{figures/cow/zebra/frame_2.pdf}
\includegraphics[width=0.12\textwidth]{figures/cow/zebra/frame_3.pdf}
\includegraphics[width=0.12\textwidth]{figures/cow/zebra/frame_4.pdf}
\includegraphics[width=0.12\textwidth]{figures/cow/zebra/frame_5.pdf}
\includegraphics[width=0.12\textwidth]{figures/cow/zebra/frame_6.pdf}
\includegraphics[width=0.12\textwidth]{figures/cow/zebra/frame_7.pdf}

\makebox[0.12\textwidth]{A \textcolor{blue}{\textbf{bull}} is walking.}\\
\includegraphics[width=0.12\textwidth]{figures/cow/bull/frame_0.pdf}
\includegraphics[width=0.12\textwidth]{figures/cow/bull/frame_1.pdf}
\includegraphics[width=0.12\textwidth]{figures/cow/bull/frame_2.pdf}
\includegraphics[width=0.12\textwidth]{figures/cow/bull/frame_3.pdf}
\includegraphics[width=0.12\textwidth]{figures/cow/bull/frame_4.pdf}
\includegraphics[width=0.12\textwidth]{figures/cow/bull/frame_5.pdf}
\includegraphics[width=0.12\textwidth]{figures/cow/bull/frame_6.pdf}
\includegraphics[width=0.12\textwidth]{figures/cow/bull/frame_7.pdf}

\makebox[0.12\textwidth]{A cow is walking, \textcolor{blue}{\textbf{on the snow}}.}\\
\includegraphics[width=0.12\textwidth]{figures/cow/snow/frame_0.pdf}
\includegraphics[width=0.12\textwidth]{figures/cow/snow/frame_1.pdf}
\includegraphics[width=0.12\textwidth]{figures/cow/snow/frame_2.pdf}
\includegraphics[width=0.12\textwidth]{figures/cow/snow/frame_3.pdf}
\includegraphics[width=0.12\textwidth]{figures/cow/snow/frame_4.pdf}
\includegraphics[width=0.12\textwidth]{figures/cow/snow/frame_5.pdf}
\includegraphics[width=0.12\textwidth]{figures/cow/snow/frame_6.pdf}
\includegraphics[width=0.12\textwidth]{figures/cow/snow/frame_7.pdf}


\makebox[0.12\textwidth]{A cow is walking, \textcolor{blue}{\textbf{on the desert}}.}\\
\includegraphics[width=0.12\textwidth]{figures/cow/desert/frame_0.pdf}
\includegraphics[width=0.12\textwidth]{figures/cow/desert/frame_1.pdf}
\includegraphics[width=0.12\textwidth]{figures/cow/desert/frame_2.pdf}
\includegraphics[width=0.12\textwidth]{figures/cow/desert/frame_3.pdf}
\includegraphics[width=0.12\textwidth]{figures/cow/desert/frame_4.pdf}
\includegraphics[width=0.12\textwidth]{figures/cow/desert/frame_5.pdf}
\includegraphics[width=0.12\textwidth]{figures/cow/desert/frame_6.pdf}
\includegraphics[width=0.12\textwidth]{figures/cow/desert/frame_7.pdf}

\caption{\textbf{Qualitative results} Additional selected samples for our model.}
\label{fig:supp_qual4}
\end{center}
\end{figure*}

\begin{figure*}
\vspace{0.8em}
\begin{center}
\makebox[0.12\textwidth]{\colorbox{pink}{\textbf{Training video}} A man is boxing}\\

\includegraphics[width=0.12\textwidth]{figures/boxing/src/00000.pdf}
\includegraphics[width=0.12\textwidth]{figures/boxing/src/00001.pdf}
\includegraphics[width=0.12\textwidth]{figures/boxing/src/00002.pdf}
\includegraphics[width=0.12\textwidth]{figures/boxing/src/00003.pdf}
\includegraphics[width=0.12\textwidth]{figures/boxing/src/00004.pdf}
\includegraphics[width=0.12\textwidth]{figures/boxing/src/00005.pdf}
\includegraphics[width=0.12\textwidth]{figures/boxing/src/00006.pdf}
\includegraphics[width=0.12\textwidth]{figures/boxing/src/00007.pdf}

\makebox[0.12\textwidth]{\colorbox{yellow}{\textbf{Edit-A-Video (Ours)}} A \textcolor{blue}{\textbf{Bat Man}} is boxing}\\

\includegraphics[width=0.12\textwidth]{figures/boxing/ours/frame_0.pdf}
\includegraphics[width=0.12\textwidth]{figures/boxing/ours/frame_1.pdf}
\includegraphics[width=0.12\textwidth]{figures/boxing/ours/frame_2.pdf}
\includegraphics[width=0.12\textwidth]{figures/boxing/ours/frame_3.pdf}
\includegraphics[width=0.12\textwidth]{figures/boxing/ours/frame_4.pdf}
\includegraphics[width=0.12\textwidth]{figures/boxing/ours/frame_5.pdf}
\includegraphics[width=0.12\textwidth]{figures/boxing/ours/frame_6.pdf}
\includegraphics[width=0.12\textwidth]{figures/boxing/ours/frame_7.pdf}

\makebox[0.12\textwidth]{\colorbox{green}{\textbf{Frame-wise Editing}} A \textcolor{blue}{\textbf{Bat Man}} is boxing}\\

\includegraphics[width=0.12\textwidth]{figures/boxing/framewise/frame_0.pdf}
\includegraphics[width=0.12\textwidth]{figures/boxing/framewise/frame_1.pdf}
\includegraphics[width=0.12\textwidth]{figures/boxing/framewise/frame_2.pdf}
\includegraphics[width=0.12\textwidth]{figures/boxing/framewise/frame_3.pdf}
\includegraphics[width=0.12\textwidth]{figures/boxing/framewise/frame_4.pdf}
\includegraphics[width=0.12\textwidth]{figures/boxing/framewise/frame_5.pdf}
\includegraphics[width=0.12\textwidth]{figures/boxing/framewise/frame_6.pdf}
\includegraphics[width=0.12\textwidth]{figures/boxing/framewise/frame_7.pdf}

\makebox[0.12\textwidth]{\colorbox{green}{\textbf{Tune-A-Video}} A \textcolor{blue}{\textbf{Bat Man}} is boxing}\\
\includegraphics[width=0.12\textwidth]{figures/boxing/tuneavideo/frame_0.pdf}
\includegraphics[width=0.12\textwidth]{figures/boxing/tuneavideo/frame_1.pdf}
\includegraphics[width=0.12\textwidth]{figures/boxing/tuneavideo/frame_2.pdf}
\includegraphics[width=0.12\textwidth]{figures/boxing/tuneavideo/frame_3.pdf}
\includegraphics[width=0.12\textwidth]{figures/boxing/tuneavideo/frame_4.pdf}
\includegraphics[width=0.12\textwidth]{figures/boxing/tuneavideo/frame_5.pdf}
\includegraphics[width=0.12\textwidth]{figures/boxing/tuneavideo/frame_6.pdf}
\includegraphics[width=0.12\textwidth]{figures/boxing/tuneavideo/frame_7.pdf}

\makebox[0.12\textwidth]{\colorbox{green}{\textbf{SDEdit}} A \textcolor{blue}{\textbf{Bat Man}} is boxing}\\
\includegraphics[width=0.12\textwidth]{figures/boxing/sdedit/frame_0.pdf}
\includegraphics[width=0.12\textwidth]{figures/boxing/sdedit/frame_1.pdf}
\includegraphics[width=0.12\textwidth]{figures/boxing/sdedit/frame_2.pdf}
\includegraphics[width=0.12\textwidth]{figures/boxing/sdedit/frame_3.pdf}
\includegraphics[width=0.12\textwidth]{figures/boxing/sdedit/frame_4.pdf}
\includegraphics[width=0.12\textwidth]{figures/boxing/sdedit/frame_5.pdf}
\includegraphics[width=0.12\textwidth]{figures/boxing/sdedit/frame_6.pdf}
\includegraphics[width=0.12\textwidth]{figures/boxing/sdedit/frame_7.pdf}

\caption{\textbf{Baseline comparison} Additional samples for our model and baselines.}
\label{fig:supp_comparison1}
\end{center}
\end{figure*}

\begin{figure*}
\vspace{0.8em}
\begin{center}
\makebox[0.12\textwidth]{\colorbox{pink}{\textbf{Training video}} A man is riding a motorcycle}\\

\includegraphics[width=0.12\textwidth]{figures/motorcycle/src/00000.pdf}
\includegraphics[width=0.12\textwidth]{figures/motorcycle/src/00001.pdf}
\includegraphics[width=0.12\textwidth]{figures/motorcycle/src/00002.pdf}
\includegraphics[width=0.12\textwidth]{figures/motorcycle/src/00003.pdf}
\includegraphics[width=0.12\textwidth]{figures/motorcycle/src/00004.pdf}
\includegraphics[width=0.12\textwidth]{figures/motorcycle/src/00005.pdf}
\includegraphics[width=0.12\textwidth]{figures/motorcycle/src/00006.pdf}
\includegraphics[width=0.12\textwidth]{figures/motorcycle/src/00007.pdf}

\makebox[0.12\textwidth]{\colorbox{yellow}{\textbf{Edit-A-Video (Ours)}} A \textcolor{blue}{\textbf{gorilla}} is riding a motorcycle}\\

\includegraphics[width=0.12\textwidth]{figures/motorcycle/ours/frame_0.pdf}
\includegraphics[width=0.12\textwidth]{figures/motorcycle/ours/frame_1.pdf}
\includegraphics[width=0.12\textwidth]{figures/motorcycle/ours/frame_2.pdf}
\includegraphics[width=0.12\textwidth]{figures/motorcycle/ours/frame_3.pdf}
\includegraphics[width=0.12\textwidth]{figures/motorcycle/ours/frame_4.pdf}
\includegraphics[width=0.12\textwidth]{figures/motorcycle/ours/frame_5.pdf}
\includegraphics[width=0.12\textwidth]{figures/motorcycle/ours/frame_6.pdf}
\includegraphics[width=0.12\textwidth]{figures/motorcycle/ours/frame_7.pdf}

\makebox[0.12\textwidth]{\colorbox{green}{\textbf{Frame-wise Editing}} A \textcolor{blue}{\textbf{gorilla}} is riding a motorcycle}\\

\includegraphics[width=0.12\textwidth]{figures/motorcycle/framewise/frame_0.pdf}
\includegraphics[width=0.12\textwidth]{figures/motorcycle/framewise/frame_1.pdf}
\includegraphics[width=0.12\textwidth]{figures/motorcycle/framewise/frame_2.pdf}
\includegraphics[width=0.12\textwidth]{figures/motorcycle/framewise/frame_3.pdf}
\includegraphics[width=0.12\textwidth]{figures/motorcycle/framewise/frame_4.pdf}
\includegraphics[width=0.12\textwidth]{figures/motorcycle/framewise/frame_5.pdf}
\includegraphics[width=0.12\textwidth]{figures/motorcycle/framewise/frame_6.pdf}
\includegraphics[width=0.12\textwidth]{figures/motorcycle/framewise/frame_7.pdf}

\makebox[0.12\textwidth]{\colorbox{green}{\textbf{Tune-A-Video}} A \textcolor{blue}{\textbf{gorilla}} is riding a motorcycle}\\
\includegraphics[width=0.12\textwidth]{figures/motorcycle/tuneavideo/frame_0.pdf}
\includegraphics[width=0.12\textwidth]{figures/motorcycle/tuneavideo/frame_1.pdf}
\includegraphics[width=0.12\textwidth]{figures/motorcycle/tuneavideo/frame_2.pdf}
\includegraphics[width=0.12\textwidth]{figures/motorcycle/tuneavideo/frame_3.pdf}
\includegraphics[width=0.12\textwidth]{figures/motorcycle/tuneavideo/frame_4.pdf}
\includegraphics[width=0.12\textwidth]{figures/motorcycle/tuneavideo/frame_5.pdf}
\includegraphics[width=0.12\textwidth]{figures/motorcycle/tuneavideo/frame_6.pdf}
\includegraphics[width=0.12\textwidth]{figures/motorcycle/tuneavideo/frame_7.pdf}

\makebox[0.12\textwidth]{\colorbox{green}{\textbf{SDEdit}} A \textcolor{blue}{\textbf{gorilla}} is riding a motorcycle}\\
\includegraphics[width=0.12\textwidth]{figures/motorcycle/sdedit/frame_0.pdf}
\includegraphics[width=0.12\textwidth]{figures/motorcycle/sdedit/frame_1.pdf}
\includegraphics[width=0.12\textwidth]{figures/motorcycle/sdedit/frame_2.pdf}
\includegraphics[width=0.12\textwidth]{figures/motorcycle/sdedit/frame_3.pdf}
\includegraphics[width=0.12\textwidth]{figures/motorcycle/sdedit/frame_4.pdf}
\includegraphics[width=0.12\textwidth]{figures/motorcycle/sdedit/frame_5.pdf}
\includegraphics[width=0.12\textwidth]{figures/motorcycle/sdedit/frame_6.pdf}
\includegraphics[width=0.12\textwidth]{figures/motorcycle/sdedit/frame_7.pdf}

\caption{\textbf{Baseline comparison} Additional samples for our model and baselines.}
\label{fig:supp_comparison2}
\end{center}
\end{figure*}


\begin{figure*}[t]
\vspace{-1.8em}
\begin{center}
\makebox[0.12\textwidth]{\colorbox{pink}{\textbf{Training video}} A man is skiing }\\
\rotatebox{90}{\parbox{0.11\textwidth}{\centering ~ \\ ~ }}
\includegraphics[width=0.11\textwidth]{figures/attn_anal/src/00000.pdf}
\includegraphics[width=0.11\textwidth]{figures/attn_anal/src/00001.pdf}
\includegraphics[width=0.11\textwidth]{figures/attn_anal/src/00002.pdf}
\includegraphics[width=0.11\textwidth]{figures/attn_anal/src/00003.pdf}
\includegraphics[width=0.11\textwidth]{figures/attn_anal/src/00004.pdf}
\includegraphics[width=0.11\textwidth]{figures/attn_anal/src/00005.pdf}
\includegraphics[width=0.11\textwidth]{figures/attn_anal/src/00006.pdf}
\includegraphics[width=0.11\textwidth]{figures/attn_anal/src/00007.pdf}

\makebox[0.12\textwidth]{\colorbox{green}{\textbf{Cross attention}} A \textcolor{blue}{\textbf{Spider Man}} is skiing }\\

\rotatebox{90}{\parbox{0.11\textwidth}{\centering \textbf{duration \\ 0.2}}}
\includegraphics[width=0.11\textwidth]{figures/attn_anal/self0.8_sparse0.5_cross0.2/frame_0.pdf}
\includegraphics[width=0.11\textwidth]{figures/attn_anal/self0.8_sparse0.5_cross0.2/frame_1.pdf}
\includegraphics[width=0.11\textwidth]{figures/attn_anal/self0.8_sparse0.5_cross0.2/frame_2.pdf}
\includegraphics[width=0.11\textwidth]{figures/attn_anal/self0.8_sparse0.5_cross0.2/frame_3.pdf}
\includegraphics[width=0.11\textwidth]{figures/attn_anal/self0.8_sparse0.5_cross0.2/frame_4.pdf}
\includegraphics[width=0.11\textwidth]{figures/attn_anal/self0.8_sparse0.5_cross0.2/frame_5.pdf}
\includegraphics[width=0.11\textwidth]{figures/attn_anal/self0.8_sparse0.5_cross0.2/frame_6.pdf}
\includegraphics[width=0.11\textwidth]{figures/attn_anal/self0.8_sparse0.5_cross0.2/frame_7.pdf}

\rotatebox{90}{\parbox{0.11\textwidth}{\centering duration \\ 0.5}}
\includegraphics[width=0.11\textwidth]{figures/attn_anal/self0.8_sparse0.5_cross0.5/frame_0.pdf}
\includegraphics[width=0.11\textwidth]{figures/attn_anal/self0.8_sparse0.5_cross0.5/frame_1.pdf}
\includegraphics[width=0.11\textwidth]{figures/attn_anal/self0.8_sparse0.5_cross0.5/frame_2.pdf}
\includegraphics[width=0.11\textwidth]{figures/attn_anal/self0.8_sparse0.5_cross0.5/frame_3.pdf}
\includegraphics[width=0.11\textwidth]{figures/attn_anal/self0.8_sparse0.5_cross0.5/frame_4.pdf}
\includegraphics[width=0.11\textwidth]{figures/attn_anal/self0.8_sparse0.5_cross0.5/frame_5.pdf}
\includegraphics[width=0.11\textwidth]{figures/attn_anal/self0.8_sparse0.5_cross0.5/frame_6.pdf}
\includegraphics[width=0.11\textwidth]{figures/attn_anal/self0.8_sparse0.5_cross0.5/frame_7.pdf}

\rotatebox{90}{\parbox{0.11\textwidth}{\centering duration \\ 0.8}}
\includegraphics[width=0.11\textwidth]{figures/attn_anal/self0.8_sparse0.5_cross0.8/frame_0.pdf}
\includegraphics[width=0.11\textwidth]{figures/attn_anal/self0.8_sparse0.5_cross0.8/frame_1.pdf}
\includegraphics[width=0.11\textwidth]{figures/attn_anal/self0.8_sparse0.5_cross0.8/frame_2.pdf}
\includegraphics[width=0.11\textwidth]{figures/attn_anal/self0.8_sparse0.5_cross0.8/frame_3.pdf}
\includegraphics[width=0.11\textwidth]{figures/attn_anal/self0.8_sparse0.5_cross0.8/frame_4.pdf}
\includegraphics[width=0.11\textwidth]{figures/attn_anal/self0.8_sparse0.5_cross0.8/frame_5.pdf}
\includegraphics[width=0.11\textwidth]{figures/attn_anal/self0.8_sparse0.5_cross0.8/frame_6.pdf}
\includegraphics[width=0.11\textwidth]{figures/attn_anal/self0.8_sparse0.5_cross0.8/frame_7.pdf}

\makebox[0.12\textwidth]{\colorbox{green}{\textbf{Temporal attention}} A \textcolor{blue}{\textbf{Spider Man}} is skiing }\\
\rotatebox{90}{\parbox{0.11\textwidth}{\centering duration \\ 0.2}}
\includegraphics[width=0.11\textwidth]{figures/attn_anal/self0.2_sparse0.5_cross0.2/frame_0.pdf}
\includegraphics[width=0.11\textwidth]{figures/attn_anal/self0.2_sparse0.5_cross0.2/frame_1.pdf}
\includegraphics[width=0.11\textwidth]{figures/attn_anal/self0.2_sparse0.5_cross0.2/frame_2.pdf}
\includegraphics[width=0.11\textwidth]{figures/attn_anal/self0.2_sparse0.5_cross0.2/frame_3.pdf}
\includegraphics[width=0.11\textwidth]{figures/attn_anal/self0.2_sparse0.5_cross0.2/frame_4.pdf}
\includegraphics[width=0.11\textwidth]{figures/attn_anal/self0.2_sparse0.5_cross0.2/frame_5.pdf}
\includegraphics[width=0.11\textwidth]{figures/attn_anal/self0.2_sparse0.5_cross0.2/frame_6.pdf}
\includegraphics[width=0.11\textwidth]{figures/attn_anal/self0.2_sparse0.5_cross0.2/frame_7.pdf}

\rotatebox{90}{\parbox{0.11\textwidth}{\centering duration \\ 0.5}}
\includegraphics[width=0.11\textwidth]{figures/attn_anal/self0.5_sparse0.5_cross0.2/frame_0.pdf}
\includegraphics[width=0.11\textwidth]{figures/attn_anal/self0.5_sparse0.5_cross0.2/frame_1.pdf}
\includegraphics[width=0.11\textwidth]{figures/attn_anal/self0.5_sparse0.5_cross0.2/frame_2.pdf}
\includegraphics[width=0.11\textwidth]{figures/attn_anal/self0.5_sparse0.5_cross0.2/frame_3.pdf}
\includegraphics[width=0.11\textwidth]{figures/attn_anal/self0.5_sparse0.5_cross0.2/frame_4.pdf}
\includegraphics[width=0.11\textwidth]{figures/attn_anal/self0.5_sparse0.5_cross0.2/frame_5.pdf}
\includegraphics[width=0.11\textwidth]{figures/attn_anal/self0.5_sparse0.5_cross0.2/frame_6.pdf}
\includegraphics[width=0.11\textwidth]{figures/attn_anal/self0.5_sparse0.5_cross0.2/frame_7.pdf}

\rotatebox{90}{\parbox{0.11\textwidth}{\centering \textbf{duration \\ 0.8}}}
\includegraphics[width=0.11\textwidth]{figures/attn_anal/self0.8_sparse0.5_cross0.2/frame_0.pdf}
\includegraphics[width=0.11\textwidth]{figures/attn_anal/self0.8_sparse0.5_cross0.2/frame_1.pdf}
\includegraphics[width=0.11\textwidth]{figures/attn_anal/self0.8_sparse0.5_cross0.2/frame_2.pdf}
\includegraphics[width=0.11\textwidth]{figures/attn_anal/self0.8_sparse0.5_cross0.2/frame_3.pdf}
\includegraphics[width=0.11\textwidth]{figures/attn_anal/self0.8_sparse0.5_cross0.2/frame_4.pdf}
\includegraphics[width=0.11\textwidth]{figures/attn_anal/self0.8_sparse0.5_cross0.2/frame_5.pdf}
\includegraphics[width=0.11\textwidth]{figures/attn_anal/self0.8_sparse0.5_cross0.2/frame_6.pdf}
\includegraphics[width=0.11\textwidth]{figures/attn_anal/self0.8_sparse0.5_cross0.2/frame_7.pdf}

\makebox[0.12\textwidth]{\colorbox{green}{\textbf{Sparse-Causal attention}} A \textcolor{blue}{\textbf{Spider Man}} is skiing }\\
\rotatebox{90}{\parbox{0.11\textwidth}{\centering duration \\ 0.2}}
\includegraphics[width=0.11\textwidth]{figures/attn_anal/self0.8_sparse0.2_cross0.2/frame_0.pdf}
\includegraphics[width=0.11\textwidth]{figures/attn_anal/self0.8_sparse0.2_cross0.2/frame_1.pdf}
\includegraphics[width=0.11\textwidth]{figures/attn_anal/self0.8_sparse0.2_cross0.2/frame_2.pdf}
\includegraphics[width=0.11\textwidth]{figures/attn_anal/self0.8_sparse0.2_cross0.2/frame_3.pdf}
\includegraphics[width=0.11\textwidth]{figures/attn_anal/self0.8_sparse0.2_cross0.2/frame_4.pdf}
\includegraphics[width=0.11\textwidth]{figures/attn_anal/self0.8_sparse0.2_cross0.2/frame_5.pdf}
\includegraphics[width=0.11\textwidth]{figures/attn_anal/self0.8_sparse0.2_cross0.2/frame_6.pdf}
\includegraphics[width=0.11\textwidth]{figures/attn_anal/self0.8_sparse0.2_cross0.2/frame_7.pdf}

\rotatebox{90}{\parbox{0.11\textwidth}{\centering \textbf{duration \\ 0.5}}}
\includegraphics[width=0.11\textwidth]{figures/attn_anal/self0.8_sparse0.5_cross0.2/frame_0.pdf}
\includegraphics[width=0.11\textwidth]{figures/attn_anal/self0.8_sparse0.5_cross0.2/frame_1.pdf}
\includegraphics[width=0.11\textwidth]{figures/attn_anal/self0.8_sparse0.5_cross0.2/frame_2.pdf}
\includegraphics[width=0.11\textwidth]{figures/attn_anal/self0.8_sparse0.5_cross0.2/frame_3.pdf}
\includegraphics[width=0.11\textwidth]{figures/attn_anal/self0.8_sparse0.5_cross0.2/frame_4.pdf}
\includegraphics[width=0.11\textwidth]{figures/attn_anal/self0.8_sparse0.5_cross0.2/frame_5.pdf}
\includegraphics[width=0.11\textwidth]{figures/attn_anal/self0.8_sparse0.5_cross0.2/frame_6.pdf}
\includegraphics[width=0.11\textwidth]{figures/attn_anal/self0.8_sparse0.5_cross0.2/frame_7.pdf}

\rotatebox{90}{\parbox{0.11\textwidth}{\centering duration \\ 0.8}}
\includegraphics[width=0.11\textwidth]{figures/attn_anal/self0.8_sparse0.8_cross0.2/frame_0.pdf}
\includegraphics[width=0.11\textwidth]{figures/attn_anal/self0.8_sparse0.8_cross0.2/frame_1.pdf}
\includegraphics[width=0.11\textwidth]{figures/attn_anal/self0.8_sparse0.8_cross0.2/frame_2.pdf}
\includegraphics[width=0.11\textwidth]{figures/attn_anal/self0.8_sparse0.8_cross0.2/frame_3.pdf}
\includegraphics[width=0.11\textwidth]{figures/attn_anal/self0.8_sparse0.8_cross0.2/frame_4.pdf}
\includegraphics[width=0.11\textwidth]{figures/attn_anal/self0.8_sparse0.8_cross0.2/frame_5.pdf}
\includegraphics[width=0.11\textwidth]{figures/attn_anal/self0.8_sparse0.8_cross0.2/frame_6.pdf}
\includegraphics[width=0.11\textwidth]{figures/attn_anal/self0.8_sparse0.8_cross0.2/frame_7.pdf}



\caption{\textbf{Attention injection analysis} Samples according to attention injection hyperparameters.}
\label{fig:supp_attention}
\end{center}
\vspace{-1.5em}
\end{figure*}


\begin{figure*}
\vspace{0.8em}
\begin{center}
\makebox[0.12\textwidth]{\colorbox{pink}{\textbf{Training video}} A man is riding a bicycle}\\

\includegraphics[width=0.12\textwidth]{figures/failure_case/bicycle/src/00000.pdf}
\includegraphics[width=0.12\textwidth]{figures/failure_case/bicycle/src/00001.pdf}
\includegraphics[width=0.12\textwidth]{figures/failure_case/bicycle/src/00002.pdf}
\includegraphics[width=0.12\textwidth]{figures/failure_case/bicycle/src/00003.pdf}
\includegraphics[width=0.12\textwidth]{figures/failure_case/bicycle/src/00004.pdf}
\includegraphics[width=0.12\textwidth]{figures/failure_case/bicycle/src/00005.pdf}
\includegraphics[width=0.12\textwidth]{figures/failure_case/bicycle/src/00006.pdf}
\includegraphics[width=0.12\textwidth]{figures/failure_case/bicycle/src/00007.pdf}

\makebox[0.12\textwidth]{\colorbox{green}{\textbf{Failure case (a)}} A \textcolor{blue}{\textbf{astronaut}} is riding a bicycle}\\

\includegraphics[width=0.12\textwidth]{figures/failure_case/bicycle/astronaut/00000.pdf}
\includegraphics[width=0.12\textwidth]{figures/failure_case/bicycle/astronaut/00001.pdf}
\includegraphics[width=0.12\textwidth]{figures/failure_case/bicycle/astronaut/00002.pdf}
\includegraphics[width=0.12\textwidth]{figures/failure_case/bicycle/astronaut/00003.pdf}
\includegraphics[width=0.12\textwidth]{figures/failure_case/bicycle/astronaut/00004.pdf}
\includegraphics[width=0.12\textwidth]{figures/failure_case/bicycle/astronaut/00005.pdf}
\includegraphics[width=0.12\textwidth]{figures/failure_case/bicycle/astronaut/00006.pdf}
\includegraphics[width=0.12\textwidth]{figures/failure_case/bicycle/astronaut/00007.pdf}

\makebox[0.12\textwidth]{\colorbox{pink}{\textbf{Training video}} A bird is flying}\\

\includegraphics[width=0.12\textwidth]{figures/failure_case/bird/src/00000.pdf}
\includegraphics[width=0.12\textwidth]{figures/failure_case/bird/src/00001.pdf}
\includegraphics[width=0.12\textwidth]{figures/failure_case/bird/src/00002.pdf}
\includegraphics[width=0.12\textwidth]{figures/failure_case/bird/src/00003.pdf}
\includegraphics[width=0.12\textwidth]{figures/failure_case/bird/src/00004.pdf}
\includegraphics[width=0.12\textwidth]{figures/failure_case/bird/src/00005.pdf}
\includegraphics[width=0.12\textwidth]{figures/failure_case/bird/src/00006.pdf}
\includegraphics[width=0.12\textwidth]{figures/failure_case/bird/src/00007.pdf}

\makebox[0.12\textwidth]{\colorbox{green}{\textbf{Failure case (b)}} A \textcolor{blue}{\textbf{airplane}} is flying}\\

\includegraphics[width=0.12\textwidth]{figures/failure_case/bird/airplane/00000.pdf}
\includegraphics[width=0.12\textwidth]{figures/failure_case/bird/airplane/00001.pdf}
\includegraphics[width=0.12\textwidth]{figures/failure_case/bird/airplane/00002.pdf}
\includegraphics[width=0.12\textwidth]{figures/failure_case/bird/airplane/00003.pdf}
\includegraphics[width=0.12\textwidth]{figures/failure_case/bird/airplane/00004.pdf}
\includegraphics[width=0.12\textwidth]{figures/failure_case/bird/airplane/00005.pdf}
\includegraphics[width=0.12\textwidth]{figures/failure_case/bird/airplane/00006.pdf}
\includegraphics[width=0.12\textwidth]{figures/failure_case/bird/airplane/00007.pdf}

\makebox[0.12\textwidth]{\colorbox{pink}{\textbf{Training video}} A shark is swimming}\\

\includegraphics[width=0.12\textwidth]{figures/failure_case/shark/src/00000.pdf}
\includegraphics[width=0.12\textwidth]{figures/failure_case/shark/src/00001.pdf}
\includegraphics[width=0.12\textwidth]{figures/failure_case/shark/src/00002.pdf}
\includegraphics[width=0.12\textwidth]{figures/failure_case/shark/src/00003.pdf}
\includegraphics[width=0.12\textwidth]{figures/failure_case/shark/src/00004.pdf}
\includegraphics[width=0.12\textwidth]{figures/failure_case/shark/src/00005.pdf}
\includegraphics[width=0.12\textwidth]{figures/failure_case/shark/src/00006.pdf}
\includegraphics[width=0.12\textwidth]{figures/failure_case/shark/src/00007.pdf}

\makebox[0.12\textwidth]{\colorbox{green}{\textbf{Failure case (c)}} A \textcolor{blue}{\textbf{jellyfish}} is swimming}\\

\includegraphics[width=0.12\textwidth]{figures/failure_case/shark/jellyfish/00000.pdf}
\includegraphics[width=0.12\textwidth]{figures/failure_case/shark/jellyfish/00001.pdf}
\includegraphics[width=0.12\textwidth]{figures/failure_case/shark/jellyfish/00002.pdf}
\includegraphics[width=0.12\textwidth]{figures/failure_case/shark/jellyfish/00003.pdf}
\includegraphics[width=0.12\textwidth]{figures/failure_case/shark/jellyfish/00004.pdf}
\includegraphics[width=0.12\textwidth]{figures/failure_case/shark/jellyfish/00005.pdf}
\includegraphics[width=0.12\textwidth]{figures/failure_case/shark/jellyfish/00006.pdf}
\includegraphics[width=0.12\textwidth]{figures/failure_case/shark/jellyfish/00007.pdf}


\caption{\textbf{Failure case} Some failure examples of our model.}
\label{fig:failure_case}
\end{center}
\end{figure*}


\end{document}
