\section{INTRODUCTION}
\label{sec:introduction}





In recent years autonomous cars have become reliable enough to be deployed in the real world~\cite{teslaautopilot,waymoone}.
Nonetheless, they can currently operate safely only under limited conditions such as a mapped environment, good weather, or specific types of roads.
In the unstructuredness of the real world with unforeseeable situations, potentially caused by mistakes of other road users, the autonomous car might be required to drive at the edge of its ability under all environmental conditions in order to avoid accidents.
One such scenario is driving on varying surfaces which becomes especially challenging if their corresponding friction values differ a lot.
Maneuvers that are safe to execute on one surface can be dangerous or even impossible on a different surface.
For autonomous vehicles, this poses a safety-critical problem as the driving has to be adjusted according to the current surface and a wrong estimate about possible future trajectories can result in disastrous outcomes.
How to autonomously learn to predict features of a surface and the corresponding impact on the future trajectory without supervision by ground truth friction values for all surfaces is still an open research question.

\begin{figure}
\centering
\includegraphics[width=0.94\linewidth]{images/drifting_cover2.png}
\caption{
Overview of our approach.
During driving the latent mapper uses different modalities like sound and vision to update grid cell variables that represent information about the road and its properties at that location.  
The learned dynamics model receives the latent variable corresponding to its current location as additional input to allow for trajectory predictions that are aware of the road material.
}
\label{fig:cover}
\vspace{-0.55cm}
\end{figure}

Previous works allowing stable handling at the edge of controllability considered a single surface~\cite{cutler2016autonomous,1528936,8710588} or achieved steady state drifting but no cornering~\cite{Acosta2018TeachingAV}.
A typical approach in the case of varying surfaces is to use a separate terrain classifier~\cite{kalakrishnan2010fast,kolter2007hierarchical,thrun2006stanley}, which, however, assumes a fixed number of prespecified terrain classes.
This problem can be overcome by using conditional dynamics models~\cite{nagabandi2018learning} but the inferred information is not stored in a map and can hence not be used for later traversals of the same location.
Other works estimate the explicit coefficient of friction inside a physics model~\cite{8917024,huang2019calculation}. This, however, relies on the assumption that an accurate model can be identified, which is difficult for slippery surfaces.



We tackle this problem with a new approach that trains a dynamics model to include information from a learned and automatically updated latent map.
During inference, a latent mapper updates the map, such that the latent variable at a specific location stores valuable information about the surface material at that position. 
The dynamics model uses the latent variable for its current location as additional input to make accurate surface-dependent predictions of the next state.
Several traversals through a single location are given sequentially to the latent mapper to update the corresponding latent variable in an autoregressive fashion.
Since all model parts are differentiable, we can train the dynamics model and the latent mapper simultaneously from the same loss in a way that the latter learns to give useful surface information to the former.
Motivated by the fact that humans automatically use multiple cues to infer what driving style is appropriate for the surface they are driving on, the latent mapper receives multiple modalities like images and sound allowing it to learn a general representation of different surface patterns.

We implement and evaluate our method on a real miniature electric car.
The results show our approach is able to generate complex driving maneuvers on unknown and varying surfaces, showing the benefit of implicit terrain maps.
To summarize, our contributions are
\begin{itemize}
    \item a latent mapper trained to update a map with surface information from multiple modalities
    \item a surface-aware dynamics model using the latent map
    \item the implementation on a real electric car
    \item an extensive evaluation of the single components of our method and their combinations
    \item demonstrating the benefit of our latent mapping relative to a baseline without surface awareness in the real world.
\end{itemize}



