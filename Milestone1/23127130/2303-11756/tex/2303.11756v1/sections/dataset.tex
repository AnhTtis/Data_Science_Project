\vspace{-0.1cm}
\section{DATASET}
\label{sec:dataset}
To train and evaluate our approach we propose a novel dataset which we refer to as \textit{Dynamic FreiCar}. As an experimental vehicle, we employ a rear-driven miniature 1:8 scale car that is equipped with a computer, various sensors, and a high-torque electric motor.
To capture RGB images, we leverage a \textit{ZED} camera, while we use a \textit{Rode} compact microphone to record audio data. We further use \textit{Valve Lighthouses} to track the position of the car and gather ground-truth velocity and acceleration data.
Our dataset contains 70 minutes of expert driving along random trajectories that include drifting scenarios. All data is recorded at 100hz. 
We use two types of wood laminate and gym rubber mats for the different surface materials.
To avoid bumps at the transition we level out the different materials.
To ensure fair training and evaluation splits we create two maps for training and one map for evaluation by spatially rearranging the materials. Thus, as we train on multiple maps, we avoid overfitting to a specific map, which we validate in our experiments section. Figure \ref{drift:fig:path_overlay} shows our experimental vehicle and the driving environment.
Note that our approach does not model explicit friction values and we do not have the ground truth friction values for the different surfaces.

\begin{figure}
\setlength{\tabcolsep}{1pt}
\centering
\footnotesize
        \begin{tabular}{P{.2665\linewidth}P{.36\linewidth}P{.36\linewidth}}
        The Car & Without Map & With Map \\
        \includegraphics[width=\linewidth]{images/car2.png} & \includegraphics[width=\linewidth]{images/bs_2_5.1.1.png} & \includegraphics[width=\linewidth]{images/ours_2_5.2.1.png} \\
    \end{tabular}
    \scriptsize  % smaller
    \caption{Left: Our experimental vehicle. The background shows the various surface materials over which the driving is conducted. Center and right: the driven path of the vehicle with and without map information.}
    \label{drift:fig:path_overlay} 
    \vspace{-0.5cm}
\end{figure}
