\vspace{-0.3cm}
\section{RELATED WORK}
\label{sec:relatedworks}



\vspace{-0.1cm}
\subsection{Surface-Awareness}
\vspace{-0.1cm}
Terrain classification can be done in an automated way~\cite{zurn2020self} or from different modalities~\cite{valada2018deep}.
Several approaches use an explicit terrain classifier trained with human-specified labels and deploy a terrain-specific policy accordingly~\cite{kalakrishnan2010fast,kolter2007hierarchical,thrun2006stanley}.
More similar to our approach is the work of Nagabandi \textit{et~al.}~\cite{nagabandi2018learning} where additionally to state and action an image is given as input to the dynamics model such that it can infer the next state conditioned on the surface the robot is currently moving on.
Furthermore, meta-learning has been used for learning to adapt the dynamics model parameters during inference to a given environment including different terrains~\cite{nagabandi2018meta}.
Our approach is different in that it updates the surface embedding during inference allowing it in principle to learn the adaption to any new surface and to further store the information in a map.

\vspace{-0.1cm}
\subsection{Explicit Estimation of the Coefficient of Friction}
\vspace{-0.1cm}
One can compute possible trajectories using the coefficient of friction between the tires and the road. However, the coefficient depends on all kinds of conditions and has to be estimated from real data. 
For example, Huang \textit{et~al.}~\cite{huang2019calculation} estimate the coefficient with a limited-memory adaptive extended Kalman Filter to reduce the effect of outdated measurements on filtering. Multiple surveys~\cite{khaleghian2017technical, wang2022tire} cover these topics.
Deep neural networks can use ultrasound as input to classify the road surface from which the coefficient of friction is derived~\cite{kim2021road}.
For autonomous race cars driving several laps on a track with an inhomogeneous surface covering has been proposed to estimate the coefficient of friction and store it in a map~\cite{8917024}. 
All of these approaches have in common that they use a physical dynamics model for which they estimate the corresponding parameters. Our approach is more flexible in that it can learn any dynamics model without the need to model the underlying physics. Thus, our work is most comparable to other methods that learn dynamics models solely from data. Our focus is to improve ML-based methods by introducing surface-awareness.

\vspace{-0.1cm}
\subsection{Dynamics Models with Latent Variables}
\vspace{-0.1cm}
Some works train a network for system identification from the most recent history to output a latent variable which is then given as additional input to a model-free RL policy~\cite{yu17preparing} or the dynamics model in model-based RL~\cite{lee2020context,zhou2018environment}.
In the meta-RL setting Rakelly \textit{et~al.}~\cite{rakelly2019efficient} train a network to produce a probabilistic context variable given as additional input to an off-policy RL algorithm to adapt to a given task.

Our approach differs in that it builds a map over the state space and learns to learn different context embeddings at different locations in the state space during inference. 

