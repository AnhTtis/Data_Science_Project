%%%%%%%%%%%%%%%%%%%%%%%%%%%%%%%%%%%%%%%%%%%%%%%%%%%%%%%%%%%%%%%%%%%%%%%%%%%%%%%%
%2345678901234567890123456789012345678901234567890123456789012345678901234567890
%        1         2         3         4         5         6         7         8

\documentclass[letterpaper, 10 pt, conference]{ieeeconf}  % Comment this line out if you need a4paper

%\documentclass[a4paper, 10pt, conference]{ieeeconf}      % Use this line for a4 paper

\IEEEoverridecommandlockouts                              % This command is only needed if 
                                                          % you want to use the \thanks command

\overrideIEEEmargins                                      % Needed to meet printer requirements.

%In case you encounter the following error:
%Error 1010 The PDF file may be corrupt (unable to open PDF file) OR
%Error 1000 An error occurred while parsing a contents stream. Unable to analyze the PDF file.
%This is a known problem with pdfLaTeX conversion filter. The file cannot be opened with acrobat reader
%Please use one of the alternatives below to circumvent this error by uncommenting one or the other
%\pdfobjcompresslevel=0
%\pdfminorversion=4

% See the \addtolength command later in the file to balance the column lengths
% on the last page of the document

% The following packages can be found on http:\\www.ctan.org
%\usepackage{graphics} % for pdf, bitmapped graphics files
\usepackage{epsfig} % for postscript graphics files
%\usepackage{mathptmx} % assumes new font selection scheme installed
\usepackage{times} % assumes new font selection scheme installed
\usepackage{amsmath} % assumes amsmath package installed
\usepackage{amssymb}  % assumes amsmath package installed
\usepackage{graphicx}
\usepackage{hyperref}
\usepackage{xcolor}
\usepackage{multirow}
\usepackage{siunitx}
\usepackage{array}
\usepackage{diagbox}
\usepackage{adjustbox}
\usepackage{booktabs}
\usepackage{todonotes}
\usepackage{pifont}
\usepackage{amssymb}
\usepackage{url}
\usepackage{colortbl}
\usepackage{overpic}


 
\newcommand{\greycell}{\cellcolor[RGB]{220,220,220}}%

\newcommand{\cmark}{\ding{51}}%
\newcommand{\xmark}{\ding{55}}%

\urlstyle{same}

\newcommand{\etal}{\emph{et~al.}}
\newcommand\figref{Figure~\ref}

\newcommand*\rot{\multicolumn{1}{R{90}{1em}}}
\newcommand\myworries[1]{\textcolor{red}{#1}}

\newcolumntype{P}[1]{>{\centering\arraybackslash}p{#1}}
\newcommand\crule[3][black]{\textcolor{#1}{\rule{#2}{#3}}}
\DeclareMathOperator*{\argmin}{argmin}

\renewcommand{\baselinestretch}{0.985}

\newcolumntype{R}[2]{%
    >{\adjustbox{angle=#1,lap=\width-(#2)}\bgroup}%
    l%
    <{\egroup}%
}

\def\X#1{\raisebox{3.5em}{#1}}

\usepackage[font=small]{caption}
\captionsetup[table]{position=bottom} 
\title{\LARGE \bf
Improving Deep Dynamics Models for Autonomous Vehicles with \\ Multimodal Latent Mapping  of Surfaces}


\author{Johan Vertens$^{*}$ \and Nicolai Dorka$^{*}$ \and Tim Welschehold \and Michael Thompson \and Wolfram Burgard% <-this % stops a space
\thanks{$^{*}$These authors contributed equally. Johan Vertens, Nicolai Dorka, and Tim Welschehold are with the University of Freiburg, Germany. Michael Thompson is with Toyota Research Institute, Los Altos, USA. Wolfram
Burgard is with the University of Technology, Nuremberg, Germany. Corresponding author: {\tt\small vertensj@informatik.uni-freiburg.de}}
}


\begin{document}


\maketitle
\thispagestyle{empty}
\pagestyle{empty}


%%%%%%%%%%%%%%%%%%%%%%%%%%%%%%%%%%%%%%%%%%%%%%%%%%%%%%%%%%%%%%%%%%%%%%%%%%%%%%%%
\begin{abstract}
The safe deployment of autonomous vehicles relies on their ability to effectively react to environmental changes.
This can require maneuvering on varying surfaces which is still a difficult problem, especially for slippery terrains.
To address this issue we propose a new approach that learns a surface-aware dynamics model by conditioning it on a latent variable vector storing surface information about the current location.
A latent mapper is trained to update these latent variables during inference from multiple modalities on every traversal of the corresponding locations and stores them in a map.
By training everything end-to-end with the loss of the dynamics model, we enforce the latent mapper to learn an update rule for the latent map that is useful for the subsequent dynamics model.
We implement and evaluate our approach on a real miniature electric car.
The results show that the latent map is updated to allow more accurate predictions of the dynamics model compared to a model without this information.
 We further show that by using this model, the driving performance can be improved on varying and challenging surfaces. 
\end{abstract}


\section{Introduction}
\label{sec:introduction}
% \begin{itemize}
%     % Diffusion of FL
%     \item {\st{Diffusion of FL}}
%     % Security threats to FL
%     \item {\st{Security threats to FL with particular focus on model poisoning}}
%     % Limitations of existing countermeasures
%     \item {\st{Current countermeasures (e.g., KRUM) and their limitations}}
%     % Proposed method and its advantages
%     \item {\st{Intuitive description of the proposed method and its difference (i.e., advantages) w.r.t. state of the art}}
%     % Main contributions
%     \item {\st{Summary of the main contributions of this work}}
%     % Paper's structure and organization
%     \item {\st{Paper's structure and organization}}
% \end{itemize}

% Diffusion of FL
Recently, {\em federated learning} (FL) has emerged as the leading paradigm for training distributed, large-scale, and privacy-preserving machine learning (ML) systems~\cite{mcmahan2017googleai,mcmahan2017aistats}. 
The core idea of FL is to allow multiple edge clients to collaboratively train a shared, global model without disclosing their local private training data.
%Specifically, an FL system consists of a central server and many edge clients; 
A typical FL round involves the following steps: {\em(i)} the server randomly picks some clients and sends them the current, global model; {\em(ii)} each selected client locally trains its model with its own private data; then, it sends the resulting local model to the server;\footnote{Whenever we refer to global/local model, we mean global/local model {\em parameters}.} {\em(iii)} the server updates the global model by computing an \emph{aggregation function}, usually the average (FedAvg), on the local models received from clients.
% \begin{enumerate}
%     \item[{\em(i)}] the server sends the current, global model to the clients and appoints some of them for training;
%     \item[{\em(ii)}] each selected client locally trains its copy of the global model with its own private data; then, it sends the resulting local model back to the server;\footnote{Whenever we refer to global/local model, we mean global/local model {\em parameters}.}
%     \item[{\em(iii)}] the server updates the global model by computing an \emph{aggregation function} on the local models received from clients (by default, the average, also referred to as FedAvg~\cite{mcmahan2017aistats}).
% \end{enumerate}
This process goes on until the global model converges. %(e.g., after a certain number of rounds or other similar stopping criteria).
%\\
% The advantages of FL over the traditional, centralized learning paradigm are undoubtedly clear in terms of flexibility/scalability (clients can join/disconnect from the FL network dynamically), network communications (only model weights\footnote{We will use \textit{parameters} and \textit{weights} interchangeably.} are exchanged between clients and server), and privacy (each client's private training data is kept local at the client's end and not uploaded to the server).
\\
% Security threats to FL
%However, the growing adoption of FL also raises security concerns~\cite{costa2022covert}, particularly about its confidentiality, integrity, and availability.
Although its advantages over standard ML, FL also raises security concerns~\cite{costa2022covert}. %, particularly about its confidentiality, integrity, and availability~\cite{costa2022covert}.
% OLD, LONG VERSION
% Indeed, some work deals with privacy leakage that may expose the local data of some clients~\cite{melis2019sp}. 
% A large body of work, instead, investigates attacks that usually aim to detriment the predictive accuracy of the learned global model. For instance, \emph{data poisoning} attacks achieve this goal by letting an adversary pollute the training set of some corrupt FL clients with maliciously crafted examples~\cite{jagielski2018sp}.
% Similarly, in \emph{model poisoning} the attacker attempts to tweak the global model weights~\cite{bhagoji2019pmlr} by directly perturbing the local model's weights of some infected FL clients before these are sent to the central server for aggregation, usually via so-called Byzantine attacks. 
% It turns out that Byzantine model poisoning attacks severely impact standard FedAvg; therefore, more robust aggregation functions must be designed to make FL systems secure.
Here, we focus on \emph{untargeted model poisoning} attacks~\cite{bhagoji2019pmlr}, where an adversary attempts to tweak the global model weights %\footnote{We will use the terms \textit{parameters} and \textit{weights} interchangeably.} 
by directly perturbing the local model's parameters of some infected clients before these are sent to the central server for aggregation.
In doing so, the adversary aims to jeopardize the global model \textit{indiscriminately} at inference time.
Such model poisoning attacks severely impact standard FedAvg; therefore, more robust aggregation functions must be designed to secure FL systems.
\\
% In this paper, we focus on designing a novel robust aggregation scheme at the server's end to contrast the effect of Byzantine model poisoning attacks.
%
% Current countermeasures and their limitations
%Several countermeasures have been proposed in the literature to combat model poisoning attacks on FL systems.
% Some methods use simple statistics more robust than plain average to smooth the impact of malicious updates (e.g., Trimmed Mean and FedMedian~\cite{yin2018icml}). 
% Other defenses implement outlier detection techniques to discard malicious updates from the aggregation performed at the server's end. Those are either based on heuristics (e.g., Krum/Multi-Krum~\cite{blanchard2017nips} and Bulyan~\cite{mhamdi2018pmlr}) or data-driven approaches (e.g., K-means clustering~\cite{shen2016acm} or DnC via spectral analysis~\cite{shejwalkar2021ndss}). 
% Finally, some strategies rely on a centralized ``source of trust'' to spot potential malicious updates (e.g., FLTrust~\cite{cao2020fltrust}).
% Several countermeasures have been proposed in the literature to combat model poisoning attacks on FL systems, i.e., to discard possible malicious local updates from the aggregation performed at the server's end. 
% These techniques range from simple statistics more robust than plain average (e.g., Trimmed Mean and FedMedian~\cite{yin2018icml}) to outlier detection heuristics (e.g., Krum/Multi-Krum~\cite{blanchard2017nips} and Bulyan~\cite{mhamdi2018pmlr}) or data-driven approaches (e.g., spectral analysis via K-means clustering~\cite{shen2016acm} or spectral analysis), or methods based on ``source of trust'' (e.g., FLTrust~\cite{cao2020fltrust}).
% OLD, LONG VERSION
%Several countermeasures have been proposed in the literature to combat Byzantine model poisoning attacks on FL systems.
% Descriptive statistics
% For example, Trimmed Mean and FedMedian aggregate local model updates using more robust statistics than standard average~\cite{yin2018icml}.
%
% % Heuristics for outlier detection
% Many existing Byzantine-resilient strategies implement some outlier detection heuristics to discard the model updates sent by potentially malicious clients from the input of the aggregation function.
% One of the most popular heuristics is Krum~\cite{blanchard2017nips}.
% This strategy tries to mitigate the impact of Byzantine attacks by selecting as a global model the local model with the smallest sum of Euclidean distances to {\em all} the other local models.
% Although powerful, Krum requires the server to know (or, at least, estimate) the number of malicious FL clients upfront, which is generally impossible in a realistic attack scenario. %
% Moreover, Krum may become ineffective for complex, high-dimensional model parameter spaces due to the curse of dimensionality.
% Bulyan~\cite{mhamdi2018pmlr} tries to overcome this issue by combining Krum with a variant of Trimmed Mean.
% % Data-driven outlier detection
% Other strategies use data-driven outlier detection techniques -- e.g., via K-means clustering~\cite{shen2016acm} -- to spot potential malicious local model updates. 
% %For instance, Shen et al. propose to cluster local model updates with K-means and thus identify outliers.
%
% % Other techniques
% As far as the server is concerned, any local model received can be from a potential malicious client. 
% FLTrust~\cite{cao2020fltrust} assumes the server acts as a client, i.e., trains a local model on an additional {\em trustworthy} dataset at the server's end and compares it against all the local models from other clients. 
% This way, the server can rely on some ``source of trust'' when discarding potentially malicious clients.
%\\
% Limitations of existing Byzantine-resilient strategies
Unfortunately, existing defense mechanisms either rely on simple heuristics (e.g., Trimmed Mean and FedMedian by~\cite{yin2018icml}) or need strong and unrealistic assumptions to work effectively (e.g., foreknowledge or estimation of the number of malicious clients in the FL system, as for Krum/Multi-Krum~\cite{blanchard2017nips} and Bulyan~\cite{mhamdi2018pmlr}, which, however, cannot exceed a fixed threshold).
Furthermore, outlier detection methods using K-means clustering~\cite{shen2016acm} or spectral analysis like DnC~\cite{shejwalkar2021ndss} do not directly consider the temporal evolution of local model updates received.
Finally, strategies like FLTrust~\cite{cao2020fltrust} require the server to collect its own dataset and act as a proper client, thereby altering the standard FL protocol.
\\
% OLD, LONG VERSION
% Overall, existing Byzantine-resilient strategies are either simple heuristics (e.g., FedMedian) or, if they are more complex, they rely on strong and unrealistic assumptions to work effectively (e.g., knowing the number of malicious clients in the FL system in advance, as for Krum and alike).
% Furthermore, data-driven outlier detection methods do not consider the temporary evolution of local model updates received (e.g., K-means clustering). 
% Finally, strategies like FLTrust requires the server to collect its own dataset and act as a proper client, thereby altering the standard FL protocol.
%
% Description of the proposed method
This work introduces a novel pre-aggregation \textit{filter} robust to untargeted model poisoning attacks. Notably, this filter $(i)$ operates without requiring prior knowledge or constraints on the number of malicious clients and $(ii)$ inherently integrates temporal dependencies. 
The FL server can employ this filter as a preprocessing step before applying \textit{any} aggregation function, be it standard like FedAvg or robust like Krum or Bulyan.
Specifically, we formulate the problem of identifying corrupted updates as a multidimensional (i.e., matrix-valued) time series anomaly detection task. 
The key idea is that legitimate local updates, resulting from well-calibrated iterative procedures like stochastic gradient descent (SGD) with an appropriate learning rate, show \textit{higher predictability} compared to malicious updates. This hypothesis stems from the fact that the sequence of gradients (thus, model parameters) observed during legitimate training exhibit regular patterns, as validated in Section~\ref{subsec:intuition}. %until convergence. 
%This regularity may be more pronounced for smooth convex loss functions, but it can still be captured within an appropriate time window, even for more complex and convoluted loss surfaces. 
%We provide evidence of this claim in Appendix~B, where we show that the average mutual information (i.e., ``predictability''), calculated over pairs of legitimate model updates sent at different FL rounds, is significantly higher than the corresponding computation for a malicious client.
\\
Inspired by the matrix autoregressive (MAR) framework for multidimensional time series forecasting~\cite{chen2021je}, we propose the FLANDERS ({\em \textbf{F}ederated \textbf{L}earning meets \textbf{AN}omaly \textbf{DE}tection for a \textbf{R}obust and \textbf{S}ecure}) filter.
The main advantages of FLANDERS over existing strategies like FLDetector~\cite{zhao2020multivariate} are its resilience to large-scale attacks, where $50\%$ or more FL participants are hostile, and the capability of working under realistic non-iid scenarios.
We attribute such a capability to two key factors: $(i)$ FLANDERS works without knowing a priori the ratio of corrupted clients, and $(ii)$ it embodies temporal dependencies between intra- and inter-client updates, quickly recognizing local model drifts caused by evil players. Below, we summarize our main contributions:

\begin{itemize}
\item[{\em(i)}]
We provide empirical evidence that the sequence of models sent by legitimate clients is more predictable than those of malicious participants performing untargeted model poisoning attacks.
\\
\item[{\em(ii)}] 
We introduce FLANDERS, the first pre-aggregation filter for FL robust to untargeted model poisoning based on multidimensional time series anomaly detection.
\\
\item[{\em(iii)}] 
We integrate FLANDERS into Flower,\footnote{\scriptsize{\url{https://flower.dev/}}} a popular FL simulation framework for reproducibility.
\\
\item[{\em(iv)}] 
We show that FLANDERS improves the robustness of the existing aggregation methods under multiple settings: different datasets, client's data distribution (non-iid), models, and attack scenarios.
\\
\item[{\em(v)}] 
We publicly release all the implementation code of FLANDERS along with our experiments.\footnote{\scriptsize{\url{https://anonymous.4open.science/r/flanders_exp-7EEB}}}
\end{itemize}

% Paper's structure and organization
The remainder of the paper is structured as follows. %some related work and the current state-of-the-art solutions to security issues that FL entails. 
Section~\ref{sec:background} covers background and preliminaries. 
In Section~\ref{sec:related}, we discuss related work.
Section~\ref{sec:problem} and Section~\ref{sec:method} describe the problem formulation and the method proposed. % to tackle it. 
Section~\ref{sec:experiments} gathers experimental results. %, and Section~\ref{sec:limitations} discusses some limitations of this work.
Finally, we conclude in Section~\ref{sec:conclusion}.
 %discusses the limitations of this work and draws future research directions.
%reports conclusions and draws perspectives for future research directions.

%%%%%%% OLD %%%%%%%
%to overcome the resilience of Byzantine failures in distributed Stochastic Gradient Descent computations. 
% The strength of Krum is its time complexity, which is linear in the gradient dimension. 
% However, the robustness of the approach is guaranteed for gradient-based learning applications only when the majority of the clients are not compromised. 
% Besides, the aggregation mechanism of Krum, as well as that of similar methods, is robust from a coarse-grained perspective and does not provide solutions to errors and perturbations that may occur at inference time.
%A related approach to~\cite{blanchard2017nips} is the work of Su et al.~\cite{su2016dc}. Here, the authors propose an iterated approximate agreement to tackle a multi-layer scenario attacked by Byzantine agents. 
%However, the method works efficiently on the sole discrete context and it is inapplicable to continuous state environments.
%\gabri{Maybe, we should just talk about the main limitations of existing countermeasures without digging into their details (or, we can just mention Krum as this is the most popular one). I will move the description of all these methods to the Related Work section.}
\section{Related Works}

%\looseness=-1\paragraph{Probing for transformers}
\paragraph{(Structural) probing}
\looseness=-1Several recent works on probing have aimed to study the encoded information in BERT-like models~\citep{rogers2020primer}. \citet{hewitt2019structural,reif2019visualizing,manning2020emergent,vilares2020parsing,maudslay2020tale,maudslay2021syntactic,chen2021probing,arps2022probing,jawahar2019does} have demonstrated that it is possible to predict various syntactic information present in the input sequence, including parse trees or POS tags, from internal states of BERT. 
%However, in contrast to these existing approaches that typically utilize a pre-trained model as-is, we adopt a close environment approach to understand the relationship between the data distribution, the masked language model objective, and the architecture to its ability to do syntactic parsing. We show for one particular pre-training data distribution, the pre-trained model's representation captures quantities correlated with the quantities of an optimal algorithm. We hope that our work can motivate future work in this direction.
In contrast to existing approaches that commonly employ a model pre-trained on natural language, we pre-train our model under PCFG-generated data to investigate the interplay between the data, the MLM objective, and the architecture's capacity for parsing. 
Besides syntax, probing has also been used to test other linguistic structures like semantics, sentiment, etc.~\citep{belinkov2017neural,reif2019visualizing,kim2020pre,richardson2020probing,vulic2020probing,conia-navigli-2022-probing}.
%e.g. the syntactic (structural) information~\citep{hewitt2019structural,reif2019visualizing,manning2020emergent,vilares2020parsing,maudslay2020tale,maudslay2021syntactic,chen2021probing,arps2022probing,jawahar2019does}. 


%However as mentioned in \citet{maudslay2021syntactic}, the probing success of the previous works on syntax emerged from the model using semantic cues to parse. 
%The probed pre-trained models had been pre-trained on natural language datasets, where the semantic structures are most likely correlated with the syntactic ones. Indeed, \citet{arps2022probing} tried to separate the semantics from syntax by training the probe on a mixture of natural language and manipulated data, however, the authors acknowledged the possibility of semantics still affecting the decision of the probe trained on manipulated data. 
%\paragraph{Other probings} Besides syntax, probing has been used for other linguistic structures like semantics, sentiment, etc.~\citep{belinkov2017neural,reif2019visualizing,kim2020pre,richardson2020probing,vulic2020probing,conia-navigli-2022-probing}.

\paragraph{Expressive power of transformers}
\looseness=-1\citet{Yun2020Are,yun2020n} show that transformers are universal sequence-to-sequence function approximators. Later, \citet{perez2021attention,bhattamishra2020computational} show that attention models can simulate Turing machines, with \citet{wei2022statistically} proposing statistically meaningful approximations of Turing machines. 
%Attention models with bounded size have been shown capable of recognizing deterministic context-free languages such as bounded-depth Dyck-k~\citep{yao2021self}. 
%The size of the constructed models, however, depends on the complexity of the target function and often requires arbitrary precision to encode the target function. 
%, that also exhibit good statistical learnability. 
%\citet{liu2022transformers} constructed (by hand) transformers that can efficiently simulate automata. 
To understand the behavior of moderate-size transformer architectures, many works have investigated specific classes of languages, e.g. bounded-depth Dyck languages~\citep{yao2021self}, modular prefix sums~\citep{anil2022exploring}, adders~\citep{nanda2023progress}, regular languages~\citep{bhattamishra2020ability}, and sparse logical predicates~\citep{edelman2022inductive}. \citet{merrill2022saturated} relate saturated transformers with constant depth threshold circuits, and \citet{liu2022transformers} provide a unified theory on understanding automata within transformers.
%Compared to the existing literature, our expressiveness power results are more related to the real natural language, since the PCFG we study is learned from natural language and the transformer we construct is of moderate size. \haoyu{is previous sentence OK? Or the following sentence?} 
These works study expressive power under a class of synthetic language. Compared to the prior works, our results are more related to the natural language, as we consider not only a class of synthetic language (PCFG), but also a specific PCFG tailored to the natural language.

% for inputs of a small range of lengths. 
 %and empirically showed the existence of such solutions in pre-trained transformers. 
 %However, their short-cut solution is not robust to OOD generalization, while our probes are all OOD generalizable since we pre-train on synthetic PCFG data while probing on the \dataset{PTB} dataset.
 %However, as evident from the name, shortcut solutions aren't robust to OOD generalization.
 %Interestingly, we observe that language models pre-trained on PCFG-generated data encode relevant information from the Inside-Outside algorithm for sentences from both natural language and PCFG-generated data, which suggests an implicit bias toward learning the general algorithm (generalizing to all input lengths). 
 %This may suggest that pre-training on purely synthetic data and data closer to natural language have different regimes. Besides, these results do not shed light on the expressivity of transformers needed to encode relevant syntactic and semantic information in languages. 
 %A careful study is left for future work.
 

 


%\paragraph{Grammar induction}
\vspace{-0.1cm}
\section{TECHNICAL APPROACH}
\label{drift:sec:approach}
In our work, we employ a dynamics model that predicts a potential future state given an input action. Subsequently, the dynamics model is used to control a vehicle along pre-defined trajectories following a model predictive scheme.
Our approach does not model explicit parameters of a physical model nor does it take factors such as tire pressure or temperature into account. 
One could combine these aspects with out method but this is beyond the scope of this work.

In contrast to previous approaches, we propose a method that considers potential changes of the road material along the track, which may influence the dynamic behavior of the vehicle. 
Therefore, we first create a grid map that is defined in global coordinates and equally divides the space into quadratic cells. 
Once the vehicle traverses a cell of the grid map, a mapping neural network infers a latent vector that describes the local road surface. 
Thereby, the mapping network leverages various modalities that were recorded during the traversal of the cell, such as RGB images, acoustic spectrograms, history states, and history actions. 
Our dynamics model takes, aside from low-level state and action information, the latent vector corresponding to the current vehicle location as an input, such that it can leverage the additional mapped cues from multiple modalities to optimize the predictions of future states, making it surface-aware.
As the training of the mapping model is guided by the loss of the dynamics model predictions, we do not require any labels that associate the input modalities with specific surface characteristics.
Note, that in contrast to previous approaches, our work models the whole surface-aware dynamics as a neural network, which avoids assumptions or inductive biases for the created surface map.
Next, we unroll multiple trajectories using the dynamics model and sampling from the action space. 
Our system then employs a reward function and the cross entropy~\cite{de2005tutorial} method to score and select the trajectories that follow a reference path as fast and close as possible. 
Following a typical model predictive scheme, we execute the first action of the resulting plan and restart the process.
In the following we describe each component of our system in more detail, followed by an explanation of the loss functions and training procedure.



\vspace{-0.1cm}
\subsection{Surface-Aware Probabilistic Dynamics Model Ensemble}
\vspace{-0.1cm}
\label{drift:sec:dyn_model}
Our dynamics model predicts the next output state $s^{\mathrm{out}}_{t+1}$, given the current input state $s^{\mathrm{in}}_{t}$ and the current action $a_t$. To this end, we distinguish between input states, which are the representation of the input to the dynamics model, and output states, which are the prediction of the dynamics model.
We define the input state as the concatenation of 3D linear velocities $\mathrm{vel}_l$, angular velocities $\mathrm{vel}_a$, linear accelerations $\mathrm{acc}_l$, angular accelerations $\mathrm{acc}_a$, and motor rpm : $s^{\mathrm{in}} = [\mathrm{vel}_l^T, \mathrm{vel}_a^T, \mathrm{acc}_l^T, \mathrm{acc}_a^T, \mathrm{rpm}]^T$. 
Additionally, we define the predicted output state of the dynamics model as estimated local changes in the x-position $\Delta p_x$, y-position $\Delta p_y$ and yaw angle $\Delta \gamma$, all velocities, and all accelerations as  $s^{\mathrm{out}} = [\Delta p_x, \Delta p_y, \Delta \gamma, \mathrm{vel}_l^T, \mathrm{vel}_a^T, \mathrm{acc}_l^T, \mathrm{acc}_a^T, \mathrm{rpm}]^T$.
The action is composed of the throttle $a_\mathrm{th}$ and steering command $a_\mathrm{st}$, such that $a=[a_\mathrm{th}, a_\mathrm{st}]$.
We make our model surface-aware by using additional latent vectors as input to the dynamics model. These latent vectors describe the local learned properties of the road.
Therefore, we propose to learn a latent map $L$ that is represented as a grid map where each quadratic cell $c$ holds a distribution over the latent vector. Here, we assume each entry of the map to be a $k_l$-dimensional multivariate normal distribution that is parametrized by $l_{\theta}^c$, corresponding to cell $c$. As the vector is learned implicitly, the number of dimensions $k_l$ represents a hyperparameter. 
To predict the next output state, we first sample a latent vector $l^c$ from the latent distribution $\mathcal{N}(l_{\theta_\mu}^c, {l_{\theta_{\sigma^2}}^c})$ at the cell corresponding to the current vehicle position. 
Following, we employ an ensemble of probabilistic dynamics models~\cite{chua2018deep} with parameters $\psi$ to predict the next state, while capturing model and data uncertainties. The input of this ensemble comprises the current input state, the current action, and the sampled latent vector, which we feed by simple concatenation. 
Thus, we model the Gaussian distribution of the next state as:
\vspace{-0.1cm}
\begin{equation}
    f_{\psi}(s_{t+1} \mid s_{t}, a_t) = \mathrm{Pr}(s_{t+1} \mid s_{t}, a_t, l^c;\psi).
    \vspace{-0.1cm}
\end{equation}
For clarity, we omitted the differentiation between the input and the output state.



\vspace{-0.1cm}
\subsection{Mapping Network}
\vspace{-0.1cm}
\label{drift:sec:mapping_network}
To estimate the latent map, we propose a novel neural network architecture that takes a variety of modalities as input.
In more detail, when the car traverses a cell $c$, we leverage an RGB image $I^c$, an acoustic spectrogram $S^c$, the history of states $H_s^c$, the history of actions $H_a^c$, and the previous estimate of the mean and variance of the latent vector $l_{\theta}^c$ that were recorded in the same cell $c$. The cues are then encoded into high-level features using respective encoders.
These features are then concatenated and passed to another MLP with two output heads, which predicts the mean and variance of the latent vector respectively.
As mentioned in Sec.~\ref{drift:sec:dyn_model}, the means and variances are then aggregated to a latent map $L$.
More formally, let $\phi$ be the parameters of the mapping model and $\Tilde{l_{\theta}^{c}}$ the updated parametrization of the Gaussian distribution of the latent vector in cell $c$. We define a latent update for the cell $c$ as: 
\vspace{-0.1cm}
\begin{equation}
\Tilde{l_{\theta}^{c}} = M_{\phi}(I^c, S^c, H_{s}^c, H_{a}^c, l_{\theta}^c),
\vspace{-0.1cm}
\end{equation}
Note, that since we input previous latent estimates $l_{\theta}^c$, the latent representation of the road surface is updated iteratively. 

While the RGB image may entail visual information of the road, the acoustic spectrograms capture direct tire-road interactions that are characteristic for specific materials. Additionally, from learning about the history of states and actions, our mapping model can infer ground patterns that lead to specific state sequences given the respective actions.

Particularly at inference time, estimating the road characteristics from only the low-level state and action history would most likely fail when the vehicle is at slow speeds or stands still.
To argue about the road surface, the dynamic behavior needs to differ across different road materials due to distinct friction characteristics. This, however, is only given in cases where the maximum frictional force~\cite{mu2003estimation} that is achieved is smaller than the force that is needed to sustain the vehicle track. Thus, to enable arguing about the road material, situations are required in which the car starts slipping.
While, for training purposes of the dynamics model, this data can be collected by an expert driver, uncontrolled slipping should be avoided during inference. 
To this end, our multimodal approach allows learning a mapping that associates visual or acoustic cues to a latent representation of the road without the requirement of slipping. As an example, our network can learn to associate a slippery surface with the visually shiny appearance of the road. Further, acoustic spectrograms can contain information about the road even under low velocities.

Consequently, we require examples of aggressive driving only during training, while during inference the road representation can be estimated under all conditions.
Furthermore, the employed modalities can complement each other if one modality lacks information due to visual occlusions, low lighting,  or when external acoustic events drown out important acoustic tire-road interactions.
We show in Sec.~\ref{drift:sec:results} that leveraging multimodal data yields high gains in state prediction performance.

\vspace{-0.1cm}
\subsection{Training of the models}
\vspace{-0.1cm}
\label{drift:sec:training}
We first collect a training dataset with dynamic examples of random driving in environments with spatially changing road materials. These driving examples include situations where the car slips.
To train our networks we propose a loss function that fulfills two requirements:
\begin{itemize}
    \item Unrolling of future states may lead to querying cells that have not been observed yet. In these cases, it should be possible to inform the dynamics model of zero knowledge of the surface material.
    \item The outputs of the mapping model should represent a spatial property of the road surface that is valid for any state prediction in the respective cell. 
\end{itemize}
To accomplish these requirements, we first group the individual recorded ground-truth state transitions according to the cell $c$ in which the transition was captured. Here, we denote the $n$-th state-transition in our dataset that occurred in the cell $c$ as $s^{c^n}_{t} \rightarrow  s^{c^n}_{t+1}$. Now, having a list of all state transitions that occurred in the same cell, we select $N$ random state transitions within a cell and define the loss for training the dynamics model as:
\vspace{-0.1cm}
\begin{equation}
\label{drift:eq:main_loss}
\mathcal{L}_d = \sum_{n=[0, 1, ... N]}(\mathcal{L}_g( f_{\psi}(s_{t+1}^{c^n} \mid s_{t}^{c^n}, a_t^{c^n}, l^{c^n} ),  \Bar{s^{c^n}_{t+1}})),
\vspace{-0.1cm}
\end{equation}
where $l^{c^0} = 0$ and for $n>0$:
\vspace{-0.1cm}
\begin{equation}
% l^{n+1} = M_{\phi}\left(I^{c^{n}}, S^{c^{n}}, H_{s}^{c^{n}}, H_{a}^{c^{n}}, l_{\theta}^{c^{n}}\right),
l^{c^{n+1}} = M_{\phi}\big(I^{c^{n}}, S^{c^{n}}, H_{s}^{c^{n}}, H_{a}^{c^{n}}, l_{\theta}^{c^{n}}\big),
\vspace{-0.05cm}
\end{equation}
and where $\mathcal{L}_g(\theta, \mathrm{target}) = \frac{1}{2} (\log \theta_{\sigma}^2 + \frac{(\theta_\mu - \mathrm{target})^2}{\theta_{\sigma}^2})$ is the Gaussian negative log-likelihood loss, and $\Bar{s^{c^n}_{t+1}}$ is the ground truth target state. The order of the traversals is irrelevant to our loss. Our loss represents the update scheme of the latent vectors, in which in the first iteration no knowledge of the surface is assumed. Thus, in the first iteration, a latent vector for the dynamics model is defined as a zero-vector, which forces the dynamics model to predict a future state distribution that is broad enough to cover all road materials properties that appear during training. This is particularly useful when unrolling state sequences over cells that have not been observed yet. In these cases, a conservative estimate of the state distribution is required as unobserved cells may entail any material or surface condition. 
In the following iterations of our loss function ($n >0$), the latent vector fed to the dynamics model is updated using our mapping model. In our loss function, the latent update is calculated based on the observed data of the previous traversal $n-1$, while the dynamics model predicts the state transition for the current traversal $n$ in the same cell $c$.  Thus, we ensure that the latent vectors are independent of the currently predicted state transition and represent a joint representation that improves prediction accuracy for all transitions in the respective cell.
Note that if the dynamics model would receive a latent vector generated by the mapping network using data recorded at the same time as the inputs of the dynamics model, the mapping model could directly contribute to the prediction of the next state rather than representing a spatial property of the track.
For our experiments, we set the number of selected state transitions to $N=3$.

\subsubsection{Three Stage Training}
We optimize our model using a three-stage approach. In the first stage, we optimize the dynamics model as well as the latent vectors but without training the mapping network. Instead, we optimize the latent vectors $l^c$ directly by backpropagating into them, treating the map as model parameters. We denote the directly optimized latent parameters as $\Bar{l^{c}} \in \Bar{L}$. In contrast to the limited information of the input of the mapping network at inference time, this has the advantage that the latent vectors can be thoroughly optimized over all batches of the dataset. 
We denote the resulting loss as:
\vspace{-0.1cm}
\begin{equation}
    \mathcal{L}_{d_{\mathrm{s1}}} = \sum_{n=[0, 1, ... N]}(\mathcal{L}_g( f_{\psi}(s_{t+1}^{c^n} \mid s_{t}^{c^n}, a_t^{c^n}, \Bar{l^{c}}),  \Bar{s^{c^n}_{t+1}}))
\vspace{-0.1cm}
\end{equation}
In contrast to the later training stages, we do not inject zero-vectors, while optimizing the latent vectors directly as we presented in Eq. \ref{drift:eq:main_loss}. Experiments have shown that the training becomes instable otherwise.

% To avoid agitated latent maps, we additionally add a smoothness term that minimizes the local gradient of the latent maps:
To avoid agitated latent maps, we add a smoothness term minimizing the local gradient of the latent maps:
\vspace{-0.15cm}
\begin{equation}
\mathcal{L}_s = \; \mid \mid \nabla \Bar{L} \mid \mid^2    
\vspace{-0.15cm}
\end{equation}
The overall loss being optimized in the first stage is simply a weighted sum of both loss functions:
\vspace{-0.15cm}
\begin{equation}
    \mathcal{L}_{\mathrm{s1}} = \mathcal{L}_{d_{\mathrm{s1}}} + \lambda \mathcal{L}_s,
\vspace{-0.15cm}
\end{equation}
where $\lambda$ denotes a weighting hyper-parameter that defines the strength of the smoothness term.

However, as the parameter map $\Bar{L}$ is optimized offline, it can not be employed in practical applications as the vehicle should be capable of driving through previously unseen environments. By leveraging our multimodal mapper, new environments should be observed on-the-fly avoiding this limitation.
Thus, in the second stage, we freeze the learned latent parameters $\Bar{L}$ and the dynamics model while optimizing the mapping network. In this stage, we guide the latent predictions from our mapping model, by optimizing the negative log-likelihood of the predicted latent vector distribution $l_{\theta}^{c}$ given the learned parameter corresponding to the same cell $\Bar{l^{c}}$.
Overall, the loss for the second stage of our training scheme is defined as:
\vspace{-0.15cm}
\begin{equation}
    \mathcal{L}_{\mathrm{s2}} = \sum_{n=[1, ... N]} \mathcal{L}_g(l_{\theta}^{c^n}, \Bar{l^{c^n}}).
\vspace{-0.15cm}
\end{equation}
In the last stage, we then freeze the mapping model and refine the dynamics model by optimizing Eq.~\ref{drift:eq:main_loss} and feed the estimate $l^c$ from the mapping network into the dynamics model instead of the previously used learned parameter $\Bar{l^{c}}$.

\begin{figure*}
\centering
\includegraphics[width=0.9\linewidth]{images/scheme2.png}
\caption{In our approach the vehicle estimates a distribution of a latent representation of the road surface using different modalities, such as history state/action information, RGB images, and acoustic spectrograms. Thereby, all local latent distributions are collected in a global grid map $L$.  Once the vehicle traverses a cell $c$ of the grid map, the underlying latent distribution $l_{\theta}^c$ gets updated. For planning purposes, an ensemble of probabilistic dynamics models estimates the distribution of the future state $s_{t+1}$ given the actions $a_t$, the current input state $s_t$ and a sampled latent vector from the latent distribution of the grid-map-cell that corresponds to the input state $s_t$. As the dynamics model can be successively applied, a distribution of future trajectories can be obtained. As both mechanisms, latent mapping and state prediction can run independently from each other, the mapping is conducted asynchronously.}
\label{fig:architecture}
\vspace{-0.5cm}
\end{figure*}



\vspace{-0.1cm}
\subsection{Planning and Control}
\vspace{-0.1cm}
In order to follow a reference trajectory $T_r$, we follow a model predictive control approach. In detail, we use iCEM~\cite{pinneri2020sample}, which generates multiple future state sequences by sampling over the actions. 
The best sequences are selected using a pre-defined reward function and are refined for a specific amount of iterations. In contrast to the vanilla CEM~\cite{de2005tutorial}, iCEM provides significantly better sample efficiency and generates smoother trajectories due to enforced temporal consistency along the state sequences. These properties make the sampling-based planning real-time capable, which is a crucial requirement for high-speed autonomous driving.
\subsubsection{Trajectory Unrolling and Latent Sampling}
To generate candidate trajectories for the planning module, we start from the initial current state of the vehicle and unroll future state sequences by successively applying our dynamics model given a sequence of actions. We then accumulate all predicted local changes of the x-position $\Delta p_x$, y-position $\Delta p_y$, and yaw angle $\Delta \gamma$ to convert the state sequences into trajectories that are defined in the coordinate system of the initial state. 
Finally, we add the initial position of the vehicle to convert them into the global coordinate system. To sample multiple trajectories from our ensemble of dynamics models, we employ the \textit{TS1}-strategy \cite{chua2018deep}. In each iteration of iCEM we generate trajectories for $N_a$ distinct action sequences. As our ensemble of dynamics models predicts a single state transition at a time, we apply our dynamics model $h$ times for a trajectory with the length of $h$ time steps and the same number of actions. Further, we sample $k$ different state hypotheses for each individual action, effectively resulting in $N_a*k*h$ inferences passes of our dynamics model and $k*N_a$ trajectories. During trajectory generation, the latent vectors are sampled from the respective latent distribution for each individual state of the trajectories and leveraged for the next update using the dynamics model. Thus, intra-trajectory changes of the road materials are considered.
\subsubsection{Map Update}
We start with a map in which the parametrization of each cell is set to zero. As discussed in Sec.~\ref{drift:sec:training} this indicates zero knowledge about the map.
As the latent distribution in each cell can be updated asynchronously with respect to the prediction of the dynamics model, we run the mapping model and the controller including the dynamics model in separate threads.
This ensures that the dynamics model does not need to wait for the mapping inference to finish and allows for parallelization as we run the dynamics model on CPU and the mapping model on GPU.
\subsubsection{Reward Function}
To score all trajectory candidates, we propose a reward function that evaluates these in terms of different metrics.
The target trajectory is given by $x-y$ coordinate waypoints.
We measure the deviation of a given trajectory to the target with the cross-track error $R_{cte}$.
As we want the car to move as fast as possible we define a progress reward $R_p$ by adding up the length of all the line segments between waypoints the trajectory passes.
Further, to punish risky maneuvers and prevent shortcuts we define a binary boundary violation reward $R_b$ which is equal to $1$ if a specified boundary around the waypoints is exceeded and $0$ otherwise.
Lastly, to encourage smooth driving we define the reward $R_a$ as the absolute difference between the last executed throttle command and that of the first action of the trajectory.
The final reward for a trajectory is defined as
\vspace{-0.1cm}
\begin{equation}
    % R = w_p \cdot R_p - w_{cte} \cdot R_{cte} - w_a \cdot R_a + w_t \cdot R_t - w_b \cdot R_b,
    R = w_p \cdot R_p - w_{cte} \cdot R_{cte} - w_a \cdot R_a - w_b \cdot R_b,
\vspace{-0.1cm}
\end{equation}
where we set the weighting parameters to $w_p=40, w_{cte}=10,w_a=20, w_b=20000$.

The ARMBench dataset presents: 1) a collection of sensor data acquired by a robotic manipulation workcell performing pick-and-place operation, 2) metadata and reference images for objects in containers, 3) a set of annotations acquired either automatically, by virtue of the system design, or via manual labeling, and 4) tasks and metrics to benchmark perception algorithms for robotic manipulation. Fig.\ \ref{fig:contributions} illustrates the benchmark tasks and variety of objects captured in the dataset. The dataset captures diversity in objects with respect to Amazon product categories as well as physical characteristics such as size, shape, material, deformability, appearance, fragility, etc. 

The data collection platform is a robotic manipulation workcell performing pick-and-place operation in a warehouse \cite{Sparrow2022}. The workcell contains a robotic arm mounted with a vacuum-based end-effector. It is presented with a heterogeneous collection of objects placed in unstructured configurations within a container (storage tote). The robotic arm is tasked with picking one object at a time (singulation) and place it on moving trays until the container is empty. The empty container ejects the workcell and is replaced by a new container. While the operation is completely autonomous, it includes a human-in-the-loop to monitor the status of each pick-and-place activity, annotate, and resolve any defects during manipulation. Multiple imaging sensors are placed in the workcell to facilitate and validate the pick-and-place operation. Following is a list of sensor data (Fig.\ \ref{fig:intro}) associated with each pick activity:
\begin{itemize}
\item Pick-image: A 5\,MP camera is used to capture a top-down image of the container.
% \item Pick-3D: Two Ensenso sensors capture the 3D point cloud of the source container.
\item Transfer-images: Multiple 5\,MP cameras are placed on different sides in the workcell to capture the moving object from different viewpoints.
% \item Transfer-Barcode: Multiple Cognex barcode sensors are used to scan the barcode of the object during transfer.
\item Place-image: A top-down view of the object is captured once it is placed on the tray.
\item Video: A camera is mounted to capture 720p videos of pick-and-place manipulation processes at 30\,FPS
\end{itemize}
Additionally, the following metadata (Fig.\ \ref{fig:contributions} (b)) is available by virtue of a warehouse tracking system:
\begin{itemize}
\item Container-manifest: A list of objects present in the container along with data such as product description, coarse dimensions, and weight.
\item Reference images: One or more images of objects from previous operations within the warehouse.
\end{itemize}
The sensor data and metadata were consumed by perception algorithms required to autonomously operate the robotic workcell. Benchmarking against these algorithms would not only optimize a manipulation task such as the one used for data collection but also enable more complex and intentional manipulation. This work considers a subset of such perception tasks namely object segmentation, object identification, and defect detection. These are critical not only to make informed grasping and motion decisions but also to track the state of the objects and containers within the warehouse. The following sections will describe these tasks and present the challenges using annotations, baseline algorithms, and evaluation metrics.
\section{Results}
\label{results}

\begin{figure*}[ht]
    \centering
    \includegraphics[scale=0.15,trim={0 2.5cm 0 5cm},clip]{images/aoi-single_burst}
    \caption{The time average peak Age of Information with burst and \gls{soa} loss values against the dynamic reliability logic for different network topologies.}
    \label{fig:aoi_burst}\vspace{-0.4cm}
\end{figure*}


This paper focuses on both transport layer and application layer metrics to determine the feasibility of dynamic reliability. For this, we have selected the session packet volume, as transmitted, retransmitted, lost and backlogged packets as \glspl{kpi} for the transport layer; while focusing on the \gls{aoi} for the application layer. The \gls{aoi} was chosen as a crucial indicator for the freshness of packets in real-time applications. More specifically, this work adopts the time average peak \gls{aoi} equation \cite{aoi_equation} depicted in Eq. \ref{aoi}, where $\Delta(r_{i+1})$ is the $i$th update at the time it was received at the server, for a session time period of $\tau$.

\begin{equation}
    \label{aoi}
    \gls{aoi}_\tau = \frac{1}{n-1}\sum_{i=1}^{n-1} \Delta(r_{i+1})
\end{equation}

We include a comparison between the vanilla QUIC implementation which does not enjoy the dynamic reliability extension, with a number of dynamic reliability policies. The tests were run a number of times for statistical significance, with the mean value of vanilla implementation used as a baseline for comparison. The topology utilised both random loss and bursty loss to explore the bounds of dynamic reliability. The \gls{soa} loss in the figures correspond to the loss values presented in Table. \ref{tab:path_char}, for ease of comparison between bursty and random loss scenarios.

\subsection{Transport-Layer KPIs}

To analyse the performance gain at the transport layer due to dynamic reliability, the volume of transmitted and backlogged packets is examined. The figures are in the form of boxplots, which take the vanilla implementation as a benchmark, depicted as the red dashed line.

As seen in Fig. \ref{fig:sent_burst}, the loss plays a crucial role in the performance of the reliability policies. The policies under random loss did incredibly well for the networks with a larger capacity, namely \gls{mmwave} and Sub-6~GHz, whereas for burst loss, the lower network capacities had a larger packet reduction. With the increase in burst loss, the behaviour of the set split reliable policies became unpredictable, if a reliable assignment happened to coincide with a burst loss, the number of transmitted packets increases, and vice versa. On the other hand, in smarter policies, such as Loss-Aware, the performance lightly matched the vanilla baseline, as the reliable assignment dominated the session to compensate for a higher burst loss. Not only that but, the burst loss also impacted the variance of the transmitted packets for the policies.

Unsurprisingly, the unreliable focused policy, 80-20 split, outperformed other policies for all topologies in random and bursty loss scenarios, with an approximate reduction of 80\%. That being said, the majority of the policies reduced the transmitted packets on the link by approximately 70\% for random loss, while the reduction started at $\approx 15\%$ and decreased as the loss increased for the burst loss scenario.

The retransmitted and lost packets, not shown due to space limitations, followed the same trend as the transmitted packets for the random loss scenarios. However, for the burst loss scenarios, the larger capacity networks had a lower reduction in the retransmitted and lost packets. This can be seen as a favorable outcome since the lower capacity networks are scarce on resources. It is important to note that the Loss-Aware policy mimicked the vanilla approach as the burst loss increased, signifying the overwhelming appointment of reliable packets in adapting to the harsh burst loss conditions.
 
Alternatively, Fig. \ref{fig:backlog_burst} clearly shows a stark comparison between the policies and loss scenario in the reduction of the backlogged packets. The Loss-Aware policy for random loss scenario reduced the backlogged packets by up to 50\%, beating all other policies by approximately 30\%. Furthermore, it is clear that the unreliability focused policies resulted in the lowest backlog for the session. In comparison, we notice that the burst loss and the backlogged frequency have a positive correlation, where the maximum reduction of the backlogged packets for the policies is at most 20\%. Much like the transmitted packets, the probability of a burst loss occurrence plays a vital role in the number of retransmissions sent and by extension the number of backlogged packets. Thus, we can conclude that the stress placed on the buffer is a result of the reliable packets which is tightly coupled with the congestion on the session. Whereas, unreliable focused policies did not encounter such a phenomenon regardless if it was experiencing a burst loss.


\subsection{Application-Layer KPIs}

The feasibility of dynamic reliability for real-time applications can be determined by the \gls{aoi}, with comparison across different topologies and policies. If we take a strict approach and consider anything below $10$~ms is real-time \cite{real-time}, then all the reliability policies passed that requirement, which is attractive for real-time applications, as shown in Fig. \ref{fig:aoi_burst}. Utilising the median as an estimate of the runs, the policies in the WLAN and Sub-6~GHz topology with random loss floated around $4-5$~ms with negligible difference, while the \gls{aoi} for \gls{mmwave} was $\approx 2-3$~ms. It is clear that the \gls{aoi} and the network capacity have a negative correlation, as the network capacity decreases, the \gls{aoi} increases. The same correlation is extended to the bursty loss scenarios, where \gls{mmwave} dominated the other topologies. That being said, it is crucial to note that the \gls{aoi} for the reliability policies is often slightly better than or equal to the \gls{aoi} of the vanilla implementation, proving that dynamic reliability reduces the congestion of the session at no cost to the \gls{aoi}.

\section{Conclusions}
We consider the phase-extraction problem, and we showed that, given a unitary $U = e^{i\pi H}$ and its inverse $U^{\dag}$, we could implement a block-encoding of $\phi(H)$ for some smooth function $\phi(x)$. The word `smooth' here means existence and continuity of the derivatives: the higher the number of continuous derivatives that a function has, the faster its Fourier sum (and thus the Laurent polynomial on the eigenphases) uniformly converges to that function. We are confident this can have many more applications beyond what is shown in this work. It is also worth remarking that Jackson showed that the convergence rate of a Fourier series is almost-optimal, in the sense that no trigonometric (or, equivalently, complex exponential) series can approximate the desired function faster, up to that $\log d$ factor~\cite[p.\ 21]{jacksonTheoryApproximation1930a}. Also remember that `smoothing' a function, i.e., replacing its derivative with a continuous function, does not give faster convergence for free in general, as its derivative will become steep in the points where we smooth out discontinuities, and this translates to a high Lipschitz constant: a~clear example is given by Eq.~\ref{eq:lipschitz-constant-recurrence-solution}, but in that case, fortunately, nothing depends on the size of the input $N$, and thus does not influence the asymptotic query complexity of Algorithm~\ref{alg:prop-sampling-qsp}, although the constant factor can become large even for $p = 20$. From a theoretical point of view, this work shows that, for any $\eta > 0$, there is an algorithm with query complexity 
$$\Tilde{\bigO}\left(\frac{1}{\bar{c}^{\frac{1}{2} + \eta}} \frac{1}{\epsilon^\eta} \right)$$
solving the proportional-sampling problem. This statement seems to suggest there exists an algorithm which directly solves the problem with $\eta = 0$, and an open question would be to find such algorithm.


It is also interesting to remark that Theorems~\ref{thm:haah-construction},~\ref{thm:haah-completion} indeed allow the construction for any $\phi$, even complex-valued, provided that its absolute value is reciprocal.

One could think that, in Section~\ref{sec:prop-sampling}, instead of using the linear function in the phase-extraction subroutine, we could approximate the square root and then apply the transformation directly on $e^{i \pi c(x)}$. However, in the case of proportional sampling this would be inconvenient, as the derivative of the square root function has a discontinuity with an infinite jump around 0, and we could not choose a constant $\delta$ if we had values of the oracle that are too close to $0$.



%\addtolength{\textheight}{-12cm}   % This command serves to balance the column lengths
                                  % on the last page of the document manually. It shortens
                                  % the textheight of the last page by a suitable amount.
                                  % This command does not take effect until the next page
                                  % so it should come on the page before the last. Make
                                  % sure that you do not shorten the textheight too much.


%%%%%%%%%%%%%%%%%%%%%%%%%%%%%%%%%%%%%%%%%%%%%%%%%%%%%%%%%%%%%%%%%%%%%%%%%%%%%%%%

% \vspace{-0.5cm}
{\small
\bibliographystyle{IEEEtran}
\bibliography{egbib}
}


\end{document}
