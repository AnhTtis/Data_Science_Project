%%%%%%%% ICML 2023 EXAMPLE LATEX SUBMISSION FILE %%%%%%%%%%%%%%%%%

\documentclass{article}

% Recommended, but optional, packages for figures and better typesetting:
\usepackage{microtype}
\usepackage{graphicx}
\usepackage{subfigure}
\usepackage{booktabs} % for professional tables
\usepackage{adjustbox}
\usepackage{multirow}


% hyperref makes hyperlinks in the resulting PDF.
% If your build breaks (sometimes temporarily if a hyperlink spans a page)
% please comment out the following usepackage line and replace
% \usepackage{icml2023} with \usepackage[nohyperref]{icml2023} above.
\usepackage{hyperref}


% Attempt to make hyperref and algorithmic work together better:
\newcommand{\theHalgorithm}{\arabic{algorithm}}

% Use the following line for the initial blind version submitted for review:
% \usepackage{icml2023}

% If accepted, instead use the following line for the camera-ready submission:
\usepackage[accepted]{icml2023}

% For theorems and such
\usepackage{amsmath}
\usepackage{amssymb}
\usepackage{mathtools}
\usepackage{amsthm}

% if you use cleveref..
\usepackage[capitalize,noabbrev]{cleveref}

%%%%%%%%%%%%%%%%%%%%%%%%%%%%%%%%
% THEOREMS
%%%%%%%%%%%%%%%%%%%%%%%%%%%%%%%%
\theoremstyle{plain}
\newtheorem{theorem}{Theorem}[section]
\newtheorem{proposition}[theorem]{Proposition}
\newtheorem{lemma}[theorem]{Lemma}
\newtheorem{corollary}[theorem]{Corollary}
\theoremstyle{definition}
\newtheorem{definition}[theorem]{Definition}
\newtheorem{assumption}[theorem]{Assumption}
\theoremstyle{remark}
\newtheorem{remark}[theorem]{Remark}

% Todonotes is useful during development; simply uncomment the next line
%    and comment out the line below the next line to turn off comments
%\usepackage[disable,textsize=tiny]{todonotes}
\usepackage[textsize=tiny]{todonotes}


% The \icmltitle you define below is probably too long as a header.
% Therefore, a short form for the running title is supplied here:
\icmltitlerunning{ProtFIM: Fill-in-Middle Protein Sequence Design via Protein Language Models}

\begin{document}

\twocolumn[
\icmltitle{ProtFIM: Fill-in-Middle Protein Sequence Design \\ via Protein Language Models}

% It is OKAY to include author information, even for blind
% submissions: the style file will automatically remove it for you
% unless you've provided the [accepted] option to the icml2023
% package.

% List of affiliations: The first argument should be a (short)
% identifier you will use later to specify author affiliations
% Academic affiliations should list Department, University, City, Region, Country
% Industry affiliations should list Company, City, Region, Country

% You can specify symbols, otherwise they are numbered in order.
% Ideally, you should not use this facility. Affiliations will be numbered
% in order of appearance and this is the preferred way.
\icmlsetsymbol{equal}{*}

\begin{icmlauthorlist}
\icmlauthor{Youhan Lee}{comp}
\icmlauthor{Hasun Yu}{comp}
% \icmlauthor{Firstname3 Lastname3}{comp}
% \icmlauthor{Firstname4 Lastname4}{sch}
% \icmlauthor{Firstname5 Lastname5}{yyy}
% \icmlauthor{Firstname6 Lastname6}{sch,yyy,comp}
% \icmlauthor{Firstname7 Lastname7}{comp}
% %\icmlauthor{}{sch}
% \icmlauthor{Firstname8 Lastname8}{sch}
% \icmlauthor{Firstname8 Lastname8}{yyy,comp}
%\icmlauthor{}{sch}
%\icmlauthor{}{sch}
\end{icmlauthorlist}

% \icmlaffiliation{yyy}{Department of XXX, University of YYY, Location, Country}
\icmlaffiliation{comp}{Kakao Brain Corp, Seongnam, Gyeonggi-do, Republic of Korea}
% \icmlaffiliation{sch}{School of ZZZ, Institute of WWW, Location, Country}

\icmlcorrespondingauthor{Youhan Lee}{youhan.lee@kakaobrain.com}
\icmlcorrespondingauthor{Hasun Yu}{shawn.yu@kakaobrain.com}

% You may provide any keywords that you
% find helpful for describing your paper; these are used to populate
% the "keywords" metadata in the PDF but will not be shown in the document
\icmlkeywords{Machine Learning, Protein design, Protein Language Modeling}

\vskip 0.3in
]

% this must go after the closing bracket ] following \twocolumn[ ...

% This command actually creates the footnote in the first column
% listing the affiliations and the copyright notice.
% The command takes one argument, which is text to display at the start of the footnote.
% The \icmlEqualContribution command is standard text for equal contribution.
% Remove it (just {}) if you do not need this facility.

\printAffiliationsAndNotice{}  % leave blank if no need to mention equal contribution
% \printAffiliationsAndNotice{\icmlEqualContribution} % otherwise use the standard text.

\begin{abstract}

Protein language models (pLMs), pre-trained via causal language modeling on protein sequences, have been a promising tool for protein sequence design. In real-world protein engineering, there are many cases where the amino acids in the middle of a protein sequence are optimized while maintaining other residues. Unfortunately, because of the left-to-right nature of pLMs, existing pLMs modify suffix residues by prompting prefix residues, which are insufficient for the infilling task that considers the whole surrounding context. To find the more effective pLMs for protein engineering, we design a new benchmark, Secondary structureE InFilling rEcoveRy, SEIFER, which approximates infilling sequence design scenarios. With the evaluation of existing models on the benchmark, we reveal the weakness of existing language models and show that language models trained via fill-in-middle transformation, called ProtFIM, are more appropriate for protein engineering. Also, we prove that ProtFIM generates protein sequences with decent protein representations through exhaustive experiments and visualizations.


\end{abstract}


%%%%%%%%%%%%%%%%%%%%%%%%%%%%%%%%%%%%%%%%%%%%%%%%%%%%%%%%%%%%%%%%%%%%%%%%%%%%%%%
%%%%%%%%%%%%%%%%%%%%%%%%%%%%%%%%%%%%%%%%%%%%%%%%%%%%%%%%%%%%%%%%%%%%%%%%%%%%%%%
% main
%%%%%%%%%%%%%%%%%%%%%%%%%%%%%%%%%%%%%%%%%%%%%%%%%%%%%%%%%%%%%%%%%%%%%%%%%%%%%%%
%%%%%%%%%%%%%%%%%%%%%%%%%%%%%%%%%%%%%%%%%%%%%%%%%%%%%%%%%%%%%%%%%%%%%%%%%%%%%%%

% \begin{figure}[t]
%     % \begin{subfigure}{1\linewidth}
%     %   \centering
%     % %   \includegraphics[width=1\linewidth]{figs/fig_1_moti_textattn.pdf}  
%     % %   \includegraphics[width=1\linewidth]{figs/fig_1_moti_textattn_v2.pdf}  
%     %   \includegraphics[width=1\linewidth]{figs/fig_1_moti_textattn_v5.pdf}  
%     %   \vspace{-0.5cm}
%     %     \caption{Amount of attention added to each video clip from the source video and query text in the self-attention layers of Moment-DETR encoder.}
%     %     % \caption{Distribution of attention for source and query in Moment-DETR encoder}
%     %     % Visualization of video clip's self-attention score in Moment-DETR encoder.
%     %   \label{fig:fig1_text_attn_ex}
%     % \end{subfigure}%\hfill% or  or \hspace{0.3\textwidth}
%     \vspace{0.2cm}
%     % \begin{subfigure}{1\linewidth}
%       \centering
%     %   \includegraphics[width=1\linewidth]{figs/fig1_moti_negattn.pdf}  
%       \includegraphics[width=1\linewidth]{figs/fig1_moti_negattn_v3.pdf}  
%       \vspace{-0.4cm}
%     %   \caption{Correspondence of saliency scores on the relevance between video clips and the text query.}
%     % \caption{Predicted saliency scores against the video relevant positive query and video irrelevant negative query}
%       \label{fig:fig1_neg_attn_ex}
%     % \end{subfigure}%\hfill% or  or \hspace{0.3\textwidth}
%     \caption{
%     % 원준 원본
%     % (a) Comparison between attention scores of source and query for each video clip~(We sum the attention scores from video and text). 
%     % We observe that the attention scores are dominated by other clips in the source video. 
%     % Text queries do not account for much attention regardless of the relevance to the video clips.
%     % \textbf{(a)} Inspection of the query dependency in Moment-DETR encoder.
%     % % We visualize the attention score of video tokens in the transformer encoder and observe that text query accounts for only a low portion of attention.
%     % % This tendency occurs regardless of the relevance between the text query and video clips. 
%     % We visualize the attention score of video tokens in the transformer encoder and observe 1) text query only accounts for a low portion of attention, and 2) relevance between video-query pair does not affect the attention scores ratio of text.
%     \textbf{(b)} Comparison of highlight-ness when relevant and non-relevant queries are input.
%     As observed in , existing work only uses queries to play an insignificant role, thereby may not be capable of detecting false queries and considering the video-query relevance even when the problem in (a) is resolved. 
%     % \SE{} % 이 부분이 "not capable of" 란 용어가 세다는 피드백이 있는 듯 합니다. 이러한 능력이 없다는 것은 굉장히 강한 어조인거 같기는 하고, 이러한 경우들이 종종 있다거나 좀 약화시킬 필요가 있어보이긴 하네요.
%     On the other hand, our QD-DETR yields a query-dependent representation that the relevance between the source video and query text is updated in the saliency scores.
%     There is a large gap between positive and negative saliency scores, and scores are consistent since the clips are all highly correlated to others.
%     }
%     \label{fig:motivation_ex}
%     % \captionsetup{belowskip=13pt}
%     % \setlength{\belowcaptionskip}{-10pt}
% \end{figure}
\begin{figure}
    \centering
    \includegraphics[width=1\linewidth]{figs/fig1_moti_negattn_1111.pdf}
    % \includegraphics[width=1\linewidth]{figs/fig1_moti_negattn_1109.pdf}
    % \includegraphics[width=1\linewidth]{figs/fig1_moti_negattn_stat.pdf}
    \vspace{-0.6cm}
    \caption{
        % \SE{} % 수정 필요
        Comparison of highlight-ness~(saliency score) when relevant and non-relevant queries are given.
        We found that the existing work only uses queries to play an insignificant role, thereby may not be capable of detecting negative queries and video-query relevance; saliency scores for clips in ground-truth~(GT) moments are low and equivalent for positive and negative queries.
        % This also results in mispredicted moments when ground-truth~(GT) moment is dominated by clips unrelated to GT since their prediction is highly focused on the video.
        % \SE{} % 여기 한번 더 보면 좋을 듯 합니다. GT moment에 unrelated한 clip이 많으면? label이 틀렷을 경우를 말씀하시는건지?
        % As observed in saliency graph, existing work only uses queries to play an insignificant role, thereby may not be capable of detecting false queries and considering the video-query relevance.
        On the other hand, query-dependent representations of QD-DETR result in corresponding saliency scores to the video-query relevance and precisely localized moments.
        % On the other hand, our QD-DETR yields a query-dependent representation that the
        % saliency scores are in accordance with the relevance between the video and query.
        % text is in accordance with the saliency scores.
        % There is a large gap between positive and negative saliency scores, and scores are consistent since the clips are all highly correlated to others.
}
    \label{fig:motivation_ex}
\end{figure}


\section{Introduction}
% 원준 원본
% Along with the advance of digital devices and platforms, video is now one of the most desired data type for consumers. However, although the large information capacity of videos may be beneficial in many aspects, e.g., informative and entertaining, on the contrary perspective, videos are time-consuming, and hard to search for desirable moments. 
% This has led many creators to use extra manpower to crop and edit the video to generate highlight clips to gain the consumer’s attention.
Along with the advance of digital devices and platforms, video is now one of the most desired data types for consumers~\cite{apostolidis2021video,wu2017deep}.
% SE: Video aware deep learning application & survey papers?
Although the large information capacity of videos might be beneficial in many aspects, e.g., informative and entertaining, inspecting the videos is time-consuming, so that it is hard to capture the desired moments~\cite{anne2017localizing,apostolidis2021video}. 
% This has led many creators to use extra manpower to crop and edit the video to generate highlight clips to gain the consumer’s attention.


% On the other side, 
Indeed, the need to retrieve user-requested or highlight moments within videos is greatly raised.
Numerous research efforts were put into the search for the requested moments in the video~\cite{anne2017localizing, gao2017tall, liu2015multi, escorcia2019temporal} and summarizing the video highlights~\cite{zhang2016video, mahasseni2017unsupervised, badamdorj2022contrastive, wei2022learning}.
% Numerous research efforts were put into the search for the requested moments in the video~\cite{anne2017localizing, gao2017tall, liu2015multi, escorcia2019temporal}, summarizing the video to generate highlights was another popular topic~\cite{zhang2016video, mahasseni2017unsupervised, badamdorj2022contrastive, wei2022learning}.
Recently, Moment-DETR~\cite{momentdetr} further spotlighted the topic by proposing a QVHighlights dataset that enables the model to perform both tasks, retrieving the moments with their highlight-ness, simultaneously.

% 원준 원본
% To detect the desired moments, previous works employed transformer encoder-decoder architectural designs to fuse the text query into the video representations. Moment-DETR~\cite{mDETR} modified detection transformer to process capture the moment as a set, and UMT~\cite{umt} implemented transformer decoder as to output clip-wise saliency. 
% Yet to their outstanding breakthroughs in the literature of moment retrieval with the seminal architectures, their limitation is that the role of the given text query is insignificant in representing the query-conditioned video representation; the attention mechanism of moment DETR is not explicitly conditioned on the text query, and the text query is conditioned on multi-modal clips where the differences between the clips are smoothed after encoding process in UMT.



% \begin{figure}[t]
% \centering
%     \begin{subfigure}[l]{0.37\linewidth}
%       \centering
%       \vspace{0.20cm}
%     %   \includegraphics[width=1\linewidth]{figs/fig_1_moti_textattn.pdf}  
%     %   \includegraphics[width=1\linewidth]{figs/fig_1_moti_textattn_v2.pdf}  
%       \includegraphics[width=1\linewidth]{figs/fig1_moti_violin_a.pdf}  
%       \vspace{-0.60cm}
%     %   \caption{text attention}
%         \caption{Importance of queries in video representation}
%       \label{fig:fig1_text_attn}
%     \end{subfigure}%\hfill% or  or \hspace{0.3\textwidth}
%     \vspace{0.2cm}
%     \begin{subfigure}[r]{0.61\linewidth}
%       \centering
%     %   \includegraphics[width=1\linewidth]{figs/fig1_moti_negattn.pdf}  
%       \includegraphics[width=1\linewidth]{figs/fig1_moti_violin_b.pdf}  
%     %   \caption{neg attention}
%         % \caption{Relation between the highlight-ness and the relevance between videos and query texts.}
%         \caption{Highlight-ness~(saliency) histogram of positive and negative video-query pairs\SE{}}
%       \label{fig:fig1_neg_attn}
%     \end{subfigure}%\hfill% or  or \hspace{0.3\textwidth}
%     % \vspace{-0.2cm}
%     \caption{Overall statistics for attention scores in Fig.~\ref{fig:motivation_ex} in QVHighlights dataset. 
%     (a) For the attention scores that measure how much the text query is generally involved in video representation, we use violin plots to show the probability density. We plot the score for each layer in the encoder.
%     % (b) Using the histogram, we compare how the baseline and QD-DETR yield different salient scores given the positive and negative video-text pairs.
%     (b) Saliency histogram shows the distributional gap between positive and negative video-text query pairs of baseline~(Moment-DETR) and proposed QD-DETR.\SE{}
%     }
%     \label{fig:motivation}
%     % \captionsetup{belowskip=13pt}
%     % \setlength{\belowcaptionskip}{-10pt}
% \end{figure}

% \begin{figure}[t]
% \centering

%     \begin{subfigure}[r]{1\linewidth}
%       \centering
%       \hspace{-0.2cm}
%     %   \includegraphics[width=1\linewidth]{figs/fig1_moti_negattn.pdf}  
%       \includegraphics[width=1.1\linewidth]{figs/fig1_moti_violin_a_v2.pdf}  
%     %   \caption{neg attention}
%         % \caption{Relation between the highlight-ness and the relevance between videos and query texts.}
%         \vspace{-0.5cm}
%         % \caption{Saliency histogram of positive and negative video-query pairs}
%         \caption{We plot the histograms and its average value~(dotted line) to compare saliency scores when true and false text queries are given for each method. (left) Since the video representations do not include much textual information, both the true and false queries yield similar saliency scores. (Middle) Even when the video representation is enforced to be updated with the textual information, the issue is not much resolved. (Right) By extracting discriminative features in the text query, distributions are differentiated.
%         % \SE{} % R1@0.5 설명
%         Also, R1@0.5 indicates evaluation metric, Recall at 1 with IoU 0.5 threshold on QVhighlight \textit{val} set.
%         }
%       \label{fig:fig1_neg_attn}
%     \end{subfigure}%\hfill% or  or \hspace{0.3\textwidth}
%     \\
%     \begin{tabular}{cc}
%     \hspace{-0.2cm}
%         \begin{minipage}{.4\linewidth}
%             \begin{subfigure}[l]{1\linewidth}
%               \centering
%             %   \vspace{0.20cm}
%             %   \includegraphics[width=1\linewidth]{figs/fig_1_moti_textattn.pdf}  
%             %   \includegraphics[width=1\linewidth]{figs/fig_1_moti_textattn_v2.pdf}  
%               \includegraphics[width=1\linewidth]{figs/fig1_moti_violin_a.pdf}  
%               \vspace{-0.60cm}
%             %   \caption{text attention}
%                 \caption{Importance of queries in video representation}
%               \label{fig:fig1_text_attn}
%             \end{subfigure}%\hfill% or  or \hspace{0.3\textwidth}
%         \end{minipage}
        
%         \begin{minipage}{.6\linewidth}
%             \vspace{-0.2cm}
%             \caption{Overall statistics of Fig.~\ref{fig:motivation_ex} in QVHighlights dataset. 
%             (a) Saliency histogram shows the distributional gap between positive and negative video-text query pairs.
%             % (a) For the attention scores that measure how much the text query is generally involved in video representation, we use violin plots to show the probability density. We plot the score for each layer in the encoder.
%             % (b) Using the histogram, we compare how the baseline and QD-DETR yield different salient scores given the positive and negative video-text pairs.
%             % (b) Text ratio in self-attention layer to  of Moment-DETR
%             % (b) Ratio of text when representing video tokens in self-attention of Moment-DETR.
%             % (b) Magnitude of attention text query involved.
%             % (b) Attention score of video tokens
%             % (b) Magnitude of text query to refine the video tokens in self-attention layer of Moment-DETR.
%             (b) Probability density depicting the weight of the text query in attention score for video clips. Scores are from the self-attention layers in Moment-DETR encoder.
%             % (b) The text query ratio in attention score of video clips (Self-attention layer in Moment-DETR encoder). We use violin plots to show probability density.
%             % 텍스트 쿼리가, 비디오 피쳐에 얼만큼 attend 하는지
%             }
%         \end{minipage}
    
%     \end{tabular}
%     \vspace{-0.5cm}
%     \label{fig:moti}
%     % \captionsetup{belowskip=13pt}
%     % \setlength{\belowcaptionskip}{-10pt}
% \end{figure}


% \begin{figure}
%     \centering
%     % \includegraphics[width=1\linewidth]{figs/fig1_moti_negattn_1109.pdf}
%     \includegraphics[width=1\linewidth]{figs/fig1_moti_negattn_stat_v2.pdf}
%     \vspace{-0.8cm}
%     \caption{
%         Histogram of saliency when the positive and negative queries are given. We plot the histograms and its average value~(dotted line) to compare saliency scores when relevant~(positive) and irrelevant~(negative) text queries are given for each method. (Left) Since the video representations do not properly reflect textual information, both the positive and negative queries yield similar saliency scores. 
%         % (Middle) Even when the video representation is enforced to be updated with the textual information, the issue is not much resolved. 
%         (Right) By representing video clips in query-dependent manner, distributions are differentiated.
%     }
%     \vspace{-0.6cm}
%     \label{fig:motivation}
% \end{figure}


% One of the demanding task is moment retrieval task, which is detecting the desired moments from the given query, typically the text query.
When describing the moment, one of the most favored types of query is the natural language sentence~(text)\cite{anne2017localizing}. 
While early methods utilized convolution networks~\cite{zhang2020learning, gao2021fast, wang2020temporally}, recent approaches have shown that deploying the attention mechanism of transformer architecture is more effective to fuse the text query into the video representation.
% To handle these modalities, previous works simply employed the attention mechanism of transformer architecture to fuse the text query into the video representation.
For example, Moment-DETR~\cite{momentdetr} introduced the transformer architecture which processes both text and video tokens as input by modifying the detection transformer~(DETR), and UMT~\cite{umt} proposed transformer architectures to take multi-modal sources, e.g., video and audio. 
Also, they utilized the text queries in the transformer decoder.
Although they brought breakthroughs in the field of MR/HD with seminal architectures, they overlooked the role of the text query.
To validate our claim, we investigate the Moment-DETR~\cite{momentdetr} in terms of the impact of text query in MR/HD~(Fig.\ref{fig:motivation_ex}).
Given the video clips with a relevant positive query and an irrelevant negative query, we observe that the baseline often neglects the given text query when estimating the query-relevance scores, i.e., saliency scores, for each video clip.
% the output saliency score, i.e. query-relevance scores.
% Based on the observation, we traced the actual saliency prediction of the model against both the video-relevant query and the irrelevant dummy one where we find that the baseline often neglects the given text query when estimating the query-relevance scores of video clips.
% For example, in Fig.~\ref{fig:motivation_ex}, saliency scores are not affected even when the query is substituted with the dummy.
% % General statistics for Fig.~\ref{fig:motivation_ex} is shown in Fig.~\ref{fig:motivation}. 
% General statistics corresponding to Fig.~\ref{fig:motivation_ex} are also shown in Fig.~\ref{fig:motivation}.



% The limitation of the concrete baseline~\cite{momentdetr} is inspected in two different aspects; 1) Utilization of text-query in the encoding process and 2) the output saliency score, i.e. query-relevance scores.
% Firstly, we visualize the attention score when video clips are given as a query in self-attention. 
% We observe that the text queries have relatively small impacts compared to other video features, as shown in Fig.~\ref{fig:fig1_text_attn_ex}.
% That is, the text does not account for much in representing every video clip, although the goal of MR/HD is to detect query-relevant moments.
% Based on the observation, we traced the actual saliency prediction of the model against both the video-relevant query and the irrelevant dummy one where we find that the baseline often neglects the given text query when estimating the query-relevance scores of video clips.
% For example, in Fig.~\ref{fig:motivation_ex}, saliency scores are not affected even when the query is substituted with the dummy.
% % General statistics for Fig.~\ref{fig:motivation_ex} is shown in Fig.~\ref{fig:motivation}. 
% General statistics are also shown in Fig.~\ref{fig:motivation}.

% Consequently, in Fig.~\ref{fig:fig1_neg_attn_ex}~(b), we found that the baseline often neglects the given text query when estimating the query-relevance scores of video clips; 
% For example, 


% We validate the previous work sometimes neglects the given query when estimating the saliency of video clips.
% For example, there is an example that the saliency scores from positive and negative queries cannot be distinguishable, as shown in Fig.~\ref{fig:fig1_neg_attn_ex}.
% % 우리는 추가로 text attention을 추가도 해봤지만, 효과가 있긴 했으나, still 이슈가 있는 것을 확인하였다?
% % Still, we observe that assuring the high attendance of text queries does not resolve the overlap which motivates us to question the quality of the naive use of task-agnostic text representation~\cite{momentdetr, umt}.
% We found that introducing the text-attention for ensuring the high attendance of text queries relieve the overlap, but there still be a severe overlap.


% To validate their limitations, we inspect the impacts of text queries in the concrete baseline~\cite{momentdetr} with the two different aspects, 1) tendency of attention in self-attention layer and 2) saliency score, i.e. query-relevance scores. \SE{} % attention 이 갑자기 등장하는가?
% Firstly, we visualize the attention score when video clips are given as a query in self-attention. We observe the text queries have relatively low attention scores compared to the video features, as shown in Fig.~\ref{fig:fig1_text_attn_ex}.
% That is, the text does not account for much in representing every video clip, although the goal of MR/HD is to detect query-relevant moments.
% Based on this observation, we trace the actual saliency prediction of the model against both positive and negative text queries.
% We validate the previous work sometimes neglects the given query when estimating the saliency of video clips.
% For example, there is an example that the saliency scores from positive and negative queries cannot be distinguishable, as shown in Fig.~\ref{fig:fig1_neg_attn_ex}.
% % 우리는 추가로 text attention을 추가도 해봤지만, 효과가 있긴 했으나, still 이슈가 있는 것을 확인하였다?
% % Still, we observe that assuring the high attendance of text queries does not resolve the overlap which motivates us to question the quality of the naive use of task-agnostic text representation~\cite{momentdetr, umt}.
% We found that introducing the text-attention for ensuring the high attendance of text queries relieve the overlap, but there still be a severe overlap.



% Thus, we 
% query dependency를 높이기 위해 
% Cross-attention? text-attention? detailed explanation on text-attention should be needed?
% By handling these two issues, we find that more precise retrieval can be achieved.
% 
% 
%
% By projecting video-discriminative text features with high text attendance to source video, we f 
% We also find the need to improve the quality of query features since assuring high text attendance also results in...
% pairs are not finetuned to be discriminative that even the similarity within the pairs does not reflect the relevance between the query and the video clips.
% General statistics for Fig.~\ref{fig:motivation_ex} is shown in Fig.~\ref{fig:motivation}. 
% \SE{} % 이거 ??로 뜨는데, 위처럼 figure 그리면 label이 안되는걸까요
% \SE{}
% 형님 아래 사항 생각 좀 해보는게 좋을 거 같아요.
% fig 1. (a) 그림만 봤을 때 모든 clip에 대해 text attention이 일정이상 존재하긴 하니까, 뭔가 not assured to be conditioned가 와닿지 않는거 같아요.
% + 왜 text가 항상 attend 해야하나?
% not assured to be conditioned --> text shows relatively low affects compared to video 같이 실제 나타난 현상까지 같이 적으면 어떨까 싶어요.
% fig 1. (b) 덜 반영한다?

% \SU{}
% 일단 text가 attend 잘 되어야 한다는 것에 좀 궁금점이 생깁니다. 결국에는 text와 관련있는 frame들을 attend해서 higlight를 찾아야 하는게 아닐까요? 그리고, 현제 저희의 모델 구조상 text query가 Key와 Value로 거의 활용되고 있는데 그렇다면 결국에는 해당 모델은 text에 대한 attention이 전혀 없다고 봐도 무방하지 않을까요? 그런 면에서 text attention을 강조하는게 좀 걸리긴 합니다.

% Specifically, the text query is not assured to be explicitly conditioned on every clip of the video, and as the query texts are evenly treated, discriminative keywords may not be spotlighted.
% attention mechanism of Moment-DETR is not explicitly conditioned on the text query as shown in Fig~\ref{}(d), and in UMT, the text are only used for conditioning the queries while the video representation are refined itself by self-attention.

% \begin{figure}[t]
%     \begin{subfigure}{1\linewidth}
%       \centering
%     %   \includegraphics[width=1\linewidth]{figs/fig_1_moti_textattn.pdf}  
%     %   \includegraphics[width=1\linewidth]{figs/fig_1_moti_textattn_v2.pdf}  
%       \includegraphics[width=1\linewidth]{figs/fig_1_moti_textattn_v4.pdf}  
%       \vspace{-0.5cm}
%     %   \caption{text attention}
%         \caption{Distribution of attention scores in Moment-DETR encoder}
%       \label{fig:fig1_text_attn}
%     \end{subfigure}%\hfill% or  or \hspace{0.3\textwidth}
%     \vspace{0.2cm}
%     \begin{subfigure}{1\linewidth}
%       \centering
%     %   \includegraphics[width=1\linewidth]{figs/fig1_moti_negattn.pdf}  
%       \includegraphics[width=1\linewidth]{figs/fig1_moti_negattn_v2.pdf}  
%       \vspace{-0.5cm}
%     %   \caption{neg attention}
%         \caption{Saliency score against positive and negative text queries}
%       \label{fig:fig1_neg_attn}
%     \end{subfigure}%\hfill% or  or \hspace{0.3\textwidth}
%     \vspace{0.2cm}
%     \begin{subfigure}{1\linewidth}
%       \centering
%     %   \includegraphics[width=1\linewidth]{figs/fig1_moti_violin.pdf}  
%       \includegraphics[width=1\linewidth]{figs/fig1_moti_violin_v2.pdf}  
%       \vspace{-0.5cm}
%       \caption{violin}
%       \label{fig:fig1_violin}
%     \end{subfigure}%\hfill% or  or \hspace{0.3\textwidth}
%     \vspace{-0.2cm}
%     \caption{(a) 1. portion of text attention vs. video attention 2. relation with text query and content (e.g. fg, bg) of clip seems not to affect the attention score
%     (b) 1. high variability even though entire clips are highly correlated with the given text query 2. positive and negative query makes overlaps on saliency score distribution
%     (3) actual distribution on validation dataset.}
%     \label{fig:motivation}
%     % \captionsetup{belowskip=13pt}
%     % \setlength{\belowcaptionskip}{-10pt}
% \end{figure}

To this end, we propose Query-Dependent DETR~(QD-DETR) that produces query-dependent video representation.
% Our key focus is to ensure each clip in predicted moments is explicitly conditioned by the query, particularly on the video-descriptive portion of the text query.
% Our key focus is to ensure that query-relevant clips are predicted by enforcing each clip to be explicitly conditioned by the query.
%Our key focus is to ensure that the model prediction for each clip is highly relevant to the query.
Our key focus is to ensure that the model's prediction for each clip is highly dependent on the query.
% by enforcing each clip to be explicitly conditioned by the query. :)
% hmm...
% \SE {} % "query-relevant clips are predicted" 이 문장이 좀 애매한거 같습니다. relevant 클립을 놓지지 않고 찾는 것을 보장한다? 이런 느낌인지 아니면 높은 saliency 를 주는게 목적이다? model prediction이 query-relevance를 반영하는 것을 보장한다?
% Our key focus is to ensure that the model prediction reflects query-relevance of clips by enforcing each clip to be explicitly conditioned by the query.
First, to fully utilize the contextual information in the query, we revise the transformer encoder to be equipped with cross-attention layers at the very first layers.
% 상익's thought :  single video - query간의 관계만 고려 - 같은 word가 더 많이 쓰이는 것을 보고 
% 교수님's thought : neg pair 를 쓰면 쿼리를 보지 않고서는 video clip간만 고려하는 것이 사라짐. 왜냐면 0으로 내보내야 하기 때문. --> SE: relative difference 만 고려하다가, 
By inserting a video as the query and a text as the key and value of the cross-attention layers, our encoder enforces the engagement of the text query in extracting video representation.
% 원준 교수님 코멘트 반영해서 다시
Then, in order to not only inject a lot of textual information into the video feature but also make it fully exploited, we leverage the negative video-query pairs generated by mixing the original pairs.
Specifically, the model is learned to suppress the saliency scores of such  negative~(irrelevant) pairs.
Our expectation is the increased contribution of the text query in prediction since the videos will be sometimes required to yield high saliency scores and sometimes low ones depending on whether the text query is relevant or not.
% \SE{}
% learns to?
% By suppressing the saliency scores of the irrelevant video-query pairs, the model learns to spotlight only the video-specific discriminative words in the query.
% % \SE{} % ====================== 상익 수정 ========================
% However, this architectural design still lacks the capability of identifying the video-descriptive keywords in the query.
% % However, this architectural design still lacks in identifying proper query relevance.
% This is because the current training scheme only focuses on the interactions of video and clips within a single video while neglecting information shared throughout the entire video.
% % We argue the problem of the current training scheme that only focuses on distinguishing the clips in a single video while neglecting information shared throughout the entire video.
% Therefore, we leverage the negative video-query relationships to enhance the capability of identifying the contextual similarity of query and video clips.
% 
% 원준 원본 
% However, this architectural design heavily relies on the quality of the text query.
% Therefore, we leverage the negative video-query relationships to enable the model to emphasize key corresponding query features.
% By suppressing the saliency scores of the irrelevant video-query pairs, the model learns to spotlight only the video-specific discriminative words in the query.
% =========================================================
Lastly, to apply the dynamic criterion to mark highlights for each instance, we deploy a saliency token to represent the entire video and utilize it as an input-adaptive saliency criterion. 
With all components combined, our QD-DETR produces query-dependent video representation by integrating source and query modalities.
This further allows the use of positional queries~\cite{dabdetr} in the transformer decoder.
% Furthermore, we can exploit the advanced DETR decoder architectures using the positional information, e.g., DAB-DETR, since our encoded tokens consist of identical position representations from a single modality.
% \SE{} % ====================== 상익 수정 ========================
% Furthermore, we can exploit the advanced DETR decoder architectures using the positional information, e.g., DAB-DETR, since our video clip tokens consist of identical position representations from a single modality.
% 원준 원본
% It also enables the use of advanced DETR decoder architectures, e.g., DAB-DETR, for the first time, as these works exploit the position information within a single modality.
% =========================================================
Overall, our superior performances over the existing approaches validate the significance of the role of text query for MR/HD.
% Our extensive experiments on QVHighlights, TVSum, and Charades-STA datasets validate the significance of considering the role and the quality of text query.

% All components combined with dynamic anchor moments for the query of decoder, our FOQUE fosters the query-dependent video representation, thereby making the 
% All components combined, our modified transformer encoding process fosters the query-dependent video representation thereby achieving the state-of-the-art results on various benchmarks of moment-retrieval and highlight detection.
	
% -	Video Platform & Streamer & Consumer의 증가. 
% Video는 다른 데이터 타입보다 정보가 많아 유용하지만, 이는 다른 말로 해석하면 video를 보는 것은 time-consuming 하고, 원하는 것을 찾아보기에는 힘들 수 있음.
% 따라서, 많은 매체에서는 사람들의 더 많은 이목을 끌기 위해 highlight 비디오라는 것을 편집하여 공유도 함.
% 하지만, highlight video를 만들기 위해 사람의 노력이 필요한 현 시점에서, This spotlights the need to retrieve the user-requested / Highlight moments in the video.

% -	이전에도 이러한 문제를 해결하기 위해 (asdfasdf) for moment retrieval, (asdfasdf) for highlight detection 등이 제안 되었지만, 이들은 비디오의 특정 영역을 찾는다는 공통된 목적을 가지고 있으면서도, 데이터 셋의 한계로 인해 따로 연구되었음. 이를 문제 삼으며, 최근에는 두 task를 동시에 학습할 수 있는 dataset이 소개 되었는데, 컴퓨터비전에서 최근 각광을 받고 있는 Transformer 모델 도입과 함께 큰 발전을 거듭하고 있음.

% -	구체적으로, 이 두가지 task를 수행하기 위해서는 transformer를 두가지 방법으로 이용할 수 있는데, moment-DETR 처럼 moment 를 clip의 set 단위로 예측할 수 있고, UMT 처럼 clip-wise prediction을 할 수 있음. 하지만, 이들은 query를 condition이 아닌 video와 동등한 레벨로 취급하거나 [mDETR], 매 클립이 self-attention으로 mixing 된 후에 condition을 걸어주어 clip간의 차이를 확실하지 이용하지 못하였고, 또한, 확실하게 condition으로 주지 못하였고, video와 query 사이의 관계를 한정적으로만 이용하였다.

% -	따라서, we explore three different ways to fully exploit query information. First, we design one-way cross-attention layer to condition every clip with the query features. Then, we utilized the negative video-text pairs to better model the relationships between the video and the text embeddings. Lastly, we define the saliency token to be the video-query dependent saliency estimator.


















% ===================== neg pair 부분 ===========================
% Nevertheless, the current training scheme, only considering the given video-query pair, still disturbs the model from identifying proper query-relevance prediction.
% In detail, the model focus on learning the fine-grained discrepancy between video clips, while neglecting the information they share, which contains significant clues to understand the context of video.
% Therefore, we leverage the negative video-query relationships to enhance the capability of identifying the contextual similarity of query and video clips.
% Therefore, we leverage the negative video-query relationships by suppressing those pairs, so that enhance the capability of identifying the contextual similarity of query and video clips.
% We hypothsize the diversity in query-video pairs are insufficient to learn the general relationship between text query and video.
% Therefore, we leverage the negative video-query relationships by suppressing the saliency scores of the irrelevant video-query pairs.
% However, this architectural design still lacks in identifying proper query relevance.
% We argue that the current training scheme only focuses on learning the fine-grained discrepancy between clips in a single video, while neglecting the information they share, which contains significant clues to understand the context of the video.
% Therefore, we leverage the negative video-query relationships to enhance the capability of identifying the contextual similarity of query and video clips.
% However, this architectural design still lacks in identifying proper query relevance.
% We argue the problem of the current training scheme that only focuses on learning the fine-grained discrepancy between clips in a single video.
% That is, the current design neglects the information shared throughout the video, although it contains significant clues to understand the context of the video.
\section{Related Works}

\begin{figure*}[!ht]
\centering
\includegraphics[width=\linewidth]{body/figures/data_collection2.png}
\caption{\textbf{Hardware Setup.} We use a GelSight Wedge sensor for tactile sensing, an Intel ReslSense D405 camera mounted on the side for RGB vision sensing, and an OptiTrack setup for motion capture. \textbf{Data Collection.} The tactile finger and the camera are fixed to the table at all times. A human operator moves a test object and presses it against the finger. We show sampled tactile and RGB images as well as a reconstructed local tactile depth map on the right.}
\label{datacollection}
\end{figure*}

% \textbf{Tactile sensors}:
% Over the years, researchers have developed tactile sensors working on different sensing principles, such as resistance, capacitance, magnetic, barometric, and optic.
% We refer readers to \cite{kappassov2015tactile} for an in-depth review of different types of tactile sensors and their applications.
% Compared to other sensing principles, GelSight tactile sensors have the advantage of providing high-resolution geometrical information of the contact surface.
% They are usually constructed with an elastic silicone gel, directional colored LEDs, and a camera pointing at the gel.
% The gels are usually coated with reflective paint with printed dots.
% When in contact, the gel deforms and takes the shape of the contact surface.
% Shear force can be retrieved by tracking dot movements.
% Furthermore, the color value of a pixel is correlated with the gradient of the height of the contact surface at the specific location. 
% With a pre-calibrated color table, a depth map can be reconstructed from the color image.
% This type of tactile sensor is selected for our work for its rich output and ease to use.

Researchers of the robotics community have put forward a wide range of tactile sensing solutions.
Sensors working on different sensing principles have been adopted to solve a large set of manipulation tasks.
Among different types of tactile sensors, vision-based ones such as GelSight \cite{yuan2017gelsight} and GelSlim \cite{donlon2018gelslim} stand out for their rich output, ease to use, and affordability.
While we focus on the pose estimation and shape reconstruction task using vision-based tactile sensors, we refer readers to \cite{kappassov2015tactile} for an in-depth review of different types of tactile sensing and their applications.
In this section, we review works on three typical tasks that are most relevant to our solution: slip detection, object property inference, and SLAM.

% Researchers have found that using this class of vision-based tactile sensors can greatly increase the accuracy when reasoning about the contact surface, compared to traditional tactile sensors that are constructed with normal direction force sensors [\todo{add citation}].

\textbf{Slip detection and estimation}:
Using a similar sensor to ours, Yuan \etal compared and analyzed a GelSight tactile sensor's images collected at different stages of slip in \cite{yuan2015measurement} and showed this type of sensors' capability in detecting micro scale movements.
Li \etal and Zhang \etal trained recurrent neural networks on tactile images to detect slip between multiple time steps in a manipulation sequence \cite{li2018slip, zhang2018fingervision}.
Built on their binary slip detection model in \cite{li2018slip}, Li further added rotational slip direction prediction in \cite{li2019rotational}.
Calandra \etal improved a grasp planner for the classic robot bin-picking problem by incorporating slip detection and achieved a higher grasp success rate \cite{calandra2017feeling}. 
However, those methods only detect slip without localizing the object after the slip.
In many precision manipulation tasks we are also interested in the amount of the displacement.

\textbf{Object property inference and localization}:
% With detailed information on the contact surface provided by high-resolution tactile sensors, 
Many works have focused on inferring properties of the in-contact object, such as shape \cite{strub2014using, luo2015tactile, luo2019iclap}, texture \cite{luo2018vitac, yuan2017connecting}, and material \cite{yuan2017connecting, kroemer2011learning, kerr2018material}.
Those learned object properties can be further used for localization.
In order to localize current grasps, Bauza \etal proposed to match new tactile imprints with previously collected tactile imprints \cite{bauza2019tactile}, while Luo \etal learned to match tactile imprints directly to visual images of the whole object \cite{luo2015localizing}.
Assuming known CAD models, Bauza \etal proposed to localize by comparing contact masks generated from tactile images with a large bank of random projections of the CAD model \cite{bauza2022tac2pose}.
To solve the reverse problem, i.e. what a tactile image looks like given an object and a pose, several tactile simulators have been built to automatically generate tactile images given an object's CAD model and a finger pose \cite{si2022taxim, wang2022tacto}.
One major limitation for this category of works is that they all require a known calibrated geometry of the object: a pre-collected tactile map \cite{bauza2019tactile}, a model of the object \cite{bauza2022tac2pose}, or a global image with known geometry \cite{luo2015localizing}.
This requirement can be hard to meet in less constraint environments.

\textbf{Tactile SLAM}:
Recent studies have shown interests in working with unknown objects by leveraging methods from the SLAM problem.
With a focus on 2D shapes, Suresh \etal parameterized shapes as Gaussian Process Implicit Surfaces (GPIS), and learned its parameters from tactile signals collected during pushing \cite{suresh2021tactile}.
Assuming known contact poses, authors of \cite{suresh2022shapemap} first learned a noisy mapping from known surface geometries to corresponding tactile images, then reconstructed an object by combining many noisy local tactile measurements into an optimized global shape using factor graph optimization.
The closest prior work to ours is \cite{sodhi2022patchgraph}, where the authors learned to estimate 6D poses and 3D shapes simultaneously for unknown objects. 
They constructed a pose estimator based on tactile sensing, and a shape reconstruction pipeline that added in new tactile point clouds incrementally on the run.
However, this approach heavily relies on the performance of the tactile pose estimator, which lacks a global understanding of the object and can suffer from repeated patterns or smooth surfaces.
In contrast, our work combines vision and tactile sensing which provides us with both global and local understandings of the scene without requiring any other domain knowledge.
Furthermore, we designed a loop closure mechanism that periodically matches current tactile and vision images to stored key-frames, which significantly reduced accumulated errors.
With this, FingerSLAM is able to produce realistic reconstructions even in long sequences. 
\section{Methods}
\label{sec:methods}
\subsection{Preliminary}
In this section, we introduce each component of MAPSeg (\hyperref[fig2]{Fig.2}) and how MAPSeg can serve as a unified solution to centralized, federated, and test-time UDA (\hyperref[fig:overview]{Fig.1b}). We deploy MAPSeg for domain adaptative 3D segmentation of heterogeneous medical images and it consists of three components: (1) 3D masked multi-scale autoencoding for self-supervised pre-training, (2) 3D masked pseudo-labeling for domain adaptive self-training, and (3) global-local feature collaboration to fuse global and local contexts for the final segmentation task. The hybrid cross-entropy and Dice loss (\hyperref[eq:L_seg]{Eq.1}) is often adopted for regular supervised segmentation training, and we employ it as the basic component of the objective functions for MAPSeg:
\begin{equation}
    \label{eq:L_seg}
    \mathcal{L}_{seg}(\hat{y},y) = -\frac{1}{n}\sum_i\sum_jy_{i,j}\log(\hat{y}_{i,j}) -\frac{2\sum y\hat{y}+\epsilon}{\sum y+\sum \hat{y}+\epsilon}
\end{equation}
where $n$ denotes the number of pixels, $y_{i,j}$ and $\hat{y}_{i,j}$ represent the ground truth label and predicted probability for the $i$th pixel to belong to the $j$th class, and $\epsilon$ is used to prevent zero-division. 

In the following sections, notations are defined as: $x$ and $y$ indicate the original image and label of the randomly sampled local patch; $X$ and $Y$ refer to downsampled global scan and label; the subscripts $s$ and $t$ refer to the source and target domains, respectively; the superscript $M$ indicates the image is masked (\eg, $x_t^M$ refers to a masked local patch from the target domain).

\begin{figure*}
\centering
\includegraphics[width=0.85\linewidth]{./figs/fig2-11.pdf}
\caption{Components of the proposed MAPSeg framework. (a) 3D multi-scale masked autoencoding. (b) 3D masked pseudo labeling in source and target domains. (c) 3D Global-local collaboration.} 
\label{fig2}
\end{figure*}
\subsection{3D Multi-Scale Masked Autoencoder (MAE)}
In this study, we propose a 3D variant of MAE using a 3D CNN backbone (\hyperref[fig2]{Fig.2a}). The detailed configuration can be found in Appendix \cref{sec:archite}. Training is jointly performed on two image sources with identical size ($96^3$ voxels): local patches $x$ randomly sampled from the volumetric scan, and the whole scan downsampled to the same size, denoted as $X$. 
Both $x$ and $X$ are masked before feeding into the MAE: $x$ is divided into non-overlapping 3D sub-patches with size $8^3$, of which 70\% are masked out randomly based on a uniform distribution (\hyperref[fig2]{Fig.2a}); The same procedure is applied to $X$ with patch size $4^3$ since it contains a larger field-of-view (FOV). The masked versions of $x$ and $X$ are denoted as $x^M$ and $X^M$, respectively. We train the MAE encoder and decoder to reconstruct $x/X$ based on $x^M/X^M$ using mean squared error on the masked-out regions as the objective function.

\subsection{3D Masked Pseudo-Labeling (MPL)}
MPL uses a teacher-student framework which is a standard strategy in semi-/self-supervised learning~\cite{grill2020bootstrap,NIPS2017_68053af2} to provide stable pseudo labels on an unlabeled target domain during training. 
After MAE pre-training, we keep the MAE encoder $g$ and append a segmentation decoder $h$ to build the segmentation model $f=h\circ g$ (\hyperref[fig2]{Fig.2b-c}). Given an input image $x_s$ and label $y_s$ from the source domain and an input image $x_t$ from the target domain, the teacher model $f_\theta$ takes as input the target image $x_t$ and generates pseudo labels $f_\theta(x_t)$, with gradient detached. The student model $f_\phi$ is then optimized by minimizing the segmentation loss between the predictions of $x_t^M$/$x_s^M$ and $f_{\theta}(x_t)$/$y_s$, which can be formulated as:  
\begin{equation}
\label{eq:L_mpl}
\mathcal{L}_{MPL} = \mathcal{L}_{Seg}(f_{\phi}(x_t^M),f_{\theta}(x_t))+\beta\mathcal{L}_{Seg}(f_{\phi}(x_s^M),y_s)
\end{equation}
where $\beta$ is the weight of source prediction and set as 0.5. 
The teacher model's parameters $\theta$ are then updated during training via exponential moving average (EMA) based on the student model's parameters $\phi$~\cite{NIPS2017_68053af2}.

\begin{equation}
\label{eq:ema_update}
\theta_{t+1} \gets \alpha \theta_{t} + (1-\alpha)\phi_t, 
\end{equation}
where $t$ and $t+1$ indicate training iterations and $\alpha$ is the EMA update weight. For model initialized from the large-scale MAE pretraining, we set $\alpha$ as 0.999 during the first 1,000 steps and 0.9999 afterwards. For model pretrained on small-scale source and target datasets (\eg, only dozens of scans), we set $\alpha$ as 0.99 during the first 1,000 steps, 0.999 during the next 2,000 steps, and 0.9999 for the remaining training. The teacher model $f_{\theta}$ is initialized with student model's parameters $\phi$ after some warm-up training (\eg, 1,000 iterations) on the source-domain data. 

\subsection{3D Global-Local Collaboration (GLC)}
Directly applying MPL for UDA segmentation with large domain shift (\eg, cross-modality/sequence) may lead to unreliable pseudo-label and disrupt the training. Therefore, we design a GLC module (\hyperref[fig2]{Fig.2c}) to improve pseudo-labeling by leveraging the spatial global-local contextual relations induced by the inherent anatomical distribution prior in medical images. With the image encoder pretrained to extract image features at both local and global levels during multi-scale MAE, we take advantage of the global-local contextual relations by concatenating local and global semantic features in the latent space and make prediction based on the fused features. We differ from previous study~\cite{Chen_2019_CVPR} by only applying GLC on the output of the encoder $g$ instead of all layers to save computation cost and employing a different regularization to prevent segmentation decoder from predicting solely based on local features. 

In GLC, a binary mask $M$ is used to indicate the corresponding location of the local patch $x$ inside the downsampled global volume $X$. The encoder $g$ takes as input $x$ and $X$ and generates the local latent feature $\chi_{loc} = g(x)$ as well as cropped and resized global latent feature $\chi_{glo}=\mathit{upsample}(M \odot g(X))$, where $\odot$ indicates cropping $g(X)$ based on $M$ followed by upsampling to match the spatial size of $\chi_{loc}$. Therefore, segmenting a local patch $x$ can be rewritten as $f(x)=h(\chi_{loc}\oplus\chi_{glo})$, where $\oplus$ is the concatenation along channel dimension (\hyperref[fig2]{Fig.2c}). In addition, $f$ is also trained on downsampled global volume $X$ with $\mathcal{L}_{Seg}(f(X),Y)$), in which the global latent feature $g(X)$ is duplicated and $f(X) = h(g(X)\oplus g(X))$, to prevent model from solely relying on local semantic features and encourage the encoder to extract meaningful semantic features from both local and global levels.

We also add a regularization term between the $\chi_{loc}$ and $\chi_{glo}$ to maintain their similarity following~\cite{Chen_2019_CVPR}. Instead of the $\mathcal{L}_2$ regularization used in~\cite{Chen_2019_CVPR}, we maximize the cosine similarity between the $\chi_{loc}$ and $\chi_{glo}$ as:
\begin{equation}
\mathcal{L}_{cos}(x, X) = 1 - \frac{\chi_{loc}\cdot\chi_{glo}}{\max(\| \chi_{loc} \|_2, \| \chi_{glo} \|_2, \epsilon)}
\end{equation}
where $\epsilon$ is used to prevent zero-division. The loss function for GLC calculated on the source data is formulated as: 
\begin{align}
\label{eq.L_gs}
\mathcal{L}_{GLC}^{S} &= \gamma(\mathcal{L}_{Seg}(f_{\phi}(X_s),Y_s)+\mathcal{L}_{Seg}(f_{\phi}(X_s^M),Y_s))
\nonumber\\
&+\delta(\mathcal{L}_{cos}(x_s, X_s) + \mathcal{L}_{cos}(x_s^M, X_s^M))
\end{align}
where $\gamma$ and $\delta$ are the weights of the auxiliary global loss and cosine similarity, and set as $\gamma=0.05$ and $\delta= 0.025$ in our experiments. Similarly, the GLC loss is also calculated on the target data based on pseudo-label $f_{\theta}(X_t)$ and formulated as:
\begin{align}
\label{eq.L_gt}
\mathcal{L}_{GLC}^{T} &= 2\gamma\mathcal{L}_{Seg}(f_{\phi}(X_t^M),f_{\theta}(X_t)) + 2\delta\mathcal{L}_{cos}(x_t^M, X_t^M)
\end{align}
Therefore, the overall loss function of GLC is:
\begin{align}
\label{eq.L_global}
\mathcal{L}_{GLC} &= \mathcal{L}_{GLC}^{S}+\mathcal{L}_{GLC}^{T}
\end{align}
With the regular fully-supervised segmentation loss on source data $\mathcal{L}_{FSS} = \beta\mathcal{L}_{Seg}(f_{\phi}(x_s),y_s)$, where $\beta$ is defined as in \hyperref[eq:L_mpl]{Eq.2}, the overall objective function $\mathcal{L}$ for centralized UDA is formulated as:
\begin{equation}
\label{eq.L_center}
\mathcal{L} = \mathcal{L}_{FSS}+\mathcal{L}_{MPL}+\mathcal{L}_{GLC}
\end{equation}
It is clear that \hyperref[eq.L_center]{Eq.8} requires centralized and synchronous access to source and target data. In the section \hyperref[sec.fuda]{3.5} and \hyperref[sec.ttuda]{3.6}, we demonstrate how MAPSeg can be adapted to federated (decentralized and synchronous access to data) and test-time (decentralized and asynchronous access to data) UDA scenarios. 

\subsection{Extension to Federated UDA}
\label{sec.fuda}
In reality, labeled source-domain data and unlabeled target-domain data are often collected at different sites. We consider a practical scenario where a server (\eg a major hospital) hosts potentially large amount of both labeled and unlabeled scans, and distributed clients (\eg clinics or imaging sites) possess only unlabeled images. This is an under-explored scenario as FL typically assumes either fully or partially labeled data from all clients. We extend MAPSeg to solve this federated multi-target UDA problem according to the details in Algorithm 1 of Appendix \cref{sec:recipe}. Specifically, the server updates the student model $f_\phi$ by minimizing the loss for the labeled source-domain data $D_S$:
\begin{align}
    \mathcal{L}_s 
    &= \beta(\mathcal{L}_{seg}(f_\phi(x_s), y_s)+\mathcal{L}_{seg}(f_\phi(x_s^M), y_s)) \nonumber\\
    &+\gamma(\mathcal{L}_{seg}(f_\phi(X_s), Y_s)+\mathcal{L}_{seg}(f_\phi(X_s^M), Y_s)) \nonumber\\
    &+ \delta(\mathcal{L}_{cos}(x_s, X_s) + \mathcal{L}_{cos}(x_s^M, X_s^M)) \label{eq:loss_server}
\end{align}
The clients update the student model $f_\phi$ by minimizing the loss for its own unlabeled target-domain data $D_T^k$:
\begin{align}
    \mathcal{L}_u
    &= \beta(\mathcal{L}_{seg}(f_\phi(x_t^M), f_\theta(x_t))+\mathcal{L}_{seg}(f_\phi(x_t), f_\theta(x_t))) \nonumber\\
    &+ \gamma(\mathcal{L}_{seg}(f_\phi(X_t^M), f_\theta(X_t))+\mathcal{L}_{seg}(f_\phi(X_t), f_\theta(X_t))) \nonumber\\
    &+ \delta(\mathcal{L}_{cos}(x_t, X_t) + \mathcal{L}_{cos}(x_t^M, X_t^M)) \label{eq:loss_client}
\end{align}
Comparing to the centralized UDA loss (\hyperref[eq.L_center]{Eq.8}), we decompose it into two components: fully supervised loss for server training (\hyperref[eq:loss_server]{Eq.9}) and self-supervised loss for client updates (\hyperref[eq:loss_client]{Eq.10}), which avoids the need for centralized data. After each local update, each client sends the EMA teacher model parameters $\theta$ to the server for aggregation following typical federated averaging\cite{mcmahan2017communication}.

\subsection{Extension to Test-time UDA}
\label{sec.ttuda}
Test-time UDA often involves two separate stages of training, including the source-only training at one center and the target-only finetuning at another site. In the federated UDA setting, \hyperref[eq:loss_server]{Eq.9} and \hyperref[eq:loss_client]{Eq.10} are jointly used to update the server model through synchronous federated averaging after each round. We can further ease the constraint of synchronous communication between source and target sites by training $f_\phi$ on the source data using \hyperref[eq:loss_server]{Eq.9} for some (\eg 1,000) warm-up steps before distributing the model parameters $\phi$ to the target site for initializing the teacher model $f_\theta$. On the target site, $f_\theta$ provides stable pseudo-labels to guide the self-supervised training with \hyperref[eq:loss_client]{Eq.10} and is updated by the EMA of $\phi$ following \hyperref[eq:ema_update]{Eq.3}. We find that in this asynchronous setting MAPSeg still performs well on the target-domain data, albeit with a minor performance tradeoff on the source-domain data (see \hyperref[tab:testtime]{Tab.3}).

%-------------------------------------------------------------------------
\subsection{Implementation Details} \label{section:2.1}

\noindent\textbf{Model architecture and implementation.} We implement the encoder backbone $g$ using 3D-ResNet-like CNN. The segmentation decoder $h$ is adapted from DeepLabV3~\cite{chen2017rethinking}. The framework is implemented using PyTorch. More details of the model and the training procedure are provided in Appendix \cref{sec:archite} and \cref{sec:recipe}. 

\noindent\textbf{Selecting the best model.} For choosing the best model during training, some studies choose to train for fixed iterations and use the last checkpoint. On the other hand, some of the previous UDA studies~\cite{8988158,Chen_Dou_Chen_Qin_Heng_2019} face a dilemma in selecting the best model during training by validating against a hold-out portion of target-domain labels, which is unrealistic as UDA assumes full absence of target labels. We demonstrate that MPL not only provides an efficient pathway to domain adaptative segmentation but also serves as an indicator of how well the model is being adapted to the target domain. We validate the model after each epoch and the best model is selected based on the score: 
$\mathit{Score}=\mathit{Dice}_{Src}-0.5\times\overline{\mathcal{L}_{Seg}}(f_{\phi}(x_t^M),f_{\theta}(x_t))$, where $\mathit{Dice}_{Src}$ is the Dice score on source-domain validation set and $\overline{\mathcal{L}_{Seg}}(f_{\phi}(x_t^M),f_{\theta}(x_t))$ is the mean of $\mathcal{L}_{Seg}(f_{\phi}(x_t^M),f_{\theta}(x_t))$ during the last training epoch. From \hyperref[eq:L_seg]{Eq.1}, it is clear that $ \lim_{\hat{y}\to y} \mathcal{L}_{seg}(\hat{y},y)=-1$, therefore, $Score$ has an upper bound of $1.5$. We demonstrate in \hyperref[tab:cardiac]{Tab.4} that the difference between validation using target labels versus $Score$ is acceptable (81.2 vs. 80.3). Even without accessing target labels for validation, MAPSeg still surpasses the previous SOTA results that use target labels for validation. It is worth noting that we only use target labels for validation in \hyperref[tab:cardiac]{Tab.4} for a fair comparison with previously reported results; other results presented use $Score$ for validation by default. For federated and test-time UDA, $\mathit{Score} = -\overline{\mathcal{L}_{Seg}}(f_{\phi}(x_t^M),f_{\theta}(x_t))$.

\section{Experimental Setup}
\label{sec:experiments}
\begin{figure}[t]
    \centering 
    \hspace{-.04\columnwidth}
    \includegraphics[width=1.025\columnwidth]{results/VOC/figures/pareto_example.pdf}
    \caption{\textbf{Selecting models for evaluation.} For each configuration, we evaluate every model at every checkpoint and measure its performance across various metrics (\fone, \epg, \iou) on the validation set; \ie every point in the left graph corresponds to one model (for \bcos models optimized via the \epgloss loss at the input layer). Instead of evaluating a single model on the test set, we evaluate \emph{all Pareto-dominant} models, as indicated in the center and right plot.
    % \moritz{Did we not update the results to be consistent with this? I distinctly remember creating the plots for this. (The Pareto front here as a lot more points than those in the result figures...)}
    }
    \label{fig:pareto_example}
\end{figure}

In this section, we describe our experimental setup
and how we select the best models across metrics. {Full training details can be found in the supplement.} We evaluate across the full sweep of combinations of choices for each category, and discuss our results in \cref{sec:results}. 

\myparagraph{Datasets:} We evaluate on \voc \citeMain{everingham2009pascal} and \coco \citeMain{lin2014microsoft} for multi-label image classification. {In \cref{sec:results:waterbirds}, to understand the effectiveness of model guidance in mitigating spurious correlations, we also evaluate on the synthetically constructed Waterbirds-100 dataset \citeMain{sagawa2019distributionally,petryk2022guiding}, where landbirds are perfectly correlated with land backgrounds on the training and validation sets, but are equally likely to occur on land or water in the test set (similar for waterbirds and water). With this dataset, we evaluate model guidance for suppressing undesired features.}

\myparagraph{Attribution Methods and Architectures:} As described in \cref{sec:method:attributions}, we evaluate with \ixg \citeMain{shrikumar2017learning}, \intgrad \citeMain{sundararajan2017axiomatic}, \bcos \citeMain{bohle2022b}, and \gradcam \citeMain{selvaraju2017grad} using models with a \resnet \citeMain{he2016deep} backbone. For \intgrad, we use an \xdnn \resnet \citeMain{hesse2021fast} to reduce the computational cost, and a \bcos \resnet for the \bcos attributions. We optimize the attributions at the input and final layer\footnote{As typically used in \ixg (input) and \gradcam (final) respectively.}; for intermediate layer results, see supplement. Given the similarity of the results between \gradcam and \ixg, and since \bcos attributions performed better than \gradcam for \bcos models, we show \gradcam results in the supplement. 
All models were pretrained on \imagenet \citeMain{imagenet}, and model guidance was performed starting from a baseline model fine-tuned on the target dataset.

\myparagraph{Localization Losses:} As described in \cref{sec:method:losses}, we compare four localization losses in our evaluation: (i) \energyloss, (ii) \loneloss \citeMain{gao2022aligning,gao2022res}, (iii) \ppceloss \citeMain{shen2021human}, and (iv) \rrrloss (cf.~\cref{sec:method:losses}, \citeMain{ross2017right}).

\myparagraph{Evaluation Metrics:} As discussed in \cref{sec:method:metrics}, we evaluate both for classification and localization performance of the models. For classification, we report the F1 scores, similar results with \map scores can be found in the supplement. For localization, we evaluate using the \epg and \iou scores.

\myparagraph{Selecting the best models:} As we evaluate for two distinct objectives (classification and localization), it is non-trivial to decide which models to select during training. \Eg, a model that provides the best classification performance might provide significantly worse localization performance than a model that provides slightly lower classification performance but much better localization. Finding the right balance and deciding which of those models in fact constitutes the `better' model depends on the preference of the end user. 
Hence, instead of selecting models based on a single metric, we select the set of Pareto-dominant models \citeMain{pareto1894massimo,pareto2008maximum,backhaus1980pareto} across three metrics---F1, \epg, and \iou---for each training configuration, as defined by a combination of attribution method, layer, and loss. Specifically, as shown in \cref{fig:pareto_example}, we train for each configuration using three different choices of $\lambda_\text{loc}$, and select the set of Pareto-dominant models among all checkpoints (epochs and $\lambda_\text{loc}$). This provides a more holistic view of the general trends on the effectiveness of model guidance for each configuration.
\section{Discussion}

\noindent\textbf{Why does \NAME work better than CLIPScore?}
Our experimental results and human evaluations in Section~\ref{sec:exp} show that \NAME is more accurate than prior metrics (CLIP~\cite{radford2021learning} and captioning-based approaches) for evaluating text-to-image faithfulness. We hypothesize that the major challenge of these prior metrics is that they summarize the image outputs and text inputs into a single representation (embedding/caption). In contrast, \NAME exploits the power of the language models to decompose the text input into fine-grained probes, which allows VQA to capture more nuanced aspects of the text input and the generated image.

\noindent\textbf{How do current VQA models perform on \NAME?}
One limitation is that \NAME requires VQA models to work reasonably well, which is true given the current models and \NAME v1.0, as shown in Section~\ref{sec:exp}. 
Nevertheless, the assumption might not hold for current models in domains like anime and abstract art.
TIFA is a modularized evaluation framework.
The VQA models used within the framework can be updated as stronger VQA models become available in the future. 
For instance, we plan to incorporate GPT-4 once its image API is made public since it is likely to improve TIFA. Another possible solution is to ensemble multiple image understanding models. For example, one may employ expert models on art concepts. We leave this for future work.

\vspace{1mm}
\noindent\textbf{Other limitations.}
Another limitation of \NAME is its runtime. Answering multiple visual questions is slower than one CLIP inference. In the scenario described in Section~\ref{sec:exp:comparisonVQA}, mPLUG takes 1.6s to evaluate one image (without batching). Also, our question generation pipeline needs one inference on a modern language model for each text input.
The run time is not a critical issue for benchmarking purposes, but may not be computationally feasible for the kind of large-scale data filtering done, for example, in LAION-5B~\cite{schuhmann2022laion}.
Nevertheless, we would like to point out that our evaluation is much faster than the image generation process of diffusion models.
Thus, we believe it is feasible to perform reranking and reinforcement learning with \NAME on diffusion models.


 \section{Conclusion}
 In this paper, we have presented a tactile manipulation system that is able to rotate different objects without vision. We showed an end-to-end reinforcement learning framework to learn tactile dexterity over the proposed system. We carried out experiments both in simulation and real to demonstrate its effectiveness. Our work demonstrated that we are able to achieve tactile dexterity as humans in real for the first time. In the future, there are many promising future directions to investigate, such as exploring the use of a more dense contact sensor array and scaling up the system to solve more diverse tasks. We hope that our work can pave the way for more intelligent robot hands.




\bibliography{example_paper}
\bibliographystyle{icml2023}

%%%%%%%%%%%%%%%%%%%%%%%%%%%%%%%%%%%%%%%%%%%%%%%%%%%%%%%%%%%%%%%%%%%%%%%%%%%%%%%
%%%%%%%%%%%%%%%%%%%%%%%%%%%%%%%%%%%%%%%%%%%%%%%%%%%%%%%%%%%%%%%%%%%%%%%%%%%%%%%
% APPENDIX
%%%%%%%%%%%%%%%%%%%%%%%%%%%%%%%%%%%%%%%%%%%%%%%%%%%%%%%%%%%%%%%%%%%%%%%%%%%%%%%
%%%%%%%%%%%%%%%%%%%%%%%%%%%%%%%%%%%%%%%%%%%%%%%%%%%%%%%%%%%%%%%%%%%%%%%%%%%%%%%
\newpage
\appendix
\onecolumn

Next, we present the Supplementary Materials for the paper ``Re-ReND: Real-time Rendering of NeRFs across Devices''.
Specifically, in addition to the results reported in the paper, we report results of \methodname w.r.t. Image Quality~(Section~\ref{sec:im_qual}) and (Section~\ref{sec:quali}), Rendering Speed~(Section~\ref{sec:fps}), Mesh Size~(Section~\ref{sec:mesh_size} and Section~\ref{sec:meshi}), Disk Space~(Section~\ref{sec:disk_space}), validation of view-dependent effects (Section~\ref{sec:val}),  sensitivity to geometry variations (Section~\ref{sec:geo}) and Photo-metric quality w.r.t. embedding dimensionality $D$ (Section~\ref{sec:dim}).
Furthermore, we encourage the reviewers to watch the \textbf{associated video}, \texttt{Re-ReND.mp4}, demonstrating \methodname's capabilities of real-time rendering across devices.
% In particular, please refer to .
This video demonstrates how \methodname can render, in real time, a scene composed of tens (\Figure{composit}) or even thousands (\Figure{many_objects}) of objects. % , respectively. %  , or even with thousands of . %  in an AR headset.
\Figure{composit} illustrates such a scene, composed of moving chairs, hotdogs, the drumset, and a microphone.


% Finally, we also provide the PyTorch~\cite{NEURIPS2019_9015} and GLSL implementations of our method inside the folders called \texttt{Re-ReND\_Pytorch\_code} and \texttt{Re-ReND\_GLSL\_code}.

% \thispagestyle{empty}
% \appendix

%%%%%%%%% BODY TEXT - ENTER YOUR RESPONSE BELOW
% \section{The PyTorch code and GLSL code}

%  \begin{itemize}
%     \item Clean and README.md
%     \item Should I upload only pur method or MipNeRF and NeRF++?
%     \item Should I upload the generated data and the meshes in a google drive? What happens with anonymity?
% \end{itemize}

% \section{A video showing how we were measuring the FPS}
% \section{A video showing real scenes in comparison with MobileNeRF and SNeRG}
% \section{Qualitative Results}

%  \begin{itemize}
%     \item all objects visualizations 
% \end{itemize}

%-------------------------------------------------------------------------


\begin{figure}
    \centering
    \includegraphics[width=\linewidth]{pics/quantitative.pdf}
    \caption{Box plots of quantitative benchmarks MIG, FactorVAE, Disentanglement, and reconstruction error on dSprites and Shapes3D.}\label{fig:quantitative}
\end{figure}

%%%%%%%%%%%%%%%%%%%%%%%%%%%%%%%%%%%%%%%%%%%%%%%%%%%%%%%%%%%%%%%%%%%%%%%%%%%%%%%
%%%%%%%%%%%%%%%%%%%%%%%%%%%%%%%%%%%%%%%%%%%%%%%%%%%%%%%%%%%%%%%%%%%%%%%%%%%%%%%


\end{document}

