\section{Sequence design analysis}

% \begin{table*}[t]
% \caption{Zero-shot fitness prediction on FLIP tasks. All scores are Spearman correlation.}
% \label{tab:flip}
% \vskip 0.15in
% \begin{center}
% \begin{small}
% \begin{sc}
% \begin{adjustbox}{width=360pt}
% \begin{tabular}{l|cccccc}
%       \toprule % <-- Toprule here
%       Model & \#Params & Objective &  AAV & GB1 & Meltome & Meta Avg.\\
%       \midrule % <-- Midrule here
%        ESM-1b (mean)  & 750M & MLM & 0.36 & 0.34 & 0.71 & 0.47\\
%        ESM-1v (mean)  & 750M & MLM & 0.33 & 0.38 & 0.72 & 0.48\\
%        ProGen2-small  & 151M & CLM & 0.39 & -0.21 & 0.56 & 0.25\\
%        ProGen2-medium & 764M & CLM & 0.18 & -0.11 & 0.59 & 0.22\\
%        ProGen2-large  & 2.7B & CLM & 0.41 & 0.24 & 0.68 & 0.44\\
%        ProtXLNet      & 409M & PLM & 0.33 & 0.29 & 0.47 & 0.36\\
%        ProtGPT2-C & 80M & CLM & 0.40 & 0.18 & 0.53 & 0.37\\
%        ProtFIM        & 80M & FIM & 0.39 & 0.25 & 0.60 & 0.41\\
% \bottomrule
% \end{tabular}
% \end{adjustbox}
% \end{sc}
% \end{small}
% \end{center}
% \vskip -0.1in
% \end{table*}




\subsection{pLDDT change}
% AlphaFold2 gives a per-residue confidence metric called the predicted local distance difference test (pLDDT) ranging from 0 to 100. Recently, several works have used the metric as a scoring criterion to assess designed protein sequence by assuming that the higher pLDDT, the better and more plausible structure~\citep{moffat2022design,wang2022scaffolding}. To assess the FIM engineering performance of models in terms of pLDDT, we visualize the difference between pLDDT of the structure of both new sequences and the corresponding original sequence using a cumulative density plot. Figure~\ref{pLDDT_diff} reveals that positive cases where pLDDT increases after FIM engineering are rare for all models, but pLMs have more chance to get sequences with higher pLDDT.

AlphaFold2 gives a per-residue confidence metric called the predicted local distance difference test (pLDDT) ranging from 0 to 100. Recently, several works have used this metric as a scoring criterion to assess designed protein sequences by assuming that the higher pLDDT, the better and more plausible structure~\cite{moffat2022design,wang2022scaffolding}. To assess the FIM engineering performance of models in terms of pLDDT, we collect all pLDDT scores of all created structures in SEIFER and compute the differences of pLDDT between designed structures and corresponding original structures. Finally, we visualize the difference using a cumulative density plot. We compare ProtFIM with ProtGPT2-C to provide a fair comparison. Figure~\ref{pLDDT_diff} reveals that ProtFIM has the largest positive cases, where pLDDT increases ($\Delta$pLDDT$>$0), compared to both ProtGPT2-C and random mutation, demonstrating the specialty of ProtFIM in real-world protein sequence optimization.

\begin{figure}[h]
\vskip 0.2in
\begin{center}
\centerline{\includegraphics[width=200pt]{./figures/pLDDT_diff.pdf}}
\caption{Cumulative density plot on pLDDT change.}
\label{pLDDT_diff}
\end{center}
\vskip -0.2in
\end{figure}


\subsection{Case study}
 We cherry-pick a protein and visualize the original and modified structures through ProtFIM as shown in Figure~\ref{figure6}. The new two sequences of middle sites are different from the original sequences, but all have $\alpha$-helix. Interestingly, in-depth visualization considering the side-chain unveils the subtle difference, resulting in well or poorly-optimized sequences. All the above results demonstrate that our model can serve as a sequence design framework, which optimizes the target sequence while maintaining the structures. 

 % We cherry-pick a protein and visualize the original structure and modified structures through ProtFIM as shown in Figure~\ref{figure6}. The new two sequences of middle sites are different from the original sequences, but all have $\alpha$-helix. Interestingly, in-depth visualization considering the side-chain unveils the subtle difference, resulting in well or poorly-optimized sequences. All the above results demonstrate that our model, with the help of AlphaFold2, can serve as a sequence design framework, which optimizes the target sequence while maintaining the structures essential for the protein's function. 

\begin{figure}[h]
\vskip 0.2in
\begin{center}
\centerline{\includegraphics[width=200pt]{./figures/figure6.pdf}}
\caption{An example of a case where pLDDT increases or decreases after protein sequence design via ProtFIM.}
\label{figure6}
\end{center}
\vskip -0.2in
\end{figure}

