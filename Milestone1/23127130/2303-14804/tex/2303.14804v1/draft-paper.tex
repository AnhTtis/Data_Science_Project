\documentclass[final,5p,times,twocolumn]{elsarticle}
%\documentclass[review,5p]{elsarticle}
\usepackage{color}
\usepackage{graphicx,amssymb,epsfig,epsf,amsmath,fontenc}
\usepackage{caption, subcaption}
%\usepackage{tabular}
%\usepackage{multirow}
%\usepackage{fullpage}
\usepackage{tabularx}
%\usepackage{booktabs}
\usepackage{hyperref}
\usepackage{cleveref}
\usepackage{epstopdf}
%\usepackage[backend=biber,style=numeric-comp,sorting=none]{biblatex}
\usepackage{natbib}
%\usepackage[english]{babel}
%\addbibresource{ref.bib}
\bibliographystyle{elsarticle-num} 
%\bibliographystyle{unsrtnat}
\def\beqn{\begin{eqnarray}}
\def\eeqn{\end{eqnarray}}
\date{\today}%\vspace{-5ex}}
%\author{\vspace{-5eX}}
%\newwatermark[allpages,color=black!50,angle=45,scale=3,xpos=0,ypos=0]{DRAFT}

%opening
\def\beq{\begin{equation}}
\def\eeq{\end{equation}}
\def\beqn{\begin{eqnarray}}
\def\eeqn{\end{eqnarray}}

\newcounter{saveeqn}
\newcommand{\alpheqn}{\setcounter{saveeqn}{\value{equation}}%
  \stepcounter{saveeqn}\setcounter{equation}{0}%
  \renewcommand{\theequation}{%
  \mbox{\arabic{saveeqn}\alph{equation}}}}%
\newcommand{\reseteqn}{\setcounter{equation}{\value{saveeqn}}%
\renewcommand{\theequation}{\arabic{equation}}}



\def\souligne#1{$\underline{\smash{ \hbox{#1}}}$}

\def\x {{\bf x}}
\def\r {{\bf r}}
\def\n {{\bf n}}
\def\d {{\bf d}}

\def\C {{\bf C}}
\def\U {{\bf U}}
\def\D {{\bf D}}
\def\T {{\bf T}}
\def\O {{\bf O}}


\def\bcalH {\mbox{\boldmath $\mathcal H$}}

\def\brho {\mbox{\boldmath $\rho$}}

\def\bphi {\mbox{\boldmath $\phi$}}

\journal{Physics Letters A}
\begin{document}
\begin{frontmatter}
\title{Many-body effects in the out-of-equilibrium dynamics of a composite bosonic Josephson junction}
\author[1]{Sudip Kumar Haldar\corref{cor1}}%\fnref{label2}}
\ead{sudip.k@srmuniversity.ac.in}
%% \ead[url]{home page}
%%\fntext[label2]{}
 \cortext[cor1]{Corresponding Author}
\author[2,3]{Anal Bhowmik}
\author[2,3]{Ofir E. Alon}
\affiliation[1]{Organiszation={Department of Physics, SRM University Delhi-NCR}, 
addressline={Rajiv Gandhi Education city}, 
city={Sonepat}, 
postcode={131029},
country={India}}

\affiliation[2]{Organization={Department of Physics, University of Haifa}, 
city={Haifa}, 
postcode={3498838}, 
country={Israel}}


\affiliation[3]{Organization={Haifa Research Center for Theoretical Physics and Astrophysics, University of Haifa}, 
city={Haifa},
postcode={3498838}, 
country={Israel}}


\begin{abstract}
The out-of-equilibrium many-body quantum dynamics of an interacting Bose gas trapped in a one-dimensional composite double-well potential is studied by solving the many-body Schr\"odinger equation numerically accurately by employing  the multiconfigurational time-dependent Hartree for bosons (MCTDHB) method. The composite double-well is formed by merging two deformed harmonic wells having a hump at their centre. We characterised the dynamics by the time evolution of survival probability, fragmentation, and many-particle position and momentum variances. Our study demonstrates the prominent role played by the higher orbitals in the dynamics and thereby highlighted the necessity of a many-body technique like MCTDHB which can take into account all the relevant orbitals for the accurate description of complex many-body dynamics. Further, we showed that the universality of fragmentation with respect to the number of particles corresponding to a particular interaction strength is also exhibited by the higher-order orbitals. Therefore, it is a robust phenomenon not limited to systems that can be described by two orbitals only.
\end{abstract}

\end{frontmatter}

\section{Introduction}
Ultra-cold atomic systems and quantum gases have emerged as popular tools for the quantum simulation of condensed-matter problems due to their exceptional controllability  over the system parameters.  Various trapping geometries \cite{Petrov2004, Henderson2009},   created utilizing the advanced technique \cite{Ye1999, Wieman1999, Pinkse2000, Bhowmik2018}, and controlling the underlying inter-particle interactions via  Feshbach resonances \cite{Chin2010}  unveil numerous fundamental and intriguing physical phenomena which were elusive until recently. 

Trapping of ultra-cold bosons in a symmetric double-well potential is popularly known as the bosonic Josephson junction (BJJ) \cite{Sakmann2009, Haldar2018, Bhowmik2020},  which represents a prototype of the Josephson effect initially predicted for tunnelling of  Cooper pair between two weakly linked superconductors \cite{Josephson1962}. Ultra-cold many-body bosons in BJJ provide insight into the tunnelling of correlated quantum dynamics and naturally attract immense interest from the physics community. Examples are  fragmentations \cite{Sakmann2014,Theel2020,Vargas2021},   Josephson oscillations \cite{Levy2007}, macroscopic self-trapping \cite{Albiez2005}, collapse and revival sequences \cite{Milburn1997}, matter wave interferometry \cite{Schumm2005},  and atomic squeezing state \cite{Orzel2001, Wu2022}, etc..

 The BJJ has been extensively studied both theoretically and experimentally \cite{Smerzi1997, Albiez2005, Gati2007, LeBlanc2011, Gillet2014, Burchinati2017, Hou2018}. Theoretically, two commonly used methods for the study of BJJ dynamics are the mean-field method  and the Bose-Hubbard model. As previously mentioned several features of BJJ dynamics, such as the collapse and revival of the density oscillations  and the fragmentation are beyond the scope of the mean-field level of theory. While the two-mode Bose-Hubbard model can explain those features accurately as long as the dynamics of the BJJ involve only the lowest energy band. Therefore, in order to accurately study the BJJ dynamics when the higher energy bands participate, one requires to solve the full many-body Schrodinger equation. In this context, our group has developed a numerically exact many-body method, called the Multiconfigurational time-dependent Hartree method for bosons (MCTDHB),  which takes into account all the participating energy bands, unlike the two-mode Bose-Hubbard model where only the lowest band participates.  

In recent times, MCTDHB has been thoroughly used to study the BJJ dynamics in different situations, viz., in 1D as well as in 2D; with repulsive and attractive interactions \cite{Sakmann2010}; with contact and finite range interactions \cite{Tunneling_Rapha, Haldar2018, Bhowmik2022arxiv}; and in presence of an asymmetry in the trap \cite{Haldar2019, Bhowmik2022}, etc. Fragmentation and the uncertainty product of the many-particle position and momentum operators have also been studied \cite{Klaiman2016} which reasserted the importance of using a full many-body model such as MCDTHB. The many-body uncertainty product of the BEC in a double well was shown to grow with time $t$ as $t^2$ (up to leading order in $t$). It has been further shown that, contrary to the BH dimer, the full many-body dynamics of BJJ for the repulsive and attractive interactions are not equivalent \cite{Sakmann2010}.  A universality of the degree of fragmentation with respect to different particle numbers $N$ (i.e., fragmentation of the systems to the same value of natural occupations), keeping the interaction parameter $\Lambda =\lambda_0 (N-1)$ fixed ($\lambda_0$ being the strength of interaction) has been predicted in the dynamics of interacting bosons in a symmetric as well as an asymmetric double well for the sufficiently strong interaction. Moreover, resonantly enhanced tunnelling which was earlier experimentally observed  has also been demonstrated for an asymmetric BJJ with MCTDHB \cite{Haldar2019}.  Moreover, on a different note, the Josephson effects have been investigated  in various complex systems,  such as, two-component BEC \cite{Mondal2022}, spinor condensates \cite{Zibold2010}, polariton condensates \cite{Abbarchi2013}, fermionic superfluid \cite{Valtolina2015}, and spin-orbit coupled BEC \cite{Hou2018}.

Although  various  intriguing many-body effects have already been observed in one-dimensional ultra-cold bosonic ensembles, in all those studies  only the lowest band had a significant role in the investigated parameter regime.  In order to explore the role of higher energy bands in the many-body dynamics in one spatial dimension, in this work,  we propose to study the many-body dynamics  of an interacting bosonic system in a composite double well. A composite double well is formed by merging two wells and each of which has a small hump at its centre. To characterise the dynamics, we studied the time evolution of  several dynamical quantities such as survival probability, fragmentation, many-particle position and momentum variances, out of which fragmentation and many-particle position and momentum variances are purely many-body quantities. In this work, we showed that an accurate description of the complex many-body dynamics requires a many-body method that takes into account all relevant orbitals. Additionally, we showed that the universality of fragmentation is a global many-body phenomenon.

We organise the paper as follows. Section~\ref{method} introduces the system studied here and gives an outline of the methodology used. We discuss our findings in section~\ref{result}. We summarise our main findings and draw our conclusions in section~\ref{conclusion}.  Further details of the methodology as well as the discussion about the numerical convergence of our results are discussed in Appendix.

\section{Theoretical framework}~\label{method}
Here we are interested in the dynamics of a system of $N$ interacting structureless bosons in a one-dimensional (1D) potential given by 
\begin{equation}
	V_{Trap}(x)=V_T(x)+V_0 \exp[-a(x+2)^2]+V_0 \exp[-a(x-2)^2]
\end{equation}
where
\begin{equation}\label{cdw}
	V_T(x) = \left\{
	\begin{matrix}
		\frac{1}{2}(x + 2)^2 , \hspace*{1cm} x < -\frac{1}{2} \cr 
		%\vspace*{0.3cm}
		\frac{3}{2}(1-x^2) , \hspace*{1cm} |x| \le \frac{1}{2} \cr 
		%\vspace*{0.3cm}
		\frac{1}{2}(x - 2)^2 , \hspace*{1cm} x > \frac{1}{2} \cr  
	\end{matrix}
	\right.\,.
\end{equation} 
We prepare the initial state of the system in the ground state of 
\begin{equation}\label{eq-ini}
	V_{Trap}(x)=\frac{1}{2}(x + 2)^2 +V_0 \exp[-a(x+2)^2]
\end{equation}
and make it propagate in $V_{Trap}(x)$. The dynamics of this many-body system are governed by the time-dependent many-body Schr\"odinger equation:

\begin{gather}\label{MBSE}
\hat H \Psi = i \frac{\partial \Psi}{\partial t},\\ \nonumber
\hat H(x_1,x_2,\ldots,x_N) =  \sum_{j=1}^{N} \hat h(x_j) +  \sum_{k>j=1}^N W(x_j-x_k).    
\end{gather}

Here $x_j$ {is} the coordinate of the $j$-th boson, $\hat h(x) = \hat T(x) + V_{Trap}(x)$ is the one-body Hamiltonian 
containing kinetic energy $T(x)$ and a trapping potential $V_{Trap}(x)$ terms, {and}
$W(x_j-x_k)$ is the pairwise interaction between the $j$-th and $k$-th bosons. Dimensionless units are employed throughout this work. We used a dimensionless unit in this work.

For our present system,  the time-dependent many-boson Schr\"odinger equation (\ref{MBSE}) cannot be solved exactly (analytically). Thus, we have deployed an in-principle numerically exact many-body method, MCTDHB,~\cite{Streltsov2007, Ofir2008} which has been developed by our group and {bench-marked with an exactly-solvable model~\cite{Lode2012, Axel_MCTDHF_HIM}. This method has already} been extensively used in the literature 
\cite{Sakmann2009,MCTDHB_OCT,MCTDHB_Shapiro,Tunneling_Rapha,Axel2016,Axel2017,NJP2018Axel,NJP2019Axel,Cosme2017,NJP2015Schmelcher,NJP2017Camille,NJP2017Schmelcher, Haldar2018, Bhowmik2018, Haldar2019, ofir_review2020, Bhowmik2020, Bhowmik2022arxiv, Dutta2023}. %, NJP2013Schmelcher, NJP2015Schmelcher}.
A detailed derivation of the MCTDHB equation of motions can be found in~\cite{Ofir2008, ofir_review2020}. However, for completeness, we provide a brief description of the method in the Appendix.

%In this work, we characterized the many-body dynamics of our system in terms of the time evolution of a few dynamical quantities such as the survival probability, depletion and fragmentation, and the many-particle position and momentum variances. %The definitions of these quantities in terms of the reduced one-body and the two-body density matrices and their connection with related experimentally relevant quantities are discussed in the Appendix. 
 
\section{Result}~\label{result}
In this section, we present our findings of the study on the many-body dynamics of an interacting bosonic gas in a composite double well as defined in Eq.~(\ref{cdw}). Our main purpose of this study is to explore the role of higher energy bands in the many-body dynamics. To understand the characteristics that are exclusively due to the participation of the higher-order energy bands, we will compare our results with the many-body dynamics in a symmetric double well in which only the lowest energy band has a significant role. 

\subsection{Many-body dynamics}
 As mentioned earlier, we prepared the initial state in the ground state of the left well [Eq.~\ref{eq-ini}] of the composite double well. Owing to the different values of $V_0$,  the ground states have acquired complicated shapes.  Here it is worth mentioning that due to the symmetry of the trapping potential with respect to reflection, the many-body dynamics are independent of the choice of whether the ground states are prepared in  the right-hand side well or the left-hand side well. In this work, we considered a system of $N=100$ bosons interacting via a contact $\delta$-potential of interaction parameter $\Lambda=0.1$. In order to understand the participation of the higher-order energy bands in the dynamics, we varied the trap geometry by tuning $V_0$ while keeping $a=10$ fixed and bench-marked our findings with the corresponding results of the BJJ. 
We will characterize the dynamics through the time evolution of several physical quantities such as the survival probability, depletion and fragmentation, and many-particle position and momentum variances. 

\subsubsection{Survival probability}
The survival probability $p_L(t)$, which is a measure of density oscillations between the composite double wells, can be defined as
\begin{equation}
	p_L(t)=\int_{-\infty}^0 {\rm d}x \frac{\rho(x;t)}{N},
\end{equation}
where $\rho(x;t)$ is the density in the left composite well. It is closely connected to the population imbalance which is routinely studied in the experiments.  In Fig.~\ref{fig-pl}, we plot the survival probability $p_L(t)$ in the left well for different values of $V_0$. For $V_0=0$,  we get back the standard BJJ comprising of two symmetrical wells connected through a barrier and accordingly, $p_L(t)$ also exhibits the usual collapse of density oscillations with time. As $V_0$ is increased from zero, a hump appears in either well of the BJJ and it turns into the composite BJJ. We observe that irregularity appears in the oscillations of the $p_L(t)$ for $V_0 \neq 0$. Further, the irregularities enhance with the increase in $V_0$. With further increase in $V_0$, we observe two distinct oscillations for $V_0=10$. The high-frequency oscillations indicate the tunnelling between the two adjacent shallow wells whereas the low-frequency oscillations, appearing as the envelope to the high-frequency oscillations, can be attributed to the tunnelling of the whole system between the two composite double wells.   The different rates of decay of the density oscillations depending on the values of $V_0$ come from the initial shape of the ground orbital.  

\begin{figure}[!h]
\begin{center}
\begin{subfigure}{0.5\textwidth}
\centering
\includegraphics[width=\textwidth]{surv-prob-cbjj_N_100-M_4.pdf}
\caption{}
\end{subfigure}
%\hfill
%\vglue -1.25truecm
\begin{subfigure}{0.5\textwidth}
\centering
    \includegraphics[width=\textwidth]{surv-prob-cbjj_N_100-M_4-short.pdf}
    \caption{}
\end{subfigure}
\vglue -0.5truecm
%\hspace*{1cm}(a)	\hspace*{8.5cm}(b)\\
\end{center} 
%	\vglue -0.50truecm
\caption{Plot of the $p_L(t)$ as a function of time $t$ for various barrier heights $V_0$ [panel (a)]. The magnified view of the first few oscillations of $p_L(t)$ is shown in panel (b). Colour codes are explained in panel (a). These results are obtained with $M=4$ orbitals for $N=100$ particles with interaction strength $\Lambda=0.1$. }
\label{fig-pl}
\end{figure}

\subsubsection{Fragmentation}
Fragmentation of BEC provides information about the relative macroscopic populations in higher-order energy levels. The connection between the damping of  density oscillations and the
fragmentation for a BJJ is already well demonstrated in Ref.~\cite{Sakmann2009}.  Naturally, our findings for the $p_L(t)$ mentioned above demand a study of the fragmentation as a function of time for various heights of the hump $V_0$. This would help us to correlate the changes in density oscillations with the participation of higher-order energy bands. 

 The development of fragmentation in the system is characterized by the time evolution of the natural occupations per particle $\frac{n_i}{N}$.  The system is said to be condensed if $\frac{n_1}{N} \sim \mathcal{O}(1)$. %In that case, the sum over all the microscopic fractions of occupations in the higher orbitals $f = \sum_{j=2}^{N}\frac{n_j}{N}$ gives the depletion of BEC. 
On the other hand, the system is fragmented if $\frac{n_i}{N} \sim \mathcal{O}(1)$ for more than one orbital.

\begin{figure}[!h]
\begin{center}		
\includegraphics[width=0.5\textwidth]{NO-cbjj_N_100-M_4.pdf}
\end{center} 
\vglue -1.0truecm
\caption{Plot of $\frac{n_i}{N}$ $(i=1,2,3, \mathrm{and}  4)$ for BJJ $(V_0=0)$ and the composite BJJ with various barrier heights $V_0$.These results are obtained with $M=4$ orbitals for a system of $N=100$ particles with interaction strength $\Lambda=0.1$.}%Plot of the  $p_L(t)$ for  }
\label{fig-no}
\end{figure}

We present our results in Fig.~\ref{fig-no}. It is found that at $t=0$, for all the values of $V_0$,  only the ground orbital is occupied. It signifies that although the shape of the ground state is modified due to different values of $V_0$, the coherency of it is unchanged. At $t>0$, we find that the system gradually becomes fragmented for all $V_0$. While for BJJ ($V_0=0$) the system becomes two-fold fragmented, the degree of fragmentation increases for $V_0 \neq 0$.  For BJJ ($V_0 = 0$), after a sufficiently long time, the system becomes two-fold fragmented with $\frac{n_1}{N} \approx 60\%$ and $\frac{n_2}{N} \approx 40\%$ while the occupations in all higher orbitals are negligibly small. Therefore, a BJJ can be accurately described with only $M=2$ orbitals. However, the occupations in the third and fourth orbitals start increasing with an increase in $V_0$ at the cost of occupations in the first and second orbitals. While for  $V_0=2.5$, we see $\frac{n_3}{N} \approx 5\%$, it grows to over $\frac{n_3}{N} \approx 20\%$ for $V_0=10$. Similarly, $\frac{n_4}{N}$ which is very small for $V_0=2.5$ grows to $\frac{n_4}{N} \approx 5\%$ for $V_0=10$. Accordingly, the occupations in the first orbital reduces from $\frac{n_1}{N}\sim 60\%$ for $V_0=0$ to $\frac{n_1}{N}\sim 50\%$ for $V_0=2.5$, $\frac{n_1}{N}\sim 45\%$ for $V_0=5$, $\frac{n_1}{N}\sim 37\%$ for $V_0=7.5$. With further increase in $V_0$ to $V_0=10$, there is no significant variation in $\frac{n_1}{N}$. Similarly, $\frac{n_2}{N}$ first increases from $\frac{n_2}{N} \sim 40\%$ for $V_0=0$ to $\frac{n_2}{N} \sim 50\% $ for $V_0=2.5$, and then gradually decreases to $\frac{n_2}{N} \sim 45\% $ for $V_0=5$, $\frac{n_2}{N} \sim 35\% $ for $V_0=7.5$, and finally to $\frac{n_2}{N} \sim 32\% $ for $V_0=10$.  So initially, primarily the second orbital gains population at the cost of the first orbital and both the orbital achieve similar occupations for $V_0=2.5$. With further increase in  $V_0$, particles are transferred to the third and fourth orbitals from the first two orbitals. So, for $V_0=5$ the occupations in the first and second orbitals become closer and at the same time occupations in the third and fourth orbitals also increase. While there is a further reduction in occupations in both of the first two orbitals, more particles transfer from the second orbital resulting in reduced occupations in the second orbital. Finally, as $V_0$ increases to $V_0=10$ from $V_0=7.5$, there is further redistribution of particles among the second, third and fourth orbitals so that  $\frac{n_1}{N}$ remains practically the same.  Therefore it is evident that there is a redistribution of the particles among the orbitals and finally, for significantly large $V_0$ there is a more equitable distribution of the population. This  clearly indicates the greater role played by the higher order orbitals in the tunnelling dynamics of the CBJJ ($V_0 \neq 0$).  

In this connection, it is instructive to note that the density oscillations die out in BJJ as the system becomes correlated. However, the damping slows down for the CBJJ even though the CBJJ is more fragmented than BJJ. This may be attributed to the delay in the growth of fragmentation in CBJJ for smaller values of $V_0$. But for $V_0 \ge 7.5$, the damping is slower even though the correlation in the system grows faster than BJJ. This shows the complicated relationship between the damping of density oscillations and the growth of correlation in the system in the case of CBJJ.   

\subsubsection{Many-particle position and momentum variances}
Next, we consider the many-particle position and momentum variances of the system which is a measure of the quantum resolution of measurement of any observable. Although these cannot be measured easily, these are fundamental quantities due to their connection with the uncertainty principle. Contrary to fragmentation, which reflects only relative occupations, many-body variances are dependent on the actual number of fragmented atoms. Therefore, these are expected to bear more prominent signatures of the actual occupations in higher orbitals and thereby may throw more light on the role of higher-order orbitals in the dynamics. 

One can calculate the variance per particle $\frac{1}{N}\Delta_{\hat A}^2(t)$ for any many-body operator $\hat A=\sum_{j=1}^N \hat a(x_j)$, which is obtained from single-particle Hermitian operator $\hat a(x_j)$, by 

%\begin{strip}
\begin{gather}\label{dis}
%\begin{split}
\frac{1}{N}\Delta_{\hat A}^2(t)   =  \frac{1}{N} 
\left[\langle\Psi(t)|\hat A^2|\Psi(t)\rangle - \langle\Psi(t)|\hat A|\Psi(t)\rangle^2\right]  \\ \nonumber
\equiv \Delta_{\hat a, density}^2(t) + \Delta_{\hat a, MB}^2(t),  \\ \nonumber
\Delta_{\hat a, density}^2(t) = 
\int d x \frac{\rho(x;t)}{N} a^2(x) - \left[\int dx \frac{\rho(x;t)}{N} a(x) \right]^2, \\   \nonumber
 \Delta_{\hat a, MB}^2(t)  = \frac{\rho_{1111}(t)}{N} \left[\int d x |\phi^{NO}_1(x;t)|^2 a(x) \right]^2\\ \nonumber
 - (N-1) \left[\int dx \frac{\rho(x;t)}{N} a(x) \right]^2\\ \nonumber
 + \sum_{jpkq\ne 1111}\frac{\rho_{jpkq}(t)}{N} \left[\int d x \phi^{\ast{NO}}_j(x;t) \phi^{NO}_k(x;t) a(x) \right] \\ \nonumber
 \times \left[\int dx \phi^{\ast{NO}}_p(x;t) \phi^{NO}_q(x;t) a(x)\right].
%\end{split}     
\end{gather} 
%\end{strip}
Accordingly in Fig.~\ref{fig-var} (a), we present our results for the many-particle position variance while  Fig.~\ref{fig-var} (b) depicts the results for the many-particle momentum variance.

For $V_0=0$, we again get back the oscillatory growth of  $\frac{1}{N}\Delta_{\hat X}^2$ with time followed by the saturation as the density oscillation collapses. For $V_0 \neq 0$, we get the overall behaviour quite similar to the BJJ  but the main difference lies in the details. For a small value of $V_0=2.5$, the growth rate of  $\frac{1}{N}\Delta_{\hat X}^2$ becomes slower which is consistent with the fragmentation dynamics. However, the fluctuations in the saturation value are very high for such a small value of $V_0$. With a further increase of $V_0$, the growth rate of $\frac{1}{N}\Delta_{\hat X}^2$ increases but still remains slower than BJJ. Also, the fluctuations reduce drastically for higher values of $V_0$.

The fluctuations in $\frac{1}{N}\Delta_{\hat X}^2$ about the saturation value are the manifestation of the breathing oscillations within the two wells of each of the compound wells. For small intra-well barrier height $V_0$, large breathing oscillations between the two wells of each compound wells lead to large fluctuations in $\frac{1}{N}\Delta_{\hat X}^2$. With increasing $V_0$, the breathing oscillations get damped leading to smaller fluctuations in $\frac{1}{N}\Delta_{\hat X}^2$ for higher values of $V_0$.
\begin{figure}[!h]
\begin{center}
\begin{subfigure}{0.5\textwidth}
\centering
    \includegraphics[width=\textwidth]{pos_var-cbjj_N_100-M_4.pdf}
    \caption{}
\end{subfigure}	
 \begin{subfigure}{0.5\textwidth}
 \centering
     \includegraphics[width=\textwidth]{mom_var-cbjj_N_100-M_4.pdf}
    \caption{}
 \end{subfigure}
	\end{center} 
	%	\vglue -1.35truecm
	\caption{Plot of the many-particle position [panel (a)] and momentum variances [panel (b)] for various barrier heights $V_0$. Here the results are obtained with $M=4$ orbitals for $N=100$ particles and interaction strength $\Lambda=0.1$.}%Plot of the  $p_L(t)$ for (a) Composite BJJ [top] and (b) BJJ [bottom] for $\Lambda=0.1$. }
	\label{fig-var}
\end{figure}

The dynamics of atoms will have a more prominent impact on the many-particle momentum variance. Naturally, we again observe fluctuations in $\frac{1}{N}\Delta^2_{\hat{P}_X}$ around a mean value for BJJ ($V_0=0$). For a small $V_0$, we observe very large fluctuations in $\frac{1}{N}\Delta^2_{\hat{P}_X}$ due to the large breathing oscillations. With further increase in $V_0$, the damping in the intra-well breathing oscillations results in the gradual reduction in fluctuations of $\frac{1}{N}\Delta^2_{\hat{P}_X}$. For such values of $V_0$, $\frac{1}{N}\Delta^2_{\hat{P}_X}$ fluctuates  around a mean value which initially increases slightly with time before stabilising to a saturation value. This saturation value also depends on $V_0$ and first increases before gradually decreasing with increasing $V_0$. 

\subsection{Universality of fragmentation}\label{univ}
Since the participation of the higher-order energy bands significantly affected the growth of fragmentation in the system, it is natural to ask about its impact on the universality of the degree of fragmentation with respect to $N$. It is a novel phenomenon predicted for the BJJ dynamics when only a single band plays the dominant role in the dynamics. Therefore, next, we examine the degree of fragmentation of the CBJJ when the second band (in addition to the lowest band) also plays a significant role in the dynamics. We present our results for different $V_0$  in Fig.~4. The panels in the left column show $\frac{n_i}{N}$ for the first two orbitals while those on the right column exhibit the same for the third and fourth orbitals. 

\begin{figure*}[!h]
\centering
\begin{subfigure}{0.45\textwidth}
    \includegraphics[width=\textwidth]{bjj-univ.pdf}
\caption{}
\end{subfigure}
\begin{subfigure}{0.45\textwidth}
    \includegraphics[width=\textwidth]{bjj-univ1.pdf}
    \caption{}
\end{subfigure}
\begin{subfigure}{0.45\textwidth}
    \includegraphics[width=\textwidth]{cbjj-univ-v0=5.pdf}
    \caption{}
\end{subfigure}		
	\begin{subfigure}{0.45\textwidth}
 \includegraphics[width=\textwidth]{cbjj-univ1-v0=5.pdf}
	    \caption{}
	\end{subfigure}
\begin{subfigure}{0.45\textwidth}
    \includegraphics[width=\textwidth]{cbjj-univ-v0_7p5.pdf}
\caption{}
\end{subfigure}	
\begin{subfigure}{0.45\textwidth}
    \includegraphics[width=\textwidth]{cbjj-univ1-v0_7p5.pdf}
    \caption{}
\end{subfigure}
		
\caption{Universality of fragmentation in a symmetric double well (i.e. $V_0=0$) [panel (a) \& panel (b)] and composite double well with $V_0=5$ [panel (c) \& panel (d)], and $V_0=7.5$ [panel (e) \& panel (f)]. In each row, the left panels show the occupations for the first two orbitals while the right panels show the occupation for the third and fourth orbitals. Since the occupations in higher orbitals are small, these are shown in the log scale. Colour codes are explained in panel (b). Results shown here are obtained with $M=4$ for interaction strength $\Lambda = 0.1 $.}
\label{fig-uni}
\end{figure*}

For the symmetric double well ($V_0=0$), we get back the usual two-fold fragmentation with the $\frac{n_1}{N} \sim 60\%$ and $\frac{n_2}{N} \sim 40\%$ for all $N$ keeping $\Lambda$ fixed at $0.1$. Although there are no macroscopic occupations in the higher orbitals for BJJ, we still observe the depletion  of  the same order in the third ($\frac{n_3}{N} \sim 10^{-3}$) and fourth ($\frac{n_4}{N} \sim 10^{-4}$) orbitals for all $N$.

For a finite $V_0$, the system is now four-fold fragmented. However, we still find that over time $\frac{n_i}{N}$ for all four orbitals reaches a fixed value for various $N$  corresponding to the same $\Lambda$  for all $V_0$ studied here.  However, as discussed above, the final values of $\frac{n_i}{N}$ change with an increase in $V_0$ and  the particles of  the system become more evenly distributed among the four orbitals with reduced occupations in higher orbitals and increased occupations in lower orbitals. Therefore, even though the details of fragmentation vary for various trap geometry, the systems exhibit the universality of fragmentation for all four orbitals with respect to $N$ corresponding to a fixed $\Lambda=0.1$ irrespective of the value of $V_0$.  However, we have noticed that the pace of appearance of the universality of fragmentation depends on $V_0$. As we switch on the internal barrier $V_0$, for a small value of $V_0$ it takes longer to observe the universality of fragmentation than the symmetric double well $(V_0=0)$. Then, as $V_0$ is increased, the universality appears faster. Furthermore, for a fixed value of $V_0$, the system fragments later for larger $N$. This is expected as the many-body effects are reduced with increasing $N$ keeping $\Lambda$ fixed.

In our earlier study~\cite{Haldar2019}, we demonstrated that the universality of fragmentation is preserved in asymmetric BJJ. So, our present study in combination with the earlier findings indicates that the universality of fragmentation with respect to $N$ is neither limited to systems where only two orbitals play a dominant role nor is susceptible to the geometry of the double well potential. Thus it is indeed a robust many-body effect.

\section{Summary and conclusion}~\label{conclusion}
In this work, we reviewed the many-body tunnelling dynamics of an interacting Bose gas in a composite double well. The composite double well is formed by merging two deformed harmonic wells which have a hump at their centre. We examined the dynamics through the time evolution of survival probability, fragmentation, and many-particle variances. While survival probability may be studied at the mean-field level, fragmentation and many-particle variances are of purely many-body nature.

We found irregularity in the density oscillations for very small hump and then, two distinct oscillations for bigger hump. While the fast high-frequency oscillation is due to the tunnelling through the hump within the same well, the low-frequency tunnelling is due to the tunnelling between the two wells. Evidently, we need more than two orbitals to accurately describe such complicated many-body tunnelling. We observed (see appendix) that at least four orbitals are required for the accurate description of the many-body dynamics of the system contrary to the standard BJJ dynamics where two orbitals are sufficient. 

The correlation between the damping of the density oscillation and the growth of fragmentation is already well-established. We observed that the same correlations still exist even for composite BJJ. However, the degree of fragmentation as well as the growth of the fragmentation significantly differ in the two cases. While the BJJ becomes two-fold fragmented, CBJJ is four-fold fragmented. On the other hand, the growth of fragmentation becomes slower in CBJJ except for very high hump $V_0$.  Accordingly, damping of density oscillations becomes slower in CBJJ. Furthermore, the occupations of different orbitals $\frac{n_i}{N}$ clearly show the more prominent participation of the third and fourth orbitals in the case of CBJJ. All these observations are further strengthened by the study of the many-particle position and momentum variances which exhibit higher fluctuations but slower growth.

Another interesting finding of this study is that the universality of the fragmentation with respect to the occupation in an orbital corresponding to a fixed interaction strength $\Lambda$ is not limited to systems where only two orbitals play dominant roles. Furthermore, not only universality is observed for orbitals with macroscopic occupations but even orbitals with microscopic occupations also have the same order of occupations with respect to different $N$ but with a fixed $\Lambda$. And these features are observed for all $V_0$ which reaffirms that the universality of fragmentation is a global many-body phenomenon. 

Thus we see that depending on the systems, the role of a different number of orbitals in the dynamics becomes dominant. Therefore, it is absolutely necessary to have a many-body method capable of taking into account all relevant orbitals for an accurate description of the complex many-body dynamics. The MCTDHB method is an important and promising many-body technique in this direction. The present work can inspire us to investigate the tunnelling dynamics of even more intriguing ground states of bosonic and fermionic ensembles.  
%\end{document}

\vspace*{0.5cm}
\appendix%{Numerical computations and their convergence}
%\begin{Large}
%	\textbf{Appendix: }
%\end{Large}
\section{The multi-configurational time-dependent Hartree method for bosons (MCTDHB)}
\label{MCTDHB}
%Hence, {to solve Eq.~(\ref{MBSE}) in-principle numerically exactly,  
Below we briefly describe the basic idea behind the method. In MCTDHB, the ansatz for solving Eq.~(\ref{MBSE}) {is obtained by the superposition of all possible
$\begin{pmatrix} 
 N+M-1\\
 N
\end{pmatrix}$
configurations, obtained by distributing $N$ bosons in $M$ time-dependent single{-}particle states $\phi_k(x,t)$, which we call orbitals, i.e,}
\begin{equation}\label{MCTDHB_Psi}
 \left|\Psi(t)\right> = 
\sum_{\vec{n}}C_{\vec{n}}(t)\left|\vec{n};t\right>, 
\end{equation}
where the occupations $\vec{n}=(n_1,n_2,\cdots,n_M)$ preserve the total number of bosons $N$. {For an exact theory, $M$ should be infinitely large. However, for numerical computations, one has to truncate the series at a finite $M$. In actual calculations, we keep on increasing $M$ until we reach the convergence with respect to $M$ and thereby we obtain a numerically-exact result. In the context of bosons in a double well, the latter implies that the MCTDHB theory effectively takes all required bands into account. Here we would like to point out that for $M=1$, the ansatz Eq.~(\ref{MCTDHB_Psi}) gives back the ansatz for the Gross Pitaevskii theory~\cite{rev1}.} {Therefore, solving for the time-dependent wavefunction $\Psi(t)$ boils down to  the determination of the time-dependent coefficients $\{C_{\vec{n}}(t)\}$ and the time-dependent orbitals $\{\phi_k(x,t)\}$.} Employing the usual Lagrangian formulation of the time-dependent variational principle \cite{LF1,LF2} subject to the orthonormality between the orbitals, {the working equations of the MCTDHB are obtained as follows} 
    

\begin{eqnarray}\label{MCTDHB1_equ}
&&i\left|\dot\phi_j\right\rangle  =  \hat {\mathbf P} \left[\hat h \left|\phi_j\right\rangle  + \sum^M_{k,s,q,l=1} 
  \left\{\brho(t)\right\}^{-1}_{jk} \rho_{ksql} \hat{W}_{sl} \left|\phi_q\right\rangle \right]; \nonumber\\
%& &  
%\qquad 
&&\hat {\mathbf P} = 1-\sum_{j^{\prime}=1}^{M}\left|\phi_{j^{\prime}}\left>\right<\phi_{j^{\prime}}\right| \nonumber\\
 &&{\mathbf H}(t)\C(t) = i\frac{\partial \C(t)}{\partial t}.
 \end{eqnarray}
Here, $\brho(t)$ is the reduced one-body density matrix [see Eq.~(\ref{1RDM}) below], $\rho_{ksql}$ are the elements of the two-body reduced density matrix [see Eq.~(\ref{2RDM}) below], and ${\mathbf H}(t)$ is the Hamiltonian matrix {with the elements} $H_{\vec{n}\vec{n}'}(t) = \left<\vec{n};t\left|\hat H\right|\vec{n}';t\right>$. The one-body reduced density matrix used in Eq.~(\ref{MCTDHB1_equ}) $\rho^{(1)}(x_1|x_1^{\prime};t)$ can be defined through the normalized many-body wavefunction $\Psi(t)$ as 
%Eqn(6)
\begin{multline}\label{1RDM}
 \rho^{(1)}(x_1|x_1^{\prime};t)  =    \\ 
N \int d x_2 \ldots d x_N \, \Psi^\ast(x_1^{\prime},x_2,\ldots,x_N;t)   \Psi(x_1,x_2,\ldots,x_N;t)  \\
 = \sum_{j=1}^{M} n_j(t) \, \phi^{\ast{NO}}_j(x_1^{\prime},t)\phi^{NO}_j(x_1,t).   
\end{multline}

Here, the time-dependent eigenfunctions $\phi^{NO}_j(x_1,t)$ of $\rho^{(1)}(x_1|x_1^{\prime};t)$ are the natural orbitals and $n_(t)$ the time-dependent eigenvalues of $\rho^{(1)}(x_1|x_1^{\prime};t)$ are the natural occupation numbers.  The diagonal of the $\rho^{(1)}(x_1|x_1^{\prime};t)$ gives the density of the system $\rho(x;t) \equiv \rho^{(1)}(x|x^{\prime}=x;t) \label{1RDMdiag}$.

Similarly, the two-body density can be calculated as  
\begin{equation}\label{2RDM}
\begin{split}
\rho^{(2)}(x_{1},x_{2} \vert x_{1}^{\prime}, x_{2}^{\prime};t)=
\hspace*{4cm}\\
N(N-1)\int_{}^{}d x_{3} \ldots d x_{N}		
\Psi^{*}(x_{1}^{\prime},x_{2}^{\prime},x_{3},\ldots,x_{N};t) \\
\times \Psi(x_{1},x_{2},x_{3}, \ldots, x_{N};t).
\end{split}
\end{equation} 
Therefore, the matrix elements of the two-body reduced density matrix are given by 
$\rho_{ksql}=\left<\Psi\left|b_k^\dag b_s^\dag b_q b_l\right|\Psi\right>$ where $b_k$ and $b_k^\dag$
are the bosonic annihilation and creation operators, respectively.

\section{Numerical implementation of MCTDHB}
A parallel version of MCTDHB has been implemented using a novel mapping technique~\cite{Streltsov1, Streltsov2}. We mention that by propagating in imaginary time the MCTDHB equations also allow one to determine the ground and excited state of interacting many-boson systems, see~\cite{MCHB, Lode2012}. In our present work, we have performed all computations with $M=4$ time-adaptive orbitals. {By repeating our computations with $M=6,8,10$, and $12$ orbitals the results have been verified and found to be highly accurate for the quantities and propagation {times} considered here.

In our numerical {calculations}, {the many-body Hamiltonian is represented by 256 exponential discrete-variable-representation (DVR) grid points (using a Fast Fourier transformation routine) in a box size $[-10,10)$. %We also checked for convergence with respect to DVR grid points by repeating computations  with $128$ and $512$ DVR grid points (see below). 
We obtain the initial state for the time propagation, the many-body ground state of the BEC by propagating the MCTDHB equations of motion {[Eq.~(\ref{MCTDHB1_equ})]} in imaginary time~\cite{MCHB, Lode2012}. For our numerical computations, we use the numerical implementation in~\cite{Streltsov1, Streltsov2}.} 
We keep on {repeating the computation with} increasing $M$ until convergence is reached {and thereby obtain the numerically accurate results}.

\section{Numerical convergence}

As explained above, in actual calculation Eq.~(\ref{MCTDHB_Psi}) is truncated at a finite $M$ which is just sufficient for numerical convergence and thereby provides a numerically accurate description of the many-body dynamics. Here we have truncated Eq.~(\ref{MCTDHB_Psi}) at $M=4$. Therefore, now we demonstrate that $M=4$ is enough for achieving a good degree of numerical convergence of our results with respect to $M$. %Further, since in our numerical calculations, the potential is represented by exponential discrete-variable-representation (DVR) grid points, it is also required to show that the trapping potential is adequately sampled. 
%Therefore, in this section, we demonstrate the numerical convergence of our results with respect to the number of orbitals $M$. %as well as the no of grid points used to sample the potential in the DVR representation. 
Here, we will explicitly show the numerical convergence of the dynamical quantities for a system of $N=10$ only. Since in the limit $N \rightarrow \infty$, keeping $\Lambda$ fixed, the many-body effects diminish {and the density per particle of the system} converge to its corresponding mean-field values~\cite{Lieb_PRA, Lieb_PRL, Erdos_PRL, MATH_ERDOS_REF}, the convergence of the quantities improves for higher $N$ values for a fixed $\Lambda$. Therefore, the accuracy of our actual results discussed above is better than the results shown in this section.

%\subsection{Convergence with respect to orbital nos $M$}
%We start our discussion of convergence by demonstrating the convergence of our numerical results with respect to the orbital numbers $M$ for $256$ DVR grid points. %Next, we will demonstrate convergence with respect to the number of grid points. 

%\subsection{Convergence of initial state}
%We first discuss the convergence of the initial state with respect to $M$.
%As mentioned above, the initial state, i.e., the ground state of the system, is obtained by the imaginary time propagation of the MCTDHB equation Eq~(\ref{MCTDHB1_equ}). %To prove the convergence of the initial state, we explicitly show the convergence of ground state energy $E_g$, occupation numbers $\frac{n_1}{N}$ and the many-particle position variance $\frac{1}{N^2}\Delta_{\hat X}^2$ and momentum variance $\frac{1}{N^2}\Delta_{\hat P_X}^2$ of the initial states for both the BJJ and CBJJ  in Table~\ref{symm-DW}. 
As demonstrated in previous studies~\cite{Sakmann2009, Haldar2018}, $M=2$ orbitals are enough for an accurate description of the time evolution of all the dynamical quantities viz. $p_L(t)$, $\frac{n_i}{N}$, $\frac{1}{N^2}\Delta_{\hat X}^2$ and $\frac{1}{N^2}\Delta_{\hat P_X}^2$ of the BJJ even with a long-range interaction.
%For BJJ, the initial state is the ground state of the BEC in a harmonic trap while 
Since the initial state for CBJJ is the ground state of BEC in a symmetric double well (BJJ), $M=2$ should be enough for an accurate description of the initial state for the dynamics of CBJJ. However, here we computed the initial states with $M=4$ orbitals as for studying the dynamics we would need at least $M=4$ orbitals for CBJJ. Therefore, at the onset of our study, we can rest assured about the convergence of our initial state with respect to $M$.
%\begin{table*}[h!]
%\begin{center}
%\caption{Convergence of the initial state with respect to no of orbitals $M$. The Data presented here are obtained for $N=10$ bosons trapped in a BJJ and in a CBJJ with $V_0=10$} \label{symm-DW}
%\begin{tabular}{|c|c|c|c|c|c|} \hline
%				System & $M$ & $ E_g$  & $\frac{n_1}{N}$ & $\frac{1}{N}\Delta^2_{\hat{X}}$ & $\frac{1}{N}\Delta^2_{\hat{P}_X}$\\ \hline
%				BJJ &	2   &  5.1981862673641537  &  9.9998742099188416  &  0.5013327540849772 & 0.4986911823717152 \\ \cline{2-6}
%				& 4   &  5.1980426773595259  &  9.9998327041790045  &  0.5004567087181968 & 0.4995510366734439 \\ \cline{2-6}
%				&	6   & 5.1979927590962198   &  9.9998236168230576  &  0.5002211723366230 & 0.4997827371999016 \\ \cline{2-6}
%				&	8   & 5.1979677103288973   &  9.9998204110783355  &  0.5001280466306142 & 0.4998743598655270  \\ \cline{2-6}
%				&  10 & 5.1979527257359299  &   9.9998189537860753 &  0.5000825896816181 & 0.4999190641006763   \\ \cline{2-6}
%				& 12 & 5.1979427786584758  &   9.9998181807290187 &  0.5000572513297428  & 0.4999439671993041 \\ \hline			
%				CBJJ & 2    &   13.7166066766799286        & 9.9972388487872870 &  1.145629226961262 &  0.6593375473099641\\ \cline{2-6}  
%				& 4    &   13.7165079082761903        & 9.9972029921605898 &  1.144706377192293 &  0.6628162942142350\\  \cline{2-6}  
%				&  6    &   13.7164620221503757        & 9.9971950394698919 &  1.144577337254198  &  0.6628981080707090 \\ \cline{2-6}  
%				& 8    &   13.7164390220376422       & 9.9971917436650735  &  1.144510669231387 &   0.6631242241038646\\ \cline{2-6}  
%				& 10 &    13.7164243329438627      & 9.9971905820728377   &  1.144490356611733 &  0.6631422036062781\\ \cline{2-6}         
%				& 12 &    13.7164146437972487      &   9.9971899506099611  & 1.144476210483205 &  0.6631888775940531 \\  \hline	   
%					\end{tabular}
%				\end{center}
%	\end{table*}

\subsection{Convergence of dynamical quantities}
Now, we will discuss the convergence of the dynamics with respect to the $M$. We characterized the dynamics of the systems by the time evolution of the survival probability, fragmentation, and the many-particle position and momentum variances. However, it is well demonstrated that achieving convergence for the many-particle position and momentum variances are more challenging. Therefore, the demonstration of convergences for the many-particle position and momentum variances should automatically demonstrate the convergence of the survival probability as well as the fragmentation. Still, we will explicitly show the convergence of occupation numbers in addition to the convergence of the many-particle position and momentum variances as the characteristics of the time evolution of the fragmentation are directly related to the features of the many-particle position and momentum variances. Moreover, demonstration of convergence for systems with a higher degree of fragmentation would automatically imply convergence for systems with a lower degree of fragmentation as the many-body effects are diminished with the decrease in the occupations in higher orbitals. Therefore, demonstration of convergence of various quantities for $V_0 = 10$ would automatically imply the convergence for other $V_0$ considered in this work for which the system is less fragmented. We will discuss the convergence of these quantities with respect to $M$ for the time duration discussed in this work. 
%\begin{enumerate}
%	\item[1.] \textbf{Convergence of survival Probability:}
%	Here we plot the survival probability $p_L(t)$  in the left well for both the composite BJJ and standard BJJ. In the same figure convergence of numerical results with respect to the orbital number, $M$ is also shown.
%	\begin{figure}
		%\begin{center}
%		\includegraphics[width=0.50\linewidth]{conv-surv-prob.eps}
		%		(a)\\
%		\vglue -1.25truecm
%		\includegraphics[width=0.50\linewidth]{conv-surv-prob-symm-DW.eps}\\
		%	(b)\\
		%\end{center} 
%		\vglue -1.35truecm
%		\caption{Plot of the  $p_L(t)$ for (a) Composite BJJ [top] and (b) BJJ [bottom] for $\Lambda=0.1$. }
%		\label{fig-pl-conv}
%	\end{figure}
	
	%\begin{figure}
	%	\includegraphics[width=1.0\linewidth]{conv-surv-prob-short.eps}
	%	\vglue -0.75truecm
	%	\includegraphics[width=1.0\linewidth]{conv-surv-prob-symm-DW-short.eps}
	%	\vglue -1.0truecm
	%\caption{Plot of the  $p_L(t)$ for (a) Composite BJJ [top] and (b) BJJ [bottom] for $\Lambda=0.1$ for a very short time.}	
	%\end{figure}
	
	%\begin{figure}
	%	\includegraphics[width=1.0\linewidth]{conv-surv-prob-long.eps}
	%	\vglue -0.75truecm
	%	\includegraphics[width=1.0\linewidth]{conv-surv-prob-symm-DW-long.eps}
	%	\vglue -1.0truecm
	%	\caption{Plot of the  $p_L(t)$ for (a) Composite BJJ [top] and (b) BJJ [bottom] for $\Lambda=0.1$ for a longer time.}	
	%\end{figure}
\subsubsection{Convergence of fragmentation and depletion}
We start with the demonstration of convergence of the occupation numbers of different orbitals for composite BJJ. Although the demonstration of convergence for $V_0=10$ is sufficient, we still explicitly show the convergence for $V_0=7.5$ in addition to $V_0=10$ since in section~\ref{univ} we showed the universality of fragmentation for $V_0$ up to $V_0=7.5$.  
\begin{figure*}[!h]
\centering
\begin{subfigure}{0.45\textwidth}
\includegraphics[width=\textwidth]{n_1_2_V_7.5_N_10_converge.pdf}
\caption{}
\end{subfigure}		
\begin{subfigure} {0.45\textwidth}
\includegraphics[width=\textwidth]{n_3_4_V_7.5_N_10_converge.pdf}  
\caption{}
\end{subfigure}	
\begin{subfigure}{0.45\textwidth}
\includegraphics[width=\textwidth]{n_1_2_V_10_N_10_converge.pdf}
\caption{}
\end{subfigure}		
\begin{subfigure} {0.45\textwidth}
\includegraphics[width=\textwidth]{n_3_4_V_10_N_10_converge.pdf} 
\caption{}
\end{subfigure}	
%		\vglue -0.5truecm
%		\includegraphics[width=0.50\linewidth]{conv-frag-symm-DW.eps}
%		\vglue -0.5truecm	
%\hspace*{1cm}(a)  \hspace*{8cm}	(b)
\vglue -0.45truecm
\caption{Plot of the convergence of the occupations of  (a) the first and second orbitals and (b) the third and fourth orbitals with respect to $M$ for a CBJJ of barrier height $V_0=7.5$, $N=10$, and  $\Lambda=0.1$. Convergence of occupation numbers for $V_0=10$, $N=10$, and  $\Lambda=0.1$ are shown in panel (c) (first and second orbitals) and (d) (third and fourth orbitals) respectively.}
\label{fig-frag-conv}
\end{figure*}\\
First, we consider the first four orbitals which have macroscopic occupations for CBJJ (though the third and fourth orbitals for BJJ still have microscopic occupations). In Fig.~\ref{fig-frag-conv} we plot $\frac{n_i}{N}[\%]$ for the first $4$ orbitals for different $M \ge 4$ to exhibit the convergence of our results. We observe that the results for %$M=8$ and $M=10$ completely overlap with each other while the results obtained for 
$M=4$ and $M=8$ essentially overlap with each other for both $V_0 = 7.5$ and $V_0 = 10$. Since the many-body results smoothly approach the mean-field limit as $N$ increases,  
%lies very close to that of $M=8$. %and $M=10$. This small deviation for $N=10$ implies that 
our results for $N=100$ accurately demonstrate the development of fragmentation in the course of the out-of-equilibrium many-body dynamics. %and also compare composite BJJ and BJJ.
Further, in Fig.~\ref{fig-depl}, we explicitly show that the occupations for the higher orbitals for $V_0 = 10$ are microscopically small.

\begin{figure}[!h]
\begin{center}		
\includegraphics[width=0.5\textwidth]{n_5_to_8_V_10.pdf}
\end{center} 
\vglue -1.0truecm
\caption{Plot of occupations in higher $M > 4$ orbitals viz. $\frac{n_i}{N}$ $(i=5,6,7, \mathrm{and} 8)$ for the composite BJJ with $V_0 = 10$.}%Plot of the  $p_L(t)$ for  }
\label{fig-depl}
\end{figure}
%Further, we highlight the differences between the occupations in the $3$rd and $4$th orbital for BJJ and composite BJJ along with the convergence with respect to $M$ in Fig.~\ref{fig-n3-n4}.
%\begin{figure}[!h]
%	\begin{center}
%		\includegraphics[width=0.95\linewidth]{n_3_4_V_10_N_10_converge.eps}
%		\vglue -0.75truecm
%		(a)\\			
%		\includegraphics[width=0.50\linewidth]{conv-n3-n4-symm-DW.eps}
%		\vglue -0.75truecm
%		(b)\\
%	\end{center} 
%	\vglue -0.75truecm
%	\caption{Plot of the convergence of occupations of  the third and fourth orbitals with respect to $M$ (a) in composite BJJ and (b) in BJJ for $\Lambda=0.1$}
%	\label{fig-n3-n4}
%\end{figure}
%Similarly, in Fig.~\ref{fig-dep} we plot the depletion $\frac{n_i}{N}$ for the higher orbitals for different $M$ for both BJJ and composite BJJ. We also plotted results for several $M$ to exhibit convergence.\\

%\begin{figure}[!h]
%	\begin{center}
%		\includegraphics[width=0.5\linewidth]{conv-n5-n6.eps}%\\
%		\includegraphics[width=0.5\linewidth]{conv-n5-n6-symm-DW.eps}
%		\vglue -0.5truecm		
		%		(a)    \\
%		\vglue -0.25truecm		
%		\includegraphics[width=0.49\linewidth]{conv-n7-n8.eps}
%		\includegraphics[width=0.49\linewidth]{conv-n7-n8-symm-DW.eps}
%		\vglue -0.5truecm		
		%		(b)
%	\end{center}
%	\vglue -0.75truecm
%	\caption{Plot of the convergence of occupation no with respect to $M$ for fifth and sixth orbitals and (b) higher orbitals for a composite BJJ (left panels ) and symmetric double well (right well) with $\Lambda=0.1$.}
	%\label{fig-dep}
%\end{figure}

\subsubsection{Convergence of many-particle position and momentum variances}
Next, we present the convergence results for $\frac{1}{N^2}\Delta_{\hat X}^2$ and $\frac{1}{N}\Delta^2_{\hat{P}_X}$ [Fig~\ref{fig-var-conv}]. For both $M=4$ and $M=8$ orbitals, we observe the same qualitative feature that $\frac{1}{N^2}\Delta_{\hat X}^2$ increases with the growth of fragmentation and then exhibits an irregular oscillation [Fig~\ref{fig-var-conv}(a)]. In conformity with our findings of the fragmentation, here also we find a good degree of overlap between the fluctuations of $\frac{1}{N^2}\Delta_{\hat X}^2$ for $M=4$ and $M=8$. %the complete overlap between $\frac{1}{N^2}\Delta_{\hat X}^2$ obtained with $M=8$ and $M=10$ orbitals while the result for $M=4$ slightly deviates. From Fig~\ref{fig-var}(b), 
Similarly, we notice a very high degree of overlapping between $\frac{1}{N}\Delta^2_{\hat{P}_X}$ for $M=4$ and $M=8$ [Fig~\ref{fig-var-conv}(b)]. However, the deviation between the $\frac{1}{N}\Delta^2_{\hat{P}_X}$ for $M=4$ and $M=8$ is slightly higher than $\frac{1}{N^2}\Delta_{\hat X}^2$. This is expected as achieving convergence for $\frac{1}{N}\Delta^2_{\hat{P}_X}$ is more difficult. 

As the convergence significantly improves with increasing $N$, we can safely assume that the main findings for the system size $N=100$ accurately demonstrate the many-body features of the out-of-equilibrium dynamics of a fragmented system.  

\begin{figure}[!h]
\centering
 \begin{subfigure}{0.5\textwidth}    
 \includegraphics[width=\textwidth]{var_x_V0_10_N_10_converge.pdf}
 \caption{}
 \end{subfigure}
 \begin{subfigure}{0.5\textwidth}		
\includegraphics[width=\textwidth]{var_px_V0_10_N_10_converge.pdf}
\caption{}
\end{subfigure}
%		\vglue -0.75truecm
%	\hspace*{1cm}	(a) \hspace*{8cm} (b)\\
%		\includegraphics[width=0.50\linewidth]{conv-var-symm-DW.eps}%\\
%		\vglue -0.75truecm
%		(b)
%\vglue -0.5truecm
\caption{Plot of the convergence of (a) many-particle position variance and (b) many-particle momentum variance with respect to $M$ for a CBJJ of barrier height $V_0=10.0$, $N=10$, and  $\Lambda=0.1$.}
\label{fig-var-conv}
\end{figure}	

%\begin{figure}[!h]
%\begin{center}
%\includegraphics[width=0.9\linewidth]{conv-var-short.eps}\\
%	(a)\\
%\includegraphics[width=0.9\linewidth]{conv-var-short-symm-DW.eps}\\
%	(b)
%\end{center} 
%	\vglue -0.75truecm
%	\caption{Plot of the many-particle position variance for (a) Composite BJJ  and (b) BJJ  for $\Lambda=0.1$. for short time.}
%	\label{fig-var-short}
%\end{figure}	

%\begin{figure}
%	\begin{center}
%	\includegraphics[width=0.950\linewidth]{var_px_V_10_N_10_converge.eps}
%		\includegraphics[width=0.50\linewidth]{}%\\
%		\vglue -0.75truecm
%		(a)\\
%		\includegraphics[width=0.50\linewidth]{conv-momvar-symm-DW.eps}%\\
%		\vglue -0.75truecm
%		(b)\\
%	\end{center} 
%	\vglue -0.75truecm

%	\caption{Plot of the many-particle momentum variance for (a) Composite BJJ  and (b) BJJ for $\Lambda=0.1$.}
%	\label{fig-momvar}
%\end{figure}

%\begin{figure}[!h]
%	\begin{center}
	%	\includegraphics[width=1.0\linewidth]{conv-momvar-short.eps}\\
	%	\vglue -0.75truecm
	%		(a)\\
	%	\includegraphics[width=1.0\linewidth]{conv-momvar-short-symm-DW.eps}\\
	%	\vglue -0.75truecm
	%		(b)\\
	%	\end{center} 
%	\vglue -0.75truecm
%	\caption{Plot of the many-particle momentum variance for (a) Composite BJJ and (b) BJJ for $\Lambda=0.1$ for short time.}
%	\label{fig-momvar-short}
%\end{figure}

\section*{Acknowledgements}

SKH acknowledges the support from the Department of Science and Technology (DST), India, through TARE Grant No.: TAR/2021/000136. Work performed in Haifa is supported by the Israel Science Foundation (grant number 1516/19). Computation time on the High-Performance Computing system Hive of the Faculty of Natural Sciences at the University of Haifa and at the High-Performance Computing Center Stuttgart (HLRS) is gratefully acknowledged.	


%\printbibliography
\bibliography{ref}
\end{document}
 