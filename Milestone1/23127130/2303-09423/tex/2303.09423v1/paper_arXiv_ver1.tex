\documentclass[aps,pra,reprint,nofootinbib,floatfix,superscriptaddress,fleqn]{revtex4-2}
\usepackage{mathtools,amssymb,amsthm,bm,bbm,xcolor,mathdots,stmaryrd}
\usepackage[colorlinks=true, linkcolor=blue, citecolor=magenta, urlcolor=blue]{hyperref}
\pdfoutput=1

\usepackage[T1]{fontenc}
\usepackage{lmodern}
\usepackage[utf8]{inputenc}

\usepackage{microtype}
\usepackage{orcidlink}

\usepackage[caption=false]{subfig}
\usepackage{graphicx}

\theoremstyle{remark}
\newtheorem{rmk}{Remark}
\newtheorem{ex}{Example}

\newcommand{\HH}{\mathcal{H}}
\newcommand{\PP}{\mathcal{P}}
\renewcommand{\AA}{\mathcal{A}}
\newcommand{\JJ}{\mathcal{J}}
\renewcommand{\SS}{\mathcal{S}}
\newcommand{\tr}{\operatorname{tr}}
\renewcommand{\d}{\mathrm{d}}
\newcommand{\bra}[1]{\langle #1|}
\newcommand{\ket}[1]{|#1\rangle}
\newcommand{\braket}[2]{\langle #1 | #2 \rangle}
\newcommand{\ketbra}[2]{|#1 \rangle\langle #2|}
\newcommand{\dt}{\mathrm{d}t}
\newcommand{\1}{\mathbbm{1}}
\newcommand{\tauMT}{\tau_\textsc{mt}}
\newcommand{\tauclMT}{\bar\tau_\textsc{mt}}
\newcommand{\tauML}{\tau_\textsc{ml}}
\newcommand{\tauBD}{\tau_\textsc{bd}}
\newcommand{\tauclBD}{\bar\tau_\textsc{bd}}
\newcommand{\dist}{\operatorname{dist}}
\newcommand{\llangle}{\langle\hspace{-2pt}\langle}
\newcommand{\rrangle}{\rangle\hspace{-2pt}\rangle}
\newcommand{\emax}{\epsilon_\mathrm{max}}
\newcommand{\emin}{\epsilon_\mathrm{min}}
\newcommand{\emint}{\epsilon_{\mathrm{min};t}}
\newcommand{\emaxt}{\epsilon_{\mathrm{max};t}}

\begin{document}
\title{Closed systems refuting quantum speed limit hypotheses}
\author{Niklas H{\"o}rnedal\,\orcidlink{0000-0002-2005-8694}\,}
\email{niklas.hornedal@uni.lu}
\affiliation{Department of Physics and Materials Science, University of Luxembourg, L-1511 Luxembourg, Luxembourg}
\author{Ole S{\"o}nnerborn\,\orcidlink{0000-0002-1726-4892}\,}
\email{ole.sonnerborn@kau.se}
\affiliation{Department of Mathematics and Computer Science, Karlstad University, 651 88 Karlstad, Sweden}
\affiliation{Department of Physics, Stockholm University, 106 91 Stockholm, Sweden}

\begin{abstract}
Quantum speed limits for isolated systems that take the form of a distance divided by a speed extend straightforwardly to closed systems. This is, for example, the case with the well-known Mandelstam-Tamm quantum speed limit. Margolus and Levitin derived an equally well-known and ostensibly related quantum speed limit, and it seems to be widely believed that the Margolus–Levitin quantum speed limit can be similarly extended to closed systems. However, a recent geometrical examination of this limit reveals that it differs significantly from most quantum speed limits. In this paper, we show contrary to the common belief that the Margolus-Levitin quantum speed limit does not extend to closed systems in an obvious way. More precisely, we show that there exist closed systems that evolve between states with any given fidelity in an arbitrarily short time while keeping the normalized expected energy fixed at any chosen value. We also show that for isolated systems, the Mandelstam-Tamm quantum speed limit and a slightly weakened version of this limit that we call the Bhatia-Davies quantum speed limit always saturate simultaneously. Both of these evolution time estimates extend straightforwardly to closed systems. We demonstrate that there are closed systems that saturate the Mandelstam-Tamm quantum speed limit but not the Bhatia-Davies quantum speed limit.
\end{abstract}
\date{\today}
\maketitle

\section{Introduction}
Many quantum speed limits (QSLs) for isolated systems take the form of a distance divided by a speed \cite{PiCiCeAdSo-Pi2016,Fr2016,DeCa2017}. Such evolution time estimates can be straightforwardly extended to closed systems.\footnote{An \emph{isolated system} is a system that evolves according to the von Neumann equation with a time-independent Hamiltonian, and a \emph{closed system} is one that evolves according to the von Neumann equation with a  time-varying Hamiltonian.} The famous Mandelstam-Tamm QSL is an estimate of this kind \cite{MaTa1945,AnAh1990}. The Mandelstam-Tamm QSL states that the time it takes for an isolated system to evolve between two fully distinguishable states is bounded from below by\footnote{\emph{State} will always refer to a pure quantum state, that is, a state that can be represented by a density operator of rank $1$.}\textsuperscript{,}\footnote{All quantities are expressed in units such that $\hbar=1$.}
\begin{equation}\label{MT}
    \tauMT
    = \frac{\pi}{2\Delta H},
\end{equation}
where $\Delta H$ is the energy uncertainty. More generally,
the time it takes for an isolated system to evolve between two states with fidelity $\delta$ is bounded from below by\footnote{The \emph{fidelity} or \emph{overlap} between two states $\rho_1$ and $\rho_2$ is $\tr(\rho_1\rho_2)$.} 
\begin{equation}\label{isolatedMT}
	\tauMT(\delta)
	= \frac{\arccos\sqrt{\delta}}{\Delta H}.
\end{equation}
This estimate is also due to Mandelstam and Tamm but was rediscovered and formulated more concisely in \cite{Fl1973}. 

The Mandelstam-Tamm QSL can be extended to closed systems by replacing the denominator in \eqref{isolatedMT} with the corresponding time average. Thus, the evolution time of a closed system evolving between two states with fidelity $\delta$ is bounded from below by 
\begin{equation}\label{closedMT}
	\tauclMT(\delta)
	= \frac{\arccos\sqrt{\delta}}{\llangle\Delta H_t\rrangle},
\end{equation}
with $\llangle\Delta H_t\rrangle$ being the time average of the energy uncertainty. Since the Fubini-Study distance between two states with fidelity $\delta$ is $\arccos\sqrt{\delta}$ and the Fubini-Study speed with which a state evolves is $\Delta H_t$ \cite{AnAh1990,HoAlSo2022}, the Mandelstam-Tamm QSL is saturated if and only if the state follows a Fubini-Study geodesic in the projective Hilbert space. Mandelstam and Tamm's QSL has been extended to systems in mixed states \cite{Uh1992, DeLu2013a, AnHe2014, HoAlSo2022}.

Margolus and Levitin \cite{MaLe1998} derived a seemingly similar evolution time estimate. The Margolus-Levitin QSL states that the time it takes for an isolated system to evolve between two fully distinguishable states is greater than or equal to
\begin{equation}\label{ML}
    \tauML=\frac{\pi}{2\langle H-\emin\rangle},
\end{equation}
where $\langle H-\emin\rangle$ is the expected energy $\langle H\rangle$ shifted by the smallest occupied energy $\emin$, hereafter called the normalized expected energy. A more general result states that the time it takes for an isolated system to evolve between two states with fidelity $\delta$ is lower bounded by
\begin{equation}\label{extML}
    \tauML(\delta)
    =\frac{\alpha(\delta)}{\langle H-\emin\rangle},
\end{equation}
where 
\begin{equation}\label{alpha}
    \alpha(\delta)
    =\min_{z^2\leq \delta}\frac{1+z}{2}\arccos\Big(\frac{2\delta-1-z^2}{1-z^2}\Big).
\end{equation}
Like $\tauMT(\delta)$, the bound $\tauML(\delta)$ is tight, and $\tauML(0)=\tauML$. The bound $\tauML(\delta)$ was established numerically in \cite{GiLlMa2003} and derived analytically in \cite{HoSo2023}. Reference \cite{HoSo2023} also contains a geometric interpretation of $\tauML(\delta)$ and a complete description of the systems that reach the bound.

A natural guess is that the Margolus-Levitin QSL is also valid for closed systems, provided one puts the time average of the normalized expected energy in the denominator. More generally, one might expect that the evolution time of a closed system is lower bounded by a quantity of the form $\mathcal{L}(\delta)/\llangle H_t-\emint\rrangle$ where $\mathcal{L}$ is some positive function that depends only on the fidelity $\delta$ between the initial and the final state. In the next section, we show that this is not the case:

\vspace{2pt}
\noindent \emph{We show that for each state $\rho$ and $0\leq \delta\leq 1$, there exists a Hamiltonian $H_t$ that evolves $\rho$ to a state with fidelity $\delta$ relative to $\rho$ in an arbitrarily short time while keeping the normalized expected energy fixed at an arbitrary predetermined value.}

\vspace{2pt}
Lui et al.\ \cite{LiMiFuWa2021} used the Bhatia–Davies inequality to transform the Mandelstam–Tamm QSL into an upper bound for a proposed operationally defined QSL \cite{ShLiZhYuLi2020}. This upper bound is a new QSL that we call the Bhatia-Davies QSL, although one should rightly attribute it to the authors of \cite{LiMiFuWa2021}. The Bhatia-Davies QSL states that the time it takes for an isolated system to evolve between two states with fidelity $\delta$ is bounded from below by
\begin{equation}\label{isolatedBD}
    \tauBD(\delta) = \frac{\arccos\sqrt{\delta}}{\sqrt{\langle\emax -  H\rangle\langle H - \emin\rangle}},
\end{equation}
where $\emax$ is the largest and $\emin$ is the smallest occupied energy. The Bhatia-Davies QSL also extends straightforwardly to closed systems:
\begin{equation}\label{closedBD}
	\tauclBD(\delta)
    = \frac{\arccos\sqrt{\delta}}{\llangle\sqrt{\langle\emaxt -  H_t\rangle\langle H_t- \emint\rangle}\,\rrangle}
\end{equation}

The Bhatia-Davies QSL is weaker than that of Mandelstam and Tamm in the sense that $\tauclMT(\delta)\geq \tauclBD(\delta)$ with a strict inequality in general for both isolated and closed systems. We show that the Mandelstam-Tamm and the Bhatia-Davies QSLs are always saturated simultaneously for isolated systems but that this need not be the case for closed systems: 

\vspace{2pt}
\noindent\emph{We provide an example of a closed system that saturates the Mandelstam-Tamm but not the Bhatia-Davies QSL.}

\section{Time-dependent systems that disprove common belief}
One obtains a relatively simple type of time-dependent Hamiltonian if one conjugates a time-independent Hamiltonian $H$ with a one-parameter group of unitaries generated by a Hermitian operator $A$:
\begin{equation}
    H_t=e^{-iAt}H e^{iAt}.
\end{equation}
Such a group action will preserve the eigenvalues but rotate the eigenvectors of $H$. If a state $\rho$ evolves under the influence of $H_t$,
\begin{equation}
    \dot\rho_t=-i[H_t,\rho_t],\qquad \rho_0=\rho,
\end{equation}
the state in the rotating frame picture,
\begin{equation}
    \rho^{\textsc{rf}}_t=e^{iAt} \rho_t e^{-iAt},
\end{equation}
evolves as if the time-independent Hamiltonian $A-H$ governed the dynamics:
\begin{equation}\label{rotating}
    \dot\rho^{\textsc{rf}}_t=-i[A-H,\rho^{\textsc{rf}}_t],\qquad \rho^{\textsc{rf}}_0=\rho.
\end{equation}
As a consequence, in the Schrödinger picture, 
\begin{equation}\label{evolution}
    \rho_t=e^{-iAt} e^{-i(A-H)t} \rho e^{i(A-H)t} e^{iAt}.
\end{equation}
In general, the behavior of $\rho_t$ can be quite complex even though $H_t$ has a relatively simple time dependence. However, equation \eqref{evolution} tells us that if $\rho$ commutes with $A-H$, the evolving state will behave as if the time-independent `effective' Hamiltonian $A$ generates it:
\begin{equation}
    \rho_t=e^{-iAt} \rho e^{iAt}.
\end{equation}
This observation will be of central importance below. 

The eigenvectors of $H$ will also evolve with $A$ as effective Hamiltonian: If $\ket{j}$ is an eigenvector of $H$ with eigenvalue $\epsilon_j$, then $\ket{j;t}=e^{-iAt}\ket{j}$ is an eigenvector of $H_t$ with the eigenvalue $\epsilon_j$. As a result, the occupations of the energy levels are constant over time:
\begin{equation}
    \bra{j;t}\rho_t\ket{j;t}=\bra{j}\rho\ket{j}.
\end{equation}
This means that the expected energy $\langle H_t\rangle$, the energy uncertainty $\Delta H_t$, and the normalized expected energy $\langle H_t-\emint\rangle$ and its `dual' $\langle \emaxt-H_t\rangle$ are conserved quantities; see \cite{NeAlSa2022,HoSo2023} for a QSL involving the dual of the normalized expected energy. 

Another important fact is that $\rho_t$ is a Fubini-Study geodesic if $A\rho+\rho A=A$; see Appendix A in \cite{HoAlSo2022}. If such is the case, the Mandelstam-Tamm QSL is saturated, and the system evolves between two states with fidelity $\delta$ in time $\tauclMT(\delta)$. Interestingly, given an initial state $\rho$ and a Hamiltonian $H$, there is an elegant way to construct an $A$ such that $[A-H,\rho]=0$ and $A\rho+\rho A=A$: Write $\rho=\ketbra{u}{u}$, let $\epsilon = \bra{u}H\ket{u}$, and define
\begin{equation}\label{elegant}
    A=(H-\epsilon)\ketbra{u}{u}+\ketbra{u}{u}(H-\epsilon).
\end{equation}
Below we show how to disprove two hypotheses about QSLs with appropriate choices of $\rho$ and $H$, and $A$ defined as in \eqref{elegant}.

\subsection{The non-existence of a time-dependent Margolus-Levitin QSL}\label{sec:No timeMLQSL}
The Mandelstam-Tamm and Margolus-Levitin QSLs say that if one requires the state to follow a geodesic, one cannot modify a time-independent Hamiltonian in such a way that the energy uncertainty takes on an arbitrarily large value without the normalized expected energy also doing so. Interestingly, this does not hold for time-dependent Hamiltonians. Below we give an example of a closed system whose state follows a geodesic and in which the energy uncertainty decouples from the normalized expected energy so that one can let the energy uncertainty assume arbitrarily large values while the normalized expected energy remains at a fixed, predetermined value.

\vspace{2pt}
\noindent \emph{The consequence is that one can make the system evolve between two states with a given fidelity $\delta$ in an arbitrarily short time and, at the same time, keep the normalized expected energy fixed at a finite value.}

\vspace{2pt}
\noindent Consider a quantum system in a state $\rho=\ketbra{u}{u}$. Let $H$ be a Hamiltonian, to be specified, and define $A$ as in \eqref{elegant}. Further, let $H_t=e^{-iAt}He^{iAt}$ and let $\rho_t$ be the state at time $t$ generated from $\rho$ by $H_t$. Then $\rho_t=e^{-iAt}\rho e^{iAt}$, and $\rho_t$ follows a Fubini-Study geodesic.

\begin{figure}[t]
	\centering
	\includegraphics[width=0.9\linewidth]{Fig0.pdf}
	\caption{Graphs illustrating the dependence of the normalized expected energy (red) and energy uncertainty (blue) on the angle $\theta$. The requirement that the normalized expected energy be constant forces the energy uncertainty to grow toward infinity with decreasing angle.}
	\label{fig0}
\end{figure}
To specify $H$ let $\ket{v}$ be a unit vector perpendicular to $\ket{u}$ and define the Pauli operators $X$ and $Z$ as
\begin{align}
    X &= \ketbra{u}{u}-\ketbra{v}{v}, \label{X}\\
    Z &= \ketbra{u}{v}+\ketbra{v}{u}. \label{Z}
\end{align}
Fix the value $E>0$ that the normalized expected energy should have, let $\mu$ be a positive function on the interval $0 < \theta <\pi$, and define 
\begin{equation}
    H=\mu(\theta)(\sin\theta Z - \cos\theta X).
\end{equation}
The largest and the smallest eigenvalues of $H$, and thus of $H_t$, are $\mu(\theta)$ and $-\mu(\theta)$, respectively, both of which are occupied by $\rho_t$. Furthermore, the normalized expected energy and the energy uncertainty are 
\begin{align}
    &\langle H_t - \emint\rangle = \mu(\theta)(1-\cos\theta), \\
    &\Delta H_t = \mu(\theta)\sin\theta.
\end{align}
Since we want the normalized expected energy to be $E$, we must define $\mu$ as $\mu(\theta)=E/(1-\cos\theta)$, implying that
\begin{equation}
    \Delta H_t = E\cot(\theta/2).
\end{equation}
Figure \ref{fig0} shows how the normalized expected energy and the energy uncertainty depend on the angle $\theta$.
\begin{figure}[t]
	\centering
	\includegraphics[width=0.9\linewidth]{fig2test3.pdf}
	\caption{Bloch vector representations of $H$, $A$, and $\rho$. The vector representing $\rho$ points along the positive $x$-axis, and the vector representing $H$ makes the angle $\theta$ with the negative $x$-axis. The purple circle represents the expected energy level to which $\rho$ belongs. As time passes, the state and the Hamiltonian rotate around the $z$-axis with the same angular velocity. The dashed vectors represent $H_t$ and $\rho_t$ at a time $t>0$. The expected energy level rotates with the state.}
	\label{fig1}
\end{figure}

Let $\tau(\delta)$ be the first time the system reaches a state having fidelity $\delta$ with the initial state $\rho$. Since the state follows a Fubini-Study geodesic, the Mandelstam-Tamm QSL is saturated: 
\begin{equation}
    \tau(\delta)=\tauclMT(\delta)=\frac{\arccos\sqrt{\delta}}{E\cot(\theta/2)}.
\end{equation}
This evolution time can be made arbitrarily small by choosing $\theta$ sufficiently close to $0$. However, regardless of the value of $\theta$, the normalized average energy is preserved with the prescribed value $E$. We conclude that irrespective of a required fidelity $\delta$ between the initial and the final states, a Hamiltonian exists that evolves the system between two states with fidelity $\delta$ in an arbitrarily short time and along a trajectory such that the normalized expected energy is conserved with a prescribed value. 

\vspace{2pt}
\noindent\emph{Consequently, the Margolus-Levitin QSL does not obviously extend to closed systems.}

\vspace{2pt}
In Figure \ref{fig1}, we have represented $H$, $A$, and $\rho$ as Bloch vectors relative to $X$, $Y$, and $Z$, with $Y=i(\ketbra{u}{v}-\ketbra{v}{u})$. The angle between $H$ and the negative $x$-axis is $\theta$. As time passes, the state and the Hamiltonian rotate around the $z$-axis with the same angular velocity. Note that $\rho_t$ moves along the equator in the Bloch sphere and thus is a Fubini-Study geodesic. The dotted vectors represent the state and the Hamiltonian at a time $t>0$.

The purple circle, formed by intersecting the Bloch sphere with a plane perpendicular to the extension of the vector representing $H$, represents the expected energy level to which $\rho$ belongs. This circle rotates together with $H_t$ and always lies in a plane perpendicular to the vector representing $H_t$. The key observation is that this circle corresponds to the normalized expected energy $E$ irrespective of the value of angle $\theta$, and $\rho_t$ will evolve together with that circle.

Most initial states will not evolve in such a well-behaved manner as those located on the equator of the Bloch sphere. In Figure \ref{fig2}, we have drawn the evolution curve of a state not on the equator.
\begin{figure}[t]
	\centering
	\includegraphics[width=1.0\linewidth]{fig3.jpg}
	\caption{An evolution curve starting from a state not on the equator of the Bloch sphere. In this case, $\theta=30^\circ$ and $E=1$. The left figure shows the evolution curve in the Schrödinger picture, and the right figure shows the same curve in the rotating frame picture. The warmer colors indicate more recent times, and the blue arrow represents the state at the final time.}
	\label{fig2}
\end{figure}
In the rotating frame picture \eqref{rotating}, the evolution curve forms a circle around the $x$-axis. This is since $A-H\propto X$.

\subsection{The Bhatia-Davies QSL}
The example in the previous section shows that the normalized expected energy alone does not necessarily  limit the evolution time from below for closed systems. In the example, however, an arbitrary width of the energy spectrum was permitted. If we require that the spectral width does not exceed a given value, the evolution time cannot be made arbitrarily small. This since the energy uncertainty cannot exceed the spectral width.

The Bhatia-Davies inequality \cite{BhDa2000} provides a tighter bound on the energy uncertainty than the spectral width. The Bhatia-Davies inequality states that the variance of any observable $B$ is bounded from above according to 
\begin{equation}\label{Bhatia-Davies ineq}
    \Delta^2 B\leq \langle b_{\mathrm{max}} -  B\rangle\langle B - b_{\mathrm{min}}\rangle,
\end{equation}
with $b_{\mathrm{max}}$ and $b_{\mathrm{min}}$ being the largest and the smallest occupied eigenvalues of $B$. Consequently, the evolution time of an isolated system is bounded by $\tauBD(\delta)$ defined in \eqref{isolatedBD}, and the evolution time of a closed system is bounded by $\tauclBD(\delta)$ defined in \eqref{closedBD}. 

Equality holds in the Bhatia-Davies inequality if and only if the state occupies at most two eigenvalues of $B$. Since the state of an isolated system saturating the Mandelstam-Tamm QSL occupies only two energy levels \cite{Br2003,HoAlSo2022}, the Mandelstam-Tamm and Bhatia-Davies QSLs are always saturated simultaneously for isolated systems.

The Mandelstam-Tamm and Bhatia-Davies QSLs generalize to closed systems as in \eqref{closedMT} and \eqref{closedBD}, respectively, and a natural guess would be that also these QSLs are always saturated simultaneously. However, as we will see, a time-dependent Hamiltonian can evolve a state at a constant speed along a Fubini-Study geodesic in such a way that the state during the entire evolution occupies more than two energy levels. Such an evolution will saturate the Mandelstam–Tamm QSL but not the Bhatia–Davies QSL. This is because the Bhatia-Davies inequality will be strict over the entire evolution time interval, which means that the denominator in \eqref{closedBD} is strictly greater than the denominator in \eqref{closedMT}.

\subsection{A non-saturation of the Bhatia-Davies QSLs}
Let $H$ be a Hamiltonian for a system with at least three distinct eigenvalues, and let $\rho=\ketbra{u}{u}$ be any state occupying at least three of those. Define $A$ as in \eqref{elegant}, let $H_t=e^{-iAt}He^{iAt}$, and let $\rho_t$ be the state at time $t$ generated from $\rho$ by $H_t$. Since $[A-H,\rho]=0$ and $A\rho+\rho A=A$, the Mandelstam-Tamm QSL is saturated, and the system will evolve between two states with fidelity $\delta$ in time $\tauclMT(\delta)$. Furthermore, since $\rho_t$ always occupies at least three different energy levels, 
\begin{equation}
    \Delta^2H_t<\langle\emaxt -  H_t\rangle\langle H_t - \emint\rangle.
\end{equation}
Therefore, $\tauclMT(\delta)>\tauclBD(\delta)$, and the Mandelstam-Tamm QSL is saturated but not the Bhatia-Davies QSL.

\section{Summary}
A common view is that the Margolus-Levitin quantum speed limit extends to an evolution time estimate for closed systems of the form $\mathcal{L}(\delta)/\llangle H_t-\emint\rrangle$ where $\mathcal{L}$ is a positive function that only depends on the fidelity $\delta$ between the initial and final states and $\llangle H_t-\emint\rrangle$ is the time average of the normalized expected energy. We have shown with a counterexample that this is not the case. More precisely, we have constructed a closed system that evolves between two states with fidelity $\delta$ in an arbitrarily short time while keeping the normalized expected energy fixed at an arbitrary predetermined value.

We have also considered a QSL for isolated systems called the Bhatia-Davies QSL. This QSL extends straightforwardly to closed systems. We have shown that the Bhatia-Davies and Mandelstam-Tamm QSLs are always simultaneously saturated for isolated systems but that this need not be the case for closed systems.

\begin{thebibliography}{99}

\bibitem{PiCiCeAdSo-Pi2016}
D. P. Pires, M. Cianciaruso, L. C. Céleri, G. Adesso, and D. O. Soares-Pinto, 
Generalized geometric quantum speed limits, 
\href{https://doi.org/10.1103/PhysRevX.6.021031}{Phys. Rev. X \textbf{6}, 021031 (2016)}.

\bibitem{Fr2016}
M. R. Frey, 
Quantum speed limits–primer, perspectives, and potential future directions,
\href{https://doi.org/10.1007/s11128-016-1405-x}{Quantum Information Processing \textbf{15}, 3919 (2016)}.

\bibitem{DeCa2017}
S. Deffner and S. Campbell, 
Quantum speed limits: from Heisenberg’s uncertainty principle to optimal quantum control, \href{https://doi.org/10.1088/1751-8121/aa86c6}{J. Phys. A: Math. Theor. \textbf{50}, 453001 (2017)}.

\bibitem{MaTa1945}
L. Mandelstam and I. Tamm,
The uncertainty relation between energy and time in non-relativistic quantum mechanics,
\href{https://doi.org/10.1007/978-3-642-74626-0_8}{J. Phys. (USSR) \textbf{9}, 249 (1945)}.

\bibitem{AnAh1990}
J. Anandan and Y. Aharonov,
Geometry of quantum evolution,
\href{https://doi.org/10.1103/PhysRevLett.65.1697}{Phys. Rev. Lett. \textbf{65}, 1697 (1990)}.

\bibitem{Fl1973}
G. N. Fleming,
A unitarity bound on the evolution of nonstationary states,
\href{https://doi.org/10.1007/BF02819419}{Nuov Cim A \textbf{16}, 232–240 (1973)}. 

\bibitem{HoAlSo2022}
N. H\"ornedal, D. Allan, and O. S\"onnerborn,
Extensions of the Mandelstam–Tamm quantum speed limit to systems in mixed states, \href{https://doi.org/10.1088/1367-2630/ac688a}{New J. Phys. \textbf{24}, 055004 (2022)}.

\bibitem{DeLu2013a}
S. Deffner and E. Lutz,
Energy-time uncertainty relation for driven quantum systems,
\href{https://doi.org/10.1088/1751-8113/46/33/335302}{J. Phys. A: Math. Theor. \textbf{46}, 335302 (2013)}.

\bibitem{Uh1992}
A. Uhlmann, 
An energy dispersion estimate, 
\href{https://doi.org/10.1016/0375-9601(92)90555-Z}{Phys. Lett. A \textbf{161}, 329 (1992)}.

\bibitem{AnHe2014}
O. Andersson and H. Heydari,
Quantum speed limits and optimal Hamiltonians for driven systems in mixed states, \href{https://doi.org/10.1088/1751-8113/47/21/215301}{J. Phys. A: Math. Theor. \textbf{47}, 215301 (2014)}.

\bibitem{MaLe1998}
N. Margolus and L. B. Levitin, The maximum speed of dynamical evolution, \href{https://doi.org/10.1016/S0167-2789(98)00054-2}{Physica D \textbf{120}, 188 (1998)}.

\bibitem{GiLlMa2003}
V. Giovannetti, S. Lloyd, and L. Maccone,
Quantum limits to dynamical evolution,
\href{https://doi.org/10.1103/PhysRevA.67.052109}{Phys. Rev. A \textbf{67}, 052109 (2003)}.

\bibitem{HoSo2023}
N.\ H\"ornedal and O. S\"onnerborn, The Margolus-Levitin quantum speed limit for an arbitrary fidelity, \href{https://doi.org/10.48550/arXiv.2301.10063}{arXiv:2301.10063 (2023)}.

\bibitem{LiMiFuWa2021}
J. Liu, Z. Miao, L. Fu, and X. Wang, Bhatia-Davis formula in the quantum speed limit, \href{https://doi.org/10.1103/PhysRevA.104.052432}{Phys. Rev. A \textbf{104}, 052432 (2021)}.

\bibitem{ShLiZhYuLi2020}
Y. Shao, B. Liu, M. Zhang, H. Yuan, and J. Liu, Operational definition of a quantum speed limit, 
\href{https://doi.org/10.1103/PhysRevResearch.2.023299}{Phys. Rev. Res. \textbf{2}, 023299 (2020)}.

\bibitem{NeAlSa2022}
G. Ness, A. Alberti, and Y. Sagi,
\textit{Quantum Speed Limit for States with a Bounded Energy Spectrum},
\href{https://doi.org/10.1103/PhysRevLett.129.140403}{Phys. Rev. Lett. {129}, 140403 (2022)}.

\bibitem{BhDa2000}
R. Bhatia and C. Davis,
A Better Bound on the Variance,
\href{https://doi.org/10.1080/00029890.2000.12005203}{The American Mathematical Monthly \textbf{107}, 353 (2000)}.

\bibitem{Br2003}
D. C. Brody, 
Elementary derivation for passage times, 
\href{https://doi.org/10.1088/0305-4470/36/20/314}{J. Phys. A: Math. Gen. \textbf{36}, 5587 (2003)}.

\end{thebibliography}
\end{document}