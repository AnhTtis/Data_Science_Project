\documentclass[10pt,twocolumn,letterpaper]{article}
\usepackage[pagenumbers]{cvpr} % To force page numbers, e.g. for an arXiv version
\usepackage{graphicx}
\usepackage{amsmath}
\usepackage{amssymb}
\usepackage{booktabs}
\usepackage{tabularx}
\usepackage{float}
\usepackage{multirow}
\usepackage{siunitx}
\usepackage{hyperref}
\hypersetup{colorlinks,allcolors=black}
\usepackage{caption}  % for small gap between table and caption 
\captionsetup[table]{skip=3pt}
\captionsetup[figure]{skip=3pt}
\usepackage[capitalize]{cleveref}
\usepackage{pifont}%for xmark
\newcommand{\cmark}{\ding{51}}%
\newcommand{\xmark}{\ding{55}}%

\crefname{section}{Sec.}{Secs.}
\Crefname{section}{Section}{Sections}
\Crefname{table}{Table}{Tables}
\crefname{table}{Tab.}{Tabs.}

\setlength {\marginparwidth }{2cm}
\begin{document}

%%%%%%%%% TITLE - PLEASE UPDATE
\title{\ MAIR: Multi-view Attention Inverse Rendering with 3D Spatially-Varying Lighting Estimation}

\author{
{JunYong Choi$^{1,2}$}\quad 
{SeokYeong Lee$^{1,2}$}\quad
{Haesol Park$^{1}$}\quad 
{Seung-Won Jung$^{2}$}\quad
{Ig-Jae Kim$^{1,3,4}$}\quad
{Junghyun Cho$^{1,3,4}$} 
\\[2mm]
{$^{1}$Korea Institute of Science and Technology(KIST)}\quad 
{$^{2}$Korea University} \\
{$^{3}$AI-Robotics, KIST School, University of Science and Technology} \\
{$^{4}$Yonsei-KIST Convergence Research Institute, Yonsei University} \\
\vspace{-3mm}
{\tt\small \{happily,shapin94,haesol,drjay,jhcho\}@kist.re.kr}\quad
{\tt\small swjung83@korea.ac.kr}
\renewcommand\footnotemark{}
\renewcommand\footnoterule{}
\thanks{This work was partly supported by Institute of Information \& communications Technology Planning \& Evaluation (IITP) grant funded by the Korea government(MSIT)(No.2020-0-00457, 50\%) and KIST Institutional Program(Project No.2E32301, 50\%).}
}
\maketitle

%%%%%%%%% ABSTRACT
\begin{abstract}
We propose a scene-level inverse rendering framework that uses multi-view images to decompose the scene into geometry, a SVBRDF, and 3D spatially-varying lighting. Because multi-view images provide a variety of information about the scene, multi-view images in object-level inverse rendering have been taken for granted. However, owing to the absence of multi-view HDR synthetic dataset, scene-level inverse rendering has mainly been studied using single-view image. We were able to successfully perform scene-level inverse rendering using multi-view images by expanding OpenRooms dataset and designing efficient pipelines to handle multi-view images, and splitting spatially-varying lighting. Our experiments show that the proposed method not only achieves better performance than single-view-based methods, but also achieves robust performance on unseen real-world scene. Also, our sophisticated 3D spatially-varying lighting volume allows for photorealistic object insertion in any 3D location.


\end{abstract}

%------------------------------------------------------------------------


%%%%%%%%% BODY TEXT
\section{Introduction}
\label{sec:intro}
Inverse rendering is a technology used to estimate material, lighting, and geometry from RGB color images. Decomposing a scene through inverse rendering enables various applications such as object insertion, relighting, and material editing in VR and AR. However, since inverse rendering is an ill-posed problem, previous studies have focused only on a part of the inverse rendering, such as intrinsic image decomposition~\cite{id1,id2,id3,id4}, shape from shading~\cite{sfs1,sfs2,sfs3}, and material estimation~\cite{sime1, sime2, sime3, sime4, sime5}. 

Recent advances in GPU-accelerated physically-based rendering algorithms have made constructing large-scale photorealistic indoor high dynamic range (HDR) image dataset that include geometries, materials, and spatially-varying lighting~\cite{openrooms2021}. The availability of such dataset and the recent success of deep learning technology have enabled seminal works on single-view-based inverse rendering~\cite{cis2020, irisformer2022, vsg, phyir, Li22, zhu2022montecarlo}. These methods have fundamental limitations in that they are prone to bias in training dataset despite having shown promising results. Specifically, single-view-based inverse rendering must refer to specular reflectance from the contextual information of the image, making it less reliable for predicting complex SVBRDF(spatially-varying bidirectional reflectance distribution function) in the real-world. Fig.~\ref{fig:main} shows such an example, where decomposition has severely failed owing to the complexity of the real-world scene. In addition, depth-scale ambiguity makes it challenging to employ these methods to 3D applications such as object insertion. 

\begin{figure}[t!]
  \centering
  \includegraphics[width=\linewidth]{figure/main.pdf}
  \caption{The result of inverse rendering and floating chrome sphere insertion in the unseen real-world scene. Since the single-view-based method\cite{cis2020} relies only on contextual information, it has difficulty estimating the complex material, geometry of the real-world. Notice the spatially-consistent albedo of apples, eloborated normal, and realistic lighting reflected on the inserted object. In previous work\cite{cis2020}, they have limitations on floating object insertion because they use per-pixel lighting.}
  \label{fig:main}
  \vspace{-2mm}
\end{figure}

In this paper, we introduce \textbf{MAIR}, the scene-level \textbf{M}ulti-view \textbf{A}ttention \textbf{I}nverse \textbf{R}endering pipeline. MAIR exploits multiple RGB observations of the same scene point from multiple cameras and, more importantly, it utilizes multi-view stereo (MVS) depth as well as scene context to estimate SVBRDF. As a result, the method becomes less dependent on the implicit scene context, and shows better performance on unseen real-world images. However, the processing of multi-view images inherently requires a high computational cost to handle multiple observations with occlusions and brightness mismatches. To remedy this, we design a three-stage training pipeline for MAIR that can significantly increase training efficiency and reduce memory consumption. Spatially-varying lighting consists of direct lighting and indirect lighting. Indirect lighting affected by the surrounding environment makes inverse rendering difficult. Therefore, in Stage 1, we first estimate the direct lighting and geometry, which reflect the amount of light entering each point and in which direction the specular reflection appears. We estimate the material in Stage 2 using the estimates of the direct lighting, geometry, and multi-view color images. In Stage 3, we collect all the material, geometry, and direct lighting information and finally estimate 3D spatially-varying lighting, including indirect lighting. The MAIR pipeline is shown in Fig.~\ref{fig:whole}. We created the OpenRooms Forward Facing(OpenRooms FF) dataset as an extension of OpenRooms~\cite{openrooms2021} to train the proposed network.

Our contribution can be summarized as follows:
\begin{itemize}
\item As summarized in Tab.~\ref{tab:compare_prior}, we believe this is the first demonstration of using multi-view images to decompose the scene into geometry, complex material, and 3D spatially-varying lighting without test-time optimization. Also, we release OpenRooms FF dataset.
% Our method can robustly perform inverse rendering compared with the single-view context-based method and we release OpenRooms FF dataset.

\item We propose a framework that can efficiently train multi-view inverse rendering networks. Our framework increases the training efficiency by decomposing lighting and separating the scene components by stage.
\item Our method achieves better inverse rendering performance than the existing single-view-based method, and realistic object insertion in real-world is possible by reproducing 3D lighting of the real-world.
\vspace{-2mm}
\end{itemize}

\begin{figure*}[t!]
  \centering
  \includegraphics[width=\linewidth]{figure/entire.pdf}
   \caption{MAIR's entire pipeline. Our method has reduced the difficulty of inverse rendering by splitting the scene components as small as possible, and progressively estimating the scene components.}
   \label{fig:whole}
\end{figure*}


\section{Related Works}
\noindent\textbf{Inverse rendering.}
Research on inverse rendering has received significant attention in recent years owing to the development of deep learning technology. Yu \etal\cite{outdoorinv} performed outdoor inverse rendering with multi-view self-supervision, but their lighting is simple distant lighting. A pioneering work by Li \etal \cite{cis2020} conducted inverse rendering on a single image, and IRISformer\cite{irisformer2022} further improved the performance by replacing convolutional neural networks (CNNs) with a Transformer\cite{vit}. PhyIR \cite{phyir} addressed this problem using panoramic image. However, the lighting representation in~\cite{cis2020,irisformer2022, phyir} is a 2D per-pixel environment map, which is insufficient for modeling 3D lighting. Li \etal \cite{Li22} adopted a parametric 3D lighting representation; however, it fixed with two types of indoor light sources. The recently introduced Zhu \etal\cite{zhu2022montecarlo} demonstrated realistic 3D volumetric lighting based on ray tracing. But, since these prior works\cite{outdoorinv, cis2020,irisformer2022,Li22, phyir, zhu2022montecarlo} are all single-view-based, they inherently rely on the scene context, making these methods less reliable for unseen images. In contrast, the proposed method can estimate BRDF and geometry more accurately by utilizing the multi-view correspondences as additional cues for inverse rendering. 

\begin{table}[t!]
\footnotesize
\centering
\resizebox{\linewidth}{!}{%
\begin{tabular}{|l|c|c|c|c|}
\hline
Method & \multicolumn{1}{c|}{Input} & \multicolumn{1}{c|}{Material} & \multicolumn{1}{c|}{Lighting} & \multicolumn{1}{c|}{Insertion} \\
\hline
VSG\cite{vsg} &single &diffuse &volume &any\\
Li~\etal\cite{cis2020} &single &microfacet &per-pixel &surface\\
IRISformer\cite{irisformer2022} &single  &microfacet &per-pixel &surface\\
PBE\cite{Li22} &single &microfacet &parametric &any\\
SOLD \cite{SOLD-Net} & single &\xmark &per-pixel &surface\\
Zhu~\etal\cite{zhu2022montecarlo} &single &microfacet(metalic) &volume &any\\
lighthouse~\cite{lighthouse} &stereo &\xmark &volume &any\\
FreeView~\cite{philip2021} &multi &glossy &irradiance & \xmark\\
Zhang~\etal \cite{invrender2022} &multi &microfacet &object & \xmark\\
PhotoScene~\cite{photoscene} &any &microfacet &parameteric &any\\
Intrinsic3D~\cite{maier2017intrinsic3d} &multi RGBD &diffuse &volume &any\\
Zhang \etal\cite{zhang2016emptying} &multi RGBD &diffuse &parameteric &any\\
\hline
MAIR (Ours) &multi &microfacet &volume &any\\
\hline
\end{tabular}%
}
\caption{Compared to previous works, our work is the first demonstration to perform multi-view scene-level inverse rendering.}
\vspace{-3mm}
\label{tab:compare_prior}
\end{table}

\noindent\textbf{Lighting estimation.}
Lighting estimation has been studied not only as a sub-task of the inverse rendering but also an important research topic\cite{le1,le2,le3,Song19}. In Lighthouse \cite{lighthouse}, 3D spatially-varying lighting was obtained by estimating the RGBA lighting volume. Wang \etal\cite{vsg} estimated a more sophisticated 3D lighting volume by replacing RGB with a spherical Gaussian in lighting volume. However, because the methods in~\cite{lighthouse,vsg} assume Lambertian reflectance, they cannot represent complex indirect lighting and have limitations in expressing HDR lighting due to weak-supervision with LDR dataset\cite{interiornet}. On the other hand, the proposed method can handle complex SVBRDF and HDR lighting well because we trained our model on the large indoor HDR dataset\cite{openrooms2021}. Recently, a study on spatially-varying lighting estimation in the outdoor scenes\cite{wang2022neural,SOLD-Net} was introduced. Because they focus on outdoor street scenes, they cannot clearly reproduce the indirect lighting by the scene material. 

\noindent\textbf{Multi-view inverse rendering and neural rendering.}
Earlier works such as Intrinsic3D~\cite{maier2017intrinsic3d} and Zhang~\etal \cite{zhang2016emptying} perform multi-view inverse rendering without deep learning, but require additional equipment to obtain RGB-D images. 3D geometry-based methods~\cite{cis-texture, invpath, kim2016multi} require additional computation to generate mesh. PhotoScene~\cite{photoscene}, in particular, requires external CAD geometry, which should be manually aligned for each object. Moreover, all of these methods require test-time optimization. In contrast, we only need per-view depth maps and we does not require 3D geometry or test-time optimization, making it more computationally efficient and much easier to apply to more general scenes. Philip \etal\cite{philip2019, philip2021} demonstrated a successful relighting with multi-view images; however, these methods cannot be used for applications such as object insertion because they use trained neural renderers. The advent of NeRF\cite{nerf, mipnerf, mip360} has led to a breakthrough in the field of neural rendering research. Several recent methods\cite{nerd, nerv, physg, zhang2021nerfactor, refnerf, invrender2022} have successfully performed inverse rendering using multi-view images; however, they are difficult to apply to scene-level inverse rendering because they are only trained and tested on object-centric images.

% Various research results\cite{nerfeditable, nerfcomposite, nerfobjcen, giraffe}, such as scene editing or object insertion have been also introduced. However, they did not consider the materials and lighting of the scene. In contrast, our work focuses on scene-level inverse rendering, which enables physically-meaningful object insertion.
%%%%%%%%%%%%%%%%%%%%%%%%%%%%%%%%%%%%%%%%%%%%%%%
%%%%%%%        2. Proposed Approach         %%%%%%%
%%%%%%%%%%%%%%%%%%%%%%%%%%%%%%%%%%%%%%%%%%%%%%%

\section{Proposed Approach}
\label{sec:methods}

A depiction of our approach is shown in Figure \ref{fig:model}. The model consists of a MOS prediction model (shown left) and a speech enhancement model (shown right). Our MOS prediction model is tailored to provide estimates for subjective-MOS (as rated by humans), and going forward, we will use MOS to refer to subjective-MOS unless explicitly stated otherwise, for ease of understanding. We next will provide notation and then describe each of these sub-modules.

\subsection{Notation}

We define a clean speech signal as $s_t$ and background noise as $n_t$ at time $t$. The mixture of clean speech and noise is denoted as $m_t=s_t+n_t$. We aim to extract the speech from the mixture by removing the unwanted noise. The short-time Fourier transform (STFT) converts the time-domain mixture into a T-F representation, $M_{t,f}$, that is defined at time $t$ and frequency $f$. The complex-valued STFT matrix, $\bm{M}$, can be written as $\bm{M}=|\bm{M}|e^{i\bm{\theta}^M}$ with magnitude $|\bm{M}|\in \bm{\Re}^{T\times F}_+$ and phase $\bm{\theta}^M \in \bm{\Re}^{T\times F}$, where $T$ is the length of speech in time and $F$ is the total number of frequency channels.

Enhancing the magnitude response of noisy speech results in an estimate of the clean speech magnitude response, $|\hat{\bm{S}}|$, using an enhancement function $\mathcal{F}_\delta$ such that $|\hat{\bm{S}}| =\mathcal{F}_\delta(|\bm{M}|)$. The enhancement function is modeled with a deep neural network which is trained to maximize the conditional log-likelihood of the training dataset, 
\begin{align*}
    &\max \frac{1}{N} \sum^N \log P\Big( |{\bm{S}}| \, \Big| \, |\bm{M}|\Big) \\
    \Rightarrow &\max_\delta \frac{1}{N} \sum^N \log P\Big( \mathcal{F}_\delta(|\bm{M}|) \, \Big| \, |\bm{M}|\Big)
\end{align*}
% $$\max \frac{1}{N} \sum^N \log P\Big( |{\bm{S}}| \, \Big| \, |\bm{M}|\Big) \Rightarrow \max_\delta \frac{1}{N} \sum^N \log P\Big( |\hat{\bm{S}}| \, \Big| \, |\bm{M}|\Big) $$
where $\delta$ denotes the set of tunable parameters and $N$ is the number of training examples. The estimated magnitude response $|\hat{\bm{S}}|$ is then combined with the noisy phase, $\bm{\theta}^M$, where the inverse STFT produces an enhanced speech signal in the time domain, $\hat{s}_t$. 

\subsection{Speech quality assessment model}
\label{subsec:mos_model}

A MOS prediction model proposed by \cite{dong2020pyramid} is adapted to estimate the MOS from noisy speech. This model has been developed with real-world captured data and it has been shown to outperform comparison approaches~\cite{fu2018quality, avila2019non, mittag2019non}, according to multiple metrics. The MOS prediction model consists of an attention-based encoder-decoder structure that uses stacked pyramid bi-directional long-short term memory (pBLSTM)~\cite{chan2016listen} networks in the encoder. We denote this model as Pyramid-MOS (PMOS). A pBLSTM architecture gives the advantages of processing sequences at multiple time resolutions, which effectively captures short- and long-term dependencies. Speech has spectral and temporal dependencies over short and long durations, and a multi-resolution framework is effective in learning these complex relations. 


A single T-F frame of the noisy-speech mixture, $|\bm{M}_t|$, is the input to the PMOS encoder. In a pyramid structure, the lower layer outputs from $\Upsilon$ consecutive time frames are concatenated and used as inputs to the next pBLSTM layer, along with the recurrent hidden states from the previous time step. The output of a pBLSTM node is an embedding vector, $h^l_t$, that is as defined below:
\begin{align}
    h^l_t &= pBLSTM\Big( h^l_{t-1}, \big[ h^{l-1}_{\Upsilon\times t -\Upsilon+1}, h^{l-1}_{\Upsilon\times t}\big] \Big)
\end{align}
where $\Upsilon$ is the reduction factor (e.g., number of concatenated frames) between successive pBLSTM layers and $l$ is the layer number. A pBLSTM reduces the time resolution from the input speech to the final latent representation $\bm{H}$. Figure~\ref{fig:pBLSTM} shows the internal structure of pBLSTM module.
This compressed vector accumulates the useful features for measuring speech perceptual quality that resides in a range of time-frames and ignores the least important features.
The encoder outputs a concatenated version of the hidden states of the last pBLSTM layer as vector $\bm{H}=\{\bm{h}_1, \dotsb, \bm{h}_\tau, \dotsb, \bm{h}_\wp\}$, where $\wp$ is the total number of final embedding vectors with time index $\tau$.

The output of the PMOS encoder becomes the input to the PMOS decoder unit. This decoder is implemented as an attention layer followed by a fully-connected (FC) layer and it outputs an estimated MOS of the input speech utterance. Attention models learn key attributes of a latent sequence, since adjacent time frames can provide important information, which is particularly crucial for our task.  
The attention mechanism~\cite{luong2015effective} uses the pyramid encoder output at the $i$-th and $k$-th time steps to compute the attention weights, $\alpha^{PMOS}_{i,k}$. Attention weights are used to compute a context vector, $c^{PMOS}_i$, using the following equations:
\begin{align}
    \alpha^{PMOS}_{i,k} &= \frac{\exp{(\bm{h}_i^\top \bm{Q} \bm{h}_k)}}{\sum^{\wp}_{\phi=1} \exp{(\bm{h}_i^\top \bm{Q} \bm{h}_\phi)}}\\
    % \alpha^{PMOS}_{i,k} &= Attention(\bm{h}_i, \bm{h}_k)\\
    c^{PMOS}_i &= \sum^\wp_{k=1} \alpha^{PMOS}_{i,k} \cdot \bm{h}_k
\end{align}
$\bm{Q}^{\wp\times\wp}$ is the trainable PMOS attention weight matrix. We learn $\bm{Q}$ using a feed-forward neural network that attempts to capture the alignment between the embeddings $\bm{h}_i$ and $\bm{h}_k$. 

The context vector is provided to a fully-connected layer to estimate the MOS. Note that the pyramid structure of the encoder results in a shorter sequence of latent representations than the original input sequence, and it leads to fewer encoding states for attention calculation at the decoding stage. Therefore, strictly  $\wp<T$, and in our case $\wp = \lceil T/\Upsilon^L \rceil$, where $L$ is the number of pBLSTM layers.
We train the PMOS model separately with the parameters defined in~\cite{nayem2019incorporating}. After training, this model is held frozen during inference.

\begin{figure}[t!]
    \centering
    \includegraphics[width = 0.95\linewidth]{figs/pBLSTM.png}
    \caption{Illustration of pBLSTM structure with reduction factor $\Upsilon=2$ and number of layer $L=2$.}
    % \vspace{-2em}
    \label{fig:pBLSTM}
    % \vspace{-0.4cm}
\end{figure}

\subsection{Proposed speech enhancement model}
\label{subsec:se_model}
Our proposed speech-enhancement (SE) model follows an encoder-decoder structure, and it is shown in Figure \ref{fig:model} (right). The SE encoder takes a single T-F frame of a noisy-speech mixture, $|\bm{M}_t|$, as input and multiple BLSTM layers, are stacked together to create a hidden representation of the frame, $\bm{g}_t$. In our SE encoder, we utilize BLSTM layers instead of pBLSTM layers since we aim to estimate an embedding frame for each time frame and pBLSTM layers reduce the number of output frames. 
An attention mechanism is applied using the mixture encoding from the SE model, $\bm{G}=\{\bm{g}_1, \bm{g}_2, \dotsb, \bm{g}_T\}$, and the PMOS encoding, $\bm{H}$, from the MOS prediction model. This allows the SE model to exploit the MOS estimator's encoding and utilize the important perceptual feature embedding that correlates with human assessment. Considering that the pBLSTM structure of the PMOS encoder condenses the final encoding vector $\bm{H}$ along time, PMOS yields a smaller time resolution than the encoding from the SE encoder, so we compute a score for each embedding vector $\bm{h}_{\tau}$ using an alignment  weight matrix, $\bm{W}^{T\times\wp}$. Then the attention weights for the SE model, $\alpha_{t,\tau}$, are obtained using a softmax operation over the scores of all $\bm{h}_\tau$. Now, the PMOS encoding is summarized in a context vector $\bm{c}_t$ for each mixture frame $\bm{g}_t$. Prior to computing $\bm{c}_t$, $\bm{h}_\tau$ passes through a linear layer $\ell$, so that we learn a different representation for the SE task. The computations are below:
\begin{align}
    \alpha_{t,\tau} &= \frac{\exp{(\bm{g}_t^\top \bm{W} \bm{h}_\tau})}{\sum^{\wp}_{\phi=1} \exp{(\bm{g}_t^\top \bm{W} \bm{h}_\phi)}} \\
    \bm{c}_t &= \sum_{\tau=1}^\wp \alpha_{t,\tau} \cdot \ell (\bm{h}_\tau)
\end{align}
\noindent
Then, the context vector and SE-model embedding vector are concatenated (e.g., $[\bm{c}_t, \bm{g}_t]$) and passed to the decoder module. The SE-decoder module follows the network structure from \cite{schulze2020joint}. It consists of a linear layer with a $tanh(\cdot)$ activation function, two BLSTM layers, and a linear layer with ReLU activation. It outputs the estimated enhanced speech $|\hat{\bm{S}}|$. This estimated speech magnitude with noisy phase produce the estimated clean speech, i.e. $\hat{\bm{S}} = |\hat{\bm{S}}|e^{i\bm{\theta}^M}$. Since we are estimating two targets MOS and enhanced speech simultaneously, the unified model will learn different representations for these tasks. Thus both PMOS and SE models will learn their corresponding targets with perceptual feature sharing. We freeze the PMOS model while training this SE model.


\subsection{Joint-learning of PMOS and SE model}
\label{subsec:joint_model}
We also develop an approach that allows the PMOS and SE models to be jointly trained. Our joint-learning objective function uses a weighted average of a {time-domain} signal-approximation loss $\mathcal{L}_{sa}$ (from the SE model), the MSE of the magnitude spectrum $\mathcal{L}_{mse}$ (from the SE model) and the MSE of the MOS estimation $\mathcal{L}_{mos}$ (from the PMOS model). We compute the signal-approximation loss from the time-domain signal difference between the reference speech $s$ and enhanced speech $\hat{s}$. The overall loss function of our network is defined as below, using hyper-parameters $\lambda_1$ and $\lambda_2$ that control the impact of individual loss terms:
\begin{align}
    \mathcal{L} &= \lambda_1\left[\lambda_2\mathcal{L}_{mse} + (1-\lambda_2)\mathcal{L}_{sa}\right] + (1-\lambda_1)\mathcal{L}_{mos}
    \label{eq:loss}
\end{align}
\noindent
The model training order is as such. First, we train the PMOS model using $\mathcal{L}_{mos}$ (e.g. $\lambda_1 = 0$). Then we train the SE model using $\lambda_1 = 1$, while running the PMOS model in inference mode (e.g. it is held fixed). This is done to ensure that the trained PMOS model effectively encodes the key features in the embedding vector that are important to perceptual speech quality. Finally, we train both the models jointly (e.g. $0 < \lambda_1 < 1$) using $\mathcal{L}$ to further reduce any correctional differences between the true and estimated MOS in the PMOS model, and to increase the perceptual quality of the enhanced speech.
\begin{figure}[t!]
    \centering
    \includegraphics[width = 0.8\linewidth]{figs/quant_fig2.png}
    \caption{Quantization of a clean magnitude spectrum.}
    % \vspace{-2em}
    \label{fig:quant}
    % \vspace{-0.4cm}
\end{figure}
\subsection{Quantized Spectral Model}
\label{subsec:QSM}
% An external language model can integrate additional information regarding speech correlation which is helpful for improving enhancement performance. Typical LM is applied at the phoneme or word level and the performance of the LM depends on the text and its vocabulary. Additionally, parallel corpus of speech and text is a requirement for training which rules out a huge number of corpus from usage. 
%A language model (LM) serves as prior knowledge on acoustic input that constrains the alternative word (or phonetic) hypothesis during speech recognition by learning which sequences of words (or phonetics) are most likely to be spoken. LM predicts which words will follow on from the current words and with what probability. $\mathbb{N}$-gram LM is a widely used approach which estimates the probability of a given sequence of words $w_{1\cdots\Omega}$ within the assumption that the probability of word $w_\delta$ depends only on previous $(\mathbb{N}-1)$ words, and the probability can be expressed as: 
%\begin{align}
%    P(w_{1\cdots\Omega})=\prod_{w_\delta} P(w_\delta|w_{\delta-1}, w_{\delta-2},\cdots,w_{\delta-\mathbb{N}+1})
%\end{align}
%Compared to conventional ASR approaches, deep ASR systems model learn in-house LM \cite{yu2016automatic}; and they can be coupled with SE task~\cite{weninger2015speech, wang2020complex}. LM helps a SE model by predicting probability of next utterance, which is otherwise will be any utterance in the whole speech spectrum. However, deep LM typically require more data to achieve comparable results. Additionally, parallel corpus of speech and text is a requirement for training which rules out a huge number of raw audio collections from usage.
%Therefore, we adapt an alternative view of a LM from \cite{nayem2021towards}, where quantized t-f values are considered as word. 
From written and spoken language, we can determine the sequences of words that are most likely to occur. This knowledge is captured by a language model (LM) of an automatic speech recognition system which we can expressed as,
\begin{align}
    \hat{words}=\argmax_{words\in Language} \overbrace{P(input|words)}^{acoustic\;model} \overbrace{P(words)}^{language\;model}
\end{align}
%Here, the most likely word sequence, $\hat{words}$, is estimated by an acoustic model that calculates the probability of the input audio given the word sequence $words$, and by a language model that gives the likelihood of the word sequence. Hence, the LM predicts the probability of a sequence of words. 
The LM is useful in eliminating rare and grammatically incorrect word sequences, and it enhances the performance of ASR systems. In the case of speech enhancement, models learn spectral information within frames over time, but they often neglect the temporal correlations. Our approach, as proposed in \cite{nayem2021towards}, suggests incorporating a ``LM" to fuse temporal correlations and overcome this limitation. Therefore, we construct a bi-gram Quantized Spectral Model (QSM), which functions in a similar way to a language model (LM), in order to produce more realistic spectra. The QSM estimates the probability of spectral magnitudes along time for each frequency channel conditioned on its previous T-F spectral magnitude. %Range of T-F unit values is constrained in a signal approximating SE system and is far smaller than typical spoken language vocabulary size. As a result, the training time and computational resource requirement are quite small fo spectral LM.
On a reference speech corpora, we apply a normalization scaling function, $\mathcal{N}_{[o,r]}(\cdot)$, that normalizes the magnitude spectrogram and re-scales the range to $[0,r]$. Then a quantization function, $\mathcal{Q}_\chi(\cdot)$, converts the range constrained magnitude spectrogram into $\mathcal{D}$ number of bins that are $\chi$ steps apart. This produces quantized speech, i.e. $|S|^q = \mathcal{Q}_\chi\big(\mathcal{N}_{[0,r]}(|S|)\big)$. Fig.~\ref{fig:quant} shows an example of the original clean and quantized clean magnitude spectra, where $\chi=2$ for display purposes. Our proposed QSM has $\mathcal{D}$ spectral levels. We construct the QSM using quantized speech magnitudes from the clean speech corpora. The QSM is less likely to suffer from the out of vocabulary problem when the model parameters, $\chi$ and $r$, are adequately defined.

%\begin{figure}[tbh!]
%    \centering
%    \includegraphics[width = \linewidth]{IEEEtran/figs/fQSM.png}
%    \caption{Proposed Quantized Spectral Models (QSMs) for per-frequency-channel.}
%    % \vspace{-2em}
%    \label{fig:fQSM}
%    % \vspace{-0.4cm}
%\end{figure}

We compute per-frequency-channel QSMs along the time axis where each entry, $d$, refers to a quantization attenuation level. We then compute the transition probability between two time consecutive T-F units, $fQSM_f = P(d_{t+1,f}|d_{t,f})$. The probabilities are calculated by counting the level transitions, and then normalizing by the appropriate scalar. These probabilities are stored in the per-frequency-channel QSM resulting in a $F\times \mathcal{D}\times \mathcal{D}$ probability matrix. %Figure~\ref{fig:fQSM} shows proposed QSMs along per-frequency-channel. 
We re-evaluate the transition probabilities using Good-Turing smoothing~\cite{jurafskyMartin2009} to overcome the zero-probability problem in N-grams. Shallow fusion~\cite{gulcehre2015using} is a simple method to incorporate an external LM into an encoder-decoder model, and it produces better results compared to others. Hence, we use shallow fushion to combine our QSM and SE model based on log-linear interpolations at inference time. This is shown in the below equations:

\begin{align}
    P^{QSM}_f(|\hat{\bm{S}}_{:,f}|) &= \prod^T_{i=1} P(d_{i,f}|d_{i-1,f}) \\
    |\hat{\bm{S}}_{:,f}|^* = \argmax_{|\hat{\bm{S}}_{:,f}|} &\log P\big(|\hat{\bm{S}}_{:,f}| \big| |\bm{M}|\big) + \mu \log P^{QSM}_f\big(|\hat{\bm{S}}_{:,f}| \big)
    \label{eq:S_hat}
\end{align}
\noindent
Here $P^{QSM}_f$ denotes the transitional probability of QSM at frequency $f$, $P\big(|\hat{\bm{S}}_{:,f}| \big| |\bm{M}|\big)$ represents the estimated magnitude output of the LSTM layers of the SE decoder, and $\mu$ is a hyper-parameter that is tuned to maximize the performance on a development set. Note that we train our QSM in advance on a clean speech corpus and use it in inference mode during enhancement. The tunable parameter $\mu$ of (\ref{eq:S_hat}) is set to zero when we do not have a trained QSM. 





\section{Experiments}



\begin{table*}[t]
\small
\centering
\caption{\textbf{Evaluation of panel-prediction quality} on seen and unseen garment classes. M-L2: Mask L2 ; P-L2 : Panel L2; R-L2: Rotation L2; T-L2: Translation L2 . $\dagger$ represents orderless-LSTM.} %$*$ denotes the models without data-filtering.}
\setlength{\tabcolsep}{8pt}
\begin{tabular}{@{}c|ccccc|ccccc@{}}
\toprule
                                   & \multicolumn{5}{c|}{\textbf{Seen classes}}                                                                                                                                                                                                      & \multicolumn{5}{c}{\textbf{Unseen classes}}                                                                                                                                                                                                     \\ \cmidrule(l){2-11} 
\multirow{-2}{*}{\textbf{Methods}}& \textbf{P-L2}                     & \textbf{\# Panels}                    & \textbf{\# Edges}                     & \textbf{R-L2}                       & \textbf{T-L2}           & \textbf{P-L2}                     & \textbf{\# Panels}                    & \textbf{\# Edges}                     & \textbf{R-L2}                       & \textbf{T-L2}                     \\ \midrule
Baseline-I                 & 3.92                                    &  \bf 99.9\%                                    &\bf 100.0 \%                                    & 0.06                             & 0.117                                       & 6.61                                    & 94.6\%                                     &  95.4\%                                     & 0.09                                    & 0.21                                    \\
Baseline-II                                                   & 4.3                                   & 99.4\%                                   &   99.7\%                                        & 0.08                                   & 1.46                                                             & 8.1                                   & 89.3\%                                   & 90.3\%                                   & 121                                   & 1.25                                   \\
%Baseline-III                                                 & 3.91                                   & 99.9 \%                                  & 99.9 \%                                    & 0.06                                   & 0.05                                                            & 6.3                                   & 93.9  \%                                    & 94.2 \%                                    & 0.07                                   & 0.18                                   \\
LSTM                                                       & 2.71                                  & 99.8\%                                & 99.9\%                                & \bf 0.004                                 & 0.32                                                                 & 14.7                                  & 6.5\%                                 & 53.2\%                                & 0.17                                  & 6.75                                  \\
LSTM$^{\dagger}$                                                    & 2.87                                  & 99.4\%                                & 99.9\%                                & \textbf{0.004}                                 & 0.33                                                                   & 12.94                                 & 2.7\%                                 & 59.0\%                                & 0.16                                  & 7.18                                  \\
Neural-Tailor                                                      & \textbf{1.5}                                   & 99.7\%                                & 99.7\%                                & 0.04                                  & 1.46                                                         & 5.2                                   & 83.6\%                                & 87.3\%                                & 0.07                                  & 3.22                                  \\
% Neural-Tailor*                                                    & 1.53                                  & 98.8\%                                & 99.6\%                                & 0.04                                  & 1.45                                                        & 7.96                                  & 73.1\%                                & 80.5\%                                & 0.08                                  & 3.57                                  \\
% Neural-Tailor*                                                    & 1.6/1.95                              & 98.6/97.5\%                           & 99.8/99.2\%                           & 0.07/0.07                             & 2.2/2.5                                                             & 6.2/6.4                               & 81.6/75.2\%                           & 88.5/88.2\%                           & 0.08/0.10                             & 3.9/4.5                               \\ \midrule
\midrule
\textbf{Ours w/ Text}                      & \cellcolor[HTML]{FFFFDB}2.80  & \cellcolor[HTML]{FFFFDB}\textbf{99.9\%}  & \cellcolor[HTML]{FFFFDB}{99.9\%}  & \cellcolor[HTML]{FFFFDB}0.04  & \cellcolor[HTML]{FFFFDB}\textbf{0.04}    & \cellcolor[HTML]{FFFFDB}\textbf{4.20}  & \cellcolor[HTML]{FFFFDB}\textbf{99.9\%}  & \cellcolor[HTML]{FFFFDB}\textbf{99.8\%}  & \cellcolor[HTML]{FFFFDB}\textbf{0.05}  & \cellcolor[HTML]{FFFFDB}\textbf{0.05}  \\
\textbf{Ours w/ Sketch}                      & \cellcolor[HTML]{FFFFDB}2.91  & \cellcolor[HTML]{FFFFDB}\textbf{99.9\%}  & \cellcolor[HTML]{FFFFDB}{99.9\%}  & \cellcolor[HTML]{FFFFDB}0.05  & \cellcolor[HTML]{FFFFDB}{0.06}    & \cellcolor[HTML]{FFFFDB}\textbf{4.33}  & \cellcolor[HTML]{FFFFDB}\textbf{99.9\%}  & \cellcolor[HTML]{FFFFDB}\textbf{99.9\%}  & \cellcolor[HTML]{FFFFDB}\textbf{0.06}  & \cellcolor[HTML]{FFFFDB}\textbf{0.07}  \\
%\textbf{Ours w/ Overlap}                      & \cellcolor[HTML]{FFFFDB}2.80  & \cellcolor[HTML]{FFFFDB}\textbf{99.9\%}  & \cellcolor[HTML]{FFFFDB}\textbf{99.9\%}  & \cellcolor[HTML]{FFFFDB}0.04  & \cellcolor[HTML]{FFFFDB}\textbf{0.04}    & \cellcolor[HTML]{FFFFDB}\textbf{4.20}  & \cellcolor[HTML]{FFFFDB}\textbf{99.9\%}  & \cellcolor[HTML]{FFFFDB}\textbf{99.8\%}  & \cellcolor[HTML]{FFFFDB}\textbf{0.05}  & \cellcolor[HTML]{FFFFDB}\textbf{0.05}  \\

 \bottomrule 
\end{tabular}

\label{tab:main_tab}
\end{table*}

\begin{figure*}[t]
    \centering
    \includegraphics [width=\linewidth]{img/fig4_v2.pdf}
    \caption{
    Comparing our method with NeuralTailor (\texttt{NT})
    on the unseen garment classes: ‘jacket sleeveless’, ‘skirt waistband’, ‘wb jumpsuit sleeveless’ and ‘dress’.
    {\em Metric}: the average Vertex L2 error.
    {\em Ground-truth}: dash thin lines.
    % Comparison of our method with NeuralTailor (\texttt{NT}) \cite{korosteleva2022neuraltailor} on the unseen garments from ‘jacket sleeveless’, ‘skirt waistband’, ‘wb jumpsuit sleeveless’ and ‘dress’ categories of the dataset \cite{korosteleva2021generating}. The numbers show the average Vertex L2 for the shown exemplars. The colored panels indicate predicted panels, and the dash thin lines indicate the ground-truth panels.
    }
    \label{fig:main_viz}
    % % \vspace{-0.2in}
\end{figure*}
\noindent \textbf{Dataset}
We evaluate the PersonalTailor on the 3D garments dataset with sewing patterns from \cite{korosteleva2021generating}. It contains 19 garment classes with $22,000$ 3D garment-sewing pattern pairs in total, covering the variations in t-shirts, jackets, pants, skirts, jumpsuits and dresses. 
There are 10627/722/729 samples for train/val/test
% The number of samples in train/val/test is 10627/722/729 
in the filtered version. 
Following NeuralTailor \cite{korosteleva2022neuraltailor}, the classes of panels are designed based on the panel's role and location around the body across all garment classes. For example, panels located around the back of human body are grouped in the ``back panels'' class. We follow the standard panel labels, data filtering and train/test splits of garment classes. There are 7 garment classes unseen to training and used for evaluation. 



\noindent \textbf{Evaluation metrics}
We use the same evaluation metrics as in \cite{korosteleva2022neuraltailor}. 
We evaluate the accuracy in predicting the number of panels within
every pattern (\ie, \#Panels) and the number of edges within every panel
(\ie, \#Edges). To estimate the quality of panel shape predictions, we use the average distance (L2 norm) between the vertices of predicted and ground
truth panels with curvature coordinates converted to panel space,
acting as panel masks in this comparison (Panel L2). Similarly,
we report L2 normalized differences of predicted panel rotations
(Rot L2) and translations (Transl L2) with the ground truth. The
quality of predicted stitching information is measured by a mean
precision (Precision) and recall (Recall) of predicted stitches.




\noindent \textbf{Implementation details}
For language encoding, we use CLIP \cite{radford2021learning} pretrained encoder. 
For sketch encoding, we use SketchRNN \cite{yang2021sketchgnn}. 
We follow the training scheme as \cite{korosteleva2022neuraltailor}.
We set the maximum number of panels $M=23$.
There are $g=12/8$ garment classes in training/testing set. We set the feature dimension for text and the global embedding $D = 512$. % is set as 512. 
% The number of codes $K$ in codebook is set as 2000. 
% For Stage-1 training, 
Our model is trained for 250 epochs using Adam optimizer with learning rate of $10e-5$ and batch size of 15. 
% For Stage-2 training, our model
The stitching GNN is trained for 50 epochs using SGD optimizer with learning rate of $10e-4$.
%
% Specifically, %We train our stitch prediction network
% it is trained by the edge vectors from ground truth panels and edges outputted by the prediction module under the text prompt scenario. 
%
Specifically, %We train our stitch prediction network
it is trained by the predicted edges.
%
The inference threshold for panel mask head is set as 0.5 and top-$k$ is set as 14. The code will be made publicly available upon acceptance.
% Our model is implemented in Pytorch and trained with batch size of 15 on a single NVIDIA 2080GTX GPU.

% \begin{figure}[t]
%     \centering
%     \includegraphics[scale=0.35]{img/final_model.png}
%     \caption{\textbf{Examples of garment personalization} 
%     % Based on user input via sketch/text prompt, we illustrate the customization 
%     %
%     % Garment personalization 
%     from (a) Pant Straight sides to Skirt 4 Panels, (b) Skirt 4 Panels to Pant Straight Sides, (c) Dress Sleeveless to Dress Waistband Sleeveless, (d) Dress Waistband Sleeveless to Dress Sleeveless respectively. }
%     \label{fig:customized}
% \end{figure}


\subsection{Personalized pattern design evaluation}
\noindent \textbf{Setting}  To quantitatively evaluate the performance of personalization,
% based on the user input prompts (\ie, text and sketch), 
we conduct 6 garment class transfer cases (case 1\&2: Tee $\leftrightarrow$ Jacket, case 3\&4: Jumpsuit$\leftrightarrow$ Dress, case 5\&6: Jacket $\leftrightarrow$ Jacket Sleeveless)
under both text and sketch prompt. We define the \textit{Panel IOU}  metric as the mean of panel-wise IOUs between predicted panels of the source garment class and the average panels of the target garment class. Formally, we use the desired input prompts to transfer the source garment class to the target garment class. Then we compare the \textit{Panel IOU} before and after personalization against the target class panel attributes. % in personalized query. 

\noindent \textbf{Baseline} Due to lacking of competing works or open-source alternatives, % in the literature, 
% we created our own baselines. More specifically, 
we created a personalization baseline by removing the prompt embedding and cross-modal embedding module (referred as \texttt{baseline}) from our PersonalTailor. 

\noindent \textbf{Quantitative results} 
The personalization results are reported in Tab.~\ref{tab:personalization}. It can be observed that (1) our method can achieve an average panel IOU of $53\%$ over 6 cases by text and $52\%$ by sketch, outperforming the baseline method by $13\%$/$16\%$ respectively. This is because the decoder of the baseline is randomly initialized lacking the semantic and structural information of the panel attributes. Thus, it has less personalization ability. (2) Our method yields a larger gain over the baseline before and after personalization under both text ($22\%$ \vs $16\%$) and sketch prompts  $21\%$ \vs $17\%$). 
This verifies our superior personalization ability.
% of our model design.
% This showcase that our method has better personalization ability.



\noindent \textbf{Qualitative/visual results}
We show the personalized garment transfer process of case 1\&2 by text prompt (target garment’s panel classes) in Fig.~\ref{fig:editing} (a,b), case 3\&4 by sketch prompt (target garment’s average panel silhouettes) in Fig.~\ref{fig:editing} (c,d).  Overall, it is shown that our method can support panel shape editing with complex topology changes from one garment class to another using personalized prompts, even for those unseen during training, \eg, Jumpsuit and Dress. Beyond topology change, it also supports adding 
new panels (Fig.~\ref{fig:teaser} (b)), removing panels (Fig.~\ref{fig:teaser} (c)), and creating a
new design % that is not included in the dataset 
(Fig.~\ref{fig:teaser} (e)).
We also observe that our method can achieve fine-grained panel shape editing by using sketch prompts. As shown in Fig.~\ref{fig:sketch_edit}, given a 3D jacket and different users' sketch prompts, our method can produce the panels aligned with the sketch's shape while preserving the intrinsic structure of the 3D shape. 
% We provide more illustration of personalized garment transfer in Fig.~\ref{fig:customized}. 
%\paragraph{Controllable garment personalization} \annie{figure 4 and 1} We demonstrate our framework is capable of controllable garment personalization. Given a source 3D garment, our PersonalTailor can accurately edit the 2D sewing panel shapes. Additionally, our model can also perform personalization by design choice during inference. From the results in Fig~\ref{fig:editing}, we can observe that PersonalTailor supports controllable editing on the 3D garment shape and topology with
%preserved intrinsic structure. And PerosnalTailor can edit garments with significant shape variations or transfer garments from one category to another by editing on the 2D panels via mask instructions, even for some categories that are not shown in the training set. As an added benefit, our network facilitates both text and sketch based editing to give more expressive power to the users.








% \begin{figure*}
%     \centering
%     \includegraphics[scale=0.43]{img/PTailor_main.png}
%     \caption{\textbf{Examples of PersonalTailor's output (unrefined)}
%     It is shown that our method works similarly well with text and sketch/visual prompts.
%     % As seen from the figure, both text and sketch prompt predicts similar mask prediction with textual prompt marginally better. 
%     }
%     \label{fig:ptailor_main}
% \end{figure*}

\begin{figure}[t]

    \centering
    \includegraphics[scale=0.32]{img/new_fig_6.png}
    % % \vspace{-0.1in}
    \caption{\ann{Illustration of fine-grained panel editing by sketch. Given a 3D garment and different users' sketches, our method can support fine-grained panel shape editing while preserving the intrinsic structure of the 3D garment.}}
  \label{fig:sketch_edit}
  % % \vspace{-0.2in}
\end{figure}
\subsection{Standard pattern design evaluation}

\noindent \textbf{Setting} In this setting, we evaluate the standard (non-personalized) pattern design.  
% We test the open-set scenario where the training and testing garment classes do not overlap, \ie $g_{train} \cap g_{test} = \phi$, whilst the panel classes may overlap $p_{train} \cap p_{test} \neq \phi$. 
We follow the same setting and dataset splits as proposed in NeuralTailor \cite{korosteleva2022neuraltailor}. More specifically, we evaluate on two settings:
\ann{(1) Training with seen classes and evaluating on unseen data of those seen classes, \ie closed-set setting; 
(2) Training with seen classes and evaluating on unseen classes, \ie open-set setting.} 

\noindent \textbf{Competitors} We considered the following competitors for comparison: 
(a) a competitive garment pattern prediction method Neural Tailor on filtered data \cite{korosteleva2021generating}, 
(c) an LSTM \cite{graves2012long} based garment pattern prediction ,
(d) an orderless LSTM \cite{yazici2020orderless} based garment pattern prediction, 
(e) \textit{Baseline-I} we created using GCN encoder and CNN decoder, 
(f) \textit{Baseline-II} we created using PointTransformer encoder \cite{zhao2021point} and Transformer decoder \cite{vaswani2017attention} with random initialized queries. %(g) a baseline termed as \textit{Baseline-III} created using GCN encoder and Transformer decoder without language. We evaluate all of the above techniques in a similar setup. 

\noindent \textbf{Results} The results are reported in Tab.~\ref{tab:main_tab}. 
\textbf{(I) Closed-set settings:} 
% Under this setting, t
The performance of some metrics (\eg, Edges/Panels) has almost saturated.
% by different methods.
In particular, NeuralTailor has the best Panel L2 result, indicating that learning vertex is better in the closed set than mask prediction.
However, our PersonalTailor achieves the best translation prediction, suggesting the importance of global information.
%
% In this settings, our PersonalTailor has near-to 100\% edge and panel accuracy indicating the superior model design. It is interesting to note that PersonalTailor has almost 8x better translation metric indicating the importance of global information. NeuralTailor has the best Panel L2 indicating that learning vertex has better generalization in the closed set than mask prediction.
\textbf{(II) Open-set settings:} %In contrast to the closed-set setting, 
Our method achieves the state-of-the-art in all the metrics, surpassing the competitors by a large margin. This indicates the superiority of \ann{personalized prompts} in open-set generalization.
% of class agnostic masks. %Additionally, we have tested our method on the non-overlapping setting which can be found in the supplementary file.\annie{check}.



% % \vspace{-0.1in}
\paragraph{Qualitative results} 
% We first present qualitative results for our PersonalTailor in Fig.~\ref{fig:main_viz},\ref{fig:ptailor_main} whilst comparing to our closest competitor NeuralTailor \cite{korosteleva2022neuraltailor} with unseen garments. 
We present qualitative results on unseen garments. 
It can be observed from Fig.~\ref{fig:main_viz} that our PersonalTailor predicts more accurate panels over the prior art NeuralTailor \cite{korosteleva2022neuraltailor}, due to our multi-modal embedding-based design enabling the prompt bring in additional semantic information about the garment's shape.  
We also show in Fig. \ref{fig:ptailor_main} that PersonalTailor works well with both text and sketch prompts.
% Our findings are also reflected in the Panel L2 metric for each garment where our model is superior in open-set scenario. %Additionaly, our model can also handle overlapping garment problem as seen in Fig~\ref{fig:overlap} where our model has significant better mask prediction than NeuralTailor.

\begin{figure}[t]
    \centering
    \includegraphics[scale=0.35]{img/final_model.png}
    \caption{\textbf{Examples of garment personalization} 
    % Based on user input via sketch/text prompt, we illustrate the customization 
    %
    % Garment personalization 
    from (a) Pant Straight sides to Skirt 4 Panels, (b) Skirt 4 Panels to Pant Straight Sides, (c) Dress Sleeveless to Dress Waistband Sleeveless, (d) Dress Waistband Sleeveless to Dress Sleeveless respectively. }
    \label{fig:customized}
\end{figure}





\subsection{Stitching prediction} 

We evaluate the stitching module design
by comparing with NeuralTailor \cite{korosteleva2022neuraltailor}.
As shown in Tab.~\ref{tab:stitch}, 
our GNN based design is clearly superior particularly
for unseen garment classes.
This validates the efficacy of our exploiting the structural information of panels.
%
We also show that using the edge vectors of ground-truth panels
for training is inferior than using the predicted for both methods,
as the former introduces some inconsistency with model inference.


% our stitch prediction network trained on the edges predictions outperforms the second best $3.3/0.5$ and $9.0/0.3$ points on Precision and Recall for Seen Type and Unseen Type respectively.} And the models  trained by edge prediction data (OurPred, NeuralTailorPred) perform better than that of ground truth edge vectors  (OurGT, NeuralTailorGT). The noised prediction data empowers the generalization ability of the networks.


\begin{table}[h]
\caption{\textbf{Evaluation of stitching prediction} on both seen and unseen garment classes.
$^*$: Trained by the edges of GT panels.
}
\label{tab:stitch}
\resizebox{\columnwidth}{!}{
\begin{tabular}{c|cc|cc}
\hline
\multirow{2}{*}{\textbf{Method}}                 & \multicolumn{2}{c|}{\textbf{Seen classes}}     & \multicolumn{2}{c}{\textbf{Unseen classes}}   \\ \cline{2-5}
                & \textbf{Precision}       & \textbf{Recall}          & \textbf{Precision}       & \textbf{Recall}          \\ \hline
NeuralTailor$^*$ \cite{korosteleva2022neuraltailor}  & 96.6\%          & 88.6\%          & 75.3\%          & 60.6\%          \\ \hline
NeuralTailor \cite{korosteleva2022neuraltailor} & 96.3\%          & 99.4\%          & 74.7\%          & 83.9\%          \\ \hline
Ours$^*$            & 74.9\%          & 65.0\%          & 76.8\%          & 73.0\%          \\ \hline
Ours          & \cellcolor[HTML]{F6DDCC}\textbf{99.9\%} & \cellcolor[HTML]{F6DDCC}\textbf{99.9\%} & \cellcolor[HTML]{F6DDCC}\textbf{85.8\%} & \cellcolor[HTML]{F6DDCC}\textbf{84.2\%} \\ \hline
\end{tabular}
}
\label{stitchingres}
\end{table}






\begin{figure*}
    \centering
    \includegraphics[scale=0.43]{img/PTailor_main.png}
    \caption{\textbf{Examples of PersonalTailor's output (unrefined)}
    It is shown that our method works similarly well with text and sketch/visual prompts.
    % As seen from the figure, both text and sketch prompt predicts similar mask prediction with textual prompt marginally better. 
    }
    \label{fig:ptailor_main}
\end{figure*}
\subsection{Ablation studies}
We conduct ablation studies to provide insights into each
component with text prompt. \saura{More in-depth ablations are provided in the \texttt{Supplementary}.}

\noindent \textbf{Impact of cross-modal alignment}
We evaluate the importance of cross-modal alignment.
To that end, we consider two down-stripped designs:
{\bf(1)} Removing the cross modal local association block (CMLA);
{\bf(2)} Removing the multi-modal transformer (MMT).
As shown in Tab.~\ref{tab:mmemb},
we find that without CMLA, a significant drop in the Panel L2 metric 
occurs, suggesting the importance of resolving the domain gap between the point cloud and semantic information from the prompt.
It is also shown that MMT is useful in terms of 
fusing information.

\begin{table}[h]
\centering
\small
\caption{Impact of multi-modal embedding.
CMLA: Cross modal local association;
MMT: Multi-modal transformer.
}
\label{tab:mmemb}
\setlength{\tabcolsep}{3pt}
\begin{tabular}{c|cc|cc}
\hline
\multirow{2}{*}{\textbf{Model}} & \multicolumn{2}{c|}{\textbf{Seen}}     & \multicolumn{2}{c}{\textbf{Unseen}}    \\ \cline{2-5} 
                                & \textbf{Panel L2} & \textbf{\# panels} & \textbf{Panel L2} & \textbf{\# panels} \\ \hline
 Ours                            & \cellcolor[HTML]{F6DDCC}\textbf{2.80}                & \cellcolor[HTML]{F6DDCC} \textbf{99.9\%}                 & \cellcolor[HTML]{F6DDCC}\textbf{4.20}               & \cellcolor[HTML]{F6DDCC}\textbf{99.9\%}  \\ \hline
w/o CMLA                         & 5.51               & 99.8\%                & 6.5                & 99.2\%                   \\
w/o MMT                         & 4.42                &  99.9\%                  & 5.32     
& 99.6\%                  \\ 
\hline


% \hline
 % \\ 
\hline
% w/o both                         & 41                & 42                 & 43   
% & 44 \\

\end{tabular}
% % \vspace{-0.2in}
\end{table}
% as the raw prompt features lack the power of expressively to predict the panel mask.

% in the pattern prediction performance in Tab.~\ref{tab:mmemb}. We first removed the cross modal local association block (CMLA) from our model and observed a significant drop of 2.7\%/2.3\% in the Panel L2 metric for Seen and Unseen setting respectively. This signifies the importance of resolving the domain gap between the point cloud and semantic information from the prompt. Once we remove the multi-modal transformer (MMT) we observe a drop of 1.6\%/1.1\% in the Panel L2 metric for Seen and Unseen setting respectively. This indicates that the raw prompt features lack the power of expressivity to predict the panel mask.



\noindent \textbf{Impact of point-cloud encoders} We evaluate our point-cloud encoder design including 
global encoder (GE) and local encoder (LE).
As shown in Tab.~\ref{tab:ptc}, we see that 
{\bf(1)} GE is useful particularly for unseen garment classes.
This is because global information plays an important role in estimating the cutting pattern and guiding the panel-mask decoder prediction.
% during cross-attention.
{\bf (2)} LE brings in further performance gain
due to extra part-level information introduced.
%
% ({\bf \color{red} W/O LE giving better Panel L2 on Unseen? Double check!})
% experimentally prove the necessity of our point cloud encoders in Tab.~\ref{tab:ptc}. Our variant without global encoder (GE) performs the worst by a margin of 0.6\%/1.3\% in Panel L2 metric for Seen and Unseen setting respectively. This is expected as global information plays an important role in estimating the cutting pattern and also guides the panel-mask decoder prediction using cross-attention. It additionally also affects the panel mask accuracy shown by a performance drop of 4.7\% in \# panels metric for Unseen setting  without global encoder (GE). The variant without local encoder (LE) however observes a slight drop of 0.5\%/0.7\% indicating fine-grained local structure is also important for the overall performance improvement. 






\begin{table}[t]
\centering
\small
\setlength{\tabcolsep}{3pt}
\caption{Impact of global and local point cloud encoders.
LE: Local Encoder; GE: Global Encoder.}
\label{tab:ptc}
\begin{tabular}{cc|cc|cc}
\hline
\multicolumn{2}{c|}{\textbf{Model} }      & \multicolumn{2}{c|}{\textbf{Seen}} & \multicolumn{2}{c}{\textbf{Unseen}} \\
\hline
 \textbf{LE}     & \textbf{GE}                       & \textbf{Panel L2}    & \textbf{\# Panels}      & \textbf{Panel L2}     & \textbf{\# Panels}     \\ \hline
\xmark          & \cmark                         & 3.20          & 99.4\%            & 4.90               & 95.2\%            \\
\cmark         & \xmark         & 3.30          & 99.9\%            & 5.51               & 96.3\%      \\
\hline
 \cmark        & \cmark    & \cellcolor[HTML]{F6DDCC}\textbf{2.80} & \cellcolor[HTML]{F6DDCC}\textbf{99.9\%} & \cellcolor[HTML]{F6DDCC}\textbf{4.20}  & \cellcolor[HTML]{F6DDCC}\textbf{99.9\%} \\ \hline
\end{tabular}
\end{table}



\noindent \textbf{Design choice of panel decoder}
We evaluate more choices of panel mask decoder
including (1) a CNN and (2) a Transformer decoder without positional embedding (Trans. w/o PE). 
We observe in Tab.~\ref{tab:dec} that (1) CNN is least performing for instruction based mask prediction as it loses out on the panel prediction performance due to lack of interaction among the panels. 
(2) Positional encoding is important as it predicts position specific masks.

\begin{table}[h]
\centering
\setlength{\tabcolsep}{3pt}
\caption{Ablation on the design choice of decoder.
Trans.: Transformer; PE: Positional Encoding.
}
\label{tab:dec}
\begin{tabular}{c|cc|cc}
\hline
\multirow{2}{*}{\textbf{Model}}       & \multicolumn{2}{c|}{\textbf{Seen}} & \multicolumn{2}{c}{\textbf{Unseen}} \\ \cline{2-5} 
                             & \textbf{Panel L2}    & \textbf{\# Panels}      & \textbf{Panel L2}     & \textbf{\# Panels}     \\ \hline
CNN                          & 5.49          & 93.2\%          & 6.92           & 91.7\%          \\
Trans. w/o PE        & 3.30          & 99.9\%            & 5.61           & 96.1\%    \\
\hline
Ours & \cellcolor[HTML]{F6DDCC}\textbf{2.80} & \cellcolor[HTML]{F6DDCC}\textbf{99.9\%} & \cellcolor[HTML]{F6DDCC}\textbf{4.20}  & \cellcolor[HTML]{F6DDCC}\textbf{99.9\%} \\ \hline
\end{tabular}
\end{table}

% To evaluate the expressive power of our panel mask decoder, we compare it against a CNN based baseline (CNN) and a vanilla transformer decoder without positional embedding (Trans. w/o PE). From the results in Tab.~\ref{tab:dec}, it is clear that the CNN based baseline  is not suitable for instruction based mask prediction as it loses out on the panel prediction performance due to lack of self-attention among the panels. This is improved by 2.7\%/2.7\% higher Panel L2 metric with the transformer decoder under Seen and Unseen settings respectively.  Besides, positional encoding is important as it predicts position specific masks which is reflected in the drop of 1.4\%/0.5\% in overall performance without positional encoding under Seen and Unseen settings.

\subsection{In-the-wild garment pattern design}
For more extensive evaluation, we qualitatively test on the garment captures from DeepFashion3D dataset \cite{zhu2020deep}.
We observe in Fig.~\ref{fig:wild} that our model makes better panel predictions than NeuralTailor \cite{korosteleva2022neuraltailor}. For example, NeuralTailor fails to perceive the sleeves
with the T-shirt (unseen to model training), whilst our model succeeds.
In the case of jeans, our model gives better panel uniformity than \cite{korosteleva2022neuraltailor}.
% We observe in Fig.~\ref{fig:wild} that our model makes more correct panel predictions and good guesses about the panel structure than NeuralTailor \cite{korosteleva2022neuraltailor}. For example, for garments not seen in the dataset, like the t-shirt which has sleeves, was not predicted by \cite{korosteleva2022neuraltailor} but succesfully predicted by our approach. In the case of jeans, our model has better panel uniformity than \cite{korosteleva2022neuraltailor} highlighting the effectivity of our model design. The quality of prediction on the in-the-wild garment scans can be improved further and thus bridging this sim-to-real gap is part of our future work.
\begin{figure}[h]
    \centering
    \includegraphics[scale=0.28]{img/3rd_party.png}
    \caption{\textbf{In-the-wild garment evaluation} Sewing patterns predicted by NeuralTailor \cite{korosteleva2022neuraltailor} and our PersonalTailor on examples from DeepFashion3D \cite{zhu2020deep}.}
    \label{fig:wild}
\end{figure}



\section{Human Evaluation}

In additional to benchmark based assessment,
we further provide human evaluation with a thoughtful user study.
In particular, we approached 20 professional tailors to request their 
preference on the significance of personalizing garment pattern design.
In particular, we asked them two questions: 
(1) If garment personalization is necessary? 
(2) Which prompt (text or sketch) is preferred?
As shown in Fig.~\ref{fig:hstud1}, 80\% of tailors consider 
automated personalization to be useful, as it saves them time in production and the cost of business.
Besides, 40\% prefer sketch over text for instruction,
whilst 5\% are concerned with sketch to be only for professional designers.
We also collected the tailors feedback on the functions of (a) adding new garment panels, (b) removing garment panels, (c) changing dress topology, and (d) creating new designs. As shown in Fig.~\ref{fig:hstud2}, removing garment panels is most popular, which is supported by our proposed method. 

% To evaluate the performance in a human perceptive level, we conduct thoughtful user studies in this section. Human subjects \ie professional Tailors evaluation is conducted to investigate the usefullness of customized garment pattern design. We have conducted this experiments on twenty human subjects and reported the scores in percentages out of hundred. As observed from Fig~\ref{fig:hstud1}(a), majority of the tailors prefer automated personalization to save time of production and cost of business. It is also interesting to note from Fig~\ref{fig:hstud1}(b) that majority of the tailors prefer sketch as a medium of customer input for the ease of production. However, 5\% of the tailors also raised concerns on the fact that sketch is only for professional designers and not so customer friendly. We also collected tailors feedback on the most requested mode of personalization among (a) adding new garment panels (b) removing panels (c) changing dress topology and (d) creating new designs. From the results in Fig~\ref{fig:hstud2}, it can be observed that majority of customers prefer removing garment panels which is also supported by our proposed solution. 

\begin{figure}[h]
    \centering
    \includegraphics[scale=0.38]{img/human_eval.png}
    % % \vspace{-0.1in}
    \caption{\textbf{Votes on garment personalization.}}
    \label{fig:hstud1}
    % % \vspace{-0.2in}
\end{figure}

\begin{figure}
    \centering
    \includegraphics[scale=0.23]{img/human_study_2.png}
    \caption{\textbf{Votes on garment personalization functions.}}
    \label{fig:hstud2}
\end{figure}

\section{Limitations and Future Work}
We presented a personalized 2D pattern design method for 3D garments,
featuring editing capabilities. Our model is 2D panel-aware, requires no panel annotation, and leverages the Transformer architecture to
form globally coherent 2D patterns of varied topology. The network was designed to allow editing at an interactive rate, where, as demonstrated, the user can interact with the model using a simple instruction (refer to Fig 2). However, one limitation is a relatively small set of panels and
garments, lacking multiple test domains for evaluation. Our mask based panel design cannot model complicated 2D patterns like pleats or darts which is another limitation.  As this is an under-studied area, many challenges (3D optimization, fitting to different body structures etc.) still exist for future work.


\textbf{{\huge Appendix}}

\appendix

\section{Appendix Outline}
These appendices provide details about the OpenRooms FF dataset (Appendix~\ref{B}), details about direct lighting (Appendix~\ref{C}) and analysis of lighting estimation results (Appendix~\ref{D}), view synthesis applications (Appendix~\ref{E}), additional implementation details (Appendix~\ref{F}), and additional experimental results (Appendix~\ref{G}).

\section{OpenRooms FF dataset}\label{B}
We created a dataset for multi-view inverse rendering called OpenRooms Forward Facing (Openrooms FF) dataset. Openrooms FF is an extension of the existing single-view inverse rendering dataset, OpenRooms~\cite{openrooms2021}, and most of resources to build the dataset are provided by the authors of OpenRooms~\cite{openrooms2021}, including data sources and creation tools. The materials, however, were unavailable due to the licensing issue, so we had to purchase materials from Adobe Stock~\cite{adobestock} except for 200 materials that were not found from Adobe Stock; instead, we replace them with other similar materials. We selected 23,618 images from the OpenRooms dataset by filtering out the images in which the camera looks at a wall or window, lacks textures in the scene, or object is too close to the camera. Then, we rendered forward facing multi-view images of 3 $\times$ 3 arrays by moving camera in eight directions: up, right up, right, right down, down, left down, and left, left up using the OptiX-based renderer~\cite{optixrenderer}. The baseline was set proportionally to the average depth of the scene to observe the change in the specular radiance. See Fig.~\ref{fig:33} for a multi-view images sample. As a result, a total of 212,562 (9 $\times$ 23,618) images were created and 27,000 (9 $\times$ 3000) images were separated into test dataset. OpenRooms FF consists of HDR RGB images, diffuse albedo images, roughness images, normal maps, binary masks, depth maps, per-pixel environment maps. We rendered images at 640 $\times$ 480 resolution but resized to 320 $\times$ 240 with bilinear interpolation for the training/test. The OpenRooms FF is summarized in Tab.~\ref{tab:openroomsFF}.


\begin{figure}[ht]
  \centering
  \includegraphics[width=\linewidth]{figure/supp_33.png}
   \caption{Sample of forward facing multi-view images in OpenRooms FF.}
   \label{fig:33}
\end{figure}

\begin{table}[ht]
\footnotesize
\centering
\begin{tabular}{|l|c|c|}
\hline
& Dataset & Training / Test \\
\hline
HDR RGB & 640 $\times$ 480  & 320 $\times$ 240 \\
Diffuse Albedo & 640 $\times$ 480  & 320 $\times$ 240 \\
Roughness & 640 $\times$ 480  & 320 $\times$ 240 \\
Normal & 640 $\times$ 480  & 320 $\times$ 240 \\
Mask & 640 $\times$ 480  & 320 $\times$ 240 \\
Depth & 640 $\times$ 480  & Not used \\
per-pixel DL & 40 $\times$ 30 $\times$ 32 $\times$ 16  & 40 $\times$ 30 $\times$ 16 $\times$ 8 \\
per-pixel SVL & 160 $\times$ 120 $\times$ 32 $\times$ 16  & 160 $\times$ 120 $\times$ 16 $\times$ 8 \\
\hline
\end{tabular}
\caption{Data type and resolution of OpenRooms FF. Spatially-varying lighting (SVL) has a spatial resolution of 160 $\times$ 120 and an angular resolution of 32 $\times$ 16.}
\label{tab:openroomsFF}
\end{table}


\section{Direct Lighting Details}\label{C}
Since the intensity($\boldsymbol{\eta}_s$) of incident direct lighting is the intensity of the light source, it is unrelated to pixel location. Thus we use global intensities ${\boldsymbol{\eta}_s}$ rather than per-pixel intensities. Instead, per-pixel visibility ${\mu_s \in \mathbb{R}}$ was used to account for occlusion. To enhance the dynamic range of the SG lobes, we use the non-linear transformation~\cite{cis2020}. The ablation study results for $S_D$ in SVSGs of incident direct lighting are shown in Tab.~\ref{tab:abl_SD}. Please see \cref{eqn:eq_lossDL} for $\mathcal{L}_{\text{reg}}$. Direct lighting performance improved as $S_D$ increased, but GPU Memory also increased. We chose $S_D=3$ considering its performance and GPU usage. Fig.~\ref{fig:DL_example} shows the incident(SVSGs) / exitant($\Tilde{\mathrm{V}}_\text{DL}$) direct lighting estimation results. SVSGs generally performed better because $\Tilde{\mathrm{V}}_\text{DL}$ estimates 3D volume, while SVSGs directly estimates 2D per-pixel environment map(E). Also, even though the consistency between them is not considered, since they are trained with the same ground truth(GT), they are consistent enough as shown in the Fig.~\ref{fig:DL_example}.



\begin{table}[ht]
\footnotesize
\centering
\begin{tabular}{|c|c|c|c|}
        \hline
         $S_D$ & si-MSE & {$\mathcal{L}_{\text{reg}}$} & GPU Memory(GB). \\
         \hline
        1 & 0.106 & 0.136 & 10.8\\ 
        \hline
        2 & 0.103 &0.127 & 11.26\\
        \hline
        \textbf{3} & \textbf{0.101} & \textbf{0.092} & \textbf{12.72}\\
        \hline
        4 & 0.101 & 0.081 & 13.43\\
        \hline
        6 & 0.100 & 0.061 & 14.94\\
        \hline
\end{tabular}
\caption{The ablation study results for $S_D$ in SVSGs.}
\label{tab:abl_SD}
\end{table}


\begin{figure*}[ht]
  \centering
  \includegraphics[width=\linewidth]{figure/DL_example.pdf}
   \caption{Direct lighting environment map~($16\times8\times3$) estimation results for OpenRooms FF.} 
   \label{fig:DL_example}
\end{figure*}


\section{Analysis of Lighting Estimation Results}\label{D}
We have analyzed spatially-varying lighting quality in detail. Since the SVLNet implementation is quite memory-hungry, the resolution of our $\Tilde{\mathrm{V}}_\text{SVL}$ is $128^3$, which is low compared to the image resolution ($320\times240$ ). Also, because the field-of-view of our camera setup is limited, the lighting of the out-of-view area must rely on context inference about the dataset. Fig.~\ref{fig:L_analysis} shows the per-pixel lighting estimation results for the OpenRooms FF test scene. In the Fig.~\ref{fig:L_analysis}, our estimation approximates the overall outline of the GT better than Li~\etal\cite{cis2020} , but fails to mimic the high frequency details of the GT due to limitations in resolution and field-of-view.


\begin{figure*}[ht]
  \centering
  \includegraphics[width=\linewidth]{figure/L_analysis.pdf}
   \caption{Per-pixel environment map~($32\times16\times3$) estimation results for OpenRooms FF.}
   \label{fig:L_analysis}
\end{figure*}



\section{View Synthesis}\label{E}
While image-based rendering(IBR) can perform view interpolation excellently, the view-dependent effect of highly specular objects, such as chrome spheres, is difficult to reproduce using IBR. Physically-based rendering(PBR) can handle this view-dependent effect realistically, but PBR requires scene material, geometry, and spatially-varying lighting that is difficult to obtain in the real-world. Because MAIR can perform accurate inverse rendering in real-world scenes, and can be easily applied to existing view synthesis methods with multi-view images, we can take advantage of IBR and PBR. The view synthesis result of the scene with chrome sphere inserted is in the accompanied video. This application consists of two steps: (1) background rendering with NeRF~\cite{nerf}, and (2) object and mask rendering with our renderer. We render the shadow of an object in all images and we train NeRF with these images. Background including shadow in novel view is rendered with NeRF, and chrome sphere in novel view is rendered with our lighting and renderer. Among the variants of NeRF, we use DirectVoxGO~\cite{dvgo} for fast training.

 
\section{Implementation details}\label{F}
\noindent{\bf Training and architecture details.} Our experiments were conducted with 8 NVIDIA RTX A5000 (24GB). In training, we use Adam optimizer, and the binary mask image $(\mathrm{M}_o, \mathrm{M}_l)$. $\mathrm{M}_o \in \mathbb{R}^{H \times W}$ is mask on pixels of valid materials, and $\mathrm{M}_l \in \mathbb{R}^{H \times W}$ is mask on pixels of valid materials and area lighting. The binary mask image is included in the OpenRooms FF and is used only for training. First, we define masked L1 angular error function ($g_1$), masked MSE function ($g_2$), masked scale invariant MSE function ($g_3$), masked scale invariant $\log$ space MSE function ($g_4$), and regularization function ($g_5$) as follows.

\small
\begin{align}
g_1(A, B, M) = ||(\cos^{-1}(A \odot B)) \otimes M||_1, \\
g_2(A, B, M) = ||{(A - B) \otimes M}||_2^2, \\
g_3(A, B, M) = ||{(A - \tau B) \otimes M}||_2^2, \\
g_4(A, B, M) = ||(\text{log}(A+1) - \text{log}(\tau B+1)) \otimes M) ||_2^2, \\
g_5(A) = -A\log(A),
\end{align}
\normalsize
where $\odot$ is element-wise dot product, $\otimes$ is element-wise multiplication, and $\tau$ is the scale obtained by least square regression between A and B.

In stage 1, the loss function of NormalNet is as follows:
\small
\begin{equation}
\mathcal{L}_{\text{normal}} = \beta_1 g_1(\mathrm{N}, \Tilde{\mathrm{N}}, \mathrm{M}_l)
+ \beta_2 g_2(\mathrm{N}, \Tilde{\mathrm{N}}, \mathrm{M}_l).
\end{equation}
\normalsize
NormalNet has a U-Net\cite{unet} structure with 6 down-up convolution blocks. 

Since the light source is not transparent, we use a regularization $g_5$ so that the visibility $\mu_s$ of InDLNet and the opacity $\alpha$ of ExDLNet can be 0 or 1. the loss function of InDLNet and ExDLNet is as follows:
\small
\begin{align} \label{eqn:eq_lossDL}
\mathcal{L}_{\text{InDL}} = \beta_1 g_4(\mathrm{E}_{DL}, \Tilde{\mathrm{E}}_{DL}, \mathrm{M}_o) +\beta_2 g_5(\mu_s), \\
\mathcal{L}_{\text{ExDL}} = \beta_1 g_4(\mathrm{E}_{DL}, \Tilde{\mathrm{E}}_{DL}, \mathrm{M}_o) +\beta_2 g_5(\alpha),
\end{align} 
\normalsize
where $\mathrm{E}_{DL}$ is the per-pixel direct lighting environment map. InDLNet also has a U-Net structure that encoder is shared, and decoders are separated by $\lambda_s, \xi_s, \mu_s$. The light source intensity $\eta_s$ was decoded using MLP. ExDLNet follows structure of OccNet\cite{occupancy} and uses MLP with conditional batch normalization (CBN) \cite{de2017modulating}. All convolution blocks use batch normalization(BN).

In stage2, the loss function is as follows. 

\small
\begin{equation}
\mathcal{L}_{\text{BRDF}} = \beta_1 g_3(\mathrm{A}, \Tilde{\mathrm{A}}, \mathrm{M}_o) + \beta_2 g_2(\mathrm{R}, \Tilde{\mathrm{R}}, \mathrm{M}_o).
\end{equation}
\normalsize

ContextNet uses U-Net with ResNet18\cite{resnet}, SpecNet uses MLP with 3 layers, MVANet uses layer normalization (LN), and RefineNet uses U-Net with group normalization(GN).

In stage3, the loss function is as follows. 
\small
\begin{multline} 
    \mathcal{L}_{\text{SVL}} = \beta_1 g_4(\mathrm{E}_{SVL}, \Tilde{\mathrm{E}}_{SVL}, \mathrm{M}_o) + \beta_2 g_5(\alpha) \\
    + \beta_3\displaystyle\sum_{k=1}^K ||w_k{(\mathrm{I}^k- \tau_{diff}\Tilde{\mathrm{I}}_{diff} - \tau_{spec}\Tilde{\mathrm{I}}^k_{spec}) \otimes \mathrm{M}_o}||_2^2,
\end{multline} 
\normalsize
% \!\!\!\!

where $\mathrm{E}_{SVL}$ is the per-pixel lighting environment map, $\tau_{diff}$ and $\tau_{spec}$ are the scale obtained by least square regression with target image. $\mathrm{I}^k, \Tilde{\mathrm{I}}_{diff}, \Tilde{\mathrm{I}}^k_{spec}$ are $k$-view image, diffuse image, $k$-view specular image, respectively, and $w_k$ is multi-view weight. In SVLNet, visible surface volume ($\mathrm{T}$) is concatenated with $\Tilde{\mathrm{V}}_\text{DL}$ after 2 downsampling and processed with 3D U-Net. The resolution of the $\Tilde{\mathrm{V}}_\text{DL}$ is $32^3$, and the resolution of the $\mathrm{T}$ and $\Tilde{\mathrm{V}}_\text{SVL}$ is $128^3$. SVLNet uses instance normalization(IN). SVLNet needs a lot of memory when training, so we render environment map with a spatial resolution of 60$\times$80. A summary of training, number of GPUs, hyperparameter and network architecture is provided in Tab.~\ref{tab:arch}. Rendering includes the time to obtain a 60$\times$80$\times$8$\times$16 environment map from VSG and the time to re-render the input image.

\begin{table*}[htb!]
\footnotesize
\centering
\resizebox{\linewidth}{!}{
\begin{tabular}{|c|l|c|c|c|c|c|c|c|c|c|c|c|c|}
\hline
Stage &Network &input & Arch & norm &batch &epoch &$\beta_1$ &$\beta_2$ &$\beta_3$ &lr &training / GPUs &inference &output(channels)\\
\hline
\multirow{3}{*}{1}&NormalNet &$\mathrm{I},\Tilde{\mathrm{D}}, \nabla\Tilde{\mathrm{D}}, \Tilde{\mathrm{C}}$,&U-Net  & BN & 96 & 60 & 1.0 & 1.0 & - &2e-3 &7h / 4 & 3ms &$\Tilde{\mathrm{N}}(3)$  \\

&InDLNet &$\mathrm{I}, \Tilde{\mathrm{N}}, \Tilde{\mathrm{D}}, \Tilde{\mathrm{C}}$& U-Net, MLP &BN &384 &80 &1.0 &1e-3 & - &2e-4 &10h / 4 &5ms &$\boldsymbol{\xi}_s, \lambda_s, \mu_s, {\boldsymbol{\eta}_s}(8)$ \\

&ExDLNet &$\mathrm{I}, \Tilde{\mathrm{N}}, \Tilde{\mathrm{D}}, \Tilde{\mathrm{C}}$& U-Net, MLP &BN &96 &80 &1.0 &1e-4 & - & 1e-4 &1d / 8 & 6ms &$\Tilde{\text{V}}_\text{DL}(8)$\\

\hline
\multirow{4}{*}{2} &ContextNet &$\mathrm{I}, \Tilde{\mathrm{N}}, \Tilde{\mathrm{D}}, \Tilde{\mathrm{C}}$ & Res U-Net  & BN &\multirow{4}{*}{64} &\multirow{4}{*}{40} &\multirow{4}{*}{3.0} &\multirow{4}{*}{1.0} &\multirow{4}{*}{-} &\multirow{4}{*}{1e-4} &\multirow{4}{*}{1d 20h / 8} &\multirow{4}{*}{54ms} &$\boldsymbol{f}_\text{context}(32)$\\
&SpecNet &${\boldsymbol{\xi}_s}, {\lambda_s}, {\mu_s}, {\boldsymbol{\eta}_s}, \boldsymbol{v}, \Tilde{\mathrm{N}}$ & MLP & -  &&&&&&&&&$\boldsymbol{f}_\text{spec}(8)$\\
&MVANet &$\mathrm{I}, \boldsymbol{f}_\text{context}, \boldsymbol{f}_\text{spec}, \boldsymbol{w}$ & - &LN &&&&&&&&&$\boldsymbol{f}_\text{BRDF}(16)$\\
&RefineNet &$\mathrm{I}, \Tilde{\mathrm{N}}, \Tilde{\mathrm{D}}, \Tilde{\mathrm{C}}, \boldsymbol{f}_\text{context}, \boldsymbol{f}_\text{BRDF}$ & U-Net &GN &&&&&&&&&$\Tilde{\mathrm{A}}(3), \Tilde{\mathrm{R}}(1)$\\
\hline
3&SVLNet &$\mathrm{I}, \Tilde{\mathrm{N}}, \Tilde{\mathrm{D}}, \Tilde{\mathrm{C}}, \Tilde{\mathrm{A}}, \Tilde{\mathrm{R}}, \Tilde{\text{V}}_\text{DL}$ &3D U-net &IN &8 &10 &10.0 &1e-2 &1.0 &1e-4 &3d 8h / 8 &11ms &$\Tilde{\text{V}}_\text{SVL}(8)$ \\
\hline
- &Rendering & $\Tilde{\mathrm{N}}, \Tilde{\mathrm{D}}, \Tilde{\mathrm{C}}, \Tilde{\mathrm{A}}, \Tilde{\mathrm{R}}, \Tilde{\text{V}}_\text{SVL}$, &-&-&-&-&-&-&-&- &- &834ms &$\Tilde{\mathrm{I}}(3)$\\
\hline

\end{tabular}
}
\caption{The details of the network architecture, and training. Please refer to the main paper for the architecture of MVANet.}
\label{tab:arch}
\end{table*}

\noindent{\bf Test details.} 
Li \etal\cite{cis2020} and we both used an environment map with an angular resolution of 16 $\times$ 8 during training, but we created an environment map with 32 $\times$ 16 during testing because our VSG was not restricted by resolution. In training, all views are rendered for re-rendering loss, but in testing, only the target view was rendered.


\section{Additional Experimental Results}\label{G}
\subsection{Indoor Synthetic Scenes}
We provide additional inverse rendering results for OpenRooms FF test scene in Fig.~\ref{fig:supp_ir_indoor}. Our method leverage multi-view and incident direct lighting to provide more accurate material estimation results for highly specular regions. (\eg table in sample 2, chair in sample 3) Furthermore, the proposed method yields better normal estimation results especially for more complicated structures by utilizing MVS depth. As a result, our lighting is more realistic and we can re-render input image more accurately.

\subsection{Real-World Scenes}
The performance gaps between MAIR and the single-view-based methods are more distinct in the unseen real-world scene. Fig.~\ref{fig:supp_ir_real} shows that our method robustly produces reasonable normal maps even for complex scene structures, and this naturally affects the subsequent material, lighting estimation. MAIR shows better material estimation results for shadowed regions(\eg table, wall in sample 2, floor in sample 3) or specular regions(\eg drawer in sample 4). Although there are no ground truths for materials, from our experience, we know that the stones, bushes in sample 1, and the dolls in sample 5 should show high roughness, which are consistent with our high roughness estimation results.

\subsection{Object Insertion}
Inverse rendering performance of three competing methods, lighthouse~\cite{lighthouse}, Li \etal\cite{cis2020}, and MAIR, are tested by comparing the quality of object insertion. We implemented a simple renderer for object insertion by referring to Wang \etal\cite{vsg} and used it for rendering results of MAIR and lighthouse~\cite{lighthouse}. As the public implementation of Li \etal\cite{cis2020} includes a renderer of their own, results of Li \etal\cite{cis2020} were rendered using this renderer, except for the results of the chrome sphere insertion; the renderer from Li \etal\cite{cis2020} does not support the chrome sphere rendering directly, so we used our renderer for this case. It should be also noted that all results of lighthouse~\cite{lighthouse} were produced by using our scene geometries because scene geometry results from lighthouse~\cite{lighthouse} were not accurate enough to render.

We conducted a user study to evaluate the quality of object insertion from the three methods. Given a background image and an object of a particular material, users selected the most natural image among the three different results in a random order. 100 users evaluated 25 different scenes. Fig.~\ref{fig:supp_oi_chrome1}, \ref{fig:supp_oi_chrome2}, \ref{fig:supp_oi_indoor}, \ref{fig:supp_oi_real1}, and \ref{fig:supp_oi_real2} show all the scenes used in our user study. Our 3D lighting not only clearly expresses HDR lighting, but also fully reflects real-world scene geometry and material. This allowed the object to be realistically inserted into the scene, acquiring the highest score among the competing methods. 

We also provide additional object insertion results. In the accompanied video, the object can be located not only on the plane but also on any geometry, and the shadow of the object realistically appears to match the scene illumination.

% Therefore, for fairness, only the results of us and Li \etal\cite{cis2020} were compared quantitatively.
%\noindent{\bf User study.}

\begin{figure*}[t]
  \centering
  \includegraphics[width=0.97\linewidth]{figure/supp_ir_indoor_1.pdf}
  \includegraphics[width=0.97\linewidth]{figure/supp_ir_indoor_2.pdf}
  \includegraphics[width=0.97\linewidth]{figure/supp_ir_indoor_3.pdf}
  \includegraphics[width=0.97\linewidth]{figure/supp_ir_indoor_4.pdf}
   \caption{Additional inverse rendering results on OpenRooms FF. Small insets are the estimations without bilateral solver (BS).}
   \label{fig:supp_ir_indoor}
\end{figure*}

\begin{figure*}[ht]
  \centering
  \includegraphics[width=0.99\linewidth]{figure/supp_ir_real_1.pdf}
  \includegraphics[width=0.99\linewidth]{figure/supp_ir_real_2.pdf}
  \includegraphics[width=0.99\linewidth]{figure/supp_ir_real_3.pdf}
   \caption{Additional inverse rendering results on IBRNet dataset\cite{ibrnet}. Small insets are the estimations without BS.}
   \label{fig:supp_ir_real}
\end{figure*}

\begin{figure*}[ht]
  \centering
  \includegraphics[width=0.99\linewidth]{figure/supp_oi_chrome_1.pdf}
  \includegraphics[width=0.99\linewidth]{figure/supp_oi_chrome_2.pdf}
  \includegraphics[width=0.99\linewidth]{figure/supp_oi_chrome_3.pdf}
   \caption{Additional chrome sphere insertion results on IBRNet dataset\cite{ibrnet}. The number under the image is the result of user study.}
   \label{fig:supp_oi_chrome1}
\end{figure*}

\begin{figure*}[ht]
  \centering
  \includegraphics[width=\linewidth]{figure/supp_oi_chrome_4.pdf}
   \caption{Additional chrome sphere insertion results on IBRNet dataset\cite{ibrnet}. The number under the image is the result of user study.} 
   \label{fig:supp_oi_chrome2}
\end{figure*}

\begin{figure*}[ht]
  \centering
  \includegraphics[width=\linewidth]{figure/supp_oi_indoor_1.pdf}
  \includegraphics[width=\linewidth]{figure/supp_oi_indoor_2.pdf}
   \caption{Additional white sphere insertion results on OpenRooms FF. The number under the image is the result of user study.} 
   \label{fig:supp_oi_indoor}
\end{figure*}

\begin{figure*}[ht]
  \centering
  \includegraphics[width=\linewidth]{figure/supp_oi_real_1.pdf}
  \includegraphics[width=\linewidth]{figure/supp_oi_real_2.pdf}
  \includegraphics[width=\linewidth]{figure/supp_oi_real_3.pdf}
   \caption{Additional virtual object~\cite{stanford} insertion results on IBRNet dataset\cite{ibrnet}. The number under the image is the result of user study.}
   \label{fig:supp_oi_real1}
\end{figure*}

\begin{figure*}[ht]
  \centering
  \includegraphics[width=\linewidth]{figure/supp_oi_real_4.pdf}
  \includegraphics[width=\linewidth]{figure/supp_oi_real_5.pdf}
   \caption{Additional virtual object~\cite{stanford} insertion results on IBRNet dataset\cite{ibrnet}. The number under the image is the result of user study.}
   \label{fig:supp_oi_real2}
\end{figure*}


%%%%%%%%% REFERENCES

%%%%%%%%% REFERENCES
{\small
\bibliographystyle{ieee_fullname}
\bibliography{egbib}
}


\end{document}
