% ------------------------------------------------------------------------
% bjourdoc.tex for birkjour.cls*******************************************
% ------------------------------------------------------------------------
%%%%%%%%%%%%%%%%%%%%%%%%%%%%%%%%%%%%%%%%%%%%%%%%%%%%%%%%%%%%%%%%%%%%%%%%%%

\documentclass{birkjour}

\usepackage{enumitem}
%
%
% THEOREM Environments (Examples)-----------------------------------------
%
 \newtheorem{axm}{Axiom}
\newtheorem{thm}{Theorem}[section]
\newtheorem{cor}[thm]{Corollary}
\newtheorem{lem}[thm]{Lemma}
\newtheorem{prop}[thm]{Proposition}
\newtheorem{cnj}[thm]{Conjecture}
\theoremstyle{definition}
\theoremstyle{definition}
\newtheorem{defn}[thm]{Definition}
\theoremstyle{remark}
\newtheorem{rem}[thm]{Remark}
\newtheorem*{ex}{Example}
\numberwithin{equation}{section}
\newcommand{\VERBOSE}{}
\newcommand{\AUTHOR}{}


\begin{document}

%-------------------------------------------------------------------------
% editorial commands: to be inserted by the editorial office
%
%\firstpage{1} \volume{228} \Copyrightyear{2004} \DOI{003-0001}
%
%
%\seriesextra{Just an add-on}
%\seriesextraline{This is the Concrete Title of this Book\br H.E. R and S.T.C. W, Eds.}
%
% for journals:
%
%\firstpage{1}
%\issuenumber{1}
%\Volumeandyear{1 (2004)}
%\Copyrightyear{2004}
%\DOI{003-xxxx-y}
%\Signet
%\commby{inhouse}
%\submitted{March 14, 2003}
%\received{March 16, 2000}
%\revised{June 1, 2000}
%\accepted{July 22, 2000}
%
%
%
%---------------------------------------------------------------------------
%Insert here the title, affiliations and abstract:
%


\title[A New Axiom Set for Euclidean Geometry]
 {A Testable and Progressive Axiom Set for 2-Dimensional Euclidean Geometry}

\ifdefined\AUTHOR
%----------Author 1
\author[Chengpu Wang]{Chengpu Wang*}
\email{Chengpu@gmail.com}

%----------Author 2
\author{Alice Wang}
\email{alicewang05@yahoo.com}
\fi

\thanks{}

%----------classification, keywords, date
%\subjclass{Primary 51M05; Secondary 51N20}

\keywords{Euclidean, formal system}

\date{Jan 28, 2023}
%----------additions
%\dedicatory{To the students who are willing to reinvent the wheel}
%%% ----------------------------------------------------------------------

\begin{abstract}
This paper shows that the 4 axioms in the original Euclidean axiom set can be modernized as 5 axioms.  
The new axiom set introduces each axiom progressively, with no new concept in any of the axioms so that each axiom is testable. 
One of the new axiom states that Euclidean, spherical, and hyperbolic spaces are all strictly monotonic, which seems to be implied in all existing axiom sets for Euclidean geometry.

A new implicit measure called local right ratio is introduced to replace parallel postulate, Riemann curvature tensor, or Gaussian curvature, to determine if the surface is locally flat, smooth, and curved at each point.  
Euclidean is shown to be the only surface to have constant right ratios.
\end{abstract}

%%% ----------------------------------------------------------------------
\maketitle
%%% ----------------------------------------------------------------------
%\tableofcontents
\section{Introduction}

\subsection{The Existing Axiom Sets for 2-Dimensional Euclidean Geometry}

Euclidean geometry \cite{Euclid}\cite{Plane}\cite{Old and New} is the origin for modern mathematics.
The original Euclidean axiom set contains 4 axioms. 
It is regarded as insufficient \cite{Old and New}\cite{Mistakes}, so it was enhanced by the Hilbert axiom set of 21 axioms in 5 groups \cite{Hilbert}.
For common usage, a parallel postulate is added to the Euclidean axiom set as the fifth axiom in modern time  \cite{Euclid}\cite{Plane}\cite{Old and New}.
In this paper, the Euclidean axiom set does not contain the fifth axiom.

The Euclidean axiom set \cite{Euclid} has one mathematical difficulty: the measurement of angles.  
Through its fourth axiom, it uses the right angle as the unit of measurement \cite{Mistakes}, but it cannot provide any way to arbitrarily divide a right angle, so that such a unit cannot be applied generally to angle measurement. 
In practice, an angle is measured by the corresponding arc length, while the arc length is calculated by the angle measurement \cite{Plane}, which results in circular definitions \cite{Formal System}.  
Perhaps to overcome this difficulty of angle measurement, later the Birkhoff axiom set for Euclidean geometry \cite{Birkhoff} requires both distance and angle to be independently measurable, which is different from the common practice to derive angle measurement from distance measurement.  
Also, the Birkhoff axiom set is too simple to be mathematically rigorous.

Both the Hilbert axiom set and the Birkhoff axiom set belong to the second-order logic which is based on relations of point sets \cite{Old and New}. 
The Tarski axiom set \cite{Tarski} is a first-order logic \cite{Formal System}.
However, Tarski \textsl{``designed his system mainly to facilitate its analysis via the tools of mathematical logic''} \cite{Tarski}, so that it is quite difficult from normal understanding of Euclidean geometry \cite{Plane}. 
For example, in the Tarski axiom set, among the three alternative forms of the Axiom of Euclid \cite{Tarski} which distinguishes Euclidean space from other spaces:
\begin{enumerate}[label=\alph*)]
\item \textsl{``through any point in the interior of an angle there is a line that intersects both sides of the angle.''}

\item \textsl{``in any (non-degenerate) triangle there is a point that is equal distant from each of the vertices.''}

\item \textsl{``the line connecting the mid points of two sides of a triangle is half the length of the third side.''}
\end{enumerate}
a) and b) are also permissible on spherical surface, but c) is not, so it is hard to understand how these three alternatives are equivalent in that logic system.
There is even no need to interpret meaning in the Tarski axiom set because the formalism of the first-order logic will prove or disprove all statements on the subject matter \cite{Old and New}\cite{Tarski}.
the Tarski axiom set is not suitable for common understanding of Euclidean geometry.

In this paper, the Hilbert axiom set is treated as the modern standard axiom set for Euclidean geometry.



\subsection{Localness}

The Euclidean axiom set does not use infinitive geometric objects such as straight lines as its foundation, but instead states \textsl{``if the line is extended to a sufficient length"} \cite{Euclid}.
However, the first primitive notion in the Hilbert axiom set is straight line \cite{Hilbert}, which is infinitive in extend.
The parallel postulate \cite{Hilbert} in the Hilbert axiom set, which differentiates Euclidean surfaces from non-Euclidean surfaces \cite{Plane}\cite{Old and New}\cite{Non-Euclidean Geometry}, is also infinitive in extend, so that it cannot be applied locally.
Euclidean space can be established locally on a globally non-Euclidean surface, such as on a manifold \cite{Differential Geometry} which means that the space is Euclidean at each point, and can be well approximated by Euclidean at nearby.  
Thus, the straight line and the parallel postulate should not be in the new axiom set.
Other existing methods from the Hilbert axiom set for this purpose such as \textsl{``The sum of the angles of a triangle is two right angles''} \cite{Hilbert} is not truly local because it cannot measure the curvature at a point on a manifold \cite{Differential Geometry}.

This paper will show that the parallel postulate is unnecessary when the fourth Euclidean axiom, that \textsl{``All right angles are equal to each other''} \cite{Euclid}, can be interpreted as equal at different distance scales so that a right angle equals to itself at different distance scales.
This paper will provide a new implicit local measure called right ratio for the smoothness and curvature at each point of a surface.
It replaces both Gaussian curvature \cite{Non-Euclidean Geometry} and Riemann curvature tensor \cite{Differential Geometry} which require the surface to be differentiable.  
In contrast, right ratio only requires the surface to be metric \cite{Isometric}.



\subsection{Implicit Assumptions}

Axiom \ref{axm:monotonic} of this new axiom set seems an implicit assumption in all the existing axiom sets for Euclidean geometry.
 


\subsection{Logic Foundation}

The Euclidean axiom set \cite{Euclid} states the properties of Euclidean space using 4 axioms, to conclude the theorems for congruence and similarity.
In contract, both the Hilbert axiom set \cite{Hilbert} and the Tarski axiom set \cite{Tarski} start with congruence and similarity relations, so they have reversed the logic for the Euclidean axiom set.
This new axiom set follows the logic of the Euclidean axiom set.


\subsection{Testable}

Because an axiom cannot be proved or disproved, it can be neither complete nor incomplete \cite{Formal System}.  
If an axiom can assert true undoubtfully, it is \emph{testable} by experimentation to be either existing or without exception.

A primitive notion \cite{Formal System} is a concept without proper definition beforehand but is relied upon by an axiom set.
Logically, when an axiom contains one or more primitive notions, it may no longer be testable by itself.
To overcome this logic short-come, an axiom set contains more axioms than primitive notions.
Ideally, all primitive notions should be described before stating the axiom set, to avoid relying on common and perhaps even wrong understanding of these concepts.
Ideally, an axiom set should contain no more than one primitive notion, to avoid interactions between primitive notions.
Thus, the number of primitive notion(s) for an axiom set has to be minimized.

One problem of the Hilbert axiom set \cite{Hilbert} is that it contains a few primitive notions without prior descriptions, such as the axiom \textsl{``I.3. Three points not situated in the same straight line always completely determine a plane.''}, which hides three primitive notion: dimension, flat plane, and the uniqueness of the flat plane. 
With so many new concepts in one sentence, it is hard to grasp the meaning of this axiom without relying on common belief related to this axiom.

The Tarski axiom set contains point as the only primitive notion \cite{Tarski} so it is better.
Like the Hilbert axiom set, the Tarski axiom set states that three noncollinear points determine a 2-dimensional surface, but it lacks the actual definition of the 2-dimensional surface.
Ideally, what is a 2-dimensional surface should also be defined.

This paper will define a new axiom set for plane geometry using distance-based ordered set of points to define geometric objects, including 2-dimensional surfaces, with point as the only primitive notion.
In addition, each property of a surface, such as smoothness, flatness, and curvature, are mathematically defined. 
All the definitions are implicit rather than explicit \cite{Formal System} so that the axiom set is self-consistent.



\subsection{Distance}

Both Hilbert's and Tarski's axiom have a few axioms to describe distance to avoid define distance, because the number set for the distance in the Euclidean axiom set contains non-negative rational numbers plus their square roots, so that defining distance as a real number is an overkill \cite{Old and New}.
Modern Euclidean geometry is already a general geometry for flat surface, and it is beyond the plane geometry using unmarked ruler and compass \cite{Euclid}.
For example, using a marked ruler, it is now possible to achieve what the Euclidean axiom set cannot achieve, such as trisecting an angle \cite{Old and New}.
The distance measurement and the angle measurement allow more advanced operations beyond trisecting an angle.
Such extension of Euclidean geometry also means the extension of the number set for distance \cite{Old and New}.
New tools such as Gaussian curvature \cite{Non-Euclidean Geometry} or Riemann curvature tensor \cite{Differential Geometry} further requires the surface to be differentiable.
Now, it is reasonable to define distance as a generic real number for common understanding of both Euclidean and Non-Euclidean geometries.

\subsection{Betweeness}

As a mathematical insufficiency, the Euclidean axiom set contains no ordering of points in a straight line \cite{Old and New}.
Both the Hilbert axiom set and the Tarski axiom sets have a few axioms to describe ordered ``betweeness'' of points in a straight line \cite{Old and New}\cite{Hilbert}\cite{Tarski}.
Most axioms for ``congruence'' are already implied by distance, while most axioms for ``betweenness'' are already implied by distance-based real number set.
Thus, using distance-based ordered set to define geometric objects can describe ``betweeness'' more concisely and accurately.
For example, in the Tarski axiom set, ``betweenness'' can be deduced from ``congruence'' when $\leq$ relation is defined \cite{Tarski}, but such specification is weaker than using distance-based ordered set of points to specify continuity of the space, including different types of continuity  \cite{Isometric}.
When geometric objects are such defined, two geometric objects will not miss their intersection only because they are not continuous enough, such as between a straight line and a circle  \cite{Old and New}.
Using ordered sets of points to define geometric objects is already a common practice such as in differential geometry \cite{Differential Geometry}.
Using sets of points to describe a space also relates the geometric space to its topology \cite{Topology}.

Euclidean, spherical, or hyperbolic geometry can also be established from metric foundation with free mobility of geometry objects \cite{Isometric}.   
The idea that metric can be the foundation for geometries \cite{Isometric} is followed by this paper.



\subsection{Progressiveness}

Instead of stating all axioms of an axiom set together, this new axiom set specifies the different applicable scope of each axiom, and progresses by narrowing down the scopes.

 

\subsection{Educational Purpose}

In US, the public education on mathematics is not satisfactory to prepare students for science and engineering.
Best students learn mathematics by themselves, such as through the requirements of AMC, AIME and USAMO competitions, which are all way ahead of the normal curriculum.
Along this trend, and also to satisfy the AP preference in US college admissions, the best high school students in mathematics have focused on learning advanced mathematics such as advanced calculus and functional analysis, rather than thoroughly understanding basic mathematics such as Euclidean geometry.
The downside of such self-learning in rushed manner or college-level courses taught by high-school instructors, is the loss of the chance and the ability to ask naive but sometimes deep questions when encountering a new subject matter.
The virgin thoughts on a new subject matter are a major source of creativeness and understanding in mathematics, which may be more valuable than the techniques of which mathematical competitions primarily focus on.
This paper is generated in the learning process of both authors over several years when one author learned Euclidean geometry for the first time in a relaxed manner from the other author, when the naive but sometimes deep questions broke the for-granted understanding for 2-dimensional Euclidean geometry, starting from the question of how to measure an angle.
This paper will provide a thought process following the following four steps: 1. Observation, 2. Generalization, 3. Deduction, 4. Verification.
Hoping to benefit more young students learning mathematics, this paper is aimed to be understandable by high school students who have taken the AP courses on mathematics, or college students in science and engineering.



 
\section{Geometric Objects}

A geometric \emph{point} is an exact location in a space but without any other properties.
It is the only primitive notion in the new axiom set.  
The point cannot be properly defined so that it remains a presumption in this paper, relying on common understanding on what a geometric point is.

\begin{defn}[geometric object]
A \emph{geometric object} is a set of points.
\end{defn}

\begin{defn}[same]
If two geometric objects contain the same set of points, they are the \emph{same} geometric object.
\end{defn}

In this paper, the following common conventions \cite{Formal System} are used:
\begin{itemize}
\item A point is denoted by a capital letter. 

\item A set can either be expressed as a collection of points, such as $\{A,B\}$, or as all points which satisfies certain conditions, such as $\{P: |PA|=r\}$ for a round object of radius $r$ centered at $A$.

\item A point can be in ($\in$) or not in ($\notin$) a geometric object.

\item Two geometric objects can either intersect ($\cap$) or union ($\cup$) each other. 
%$+$ is used to union two sets which have at most one point in common between them.

\item $\oslash$ is the empty set.  
\end{itemize}
In addition, the following mathematical logic symbols are used \cite{Formal System}:
\begin{itemize}
\item For a mathematical statement:  $\forall$ for \emph{any}, $\exists$ for \emph{some}, $\exists !$ for \emph{only one}, and $\neg$ for \emph{none} or \emph{negation}.

\item Two mathematical statements can be combined logically to form a new statement: $\wedge$ for \emph{and}, $\vee$ for \emph{or}.

\item Two mathematical statements can have the following relations: $\Longrightarrow$ for \emph{leads to}, and $\iff$ for \emph{equivalent} (which is leads to in both ways).

\item When a mathematical statement is the common statement for the other statements, it proceeds the others with $:$.
\end{itemize}



\section{Euclidean Space Is Measurable}

\subsection{Distance}

The concept of distance is implied in the original axiom set for Euclidean geometry, as \textsl{``To describe a circle with any center and distance''} \cite{Euclid}, which can be viewed as using construction of circles to measure distance. 
The distance defined in this paper is equivalent to the concept of metric \cite{Isometric}\cite{Differential Geometry} for generic spaces. 

\begin{defn}[distance]
Between two points $A$ and $B$, the \emph{distance} $|AB|$ is defined as a non-negative real value which satisfies the following requirements:
\begin{itemize}
\item $\forall A:|AA|=0$;
\item $\forall A \;\forall B:|AB|=|BA|>0$;
\item $\forall A \;\forall B \;\forall C: |AB| \leq |AC|+|CB|$.
\end{itemize}
\end{defn}
This definition of distance differs from the normal definition of distance because it does not have unit which is the definition of $1$ for distance.
It just says that the space is measurable, without specifying on how to measure it.
In another word, it is only a necessary condition for defining distance.

Immediately from the definition for distance:
\begin{thm}
\label{thm:|AB|=0}
$\forall A \;\forall B: |AB| = 0 \;\iff\; A = B$.
\end{thm}

It is also possible to define the distance as another number set other than the whole real number set, such as the construable field \cite{Old and New} for the Hilbert axiom set.
Using the whole real number set unifies the distance definition with other branches of geometry, such as differential geometry \cite{Differential Geometry}.


\subsection{Measurable}

A space is \emph{measurable} if there is only one distance between any two points:
\begin{axm}[measurable]
\label{axm:measurable}
$\forall A \forall B \Longrightarrow \exists ! |AB|$.
\end{axm}

Axiom \ref{axm:measurable} is equivalent to \textsl{``Postulate I. Space is metric''} \cite{Isometric}, in which a measurable space is defined as a \emph{metric space}.  
It is implicit in the Hilbert axiom set \cite{Hilbert} and the Tarski axiom set \cite{Tarski} by asserting ``congruence'' relation between any two points.

The violation of Axiom \ref{axm:measurable} is used to detect gravitational waves \cite{Gravitational Waves}.


\section{Euclidean Space Is Continuous}

\subsection{Straight Segment}
 
\begin{defn}[straight segment]
A \emph{straight segment} between two points $A$ and $B$ in a measurable space is defined as \cite{Differential Geometry}:
\begin{align}
\label{eqn:segment}
\overline{AB} \equiv & \{ P: |AB| = |AP| + |PB|, |AP| \in [0, |AB|]^R \} \nonumber \\
& P \in \overline{AB} \iff |AP| \in [0, |AB|]^R 
\end{align}
\end{defn}
The definition of $\overline{AB}$ excludes when $A$ and $B$ are two opposite poles on a spherical surface, because one of such poles can be regarded as infinitively far way from the other pole so that they can only be connected by a straight line \cite{Non-Euclidean Geometry}.
It satisfies the Hilbert axiom \textsl{``I.2: For every two points there exists no more than one line that contains them both''}, which implies Axiom \ref{axm:measurable}.

Some immediate conclusions of Equation \eqref{eqn:segment} are:

\begin{thm}
\label{thm:equal segments}
$\forall A \;\forall B: \overline{AB} = \overline{BA}$
\end{thm}

\begin{thm}
\label{thm:segment inequality}
$\forall C \notin \overline{AB} \iff |AB| < |AC| + |CB|$
\end{thm}

\begin{thm}
\label{thm:split segment}
$\forall C \in \overline{AB} \iff |AB| = |AC| + |CB|$.
\end{thm}

A geodesic \cite{Non-Euclidean Geometry} is a curve representing the shortest path between two points in a surface.
Equation \eqref{eqn:segment} shows that a straight segment is a geodesic in a Euclidean space.
Theorem \ref{thm:segment inequality} and Theorem \ref{thm:split segment} have an important implication: any segment of a geodesic is also a geodesic.

\iffalse

The continuity requirement in the definition of straight segment can be relaxed to Equation \eqref{eqn:line}, which is equivalent to Equation \eqref{eqn:segment} but which is much easier to test:
\begin{equation}
\label{eqn:line}
\lim_{n \to \infty} \{ M: |AB| = |AM| + |MB|, |AM| = |AB|\frac{m}{2^n}, m=0,1,...2^n \}
\iff \overline{AB}
\end{equation}
\begin{proof}
Suppose the range $[0, |AB|]$ has only one gap: $(|AC_1|, |AC_2|)$. 
$|AB| = |AC| + |CB| \Longrightarrow |AC_2| = |AB| - |AC_1| \wedge |AC_1| < |AB|/2$, which means that the midpoint between $A$ and $B$ does not exist in $\overline{AB}$.  
This case can extend to any odd number of gaps in $[0, |AB|]$.

Suppose the range $[0, |AB|]$ has only two gaps: $(|AC_1|, |AB| - |AC_2|)$ and $(|AC_2|, |AB| - |AC_1|)$, with $|AC_1|, |AC_2| < |AB|/4$, which conflicts with Equation \eqref{eqn:line} when $n=1$. 
This case can extend to any even number of gaps in $[0, |AB|]$.

Thus $[0, |AB|]$ cannot have any gap.
\end{proof}

\fi




\subsection{Continuous}

As the embodiment of the original axiom \textsl{``To draw a straight line from any point to any point''} in the Euclidean axiom set \cite{Euclid}:
\begin{axm}[continuous]
\label{axm:continuous}
$\forall A \forall B \Longrightarrow \exists ! \overline{AB}$.
\end{axm}
Axiom \ref{axm:continuous} specifies the topological property for the space to be metric \cite{Isometric}. 
It is equivalent to Hilbert axiom \textsl{``I.1 For every two points A and B there exists a line a that contains them both''}.
One immediate conclusion of Axiom \ref{axm:continuous} is Theorem \ref{thm:line cross line}:
\begin{thm}
\label{thm:line cross line}
$\forall \overline{AB} \cap \overline{CD} \neq \oslash \Longrightarrow \exists! E: \{E\} = \overline{AB} \cap \overline{CD}$.
\end{thm}

From the definition, $\overline{AB}$ is isomorphic to $[0, |AB|]^R$.
Axiom \ref{axm:continuous} extends this relation to any two points in the Euclidean space:
\begin{itemize}
\item $[0, |AB|]$ enables set comparison between two straight segments, which leads to Theorem \ref{thm:line equal line}.

\item The space is continuous, dense, complete, compact, and connected \cite{Isometric}, with each point separable from others according to Theorem \ref{thm:|AB|=0}.

\item Same as $[0, |AB|]$, $\overline{AB}$ has inner convexity \cite{Isometric}, which is the Hilbert axiom \textsl{``II.1. If A and C are two points of a straight line, then there exists at least one point B lying between A and C''} \cite{Hilbert}.

\item Same as $[0, |AB|]$, a straight segment is an ordered set of points \cite{Isometric}, as in the Hilbert axiom \textsl{``II.3. Of any three points situated on a straight line, there is always one and only one which lies between the other two.''} \cite{Hilbert}. 
\end{itemize}

\begin{thm}
\label{thm:line equal line}
$\forall \overline{AB} \;\forall \overline{CD}: A = C \wedge B = D \iff \overline{AB} = \overline{CD}$.
\end{thm}


\subsection{Round}

As the embodiment of the original axiom \textsl{``To describe a circle with any center and distance''} \cite{Euclid} in the Euclidean axiom set but not limited to 2-dimensional:

\begin{defn}[round]
A \emph{round} geometric object is $\overline{A : r} \equiv \{ P: |PA| = r \} $, in which the point $A$ is the \emph{center}, while the non-negative real number $r$ is the \emph{radius}. 
\end{defn}

A round geometric object thus separates the space into three parts:

\begin{defn}[inside]
$B$ is \emph{inside} $\overline{A : r}$ if $|AB|<r$.
\end{defn}

\begin{defn}[outside]
$B$ is \emph{outside} $\overline{A : r}$ if $|AB|>r$.
\end{defn}

From Theorem \ref{thm:segment inequality}:

\begin{thm}
\label{thm:outside rounds}
$\forall r_A + r_B < |AB| \iff \overline{A : r_A}$ and $\overline{B : r_B}$ are outside of each other.
\end{thm}

\begin{thm}
\label{thm:inside round}
$\forall |AB| < r_B - r_A \iff \overline{A : r_A}$ is inside $\overline{B : r_B}$.
\end{thm}

\ifdefined\MULTI_SEGMENT
Let $\{\overline{AB}\}$ be the set of all the unique straight segments connection $A$ and $B$:

\begin{thm}
\label{thm:inclusive outside round}
$\forall |AB| = r_A + r_B \iff \overline{A : r_A}$ and $\overline{B : r_B}$ are outside of each other except for $\{C: |AC| = r_A, C \in \overline{AB}, \overline{AB} \in \{\overline{AB}\} \}$.
\end{thm}

\begin{thm}
\label{thm:inclusive inside round}
$\forall |AB| = r_A - r_B \iff \overline{B : r_B}$ is inside $\overline{A : r_A}$ except for 
$\{C: |AC| = r_A, C \in \overline{AB}, \overline{AB} \in \{\overline{AB}\} \}$.
\end{thm}
\else
\begin{thm}
\label{thm:inclusive outside round}
$\forall |AB| = r_A + r_B \iff \overline{A : r_A}$ and $\overline{B : r_B}$ are outside of each other except for $\exists! C: |AC| + |CB| = |AB| \wedge \{C\} = \overline{A : r_A} \cap \overline{B : r_B}$.
\end{thm}

\begin{thm}
\label{thm:inclusive inside round}
$\forall |AB| = r_A - r_B \iff \overline{B : r_B}$ is inside $\overline{A : r_A}$ except for $\exists! C: |AC| - |CB| = |AC| \wedge \{C\} = \overline{A : r_A} \cap \overline{B : r_B}$.
\end{thm}
\fi

\begin{thm}
\label{thm:intersecting rounds}
$|r_A - r_B| < |AB| < r_A + r_B \iff 
\overline{A : r_A} \cap \overline{B : r_B} \neq \oslash \wedge (\overline{A : r_A} \cap \overline{B : r_B}) \cap \overline{*AB*} = \oslash$.
\end{thm}

\ifdefined\VERBOSE
\begin{proof}
The exception of Theorem \ref{thm:outside rounds}, \ref{thm:inside round}, \ref{thm:inclusive outside round} and \ref{thm:inclusive inside round} suggests Theorem \ref{thm:intersecting rounds}.
\end{proof}
\fi





\section{Euclidean Space Is Boundless and Smooth}

As the embodiment of the original axiom \textsl{``To extend a finite straight line continuously in a straight line''} \cite{Euclid} in the Euclidean axiom set:

\begin{axm}[boundless and smooth]
\label{axm:boundless}
$\forall \overline{AB} \Longrightarrow \exists C, D \not \in \overline{AB}: A \in \overline{CB} \wedge B \in \overline{AD}$.
\end{axm}

Axiom \ref{axm:boundless} excludes the space to have an closed boundary, requiring the space to be \emph{boundless}.

On a conical surface, any straight segment with the sharp point of the cone as one end cannot be extended at that end.
Thus, Axiom \ref{axm:boundless} also tests for \emph{smoothness}.


\subsection{Ray}

\begin{defn}[ray]
A \emph{ray} starting from $A$ passing through $B$ is:
\begin{equation*}
\overline{AB*} \equiv \overline{AB} + \{ P: |AP| = |AB| + |BP| \}.  
\end{equation*}
\end{defn}

Immediately from the definition:

\begin{thm}
\label{thm:equal rays}
$\forall C \in \overline{AB*} \wedge C \neq A \Longrightarrow \overline{AC*} = \overline{AB*}$.
\end{thm}

\begin{thm}
\label{thm:ray cross round}
$\forall \overline{AB*} \;\forall r > 0 \Longrightarrow \exists! C: \{C\} = \overline{AB*} \cap \overline{A : r}$.
\end{thm}


\subsection{Straight Line}

\begin{defn}[straight line]
A \emph{straight line} passing $A$ and $B$ is:
\begin{equation*}
\overline{*AB*} \equiv \{ P: |PB| = |PA| + |AB| \} + \overline{AB} + \{ P: |AP| = |AB| + |BP| \}.  
\end{equation*}
\end{defn}

The definition of $\overline{*AB*}$ satisfies the Hilbert axiom \textsl{``I.1. Two distinct points always completely determine a straight line''} \cite{Hilbert}, and the Hilbert axiom \textsl{``I.7. Upon every straight line there exist at least two points,''} \cite{Hilbert}.
Theorem \ref{thm:equal line} is equivalent to the Hilbert axiom \textsl{``I.2. Any two distinct points of a straight line completely determine that line''} \cite{Hilbert}.

Immediately from the definition:

\begin{thm}
\label{thm:ray in line}
$\forall C \in \overline{AB} \wedge C \neq A \wedge C \neq B \Longrightarrow \overline{*AB*} = \overline{*AC} + \overline{CB*}$.
\end{thm}

\begin{thm}
\label{thm:equal line}
$\forall C,D \in \overline{*AB*} \wedge C \neq D \Longrightarrow \overline{*CD*} = \overline{*AB*}$.
\end{thm}

\begin{thm}
\label{thm:non-collinear}
$\forall C \notin \overline{*AB*} \Longrightarrow B \notin \overline{*CA*} \wedge A \notin \overline{*BC*}$.
\end{thm}

\begin{thm}
\label{thm:round away from line}
$\forall C \not \in \overline{AB} \Longrightarrow \exists r > 0: \overline{C:r} \cap \overline{*AB*} = \oslash$.
\end{thm}


\subsection{Surface as 2-Sided 2-Dimensional Space}

From observation, a straight line divides a flat surface, a cylindrical surface, a conical surface, a spherical surface, or a hyperbolic surface into two separate parts:

\begin{defn}[opposite side]
if $C, D \notin \overline{*AB*}$, and $\overline{CD} \cap \overline{*AB*} \neq \oslash$, $C$ and $D$ are defined as on the \emph{opposite sides} of $\overline{*AB*}$ , which is denoted as $C, D \div \overline{*AB*}$.
\end{defn}

\begin{defn}[2-dimensional space]
When $C, D \div \overline{*AB*}$, the \emph{2-dimensional space} $\{ C, D \div \overline{*AB*} \}$ is $ \overline{*AB*} \cup \{P: P, C \div \overline{*AB*}\} \cup \{P: P, D \div \overline{*AB*}\}$.
\end{defn}

\begin{defn}[2-sided 2-dimensional space]
If a $\{ C, D \div \overline{*AB*} \}$ further satisfies $\{P: P, C \div \overline{*AB*}\} \cap \{P: P, D \div \overline{*AB*}\} = \oslash$, it is a \emph{2-sided 2-dimensional space} $\{ C, D \div \overline{*AB*} \}_2$.  
\end{defn}

\begin{defn}[same side]
By definition, in a $\{ C, D \div \overline{*AB*} \}_2$, $C, E \div \overline{*AB*} \Longrightarrow \overline{DE} \cap \overline{*AB*} = \oslash$, or $D$ and $E$ are on the \emph{same side} of $\overline{*AB*}$.
\end{defn}

The 2-dimensional space satisfies the Hilbert axiom \textsl{``I.5. If two points of a straight line a lie in a plane, then every point of the straight line lies in the plane''} \cite{Hilbert}. 
The 2-sided 2-dimensional space satisfies the Hilbert axiom \textsl{``II, 5. Let A, B, C be three points not lying in the same straight line and let a be a straight line lying in the plane ABC and not passing through any of the points A, B, C. Then, if the straight line a passes through a point of the segment AB, it will also pass through either a point of the segment BC or a point of the segment AC. ''} \cite{Hilbert}. 

Theorem \ref{thm:line cross line} and \ref{thm:ray cross round} show that any point on a 2-dimensional space can have a local 2-side area surrounding it, as Theorem \ref{thm:local 2_sided}:
\begin{thm}
\label{thm:local 2_sided}
$\forall C \Longrightarrow \exists D \;\exists \overline{*AB*}: \{C, D \div \overline{*AB*}\}_2$.
\end{thm}

The 2-sided surfaces exclude topologically more complex surfaces such as Mobius ring and Klein bottle \cite{Non-Euclidean Geometry}.
A 2-sided 2-dimensional space is simply called a \emph{surface} in this paper.
For simplicity, the discussion of space should be limited to surfaces.





\section{Euclidean Space is Strictly Monotonic}

\begin{figure}
\includegraphics[scale=0.5]{tangent.png}
\caption{The tangent relation between $C \not \in \overline{*AB*}$.}
\label{fig:tangent}
\end{figure}
In Figure \ref{fig:tangent}, if $C \not \in \overline{*AB*}$, when $r$ increases from $0$, $\overline{C:r}$ may have $0$, $1$ or $2$ intersection points with $\overline{*AB*}$.
When $\{D\} = \overline{C:|CD|} \cap \overline{*AB*} \;\wedge\;\{E,F\} = \overline{*AB*} \cap \overline{C:|CE|}$, $|DE|$ or $|DF|$ increases strictly monotonically with $|CE|$.

However, this relation may not hold on more complicated surfaces.
When $C$ is very close to $D$, $|CE|$ probably increases monotonically with $|DE|$ on a differentiable surface, but in other cases, such monotonic relation is not guaranteed.
For an example, on a spherical surface, an equivalent straight line is a big circle, so the intersection between a straight line and the corresponding big circle is the big circle itself, which is more than a point.  
An arbitrary surface can be designed so that a straight line intersects with a circle for more than two points in all kinds of ways.
Thus, together with Theorem \ref{thm:round away from line}, Axiom \ref{axm:monotonic} captures the simplicity of Euclidean surface:
\begin{axm}[monotonic]
\label{axm:monotonic}
\begin{align*}
\forall  C & \not \in \overline{*AB*} \Longrightarrow \\
&\exists! D \in \overline{*AB*}: \{D\} = \overline{C : |CD|} \cap \overline{*AB*} \;\vee \\
&\exists E, F: \{E,F\} = \overline{*AB*} \cap \overline{C: |CE|} \iff |CD| < |CE| \;\vee \\
&\exists G, H: \{G,H\} = \overline{*AB*} \cap \overline{C: |CG|} \wedge E, G \in \overline{DA*}: \\
&\;\;\;\;\;\;\;\; |CE| < |CG|  \iff |DE| < |DG|.
\end{align*}
\end{axm}

Axiom \ref{axm:monotonic} seems to be assumed in all existing axiom sets for Euclidean geometry.
For example, the monotonic relation belongs to neither congruence relation, nor similarity relation, nor to be covered by Archimedes axiom, nor any other axiom in the Hilbert axiom set.



\subsection{Tangent}

One immediate conclusion of Axiom \ref{axm:monotonic} is that $D$ is a special point for $C \not \in \overline{*AB*}$:
\begin{defn}[tangent]
For $C \not \in \overline{*AB*}$, if $\exists! D: \{D\} = \overline{C:|CD|} \cap \overline{*AB*}$, then $D$ is the \emph{tangent} of $C$ on $\overline{*AB*}$, which is denoted as $D = C \perp \overline{*AB*}$.  
\end{defn}

From Figure \ref{fig:tangent} left, according to Axiom \ref{axm:monotonic}, $\overline{C:|CD|}$ is always inside  $\overline{C:|CE|}$, so that $\overline{CD}$ is the shortest straight segment between $C$ and $\overline{*AB*}$:
\begin{thm}
\label{thm:tangent}
\begin{align*}
\forall C \not \in \overline{*AB*} \wedge \exists D \in \overline{*AB*} \wedge \forall E \in \overline{*AB*}: D = C \perp \overline{*AB*} \iff |CD| \leq |CE|
\end{align*}
\end{thm}



\subsection{Perpendicular}

\begin{figure}
\includegraphics[scale=0.5]{perpendicular.png}
\caption{Perpendicular lines.}
\label{fig:perpendicular}
\end{figure}

Figure \ref{fig:perpendicular} shows the perpendicular relation between two lines, which is captured by Theorem \ref{thm:perpendicular}.

\begin{defn}[perpendicular]
If $\forall F \in \overline{*AB} \Longrightarrow E = F \perp \overline{*CD*}$, then $\overline{*AB*}$ is \emph{perpendicular} to $\overline{*CD*}$ at $E$, which is denoted as $E = \overline{*AB*} \perp \overline{*CD*}$.
\end{defn}

\begin{thm}
\label{thm:perpendicular}
$E = B \perp \overline{*CD*} \Longrightarrow E = \overline{*EB*} \perp \overline{*CD*} = \overline{*CD*} \perp \overline{*EB*}$.
\end{thm}
\ifdefined\VERBOSE
\begin{proof}
In Figure \ref{fig:perpendicular}, construct $\overline{B:|EB|}$ so that $\{E\} = \overline{B:|BE|} \cap \overline{*CD*}$ because $E = B \perp \overline{CD}$.  
For $\forall B' \in \overline{BE}$, construct $\overline{B':|EB'|}$ so that $\{E\} = \overline{B':|B'E|} \cap \overline{*CD*}$.
According to Theorem \ref{thm:inclusive inside round}, $\overline{B':|EB'|}$ is inside $\overline{B:|EB|}$ except at $E$.  
Thus, $E = \overline{BE} \perp \overline{*CD*}$.

If $E = \overline{EB'} \perp \overline{*CD*}$, assume $E' = \overline{EB} \perp \overline{*CD*} \Longrightarrow E' = \overline{EB'} \perp \overline{*CD*}$, which means $E' = E$. 
Thus, $E = \overline{EB*} \perp \overline{*CD*}$.

From Theorem \ref{thm:inclusive outside round}, it is consistent to assume $E = \overline{*AB*} \perp \overline{*CD*}$.
Assume $F \in \overline{EC*} \wedge F = \overline{EA*} \perp \overline{*CD*} \wedge F \neq E$, so that $|AF| < |AE| \wedge |BE| < |BF|  \wedge |AE| + |BE| < |AF| + |BF|$, which leads to the contradictory condition $|BE| < |AE|$.
Thus, $E = \overline{*AB*} \perp \overline{*CD*}$.

Because ${E} = B \perp \overline{*CD*}$:
\begin{align*}
\forall F \in \overline{EA} \Longrightarrow 
	& \exists {G} = \overline{B:|BF|} \cap \overline{EC*} \wedge |GE| < |GF| \;\vee \\
	& \exists {H} = \overline{B:|BF|} \cap \overline{ED*} \wedge |HE| < |HF|
\end{align*}
Thus, according to Theorem \ref{thm:tangent}, $E = \overline{*CD*} \perp \overline{*AB*}$.
\end{proof}
\fi


\subsection{Reflective}

\begin{defn}[reflection]
If $E = \overline{CD} \perp \overline{*AB*}$ with $\{C,D \div \overline{*AB*}\}_2$, and $|CE|=|DE|$, then $C$ and $D$ are \emph{reflection} of each other along $\overline{*AB*}$.
\end{defn}

The reflection relation can be extended to two geometric objects.




\section{Euclidean Surface Is Flat}

\subsection{Angle Measurement}

Any two rays from one point form an \emph{angle}, in which the point is called the \emph{vertex}, while the two rays are called two \emph{arm}s of the angle.  
Letting the vertex be $A$ and the two arms be $\overline{AB*}$ and $\overline{AC*}$, the angle is denoted as $\angle BAC$ or equivalently $\angle CAB$.

\begin{defn}[angle distance ratio]
To measure an angle $\angle BAC$:
\begin{enumerate}
\item Use a positive value called \emph{angle measuring distance} $u$ to construct $\overline{A : u}$, which intersects $\overline{AB*}$ and $\overline{AC*}$ at $B$ and $C$ respectively, according to Theorem \ref{thm:ray cross round}.

\item The \emph{angle distance ratio} is defined as: $|\angle BAC|_u \equiv |BC|^2/(2u)^2 \in [0,1]$, with $0$ for a ray and $1$ for a straight line.  
\end{enumerate}
\end{defn}

\begin{defn}[right ratio]
When $A = \overline{*BAD*} \perp \overline{*CAE*}$, the angle distance ratio for each of the angles $\angle BAC$, $\angle BAE$, $\angle DAE$, and $\angle DAC$ is defined as a \emph{right ratio}.  
\end{defn}

\begin{thm}
\label{thm:right ratio floor}
Any right ratio is larger than $1/4$.
\end{thm}
\ifdefined\VERBOSE
\begin{proof}
When $B = \overline{AB} \perp \overline{BC}$, if $|AB|=|BC|=u$, then $|AC| > u$.
\end{proof}
\fi



\subsection{Flat}

The fourth axiom in the Euclidean axiom set \cite{Euclid} is \textsl{``All right angles are equal to one another''}. This requirement can be quantified by:
\begin{defn}[flat]
If all right ratios on a surface are the same constant, the space is \emph{flat}.
\end{defn}

\begin{axm}[flat]
\label{axm:flat}
All right ratios are $1/2$.
\end{axm}

Axiom \ref{axm:flat} shows that Euclidean space is flat.
It could be relaxed as \emph{``Euclidean space is flat''} in the future, because Theorem \ref{thm:constant right ratios} will prove that on a surface, if it is flat at one point, the whole surface has $1/2$ as right ratios.

In contrast, Section \ref{sec:Non Euclidean} will show that non-Euclidean surfaces are not flat, with right ratios:
\begin{itemize}
\item increases with measuring distance from $1/2$ for spherical surfaces, and

\item decreases with measuring distance from $1/2$ for hyperbolic surfaces.
\end{itemize}
Because the right ratios is $1/2$ when the measuring distance approaches $0$, both surfaces are manifold \cite{Differential Geometry}.




\subsection{Square}

\begin{figure}
\includegraphics[scale=0.5]{squares.png}
\caption{ \textsl{Left:} Form a square in a Euclidean surface. \textsl{Right:} Stack identical squares side by side to form rectangles }
\label{fig:squares}
\end{figure}

In Figure \ref{fig:squares} \textsl{Left}, let $O = \overline{AB} \perp \overline{CD}$ with $|OA|=|OB|=|OC|=|OD|$.  
According to Axiom \ref{axm:flat}, $|AC|=|CB|=|BD|=|DA|=\sqrt{2}|OA|$, so that $|\angle ACB|=|\angle CBD|=|\angle BDC|=|\angle DAC|=1/2$, or the geometric object $ACBD$ is a square.  

\begin{defn}[square]
A \emph{square} is formed by four sides of equal length, with the adjacent sides perpendicular to each other.
\end{defn}

\begin{thm}
\label{thm:square}
Squares of any size and orientation can be constructed anywhere on a Euclidean surface.
\end{thm}


\subsection{Rectangle}

As shown in Figure \ref{fig:squares} \textsl{Right}, when squares $ABCD$ and $CDEF$ are stacked along side $\overline{CD}$, $|\angle ADC|=|\angle EDC|=1/2$, so that $D = \overline{AE} \perp \overline{CD}$ according to Theorem \ref{thm:tangent}.  
The geometric object $ABFE$ is a rectangle.
By stacking two squares in a row, the rectangle ABFE has $|AE|=|BF|=2|AB|=2|EF|$, or a side ratio of $2/1$.  
By stacking $M$ squares in each row and $N$ squares in each column, a rectangle of size ratio $M/N$ can be constructed.  
%According to Hurwitz's theorem \cite{Formal System}, any irrational number can be approximated by a rational number to any precision.
A square is a special case of a rectangle.

\begin{defn}[rectangle]
A \emph{rectangle} is formed by four sides of equal length between non-adjacent sides, with adjacent sides perpendicular to each other.  
\end{defn}

\begin{thm}
\label{thm:rectangle}
Rectangles of any size and orientation can be constructed anywhere on a Euclidean surface.
\end{thm}


\subsection{The ``Parallel Postulates''}

\begin{defn}[parallel]
If two straight lines are both perpendicular to a third straight line, these two straight lines are \emph{parallel} to each other, and the distance between the two intersection points on the third straight line is defined as \emph{the distance between the two parallel straight lines}. 
\end{defn}

\begin{thm}
\label{thm:parallel}
There is one distance value between two parallel straight lines.
\end{thm}
\ifdefined\VERBOSE
\begin{proof}
According to the definition, any two opposite sides of a rectangle or a square are parallel to each other. 
In Figure \ref{fig:squares} \textsl{right}, an arbitrary rectangle $ABHG$ is stacked by another arbitrary rectangle $ABFE$ along $\overline{AB}$, so that $G = \overline{GH} \perp \overline{*GE*}$, $H = \overline{GH} \perp \overline{*HF*}$,  $A = \overline{AB} \perp \overline{*GE*}$, $B = \overline{AB} \perp \overline{*HF*}$, $E = \overline{EF} \perp \overline{*GE*}$, and $F = \overline{EF} \perp \overline{*HF*}$, with $|GH|=|AB|=|EF|$. 
Thus, the distances between the two parallel straight lines $\overline{*GE*}$ and $\overline{*HF*}$ are a constant. 
\end{proof}
\fi
Theorem \ref{thm:parallel} leads to the ``parallel postulates'', such as the Hilbert axiom \textsl{``III. In a plane there can be drawn through any point A, lying outside of a straight line a, one and only one straight line which does not intersect the line a. This straight line is called the parallel to a through the given point A.''} \cite{Hilbert}.  
The ``parallel postulates'' is no longer an axiom because Axion \ref{axm:flat} interprets the fourth Euclidean postulate \cite{Euclid} \textsl{``That all right angles are equal to one another''} to be equal in all distance-scales, so that a right angle equals to itself in all different distance-scales.
Section \ref{sec:Non Euclidean} will show that such equality is not permissible in non-Euclidean spaces.


\section{Euclidean Space is Homogeneous, Isometric, Isotropic, Reflective, and Similar}

\subsection{Measuring Grid}

\begin{figure}
\includegraphics[scale=0.5]{net.png}
\caption{A square grid measures three triangles.}
\label{fig:net}
\end{figure}

The stacking of identical squares can be continuous in both directions to form a measuring grid, as shown in Figure \ref{fig:net}:

\begin{itemize}
\item 
Because Euclidean surface can be divided into identical squares, the geometric objects from different parts of the surface can be compared for equality directly. 
This means that Euclidean surface is \emph{homogeneous} \cite{Non-Euclidean Geometry} and \emph{isometric} \cite{Non-Euclidean Geometry}.

\item 
A measuring grid has only two parameters: the orientation and the size of each square, so that two measuring grids can be compared for equality after enlarging and rotating one of them if necessary.  
For example, in Figure \ref{fig:net}, $\triangle A'B'C'$ matches $\triangle ABC$ when $\triangle A'B'C'$ is enlarged by 2-fold, then rotated by a right angle. 
This means that Euclidean surface is \emph{isotropic} \cite{Non-Euclidean Geometry} and \emph{similar} \cite{Non-Euclidean Geometry}.

\item 
It may be necessary to make a reflection before two geometry objects can be compared for equality, such as $\triangle A'B'C'$ and $\triangle A"B'C'$ in Figure \ref{fig:net}.
Because reflecting twice restores the original geometric object, the reflection has only two choices.
This means that Euclidean surface is \emph{reflective} \cite{Non-Euclidean Geometry}.

\item 
By counting the grid squares which a geometric object occupies, a measuring grid defines the \emph{area} of the geometric object \cite{Plane}, e.g., $\triangle ABC$ occupies 12 squares in Figure \ref{fig:net}.  
The concept of area leads to Pythagorean Theorem in Euclidean surface \cite{Plane}.

\item 
A measuring grid obeying the Pythagorean Theorem allows for the definition of Cartesian coordinates, which is the foundation for explicit geometries studying non-Euclidean spaces \cite{Non-Euclidean Geometry}\cite{Differential Geometry}.
\end{itemize}




\subsection{Congruence and Similarity}

Because a Euclidean surface is isometric, isotropic, and reflective, the length of a straight segment is invariant under translation, rotation, and reflection.  
Because a Euclidean surface is also similar, an angle distance ratio is invariant under translation, rotation, reflection, and scaling.
The invariance of a geometric object under translation, rotation, reflection, and scaling is described by its \emph{geometric properties}:

\begin{defn}[geometric properties]
The following types of true statements about a geometric object form the \emph{geometric properties} of the object:
\begin{itemize}
\item \emph{distance-related} properties: including the length of any straight segment, and any other measurement derived from the above lengths.

\item \emph{reflection-related} properties: which reflection is chosen.

\item \emph{ratio-related} properties: including any ratio derived from the length related properties, including the angle distance ratio of any angle.
\end{itemize}
\end{defn}
  
For example, $\triangle ABC$, $\triangle A'B'C'$, and $\triangle A"B'C'$ in Figure \ref{fig:net} have the following properties:
\begin{itemize}
\item $|AB|$, $|BC|$, and $|CA|$ are distance-related properties.
$|AB| + |BC| + |CA|$ can be defined as the perimeter of $\triangle ABC$, which becomes another distance-related property.  

\item On which side of $\overline{B'C'*}$ point $A'$ or $A''$ exists is a reflection-related property.

\item $|AB|/|BC|$ and $|\angle CAB|$ are ratio-related properties, so that $|\angle CAB| = |\angle C'A'B'| = |\angle C'A''B'|$.
\end{itemize}

The original Euclidean axiom set \cite{Euclid} is in the format of stating the ability to construct simple geometric objects using deterministic rules, such as a point, a straight segment, a circle, a triangle, etc.  
Following this tradition:

\begin{defn}[constructible]
A \emph{constructible object} is a geometric object which can be constructed using predefined steps of deterministic rules. 
\end{defn}

\begin{defn}[repeatable]
If a constructible object can be constructed by a \emph{first rule} that contains a ray, which is called the \emph{starting ray} with its \emph{starting point}, while all other rules only relate to each other including the first rule, then the constructible object is called a \emph{repeatable} object.
\end{defn}
 
A repeatable object adds two new types of geometric properties:
\begin{itemize}
\item 
The starting point decides \emph{location-related} properties. The difference in location-related properties can be expressed as a translation \cite{Plane}.

\item 
The direction of the starting ray decides \emph{orientation-related} properties. The difference in orientation-related properties can be expressed as a rotation \cite{Plane}.
\end{itemize}
Euclidean geometry is about the geometric properties of repeatable objects:

\begin{thm}
\label{thm:repeated}
On a Euclidean surface, if the construction rules of two repeatable objects differ only in the first rule, then they have identical corresponding geometric properties except 1) a possible difference in location-related properties, and 2) a possible difference in orientation-related properties.
\end{thm}

\begin{thm}
\label{thm:reflective}
On a Euclidean surface, if the construction rules of two repeatable objects differ only in A) the first rule, and B) the side of the starting ray on which to carry out all the other rules, then they have identical geometric corresponding properties except for 1) a possible difference in location-related properties, and 2) a possible difference in orientation-related properties, and 3) opposite reflection-related properties.
\end{thm}

\begin{thm}
\label{thm:repeated and scaled}
On a Euclidean surface, if the construction rules of two repeatable objects differ only in A) the first rule, and B) each distance to be scaled by a positive real constant in all the other rules, then they have identical corresponding geometric properties except for 1) a possible difference in location-related properties, and 2) a possible difference in orientation-related properties, and 3) each distance-related property to be scaled by the constant.
\end{thm}

\begin{thm}
\label{thm:reflective and scaled}
On a Euclidean surface, if the construction rules of two repeatable objects differ only in A) the first rule, and B) each distance to be scaled by a positive real constant in all the other rules, and C) the side of the starting ray on which to carry out all the other rules, then they have identical corresponding geometric properties except for 1) a possible difference in location-related properties, and 2) a possible difference in orientation-related properties, and 3) each distance-related property to be scaled by the constant, and 4) opposite reflection-related properties.
\end{thm}

The combination of Theorem \ref{thm:repeated} and Theorem \ref{thm:reflective} states \emph{congruence} relations between two geometric objects. 
All axioms of group IV in the Hilbert axiom set \cite{Hilbert} are special cases for the general congruence relation defined by Theorem \ref{thm:repeated} and Theorem \ref{thm:reflective}.

The combination of Theorem \ref{thm:repeated and scaled} and Theorem \ref{thm:reflective and scaled} states \emph{similarity} relations between two geometric objects.  
Through Archimedes Axiom, the Hilbert axiom set introduces similarity relations.

Measuring grids can also be established in spherical space using equilateral triangles or polygons, but only for selected sizes \cite{Non-Euclidean Geometry}.  
Thus, a spherical space can have congruence relations but not similarity relations between two geometric objects.


\iffalse
\subsection{Some Basic Operations in Euclidean Geometry}

\begin{figure}
\includegraphics[scale=0.5]{intersecting_circles.png}
\caption{Two intersecting circles contains basic operations in Euclidean geometry.}
\label{fig:intersecting circles}
\end{figure}


Figure \ref{fig:intersecting circles} contains some basic operations in 2-dimensional Euclidean geometry:
\begin{itemize}
\item For $C \notin \overline{*AB*}$, find $D: D = C \perp \overline{*AB*}$.

\item For $C \in \overline{*AB*}$, find $\overline{*CD*}: C = \overline{*CD*} \perp \overline{*AB*}$.

\item Bisect a straight segment such as $\overline{CD}$.

\item Bisect an angle such as $\angle CAD$.

\item Construct a pair of reflective points $C$ and $D$ along $\overline{*AB*}$. 
\end{itemize}
\fi


\section{Beyond Euclidean Geometry}
\label{sec:Non Euclidean}

\subsection{Local Euclidean Region}

\begin{figure}
\includegraphics[scale=0.6]{cone.png}
\caption{ 
\textsl{Left:} On a conical surface with tip $A$, two geodesics connect $B$ and $C$.
\textsl{Right:} The angle measuring distance is limited to $|FB|$ for the right ratio at $F$ to be $1/2$.
}
\label{fig:cone}
\end{figure}

\begin{figure}
\includegraphics[scale=0.4]{conical_transition.png}
\caption{ 
\textsl{Left:} Normalized transitional distance of right ratio v.s. conical cut angle from 0 to 180 degrees, and measuring angle from 5 to 85 degrees on a conical surface.
\textsl{Right:} Normalized transitional derivative of right ratio v.s. conical cut angle from 0 to 180 degrees, and measuring angle from 5 to 85 degrees on a conical surface.
}
\label{fig:transition}
\end{figure}

It is possible that the Euclidean surface be established only within a region, such as on a conical surface.  

Figure \ref{fig:cone} \textit{Left} shows a conical surface with tip $A$, while Figure \ref{fig:cone} \textit{Right} shows how to construct the conical surface by cutting and stitching along $\overline{AC*}$ and $\overline{AC'*}$, to create $\overline{AC*}$ on Figure \ref{fig:cone} \textit{Left}.
$\angle CAC'$ is defined as the \emph{conical cut angle} for the conical surface.

Figure \ref{fig:cone} \textit{Right} shows the measurement of a right ratio on such a conical surface at $F$ when $A$ lies between the two arms of $F = \overline{FB} \perp \overline{FC}$. 
$\angle AFB$ is defined as the \emph{measuring angle} for a particular measurement.
When the angle measurement distance is larger than $|FB|=|FC|$, $\overline{FC}$ extends over the cut as $\overline{C'H}$, so that the right ratio starts to decrease from $1/2$.

To show the extent of the Euclidean region of the right ratio measurement, Figure \ref{fig:transition} \textit{Left} shows $|FB|/|FA|$ which is defined as the \emph{normalized transitional distance of right ratio} v.s. the conical cut angles and the measuring angles.
To show the rate of deviation from the Euclidean space, Figure \ref{fig:transition} \textit{Right} shows the \emph{normalized transitional derivative of right ratio} which is defined as the first derivative of the right ratio at the normalized transitional distance.
After the normalization, Figure \ref{fig:transition} applies to any right ratio measurement with the tip of the conical surface between the two arms of the right angle, for any point on any conical surface whose conical cut angle is $180^o$ or less.
Figure \ref{fig:transition} shows that when the cone becomes sharper, both the normalized transitional distances and the normalized transitional derivatives have the largest decrease at the $45^o$ measuring angle.
The $45^o$ measuring angle, which is also the bisect direction of the right angle, is thus defined as the \emph{direction of a right ratio measurement}.

The above analysis of conical surfaces showcases that Euclidean surface can be established in confined regions.



\subsection{Smoothness}

In Figure \ref{fig:cone}, the conical surface at its tip $A$ is not smooth, which can be measured using right ratios: for right ratio measurements from $F$, if $A$ is not between the two arms of the right angles at $F$, the right ratios are always $1/2$, so that there is a discontinuity of the right ratio v.s. directions and positions due to the sharp point $A$ of the conical surface.
To avoid such singular points on an otherwise smooth surface:

\begin{defn}[smooth]
If all right ratios on a surface are continuous v.s. angle measuring distance, direction, and position, the surface is \emph{smooth}.
\end{defn}

\begin{defn}[local right ratio]
The right ratio when the angle measuring distance approaches zero is defined as the \emph{local right ratio} at that point.  
\end{defn}

When $F$ is extremely close to $A$, the first derivative of the local right ratio approaches Figure \ref{fig:transition} \text{Right}; Otherwise, the first derivative of the local right ratio is always zero.  
Thus, the first derivatives of local right ratios may provide a measurement of the non-smoothness at a point.






\subsection{Flatness}

In a 3-dimensional Euclidean space, among all surfaces that contain any 3 points which are not collinear, there is only one Euclidean surface passing all these three points.  
Another difference is that all other surfaces need parameters to characterize them, such as the radius of a spherical surface, but Euclidean surface is free from any such parameter.  
Are Euclidean surfaces unique?  

\ifdefined\VERBOSE
\begin{figure}
\includegraphics[scale=0.5]{flat.png}
\caption{Construct squares from a flat point on a smooth surface.}
\label{fig:square}
\end{figure}
\fi

The answer lies in Theorem \ref{thm:constant right ratios}: only Euclidean surfaces are flat.

\begin{thm}
\label{thm:constant right ratios}
If a 2-sided smooth surface is globally flat at one point, then the surface is Euclidean.
\end{thm}
\ifdefined\VERBOSE
\begin{proof}
In Figure \ref{fig:square} \textsl{Left}, assume the right ratios are $\alpha^2/4$ for a point $O$. 
According to Theorem \ref{thm:right ratio floor}, $\alpha > 1$.
 
Because the right ratio is a constant in all the directions and for all the measuring distances, the region near the point has to be isotropic and isometric, with perpendicular straight lines dividing the side spaces  symmetrically of each other.

Construct $O = \overline{AC} \perp \overline{BD}$ with $|OA|=|OB|=|OC|=|OD|=1$.
Let $E, F, G$, and $H$ be the midpoints of $\overline{AB}, \overline{BC}, \overline{CD}$, and $\overline{DA}$, respectively.
\begin{enumerate}
\item $|AB|=|BC|=|CD|=|DA|=\alpha$.

\item $|\angle DAB|=|\angle ABC|=|\angle BCD|=|\angle CDA|= 1/ \alpha^2$.

\item $|\angle EAH| = |\angle DAB|: |HE|=|EF|=|FG|=|GH|=1$.

\item $O = \overline{EG} \perp \overline{FH}: |OE|=|OF|=|OG|=|OH|=1/ \alpha$.

\item From symmetry, $O = \overline{*EG*} \perp \overline{*FH*}$.
$|OE| = 1/\alpha$.

\item $|\angle HEF|=|\angle EFG|=|\angle FGH|=|\angle GHE|= 1/ \alpha^2$.
$EFGH$ is $ABCD$ scaled down by $1/\alpha$ for every geodesic.

\item From symmetry, $E = \overline{AB} \perp \overline{OE}, F = \overline{BC} \perp \overline{OF}, G = \overline{CD} \perp \overline{OG}$, and $H = \overline{DA} \perp \overline{OH}$.  
$\angle AOE| = \alpha^2/4 = \frac{1}{4} |AO|^2 /|OE|^2$.
Thus $|OE| = |AE|$, which means $\alpha^2 = 2$ or the right ratio is $1/2$.
\end{enumerate}
The space can be tiled with a square grid of any size and orientation, so that the space is Euclidean.
\end{proof}
\fi

According to Theorem \ref{thm:constant right ratios}, when the first derivative of the local right ratio is $0$, the right ratio approaches $1/2$.
Thus, the infinitesimal region surrounding each point on a smooth surface is Euclidean.
Such surface is a manifold \cite{Differential Geometry}.



\iffalse
\subsection{Congruence}

Let the right ratio be a function $r(u)$ in which $u$ is measuring distance, with $\lim_{u \rightarrow 0} r(u) = 1/2$.
\begin{enumerate}
\item Let $O = \overline{*AC*} \perp \overline{*BD*}$, $|OA|=|OB|=|OC|=|OD|= \alpha$, and  $|AB|=|BC|=|CD|=|DA|= 2 \beta$, so that $r(\alpha) = \beta^2 / \alpha^2$.

\item 
According to Figure \ref{thm:constant right ratios} right, due to symmetry, $O = \overline{*EG*} \perp \overline{*FH*}$.
let $|EF|=|FG|=|GH|=|HE|= 2 \gamma$, $|OE|=|OF|=|OG|=|OH|=\eta$, so that $r(\beta) = \gamma^2 / \eta^2$.

\item
According to Figure \ref{thm:constant right ratios} right, due symmetry, $E = \overline{*OE*} \perp \overline{*AB*}$.
If $\eta = |OE| = |EB| = \beta$, $E$ is equivalent to $O$, so that $\angle BAD$ is a right angle, and $ABCD$ is a square.
Thus, $|OE| \neq |EB|$.

\item
$r(\beta)$ is between $\frac{1}{4} \beta^2 / \alpha^2$ and 

\end{enumerate}

$\angle EBF$ cannot be right angle; Otherwise $OEBF$ becomes a square.
If $|OE| = |EB|$, then $|EH| =  \alpha$, which means that the surface is Euclidean.
\fi






\subsection{Curvature}

\begin{figure}
\includegraphics[scale=0.6]{right_ratios.png}
\caption{
The right ratios at the center of the inside surface of a torus are more than 1/2 and increase with the angle measuring distance, while the right ratios at the center of the outside surface of the torus are less than 1/2 and decrease with the angle measuring distance.  
The former has negative Gaussian curvature, while the latter has positive Gaussian curvature. 
}
\label{fig:right ratios}
\end{figure}


How a surface deviates from Euclidean surface at the same location is described by \emph{curvature}, which can be viewed easily from the Euclidean space of higher dimension.
For example, a torus has opposite bending at its inside and outside surfaces.
Mathematically, Gaussian curvature \cite{Non-Euclidean Geometry} quantifies the curvature of a 2-dimensional surface in the 3-dimensional Euclidean space, so it is an \emph{explicit} \cite{Non-Euclidean Geometry} measure of the surface.
The outside surface has positive Gaussian curvature, the inside surface had negative Gaussian curvature, while a flat surface has zero Gaussian curvature.

It is also possible to measure the curvature within the 2-dimensional surface, as an \emph{implicit} \cite{Non-Euclidean Geometry} measurement.
Figure \ref{fig:right ratios} shows the right ratios for the inside and outside surface of a torus: The right ratios are found to be more than 1/2 when Gaussian curvature is negative, less than 1/2 when Gaussian curvature is positive, and close to 1/2 when the angle measuring distance is small.
As shown in Figure \ref{fig:right ratios}, The second derivative of the local right ratios is negatively correlated with the Gaussian curvature. 
\iffalse 
For another example, on a spherical surface with radius $R$, the local right ratio is $1/2$, the first derivative is $0$, and the second derivative is $-\frac{1}{3 R^2}$, while the corresponding Gaussian curvature is $+ \frac{1}{R^2}$ \cite{Non-Euclidean Geometry}.
The absolute values for both curvature measurements are proportional to the bending: $\frac{1}{R^2}$.
\fi

\begin{defn}[curvature]
On a surface at a differentiable point, the second derivative of a local right ratio is defined as \emph{curvature} at the point for the direction bisecting the corresponding right angle.
\end{defn}

  


\iffalse
\subsection{Local Model}

\begin{figure}
\includegraphics[scale=0.6]{torus.png}
\caption{A torus is a doughnut-shaped surface.  Like any surface with rotational symmetry, a torus is characterized by an angular coordinate $\varphi$ in the poloidal direction (the blue arrow), and an angular coordinates $\theta$ in the toroidal direction (the red arrow). }
\label{fig:torus}
\end{figure}

This paper proposes a \emph{local model} to initiate finding the geodesic $\overline{AB}$ on a surface between any two points if the surface can be characterized by two orthogonal parameters in 3-dimensional Euclidean space.  
For example, when a surface has rotational symmetry, it is usually characterized by an angular coordinate $\varphi$ in the poloidal direction, and an angular coordinate $\theta$ in the toroidal direction, as shown in Figure \ref{fig:torus}.  
The torus can be characterized in Cartesian coordinates as:
\begin{align}
x(\theta,\varphi) &= (R + r \cos \theta) \cos \varphi, \\
y(\theta,\varphi) &= (R + r \cos \theta) \sin \varphi, \\
z(\theta,\varphi) &= r \sin \theta
\end{align} 

\begin{figure}
\includegraphics[scale=0.6]{local_model.png}
\caption{The local model for a torus with an aspect ratio of 3. To measure right ratios, $|OA|=|OB|$, and $\overline{AB}$ is represented by the dashed lines in the local model, and by the thin solid lines in the $(\varphi, \theta)$ parameter space.  \textsl{Left:} At the center of the outside surface, $\varphi = 180^o$. \textsl{Right:} At the center of the outside surface, $\varphi = 0^o$. }
\label{fig:local model}
\end{figure}

Built from two orthogonal parameters $(\varphi, \theta)$ in the real space, the local model is in X-Y coordinates specific for points A and B:
\begin{enumerate}
\item 
Its X-axis is a curve $\widehat{OA}$ along $\varphi$, and its Y-axis is a curve $\widehat{OB}$ along $\theta$, so that $O = \widehat{OA} \perp \widehat{OB}$. 
For example, in Figure \ref{fig:torus}, the blue arrow could be X-axis, and the red arrow could be the Y-axis.  

\item 
Every point P on the surface is mapped to a point $(X, Y)$ in the model space: $X = (R + r \cos \theta) \varphi, Y = r \theta$.  
The local model tries to flatten the surface along the axis $O = \widehat{OA} \perp \widehat{OB}$ while keep the two axis $\widehat{OA} \perp \widehat{OB}$ invariant between the real and the model spaces.  

\item 
$\overline{AB}$ can be approximated as a straight segment connecting $A$ and $B$ in the model X-Y coordinates, which can be mapped back to the real surface as an approximate geodesic in $(\theta, \varphi)$ coordinates. 
\end{enumerate}
For example, in Figure \ref{fig:local model}, the Y axis is in degree of $\theta$, the X-axis is in the equivalent length unit of the Y-axis, and the approximate geodesics corresponding to the dashed line in the model are drawn in solid thin lines, respectively. 
They show opposite bending on the inside vs outside surfaces of a torus. 
Such bending is proportional to the curvature of the surface with increased angle measuring distance. 
The approximate resulting right ratios at the centers of the inside and outside surfaces of the torus are displayed in Figure \ref{fig:right ratios}.
\fi



\section{Discussion}

\subsection{What are Euclidean Surfaces?}

This paper uses 5 axioms to establish 2-dimensional Euclidean geometry progressively:
\begin{itemize}
\item Axiom \ref{axm:measurable} requires Euclidean spaces to be measurable by distance.

\item Axiom \ref{axm:continuous} requires Euclidean spaces to be continuous and complete.

\item Axiom \ref{axm:boundless} requires Euclidean spaces to be boundless and smooth.

\item Axiom \ref{axm:monotonic} requires Euclidean spaces to be strictly monotonic.

\item Axiom \ref{axm:flat} requires the Euclidean spaces to be flat.
\end{itemize}

Each axiom has its own applicable range, and the axiom set narrows down the applicable ranges progressively, e.g., the first 4 axioms apply to non-Euclidean surfaces \cite{Non-Euclidean Geometry} as well.


\subsection{Comparison to the Existing Axiom Sets for Euclidean Geometry}

Compared with the Euclidean axiom set \cite{Euclid}, the new axiom set is mathematically more rigorous, because it is built upon the modern mathematical formal system \cite{Formal System}.  It can be viewed as a modern update of the original axioms:
\begin{itemize}
\item It restates two existing axioms as Axiom \ref{axm:continuous}, and Axiom \ref{axm:boundless}.

\item It discovers one implied assumption as Axiom \ref{axm:monotonic}.

\item It quantifies two existing axioms as Axiom \ref{axm:measurable} and Axiom \ref{axm:flat}.

\item It shows that the additional "parallel postulate" is indeed redundant.

\item It defines the straight line from the straight segment, as in \textsl{``if the line is extended to a sufficient length"} \cite{Euclid}, to establish Euclidean space locally.
\end{itemize}

Compared with the Hilbert axiom set \cite{Hilbert}, this new set is much simpler, because the new axiom set has much less primitive notions by utilizing real number set to define primitive notions in the Hilbert axiom set.  
The parallel postulate, the congruence relation, and the similarity relation are all the consequence of Axiom \ref{axm:flat}.
As shown in this paper, the new axiom set can conclude each 2-dimensional related axioms in the Hilbert axiom set.
In contrast, Axiom \ref{axm:monotonic} seems to be implied in Hilbert set.

The new axiom set requires distance as the only measurement, and derives angle measurement from distance, thus avoiding the conventional angle measurement problem \cite{Plane} \cite{Hilbert}, while also avoiding requiring the angle to be measurable independently \cite{Birkhoff}.

In the new axiom set, ``free mobility'' of geometric object is no longer a requirement \cite{Isometric}, but a deducted property of Euclidean surface.

In the new axiom set, the derivatives of the local right ratios provide an implicit and point-specific characterization of locally Euclidean, smoothness, and curvature of any surfaces.
It is a much simpler approach than conventional methods for the same purpose \cite{Non-Euclidean Geometry}\cite{Differential Geometry}\cite{Lie Group}.



\subsection{The Potential Importance of Right Ratio}

Measuring space curvature experimentally is both important and active research today.
All current methods are based on non-local parallel postulate so they can only measure integral effects of space curvature over long distance \cite{Space Curvature}.
Local right ratio provides space curvature measurement at each point, so it could be an important improvement. 

Theorem \ref{thm:constant right ratios} and Figure \ref{fig:right ratios} suggest one hypothesis: If at a point the surface is smooth, then the first derivative of the right ratio is zero, and the surface is a manifold at the point.

Right ratio enables studying of hyperbolic spaces directly.
So far, hyperbolic spaces are only studied using models \cite{Non-Euclidean Geometry}.
It may also possible to prove that if a surface is isometric and isotropic, it is either Euclidean or spherical.


\ifdefined\AUTHOR
\section{Acknowledgments}

This paper started as an answer to the naive but fundamental questions from one of the authors when she first studied Euclidean geometry, indeed showing the importance of the first thoughts in studying mathematics.  
The other author feels deeply grateful for:
\begin{itemize}
\item the wonderful teaching by Mr. Jianye Liu from The Middle School of Peking University,

\item the unique teaching for challenging status quo of knowledge, by Dr Paul Hough from Brookhaven National Lab, including the courage to reinvent wheels, either as a learning process or as an inventory method.
\end{itemize}    
The first attempt resulted in a fatally flawed self-published paper \textsl{How to Define a Flat Plane} (2017), although it is the origin for most of the ideas and approaches presented in this paper. 
Dr. Oliver Attie, an mathematician on differential geometry and bioinformatics, reviewed the paper carefully, and provided many valuable suggestions. 
Mr. Victor Aguilar, the author of \textsl{Geometry-Do} (2019), spent time reading the first paper and found the flaw.   
Mr. Arthur Knish, the president of Institute of Creative Problem Solving on Long Island, encouraged the authors to present in the school on Dec. 2016, and submit the the paper to journals. 
Many of our fiends help proof-reading the draft of the paper, with Ms. Lynn Ye in particular.
\fi


\iffalse
\section*{Work Sheet}
\subsection{Second Derivative of Right Ratio for Spherical Surface}

The Napier's rules for a spherical right triangle of angles $A, A, C=\pi/2$ and normalized opposite sides $a, a, c$:
\begin{equation}
\begin{split}
& \frac{\sin a}{\sin c} = \sin A \\
& \frac{\tan a}{\tan c} = \cos A \\
& \cos c = \cos^2 a \\
& \cos a = \frac{1}{\tan A}
\end{split}
\end{equation}

When $a, c \Longrightarrow 0$:
\begin{equation}
\begin{split}
& \frac{c}{a} \Longrightarrow \frac{\sin c}{\sin a} = \frac{1}{\sin A} \Longrightarrow \sqrt{2} \\
& \frac{d c}{d a} = 2 \cos a \frac{\sin a}{\sin c} = 2 \cos a \sin A \Longrightarrow \sqrt{2} \\
& \frac{d \sin A}{d a} \Longrightarrow 0 \\
& \frac{d^2 c}{d^2 a} \Longrightarrow -2 \sin a \sin A = 0 \\
& \frac{d^3 c}{d^3 a} \Longrightarrow -2 \cos a \sin A = -\sqrt{2} \\
& \frac{c}{a} \Longrightarrow \sqrt{2} - \frac{\sqrt{2}}{6} a^2 \\
& \frac{d c}{d a} \Longrightarrow \sqrt{2} - \frac{\sqrt{2}}{2} a^2 \\
& \frac{d^2 c}{d^2 a} \Longrightarrow - \sqrt{2} a \\
& r \equiv \frac{c^2}{4 a^2} \Longrightarrow \frac{1}{2} \\
& \frac{d r}{d a} = \frac{c}{2 a^2} \frac{d c}{d a} - \frac{c^2}{2 a^3}
    = \frac{c}{2 a^2}(\frac{d c}{d a} - \frac{c}{a}) \Longrightarrow 0 
\end{split}
\end{equation}

\begin{equation}
\begin{split}
& \frac{d^2 r}{d^2 a} = \frac{1}{2 a^2} \lbrace c \frac{d^2 c}{d^2 a} + (\frac{d c}{d a})^2
    - 4 \frac{c}{a} \frac{d c}{d a} + 3 \frac{c^2}{a^2} \rbrace \\
& \Longrightarrow \frac{1}{2 a^2} \lbrace - \sqrt{2} a c 
    + (\sqrt{2} - \frac{\sqrt{2}}{2} a^2)^2
    - 4 (\sqrt{2} - \frac{\sqrt{2}}{6} a^2) (\sqrt{2} - \frac{\sqrt{2}}{2} a^2)
    + 3 (\sqrt{2} - \frac{\sqrt{2}}{6} a^2)^2 \rbrace \\
& = \frac{1}{2 a^2} \lbrace  - \sqrt{2} a c - 2 a^2 + \frac{16}{3} a^2 - 2 a^2 \rbrace = - \frac{1}{3}
\end{split}
\end{equation}

\subsection{First Derivative of Right Ratio for Conical Surface}


In Figure \ref{fig:cone} \textit{Right}, let $|AF| = a, |\angle AFB| = \theta, |\angle CAC| = \phi < \pi$.  Let $\overline{FC*}$ be $x$ axis, and $\overline{FB*}$ be $y$ axis:
$A = a (\sin \theta, \cos \theta)$.

Let $c \equiv |FC| = |FB| = a \cos \theta + b$ and $\alpha \equiv \frac{b}{a \sin \theta}$
\begin{enumerate}
\item $\overline{*BAD*}: y = c - \alpha x$.

\item $\overline{*CDC'}: x = c + \alpha y$.

\item $D = \overline{*BAD*} \perp \overline{*CDC'*}: 
D = \frac{c}{1 + \alpha^2} \left( 1 + \alpha, 1 - \alpha \right)$, with $-1 < \alpha < 1$.

\item $|C'D| = |CD|$, so that:
\begin{align*}
C' =& \frac{c}{1 + \alpha^2} \left( 1 + 2 \alpha - \alpha^2, 2(1 - \alpha) \right)
\end{align*}
\end{enumerate}
To verify $C'$.
\begin{multline*}
\frac{|CD|^2}{c^2} = (\frac{1 + \alpha}{1 + \alpha^2} - 1)^2 + (\frac{1 - \alpha}{1 + \alpha^2})^2 
= (\frac{\alpha - \alpha^2}{1 + \alpha^2})^2 + (\frac{1 - \alpha}{1 + \alpha^2})^2
= \frac{(1 - \alpha)^2}{1 + \alpha^2}
\end{multline*}
\begin{multline*}
\frac{|C'C|^2}{c^2} = (\frac{1 + 2 \alpha - \alpha^2}{1 + \alpha^2} - 1)^2 + 
4(\frac{1 - \alpha}{1 + \alpha^2})^2 
= (\frac{2\alpha - 2\alpha^2}{1 + \alpha^2})^2 + 4(\frac{1 - \alpha}{1 + \alpha^2})^2
= 4\frac{(1 - \alpha)^2}{1 + \alpha^2}
\end{multline*}
\begin{multline*}
\frac{|C'B|^2}{c^2} = (\frac{1 + 2 \alpha - \alpha^2}{1 + \alpha^2})^2 + 
(\frac{2 - 2\alpha}{1 + \alpha^2} - 1)^2 
= (\frac{1 + 2 \alpha - \alpha^2}{1 + \alpha^2})^2 + (\frac{1 - 2 \alpha - \alpha^2}{1 + \alpha^2})^2
\\= 2\frac{(1 - \alpha^2)^2 + 4 \alpha^2}{(1 + \alpha^2)^2} = 2
\end{multline*}

\begin{equation*}
|AD|^2 = (c \frac{1 + \alpha}{1 + \alpha^2} - a \sin \theta)^2 +
(c \frac{1 - \alpha}{1 + \alpha^2} - a \cos \theta)^2
\end{equation*}
\begin{align*}
|AD|^2 \frac{(1 + \alpha^2)^2}{a^2} =
& ((\cos \theta + \sin \theta \alpha)(1 + \alpha) - \sin \theta (1 + \alpha^2))^2 + \\ 
& ((\cos \theta + \sin \theta \alpha)(1 - \alpha) - \cos \theta (1 + \alpha^2))^2 \\ =
& ((\cos \theta - \sin \theta) + (\cos \theta + \sin \theta) \alpha)^2 + \\ 
& ((\cos \theta - \sin \theta) + (\cos \theta + \sin \theta) \alpha)^2 \alpha^2 \\
|AD|^2 \frac{1 + \alpha^2}{a^2} =
& (1 - \sin 2\theta) + 2 \cos 2\theta \alpha + (1 + \sin 2\theta)\alpha^2 \\ =
& \frac{(\cos 2\theta + (1 + \sin 2\theta) \alpha)^2}{(1 + \sin 2 \theta)}
\end{align*}

To solve for $\alpha$:
\begin{align*}
& \tan \frac{\phi}{2} = \frac{|CD|}{|AD|}
= \sqrt{1 + \sin 2 \theta}
\frac{(1 - \alpha)(\cos \theta + \sin \theta \alpha)}{\cos 2\theta + (1 + \sin 2\theta) \alpha} \\
&\beta \equiv \tan \frac{\phi}{2} /\sqrt{1 + \sin 2 \theta} \\
&(\beta \cos 2\theta - \cos \theta) + (\beta (1 + \sin 2\theta) + (\cos \theta - \sin \theta)) \alpha + \sin \theta \alpha^2 = 0
\end{align*}

$\overline{C'H*}$ is $\overline{CE*}$ rotated by $\phi$.
Let $|BJ| = |C'H| = \gamma c$. 
\begin{align*}
H = c \left(\frac{1 + 2 \alpha - \alpha^2}{1 + \alpha^2} + \gamma \cos \phi, 
            \frac{2(1 - \alpha)}{1 + \alpha^2} + \gamma \sin \phi \right)
\end{align*}

The angle distance ratio $\tau$ is:
\begin{multline*}
4 (1 + \gamma)^2 \tau = \frac{|HJ|^2}{c^2} \\
= (\frac{1 + 2 \alpha - \alpha^2}{1 + \alpha^2} + \gamma \cos \phi)^2 
+ (\frac{2(1 - \alpha)}{1 + \alpha^2} + \gamma \sin \phi - (1 + \gamma))^2 \\
= (\frac{1 + 2 \alpha - \alpha^2}{1 + \alpha^2} + \gamma \cos \phi)^2 
+ (\frac{1 - 2 \alpha - \alpha^2}{1 + \alpha^2} + \gamma \sin \phi - \gamma)^2 \\
= 2 + 2\gamma^2 + 2\gamma \frac{
(1 + 2 \alpha - \alpha^2)\cos \phi + (1 - 2 \alpha - \alpha^2)(\sin \phi - 1)
}{1 + \alpha^2}
\end{multline*}
\begin{align*}
\eta \equiv \frac{
(1 + 2 \alpha - \alpha^2)\cos \phi + (1 - 2 \alpha - \alpha^2)(\sin \phi - 1)
}{1 + \alpha^2}: & \;
\tau = \frac{1}{2} \frac{1 + \eta \gamma + \gamma^2}{1 + 2 \gamma + \gamma^2} \\
\gamma \Longrightarrow 0:& \; \frac{\partial \tau}{\partial (\gamma + 1)} = \frac{\eta}{2} - 1
\end{align*}

In special cases
\begin{itemize}
\item $\phi = 0: \alpha = 1; \eta = 2; \frac{\partial \tau}{\partial \gamma} = 0$. 

\item $\phi = \pi/2: \eta = 0, \frac{\partial \tau}{\partial \gamma} = -1$. 

\item $\phi = \pi$:
\begin{align*}
\alpha =& - \frac{\cos 2\theta}{1 + \sin 2\theta} \\
\eta =& - \frac{(1 + 2 \alpha - \alpha^2) + (1 - 2 \alpha - \alpha^2)}{1 + \alpha^2}
= 2 \frac{\alpha^2 - 1}{\alpha^2 + 1} \\ =
& 2 \frac{\cos^2 2\theta - (1 + \sin 2\theta)^2}{\cos^2 2\theta + (1 + \sin 2\theta)^2}
= \frac{\cos 4\theta - \sin 2\theta}{\sin 2\theta + 1} - 1
\end{align*}

\item $\phi = 2\pi: \alpha = 1; \eta = 2; \frac{\partial \tau}{\partial \gamma} = 0$:

\item $\theta = 0$: The solution is not possible because it has limitation on $\phi$.
\begin{align*}
& \tan \frac{\phi}{2} = \frac{1 - \alpha}{1 + \alpha}; \;
  \sin \frac{\phi}{2} = \sqrt{\frac{1}{2} \frac{1 - \alpha}{1 + \alpha}}; \;
  \cos \frac{\phi}{2} = \sqrt{\frac{1}{2} \frac{1 + \alpha}{1 - \alpha}}; \; \\
& \sin \phi = 1; \;
  \cos \phi = 0 =  \frac{1}{2} \frac{1 + \alpha}{1 - \alpha} - \frac{1}{2} \frac{1 - \alpha}{1 + \alpha}
  = \frac{2\alpha}{1 - \alpha^2}; 
\end{align*}

\item $\theta = \pi/4$:
\begin{align*}
& \tan \frac{\phi}{2} = \frac{1 - \alpha^2}{2 \alpha}; \; 
  \sin \frac{\phi}{2} = \frac{1 - \alpha^2}{1 + \alpha^2}; \;
  \cos \frac{\phi}{2} = \frac{2 \alpha}{1 + \alpha^2}; \\
& \sin \phi = \frac{4 \alpha (1 - \alpha^2)}{(1 + \alpha^2)^2}; \;
  \cos \phi = \frac{4 \alpha^2 - (1 - \alpha^2)^2}{(1 + \alpha^2)^2}; \\
& (1 + \alpha^2)^3 \eta =
  (1 + 2 \alpha - \alpha^2)(-1 + 6\alpha^2 - \alpha^4) \\
&+ (1 - 2 \alpha - \alpha^2)(4\alpha - 4\alpha^3)
- (1 - 2 \alpha - \alpha^2)(1 + 2\alpha^2 + \alpha^4) \\
\eta =& \frac{-2 + 4 \alpha - 3 \alpha^2 - 7 \alpha^3 + 4 \alpha^4 + 8 \alpha^5 + \alpha^6}
             {(1 + \alpha^2)^3} 
\end{align*}
\end{itemize}
 





\fi


\begin{thebibliography}{1}

\bibitem{Euclid}
Euclid of Alexandria, 
\textit{Euclid's Elements of Geometry}, 
edited, and provided with a modern English translation, by Richard Fitzpatrick, ISBN 978-0-6151-7984-1.

\bibitem{Plane}
Charles Aboughantous, 
\textit{Euclidean Plane Geometry}, 
Universal Publishers, 2010, ISBN 978-1-5994-2822-2.

\bibitem{Old and New}
Marvin Jay Greenberg
\textit{Old and New Results in the Foundations of Elementary Plane Euclidean and Non-Euclidean Geometries}
The American Mathematical Monthly, Volume 117, Number 3, March 2010, pages 198-219.

\bibitem{Mistakes}
A. I. Fetisov, and Y.S. Dubnov, 
\textit{Proof in Geometry, with Mistake in Geometric Proofs}, 
Dover Publications, 2018, ISBN 978-0-4864-5354-5.

\bibitem{Hilbert}
David Hilbert, 
\textit{Foundations of Geometry}, 
The Open Court Publishing Co, 1902.

\bibitem{Formal System}
Daniel J Velleman, 
\textit{How to prove it: a structured approach}, 
Cambridge University, 1994.  ISBN 978-0-5216-7599-4.

\bibitem{Birkhoff}
George D. Birkhoff, 
\textit{A Set of Postulates for Plane Geometry (Based on Scale and Protractors)}, 
Annals of Mathematics, 1932, Volume 33, Page 329–345.

\bibitem{Tarski}
Alfred Tarski, and Steven Givant,
\textit{Tarski's system of geometry}, 
The Bulletin of Symbolic Logic, 5 (2): 175–214, (1999).

\bibitem{Non-Euclidean Geometry}
H.S.M. Coxeter, 
\textit{Non-Euclidean Geometry}, 
The Mathematical Association of America, 1998.  ISBN 0-88385-522-4.

\bibitem{Differential Geometry}
Chuan-Chih Hsiung, 
\textit{A First Course in Differential Geometry}, 
John Wiley and Sons, 1981. ISBN 0-471-07953-7.

\bibitem{Isometric}
Garrett Birkhoff, 
\textit{Metric foundations of geometry. I.}, 
Transaction of American Mathematics Society, Volume 55 (1944), Page 465-492.

\bibitem{Topology}
Theodore W. Gamelin, Robert Everist Greene
\textit{Introduction to Topology}
Dover Books on Mathematics, 1999, ISBN 0-486-406806-6

\bibitem{Gravitational Waves}
B.P. Abbott et al. (LIGO Scientific Collaboration and Virgo Collaboration)
\textit{GW170817: Observation of Gravitational Waves from a Binary Neutron Star Inspiral}
Physics Review Letter 119 (2017), 161101

\bibitem{Lie Group}
Hans Fredenthal,  
\textit{Lie Groups in the Foundations of Geometry }, 
Advances in Mathematics, Volume 1, Issue 2, 1964, Page 145-190.

\bibitem{Space Curvature}
Slava G. Turyshev (Jet Prop Lab, Cal Tech)
\textit{Experimental Tests of General Relativity}
https://arxiv.org/abs/0806.1731



\end{thebibliography}

% ------------------------------------------------------------------------
\end{document}
% ------------------------------------------------------------------------
