% ------------------------------------------------------------------------
% bjourdoc.tex for birkjour.cls*******************************************
% ------------------------------------------------------------------------
%%%%%%%%%%%%%%%%%%%%%%%%%%%%%%%%%%%%%%%%%%%%%%%%%%%%%%%%%%%%%%%%%%%%%%%%%%

\documentclass{birkjour}
%
%
% THEOREM Environments (Examples)-----------------------------------------
%
 \newtheorem{axm}{Axiom}
 \newtheorem{thm}{Theorem}[section]
 \newtheorem{cor}[thm]{Corollary}
 \newtheorem{lem}[thm]{Lemma}
 \newtheorem{prop}[thm]{Proposition}
 \newtheorem{cnj}[thm]{Conjecture}
 \theoremstyle{definition}
 \theoremstyle{definition}
 \newtheorem{defn}[thm]{Definition}
 \theoremstyle{remark}
 \newtheorem{rem}[thm]{Remark}
 \newtheorem*{ex}{Example}
 \numberwithin{equation}{section}


\begin{document}

%-------------------------------------------------------------------------
% editorial commands: to be inserted by the editorial office
%
%\firstpage{1} \volume{228} \Copyrightyear{2004} \DOI{003-0001}
%
%
%\seriesextra{Just an add-on}
%\seriesextraline{This is the Concrete Title of this Book\br H.E. R and S.T.C. W, Eds.}
%
% for journals:
%
%\firstpage{1}
%\issuenumber{1}
%\Volumeandyear{1 (2004)}
%\Copyrightyear{2004}
%\DOI{003-xxxx-y}
%\Signet
%\commby{inhouse}
%\submitted{March 14, 2003}
%\received{March 16, 2000}
%\revised{June 1, 2000}
%\accepted{July 22, 2000}
%
%
%
%---------------------------------------------------------------------------
%Insert here the title, affiliations and abstract:
%


\title[A New Axiom Set for 2-Dimensional Euclidean Geometry]
 {A Testable and Progressive Axiom Set for 2-Dimensional Euclidean Geometry}


%----------Author 1
\author[Chengpu Wang]{Chengpu Wang*}

\address{%
40 Grossman Street\\
Melville\\
NY 11747\\
United Stats of America}
\email{Chengpu@gmail.com}

%----------Author 2
\author{Alice Wang}
\address{40 Grossman Street\\
Melville\\
NY 11747\\
United Stats of America}
\email{alicewang05@yahoo.com}

\thanks{}

%----------classification, keywords, date
%\subjclass{Primary 51M05; Secondary 51N20}

\keywords{Euclidean, formal system}

\date{Jan 28, 2023}
%----------additions
%\dedicatory{To the students who are willing to reinvent the wheel}
%%% ----------------------------------------------------------------------

\begin{abstract}
This paper shows that the original Euclidean axiom set can be modernized as 6 axioms using propositional logic.  
The new axiom set introduces each axiom progressively, with no new concept in any of the axioms so that each axiom is testable. 

A new implicit measure called local right ratio is introduced, which determines if the surface is locally flat, smooth, and curved at each point.  
Euclidean is the only surfaces to have constant right ratios.
\end{abstract}

%%% ----------------------------------------------------------------------
\maketitle
%%% ----------------------------------------------------------------------
%\tableofcontents
\section{Introduction}

\subsection{The Existing Axiom Sets for 2-Dimensional Euclidean Geometry}

Euclidean geometry \cite{Euclid}\cite{Plane} is the origin for modern mathematics, containing an axiom set of 5 axioms. 
However, originally as a geometry on a flat plane using ruler and compass, its axiom set is regarded as insufficient \cite{Mistakes}, so it has been enhanced by a much larger axiom set of 21 axioms in 5 groups \cite{Hilbert}, which is criticized as the following by this paper:

\begin{itemize}

\item 
The enhanced axiom \textsl{``I.1. Two distinct points always completely determine a straight line.''}, mentions a straight line without defining it.  
Because there is no definition, the axiom set has a few descriptive statements for a straight line \cite{Hilbert}, at least some of which may be deductible if the straight line can be properly defined. 

\item 
The enhanced axioms \textsl{``I.1.Two distinct points always completely determine a straight line.''}, \textsl{``I.2. Any two distinct points of a straight line completely determine that line.''}, and \textsl{``I.7. Upon every straight line there exist at least two points, ...''} suggest a better approach of defining a straight line using the set language of the formal system \cite{Formal System} in mathematics. 
Using sets to define geometric objects is already a common practice \cite{Differential Geometry} \cite{Isometric}.

\item 
The enhanced axiom \textsl{``I.3. Three points not situated in the same straight line always completely determine a plane.''} hides three conceptions: dimension, flat plane, and the uniqueness of the flat plane. 
With so many new concepts in one sentence, it is hard to grasp the meaning of the axiom in the strict mathematical sense.

\item
As described in this paper, both axiom sets contain additional hinted or implicit assumptions.

\item 
The original Euclidean geometry \cite{Euclid} has one mathematical difficulty in modern time: the measurement of angles.  
It uses the right angle as the unit of measurement \cite{Mistakes}, but it does not provide any way to arbitrarily divide a right angle; thus such a unit of angle measurement can not be applied generally. 
Without angle measurement, two angles can not be compared or equalized.  
In practice, an angle is measured by the corresponding arc length, while the arc length is calculated by the angle measurement \cite{Plane}, which result in circular definitions \cite{Formal System}.  
This difficulty is inherited by the enhanced axiom set \cite{Hilbert}.

\end{itemize}

Perhaps to overcome the above conceptual difficulty of angle measurement, another existing axiom set for Euclidean geometry \cite{Birkhoff} requires both distance and angle to be independently measurable, which is different from the common practice that angle measurement is derived from distance measurement, such as by arc length or by trigonometry.  
So it is desirable to derive angle measurement from distance measurement in the axiom set.

The Euclidean, spherical, or hyperbolic geometry can also be established from metric foundation with free mobility of geometry objects \cite{Isometric}.  
The notion that metric can be the foundation for geometries \cite{Isometric} is followed by this paper.



\subsection{Goals for Reinventing The Wheel}

It is desirable if an axiom set:
\begin{enumerate}
\item introduces no new concept, so that the axiom can be better understood and tested.

\item specifies the different applicable scope of each axiom, and progresses by narrowing down the scopes.
\end{enumerate}
This paper will provide a toy model for such testable and progressive approach.

It is desirable that each property of a surface, such as smoothness, flatness, and curvature, should be defined \emph{mathematically}, and determined \emph{implicitly} rather that explicitly \cite{Non-Euclidean Geometry}. 
This paper will provide a new implicit local measure called local right ratio to provide local measures on the flatness, smoothness, and curvature at each point of a surface.
 
Euclidean space can be established locally on a globally non-Euclidean surface, such as on a manifold \cite{Non-Euclidean Geometry}.  
Thus, the parallel postulate and the straight line, which describe Euclidean globally, should be removed from the new axiom set.
In fact, the original Euclidean geometry does not use global objects such as straight lines as its foundation, but instead states \textsl{``if the line is extended to a sufficient length"} \cite{Euclid}. 




 

%\iffalse
\subsection{Educational Purpose}

In US, the public education on mathematics is not satisfactory to prepare students for science and engineering.
Most advanced students learn mathematics by themselves, such as through the requirements of AMC, AIME and USAMO competitions, which are all way ahead of the normal curriculum.
The downside of such learning for the competition purpose in rushed manner is the loss of the chance and the ability to ask naive but sometimes deep questions when encountering a new subject matter.
The virgin thoughts on a new subject matter is a major source of creativeness and understanding in mathematics, which may be more valuable than the techniques, of which mathematical competitions primarily focus on.
This paper is the learning process of both authors over several years when one author learned Euclidean geometry for the first time in a relaxed manner from the other author, when the naive but sometimes deep questions broke the for-granted understanding for 2-dimensional Euclidean geometry.
This paper will provide a thought process following the following four steps: 1. Observation, 2. Generalization, 3. Deduction, 4. Verification.
Hoping to benefit more young students learning mathematics, this paper is aimed to be understandable by high school students who have taken the AP courses on mathematics, or college students in science and engineering.
%\fi

\section{Geometrical Objects}

\begin{defn}[geometric object]
A \emph{geometric object} is a set of points.
\end{defn}

\begin{defn}[same]
If two geometric objects contain the same set of points, they are the \emph{same} geometric object.
\end{defn}

A \emph{point} is an exact location in a space but without any other properties.
As in all approaches in studying geometry \cite{Euclid} \cite{Plane} \cite{Mistakes} \cite{Hilbert} \cite{Differential Geometry} \cite{Non-Euclidean Geometry}, the concept of the point is the necessary presumption for geometries.  
In this paper, the following common conventions \cite{Formal System} are used:
\begin{itemize}
\item A point is denoted by a capital letter. 

\item A set can either be expressed as a collection of points, such as $\{A,B\}$, or as all points which satisfies certain conditions, such as $\{P: |PA|=r\}$ for a round object of radius $r$ centered at $A$.

\item A point can be in ($\in$) or not in ($\notin$) a geometric object.

\item Two geometric objects can either intersect ($\cap$) or union ($\cup$) each other. 
$+$ is used to union two sets which have at most one point in common between then.

\item $\oslash$ is the empty set.  The \emph{space} is the set of all points. 
\end{itemize}


\section{Euclidean Space Is Stable}

The original Euclidean axiom set \cite{Euclid} is in the format of stating the ability to construct simple geometric objects using deterministic rules, such as a point, a straight line segment, a circle, a triangle, etc.  
Following this tradition:

\begin{defn}[constructible]
A \emph{constructible object} is a geometric object which can be constructed using a predefined steps of deterministic rules. 
\end{defn}

Not only does the rule set of a constructible object need to be deterministic, the space in which the rule set is applied also has to be stable, both of which are necessary conditions for Axiom \ref{axm:stable}:  

\begin{axm}[stable]
\label{axm:stable}
If the rules for a constructible object run twice, the same object is created twice.
\end{axm}



\section{Euclidean Space Is Measurable}

\subsection{Distance}

The concept of distance is implied in the original axiom set for Euclidean geometry, as \textsl{``To describe a circle with any center and distance''} \cite{Euclid}, which can be viewed as using construction of circles to measure distance. 
The distance defined in this paper is equivalent to the concept of metric \cite{Differential Geometry} for generic spaces. 

\begin{defn}[distance]
Between two points $A$ and $B$, the \emph{distance} $|AB|$ is defined as a non-negative real value which satisfies the following requirements:
\begin{itemize}
\item If $A = B$, $|AB|=0$; Otherwise $|AB|=|BA|>0$;
\item For any other point $C$, $|AB| \leq |AC|+|CB|$.
\end{itemize}
\end{defn}

Immediately from the definition for distance:
\begin{thm}
\label{thm:|AB|=0}
$|AB| = 0 \;\iff\; A = B$.
\end{thm}

\subsection{Measurable}

A space is \emph{measurable} if:
\begin{axm}[measurable]
\label{axm:measurable}
There is only one distance between any two points.
\end{axm}

Applying Axiom \ref{axm:stable} to Axiom \ref{axm:measurable}:
\begin{thm}
\label{thm: stable distance}
The distance between any two points is a constant.
\end{thm}

Axiom \ref{axm:measurable} is equivalent to \textsl{``Postulate I. Space is metric''} \cite{Isometric}, in which a measurable space is defined as a \emph{metric space}.  
It is implicit in both the original \cite{Euclid} \cite{Plane} and the enhanced axiom set \cite{Hilbert}.  


\section{Euclidean Space Is Continuous}

\subsection{Straight Line Segment}
 
\begin{defn}[straight line segment]
A \emph{straight line segment} between two points $A$ and $B$ in a measurable space is defined as \cite{Differential Geometry}:
\begin{align}
\label{eqn:segment}
\overline{AB} \equiv & \{ P: |AB| = |AP| + |PB|, |AP| \in [0, |AB|]^R \} \nonumber \\
& P \in \overline{AB} \iff |AP| \in [0, |AB|]
\end{align}
\end{defn}

Some immediate conclusions of Equation \eqref{eqn:segment} are:

\begin{thm}
\label{thm:equal segments}
$\overline{AB} = \overline{BA}$
\end{thm}

\begin{thm}
\label{thm:segment inequality}
$ C \notin \overline{AB} \iff |AB| < |AC| + |CB|$
\end{thm}

\begin{thm}
\label{thm:split segment}
$C \in \overline{AB} \Longrightarrow \overline{AB} = \overline{AC} + \overline{CB}$.
\end{thm}

A geodesic \cite{Non-Euclidean Geometry} is a curve representing the shortest path between two points in a surface.
Equation \eqref{eqn:segment} shows that a straight line segment is a geodesic in a Euclidean space.
Theorem \ref{thm:segment inequality} and Theorem \ref{thm:split segment} have an important implication: any segment of a geodesic is also a geodesic.

Equation \eqref{eqn:line} is equivalent to Equation \eqref{eqn:segment}:
\begin{equation}
\label{eqn:line}
\lim_{n \to \infty} \{ M: |AB| = |AM| + |MB|, |AM| = |AB|\frac{m}{2^n}, m=0,1,...2^n \}
\iff \overline{AB}
\end{equation}
\begin{proof}
The left of Equation \eqref{eqn:line} denies any gap in $[0, |AB|]$:
\begin{enumerate}
\item 
Suppose the range $[0, |AB|]$ has only one gap in between: $(|AC_1|, |AC_2|)$. 
From $|AB| = |AC| + |CB|$, the range gap becomes $(|AC_1|, |AB| - |AC_1|)$ with $|AC_1| < |AB|/2$, which means that the midpoint between $A$ and $B$ does not exist, so that the assumption contradicts the left of Equation \eqref{eqn:line}.  
This case can extend to any odd number of gaps in $[0, |AB|]$.

\item 
Suppose the range $[0, |AB|]$ has only two gaps in between: $(|AC_1|, |AB| - |AC_2|)$ and $(|AC_2|, |AB| - |AC_1|)$, which means $|AC_1| < |AB|/2 < |AC_2|$, or this case conflicts with Equation \eqref{eqn:line} when $n=1$. 
This case can extend to any even number of gaps in $[0, |AB|]$.
\end{enumerate}
\end{proof}

Equation \eqref{eqn:line} is not only the minimal requirement for a line segment, but also more achievable and testable.  




\subsection{Topological Properties}

As the equivalent of \textsl{``To draw a straight line from any point to any point''} in the original Euclidean geometry \cite{Euclid}:
\begin{axm}[continuous]
\label{axm:continuous}
There exists at least one straight line segment between any two points.
\end{axm}

From the definition, $\overline{AB}$ is isomorphic to $[0, |AB|]^R$.
Axiom \ref{axm:continuous} extends this relation to any two points in the Euclidean space:
\begin{itemize}
\item The space is continuous, dense, complete, compact, and connected \cite{Isometric}, with each point separable from others according to Theorem \ref{thm:|AB|=0}.

\item Same as $[0, |AB|]$, $\overline{AB}$ has inner convexity \cite{Isometric}, which is the enhanced axiom \textsl{``II.1. If A and C are two points of a straight line, then there exists at least one point B lying between A and C''} \cite{Hilbert}.

\item Same as $[0, |AB|]$, a straight line segment is an ordered group of points \cite{Isometric}, as in the enhanced axiom \textsl{``II.3. Of any three points situated on a straight line, there is always one and only one which lies between the other two.''} \cite{Hilbert}. 
\end{itemize}

Axiom \ref{axm:continuous} allows more than one straight line segments of the same length to connect any two point, such as between a pair of poles on a spherical surface, or on a conical surface.
To distinguish one straight line segment from another, Axiom \ref{axm:continuous} has one implied necessary condition \cite{Isometric}:

\begin{thm}
\label{thm:local unique segment}
When $|AB|$ is sufficiently small, $\overline{AB}$ is unique.
\end{thm}



\subsection{Round}

As the embodiment of the original axiom \textsl{``To describe a circle with any center and distance''} \cite{Euclid} but not limited to 2-dimensional:

\begin{defn}[round]
A \emph{round} geometric object is $\overline{A : r} \equiv \{ P: |PA| = r \} $, in which the point $A$ is the \emph{center}, while the non-negative real number $r$ is the \emph{radius}. 
\end{defn}

A round geometric object thus separates the space into three parts:

\begin{defn}[inside]
$B$ is \emph{inside} $\overline{A : r}$ if $|AB|<r$.
\end{defn}

\begin{defn}[outside]
$B$ is \emph{outside} $\overline{A : r}$ if $|AB|>r$.
\end{defn}

From Theorem \ref{thm:segment inequality}:

\begin{thm}
\label{thm:outside rounds}
$r_A + r_B < |AB| \iff \overline{A : r_A}$ and $\overline{B : r_B}$ are outside of each other.
\end{thm}

\begin{thm}
\label{thm:inside round}
$|AB| < r_B - r_A \iff \overline{A : r_A}$ is inside $\overline{B : r_B}$.
\end{thm}

Let $\{\overline{AB}\}$ be the set of all the unique straight line segments connection $A$ and $B$:

\begin{thm}
\label{thm:inclusive outside round}
$|AB| = r_A + r_B \iff \overline{A : r_A}$ and $\overline{B : r_B}$ are outside of each other except for $\{C: |AC| = r_A, C \in \overline{AB}, \overline{AB} \in \{\overline{AB}\} \}$.
\end{thm}

\begin{thm}
\label{thm:inclusive inside round}
$|AB| = r_A - r_B \iff \overline{B : r_B}$ is inside $\overline{A : r_A}$ except for 
$\{C: |AC| = r_A, C \in \overline{AB}, \overline{AB} \in \{\overline{AB}\} \}$.
\end{thm}

\begin{thm}
\label{thm:intersecting rounds}
$|r_A - r_B| < |AB| < r_A + r_B \iff 
\overline{A : r_A} \cap \overline{B : r_B} \neq \oslash$, and $(\overline{A : r_A} \cap \overline{B : r_B}) \cap \overline{*AB*} = \oslash$ for $\overline{AB} \in \{\overline{AB}\}$.
\end{thm}

\begin{proof}
The exception of Theorem \ref{thm:outside rounds}, \ref{thm:inside round}, \ref{thm:inclusive outside round} and \ref{thm:inclusive inside round} suggests Theorem \ref{thm:intersecting rounds}.
\end{proof}






\section{Euclidean Space Is Boundless And Smooth}

As stated originally: \textsl{``To extend a finite straight line continuously in a straight line''} \cite{Euclid}:

\begin{axm}[boundless and smooth]
\label{axm:ray}
For any $\overline{AB}$, $\exists C, D \not \in \overline{AB}: A \in \overline{CB}, B \in \overline{AD}$.
\end{axm}

Axiom \ref{axm:ray} excludes the space to have an closed boundary, requiring the space to be \emph{boundless}.

On a conical surface, any straight line segment with the sharp point of the cone as one end can not be extended at that end.
Thus, Axiom \ref{axm:ray} also excludes such sharp points and implies the space to be \emph{smooth}.


\subsection{Ray}

\begin{defn}[ray]
When $\overline{AB}$ is unique, a \emph{ray} starting from $A$ passing through $B$ is:
\begin{equation*}
\overline{AB*} \equiv \overline{AB} + \{ P: |AP| = |AB| + |BP| \}.  
\end{equation*}
\end{defn}

Immediately from the definition:

\begin{thm}
\label{thm:equal rays}
$C \in \overline{AB*}$ and $\overline{AC}$ is unique $\Longrightarrow \overline{AC*} = \overline{AB*}$.
\end{thm}

\begin{thm}
\label{thm:ray cross round}
For any ray $\overline{AB*}$ and radius $r$, $\exists! C: \{C\} = \overline{AB*} \cap \overline{A : r}$.
\end{thm}


\subsection{Straight Line}

\begin{defn}[straight line]
When $\overline{AB}$ is unique, a \emph{straight line} passing $A$ and $B$ is:
\begin{equation*}
\overline{*AB*} \equiv \{ P: |PB| = |PA| + |AB| \} + \overline{AB} + \{ P: |AP| = |AB| + |BP| \}.  
\end{equation*}
\end{defn}

Immediately from the definition:

\begin{thm}
\label{thm:ray in line}
$C \in \overline{AB}$ and $\overline{AC}$ is unique $\Longrightarrow \overline{*AB*} = \overline{*AC} + \overline{CB*}$.
\end{thm}

\begin{thm}
\label{thm:equal line}
$C,D \in \overline{*AB*}$ and $\overline{CD}$ is unique $\Longrightarrow \overline{*CD*} = \overline{*AB*}$.
\end{thm}

\begin{thm}
\label{thm:non-collinear}
$C \notin \overline{*AB*} \Longrightarrow B \notin \overline{*CA*}$ and $A \notin \overline{*BC*}$.
\end{thm}

The definition of $\overline{*AB*}$ satisfies the enhanced axiom \textsl{``I.1. Two distinct points always completely determine a straight line''} \cite{Hilbert}, and the enhanced axiom \textsl{``I.7. Upon every straight line there exist at least two points,''} \cite{Hilbert}.
Theorem \ref{thm:equal line} is equivalent to the enhanced axiom \textsl{``I.2. Any two distinct points of a straight line completely determine that line''} \cite{Hilbert}.






\section{Dimension}

\subsection{2-Sided 2-Dimensional Space}

From observation, a straight line divides a flat plane, a cylindrical surface, a conical surface, a spherical surface, or a hyperbolic surface into two separate parts:

\begin{defn}[opposite side]
if $C, D \notin \overline{*AB*}$, and $\overline{CD} \cap \overline{*AB*} \neq \oslash$, $C$ and $D$ are defined as on the \emph{opposite sides} of $\overline{*AB*}$ , which is denoted as $C, D \div \overline{*AB*}$.
\end{defn}

\begin{defn}[2-dimensional space]
When $C, D \div \overline{*AB*}$, the \emph{2-dimensional space} $\{ C, D \div \overline{*AB*} \}$ is $ \overline{*AB*} \cup \{P: P, C \div \overline{*AB*}\} \cup \{P: P, D \div \overline{*AB*}\}$.
\end{defn}

\begin{defn}[2-sided 2-dimensional space]
If a $\{ C, D \div \overline{*AB*} \}$ further satisfies $\{P: P, C \div \overline{*AB*}\} \cap \{P: P, D \div \overline{*AB*}\} = \oslash$, it is a \emph{2-sided 2-dimensional space} $\{ C, D \div \overline{*AB*} \}_2$.  
\end{defn}

\begin{defn}[same side]
By definition, in a $\{ C, D \div \overline{*AB*} \}_2$, $C, E \div \overline{*AB*} \Longrightarrow \overline{DE} \cap \overline{*AB*} = \oslash$, or $D$ and $E$ are on the \emph{same side} of $\overline{*AB*}$.
\end{defn}

The 2-dimensional space satisfies the enhanced axiom \textsl{``I.5. If two points of a straight line a lie in a plane, then every point of the straight line lies in the plane''} \cite{Hilbert}. 
The 2-sided 2-dimensional space satisfies the enhanced axiom \textsl{``II, 5. Let A, B, C be three points not lying in the same straight line and let a be a straight line lying in the plane ABC and not passing through any of the points A, B, C. Then, if the straight line a passes through a point of the segment AB, it will also pass through either a point of the segment BC or a point of the segment AC. ''} \cite{Hilbert}. 

The 2-sided surfaces exclude more complex surfaces such as Mobius ring and Klein bottle \cite{Non-Euclidean Geometry}.
One conclusion of Theorem \ref{thm:local unique segment} is Theorem \ref{thm:line intersect line}, which says that any 2-dimensional space can also have small enough 2-side area if the surface satisfied Axiom \ref{axm:continuous}.
\begin{thm}
\label{thm:line intersect line}
When $|AB|$ is sufficiently small, 
$\overline{AB} \cap \overline{*CD*} \neq \oslash$ and $\overline{AB} \not \subset \overline{*CD*} 
\Longrightarrow \exists! E: \{E\} = \overline{AB} \cap \overline{*CD*}$.
\end{thm}
%Thus, on large scale vs on small scale, the space can be different in topology.
%, so that the characterization of a space should related to certain distance scale, which makes such characterization intrinsically non-global.
A 2-sided 2-dimensional space is simply called a \emph{surface} in this paper.

\subsection{The Existence of 2-Sided 2-Dimensional Spaces}

\begin{figure}
\includegraphics[scale=0.6]{intersecting_circles.png}
\caption{Two intersecting circles.}
\label{fig:intersecting circles}
\end{figure}

Theorem \ref{thm:intersecting rounds} needs a 2-dimensional specification. 
From observation, while a straight line segment remains the same in spaces of all dimensions, a round object has different embodiments in spaces of different dimensions, such as a circle in 2-dimensional space, and a sphere in 3-dimensional space.  
Two circles intersect at two points according to Figure \ref{fig:intersecting circles}, while two spheres intersect at one circle. 
Thus, Figure \ref{fig:intersecting circles} can be used to judge 2-sided 2-dimensional space. 
 
\begin{axm}[2-sided 2-dimensional]
\label{axm:2D}
$C \not \in \overline{*AB*}$ with $\overline{AB}$, $\overline{AC}$ and $\overline{BC}$ all unique
$\Longrightarrow \exists! D: \{C, D\} = \overline{A : |AC|} \cap \overline{B : |BC|}$ 
and $\{ C,D \div \overline{*AB*} \}_2$.
\end{axm}

Axiom \ref{axm:2D} satisfies the enhanced axioms \textsl{``I.3. Three points not situated in the same straight line always completely determine a plane.''}, and \textsl{``I.4. Any three points of a plane, which do not lie in the same straight line, completely determine that plane.''} \cite{Hilbert}.


\subsection{Tangent}

\begin{figure}
\includegraphics[scale=0.75]{tangent.png}
\caption{The tangent relation between $C \not \in \overline{*AB*}$.}
\label{fig:tangent}
\end{figure}

In Figure \ref{fig:tangent}, if $C \not \in \overline{*AB*}$, when $r$ increases from $0$, $\overline{C:r}$ may have $0$, $1$ or $2$ intersection points with $\overline{*AB*}$:  

\begin{defn}[tangent]
For $C \not \in \overline{*AB*}$, if $\exists! D: \{D\} = \overline{C:|CD|} \cap \overline{*AB*}$, then $D$ is the \emph{tangent} of $C$ on $\overline{*AB*}$, which is denoted as $D = C \perp \overline{*AB*}$.  
\end{defn}

If $C \not \in \overline{*AB*}$, will $\exists! D: D = C \perp \overline{*AB*}$ as demonstrated in Figure \ref{fig:tangent}? 
The answer is: not in all the cases, such as on a spherical surface when $C$ is the corresponding pole for $\overline{*AB*}$, or on an otherwise flat surface with partial spherical bumps.
Also, the surface can be constructed in any way so that $\{|CE|: E \in \overline{*AB*}\}$ may have more than one local minimums.



\subsection{Perpendicular}

\begin{figure}
\includegraphics[scale=0.75]{perpendicular.png}
\caption{Perpendicular lines.}
\label{fig:perpendicular}
\end{figure}

\begin{defn}[perpendicular]
For $\{E\} = \overline{AB} \cap \overline{CD}$, if for every $F \in \overline{AB}$ and $F \neq E$, $E = F \perp \overline{CD}$, then $\overline{AB}$ is \emph{perpendicular} to $\overline{CD}$ at $E$, which is denoted as $E = \overline{AB} \perp \overline{CD}$, and further $E = \overline{*AB*} \perp \overline{*CD*}$.
\end{defn}

\begin{thm}
\label{thm:perpendicular ray}
$E = B \perp \overline{CD} \Longrightarrow E = \overline{EB} \perp \overline{CD}$.
\end{thm}
\begin{proof}
In Figure \ref{fig:perpendicular} \textsl{Left}, from $E = B \perp \overline{CD}$, construct $\overline{B:|EB|}$ so that $\{E\} = \overline{B:|BE|} \cap \overline{CD}$.  
For any $B' \in \overline{BE}$, construct $\overline{B':|EB'|}$.
According to Theorem \ref{thm:inclusive inside round}, $\overline{B':|EB'|}$ is inside $\overline{B:|EB|}$ except at $E$.  Thus $\{E\} = \overline{B':|EB'|} \cap \overline{CD}$ and $E = \overline{EB} \perp \overline{CD}$. 
\end{proof}

\begin{thm}
\label{thm:perpendicular line}
$A, B \div \overline{*CD*}: E = \overline{EB} \perp \overline{CD} = \overline{EA} \perp \overline{CD} \Longrightarrow A \in \overline{*EB*}$.
\end{thm}
\begin{proof}
In Figure \ref{fig:perpendicular} \textsl{Right}, Let $A,B \div \overline{*CD*}$.
Construct $\overline{A:|EA|}$ and $\overline{B:|EB|}$, which are on the opposite sides of $\overline{*CD*}$ except intersecting at $E$. 
According to Theorem \ref{thm:inclusive outside round}, $A \in \overline{*EB*}$.
\end{proof}

Theorem \ref{thm:perpendicular ray} and \ref{thm:perpendicular line} together describe $E = \overline{AB} \perp \overline{CD}$.

\begin{figure}
\includegraphics[scale=0.75]{perpendicular_2D.png}
\caption{Construct a perpendicular line from a point of another line.}
\label{fig:perpendicular_2D}
\end{figure}

\begin{thm}
\label{thm:tangent}
In each $\{C,D \div \overline{*AB*}\}_2$, $\exists! \overline{*AE*}: A = \overline{*AB*} \perp \overline{*AE*}$.
\end{thm}

\begin{proof}
In Figure \ref{fig:perpendicular_2D}, for simplicity, let $\{C, D \div \overline{*AB*}\}_2$ further satisfies $\{C, D\} = \overline{B:|AB|} \cap \overline{A:|AC|}$ according to Axiom \ref{axm:2D}.
Let $\{D, F\} = \overline{DA*} \cap \overline{A:|AC|}$, and construct $\overline{CF}$.
$F$ is outside $\overline{B:|AB|}$. 
Denote $C$ as $G$ and $F$ as $H$, respectively:
\begin{enumerate}
\item Let the mid point of $\overline{GH}$ be $E'$.
\begin{itemize}
\item If $A = B \perp \overline{*AE'*}$, let $E'$ be both the new $G$ and $H$.

\item If $\overline{*AE'*}$ intersects $\overline{B:|AB|}$ between $A$ and $C$, let $E'$ be the new $G$.

\item If $\overline{*AE'*}$ intersects $\overline{B:|AB|}$ between $A$ and $D$, let $E'$ be the new $H$.
\end{itemize}

\item Repeat the about process, until $|GH| = 0 \Longrightarrow E = G = H$
\end{enumerate}
The above procedure construct the only $\overline{*AE*}$.
\end{proof}

According to Theorem \ref{thm:tangent}, in $\{C,D \div \overline{*AB*}\}_2$, for $E \in \overline{*AB*}$, only one $\overline{*C'D'*}$ can be constructed so that  $E = \overline{*AB*} \perp \overline{*C'D'*}$, which gives the meaning of the ``2-dimensional space'' in $\{C,D \div \overline{*AB*}\}_2$.






\subsection{Reflective}

\begin{defn}[reflection]
If $E = \overline{CD} \perp \overline{*AB*}$ with $C,D \div \overline{*AB*}$, and $|CE|=|DE|$, then $C$ and $D$ are \emph{reflection} of each other along $\overline{*AB*}$.
\end{defn}

The reflection relation can be extended to two geometric objects.



\subsection{Angle Measurement}

When any two rays $\overline{AB*}$ and $\overline{AC*}$ start from a same point $A$, they form an \emph{angle} $\angle BAC$, in which point $A$ is called the \emph{vertex}, while $\overline{AB*}$ and $\overline{AC*}$ are called two \emph{arm}s of the angle.  
Figure \ref{fig:intersecting circles} contains several such angles, include  $\angle BAC$.

\begin{defn}[angle distance ratio]
To measure an angle:
\begin{enumerate}
\item Use a positive value called \emph{angle measuring distance} $u$ to construct $\overline{A : u}$, which intersects $\overline{AB*}$ and $\overline{AC*}$ at $B$ and $C$ respectively, according to Theorem \ref{thm:ray cross round}.

\item The \emph{angle distance ratio} is defined as: $|\angle BAC|_u \equiv |BC|^2/(2u)^2 \in [0,1]$, with $0$ for a ray and $1$ for a straight line.  
\end{enumerate}
\end{defn}

The angle distance ratio at the same angle measuring distance measures how large an angle is. 
An an angle can be constructed if any of its angle distance ratios is known, such as constructing $\angle CAD$ or $\angle CBD$ in Figure \ref{fig:intersecting circles}.





\section{Euclidean Space Is Flat}

\begin{defn}[right ratio]
When $A = \overline{*BAD*} \perp \overline{*CAE*}$, the angle distance ratio for each of the angles $\angle BAC$, $\angle BAE$, $\angle DAE$, and $\angle DAC$ is defined as a \emph{right ratio}.  
\end{defn}

\begin{thm}
\label{thm:right ratio floor}
Any right ratio is larger than $1/4$.
\end{thm}
\begin{proof}
When $B = \overline{AB} \perp \overline{BC}$, if $|AB|=|BC|=u$, then $|AC| > u$.
\end{proof}

One of the original axioms for Euclidean geometry \cite{Euclid} is \textsl{``That all right angles are equal to one another''}. This requirement can be quantified by:
\begin{defn}[2-dimensional Euclidean space]
\label{def: Euclidean flat}
A \emph{2-dimensional Euclidean space} is a 2-sided 2-dimensional space in which all right ratios are $1/2$.
\end{defn}

The above definition can be viewed as the last axiom necessary to define 2-dimensional Euclidean spaces.

\begin{defn}[flat]
If all right ratios on a surface are the same constant, the surface is \emph{flat}.
\end{defn}

Theorem \ref{thm:constant right ratios} will prove that only Euclidean surfaces are flat.


\subsection{Square}

\begin{figure}
\includegraphics[scale=0.5]{squares.png}
\caption{ \textsl{Left:} Form a square in a 2-dimensional Euclidean space. \textsl{Right:} Stack identical squares side by side to form rectangles }
\label{fig:squares}
\end{figure}

In Figure \ref{fig:squares} \textsl{Left}, let $O = \overline{AB} \perp \overline{CD}$ with $|OA|=|OB|=|OC|=|OD|$.  
$|AC|=|CB|=|BD|=|DA|=\sqrt{2}|OA|$, so that $|\angle ACB|=|\angle CBD|=|\angle BDC|=|\angle DAC|=1/2$, or the geometric object $ACBD$ is a square.  

\begin{defn}[square]
A \emph{square} is formed by four sides of equal length, with the adjacent sides perpendicular to each other.
\end{defn}

\begin{thm}
\label{thm:square}
Squares of any size and orientation can be constructed anywhere on a Euclidean surface.
\end{thm}


\subsection{Rectangle}

As shown in Figure \ref{fig:squares} \textsl{Right}, when squares $ABCD$ and $CDEF$ are stacked along side $\overline{CD}$, $|\angle ADC|=|\angle EDC|=1/2$, so that $D = \overline{AE} \perp \overline{CD}$ according to Theorem \ref{thm:tangent}.  
The geometric object $ABFE$ is a rectangle.
By stacking two squares in a row, the rectangle ABFE has $|AE|=|BF|=2|AB|=2|EF|$, or a side ratio of $2/1$.  
By stacking $M$ squares in each row and $N$ squares in each column, a rectangle of size ratio $M/N$ can be constructed.  
According to Hurwitz's theorem \cite{Formal System}, any irrational number can be approximated by a rational number to any precision.
A square is a special case of a rectangle.

\begin{defn}[rectangle]
A \emph{rectangle} is formed by four sides of equal length between non-adjacent sides, with adjacent sides perpendicular to each other.  
\end{defn}

\begin{thm}
\label{thm:rectangle}
Rectangles of any size and orientation can be constructed anywhere on a Euclidean surface.
\end{thm}


\subsection{The ``Parallel Postulates''}

\begin{defn}[parallel]
If two straight lines are both perpendicular to a third straight line, these two straight lines are \emph{parallel} to each other, and the distance between the two intersection points on the third straight line is defined as \emph{the distance between the two parallel straight lines}. 
\end{defn}

According to the definition, any two opposite sides of a rectangle or a square are parallel to each other. 
In Figure \ref{fig:squares} \textsl{right}, an arbitrary rectangle $ABHG$ is stacked by another arbitrary rectangle $ABFE$ along $\overline{AB}$, so that $G = \overline{GH} \perp \overline{*GE*}$, $H = \overline{GH} \perp \overline{*HF*}$,  $A = \overline{AB} \perp \overline{*GE*}$, $B = \overline{AB} \perp \overline{*HF*}$, $E = \overline{EF} \perp \overline{*GE*}$, and $F = \overline{EF} \perp \overline{*HF*}$, with $|GH|=|AB|=|EF|$. 
Thus the distance between the two parallel straight lines $\overline{*GE*}$ and $\overline{*HF*}$ is a constant. 
Because the two parallel lines have a constant distance between them, they satisfy the ``parallel postulates'', such as the enhanced axiom \textsl{``III. In a plane there can be drawn through any point A, lying outside of a straight line a, one and only one straight line which does not intersect the line a. 
This straight line is called the parallel to a through the given point A.''} \cite{Hilbert}.  

Because the parallel postulate is not listed together with the first four Euclidean postulates as the presumptions in the Element \cite{Euclid}, but as a much later discovery \cite{Plane}, historically, there were many failed attempts to deduce the parallel postulate from the first four Euclidean postulates, until it was recognized that spherical or hyperbolic surfaces also satisfy the first four Euclidean postulates \cite{Non-Euclidean Geometry}.
The secret to let the first four Euclidean postulates to specify Euclidean surfaces lies in how to interpret the forth Euclidean postulate \cite{Euclid} \textsl{``That all right angles are equal to one another''}:
\begin{itemize}
\item If it means to be equal in all distance-scale, so that a right angle equals to itself in all different distance-scale, then the space is similar and it is Euclidean.  
Definition \ref{def: Euclidean flat} implies such equality.  

\item If it means to be equal only in each distance-scale, then the space could be either Euclidean or spherical or hyperbolic.
\end{itemize}  
Perhaps it is necessary to trace back to the original ancient Greek text to see if Euclid himself realized these differences.  
  


\section{Euclidean Space is Homogeneous, Isometric, Isotropic, Reflective, and Similar}

\subsection{Measuring Grid}

\begin{figure}
\includegraphics[scale=0.6]{net.png}
\caption{A square grid measures three triangles.}
\label{fig:net}
\end{figure}

The stacking of identical squares can be continuous in both directions to from a measuring grid, as shown in Figure \ref{fig:net}:

\begin{itemize}
\item 
Because 2-dimensional Euclidean space can be divided into identical squares, the geometric objects from different parts of the space can be compared directly. 
One example of this comparison is the Archimedes' axiom in the enhanced axiom set \cite{Hilbert}.
This means that 2-dimensional Euclidean space is \emph{homogeneous} \cite{Non-Euclidean Geometry} and \emph{isometric} \cite{Non-Euclidean Geometry}.

\item 
A measuring grid has only two parameters: the orientation and the size of each square, so that two measuring grids can be compared after enlarging and rotating one of them if necessary.  
For example, in Figure \ref{fig:net}, $\triangle A'B'C'$ matches $\triangle ABC$ when $\triangle A'B'C'$ is enlarged by 2-fold, then rotated by a right angle. 
This means that 2-dimensional Euclidean space is \emph{isotropic} \cite{Non-Euclidean Geometry} and \emph{similar} \cite{Non-Euclidean Geometry}.

\item 
Two geometric objects can also be reflection of each other along a grid line, such as $\triangle A'B'C'$ and $\triangle A"B'C'$.  
This means that 2-dimensional Euclidean space is \emph{reflective} \cite{Non-Euclidean Geometry}.

\item 
By counting the grid squares which a geometric object occupies, a measuring grid defines the \emph{area} of the geometric object \cite{Plane}, e.g., $\triangle ABC$ occupies 12 squares in Figure \ref{fig:net}.  
The concept of area leads to Pythagorean Theorem in 2-dimensional Euclidean space \cite{Plane}.

\item 
A measuring grid obeying the Pythagorean Theorem allows for the definition of Cartesian coordinates, which is the foundation for explicit geometries studying non-Euclidean spaces \cite{Differential Geometry}\cite{Non-Euclidean Geometry}.
\end{itemize}

\begin{thm}
\label{thm:square net}
On a Euclidean surface, a grid of identical squares can be composed of any orientation and any size starting from any point.
\end{thm}

The remaining question is how to apply Axiom \ref{axm:stable} to this space with such measuring grids. 



\subsection{Congruence and Similarity}

Because a Euclidean surface is isometric and isotropic, the length of a straight line segment is invariant under translation and rotation.  
An angle distance ratio thus each angle itself is also invariant.
\begin{defn}[geometric properties]
The following types of true statements about a geometric object form the \emph{geometric properties} of the object:
\begin{itemize}
\item \emph{distance-related} properties: including the length of any straight line segment, and any other measurement derived from the above lengths.

\item \emph{side-related} properties: on which side of each straight line segment any other point exist.
Each side related property chooses one of the two reflection relations.

\item \emph{ratio-related} properties: including any ratio derived from the length related properties, and the angle distance ratio of any angle.
\end{itemize}
\end{defn}
  
For example, $\triangle ABC$ in Figure \ref{fig:intersecting circles} has the following properties:
\begin{itemize}
\item $|AB|$, $|BC|$, and $|CA|$ are distance-related properties.
$|AB| + |BC| + |CA|$ can be defined as the perimeter of $\triangle ABC$, which becomes another distance-related property.  

\item On which side of $\overline{*AB*}$ point C exists is a side-related property.

\item $|AB|/|BC|$ and $|\angle CAB|$ are ratio-related properties.
\end{itemize}

\begin{defn}[repeatable]
If a constructible object can be constructed by a \emph{first rule} that contains a ray, which is called the \emph{starting ray} with its \emph{starting point}, while all other rules only relate to each other including the first rule, then the constructible object is called a \emph{repeatable} object.
\end{defn}
 
A repeatable object adds two new types of geometric properties:
\begin{itemize}
\item 
The starting point decides \emph{location-related} properties. The difference in location-related properties can be expressed as a translation \cite{Plane}.

\item 
The direction of the starting ray decides \emph{orientation-related} properties. The difference in orientation-related properties can be expressed as a rotation \cite{Plane}.
\end{itemize}


One example is to construct $\triangle ABC$ given the three side lengths using $\overline{AB*}$ as the starting ray in Figure \ref{fig:intersecting circles}.

From the properties of the measuring grid, Axiom \ref{axm:stable} can be applied:

\begin{thm}
\label{thm:repeated}
On a Euclidean surface, if the construction rules of two repeatable objects differ only in the first rule, then they have identical corresponding geometric properties except 1) a possible difference in location-related properties, and 2) a possible difference in orientation-related properties.
\end{thm}

\begin{thm}
\label{thm:reflective}
On a Euclidean surface, if the construction rules of two repeatable objects differ only in A) the first rule, and B) the side of the starting ray on which to carry out all the other rules, then they have identical geometric corresponding properties except for 1) a possible difference in location-related properties, and 2) a possible difference in orientation-related properties, and 3) opposite side-related properties.
\end{thm}

\begin{thm}
\label{thm:repeated and scaled}
On a Euclidean surface, if the construction rules of two repeatable objects differ only in A) the first rule, and B) each distance to be scaled by a positive real constant in all the other rules, then they have identical corresponding geometric properties except for 1) a possible difference in location-related properties, and 2) a possible difference in orientation-related properties, and 3) each distance-related property to be scaled by the constant.
\end{thm}

\begin{thm}
\label{thm:reflective and scaled}
On a Euclidean surface, if the construction rules of two repeatable objects differ only in A) the first rule, and B) each distance to be scaled by a positive real constant in all the other rules, and C) the side of the starting ray on which to carry out all the other rules, then they have identical corresponding geometric properties except for 1) a possible difference in location-related properties, and 2) a possible difference in orientation-related properties, and 3) each distance-related property to be scaled by the constant, and 4) opposite side-related properties.
\end{thm}

The combination of Theorem \ref{thm:repeated} and Theorem \ref{thm:reflective} states \emph{congruence} relations between two geometric objects. 
All axioms of group IV in the enhanced axiom set \cite{Hilbert} are special cases for the the general congruence relation defined by Theorem \ref{thm:repeated} and Theorem \ref{thm:reflective}.

The combination of Theorem \ref{thm:repeated and scaled} and Theorem \ref{thm:reflective and scaled} states \emph{similarity} relations between two geometric objects.  

Measuring grids can also be established in spherical space using equilateral triangles or polygons, but only for selected sizes \cite{Non-Euclidean Geometry}.  
Thus, a spherical space can have congruence relations but not similarity relations between two geometric objects.


%\iffalse
\subsection{Some Basic Operations in Euclidean Geometry}

Figure \ref{fig:intersecting circles} contains some basic operations in 2-dimensional Euclidean geometry:
\begin{itemize}
\item For $C \notin \overline{*AB*}$, find $D: D = C \perp \overline{*AB*}$.

\item For $C \in \overline{*AB*}$, find $\overline{*CD*}: C = \overline{*CD*} \perp \overline{*AB*}$.

\item Bisect a straight line segment such as $\overline{CD}$.

\item Bisect an angle such as $\angle CAD$.

\item Construct a pair of reflective points $C$ and $D$ along $\overline{*AB*}$. 
\end{itemize}
%\fi


\section{Beyond Euclidean Geometry}

\subsection{Local Euclidean Region}

\begin{figure}
\includegraphics[scale=0.6]{cone.png}
\caption{ 
\textsl{Left:} On a conical surface with tip $A$, two geodesics connect $B$ and $C$.
\textsl{Right:} The angle measuring distance is limited to $|FB|$ for the right ratio at $F$ to be $1/2$.
}
\label{fig:cone}
\end{figure}

\begin{figure}
\includegraphics[scale=0.4]{conical_transition.png}
\caption{ 
\textsl{Left:} Normalized transitional distance of right ratio v.s. conical cut angle from 0 to 180 degrees, and measuring angle from 5 to 85 degrees on a conical surface.
\textsl{Right:} Normalized transitional derivative of right ratio v.s. conical cut angle from 0 to 180 degrees, and measuring angle from 5 to 85 degrees on a conical surface.
}
\label{fig:transition}
\end{figure}

It is possible that the Euclidean surface be established only within a region, such as on a conical surface.  

Figure \ref{fig:cone} \textit{Left} shows a conical surface with tip $A$, while Figure \ref{fig:cone} \textit{Right} shows how to construct the conical surface by cutting and stitching along $\overline{AC*}$ and $\overline{AC'*}$, to create $\overline{AC*}$ on Figure \ref{fig:cone} \textit{Left}.
$\angle CAC'$ is defined as the \emph{conical cut angle} for the conical surface.

Figure \ref{fig:cone} \textit{Right} shows the measurement of a right ratio on such a conical surface at $F$ when $A$ lies between the two arms of $F = \overline{FB} \perp \overline{FC}$. 
$\angle AFB$ is defined as the \emph{measuring angle} for a particular measurement.
When the angle measurement distance is larger than $|FB|=|FC|$, $\overline{FC}$ extends over the cut as $\overline{C'H}$, so that the right ratio starts to decrease from $1/2$.

To show the extend of the Euclidean region of the right ratio measurement, Figure \ref{fig:transition} \textit{Left} shows $|FB|/|FA|$ which is defined as the \emph{normalized transitional distance of right ratio} v.s. the conical cut angles and the measuring angles.
To show the rate of deviation from the Euclidean space, Figure \ref{fig:transition} \textit{Right} shows the \emph{normalized transitional derivative of right ratio} which is defined as the first derivative of the right ratio at the normalized transitional distance.
After the normalization, Figure \ref{fig:transition} applies to any right ratio measurement with the tip of the conical surface between the two arms of the right angle, for any point on any conical surface whose conical cut angle is $180^o$ or less.
Figure \ref{fig:transition} shows that when the cone becomes sharper, both the normalized transitional distances and the normalized transitional derivatives have the largest decrease at the $45^o$ measuring angle.
The $45^o$ measuring angle, which is also the bisect direction of the right angle, is thus defined as the \emph{direction of a right ratio measurement}.

Besides conical surfaces, Euclidean surface can be established in confined regions in other surfaces.



\subsection{Smoothness}

In Figure \ref{fig:cone}, the conical surface at its tip $A$ is not smooth, which can be measured using right ratios: for right ratio measurements from $F$, if $A$ is not between the two arms of the right angles at $F$, the right ratios are always $1/2$, so that there is a discontinuity of the right ratio v.s. directions and positions due to the sharp point $A$ of the conical surface.
To avoid such singular point on an otherwise smooth surface:

\begin{defn}[smooth]
If all right ratios on a surface are continuous v.s. angle measuring distance, direction, and position, the surface is \emph{smooth}.
\end{defn}

\begin{defn}[local right ratio]
The right ratio when the angle measure distance approaches zero is defined as the \emph{local right ratio} at that point.  
\end{defn}

When $F$ is extremely close to $A$, the first derivative of the local right ratio approaches Figure \ref{fig:transition} \text{Right}; Otherwise, the first derivative of the local right ratio is always zero.  
Thus, the first derivatives of local right ratios may provide a measurement of the non-smoothness at a point.

\iffalse
\begin{cnj}
\label{cnj:smoothness}
The first derivatives of the local right ratios to the angle measuring distance are always zero on a smooth surface.
\end{cnj}
\fi




\subsection{Flatness}

In a 3-dimensional Euclidean space, among all surfaces that contain any 3 points which are not collinear, there is only one 2-dimensional Euclidean surface.  
Another difference is that all other surfaces need parameters to characterize them, such as the radius of a spherical surface, but Euclidean surface is free from any such parameter.  
Are Euclidean surfaces unique?  

\begin{figure}
\includegraphics[scale=0.5]{flat.png}
\caption{Construct squares from a flat point on a smooth surface.}
\label{fig:square}
\end{figure}

The answer lies in Theorem \ref{thm:constant right ratios}: only 2-dimensional Euclidean surfaces are flat.

\begin{thm}
\label{thm:constant right ratios}
If a 2-sided smooth surface is globally flat at one point, then the surface is Euclidean.
\end{thm}

\begin{proof}
In Figure \ref{fig:square} \textsl{Left}, assume the right ratios are $\alpha^2/4$ for a point $O$. 
According to Theorem \ref{thm:right ratio floor}, $\alpha > 1$.
 
Because the right ratio is a constant in all the directions and for all the measuring distances, the region near the point has to be isotropic and isometric, with perpendicular straight lines dividing the side spaces  symmetrically of each other.

Construct $O = \overline{AC} \perp \overline{BD}$ with $|OA|=|OB|=|OC|=|OD|=1$.
Let $E, F, G$, and $H$ be the midpoints of $\overline{AB}, \overline{BC}, \overline{CD}$, and $\overline{DA}$, respectively.
\begin{enumerate}
\item $|AB|=|BC|=|CD|=|DA|=\alpha$.

\item $|\angle DAB|=|\angle ABC|=|\angle BCD|=|\angle CDA|= 1/ \alpha^2$.

\item $|\angle EAH| = |\angle DAB|: |HE|=|EF|=|FG|=|GH|=1$.

\item $O = \overline{EG} \perp \overline{FH}: |OE|=|OF|=|OG|=|OH|=1/ \alpha$.

\item $|\angle HEF|=|\angle EFG|=|\angle FGH|=|\angle GHE|= 1/ \alpha^2$.

\item $EFGH$ is $ABCD$ scaled down by $1/\alpha$ for every geodesic.

\item From the symmetry, $E = \overline{AB} \perp \overline{OE}, F = \overline{BC} \perp \overline{OF}, G = \overline{CD} \perp \overline{OG}$, and $H = \overline{DA} \perp \overline{OH}$, with the corresponding right ratios denoted as $\beta$.

\item Similarly, $JKLM$ is $EFGH$ scaled down by $1/\alpha$ for every geodesic.

\item Keep on such scaling down, until the $\beta$ right ratios are infinitely close to $O$, which has right ratios of $\alpha^2/4$.

\item Because the the surface is smooth, $\beta = \alpha^2/4$.

\item $|EA| = \alpha / 2 = 1/\alpha$, which means $\alpha^2=2$.

\item $OHAE, OEBF, OFCG$, and $OGDH$ are all squares.

\item Similarly, every smallest quadrilateral in Figure \ref{fig:square} \textsl{Right} is a square.  

\item The space can be tiled with a square grid of any size and orientation, so that the space is Euclidean.
\end{enumerate}
\end{proof}



\subsection{Curvature}

\begin{figure}
\includegraphics[scale=0.6]{torus.png}
\caption{A torus is a doughnut-shaped surface.  Like any surface with rotational symmetry, a torus is characterized by an angular coordinate $\varphi$ in the poloidal direction (the blue arrow), and an angular coordinates $\theta$ in the toroidal direction (the red arrow). }
\label{fig:torus}
\end{figure}

\begin{figure}
\includegraphics[scale=0.6]{right_ratios.png}
\caption{The right ratios at the center of the inside surface of a torus are more than 1/2 and increase with the angle measuring distance, while the right ratios at the center of the outside surface of the torus are less than 1/2 and decrease with the angle measuring distance.  
The former has negative Gaussian curvature, while the latter has positive Gaussian curvature. 
Subsection \emph{Local Model} will give details on the unit of the angle measuring distance.}
\label{fig:right ratios}
\end{figure}

According to Theorem \ref{thm:constant right ratios}, when the first derivative of the local right ratio is $0$, the right ratio approaches $1/2$.
Thus, the infinitesimal region surround each point on a smooth surface is Euclidean.
Such surface is a manifold \cite{Differential Geometry}.

How a locally Euclidean space deviates from a globally Euclidean space is described by \emph{curvature}, which can be viewed easily from the Euclidean space of higher dimension.
For example, in Figure \ref{fig:torus}, the 3-dimensional drawing of a torus shows that it is not flat, with opposite bending at its inside and outside surfaces.
Mathematically, Gaussian curvature \cite{Non-Euclidean Geometry} quantifies the curvature of a 2-dimensional surface in the 3-dimensional Euclidean space, so it is an \emph{explicit} \cite{Non-Euclidean Geometry} measure of the surface.
The outside surface had positive Gaussian curvature, the inside surface had negative Gaussian curvature, while a flat surface has zero Gaussian curvature.

It is also possible to measure the curvature within the 2-dimensional surface, as an \emph{implicit} \cite{Non-Euclidean Geometry} measurement.
Figure \ref{fig:right ratios} shows the right ratios for the inside and outside surface of a torus: The right ratios are found to be more than 1/2 when Gaussian curvature is negative, less than 1/2 when Gaussian curvature is positive, and close to 1/2 when the angle measuring distance is small.
As shown in Figure \ref{fig:right ratios}, The second derivative of the local right ratios is negatively correlated with the Gaussian curvature.  
For another example, on a spherical surface with radius $R$, the local right ratio is $1/2$, the first derivative is $0$, and the second derivative is $-\frac{1}{3 R^2}$, while the corresponding Gaussian curvature is $+ \frac{1}{R^2}$ \cite{Non-Euclidean Geometry}.
The absolute values for both curvature measurements are proportional to the bending: $\frac{1}{R^2}$.
So the second derivative of the local right ratio at a point gives an implicit measure of the space curvature at the point for a particular direction.
In contrast, Using the sum of the triangle inner angles is a traditional implicit measurement of curvature, but it is not a local measurement.

Gaussian curvature can lead to the second derivative of local right ratio, but not reversely in concept.  Thus, using right ratio is a weaker requirement to measure curvature.



  


\iffalse
\subsection{Local Model}

This paper proposes a \emph{local model} to initiate finding the geodesic $\overline{AB}$ on a surface between any two points if the surface can be characterized by two orthogonal parameters in 3-dimensional Euclidean space.  
For example, when a surface has rotational symmetry, it is usually characterized by an angular coordinate $\varphi$ in the poloidal direction, and an angular coordinate $\theta$ in the toroidal direction, as shown in Figure \ref{fig:torus}.  
The torus can be characterized in Cartesian coordinates as:
\begin{align}
x(\theta,\varphi) &= (R + r \cos \theta) \cos \varphi, \\
y(\theta,\varphi) &= (R + r \cos \theta) \sin \varphi, \\
z(\theta,\varphi) &= r \sin \theta
\end{align} 

\begin{figure}
\includegraphics[scale=0.6]{local_model.png}
\caption{The local model for a torus with an aspect ratio of 3. To measure right ratios, $|OA|=|OB|$, and $\overline{AB}$ is represented by the dashed lines in the local model, and by the thin solid lines in the $(\varphi, \theta)$ parameter space.  \textsl{Left:} At the center of the outside surface, $\varphi = 180^o$. \textsl{Right:} At the center of the outside surface, $\varphi = 0^o$. }
\label{fig:local model}
\end{figure}

Built from two orthogonal parameters $(\varphi, \theta)$ in the real space, the local model is in X-Y coordinates specific for points A and B:
\begin{enumerate}
\item 
Its X-axis is a curve $\widehat{OA}$ along $\varphi$, and its Y-axis is a curve $\widehat{OB}$ along $\theta$, so that $O = \widehat{OA} \perp \widehat{OB}$. 
For example, in Figure \ref{fig:torus}, the blue arrow could be X-axis, and the red arrow could be the Y-axis.  

\item 
Every point P on the surface is mapped to a point $(X, Y)$ in the model space: $X = (R + r \cos \theta) \varphi, Y = r \theta$.  
The local model tries to flatten the surface along the axis $O = \widehat{OA} \perp \widehat{OB}$ while keep the two axis $\widehat{OA} \perp \widehat{OB}$ invariant between the real and the model spaces.  

\item 
$\overline{AB}$ can be approximated as a straight line segment connecting $A$ and $B$ in the model X-Y coordinates, which can be mapped back to the real surface as an approximate geodesic in $(\theta, \varphi)$ coordinates. 
\end{enumerate}
For example, in Figure \ref{fig:local model}, the Y axis is in degree of $\theta$, the X-axis is in the equivalent length unit of the Y-axis, and the approximate geodesics corresponding to the dashed line in the model are drawn in solid thin lines, respectively. 
They show opposite bending on the inside vs outside surfaces of a torus. 
Such bending is proportional to the curvature of the surface with increased angle measuring distance. 
The approximate resulting right ratios at the centers of the inside and outside surfaces of the torus are displayed in Figure \ref{fig:right ratios}.
\fi



\section{Discussion}

\subsection{What are 2-Dimensional Euclidean Spaces?}

This paper uses five axioms and a definition to establish 2-dimensional Euclidean geometry progressively:
\begin{itemize}
\item Axiom \ref{axm:stable} requires Euclidean spaces to be stable.

\item Axiom \ref{axm:measurable} requires Euclidean spaces to be measurable by distance.

\item Axiom \ref{axm:continuous} requires Euclidean spaces to be continuous and complete.

\item Axiom \ref{axm:ray} requires Euclidean spaces to be boundless and smooth.

\item Axiom \ref{axm:2D} requires Euclidean spaces to be 2-sided.

\item Definition \ref{def: Euclidean flat} requires the 2-dimensional Euclidean spaces to be flat and similar.
\end{itemize}

Each axiom has its own applicable range, and the axiom set narrows down the applicable ranges progressively, e.g., the 5 axioms applies to non-Euclidean surfaces \cite{Non-Euclidean Geometry} as well.


\subsection{Comparison To the Existing Axiom Sets For Euclidean Geometry}

Compared with the original axioms \cite{Euclid}, the new axiom set is mathematically more complete and strict, because it is built upon the modern mathematical formal system \cite{Formal System}.  It can be viewed as a modern update of the original axioms:
\begin{itemize}
\item It follows the constructive approach as Axiom \ref{axm:stable}, which further leads to the commonly used judgments \cite{Plane} on congruence and similarity.

\item It restates two existing postulations as Axiom \ref{axm:continuous} and Axiom \ref{axm:ray}.

\item It discovers two implied assumptions as Axiom \ref{axm:stable} and Axiom \ref{axm:2D}.

\item It quantifies two existing postulations as Axiom \ref{axm:measurable} and the definition.

\item It shows that the additional "parallel postulate" is indeed redundant.

\item It defines the straight line from the straight line segment, as in \textsl{``if the line is extended to a sufficient length"} \cite{Euclid}, to establish Euclidean space locally rather than globally.

\item It deducts the free mobility \cite{Isometric} of geometric objects on a Euclidean surface, which is implied in the original Euclidean geometry.

\item Both introduce the Euclidean geometry \emph{progressively}, with a later axiom limits its earlier axiom step by step, and with each subset of the axiom set stands by itself.
\end{itemize}

Compared with the enhanced axiom set \cite{Hilbert}, this new set is much simpler, because the new axiom set contains the definition for both straight line segments and straight lines.  
As shown in this paper, each 2-dimensional related axioms in the enhanced axiom set has the corresponding definition or axiom or theorem in this new axiom set.
Both the parallel postulate and the congruence are the consequence of the constant right ratio of Euclidean surfaces, thus they are no longer axioms.

Similar to \cite{Isometric}, the new axiom set requires distance as the only measurement, and derives angle measurement from distance, thus avoiding the conventional angle measurement problem \cite{Plane} \cite{Hilbert}, while also avoiding requiring the angle to be measurable independently \cite{Birkhoff}.

Contrary to \cite{Isometric}, ``free mobility'' of geometric object is no longer a requirement, but a deducted property in this new axiom set.

%When the Gaussian curvature \cite{Non-Euclidean Geometry} is a constant, for any two angles $\angle \alpha$ and $\angle \beta$, $\exists u_0 > 0: |\angle \alpha|_{u_0} = \angle \alpha|_{u_0} \Longrightarrow  |\angle \alpha|_u = \angle \alpha|_u$ for any $ u > 0$, which leads to the SAS (Side-Angle-Side) congruence relation \cite{Hyperbolic} between two triangles.
%The SAS congruent relation leads to all other congruence relations \cite{Hyperbolic}, and the free mobility of geometric objects in the 2-dimensional spaces.
%On the other hand, free mobility leads to the constant Gaussian curvature of the surface \cite{Lie Group}. 
%Thus, the SAS congruence relation, the free mobility, and the constant Gaussian curve are all equivalent on a surface.
%In contrast, without limited to the Euclidean surfaces, the new axiom set can apply to non-Euclidean surfaces of non-constant Gaussian curves, and deduct the free mobility on surfaces of constant Gaussian curves.

The derivatives of the local right ratios provide an implicit and point-specific characterization of locally Euclidean, smoothness, and curvature of any surfaces.

Unlike other axiom sets, the new axiom set contains no new concept in any of its axioms, so each axiom can be understood thoroughly and tested rigorously.


\subsection{The Potential Importance of Right Ratio}

Gaussian curvature needs to be defined in higher dimensions, so it is an explicit measure of space curvature.
The most widely known two implicit measures of surface curvature are the sum of the inner angles of a triangle, and the Pythagorean theorem \cite{Non-Euclidean Geometry}, which are also currently used experimentally to measure space curvature or to detect gravitational waves in astrophysics.
However, these measures fail to distinguish between Euclidean, elliptical, and hyperbolic surfaces on any manifold locally by definition.

In contrast, using local right ratio as another implicit measure:
\begin{itemize}
\item If on a surface area all the right ratios are a constant, then the constant is 1/2, and the surface is Euclidean. 

\item If on a surface area the right ratios are smooth, then the surface area is smooth. 

\item If on a smooth surface area, a local right ratio is $1/2$, then the point is at a manifold.

\item If a point is on a manifold, the second derivative of a local right ratio to its measuring distance measures the curvature of the space at the point:
\begin{itemize}
\item If it is less than $0$, the point is on a local spherical surface.

\item If it is more than $0$, the point is on a local hyperbolic surface.
\end{itemize}
\end{itemize} 
Thus, right ratio provides an implicit measure of surfaces at each point in the simplest way.



\section{Acknowledgments}

This paper started as an answer to the naive but fundamental questions from one of the authors when she first studied Euclidean geometry, indeed showing the importance of the first thoughts in studying mathematics.  
The other author feels deeply grateful for:
\begin{itemize}
\item the wonderful teaching by Mr. Jianye Liu from The Middle School of Peking University,

\item the unique teaching for challenging status quo of knowledge, by Dr Paul Hough from Brookhaven National Lab, including the courage to reinvent wheels, either as a learning process or as an inventory method.
\end{itemize}    
The first attempt resulted in a fatally flawed self-published paper \textsl{How to Define a Flat Plane} (2017), although it is the origin for most of the ideas and approaches presented in this paper. 
Dr. Oliver Attie, an mathematician on differential geometry and bioinformatics, reviewed the paper carefully, and provided many valuable suggestions. 
Mr. Victor Aguilar, the author of \textsl{Geometry-Do} (2019), spent time reading the first paper and found the flaw.   
Mr. Arthur Knish, the president of Institute of Creative Problem Solving on Long Island, encouraged the authors to present in the school on Dec. 2016, and submit the the paper to journals. 
Many of our fiends help proof-reading the draft of the paper, with Ms. Lynn Ye in particular.

\section{Statements and Declarations}

This work is done without official funding and without known conflicts of interests.

This paper and all of its supplementary proofs, texts, figures and codes are available upon request and can be published.
This paper requires no external data.
All supplementary proofs, texts, figures and codes have been included in the submission for this paper.


\iffalse
\section*{Work Sheet}
\subsection{Second Derivative of Right Ratio for Spherical Surface}

The Napier's rules for a spherical right triangle of angles $A, A, C=\pi/2$ and normalized opposite sides $a, a, c$:
\begin{equation}
\begin{split}
& \frac{\sin a}{\sin c} = \sin A \\
& \frac{\tan a}{\tan c} = \cos A \\
& \cos c = \cos^2 a \\
& \cos a = \frac{1}{\tan A}
\end{split}
\end{equation}

When $a, c \rightarrow 0$:
\begin{equation}
\begin{split}
& \frac{c}{a} \rightarrow \frac{\sin c}{\sin a} = \frac{1}{\sin A} \rightarrow \sqrt{2} \\
& \frac{d c}{d a} = 2 \cos a \frac{\sin a}{\sin c} = 2 \cos a \sin A \rightarrow \sqrt{2} \\
& \frac{d \sin A}{d a} \rightarrow 0 \\
& \frac{d^2 c}{d^2 a} \rightarrow -2 \sin a \sin A = 0 \\
& \frac{d^3 c}{d^3 a} \rightarrow -2 \cos a \sin A = -\sqrt{2} \\
& \frac{c}{a} \rightarrow \sqrt{2} - \frac{\sqrt{2}}{6} a^2 \\
& \frac{d c}{d a} \rightarrow \sqrt{2} - \frac{\sqrt{2}}{2} a^2 \\
& \frac{d^2 c}{d^2 a} \rightarrow - \sqrt{2} a \\
& r \equiv \frac{c^2}{4 a^2} \rightarrow \frac{1}{2} \\
& \frac{d r}{d a} = \frac{c}{2 a^2} \frac{d c}{d a} - \frac{c^2}{2 a^3}
    = \frac{c}{2 a^2}(\frac{d c}{d a} - \frac{c}{a}) \rightarrow 0 
\end{split}
\end{equation}

\begin{equation}
\begin{split}
& \frac{d^2 r}{d^2 a} = \frac{1}{2 a^2} \lbrace c \frac{d^2 c}{d^2 a} + (\frac{d c}{d a})^2
    - 4 \frac{c}{a} \frac{d c}{d a} + 3 \frac{c^2}{a^2} \rbrace \\
& \rightarrow \frac{1}{2 a^2} \lbrace - \sqrt{2} a c 
    + (\sqrt{2} - \frac{\sqrt{2}}{2} a^2)^2
    - 4 (\sqrt{2} - \frac{\sqrt{2}}{6} a^2) (\sqrt{2} - \frac{\sqrt{2}}{2} a^2)
    + 3 (\sqrt{2} - \frac{\sqrt{2}}{6} a^2)^2 \rbrace \\
& = \frac{1}{2 a^2} \lbrace  - \sqrt{2} a c - 2 a^2 + \frac{16}{3} a^2 - 2 a^2 \rbrace = - \frac{1}{3}
\end{split}
\end{equation}

\subsection{First Derivative of Right Ratio for Conical Surface}


In Figure \ref{fig:cone} \textit{Right}, let $|AF| = a, |\angle AFB| = \theta, |\angle CAC| = \phi < \pi$.  Let $\overline{FC*}$ be $x$ axis, and $\overline{FB*}$ be $y$ axis:
$A = a (\sin \theta, \cos \theta)$.

Let $c \equiv |FC| = |FB| = a \cos \theta + b$ and $\alpha \equiv \frac{b}{a \sin \theta}$
\begin{enumerate}
\item $\overline{*BAD*}: y = c - \alpha x$.

\item $\overline{*CDC'}: x = c + \alpha y$.

\item $D = \overline{*BAD*} \perp \overline{*CDC'*}: 
D = \frac{c}{1 + \alpha^2} \left( 1 + \alpha, 1 - \alpha \right)$, with $-1 < \alpha < 1$.

\item $|C'D| = |CD|$, so that:
\begin{align*}
C' =& \frac{c}{1 + \alpha^2} \left( 1 + 2 \alpha - \alpha^2, 2(1 - \alpha) \right)
\end{align*}
\end{enumerate}
To verify $C'$.
\begin{multline*}
\frac{|CD|^2}{c^2} = (\frac{1 + \alpha}{1 + \alpha^2} - 1)^2 + (\frac{1 - \alpha}{1 + \alpha^2})^2 
= (\frac{\alpha - \alpha^2}{1 + \alpha^2})^2 + (\frac{1 - \alpha}{1 + \alpha^2})^2
= \frac{(1 - \alpha)^2}{1 + \alpha^2}
\end{multline*}
\begin{multline*}
\frac{|C'C|^2}{c^2} = (\frac{1 + 2 \alpha - \alpha^2}{1 + \alpha^2} - 1)^2 + 
4(\frac{1 - \alpha}{1 + \alpha^2})^2 
= (\frac{2\alpha - 2\alpha^2}{1 + \alpha^2})^2 + 4(\frac{1 - \alpha}{1 + \alpha^2})^2
= 4\frac{(1 - \alpha)^2}{1 + \alpha^2}
\end{multline*}
\begin{multline*}
\frac{|C'B|^2}{c^2} = (\frac{1 + 2 \alpha - \alpha^2}{1 + \alpha^2})^2 + 
(\frac{2 - 2\alpha}{1 + \alpha^2} - 1)^2 
= (\frac{1 + 2 \alpha - \alpha^2}{1 + \alpha^2})^2 + (\frac{1 - 2 \alpha - \alpha^2}{1 + \alpha^2})^2
\\= 2\frac{(1 - \alpha^2)^2 + 4 \alpha^2}{(1 + \alpha^2)^2} = 2
\end{multline*}

\begin{equation*}
|AD|^2 = (c \frac{1 + \alpha}{1 + \alpha^2} - a \sin \theta)^2 +
(c \frac{1 - \alpha}{1 + \alpha^2} - a \cos \theta)^2
\end{equation*}
\begin{align*}
|AD|^2 \frac{(1 + \alpha^2)^2}{a^2} =
& ((\cos \theta + \sin \theta \alpha)(1 + \alpha) - \sin \theta (1 + \alpha^2))^2 + \\ 
& ((\cos \theta + \sin \theta \alpha)(1 - \alpha) - \cos \theta (1 + \alpha^2))^2 \\ =
& ((\cos \theta - \sin \theta) + (\cos \theta + \sin \theta) \alpha)^2 + \\ 
& ((\cos \theta - \sin \theta) + (\cos \theta + \sin \theta) \alpha)^2 \alpha^2 \\
|AD|^2 \frac{1 + \alpha^2}{a^2} =
& (1 - \sin 2\theta) + 2 \cos 2\theta \alpha + (1 + \sin 2\theta)\alpha^2 \\ =
& \frac{(\cos 2\theta + (1 + \sin 2\theta) \alpha)^2}{(1 + \sin 2 \theta)}
\end{align*}

To solve for $\alpha$:
\begin{align*}
& \tan \frac{\phi}{2} = \frac{|CD|}{|AD|}
= \sqrt{1 + \sin 2 \theta}
\frac{(1 - \alpha)(\cos \theta + \sin \theta \alpha)}{\cos 2\theta + (1 + \sin 2\theta) \alpha} \\
&\beta \equiv \tan \frac{\phi}{2} /\sqrt{1 + \sin 2 \theta} \\
&(\beta \cos 2\theta - \cos \theta) + (\beta (1 + \sin 2\theta) + (\cos \theta - \sin \theta)) \alpha + \sin \theta \alpha^2 = 0
\end{align*}

$\overline{C'H*}$ is $\overline{CE*}$ rotated by $\phi$.
Let $|BJ| = |C'H| = \gamma c$. 
\begin{align*}
H = c \left(\frac{1 + 2 \alpha - \alpha^2}{1 + \alpha^2} + \gamma \cos \phi, 
            \frac{2(1 - \alpha)}{1 + \alpha^2} + \gamma \sin \phi \right)
\end{align*}

The angle distance ratio $\tau$ is:
\begin{multline*}
4 (1 + \gamma)^2 \tau = \frac{|HJ|^2}{c^2} \\
= (\frac{1 + 2 \alpha - \alpha^2}{1 + \alpha^2} + \gamma \cos \phi)^2 
+ (\frac{2(1 - \alpha)}{1 + \alpha^2} + \gamma \sin \phi - (1 + \gamma))^2 \\
= (\frac{1 + 2 \alpha - \alpha^2}{1 + \alpha^2} + \gamma \cos \phi)^2 
+ (\frac{1 - 2 \alpha - \alpha^2}{1 + \alpha^2} + \gamma \sin \phi - \gamma)^2 \\
= 2 + 2\gamma^2 + 2\gamma \frac{
(1 + 2 \alpha - \alpha^2)\cos \phi + (1 - 2 \alpha - \alpha^2)(\sin \phi - 1)
}{1 + \alpha^2}
\end{multline*}
\begin{align*}
\eta \equiv \frac{
(1 + 2 \alpha - \alpha^2)\cos \phi + (1 - 2 \alpha - \alpha^2)(\sin \phi - 1)
}{1 + \alpha^2}: & \;
\tau = \frac{1}{2} \frac{1 + \eta \gamma + \gamma^2}{1 + 2 \gamma + \gamma^2} \\
\gamma \rightarrow 0:& \; \frac{\partial \tau}{\partial (\gamma + 1)} = \frac{\eta}{2} - 1
\end{align*}

In special cases
\begin{itemize}
\item $\phi = 0: \alpha = 1; \eta = 2; \frac{\partial \tau}{\partial \gamma} = 0$. 

\item $\phi = \pi/2: \eta = 0, \frac{\partial \tau}{\partial \gamma} = -1$. 

\item $\phi = \pi$:
\begin{align*}
\alpha =& - \frac{\cos 2\theta}{1 + \sin 2\theta} \\
\eta =& - \frac{(1 + 2 \alpha - \alpha^2) + (1 - 2 \alpha - \alpha^2)}{1 + \alpha^2}
= 2 \frac{\alpha^2 - 1}{\alpha^2 + 1} \\ =
& 2 \frac{\cos^2 2\theta - (1 + \sin 2\theta)^2}{\cos^2 2\theta + (1 + \sin 2\theta)^2}
= \frac{\cos 4\theta - \sin 2\theta}{\sin 2\theta + 1} - 1
\end{align*}

\item $\phi = 2\pi: \alpha = 1; \eta = 2; \frac{\partial \tau}{\partial \gamma} = 0$:

\item $\theta = 0$: The solution is not possible because it has limitation on $\phi$.
\begin{align*}
& \tan \frac{\phi}{2} = \frac{1 - \alpha}{1 + \alpha}; \;
  \sin \frac{\phi}{2} = \sqrt{\frac{1}{2} \frac{1 - \alpha}{1 + \alpha}}; \;
  \cos \frac{\phi}{2} = \sqrt{\frac{1}{2} \frac{1 + \alpha}{1 - \alpha}}; \; \\
& \sin \phi = 1; \;
  \cos \phi = 0 =  \frac{1}{2} \frac{1 + \alpha}{1 - \alpha} - \frac{1}{2} \frac{1 - \alpha}{1 + \alpha}
  = \frac{2\alpha}{1 - \alpha^2}; 
\end{align*}

\item $\theta = \pi/4$:
\begin{align*}
& \tan \frac{\phi}{2} = \frac{1 - \alpha^2}{2 \alpha}; \; 
  \sin \frac{\phi}{2} = \frac{1 - \alpha^2}{1 + \alpha^2}; \;
  \cos \frac{\phi}{2} = \frac{2 \alpha}{1 + \alpha^2}; \\
& \sin \phi = \frac{4 \alpha (1 - \alpha^2)}{(1 + \alpha^2)^2}; \;
  \cos \phi = \frac{4 \alpha^2 - (1 - \alpha^2)^2}{(1 + \alpha^2)^2}; \\
& (1 + \alpha^2)^3 \eta =
  (1 + 2 \alpha - \alpha^2)(-1 + 6\alpha^2 - \alpha^4) \\
&+ (1 - 2 \alpha - \alpha^2)(4\alpha - 4\alpha^3)
- (1 - 2 \alpha - \alpha^2)(1 + 2\alpha^2 + \alpha^4) \\
\eta =& \frac{-2 + 4 \alpha - 3 \alpha^2 - 7 \alpha^3 + 4 \alpha^4 + 8 \alpha^5 + \alpha^6}
             {(1 + \alpha^2)^3} 
\end{align*}
\end{itemize}
 





\fi


\begin{thebibliography}{1}

\bibitem{Euclid}
Euclid of Alexandria, 
\textit{Euclid's Elements of Geometry}, 
edited, and provided with a modern English translation, by Richard Fitzpatrick, ISBN 978-0-6151-7984-1.

\bibitem{Plane}
Charles Aboughantous, 
\textit{Euclidean Plane Geometry}, 
Universal Publishers, 2010, ISBN 978-1-5994-2822-2.

\bibitem{Mistakes}
A. I. Fetisov, and Y.S. Dubnov, 
\textit{Proof in Geometry, with Mistake in Geometric Proofs}, 
Dover Publications, 2018, ISBN 978-0-4864-5354-5.

\bibitem{Hilbert}
David Hilbert, 
\textit{Foundations of Geometry}, 
The Open Court Publishing Co, 1902.

\bibitem{Formal System}
Daniel J Velleman, 
\textit{How to prove it: a structured approach}, 
Cambridge University, 1994.  ISBN 978-0-5216-7599-4.

\bibitem{Differential Geometry}
Chuan-Chih Hsiung, 
\textit{A Firt Course in Differential Geometry}, 
John Wiley and Sons, 1981. ISBN 0-471-07953-7.

\bibitem{Isometric}
Garrett Birkhoff, 
\textit{Metric foundations of geometry. I.}, 
Transaction of American Mathematics Society, Volume 55 (1944), Page 465-492.

\bibitem{Non-Euclidean Geometry}
H.S.M. Coxeter, 
\textit{Non-Euclidean Geometry}, 
The Mathematical Association of America, 1998.  ISBN 0-88385-522-4.

\bibitem{Birkhoff}
George D. Birkhoff, 
\textit{A Set of Postulates for Plane Geometry (Based on Scale and Protractors)}, 
Annals of Mathematics, 1932, Volume 33, Page 329–345.

%\bibitem{Hyperbolic}
%Arlan Ramsay, and Robert D. Richtmyer, 
%\textit{Introduction to hyperbolic geometry}, 
%Springer, 1995, ISBN 978-0-387-94339-8.

%\bibitem{Lie Group}
%Hans Fredenthal,  
%\textit{Lie Groups in the Foundations of Geometry }, 
%Advances in Mathematics, Volume 1, Issue 2, 1964, Page 145-190.


\end{thebibliography}

% ------------------------------------------------------------------------
\end{document}
% ------------------------------------------------------------------------
