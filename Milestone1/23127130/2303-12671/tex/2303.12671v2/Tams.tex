%\input TinLatex.tex
\NeedsTeXFormat{LaTeX2e}

\documentclass[11pt]{TinLatex} 
\textwidth15.4cm

\textheight21.8cm

\oddsidemargin0.0cm 

\evensidemargin0.0cm

\setlength{\topmargin}{0cm}
\setlength{\headheight}{0.4cm}
\setlength{\headsep}{.5cm}



%\topmargin

\usepackage{graphicx}
\usepackage{epstopdf}

\usepackage{amsmath,amsxtra,amssymb,latexsym, amscd}

\usepackage[mathscr]{eucal}

\usepackage{hyperref}
\hypersetup{
 colorlinks=true,
 linkcolor=blue,
}


\usepackage{xspace}
\usepackage{algorithmic}
\usepackage{graphicx}
\usepackage{textcomp}
%\usepackage[T5]{fontenc}
%\usepackage[utf8]{inputenc}
\usepackage{tabularx,ragged2e}
%\usepackage{biblatex}
%\usepackage[sorting=none]{biblatex}
%\usepackage{hyperref}
\usepackage{indentfirst}
\usepackage{subcaption}
\usepackage{colortbl}
\usepackage{diagbox}
\usepackage[table,xcdraw]{xcolor}
\usepackage{amsfonts}
\usepackage{booktabs}
\usepackage{siunitx}
\usepackage{array}
\usepackage{amsmath}
\usepackage{float}
%\usepackage[utf8]{inputenc}
\usepackage{booktabs, caption, makecell}
\usepackage[symbol]{footmisc}
\renewcommand\theadfont{\bfseries}
\usepackage{threeparttable}
\usepackage[linesnumbered,ruled]{algorithm2e}
%\addbibresource{refs.bib}

% \input setbmp
% \input seteps
% \input setwmf
% \input setps
%\input dih.tex

\font\elevent=cmr12 at 11pt
%\parskip=2pt
\font\it=cmti12 at 11pt
\font\bf=cmbx12 at 11pt
\font\cv=cmr10 scaled \magstephalf
\advance\hoffset0.4cm
%\font\bfv=cmbx10 %scaled \magstephalf
\font\itv=cmti10 %scaled \magstephalf
\font\bfh=cmssbx10 scaled 1400
\font\bfl=cmssbx10 scaled \magstep1
\font\beo=cmr10 scaled\magstep1
\font\chu=cmr8
\font\dsl=cmitt10
\font\cvv=cmr10 scaled \magstephalf
\font\be=cmr6
%%%%%%%%%%
\font\itn=cmti10
\font\bfn=cmbx10
\font\bfitt = cmbxti10 
\font\cn=cmr10
%%%%%%%%%%
\font\itsl=cmsl9
\font\nn=cmsl8
\font\go=eufm10
\font\gotic=eufm8
\font\goticc=eufm6
%%%%%%%%%%%%
\def\vs{\vskip-0.4cm}
\def\vhs{\vskip-0.18cm}
\def\vss{\vskip-0.7cm}
\def\vt{\vskip0.5cm}
\def\vl{\vskip0.3cm}
\def\vlv{\vskip0.27cm}
\def\vv{\vskip0.25cm}%0.25
\def\vvb{\vskip0.21cm}
\def\vb{\vskip0.17cm}%.17
\def\vn{\vskip0.11cm}
\def\vnn{\vskip0.05cm}
\def\n{\noindent}
\def\ce{\centerline}
\def\dis{\displaystyle}
\def\ed{\end{document}}
\def\vuong{\raise-0.16cm\hbox{$^\blacksquare$}}

% Add a period to the end of an abbreviation unless there's one
% already, then \xspace.
\makeatletter
\DeclareRobustCommand\onedot{\futurelet\@let@token\@onedot}
\def\@onedot{\ifx\@let@token.\else.\null\fi\xspace}

\def\eg{e.g\onedot} \def\Eg{E.g\onedot}
\def\ie{i.e\onedot} \def\Ie{I.e\onedot}
\def\cf{c.f\onedot} \def\Cf{C.f\onedot}
\def\etc{etc\onedot} \def\vs{vs\onedot}
\def\wrt{w.r.t\onedot} \def\dof{d.o.f\onedot}
\def\etal{et al\onedot}
\makeatother

\newcommand\blankfootnote[1]{%
	\begingroup
		\renewcommand\thefootnote{}\footnote{#1}%
		\addtocounter{footnote}{-1}%
	\endgroup
}


\usepackage{cite}
\usepackage{multirow}
%\usepackage[flushmargin]{footmisc}
\usepackage[numbers]{natbib}
\usepackage{tablefootnote}

\begin{document}
\makeatletter	   % `@' is now a normal "letter' for LaTeX
\setcounter{page}{1}
% \renewcommand{\ps@plain}{%
% 	%\renewcommand{\@oddhead}{\small\hfill\itsl{T\ang p ch\is\ Tin h\ong c v\ah\ \DD i\eeh u khi\eehoi n h\ong c, T.25, S.1 (2009), 1--}}
% 	\renewcommand{\@oddhead}{\hfill\begin{tabular}{r}
% 			\small{\emph{Journal of Computer Science and Cybernetics, V.xx, N.xx (20xx), 1-xx}} \\
% 			\footnotesize{DOI no 10.15625/1813-9663/xx/x/xxxx}
% 		\end{tabular}
% 	}
	
% 	\renewcommand{\@evenhead}{\@oddhead}
% 	\renewcommand{\@oddfoot}{
% 		\small\hfill{\hskip7.1cm \copyright\ 2022 Vietnam  Academy of Science \& Technology}
% 		\renewcommand{\@evenfoot}{\@oddfoot}
% 	}
% }

%Lenh duoi thay doi do dai cua duong ngang
%\usepackage[flushmargin]{footmisc} - PHẢI MỞ LỆNH NÀY
\def\footnoterule{\kern-3\p@
	\hrule \@width 1.32in \kern 2.6\p@} % the \hrule is .4pt high
%Thay doi indent cua footnote
%\renewcommand{\footnotesep}{2cm}
\renewcommand{\footnotemargin}{0em}
\newcommand{\fn}[1]{\footnotetext{\hspace{-6mm}#1}}
\setlength{\skip\footins}{8mm}

\makeatother   % `@' is restored as a "non-letter" character 


%%%%%%%%%%%%%%%%%%%%%%%%
%Title & Authors
%%%%%%%%%%%%%%%%%%%%%%%%

\title{Integrating Image Features with Convolutional Sequence-to-sequence Network for Multilingual Visual Question Answering}
\author{
	{\cn TRIET M. THAI, SON T. LUU$^*$}
	\vskip.5cm
	{
	\it University of Information Technology, Ho Chi Minh City, Viet Nam\\
	\it Vietnam National University, Ho Chi Minh City, Vietnam
\fn{*Corresponding author.}
\fn{\hspace{1.7mm}{\it E-mail addresses}: \href{mailto:19522397@gm.uit.edu.vn}{19522397@gm.uit.edu.vn} (T.M.Thai); 
	\href{mailto:SONLT@uit.edu.vn}{sonlt@uit.edu.vn} (S.T. Luu).
	}
}}

\maketitle
\renewcommand\refname{\normalsize \centerline{ REFERENCES}}
% Set to use the "plain" pagestyle
\pagestyle{plain}
\pagestyle{myheadings}
\markboth{\footnotesize \chu \uppercase{ TRIET M. THAI}, SON T. LUU}
{\footnotesize  \chu \uppercase{Integrating Image Features with Convolutional Sequence-to-sequence Network}}
% {	\vspace{-.4cm}
% 	%\hspace{13.25cm}{\includegraphics[scale=.2]{crosscheck.pdf}}%\vspace{-.5cm}
% 	\hspace{13.04cm}{\includegraphics[scale=.05]{cross_check.PNG}
% 	}
% }

%=========================================================================
\begin{abstract} Visual Question Answering (VQA) is a task that requires computers to give correct answers for the input questions based on the images. This task can be solved by humans with ease but is a challenge for computers. The VLSP2022-EVJVQA shared task carries the Visual Question Answering task in the multilingual domain on a newly released dataset: UIT-EVJVQA, in which the questions and answers are written in three different languages: English, Vietnamese and Japanese. We approached the challenge as a sequence-to-sequence learning task, in which we integrated hints from pre-trained state-of-the-art VQA models and image features with Convolutional Sequence-to-Sequence network to generate the desired answers. Our results obtained up to 0.3442 by F1 score on the public test set, 0.4210 on the private test set, and placed $3^{rd}$ in the competition.
\vb

\keywords{Visual Question Answering; Sequence-to-sequence Learning; Multilingual; Multimodal}
\end{abstract}

\section{Introduction}


Recent years have witnessed the rise of human digitization~\cite{habermannDeepCapMonocularHuman2020,alexanderCREATINGPHOTOREALDIGITAL,pengNeuralBodyImplicit2021,alldieckDetailedHumanAvatars2018, rajANRArticulatedNeural2020}. This technology greatly impacts the entertainment, education, design, and engineering industry.
There is a well-developed industry solution for this task.
High-fidelity reconstruction of humans can be achieved either with full-body laser scans~\cite{saitoSCANimateWeaklySupervised2021}, dense synchronized multi-view cameras~\cite{xiangModelingClothingSeparate2021a,xiangDressingAvatarsDeep2022a}, or light stages~\cite{alexanderCREATINGPHOTOREALDIGITAL}.
However, these settings are expensive and tedious to deploy and consist of a complex processing pipeline, preventing the technology's democratization.

Another solution is to view the problem as inverse rendering and learn digital humans directly from custom-collected data.
Traditional approaches directly optimize explicit mesh representation~\cite{loperSMPLSkinnedMultiperson2015, fangRMPERegionalMultiperson2018, pavlakosExpressiveBodyCapture2019} which suffers from the problems of smooth geometry and coarse textures~\cite{prokudinSMPLpixNeuralAvatars2020,alldieckVideoBasedReconstruction2018}. Besides, they require professional artists to design human templates, rigging, and unwrapped UV coordinates.
Recently, with the help of volumetric-based implicit representations~\cite{mildenhallNeRFRepresentingScenes2020, parkDeepSDFLearningContinuous2019, meschederOccupancyNetworksLearning2019} and neural rendering~\cite{laineModularPrimitivesHighPerformance2020, liuSoftRasterizerDifferentiable2019, thiesDeferredNeuralRendering2019}, 
one can easily digitize a quality-plausible human avatar from video footage~\cite{jiangNeuManNeuralHuman2022,wengHumanNeRFFreeviewpointRendering}.
Particularly, volumetric-based implicit representations~\cite{mildenhallNeRFRepresentingScenes2020, pengNeuralBodyImplicit2021} can reconstruct scenes or objects with much higher fidelity against previous neural renderer~\cite{thiesDeferredNeuralRendering2019,prokudinSMPLpixNeuralAvatars2020}, and is more user-friendly as it does not need any human templates, pre-set rigging, or UV coordinates.
Captured visual footage and corresponding skeleton tracking are enough for training.
However, better reconstructions and more friendly usability are at the expense of the following factors.
1) \textbf{Inefficiency:}
They require longer optimization times (typically tens of hours or days) and inference slowly.
Volume rendering~\cite{mildenhallNeRFRepresentingScenes2020,lombardiNeuralVolumesLearning2019} formulates images by querying the densities and colors of millions of spatial coordinates. 
In the training stage, due to memory constraints, only a small fraction of points are sampled which leads to slow convergence speed.
2) \textbf{Entangled representations}:
The geometry, materials, and motion dynamics are entangled in the neural networks. 
Due to the implicit nature of neural nets, one can hardly edit one property without touching the others~\cite{yuanNeRFEditingGeometryEditing2022a,liuEditingConditionalRadiance2021}.
3) \textbf{Graphics incompatibility}:
Volume rendering is incompatible with the current popular graphic pipeline,
which renders triangular/quadrilateral meshes efficiently with the rasterization technique.
Many downstream applications require mesh rasterization in their workflow (\eg, editing~\cite{foundationBlenderOrgHome}, simulation~\cite{benderPositionBasedSimulationMethods2015}, real-time rendering~\cite{akenine2019real}, ray-tracing~\cite{waldRTXRayTracing}).
Although there are approaches~\cite{lorensenMarchingCubesHigh,labelleIsosurfaceStuffingFast2007} can convert volumetric fields into meshes, the gaps from discrete sampling degrade the output quality in terms of both meshes and textures.


To address these issues, we present \textbf{EMA}, a method based on \textbf{E}fficient \textbf{M}eshy neural fields to reconstruct animatable human \textbf{A}vatars.
Our method enjoys flexibility from implicit representations and efficiency from explicit meshes, yet still maintains high-fidelity reconstruction quality.
Given video sequences and the corresponding pose tracking, our method digitizes humans in terms of canonical triangular meshes, physically-based rendering (PBR) materials, and skinning weights \textit{w.r.t.} skeletons.
We jointly learn the above components via inverse rendering~\cite{laineModularPrimitivesHighPerformance2020,chenDIBRLearningPredict2021,chenLearningPredict3D2019} in an end-to-end manner.
Each of them is derived from a separate neural field, which relaxes the requirements of a preset human template, rigging, or UV coordinates.
Specifically, we predict a canonical mesh out of a signed distance field (SDF) by differentiable marching tetrahedra~\cite{shenDeepMarchingTetrahedra2021,gaoGET3DGenerativeModel,gaoLearningDeformableTetrahedral2020,munkbergExtractingTriangular3D2022}, then we extend the marching tetrahedra~\cite{shenDeepMarchingTetrahedra2021} for spatial-varying materials by utilizing a neural field to predict PBR materials \textit{on the mesh surfaces} after rasterization~\cite{munkbergExtractingTriangular3D2022,hasselgrenShapeLightMaterial2022,laineModularPrimitivesHighPerformance2020}.
To make the canonical mesh animatable, we take another neural field to model the forward linear blend skinning for the meshes. 
Given a posed skeleton, the canonical mesh is then transformed into the corresponding poses.
Finally, we shade the mesh with a rasterization-based differentiable renderer~\cite{laineModularPrimitivesHighPerformance2020} and train our models with a photo-metric loss.
After training, we export the mesh with materials and discard the neural fields.

\looseness=-1
There are several merits of our method design.
1) \textbf{Efficiency}:
Powered by efficient mesh rendering, our method can render in real-time.
Besides, the training speed is boosted as well, 
since we compute loss holistically on the whole image and the gradients only flow on the mesh surface. In contrast, volume rendering takes limited pixels for loss computation and back-propagates the gradients in the whole space.
Our method only needs about an hour of training and minutes of optimization are enough for plausible avatar reconstruction.
2) \textbf{Disentangled representations}:
Our shape, materials, and motion modules are disentangled naturally by design, which facilitates editing. 
Besides, Canonical meshes with forward skinning modeling handle the out-of-distribution poses better.
3) \textbf{Graphics compatibility}:
Our derived mesh representation is compatible with 
the prominent graphic pipeline, which leads to instant downstream applications (\eg, the shape and materials can be edited directly in design software~\cite{foundationBlenderOrgHome}).
To further improve reconstruction quality, we additionally optimize image-based environment lights and non-rigid motions.


We conduct extensive experiments on standards benchmarks H36M~\cite{ionescuHuman36MLarge2014b} and ZJU-MoCap~\cite{pengNeuralBodyImplicit2021}.
Our method achieves very competitive performance for novel view synthesis, generalizes better for novel poses, 
and significantly improves both training time and inference speed against previous arts.
Our research-oriented code reaches real-time inference speed (100+ FPS for rendering $512\times512$ images).
We in addition showcase applications including novel pose synthesis, material editing, and relighting.
\section{Related Work}
\mypara{Roadside Perception.}
Concurrent perception efforts for autonomous driving are mainly limited to the ego vehicle~\cite{caesar2020nuscenes, sun2020waymo}. While the roadside perception, which comparatively has a longer perceptual range and more robustness to occlusion and long-time event prediction, is mainly under-explored. Recently, some pioneers have present roadside datasets~\cite{ye2022rope3d, yu2022dair}, hoping to facilitate the 3D perception tasks in roadside scenarios. Compared with the vehicle perceptual system, which only observes surroundings in a short distance, the roadside cameras, mounted on poles a few meters above the ground, can provide long-range perception. However, the cameras mounted on roadside units have ambiguous mounting positions and 
%inconsistent intrinsic 
variable extrinsic parameters, which bring critical challenges to current perception models. In this paper, we take the advances and challenges of roadside cameras into account, and design an efficient and robust roadside perception framework, \name{}.


% \mypara{Vision-based Multi-View BEV perception.}
%  single-camera setting and multi-camera setting
\mypara{Vision Centric BEV Perception.}
Recent vision-centric works predict objects in 3D space, which is very suitable for applying multi-view feature aggregation under BEV for autonomous driving. Popular methods can be divided into transformer-based and depth-based schema. 
Following DETR3D~\cite{wang2022detr3d}, transformer-based detectors design a set of object queries~\cite{liu2022petr, liu2022petrv2, jiang2022polarformer, Chen2022PolarPF, Saha2022TranslatingII} or BEV grid queries\cite{li2022bevformer}, then perform the view transformation through cross-attention between queries and image features. 
Following LSS~\cite{philion2020lift}, depth-based methods ~\cite{reading2021cadnn, huang2021bevdet, huang2022bevdet4d} explicitly predict the depth distribution and use it to construct the 3D volumetric feature. Followup works introduce depth supervision from the LiDAR sensors ~\cite{li2022bevdepth} or multi-view stereo techniques ~\cite{wang2022sts, li2022bevstereo, park2022solofusion} to improve the depth estimation accuracy and achieve state-of-the-art performance. 
Additionally, transformer-based detectors' implicit 3D location information also can benefit from accurate depth cues. Inspired by~\cite{zhang2022monodetr,huang2022monodtr}, CrossDTR~\cite{tseng2022crossdtr} proposed depth-guided transformers, which compose depth-aware embedding from depth maps and are supervised by ground truth depth maps to enhance performance.
However, when applying these methods to roadside perception, the bonus of accurate depth information fades. As the complex mounting positions and variable extrinsic parameters of the roadside cameras, predicting depth from them is difficult. In this work, our \name{} utilizes the height estimation to achieve state-of-the-art performance and best robustness of roadside 3D object detection.

\section{THE DATASET}
\label{dataset}
% UIT-EVJVQA dataset with over 33K question-answer pairs on approximately 5,000 images taken in Vietnam is provided to participating teams. The dataset is stored as.json files. Several examples of the multilingual Visual Question Answering task in VLSP 2022 are shown below.

% \begin{table}[ht!]
%      \begin{center}
%      \begin{tabular}{ p{0.15\textwidth} p{0.28\textwidth} }
%      \toprule
%      \raisebox{-\totalheight}{\includegraphics[width=0.2\textwidth]{figure/00000001451.jpg}}
%       & \begin{itemize}[topsep=0pt]
%     {\scriptsize 
%     \item \textbf{Image Id:} 1451
    
%     \item \textbf{Question:} how many people are using their phones to take pictures on the boat?
%      \vspace{-1em}
%     \item \textbf{Answer:} just one
%     }
%     \end{itemize} 
%       \\\midrule
%     \raisebox{-\totalheight}{\includegraphics[width=0.2\textwidth]{figure/00000001262.png}}
%     &
%     \begin{itemize}[topsep=0pt]
%     {\scriptsize \item \textbf{Image Id:} 1262}
%     {\scriptsize \item \textbf{Question:} người đàn ông mặc áo xanh lá đang làm gì? \textit{English: what is the man in green doing ?}}
%     {\scriptsize \item \textbf{Answer:} đang quét dọn \textit{English: he is cleaning up}}
%     % \vspace{-3em}
%     \end{itemize}
    
%     \\\midrule
%      \raisebox{-\totalheight}{\includegraphics[width=0.2\textwidth]{figure/00000004990.jpg}}
%       &     \begin{itemize}[topsep=0pt]
%     {\scriptsize
%     \item \textbf{Image Id:} 4990
    
%     \item \textbf{Question:}\begin{CJK*}{UTF8}{min}
%     {\CJKfamily{goth}女の子は水に何の手を入れていますか?}\end{CJK*}
     
%     \textit{English: which hand is the girl putting onto the water?}}
    

%      {\scriptsize\item \textbf{Answer:}\begin{CJK*}{UTF8}{min}
%     {\CJKfamily{goth}少女は左手を水の中に入れます}
%     \end{CJK*}}
    
%      {\scriptsize\textit{English: the girl is putting left 
%      hand onto the water}}
    
%     \end{itemize} 
%     \\\bottomrule
%       \end{tabular}
%       \captionof{figure}{Examples of multilingual question-answer pairs from UIT-EVJVQA dataset}
%       \label{example}
%       \end{center}
% \end{table}


% \begin{table}[ht]
% \centering
% \renewcommand{\arraystretch}{0}
% \resizebox{\columnwidth}{!}{%
% \begin{tabular}{|p{0.48\columnwidth}p{0.48\columnwidth}|}
% \hline
%     \begin{itemize}[leftmargin=*]
%     \setlength\itemsep{-0.2em}
%         \item {\scriptsize \textbf{Question:} how many people are using their phones to take pictures on the boat?}
%         \item {\scriptsize \textbf{Answer:} just one}
%     \end{itemize}&
    
%     \begin{center}
%     \includegraphics[width=0.42\columnwidth]{figure/00000001451.jpg}
%     \end{center}\\\hline

%     \begin{itemize}[leftmargin=*]
%     \setlength\itemsep{-0.2em}
%         \item {\scriptsize \textbf{Question:} người đàn ông mặc áo xanh lá đang làm gì? \textit{(English: what is the man in green doing ?)}}
%         \item {\scriptsize \textbf{Answer:} đang quét dọn \textit{(English: he is cleaning up)}}
%     \end{itemize}&
%     \begin{center}
%     \includegraphics[width=0.42\columnwidth]{figure/00000001262.png}
%     \end{center}\\\hline
%     %\phantom{a}&\phantom{a}\\\hline
    
%     \begin{itemize}[leftmargin=*]
%     \setlength\itemsep{-0.2em}
%         \item {\scriptsize \textbf{Question:}\begin{CJK*}{UTF8}{min}
%         {\CJKfamily{goth}女の子は水に何の手を入れていますか?}\end{CJK*} \textit{(English: which hand is the girl putting onto the water?)}}
%         \item {\scriptsize \textbf{Answer:}\begin{CJK*}{UTF8}{min}
%         {\CJKfamily{goth}少女は左手を水の中に入れます}
%         \end{CJK*} \textit{(English: the girl is putting left hand onto the water)}}
%     \end{itemize}&
%     \begin{center}\includegraphics[width=0.42\columnwidth]{figure/00000004990.jpg}
%     \end{center}\\\hline
% \end{tabular}}%
% \captionof{figure}{Several multilingual samples from UIT-EVJVQA dataset }
% %\caption{A table arranging  images}
% \label{example}
% \vspace{-3mm}
% \end{table}


% \begin{figure}[ht!]
% % \resizebox{\textwidth}{!}{
% \centering
% \subfloat[]{%
%   \includegraphics[clip,width=0.8\columnwidth]{figure/ex1.pdf}%
%   \label{ex31}
% }

% \subfloat[]{%
%   \includegraphics[clip,width=0.8\columnwidth]{figure/ex2.pdf}%
%   \label{ex32}
% }

% \subfloat[]{%
%   \includegraphics[clip,width=0.8\columnwidth]{figure/ex3.pdf}%
%   \label{ex33}
% }
% % }

% \caption{Several examples from UIT-EVJVQA dataset}
% \label{example}

% \end{figure}


% \begin{figure}[!htb]
% \begin{tabular}{cc}
%     \begin{minipage}{0.48\textwidth} \includegraphics[width=0.9\textwidth]{figure/ex11.pdf}
%     \vspace{-0.5em}
%     \subcaption{}
%     \vspace{0.5em}
%     \includegraphics[width=0.9\textwidth]{figure/ex22.pdf}
%     \vspace{-0.5em}
%     \subcaption{}
%     \vspace{0.5em}
%     \end{minipage}& 
%     \begin{minipage}{0.5\textwidth} \includegraphics[width=0.9\textwidth]{figure/ex33.pdf}
%     \vspace{-0.5em}
%     \subcaption{}
%     \vspace{0.5em}
%     \end{minipage}
%     \end{tabular}
%         \caption{Hello\label{fig:irrfdad}}
%     \end{figure}

The dataset released for the VLSP-EVJVQA challenge, UIT-EVJVQA \cite{vlsp2022}, is the first multilingual Visual Question Answering dataset with three languages: English (en), Vietnamese (vi), and Japanese (ja). It comprises over 33,000 question-answer pairs manually annotated on approximately 5,000 images taken in Vietnam, with the answer created from the input question and the corresponding image. Besides various types of questions, the answers are constructed in a free-form structure, making it a challenge for VQA systems. To perform effectively and achieve good results on UIT-EVJVQA, the typical VQA systems must identify and predict correct answers in free-form format for multilingual questions, due to the dataset characteristics.

\begin{table}[ht!]
    \centering
    \renewcommand{\arraystretch}{1}
    \resizebox{\textwidth}{!}{%
    \begin{tabular}{lrrrcrrr}
        % \hline
        \toprule
        & \multicolumn{3}{c}{Training set} && \multicolumn{3}{c}{Public test set} \\
        \cmidrule{2-4} \cmidrule{6-8}    
        & \textbf{English} & \textbf{Vietnamese} & \textbf{Japanese} &&  \textbf{English} & \textbf{Vietnamese} & \textbf{Japanese} \\
        \midrule
        Number of samples &7,193 &8,320 &8,261 &&1,686 &1,678 &1,651 \\
        % \hline
        \textbf{Questions} & \multicolumn{3}{c}{} & \multicolumn{3}{c}{}\\
        %\hline
        Vocabulary size &2,089 &1,860 &3,035 &&1,080 &919 &1,226 \\
        Average Length &8.52 &8.70 &13.03 &&8.76 &8.87 &13.27 \\
        %\hline
        Max Length &24 &21 &45 &&26 &22 &33 \\
        %\hline
        Min Length &3 &3 &4 &&3 &4 &4 \\
        % \hline
        \textbf{Answers} & \multicolumn{3}{c}{} & \multicolumn{3}{c}{}\\
        %\hline
        Vocabulary size &2,307 &2,067 &3,534 &&1,029 &877 &1,176 \\
        Average Length &5.09 &6.04 &7.27 &&3.89 &4.54 &5.05 \\
        %\hline
        Max Length &23 &23 &30 &&19 &18 &21 \\
        %\hline
        Min Length &1 &1 &1 &&1 &1 &1 \\
        % \hline
        \cmidrule{1-8} 
    \end{tabular}}
    \caption{Statistical information about the UIT-EVJVQA dataset}
    \label{statistic_dataset}
\end{table}

The training set and public test set have total samples of 23,774 and 5,015, respectively. Table \ref{statistic_dataset} describes the statistical information about the UIT-EVJVQA dataset on the training and public test sets. The length of sentences is computed at the word level. We use Underthesea\footnote{\url{https://github.com/undertheseanlp/underthesea}}
% \footnote{\url{https://github.com/undertheseanlp/underthesea}}
and Trankit \cite{nguyen2021trankit} library for word segmentation. 
Generally, the distribution of the training and public test sets is quite similar. English has fewer training samples compared with Vietnamese and Japanese, which may affect the question answering performance on this language. The length of questions is longer than the length of answers in three languages. The questions in Japanese are significantly longer than those in the two remaining languages. For the answers, those in Japanese are still also longer than those in English and Vietnamese. However, the difference in the length is not as much as the questions. Particularly, the shortest answers in three languages have only one word. In contrast, the questions and answers in Vietnamese have fewer words than those in English and Japanese.
\section{Method}



% \subsection{Overview}

% Information visualization uses visual encodings to characterize datasets so that humans can see the visual encodings, comprehend the dataset behind the encodings, and gain further insight from the data~\cite{munzner2008process}. 


% \alexout{Since all mark types occupy a particular region in the 2D visualization in visual perception, we apply a novel masking scheme to these marks to ensure corresponding visual channels remain the same at a close viewing distance. As a result, the visual encodings are preserved because the marks and visual channels are unchanged after the maks processing. Therefore, the processed visualization, namely privacy-preserving visualization, has the same visual encoding as the original visualization. Users in proximity to the visualization are able to comprehend the dataset from the privacy-preserving visualization.}


% Existing mobile visualizations can compromise user privacy by allowing both the data owner (i.e., the user) and potential shoulder surfers to view sensitive data clearly. To address this issue, we propose a privacy-preserved visualization approach consists of two granular levels (Figure~\ref{fig:overiew}). The first level involves manipulating the spatial frequency (Section~\ref{method:sp}) and luminance contrast (Section~\ref{method:luminance}) based on the fundamental principles of the human vision system. It applies a binary mask to visual marks within the visualization to adjust their spatial frequencies, and then reduce their luminance contrast with the visualization background. Building upon this, we take into account the different needs of visual mark types. In other words, we develop customized schemes for visual marks due to their distinct characteristics.
Informed by prior research on human perception in Section~\ref{sec:background}, we propose a novel perception-driven approach to achieve privacy preservation for visualization on mobile devices.
Specifically, we present a masking scheme to process the bitmap image of an input visualization and transform it into a privacy-preserving one.
It consists of two major steps corresponding to two levels of processing granularity (Figure~\ref{fig:overiew}): \textit{coarse-grained masking} and \textit{fine-grained masking}.
Coarse-grained masking adjusts the spatial frequency of visual marks (Section~\ref{method:sp}) and their luminance contrast with the background in a visualization (Section~\ref{method:luminance}), which takes into account the fundamental principles of the human vision system.
Fine-grained masking further enhances the privacy preservation effect for visualizations by considering the distinct characteristics of different visual marks, which is informed by our Study 1 in Section~\ref{study:study1}.
Visual marks, such as circles in a scatter plot and bars in a bar chart, are fundamental elements in visualizations~\cite{munzner2014visualization,satyanarayan2015reactive,senay1994knowledge}. 
The source code for our approach is available online: \url{https://github.com/AlexanderZsh/Privacy-preserving-visualization}.

% ~\cite{satyanarayan2016vega}.
% Prior studies~\cite{munzner2014visualization, satyanarayan2015reactive,senay1994knowledge} classify visual marks as a point (zero dimensions), a line (one dimension), or an area (two dimensions) based on the required number of spatial dimensions. 
% In our case, 
% we regard both point-based marks as area-based marks (e.g., bars) because they occupy specific regions in a visualization.
By considering the areas that different visual marks occupy,
we categorize visual marks into two types: \textit{line-based marks} such as lines, texts, and axes, and \textit{area-based marks} such as bars and circles. We propose adaptive fine-grained masking schemes for them to further improve the privacy preservation of visualizations (Section~\ref{method:improvement}). 

% \yong{pls check my Overleaf comments and check it throughout the whole paper.}
% ~\alex{When will we upload the code}
% . The source code will be available once the paper has been published




% Visual visualizations consist of marks and visual channels.
% Marks are essential elements and refer to geometrically fundamental objects (e.g., points in a scatter plot, bars in a bar chart), and visual channels determine the appearance of marks such as color, size, and position~\cite{satyanarayan2016vega}. As introduced in Section~\ref{background:csf}, a human's ability to perceive a visualization is influenced by the spatial frequency and contrast of stimuli. Inspired by the characteristics of human vision, we developed a masking scheme that can change the spatial frequency (Section~\ref{method:sp}) and luminance contrast (Section~\ref{method:luminance}) of visualizations marks. Specifically, The mask area determines how many of the original pixels of the mark are retained, thereby influencing the spatial frequency of the resultant mark. Luminance contrast determines the visual contrast between the marks and the visualization background. Additionally, according to feedback both from method development and users in Study 1 (Section~\ref{study:study1}, we improve our method by categorizing the masking scheme into two parts, namely area-based and line-based marks, and adding a border (Section~\ref{method:improvement}) in area-based marks . The border can enhance the user's visibility of processed visualizations without compromising their privacy preservation. We will illustrate our method details in the following.

% A privacy-preserving visualization prevents others from viewing a person's visualization in public. In our study, we utilized the knowledge that human vision systems (HVS) display different visual sensitivity levels to information of various spatial frequencies. Regarding the visualization presented as an image on mobile devices, as demonstrated in \ref{fig:csf}, a person's ability to perceive a visualization is influenced by the contrast between its visual elements (e.g., marks) and their backgrounds (e.g., white canvas), as well as the spatial frequencies of the visualization image received by the HVS.
% Therefore, we transfer low spatial frequencies in the visualization to high spatial frequencies in spatial frequency (Section~\ref{method:sp}), because people rarely perceive high spatial frequencies at a distance. Then, since low contrast between visual elements and background can prevent peekers from immediately recognizing critical information, we also adjust  the contrast between the visualization's components and the background (Section~\ref{method:luminance}). Finally, we combine the results from the frequency-based and contrast-based approaches to create a privacy-preserving visualization.


\begin{figure}[ht!]
    \centering
    % \hspace{-2em}
    \includegraphics[width=1\linewidth]{figures/masking_v6.pdf}
    \caption{Area-based masking transforms area-based marks from low to high-frequency. (a) Examples of area-based masks with different mask areas. The pixel at the center of the mask remains unchanged, while the adjacent pixels are transformed to match the background color (e.g., white).
    % As the mask area increases, the visual marks will be transformed to a higher frequency.;
    (b) the effect of applying the area-based masking (Mask area: 5) to a bar chart ($b_1$) to convert it to its high spatial frequency version ($b_2$). (c) the frequency domain representations of ($b_1$) and ($b_2$).
    % which shows that the low-frequency information is transformed to high-frequency content. 
    % \yong{ Pls check my side comments.}
    % \yong{What is area-based masking here? Note: we have restructured the structure of our method!}
    }
    % \caption{It displays the mask example, chart's image, and spatial frequency spectrum before and after applying the masking. (a) It provides two masks with different mask areas. This type of mask works for area-based marks. (b) It displays the visualizations in the spatial domain where b$_1$ is the normal bar chart, and b$_2$ is the privacy-preserving bar chart. (c) It delineates the spatial frequency distribution of two visualizations in the frequency domain where c$_1$ refers to the original visualization and c$_2$ represents the privacy-preserving visualizations.}
    \vspace{-2em}
    \label{fig:masking}
\end{figure}

\subsection{Coarse-grained Masking}

Coarse-grained masking aims to increase the spatial frequency and reduce the luminance contrast of visual marks in an input visualization to prevent shoulder surfers from viewing the visualization at a certain distance and allow visualization owners to see it clearly.

\subsubsection{Increasing Spatial Frequency}\label{method:sp}
% \dm{add some examples of the mask in Figure 4. }

% \dm{1. talk about visulization graphs consist of background, visual marks, and texts. 2. you first need to detect the visual marks. 3. only apply the mask to the visual mark and why. 4. show examples.  }

% Visualizations consist of visual marks, backgrounds, axes, and texts. 
% \yong{Pls check my Overleaf comments.}
% Visual marks refer to geometrically fundamental objects in a visualization (e.g., points in a scatter plot, bars in a bar chart), and their appearance (e.g., color, size, and orientation) represents the visualization underlying data characteristics. Therefore, visual marks are the fundamental elements in visualizations~\cite{satyanarayan2016vega}. If shoulder surfers cannot view the visual marks, they cannot understand the complete visualization. 



Inspired by prior research on human vision (Section~\ref{background:csf}), we intend to increase the visual marks' spatial frequency using a binary mask.
First, we need to identify visual marks in an input visualization image. Given that visualizations usually have a white background, there is a clear color contrast between visual marks and the background of visualizations (Figure~\ref{fig:masking} (b$_1$)). In this paper, we leverage the Li Thresholding algorithm~\cite{li1998iterative}, which determines the color threshold between the background and visual marks. With the color threshold, we can identify visual marks from the background in visualization.
% \yong{what is ``the slope of cross entropy''?}
% First, we must distinguish the visual marks from the visualization background. \alexout{To visually emphasize the marks, the visualization makes a great contrast between its background and the corresponding visual marks.} Because visualization marks and background has a color difference (Figure~\ref{fig:masking} (b$_1$)) in terms of pixel values, we first distinguish the marks from the background using the Li Thresholding algorithm~\cite{li1998iterative}, which leverages the slope of cross entropy to determine the optimal pixel value  threshold between the background and the visual marks. 
% \alexout{Then, we apply the mask to these detected visual marks. Specifically, as shown in Figure~\ref{fig:masking} (a), the mask works on the part of visual marks, retains a pixel value in the center of the mask, and converts the color of the remaining pixels in the mask to the background color. The mask is applied to marks iteratively until it covers the entire mark. \alexin{The mask is tiled to cover the entire mark.} Thus, the visual mark, previously a complete block of one color, is divided into smaller blocks. The background color block separates the visual mark's blocks. As a result, the mask increases the processed visual mark's frequency. For example, there are three visual marks in Figure~\ref{fig:masking} (b$_1$). After processing by a mask with a specific mask area (Figure~\ref{fig:masking} (a)), these visual marks are composed of smaller blocks as shown in Figure~\ref{fig:masking} (b$_2$). The mask converts the spatial frequency of these rectangular visual marks from low spatial frequency (Figure~\ref{fig:masking} (c$_1$)) to high spatial frequency (Figure~\ref{fig:masking} (c$_2$)). Nonetheless, users are able to see the same characteristics (e.g., color, location, and size) of the processed marks (Figure~\ref{fig:masking} b$_2$) as the original mark (Figure~\ref{fig:masking} b$_1$), so the characteristics of the marks' appearance are preserved.}
% \alexin{Then we apply the mask to the detected visual marks. The mask operates on part of the visual marks and retains a pixel value in the center of the mask while converting the remaining pixels' color in the mask to the background color. The mask is tiled to cover the entire mark, effectively dividing the visual mark, which was previously a complete block of one color, into smaller blocks. Because the complete single color block is turned into multiple smaller blocks and these blocks are separated by the background, there exist alternative color changes in the processed visual mark. As a result, the frequency of the processed visual mark changes from low frequency to high frequency. For example, in Figure~\ref{fig:masking} (b$_1$), there are three bars as visual marks. After being operated by a mask with a specific mask area (e.g., mask area is 5 shown in Figure~\ref{fig:masking} (a)), these bars are transformed from a complete block with a single color (i.e., blue in Figure~\ref{fig:masking} (b$_2$)) to smaller blocks with two alternating colors (i.e., white and blue in Figure~\ref{fig:masking} (b$_2$)). Accordingly, in the frequency domain , these bars' frequency changes from low frequency (Figure~\ref{fig:masking} (c$_1$)) to high frequency (Figure~\ref{fig:masking} (c$_2$)). As a result, it is more difficult for attackers to identify the processed bars at a distance. Nonetheless, users are able to see the same characteristics of the processed bars (e.g., color, location, and size) as the original bars at close viewing distance. This means that the characteristics of the marks' appearance are preserved, despite being processed by the mask.}
% \yong{Pls check my Overleaf comments.}

% Then, we use the \textit{area-based} masking to process visual marks,
Then, we propose a masking scheme,
as shown in Figure~\ref{fig:masking}, to process which marks.
% Such a mask scheme is called \textit{area-based mask} in this paper.
Such processing is called \textit{area-based masking} in this paper.
Specifically, we overlay the mask on the areas of visual marks, where the pixel at the center of the mask is retained, and other pixels are converted to the background color (e.g., white). The mask is tiled to cover the entire mark.
Take the bar chart in Figure~\ref{fig:masking} (b$_1$) as an example, the smooth bars will be converted to dotted bars with high spatial frequency (Figure~\ref{fig:masking} (b$_1$)).
The spatial frequency distributions of the bar chart before and after being processed by our masking scheme are shown in Figure~\ref{fig:masking} (c$_1$) and Figure~\ref{fig:masking} (c$_2$) respectively, indicating the increased spatial frequency of the processed visualization.
Accordingly, it makes it difficult for shoulder surfers to see the visualization at a distance, while users at a closer viewing distance can still clearly identify all the information 
% of the original bar chart
from the processed visualization.
As the mask area increases, the bars
% s within the visualization 
become more sparsely dotted, resulting in an increase in spatial frequency and making it harder for shoulder surfers at a distance to identify the processed bars.
% Nonetheless, users who are closer to the visualization will still be able to identify the information.
% \yong{Need to add a few sentences to discuss the influence of the mask size. For example, Figure 6 shows two different masks. Then, which one is better and what is the influence of mask size?? These are the core information we need to talk about.}
% \yong{Important -- need to explicitly mention the core idea of different masks.}





% In an image, spatial frequencies correspond to changes in pixel values. When pixel values do not change significantly in an area of a picture, then that area has low frequencies. In contrast, if there is a change in an area (e.g., two adjacent pixels, one is black and one is white), the area has changed drastically and hence frequency becomes higher~\cite{gonzalez2009digital}. Inspired by the notion, we propose a masking scheme for visualization marks. Specifically, In a sub-region, the scheme retains a pixel value in the center of the mask and converts the color of the remaining masked pixels to the background color.
% Consequently, in the sub-region, a retained pixel has a significantly different color value than its adjacent pixels converted to the background color. Therefore, mark's frequencies improve.
% As a result, the processed visualization mark is no longer a complete geometry wherein all pixels have the same color, but is composed of many discrete pixel points. The processed mark is converted from low to high frequencies in its frequency domain. Figure~\ref{fig:masking} shows an example to explain the masking scheme. Three blue rectangles in Figure~\ref{fig:masking} a$_1$ are visualization marks of a bar chart, and they have strong low-frequency components in the frequency domain shown in Figure~\ref{fig:masking} b$_1$. After the masking scheme, rectangles are constructed with a lot of discrete points in Figure~\ref{fig:masking} a$_2$. In  the corresponding frequency spectrum, the low-frequency components are converted to high-frequency components in Figure~\ref{fig:masking} b$_2$. Nonetheless, users are able to see the same characteristics (e.g., color, location, and size) of the processed mark (Figure~\ref{fig:masking} a$_2$) as the original mark (Figure~\ref{fig:masking} a$_1$), so visual channels of visualization marks are kept as well.

% Figure~\ref{fig:masking}a$_1$) into different dot matrices (e.g., Figure~\ref{fig:masking}a$_2$). As shown in Figure~\ref{fig:masking}b, the bars that originally had low frequency will turn to high frequency in the frequency domain because the previously negligible adjacent pixel value difference in the original area will become significant. 

% visualization comprises marks and visual channels. The marks (e.g., bar, line) enable users to understand the underlying data in the visualization. 
% In particular, a mark is a specific geometric area where pixels have the same value. Our method can apply the masking to the marks and then convert the mark frequency to high frequency while retaining its original visual channels (e.g., size and color). 


% Also, pixels in a geometric mark usually have the same color value to ensure perception uniformity. Based on this assumption, we apply a masking scheme to the marks. Specifically, the scheme keeps the center pixels in the mask and masks the surrounding pixels, making the masked pixels the same color as the background. Therefore, these masked pixels have a considerable color value difference from their adjacent pixels that retain the original color, and then the frequencies in the mark improve. Following the masking, the mark is transformed into a dot matrix composed of the kept pixels, and the dot matrix retains the geometric shape of the mark.
% Additionally, users are able to see the same characteristics (such as color, location, and size) in the processed mask because the scheme only modifies some pixels while leaving the rest unchanged, so visual channels are kept as well. The mask will process a small area within the mark and move along the horizontal or vertical directions until it encounters the border of the mark. As a result of the masking scheme, users can distinguish the processed mark from the background and understand the underlying data from its unaffected visual channel at a close viewing distance. However, others at a distance cannot recognize the processed mark from the background due to its changed frequency. 








% \dm{1. after masking, reduce the luminance of the remaining visual marks. 2. however, convention RGB space does not allow the modification of luminance. 3. we propose to use another color space.... 4. how to adjust...}
\begin{figure}[ht!]
    \centering
    \includegraphics[width=0.8\linewidth]{figures/masking_result_v4.pdf}
    \caption{The effect of applying different luminance contrast on the converted high-frequency bar charts. The visibility reduces with the decrease in luminance contrast.
    When the luminance contrast reaches zero,
    % human cannot identify the visual mark from the background.
    we are not able to distinguish visual marks from the white background of a visualization.
    % \yong{It is better to change ``Luminance difference'' to ``luminance contrast'' in the figure.}
    }
    \label{fig:luminance}
    \vspace{-2em}
    
\end{figure}


\subsubsection{Reducing Luminance Contrast}\label{method:luminance}

% In addition to applying the mask on the visual mark, we need to adjust the luminance of the visual mark and the visualization background so that the shoulders surfer hardly perceives the resulting visual mark. 
Besides adjusting the spatial frequency of visual marks, we also decrease their luminance contrast with the visualization background to further prevent shoulder surfers from seeing the visual marks of an input visualization.
Instead of using the commonly-used RGB color space, we employ the CIELAB color space when adjusting the luminance contrast of visualizations.
The major reason is that the RGB color space cannot accurately model how human perceives luminance~\cite{munzner2014visualization}, but the L channel of CIELAB color space aligns well with the actual perception of luminance by human vision system~\cite{hanbury2002mathematical}.
% Therefore, we can change the single luminance channel in the visual marks and background in the CIELAB color space. 
Therefore, by changing the luminance of the pixels of visual marks and background in the CIELAB color space, we can accurately control the luminance contrast between them that will be perceived by users.
Figure~\ref{fig:luminance} illustrates the influence of different luminance contrasts
in terms of CIELAB's L channel on the visibility of visual marks. 
With the decrease in the luminance contrast between the visual marks and the background, it becomes increasingly difficult for humans to distinguish visual marks from the background.
% because their perceived luminance contrast decreases.



% % \yong{Reach here.}
% The contrast is a measurement of the luminance difference between an object and its background~\cite{bertalmio2019vision}. However, conventional RGB color space does not allow the modification of luminance in its space. The RGB color space specifies colors as red, green, and blue triples and is widely used in the computer graphics system. Though the color space is computationally convenient, it cannot match actual human perception~\cite{munzner2014visualization}. The RGB color space's red, green, and blue axes must combine to represent the perception of color. A change in the value of a single axis does not accurately reflect the actual change in the human perception of color. In the literature survey, we discovered that the CIELAB color space is a suitable replacement for RGB. In CIELAB color space, the L channel denotes the amount of luminance perceived by the HVS (human vision system), and the A and B channels indicate the hue in color space ~\cite{hanbury2002mathematical}. CIELAB is most impressive for its ability to model the luminance that humans perceive accurately. The change in the L channel is equal to the change in human perception. Therefore, we can change the single luminance channel in the visual marks and background in the CIELAB color space. The adjusted luminance difference between the visual marks and background corresponds to the actual human perceived luminance contrast. Figure~\ref{fig:luminance} illustrates the luminance difference between the masked visual marks and the background in terms of CIELAB's L channel. As the luminance difference between the visual marks and the background decreases, humans hardly distinguish the marks from the background because their perceived luminance contrast decreases.

% As depicted in Figure~\ref{fig:csf} (a), the spatial frequency determines the required contrast sensitivity in HVS. The lower the contrast sensitivity, the greater the contrast of the stimulus needed to be seen by the HVS~\cite{barten1999contrast}. In addition to pixel values, viewing distance also influences spatial frequency: a greater viewing distance will result in a higher spatial frequency perceived by HVS. Therefore, the significant contrast between the visualization elements and the background (Figure~\ref{fig:mask_result} (a)) is more important to viewers who are at a greater distance from the visualization than to those who are closer to it
% In light of this, we intend to adjust the contrast between the visualization elements and the background to a level that satisfies the relatively low contrast requirements of users but not the relatively high contrast requirements of peekers. In other words, people with high contrast sensitivity are able to recognize the adjusted contrast, but people with low contrast sensitivity cannot do so.

% The contrast is a measurement of the luminance difference between an object and its background~\cite{bertalmio2019vision}. Besides RGB, CIELAB provides a perceptually uniform color space, where the L channel denotes the amount of luminance perceived by the HVS, and the a and b channels indicate the hue in color space ~\cite{hanbury2002mathematical}. Furthermore, L can linearly model the luminance that we humans perceive. In other words, an equal step change of L is consistent with the equal step change of human perception~\cite{munzner2014visualization}. Therefore, we can change the luminance difference between the visual components and background in terms of CIELAB. The changed luminance difference corresponds to the adjusted contrast. As a result, users who are close to the visualization can perceive the luminance difference. In comparison, those far away cannot do so, thus confusing the components with the background as shown in Figure~\ref{fig:mask_result} (b).


\subsection{Fine-grained Masking}\label{method:improvement}
\label{sec: further_improve_area}
% \dm{Typical visualization chart types are..., which can be group as line-based, and area-based. the above mentioned two-stage scheme cannot work well with different types of visualization charts becasue. Thus, for each group, we further design some customized technique to improve the performance.}



The coarse-grained masking discussed above is designed to process all the visualizations without considering their own visual properties.
However, typical data visualization charts such as bar chart, pie chart, scatter plot, and line chart~\cite{battle2018beagle} consist of different visual marks and thus have distinct characteristics.
To further enhance privacy preservation performance, it is necessary for us to consider the unique visual properties of different visualizations.
Thus, building upon the coarse-grained masking, we further propose~\textit{fine-grained masking} to process input visualizations, as shown in Figure~\ref{fig:overiew}.
Specifically, we design adaptive masking schemes for line-based marks and area-based marks.
% \yong{Pls check my Overleaf Chinese comments and add useful comments to your check list.}

% Typical data visualization charts include bar chart, pie chart, scatter plot, and line chart~\cite{battle2018beagle}, whose visual marks can be categorized as line-based and area-based. The aforementioned \alexout{two-stage} masking scheme cannot work well with different visualization types, because \textit{line-based elements}, including line mark, text and axes, and \textit{area-based elements}, including geometric marks (e.g., bar, circle), have distinct characteristics. Thus, we further design customized techniques for each category to improve privacy-preserving visualization effectiveness \alexin{(Figure~\ref{fig:overiew})}. In the paper, the \textit{line-based masking scheme} refers to the masking method on the line-based elements, and the \textit{area-based masking scheme} is used for the area-based elements.


\begin{figure}[ht!]
    \centering
    % \hspace{-1.5em}
    \includegraphics[width=\linewidth]{figures/line_masking_v4.pdf}
    \caption{Line-based masking converts line-based marks from low to high frequency. 
    (a) An area-based mask with a mask area of 5 (a$_1$) and a line-based mask with a mask area of 5 (a$_2$).
    % Examples of the area-based masking whose mask area is 5; (a$_2$) examples of the line-based masking whose mask area is 5 as well; 
    (b) an input line chart. (c) the line chart processed with area-based masking, where the dashed red rectangles highlight the problematic regions in the processing result.
    % . \alexnewin{The region in the line chart enclosed by dashed rectangles refers to the failure result that area-based masking cannot solve well for the line-based marks;} 
    (d) the line chart processed with line-based masking, which retains more text and line information.
    % \yong{1. What is the purpose of Figure 8a here? 2. What is its relation with other subfigures b, c and d? 3. The figure is distorted! }
    }
    % \caption{This figure shows the result of the masking scheme on a line chart. (a) It provides two masks with different mask areas. This type of mask works for line-based marks. (b) It shows the original line chart, which delineates three companies' stock. (c) It shows the resulting visualization by area-based masking, where rectangles with a red dashed line refer to failure from the area-based mask. (d) It presents the resulting visualization by line-based masking.}
    \vspace{-2em}
    \label{fig:line_masking}
\end{figure}

% \subsubsection{Enhancement on the line-based element}

\subsubsection{Adaptive Masking for Line-based Marks}\label{sec-adaptive-mask-for-line}



As discussed above, we categorized visual marks into \textit{line-based marks} and \textit{area-based marks} due to their differences in the occupied areas.
During the development of our approach, we also notice that there is no one-size-fits-all solution that works well for both line-based marks and area-based marks of visualizations.
For example, Figure~\ref{fig:line_masking}(c) is the processed result of the input line chart (Figure~\ref{fig:line_masking}(b)) by using the area-based masking (Figure~\ref{fig:line_masking}(a$_1$)), where the masking scheme keeps only the pixel at the center. 
However, it is difficult to identify the lines and texts due to the obvious discontinuity in a few parts of these line-based marks (as shown within dashed rectangles in Figure~\ref{fig:line_masking}(c)). For some parts of the lines, the line segments are even broken, making it difficult to determine the trend of lines. 
Since the width of line-based marks (e.g., lines, axes, and texts) is often smaller than area-based marks, and it is essential to preserve the orientation of line-based marks,
we propose a new masking scheme for line-based marks, as shown in Figure~\ref{fig:line_masking}(a$_2$).
Figure~\ref{fig:line_masking}(d) shows the processed result of the input line chart by using the new masking scheme. Such processing is called \textit{line-based masking} in this paper.
By keeping more pixels surrounding the center,
it is clear to see that such a new masking scheme can better preserve the visual information of line-based marks while increasing their spatial frequency.

% When developing the method, we found that the coarse-granular masking scheme did not apply to the line-based elements. For example, as shown in Figure~\ref{fig:line_masking} (b,c), it is hard to identify the line marks and text in the line chart after the coarse-granular masking because the masking scheme results in the discontinuity of line marks. Discontinuities affect lines significantly more than other geometries because the direction of change is uncertain. It is difficult for the user to determine a line's trend if a line segment is missing.
% In a 2D visualization, a mark could be classified as a point (zero dimensions), a line (one dimension), or an area (two dimensions) based on the required number of spatial dimensions~\cite{munzner2014visualization, satyanarayan2015reactive,senay1994knowledge}. 
% In our case, we regard both point-based marks as area-based marks (e.g., bars) because they occupy specific regions in a visualization. Then we apply different masking schemes to area-based and line-based elements. 
% In addition to the line marks, the horizontal and vertical axes and text in visualizations on also use a line-based masking scheme because they are naturally made of lines. 
% In addition to the line marks, we apply the line-based masking scheme to the axes and text in visualizations because they are made of lines by nature.
% The line-based masking scheme aims to raise the frequencies of the axes, text, and line marks so that peekers cannot see them from a distance. The reason why we apply masking schemes to axes and text is that they can indicate the essential information of the visualization. For example, the visualization title enables users to immediately understand what the visualization is about by reading it because the title provides a clear overview. In conjunction with the visualization title, axes titles and axes help users quickly recognize the specific information encoded in the data points and associate the data points with values. Therefore, preventing these information leakages from peeking by others is necessary.
% Unlike area-based elements, the line width is small, and some pixel points in the line are more important than others. 
% Therefore, we need to propose a new mask design that keeps more pixels in the line-based mask, as shown in Figure~\ref{fig:line_masking}. Otherwise, users hardly perceive the line. Moreover, the line trend may vary in any orientation. For instance, the line in a line chart represents the underlying data's trend, and the turning points in the line denote a change in trend ~\cite{udagawa2018predicting}.  Line-based masks process a small region of the line at a time and move in the direction of the line. Therefore, the change of line is preserved in the resulting privacy-preserving visualizations.
% \alexout{
% Additionally, since the axes and line marks in visualization are consistent in size, but text font sizes vary, we apply an adaptive mask to the text to ensure that the line-based mask can correctly process texts with varying sizes. Specifically, we use an OCR (Optical Character Recognition) tool, EasyOCR, which leverages a neural network model to detect text region from a visualization~\cite{jaidedai}.
% Following the identification of the text region on the visualization, it is necessary for us to determine the text size in terms of its stroke width. First, we reduced the stroke width of the text to 1px, which is the minimum size for a text~\cite{van2014scikit}. Then we calculated the stroke width difference between the original text  and the minimum-size text~\cite{virtanen2020scipy}. The text stroke was obtained by averaging the stroke width differences.
% Further, we empirically determined the appropriate line-based mask area values for different text sizes.
% Thus, the mask can adjust its area depending on the text size.
% As shown in Figure~\ref{fig:line_masking} (c,d), line-based masking outperforms area-based masking on line-based elements such as axes, texts, and line marks. 
% In Study 2 (Section~\ref{sec:study2}), the text processing operations are applied to the text on the visualization test samples.
% }
Furthermore, we adaptively adjust the size of the line-based masking for line-based marks according to their width. Among all the line-based marks (e.g., lines, axes, and texts), the width of lines and axes in data visualization charts are relatively stable and consistent. However, texts can vary a lot due to different font sizes, which motivates us to adaptively vary the size of the masking scheme to process texts specifically.
To this end, we first employ EasyOCR~\cite{jaided2020easyocr}, a widely used Optical Character Recognition
(OCR) tool to detect texts in the input visualization image.
% Then, we further leverage the Li Thresholding algorithm~\cite{li1998iterative} to extract strokes of texts and determine text stroke width.
% \yong{We change it back to the commented sentence after being finally accepted.}
Then, we further extract strokes of texts and determine text stroke width by 
using the fast parallel thinning algorithm~\cite{zhang1984fast} that has been integrated to the package Scipy~\cite{virtanen2020scipy}. 
The text stroke width is used to guide our empirical configuration on the adaptive mask size of our masking scheme for line-based marks.
% Such adaptive masking for texts consists of two primary steps: text detection and text size determination. First, we employ EasyOCR~\cite{jaided2020easyocr}, a widely used Optical Character Recognition
% (OCR) tool to detect texts in the input visualization image. Once the text regions in the visualization image have been detected, the mask area is adjusted according to the text size,
% which is determined by the stroke width. The stroke width refers to the thickness of each stroke in the text and is calculated by the distance from one edge of the stroke to the other, perpendicularly crossing the centerline of the stroke. To obtain the centerlines of the strokes, we skeletonize~\cite{zhang1984fast} the detected text by reducing the stroke to its centerline. We then calculate the width of each stroke from its centerline to its edge and compute the average width of all strokes in the text to determine the text size. The adaptive masking scheme can adjust its mask area for different text sizes by empirically determining line-based mask area values for different text sizes. 
% \yong{pls check my overleaf comments.}
 
% \yong{Songheng, pls fill it. I totally cannot understand what you have written above.} 




% \subsubsection{Area-based masking with border}


\subsubsection{Customized Masking for Area-based Marks} \label{sec-mask4areamarks}


% \alexout{
% Initially, we only applied a masking scheme on the area-based and line-based elements, respectively, and then conducted Study 1 (Section~\ref{study:study1}) to measure the effects of mask area and luminance contrast on human visibility to resulting visualizations. However, according to Study 1 participants' feedback, they hardly identify detailed information on specific area-based marks. For example, a lack of border lines between slices in a pie chart reduces confidence in the chart's readability. Participants should identify different slices in the pie, but they could not find the border between slices and thus had to speculate the portion of the slice by the area of different colored pixels. The speculation result is subjective and inaccurate. Similarly, participants could identify the processed area-based markers for other visualizations, such as bar charts or scatter plots. However, they requested a reference to help them quickly compare different area-based markers, such as the comparison of bar heights. Based on participants' feedback, in the privacy-preserving visualizations, we applied line-based masking to the border of area-based marks, making them invisible at a distance, but these processed borders provide an additional reference for users to view the area-based marks at close viewing distance.
% }\textbf{}



For the area-based marks like bars and circles, we initially leverage the area-based masking (Figure~\ref{fig:masking}(a)) to process them without specifically handling the borders of area-based visual marks.
As will be introduced in Section~\ref{study:study1}, we follow such a setup to evaluate 
% the effects of mask area and luminance contrast on 
the visibility of visualizations. The participants' feedback shows that the proposed area-based masking approach can achieve a good privacy preservation effect for area-based marks. However, it also makes it difficult for participants to accurately perceive the corresponding data of area-based visualizations due to the overly discretized borders of area-based marks. 
For example, for a processed pie chart, an excessively sparse border between two adjacent slices makes it difficult for human users to accurately identify the boundary between the two adjacent pie slices even at a close viewing distance, as shown in Figure~\ref{fig:study2_sample} (c) of Appendix~\ref{sec:appendix}.
To address this issue, we apply line-based masking,
% the masking scheme for line-based masks,
as introduced in Section~\ref{sec-adaptive-mask-for-line}, to specifically process the borders of area-based marks, enhancing the accurate perception of area-based marks at a close viewing distance and guaranteeing privacy preservation for visualization above a certain viewing distance.
% making them invisible at a distance, but these processed borders provide an additional reference for users to view the area-based marks at close viewing distance.
% \yong{pls check my overleaf comments and do the changes accordingly.}




% According to the study 1 result\alex{add section reference later}, we add a high-frequency auxiliary line along the geometry's perimeter to enhance user efficiency when viewing the privacy-preserving visualization. However, this additional line is invisible to observers at a distance.


% \subsection{Visibility Calculation}

% \begin{figure}[h!]
%     \centering
%     \includegraphics[width=0.6\linewidth]{figures/visibility.jpg}
%     \caption{The figure shows a visualization representation in the 1D spatial frequency spectrum and its visibility result. In the (a), the x-axis refers to the spatial frequencies, and the y-axis refers to their amplitude (i.e., power in the spectrum). In the (b), it indicates how many frequencies in visualization can be seen by the human eye. The red curve represents the reciprocal of the CSF, and black dots above the red curve are the frequencies that the human can perceive in the HVS.}
%     \vspace{-1em}
%     \label{fig:visibility}
% \end{figure}
% Although our method can change the frequencies and luminance contrast of an original visualization to create a privacy-preserving visualization, we cannot determine the obtained visualization's visible and invisible range. Therefore, it is necessary to get a reference about how far the human eye can and cannot perceive the visualization, respectively. Additionally, since the visualization display effect is relevant to the mobile device screen, we aim to develop a formula to give a reference as to the human visibility to the visualization, using the screen resolution, viewing distance, and the resulting frequency domain visualization. In other words, the score can represent how many frequencies in the visualization can be seen by the human eye at a given distance.

% As mentioned in Section~\ref{background:sp}, a visualization image is composed of different frequencies with respect to the frequency domain. With the help of Fourier transformation, we can obtain the visualization representation in the frequency domain as shown in Figure~\ref{fig:masking} (b). Afterward, we should convert the 2D frequency domain representation into a 1D frequency series. The resulting 1D series shows the frequency power spectrum regarding the visualziation~\cite{isenberg2013hybrid}. Different from typical images (e.g., profiles, landscape pictures), the visualization is a simple image because it is composed of simple geometric shapes (e.g., line, bar) and thus there is not always power (i.e., amplitude) at every frequency as shown in Figure~\ref{fig:masking}(b2)~\cite{gircys2019image}. To avoid canceling out significant power frequencies, we sum frequencies out by radii rather than averaging them when converting a 2D frequency domain to a 1D spectrum. We then utilize Formula~\ref{formula:cpd} to transform the unit of spatial frequency from pixel per cycle (ppc) to cycle per degree (cpd). The obtained cpd takes the human viewing distance into account~\cite{isenberg2013hybrid}. We then derive the contrast of spatial frequencies by the following formula~\cite{hess1983contrast}:

% \begin{equation}\label{formula:contrast}
%     C(sp)  =\frac{2A(sp)}{DC}
% \end{equation}

% where \textit{C(sp)} denotes the contrast of a specific spatial frequency, \textit{A(sp)} denotes the amplitude (i.e., power) of the spatial frequencies, and \textit{DC} refers to the zero-frequency component~\cite{nunez2017elegant}.

% With Formula~\ref{formula:contrast}, we can get the contrast with respect to the frequency and know how many frequencies in a visualization image the human eye can see at a given distance. Specifically, since the reciprocal of contrast sensitivity is threshold contrast~\cite{national1985emergent}, we can know how many frequencies in a visualization a human can see at a given distance, as shown in Figure~\ref{fig:visibility}.
\section{EXPERIMENTS AND ANALYSIS}
\label{results}
\subsection{Experiment Settings}
The ConvS2S model has 512 hidden units for both encoders and decoders. All embeddings, including the output produced by the decoder before the final linear layer, have a dimensionality of 768. This setup allows the encoders to concatenate with patch embeddings from ViT model. To avoid overfitting, dropout is applied on the embeddings, decoder output, and the input of the convolutional blocks with a retaining probability of 0.5.


% We train the convolutional model using Adam optimizer with a fixed learning rate 2.50e-4.
Many experiments are carried out in order to evaluate the proposed approach toward the VLSP-EVJVQA challenge. We begin by initializing the baseline result of ConvS2S without using any image information. This mean that the generated answers are completely based on the answer-question dependencies learned by the model during the training phase. We then sequentially add hint and image features to the input sequence and study their effect on the overall performance. Because of the limitation in computational resources as well as the strict timeline of the competition, we only deploy the fine-tuned ViLT-B/32 with 200K pretraining steps and pre-trained OFA$_{\mathrm{large}}$ with 472M parameters for hints inference given the question and image.
To have the comparative result, we set up the same hyperparameters for all experiments. The models are trained in 30 epochs using Adam optimizer with a fixed learning rate of 2.50e-4 and batch size of 128. After each epoch, the performance loss on the train and development sets is calculated using the Cross-Entropy Loss function.

The proposed architecture and SOTA vision and language models are implemented in PyTorch and trained on the Kaggle platform with hardware specifications: Intel(R) Xeon(R) CPU @ 2.00GHz; GPU Tesla P100 16 GB with CUDA 11.4.

\subsection{Experimental Results}
\begin{table}[H]

    \centering
    \resizebox{\columnwidth}{!}{%
    \setlength{\tabcolsep}{5pt}
    \renewcommand{\arraystretch}{1.2}
    \begin{tabular}{lcccccc}
    \toprule
        \textbf{Model} & \textbf{F1} & \textbf{BLEU-1} & \textbf{BLEU-2} & \textbf{BLEU-3} & \textbf{BLEU-4} & \textbf{BLEU (Avg.)}  \\ \midrule
        ConvS2S (no image features) & 0.3005 &0.2592	&0.2034	&0.1677	&0.1425& 0.1932  \\ \midrule
        ConvS2S + ViLT-B/32 & 0.3294 &0.2692	&0.2109	&0.1723	&0.1446& 0.1993  \\ 
        ConvS2S + OFA$_{\mathrm{large}}$ & 0.3331 &0.2858	&0.2269	&0.1876	&0.1598 & 0.2150  \\ 
        \textbf{ConvS2S + ViLT-B/32 + OFA$_{\mathbf{large}}$}
        % \tablefootnote{This model is not yet evaluated on the private test set \label{note1}}
        & \textbf{0.3442} &0.2797	&0.2205	&0.1808	&0.1529& \textbf{0.2085}  \\ 
        \midrule
                ConvS2S + ViT-B/16 & 0.3109 &0.2683	&0.2119	&0.1747	&0.1480 & 0.2007  \\ %\midrule
        ConvS2S + ViT-B/16 + ViLT-B/32 & 0.3361 &0.2833	&0.2243	&0.1845	&0.1564 & 0.2122  \\ 
        ConvS2S + ViT-B/16 + OFA$_{\mathrm{large}}$ & 0.3390 &0.2877	&0.2276	&0.1877	&0.1593 & 0.2156  \\
        \textbf{ConvS2S + ViT-B/16 + ViLT-B/32 + OFA$_{\mathbf{large}}$}
        % \textsuperscript{\ref{note1}}
        & \textbf{0.3442} & 0.2747	&0.2148	&0.1747	& 0.1465& \textbf{0.2027} \\ \bottomrule
    \end{tabular}}
    \caption{Performance of ConvS2S with different combinations of pre-trained models on the public test set.}
    \label{result_public}
\end{table}

\begin{figure}[ht]
\centering
% \subfloat[ConvS2S training loss per epoch]{%
%   \includegraphics[width=0.495\textwidth]{figure/train_loss1.pdf}%
% }
% \hspace{-0.2em}
% \subfloat[ConvS2S testing loss per epoch]{%
%   \includegraphics[width=0.495\textwidth]{figure/test_loss1.pdf}%
% }
\includegraphics[width=\textwidth]{figure/all_loss.pdf}
\caption{Training loss and public testing loss comparison of ConvS2S model with different combinations of hint and image features}
\label{loss}
\end{figure}


The two metrics: F1 and BLEU, are used in the challenge to evaluate the results. The BLEU score is the average of BLEU-1, BLEU-2, BLEU-3, and BLEU-4. F1 is used for ranking the final results. Table \ref{result_public} presents the performance of the proposed ConvS2S model with different combinations of pre-trained models on the UIT-EVJVQA public test set.

% First, with only question as input, ConvS2
According to Table \ref{result_public}, the original ConvS2S model without image features but using only question obtained 0.3005 by F1 and 0.1932 by BLEU. When integrating hint features from images, the F1 score improved at least 2.89\% and achieve highest result with 0.3442 by F1 and 0.2085 by BLEU when using both ViLT and OFA hints. After adding image feature from ViT-B/16, the performance of previous models tend to improve. However the final ensemble does not surpass the ConvS2S{\tiny~}+{\tiny~}ViLT-B/32{\tiny~}+{\tiny~}OFA$_{\mathrm{large}}$ ensemble on F1 metrics and even give lower BLEU score. Based on F1, these two ensembles are considered as our best models on the public test set. 
Figure \ref{loss} depicts the gradual improvement in both training loss and testing loss as more image features are added to the ConvS2S model. Memory-based ConvS2S does not catch the image context and thus have the highest loss. Though ConvS2S with ViT+VILT features does not obtained a competitive result on F1 and BLEU score, it has the best loss among methods in the public test phase. In general, the optimal testing loss of methods is achieved between 14th and 20th epoch, then the models tend to be overfitting.


% \begin{figure}
%     \centering
%     \includegraphics[width=\textwidth]{figure/train_loss.pdf}
%     \caption{tmp}
%     \label{100score}
% \end{figure}
% \subsubsection{Qualitative analysis}
% \label{quali_analysis}

% \begin{figure}
%     \centering
%     \includegraphics[width=\textwidth]{figure/test_loss.pdf}
%     \caption{tmp}
%     \label{100score}
% \end{figure}
% \subsubsection{Qualitative analysis}
% \label{quali_analysis}

We manage to deploy two ensembles of ConvS2S using features from ViT-B/16 combined with hints from {\tiny~}ViLT-B/32 and {\tiny~}OFA$_{\mathrm{large}}$, respectively, for the final evaluation on private test set. As shown in Table \ref{result_private}, the ConvS2S{\tiny~}+{\tiny~}ViT-B/16{\tiny~}+{\tiny~}OFA$_{\mathrm{large}}$ model obtained the better result, which is 0.4210 by F1 and 0.3482 by BLEU, and ranked $3^{rd}$ in the challenge. Table \ref{ranking} shows the final standing at the EVLSP-EVJVQA competition, in which our best model perform poorer 1.82\% and 1.39\% by F1 compared with the first and second place solutions. Overall, there is a gap between F1 and BLEU scores.



\begin{table}[H]
    \centering
    \small
    %\resizebox{\columnwidth}{!}{%
    \setlength{\tabcolsep}{5pt}
    \renewcommand{\arraystretch}{1.2}
    \begin{tabular}{lcc}
    \toprule
        \textbf{Model} & \textbf{F1} & \textbf{BLEU}  \\ \midrule
        ConvS2S + ViT-B/16 + ViLT-B/32 &0.4053  &0.3228  \\
        \textbf{ConvS2S + ViT-B/16 + OFA$_{\mathbf{large}}$} & \textbf{0.4210}  & \textbf{0.3482}
  \\ \bottomrule
    \end{tabular}
    \caption{Performance on the private test set.}
    \label{result_private}
\end{table}

\begin{table}[!htbp]
\small
%\resizebox{\columnwidth}{!}{%
\centering
\begin{tabular}{clccccc}
\toprule
\multirow{2}{*}{\textbf{No.}} & \multirow{2}{*}{\textbf{Team name}} & \multicolumn{2}{c}{\textbf{Public Test}} && \multicolumn{2}{c}{\textbf{Private Test}} \\\cmidrule{3-4} \cmidrule{6-7}
                             &                                     & \textbf{F1}         & \textbf{BLEU}      && \textbf{F1}         & \textbf{BLEU}       \\\midrule
1                            & CIST AI                             & 0.3491              & 0.2508             && 0.4392              & 0.4009              \\
2                            & OhYeah                              & 0.5755              & 0.4866             && 0.4349              & 0.3868              \\
3                            & \textbf{DS\_STBFL}                  & \textbf{0.3390}     & \textbf{0.2156}    && \textbf{0.4210}     & \textbf{0.3482}     \\
4                            & FCoin                               & 0.3355              & 0.2437             && 0.4103              & 0.3549              \\
5                            & VL-UIT                              & 0.3053              & 0.1878             && 0.3663              & 0.2743              \\
6                            & BDboi                               & 0.3023              & 0.2183             && 0.3164              & 0.2649              \\
7                            & UIT\_squad                          & 0.3224              & 0.2238             && 0.3024              & 0.1667              \\
8                            & VC\_Internship                      & 0.3017              & 0.1639             && 0.3007              & 0.1337
\\\bottomrule       
\end{tabular}
\caption{Our performance compared with other teams at VLSP2022-EVJVQA}
\label{ranking}
\end{table}

\subsection{Performance Analysis}

According to the final result in the private test phase, the generated output from ConvS2S
+ViT-B/16+OFA$_{\mathrm{large}}$ model are chosen for further analysis. Generally, the model manages to generate answers with correct language with the input question.
\subsubsection{Quantitative analysis}
We randomly choose 100 samples from the generated result to perform quantitative analysis. The average length, vocabulary size, and the number of POS tags in the ground truth and generated answers are calculated for each language. Table \ref{quanti} shows the statistics of the ground truth answer compared with the predicted answer by the model.

% \begin{table}[ht]
% \centering
% %\resizebox{\columnwidth}{!}{%
% \begin{tabular}{llrr}
% \toprule
% &Language&Ground Truth&Predicted\\\midrule

% \multirow{ 4}{*}{Avg. length} & English & 3.74 & 6.18 \\
% & Vietnamese & 4.42 & 5.97\\
% & Japanese & 4.67 & 8.43\\
% & All &4.26&6.78\\\midrule

% \multirow{ 4}{*}{Vocab. size} & English & 78 & 72 \\
% & Vietnamese & 97 & 101\\
% & Japanese & 77 & 83\\
% & All &252&256\\\midrule

% \multirow{ 4}{*}{\# POS tag} & English & 12 & 9 \\
% & Vietnamese &10  &9 \\
% & Japanese & 10 & 11\\
% & All &14 &14\\
% \bottomrule
% \end{tabular}
% \caption{The quantitative statistic of 100 generated samples compared with the ground truth}
% \label{quanti}
% \end{table}

\begin{table}[ht]
\centering
%\resizebox{\columnwidth}{!}{%
\begin{tabular}{llrr}
\toprule
Language&Stats.&Ground Truth&Predicted\\\midrule
\multirow{ 3}{*}{English} & Avg.length  & 3.74 & 6.18 \\
& Vocab. size & 78 & 72 \\
& \# POS tag  & 12 & 9 \\\midrule

\multirow{ 3}{*}{Vietnamese} & Avg.length  & 4.42 & 5.97 \\
& Vocab. size  & 97 & 101 \\
& \# POS tag &10  &9 \\\midrule

\multirow{ 3}{*}{Japanese} & Avg.length   & 4.67 & 8.43 \\
& Vocab. size & 77 & 83 \\
& \# POS tag  & 10 & 11 \\\midrule\midrule

\multirow{ 3}{*}{All} & Avg.length  &4.26 &6.78 \\
& Vocab. size &252 &256 \\
& \# POS tag  &14 &14 \\

\bottomrule
\end{tabular}
\caption{The quantitative statistic of 100 generated samples compared with the ground truth}
\label{quanti}
\end{table}

From Table \ref{quanti}, it can be seen that although the model gave the answers longer than the ground truth answers, the semantics is not as much as the ground truth. It can be seen from Table \ref{quanti} that the predicted answers in English have an average length higher than the ground truth answers. Also, the vocabulary in the generated answers is more than the original. In contrast, the number of POS tag components in the predicted answers is lower than the ground truth. This is similar to the answers in Vietnamese. For the Japanese, the characteristics of the predicted answers in average length and vocabulary size are the same as the two remaining languages. However, the number of POS tags in the predicted answers is more than in the ground truth answers. To make it clear, we propose three types of error on our model in Section \ref{quali_analysis}.

In addition, Figure \ref{100score} illustrates the distributions of F1 and BLEU scores for each language. Generally, the histograms skewed to the right and the model  performs inconsistently across languages. The proportion of samples with F1 and BLEU scores less than 0.2 dominates the overall result across all three languages. In Vietnamese, the number of generated samples with F1 and BLEU scores greater than 0.4 is significantly higher than in other languages. Meanwhile, English and Japanese responses rarely score greater than 0.6 on both metrics, furthermore, no Japanese samples scoring greater than 0.8 in BLEU. This illustrates that our model faces numerous challenges in producing the desired responses, with specific limitations on each language.

\begin{figure}[!ht]
    \centering
    \includegraphics[width=\textwidth]{figure/hist.pdf}
    \caption{Distributions of F1 and BLEU scores for each language from 100 generated samples}
    \label{100score}
\end{figure}

\begin{figure}[!htbp]
\centering
\subfloat[]{%
  \includegraphics[width=0.8\textwidth]{figure/attns1.pdf}%
  \label{attn1}
}

\subfloat[]{%
  \includegraphics[width=0.8\textwidth]{figure/attns2.pdf}%
  \label{attn2}
}

\subfloat[]{%
  \includegraphics[width=0.8\textwidth]{figure/attns3.pdf}%
  \label{attn3}
}

\subfloat[]{%
  \includegraphics[width=0.8\textwidth]{figure/attns4.pdf}%
  \label{attn4}
}

\subfloat[]{%
  \includegraphics[width=0.8\textwidth]{figure/attns5.pdf}%
  \label{attn5}
}
\caption{Numerous samples of attention alignment from ConvS2S and the changes in attention when adding features from ViT-B/16 and OFA$_{\mathrm{large}}$. The x-axis and y-axis of each plot correspond to the words in the question and the generated answer, respectively, while each pixel illustrates the weight $w_{ij}$ of the assignment of the j-th question word for the i-th
answer word.}
\label{attn}
\end{figure}

\subsubsection{Qualitative analysis}
\label{quali_analysis}
\paragraph{Attention visualization}

Figure \ref{attn} shows several samples of attention weights between each element from the generated answer with those in the input sequence that contains no image features, OFA hints, and OFA+ViT features, respectively. The visualization provided an intuitive way to discover which positions in the input sequence were considered more important when generating the target answer word. The brighter a pixel's color, the more important the word in the input sequence is in producing the respect answer word. Through this, we study that OFA hint is importance feature to model's attention as it provide the near-correct insight for the question and reduce the reliance on question words when generating the answer. However, in some cases, the model focuses too much on a specific element of the hint, which may lead to bias. ViT features has shown to control the affection of OFA hint, neutralizing it with other elements from question if hint appears to be off-topic. It may enhance the attention, making the model focus stronger on specific parts of the provided hint, for instance, the hint token ``nhà hàng'' (\textit{restaurant}) in Figure \ref{attn3} is given more attention when adding ViT image features. These features can also reduce the attention in one element and distributes concentration on other parts of the sequence. Figures \ref{attn1} and \ref{attn2} depict the reduction in hint attention into question elements, while Figures \ref{attn4} and \ref{attn5} show the changes in attention weight distribution among hint tokens.

\paragraph{Error analysis}
\begin{figure}[!ht]
\centering
\subfloat[Error Case I]{%
  \includegraphics[width=\textwidth]{figure/err1.pdf}%
\label{fig:1a}}
\vspace{1em}
\subfloat[Error Case II]{%
  \includegraphics[width=\textwidth]{figure/err2.pdf}%
    \label{fig:1b}
}
\vspace{1em}
\subfloat[Error Case III]{%
  \includegraphics[width=\textwidth]{figure/err3.pdf}%
\label{fig:1c}}
\caption{Three typical error cases from generated results.}
\label{fig:1}
\end{figure}

% \begin{figure}[H]
% \centering
% \resizebox{\textwidth}{!}{
%     \begin{subfigure}[b]{.3\linewidth}
%     \centering
%     \includegraphics[width=0.99\textwidth]{figure/00000001682.jpg}
%     \raggedright
%     { \scriptsize \textbf{Question}: what hat does the narrator of the 
%     historical site wear?}\\
%     {\scriptsize \textbf{Groundtruth}: non la}\\
%     {\scriptsize \textbf{Predicted}: the boy wears a white shirt and white and white}\\
%     {\scriptsize \textbf{F1:}  0.0000}\\
%     {\scriptsize \textbf{BLEU:} 0.0000
%     ~~~~~~~~~~~~~~~~~~~~~~~~~~~~~~~~~~~~~~~~~~~~~~~~~~~~~~~~~~~~~~~~~~~~~~~~~~~~~~~~~~~~~~~~~~~~~~~~~~ }
%     \caption{Error Type I}
%     \label{fig:1a}
%   \end{subfigure}%
%   \hspace{0.5em}
  
%  %\hspace*{\fill}
%   \begin{subfigure}[b]{.35\linewidth}
%     \centering
%     \includegraphics[width=0.99\textwidth]{figure/00000004737.jpg}
%     \raggedright
    
%     {\scriptsize \textbf{Question}: có bao nhiêu người đứng bên phải chàng trai? (\textit{English: How many people on the right of the man?})}\\
    
%     {\scriptsize \textbf{Groundtruth}: có ba người đứng bên phải chàng trai (\textit{English: There are three people on the right of the man})}\\
    
%     {\scriptsize \textbf{Predicted}: có hai người đứng bên phải chàng trai (\textit{English: There are two people on the right of the man})}\\
    
%     {\scriptsize F1:  0.8750}\\
    
%     {\scriptsize BLEU: 0.7799}
%     \caption{Error Type II}
%     \label{fig:1b}
%   \end{subfigure}%
%   \hspace{0.5em}
%   %\hspace*{\fill}
%   \begin{subfigure}[b]{0.35\linewidth}
%      \centering
%     \includegraphics[width=0.99\textwidth]{figure/00000000111.jpg}
    
%     \raggedright {\scriptsize \textbf{Question}:}
%     {\tiny
%     \begin{CJK*}{UTF8}{min}
%     {\CJKfamily{goth}小船手は何本のオールを使っていますか? (\scriptsize \textit{English: How many paddles does the boatman use?})}
%     \end{CJK*}}\\
%     {\scriptsize \textbf{Groundtruth}: 2}\\
%     {\scriptsize \textbf{Predicted}:}
%     {\tiny
%     \begin{CJK*}{UTF8}{min}
%     {\CJKfamily{goth}2本の船を使っています (\scriptsize \textit{ English: using two boats})}
%     \end{CJK*}}\\
%     {\scriptsize \textbf{F1:} 0.0000}\\
%     {\scriptsize \textbf{BLEU:} 0.0000 ~~~~~~~~~~~~~~~~~~~~~~~~~~~~~~~~~~~~~~~~~~~~~~~~~~~~~~~~~~~~~~~~~~~~~~~~~~~~~~~~~~~~~~~~~~~~~~~~~~ }\\
%     \caption{Error Type III}
%     \label{fig:1c}
%   \end{subfigure}%  
% }
%   \caption{Example of generated answers that contains errors.(b) the keyword 'hai người' (two people) is given  instead of 'ba người' (three people). Coincidentally, the question and groundtruth in this case both share the same phrase "đứng bên phải chàng trai" ("on the right of the man"), }\label{fig:1}
% \end{figure}


For better understand the generation performance on the VQA task, we examine the generated answers of our best ensemble, ConvS2S
+ViT-B/16+OFA$_{\mathrm{large}}$, to identify the limitations and analyze factors that may cause the model to perform poorly.
Through the error analysis process, various errors and mistakes have been pointed out in the outputs of the model. The typical examples of various types of errors are illustrated in Figure \ref{fig:1}. In summary, we divide these errors into three groups:

\begin{itemize}
    \item The generated answer does not match the question and has no correct tokens compared with the ground truth answer, as shown in Figure \ref{fig:1a}. This error case sometimes accompanied by text degeneration.
    \item The generated response gives the wrong answer to the question but share some insignificant tokens with the ground truth answer, as shown in Figure \ref{fig:1b}, which significantly improves the evaluation score. This incorrect scenario exemplifies the limitation of the evaluation measures.
    \item The model managed to generate the correct key answer while also adding unnecessary information compared to the ground truth, which may lead to the response's meaning being distorted. 
    As shown in Figure \ref{fig:1c}, the model correctly predicted quantity but then added unnecessary tokens afterward, resulting in a low score on both evaluation metrics.
\end{itemize}


% \begin{figure*}[h]
% \centering
%   \begin{tabular}{@{}ccc@{}}
%     \includegraphics[width=0.3\textwidth]{figure/5.3_ex/00000000111.jpg}
%     \includegraphics[width=0.3\textwidth]{example-image-b} &
%     \includegraphics[width=0.3\textwidth]{example-image-b} \\
%   \end{tabular}
%   \caption{This is some figure side by side}
% \end{figure*}


\section{Conclusion and Future Work}
\label{sec:conclusion}

We have presented a novel neural network that successively learns shape sketch and extrusion without any expensive annotations of shape segmentation and labels as the supervision.
%Without the guidance of sketch labels, 
Our approach is able to learn smooth sketches, followed by the differentiable extrusion to reconstruct CAD models that are close to the ground truth. 
We evaluate SECAD-Net using diverse CAD datasets and demonstrate the advantages of our approach by ablation studies and comparing it to the state-of-the-art methods. 
We further demonstrate our method’s applicability in single-image CAD reconstruction. 
Additionally, the CAD shapes generated by our approach can be directly fed into off-the-shelf CAD software for sketch-level or cylinder primitive-level editing. 

% We tested SE-Net on ABC dataset and Fusion 360 dataset. Quantitative results demonstrate that SE-Net can efficiently reconstruct 3D CAD shapes. Qualitative results show that our model can learn fine 2d sketches without any associated ground-truth.


% We propose SE-Net, a network that successively learns shape sketch and extrusion in an unsupervised manner. The CAD shapes generated by the network can be directly sent to off-the-shelf CAD software for sketch-level or cylinder primitive-level editing. SE-Net can be reconstructed to generate smooth sketches and the reconstruction effect is due to the current state-of-the-art, including supervised methods. Additionally, our method is the first to learn sketches from raw shapes without the guidance of sketch labels.

In future work, we plan to extend our approach to learn more CAD-related operations such as \emph{revolve, bevel, and sweep}. %using neural methods. 
Besides, we find that current deep learning models perform poorly on datasets with large differences in shape geometry and structure. %structural and topological variations
Therefore, another promising direction is to explore how to improve the generalization of neural networks and enhance the realism of the generated shapes by learning structural and topological information.

\section*{ACKNOWLEDGMENT}
We would like to thank and give special respect to VLSP organizers for providing the valuable dataset for this challenge. 

% This work was supported by the Multimedia Processing Lab
% (MMLab) at the University of Information Technology, VNUHCM. 
% We give special respect to MMLab for providing resources and discussing lots of valuable ideas that help our study much.

%\begin{figure}[H]
  \centering
  \begin{subfigure}[b]{0.5\linewidth}
    \centering
    \includegraphics[width=0.75\linewidth]{images/GoodCaseSeaShips/002109.jpg}
    \label{fig:badcase1} 
    \vspace{4ex}
  \end{subfigure}%% 
  \begin{subfigure}[b]{0.5\linewidth}
    \centering
    \includegraphics[width=0.75\linewidth]{images/GoodCaseSeaShips/006637.jpg} 
    \label{fig:badcase2} 
    \vspace{4ex}
  \end{subfigure} 
  
  \begin{subfigure}[b]{0.5\linewidth}
    \centering
    \includegraphics[width=0.75\linewidth]{images/GoodCaseSeaShips/005455.jpg}
    \label{fig:badcase3} 
  \end{subfigure}%%
  \begin{subfigure}[b]{0.5\linewidth}
    \centering
    \includegraphics[width=0.75\linewidth]{images/GoodCaseSeaShips/006950.jpg}
    \label{fig:badcase4} 
  \end{subfigure}
  \label{fig7} 
  \caption{Some perfect cases of DLAFS Cascade R-CNN on SeaShips dataset}
  \label{fig:goodcases}
\end{figure}

\begin{figure}[H]
  \centering
  \begin{subfigure}[b]{0.5\linewidth}
    \centering
    \includegraphics[width=0.75\linewidth]{images/BadCaseSeaShips/000654.jpg}
    \label{fig7:a} 
    \caption{SeaShip bad case 1}
    \vspace{4ex}
  \end{subfigure}%% 
  \begin{subfigure}[b]{0.5\linewidth}
    \centering
    \includegraphics[width=0.75\linewidth]{images/BadCaseSeaShips/006410.jpg} 
    \label{fig7:b} 
    \caption{SeaShip bad case 2}
    \vspace{4ex}
  \end{subfigure} 
  \begin{subfigure}[b]{0.5\linewidth}
    \centering
    \includegraphics[width=0.75\linewidth]{images/BadCaseSeaShips/006500.jpg}
    \caption{SeaShip bad case 3}
    \label{fig7:c} 
  \end{subfigure}%%
  \begin{subfigure}[b]{0.5\linewidth}
    \centering
    \includegraphics[width=0.75\linewidth]{images/BadCaseSeaShips/002378.jpg}
    \caption{SeaShip bad case 2}
    \label{fig7:d} 
  \end{subfigure}
  \label{fig7} 
  \caption{Some bad cases of DLAFS Cascade R-CNN on SeaShips dataset. Ignored objects and wrong objects are circled in red and white respectively.}
%   (a) Distant ships are unrecognizable; (c) A distant bulk cargo carrier is mistaken with the other shore; (d) Ships that are overlapped each other will be misrecognized
  \label{fig:badcases}
\end{figure}

\begin{figure}[H]
  \centering
  \begin{subfigure}[b]{0.55\linewidth}
    \centering
    \includegraphics[width=0.85\linewidth]{images/CascadeDODV/104.jpg}
    \label{fig:badcaseDODV1} 
    \vspace{4ex}
  \end{subfigure}%% 
  \begin{subfigure}[b]{0.55\linewidth}
    \centering
    \includegraphics[width=0.8\linewidth]{images/CascadeDODV/CTU_060400067.jpg}
    \label{fig:badcaseDODV2} 
    \vspace{4ex}
  \end{subfigure} 
\vspace{-1.3cm}
  \caption{Some perfect cases of DLAFS Cascade R-CNN on DODV dataset}
  \label{fig:goodcasesDODV}
  \vspace{-1.5cm}
\end{figure}



\begin{figure}[H]
	  \centering
  \begin{subfigure}[b]{0.55\linewidth}
    \centering
    \includegraphics[width=0.75\linewidth]{images/CascadeDODV/Tep_001.jpg}
    \label{fig:badcaseDODV1} 
    %\vspace{4ex}
    %\vspace{-.5cm}
    \caption{DODV bad case 1}
  \end{subfigure}%% 
  \begin{subfigure}[b]{0.5\linewidth}
    \centering
    \includegraphics[width=.91\linewidth]{images/CascadeDODV/CTU_060400071.jpg}
    \label{fig:badcaseDODV2} 
    %\vspace{4ex}
    \vspace{-.5cm}
    \caption{DODV bad case 2}
  \end{subfigure} 
  \caption{Some bad cases of DLAFS Cascade R-CNN on DODV dataset. Wrong predictions are zoned in red color.}
  \label{fig:badcasesDODV}
\end{figure}

\begin{figure*}[H]
     \centering
     \begin{subfigure}[b]{0.45\textwidth}
        \centering
        \includegraphics[width=\textwidth]{images/MS-COCO/coco_example_1-1.pdf}
     \end{subfigure}
     \hspace{6mm}
     \begin{subfigure}[b]{0.45\textwidth}
         \centering
         \includegraphics[width=\textwidth]{images/MS-COCO/coco_example_2-2.pdf}
     \end{subfigure}
     \vspace{6mm}
     
     
     \begin{subfigure}[b]{0.45\textwidth}
        \centering
        \includegraphics[width=\textwidth]{images/MS-COCO/coco_example_3-3.pdf}
     \end{subfigure}
     \hspace{6mm}
     \begin{subfigure}[b]{0.45\textwidth}
         \centering
         \includegraphics[width=\textwidth]{images/MS-COCO/coco_example_4-4.pdf}
     \end{subfigure}
     \vspace{6mm}
     
     
    \begin{subfigure}[b]{0.45\textwidth}
        \centering
        \includegraphics[width=\textwidth]{images/MS-COCO/coco_example_5-5.pdf}
     \end{subfigure}
     \hspace{6mm}
     \begin{subfigure}[b]{0.45\textwidth}
         \centering
         \includegraphics[width=\textwidth]{images/MS-COCO/coco_example_6-6.pdf}
     \end{subfigure}
     \vspace{6mm}
     
     
    \begin{subfigure}[b]{0.45\textwidth}
        \centering
        \includegraphics[width=\textwidth]{images/MS-COCO/coco_example_7-7.pdf}
     \end{subfigure}
     \hspace{6mm}
     \begin{subfigure}[b]{0.45\textwidth}
         \centering
         \includegraphics[width=\textwidth]{images/MS-COCO/coco_example_8-end.pdf}
     \end{subfigure}
     \vspace{6mm}
     
     %chúng tôi trực quan một phần của ảnh
     \caption{Visualization results of DLAFS Cascade R-CNN on MS-COCO }
     \label{fig:ms-coco}
\end{figure*}

%\begin{thebibliography}{99}
	{\small
		\bibliographystyle{IEEEtranS} % sorted IEEE style
		\bibliography{refs.bib} % name your BibTeX data base
	}
%\end{thebibliography}

%\printbibliography

% \hfill {\it Received on July 04, 2021}

% \hfill {\it Accepted on May 24, 2022}
\end{document}