% ****** Start of file apssamp.tex ******
%
%   This file is part of the APS files in the REVTeX 4.2 distribution.
%   Version 4.2a of REVTeX, December 2014
%
%   Copyright (c) 2014 The American Physical Society.
%
%   See the REVTeX 4 README file for restrictions and more information.
%
% TeX'ing this file requires that you have AMS-LaTeX 2.0 installed
% as well as the rest of the prerequisites for REVTeX 4.2
%
% See the REVTeX 4 README file
% It also requires running BibTeX. The commands are as follows:
%
%  1)  latex apssamp.tex
%  2)  bibtex apssamp
%  3)  latex apssamp.tex
%  4)  latex apssamp.tex
%
\documentclass[%
% reprint,
superscriptaddress,
%groupedaddress,
%unsortedaddress,
%runinaddress,
%frontmatterverbose, 
preprint,
%preprintnumbers,
%nofootinbib,
%nobibnotes,
%bibnotes,
 amsmath,amssymb,
 aps,
%pra,
prb,
%rmp,
%prstab,
%prstper,
%floatfix,
]{revtex4-2}

\usepackage{graphicx}% Include figure files
\usepackage{dcolumn}% Align table columns on decimal point
\usepackage{bm}% bold math

\usepackage{hyperref}% add hypertext capabilities
%\usepackage[mathlines]{lineno}% Enable numbering of text and display math
%\linenumbers\relax % Commence numbering lines

%\usepackage[showframe,%Uncomment any one of the following lines to test 
%%scale=0.7, marginratio={1:1, 2:3}, ignoreall,% default settings
%%text={7in,10in},centering,
%%margin=1.5in,
%%total={6.5in,8.75in}, top=1.2in, left=0.9in, includefoot,
%%height=10in,a5paper,hmargin={3cm,0.8in},
%]{geometry}

%========custom package
\usepackage{subfigure}
\usepackage{siunitx}
\usepackage{xcolor}

% \usepackage[UTF8]{ctex}

%========custom package
\newcommand{\revision}[1]{{\color{blue}{#1}}}

 
\begin{document}

\preprint{APS/123-QED}

\title{Chiral-induced angular momentum radiation in single molecular junctions}
% Force line breaks with \\
%\thanks{A footnote to the article title}%

\author{Bing-Zhong Hu}
 \affiliation{School of Physics, Institute for Quantum Science and Engineering and Wuhan National High Magnetic Field Center,\\ Huazhong University of Science and Technology, Wuhan 430074, People's Republic of China}
%\author{Gen Li}
%    \affiliation{School of Physics and Wuhan National High Magnetic Field Center,\\ Huazhong University of Science and Technology, Wuhan 430074, P. R. China}
%\author{Wen-Hao Mao}
%    \affiliation{School of Physics and Wuhan National High Magnetic Field Center,\\ Huazhong University of Science and Technology, Wuhan 430074, P. R. China}
\author{Zu-Quan Zhang}
    \affiliation{Department of Physics, Zhejiang Normal University, Jinhua 321004, People's Republic of China}
\author{Lei-Lei Nian}
    \affiliation{School of Physics and Astronomy, Yunnan University, Kunming 650091, People's Republic of China}
\author{Jing-Tao L\"u}
    \email{jtlu@hust.edu.cn}
    \affiliation{School of Physics, Institute for Quantum Science and Engineering and Wuhan National High Magnetic Field Center,\\ Huazhong University of Science and Technology, Wuhan 430074, People's Republic of China}
    

\date{\today}% It is always \today, today,
             %  but any date may be explicitly specified

\begin{abstract}
%Photon radiation can be stimulated by electrons tunneling inelastically through a junction, especially in scanning tunneling microscope(STM).
%Angular momentum radiation(AMR) is an important physical quantity of such a system. We consider the Hamiltonian of double-stranded DNA(dsDNA) in tight-binding(TB) model, including spin-orbit coupling matrix. By using NEGF-TB calculation, we analyse the electric current, photon radiation and angular momentum radiation of dsDNA. This paper discusses the relation between the angular momentum radiation and chirality and spin-orbit coupling strength. Our results show that by increasing the chain length of dsDNA, AMR can be enhanced linearly. In paricular, different parameters have been studied, such as temperature, onsite energies, inter-chain coupling, voltage bias. In addition, we introduce the concept of vortex electron, define the angular momentum of vortex electron and build up the angular momentum conservation relation between emitted photon and vortex electron.
We study angular momentum radiation from electrically-biased chiral single molecular junctions using the nonequilibrium Green's function method. Using single and double helical chains as examples, we make connections between the ability of a chiral molecule to emit photons with angular momentum to the geometrical factors of the molecule. We point out that the mechanism studied here does not involve the magnetic dipole momentum. Rather, it relies on inelastic transitions between scattering states originated from two electrodes with different chiral properties and chemical potentials. Thus, the required time-reversal symmetry breaking is provided by nonequilibrium electron transport. 
\end{abstract}

\keywords{Angular momentum radiation, }
%Use showkeys class option if keyword
%display desired
\maketitle

%\tableofcontents

%%%%%%%%%%%%%%%%%%%%%%%%%%%%%%%%%%%%%%
\section{\label{sec:Intro} Introduction}
%%%%%%%%%%%%%%%%%%%%%%%%%%%%%%%%%%%%%%
Angular momentum(AM) is a fundamental property of light\cite{poynting_wave_1909,beth_mechanical_1936,coullet_optical_1989,allen_orbital_1992,bliokh_transverse_2015,shen_optical_2019}, whose generation and manipulation is of vital importance for their applications in optoelectronics, quantum information science, and so on\cite{jack_holographic_2009,bozinovic_terabit-scale_2013,andrews_optical_2004,krenn_generation_2014,shen_optical_2019}.
Light with AM can be generated by physical objects with vastly different scales, from as small as synchrotron in particle physics\cite{katoh_helical_2017,katoh_angular_2017,epp_angular_2019,lan_electron_2020} to as large as rotating black hole in astrophysics\cite{tamburini_twisting_2011}. The AM of light can be furthermore used to probe the spin-polarized electronic structure and to study other types of chiral excitations. 

The magneto-electric coupling, depending on both the magnetic and electric dipole transition elements, is a key factor that determines the magnitude or efficiency of many of the above-mentioned processes\cite{polavarapu_chiroptical_2017}. Unfortunately, the magnetic dipole transition is much weaker than the corresponding electric one, resulting in a small magneto-electric coupling. Employing the chiral geometric or electronic structure in electric dipole transitions is a promising approach to avoid the weak magnetic dipole transition, given that the time-reversal symmetry breaking is provided by other mechanisms.  
Recently, it has been shown theoretically that coupling of electron orbital motion with light in current-carrying molecular junctions can lead to  AM radiation (AMR)\cite{zhang_angular_2020,zhang_far-field_2020,zhang_electroluminescence_2021,ridley_quantum_2021,zhang_microscopic_2022}. Electroluminescence from single molecules or localized gap plasmons in a scanning tunneling microscope (STM) has been studied for decades \cite{berndt1993photon,qiu2003vibrationally,dong_generation_2010,kazuma_real-space_2018,doppagne_electrofluorochromism_2018,zhang2017sub,imada2016real}. Different quantum statistical properties of emitted light has been characterized using STM setup\cite{zhang2017electrically,merino2018bimodal,leon2019photon,nian_fano_2018,nian_plasmon_2022}. Thus, it is also an ideal experimental candidate to study AMR at the single molecular scale.

% chiral induced effect
On the other hand, molecular electronics and optoelectronics using chiral molecules such as DNA have been the focus of recent intense research\cite{zhang_recent_2020}. In the phenomenon of chiral-induced spin selectivity (CISS)\cite{gohler_spin_2011,naaman_chiral-induced_2012,dalum_theory_2019,naaman_chiral_2020,liu_chirality-driven_2021,evers_theory_2022,naskar_common_2022}, spin-polarized electrons can be generated from chiral molecular structure driven by electrical or optical stimuli. Spin-orbit interaction is argued to play an important role, although the exact mechanism is still under debate. In light emitting diode, large chiroptical effects are observed from chiral molecular structures\cite{greenfield_pathways_2021,liu_chirality-driven_2021}. It origin is attributed to either the magneto-electric coupling (natural optical activity), or structural chirality. Notably, a recent work proposed an electronic mechanism employing the topological electronic structure for circular polarized light emission under electrical  current flow\cite{wan_anomalous_2023}. 
The common trends of CISS and optical dichroism in helical structures is also studied very recently\cite{naskar_common_2022}. 


In this work, we study theoretically AMR from junctions of model helical chains using nonequilibrium Green's function (NEGF) method\cite{haug2008quantum,zhang_angular_2020}. We analyze in details how the molecular geometry, electronic structure and molecule-electrode coupling influence the spectrum and efficiency of AMR. Suitable conditions for enhancing the AMR are proposed based on the numerical calculation.




%%%%%%%%%%%%%%%%%%%%%%%%%%%%%%%%%%%%%%
%\section{\label{sec:Intro} Introduction}
%%%%%%%%%%%%%%%%%%%%%%%%%%%%%%%%%%%%%%
% brief history of angular momentum radiation
%Angular momentum(AM) was proposed as a fundamental property of light by Poynting in 1909\cite{poynting_wave_1909}, and experimentally observed by Beth\cite{beth_mechanical_1936}. Bateman discussed the radiation of angular momentum according to classical theory\cite{bateman_radiation_1926}. 
%In 1989, the concept of optical vortices was first proposed as photons carrying OAM \cite{coullet_optical_1989}. After that, AM was seperated into spin angular momentum(SAM) and orbital angular momentum (OAM) in expermiment\cite{allen_orbital_1992}, opening a new path for understanding AM of light. 
%Since then, light with angular momentum has grown into a vast regime with various topics, \cite{bliokh_transverse_2015,shen_optical_2019} reviewed the progress for light with angular momentum.

 


% application of light with angular momentum

%Emission of light with angular momentum has promising applications in optoelectronics and quantum information\cite{jack_holographic_2009,bozinovic_terabit-scale_2013,andrews_optical_2004,krenn_generation_2014,shen_optical_2019}. OAM was uesd in holographic ghost imaging to violate Bell inequality\cite{jack_holographic_2009}.  Zeilinger's group realized the quantum entanglement of OAM states\cite{mair_entanglement_2001}.  Photons carry a lot of information since one photon can carry any integer number OAM $l \hbar$, which can be used in optical and quantum communication\cite{bozinovic_terabit-scale_2013,chen_li-xiang_research_2015}.  Light with angular momentum can be emitted by various physical objects, from as small as synchrotron in particle physics\cite{katoh_helical_2017,katoh_angular_2017,epp_angular_2019,lan_electron_2020} to as large as rotating black hole in astrophysics\cite{tamburini_twisting_2011}. While circular polarized photons can be generated by breaking time-reversal symmetry of the system via applying a static magnetic field, different approaches have been proposed, such as semiconductor planar chiral nanostructures inducing imbalance of circular polarization in vacuum field\cite{konishi_circularly_2011}, angular momentum radiation from topological structures with intrinsic time reversal breaking\cite{maghrebi_fluctuation-induced_2019}. In 2D material, electrically controlled polarized light emission was realized experimentally by utilizing the valley degree of freedom of the material\cite{zhang_electrically_2014,onga_high_2016,gong_nanoscale_2018}. In optics, spiral antenna have been proposed\cite{rui_beaming_2012} and utilized\cite{rui_demonstration_2013} to create photons with angular momentum. Recently, coupling of the orbital degree of freedom of electrons with light in current-carrying molecular junctions leading to angular momentum radiation has been studied theoretically\cite{zhang_angular_2020,zhang_far-field_2020,zhang_electroluminescence_2021,zhang_energy_2021,zhang_microscopic_2022}.
%Find ways to produce and manipulate light carrying AM is an interesting topic.  

% Electroluminescence in molecular electronics
%In molecular elctronics, scanning tuneling microscope(STM) was invented in 1980s\cite{binnig_tunneling_1982}. After that, IBM positioned single atom with STM\cite{eigler_positioning_1990}. Then, tip-enhanced Raman spectroscopy(TERS) was proposed by Wessel\cite{wessel_surface-enhanced_1985} , which opened the opportunity to observe single macro-molecules. From another perspective, that means we can detect photons emitted from molecules directly. Various works was reviewed in \cite{rossel_luminescence_2010}. 


% chiral induced effect
%Chiral molecules such as DNA gains broad interest due to the phenomenon of electron transport property of chiral-induced spin selectivity\cite{gohler_spin_2011,naaman_chiral-induced_2012,dalum_theory_2019,liu_chirality-driven_2021,evers_theory_2022,naskar_common_2022}. In light emitting field, the magnetic transition dipoles in organic chiral emissive materials are widely studied and are considered playing a key role in circularly polarized organic light-emitting diode\cite{greenfield_pathways_2021,liu_chirality-driven_2021}. In contrast, a recent work proposed that the topological electric property played an important role for the net circular polarization in chiral materials carrying electric current, where the polarization was enhanced due to opposite handedness in forward and backward emission with finite angular momentum transfers from electrons to photons in the optical transition process\cite{wan_anomalous_2022}.  
%However, the detailed mechanism of angular momentum transfer in chiral molecules needs to be further discussed. There is still much to explore about the effect of chiral induced angular momentum radiation.  

%In this work, we focus on angular momentum radiation from a molecular junction composed of a helical chain theoretically. We provide an intuitive picture of angular momentum transfer during inelastic transition by introducing AM transition dipole moment $X$. The AM transition dipole moment can be obtained directly from NEGF calculation. By exploring the parameter space of AM transition dipole moment, this method offers a simple way to determing AM transfer in any transition from one energy level to another. It may be used to understand selection rules and other effects during light emission in almost all molecule junction systems.

%This paper is organized as follows.
%Section \ref{sec:thoery} gives the abstract model. The helical chains are described using a tight-binding model. NEGF method is introduced. And the definition of AM transition dipole moment, angular momentum radiation and circular/vortex transmission is then derived.
%Section \ref{sec:results} starts with a minimized helical chain model with only 3 sites. Then, we extend it into $N=10$ helical chain, the single-stranded helical chain serves as a simple structure to analyze the role of the helical structure on the angular momentum radiation under a bias voltage. After that, we study radiation from the double-stranded helical chain model, the role of the interchain hopping, second nearest-neighbor hopping and spin-orbital interaction on the angular momentum radiation are considered.  Finally, we consider an actual double-stranded DNA models with specific parameter. Suitable conditions for enhancing the circularly polarized emission are proposed based on the numerical calculation. 



%\section{\label{sec:intro2}Introduction2}
%Chiral molecules such as DNA gains broad interest due to the phenomenon of electron transport property of chiral-induced spin selectivity\cite{liu_chirality-driven_2021}. In light emitting field, the magnetic transition dipoles in organic chiral emissive materials are widely studied and are considered playing a key role in circularly polarized organic light-emitting diode\cite{greenfield_pathways_2021,frederic_designs_2021}. In contrast, a recent work proposed that the topological electric property played an important role for the net circular polarization in chiral materials carrying electric current, where the polarization was enhanced due to opposite handedness in forward and backward emission due to finite angular momentum transfers from electrons to photons in the optical transition process\cite{wan_anomalous_2022}.








%CISS: experiments[R. Naaman, Science, 283, 814 (1999); 331, 894 (2011) \cite{gohler_spin_2011}], review [F. Evers, Adv. Mat. (2022) \cite{evers_theory_2022}]
%
%Experiment: chiral valley photon \cite{gong_nanoscale_2018,onga_high_2016,zhang_electrically_2014}
%
%Experiment: dipole-coupled plasmonic spiral antenna \cite{rui_beaming_2012}



\begin{figure}[b]
    \includegraphics[height=0.5\textwidth]{./Figure1.pdf}
    % \includegraphics[height=0.3\textwidth]{figures/system_diagram.pdf}
    % \includegraphics[height=0.3\textwidth]{figures/geometry.pdf}


\caption{(a) Schematic diagram of angular momentum radiation from voltage-biased chiral molecule. (b) A double helical chain structure with radius $R$, pitch $h$, helix angle $\theta$, and arc length $l_a$.}
\label{fig:epsart}
\end{figure}

\section{\label{sec:thoery}Model and theory}

\subsection{Hamiltonian}
We use a tight binding model to write the Hamiltonian of the molecule as \cite{zhang_angular_2020,zhang_far-field_2020}
\begin{align}
    H_{\rm mol} = \sum_{ij} H_{ij}c_i^\dagger c_j e^{i\theta_{ij}} .
\end{align}
Molecular coupling to the radiation fields is taken into account by the Peierls substitution\cite{graf_electromagnetic_1995} with the phase factor $\theta_{ij}$ between sites $i$ and $j$
\begin{align}
\theta_{ij}=\frac{e}{\hbar} \int_{r_j}^{r_i} \boldsymbol{A} \cdot d \boldsymbol{l}. 
\end{align}
Here $e$ is the elementary charge, $\hbar$ is the reduced Planck constant, and $\boldsymbol{A}$ is the vector potential of the electromagnetic field, 
which is described by the following Hamiltonian
\begin{align}
    H_{\mathrm{rad}}=\frac{1}{2} \int d^3 \boldsymbol{r}\left(\varepsilon_0 \boldsymbol{E}_{\perp}^2+\frac{1}{\mu_0} \boldsymbol{B}^2\right),
\end{align}
where $\varepsilon_0, \mu_0$ the vacuum permittivity and permeability, respectively.  $\boldsymbol{E}_\perp$ and $\boldsymbol{B}$ are electric field and magnetic field in the transverse gauge $\nabla \cdot \boldsymbol{A}=0$. They are written in terms of $\boldsymbol{A}$ as  $\boldsymbol{E}_{\perp}=-\partial_t \boldsymbol{A}$, $\boldsymbol{B}=\nabla \times \boldsymbol{A}$. 
%
%
%
%Angular momentum radiation stimulated by voltage bias in the dsDNA molecule can be described by Haldane model with electron-photon interaction. Applying Peierls' substitution\cite{graf_electromagnetic_1995}, we can write system Hamiltonian\cite{zhang_angular_2020,zhang_far-field_2020},
%$$
%\begin{aligned}
%H =  H_{rad} +  \sum_{ij} H_{ij} c_{i}^{\dagger}c_{j} e^{i \theta_{ij}}.
%\end{aligned}
%$$
%The first term is radiation field $H_{\mathrm{rad}}=\frac{1}{2} \int d^3 \boldsymbol{r}\left(\varepsilon_0 \boldsymbol{E}_{\perp}^2+\frac{1}{\mu_0} \boldsymbol{B}^2\right)$, where $\varepsilon_0, \mu_0$ the vacuum permittivity and permeability. $\boldsymbol{E},\boldsymbol{B}$ is electric field and magnetic field in transverse gauge $\nabla \cdot A=0$, where $\boldsymbol{A}$ is magnetic vector potential. The transverse electric field and magnetic field can express with $\boldsymbol{E}_{\perp}=-\partial_t \boldsymbol{A},B=\nabla \boldsymbol{A}$. 
%
% And the second term $H_{ep} = \sum_{ij} H_{ij} c_{i}^{\dagger}c_{j} e^{i \theta_{ij}}$ represents molecule consisting electrons and its coupling with radiation field, where $H_{ij}$ is hopping parameter between different site $\theta_{ij}=\frac{e}{\hbar} \int_{r_j}^{r_i} \boldsymbol{A} \cdot d \boldsymbol{l}$ is phase factor due to coupling to the radiation field, $e$ is elementary charge, $\hbar$ is reduced Planck constant.
%
By expanding the exponential part to the first order in $\boldsymbol{A}$, we can divide the molecule Hamiltonian into two terms
\[ H_{\rm mol} = H_0 + H_{\rm int}. \]
The first term $H_0$ is the non-interacting part
\begin{align}
    H_0 = \sum_{i,j}H_{ij}c_i^\dagger c_j,
\end{align}
and $H_{\rm int}$ is the interacting part
\[H_{int} \approx \sum_{i j} \sum_k \sum_{\mu=x, y, z} M_{i j}^{k \mu} c_i^{\dagger} c_j A_\mu\left(\boldsymbol{r}_k\right), \]
where 
\[M_{i j}^{k \mu}= \frac{i e}{2 \hbar} H_{i j}\left(\boldsymbol{r}_i-\boldsymbol{r}_j\right)_\mu\left(\delta_{k i}+\delta_{k j}\right)\]
is electron-photon coupling matrix.
%Non-interacting Hamiltonian $H_0=H_{mol}+H_{ec}$. $H_{mol}$ is molecule Hamiltonian while $H_{ec}$ is electrode coupling Hamiltonian.

As an example, 
the noninteracting molecular Hamiltonian for a double helical chain simulating double-stranded DNA is
%Molecule Hamiltonian can be expressed by 4 part: onsite, intersite interaction on one chain, interchain interaction on corresponding site representing base-pairs coupling and spin-orbit coupling,
%\begin{align}
%    H_{0, {\rm ds}}=& \sum_{j=1}^{2}\left\{\sum_{n=1}^{N} \varepsilon_{j} c_{j n}^{\dagger} c_{j n}+\sum_{n=1}^{N-1}\left[i t^{\rm SO} c_{j n}^{\dagger}\left(\sigma_{n}^{(j)}+\sigma_{n+1}^{(j)}\right) c_{j n+1}\right.\right.\nonumber\\
%    &\left.\left.+t_{j} c_{j n}^{\dagger} c_{j n+1}+\text { H.c. }\right]\right\}+\sum_{n=1}^{N}\left[t^{IC} c_{1 n}^{\dagger} c_{2 n}+\text { h.c. }\right]
%\end{align}
\begin{align}
    H_{0, {\rm ds}}=& \sum_{j=1}^{2}\left[\sum_{n=1}^{N} \varepsilon_{jn} c_{j n}^{\dagger} c_{j n}+\sum_{n,m}\left(
    t_{j} c_{j n}^{\dagger} c_{j m}+\text { h.c. }\right)\right]+\sum_{n=1}^{N}\left(t_n^{IC} c_{1 n}^{\dagger} c_{2 n}+\text { h.c. }\right).
\end{align}
Here, $j$ is the chain index, and $m$, $n$ are site indices, $\varepsilon_{jn}$ is onsite energy of site $n$ in chain $j$, $t_{j,nm}$ is inter-site hopping within chain $j$, $t_n^{\rm IC}$ is inter-chain hopping, $c_{jn}$ ($c^{\dagger}_{jn}$) is annihilation (creation) operator of electron at site $n$ in chain $j$. This Hamiltonian has been used in previous works to study spin-dependent electron transport in ds-DNA \cite{guo_spin-selective_2012,guo_contact_2014-1}. 
For single helical chain, we can simply set $j=1$ and drop the inter-chain hopping term. Note that we have ignored the spin-orbit coupling in this work. We have checked that it does not bring any new physical effect, contrary to the CISS effect. 



To consider electron transport process, we take into account the molecule-electrode coupling in the wide band limit. Two characteristic quantity $\Gamma_L$ and $\Gamma_R$ are used to model its coupling to the left and right electrodes, respectively. We make further simplification to write $\Gamma_\alpha$ as a diagonal matrix, whose non-zero diagonal elements are of the same magnitude $\gamma_\alpha$. The positions of the non-zero diagonals are determined by the way how the molecule couples to the electrode, i.e., all the degrees of freedom that couple directly to electrode has a non-zero element, and the rest elements are zero.




% \begin{figure}[b]
% \includegraphics[width=2in]{figures/ladder}% Here is how to import EPS art
% \caption{\label{fig:ladder} Schematic view of ladder model. The geometry structure of DNA is flatten to 2D. Center(C) represent the molecule part, while Left(L)/Right(R) are electrode parts. }
% \end{figure}

% Detailed total Hamiltonian can be described by ladder model in Fig. \ref{fig:ladder}.


% \subsection{Electrode setup}




 



%\subsection{Hamiltonian}
%Angular momentum radiation stimulated by voltage bias in the dsDNA molecule can be described by Haldane model with electron-photon interaction. Applying Peierls' substitution\cite{graf_electromagnetic_1995}, we can write system Hamiltonian\cite{zhang_angular_2020,zhang_far-field_2020},
%\begin{equation}
%\begin{aligned}
%H =  H_0 + H_{rad} + H_{int}
%\end{aligned}  
%\end{equation}
%
%The first term is non-interacting electron part,
%\begin{equation}
%H_0 = \sum_{ij} t_{ij} c_{i}^{\dagger}c_{j}
%\end{equation}
%where $t_{ij}$ is hopping parameter between different site, $a^{(\dagger)}_{i}$ is annihilation/creation operator of electron in electrodes.
%
%The second term is radiation field,
%\begin{equation}
%H_{\mathrm{rad}}=\frac{1}{2} \int d^3 \boldsymbol{r}\left(\varepsilon_0 \boldsymbol{E}_{\perp}^2+\frac{1}{\mu_0} \boldsymbol{B}^2\right)
%\end{equation}
%where $\varepsilon_0, \mu_0$ the vacuum permittivity and permeability. $\boldsymbol{E},\boldsymbol{B}$ is electric field and magnetic field in transverse gauge $\nabla \cdot A=0$, where $\boldsymbol{A}$ is magnetic vector potential. The transverse electric field and magnetic field can express with $\boldsymbol{E}_{\perp}=-\partial_t \boldsymbol{A},B=\nabla \boldsymbol{A}$. 
%
%And the third term represents molecule consisting electrons and its coupling with radiation field, by expanding the exponential part of $e^{i \theta_{i j}}$ to the first order of $\boldsymbol{A}$, we can get pertubation expansion, 
%
%\begin{equation}
%H_{int} \approx \sum_{i j} \sum_k \sum_{\mu=x, y, z} M_{i j}^{k \mu} c_i^{\dagger} c_j A_\mu\left(\boldsymbol{r}_k\right) 
%\end{equation}
%where $\theta_{ij}=\frac{e}{\hbar} \int_{r_j}^{r_i} \boldsymbol{A} \cdot d \boldsymbol{l}$ is phase factor due to coupling to the radiation field, $e$ is elementary charge, $\hbar$ is reduced Planck constant, $M_{i j}^{k \mu}=i \frac{e}{2 \hbar} t_{i j}\left(\boldsymbol{r}_i-\boldsymbol{r}_j\right)_\mu\left(\delta_{k i}+\delta_{k j}\right)$ is electron-photon coupling matrix.
%
%\subsection{Molecule setup}
%We consider different molecules, leading different specific Hamiltonian $H_0$
%
%For SSDNA, we consider nearest-neighbor(NN) and second nearest-neighbor(SNN) hopping, corresponding parameter $t_1$ and $t_2$. Hamiltonian
%\begin{equation}
%\begin{aligned}
%H_{0}= \sum_{j=1}^2\sum_{i=1}^{N-j} t_{j} c_{i+j}^{\dagger} c_{i}
%\end{aligned}  
%\end{equation}
%
%For DSDNA, Hamiltonian can be expressed by 4 part: onsite, intersite interaction on one chain, interchain interaction on corresponding site representing base-pairs coupling and spin-orbit coupling,
%\begin{equation}
%\begin{aligned}
%    H_{0}=& \sum_{j=1}^{2}\left\{\sum_{n=1}^{N} \varepsilon_{j} c_{j n}^{\dagger} c_{j n}+\sum_{n=1}^{N-1}\left[i t^{SO} c_{j n}^{\dagger}\left(\sigma_{n}^{(j)}+\sigma_{n+1}^{(j)}\right) c_{j n+1}\right.\right.\\
%&\left.\left.+t_{j} c_{j n}^{\dagger} c_{j n+1}+\text { H.c. }\right]\right\}+\sum_{n=1}^{N}\left[t^{IC} c_{1 n}^{\dagger} c_{2 n}+\text { H.c. }\right]
%\end{aligned}
%\end{equation}
%
%Here, $j$ is chain index, $n$ is site index (note we have 2 indices here, for the sake of simplicity, we merge indices into one for later discussion or calculation if unmentioned),  $\epsilon_i$ is onsite energy, $t_i$ is intersite hopping, $t^{IC}$ is interchain hopping and spin-orbit coupling constant $t^{SO}$, $c^{(\dagger)}_{jn}$ is annihilation/creation operator of electron in corresponding site. Several researchs about spin-selective electron transport used this model \cite{guo_spin-selective_2012,guo_contact_2014-1}. 
%
%% Electrode coupling Hamiltonian, containing electrodes and couplings between molecule and electrodes,
%% \begin{equation}
%% H_{ec}=\sum_{k, \beta(\beta=L, R)}\left[\varepsilon_{\beta k} a_{\beta k}^{\dagger} a_{\beta k}+t_\beta a_{\beta k}^{\dagger}\left(c_{1 n_\beta}+c_{2 n_\beta}\right)+\text { H.c. }\right]
%% \end{equation}
%% Here, $\beta$ is electrode L or R, $k$ is electron site index in electrodes, $\varepsilon_{\beta k}$ is onsite energy in electrodes, $t_\beta$ is couplings between electrode, $n_{\beta}$ is site index in molecule, contacting with electrodes $\beta$, $a^{(\dagger)}_{jn}$ is annihilation/creation operator of electron in electrodes.
%
%
%
%
%
%
%% \begin{figure}[b]
%% \includegraphics[width=2in]{figures/ladder}% Here is how to import EPS art
%% \caption{\label{fig:ladder} Schematic view of ladder model. The geometry structure of DNA is flatten to 2D. Center(C) represent the molecule part, while Left(L)/Right(R) are electrode parts. }
%% \end{figure}
%
%% Detailed total Hamiltonian can be described by ladder model in Fig. \ref{fig:ladder}.
%
%
%% \subsection{Electrode setup}
%
%
\subsection{Light and angular momentum radiation}
By applying the non-equilibrium Green's function (NEGF) method, we can calculate various physical quantities such as electrical current, light and angular momentum radiation, and so on. 
%In this work, we mainly focus on the AMR.
%
The radiated power $P$, angular momentum flux $J_\gamma$ and the photon flux $J_N$ can then be written in terms of the self-energy\cite{zhang_angular_2020,wang2022transport}
\begin{equation}
P = \frac{d W}{d t} =-\sum_{\mu}\int_{0}^{\infty} \frac{d \omega}{2 \pi} \frac{\hbar \omega^{2}}{3 \pi \varepsilon_{0} c^{3}} \operatorname{Im}\left[\Pi_{\mu \mu}^{\mathrm{tot},<}(\omega)\right],
\end{equation}
\begin{equation}
J_\gamma = \frac{d L_{\gamma}}{d t}=\int_{0}^{\infty} \frac{d \omega}{2 \pi} \frac{\hbar \omega}{3 \pi \varepsilon_{0} c^{3}} \epsilon_{\gamma \mu \nu} \operatorname{Re}\left[\Pi_{\mu \nu}^{\mathrm{tot},<}(\omega)\right],
\end{equation}
\begin{equation}
J_N = \frac{d N}{d t} =-\sum_{\mu}\int_{0}^{\infty} \frac{d \omega}{2 \pi} \frac{ \omega}{3 \pi \varepsilon_{0} c^{3}} \operatorname{Im}\left[\Pi_{\mu \mu}^{\mathrm{tot},<}(\omega)\right].
\label{equ:JN}
\end{equation}
Here, $\mu$, $\nu$ and $\gamma$ are indices for the Cartesian coordinate, $\varepsilon_0$, $c$ and $\hbar$ are the vacuum permittivity, the speed of light in the vacuum, and the reduced Planck constant, respectively. The superscript `tot' means summation over all the sites in the system
$\Pi^{\rm tot, <}_{\mu\nu}(\omega) = \sum_{i,j}\Pi^<_{\mu\nu}(\boldsymbol{r}_i, \boldsymbol{r}_j,\omega)$,
%
where $\Pi^<_{\mu\nu}$ is photon self-energy due to interaction with electrons. 

% Under the random phase approximation,
% \begin{equation}
% \Pi_{\mu \nu}^{<}\left(\boldsymbol{r}_i, \boldsymbol{r}_j ; \omega\right)  =
% \label{equ:Pi}
% \end{equation}
Under the random phase approximation, the photon self-energy is written as 
% By summing over all $i,j$ indices in Eq.~(\ref{equ:Pi}), the total photon self-energy is expressed as 
\begin{align}
    \Pi^{<}_{\mu\nu}(\boldsymbol{r}_i,\boldsymbol{r}_j,\omega) =-i \hbar \int_{-\infty}^{\infty} \frac{d E}{2 \pi \hbar} \operatorname{Tr}\left[M^{i \mu} G^{<}(E) M^{j \nu} G^{>}(E-\hbar \omega)\right]
\end{align}
%
% \begin{align}
% \Pi_{\mu \nu}^{\rm tot,<} (\omega) &= -i \hbar \int_{-\infty}^{\infty} \frac{d E}{2 \pi \hbar} \sum_{i,j} \operatorname{Tr}\left[M^{i \mu} G^{<}(E) M^{j \nu} G^{>}(E-\hbar \omega)\right] \\
% &= i e^2 \hbar \int_{-\infty}^{+\infty} \frac{d E}{2 \pi \hbar} \operatorname{Tr}\left[V^{ \mu} G^{<}(E) V^{ \nu} G^{>}(E^-)\right].
% \end{align}
%
Here, ${\rm Tr}[...]$ is trace over all electronic degrees of freedom. The greater/lesser Green's function of non-interacting electrons is given by $G^{>/<}(E) = G^r(E) \Sigma^{>/<}(E) G^a(E)$, with $\Sigma^{>/<}(E)=\Sigma^{>/<}_L(E)+\Sigma^{>/<}_R(E)$ the greater/lesser self-energy due to electron coupling to left and right electrodes, and $E^-=E-\hbar\omega$. In the wide band limit, the self-energy is energy independent $\Sigma_\alpha^r= -i\Gamma_\alpha/2$. 
%
%From Eqs.~(\ref{eq:x2}-\ref{eq:pi2}), the physical meaning of the self-energy becomes 
%

%
For the ease of analysis, 
we define 
\begin{align}
    X^{\alpha \beta}_{\mu\nu}(E, E^-)&=\operatorname{Tr}\left[V^\mu A^\alpha(E) V^\nu A^\beta(E^-)\right]\\
    &=2\pi \sum_{m,n}  \langle \psi_{\alpha,m} (E) |V^\nu|\psi_{\beta,n} (E^-)\rangle \langle \psi_{\beta,n} (E^-)|V^\mu|\psi_{\alpha,m} (E) \rangle.
    \label{eq:x2}
\end{align}
where  $A^\alpha(E) = G^r(E) \Gamma^\alpha(E) G^a(E)$ is spectral function contributed by scattering states from electrode $\alpha$, $V^\mu$ is electron velocity matrix $V^\mu = \frac{1}{ie} \sum_{k} M^{k\mu} $, with $V^{\mu}_{ij} = H_{ij} (r^{\mu}_i-r^{\mu}_j)/\hbar$. 
In the second equation, we have written it in terms of velocity matrix elements between scattering states. This form highlights the origin of AMR as inelastic transitions from scattering states of one electrode to those of the other.  We can show that the following relations hold: (i)
    $X^{\alpha\beta}_{\mu\nu}(E,E^-) = X^{\beta\alpha}_{\nu\mu}(E^-,E)$, (ii)
$X^{\alpha\beta}_{\mu\nu} (E,E^-)=[X^{\alpha\beta}_{\nu\mu}(E,E^-)]^*$.
%, with 
%It can also be written in terms of velocity matrix elements between scattering states
%\begin{equation}
%    X^{\alpha \beta}_{\mu\nu}(E, E^-) =2\pi \sum_{m,n}  \langle \psi_{\alpha,m} (E) |V^\nu|\psi_{\beta,n} (E^-)\rangle \langle \psi_{\beta,n} (E^-)|V^\mu|\psi_{\alpha,m} (E) \rangle.
%    \label{eq:x2}
%\end{equation}
%$m$ and $n$ the corresponding labels for the scattering states. 
%
The photon less self-energy in the zero temperature limit can then be written as
\begin{align}
   % \Pi_{\mu \nu}^{\rm tot,<}\left( \omega\right) &= -i e^2 \hbar  \sum_{\alpha,\beta=L,R} \int_{-\infty}^{+\infty} \frac{d E}{2 \pi \hbar} X^{\alpha\beta}_{\mu\nu}(E,E^-) f^\alpha(E) (1-f^\beta(E^-))\nonumber\\
     \Pi_{\mu \nu}^{\rm tot,<}\left( \omega\right) &= -ie^2\hbar \sum_{\alpha,\beta=L,R}\int_{\mu_\beta+\hbar\omega}^{\mu_\alpha} \frac{dE}{2\pi\hbar} \Theta(\mu_\alpha-\mu_\beta-\hbar\omega) \nonumber\\
    &\times X_{\mu\nu}^{\alpha\beta}(E,E^-)(f^\alpha(E)-f^\beta(E^-))
    \label{eq:pi2}
\end{align}
Here, $f^\alpha(E)=1/({\rm exp}((E-\mu_\alpha)/k_BT)+1)$ is the Fermi-Dirac distribution function of electrode $\alpha$, and $\Theta(E)$ is the Heaviside step function, which gives the energy range where the inelastic transitions can take place. 
%$P,J_\gamma,J_N$ are radiation intensity, angular momentum flux, photon flux. We focus on angular momentum along z direction, we define angular momentum flux in z direction as $J_A = J_z$.
%
%
%--------------------------
%By applying Non-equilibrium Green's function(NEGF) method, we can calculate various physical quantities such as density of states(DOS), transmission, electric current, photon radiation itensity, photon radiation flux, angular momentum flux, and so on. In this article, we mainly focus on angular momentum radiation flux.
%
%Using random phase approximation, we calculate interacting photon self energy from non-interacting electron green function
%
%\begin{equation}
%\begin{aligned}
%\Pi_{\mu \nu}^{<}\left( \omega\right) = -i e^2 \hbar \int_{-\infty}^{+\infty} \frac{d E}{2 \pi \hbar} \operatorname{Tr}\left[V^{ \mu} g^{<}(E) V^{ \nu} g^{>}(E-\hbar \omega)\right]
%\end{aligned}
%\end{equation}
%
%with  $Tr[...]$ over all electron degrees of freedom, $g^{>/<}$ the greater/lesser Green's function of non-interacting electrons $g^{>/<} = g^r \Sigma^{>/<} g^a$,  $\Sigma^{>/<}$ is greater/lesser self-energy due to electron coupling with all leads. Here, we take wide-band limit, surface self energy is energy indepent constant $\Sigma^r= -\frac{i}{2}\Gamma$. $V_\mu$ is electron volecity matrix $V_\mu = \frac{1}{e} \sum_{k} M^{k\mu} $, for component $V^{\mu}_{ij} = -i t_{ij} (r^{\mu}_i-r^{\mu}_j)/\hbar$. 
%
%For better understanding, we define trace term \textcolor{red}{Currently, called cross velocity density} as
%\begin{equation}
%X^{\alpha \beta}_{\mu\nu}(E, E^-)=\operatorname{Tr}\left[V^\mu A^\alpha(E) V^\nu A^\beta(E^-)\right]
%\end{equation}
%where  $A^\alpha = g^r \Gamma^\alpha g^a$ is spectral function and $\alpha,\beta$ are leads $L,R$, $E$ and $E^-$ represent electron and hole energy. $X^{\alpha\beta}_{\mu\nu} = (X^{\alpha\beta}_{\nu\mu})^*$ from Equ. \ref{equ:X_conj}.  Previous work \cite{zhang_angular_2020,...} analysed $J_A$ spectrum for emission energy. Compare with $J_A$ spectrum,  $X$ contains more information, we can obtain the electron and hole energy before emission than just emission energy.
%
%
%
%
%
%
%Rewrite photon self energy,
%\begin{equation}
%\Pi_{\mu \nu}^{<}\left( \omega\right)=-i e^2 \hbar  \sum_{\alpha,\beta=L,R} \int_{-\infty}^{+\infty} \frac{d E}{2 \pi \hbar} X^{\alpha\beta}_{\mu\nu}(E,E-
%\hbar\omega) f^\alpha(E) (1-f^\beta(E-\hbar\omega))
%\label{equ:photon_self2}
%\end{equation}
%it can be shown that integrand is seperated into two part, $X$ and  $f^\alpha(E)(1-f^\beta(E-\hbar\omega))$, where $f$ is fermi distribution $f^\alpha=1/(exp((E-\mu_{\alpha})/k_B T^\alpha)+1)$, $k_B$ is Boltzmann constant.
%
%If $\mu_L>\mu_R$ and $T^{L/R}=0$ , the only working $f$ part is $f^L(E)(1-f^R(E-\hbar\omega))$. Equ.\ref{equ:photon_self2} reduce to 
%\begin{equation}
%\Pi_{\mu \nu}^{<}\left( \omega\right)=-i e^2 \hbar  \int_{\mu_R}^{\mu_L} \frac{d E}{2 \pi \hbar} X^{LR}_{\mu\nu}(E,E-
%\hbar\omega)
%\label{equ:photon_self3}
%\end{equation}
%
%In the far-field region, 
%
%\begin{equation}
%P = \frac{d W}{d t} =-\sum_{\mu}\int_{0}^{\infty} \frac{d \omega}{2 \pi} \frac{\hbar \omega^{2}}{3 \pi \varepsilon_{0} c^{3}} \operatorname{Im}\left[\Pi_{\mu \mu}^{<}(\omega)\right]
%\end{equation}
%\begin{equation}
%J_\gamma = \frac{d L_{\gamma}}{d t}=\int_{0}^{\infty} \frac{d \omega}{2 \pi} \frac{\hbar \omega}{3 \pi \varepsilon_{0} c^{3}} \epsilon_{\gamma \mu \nu} \operatorname{Re}\left[\Pi_{\mu \nu}^{<}(\omega)\right]
%\end{equation}
%\begin{equation}
%J_N = \frac{d N}{d t} =-\sum_{\mu}\int_{0}^{\infty} \frac{d \omega}{2 \pi} \frac{ \omega}{3 \pi \varepsilon_{0} c^{3}} \operatorname{Im}\left[\Pi_{\mu \mu}^{<}(\omega)\right]
%\label{equ:photon_amr}
%\end{equation}
%
%$P,J_\gamma,J_N$ are radiation intensity, angular momentum flux, photon flux. 
We focus on AMR along $z$ direction in the case of $eV= \mu_L - \mu_R >0 $. Writing $J_A = J_z$, we get
\begin{align}
J_A &= \int_{0}^{\infty} d \omega J_A(\omega) \nonumber\\ %=\int_{0}^{\infty} \frac{d \omega}{2 \pi} \frac{\hbar \omega}{3 \pi \varepsilon_{0} c^{3}}  \operatorname{Re}\left[\Pi_{xy}^{<}(\omega)-\Pi_{yx}^{<}(\omega)\right]\\
&=\frac{ 4 \alpha}{3 \pi c^2}  \int_{0}^{\infty} \frac{ d \hbar\omega}{2 \pi} \omega\Theta(eV-\hbar\omega)  \int_{\mu_R+\hbar\omega}^{\mu_L} \frac{d E}{2 \pi } j_A(E,E^-)
\label{equ:JA}
\end{align}
%% \textcolor{red}{Should prove $Re \Pi_{xy}=-Re\Pi_{yx}$ here}
%
%
%
%Combine Equ. (\ref{equ:X_conj}),(\ref{equ:photon_self3}) and (\ref{equ:JA}), $J_A$ can be expressed as a double integral of $\operatorname{Im} X^{LR}_{xy}$,
%\begin{equation}
%    J_A =\frac{ 8 \alpha}{3 c^2}  \int_{0}^{\infty} \frac{ d \hbar\omega}{2 \pi} \hbar\omega  \int_{\mu_R}^{\mu_L} \frac{d E}{2 \pi } \operatorname{Im} X_{xy}^{L R}(E, E-\hbar \omega)
%\end{equation}
with fine-structure constant $\alpha = e^2/(4\pi \varepsilon_0 \hbar c) \approx 1/137$ and AMR contribution 
\begin{align}
j_A(E,E^-) = 2 \hbar \operatorname{Im}X_{xy}^{L R}(E,E^-).
\label{eq:jax}
\end{align}
Similarly, 
\begin{align}
J_N = \frac{ 4 \alpha}{3 \pi c^2} \int_{0}^{\infty} \frac{ d \hbar\omega}{2 \pi} \omega\Theta(\mu_L-\mu_R-\hbar\omega)  \int_{\mu_R+\hbar\omega}^{\mu_L} \frac{d E}{2 \pi } j_N(E,E^-)
\end{align}
with 
\begin{align}
j_N(E,E^-) = {\rm Re}\left\{X^{LR}_{xx}(E,E^-)+X^{LR}_{yy}(E,E^-)+X^{LR}_{zz}(E,E^-)\right\}.
\end{align}
%We will analyze $j_A$ in the following to understand the dependence of $J_A(\omega)$ on the system parameters.  
We will apply the above theory to helical chain structures to study how the AMR depends on the molecule parameters and its coupling to the electrodes. 
%

%Electron current,
%\begin{equation}
%J_E = \int_{-\infty}^{\infty} \frac{dE}{2\pi} T(E) (f^L(E)-f^R(E))
%\end{equation}


%\subsection{Angular momentum of vortex electron}
%The momentum of ground state of the electrons bound to the site is expected to be zero in space and time. And the expection value of the atomic orbital angular momentum is also zero. Even if it is not the ground state, it only moves around the site center/nucleus instead of the molecular origin. It obviously does not contribute to the total molecular orbital angular momentum. The total orbital angular momentum of a molecule can only be contributed by the movement of electrons between different sites.
%We temporarily use the classical approximation, which can directly use the electric current to perform calculations, which is very convenient.
%
%\begin{equation}
%\label{equ:lz}
%\tilde{L}^e_z(E)=\sum_{ij} \mathbf{a}_{ij} \times (m_e \tilde{I}^{N,e}_{ij} * \mathbf{r}_{ij})\cdot \mathbf{e}_z.
%\end{equation}
%where $\mathbf{a}_{ij} = (\mathbf{r}_i+\mathbf{r}_j)/2$ is average position, $\times$ represent vector cross product, $\tilde{I}^{N,e}_{ij}$ is local current spectrum from site $j$ to $i$, abbr. 
%\begin{equation}
%\label{equ:local-current}
%\tilde{I}_{ij}=\frac{i}{\hbar}T_{ij}
%\end{equation}
%$T_{ij}$ is local current spectrum/transimission, by integrate $T_{ij}$ over the energy window, we can get $I_{ij}$ \cite{stegmann_current_2020}.
%$$T_{ij} = \operatorname{Im}[H^*_{ij} G^n_{ij}],$$ $G^n = g^r (\sum_\alpha \Gamma_\alpha f_\alpha) g^a $ is correlated Green's function, which is related to spectral function $A$.
%
%
%Local current is summation of local current spectrum in energy window, 
%\begin{equation}
%I^{N,e}_{ij} = \int \frac{dE}{2\pi} \tilde{I}^{N,e} = \frac{i}{\hbar} \int \frac{dE}{2\pi} (f_L-f_R) T_{ij}
%\end{equation}
%
%Combine Equ.(\ref{equ:lz}) and Equ.(\ref{equ:local-current}), we have 
%
%\begin{equation}
%\tilde{L}^e_{z,ij}(E)= \frac{i m_e}{\hbar} \sum_{ij}  (\mathbf{a}_{ij} \times \mathbf{r}_{ij} \cdot \mathbf{e}_z )T_{ij}.
%\end{equation}
%Basically, $\tilde{L}^e_{z}$ is positive when local vortex electron is in counter-clockwise movement.
%
%Derivation of angular momentum density is inspired by aromatic ring circular transmission \cite{stegmann_current_2020}.
%\begin{equation}
%T^C = \sum_{i<j} \operatorname{Sign}(\mathbf{a}_{ij} \times \mathbf{r}_{ij} \cdot \mathbf{e}_z ) T_{ij}.
%\end{equation}
%Apparently, in helix structure with a fixed radius, $T^C \propto \tilde{L}_z$, but $\tilde{L}_z$ brings intuition to angular momentum transfer progress. 
%
%
% % For the sake of simplicity, we use $T^C$ to represent vortex electron angular momentum.


\section{\label{sec:results}Results}


%It may cause calculation error in some cases.

%In the numerical calculation, we set 
%a symmetric bias between the two leads, with chemical potential $\mu_L = - \mu_R = 2 eV$, temperature $T_L=T_R=0$, and coupling between molecule and electrodes $\Gamma_L=\Gamma_R=\Gamma_0=0.4eV$. The radius of DNA is $r_{g}= \SI{7}{\angstrom} $, length is $l_a = \SI{5.6}{\angstrom}$, helix angle $\theta \approx 0.66$ and phase angle $\Delta\phi = \pi/5$.
%The default molecule setup is symmetric/indistinguishable DNA, $\epsilon_1 = \epsilon_2 = 0, t_1 = t_2 = 0.1eV, t_{IC} = 0.5eV, t_{SO} = 0$. We call it "symmetric point". Currently, restricted by computational cost and limited time, we using energy step $1 \times 10^{-4}eV$ and energy range $-4\to4eV$. It may cause calculation error in some cases.
%A previous work discussed that base pairing is the key transport distinction of DNA when interchain hopping is turning on \cite{caetano_sequencing-independent_2005}. We choose a large interchain hopping to stimulate the strong coupling for base-pairs to get two seperated bands.

%\subsection{Single helical chain}
\subsection{Angular momentum radiation spectra}
\label{para:Term X for default configuration}
We now present numerical results for model helical chain structures. We perform dimensionless calculation with $\hbar=e=l_a=t=1$. The default parameters are following: the system temperature $T_L=T_R=0$, the chemical potential $\mu_L=4t,\mu_R=-4t$ ,the coupling larameter $\gamma_L=\gamma_R=0.5$ t, 
radius of the chain $r_g = 7 \textup{~\AA} $, arc length $l_a =5.6\textup{~\AA}$,
helix angle $\theta \approx 0.66$, and phase angle $\Delta\phi = \pi/5$. 
The electronic structure is modeled by using tight-binding parameters, with onsite energy set to zero, the nearest neighbour (NN) hopping $t_{\rm NN}=t$, different values of the next nearest neighbour (NNN) hopping $t_{\rm NNN}$ will be used. 
An energy step of $10^{-4} t$ is used to do the numerical integration, and the energy range is set to $[-5t, 5t]$. 
%\revision{Please check the default parameters.}

We start from the simplest structure of a single helical chain with length $N=3$. The AMR can be analyzed through the energy dependence of $j_A(E,E^-)$. 
From Eq.~(\ref{equ:JA}) we see that the emission spectrum of the system at given energy $\hbar\omega$ can be obtained by integrating along a line cut over an effective bias window where the inelastic optical transition can take place. This bias window is determined by the relative positions between the two electrode chemical potentials, controlled by the $\Theta$ function in Eq.~(\ref{equ:JA}). Thus, $j_A (E,E^{-})$ can be used to characterize the ability of the system to emit radiation with angular momentum. This is especially useful in molecular junctions where the rotational symmetry is broken and orbital angular momentum can no longer be used to characterize the symmetry property of molecular orbitals, as in simple molecules\cite{zhang_angular_2020}. The total AMR is obtained by integrating $j_A$ over $E$ and $\omega$. 
%
Figure~\ref{fig:mod1} summarizes the main results with NN (upper row) and NNN (lower row) hopping, respectively.
%Two pairs of peaks and dips can be observed in Fig.~\ref{fig:mod1}(a), which contribute dominantly to AMR. The peak and dip correspond to positive and negative AMR, respectively. 
%We focus on the case $\hbar\omega >0$. The positive (negative) peak corresponds to inelastic transition from level $3 (2)$ to $2 (1)$ in Fig.~\ref{fig:mod1}(c). Both of them lie on the line with $\hbar\omega=t$ (red dashed line), since the three levels are equally spaced. 
%Fig.~\ref{fig:ssdna_xlr}(a) shows $ImX$ as a function of electron and hole energy for ssDNA with chain length $N=3$ and nearest neighbor hopping. 
We can identify several noteworthy points:
\begin{itemize}
    \item  Sharp peaks are observed and dominate the contribution in the parameter space ($E$, $E^-$). Their positions correspond to $\hbar\omega=E_{32}=E_{21}$ [Fig.~\ref{fig:mod1}(d)]. The positive and negative values correspond to opposite angular momentum radiation, i.e.,
    inelastic transition from state $3$ to $2$ and from $2$ to $1$ in Fig.~\ref{fig:mod1}(d) contributes oppositely. 
    \item  Switching electron and hole energy, $j_A$ remains unchanged. This can be shown analytically (Appendix \ref{sec:x_discuss}) and is reflected by the symmetry about line $E=E^{-}$. 
    For a given photon energy $\hbar\omega$, the integral in Eq.~(\ref{equ:JA}) over $E$ is along a line (red dashed line for example). $j_A$ is odd about the point where this line crosses with line $E+E^-=0$. Inclusion of NNN hopping breaks the second symmetry, while keeping the first intact.
    %\item The peak and dip for $\omega >0$ correspond to inelastic transition from state $3 (2)$ to $2 (1)$ in Fig.~\ref{fig:mod1}(d). They contribute to positive and negative AMR, respectively. 
    \item Inelastic transition from state 3 to 1 can not generate AMR in the NN case. But this `selection rule' is broken once NNN is included.
    \item According to these symmetries, in the large bias limit $\mu_L \gg E_n \gg \mu_R$, the total AMR is zero. However, if it is possible to include only part of the region, AMR becomes possible. 
    In realistic structure, this may be achieved by: (1) selective enhancement of certain spectral range, i.e., via localized gap plasmon modes in a junction\cite{kaasbjerg_theory_2015,nian_fano_2018}, (2) electrical tuning of molecular levels through gating or source-drain voltage in a transistor setup.
\end{itemize}

%From Eq.~(\ref{eq:x2}) we see that $X$ is proportional to two dipole matrix elements between a pair of occupied and unoccupied scattering states from the two electrodes, respectively. 



\begin{figure}[ht]
    \centering
    \includegraphics[width=1.0\textwidth]{./Figure2.pdf}
    % \includegraphics[height=0.2\textwidth]{./figures/s3_nn_trans.pdf}
    % \includegraphics[height=0.2\textwidth]{./figures/s3_nn_xlr2d.pdf}
    % \includegraphics[height=0.2\textwidth]{./figures/s3_nn_xlr1d.pdf}
    % \includegraphics[height=0.2\textwidth]{./figures/IET_SSDNA.drawio.pdf}
    % \includegraphics[height=0.2\textwidth]{./figures/s3_nnn_trans.pdf}
    % \includegraphics[height=0.2\textwidth]{./figures/s3_nnn_xlr2d.pdf}
    % \includegraphics[height=0.2\textwidth]{./figures/s3_nnn_xlr1d.pdf}
    % \includegraphics[height=0.2\textwidth]{./figures/IET_SSDNA_SNN.drawio.pdf}
    \caption{ Numerical results for a single helical chain with $N=3$. In the upper row, only the nearest neighbour (NN) hopping $t_{\rm NN}=t$ is included. In the lower row, in additional to $t_{\rm NN}$, next nearest neighbour (NNN) hopping $t_{\rm NNN}=0.4 t$ is included. 
        (a, e) Electron transmission spectrum. (b, f) $j_A$ as a function of energy $E$ and $E^-$ within the range [-2.5t,2.5t] . The velocity is $v_0 = l_a t /\hbar$. (c, g) Line cuts of the plot in (b, f). (d, h) IET Diagram. Red (blue) arrow represents transition with positive (negative) AMR, while dashed arrow corresponds to zero AMR.
    }
    \label{fig:mod1}
\end{figure}
%In Fig. \ref{fig:ssdna_xlr}(b), in full bias situation $\mu_L >> E_n >> \mu_R$,  for $\hbar\omega= E_{32}$, it can be shown that the integral of $Im X$ is canceled by odd symmetry. In this case, the physical picture is clear in Fig. \ref{fig:ssdna_xlr}(c), IET from level $2\to1$ and $3\to2$ produce left and right circularly polarized photon \textcolor{red}{circularly polarized photon or photon with AM?}   Angular momentum emitted from left and right circularly polarized light is canceled by each other. We prove it analytically in \ref{sec:x_discuss}.

%------------------------------------------------------------
%\newpage
%\label{para:Influence of term X in SNN hopping}
%\begin{figure}[ht]
%    \centering
%
%    \includegraphics[height=0.3\textwidth]{./figures/ssdna_neighbor_f2c.pdf}
%    \includegraphics[height=0.3\textwidth]{./figures/ssdna_neighbor_f2d.pdf}
%    \includegraphics[height=0.3\textwidth]{./figures/IET_SSDNA_SNN.drawio.pdf}
%    \caption{ 
%         (a) 2D plot of $Im X$ as a function of energy $E$ and $E-\hbar\omega$  $\lambda=0.1,N=3,\Gamma=0.05$.  (b) Line cut of the plot in (a) for different $\hbar\omega$ (0.11 for blue line, 0.14 for orange line, 0.24 for green line).  (c) IET diagram for $\lambda=0.1$
%    }
%    \label{fig:ssdna_xlr_snn}
%\end{figure}
%We now include the NNN hopping in the chain. 
%The corresponding results are shown in Fig. \ref{fig:ssdna_xlr_snn}. $j_A$ is not symmetric due to the NNN hopping term. IET $3\to1$ is also activated by this symmetry breaking term. 

%\subsubsection{Single helix chain with $N=10$}
% {\bf{RESULTS OF THE N=10 CHAIN IN THE NN LIMIT?}}
%   \begin{figure}[ht]
%     \centering
%     \includegraphics[height=0.3\textwidth]{./figures/s10_nn_trans.pdf}
%     \includegraphics[height=0.3\textwidth]{./figures/s10_nn_xlr.pdf}
%     \includegraphics[height=0.3\textwidth]{./figures/s10_nn_xlr1d.pdf}
%     \includegraphics[height=0.3\textwidth]{./figures/s10_nn_iets.pdf}


    % \caption{ Results for single helix chain with $N=10$ in NN hopping.
    %     (a) Transmission spectrum. (b) 2D plot of $j_A$ as a function of energy $E$ and $E^-$ . Grid lines represent the eigen energy levels. (c)  Line cut of the plot in (b) for different $\hbar\omega$, blue for $0.23$, orange for $0.61$, green for $1.08$, red for $1.63$    (d) IET diagram.
    % }
    % \label{fig:s10_nn}
    % \end{figure}

    \begin{figure}[ht]
        \centering
        \includegraphics[width=1.0\textwidth]{./Figure3.pdf}
        % \includegraphics[height=0.3\textwidth]{./figures/s10_nn_trans.pdf}
        % \includegraphics[height=0.3\textwidth]{./figures/s10_nn_xlr.pdf}
        % \includegraphics[height=0.3\textwidth]{./figures/s10_nn_xlr1d.pdf}
        %\includegraphics[height=0.2\textwidth]{./figures/s10_nn_eta.pdf}
        % \includegraphics[height=0.2\textwidth]{./figures/s10_nn_iets.pdf}
        % \includegraphics[height=0.3\textwidth]{./figures/s10_nnn_trans.pdf}
        % \includegraphics[height=0.3\textwidth]{./figures/s10_nnn_xlr.pdf}
        % \includegraphics[height=0.3\textwidth]{./figures/s10_nnn_xlr1d.pdf}
        %\includegraphics[height=0.2\textwidth]{./figures/s10_nnn_eta.pdf}
        % \includegraphics[width=0.2\textwidth]{./figures/s10_snn_f1e.pdf}
        % \includegraphics[height=0.3\textwidth]{./figures/s10_nnn_iets.pdf}    
        \caption{ Results for single helical chain with $N=10$. Results with NN hopping are shown in the upper row (a-c). NNN hopping of $t_{\rm NNN}=0.4 t$ is included for results shown in the lower row (d-f).  
            (a, d) Electron transmission spectrum. (b, e) 2D plot of $j_A$ as a function of energy $E$ and $E^-$ . Grid lines represent the eigen energy levels. (c, f)  Line cut of the plot in (b, e) for different $\hbar\omega$, blue for $0.23$, orange for $0.61$, green for $1.08$, red for $1.63$.
        }
        \label{fig:s10_nn_nnn}
    \end{figure}

    
To make the system more realistic, we increase the chain length to $N=10$.
The AMR distribution represented by $j_A(E, E^-)$ spreads to much larger regions (Fig.~\ref{fig:s10_nn_nnn}). 
%Correspondingly, numbers of peaks in transmission spectrum and $ImX$ increase while symmetric properties remains as well as  $N=3$. {\bf MORE DISCUSSIONS HERE!}
The effect of NNN hopping on angular momentum radiation is more dramatic than the shorter chain. The negative regions shrink and the whole distribution is dominated by the positive regions. This is also reflected in the asymmetric distribution of the electron transmission in the positive and negative energy range.  We have shown in Fig.~\ref{fig:ssdna_neighbor} the dependence of $J_A$ on the NNN hopping in the full bias regime, where the bias window encloses all the molecular orbitals. We observe increase of both magnitude and efficiency of AMR, and the efficiency $J_A/J_N$ saturates at around $0.3 \hbar$.
%By introducing SNN hopping, the electron-hole symmetry is breaking\cite{guo_spin-dependent_2014}. As a result, the angular momentum is not canceled with each other.



%------------------------------------------------------------
%\newpage
\label{para:Introduce NNN hopping}

    \begin{figure}[ht]
    \centering
    \includegraphics[width=1.0\textwidth]{./Figure4.pdf}
    % \includegraphics[width=0.3\textwidth]{./figures/s10_snn_f1a.pdf}
    % \includegraphics[height=0.3\textwidth]{./figures/s10_snn_f1b.pdf}
    % \includegraphics[width=0.3\textwidth]{./figures/s10_lambda_JA.pdf}
    % \includegraphics[width=0.3\textwidth]{./figures/s10_lambda_JAJN.pdf}
    % \includegraphics[width=0.3\textwidth]{./figures/s10_lambda_JAspec.pdf}

    \caption{ AMR $J_A$ (a), AMR per photon $J_A/J_N$ (b) for single helical chain with $N=10$ as a function of $\lambda=t_{\rm NNN}/t_{\rm NN}$ in the full bias case $\mu_L > {\rm max}\{\varepsilon_i\} > {\rm min}\{\varepsilon_i\} > \mu_R$, with $\{\varepsilon_i\}$ the set of eigen energies of the molecular orbitals. 
     (c) AMR spectrum $J_A(\omega)$ from $\lambda=0$ (bottom) to $0.4$ (top). }
    \label{fig:ssdna_neighbor}
    \end{figure}
%------------------------------------------------------------
\subsection{Geometrical dependence}
\label{para:Geometry Chirality}
\begin{figure}[ht]
    \centering
    \includegraphics[width=1.0\textwidth]{./Figure5.pdf}
    % \includegraphics[height=0.3\textwidth]{./figures/ssdna_phase_f1a.pdf}
    % \includegraphics[height=0.3\textwidth]{./figures/ssdna_phase_f1b.pdf}
    % \includegraphics[height=0.3\textwidth]{./figures/ssdna_phase_f1c.pdf}
    \caption{ 
        Dependence of $J_A$ (a), AMR per photon $J_A/J_N$ (b) , AMR per electron $J_A/J_E$ (c) as a function of phase angle $\Delta\phi$ for parameters $t_{\rm NN}=t, \lambda=0.2, N=12, \gamma_L = \gamma_R = 0.1t, \mu_L=2t, \mu_R=-2t$. $I$ is the current through the molecule. Insets of (a) show single helical chains with phase angle $\Delta\phi = -\pi/4, 0, \pi/4$, respectively.
        %The phase angles $\Delta\phi$ of molecules in in set of (a) are $-\pi/4$, $0$, $\pi/4$ from left to right.
        %\revision{In (a) the result is normalized by $J_0 \approx 2.1 \times 10^{-8} eV = 3.192 \times 10^7 \hbar/s$ which is a typical value for benzene\cite{zhang_angular_2020}.} 
    }
    \label{fig:ssdna_phase_angle}
\end{figure}

%In this subsection, we 
To show the geometrical origin of the AMR in chiral molecules, 
dependence of AMR in the high bias limit on phase angle $\Delta\phi$ is shown in Fig.~\ref{fig:ssdna_phase_angle}. It can be seen that AMR increases with the absolute value of phase angle. It is exactly zero (Appendix \ref{sec:x_discuss}) for achiral straight chain and changes sign when $\Delta \phi$ goes from positive to negative. 
Further analysis of $j_A (E, E^-)$ shows that the whole distribution in energy space reverses sign when $\Delta\phi$ changes sign. 
This positive correlation between AMR and chirality of the chain is one evidence of geometrical origin of AMR studied here. 
%Analytically, we prove $J_A=0$ for straight chain with any direction in Sec. \ref{sec:straight_chain}. We can treat $\Delta\phi$ as a measurement of chirality. AMR get larger With higher chirality.

We depict the length ($N$) and radius ($R$) dependence of the AMR in Fig.~\ref{fig:s10_geo_N_R}. We have integrated all the positive (termed Right-handed (RH), $j_{AR}$) and negative (termed Left-handed (LH), $j_{AL}$) regions of $j_A$ in the energy space to characterize the system's ability to radiate angular momentum. When $\lambda =0$, corresponding to zero NNN hopping, the positive and negative regions are of the same magnitude. They become asymmetric with enhanced positive AMR once $\lambda \neq 0$. In any case, the AMR grows linearly with chain length $N$ and quadratically with radius $R$. This is the second evidence of geometrical nature of AMR. 

%\begin{figure}[ht]
%    \centering
%
%    \includegraphics[height=0.3\textwidth]{./figures/ssdna_length_f1a.pdf}
%    \includegraphics[height=0.3\textwidth]{./figures/ssdna_length_f1b.pdf}
%    \includegraphics[height=0.3\textwidth]{./figures/ssdna_length_f1c.pdf}
%    \caption{ 
%        Plot of (a) angular momentum radiation $J_A$ (b) AMR per photon $J_A/J_N$ (c) AMR per electron $J_A/J_E$ as a function of $N$ for $\Gamma=0.1$(blue line) and $\Gamma=0.4$(orange line). 
%    }
%    \label{fig:ssdna_length}
%\end{figure}
%Fig. \ref{fig:ssdna_length} shows that AMR increases linearly with chain length $N$. And angular momentum per photon is also increasing but the slope gradually declined. AM per photon is reaching a saturation value of $0.4\hbar$ at $\Gamma=0.1eV$.

\begin{figure}[ht]
    \centering
    \includegraphics[width=1.0\textwidth]{./Figure6.pdf}    
    % \includegraphics[height=0.23\textwidth]{./figures/s10-t2-N-jar.pdf}
    % \includegraphics[height=0.23\textwidth]{./figures/s10-t2-N-jal.pdf}
    % \includegraphics[height=0.23\textwidth]{./figures/s10-t2-rog-jar.pdf}
    % \includegraphics[height=0.23\textwidth]{./figures/s10-t2-rog-jal.pdf}
    \caption{ 
        Angular momentum radiation as a function of chain length $N$ (a-b) and radius of gyration $r_g$ (c-d) with different $\lambda$ in the full bias regime. Default values are used for other parameters. 
        $J_{\rm AR}$: RH radiation corresponding to $j_A>0$,  $J_{\rm AL}$: LH radiation  corresponding to $j_A<0$. Note the different scales of $J_{\rm AR}$ and $J_{\rm AL}$.
    }
    \label{fig:s10_geo_N_R}
\end{figure}


%----------------------
%\newpage
%\label{para:Chemical Potential}
%\begin{figure}[ht]
%    \centering
%
%    \includegraphics[height=0.2\textwidth]{./figures/ssdna_potential_fa.pdf}
%    \includegraphics[height=0.2\textwidth]{./figures/ssdna_potential_fb.pdf}
%    \includegraphics[height=0.2\textwidth]{./figures/ssdna_potential_fc.pdf}
%    \includegraphics[height=0.2\textwidth]{./figures/ssdna_potential_fd.pdf}
%    \caption{ 
%        Plot of (a) angular momentum radiation $J_A$ (b) AMR per photon $J_A/J_N$ (c) AMR per electron $J_A/J_E$ as a function of chemical potential $\mu_R$ for different $\Gamma$ (d) 2D plot of AMR $J_A/J_0$ as a function of potential $\mu_L$ and $\mu_R$ at $\Gamma=0.01,N=3,\lambda=0$. 
%    }
%    \label{fig:ssdna_potential}
%\end{figure}

%------------------------------------------------------------
%In Fig. \ref{fig:ssdna_xlr}, we take full bias to select all IETs. Here, we fix $\mu_L=2eV$ and raise $\mu_R$. When $\mu_R$ close to $E_1$, the negative X contribution part is losing from energy window. After that, when $\mu_R$ close to $E_1$, the positive X contribution part is also losing, leading to zero AMR again. We call it energy window selection effect.
%Fig. \ref{fig:ssdna_potential}(d), we illustrate the chemical potential dependence. We observe a four-line-segment feature(FLSF) as the same as benzene \cite{zhang_angular_2020} and graphene nanoribbon\cite{zhang_energy_2021}. But here's something different, it forms four platform with small lead coupling $\Gamma$ due to energy window selection effect mentioned before.
%
%
%----------------------

%------------------------------------------------------------
%\newpage
%\label{para:Lead Coupling}
%\begin{figure}[ht]
%    \centering
%
%    \includegraphics[height=0.2\textwidth]{./figures/ssdna_lead_cp_f1a.pdf}
%    \includegraphics[height=0.2\textwidth]{./figures/ssdna_lead_cp_f1b.pdf}
%    \includegraphics[height=0.2\textwidth]{./figures/ssdna_lead_cp_f1c.pdf}
%    \includegraphics[height=0.2\textwidth]{./figures/ssdna_lead_cp_f1d.pdf}
%    \caption{ 
%        Plot of (a) angular momentum radiation $J_A$ (b) AMR per photon $J_A/J_N$ (c) AMR per electron $J_A/J_E$ as a function of $\Gamma_R/\Gamma_L$, fix $\Gamma_L=0.1$. 
%        (d)2D plot of $Im X$ as a function of energy $E$ and $E-\hbar\omega$ at $\Gamma_R/\Gamma_L=0.7,N=3$
%    }
%    \label{fig:ssdna_lead_cp}
%\end{figure}

%\revision{
%In default case, the electrode coupling of $L,R$ are the same. In Fig.\ref{fig:ssdna_lead_cp}, We change $\Gamma_R$ to break symmetry, AMR gained. AMR is positive when $\Gamma_L/\Gamma_R<1$. With  $\Gamma_L/\Gamma_R$ increasing,  For detailed information, we explore $ImX$, it breaks symmetric in both two directions $E=E^-$ and $E=-E^-$. That means asymmetric coupling can break two kinds of symmetry.
%}

%\subsubsection{Length dependence}
\subsection{Double-helical chain}
%\subsection{Efficiency}
%In this part, we discuss how the AMR efficiency depends on the parameters and structures.
%By applying Eqs.(\ref{equ:JN}) and (\ref{equ:JA}), AMR efficiency can be expressed using the transition dipole moment $X$. For a specific transition $E\to E^-$, we can analyse photon AMR efficiency:
%\begin{equation}
%   \eta^{A/N} = j_A / j_N.
%\end{equation}
% with AM radiation rate $j_A = 2 \hbar Im X^{LR}_{xy}$, photon radiation rate $j_N = Re(X^{LR}_{xx}+X^{LR}_{yy}+X^{LR}_{zz}).$
%Also, the result we intended is the average AM radiation efficiency, which is a weighted average of $\eta$, 
%A more accessible quantity is the total AMR efficiency defined as
%    \begin{align}
%        \bar{\eta}^{A/N}=J_A/J_N&=\int d E dE^-  \Theta(E-E^-)  (j_N/J_N) \eta^{A/N}
%    \end{align}
%We have written it as a weighted integral of the individule efficiency with the weighting factor
%$j_N/J_N$. 

%In this section, we consider both single and double helical chains and compare their AMR efficiency\cite{guo_contact_2014-1}.
%For the single helical chain, we take the following default parameters: NN hopping $t_1=1.0eV$, onsite energy $\varepsilon_0 = 0$, NNN hopping $t_2=0$.
Double-stranded DNA is a typical chiral molecule that has received considerable attention in molecular electronics\cite{xie2011spin,xiang2017gate,guo_spin-selective_2012}. We now use a commonly adopted tight-binding model of double helical chain to study its AMR property. 
Specifically, we take the following model parameters that resemble DNA\cite{dong-sheng_gap_2008}:
NN hopping $t_{\rm NN} = 1.0$ eV, inter-chain hopping $t_{IC}=2.4$ eV, onsite energy $\varepsilon_{1,2} = \pm 0.56$ eV, NNN hopping $t_{\rm NNN}=0$ eV, electrode coupling  $\Gamma=0.5$ eV.

We focus on the coupling of the molecule to the two electrodes, 
by consider two types of coupling [(I), (II) in Fig.\ref{fig:dna_elec}]. The first type is coherent coupling (I), whose `off-diagonal' elements are non-zero
\begin{align}
    \Gamma^L_{ij} = \Gamma (\delta_{i1}+\delta_{i,N+1})(\delta_{j1}+\delta_{j,N+1}),  \\
    \Gamma^R_{ij} = \Gamma (\delta_{iN}+\delta_{i,2N})(\delta_{jN}+\delta_{j,2N}). 
\end{align}
This introduces coherent coupling between the two chains.
The second type is the incoherent type (II) with zero `off-diagonal' elements
\begin{align}
    \Gamma^{L,inco}_{ij} = \Gamma (\delta_{i1}+\delta_{i,N+1})\delta_{ij}, \\
    \Gamma^{R,inco}_{ij} = \Gamma (\delta_{iN}+\delta_{i,2N})\delta_{ij}. 
\end{align}
In this case, electron transport and light emission processes are independent for the two chains. The final result is simply sum of contributions from each single helical chain. 
%
% \begin{center}
%     \begin{figure}[ht]
%     \includegraphics[height=0.3\textwidth]{./figures/contact_setup.png}
%     \includegraphics[height=0.3\textwidth]{./figures/contact_setup_inco.png}
%     \caption{(a-b) Schematic views of DNA molecular devices coupled to electrodes. Each sphere denotes a site, with red(blue) lines assembling the first(second) helical chain. DNA is connected with shared electrodes in (a), inducing a coherent effect. While individual electrodes in (b).}
%     \label{fig:dna_contact_setup}
%     \end{figure}
% \end{center}
%

   % \begin{enumerate}
   %          \item NN  hopping $t_1=1.0eV$.
   %          \item inter-chain hopping $t_{IC}=2.4eV$.
   %          \item onsite energy difference $\Delta\varepsilon =0.56eV$.
   %          \item NNN hopping $t_2=0eV$.
   %          \item phase difference $\Delta\theta = 2/3 \pi$.
   %          \item chemical potential $\mu_L=1.5eV,\mu_R=-1.5eV$.
   %          \item temperature $T_L=T_R=0$.
   %          \item coherent lead coupling  $\Gamma=1eV$.
   %  \end{enumerate}
% The double helical structure of DNA contains a major groove and minor groove, which lengths are determined by phase angle difference $\Delta\theta$. If $\Delta\theta=\pi$, the lengths of grooves are the same. 
%================================================
% RNA electronic strucuture
%
%\begin{figure}[ht]
%
%
%\includegraphics[height=0.2\textwidth]{./figures/rna-trans.pdf}
%\includegraphics[height=0.2\textwidth]{./figures/rna-jA.pdf}
%\includegraphics[height=0.2\textwidth]{./figures/rna-jN.pdf}
%\includegraphics[height=0.2\textwidth]{./figures/rna-etaAN.pdf}
%
%
%
%\caption{(a) Electron transmission spectrum for RNA.   (b-d) 2D Plot of $j_A$, $j_N$ and $\eta^{A/N}$ as a function of $E$ and $E^-$, respectively.  
%}
%\label{fig:rna_elec}
%\end{figure}



%================================================
% DNA electronic strucuture
\begin{figure}[ht]
    \includegraphics[width=1.0\textwidth]{./Figure7.pdf}
% \includegraphics[height=0.18\textwidth]{./figures/contact_setup.png}
% \includegraphics[height=0.18\textwidth]{./figures/dna_trans.pdf}
% \includegraphics[height=0.18\textwidth]{./figures/dna_jA.pdf}
% \includegraphics[height=0.18\textwidth]{./figures/dna_jN.pdf}
% \includegraphics[height=0.18\textwidth]{./figures/dna_etaAN.pdf}

% \includegraphics[height=0.18\textwidth]{./figures/contact_setup_inco.png}
% \includegraphics[height=0.18\textwidth]{./figures/dna_inco_trans.pdf}
% \includegraphics[height=0.18\textwidth]{./figures/dna_inco_jA.pdf}
% \includegraphics[height=0.18\textwidth]{./figures/dna_inco_jN.pdf}
% \includegraphics[height=0.18\textwidth]{./figures/dna_inco_etaAN.pdf}



\caption{Results for the double-helical chain with coherent (I, upper) and incoherent (II, lower) coupling to the two electrodes. (I, II) Schematic views of two types of coupling of the double helical chain to electrodes. (a-d) Electron transmission spectrum (a), 2D Plot of $j_A$ (b), $j_N$ (c) and $\eta^{A/N}=j_A/j_N$ (d) as a function of $E$ and $E^-$, respectively. (e-h) The same as (a-d) but for incoherent coupling to electrodes. 
}
\label{fig:dna_elec}
\end{figure}

The final results are summarized in Fig.~\ref{fig:dna_elec}. 
The first column compares electron transmission spectra for the two types of coupling to the electrodes. For coherent/incoherent coupling, the electron-hole pair symmetry in the transmission is broken/preserved. This has important consequence on photon and AMR spectra shown in the following columns. 
For incoherent coupling, $j_A$ is anti-symmetric along diagonal lines $E=-E^-$. Integration over $E$ and $E^-$ leads to cancellation of contributions from opposite sides of the $E=-E^-$ line. This can be avoided in the coherent coupling case due to the breaking of electron-hole symmetry. The resulting AMR efficiency $\eta_{A/N} = j_A/j_N$ is enhanced in a wide range of $(E,E^-)$ space. 
%Similar results are shown in Fig.~\ref{fig:dna_vsd} where $J_A$ (a), $J_N$ (b) and $\eta$ (c) are plotted as a function of source-drain bias $V_{sd}$. 
%These results demonstrate the importance of coherent inter-chain coupling in enhancing the AMR efficiency. 

%vlIn Fig.\ref{fig:dna_elec}(a,e), it can be shown that the transmission spectrum is separated into two bands due to high inter-chain hopping $t_{IC}$. DNA behaves as a semiconductor in this case. 
%The transmission is symmetric for incoherent coupling as shown in (b).
%However, transmission is depressed above the Fermi level due to the coherent coupling as shown in (a).  

%In (b,f), the peaks of $j_A$ in the 2D plot occur in different energy levels, corresponding to different transitions. The peaks are larger between inter-band transitions than between intra-band transitions. It indicates a larger bias than the gap is necessary for large AM radiation. For incoherent transport, $j_A$ is odd symmetric along diagonal lines $E=E^-$, while it is broken by coherent coupling as shown in (c).
%In (c,g), The peaks of $j_N$ are larger between inter-band transitions than between intra-band transitions for both cases. In (d,h), the efficiency can reach $\pm \hbar$ for different transitions. That means for specific transitions, the photon can be emitted fully "polarized". "Polarization" is stronger in coherent coupling than incoherent.
\section{Discussions}
We can try to understand these numerical results using Eqs.~(\ref{eq:x2}) nad (\ref{eq:jax}). We see that the AMR in $z$ direction is proportional to the imaginary part of the two electric dipole matrix elements in the $x$ and $y$ directions. Geometrical chiral properties of the molecule are encoded in the velocity matrix $V^\mu, \mu=x, y$. Both of them involve occupied scattering state from one electrode and unoccupied state from the other. Importantly, it does not involve any magnetic dipole matrix element. This is in contrast to circular dichroism and optical rotation in chiral molecules, where the magnetic dipole transition is critical\cite{polavarapu_chiroptical_2017}. Non-zero AMR needs breaking of time-reversal symmetry (TRS). In optical rotation and circular dichroism, the molecular eigen states are time-reversal symmetric. Breaking of TRS is realized by the external magnetic field. Meanwhile,  the biased chiral molecular junction studied here is an open system. The electronic states participating the inelastic transition are scattering states. The involved scattering states are determined by current direction. 
TRS breaking is realized by external bias and resulting electrical current. Thus, magnetic dipole transition is not necessary. This is the central observation of present study. It enables electrical generation of optical angular momentum utilizing the chiral geometric properties of the molecule without introducing magnetic field. It also differs from other approaches where optical angular momentum is generated by chiral wave guide from initially linear polarized light. 


%================================================
% DNA bias dependence
%\begin{figure}[ht]
%
%\includegraphics[height=0.3\textwidth]{./figures/dna-vsd-JA.pdf}
%\includegraphics[height=0.3\textwidth]{./figures/dna-vsd-JN.pdf}
%\includegraphics[height=0.3\textwidth]{./figures/dna-vsd-eta.pdf}
%
%
%\caption{  Plot of (a) angular momentum radiation $J_A$ (b) photon radiation $J_N$ (c) average photon AMR efficiency $\bar{\eta}^{A/N}$ as a function of bias $Vsd$ for coherent electrode (blue lines) , incoherent electrode (orange lines), NNN hopping with coherent electrode (green lines) and , NNN hopping with incoherent electrode (red lines), other parameters are default.
%}
%
%\label{fig:dna_vsd}
%\end{figure}

% In Fig.\ref{fig:dna_vsd}, we study the bias $Vsd$ dependence. The chemical potential is determined by  $\mu_L=Vsd/2,\mu_R=-Vsd/2$.In Fig.\ref{fig:dna_vsd}(a), the AM radiation is opened around $Vsd=3eV$, which is greatly increased from $10^{-10}eV$ to almost $10^{-6}eV$. And the radiation is close to zero for incoherent coupling. Due to the odd symmetry for inter-band transition, $j_A$ can be canceled greatly under symmetric bias. In Fig.\ref{fig:dna_vsd}(b), it can be shown that photon radiation for both coherent and incoherent leads is almost the same. And the average efficiency reaches nearly $80\%$ around $Vsd=5eV$, which is higher than benzene (around $30\%$).  And coherence of leads also greatly enhanced the efficiency.  Under $Vsd=5eV$, we can see the energy window perfectly covered the negative region of $j_A$, thus "polarization" is more focused under this bias, leading to higher average efficiency.

%================================================
% DNA electroluminescence spectrum
%\begin{figure}[ht]
%\includegraphics[height=0.2\textwidth]{./figures/dna_JAspec.pdf}
%\includegraphics[height=0.2\textwidth]{./figures/dna_JNspec.pdf}
%\includegraphics[height=0.2\textwidth]{./figures/dna_JAwavespec.pdf}
%\includegraphics[height=0.2\textwidth]{./figures/dna_JNwavespec.pdf}
%\caption{ Plot of photon energy spectrum and wavelength spectrum. Note the spectrum should scaled by Jocobian transformation $J(E)=J(\lambda) \frac{h c}{E^2}$  (a) Angular momentum radiation energy spectrum $J_A(E)$ (b) Photon radiation energy spectrum $J_N(E)$ (c) Angular momentum radiation wavelength spectrum $J_A(\lambda)$  (D) Photon radiation wavelength spectrum $J_N(\lambda)$. }
%\label{fig:dna_spec}
%\end{figure}

%In Fig.\ref{fig:dna_spec}, we plot the spectrum for default case. The wavelength spectrum is similar with experiments\cite{koyama_electroluminescence_2002,liang_enhanced_2022} for the significant wavelength around $500-600nm$, corresponding to $2-3eV$ photon energy.


%================================================
\if false
\newpage
\label{para:Onsite energy}

\subsection{Double-helical chain}

\begin{figure}[ht]

\includegraphics[height=0.2\textwidth]{./figures/dsdna_onsite_f1a.pdf}
\includegraphics[height=0.2\textwidth]{./figures/dsdna_onsite_f1b.pdf}
\includegraphics[height=0.2\textwidth]{./figures/dsdna_onsite_f1c.pdf}
\includegraphics[height=0.2\textwidth]{./figures/dsdna_onsite_f1d.pdf}
\includegraphics[height=0.1\textwidth]{./figures/IETS.drawio.pdf}

\caption{ 
Plot of (a) angular momentum radiation $J_A$ (b) AMR per photon $J_A/J_N$ (c) AMR per electron $J_A/J_E$ as a function of $\Delta_\epsilon$,  $\Gamma=0.4, N=2$. 
        (d)2D plot of $Im X$ as a function of energy $E$ and $E-\hbar\omega$ at $\Gamma=0.01$ (e) IETS diagram for DSDNA.
        }

\label{fig:onsite}
\end{figure}
$\Delta \epsilon = \epsilon_1 = -\epsilon_2$,
exactly same onsite energy  $\epsilon_1=\epsilon_2$, no radiation comes out.




%\newpage
%
%\begin{figure}[ht]
%    \includegraphics[width=0.3\textwidth]{./figures/dsdna_potential_f1a}
%    \includegraphics[width=0.25\textwidth]{./figures/dsdna_potential_f1b}
%    \includegraphics[width=0.25\textwidth]{./figures/benzene_mu}
%\caption{ 
%(a) AM flux for different chemical potential $\mu_L, \mu_R$ at $\epsilon_1=0.1,\epsilon_2=-0.1, \Gamma=0.01, N=2$
%(b)  AM flux vs. $\mu_R$ at $\mu_L=-0.8,0.8$ blue, orange line cut from (a)
%(c) Zhang's result for Benzene AM flux
%}
%\label{fig:lock}
%\end{figure}

In double-helical DNA, we can make unequal DOS for positive/negative AM state by breaking translation-reversal symmetry(TRS) to lock AM state. AM get distinct at different energy without energy window selection.

We can get AM radiation flux by NEGF calculation. left-top and right-bottom area represent full bias case, which is non-zero. The result is zero for benzene.



\newpage
\label{para:Extendability for chain length}

\begin{figure}[h]

    \includegraphics[width=0.2\textwidth]{./figures/dsdna_length_f1a}
    \includegraphics[width=0.2\textwidth]{./figures/dsdna_length_f1b}
    \includegraphics[width=0.2\textwidth]{./figures/dsdna_length_f1c}
\caption{Plot of (a) angular momentum radiation $J_A$ (b) AMR per photon $J_A/J_N$ (c) AMR per electron $J_A/J_E$ as a function of $N$ at  $\Delta\epsilon=0.1,\Gamma=0.4$. 
}
\label{fig:nl}
\end{figure}

DNA structure is extendable, the angular momentum radiation is proportional to $N_L-1$. more peaks appear and at the same time the peaks increase.
A closer look reveals that the peak count of circular transmission $T_C$ is exactly $N_L-1$ And we can get the linear relationship between angular momentum radiation and coverage area and chain length at this time:
$$\frac{J_A}{S N_L} = 2.5998*10^{-10} eV/(Ang^2) = 5.37*10^{-5} eV T/(\Phi_0)$$


\newpage
\label{para:Spin-orbit coupling}

\begin{figure}[ht]
\includegraphics[height=0.2\textwidth]{./figures/dsdna_soc_f1a.pdf}
\includegraphics[height=0.2\textwidth]{./figures/dsdna_soc_f1b.pdf}
\includegraphics[height=0.2\textwidth]{./figures/dsdna_soc_f1c.pdf}
\includegraphics[height=0.2\textwidth]{./figures/dsdna_soc_f1d.pdf}
    \caption{
    Plot of (a) angular momentum radiation $J_A$ (b)  AMR per photon $J_A/J_N$ (c) AMR per electron $J_A/J_E$ as a function of  $t_{SO}$  (d) Frequency-resolved spectrum of angular momentum radiation for different $t_{SO}$.
    }
    \label{fig:dsdna_soc}
\end{figure}
We also plot AM flux spectrum with different $t_{SO}$, the peaks are enchanced by $t_{SO}$ when electrode coupling is large .


\newpage
\section{Unused Material}
\subsection{Transition between different levels }

\begin{figure}[ht]
    \includegraphics[width=0.45\textwidth]{./figures/IETS.drawio}
    \includegraphics[width=0.45\textwidth]{./figures/iets2}
    % \includegraphics[width=0.3\textwidth]{./figures/p2}
    \caption{ 
    (a) IET diagram for DSDNA ($N_L=2$). Red arrow respresent positive AM generation, while blue arrow negative, and thin dashed arrow barely no AM generation. 
    (b) AMR spectrum for different chemical potential with $\Delta \varepsilon = 0.1, \Gamma=0.01$.
    }
    \label{fig:iets}
\end{figure}



The main process in this article: inelastic electron transition (IET) from high energy level to low energy level, emitting photon with angular momentum.  

IET between different levels can generate different AM. Some are positive, some are negative, some are nearly zero. Here, we take length 2 DSDNA chain as example, the four energy level of the molecule $E_n \approx -0.6, -0.4, 0.4, 0.6$, electrode coupling $\Gamma=0.01$ for small energy broadening.  By applying different bias between levels and performing NEGF calculation, we can find out contribution for all transitions. In Fig.\ref{fig:iets}, transition $4\to1,3\to1$ is positive, $4\to2,3\to2$ is negative.For example, let chemical potential $\mu_L = 0.7, \mu_R = - 0.5$, then energy level $2,3,4$ is in energy window. Transitions in level $2,3,4$ will be activated as shown as Fig.\ref{fig:iets} (b) green line.
In Fig.\ref{fig:iets} (a), transition $4\to1,3\to1$ is positive, $4\to2,3\to2$ is negative, corresponding to peaks in Fig.\ref{fig:iets}(b).

In Fig.\ref{fig:iets}(b), we can see there're 3 peaks for different chemical potential pairs(CPP). All 4 CPPs get peak around 0.8, corresponding $3\to2$.
All CPPs get peak around 1.0, except 0.5~-0.5. We can see corresponding transitions for peak around 1.0 are $4\to2, 3\to1$. And to be noticed, that peak for 0.7 -0.7 is a summation of 0.7 -0.5 and 0.5 -0.7. Last of all, peak around 1.2 is only taken by 0.7 -0.7 CPP. So transition with largest energy difference contribute to AMR mostly.



% In Fig.\ref{fig:iets}(c), we fix $\mu_L=-1,1$.  When $\mu_L=1$, scanning $\mu_R$ from $0\to -0.5$, level 2 is included. 2 negative AM flux transitions are stimulated, causing a negative platform in . While $-0.5 \to-0.7$, level 1 is included, 2 positive transitions are stimulated, making the platform positve.

% \newpage
% \subsection{Straight chain}


\newpage
\subsection{Longer chain}
\begin{figure}[ht]
    \includegraphics[width=0.45\textwidth]{./figures/iets3}
    \includegraphics[width=0.45\textwidth]{./figures/platform}
    
    % \includegraphics[width=0.3\textwidth]{./figures/p2}
    \caption{ 
    (a) AMR spectrum for different chain length ($N_L =2,3,4$)
    (b) AMR vs. $\mu_R$ for different chain length, forming various platform with $\mu_L=2$
    }
    \label{fig:iets3}
\end{figure}

Last subsection we disscussed DSDNA with NL=2. Here we compare different chain length. With length increasing, the spectrum get complicated, but the maxium value of peak increase as longer chain.  We can see platform for different chain length. Positive AMR require full bias.

\newpage
\subsection{Extendability}

\begin{figure}[ht]
\subfigure{
    \begin{minipage}{0.5\textwidth}
    \includegraphics[width=0.45\textwidth]{./figures/nl_pa}% Here is how to import EPS art
    \includegraphics[width=0.45\textwidth]{./figures/nl_pb}
    \includegraphics[width=0.45\textwidth]{./figures/nl_pc}
    \includegraphics[width=0.45\textwidth]{./figures/nl_pd}
    \end{minipage}
}
\caption{Applying full bias for $\mu_L=2,\mu_R=-2$ (a)chain length $N_L$ dependence for AMR. $N_L$(black for 4, red for 10)(b) AMR spectrum (c) circular transmission  $T_C$ (d) transmission spectrum }

\end{figure}

DNA structure is more extendable than benzene, AM radiation can be enchanced with chain length and projection area. As shown in Fig.\ref{fig:nl}(a), the angular momentum radiation is proportional to $N_L-1$. From Fig.\ref{fig:nl}(b), it can be seen that the AMR spectrum becomes larger with increasing chain length, that is, more peaks appear and at the same time the peaks increase. Combining the Fig.\ref{fig:nl}(c)(d), it can be seen that these peaks correspond to the energy differences of different energy levels one-to-one.
A closer look reveals that the peak count of circular transmission $T_C$ is exactly $N_L-1$, while the peak value of the transport spectrum remains unchanged. This shows that the angular momentum radiation is closely related to the circular transmission. And we can get the linear relationship between angular momentum radiation and coverage area and chain length at this time:
$$\frac{J_A}{S N_L} = 2.5998*10^{-10} eV/(Ang^2) = 5.37*10^{-5} eV T/(\Phi_0)$$
with $\Phi_0$ magnetic flux quantum.



\newpage
\subsection{Local AM state}

\begin{figure}[ht]
    \includegraphics[width=8.8cm]{./figures/am_sd_dsdna.pdf}
    \caption{ 
    (a) SD Energy and Lz in isolated benzene. 
    (b) Lz density at every mode coupling with single electrode  $\Gamma=0.01eV$.
    (c) Lz density at different  $\Gamma$.  
    (d) Total Lz vs. $\Gamma$
    }
    \label{fig:am_sd_benzene}
\end{figure}

% \begin{figure}[ht]
%     \includegraphics[width=8.8cm]{./am_sd/am_flux_dsdna.pdf}
%     \caption{ (a) AM flux spectrum $J_A$
%     (b) Photon flux spectrum $J_N$
%     (c) Polarization ($J_A/J_N$ in $\hbar$)
%     (d) Angular momentum density 
%     }
%     \label{fig:am_flux_dsdna}
% \end{figure}

We discuss the relation between vortex electron angular momentum and angular momentum radiation. From the primitive idea, the electron-photon interaction should obey conservation of angular momentum. So, it's a good start point to study from electron AM. We get infinite long DSDNA chain at first, consider only isolated molecule, we can simultaneously diagonalize $H$ and $L_z$ operator and get seperate mode, each has its own energy and angular momentum in Fig.\ref{fig:am_sd_benzene}(a). Then, we can stimulated each mode by contacting with virtual electrode $\Gamma = S \Gamma_n S^\dagger$, $n$ is mode index, $S$ is SD transformation matrix. The stimulated mode will broaden its own angular momentum, from discrete to continuous. By changing electrode coupling, total angular momentum keeps constant. Infinite long chain helps us understanding how angular momentum spread in energy space. 

% After that, we consider finite DSDNA chain in Fig.\ref{fig:am_flux_dsdna}. We can see clearly that electrons with higher energy have greater angular momentum. So totally, more positive angular momentum generated, carried by emitting photon.




\newpage
\subsection{AM state locking}

\begin{figure}[ht]
\subfigure{
    \begin{minipage}{0.9\textwidth}
    \includegraphics[width=0.45\textwidth]{./figures/p1}% Here is how to import EPS art
    \includegraphics[width=0.45\textwidth]{./figures/p2}
    \includegraphics[width=0.45\textwidth]{./figures/benzene_mu}
    \end{minipage}
}
\caption{ 
(a) AM flux for different chemical potential $\mu_L, \mu_R$
(b) AM flux vs. $\mu_R$ at $\mu_L=-1,1$ 
(c) Zhang's result for Benzene AM flux
}
\label{fig:lock}
\end{figure}

AM state locking can stimulate AM radiation at any high bias energy window.

Under tight-binding model, benzene AM state degenerate at certain energy level, total AM will get cancelled. By applying bias and stop chemical potential to that energy, we can choose certain AM state to stimulate AM radiation. However, in high bias regime, energy window will cover those degenerated states and states will not get chosen. Thus, AM flux decrease to zero at full bias.

While in double-helical DNA, we can make unequal DOS for positive/negative AM state by breaking translation-reversal symmertry(TRS) to lock AM state. AM get distinct at different energy without energy window selection.

We can get AM radiation flux by NEGF calculation. As shown as Fig. \ref{fig:lock}, left-top and right-bottom area represent full bias case, which is non-zero. The result is zero for benzene\cite{zhang_angular_2020}.

% 角动量态锁定可以使任意高偏压窗口出现角动量辐射。
% 紧束缚模型下,苯分子中的角动量态会在某些能级的地方简并,通过偏压可以对角动量态进行筛选,从而使系统辐射出角动量。但是,一旦偏压完全覆盖这个两个态,将不能起到筛选作用。也就是说,施加全覆盖的偏压时,角动量辐射为零。


% 双螺旋DNA中,可以通过破坏空间反演对称性,使正反角动量态占据数不相等,即角动量锁定。角动量态无需偏压筛选,就出现极化现象,这些极化的角动量态便可以通过IET辐射出带有角动量的光子。

% 通过NEGF计算可以得出角动量辐射。如图\ref{fig:lock},左上和右下即代表角动量态被能量窗口完全覆盖的情况,在苯的例子中,左上右下两块区域为零。而在双螺旋DNA中,非零。


\subsection{Energy induced TRS breaking}

Last subsection, we discuss AM locking effect, causing by geometry induced TRS breaking. Here, we show energy difference between two chain is also key to induce TRS breaking.

We compare AM radiation in Fig.\ref{fig:onsite} with different onsite energy. As shown in Fig.\ref{fig:onsite}(a), with exactly same onsite energy  $\epsilon_1=\epsilon_2$, no radiation comes out. To measure the asymmetry, we set $\epsilon_1=-\epsilon_2=\Delta\epsilon$. It's clear that AM flux is tuned by $\Delta\epsilon$ in Fig.\ref{fig:onsite}(b). In Fig.\ref{fig:onsite}(c)(d), we consider the spectrum of AM flux and photon radiation at different $\Delta\epsilon$. The spectrum is totally zero with $\Delta\epsilon=0$, both AM flux and photon radiation. The peak is slightly shifted when $\Delta\epsilon$.

% Then, we disscuss the spin-orbit coupling $t_{SO}$ in Fig.\ref{fig:so} vs other parameter $\Delta\epsilon, t, t_{IC}, \Gamma$. The $t_{SO}$ is more important than $\Delta\epsilon, t, \Gamma$. But $t_{IC}$ is an exception, without $t_{IC}$, the double helix splitted into two standalone single helix chains. Thus $t_{IC}$ is necessary.



\begin{figure}[ht]

\includegraphics[width=8.8cm]{figures/ponsite.pdf}

\caption{ (a) Angular momentum flux $J_A/J_0$ as a function of $\epsilon_1, \epsilon_2$ 
(b) AM flux dependence of $\Delta \epsilon = \epsilon_1 = -\epsilon_2$ with different energy step in $\Delta \epsilon = 0.1eV$ 
(c) AM flux spectrum with $\Delta \epsilon = 0/0.1/0.2/0.3/0.4eV$ 
(d) Photon radiation spectrum at different $\Delta \epsilon = 0/0.1/0.2/0.3/0.4eV$}
\label{fig:onsite}
\end{figure}




% \begin{figure}[ht]
% \includegraphics[width=8.8cm]{codes/prefigs/p2.pdf}

% \caption{  AM flux vs $t_{SO}$ with (a) different $\Delta \epsilon$ 
% (b) different $t=t_1=t_2 $
% (c) different $t_{IC} $
% (d) different $\Gamma/\Gamma_0 $.
% To avoid symmetric point, we set $\Delta \epsilon=0.2eV$ for (b)(c)(d).
% }
% \label{fig:so}
% \end{figure}

\newpage
\subsection{Spin-orbit coupling}
\begin{figure}[ht]
    \includegraphics[width=8.8cm]{figures/amspec_so1}% Here is how to import EPS art
    \includegraphics[width=8.8cm]{figures/amspec_so2}% Here is how to import EPS art
    \caption{ AM spectrum with different $t_{SO}$  in (a) $\Delta \epsilon = 0.1, \Gamma=0.01$ (b) $\Delta \epsilon = 0.1, \Gamma=0.4$ }
    \label{fig:so-amspec}
\end{figure}
We also plot AM flux spectrum with different $t_{SO}$ in Fig.\ref{fig:so-amspec}, the peaks are enchanced by $t_{SO}$ when electrode coupling is large .




\newpage
\subsection{Other parameters}


\begin{figure}[ht]
\includegraphics[width=8.8cm]{figures/nl.pdf}
\caption{ (a) AM flux vs different NL
(b) AMR spectrum in different NL
(c) Total TC vs different NL
(d) TC
}
\label{fig:nl}
\end{figure}

In Fig.\ref{fig:nl}, the relation between NL and AMR is almost linear. Also, between $T^C$ and AMR. 

\begin{figure}[ht]
\includegraphics[width=8.8cm]{figures/rg.pdf}
\caption{ (a) AMR vs different rg
(b) AMR spectrum in different rg
(c) AMR vs projection area
}
\label{fig:rg}
\end{figure}
In Fig.\ref{fig:rg}, we discuss rg. This is trivial, since electronic structure doesn't change with pure geometric parameter. The only quantity affect AMR is volecity matrix  $V_\mu$, which is propotional to rg. Consider the trace part $V_x g^< V_y g^>$, we can get the relation $J_A \propto r_g^2$ without numeric calculation. The relation between NL and AMR is linear. Also, between $T^C$ and AMR. 

\begin{figure}[ht]
\includegraphics[width=8.8cm]{figures/vc_ic.pdf}
\caption{ (a) AMR vs different $t_{IC}$
(b) AMR spectrum in different $t_{IC}$
(c) $T_C$ in $t_{IC}=0.1eV$ (d) $T_C$ in $t_{IC}=0.4eV$
}
\label{fig:vc_ic}
\end{figure}

\fi


\section{\label{sec:conclusion}Conclusion}
% In conclusion, we propose an extendable model to manipulate angular momentum radiation. We consider different parameters that can affect AM radiation.
% \begin{itemize}
%     \item Without SO, onsite and in-chain hopping must be unequal, at least one of them. In-chain hopping must be non-zero, to turn on radiation.
%     \item Inter-chain hopping is necessary, can greatly enhance radiation.
%     \item Spin-orbit coupling is not necessary when far from symmetric point, but can greatly enhance angular momentum radiation. 
% \end{itemize}
In summary, we have studied electrically driven AMR from helical chains using the nonequilibrium Green's function method. 
The ability of AMR is characterized by the imaginary part of a joint optical transition matrix element between scattering states originated from the two electrodes [Eq.~(\ref{eq:x2})]. 
We have made direct connection between the geometrical factors and the radiation properties. 
The most important property of this chiral-induced AMR is that it does not rely on the magnetic dipole transition moment, which is normally much smaller compared to the electric counterpart and hinders the radiation efficiency. Rather, the mechanism studied here relies on the electrical dipole transitions at two different directions, from filled to empty scattering states originated from two different electrodes. We have also shown the dependence of AMR on the tight-binding parameters and the coupling to electrodes. These parameters allow electrical engineering of molecule's AMR property. 
%This fully electrical way of generating AMR employing chiral molecules may find its usefulness in the development of chiral single molecule light sources.


\begin{acknowledgments}
We thank Prof. Jian-Sheng Wang (NUS) for valuable discussions. This work was supported by National Natural Science Foundation of China under Grants No. 22273029 and No. 21873033
\end{acknowledgments}

% \appendix

% \section{Relativistic cyclotron radiation}
% V. Epp derivated energy and angular momentum radiation from cyclotron (electron in cyclic motion).
% For each single harmonic, the flux of angular momemtum is
% \begin{equation}
% \frac{\mathrm{d} L_{n z}}{\mathrm{~d} t}=\frac{e^{2} n \omega}{c \beta}\left[2 \beta^{2} J_{2 n}^{\prime}(2 n \beta)-\left(1-\beta^{2}\right) \int_{0}^{2 n \beta} J_{2 n}(x) \mathrm{d} x\right].
% \end{equation}
% with $n$ the harmonic mode, $\beta$ rate of electron speed and light speed, $J_n$ first Bessel function. In low speed case, radiation focus on first mode, while in ultra-relativistic case, synchrotron radiation falls on very high harmonic numbers n as shown as Fig. \ref{fig:cyclotron-am} \cite{epp_angular_2019} .

% \begin{figure}[b]
% \subfigure{
%     \begin{minipage}{5cm}
%     \includegraphics[width=1.8in]{figures/cyclotron_am1}% Here is how to import EPS art
%     \end{minipage}
% }
% \subfigure{
%     \begin{minipage}{5cm}
%     \includegraphics[width=1.8in]{figures/cyclotron_am2}
%     \end{minipage}
% }
% \caption{ AM flux of cyclotron with mode n when (a) $\beta = 0.9$ (b) $\beta = 0.1$}
% \label{fig:cyclotron-am}
% \end{figure}



% The intensity of synchrotron radiation in mode $n$ is just the multiplier of $\omega$
% \begin{equation}
% \frac{\mathrm{d} W_{n}}{\mathrm{~d} t}
% =\omega \frac{\mathrm{d} L_{n z}}{\mathrm{~d} t}
% =\hbar \omega \frac{\mathrm{d} N_{n z}}{\mathrm{~d} t},
% \end{equation}
% meanwhile, each photon carries $n \hbar$ angular momentum and $n \hbar \omega$ energy. 

% Summing up over $n$ we obtain the total flux of angular momentum of synchrotron radiation, 
% \begin{equation}
% \frac{\mathrm{d} L_{z}}{\mathrm{~d} t}=\frac{2 e^{2} \gamma^{4} \beta^{2} \omega}{3 c},
% \end{equation}
% with $\gamma = (1-\beta^2)^{-1/2}$.

% Apply the formula into benzene case (radius is benzene edge length $a_0 = \SI{1.4}{\angstrom}$, electron speed $\beta = a_0 t / (c\hbar) \approx 0.001$, angular frequency $\omega = \beta c  / a_0 \approx  3.80 \times 10^{15} s^{-1}$), we get the flux of angular momentum thoroughly, 
% \begin{equation}
% \frac{\mathrm{d} L_{z}}{\mathrm{~d} t} \approx 1.28 \times 10^{-8} eV,
% \end{equation}
% which is slightly smaller than Zhang's benzene result $2.1 \times 10^ {-8}$ eV.

% But to implement such a cyclic motion under magnetic field, corresponding magnetic field strength should be impossibly large, 
% \begin{equation}
% B = m_e \omega / e = 2.16 \times 10^{4} T.
% \end{equation}
% But in such a molecule junction system, we can simply do this by applying voltage bias. This is our advantage.

% \section{\label{benzene}Benzene}

% We reproduce Zhang's benzene results as shown in Fig. \ref{fig:benzene} , but there's a little differnce near resonant peak. Our peak is sharp while Zhang's is plain. By analysing vortex current, we can clearily understand why the peak is sharp.

% Here, we set $\mu_R$ fixed at $-6eV$. By scanning $\mu_L$ we can get a sharp peak at $2.5eV$. We can see clearly how the vortex electron contribute AM radiation. When electron energy is slightly lower than $2.5eV$, the angular momentum is suddenly increasing to a high level, generating high AM radiation. After crossing $2.5eV$. Vortex electron got opposite angular momentum immediately, cancelling the contribution under $2.5eV$. That explains why AMR drop back and become a peak. Vortex electron angular momentum is discontinuous at $2.5eV$, so it is sharp peak.

% \begin{figure}[b]
% \subfigure{
%     \begin{minipage}{4cm}
%     \includegraphics[width=1.5in]{figures/benzene_am1}% Here is how to import EPS art
%     \end{minipage}
% }
% \subfigure{
%     \begin{minipage}{4cm}
%     \includegraphics[width=1.5in]{figures/benzene_am2}
%     \end{minipage}
% }

% \subfigure{
%     \begin{minipage}{4cm}
%     \includegraphics[width=1.5in]{figures/benzene_am3}
%     \end{minipage}
% }
% \subfigure{
%     \begin{minipage}{4cm}
%     \includegraphics[width=1.5in]{figures/benzene_am4}
%     \end{minipage}
% }
% \caption{ For benzene system, (a) $\mu_L$ dependence of AM near resonant level. Our result(blue solid line) and Zhang's result(red dashed line) are not in agreement at peak. Our result get a sharp peak.  (b) $\mu_L$ dependence of AM in full range (c) $\mu_L$ dependence of first order derivative of AM as $\mu_L$ (d) energy  dependence of vortex current $T^C$
% }
% \label{fig:benzene}
% \end{figure}


% The \nocite command causes all entries in a bibliography to be printed out
% whether or not they are actually referenced in the text. This is appropriate
% for the sample file to show the different styles of references, but authors
% most likely will not want to use it.
% \nocite{*}

% \bibliography{main}% Produces the bibliography via BibTeX.

\appendix
\section{Properties for X}
\label{sec:x_discuss}

In this section, we derive the properties for $X$ in Eq.~(\ref{eq:x2}) in the main text. Using the cyclic property of the trace, we get
\begin{equation}
    X^{\alpha\beta}_{\mu\nu}(E,E^-) = \operatorname{Tr} [V^\mu A^\alpha(E) V^\nu A^\beta(E^-)]=\operatorname{Tr}[V^\nu A^\beta(E^-) V^\mu A^\alpha(E)]= X^{\beta\alpha}_{\nu\mu}(E^-,E)
\end{equation}
By noting that $A^\alpha, V^\mu$ are Hermitian, we obtain another relationship by performing conjugate of $X$
\begin{equation}
X^{\alpha\beta}_{\mu\nu} (E,E^-) = \operatorname{Tr}[V^\mu A^\alpha V^\nu A^{\beta-}] =\operatorname{Tr} [V^{\nu}A^{\alpha}V^{\mu}A^{\beta-}]^* =[X^{\alpha\beta}_{\nu\mu}(E,E^-)]^*
\label{equ:X_conj}
\end{equation}
We use the above two equations and derive useful properties for $X$, which can help us to understand the results in the main text.

\subsection{Single helical chain with NN hopping}
\label{sec:zero_amr_nn}
The system under this case obeys particle-hole symmetry, from which we get one additional condition
% \subsection{Prove zero AMR for SSDNA NN hopping}

\begin{equation}
    g^r(-E)=g^a(E)
\end{equation}
In a physical sense, this means a process involving an energy $E$ and a time-reversed process involving an energy $-E$ are equivalent. Similarly, The spectral function is also symmetric
\begin{equation}
A^{\alpha\top}(E)=-A^\alpha(-E).
\end{equation}
The above results lead to 
\begin{equation}
\begin{aligned}
X^{LR}_{xy}(-E^-,-E)&= {\rm Tr}[V^xA^L(-E^-)V^yA^R(-E)]\\
&= {\rm Tr}[V^xA^{L\top}(E^-) V^y A^{R\top}(E)]\\
&= {\rm Tr}[V^xA^R(E)V^yA^L(E^-)]\\
&=X^{RL}_{xy}(E,E^-)
\end{aligned}
\end{equation}
This means that for any transition $E \to E^-$, the AMR contribution is exactly the opposite of the corresponding transition $-E^- \to -E$. If both transitions are allowed in the large bias limit, their AMR contribution cancels.



% Ingore subscript xy for convenience.

% Proof 0: Prove spectral function is same by reverse atom index,
% $$ A^L_{ij} =A^R_{i'j'}$$

% with $i'= n+1-i, \Gamma^{L}_{11}=\Gamma^{R}_{nn}=\Gamma.$ 

% Since 
% \begin{equation}
% H_{ij}=H_{i'j'},\Sigma_{ij}=\Sigma_{i'j'}
% \end{equation}
% So

% \begin{equation}
% g^r_{ij}=g^r_{i'j'}
% \end{equation}

% Finally,
% \begin{equation}
% A^L_{ij}= g^r_{i1} \Gamma^L_{11}g^a_{1j} \\
% =g^r_{i'n} \Gamma^L_{n}g^a_{nj'}=A^R_{i'j'}
% \end{equation}



% Proof 1: Prove trace of spectral functions multiplying any operator are the same.
% \begin{equation}
% Tr[A^L B]=Tr[A^R B]
% \end{equation}

% \begin{equation}
% A^L_{ij} B_{ji} = A^R_{i'j'} B_{j'i'}
% \end{equation}


% Proof 2: Prove imaginary part of spectral functions are opposite.
% Since total spectral funtion $A$ is real symmetric, we have
% \begin{equation}
% Im (A^L+A^R)=Im A = 0
% \end{equation}

% Proof 3: Prove imaginary part of X is opposite when switching electrodes  
% Consider X Since $V^\mu$ is always imaginary, elements in $V^x V^y$ are real. 
% \begin{equation}
% \begin{aligned}
% Im X^{LR}(E, E-\hbar\omega) &= Tr[V^x Re A^L V^y Im A^R+V^x Im A^L V^y Re A^R]\\
% &= -Tr[V^x Re A^R V^y Im A^L+V^x Im A^R V^y Re A^L]\\
% &=- Im X^{RL}(E, E-\hbar\omega)
% \end{aligned}
% \end{equation}
% Thus, $\Pi^{<,\alpha\beta}= \int \frac{d E}{2\pi} i X^{\alpha\beta} f^{\alpha}\overline{f^{\beta}}$ get opposite re part when switching electrodes, emit opposite AMR.

% Final proof: 
% For NN tight-binding, additional condition is reversal-symmetric GF (GF in second neighbor are not reversal-symmetric),
% \begin{equation}
% g^r(-E)=g^a(E)
% \end{equation}

% Prove total AMR is zero under symmetric chemical potential $\mu_L = - \mu_R = \mu_0.$

% Since
% \begin{equation}
% g^r(-E)=g^a(E)
% \end{equation}
% spectral function is also symmetric
% \begin{equation}
% A^{L\top}(E)=-A^L(-E)
% \end{equation}




% For any photon energy $\hbar\omega$, consider energy $\hbar\omega-E$
% \begin{equation}
% \begin{aligned}
% X^{LR}(\hbar\omega-E,-E)&=Tr[V^xA^L(\hbar\omega-E)V^yA^R(-E)]\\
% &=Tr[V^xA^{L\top}(E-\hbar\omega) V^y A^{R\top}(E)]\\
% &=Tr[V^xA^R(E)V^yA^L(E-\hbar\omega)]\\
% &=X^{RL}(E,E-\hbar\omega)
% \end{aligned}
% \end{equation}


% We can make energy pair $E,\hbar\omega-E$ in energy window,
% \begin{equation}
% Im X^{LR}(E,E-\hbar\omega)+X^{LR}(\hbar\omega-E,-E)= Im X^{LR}(E,E-\hbar\omega)+X^{RL}(E,E-\hbar\omega)=0
% \end{equation}

% Integrate over energy window, 
% \begin{equation}
% Re \Pi^{<,LR}= Re \int \frac{d E}{2\pi} i X^{LR} f^{L}\overline{f^{R}}=0
% \end{equation}

\subsection{Analysis of AMR for straight chain}
\label{sec:straight_chain}
For straight chain, $r^x_{ij}/r^y_{ij}$ is constant, so is $V^x/V^y$. That means $V^x$ and $V^y$ are switchable by multiplying a constant. Then, we have $X^{LR}_{xy}=X^{LR}_{yx}$. 
Combining with Eq. (\ref{equ:X_conj}), we get
\begin{equation}
X^{LR}_{xy} = (X^{LR}_{yx})^* = X^{LR}_{yx}
\end{equation}
So ${\rm Im} X^{LR}_{xy} =0$, meaning that AMR contribution $j_A$ is zero for straight chain. 

% \subsection{Rotational invariance of X}
% We define the rotation operator as follows,
% Real space ration about $z$ axis is represented by the rotational matrix
% \begin{align}
%     \left(
%     \begin{array}{c}
%     x' \\ y' \\ z'
%     \end{array}
%     \right)
%     =
%     \left(
%     \begin{array}{ccc}
%     \cos\theta & -\sin\theta & 0 \\
%     \sin\theta & \cos\theta & 0 \\
%     0 & 0 & 1
%     \end{array}
%     \right)
%     \left(
%     \begin{array}{c}
%     x \\ y \\ z
%     \end{array}
%     \right)
% \end{align}
% Since $V^{\mu}_{ij}\propto \mu_{ij} $, the transformation of ${\bf V}_{ij}$ follows the same form. 
% Thus,
% $$
% \begin{aligned}V^{x \prime} A V^{y \prime} B 
% &= (V^x \cos \theta-V^y \sin \theta) A (V^x \sin \theta+V^y \cos \theta )B  \\
% &=V^x A V^x B \cos \theta \sin \theta + V^x A V^y B \cos \theta \cos \theta \\
% &-V^y A V^x B \sin \theta \sin \theta + V^y A V^y B \cos \theta \sin \theta
% \end{aligned}
% $$
% Switch same $V$ in the first and last term, the imaginary part of trace must be zero.
% $$
% \begin{aligned}
% ImTrV^{x \prime} A V^{y \prime} B 
% &=ImTrV^x A V^y B \cos \theta \cos \theta +V^x A V^y B \sin \theta \sin \theta\\
% &=ImTrV^x A V^y B
% \end{aligned}
% $$
% Thus, $Im X$ is rotational invariant, so is $j_A$.

% $ReTrV^x A V^x B+V^y A V^y B$ 

% For the same procedure, we note  $x:V^x,y:V^y,s:\sin,c:\cos$ and ignore $AB$ currently, we get
% $$
% \begin{aligned}
% x' x' +y'y'  
% &=xcxc-cxys-ysxc+ysys+xsxs+xsyc+ycxs+ycyc\\
% &=xcxc+ysys+xsxs+ycyc\\
% &=xx+yy
% \end{aligned}
% $$
% Recover from short note, 
% $$
% \begin{aligned}
% ReTrV^{x\prime} A V^{x\prime} B+V^{y\prime} A V^{y\prime} B =ReTrV^x A V^x B+V^y A V^y B 
% \end{aligned}
% $$

%\section{Resonant effect for non-degenerate levels}
%In weak coupling limit, spectral function $A_\alpha$ focus on level $E_m$, real part of diagonal element $Re A^\alpha_{mm}$ will be much larger than the other matrix elements, and the other matrix elements can be ignored and the following approximation is made,
%
%\begin{align}
%\centering
%Re A^\alpha_{mm} &\approx \frac{\tilde{\Gamma}^\alpha_{mm}} {  (E-E_m)^2 + 1/4 \tilde{\Gamma}^2_{mm}} \\
%j_{N} &\approx \sum_{m,n,m\neq n}  W_{N,mn}  \frac{\tilde{\Gamma}^L_{mm}} {  (E-E_m)^2 + 1/4 \tilde{\Gamma}^2_{mm}}  \frac{\tilde{\Gamma}^R_{nn}} {  (E^- -E_n)^2 + 1/4 \tilde{\Gamma}^2_{nn}} 
%\end{align}
%where, $W_{N}=\sum_{\mu=x,y,z} V^2_\mu$. The result of $j_N$ shows a star pattern on the 2D plot, that coincides with Fig.~\ref{fig:dna_elec}(g). For specific transition $E_m \to E_n$, we can get the peak value
%
%
%\begin{align}
%j_{N,mn}(E_m,E_n) \approx 16 W_{N,mn}  \frac{\tilde{\Gamma}^L_{mm} \tilde{\Gamma}^R_{nn}} { \tilde{\Gamma}^2_{mm}\tilde{\Gamma}^2_{nn}} 
%\end{align}
%
%

\bibliographystyle{apsrev4-2}
\bibliography{main}

\end{document}
%
% ****** End of file apssamp.tex ******
