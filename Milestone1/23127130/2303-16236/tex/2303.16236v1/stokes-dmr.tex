\documentclass[a4paper]{scrartcl}


% \usepackage{graphicx}
% \graphicspath{{figures/}}

\usepackage{amsmath}
\usepackage{amssymb}
\usepackage[hypertexnames=false]{hyperref}
\usepackage{cleveref}
\usepackage{autonum}
% \usepackage[color]{showkeys}
% \usepackage{refcheck}
\usepackage[vmargin=3cm]{geometry}
\usepackage[svgnames,dvipsnames]{xcolor}
\usepackage{enumitem}
\usepackage{amsthm}
\usepackage{mathtools}
\usepackage{mleftright}
\usepackage{bm}
\usepackage{xparse}

\hypersetup{
    colorlinks=true,
    linkcolor=RoyalBlue,
    citecolor=Green
}

\colorlet{refkey}{pink!90!red}
\colorlet{labelkey}{JungleGreen!80!yellow}

\newcommand{\crefrangeconjunction}{--}
\newtheorem{theorem}{Theorem}[section]
\crefname{theorem}{Theorem}{Theorems}
\newtheorem{proposition}{Proposition}[section]
\crefname{proposition}{Proposition}{Propositions}
\newtheorem{lemma}{Lemma}[section]
\crefname{lemma}{Lemma}{Lemmas}
\newtheorem{corollary}{Corollary}[section]
\crefname{corollary}{Corollary}{Corollaries}

\theoremstyle{definition}
\newtheorem{remark}{Remark}[section]
\crefname{remark}{Remark}{Remarks}
\newtheorem{definition}{Definition}
\crefname{definition}{Definition}{Definitions}

\crefname{section}{Section}{Sections}

\numberwithin{equation}{section}





\DeclarePairedDelimiterXPP{\inner}[2]{}{\lparen}{\rparen}{}{#1,#2}
\DeclarePairedDelimiterXPP{\innerh}[2]{}{\lparen}{\rparen}{_h}{#1,#2}
\DeclarePairedDelimiterXPP{\stinner}[2]{}{\lbrack}{\rbrack}{}{#1,#2}
\DeclarePairedDelimiterXPP{\stinnerh}[2]{}{\lbrack}{\rbrack}{_h}{#1,#2}

\DeclarePairedDelimiterXPP{\norm}[2]{}{\lVert}{\rVert}{_{#2}}{#1}
\DeclarePairedDelimiterXPP{\Lpnorm}[3]{}{\lVert}{\rVert}{_{L^{#2}({#3})}}{#1}
\DeclarePairedDelimiterXPP{\LpLqnorm}[5]{}{\lVert}{\rVert}{_{L^{#2}({#3}; L^{#4}({#5}))}}{#1}


\DeclarePairedDelimiter{\abs}{\lvert}{\rvert}
\DeclarePairedDelimiter{\paren}{\lparen}{\rparen}
\DeclarePairedDelimiter{\bracket}{\lbrack}{\rbrack}

% just to make sure it exists
\providecommand\given{}
% can be useful to refer to this outside \Set
\newcommand\SetSymbol[1][]{%
    \nonscript\:#1\vert
    \allowbreak
    \nonscript\:
    \mathopen{}
}
\DeclarePairedDelimiterX{\Set}[1]{\{}{\}}{%
    \renewcommand\given{\SetSymbol[\delimsize]}
    #1
}



\DeclareMathOperator{\dv}{div}
\DeclareMathOperator{\supp}{supp}
\DeclareMathOperator{\dist}{dist}

\newcommand{\Vhs}{V_{h,\sigma}}
\newcommand{\Qhm}{Q_h^0}
\newcommand{\Bhs}{B^{p'}_{h,\sigma}}


\newcommand{\projs}{P_{h,\sigma}}

\newcommand{\cTh}{\mathcal{T}_h}
\newcommand{\Omegah}{{\Omega_h}}
\newcommand{\skin}{{\Omega\setminus\Omegah}}

\newcommand{\bC}{\mathbb{C}}
\newcommand{\bR}{\mathbb{R}}
\newcommand{\bN}{\mathbb{N}}
\newcommand{\bZ}{\mathbb{Z}}
\newcommand{\cT}{\mathcal{T}}

\newcommand{\intervalq}{\paren*{\frac{2N}{N+2},\frac{2N}{N-2}}}

\allowdisplaybreaks



\title{Discrete maximal regularity for the finite element approximation of the Stokes operator and its application}

\author{Tomoya Kemmochi%
    \thanks{%
    Graduate School of Engineering, Nagoya University.
    % \\
    Email: \texttt{kemmochi@na.nuap.nagoya-u.ac.jp}, Web: \url{https://t-kemmochi.github.io/en/}
    }
}

% \author{Tomoya Kemmochi}

\hypersetup{
    pdftitle={Discrete maximal regularity for the finite element approximation of the Stokes operator and its application},
    pdfauthor={T. Kemmochi}
}

\begin{document}
\mleftright

\maketitle

\begin{abstract}
    Maximal regularity for the Stokes operator plays a crucial role in the theory of the non-stationary Navier--Stokes equations.
    In this paper, we consider the finite element semi-discretization of the non-stationary Stokes problem and establish the discrete counterpart of maximal regularity in $L^q$ for $q \in \intervalq$.
    For the proof of discrete maximal regularity, we introduce the temporally regularized Green's function.
    With the aid of this notion, we prove discrete maximal regularity without the Gaussian estimate.
    As an application, we present $L^p(0,T;L^q(\Omega))$-type error estimates for the approximation of the non-stationary Stokes problem.
\end{abstract}





\section{Introduction}
\label{sec:intro}

In this paper, we consider the finite element approximation of the non-stationary Stokes problem
\begin{equation}
    \begin{cases}
        \partial_t u - \Delta u + \nabla \varphi = f, & \text{in } \Omega \times J, \\
        \dv u = 0, & \text{in } \Omega \times J, \\
        u = 0, & \text{on } \partial\Omega \times J, \\
        u(0) = 0, & \text{in } \Omega,
    \end{cases}
    \label{eq:Stokes}
\end{equation}
where $\Omega \subset \bR^N$ ($N \ge 2$) is a bounded convex domain with smooth boundary, 
$J=(0,T) \subset \bR$ is an interval with $T>0$, 
$u \colon \Omega \times J \to \bR^N$ is an unknown velocity vector,
$\varphi \colon \Omega \times J \to \bR$ is an unknown pressure,
and $f \in L^p(J; L^q(\Omega)^N)$ is a given vector field with $p,q \in (1,\infty)$.
Throughout this paper, we write $v(t) = v(\cdot,t)$ ($t \in J$) for a space-time function $v$.
Let us moreover define the Stokes operator $A = A_q$ on $L^q(\Omega)^N$ by
\begin{equation}
    D(A) \coloneqq W^{2,q}(\Omega)^N \cap W^{1,q}_0(\Omega)^N \cap L^q_\sigma(\Omega),
    \qquad 
    Av = - P_\sigma \Delta v, \quad v \in D(A),
\end{equation}
where $L^q_\sigma(\Omega)$ is the space of solenoidal vector fields in $L^q(\Omega)^N$ and $P_\sigma \colon L^q(\Omega)^N \to L^q_\sigma(\Omega)$ is the Helmholtz projection.
Throughout this paper, we do not distinguish $A_q$ for different $q$'s and we just denote it by $A$.
Then, the Stokes problem \eqref{eq:Stokes} is expressed as an abstract Cauchy problem
\begin{equation}
    \begin{cases}
        \partial_t u(t) + Au(t) = P_\sigma f(t), & t \in J, \\
        u(0) = 0
    \end{cases}
    \label{eq:Stokes-Cauchy}
\end{equation}
in $L^q_\sigma(\Omega)$.



It is well-known that $u$ and $\varphi$ satisfy the maximal regularity estimate
\begin{equation}
    \LpLqnorm{\partial_t u}{p}{J}{q}{\Omega}
    + \LpLqnorm{\nabla^2 u}{p}{J}{q}{\Omega}
    + \LpLqnorm{\nabla \varphi}{p}{J}{q}{\Omega}
    \le C \LpLqnorm{f}{p}{J}{q}{\Omega}
    \label{eq:mr}
\end{equation}
for $p,q \in (1,\infty)$ (see, e,g.,~\cite{zbMATH03627669,MR2602914}).
The estimate \eqref{eq:mr} is also described by
\begin{equation}
    \LpLqnorm{\partial_t u}{p}{J}{q}{\Omega}
    + \LpLqnorm{A u}{p}{J}{q}{\Omega}
    \le C \LpLqnorm{f}{p}{J}{q}{\Omega},
    \label{eq:mr-operator}
\end{equation}
and thus it is understood as a property of the Stokes operator $A$.

Maximal regularity plays an important role in the theory of the non-stationary Navier--Stokes equations.
For example, in \cite{zbMATH03627669}, the local well-posedness of the Navier--Stokes equations are derived via the maximal regularity.
Moreover, it is known that, if the weak solution of the three-dimensional Navier--Stokes equations in the sense of Leray--Hopf belongs to $L^p(J; L^q(\Omega)^3)$ with $p,q$ satisfying
\begin{equation}
    \frac{2}{p} + \frac{3}{q} = 1, \qquad q \in [3,\infty],
\end{equation}
then the weak solution is the unique strong solution \cite{MR0150444,MR767409,MR762786,MR635201,MR1992563}.
This class is known as the Serrin class today.
It is suggested that $L^p(J; L^q(\Omega)^N)$-norm may play an important role in the theory of the Navier--Stokes equations.


In this paper, we consider the finite element semi-discretization of the Stokes equations \eqref{eq:Stokes}.
Let $\cTh$ be a shape-regular and quasi-uniform triangulation of $\Omega$, 
$\Omegah \subset \Omega$ be a polygonal approximation of $\Omega$ consisting of elements in $\cTh$, and
$(V_h, Q_h) \subset H^1_0(\Omegah)^N \times H^1(\Omegah)$ be a pair of conforming finite element spaces associated to $\cTh$ with suitable assumptions.
Then, finite element semi-discretization of \eqref{eq:Stokes} is formulated as follows.
Find $u_h \colon J \to V_h$ and $\varphi_h \colon J \to Q_h$ that satisfy
\begin{equation}
    \begin{cases}
        \innerh{\partial_t u_h}{v_h} + \innerh{\nabla u_h}{\nabla v_h} -\innerh{\varphi_h}{\dv v_h} = \innerh{f}{v_h}, & \forall v_h \in V_h,\\
        \innerh{\dv u_h}{\psi_h} = 0, & \forall \psi_h \in Q_h, \\
        u_h(0) = 0,
    \end{cases}
    \label{eq:disc-Stokes}
\end{equation}
where $\innerh{\cdot}{\cdot}$ is the $L^2$-inner product over $\Omegah$.
We moreover denote the discrete counterpart of the Stokes operator $A$ by $A_h$, which are defined in the subsequent section.


The aim of this paper is to derive the finite element counterpart of the maximal regularity \eqref{eq:mr} and \eqref{eq:mr-operator}, namely, 
\begin{equation}
    \LpLqnorm{\partial_t u_h}{p}{J}{q}{\Omegah} + \LpLqnorm{A_h u_h}{p}{J}{q}{\Omegah} 
    + \LpLqnorm{\nabla \varphi_h}{p}{J}{q}{\Omegah}
    \le C \LpLqnorm{f}{p}{J}{q}{\Omega},
    \label{eq:dmr-intro}
\end{equation}
with $C$ independent of $h$ and $f$, where $u_h$ and $\varphi_h$ are the solution of \eqref{eq:disc-Stokes}.
We succeeded in showing \eqref{eq:dmr-intro} for $p \in (1,\infty)$ and $q \in \intervalq$.
The restriction for $q$ stems from the result of \cite{Kemmochi-arxiv}, in which the author proved the $L^q$-resolvent estimate for $A_h$ when $q \in \intervalq$.
As an application of \eqref{eq:dmr-intro}, we show error estimates for $u_h$ and $\varphi_h$ in the $L^p(J;L^q(\Omegah))$-type norms.




The result of the present paper is the first result on the estimate of the form \eqref{eq:dmr-intro} for $q \ne 2$, to the best of our knowledge.
When $q = 2$, $A_h$ has maximal regularity in $L^2$ since $A_h$ generates an analytic semigroup and $L^2$ is a Hilbert space. 
However, for general $q \ne 2$, a different approach is necessary, as in the literature on $L^q$-theory of finite element methods.
For the parabolic case, discrete counterpart of the maximal regularity is well-studied for both spatial and temporal discretization (e.g.~\cite{MR2218965,MR3395142,MR3614012,MR3549514,MR3854049,MR4081914,MR4410757,MR3606467,MR3582825,MR3620143,MR4368992}).
To establish the spatially discrete maximal regularity, the Gaussian estimate for the parabolic Green's function plays an essential role (see e.g.~\cite{MR2218965,MR3395142,MR4081914}).
For the Stokes case, however, such an estimate is unavailable since the Stokes operator does not generate an analytic semigroup in $L^1$ (see \cite{MR1838323}).
Because of this difficulty, \eqref{eq:dmr-intro} for $q \ne 2$ has not been established, and only $L^q$-resolvent estimate for $A_h$ was established recently \cite{Kemmochi-arxiv}.



It would be possible to apply our result to the fully discretized case.
Indeed, an $L^\infty(J;L^2)$-error estimate for full discretization of the non-stationary Stokes problem was recently established in \cite{BVL22}.
To show the error estimate, the authors used maximal regularity for $A_h$ in $L^2$.
Thus our result makes it possible to generalize the results in the $L^q$-setting.
Namely, an $L^\infty(J;L^q)$-error estimate would be obtained for $q \in \intervalq$.



Moreover, it is expected that the discrete maximal regularity \eqref{eq:dmr-intro} can be applied to numerical analysis of the non-stationary Navier--Stokes equations.
Indeed, for the parabolic problems, the results of discrete maximal regularity mentioned above are successfully applied to numerical analysis of nonlinear parabolic equations (e.g.~\cite{MR2350187,MR3683415,MR3778340,MR3857906,MR4246878}).
Therefore, it would be interesting to proceed numerical analysis for the Navier--Stokes problem in this direction.

As possible applications of our result, we would like to introduce related studies.
In \cite{MR2461254}, it is shown that $A_h$ is stable in fractional Sobolev spaces.
This result is applied to show boundedness in the space-time fractional Sobolev space of the numerical solution to the Navier--Stokes equations in \cite{MR2334774}.
In these estimates, fractional Sobolev spaces of $L^2$-type are addressed, since $L^q$-type estimates for the non-stationary Stokes problems were not developed at that time.
Therefore, it would be possible to refine the results by using our estimate \eqref{eq:dmr-intro}.
In \cite{MR4215339}, it is shown that, if a numerical solution of the three-dimensional Navier--Stokes problem is bounded in $L^\infty(J; L^4(\Omega))$, then there exists a unique strong solution to the original problem.
Although the setting of discretization is different from ours, discrete maximal regularity may be helpful to show the boundedness assumption.
We here remark that the required regularity is stronger than the Serrin class.


The outline of the proof of \eqref{eq:dmr-intro} is totally different from that of the parabolic problems.
In the literature on (spatially) discrete maximal regularity for parabolic equations, the estimate is reduced to an $L^1$-type estimate of the so-called regularized Green's function (cf.~\cite{MR2218965,MR3395142,MR4081914}),
and to show such an estimate, the Gaussian estimate for the parabolic Green's function is used as mentioned above, which is unavailable in our case.

Instead of this, we introduce the ``temporally regularized'' Green's function.
Let $\eta = \eta_{a,\varepsilon} \in C^\infty_0(J)$ be an $\varepsilon$-approximation of the delta function with respect to a time $a \in J$.
Then, we consider the dual Stokes problem with an external force $\eta \phi_h$ for arbitrary $\phi_h \in \Vhs$, where $\Vhs$ is the discrete counterpart of $L^p_\sigma(\Omega)$ (precise definition will be given later).
We call the solution to this problem the temporally regularized Green's function and denote it by $g = g_{a,\varepsilon}$.
We also consider the finite element approximation of $g$ and denote it by $g_h$.
Then, the desired estimate \eqref{eq:dmr-intro} is reduced to a weighted estimate for $g-g_h$.
In order to show the weighted estimate, we use the $L^q$-resolvent estimate for $A_h$ repeatedly, and this is why we assume $q \in \intervalq$.




We finally mention the temporal discretization.
There are several results stating that, roughly speaking, (continuous) maximal regularity implies temporally discrete maximal regularity (cf.~\cite{MR1299329,MR1853519,MR3582825,MR3549514,MR3606467,MR4410757,MR4368992}).
Since some of these results are established in an abstract setting, it is possible to establish full discretization of maximal regularity for the Stokes problem straightforwardly.


The remainder of the present paper is as follows.
In Section~\ref{sec:main}, we present the precise settings of this study and state our main results.
In Section~\ref{sec:preliminaries}, we collect some preliminary estimates related to the discrete Stokes operator and the discrete Helmholtz projection.
The proof of discrete maximal regularity \eqref{eq:dmr-intro} is given in Section~\ref{sec:proof-dmr}, and finally we show error estimates for \eqref{eq:disc-Stokes} in Section~\ref{sec:proof-error}.


\subsection*{Notation}

Throughout this paper, $C$ denotes general constants that are independent of parameters such as $h$.
The value of $C$ may be different at each appearance.
For $p \in [1,\infty]$, we denote the H\"older conjugate by $p' = p/(p-1)$.
The Lebesgue and the Sobolev spaces are denoted by $L^p(\Omega)$ and $W^{s,p}(\Omega)$, respectively, for $p \in [1,\infty]$ and $s > 0$.
Moreover, $L^p_\sigma(\Omega)$, $L^p_0(\Omega)$, and $W^{1,p}_0(\Omega)$ are as usual 
and $W^{-1,p}(\Omega)$ is the dual of $W^{1,p'}_0(\Omega)$.



\section{Main result}
\label{sec:main}

In this section, we will present our main result.
We start by giving the settings of this study.
Let $\Omega \subset \bR^N$ ($N \ge 2$) be a bounded convex domain with smooth boundary, 
$\{ \cT_h \}_h$ be a family of triangulations of $\Omega$,
and $\Omegah \subset \Omega$ be a polygonal approximation of $\Omega$ consisting of elements in $\cTh$.
For $v \in W^{1,1}_0(\Omegah)$, we extend $v$ by zero outside $\Omegah$ and we identify $v \in W^{1,1}_0(\Omega)$.
We then let $V_h \subset H^1_0(\Omegah)^N$ and $Q_h \subset H^1(\Omegah)$ be the conforming finite element spaces for the velocity and the pressure, respectively.


As in \cite{Kemmochi-arxiv}, we assume the following hypotheses. 
Throughout this paper, $\innerh{\cdot}{\cdot}$ denotes the $L^2$-inner product over $\Omegah$.
\begin{enumerate}[label=\textbf{(H\arabic*)},ref=\upshape{\color{black}(H\arabic*)},leftmargin=*,series=assumption]
    \item\label{assum:inv}
    We assume that the family $\{ \cTh \}_h$ is shape-regular and quasi-uniform. Namely, there exists $C>0$ such that 
    $C h \le \rho_T$ for all $T \in \cTh$ and $h>0$, where $\rho_T$ is the radius of the maximum inscribed ball.
    % 
    \item\label{assum:Ih}
    There exists a quasi-interpolation operator $I_h \colon W^{1,1}(\Omega)^N \to V_h$ that satisfies the following properties:
    \begin{enumerate}[label=(\roman*)]
        \item For all $v \in W^{1,1}(\Omega)^N $,
        \begin{equation}
            \innerh{\dv (v - I_h v)}{\psi_h} = \inner{v \cdot n}{\hat{\psi}_h}_{\partial\Omegah}, \quad \forall \psi_h \in Q_h,
            \label{eq:preserve-div}
        \end{equation}
        where $\hat{\psi}_h \in L^\infty(\Omegah)$ is a function determined by $\psi_h$ satisfying $\supp \hat{\psi}_h = \supp \psi_h$ and
        \begin{equation}
            \Lpnorm{\hat{\psi}_h}{p}{\partial\Omegah} \le C h^{-\frac{1}{p}} \Lpnorm{\psi_h}{p}{\Omegah}.
            \label{eq:preserve-div-hat}
        \end{equation}  
        % 
        \item For each $T \in \cTh$, there exists a macro-element $\Delta_T$ including $T$ that satisfies 
        \begin{align}
            \Lpnorm{v - I_h v}{p}{T} &\le C h \Lpnorm{\nabla v}{p}{\tilde{\Delta}_T}, \label{eq:int-err-v-Lp}\\
            \Lpnorm{\nabla I_h v}{p}{T} &\le C \Lpnorm{\nabla v}{p}{\tilde{\Delta}_T}, \label{eq:int-stab-v} \\
        \shortintertext{for all $p \in [1,\infty]$ and $v \in W^{1,p}_0(\Omega)^N$, and }
            \Lpnorm{v - I_h v}{p}{T}  &\le C h^2 \norm{\nabla v}{W^{1,p}(\tilde{\Delta}_T)},  \label{eq:int-err-v-Lp-h2} \\ 
            \Lpnorm{\nabla (v - I_h v)}{p}{T} &\le C h^2 \norm{\nabla v}{W^{1,p}(\tilde{\Delta}_T)},  \label{eq:int-err-v-W1p}
        \end{align}
        if $v \in W^{2,p}(\tilde{\Delta}_T)^N$ in addition, where 
        \begin{equation}
            \tilde{\Delta}_T = \Delta_T \cup \Set*{x \in \Omega\setminus\Omegah \given \operatorname*{argmin}_{S \in \cTh} \dist(x,S) \subset \Delta_T}
        \end{equation}
        and each $\Delta_T$ includes at most $L$ elements of $\cTh$ with $L$ independent of $h,T$.
    \end{enumerate}
    % 
    \item\label{assum:Jh}
    There exists a quasi-interpolation operator $J_h \colon W^{1,1}(\Omega) \to Q_h$ that satisfies the following estimates. 
    For each $T \in \cTh$, there exists a macro-element $\Delta_T$ including $T$ that satisfies 
    \begin{align}
        \Lpnorm{\psi - J_h \psi}{p}{T} &\le C h \Lpnorm{\nabla \psi}{p}{\Delta_T}, \label{eq:int-err-p-Lp}\\
        \Lpnorm{\nabla J_h \psi}{p}{T} &\le C \Lpnorm{\nabla \psi}{p}{\Delta_T}, \label{eq:int-stab-p} \\
        \Lpnorm{\nabla (\psi - J_h \psi)}{p}{T} &\le C h \Lpnorm{\nabla^2 \psi}{p}{\Delta_T}, \quad \text{if } \psi \in W^{2,p}(\Delta_T) 
    \end{align}
    for all $p \in [1,\infty]$ and any $\psi \in W^{1,p}(\Delta_T)$, where $\Delta_T$ includes at most $L$ elements of $\cTh$ with $L$ independent of $h,T$.
    % 
    \item\label{assump:tIh}
    There exists another quasi-interpolation operator $\tilde{I}_h \colon C^0(\overline{\Omegah})^N \to V_h$ that satisfies
    \begin{enumerate}[label=(\roman*)]
        \item For all $v \in C^0(\overline{\Omegah})^N \cap W^{1,1}_0(\Omegah)^N$,
        \begin{equation}
            \innerh{\dv (v - \tilde{I}_h v)}{\psi_h} = 0, \quad \forall \psi_h \in Q_h.
            \label{eq:preserve-div-2}
        \end{equation} 
        \item For each $T \in \cTh$, there exists a macro-element $\Delta_T$ including $T$ that satisfies 
        \begin{equation}
            \Lpnorm{\nabla \tilde{I}_h v}{p}{T} \le C \Lpnorm{\nabla v}{p}{\Delta_T}, \quad \forall v \in W^{1,p}(\Delta_T)^N, 
            \label{eq:int-stab-v-2} 
        \end{equation}
        for all $p \in [1,\infty]$, where $\Delta_T$ includes at most $L$ elements of $\cTh$ with $L$ independent of $h,T$.
    \end{enumerate} 
    % 
    \item\label{assum:SA}
    The above operators $\tilde{I}_h$ and $J_h$ satisfy the following super-approximation properties.
    Let $D \subset \Omegah$ be a subdomain and let 
    \begin{equation}
        D_d \coloneqq \Set{x \in \Omegah \given \dist(x,D) < d}
    \end{equation}
    for $d>0$. Furthermore, let $\omega \in C^\infty(\Omegah)$ be a cut-off function that satisfies
    \begin{equation}
        \supp\omega \subset D_d, \quad 0 \le \omega \le 1, \quad \omega|_D \equiv 1, \quad \abs{\nabla \omega} \le Cd^{-1}.
    \end{equation}
    Then, there exists $c_0 > 0$ such that for all $d \ge c_0 h$, $v_h \in V_h$ and $\psi_h \in Q_h$, 
    \begin{align}
        \norm{\omega^2 v_h - \tilde{I}_h(\omega^2 v_h)}{D_{2d}} &\le C hd^{-1} \norm{v_h}{D_{3d}}, \\ 
        \norm{\nabla \paren{\omega^2 v_h - \tilde{I}_h(\omega^2 v_h)}}{D_{2d}} &\le C d^{-1} \norm{v_h}{D_{3d}},  \\
        \norm{\omega^2 \psi_h - J_h(\omega^2 \psi_h)}{D_{2d}} &\le C hd^{-1} \norm{\psi_h}{D_{3d}}, 
    \end{align}
    where $c_0$ and $C$ are independent of the domain $D$ and parameters $d,h$.
\end{enumerate}

We refer the reader to \cite{Kemmochi-arxiv} for the construction of the above interpolation operators.
We notice that \eqref{eq:int-err-v-Lp-h2} is not assumed in \cite{Kemmochi-arxiv}; however, it holds true at least for the Taylor--Hood and MINI elements (see \cite[Appendix]{Kemmochi-arxiv}).
By assumption \ref{assum:inv}, the inverse inequality
\begin{equation}
    \Lpnorm{\nabla v_h}{p}{T} \le C h^{-1} \Lpnorm{v_h}{p}{T}
    \label{eq:inverse-W1p} 
\end{equation}
holds for all $p \in [1,\infty]$, $v_h \in V_h$, and $T \in \cTh $.








Let us further introduce the discrete counterparts of $L^p_\sigma(\Omega)$, the Helmholtz projection, and the Stokes operator as follows.
\begin{definition}\label{def:discrete}
    \begin{enumerate}[label=\upshape{(\roman*)}]
        \item We define the space $\Vhs \subset V_h$ by 
        \begin{equation}
            \Vhs \coloneqq \Set{v_h \in V_h \given \innerh{\dv v_h}{\psi_h} = 0, \, \forall \psi_h \in Q_h}.
        \end{equation}
        \item We define the discrete Helmholtz projection $\projs \colon L^p(\Omega)^N \to \Vhs$ by
        \begin{equation}
            \innerh{\projs v}{w_h} = \innerh{v}{w_h}, \qquad \forall w_h \in \Vhs 
        \end{equation} 
        for each $v \in L^p(\Omega)^N$ and $p \in (1,\infty)$
        \item We define the discrete Stokes operator $A_h \colon \Vhs \to \Vhs$ by 
        \begin{equation}
            \innerh{A_h v_h}{w_h} = \innerh{\nabla v_h}{\nabla w_h}, \qquad \forall w_h \in \Vhs,
        \end{equation}
        for each $v_h \in \Vhs$.
    \end{enumerate}
\end{definition}

Then, the finite element approximation of the non-stationary Stokes equations \eqref{eq:Stokes} is the following.
\begin{center}
    Find $u_h \in C^0(\bar{J}; V_h) \cap C^1(J; V_h)$ and $\varphi_h \in L^2(J; Q_h)$ that satisfy \eqref{eq:disc-Stokes}.
\end{center}
The solution $u_h$ and $\varphi_h$ satisfy the Galerkin orthogonality (compatibility)
\begin{equation}
    \begin{cases}
            \innerh{\partial_t(u-u_h)}{v_h} + \innerh{\nabla(u-u_h)}{\nabla v_h} - \innerh{\varphi-\varphi_h}{\dv v_h} = 0, & \forall v_h \in V_h, \\
            \innerh{\dv(u-u_h)}{\psi_h} = 0, & \forall \psi_h \in Q_h.
    \end{cases}
    \label{eq:Galerkin-orthogonality}
\end{equation}
Moreover, with the notation of \cref{def:discrete}, equation \eqref{eq:disc-Stokes} can be described as the Cauchy problem
\begin{equation}
    \begin{cases}
        \partial_t u_h + A_h u_h = \projs f, \\
        u_h(0) = 0.
    \end{cases}
    \label{eq:disc-Stokes-Cauchy}
\end{equation}
Now, our main result, namely maximal regularity for $A_h$, is stated as follows.

\begin{theorem}\label{thm:main}
    Let $p \in (1,\infty)$, $q \in \intervalq$, and $f \in L^p(J; L^q(\Omega)^N)$.
    Assume that hypotheses \ref{assum:inv}--\ref{assum:SA} hold.
    Then, the solution $(u_h,\varphi_h)$ of the discrete Stokes problem \eqref{eq:disc-Stokes} satisfies 
    \begin{equation}
        \LpLqnorm{\partial_t u_h}{p}{J}{q}{\Omegah} + \LpLqnorm{A_h u_h}{p}{J}{q}{\Omegah} 
        \le C \LpLqnorm{f}{p}{J}{q}{\Omega},
        \label{eq:dmr}
    \end{equation}
    where $C$ is independent of $h$ and $f$.
\end{theorem}

\begin{remark}
    The restriction $q \in \intervalq$ stems from the assumption to show the analyticity of the discrete Stokes semigroup $e^{-tA_h}$ in $L^q$.
    See \cite[Theorem~2.1]{Kemmochi-arxiv} and \cref{lem:semigroup} below.
\end{remark}

As an application of the discrete maximal regularity \eqref{eq:dmr}, we obtain the following error estimates.
\begin{theorem}\label{thm:error}
    Let $(u,\varphi)$ be the solution of \eqref{eq:Stokes}. 
    Then, under the same hypotheses as in \cref{thm:main}, we have
    \begin{align}
        \LpLqnorm{u-u_h}{p}{J}{q}{\Omegah} &\le C h^2 \LpLqnorm{f}{p}{J}{q}{\Omega}, \label{eq:error-Lp} \\ 
        \LpLqnorm{\nabla (u-u_h)}{p}{J}{q}{\Omegah} &\le C h \LpLqnorm{f}{p}{J}{q}{\Omega}, \label{eq:error-W1p} \\ 
        \norm{\partial_t (u-u_h)}{W^{-1,p}(J; L^q(\Omegah))} &\le C h \LpLqnorm{f}{p}{J}{q}{\Omega}, \label{eq:error-W-1p} \\ 
        \LpLqnorm{J_h \varphi-\varphi_h}{p}{J}{q}{\Omegah} &\le C h \LpLqnorm{f}{p}{J}{q}{\Omega}, \label{eq:error-zeta-Lp}
    \end{align}
    where $C$ is independent of $h$ and $f$.
\end{theorem}

The estimate for the pressure \eqref{eq:error-zeta-Lp} immediately implies the error estimate for the pressure and the a priori estimate for $\nabla \varphi_h$.
\begin{corollary}
    Under the same hypotheses as in \cref{thm:main}, we have
    \begin{equation}
        \LpLqnorm{\varphi-\varphi_h}{p}{J}{q}{\Omegah} \le C h \LpLqnorm{f}{p}{J}{q}{\Omega} \label{eq:error-pressure-Lp}
    \end{equation}
    and
    \begin{equation}
        \LpLqnorm{\nabla \varphi_h}{p}{J}{q}{\Omegah}
        \le C \LpLqnorm{f}{p}{J}{q}{\Omega},
        \label{eq:dmr-pressure}
    \end{equation}
    where $C$ is independent of $h$ and $f$.
\end{corollary}

\begin{proof}
    The error estimate \eqref{eq:error-pressure-Lp} is an immediate consequence of \eqref{eq:error-zeta-Lp}, \eqref{eq:int-err-p-Lp}, and \eqref{eq:mr}.
    Moreover, by \eqref{eq:inverse-W1p}, \eqref{eq:int-stab-p}, \eqref{eq:error-zeta-Lp}, and \eqref{eq:mr}, we obtain
    \begin{align}
        \LpLqnorm{\nabla \varphi_h}{p}{J}{q}{\Omegah}
        &\le \LpLqnorm{\nabla (\varphi_h - J_h \varphi)}{p}{J}{q}{\Omegah}
           + \LpLqnorm{\nabla J_h \varphi}{p}{J}{q}{\Omegah} \\
        &\le Ch^{-1} \LpLqnorm{\varphi_h - J_h \varphi}{p}{J}{q}{\Omegah}
           + C \LpLqnorm{\nabla \varphi}{p}{J}{q}{\Omegah} \\
        &\le C \LpLqnorm{f}{p}{J}{q}{\Omega}.
    \end{align} 
    Hence we complete the proof.
\end{proof}

\section{Preliminaries}
\label{sec:preliminaries}

Under the assumptions \ref{assum:inv}--\ref{assum:SA}, the resolvent estimate for $A_h$ holds \cite[Theorems~2.1 and 2.2]{Kemmochi-arxiv}.
With the notation of \cref{def:discrete}, the result is summarized as follows.
\begin{lemma}\label{lem:semigroup}
    Let $q \in \intervalq$ and $\delta \in (0,\pi)$.
    Assume \ref{assum:inv}--\ref{assum:SA} hold.
    Then, for any $\lambda \in \bC \setminus \{0\}$ with $|\arg \lambda| < \pi-\delta$ and any $g \in L^q(\Omega)^N$, we have
    \begin{align}
        \Lpnorm*{(\lambda + A_h)^{-1} \projs g}{q}{\Omegah} 
        &\le C (1+|\lambda|)^{-1} \Lpnorm{g}{q}{\Omega},
        \label{eq:resolvent-Ah} \\
        \Lpnorm*{\bracket*{ (\lambda + A_h)^{-1} \projs - (\lambda + A)^{-1} P_\sigma } g}{q}{\Omegah} 
        &\le C h^2 \Lpnorm{g}{q}{\Omega},
        \label{eq:error-resol-Ah} \\
        \norm*{\bracket*{ (\lambda + A_h)^{-1} \projs - (\lambda + A)^{-1} P_\sigma } g}{W^{-1,q}(\Omega)} 
        &\le C h \abs{\lambda}^{-1} \Lpnorm{g}{q}{\Omega},
        \label{eq:error-resol-Ah-W-1p} 
    \end{align}
    where $C$ is independent of $h$, $\lambda$, and $g$.
\end{lemma}

From this result, we derive some preliminary estimates for the proofs of \cref{thm:main,thm:error}.


\subsection{Further estimates on the discrete Stokes resolvent problem}

\begin{lemma}
    Let $q \in \intervalq$ and $\delta \in (0,\pi)$.
    Assume \ref{assum:inv}--\ref{assum:SA} hold.
    Then, for any $\lambda \in \bC \setminus \{0\}$ with $|\arg \lambda| < \pi-\delta$ and any $g \in W^{1,q}_0(\Omega)^N$, we have
    \begin{equation}
        \Lpnorm*{\bracket*{ (\lambda + A_h)^{-1} \projs - (\lambda + A)^{-1} P_\sigma } g}{q}{\Omegah} 
        \le C h \abs{\lambda}^{-1} \Lpnorm{\nabla g}{q}{\Omega},
        \label{eq:error-resol-Ah-W1p} 
    \end{equation}
    where $C$ is independent of $h$, $\lambda$, and $g$.
\end{lemma}

\begin{proof}
    Let $u = (\lambda + A)^{-1} P_\sigma g$ and $u_h = (\lambda + A_h)^{-1} \projs g$.
    Then, $u$ solves the Stokes resolvent problem
    \begin{equation}
        \begin{cases}
            \lambda u - \Delta u + \nabla \varphi = g, & \text{in } \Omega,\\
            \dv u = 0, & \text{in } \Omega,\\
            u = 0, & \text{on } \partial\Omega
        \end{cases}
    \end{equation}
    for some $\varphi$, and $u_h$ satisfies
    \begin{equation}
        \begin{cases}
            \lambda \innerh{u_h}{v_h} + \innerh{\nabla u_h}{\nabla v_h} - \innerh{\varphi_h}{\dv v_h} = \innerh{g}{v_h}, & \forall v_h \in V_h,\\
            \innerh{\dv u_h}{\psi_h} = 0, & \forall \psi_h \in Q_h 
        \end{cases}
    \end{equation}
    for some $\varphi_h \in Q_h$.
    Moreover, let $\phi \in C^\infty_0(\Omegah)^N$ be arbitrary and consider the dual problems
    \begin{equation}
        \begin{cases}
            \bar{\lambda} U - \Delta U + \nabla \Phi = \phi, & \text{in } \Omega,\\
            \dv U = 0, & \text{in } \Omega,\\
            U = 0, & \text{on } \partial\Omega
        \end{cases}
    \end{equation}
    and    
    \begin{equation}
        \begin{cases}
            \lambda \innerh{v_h}{U_h} + \innerh{\nabla v_h}{\nabla U_h} - \innerh{\dv v_h}{\Phi_h} = \innerh{v_h}{\phi}, & \forall v_h \in V_h,\\
            \innerh{\psi_h}{\dv U_h} = 0, & \forall \psi_h \in Q_h .
        \end{cases}
    \end{equation}
    Then, as discussed in \cite[\S 8.2]{Kemmochi-arxiv}, we have
    \begin{equation}
        \innerh{u-u_h}{\phi} = \inner{u-u_h}{\phi} = \inner{g}{U-U_h},
    \end{equation}
    which implies
    \begin{equation}
        \abs*{\innerh{u-u_h}{\phi}}
        \le \Lpnorm{\nabla g}{q}{\Omega} \norm{U-U_h}{W^{-1,q'}(\Omega)}
        \le C h \abs{\lambda}^{-1} \Lpnorm{\nabla g}{q}{\Omega} \Lpnorm{\phi}{q'}{\Omega}
    \end{equation}
    owing to \eqref{eq:error-resol-Ah-W-1p}.
    Since $\phi \in C^\infty_0(\Omegah)^N$ is arbitrary, we obtain \eqref{eq:error-resol-Ah-W1p}.
\end{proof}


\begin{corollary}
    Under the same assumptions as above, we have
    \begin{equation}
        \Lpnorm{\nabla (\lambda I + A_h)^{-1} \projs g}{q}{\Omegah} 
        \le C \abs{\lambda}^{-1} \Lpnorm{\nabla g}{q}{\Omega}
        \label{eq:resol-Ah-W1p}
    \end{equation}
    for $g \in W^{1,q}_0(\Omega)$.
\end{corollary}

\begin{proof}
    Let $u = (\lambda + A)^{-1} P_\sigma g$ and $u_h = (\lambda + A_h)^{-1} \projs g$.
    Then, $u$ satisfies the resolvent estimate of the form
    \begin{equation}
        \Lpnorm{\nabla u}{q}{\Omega} \le C \abs{\lambda}^{-1} \Lpnorm{\nabla g}{q}{\Omega},
    \end{equation}
    which is proved in \cite[Lemma 3.2]{Kemmochi-arxiv}.
    Therefore, by \eqref{eq:inverse-W1p}, \eqref{eq:int-stab-v}, \eqref{eq:int-err-v-Lp}, and \eqref{eq:error-resol-Ah-W1p}, we have
    \begin{align}
        \Lpnorm{\nabla u_h}{q}{\Omegah}
        &\le \Lpnorm{\nabla (I_h u - u_h)}{q}{\Omegah} + \Lpnorm{\nabla I_h u}{q}{\Omegah} \\
        &\le C h^{-1}\Lpnorm{I_h u - u_h}{q}{\Omegah} + C \Lpnorm{\nabla u}{q}{\Omegah} \\
        &\le C h^{-1}\Lpnorm{u - u_h}{q}{\Omegah} + C h^{-1}\Lpnorm{u - I_h u}{q}{\Omegah} + C \Lpnorm{\nabla u}{q}{\Omegah} \\
        &\le C \abs{\lambda}^{-1} \Lpnorm{\nabla g}{q}{\Omegah},
    \end{align}
    which is the desired estimate.
\end{proof}




The error estimate \eqref{eq:error-resol-Ah} implies an error estimate for the initial value problem
\begin{equation}
    \begin{cases}
        \partial_t u - \Delta u + \nabla \varphi = 0, & \text{in } \Omega \times J, \\ 
        u(0) = v
    \end{cases}
    \label{eq:Stokes-initial}
\end{equation}
for given $v \in L^p_\sigma(\Omega)$ (cf.~\cite[Lemma~5.1]{MR657878}).
\begin{lemma}\label{lem:err-initial}
    Let $q \in \intervalq$ and $v \in L^q(\Omega)^N$. 
    Then, the error estimate 
    \begin{equation}
        \Lpnorm*{\partial_t \paren*{e^{-tA} P_\sigma - e^{-tA_h} \projs}v}{q}{\Omegah}
        \le C h^2 t^{-2} \Lpnorm{v}{q}{\Omega},
        \label{eq:err-initial}
    \end{equation}
    for all $t > 0$, where $C$ is independent of $h$.
\end{lemma}

\begin{proof}
    Since $e^{-tA}$ and $e^{-tA_h}$ are analytic semigroups, we have
    \begin{equation}
        \partial_t \paren*{e^{-tA} P_\sigma - e^{-tA_h} \projs}v 
        = \frac{1}{2\pi i} \int_\Gamma \lambda e^{t\lambda} \bracket*{ (\lambda I +A)^{-1} P_\sigma - (\lambda I + A_h)^{-1} \projs } v d\lambda,
    \end{equation}
    where $\Gamma = \Set{ re^{\pm i (\pi - \delta)} \given r \ge 0}$ for some $\delta \in (0,\pi/2)$ and $\Gamma$ is oriented so that $\operatorname{Im}\lambda$ decreases along $\Gamma$.
    Therefore, owing to \eqref{eq:error-resol-Ah}, we obtain
    \begin{align}
        \Lpnorm*{\partial_t \paren*{e^{-tA} P_\sigma - e^{-tA_h} \projs}v}{q}{\Omegah}
        &\le C h^2 \Lpnorm{v}{q}{\Omegah} \int_0^\infty r e^{-tr \cos \delta} dr \\
        &\le C h^2 t^{-2} \Lpnorm{v}{q}{\Omega},
    \end{align}
    which is the desired estimate \eqref{eq:err-initial}.
\end{proof}



\subsection{Properties of the discrete Helmholtz projection}

We derive some estimates for the discrete Helmholtz projection $\projs$.
We define 
\begin{equation}
    R_h v \coloneqq A_h^{-1} \projs A v
    \label{eq:Rh}
\end{equation}
for $v \in D(A)$ and begin with the error estimate for $R_h$.

\begin{lemma}\label{lem:error-Stokes-proj}
    Let $q \in \intervalq$ and assume \ref{assum:inv}--\ref{assum:SA} hold. 
    Then, we have
    \begin{equation}
        \Lpnorm{(A^{-1} P_\sigma - A_h^{-1} \projs) g}{q}{\Omegah} \le C h^2 \Lpnorm{g}{q}{\Omega}
        \label{eq:error-Stokes-steady}
    \end{equation}
    for $g \in L^q(\Omega)^N$.
    Moreover, we have
    \begin{equation}
        \Lpnorm{v - R_h v}{q}{\Omegah} \le C h^2 \norm{v}{W^{2,q}(\Omega)}.
        \label{eq:error-Stokes-proj}
    \end{equation}
    for $v \in D(A)$. 
\end{lemma}

\begin{proof}
    Let $g \in L^q(\Omega)^N$. Then, letting $\lambda \to 0$ in the estimate \eqref{eq:error-resol-Ah}, we obtain \eqref{eq:error-Stokes-steady}.
    Moreover, let $v \in D(A)$. Then, setting $g = Av$ in \eqref{eq:error-Stokes-steady}, we obtain \eqref{eq:error-Stokes-proj} since $A^{-1} P_\sigma A v = v$.
\end{proof}

We then have the following estimates for $\projs$.
Similar estimates for $q=2$ are already given in \cite{MR687594,MR650052}.

\begin{lemma}\label{lem:disc-Helmholtz}
    Let $q \in \intervalq$ and assume \ref{assum:inv}--\ref{assum:SA} hold. 
    Then, we have the following estimates.
    \begin{enumerate}[label=\upshape{(\roman*)}]
        \item For $v \in L^q(\Omega)^N$, 
        \begin{equation}
            \Lpnorm{\projs v}{q}{\Omegah} \le C \Lpnorm{v}{q}{\Omega}.
            \label{eq:stab-disc-Helmholtz-Lq}
        \end{equation}
        \item For $v \in W^{1,q}_0(\Omega)^N$, 
        \begin{align}
            \Lpnorm{\nabla \projs v}{q}{\Omegah} &\le C \Lpnorm{\nabla v}{q}{\Omega},
            \label{eq:stab-disc-Helmholtz-W1q} \\ 
            \Lpnorm{(P_\sigma - \projs) v}{q}{\Omegah} &\le C h \Lpnorm{\nabla v}{q}{\Omega}.
            \label{eq:error-disc-Helmholtz-W1q}
        \end{align}
        \item For $v \in D(A)$, 
        \begin{equation}
            \Lpnorm{v - \projs v}{q}{\Omegah} \le C h^2 \norm{v}{W^{2,q}(\Omega)}.
            \label{eq:error-disc-Helmholtz-W2q}
        \end{equation}
    \end{enumerate}
    Here, each $C$ is independent of $h$ and $g$.
\end{lemma}

\begin{proof}
    Let $\lambda > 0$. Then, one obtains
    \begin{equation}
        \lambda (\lambda + A)^{-1}v \to v \quad (\forall v \in L^p_\sigma(\Omega)^N), \qquad  
        \lambda (\lambda + A_h)^{-1}v_h \to v_h \quad (\forall v_h \in \Vhs),
    \end{equation}
    in $L^q$ as $\lambda \to \infty$, by the general theory of semigroups (cf.~\cite[Lemma~3.2 in Chapter~1]{MR710486}).
    Therefore, we obtain \eqref{eq:stab-disc-Helmholtz-Lq}, \eqref{eq:stab-disc-Helmholtz-W1q}, and \eqref{eq:error-disc-Helmholtz-W1q} by \eqref{eq:resolvent-Ah}, \eqref{eq:resol-Ah-W1p}, and \eqref{eq:error-resol-Ah-W1p}, respectively.

    We show \eqref{eq:error-disc-Helmholtz-W2q}.
    Recalling that $R_h v \in \Vhs$ for $v \in D(A)$, we have
    \begin{equation}
        v - \projs v = (I-\projs)(v - R_h v).
    \end{equation}
    This implies \eqref{eq:error-disc-Helmholtz-W2q} together with \eqref{eq:stab-disc-Helmholtz-Lq} and \eqref{eq:error-Stokes-proj}.
\end{proof}


The stability property of $\projs$ implies the duality estimate for $\Vhs$.
\begin{corollary}\label{cor:duality-Vhs}
    Let $q \in \intervalq$ and assume \ref{assum:inv}--\ref{assum:SA} hold. 
    Then, 
    \begin{equation}
        \Lpnorm{v_h}{q}{\Omegah} \le C \sup_{w_h \in \Vhs} \frac{\innerh{v_h}{w_h}}{\Lpnorm{w_h}{q'}{\Omegah}},
        \qquad \forall v_h \in \Vhs,
        \label{eq:duality-Vhs}
    \end{equation}
    where $C$ is independent of $h$.
\end{corollary}
\begin{proof}
    Let $v_h \in \Vhs$ and set $v^*_h = \projs(|v_h|^{q-2}v_h) \in \Vhs$.
    Then, by \eqref{eq:stab-disc-Helmholtz-Lq}, we have
    \begin{equation}
        \Lpnorm{v^*_h}{q'}{\Omegah} \le C \Lpnorm{v_h}{q}{\Omegah}^{q-1}.
    \end{equation}
    This implies \eqref{eq:duality-Vhs} since $\Lpnorm{v_h}{q}{\Omegah}^q = \innerh{v_h}{v^*_h}$.
\end{proof}


\subsection{Estimates for the discrete Stokes operator}


The estimate \eqref{eq:duality-Vhs} gives the operator norm of the discrete Stokes operator $A_h$.
\begin{corollary}
    Let $q \in \intervalq$ and assume \ref{assum:inv}--\ref{assum:SA} hold. 
    Then, 
    \begin{equation}
        \Lpnorm{A_h v_h}{q}{\Omegah} \le C h^{-2} \Lpnorm{v_h}{q}{\Omegah},
        \qquad \forall v_h \in \Vhs, 
        \label{eq:norm-Ah}
    \end{equation}
    where $C$ is independent of $h$. 
\end{corollary}

\begin{proof}
    Let $v_h \in \Vhs$. Then, for any $w_h \in \Vhs$,  we have 
    \begin{equation}
        \innerh{A_h v_h}{w_h} = \innerh{\nabla v_h}{\nabla w_h} 
        \le C h^{-2} \Lpnorm{v_h}{p}{\Omegah} \Lpnorm{w_h}{p}{\Omegah}
    \end{equation}
    by the inverse inequality \eqref{eq:inverse-W1p}.
    This implies \eqref{eq:norm-Ah} together with \eqref{eq:duality-Vhs}.
\end{proof}


The resolvent estimate \eqref{eq:resolvent-Ah} also implies that the fractional power $A_h^\theta$ is well-defined for any $\theta \in \bR$,
and since $\Vhs$ is finite dimensional, $A_h^\theta v_h$ is well-defined for any $v_h \in \Vhs$.
Moreover, the following estimates hold.
The proof is clear by the interpolation estimates and thus we omit it.
\begin{lemma}\label{lem:Ah-fracrtional}
    Let $\theta \in [0,1]$ and $q \in \intervalq$. Then, for any $v_h \in \Vhs$, we have
    \begin{align}
        \Lpnorm{A_h^\theta v_h}{q}{\Omegah} &\le C h^{-2\theta} \Lpnorm{v_h}{q}{\Omegah}, \\ %\label{eq:Ah-fractional}\\ 
        \Lpnorm{A_h^\theta e^{-tA_h} v_h}{q}{\Omegah} &\le C t^{-\theta} \Lpnorm{v_h}{q}{\Omegah}, 
        \qquad \forall t>0, %\label{eq:Ah-fractional-semigroup}
    \end{align}
    where $C$ is independent of $h$.
\end{lemma}





\section{Discrete maximal regularity}
\label{sec:proof-dmr}

This section is devoted to the proof of \cref{thm:main}.

\subsection{Temporally regularized Green's function}

Let $a \in J$ be an arbitrary time and $\varepsilon > 0$ be sufficiently small.
Then, we introduce an approximated delta-function $\eta = \eta_{a,\varepsilon} \in C^\infty_0(J)$ as a function that satisfies
\begin{equation}
    \eta \ge 0, \quad \supp\eta \subset J \cap [a-\varepsilon, a+\varepsilon], \quad 
    \int_J \eta dt = 1, 
    \label{eq:reg-delta}
\end{equation}
and 
\begin{equation}
    \Lpnorm*{\eta^{(l)}}{p}{J} \le C \varepsilon^{-l-1-\frac{1}{p}}
    \label{eq:reg-delta-norm}
\end{equation}
for $l \in \bN$ and $p \in [1,\infty]$.
For example, the classical Friedrichs mollifier satisfies the above requirements.

We further fix $\phi_h \in \Vhs$ arbitrarily and then we define the temporally regularized Green's function $g=g_{a,\varepsilon} \colon J \to L^p_\sigma(\Omega)$ as the solution of the dual problem
\begin{equation}
    \begin{cases}
        -\partial_t g + A g = \eta P_\sigma \phi_h, & \text{in } J, \\
        g(T) = 0.
    \end{cases}
\end{equation}
We denote by $g_h$ the finite element approximation of $g$. 
Namely, $g_h$ satisfies 
\begin{equation}
    \begin{cases}
        -\partial_t g_h + A_h g_h = \eta \phi_h, & \text{in } J, \\
        g_h(T) = 0.
    \end{cases}
\end{equation}
Here we recall that $\phi_h \in \Vhs$ and thus $\projs \phi_h = \phi_h$.


Hereafter, we assume $q = p$ according to the general theory of maximal regularity (cf.~\cite[Theorem~4.2]{MR1225809}).
Then, keeping \eqref{eq:duality-Vhs} in mind, we address $\innerh{A_h u_h(a)}{\phi_h}$ for $\phi_h \in \Vhs$ as above.
We show the following expression formula.

\begin{lemma}\label{lem:Ahuh-expression}
    Let $a \in J$ and $\phi_h \in \Vhs$. 
    Let further $u$ and $u_h$ be the solution of \eqref{eq:Stokes} and \eqref{eq:disc-Stokes}, respectively.
    Then, we have
    \begin{equation}
        \innerh{A_h u_h(a)}{\phi_h} = I_1(a; \phi_h) + I_2(a; \phi_h) + I_3(a; \phi_h),
        % \label{eq:Ahuh-expression}
    \end{equation}
    where 
    \begin{align}
        I_1(a; \phi_h) &\coloneqq - \int_J \innerh{\projs f}{\partial_t (g-g_h)} dt, \\
        I_2(a; \phi_h) &\coloneqq \int_J \inner{-\partial_t u + f}{\eta \phi_h} dt, \\
        I_3(a; \phi_h) &\coloneqq \int_J \innerh*{A_h u_h(a) - A_h u_h(t)}{\eta(t)\phi_h} dt.
    \end{align}
\end{lemma}

\begin{proof}
    Recalling \eqref{eq:reg-delta}, we have 
    \begin{align}
        \innerh{A_h u_h(a)}{\phi_h}
        &= \int_J \innerh{A_h u_h(t)}{\eta(t)\phi_h} dt 
        + \int_J  \innerh*{A_h u_h(a) - A_h u_h(t)}{\eta(t)\phi_h} dt \\
        &= \int_J \innerh{A_h u_h(t)}{\eta(t)\phi_h} dt + I_3(a; \phi_h).
    \end{align}
    By the definition of $g_h$ and integration by parts in time, we have
    \begin{align}
        \int_J \innerh{A_h u_h(t)}{\eta(t)\phi_h} dt 
        &= \int_J \innerh{A_h u_h(t)}{-\partial_t g_h + A_h g_h} dt \\
        &= \int_J \innerh{\partial_t u_h + A_h u_h}{A_h g_h} dt \\
        &= \int_J \innerh{f}{\partial_t g_h + \eta \phi_h} dt \\
        &= \int_J \inner{f}{\partial_t g + \eta \phi_h} dt + I_1(a; \phi_h).
    \end{align}
    Here we used the fact that $\partial_t g_h (t) + \eta(t) \phi_h \in V_{h,\sigma}$ for each $t \in J$.
    Let us now address $\int_J \inner{f}{\partial_t g} dt$.
    Since $u$ satisfies \eqref{eq:Stokes-Cauchy} and $\partial_t g(t) \in L^{p'}_\sigma(\Omega)$ for each $t \in J$, we have
    \begin{align}
        \int_J \inner{f}{\partial_t g} dt
        &= \int_J \inner{\partial_t u + A u}{\partial_t g} dt \\
        &= \int_J \inner{\partial_t u}{\partial_t g - A g} dt \\
        &= \int_J \inner{\partial_t u}{- \eta \phi_h} dt
    \end{align}
    by integrating by parts in time again.
    Hence we have 
    \begin{equation}
        \int_J \inner{f}{\partial_t g + \eta \phi_h} dt 
        = \int_J \inner{-\partial_t u + f}{\eta \phi_h} dt 
        = I_2(a; \phi_h),
    \end{equation}
    and we thus obtain the assertion.
\end{proof}


Let $\Bhs = \Set{\phi_h \in \Vhs \given \Lpnorm{\phi_h}{p'}{\Omegah} \le 1}$ and we will address
\begin{equation}
    \paren*{ 
        \int_J \sup_{\phi_h \in \Bhs} \abs*{I_j(a; \phi_h)}^p da 
    }^{1/p}
\end{equation}
for $j=1,2,3$.
We first show the estimates for $I_2$ and $I_3$.
\begin{lemma}\label{lem:I2I3}
    For any $\varepsilon>0$, we have
    \begin{align}
        \paren*{ 
            \int_J \sup_{\phi_h \in \Bhs} \abs*{I_2(a; \phi_h)}^p da 
        }^{1/p}
        &\le C \Lpnorm{f}{p}{\Omega \times J}, 
        \label{eq:I2} \\
        \paren*{ 
            \int_J \sup_{\phi_h \in \Bhs} \abs*{I_3(a; \phi_h)}^p da 
        }^{1/p}
        &\le C \varepsilon h^{-2} \paren*{ \Lpnorm{f}{p}{\Omega \times J} + \Lpnorm{A_h u_h}{p}{\Omegah \times J} }.
        \label{eq:I3-epsilon}
    \end{align}
    In particular, letting $\varepsilon = \gamma h^2$ with $\gamma>0$ sufficiently small, we obtain
    \begin{equation}
        \paren*{ 
            \int_J \sup_{\phi_h \in \Bhs} \abs*{I_3(a; \phi_h)}^p da 
        }^{1/p}
        \le C \Lpnorm{f}{p}{\Omega \times J} + \frac{1}{2} \Lpnorm{A_h u_h}{p}{\Omegah \times J}.
        \label{eq:I3}
    \end{equation}
\end{lemma}

\begin{proof}
    Recalling that $\supp \eta \subset J \cap [a-\varepsilon,a+\varepsilon]$ and $\Lpnorm{\eta}{\infty}{J} \le C \varepsilon^{-1}$ (see \eqref{eq:reg-delta} and \eqref{eq:reg-delta-norm}), we have 
    \begin{align}
        \abs{I_2(a; \phi_h)}
        &\le \int_{J \cap [a-\varepsilon,a+\varepsilon]} \Lpnorm{(-\partial_t u + f)(t)}{p}{\Omega} |\eta(t)| \Lpnorm{\phi_h}{p'}{\Omegah} dt \\
        &\le C \varepsilon^{-1+ \frac{1}{p'}} \Lpnorm{\phi_h}{p'}{\Omegah} \paren*{ 
            \int_{J \cap [a-\varepsilon,a+\varepsilon]} \Lpnorm{(-\partial_t u + f)(t)}{p}{\Omega}^p dt 
         }^{1/p},
    \end{align}
    which implies
    \begin{align}
        \int_J \sup_{\phi_h \in \Bhs} \abs*{I_2(a; \phi_h)}^p da 
        &\le C \paren*{ \varepsilon^{-1 + \frac{1}{p'}} }^p
            \int_J \int_{J \cap [a-\varepsilon,a+\varepsilon]} \Lpnorm{(-\partial_t u + f)(t)}{p}{\Omega}^p dt da
        \\
        &\le C \varepsilon^{-1} 
            \int_J \Lpnorm{(-\partial_t u + f)(t)}{p}{\Omega}^p \int_{J \cap [t-\varepsilon,t+\varepsilon]} da dt
        \\
        &\le C \Lpnorm{-\partial_t u + f}{p}{\Omega \times J}^p.
    \end{align}
    Therefore, by the maximal regularity \eqref{eq:mr}, we obtain the first assertion \eqref{eq:I2}.
    
    We next address $I_3$.
    It is clear that 
    \begin{align}
        \abs*{I_3(a; \phi_h)}
        &= \abs*{\int_J \int_t^a \innerh*{\partial_t A_h u_h(s)}{ \eta(t) \phi_h} ds dt} \\
        &\le \int_J \int_J \chi_\varepsilon(s,t,a) \Lpnorm{\partial_t A_h u_h(s)}{p}{\Omegah} \abs*{\eta(t)} \Lpnorm{\phi_h}{p'}{\Omegah} ds dt 
    \end{align}
    holds, where 
    \begin{equation}
        \chi_\varepsilon(s,t,a) = 
        \begin{cases}
            1 & \text{if } |t-a| \le \varepsilon \text{ and } |s-a| \le \varepsilon, \\
            0 & \text{otherwise}.
        \end{cases}
    \end{equation}
    Noticing that 
    \begin{equation}
        \operatorname{meas} \paren*{ \Set*{(t,s) \in J \times J \given \chi_\varepsilon(t,s,a) = 1} } \le 4 \varepsilon^2 
    \end{equation}
    for each $a$, where $\operatorname{meas}(\cdot)$ denotes the Lebesgue measure, we have 
    \begin{equation}
        \abs*{I_3(a; \phi_h)}
        \le C \varepsilon^{\frac{2}{p'}}
        \paren*{ \int_J \int_J \chi_\varepsilon(s,t,a) \Lpnorm{\partial_t A_h u_h(s)}{p}{\Omegah}^p \abs*{\eta(t)}^p ds dt }^{1/p} \Lpnorm{\phi_h}{p'}{\Omegah}.
    \end{equation}
    This implies 
    \begin{multline}
            \int_J \sup_{\phi_h \in \Bhs} \abs*{I_3(a; \phi_h)}^p da 
        \\ 
        \begin{aligned}
            &\le C \paren*{ \varepsilon^{\frac{2}{p'}} }^p 
                \int_J \int_J \int_J \chi_\varepsilon(s,t,a) \Lpnorm{\partial_t A_h u_h(s)}{p}{\Omegah}^p \abs*{\eta(t)}^p ds dt da 
            \\
            &\le C \varepsilon^{p-2} 
                \int_J \Lpnorm{\partial_t A_h u_h(s)}{p}{\Omegah}^p \int_J \int_J \chi_\varepsilon(s,t,a) dt da ds 
            \\ 
            &\le C \varepsilon^p \Lpnorm{\partial_t A_h u_h}{p}{\Omegah \times J}.
        \end{aligned}
    \end{multline}
    Here we used $|\eta| \le C \varepsilon^{-1}$ and 
    \begin{equation}
        \operatorname{meas} \paren*{ \Set*{(t,a) \in J \times J \given \chi_\varepsilon(t,s,a) = 1} } \le 4 \varepsilon^2
    \end{equation}
    for each $s$.
    Moreover, by \eqref{eq:disc-Stokes-Cauchy}, \eqref{eq:stab-disc-Helmholtz-Lq}, and \eqref{eq:norm-Ah}, we have
    \begin{align}
        \Lpnorm{\partial_t A_h u_h}{p}{\Omegah \times J}
        &= \Lpnorm{A_h (\projs f - A_h u_h)}{p}{\Omegah \times J} \\
        &\le C h^{-2} \paren*{ \Lpnorm{f}{p}{\Omega \times J} + \Lpnorm{A_h u_h}{p}{\Omegah \times J} }.
    \end{align}
    Hence we obtain \eqref{eq:I3-epsilon}.
\end{proof}

Let us then consider $I_1$. 
We introduce a weight function 
\begin{equation}
    \omega(t) = \omega_a(t) = \sqrt{(t-a)^2 + \varepsilon^2}, \qquad t \in \bR.
\end{equation}
It is easy to see that, for $r \in [1,\infty]$ and $\beta > 1/r$,
\begin{equation}
    \Lpnorm{\omega_a^{-\beta}}{r}{J} \le C \varepsilon^{-\beta+\frac{1}{r}}
    \label{eq:weight-Lp}
\end{equation}
holds uniformly in $a$.
Then, we have the following estimate.
\begin{lemma}\label{lem:I1}
    Let 
    \begin{equation}
        M_\alpha \coloneqq \sup_{a \in J, \, \phi_h \in \Bhs} \Lpnorm{\omega^\alpha \partial_t(g-g_h)}{p'}{\Omegah \times J}
    \end{equation}
    for $\alpha > 0$.
    Then, for $\alpha > 1/p$, we have
    \begin{equation}
        \paren*{ 
            \int_J \sup_{\phi_h \in \Bhs} \abs*{I_1(a; \phi_h)}^p da 
        }^{1/p}
        \le C \varepsilon^{-\alpha+\frac{1}{p}} M_\alpha \Lpnorm{f}{p}{\Omega \times J}.
        \label{eq:I1}
    \end{equation}
\end{lemma}

\begin{proof}
    By the H\"older inequality and the definition of $M_\alpha$, we have
    \begin{equation}
        |I_1(a;\phi_h)| \le M_\alpha \Lpnorm{\omega_a^{-\alpha} \projs f}{p}{\Omegah \times J},
    \end{equation}
    which implies 
    \begin{align}
        \paren*{ 
            \int_J \sup_{\phi_h \in \Bhs} \abs*{I_1(a; \phi_h)}^p da 
        }^{1/p}
        &\le M_\alpha \paren*{
            \int_J \int_J \omega_a(t)^{-\alpha p} \Lpnorm{\projs f(t)}{p}{\Omegah} dt da
        }^{1/p} \\
        &= M_\alpha \paren*{
            \int_J \Lpnorm{\projs f(t)}{p}{\Omegah} \int_J \omega_a(t)^{-\alpha p} da dt
        }^{1/p}.
    \end{align}
    Recall that, by \eqref{eq:weight-Lp}, 
    \begin{equation}
        \int_J \omega_a(t)^{-\alpha p} da
        = \int_J \omega_t(a)^{-\alpha p} da
        = \Lpnorm{\omega_t^{-\alpha}}{p}{J}^p
        \le C \varepsilon^{-\alpha p + 1}
    \end{equation}
    holds for $\alpha>1/p$. Therefore, we establish \eqref{eq:I1} owing to the above estimate.
\end{proof}


Let $\varepsilon = \gamma h^2$ with $\gamma$ sufficiently small so that \eqref{eq:I3} holds.
Then, from \cref{lem:Ahuh-expression,lem:I2I3,lem:I1} and \eqref{eq:duality-Vhs}, we have
\begin{equation}
    \Lpnorm{A_h u_h}{p}{\Omegah \times J}
    \le C \paren*{ 1 + \varepsilon^{-\alpha + \frac{1}{p}} M_\alpha } \Lpnorm{f}{p}{\Omega \times J}
    \label{eq:Ahuh-weighted}
\end{equation}
for $\alpha > 1/p$.
Therefore, it suffices to show the weighted error estimate
\begin{equation}
    \Lpnorm{\omega_a^\alpha \partial_t(g-g_h)}{p'}{\Omegah \times J}
    \le C \varepsilon^{\alpha - \frac{1}{p}} \Lpnorm{\phi_h}{p'}{\Omegah}
    \label{eq:weighted-error}
\end{equation}
for some $\alpha > 1/p$.



\subsection{Proof of the discrete maximal regularity}

We prove \cref{thm:main} by showing \eqref{eq:weighted-error}.
Recall that it suffices to consider the case $q=p$ by the general theory of maximal regularity.

\begin{proof}[Proof of \cref{thm:main} for $q=p$.]
    Let $q = p \in \intervalq$, $\phi_h \in \Bhs$, $a \in J$, and $\varepsilon = \gamma h^2$ be as above. 
    We divide the interval $J$ into three parts
    \begin{equation}
        J_0 = J \cap [a-2\varepsilon, a+2\varepsilon],
        \qquad 
        J_1 = J \cap (-\infty,a-2\varepsilon),
        \qquad 
        J_2 = J \cap (a+2\varepsilon, \infty).
    \end{equation}
    Then, by the definition of $g$ and $g_h$, it is clear that $g(t) \equiv g_h(t) \equiv 0$ for $t \in J_2$.
    We now consider $\Lpnorm{\omega_a^\alpha \partial_t(g-g_h)}{p'}{\Omegah \times J_1}$.
    Let $t \in J_1$. Then, by the Duhamel formula and $t < a-2\varepsilon$, we have
    \begin{equation}
        \partial_t (g - g_h) (t) = \int_{J \cap [a-\varepsilon,a+\varepsilon]} \partial_t \bracket*{e^{-(s-t)A} P_\sigma - e^{-(s-t)}} \phi_h \eta(s) ds,
    \end{equation}
    which, together with \cref{lem:err-initial}, yields
    \begin{equation}
        \Lpnorm{\partial_t (g-g_h)(t)}{p'}{\Omegah} 
        \le C h^2 \int_{J \cap [a-\varepsilon,a+\varepsilon]} (s-t)^{-2} \Lpnorm{\phi_h}{p'}{\Omegah} \eta(s) ds.
    \end{equation}
    Moreover, recalling \eqref{eq:reg-delta-norm} and $s-t \ge a-\varepsilon-t$ for $t \in J_1$, we have
    \begin{equation}
        \Lpnorm{\partial_t (g-g_h)(t)}{p'}{\Omegah} 
        \le C h^2 (a-\varepsilon-t)^{-2} \Lpnorm{\phi_h}{p'}{\Omegah} .
    \end{equation}
    Therefore, we obtain
    \begin{equation}
        \Lpnorm{\omega_a^\alpha \partial_t (g-g_h)}{p}{\Omegah \times J_1}
        \le C h^2 \Lpnorm{\phi_h}{p'}{\Omegah} \paren*{
            \int_0^{a-2\varepsilon} \omega_a(t)^{\alpha p'} (a-\varepsilon-t)^{-2p'} dt
        }^{1/p'}.
    \end{equation}
    Let $a-t=\varepsilon s$ in the last integration and assume now that $\alpha < \frac{1}{p} + 1$. 
    Then, we have
    \begin{align}
        \int_0^{a-2\varepsilon} \omega_a(t)^{\alpha p'} (a-\varepsilon-t)^{-2p'} dt
        &\le C \varepsilon^{(\alpha-2)p' + 1} \int_2^\infty (s^2+1)^{\alpha p'/2} (s-1)^{-2p'} ds \\
        &= C \varepsilon^{(\alpha-2)p' + 1}
    \end{align}
    since $\alpha < \frac{1}{p} + 1$ is equivalent to $(\alpha-2)p' + 1 < 0$.
    Hence we have
    \begin{equation}
        \Lpnorm{\omega_a^\alpha \partial_t (g-g_h)}{p}{\Omegah \times J_1}
        \le C h^2 \varepsilon^{\alpha-2+\frac{1}{p'}} \Lpnorm{\phi_h}{p'}{\Omegah}  
        = C \varepsilon^{\alpha-\frac{1}{p}} \Lpnorm{\phi_h}{p'}{\Omegah}
        \label{eq:J1}
    \end{equation}
    for $\alpha \in \paren*{\frac{1}{p}, \frac{1}{p}+1}$, since $h^2 = \gamma^{-1} \varepsilon$.

    We next address $\Lpnorm{\omega_a^\alpha \partial_t(g-g_h)}{p'}{\Omegah \times J_0}$.
    By the maximal regularity for $g$ and \eqref{eq:reg-delta-norm}, we have
    \begin{align}
        \Lpnorm{\omega_a^\alpha \partial_t(g-g_h)}{p'}{\Omegah \times J_0}
        &\le C \varepsilon^\alpha \Lpnorm{\partial_t g}{p'}{\Omegah \times J}
        + C \varepsilon^\alpha \Lpnorm{\partial_t g_h}{p'}{\Omegah \times J_0} \\
        &\le C \varepsilon \Lpnorm{\eta \phi_h}{p'}{\Omegah \times J}
        + C \varepsilon^\alpha \Lpnorm{\partial_t g_h}{p'}{\Omegah \times J_0} \\
        &\le C \varepsilon^{\alpha-\frac{1}{p}} \Lpnorm{\phi_h}{p'}{\Omegah}
        + C \varepsilon^\alpha \Lpnorm{\partial_t g_h}{p'}{\Omegah \times J_0}.
    \end{align}
    Here we used the fact that $\omega_a(t) \le C \varepsilon$ for $t \in J_0$.
    Thus we address $\Lpnorm{\partial_t g_h}{p'}{\Omegah \times J_0}$.
    Let $t \in J_0$. In this case, we have
    \begin{equation}
        \partial_t g_h(t)
        = - \eta(t) \phi_h + \int_t^{a+\varepsilon} A_h e^{-(s-t)A_h} \phi_h \eta(s) ds.
    \end{equation}
    Now, by \cref{lem:Ah-fracrtional} and $h^2 = \gamma^{-1} \varepsilon$, we have
    \begin{align}
        \Lpnorm{A_h e^{-(s-t)A_h} \phi_h}{p'}{\Omegah}
        &= \Lpnorm{A_h^{1-\theta} e^{-(s-t)A_h} A_h^\theta \phi_h}{p'}{\Omegah} \\
        &\le C (s-t)^{-1+\theta} h^{-2\theta} \Lpnorm{\phi_h}{p'}{\Omegah} \\
        &\le C (s-t)^{-1+\theta} \varepsilon^{-\theta} \Lpnorm{\phi_h}{p'}{\Omegah} 
    \end{align}
    for any $\theta \in (0,1)$, which implies
    \begin{align}
        \Lpnorm*{ \int_t^{a+\varepsilon} A_h e^{-(s-t)A_h} \phi_h \eta(s) ds }{p'}{\Omegah}
        &\le C \varepsilon^{-\theta-1} \Lpnorm{\phi_h}{p'}{\Omegah} \int_t^{a+\varepsilon} (s-t)^{-1+\theta} ds \\
        &\le C \varepsilon^{-\theta-1} (a+\varepsilon-t)^\theta \Lpnorm{\phi_h}{p'}{\Omegah}
    \end{align}
    owing to $|\eta| \le C \varepsilon^{-1}$.
    Therefore, we obtain
    \begin{align}
        \Lpnorm{\partial_t g_h}{p'}{\Omegah \times J_0}
        &\le \Lpnorm{\eta \phi_h}{p'}{\Omegah \times J_0} + C \varepsilon^{-\theta-1} \Lpnorm{\phi_h}{p'}{\Omegah} 
        \paren*{
            \int_{a-2\varepsilon}^{a+2\varepsilon} (a+\varepsilon-t)^{\theta p'} dt
        }^{1/p'} \\
        &\le C \varepsilon^{-\frac{1}{p}} \Lpnorm{\phi_h}{p'}{\Omegah}
        + C \varepsilon^{-\theta-1} \Lpnorm{\phi_h}{p'}{\Omegah} \cdot \varepsilon^{\theta + \frac{1}{p'}} \\
        &\le C \varepsilon^{-\frac{1}{p}} \Lpnorm{\phi_h}{p'}{\Omegah}.
    \end{align}
    Hence we have
    \begin{equation}
        \Lpnorm{\omega_a^\alpha \partial_t(g-g_h)}{p'}{\Omegah \times J_0}
        \le C \varepsilon^{\alpha - \frac{1}{p}} \Lpnorm{\phi_h}{p'}{\Omegah}.
        \label{eq:J0}
    \end{equation}

    Summarizing \eqref{eq:J1} and \eqref{eq:J0}, we establish \eqref{eq:weighted-error} for $\alpha \in \paren*{\frac{1}{p} , \frac{1}{p} + 1}$.
    Therefore, we obtain
    \begin{equation}
        \Lpnorm{A_h u_h}{p}{\Omegah \times J} \le C \Lpnorm{f}{p}{\Omega \times J}
    \end{equation}
    owing to \eqref{eq:Ahuh-weighted}.
    Since $\partial_t u_h = - A_h u_h + \projs f$, we immediately obtain
    \begin{equation}
        \Lpnorm{\partial_t u_h}{p}{\Omegah \times J} \le C \Lpnorm{f}{p}{\Omega \times J}
    \end{equation}
    owing to \eqref{eq:stab-disc-Helmholtz-Lq}.
    Hence we complete the proof of \cref{thm:main}.
\end{proof}

\begin{remark}
    In the proof of discrete maximal regularity \eqref{eq:dmr}, we used the following two facts only:
    \begin{enumerate}[label=(\roman*)]
        \item $A$ has maximal regularity,
        \item $A_h$ generates an analytic semigroup (uniformly in $h$).
    \end{enumerate}
    Therefore, if one applies the same idea to the parabolic case, one obtains another proof of discrete maximal regularity for elliptic operators without the Gaussian estimate.
\end{remark}

\section{Application to error estimates}
\label{sec:proof-error}

The aim of this section is to show \cref{thm:error}.

\begin{proof}[Proof of \cref{thm:error}]
    Let $(u,\varphi)$ and $(u_h,\varphi_h)$ be the solutions of \eqref{eq:Stokes} and \eqref{eq:disc-Stokes}, respectively.

    (1) $L^p(J;L^q(\Omegah))$-error estimate for the velocity.
    We first show \eqref{eq:error-Lp} by the argument similar to \cite[Theorem~4.9]{MR2350187}.
    Let $\rho = u - \projs u$ and $\rho_h = \projs u - u_h$.
    Then, by \eqref{eq:error-disc-Helmholtz-W2q} and \eqref{eq:mr}, we have
    \begin{equation}
        \LpLqnorm{\rho}{p}{J}{q}{\Omegah} \le C h^2 \LpLqnorm{f}{p}{J}{q}{\Omega}.
        % \label{eq:rho}
    \end{equation}
    We now address $\rho_h$. Since $u$ and $u_h$ satisfy \eqref{eq:Stokes-Cauchy} and \eqref{eq:disc-Stokes-Cauchy} respectively, one has
    \begin{align}
        \partial_t \rho_h + A_h \rho_h
        &= (\projs \partial_t u + A_h \projs u)  - (\partial_t u_h + A_h u_h) \\ 
        &= \projs (P_\sigma f - A u) + A_h \projs u  - \projs f \\
        &= \projs (P_\sigma -I) f + A_h (R_h - \projs) u .
    \end{align}
    Therefore, $A_h^{-1} \rho_h$ satisfies the Cauchy problem
    \begin{align}
        \partial_t (A_h^{-1} \rho_h) + A_h (A_h^{-1} \rho_h)
        &= A_h^{-1} \projs (P_\sigma -I) f + (R_h - \projs) u \\
        &= (A_h^{-1} \projs - A^{-1} P_\sigma) (P_\sigma -I) f + (R_h - \projs) u ,
    \end{align}
    since $P_\sigma (P_\sigma -I) f = 0$.
    Hence, by the discrete maximal regularity \eqref{eq:dmr} and error estimates \eqref{eq:error-Stokes-steady} and \eqref{eq:error-Stokes-proj}, we obtain
    \begin{equation}
        \LpLqnorm{\rho_h}{p}{J}{q}{\Omegah}
        = \LpLqnorm{A_h (A_h^{-1} \rho_h)}{p}{J}{q}{\Omegah}
        \le C h^2 \LpLqnorm{f}{p}{J}{q}{\Omega},
        % \label{eq:rho_h}
    \end{equation}
    which implies the error estimate \eqref{eq:error-Lp}.

    (2) $L^p(J;W^{1,q}(\Omegah))$-error estimate for the velocity.
    Combining \eqref{eq:error-Lp} with the inverse inequality \eqref{eq:inverse-W1p} and the interpolation error estimates \eqref{eq:int-err-v-Lp-h2} and \eqref{eq:int-err-v-W1p},
    we obtain
    \begin{align}
        \LpLqnorm{\nabla (u-u_h)}{p}{J}{q}{\Omegah}
        &\le \LpLqnorm{\nabla (u - I_h u)}{p}{J}{q}{\Omegah} + C h^{-1} \LpLqnorm{I_h u - u_h}{p}{J}{q}{\Omegah} \\
        &\le C h \LpLqnorm{f}{p}{J}{q}{\Omega}.
    \end{align}
    Hence we obtain \eqref{eq:error-W1p}.

    (3) $L^p(J;W^{-1,q}(\Omegah))$-error estimate for $\partial_t u$.
    Fix $\phi \in L^{p'}(J; W^{1,q'}_0(\Omegah)^N)$ arbitrarily. 
    Then, almost everywhere in $J$, we have
    \begin{align}
        & \innerh{\partial_t (u - u_h)}{\phi} \\
        ={}& \innerh{(P_\sigma f - Au) - (\projs f - A_h u_h)}{\phi} \\
        ={}& \innerh{f}{(P_\sigma - \projs)\phi} + \innerh{\Delta u}{P_\sigma \phi - \projs \phi} - \innerh{-\Delta u + A_h u_h}{\projs \phi} \\
        ={}& \innerh{f+\Delta u}{(P_\sigma - \projs)\phi} - \innerh{\nabla (u-u_h)}{\nabla \projs \phi}.
    \end{align}
    Therefore, by \eqref{eq:mr}, \eqref{eq:error-disc-Helmholtz-W1q}, \eqref{eq:error-W1p}, and \eqref{eq:stab-disc-Helmholtz-W1q}, we obtain
    \begin{equation}
        \int_J \innerh{\partial_t (u - u_h)}{\phi} dt
        \le C h \LpLqnorm{f}{p}{J}{q}{\Omegah} \LpLqnorm{\nabla \phi}{p'}{J}{q'}{\Omegah},
    \end{equation}
    which implies \eqref{eq:error-W-1p} since $\phi$ is arbitrary.

    (4) Pressure estimate. 
    We first recall that the discrete inf-sup condition
    \begin{equation}        
        \inf_{\psi_h \in Q_h^0} \sup_{v_h \in V_h} \frac{\innerh{\psi_h}{\dv v_h}}{\Lpnorm{\nabla v_h}{q'}{\Omegah} \Lpnorm{\psi_h}{q}{\Omegah}}
        \ge C
        \label{eq:inf-sup}
    \end{equation}
    holds true owing to \eqref{eq:preserve-div}, \eqref{eq:int-stab-v}, and the existence of the Bogovski\u\i{} operator in $\Omegah$ (cf.~\cite{MR553920} and \cite[Lemma III.3.1]{MR2808162}),
    where $Q_h^0 \coloneqq Q_h \cap L^2_0(\Omegah)$.
    Now, let $\zeta_h = J_h \varphi - \varphi_h \in Q_h$ and
    \begin{equation}
        \tilde{\zeta}_h = \zeta_h + \zeta_0,\qquad 
        \zeta_0 = \frac{-1}{|\Omegah|} \int_\Omegah \zeta_h dx
    \end{equation}
    so that $\tilde{\zeta}_h \in Q_h^0$.
    Then, as discussed in \cite[\S 8.5]{Kemmochi-arxiv}, we have
    \begin{equation}
        \Lpnorm{\zeta_0}{q}{\Omegah} \le Ch \norm{\varphi}{W^{1,q}(\Omega)}
    \end{equation}
    almost everywhere in $J$, which yields
    \begin{equation}
        \LpLqnorm{\zeta_0}{p}{J}{q}{\Omegah} \le Ch \LpLqnorm{f}{p}{J}{q}{\Omegah}
        \label{eq:zeta-0}
    \end{equation}
    by \eqref{eq:mr}.
    Moreover, by the discrete inf-sup condition \eqref{eq:inf-sup}, we have
    \begin{equation}
        \Lpnorm{\tilde{\zeta}_h}{q}{\Omegah}
        \le C \sup_{v_h \in V_h} \frac{\innerh{\tilde{\zeta}_h}{\dv v_h}}{\Lpnorm{\nabla v_h}{q'}{\Omegah}}
    \end{equation}
    almost everywhere in $J$.
    For $v_h \in V_h$, we have
    \begin{align}
        & \innerh{\tilde{\zeta}_h}{\dv v_h} \\ 
        ={}& \innerh{\zeta_h}{\dv v_h} \\
        ={}& \innerh{J_h \varphi - \varphi}{\dv v_h} + \innerh{\partial_t (u - u_h)}{v_h} + \innerh{\nabla(u-u_h)}{\nabla v_h} \\
        \le{}& C \paren*{ h \Lpnorm{\nabla \varphi}{q}{\Omegah} + \norm{\partial_t (u - u_h)}{W^{-1,q}(\Omegah)} + \Lpnorm{\nabla(u-u_h)}{q}{\Omegah} }
        \Lpnorm{\nabla v_h}{q'}{\Omegah}
    \end{align}
    by \eqref{eq:int-err-p-Lp} and \eqref{eq:Galerkin-orthogonality}. 
    Therefore, we obtain
    \begin{equation}
        \Lpnorm{\tilde{\zeta}_h}{q}{\Omegah}
        \le C \paren*{ h \Lpnorm{\nabla \varphi}{q}{\Omegah} + \norm{\partial_t (u - u_h)}{W^{-1,q}(\Omegah)} + \Lpnorm{\nabla(u-u_h)}{q}{\Omegah} }
    \end{equation}
    almost everywhere in $J$.
    Integrating this in time, we obtain 
    \begin{equation}
        \LpLqnorm{\tilde{\zeta}_h}{p}{J}{q}{\Omegah}
        \le C h \LpLqnorm{f}{p}{J}{q}{\Omega}
        % \label{eq:zeta-tilde}
    \end{equation}
    by \eqref{eq:mr}, \eqref{eq:error-W-1p}, and \eqref{eq:error-W1p}, which implies \eqref{eq:error-zeta-Lp} together with \eqref{eq:zeta-0}.
    Hence we complete the proof of \cref{thm:error}.
\end{proof}

\bibliographystyle{plain}
\bibliography{reference}
\end{document}