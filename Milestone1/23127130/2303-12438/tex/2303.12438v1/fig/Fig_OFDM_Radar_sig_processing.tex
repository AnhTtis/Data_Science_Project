

%\newcommand\DELTA{0.5}
\begin{figure*}[!t]
\centering
\begin{tikzpicture}[scale=0.8, style=thick, rounded corners=1pt,inner sep=3.2pt,node distance=.8cm,every text node part/.style={align=center},
decoration = {snake,   % <-- added
                    pre length=3pt,post length=7pt,% <-- for better looking of arrow,
                    }]
%\tikzset{
%    mynode/.style={sharp corners=2pt,inner sep=7pt,node distance=.8cm,every text node part/.style={align=center}},
%    myarrow/.style={->, >=latex', shorten >=1pt, thick},
%    mylabel/.style={text width=7em, text centered} ,
%    classical/.style={thick,->,>=stealth},
%    cirarrow/.style={thick,->,>=stealth, dashed,shorten >=4pt,shorten <=4pt},
%}

% Draw rectangular nodes (switch sharp to smooth for different corners)
\node[ minimum height = 1cm, minimum width = 1cm] (state0){\small Transmit data \\ \small $\m{S} \in \mathbb{C}^{N_\text{c} \times N_\text{sym}}$};
\node[draw,right=1cm of state0, minimum height = 1cm, minimum width = 1cm](state1){\small $\m{F}_{N_\text{c}}^{-1} \downarrow$};

\draw[->] (state0.east)  -- (state1.west);

\node[draw,right=1cm of state1, minimum height = 1cm, minimum width = 1cm](state2){\small Add CP};

\draw[->] (state1.east)  -- (state2.west);

%\node[draw,right=1cm of state2, minimum height = 1cm, minimum width = 1cm](state3){\small Analog \\ front-end};

%\draw[->] (state2.east)  -- (state3.west);

\draw (state2.east)  -- ++(0.5,0) coordinate (Ant1);
\draw (Ant1)  -- ++(0,0.4) coordinate (IntAnt1);
\draw (IntAnt1)  -- ++(-0.2,0.3);
\draw (IntAnt1)  -- ++(0.2,0.3);


%\node[draw,below=0.5cm of state3, minimum height = 1cm, minimum width = 1cm](state4){\small Analog \\ front-end};
\node[draw,below=0.5cm of state2, minimum height = 1cm, minimum width = 1cm](state5){\small Remove \\ \small CP};


\draw (state5.east)  -- ++(0.5,0) coordinate (Ant2);
\draw (Ant2)  -- ++(0,0.4) coordinate (IntAnt2);
\draw (IntAnt2)  -- ++(-0.2,0.3);
\draw (IntAnt2)  -- ++(0.2,0.3);

\node[above right =-0.5cm and 2cm of state2.east, minimum height = 0cm, minimum width = 0cm, , rotate=90, anchor=north](Object){ Objects};

\draw [->, shorten <=0.2cm, shorten >=0.2cm] (IntAnt1) to (Object.north);
\draw [->, shorten <=0.2cm, shorten >=0.2cm] (Object.north) to (IntAnt2);

\node[left=0.4cm of state5](yarr1){};
\node[above=0.4cm of yarr1](yarr2){$\m{Y}_{\text{tf,ts}}$};
\draw [->, shorten <=0.01cm, shorten >=0.01cm, style=solid] (yarr1) to [out=80,in=260] (yarr2);

%\draw[->] (state4.west)  -- (state5.east);

\node[draw,left=1cm of state5, minimum height = 1cm, minimum width = 1cm](state6){\small $\m{F}_{N_\text{c}} \downarrow$};

\draw[->] (state5.west)  -- (state6.east);

\node[draw,left=1cm of state6, minimum height = 1cm, minimum width = 1cm](state7){\small Element-wise \\ \small  division \\ \small with $\m{S}$ };

\draw[->] (state6.west)  -- (state7.east);

\node[draw,left=1cm of state7, minimum height = 1cm, minimum width = 1cm](state8){\small $\m{F}_{N_\text{c}}^{-1} \downarrow$};

\draw[->] (state7.west)  -- (state8.east);

\node[draw,left=1cm of state8, minimum height = 1cm, minimum width = 1cm](state9){\small $\m{F}_{N_\text{sym}} \rightarrow$};

\draw[->] (state8.west)  -- (state9.east);

\node[ left=1cm of state9, minimum height = 1cm, minimum width = 1cm] (state10){\small RDM of size \\ \small  $N_\text{c} \times N_\text{sym}$
};

\draw[->] (state9.west)  -- (state10.east);

%
%	\node[left] (h0) at (-0.5,0) {$\m{S}$};
%	\node[mynode] (Tx1) at (2,1.5) {\footnotesize $\m{D}_{N_\text{c}}^* \left( \frac{\Delta \varphi_0}{2 \pi} \right)$};  
%	\node at (2,0.2) {$\vdots$}; 
%	\node[mynode] (Tx2) at (2,-1.5) {\footnotesize $\m{D}_{N_\text{c}}^* \left( \frac{\Delta \varphi_{N_\text{Tx}-1}}{2 \pi} \right)$ }; 
%	\node[mynode] (Rx) at (8.0,0) {\footnotesize Receiver}; 
%	
%	\draw[style=thick] (Tx1.east)  -- ++(0.5,0) coordinate (Ant1);
%	\draw[style=thick] (Ant1)  -- ++(0,0.4) coordinate (IntAnt1);
%	\draw[style=thick] (IntAnt1)  -- ++(-0.2,0.3);
%	\draw[style=thick] (IntAnt1)  -- ++(0.2,0.3);
%	
%	\draw[style=thick] (Tx2.east)  -- ++(0.5,0) coordinate (Ant2);
%	\draw[style=thick] (Ant2)  -- ++(0,0.4) coordinate (IntAnt2);
%	\draw[style=thick] (IntAnt2)  -- ++(-0.2,0.3);
%	\draw[style=thick] (IntAnt2)  -- ++(0.2,0.3);
%
%	\draw[style=thick] (Rx.west)  -- ++(-0.5,0) coordinate (RxAnt1);
%	\draw[style=thick] (RxAnt1)  -- ++(0,0.4) coordinate (RxIntAnt1);
%	\draw[style=thick] (RxIntAnt1)  -- ++(-0.2,0.3);
%	\draw[style=thick] (RxIntAnt1)  -- ++(0.2,0.3);
%	
%\draw[style=thick] (h0.east)  -- ++(0.75,0) coordinate (IntS);
%\draw[style=thick] (IntS)  -- ++(0,1.5) coordinate (IntS1);
%\draw[style=thick] (IntS)  -- ++(0,-1.5) coordinate (IntS2);
%\draw[classical] (IntS1) --  (Tx1.west);
%\draw[classical] (IntS2) -- (Tx2.west);
%
%\draw[cirarrow] (IntAnt1) -- (RxIntAnt1);
%\draw[cirarrow] (IntAnt2) -- (RxIntAnt1);
%
%\draw[draw=black, style=dashed, fill=none] (3.7,-1.6) rectangle ++(2.6, 3.9);
%\node at (5.2,-1.35) {\small CIRs/CFRs};
%
%\draw[draw=black, style=dashed, fill=none] (0.1,-2.4) rectangle ++(6.25, 4.8);
%\node at (5.2,-2.2) {\small ECIR/ECFR};
%
%\node[below of = IntAnt1, node distance=1.1cm] {$\vdots$}; 

\end{tikzpicture}
\caption{SISO OFDM radar signal processing chain, where the parallel-to-serial conversion, the ADC and DAC, and the analog front-end are not shown for simplicity. The arrows $\downarrow$ / $\rightarrow$ represent the dimension of a matrix on which the operation is applied. Figure taken from \cite{Lang_RDM_JP}.}
\label{fig:SISO_processing_steps}
\end{figure*}





