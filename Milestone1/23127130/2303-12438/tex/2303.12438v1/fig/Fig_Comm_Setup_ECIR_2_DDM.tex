

%\newcommand\DELTA{0.5}
\begin{figure*}[!t]
\centering
\begin{tikzpicture}[scale=0.8, style=thick, rounded corners=1pt,inner sep=3.2pt,node distance=.8cm,every text node part/.style={align=center}]
%\tikzset{
%    mynode/.style={sharp corners=2pt,inner sep=7pt,node distance=.8cm,every text node part/.style={align=center}},
%    myarrow/.style={->, >=latex', shorten >=1pt, thick},
%    mylabel/.style={text width=7em, text centered} ,
%    classical/.style={thick,->,>=stealth},
%    cirarrow/.style={thick,->,>=stealth, dashed,shorten >=4pt,shorten <=4pt},
%}

% Draw rectangular nodes (switch sharp to smooth for different corners)
\node[ minimum height = 1cm, minimum width = 1cm] (state0){\small Transmit data \\ $\m{S} \in \mathbb{C}^{N_\text{c} \times N_\text{sym}}$};

\node[above right =-0.8cm and 1.8cm of state0, minimum height = 0cm, minimum width = 0cm]{ $\vdots $};
\node[above right =-0.8cm and 3.9cm of state0, minimum height = 0cm, minimum width = 0cm]{ $\vdots $};
\node[above right =-0.8cm and 5.6cm of state0, minimum height = 0cm, minimum width = 0cm]{ $\vdots $};

\node[draw, above right =-0.2cm and 1cm of state0, minimum height = 1cm, minimum width = 1cm](state1u){\small $ \cdot\m{D}_{N_\text{sym}} \left( \frac{\Delta \psi_0}{2 \pi} \right)  $};

\node[draw, below right =-0.2cm and 1cm of state0, minimum height = 1cm, minimum width = 1cm](state1d){\small $ \cdot \m{D}_{N_\text{sym}} \left( \frac{\Delta \psi_3}{2 \pi} \right)  $};

\draw[->] (state0.east)  -- ++(0.5cm,0) -- ++(0,1.035cm) -- (state1u.west);
\draw[->] (state0.east)  -- ++(0.5cm,0) -- ++(0,-1.035cm) -- (state1d.west);

\node[draw,right=0.5cm of state1u, minimum height = 1cm, minimum width = 1cm](state2u){\small $\m{F}_{N_\text{c}}^{-1} \downarrow$};
\node[draw,right=0.5cm of state1d, minimum height = 1cm, minimum width = 1cm](state2d){\small $\m{F}_{N_\text{c}}^{-1} \downarrow$};

\draw[->] (state1u.east)  -- (state2u.west);
\draw[->] (state1d.east)  -- (state2d.west);

\node[draw,right=0.5cm of state2u, minimum height = 1cm, minimum width = 1cm](state3u){\small Add CP};
\node[draw,right=0.5cm of state2d, minimum height = 1cm, minimum width = 1cm](state3d){\small Add CP};

\draw[->] (state2u.east)  -- (state3u.west);
\draw[->] (state2d.east)  -- (state3d.west);

%\node[draw,right=0.5cm of state3u, minimum height = 1cm, minimum width = 1cm](state4u){\small Analog \\ front-end};
%\node[draw,right=0.5cm of state3d, minimum height = 1cm, minimum width = 1cm](state4d){\small Analog \\ front-end};
%
%\draw[->] (state3u.east)  -- (state4u.west);
%\draw[->] (state3d.east)  -- (state4d.west);

\draw (state3u.east)  -- ++(0.5,0) coordinate (Ant1u);
\draw (Ant1u)  -- ++(0,0.4) coordinate (IntAnt1u);
\draw (IntAnt1u)  -- ++(-0.2,0.3);
\draw (IntAnt1u)  -- ++(0.2,0.3);

\draw (state3d.east)  -- ++(0.5,0) coordinate (Ant1d);
\draw (Ant1d)  -- ++(0,0.4) coordinate (IntAnt1d);
\draw (IntAnt1d)  -- ++(-0.2,0.3);
\draw (IntAnt1d)  -- ++(0.2,0.3);


%\node[draw,below right =0cm and 2cm of state3u, minimum height = 1cm, minimum width = 1cm](state5){\small Analog \\ front-end};

\node[draw,below right =-0.2cm and 2cm of state3u, minimum height = 1cm, minimum width = 1cm](state6){\small Remove \\ CP};

\draw (state6.west)  -- ++(-0.5,0) coordinate (Ant2);
\draw (Ant2)  -- ++(0,0.4) coordinate (IntAnt2);
\draw (IntAnt2)  -- ++(-0.2,0.3);
\draw (IntAnt2)  -- ++(0.2,0.3);

\draw [->, shorten <=0.2cm, shorten >=0.2cm, style=dashed] (IntAnt1u) to [out=-20,in=160] (IntAnt2);
\draw [->, shorten <=0.2cm, shorten >=0.2cm, style=dashed] (IntAnt1d) to [out=20,in=200] (IntAnt2);



%\draw[->] (state5.east)  -- (state6.west);

\node[draw,right=0.5cm of state6, minimum height = 1cm, minimum width = 1cm](state7){\small $\m{F}_{N_\text{c}} \downarrow$};

\draw[->] (state6.east)  -- (state7.west);

\node[right=0.5cm of state7, minimum height = 0cm, minimum width = 0cm](state8){};

\draw[->] (state7.east)  -- (state8.west);


\node[below right =0.05cm and 0.0cm of state3d.east, minimum height = 0cm, minimum width = 0cm](cb1){};

\draw [decorate,decoration={brace,mirror, amplitude=8pt},xshift=0pt,yshift=0pt]
(cb1) -- ++(2.1,0.0) node [black,midway,yshift=-0.45cm] 
{\footnotesize CIR $\ve{f}_k$};


\node[below right =0.65cm and 0.0cm of state1d.east, minimum height = 0cm, minimum width = 0cm](cb1){};

\draw [decorate,decoration={brace,mirror, amplitude=8pt},xshift=0pt,yshift=0pt]
(cb1) -- ++(10.5,0.0) node [black,midway,yshift=-0.45cm] 
{\footnotesize CFR $\ve{p}_k$};


\node[below right =2.10cm and -0.2cm of state0.east, minimum height = 0cm, minimum width = 0cm](cb1){};

\draw [decorate,decoration={brace,mirror, amplitude=8pt},xshift=0pt,yshift=0pt]
(cb1) -- ++(14.5,0.0) node [black,midway,yshift=-0.45cm] 
{\footnotesize ECFR  $\m{H}$};


%
%\draw[cirarrow] (IntAnt1) -- (RxIntAnt1);
%\draw[cirarrow] (IntAnt2) -- (RxIntAnt1);
%
%\draw[draw=black, style=dashed, fill=none] (3.7,-1.6) rectangle ++(2.6, 3.9);
%\node at (5.2,-1.35) {\small CIRs/CFRs};
%
%\draw[draw=black, style=dashed, fill=none] (0.1,-2.4) rectangle ++(6.25, 4.8);
%\node at (5.2,-2.2) {\small ECIR/ECFR};
%
%\node[below of = IntAnt1, node distance=1.1cm] {$\vdots$}; 

\end{tikzpicture}
\caption{Transmitter and receiver processing chains including the CIR, the CFR, and the ECFR with their covered processing blocks. The parallel-to-serial conversions, the DACs, the ADC, and the analog front-ends are not shown for simplicity. The figure is based on a similar figure in \cite{Lang_RDM_JP}.}
\label{fig:Comm_setup_ECIR}
\end{figure*}



