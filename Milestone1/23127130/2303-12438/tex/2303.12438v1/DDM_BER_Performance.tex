

\section{BER Performance Comparison} \label{sec:BER_performance}


%In this section, the \ac{ber} performance of the proposed \ac{mimo} \ac{ofdm} system using \ac{rdm} is investigated. This \ac{ber} performance is compared with that of a \ac{siso} \ac{ofdm} system \cite{Van_Nee_OFDM, salehi2007digital, Diss_Hofbauer}, with that of a \ac{mimo} \ac{ofdm} system using \ac{esi}, and with that of a \ac{mimo} \ac{ofdm} system using the \ac{dsi} technique proposed in \cite{Hakobyan_A_novel_OFDM_MIMO_wo_CS}. For the latter one, we assume that the receiver perfectly knows the allocation of the Tx antennas to the subcarriers from the transmitter. 


In this section, the \ac{ber} performance of the proposed \ac{mimo} \ac{ofdm} system using \ac{ddm} is compared with that of a \ac{siso} \ac{ofdm} system and with that of \ac{mimo} \ac{ofdm} systems utilizing \ac{esi}, \ac{rdm}, and \ac{dsi} investigated in \cite{Hakobyan_A_novel_OFDM_MIMO_wo_CS}. For the latter one, it is assumed that the receiver knows the assignment of the subcarrier sets to the individual Tx antennas.
%The chosen parameters and additional processing blocks correspond to those utilized in \cite{Lang_RDM_JP} and are repeated in the following for the sake of completeness. 
All considered systems employ the system parameters listed in Tab.~\ref{Tab:com_sys_paramters} except for the \ac{siso} \ac{ofdm} system for which $N_\text{Tx} = 1$. 
%The \ac{siso} \ac{ofdm} system and the  \ac{mimo} \ac{ofdm} system utilizing \ac{esi} do  not add redundancy by means of \eqref{equ:Comm_OFDM_00111} since they do not face the problem of the constructive/destructive interference pattern described in Sec.~\ref{sec:Discussion_ECFR}.


Additional processing blocks, e.g., the channel coder/decoder \cite{Viterbi, salehi2007digital, Diss_Hofbauer}, the mapper/demapper \cite{allpress2004exact, Haselmayr_LLRs, Lang_Asilomar_2016_LLRs}, interleaver/deinterleaver, randomized \ac{cir} generation \cite{Rappaport_SISO_Channel, Rappaport_SISO_Channel_JP, Code_MIMO_Channels}, correspond to those utilized in \cite{Lang_RDM_JP} such that a detailed description can be omitted in this work. 


%Additional processing blocks correspond to those utilized in \cite{Lang_RDM_JP} and are briefly repeated in the following for the sake of completeness. All systems employ a convolutional channel code with coding rate $r = 1/2$, constraint length 7 and the generator polynomials $(133, 171)_8$ in octal representation in the transmitter \cite{salehi2007digital, Diss_Hofbauer}. The encoded bits are fed into a random interleaver with a block length equal to the number of coded bits per \ac{ofdm} symbol. The interleaved bits are mapped into binary data \ac{qpsk} symbols. At the receivers of all systems, a demapper evaluates the \acp{llr} from the estimated data symbols \cite{allpress2004exact, Haselmayr_LLRs, Lang_Asilomar_2016_LLRs}. A follow-up deinterleaver undoes the manipulations of the interleaver. The last processing block is a Viterbi decoder with soft decision decoding \cite{Viterbi}. 

%The employed model for randomized \ac{cir} generation is described in \cite{Rappaport_SISO_Channel, Rappaport_SISO_Channel_JP} and code for automated channel  generation is available in \cite{Code_MIMO_Channels}. In this code, we used the same parametrization as described in \cite{Lang_RDM_JP} such that a detailed description can be omitted in this work. The final \ac{ber} performances are evaluated by averaging over $5\,000$  \ac{los} and $5\,000$ \ac{nlos} channel realizations. For each one, a relative velocity is drawn from a uniform distribution between $\pm 60\, \text{m/s}$. The effects of this relative velocity on the channel are modelled in form of \ac{ici} and a \ac{cpe} \cite{Diss_Hofbauer}. One exception is the very last simulation, where \ac{ici} was disabled in order to investigate its effects on the \ac{ber} performance. 


The simulations are carried out for fixed values of $E_\text{b}/N_\text{0}$, where $E_\text{b}$ is the average energy per bit of information, and where $N_\text{0}/2$ is the double-sided noise power spectral density of a bandpass noise signal \cite{Diss_Hofbauer}. To obtain the desired value of $E_\text{b}/N_\text{0}$ the noise variance $\sigma_\text{n}^2$ of the complex-valued \ac{awgn} at the receiver input is chosen according to \cite{Miller_compl_BB, Diss_Hofbauer}
\begin{align}
	\sigma_\text{n}^2 = \frac{P_\text{s}}{(E_\text{b}/N_\text{0})br  \zeta \nu }. \label{equ:IEEE_CIR015a}	
\end{align}
There, $P_\text{s}$ represents the average signal power per time-domain sample measured at the receiver input. Moreover, $b$ is the number of bits per data symbol ($b=2$ for \ac{qpsk}), $r$ represents the code rate of the channel code, and  $\zeta= N_\text{c} / (N_\text{cp} + N_\text{c})$ accounts for the time domain samples in the \ac{cp}. The parameter $\nu$ accounts for the additional redundancy discussed in Sec.~\ref{sec:Robust_DDM}. Thus, $\nu=\frac{1}{4}$ for the \ac{mimo} \ac{ofdm} system employing \ac{ddm}. \ac{rdm} adds a  similar redundancy \cite{Lang_RDM_JP}, such that $\nu=\frac{1}{4}$ is chosen also for the \ac{mimo} \ac{ofdm} system utilizing \ac{rdm}. All other considered systems do not add any additional redundancy such that $\nu=1$ is chosen for them. As a result, \ac{mimo} \ac{ofdm} systems employing \ac{ddm} and \ac{rdm} observe a higher noise variance $\sigma_\text{n}^2$. 

The \ac{ber} curves are simulated for three simulation scenarios detailed in the following.

\subsubsection{Perfect Channel Knowledge; Perfect Synchronization}

In this first simulation, the receiver perfectly knows the channel between transmitter and receiver. The observed \ac{ber} curves for uncoded and coded transmission are shown in Fig.~\ref{fig:BER_optimal}. While for the uncoded case \ac{rdm} has a small  advantage in \ac{ber} performance over \ac{ddm}, both systems feature approximately the same \ac{ber} performance in the coded case and outperform the remaining systems by approximately $1.6\, \text{dB}$. This gain in \ac{ber} performance is a result of the diversity gain elaborated on in Sec.~\ref{sec:Robust_DDM}.  

As argued in \cite{Lang_RDM_JP}, granting this diversity gain also to \ac{mimo} systems utilizing \ac{esi} and \ac{dsi} by means of adding additional redundancy would increase their \ac{ber} performances as well at the cost of a reduced data rate. %In fact, adding the same redundancy also to \ac{esi} would lead to a quite similar \ac{ber} performance as observed for \ac{rdm} and \ac{ddm}.


The remaining simulations are shown for coded transmission only, since uncoded transmission is not relevant for real-world applications. 

\input{./fig/Fig_BER_uncoded_optimal}

\subsubsection{Perfect Synchronization; Imperfect Channel Estimation based on Preamble OFDM Symbols}

Now, the channel is not perfectly known but rather estimated using the procedure derived in Sec.~\ref{sec:channel_estimation}. The channel estimation procedure for the \ac{siso} \ac{ofdm} system is described in \cite{Diss_Hofbauer, Lang_Asilomar_2014}. The channels for the \ac{mimo} \ac{ofdm} systems utilizing \ac{esi} and \ac{dsi} are estimated with the \ac{blue} \cite{Kay-Est.Theory, Diss_Lang_Oliver}, whose derivations are omitted in this work.

For a fair comparison by means of having the similar distortions on the channel estimates, all three systems shall have the same effective \ac{snr} for the averaged preamble \ac{ofdm} symbols \cite{Lang_RDM_JP}. Hence, the increased noise variance $\sigma_\text{n}^2$ for the \ac{mimo} \ac{ofdm} systems with \ac{ddm} and \ac{rdm} is compensated by employing $N_\text{pr}=16$ preamble \ac{ofdm} symbols, while the other systems use $N_\text{pr}=4$. 

The resulting \ac{ber} curves are shown in Fig.~\ref{fig:BER_channel_estimation}. This figure also visualizes the simulation results for the case of perfect channel knowledge from Fig.~\ref{fig:BER_optimal} as reference. While the \ac{mimo} systems are less prone to imperfect channel knowledge, the loss in performance for all three systems is moderate\footnote{We note that in practice, not only the ECIRs but also the underlying CIRs may become highly time-varying for the assumed relative velocity between $\pm 60\,\text{m/s}$. This may entail the necessity of more frequent channel estimation or advanced channel tracking algorithms, whose analysis is beyond the scope of this work.}. 

\input{./fig/Fig_BER_channel_estimation}


\subsubsection{Perfect Channel Knowledge; Imperfect Synchronization using Pilot Subcarriers}

Now, the channels are perfectly known, but the synchronization is performed using pilot subcarriers rather than having perfect synchronization. The \ac{siso} \ac{ofdm} system and the \ac{mimo} \ac{ofdm} systems utilizing \ac{esi} and \ac{dsi} employ $N_\text{p}=16$ pilot subcarriers. The \ac{mimo} \ac{ofdm} systems with \ac{ddm} and \ac{rdm} employ $N_\text{p}=64$ pilot subcarriers in order to ensure the same effective \ac{snr} as argued previously. The \ac{ber} curves visualized in Fig.~\ref{fig:BER_synchronization} show that the loss in \ac{ber} performance is minor for all considered systems. 


\input{./fig/Fig_BER_synchronization}



\subsubsection{Perfect Channel Knowledge; Perfect CPE Synchronization; Disabled ICI}

The loss in \ac{ber} performance due to \ac{ici}-induced distortions is analyzed by disabling \ac{ici}, while the channel is assumed to be known and the \ac{cpe} is compensated perfectly. The resulting \ac{ber} curves in Fig.~\ref{fig:BER_woICI} indicate that the loss in \ac{ber} performance is negligible for all considered multiplexing methods with the chosen parametrization. 
%The reason for this behavior is the additional redundancy by means of transmitting the same data on 4 subcarriers in parallel. This causes an averaging of the \ac{ici}-induced distortions.  

\begin{figure}[!t]
\begin{center}
\begin{tikzpicture}
\begin{axis}[compat=newest, 
width=.98\columnwidth, height = .5\columnwidth, grid, xlabel={$E_\text{b}/N_0$ (dB)}, 
ylabel={$\text{log}_{10}(\text{BER})$}, 
%y label style={at={(-0.15,0.5)}},
legend pos=south east, 
legend cell align=left,
%legend style={text width=3.0cm},
legend columns=5, 
xmin = 0,
xmax = 12,
ymin = -6,
ymax = 0,
%restrict y to domain=-inf:20,
ytick={0, -1, -2, -3, -4, -5, -6},
legend style={
at={(1.0, 1.27)},
anchor=north east,font=\tiny}
]



% ESI QPSK
\addplot[line width=1pt][color=gray, solid, every mark/.append style={solid},mark=star,mark repeat = 2] table[x index =0, y index =1] {SimData/FP_ESI_Ntx4_Nc1024_Nsym256_R12_CI1_CK0_CD2_Nbps2_Nprea4_Npi16.dat};
\addlegendentry{{\footnotesize ESI}}

% NeqDySI QPSK
\addplot[line width=1pt][color=gray, solid, every mark/.append style={solid},mark=square,mark repeat = 3] table[x index =0, y index =1] {SimData/FP_DSI_Ntx4_Nc1024_Nsym256_R12_CI1_CK0_CD2_Nbps2_Nprea4_Npi16.dat};
\addlegendentry{{\footnotesize NeqDySI}}

% SISO QPSK
\addplot[line width=1pt][color=black, solid] table[x index =0, y index =1] {SimData/FP_SISO_Ntx4_Nc1024_Nsym256_R12_CI1_CK0_CD2_Nbps2_Nprea4_Npi16.dat};
\addlegendentry{{\footnotesize SISO}}

% RDM QPSK
\addplot[line width=1pt][color=lightgray, solid, every mark/.append style={solid},mark=+,mark repeat = 3] table[x index =0, y index =1] {SimData/FP_RDMult_Ntx4_Nc1024_Nsym256_R12_CI1_CK0_CD2_Nbps2_Nprea4_Npi16.dat};
\addlegendentry{{\footnotesize RDMult}}

% DDM QPSK
\addplot[line width=1pt][color=black, solid, every mark/.append style={solid},mark=o,mark repeat = 2] table[x index =0, y index =1] {SimData/FP2_DDM_Ntx4_Nc1024_Nsym256_R12_CI1_CK0_CD2_Nbps2_Nprea4_Npi16.dat};
\addlegendentry{{\footnotesize DDM}}





% ESI QPSK uncoded
\addplot[line width=1pt][color=gray, dashed, every mark/.append style={solid},mark=star,mark repeat = 2] table[x index =0, y index =1] {SimData/FP_ESI_Ntx4_Nc1024_Nsym256_R12_CI0_CK0_CD2_Nbps2_Nprea4_Npi16.dat};
%\addlegendentry{{\footnotesize SISO}}

% NeqDySI QPSK
\addplot[line width=1pt][color=gray, dashed, every mark/.append style={solid},mark=square,mark repeat = 3] table[x index =0, y index =1] {SimData/FP_DSI_Ntx4_Nc1024_Nsym256_R12_CI0_CK0_CD2_Nbps2_Nprea4_Npi16.dat};
%\addlegendentry{{\footnotesize NeqDySI}}

% SISO QPSK uncoded
\addplot[line width=1pt][color=black, dashed] table[x index =0, y index =1] {SimData/FP_SISO_Ntx4_Nc1024_Nsym256_R12_CI0_CK0_CD2_Nbps2_Nprea4_Npi16.dat};
%\addlegendentry{{\footnotesize SISO}}

% RDM QPSK uncoded
\addplot[line width=1pt][color=lightgray, dashed, every mark/.append style={solid},mark=+,mark repeat = 3] table[x index =0, y index =1] {SimData/FP_RDMult_Ntx4_Nc1024_Nsym256_R12_CI0_CK0_CD2_Nbps2_Nprea4_Npi16.dat};
%\addlegendentry{{\footnotesize RDMult}}

% DDM QPSK uncoded
\addplot[line width=1pt][color=black, dashed, every mark/.append style={solid},mark=o,mark repeat = 2] table[x index =0, y index =1] {SimData/FP_DDM_Ntx4_Nc1024_Nsym256_R12_CI0_CK0_CD2_Nbps2_Nprea16_Npi16.dat};
%\addlegendentry{{\footnotesize DDM}}



\end{axis}
\end{tikzpicture}
\caption{BER curves for code rate $r=1/2$, perfect synchronization and perfect channel knowledge. Simulation results are shown for activated (solid) and deactivated (dashed) ICI. The curves for DDM and RDMult as well as the curves for SISO, ESI and NeqDySI lie almost on top of each other.   
\label{fig:BER_woICI} }
\end{center}
\end{figure}
 