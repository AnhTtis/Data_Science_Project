
\section{Preliminaries}
\label{sec:preliminaries}
%\paragraph{Satisfiability Modulo Theories.}
We work in the context of \emph{Satisfiability Modulo Theories} (SMT). Our theory of interest is the quantifier-free theory of non-linear real arithmetic augmented with trigonometric and exponential transcendental functions, \smtnta. We assume that the reader is familiar with  standard SMT terminology~\cite{SMTBackground}.
% notions of interpretation, model, satisfiability and boolean abstraction. 

\emph{Notation.} We denote \smtnta-formulas by $\phi, \psi$, clauses by $C_1, C_2$, literals by $l_1, l_2$, real-valued variables by $x_1, x_2,\dots$,
 constants by $a, b$, intervals of real values by $I = [a,b]$, boxes by  $B=I_1 \times \cdots \times I_n$, 
 logical terms with addition, multiplication and transcendental function symbols by $f, g$, and multivariate real functions with $F, G, H$. For any formula $\phi$, we denote by $\Vars{\phi}$ the set of its real-valued variables. When there is no risk of ambiguity we write $f, g$ to also denote  the real-valued functions corresponding to the standard interpretation of the respective terms. We assume that formulas are in Conjunctive Normal Form (CNF) and that their atoms are in the form $f \bowtie 0$, with $\bowtie\;\in \{=, \leq, <\}$. We remove the negation symbol by rewriting every occurrence of $\neg (f = 0)$ as $(f<0 \lor 0< f)$ and distributing $\neg$ over inequalities.


\myparagraph{Points and boxes.} Since we have an order on the real-valued variables $x_1,x_2,\dots$, for any set of variables $V\subseteq \{x_1,x_2,\dots\}$ we can view an assignment $p: V\rightarrow \mathbb{R}$ equivalently as the $|V|$-dimensional point $p\in \mathbb{R}^{|V|}$, and an \textit{interval assignment} $\beta : V \to \{[a,b] : a,b\in \R\}$ equivalently as the $|V|$-dimensional box $B \subseteq \R^{|V|}$. By abuse of notation, we will use both representations interchangeably, using the type $\mathcal{R}^{V}$ both for assignments in $V\rightarrow \mathbb{R}$ and points in $\mathbb{R}^{|V|}$, and the type $\mathcal{B}^{V}$ both for interval assignments in $V\rightarrow \{[a,b] : a,b\in \R\}$ and corresponding boxes. This
will allow us to apply mathematical notions usually defined on points or boxes to such assignments, as well. Given a point  $p\in \mathcal{R}^{V}$, and a subset $V'\subseteq V$, we denote by $\mathit{proj}_{V'}(p)\in \mathcal{R}^{V'}$ the projection of $p$ to the variables in $V'$, that is, for all $v\in V'$, $\mathit{proj}_{V'}(p)(v):= p(v)$. Finally, % 
we will use variable assigments $\nu: V\rightarrow \mathbb{Q}$ as substitutions, denoting by $\nu(\phi)$ the result of replacing every variable $v\in V\cap \Vars{\phi}$ in $\phi$ by $\nu(v)$.

%Analogically, we also define the projection of boxes. 

\myparagraph{Systems of equations and inequalities.} We say that a formula $\phi$ %with no boolean variables
that contains only conjunctions of atoms in the form $f=0$ and $g\leq 0$  is a \emph{system of equations and inequalities}. If $\phi$ contains only equations (inequalities) then we say it is a \emph{system of equations} (\emph{inequalities}). A system of equations ${f_1 = 0 \land \cdots \land f_n=0}$, where the $f_1,\cdots,f_n$ are terms in the variables $x_1, \cdots, x_m$, can be seen in an equivalent way as the equation $F=0$, where $F$ is the real-valued function ${F := f_1 \times \cdots \times f_n : \R^m \to \R^n}$ and $0$ is a compact way to denote the point $(0,\cdots,0)\in \R^n$. 
Analogously, we can see a system of inequalities ${g_1 \leq 0 \land \cdots \land g_k \leq 0}$ as the inequality $G\leq0$,  where $G$ is the real-valued function ${G := g_1 \times \cdots \times g_k : \R^m \to \R^k}$ and $\leq$ is defined element-wise. We will write $eq(\phi)$ for the function $F$ defined by the equations in the formula $\phi$ and ${ineq(\phi)}$ for the function $G$ defined by the inequalities in~$\phi$. We say that a system of equations and inequalities is \emph{bounded} if every free variable appearing in the system has an associated bound consisting of a closed interval with rational endpoints. The handling of strict inequalities would be an easy, but technical extension of our method, which we avoid to stream-line the presentation.



\myparagraph{Dulmage–Mendelsohn decomposition.} Given a system of equations~$\phi$, it is possible to construct an associated bipartite graph $\mathcal{G}_\phi$ that
represents important structural properties of the system. This graph
has one vertex per equation, one vertex
per variable, and an edge between a variable $x_i$ and an equation $f_j=0$ iff $x_i$ appears in $f$. The Dulmage–Mendelsohn decomposition~\cite{dulmage_mendelsohn_1958,dulmage_mendelsohn_system_of_equations} is a canonical decomposition from the field of matching theory that partitions the system into three sub-systems: an over-constrained one (more equalities than variables), an under-constrained one (less equalities than variables), and a well-constrained one (as many equalities as variables, and contains no over-constrained subsystem, i.e. it satisfies the Hall property~\cite{Hall:35}). 


%\example
\begin{example}
	Let $\phi:= x-\tan(y)=0 \land z^2=0 \land w=0 \land \sin(w)=0$. Through the Dulmage–Mendelsohn decomposition we obtain an under-constrained sub-system $x-\tan(y)=0$ (two variables, one equation), a well-constrained sub-system $z^2=0$ (one variable, one equation), and an over-constrained sub-system $w=0 \land \sin(w)=0$ (one variable, two equations).
\end{example}
	



\myparagraph{Topological degree.} The notion of the degree of a continuous function (also called the topological degree) comes from differential topology~\cite{Fonseca:95,BrouwerDegreeDincaMawhin}. 
For a continuous function $F: B\subseteq\mathbb{R}^n\rightarrow \mathbb{R}^n$ and a point $p\in \mathbb{R}^n$ such that $p\not\in F(\partial B)$ (where $\partial B$ is the topological boundary of $B$), the degree $\deg(F, B, p)$
is a computable~\cite{Aberth:94,Franek:12b} integer
that  satisfies several interesting properties. The one that plays a pivotal role in our work is the \emph{topological degree test}, which states that: if ${\deg(F, B, 0) \neq 0}$, then the equation $F = 0$ has a solution in $B$.

%For $p = 0$, this integer provides information about the roots of $F$ in $B$. 

In the one-dimensional case (i.e., $n=1$ and $B$ being an interval $[a,b]$ with $f(a)\neq p$ and $f(b)\neq p$), the topological degree test is analogous to the corollary of the Intermediate Value Theorem commonly known as Bolzano's Theorem since
 $\deg(F, B, 0)= \frac{\mathrm{sgn}(F(b))-\mathrm{sgn}(F(a))}{2}$. 
 So, for $F(x)=x$, ${\deg(F, [-1,1], 0)=\frac{1-(-1)}{2}=1}$, 
 %for $F(x)=-x$, $\deg(F, [-1,1], 0)=-1$, and 
 while for $F(x)=x^2$, $\deg(F, [-1,1], 0)=0$. 

The topological degree generalizes this to functions with $n>1$.
In particular, if $p$ is regular (i.e., for all $y\in B \cap F^{-1}(p)$, $\det F'(y)\neq 0$) then the topological degree of $F$ at $p$ can be defined as
 $\deg(F,B, p) \defas  \sum\limits_{y\in B \cap F^{-1}(p)} \mathrm{sgn}\det F'(y)$. 
This definition can be further extended to non-regular values $p$ in a unique way by continuity of $\deg(F,\Omega, p)$ as a function in $p$, or alternatively, be defined axiomatically (see~\cite{topDegBook} for more details).

While
$\deg(F, B, 0)\neq 0$ implies that $F$ has a root in $B$,
the converse is not true, and the existence of a root does not imply nonzero degree in general. Still, 
if a box contains one isolated zero with non-singular Jacobian matrix, then the topological degree is non-zero~\cite{Fonseca:95}. For alternatives to the topological degree test see our discussion of related work.

\myparagraph{Interval Arithmetic.}
The basic algorithmic tool that underlies our approach is floating point interval arithmetic ($\intervalArithmOperator$)
~\cite{Rump:10,Neumaier:90,Moore:09}
which, given a box~$B$ and an $\nta$-term representing a  function $H$, is able to compute an interval $\intervalArithmOperator_H(B)$ that over-approximates the range $\{ H(x) \mid x\in B\}$ of $H$ over $B$. Since this is based on floating point arithmetic, the time needed for computing $\intervalArithmOperator_H(B)$ does not grow with the size of the involved numbers. Moreover conservative rounding guarantees correctness under the presence of round-off errors.
In the paper, we will use interval arithmetic within topological degree computation~\cite{Franek:12b}, and as a tool to prove the validity of inequalities on boxes.



\myparagraph{Robustness.} We say that a formula $\phi$ is robust if there exists some $\epsilon>0$ such that either every formula that is the result of an $\epsilon$-small perturbation of $\phi$ 
(i.e. a formula whose \emph{distance} from $\phi$ is less than $\epsilon$%, for details see Section \ref{subsec:BackgroundRobustness}
) 
is satisfiable, or either every such formula is unsatisfiable. For example, the satisfiable formula $x^2=0$ is not robust, since for every $\epsilon>0$, the perturbed formula $x^2+\epsilon/2=0$ is unsatisfiable. In contrast to that, the satisfiable formula $x^3=0$ is robust.  If $\phi$ is both robust and (un)satisfiable, we say that it is robustly (un)sat. Hence, $x^3=0$ is robustly sat.

For the first part of the paper, an intuitive understanding of the notion of $\epsilon$-small perturbation suffices. In Section~\ref{subsec:BackgroundRobustness}, we will make this more precise for a more formal analysis of our approach. For more details, we refer the reader to the literature~\cite{Franek:12}.
\\
\indent \emph{Relation between robustness and system of equations}: An over-constrained system of equations is never robustly sat~\cite[Lemma 5]{Franek:12}. It easily follows that a system of equations that contains an over-constrained sub-system (in the sense of the Dulmage–Mendelsohn decomposition) is never robustly sat as well.
\\
\indent \emph{Relation between robustness and topological degree}: Even in the case of an isolated zero, the test for non-zero topological degree can fail if the system is non-robust. For example, the function $F(x) \equiv x^2$ has topological degree $0$ in the interval $[-1,1]$%
, although the equality $x^2=0$ has an isolated zero in this interval.
% However, the zero of $x^2=0$ is not robust: it can vanish under arbitrarily small changes of the function denoted by the left-hand side $x^2$.
It can be shown that the topological degree test is able to prove satisfiability in all robust cases for a natural formalization of the notion of robustness~\cite{Franek:12}. We will not provide such a formalization, here, but use robustness as an intuitive measure for the potential success when searching for a certificate. 


\myparagraph{Logic-To-Optimization.}  
While symbolic methods usually struggle dealing with \nta, numerical methods, albeit inexact, can handle transcendental functions efficiently. For this reason, an SMT solver can benefit from leveraging numerical techniques. In the Logic-To-Optimization approach~\cite{ATVApaper,xsat,SETTA}, 
an \smtnta-formula $\phi$ in $m$ variables 	is translated into a real-valued non-negative function $\LtoO(\phi) \equiv H: \R^{m} \mapsto \R^{\ge 0}$ 
such that---up to a simple translation between Boolean and real values for Boolean variables---each model of $\phi$ is a zero of $H$ (but not vice-versa). When solving a satisfiability problem, one can try to first minimize this function through numerical methods,  then use the obtained numerical (approximate) solution to prove, through exact %symbolic
methods, that the logical formula has indeed a model.
 
While for the complete definition of the $\LtoO$ operator we refer to~\cite[Section 3]{ATVApaper}, 
we now provide a simple example to give an intuition on how the operator works. 
Given a formula of the form $F=0$, we have that $\LtoO(F=0) \equiv  F^2$, i.e.,  
for every $x$ in the domain of $F$,  $(\LtoO (F))(x) = F(x)^2$. 
For conjunctions, $\LtoO(F_1 \land F_2) \equiv \LtoO(F_1) + \LtoO(F_2)$.
For disjunctions,  $\LtoO(F_1 \lor F_2) \equiv \LtoO(F_1) * \LtoO(F_2)$

Now, consider for example $F_1, F_2:\mathbb{R}\to \mathbb{R}$ defined by $F_1(x) \equiv x^2-2x$ and $F_2 \equiv \sin(x)$. The formula $F_1=0$ has exactly 2 solutions, $\{0, 2\}$, which are exactly the zeros (hence the global minima) of the non-negative function $F_1^2: \mathbb{R}\to \mathbb{R}$, while the formula $F_2 = 0$ has infinitely many solutions $\{k\pi \ |\  k\in \mathbb{Z} \}$, which are exactly the zeros (and global minima) of $F_2^2: \mathbb{R}\to \mathbb{R}$. Then, in order to find the solutions of $\phi \equiv (F_1 = 0 \land F_2 = 0)$ (which, in this case, consist of the singleton $\{0\}$), we can search for the zeros of $\LtoO(\phi)$. 
%When the complexity of the constraints is huge, by using $\LtoO$ we can leverage numerical optimization to quickly get  candidate solutions,  and then only use exact methods on local subproblems of the original formula.




%%% Local Variables:
%%% mode: latex
%%% TeX-master: "./main.tex"
%%% End:

