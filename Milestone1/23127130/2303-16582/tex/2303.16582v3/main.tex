%\documentclass[11pt]{article}
\documentclass{llncs}
%\documentclass{sn-jnl}
\usepackage{orcidlink}

\title{Satisfiability of
  Non-Linear Transcendental Arithmetic as a
  Certificate Search Problem}

%corresponding?
%emails?

\usepackage{url}
%\usepackage[toc,page]{appendix}
\usepackage{todonotes}
\usepackage{hyperref}


\usepackage{tableof}

% To disable colors, comment the following row and de-comment the next one
\usepackage{xcolor}
%\usepackage[monochrome]{xcolor}

%\usepackage{amsthm}
\usepackage{amsmath}
\usepackage{amsfonts}
\usepackage{graphicx}
\usepackage{xspace}
\usepackage{hhline}
\usepackage{makecell}
\usepackage{colortbl} %used only for coloring table in draft

\usepackage{amssymb}

\definecolor{purple}{rgb}{1, 0, 1}

\newcommand{\ie}{\emph{i.e.,}\xspace}
\newcommand{\eg}{\emph{e.g.,}\xspace}
\newcommand{\abr}{\emph{abbr.}\xspace}
\newcommand{\ea}{\emph{et al.}\xspace}
\newcommand{\gensync}{\emph{GenSync}\xspace}
\newcommand{\colosseum}{\emph{Colosseum}\xspace}
\newcommand{\srep}{\emph{SREP}\xspace} % Set Reconciliation Enhances
\newcommand{\srepsim}{\emph{SREPSim}\xspace}
% Propagation
\newcommand{\esrep}{\emph{E-SREP}\xspace}
\newcommand{\epsrep}{\emph{EP-SREP}\xspace}
\newcommand{\mesrep}{\emph{ME-SREP}\xspace}
\newcommand{\mempoolsync}{\emph{MempoolSync}}

\newcommand{\fref}[1]{Fig.~\ref{#1}}
\newcommand{\tref}[1]{Table~\ref{#1}}
\newcommand{\aref}[1]{Algorithm~\ref{#1}}
\newcommand{\procref}[1]{Procedure~\ref{#1}}
\newcommand{\sref}[1]{Section~\ref{#1}}
\newcommand{\lineref}[1]{line~\ref{#1}}
\newcommand{\appref}[1]{Appendix~\ref{#1}}

% Change \eqref
\LetLtxMacro{\originaleqref}{\eqref}
\renewcommand{\eqref}{Eq.~\originaleqref}

% Theorems and corollaries
\newcounter{theoremcount}
\setcounter{theoremcount}{0}
\DeclareRobustCommand{\theorem}[1]{%
  \refstepcounter{theoremcount}%
  \noindent\textit{\textbf{Theorem \thetheoremcount\label{theorem:#1}: }}%
}
\DeclareRobustCommand{\theoremref}[1]{Theorem~\ref{theorem:#1}}

\DeclareRobustCommand{\proof}{\emph{Proof:}\xspace}
\DeclareRobustCommand{\qqed}{\hfill$\blacksquare$}

\newcounter{corollcount}
\setcounter{corollcount}{0}
\DeclareRobustCommand{\coroll}[1]{%
  \refstepcounter{corollcount}%
  \noindent\textit{\textbf{Corollary \thecorollcount\label{coroll:#1}: }}%
}
\DeclareRobustCommand{\corollref}[1]{Corollary~\ref{coroll:#1}}

\newcounter{lemmacount}
\setcounter{lemmacount}{0}
\DeclareRobustCommand{\lemma}[1]{%
  \refstepcounter{lemmacount}%
  \noindent\textit{\textbf{Lemma \thelemmacount\label{lemma:#1}: }}%
}
\DeclareRobustCommand{\lemmaref}[1]{Lemma~\ref{lemma:#1}}

\newcounter{definitioncount}
\setcounter{definitioncount}{0}
\DeclareRobustCommand{\definition}[1]{%
  \refstepcounter{definitioncount}%
  \noindent\textit{\textbf{Definition \thedefinitioncount\label{definition:#1}: }}%
}
\DeclareRobustCommand{\defref}[1]{Definition~\ref{definition:#1}}

%notes of different authors
\newif\ifnotes
\notestrue
\notesfalse

\newif\ifdiff
\difftrue
\difffalse

\newcommand{\anote}[1]{\ifnotes $\ll$\textsf{\textcolor{purple}{Ari: {#1}}}$\gg$ \fi}
\newcommand{\nnote}[1]{\ifnotes $\ll$\textsf{\textcolor{orange}{Novak: {#1}}}$\gg$ \fi}
\newcommand{\diff}[1]{\ifdiff\textcolor{orange}{#1}\else#1\fi}

%%% Local Variables:
%%% mode: latex
%%% TeX-master: "main"
%%% End:


\begin{document}


\author{Enrico Lipparini\inst{1,2}\orcidlink{0009-0009-0428-4403}\and
	Stefan Ratschan\inst{3} \orcidlink{0000-0003-1710-1513} 
}

\institute{
	University of Genoa, Italy 	\and
	University of Cagliari, Italy \and
	Institute of Computer Science of the Czech Academy of Sciences
}


% TODO
\setlength{\marginparwidth}{4cm}
\reversemarginpar




\maketitle


\begin{abstract}

\begin{center}
\end{center}
	\noindent
	For typical first-order logical theories, satisfying assignments have a straightforward finite representation that can directly serve as a certificate that a given assignment satisfies the given formula. For non-linear real arithmetic augmented with trigonometric and exponential functions (\nta), however, there is no known direct representation of satisfying assignments that allows for a simple independent check of whether the represented numbers exist and satisfy the given formula. Hence, in this paper, we introduce a different form of satisfiability certificate for \nta, and formulate the satisfiability problem as the problem of searching for such a certificate.
	% This does not only ease the independent verification of satisfiability, but also allows the design of new algorithms that show satisfiability by systematically searching for such certificates. Moreover, it provides a basis for studying algorithmically solvable sub-classes of  the undecidable theory of non-linear real arithmetic with transcendental functions, which we characterize by providing lower and upper bounds in terms of relevant well-known classes, and for which we show the existence of a terminating procedure.
	% Computational experiments document that the resulting algorithms are able to prove satisfiability of a substantially higher number of benchmark problems than existing methods.
	This does not only ease the independent verification of satisfiability, but also allows the design of new algorithms that show satisfiability by systematically searching for such certificates.          Computational experiments document that the resulting algorithms are able to prove satisfiability of a substantially higher number of benchmark problems than existing methods.         
We also characterize the formulas  whose satisfiability can be demonstrated by such a certificate, by providing lower and upper bounds in terms of relevant well-known classes.
Finally we show the existence of a procedure for checking the satisfiability of \nta-formulas that terminates for formulas that satisfy certain robustness assumptions.        
\end{abstract}

%\setcounter{tocdepth}{5} %show more in the toc
%\tableofcontents

\section{Introduction}
\label{sec:introduction}
% \begin{itemize}
%     % Diffusion of FL
%     \item {\st{Diffusion of FL}}
%     % Security threats to FL
%     \item {\st{Security threats to FL with particular focus on model poisoning}}
%     % Limitations of existing countermeasures
%     \item {\st{Current countermeasures (e.g., KRUM) and their limitations}}
%     % Proposed method and its advantages
%     \item {\st{Intuitive description of the proposed method and its difference (i.e., advantages) w.r.t. state of the art}}
%     % Main contributions
%     \item {\st{Summary of the main contributions of this work}}
%     % Paper's structure and organization
%     \item {\st{Paper's structure and organization}}
% \end{itemize}

% Diffusion of FL
Recently, {\em federated learning} (FL) has emerged as the leading paradigm for training distributed, large-scale, and privacy-preserving machine learning (ML) systems~\cite{mcmahan2017googleai,mcmahan2017aistats}. 
The core idea of FL is to allow multiple edge clients to collaboratively train a shared, global model without disclosing their local private training data.
%Specifically, an FL system consists of a central server and many edge clients; 
A typical FL round involves the following steps: {\em(i)} the server randomly picks some clients and sends them the current, global model; {\em(ii)} each selected client locally trains its model with its own private data; then, it sends the resulting local model to the server;\footnote{Whenever we refer to global/local model, we mean global/local model {\em parameters}.} {\em(iii)} the server updates the global model by computing an \emph{aggregation function}, usually the average (FedAvg), on the local models received from clients.
% \begin{enumerate}
%     \item[{\em(i)}] the server sends the current, global model to the clients and appoints some of them for training;
%     \item[{\em(ii)}] each selected client locally trains its copy of the global model with its own private data; then, it sends the resulting local model back to the server;\footnote{Whenever we refer to global/local model, we mean global/local model {\em parameters}.}
%     \item[{\em(iii)}] the server updates the global model by computing an \emph{aggregation function} on the local models received from clients (by default, the average, also referred to as FedAvg~\cite{mcmahan2017aistats}).
% \end{enumerate}
This process goes on until the global model converges. %(e.g., after a certain number of rounds or other similar stopping criteria).
%\\
% The advantages of FL over the traditional, centralized learning paradigm are undoubtedly clear in terms of flexibility/scalability (clients can join/disconnect from the FL network dynamically), network communications (only model weights\footnote{We will use \textit{parameters} and \textit{weights} interchangeably.} are exchanged between clients and server), and privacy (each client's private training data is kept local at the client's end and not uploaded to the server).
\\
% Security threats to FL
%However, the growing adoption of FL also raises security concerns~\cite{costa2022covert}, particularly about its confidentiality, integrity, and availability.
Although its advantages over standard ML, FL also raises security concerns~\cite{costa2022covert}. %, particularly about its confidentiality, integrity, and availability~\cite{costa2022covert}.
% OLD, LONG VERSION
% Indeed, some work deals with privacy leakage that may expose the local data of some clients~\cite{melis2019sp}. 
% A large body of work, instead, investigates attacks that usually aim to detriment the predictive accuracy of the learned global model. For instance, \emph{data poisoning} attacks achieve this goal by letting an adversary pollute the training set of some corrupt FL clients with maliciously crafted examples~\cite{jagielski2018sp}.
% Similarly, in \emph{model poisoning} the attacker attempts to tweak the global model weights~\cite{bhagoji2019pmlr} by directly perturbing the local model's weights of some infected FL clients before these are sent to the central server for aggregation, usually via so-called Byzantine attacks. 
% It turns out that Byzantine model poisoning attacks severely impact standard FedAvg; therefore, more robust aggregation functions must be designed to make FL systems secure.
Here, we focus on \emph{untargeted model poisoning} attacks~\cite{bhagoji2019pmlr}, where an adversary attempts to tweak the global model weights %\footnote{We will use the terms \textit{parameters} and \textit{weights} interchangeably.} 
by directly perturbing the local model's parameters of some infected clients before these are sent to the central server for aggregation.
In doing so, the adversary aims to jeopardize the global model \textit{indiscriminately} at inference time.
Such model poisoning attacks severely impact standard FedAvg; therefore, more robust aggregation functions must be designed to secure FL systems.
\\
% In this paper, we focus on designing a novel robust aggregation scheme at the server's end to contrast the effect of Byzantine model poisoning attacks.
%
% Current countermeasures and their limitations
%Several countermeasures have been proposed in the literature to combat model poisoning attacks on FL systems.
% Some methods use simple statistics more robust than plain average to smooth the impact of malicious updates (e.g., Trimmed Mean and FedMedian~\cite{yin2018icml}). 
% Other defenses implement outlier detection techniques to discard malicious updates from the aggregation performed at the server's end. Those are either based on heuristics (e.g., Krum/Multi-Krum~\cite{blanchard2017nips} and Bulyan~\cite{mhamdi2018pmlr}) or data-driven approaches (e.g., K-means clustering~\cite{shen2016acm} or DnC via spectral analysis~\cite{shejwalkar2021ndss}). 
% Finally, some strategies rely on a centralized ``source of trust'' to spot potential malicious updates (e.g., FLTrust~\cite{cao2020fltrust}).
% Several countermeasures have been proposed in the literature to combat model poisoning attacks on FL systems, i.e., to discard possible malicious local updates from the aggregation performed at the server's end. 
% These techniques range from simple statistics more robust than plain average (e.g., Trimmed Mean and FedMedian~\cite{yin2018icml}) to outlier detection heuristics (e.g., Krum/Multi-Krum~\cite{blanchard2017nips} and Bulyan~\cite{mhamdi2018pmlr}) or data-driven approaches (e.g., spectral analysis via K-means clustering~\cite{shen2016acm} or spectral analysis), or methods based on ``source of trust'' (e.g., FLTrust~\cite{cao2020fltrust}).
% OLD, LONG VERSION
%Several countermeasures have been proposed in the literature to combat Byzantine model poisoning attacks on FL systems.
% Descriptive statistics
% For example, Trimmed Mean and FedMedian aggregate local model updates using more robust statistics than standard average~\cite{yin2018icml}.
%
% % Heuristics for outlier detection
% Many existing Byzantine-resilient strategies implement some outlier detection heuristics to discard the model updates sent by potentially malicious clients from the input of the aggregation function.
% One of the most popular heuristics is Krum~\cite{blanchard2017nips}.
% This strategy tries to mitigate the impact of Byzantine attacks by selecting as a global model the local model with the smallest sum of Euclidean distances to {\em all} the other local models.
% Although powerful, Krum requires the server to know (or, at least, estimate) the number of malicious FL clients upfront, which is generally impossible in a realistic attack scenario. %
% Moreover, Krum may become ineffective for complex, high-dimensional model parameter spaces due to the curse of dimensionality.
% Bulyan~\cite{mhamdi2018pmlr} tries to overcome this issue by combining Krum with a variant of Trimmed Mean.
% % Data-driven outlier detection
% Other strategies use data-driven outlier detection techniques -- e.g., via K-means clustering~\cite{shen2016acm} -- to spot potential malicious local model updates. 
% %For instance, Shen et al. propose to cluster local model updates with K-means and thus identify outliers.
%
% % Other techniques
% As far as the server is concerned, any local model received can be from a potential malicious client. 
% FLTrust~\cite{cao2020fltrust} assumes the server acts as a client, i.e., trains a local model on an additional {\em trustworthy} dataset at the server's end and compares it against all the local models from other clients. 
% This way, the server can rely on some ``source of trust'' when discarding potentially malicious clients.
%\\
% Limitations of existing Byzantine-resilient strategies
Unfortunately, existing defense mechanisms either rely on simple heuristics (e.g., Trimmed Mean and FedMedian by~\cite{yin2018icml}) or need strong and unrealistic assumptions to work effectively (e.g., foreknowledge or estimation of the number of malicious clients in the FL system, as for Krum/Multi-Krum~\cite{blanchard2017nips} and Bulyan~\cite{mhamdi2018pmlr}, which, however, cannot exceed a fixed threshold).
Furthermore, outlier detection methods using K-means clustering~\cite{shen2016acm} or spectral analysis like DnC~\cite{shejwalkar2021ndss} do not directly consider the temporal evolution of local model updates received.
Finally, strategies like FLTrust~\cite{cao2020fltrust} require the server to collect its own dataset and act as a proper client, thereby altering the standard FL protocol.
\\
% OLD, LONG VERSION
% Overall, existing Byzantine-resilient strategies are either simple heuristics (e.g., FedMedian) or, if they are more complex, they rely on strong and unrealistic assumptions to work effectively (e.g., knowing the number of malicious clients in the FL system in advance, as for Krum and alike).
% Furthermore, data-driven outlier detection methods do not consider the temporary evolution of local model updates received (e.g., K-means clustering). 
% Finally, strategies like FLTrust requires the server to collect its own dataset and act as a proper client, thereby altering the standard FL protocol.
%
% Description of the proposed method
This work introduces a novel pre-aggregation \textit{filter} robust to untargeted model poisoning attacks. Notably, this filter $(i)$ operates without requiring prior knowledge or constraints on the number of malicious clients and $(ii)$ inherently integrates temporal dependencies. 
The FL server can employ this filter as a preprocessing step before applying \textit{any} aggregation function, be it standard like FedAvg or robust like Krum or Bulyan.
Specifically, we formulate the problem of identifying corrupted updates as a multidimensional (i.e., matrix-valued) time series anomaly detection task. 
The key idea is that legitimate local updates, resulting from well-calibrated iterative procedures like stochastic gradient descent (SGD) with an appropriate learning rate, show \textit{higher predictability} compared to malicious updates. This hypothesis stems from the fact that the sequence of gradients (thus, model parameters) observed during legitimate training exhibit regular patterns, as validated in Section~\ref{subsec:intuition}. %until convergence. 
%This regularity may be more pronounced for smooth convex loss functions, but it can still be captured within an appropriate time window, even for more complex and convoluted loss surfaces. 
%We provide evidence of this claim in Appendix~B, where we show that the average mutual information (i.e., ``predictability''), calculated over pairs of legitimate model updates sent at different FL rounds, is significantly higher than the corresponding computation for a malicious client.
\\
Inspired by the matrix autoregressive (MAR) framework for multidimensional time series forecasting~\cite{chen2021je}, we propose the FLANDERS ({\em \textbf{F}ederated \textbf{L}earning meets \textbf{AN}omaly \textbf{DE}tection for a \textbf{R}obust and \textbf{S}ecure}) filter.
The main advantages of FLANDERS over existing strategies like FLDetector~\cite{zhao2020multivariate} are its resilience to large-scale attacks, where $50\%$ or more FL participants are hostile, and the capability of working under realistic non-iid scenarios.
We attribute such a capability to two key factors: $(i)$ FLANDERS works without knowing a priori the ratio of corrupted clients, and $(ii)$ it embodies temporal dependencies between intra- and inter-client updates, quickly recognizing local model drifts caused by evil players. Below, we summarize our main contributions:

\begin{itemize}
\item[{\em(i)}]
We provide empirical evidence that the sequence of models sent by legitimate clients is more predictable than those of malicious participants performing untargeted model poisoning attacks.
\\
\item[{\em(ii)}] 
We introduce FLANDERS, the first pre-aggregation filter for FL robust to untargeted model poisoning based on multidimensional time series anomaly detection.
\\
\item[{\em(iii)}] 
We integrate FLANDERS into Flower,\footnote{\scriptsize{\url{https://flower.dev/}}} a popular FL simulation framework for reproducibility.
\\
\item[{\em(iv)}] 
We show that FLANDERS improves the robustness of the existing aggregation methods under multiple settings: different datasets, client's data distribution (non-iid), models, and attack scenarios.
\\
\item[{\em(v)}] 
We publicly release all the implementation code of FLANDERS along with our experiments.\footnote{\scriptsize{\url{https://anonymous.4open.science/r/flanders_exp-7EEB}}}
\end{itemize}

% Paper's structure and organization
The remainder of the paper is structured as follows. %some related work and the current state-of-the-art solutions to security issues that FL entails. 
Section~\ref{sec:background} covers background and preliminaries. 
In Section~\ref{sec:related}, we discuss related work.
Section~\ref{sec:problem} and Section~\ref{sec:method} describe the problem formulation and the method proposed. % to tackle it. 
Section~\ref{sec:experiments} gathers experimental results. %, and Section~\ref{sec:limitations} discusses some limitations of this work.
Finally, we conclude in Section~\ref{sec:conclusion}.
 %discusses the limitations of this work and draws future research directions.
%reports conclusions and draws perspectives for future research directions.

%%%%%%% OLD %%%%%%%
%to overcome the resilience of Byzantine failures in distributed Stochastic Gradient Descent computations. 
% The strength of Krum is its time complexity, which is linear in the gradient dimension. 
% However, the robustness of the approach is guaranteed for gradient-based learning applications only when the majority of the clients are not compromised. 
% Besides, the aggregation mechanism of Krum, as well as that of similar methods, is robust from a coarse-grained perspective and does not provide solutions to errors and perturbations that may occur at inference time.
%A related approach to~\cite{blanchard2017nips} is the work of Su et al.~\cite{su2016dc}. Here, the authors propose an iterated approximate agreement to tackle a multi-layer scenario attacked by Byzantine agents. 
%However, the method works efficiently on the sole discrete context and it is inapplicable to continuous state environments.
%\gabri{Maybe, we should just talk about the main limitations of existing countermeasures without digging into their details (or, we can just mention Krum as this is the most popular one). I will move the description of all these methods to the Related Work section.}

\section{Notation and Preliminaries}\label{sec_prel}
Let $\mathbb{Z}_{>0}$ denote the set of positive integers and let $\mathbb{Z}_{[a,b]}$ denote the set of integers in the interval $[a,b]$. The $m\times m$ identity matrix is denoted by $I_m$ and its columns by $e_i$ for $i\in\mathbb{Z}_{[1,m]}$. We use $\mathbf{0}$ to denote a vector or a matrix of zeros of appropriate dimensions. For a sequence $\{z_k\}_{k=0}^{N-1}$ with $z_k\in\mathbb{R}^\eta$, we denote its stacked vector as $z = \begin{bmatrix}z_0^\top &z_1^\top & \dots & z_{N-1}^\top\end{bmatrix}^\top$ and a stacked window of it as $z_{[l,j]} = \begin{bmatrix}z_l^\top &z_{l+1}^\top & \dots & z_{j}^\top\end{bmatrix}^\top$ with $0\leq l<j$.\par
Persistence of excitation of a sequence and its extension to multiple sequences \cite{vanWaarde20} are defined as follows.
\begin{definition} The sequence \(\{z_k\}_{k=0}^{N-1}\), $z_k\in\mathbb{R}^{\eta}$, is said to be persistently exciting of order \(L\) if \(\textup{rank}(\mathscr{H}_{L}(z))=\eta L\), where $\mathscr{H}_L(z) = \begin{bmatrix}
		z_{[0,L-1]} & z_{[1,L]} & \cdots & z_{[N-L,N-1]}
	\end{bmatrix}$.
	\label{def_PE}
\end{definition}
\begin{definition}[\cite{vanWaarde20}]\label{def_cPE}
	The sequences $\{z_k^{(j)}\}_{k=0}^{N_j-1}$, with $z_k^{(j)}\in\mathbb{R}^\eta$ and $j\in\mathbb{Z}_{[1,r]}$, are said to be \textit{collectively persistently exciting} of order $L$ if rank$(\mathcal{H}_L(\mathscr{Z}))=\eta L$, where $\mathscr{Z} = \begin{bmatrix}
		(z^{(1)})^\top & \cdots & (z^{(r)})^\top
	\end{bmatrix}^\top,$ and
	\begin{equation*}
		\mathcal{H}_L(\mathscr{Z}) = \begin{bmatrix}
			\mathscr{H}_L(z^{(1)}) & \cdots & \mathscr{H}_L(z^{(r)})
		\end{bmatrix}.
	\end{equation*}
\end{definition}


\section{Goal}
\label{sec:goal}
Consider an SMT solver that takes as input some formula $\phi$ and as output an element of $\{ \Sat,\Unknown, \Unsat\}$.  How can we gain trust in the correctness of the result of such an SMT solver? One approach would be to ensure that the algorithm itself is correct. Another option is to provide a second algorithm whose output we compare with the original one. Both approaches are, however, very costly, and moreover, the latter approach still may be quite unreliable.

Instead, roughly following McConnell et. al.~\cite{McConnell:11} (see also Figure~\ref{fig:certifying}), we require our solver to return---in addition to its result---some information that makes an independent check of this result easy:

\begin{definition}
	An SMT solver is \emph{certifying} iff for an input formula $\phi$, in addition to an element $r\in \{ \Sat, \Unknown, \Unsat \}$, it returns an object~$w$ (a \emph{certificate}) such that
	\begin{itemize}
		\item $(\phi, r, w)$ satisfies a property $W$ where $W(\phi, r, w)$ implies that $r$ is a correct result for $\phi$, and 
		\item   there is an algorithm (a \emph{certificate checker}) that
		\begin{itemize}
			\item     takes as input a triple $(\phi, r, w)$ and returns $\top$ iff $W(\phi, r, w)$, and that
			\item is simpler than the SMT solver itself.
		\end{itemize}
		
	\end{itemize}
	
      \end{definition}

\begin{figure}[tbh]
\centering
\includegraphics[clip,trim={0cm 3cm 0cm 0cm},width=8cm]{certificate.pdf}
	\caption{Certifying SMT Solver}
	\label{fig:certifying}
\end{figure}




So, for a given formula $\phi$,  one can ensure correctness of the result $(r, w)$ of a certifying SMT solver by using a certificate checker to check the property $W(\phi, r, w)$. Since the certificate checker is simpler than the SMT solver itself, the correctness check is simpler than the computation of the result itself. 

The definition leaves it open, what precisely is meant by  ``simpler''. In general, it could either refer to the run-time of the checker, or to the effort needed for implementing the certificate checker and ensuring its correctness. The former approach is taken in computational complexity theory, the latter in contexts where correctness is the main concern~\cite{McConnell:11}. Indeed, we will later see that our approach succeeds in satisfying both requirements, although we will not use complexity-theoretic measures of run-time, but will measure run-time experimentally. 

The use of such certificates is ongoing research in the unsatisfiable case~\cite{Barbosa:22}. In the satisfiable case, for most theories, one can simply use satisfying assignments (i.e., witnesses) as certificates. Here the property $W$ simply is the property that the given assignment satisfies the formula, which can be checked easily.

For \smtnta, however, the situation is different: Here, no general finite representation of  satisfying assignments is available. Hence one needs to use certificates of a different form. 
We introduce the following definition:
\begin{definition}
Let $\phi$ be a formula in \nta. A \emph{(satisfiability) certificate} for $\phi$ is a triple $(\sigma, \nu, \setOfBoxes)$ such that $W(\phi, \Sat, (\sigma, \nu, \setOfBoxes))$ iff
	\begin{itemize}
	\item $\sigma$ is a function selecting a literal from every clause of $\phi$% such that no two selected Boolean literals contradict each other
		\item $\nu$ is a variable assignment in $\mathcal{R}^{V}$ assigning floating point numbers to a subset $V\subseteq\Vars{\sigma(\phi)}$ (where $\sigma(\phi)$ is a compact way of writing $\bigwedge_{ C \in \phi} \sigma(C)$), s.t. $\sigma(\phi)$
		 contains as many equations as real-valued variables not in $V$.
		\item $\setOfBoxes$ is a finite set of interval assignments in $\mathcal{B}^{\Vars{\phi}\setminus V}$% that assign floating point intervals
		such that their set-theoretic union as boxes
		is again a box $B_\beta$ and, 
		for the system of equations  $F:= eq(\nu(\sigma(\phi)))$ 
		and the system of inequalities $G:= ineq(\nu(\sigma(\phi)))$, it holds that:
	
		\begin{itemize}
			%\item the equalities form a square system, that is $F: \mathbb{R}^k\rightarrow\mathbb{R}^k$, for some $k$,
			\item $0\not\in F(\partial B_\beta)$,
			\item $deg(F, B_\beta, 0)\neq 0$, and
			\item for every $B\in\setOfBoxes$, $\intervalArithmOperator_G(B)\leq 0$.
		\end{itemize}
	\end{itemize}

\end{definition}

\begin{example}
Consider the formula 
\begin{alignat*}{2}
	& \qquad \qquad \qquad \qquad \qquad \phi := C_1 \land C_2 \land C_3 \land C_4  \\
	& C_1 \ \equiv \ cos(y) = 0 \ \lor \ sin(y) = e^x
	&& C_3 \ \equiv \ x-y \leq cos(z) \\ 
	& C_2 \ \equiv \  sin(y)=0 \ \lor \  cos(y) = sin(8x^2-z) 
	&& C_4 \ \equiv \ x+y \geq sin(z) 
      \end{alignat*}      
The following $(\sigma, \nu, \beta)$ is a certificate:
\begin{itemize}
	\item $\sigma := \{ C_1 \mapsto sin(y) = e^x\ ;\  C_2 \mapsto cos(y)=sin(8x^2-z)\ ; \\ C_3 \mapsto C_3\ ;\  C_4 \mapsto C_4\}$ 
	\item $\nu := \{ z \mapsto 0.2 \}$
	\item $\setOfBoxes:=\{B\}$, where $B := \{x \mapsto [-0.1,0.05]\ ;\ y \mapsto [1.4, 1.9] \} $
        \end{itemize}
      \end{example}

      As can be seen in Figure~\ref{fig:example}, the solution sets of $C_1$ and $C_2$ cross at a unique point in $B$, which reflects the fact that the degree of 
      the function $(x,y)\rightarrow (sin(y)-e^x,cos(y)-sin(8x^2-0.2))$ is non-zero. Moreover, the inequalities $C_3$ and $C_4$ hold on all elements of the box.
      \begin{figure}[tb]
\centering
        \includegraphics[width=8cm]{example.pdf}       
        \caption{Solution Sets of Equalities of Example Certificate}
        \label{fig:example}
      \end{figure}
\ 

Due to the properties of the topological degree and of interval arithmetic discussed in the preliminaries, we have:
\begin{property}
	$W(\phi, \Sat, (\sigma, \nu, \setOfBoxes))$ implies that $\phi$ is satisfiable.
\end{property}

Moreover, the topological degree can be computed algorithmically~\cite{Aberth:94,Franek:12b}, and one can easily write a certificate checker based on such an algorithm. Hence such a triple can be used as a certificate for satisfiability. 

In this paper, we will show that in addition to the discussed benefits for correctness, formulating satisfiability checking as the problem of search for such certificates also is beneficial for efficiency of the SMT solver itself. Since we will concentrate on satisfiability, we will simply ignore the case when an SMT solver returns $\Unsat$, so the reader can simply assume that an SMT solver such as the one from Figure~\ref{fig:certifying} only returns an element from the set $\{\Sat, \Unknown\}$.


% For some applications it might make sense to return more information than just ``satisfiable'' or ``unknown'' (e.g., an algorithm-dependent estimate for the probability for future success, or algorithm-independent information about the distance to satisfiability).

%%% Local Variables:
%%% mode: latex
%%% TeX-master: "./main.tex"
%%% End:


\section{Method}
\label{sec:method}

% \ml{``Inconsistent'' to ``large variation''}

% In this section, we propose our methods based on the observations in Section \ref{sec:motivation}.
In this section, we propose two techniques to further enhance the strong baseline to capture the variation of activation distributions better.
We first introduce spatial re-scaling to adapt the network to pixel-to-pixel variation.
We then propose channel-wise shifting and re-scaling to better capture the channel-to-channel variation.
Meanwhile, as both of the two methods are image-dependent, the image-to-image variation can be captured naturally.
By combining the two methods with our strong baseline, we build our enhanced BNN for SR, named EBSR.

% Because the activation distributions among pixels, channels and images have large variations \red{**are highly inconsistent} in SR networks, we introduce spatial re-scaling to adapt to pixel-wise variations and channel shift and re-scaling to adapt to channel-wise variations. And both of them are image-dependent to adapt to image-wise variations, which means during inference our network re-scales and shifts the distributions of activations flexibly for different input images. Based on these methods, we build an enhanced binary neural network for image super-resolution (EBSR).

% According to [3], the difference of activation magnitudes indicates different scaling factors are needed for each pixel.

\subsection{Spatial Re-scaling}
% It is better to use different scaling factors for different pixels to reduce the quantization error and retain more detailed information for image super-resolution. 

% \ml{In the main method, we do not need to introduce the previous works but can focus on introducing our own method. Channel rescaling in Real-to-binary Net is not relevant in this context.}

% Re-scaling the output of binary convolutions was proposed at the birth of BNN in XNOR-Net \cite{rastegari2016xnor} to reduce quantization error and improve accuracy for image classification tasks.
% It is computed as below:
% \begin{equation}
% \mathcal{A} * \mathcal{W} \approx(\operatorname{sign}(\mathcal{A}) \circledast \operatorname{sign}(\mathcal{W})) \odot \mathcal{K} \alpha
% \label{eq:xnor-net rescale}
% \end{equation}
% where $\circledast$ denotes the binary convolution and $\odot$ denotes the element-wise multiplication.
% $\mathcal{A}$, $\mathcal{W}$, $\alpha$, and $\mathcal{K}$ denote the activation, weight, weight scaling factor, and activation scaling factor, respectively.
%  Later in XNOR-Net++ \cite{bulat2019xnor}, Bulat et al. fuse the activation and weight scaling factors into a single one that is learned end-to-end based on gradients and this improves the classification accuracy on ImageNet dataset.

% % It is computed as Eq.~\ref{eq:xnor-net rescale}, where $\circledast$ denotes 
% %  the binary convolution and $\odot$ denotes the element-wise multiplication. The binary convolution of $\mathcal{A}$ and $\mathcal{W}$ is rescaled by the weight scaling factor $\alpha$ and the activation scaling factor $\mathcal{K}$, both of which are calculated analytically.


% \zc{Similarly, you should explain the meaning of A, W and the operators $\circledast$ in the formula}
% Then in Real-to-binary Net \cite{martinez2020training}, Martinez et al. used a data-driven channel re-scaling module that takes the pre-convolution activations as input to predict the activation scaling factor. Unlike that in XNOR-Net++ \cite{bulat2019xnor}, these scaling factors are not fixed during inference but rather inferred from data. By doing this, they further improved the classification accuracy on ImageNet over XNOR-Net++. 
As is shown in Figure \ref{fig:pixel}, activation distributions have large pixel-to-pixel variation in SR networks
and the difference of activation magnitudes indicates different scaling factors are preferred for different pixels.
Inspired by \cite{martinez2020training}, we propose spatial re-scaling to better adapt the network to the spatial variation
of activation distributions in SR networks.
% fit the various pixel-wise distributions in SR networks.
We take the real-valued activations $A$ before convolution as input and predict pixel-wise scaling factors $S(A)$, which re-scale the binary convolution output. Spatial re-scaling process can be formulated as follows:
\begin{equation}
A * W \approx(\operatorname{sign}(A) \circledast \operatorname{sign}(W)) \odot \alpha \odot S(A)
\label{eq:spatial rescale}
\end{equation}
where $\circledast$ denotes 
the binary convolution and $\odot$ denotes the element-wise multiplication. $A$, $W$, $\alpha$, and $S\left(A\right)$ denote real-valued activations, weights, the scaling factor of weights, and the spatial-wise scaling factor of activations respectively. $S\left(A\right) \in \mathbb{R}^{1\times H\times W}$ can be calculated with a convolution and a sigmoid function.
% as $\sigma\left( CONV\left(A\right)\right)$. 
As shown in Figure \ref{fig:method}(a), real-valued activations first go through a convolution layer,
which has an input channel of $C$ and an output channel of 1, 
and then pass through a sigmoid function to produce the scaling factors $S(A)$ along the spatial dimension.
During inference, the scaling factor will change dynamically according to different input feature maps.
By re-scaling binary convolution output using $S(A)$, we can reduce the quantization error and the original pixel-wise information in FP activation
will be preserved much better.
Spatial re-scaling leads to a large PSNR improvement of 0.24 dB (from 30.30 dB to 31.54 dB) on Set5 and 0.22 dB (from 25.09 dB to 25.31 dB)
on Urban100 compared with our strong baseline. 

\subsection{Channel-wise Shifting and Re-scaling}

\begin{table}[!tb]
\centering
\caption{Comparison between whether to fuse channel-wise shifting and re-scaling or not based on our baseline with spatial re-scaling. }
\label{tab:fusing}

\scalebox{0.65}{
\begin{tabular}{c|cc|cc|cc}
\hline
\multirow{2}{*}{Method}     & \multirow{2}{*}{OPs} & \multirow{2}{*}{Params} & \multicolumn{2}{c|}{Set5} & \multicolumn{2}{c}{Urban100} \\ \cline{4-7} 
                            &                      &                         & PSNR        & SSIM        & PSNR          & SSIM         \\ \hline
Baseline + spatial re-scale & 2.16G                & 0.05M                   & 31.54       & 0.883       & 25.31         & 0.759        \\
+ channel-wise shift and re-scale             & 2.34G                & 0.09M                   & 31.61       & 0.885       & 25.35         & 0.761        \\
+ w/ fusing                   & 2.27G                & 0.08M                   & \textbf{31.64}       & \textbf{0.885}       & \textbf{25.36}         & \textbf{0.761}        \\ \hline
\end{tabular}
}
\end{table}

In SR networks, activation distributions exhibit larger channel-to-channel variation (Figure \ref{fig:chl}).
Both the mean and magnitude of the activation distributions vary significantly across channels.
% Thus we use channel-wise shifting and re-scaling to adapt to various channel-wise distributions. 
\cite{martinez2020training} has proposed the data-driven channel re-scaling, 
but our method differs from them in further introducing data-driven thresholds to handle the channel-wise variation of both mean and magnitude.
Since the blocks to generate the scaling factors and thresholds are very similar, we further propose to fuse them into one module.
% and fusing channel-wise shifting and re-scaling into one module.
We evaluate the effect of fusing the two blocks in Table \ref{tab:fusing}.
With channel-wise shifting and re-scaling fused, our models have fewer operations and parameters overhead and slightly higher performance.

For the specific process, we take the real-valued activations as input and predict different thresholds and scaling factors for each channel. They are also image dependent, e.g., $\beta_{i}$ in Eq.\ref{eq:act_binarize} is no longer fixed during inference but generated according to different input feature maps. Channel-wise shifting and re-scaling can be formulated as follows:
\begin{equation}
A * W \approx(\operatorname{sign}(A-C_s(A)) \circledast \operatorname{sign}(W)) \odot \alpha \odot C_r(A)
\label{eq:channel-wise_shift_and_rescale}
\end{equation}
where $\circledast$ denotes 
the binary convolution and $\odot$ denotes the element-wise multiplication. $C_s(A), C_r(A) \in \mathbb{R}^{C\times1\times1}$ denote the channel-wise threshold and scaling factor, respectively. 
We show the block diagram in Figure \ref{fig:method}(b).
The real-valued input feature map is first squeezed to a ${C\times1\times1}$ vector by a global average pooling (GAP) layer.
The subsequent fully connected layers and ReLU learn the channel-wise information and output a ${2C\times1\times1}$ vector.
Then the ${2C\times1\times1}$ vector is split into two ${C\times1\times1}$ vectors.
We use the first $C$ channels as the channel-wise bias and pass the last $C$ channels through a sigmoid layer 
as the channel-wise scaling factor, which are used to shift the real-valued activations and re-scale the binary convolution output, respectively. 


% \ml{We can mention previously, channel-wise re-scale has been proposed. We propose to fuse them. Add the comparison between fuse v.s. no fuse.}

\begin{figure}[!tbp]%
  \centering
    \includegraphics[width=0.4\textwidth]{fig/methods.png}
  
% \subfloat[channel-wise shifting\&re-scale]{
%     \label{subfig:channel-wise shifting and re-scale}
%     \includegraphics[width=0.2\textwidth]{fig/chl shift and rescale.png}
%   }

  \caption{Block diagram for spatial re-scaling, and channel-wise shifting and re-scaling.} 
  % Input A is the real-valued activation tensor and C, H, and W denote its dimension. GAP stands for global average pooling. The reduction ratio r is set to 16 for a better trade-off between the performance and the number of operations and parameters.}
  \label{fig:method}
\end{figure}


\subsection{Network Structure}

Combining the spatial re-scaling and the channel-wise shifting and re-scaling methods, we construct the enhanced convolution layer (E-Conv).
Then we build our EBSR model based on E-Conv.
In Figure \ref{fig:E-conv}, we compare the binary convolution layer used in the baseline network and our proposed E-Conv.
We use spatial and channel-wise scaling factors to re-scale the binary convolution output,
and use channel-wise shifting to learn appropriate thresholds for each channel before binarization.
The scaling factors and threshold used in E-Conv are learnable and depend on the real-valued input activations.
In this way, our proposed EBSR can adapt to pixel-to-pixel, channel-to-channel, and image-to-image variations
to reduce the large binarization error and preserve more details.
% In this way, our proposed E-Conv reduces the large quantization error caused by binarization and keeps the original information of input feature maps to a large extent.


\begin{figure}[!tb]%
  \centering

    \includegraphics[width=0.5\textwidth]{fig/E-conv.png}

  \caption{Comparison of (a) the binary convolution layer with a skip connection used in our baseline network and (b) the proposed E-Conv.}
  \label{fig:E-conv}
\end{figure}


Figure \ref{fig:network} shows the basic block based on the E-Conv and our EBSR composed of the basic blocks. Following existing works, the convolution layers in the head and tail modules are not binarized. We choose the lightweight EDSR which has 16 basic blocks and 64 channels, and EDSR which has 32 basic blocks and 256 channels as our backbones, which correspond to EBSR-light and EBSR, respectively.

\begin{figure}[!tb]%
  \centering
  {
    \includegraphics[width=0.35\textwidth]{fig/network.png}
  }
  
  \caption{The structure of our proposed EBSR.  Convolution layers in purple are real-valued vanilla 3x3 convolutions.}
  \label{fig:network}
\end{figure}



\section{Certificate Search}
\label{sec:certificate-search}

In this section, we will discuss possibilities for search strategies by defining for every search tree node labeled with tuple  $\tau$, the ordered sequence $\Children{\tau}$ of choices for the next tuple element. Our framework allows for many more possibilities from which we choose strategies that both demonstrate the applicability of the framework to different search strategies, and allow for efficient search, as will be demonstrated by the computational experiments in Section~\ref{sec:experiments}.

In order to be able to refer to different variants of the search strategy in the description of computational experiments, we will introduce keywords for those variants that we will write in teletype font.

Here we will focus on strategies of the two following basic types: 
\begin{itemize}
\item Filtering: We skip elements from the set of possible choices that cannot result in a certificate or for which the probability of resulting in a certificate is negligible.
\item Ordering: We choose elements from the set of possible choices in a certain order that tries the reflect the probability of resulting in a certificate.
\end{itemize}

\subsection{Points}

The points $\Children{}=\langle p_1,\dots, p_k\rangle$ determining the first level of the search tree are generated by an optimization problem defined on the formula $\phi$
%in a way already described earlier
following the Logic-To-Optimization approach~\cite{ATVApaper}. Here we translate the satisfiability problem into a numerical minimization problem, mapping the logic formula $\phi$ into the non-negative real-valued function $\LtoO(\phi) \equiv H: \mathbb{R}^n\rightarrow\mathbb{R}_{\geq 0}$
(called the \textit{objective function}) such that for every satisfying assignment, this objective function is zero, and for assignments that do not satisfy the formula, the objective function is
typically (but not always) non-zero. Then we find local minima of $H$ through an unconstrained optimization algorithm such as basin hopping~\cite{basinhoppin}, a two-phases Monte Carlo Markov Chain method that alternates local minimization with random jumps. In our implementation, we compute $k=100$ local minima, and process them in the order of their value.


\subsection{Literals}


Given a point $p$, we now discuss how choose literal selector functions $\Children{p}=\langle \sigma_1,\dots, \sigma_k\rangle$.
For filtering the set of literal selector functions, we will restrict ourselves, 
for each clause $C\in\phi$, to the literals $l$ for which the objective function restricted to $l$ and evaluated in the point~$p$ is below a certain threshold. That is, we determine the set of approximately satisfiable literals $$L_C := \{l\in C \mid \LtoO(l)(p)\leq \epsilon  \}.$$ 
Our literal selector functions will then correspond to the set of all approximately satisfiable combinations $$\{\sigma \mid \text{for all } C\in\phi, \sigma(C)\in L_C \},$$ that is, each $\sigma$ selects exactly one approximately satisfiable literal from each clause. In order to maximize the chances of choosing a better literal combination, 
%we can sort each $L_C$
we can sort the elements of $L_C$
according to the value of the respective objective functions and then choose literal combinations using the corresponding lexicographic order (we will refer to this heuristic as \sortWrtCost).

While the point $p$ is usually a good candidate in terms of \emph{distance from a zero}, it can sometimes lead to an inconsistent problem:
\begin{example}Consider the formula	
	\begin{alignat*}{2}
		& \qquad \qquad \qquad \qquad \qquad \qquad \phi := C_1 \land C_2  \\
		& C_1 \equiv (x+y=0) \lor (x=\e^{10^6*y})
		&& C_2 \equiv (x+y \geq \epsilon_1) \lor (x=\tan(y+\epsilon_1)) 
	\end{alignat*}  	
	The numerical optimizer will be tempted to return first some point $p_1$ such as $\{x\mapsto 1; y\mapsto -1\}$, that \emph{almost} satisfies $(x+y=0) \land (x+y \geq \epsilon_1)$, instead of a harder approximate solution involving transcendental functions and heavy approximations, such as $(x=\e^{10^6*y}) \land (x=\tan(y+\epsilon_1))$, that is exactly satisfiable in a point $p_2$ near $(0, -\pi)$.
\end{example}


Such inconsistencies may occur in many combinations of literals. We use a strategy that detects them in situations where for certain clauses $C$, the set $L_C$ contains only one literal $l$. We will call such a literal $l$ a \emph{forced literal}, since, for every literal selector function~$\sigma$, $\sigma(\phi)$ will include $l$. Before starting to tackle every approximately satisfiable literal combination, we first analyze the set of forced literals. We do symbolic simplifications (such as rewriting and Gaussian elimination) to check whether the set has inconsistencies that can be found at a symbolic level (as in the previous example). If the symbolic simplifications detect that the forced literals are inconsistent then we set $\Children{p}$ to the empty sequence $\langle\rangle$ which causes backtracking in depth-first search. We refer to the variant of the algorithm using this check as \checkForcedLiterals. 


\ 

\emph{Filtering out over-constrained systems}. Given a literal selector function~$\sigma$, we analyze the structure of the system of equations formed by the equations selected by $\sigma$ through the Dulmage–Mendelsohn decomposition, that uniquely decomposes the system into a well-constrained subsystem, an over-constrained subsystem and an under-constrained subsystem.
We filter out every literal combination having a non-empty over-constrained subsystem, since this leads to a non-robust sub-problem, referring to this heuristic as \filterOverconstr.


%\subsection{Reduction to square system}
%\subsubsection{Instantiations}
\subsection{Instantiations}

\label{subsec:instantiations}
We define the instantiations $\Children{p, \sigma}=\langle \nu_1,\dots,\nu_k\rangle$ based on a sequence of sets of variables $V_1,\dots,V_k$ to instantiate, and define $\nu_i:= \mathit{proj}_{V_i}(p)$. The uninstantiated part of $p$ after projection to a set of variables $V_i$ is  then $\mathit{proj}_{\Vars{\phi}\setminus V_i}(p)$, which we will denote by $p_{\neg V_i}$.

For searching for the variables to instantiate, we use the Dulmage–Mendelsohn decomposition constructed in the previous level of the hierarchy. We do not want to instantiate variables appearing in the well-constrained sub-system, since doing so would make the resulting system after the instantiation over-constrained. Hence the variables to be instantiated should be chosen only from the variables occurring in the under-constrained subsystem. This substantially reduces the number of variable combinations that we can try. 
%Moreover, if the under-constrained subsystem is not connected, we can find its connected components to further reduce the number of feasible assignments.  
Denoting the variables satisfying this criterion by $V_{under}$, this restricts $V_i\subseteq V_{under}$, for all $i\in \{1, \dots, k\}$. 
This does not yet guarantee that every chosen variable combination leads to a well-constrained system after the instantiation. For example
, the under-determined system of equations  $x+y = 0 \land z+w=0$ has four variables and two equations, but becomes over-constrained after instantiating either the two variables $x$ and $y$, or the variables $z$ and $w$. So, for each $V_i$, we further check whether the system obtained after the instantiation is well-constrained (we refer to this heuristic as \filterOverconstrV). 

The method described in the previous paragraph only uses information about which equations in the system contain which variables (i.e., it deals only with the \textit{structure} of the system, not with its \textit{content}). Indeed, it ignores the point~$p$. 

To extract more information, we use the following fact: If a zero of a function has non-singular Jacobian matrix, then every box containing this zero and no other zeros has a non-zero topological degree~\cite{Fonseca:95}.
%To extract more information, we use the fact that a non-singular Jacobian matrix of a function at one of its zeros implies a non-zero topological degree wrt. every box containing this single zero~\cite{Fonseca:95}. 
So we compute a floating point approximation of the Jacobian matrix at point $p$ (note that, in general, this matrix is non-square). Our goal is to find a set of variables $V$ to instantiate such that the Jacobian matrix corresponding to the resulting square system at the point $p_{\neg V}$ has full rank. This matrix is the square sub-matrix of the original Jacobian matrix that is the result of removing the instantiated columns.



A straight-forward way of applying the Jacobian criterion is, given random variable instantiations, to filter out instantiations whose corresponding Jacobian matrix is rank-deficient \filterRankDeficient, similarly to what is done in the previous paragraph with the overconstrained filter. 
Note that, as the Jacobian matrix of non-well-constrained system of equations is always rank-deficient,
this filter is stronger than the previous one. However, it may filter out variable instantiations that result in a non-zero degree (e.g., the function $x^3$ has non-zero degree in $[-1,1]$, but its Jacobian matrix at the origin is rank deficient since $f'(0)=0$).

We can further use the information given by the Jacobian matrix not only to filter out bad variable instantiations, but also to maximize the chance of choosing good variable instantiations from the beginning. Indeed, not all variable instantiations will be equally promising, and it makes sense to head for an instantiation such that the resulting square matrix not only has full rank, but---in addition---is far from being rank-deficient (i.e., it is as robust as possible). 
We can do so by modifying Kearfott's method~\cite[Method 2]{Kearfott:98}, which fixes the coordinates most tangential to the orthogonal hyperplane of $F$ in $p$ by  first computing an approximate basis of the null space of the Jacobian matrix in the point, and then choosing the variables corresponding to the coordinates for which the sum of the absolute values in the basis is maximal.
Since we are interested in more than just a single variable choice, we order all the variables w.r.t. to this sum. Then, we extract the  sets of variables $V_1, V_2, \dots$ through a lexicographic combinatorial algorithm. We refer to this heuristic as \KearfottOrdering.


\myparagraph{Adding equations.} 
\label{subsubsec:orth}
An alternative approach for reducing from an underconstrained system of equations to a square one is, instead of instantiating variables, to add equations. 
This approach is justified by the fact that 
each variable instantiation can be seen as a system of equalities (while the vice-versa is not true). 
%(only equations of the form $x_i - p_i = 0$ can be seen as variable instantiations).
%In this subsection we explain how to find suitable equations to add.
 
While discussing Kearfott's method, we showed that, given a point $p$, it is better to choose variable instantiations that are the most orthogonal possible to the tangent hyperplane of $F$ in $p$. 
With the equations adding approach we can go further: we can directly choose the linear equations that describe the hyperplane orthogonal to the tangent space of $F$ in $p$. 
These equations can be found through the QR-decomposition of the Jacobian matrix of $F$ in $p$.
We can then add these equations to $F$ in order to obtain a square system of equations. We refer to this heuristic as \orthogonal.

Since the found equations are linear, we can further modify the previous heuristic by applying Gaussian elimination to the linear part of the square system obtained, thus reducing the dimension of the system of equations. We refer to this sub-heuristic as \gaussel.

 


\subsection{Boxes}
\label{subsec:box}
We construct boxes around $p_{\neg V}$, where $V$ is the set of variables $\nu$ instantiates, that is, $\nu\in\mathcal{R}^{V}$. So we define $\Children{p, \sigma,\nu}:=\langle \beta_1,\dots,\beta_k\rangle$ s.t. for all $i\in\{1,\dots,k\}$, for all $B\in\beta_i$, $B\in \mathcal{B}^{\Vars{\phi}-V}$ and $p_{\neg V}\in\bigcup_{B\in \beta_i} B$.

We use two different methods, \epsInflation and \boxGridding:
\begin{itemize}
	\item Epsilon-inflation~\cite{Mayer:94} is a method to construct incrementally larger boxes around a point. In this case, the $\beta_1,\dots, \beta_k$ will each just contain one single box $B_i$ defined as the box centered at $p_{\neg V}$ having side length $2^i\epsilon$, where, in our setting, $\epsilon=10^{-20}$. We terminate the iteration if either $\intervalArithmOperator_G(B_i)\leq0$ and $\deg(F,B_i,0)\neq0$, in which case we found a certificate, or we reach an iteration limit (in our setting when $2^i\epsilon > 1$).
	\item Box-gridding is a well-known technique from the field of interval arithmetic based on iteratively refining a starting box into smaller sub-boxes. Here we use a specific version, first proposed in \cite{Franek:12} and then implemented with some changes in \cite{ATVApaper}. In the following we roughly outline the idea behind the algorithm, and refer to the other two papers for details. We start with a grid that initially contains a starting box (in our setting, having side length $1$). We then iteratively refine the grid by splitting the starting box into smaller sub-boxes. At each step, for each sub-box~$B$ 
	we first check whether interval arithmetic can prove that the inequalities or the equations are unsatisfiable, and, if so, we remove $B$ from the grid.
	We check also whether $\deg(F,B,0)\neq0$ and interval arithmetic can prove the satisfiability of the inequalities, and, if so, then we  terminate our search, finding a certificate with the singleton $\beta_i=\{ B\}$. 
	In some cases, in order to verify  the satisfiability of the inequalities, we will have to further split the box $B$ into sub-boxes, using the set of resulting sub-boxes instead of the singleton $\{ B \}$. 
	After each step, if there are sub-boxes left in the grid, we continue the refinement process. Otherwise, if the grid is empty, we conclude that there cannot be solutions in the starting box. If a certain limit to the grid size is exceeded, we also stop the box gridding procedure without success.
      \end{itemize}

For both methods, if the method stops without success, we have arrived at the last element of the sequence of choices $\langle \beta_1,\dots,\beta_k\rangle$ without finding a certificate, which results in backtracking of the depth-first search for a certificate.

Both mentioned methods have their advantages, and can be seen as complementary. Epsilon-inflation is quite fast, and performs particularly well if the solution is isolated and is near the center. However, if there are multiple solutions in a box, the topological degree test can potentially fail to detect them\footnote{For example, for $f(x)=x^2-1$, $deg(f, [-10,10],0)=0$, while $deg(f,[-10,0],0)=-1$, and $deg(f,[0,10],0)=1$.}, and if the solution is far from the center then we need a bigger box to encompass it, which is less likely  to be successful than a smaller box, as we require the inequalities to hold everywhere in the box, and, moreover, the chance of encompassing other solutions (thus incurring in the previous problem) grows. 

The box-gridding procedure, on the other side, can be quite slow, as in the worst case the number of sub-boxes explodes exponentially. However,
grid refinement leads to a very accurate box search, which allows us to avoid the issues faced with epsilon inflation (i.e. multiple solutions, or a solution far from the center). Moreover, if the problem is robust, we have the theoretical guarantee that the procedure will eventually converge to a solution~\cite{Franek:12}, although this does not hold in practice due to the introduced stopping criterion.

Indeed, a third approach is to combine the two methods: first use epsilon inflation, that is often able to quickly find a successful box, and, if it fails, then use the more accurate box-gridding procedure.  



%%% Local Variables:
%%% mode: latex
%%% TeX-master: "./main.tex"
%%% End:



We present in section~\ref{ssec:faces} an application of PnP-HVAE on face images, using a pretrained state-of-the-art hierarchical VAE. 
Next, we study the application of our framework to natural images. To that end, we introduce  in section~\ref{ssec:patchVDVAE}  a patch hierachical VAE architecture, that is able to model natural images of different resolutions. In section~\ref{ssec:app_nat}, we provide deblurring, super-resolution and inpainting experiments to demonstrate the relevance of the proposed method.

Additional results are presented in Appendix~\ref{app:add}. All experiments can be reproduced using the code available at \url{https://github.com/jprost76/PnP-HVAE}.



\subsection{Face Image restoration (FFHQ)}\label{ssec:faces}
We first demonstrate the effectiveness of PnP-HVAE on highly structured data, by performing face image restoration.
Latent variable generative models can accurately model structured images such as face images \cite{karras2019style,vahdat2020nvae,child2021very,kingma2018glow}, and then be used to produce high quality restoration of such data. 
In our experiments, we use the VDVAE model of~\cite{child2021very}, pre-trained on the FFHQ dataset~\cite{karras2019style}, as our hierarchical VAE prior.
VDVAE has $L=66$ latent variable groups in its hierarchy and generates images at resolution $256\times256$.

We compare PnP-HVAE with the intermediate layer optimization algorithm (ILO)~\cite{daras2021intermediate} that is based on a different class of generative models than HVAE. ILO is a GAN inversion method which optimizes the image latent code along with the intermediate layer representation of a StyleGAN to generate an image consistent with a degraded observation.
We use the official implementation of ILO, along with a StyleGAN2 model~\cite{karras2020analyzing, stylegan2pytorch}, that was trained for 550k iterations on images of resolution $256\times256$ from FFHQ.  
As VDVAE and StyleGAN models are not trained on the same train-test split of FFHQ, we chose to evaluate the methods on a subset of 100 images from the CelebA dataset~\cite{liu2018large}. 
For super-resolution, the degradation model corresponds to the application of a gaussian low-pass filter followed by a $\times 4$ sub-sampling, and the addition of a gaussian white noise with $\sigma=3$.
For the deblurring, we considered motion blur and  gaussian kernels, both with a noise level $\sigma=8$. %

We provide quantitative comparisons in table~\ref{table:comp_ILO}, along with a visual comparison of the results in figure~\ref{fig:face_restoration}.
PnP-HVAE has the best  PSNR and SSIM results for all the considered restoration tasks, while ILO provides better results  for the perceptual distance.
By jointly optimizing the image and its latent variable, PnP-HVAE provides  results that are both realistic and consistent with the degraded observation.
On the other hand,  ILO  only optimizes on an extended latent space. This method generates  sharp and realistic images with better LPIPS scores,   
but the results lack  of consistency with respect to the observation, which explains the overall lower PSNR performance. 






\subsection{PatchVDVAE: a HVAE for natural images}\label{ssec:patchVDVAE}
Available generative models in the literature operate on images of  fixed resolutions and
are either restrained to datasets of limited diversity, or even to registered face images~\cite{kingma2018glow,child2021very, vahdat2020nvae, karras2019style}, or requiring additional class information~\cite{brock2018large, dhariwal2021diffusion, song2020score, luhman2022optimizing}.
Fitting an unconditional model on natural images appears to be a more difficult task, as their resolution can change, and their content is highly diverse.
The complexity of the problem can be reduced by learning a prior model on patches of reduced dimension. 
For image restoration problems, the patch model can be reused on images of higher dimensions~\cite{zoran2011learning,prost2021learning,altekruger2022patchnr}. When the model is a full CNN, the prior on the set of the  patches can  be computed efficiently by applying the network on the full image~\cite{prost2021learning}.

We thus introduce  patchVDVAE, a fully convolutional hierarchical VAE.
Contrary to existing HVAE models whose resolution is constrained by the constant tensor at the input of the top-down block, patchVDVAE can generate images of different resolutions by controlling the dimension of the input latent. 
This amounts to defining a prior on patches whose dimension corresponds to the receptive field of the VAE. A similar model is used for image denoising in~\cite{prakash2021interpretable}.

 
For PatchVDVAE architecture, we use the same bottom-up and top-down blocks as VDVAE~\cite{child2021very}, and replace the constant trainable input in the first top-down block by a latent variable, to make the model fully convolutional (details on the  architecture are given in Appendix~\ref{app:details}). 
The training dataset is composed of $128\times 128$ patches extracted from a combination of DIV2K~\cite{agustsson2017ntire} and Flickr2K~\cite{Lim_2017_CVPR_workshops} datasets.
We perform data augmentation by extracting  patches at $3$ resolutions: HR-images and $\times 2$ and $\times 4$ downscaled images. 
The model is trained for $7.10^5$ iterations with a batch size of $64$. Following the recommendation of~\cite{hazami2022efficient}, we use Adamax optimizer with an exponential moving average and gradient smoothing of the variance.
We set the decoder model to be a gaussian with diagonal covariance, as in~\cite{luhman2022optimizing}.
PatchVDVAE is fully convolutional and can generate images of dimension that are multiples of $64$ as illustrated by
figure~\ref{fig:vdvae}.

\newlength{\patchwidth}
\setlength{\patchwidth}{0.135\columnwidth}
\begin{figure}[!ht]
    \centering
    \begin{subfigure}[t]{.34\columnwidth}\hspace{0.1cm}
        \setlength{\tabcolsep}{0.02pt}
\renewcommand{\arraystretch}{0}
        \begin{tabular}{*{2}{p{1.03\patchwidth}}}
            \includegraphics[width=\patchwidth]{figures_arxiv/patchVDVAE/samples/generated/64x64/setup-5-image-0018.png} &
            \includegraphics[width=\patchwidth]{figures_arxiv/patchVDVAE/samples/generated/64x64/setup-5-image-0016.png} \\
            \includegraphics[width=\patchwidth]{figures_arxiv/patchVDVAE/samples/generated/64x64/setup-5-image-0008.png} &
            \includegraphics[width=\patchwidth]{figures_arxiv/patchVDVAE/samples/generated/64x64/setup-5-image-0019.png}   
        \end{tabular}
    \end{subfigure}\hspace{-0.15cm}
    \begin{subfigure}[t]{.64\columnwidth}
\begin{tabular}{cc}\vspace{-0.1cm}
\includegraphics[width=2\patchwidth]{figures_arxiv/patchVDVAE/samples/generated/256x256/setup-2-image-0009.png}&
        \includegraphics[width=2\patchwidth]{figures_arxiv/patchVDVAE/samples/generated/256x256/setup-2-image-0002.png}\end{tabular}

    \end{subfigure}
    \caption{\label{fig:vdvae} Left: $64\times64$ patches samples from our patchVDVAE model trained on patches from natural images.
    Right: PatchVDVAE is fully convolutional and it can generate images of higher resolution (here: $128\times128$).\vspace{-0.2cm}}
\end{figure}

\subsection{Natural images restoration}\label{ssec:app_nat}
We  evaluate PnP-HVAE on natural image restoration.
For each task, we report the average value of the PSNR, the SSIM, and the LPIPS metrics on $20$ images from the test set of the BSD dataset~\cite{MartinFTM01}.\\


\noindent
{\bf Image deblurring.}
In the experiments, we consider $2$ gaussian kernels and $2$ motion blur kernels from~\cite{levin2009understanding}, with $3$ different noise levels 
$\sigma \in \{2.55, 7.65, 12.75\}$.
As a baseline we consider  EPLL~\cite{zoran2011learning}, which learns a prior on image patches with a gaussian mixture model.
We also compare PnP-HVAE  with PnP-MMO and GS-PnP, $2$ competing convergent Plug-and-Play methods based on CNN denoisers.
PnP-MMO~\cite{pesquet2021learning} restricts the denoiser to be contraction in order to guarantee the convergence of the PnP forward-backard algorithm. GS-PnP~\cite{hurault2022gradient} considers a gradient step denoiser and reaches state-of-the-art performances of non converging methods~\cite{zhang2021plug}.
We set the temperature $\tau$  in our method as $0.95$, $0.8$ and $0.6$ for noise levels $2.55$, $7.65$ and $12.75$ respectively, and we let it run for a maximum of $50$ iterations. 
For the three compared methods we use the official implementations and pre-trained models provided by the respective authors. 
Details on the choice of hyperparameters for the concurrent methods are provided in the Appendix~\ref{app:details}
Figure~\ref{fig:deblurring_bsd} illustrates that our method provides correct deblurring results. 

According to table~\ref{tab:deb}, the performance of PnP-HVAE is between those of EPLL and GS-PnP and it outperforms PnP-MMO for large noise levels.\\

\begin{table}
\begin{center}\footnotesize
    \begin{tabular}{>{\centering}m{.3cm}*{5}{c}}
    $\sigma$ &Method & PSNR$\uparrow$ & SSIM$\uparrow$ & LPIPS$\downarrow$  \\ 
    \hline
    \multirow{4}{*}{\vcell{$2.55$}}
    & PnP-HVAE & $27.75$ & $0.79$ & $0.31$\\
    & GS-PNP \cite{hurault2022gradient} & $\mathbf{29.59}$ & $\mathbf{0.84}$ & $\mathbf{0.22}$\\
    & EPLL \cite{zoran2011learning} & $26.49$ & $0.71$ & $0.36$\\ 
    & PnP-MMO \cite{pesquet2021learning} & $\underbar{29.50}$ & $\underbar{0.83}$ & $\underbar{0.20}$ \\ \hline
    \multirow{4}{*}{\vcell{$7.65$}}
    & PnP-HVAE & $\underbar{26.36}$ & $\underbar{0.72}$ & $\underbar{0.40}$\\
    & GS-PNP \cite{hurault2022gradient} & $\mathbf{27.33}$ & $\mathbf{0.77}$ & $\mathbf{0.31}$\\
    & EPLL \cite{zoran2011learning} & $24.04$ & $0.66$ & $0.45$ \\ 
    & PnP-MMO \cite{pesquet2021learning} & $25.34$ & $0.69$ & $0.34$\\
    \hline
    \multirow{4}{*}{\vcell{$12.75$}}
    & PnP-HVAE & $\underbar{25.12}$ & $\mathbf{0.73}$ & $\underbar{0.47}$\\
    & GS-PNP \cite{hurault2022gradient} & $\mathbf{26.32}$ & $\mathbf{0.73}$ & $\mathbf{0.37}$\\
    & EPLL \cite{zoran2011learning} & $23.28$ & $0.61$ & $0.51$ \\ 
    & PnP-MMO \cite{pesquet2021learning} & $22.42$ & $0.53$& $0.54$ \\
    \hline
    &\vspace*{-.3cm}\\
            \multicolumn{2}{c}{Blur and motion kernels}& \multicolumn{3}{c}{
        \includegraphics*[scale=1]{figures_arxiv/kernels/4.png}\;\includegraphics*[scale=1]{figures_arxiv/kernels/7.png}\;\includegraphics*[scale=1]{figures_arxiv/kernels/9.png}\;\includegraphics*[scale=1]{figures_arxiv/kernels/11.png}} 
    \end{tabular}
        \caption{\label{tab:deb}Comparison  of PnP-HVAE  and other restoration methods on deblurring. Results are averaged on $4$ kernels.\vspace{-0.2cm}}% on image deblurring.}
    \end{center}
\end{table}

\begin{figure}
    
    \begin{subfigure}[h]{\linewidth}
        \centering
        \includegraphics*[width=\columnwidth]{figures_arxiv/deb_s255_k7.pdf}\vspace{-0.1cm}
        \caption{Gaussian blur, $\sigma=2.55$}
    \end{subfigure}
    \begin{subfigure}[h]{\linewidth}
        \centering
        \includegraphics*[width=\columnwidth]{figures_arxiv/deb_s765_k11.pdf}\vspace{-0.1cm}
        \caption{Motion blur, $\sigma=7.65$}
    \end{subfigure}\vspace*{-0.1cm}
    \caption{\label{fig:deblurring_bsd} Natural image deblurring\vspace{-0.1cm}}
\end{figure}

\noindent {\bf Effect of the temperature.}
PnP-HVAE gives control on the temperature of the prior over the latent space.
In figure~\ref{fig:temp_effect}, we illustrate that reducing the temperature increases the strength of the regularization prior. In this example the tuning $\tau=0.7$ produces the best performance.\\
\begin{figure}[!ht]
   
    \includegraphics[width=\columnwidth]{figures_arxiv/demo_temp.pdf}\vspace{-0.15cm}
    \caption{ \label{fig:temp_effect} Effect of the temperature in PnP-VAE on a deblurring problem, with $\sigma=7.65$.\vspace{-0.15cm}}
\end{figure}


\noindent
{\bf Image inpainting.}
Next we consider the task of noisy image inpainting. 
We compose a test-set of 10 images from the validation set of BSD~\cite{MartinFTM01} and we create masks
  by occluding diverse objects of small size in the images. 
A gaussian white noise with $\sigma=3$ is added to the images.
As a comparaison, we still consider GS-PnP and EPLL.
For PnP-HVAE, the temperature is set to $\tau=0.6$, and the algorithm is run for a maximum of $200$ iterations, unless the residual $||\x_{k+1}-\x_k||$ is on a plateau.
We provide on Table~\ref{tab:inpainting_bsd} the distortion metrics with the ground truth, as well as a visual
\begin{table}



\begin{center}
    \begin{tabular}{cccc}
        & PSNR$\uparrow$ & SSIM$\uparrow$ &LPIPS$\downarrow$ \\\hline
        PnP-HVAE  & $\mathbf{29.54}$ & $\mathbf{0.93}$ & $\mathbf{0.06}$\\
        GS-PNP & $28.52$ & $\mathbf{0.93}$ & $0.09$\\
        EPLL & $\underline{29.16}$ & $\mathbf{0.93}$ & $\mathbf{0.06}$\\
    \end{tabular}
    \caption{\label{tab:inpainting_bsd}Quantitative evaluation for inpainting on BSD.}
    \end{center}
\end{table}
comparison on figure~\ref{fig:inpainting_bsd}. 
With its hierarchical structure,  PnP-HVAE outperforms the compared methods. \vspace{0.05cm}



\begin{figure}[!h]
    \includegraphics[width=\columnwidth]{figures_arxiv/demo_inp_bsd2.pdf}\vspace{-0.1cm}
    \caption{\label{fig:inpainting_bsd}Natural image inpainting\vspace{-0.3cm}}
\end{figure}












\section{Theoretical Characterization}
\label{sec:th_considerations}


Since the problem addressed by this paper is undecidable, the success of any algorithmic approach to solving the problem must necessarily depend on heuristics. Still, in this section, we contribute some results that provide insight into when one can reasonably expect an approach such as the one presented in this paper to succeed. 

Especially, we address a sensitive part of our method---the reduction from an under-constrained system of equations to a well-constrained subsystem (Section \ref{subsec:instantiations}). Indeed, given a system of equations in $m$ variables and $n$ equations ($m>n$), %and a candidate approximate solution $p$, 
in order to obtain a well-constrained system, we need to either instantiate $k:= m-n$ variables, or, alternatively, to add $k$ equations. The contribution of this section is threefold:
\begin{itemize}
	\item We bound the class of problems solvable through the variable instantiation method, both from below and from above.
	\item We prove that the class of problems solvable through the variable instantiation method is a (possibly non-strict) subset of the class of problems solvable through the equation adding method.
	\item We show that for bounded systems of equations and inequalities a certain strategy in the certificate search method presented in Section~\ref{sec:method} will always succeed in determining satisfiability under certain robustness assumptions.%
\end{itemize}

For ease of discussion, first, we will 	consider only systems of equations (i.e., without inequalities). It is however straightforward to see that including inequalities does not change any of the results. 
We will then treat the general case of conjunction and disjunctions of systems of equations and inequalities 	when discussing the last contribution. 
%



We will start by introducing some notation for the relevant classes of problems. For now, we will introduce these classes only informally, and define them precisely, later. We will denote by $\FRobI$ and $\FRobLEq$ the problem classes, for which the two methods (instantiation and adding equations, respectively) result in a robust system. More precisely, we will denote by  $\FRobI$ 
the class of \Cone functions $F:B\subseteq \mathbb{R}^{n+k}\to \mathbb{R}^{n}$ 
for which there exists a point $p\in \mathbb{R}^{n+k}$ such that the instantiation of $k$ variables to the corresponding $k$ values of $p$ leads to a robust system in $\mathbb{R}^n\rightarrow\mathbb{R}^n$ (we will call such functions \emph{robust under instantiation}), and we will denote by $\FRobLEq$ the class of \Cone functions $F:B\subseteq \mathbb{R}^{n+k}\to \mathbb{R}^{n}$  for which adding $k$ linear equations leads to a robust system in  $\mathbb{R}^{n+k}\rightarrow\mathbb{R}^{n+k}$.

First, we will bound $\FRobI$ from above by the class $\FRob$
of \Cone functions $F:B\subseteq \mathbb{R}^{n+k}\to \mathbb{R}^{n}$ that have 
 a robust solution, and from below by
the class $\FReg$ of \Cone functions 
$F:B\subseteq \mathbb{R}^{n+k}\to \mathbb{R}^{n}$ 
that have a solution that is regular in the sense of topology.

Based on this, we will prove the following:
\begin{theorem}
	\label{thm:FRobIbounds}
	%$\FReg \mysubsetneq  \FRobI = \FRobLEq \mysubsetneq \FRob $
	$\FReg \mysubsetneq  \FRobI  \mysubsetneq \FRob $ 
\end{theorem}

As can be seen from the disequalities, the lower and upper bounds are strict, here.

% This theorem effectively provides strict lower and upper bounds to the class of problems that our method is able to solve. The existence of stricter bounds is an open problem. 

Secondly, we prove that every problem that can be solved via variable instantiation can be solved via adding equation, i.e. we have the following theorem:

\begin{theorem}
\label{thm:instVersusEq}      
	$  \FRobI  \subseteq \FRobLEq $ 
\end{theorem}

Note that the inclusion, in this case, is not strict. Indeed, we conjecture that the equality holds, but will leave the proof to future work.

Finally, we present a variation of our method that is guaranteed to always terminate on problems in $\FRobI$, and that will serve to prove the following theorem:

\begin{theorem}
	\label{thm:quasiquasidecidability}
	There exists a procedure that, given a bounded system of equations and inequalities $F=0 \wedge G \leq 0$,
	\begin{itemize}
        \item always returns the correct answer ``satisfiable'' or ``unsatisfiable'', if it terminates, 
        \item always terminates successfully when $F=0 \wedge G \leq 0$ is robustly satisfiable  and  $F\in \FRobI$,
        \item always terminates successfully when $F=0 \wedge G \leq 0$ is robustly unsatisfiable.
	\end{itemize}
\end{theorem}

This theorem can be seen as an extension of an earlier result~\cite{Franek:12} showing
that the class of bounded systems of equations and inequalities in $m$ variables and $n$ equations, with $n\geq m$ or $n=0$, is quasi-decidable in the sense that there exists a procedure that always terminates on robust instances, and that never returns a wrong answer.
 
Our contribution is to cover the case of under-constrained systems (i.e. when $n < m$), to a certain extent. 
Indeed, it is not possible to simply remove the restriction on the number of equations versus the number of variables~\cite[Theorem 2]{Franek:12}. 
We overcome this by guaranteeing termination in the satisfiable case only for problems for which the system of equations is robust under instantiation (which is a stricter condition than general robustness). 
In this sense, our procedure is not a quasi-decision procedure, 
as it does not terminate for \emph{all} robust instances, 
but it will cover a meaningful sub-class.%(known problems that are robust but not robust under instantiation derive more from mathematical counter-examples rather than from real-world verification cases)
%\footnote{We could say that our procedure is \emph{quasi} a quasi-decidability procedure. }. 

\

% More compactly, we have:

% \begin{itemize}
% 	\item $\mathcal{F}_{RobI} \defas \{F:B\subseteq \mathbb{R}^{n+k}\to \mathbb{R}^{n} | \text{ $F$ has a solution robust under instantiation} \}$
% 	\item $\mathcal{F}_{Rob} \defas \{F:B\subseteq \mathbb{R}^{n+k}\to \mathbb{R}^{n} | \text{ $F$ has a robust solution} \}$
% 	\item $\mathcal{F}_{Reg} \defas \{F:B\subseteq \mathbb{R}^{n+k}\to \mathbb{R}^{n} | \text{ $F$ has a regular solution} \}$
% 	\item $\mathcal{F}_{RobLEq} \defas \{F:B\subseteq \mathbb{R}^{n+k}\to \mathbb{R}^{n} | \text{there exist $k$ linear eq. that added to $F$  yield a  system with a robust solution} \}$
% \end{itemize}

% First, we will show that for every system that has a regular solution, there exists a candidate approximate solution and a variable instantiation that yield a robust sub-system ($\FReg \subseteq \FRobI$). Then, we will show that the variable instantiation method is not limited to succeed only on systems having a regular solution, i.e. we have a strict lower bound ($\FReg \mysubsetneq \FRobI$). We also provide a strict upper bound: indeed, the variable instantiation method is able to solve only systems that admit at least one robust solution ($\FRobI \subseteq \FRob$), and there exist counter-examples of systems that have a robust solution but for which no variable instantiation yields a robust sub-system (i.e. $\FRob \not\subset \FRobI$).  Finally, we will show that the class of system of equations that can be solved via variable instantiation  coincides with the class of system of equations that can be solved via equation adding ($\FRobI = \FRobLEq$).




% \

The section is organized as follows. In Section \ref{subsec:BackgroundRobustness}, we formalize
the problem classes mentioned by the two theorems and provide some further 
definitions and properties that will be useful in the following subsections.
%introduce some definitions regarding  robustness and robust solutions. 
In Section \ref{subsec:GuaranteesRegularity},  we will prove the lower bound $\FReg \subseteq \FRobI$ (Lemma \ref{thm:FRegsubsetFRob}) and that ${\FRobI \not\subseteq \FReg}$ (Lemma~\ref{lemma:FRobInotsubsetFReg}). In Section~\ref{subsec:RobPreservationAfterVarInst} we will prove the upper bound $\FRob \not\subseteq \FRobI$ (Lemma~\ref{lemma:FRobnotsubsetFRobI}) and that $\FRobI \subseteq \FRob$ (Lemma~\ref{lemma:FRobIimpliesFRob}). Then, in Section \ref{subsec:InstVarVsAddEq}, we prove that $\FRobI \subseteq \FRobLEq$.
Finally, in Section~\ref{subsec:termination},  we prove Theorem \ref{thm:quasiquasidecidability}.
%First, we prove that a system obtained by instantiating $\{x_i \mapsto p_i \}$ is equi-robustly sat to the system obtained by adding equations $\{x_i - p_i = 0\}$ (Lemma~\ref{lemma:instVarVsAddEq}, one side still not proved), which implies that $\FRobI \subseteq \FRobLEq$.\\
%Then we want to prove that $\FRobLEq \subseteq \FRobI$. This proof is not concluded yet.






\subsection{Background on robustness and regularity}
\label{subsec:BackgroundRobustness}




In this section, we first give a formal definition of robustness, and then proceed to provide all the definitions needed to formally define our four classes of interest. 
We will also present some results regarding these definitions that will be used for proving the main theorem.

\

\emph{Notation}: Given a multivalued function $F:\Omega \subseteq \mathbb{R}^m\to \mathbb{R}^n$, we will denote with $F_1, \dots, F_n: \Omega \subseteq \mathbb{R}^m\to \mathbb{R}$ the univalued functions such that $F= (F_1, \dots, F_n)$. 
With a slight abuse of terminology, we will say that a function $F:\Omega \subseteq \mathbb{R}^m\to \mathbb{R}^n$ is \emph{satisfiable} %in $\Omega'\subseteq \Omega$ 
if and only if it has a zero.% in $\Omega$.

%\

\subsubsection{Robustness.}
First, we provide a formal definition of the concept of robustness. Since our main focus are systems of equations, we will provide a definition of robustness only in terms of multivalued functions. This concept, however, can be generalized and formalized for general formulas. The definition that we give here is just a special case of the more general definition presented in  \cite{Franek:12}. In fact, a multivalued function $F$ is robustly satisfiable if and only if the logical formula representing the equation $F=0$ is. We will make use of the general definition of robustness only in the last section, when discussing Theorem \ref{thm:quasiquasidecidability}.

We first introduce the concept of distance between functions.


\begin{definition}[Distance between two functions]
	Let $F:\Omega_1 \subseteq \mathbb{R}^m\to \mathbb{R}^n$  and $F':\Omega_2 \subseteq \mathbb{R}^m\to \mathbb{R}^n$ be two multivalued continuous functions. 	Given $\Omega \subseteq \mathbb{R}^m$ such that $\Omega \subseteq \Omega_1$ and $\Omega \subseteq \Omega_2$,
	we define the distance between $F$ and $F'$ in $\Omega$ as
	$$\dist_{\Omega}(F,F') \defas \underset{k\in [1,n]}{\max}( \|F_k-F'_k\|_{\Omega} )$$	
	where $\|F_k-F'_k\|_{\Omega} \defas \sup \{|F_k({x})-F'_k({x})| : {x} \in {\Omega}\}$.
\end{definition}

When $\Omega$ is clear from the context, with an abuse of notation we will 
drop the subscript and just write $\dist(F,F')$. 
We say that $F'$ is an \emph{$\epsilon$-small perturbation} of $F$ if $\dist(F,F')< \epsilon$.


\begin{definition}[Robustness of a function]
	Given $\alpha\in \mathbb{R}_{> 0}$, we say that a continuous function $F:\Omega\subseteq \mathbb{R}^m\to \mathbb{R}^n$ is $\alpha$-robust iff for every continuous function $F'$ s.t. $\dist_{\Omega}(F,F') < \alpha$, either both $F$ and $F'$ have a zero in $\Omega$, or none of them has.
	A function $F$ is\textbf{ robust} iff there exists $\alpha\in \mathbb{R}_{> 0}$ s.t. $F$ is $\alpha$-robust in $\Omega'$.
	
\end{definition}

For $\Omega'\subseteq \Omega$, we say that $F$ is robust in $\Omega'$ iff $F_{|\Omega'}:\Omega'\subseteq \mathbb{R}^m\to \mathbb{R}^n$ is robust.



\begin{definition}[Robustly satisfiable function]
	\label{def:robustlysat}
	A function $F$ is \textbf{robustly satisfiable} iff
	% it is satisfiable and $\exists \delta>0$ s.t. $\forall \phi'$ with $d(\phi, \phi')<\delta$, $\phi'$ is satisfiable.
	it is robust and satisfiable.	
\end{definition}



\subsubsection{Robust solutions ($\FRob$).}
Robust satisfiability of a function---as defined by Definition~\ref{def:robustlysat}---does not depend on any specific solution. Indeed, it is the entirety of the solution set that accounts for the robust satisfiability of the function.  However, it will be useful to talk about the robustness of a function around a specific solution, which also formalizes the definition of the class $\FRob$.
% (i.e., \emph{local} robustness).

\

	\begin{definition}[Robust solution]
		Given a function $F:\mathbb{R}^m\to \mathbb{R}^n$, we say that a point $p\in \mathbb{R}^m$ is a \textbf{robust solution} iff
		\begin{enumerate}
			\item $F(p)=0$, and
			\item for all $\epsilon>0$ there exists a $\delta > 0$ such that for all $F'$ with
			$\dist(F,F')<\delta$, there exists $p'$ such that 
			$F'(p')=0$ 
			%robustly satisfies $\phi'$ 
			and $\dist(p,p')<\epsilon$.
		\end{enumerate}
	\end{definition}
	
	It follows by the definition of robust solution that if $F$ has a robust solution, then $F$ is robustly sat. 
	
Note that the converse, however, is not true in general:

\begin{example}[Robustly sat formula with no robust solution]
	Let $F: \mathbb{R}\to \mathbb{R}$ defined by 
	$$F:x\mapsto
	\begin{cases}
		-x^2 & \text{ if } x<0 \\
		0        & \text{ if } 0 \leq x\leq 1 \\
		(x-1)^2        & \text{ if } 1<x\\
	\end{cases} $$
This function is robustly sat, since every $F'$ obtained by a small perturbation of $F$ has still a solution either near $0$ or near $1$. 
But no point in the solution set of $F$ (i.e. the points in $[0,1]$) is a robust solution.
Indeed, $0$ is not a robust solution, since, given $\epsilon = 0.1$, for every $\delta>0$, the function $F'_\delta$ defined by $F'_\delta(x)\mapsto F(x)-\delta$; 
for $0$ has no solution in the open ball $\openball{B}{0.1}(0)$. Symmetrically, for $1$ we can take $F''_\delta$ defined by $F''_\delta(x)\mapsto F(x)+\delta$. 
And for every point $p\in(0, 1)$, we can take $\epsilon=\min(\dist(p,0), \dist(p,1))$, and for every $\delta$ either $F'_\delta$ or $F''_\delta$.

\end{example}







\ 

A partial converse is the following result, which states that if a function is locally robustly satisfiable around a solution, the solution is robust.
\begin{proposition}
	\label{prop:locallyRobustlysat}
	If $p$ is a solution for $F$, and for every $\epsilon > 0$ there exists a neighborhood  $\Omega_\epsilon \subseteq \openball{B}{\epsilon}(p)$ of $p$ such that       
	$F$ is robustly sat in $\Omega_\epsilon$, then $p$ is a robust solution for $F$.
\end{proposition}

\begin{proof}
	For every $\epsilon>0$, $F$ is robustly sat in $\Omega_\epsilon$ if and only if  (by replacing the definition of robustly satisfiable function) $\forall \epsilon$, $\exists \delta$ s.t. $\forall F'$ with $\dist(F,F')<\delta$, $F'$ has a solution in $\Omega_\epsilon$.
	$F'$ has a solution in $\Omega_\epsilon$ if and only if there exists $p'$ s.t.  $F'(p')=0$ and $p'\in \Omega_\epsilon$ (i.e. $\dist(p, p')<\epsilon$).
	So we obtain: $\forall \epsilon$, $\exists \delta$ s.t. $\forall F'$ with $\dist(F,F')<\delta$, $\exists p'$ s.t. $p'$ is a solution for  $F'$ and $\dist(p, p')<\epsilon$ which is the definition of robust solution for $p$.
\end{proof}
	

%\

\subsubsection{Robust under instantiation ($\FRobI$)}

%Since in our method, given a system of equations $F=0$, with $F:\mathbb{R}^{n+k}\to \mathbb{R}^n$, we instantiate $k$ variables, the following definitions will be useful.
%Now we introduce a few definitions in order to provide a formal definition of $\FRobI$. 

In general, even if a solution is robust, after the instantiation of some variables, the projection of the solution may not be a robust solution for the function obtained after variable instantiation.

Now we provide some notation and some definitions regarding variable instantiation. We start by formalizing the fact that $k$ coordinates of a point have a finite representation in the form of a dyadic rational number.
\begin{definition}[$k$-finite point]
	A point ${p=(p_1,\cdots, p_m)\in \mathbb{R}^{m}}$ 
	is a \textbf{\kfinite{k}} 
	(with $0\leq k\leq m$)
	if for at least $k$ coordinates $i\in\{1,\dots,m\}$
 there exist integers $n_i$ and $r_i$ such that $p_i= n_i 2^{-{r_i}}$.
      \end{definition}
Here, we use base $2$ just for convenience. Any other base would work equally well for our purposes.    

%\begin{definition}[Function instantiation via an assignment] Let ${F: B\subseteq \mathbb{R}^{n+k}\to \mathbb{R}^n}$ be a smooth function and $p=(p_1, \dots, p_{n+k})\in \mathbb{R}^{n+k}$ a \kfinite{k} (with $I$ denoting the set of the $k$ indices finitely representable)
%	Then, given the partial assignment $\nu_I \defas \{x_i \mapsto p_i\}_{i \in I}$,
%	we define the \mbox{\textbf{instantiation of $F$ via $\nu_I$}} as the function
%	$F_{|\nu_I}: B_{|\mathbb{R}^n} \subseteq \mathbb{R}^n\to \mathbb{R}^n$ 
%	obtained by
%	replacing $x_i$ with $p_i$ for every $i\in I$.
	%
%end{definition}


\begin{definition}[Robust instantiation of a point]
	Let ${F: B\subseteq \mathbb{R}^{n+k}\to \mathbb{R}^n}$ be a \Cone function  and $p=(p_1, \dots, p_{n+k})\in \mathbb{R}^{n+k}$ a \kfinite{k} (with $I$ denoting a set of  $k$ finitely representable indices).
	%and  ${I\subseteq [1,n+k]}$ a set of indices,  with $|I|=k$ and $p_i$ finitely representable for every $i\in I$.
	
	Given the partial assignment $\nu_I \defas \{x_i \mapsto p_i\}_{i \in I}$,
	we define the {\emph{instantiation of $F$ via $\nu_I$}} as the function
	$F_{|\nu_I}: B_{|\mathbb{R}^n} \subseteq \mathbb{R}^n\to \mathbb{R}^n$ 
%	obtained by	replacing $x_i$ with $p_i$ for every $i\in I$.
        such that for every $(x_1,\dots,x_n)\in\mathbb{R}^n$,
        $F_{|\nu_I}(x_1,\dots,x_n)= F(y_1,\dots,y_{n+k})$, where 
                \[ \text{for } i\in \{1,\dots, n+k\},\:
          y_i=
          \left\{\begin{array}{l}
            p_i, \text{if } i\in I,\\
            x_i, \text{if } i\not\in I
          \end{array}\right.. \]

        
	We say that the partial assignment $\nu_I$ 
	is a \textbf{robust instantiation of $p$} if and only if the point $p_{|\nu_I} = (p_i)_{i\not\in I} \in \mathbb{R}^n$  
	 is a robust solution for $F_{|\nu_I}$.
      \end{definition}
      If an instantiation is not a robust instantiation, then we say it is a \emph{\mbox{non-robust} instantiation}. Note that assignments are defined as maps to the set of finitely representable values. So if $p$ is not a \kfinite{k}, then it admits no instantiations, and hence no robust instantiations.



For our method to succeed, we only need the existence of a single robust instantiation. 
So we are not interested in solutions that are robust under \emph{all} instantiations, but in solutions that are robust under \emph{at least one}  instantiation.
Finally, we have the following definition (from which the definition of $\FRobI$ follows):


\begin{definition}[Robust under instantiation] Given a continuous function ${F: B\subseteq \mathbb{R}^{n+k}\to \mathbb{R}^n}$ and a point $p$, we say that $p$ is \textbf{\mbox{robust under instantiation}}  if there exists at least one robust instantiation of $p$.
\end{definition}
If no robust instantiation of $p$ exists, then we say that $p$ is \emph{non-robust under instantiations}. If $p$ is not a \kfinite{k}, then, by definition, $p$ is non-robust under instantiations.

\

In the following, for ease of notation and without loss of generality, we will assume that ${I=\{n+1, \dots, n+k\}}$, unless otherwise specified.  
In this case, we will write 
$\nu$ instead of $\nu_I$, and we will denote
$\pn \defas (p_1 ,\dots, p_n)$ for the projection of $p$ to the first $n$ coordinates (i.e. the ones not instantiated by $\nu$), 
and $\pnk$ for the projection of $p$ to the last $k$ coordinates (i.e. the ones not instantiated by $\nu$).



\subsubsection{Robustness after equation adding ($\FRobLEq$).} 
We define $\FRobLEq$ as the set of \Cone functions $F: \mathbb{R}^{n+k}\to \mathbb{R}^{n}$ such that there exists a linear function $L: \mathbb{R}^{n+k}\to \mathbb{R}^k $ such that the function ${F_{leq}: \mathbb{R}^{n+k}\to \mathbb{R}^{n+k}}$, defined by $F_{leq}: x \mapsto (F(x), L(x))$,
has a robust solution.


\subsubsection{Regular solutions ($\FReg$).} We now define the last of our classes of interest. For doing so, we provide a brief background on the notion of regularity from the field of differential topology.
%\myparagraphB{Background on regularity.} 


\begin{definition}[Regular point]
	Let $F: B\subseteq \mathbb{R}^m \to \mathbb{R}^n$ be a \Cone function, with $m-n = k \geq 0$. 
	We say that $p\in B$ is a \textbf{regular point} for $F$ 
	%if and only if ${dF_x:T_x\interior{B}\to T_{F(x)}\mathbb{R}^n }$ is surjective, or, in equivalent, 
	if and only if the Jacobian matrix of $F$ at $x$ has maximal rank.
\end{definition}
If $F(p)=0$ and $p$ is a regular point, we will say that $p$ is a \emph{regular solution} of $F$. 
If $p$ is not a regular point, we say that $p$ is a \emph{critical point}.  
We say that $q\in \mathbb{R}^n$ is a \emph{regular value} if and only if for every $p$ such that $F(p)=q$, $p$ is a regular point.
If $q$ is not a regular value, we say it is a \emph{critical value}.


Regularity is a very meaningful property, as it guarantees the "well-behavior" of the system. In particular, a regular point $p$ satisfies the hypothesis of the Implicit Function Theorem~\cite{Munkres1991AnalysisOM}, that we now recall and that we will use in Section \ref{subsec:GuaranteesRegularity}:


\emph{Implicit Function Theorem}: If  ${F(p_1 , \dots, p_n, p_{n+1}, \dots, p_{n+k}) = 0}$, and the Jacobian matrix of $F$ with respect to the first $n$ coordinates has non-zero determinant in $p$ (i.e. $\det(J_{F, x_{|n}}(p)) \neq 0$), 
then there exists a neighborhood $U\subseteq \mathbb{R}^k$ of $(p_{n+1}, \dots, p_{n+k})$ and a \Cone function $H: U \to \mathbb{R}^n$ such that $H(p_{n+1}, \dots, p_{n+k})=(p_1 , \dots, p_n)$, 
and such that, for all $q\in U$, ${F(H(q), q)=0}$. 
%(Note that, while the Jacobian matrix of $F$ is a $(n \times m)$ rectangular matrix, we are only interested in  the determinant of $J_{F, x_{|n}}$, 
%i.e. the determinant of the left-hand $(n \times n)$ sub-matrix of $J_F$, which depends only on the first $n$ coordinates.) 



%\subsubsection{Relation with topological degree}


%\subsection{Guarantees under assumption of regularity}
\subsection{Regularity and robustness under instantiation}
\label{subsec:GuaranteesRegularity}




In this section, we prove that the existence of a regular solution is a sufficient---but not necessary---criterion for the existence of a solution robust under instantiation.
% candidate approximate solution and of a variable instantiation that leads to a robustly satisfiable sub-system.


%under the assumption of the existence of an algorithm able to find arbitrarily precise finite approximations of a solution, 
%we provide a sufficient condition for the existence 
%of an approximate solution and 
%of a variable instantiation that generates a robust subsystem.


  We prove that, if $F$ has at least one regular solution $q$, then there exists at least one regular solution $p$ such that at least $k$ coordinates of $p$ are finitely representable rational numbers, and such that the subsystem induced from the instantiation of the $k$ corresponding variables is robustly sat. 


%We first give a brief background on the notion of regular points and on the \emph{Implicit Function Theorem}, which will play an important role for proving our statement. We assume that the reader is familiar with some basic concepts of differential analysis and topology.


% differential topology, such as \emph{smooth manifolds} and the related concepts (e.g. \emph{smooth map}, \emph{tangent space}, etc.). The interested reader can find a detailed discussion of these concepts on any differential topology textbook (e.g. \cite{Milnor1965TopologyFT}\cite{Michor}\cite{TuIntroManifolds}). We then present a few Lemmas, that will be used to prove our main statement at the end of the Section. 
 


%If $c$ is a regular value for $F$, then the set $F^{-1}(c)$ is a smooth submanifold of $\mathbb{R}^{m}$ of dimension $k$. 
%This result is known in the literature as the the \emph{Regular Value Theorem} or the \emph{Preimage Theorem} \cite{TuIntroManifolds}. 
%In our case, where $c=0$, this theorem guarantees that the solution space of $F$ is a submanifold of maximum dimension.

%Note that, by the definition of regular value, in order for $0$ to be defined as such, we need every counter-image $p$ of $0$ to be a regular point. 
%But in practice, for our goal, it suffices a weaker condition, that is the existence of a single regular point $p$ such that $F(p)=0$. 
%Indeed, if $p$ is a regular point, then there exists a neighborhood $U$ of $p$ such that every $p'\in U$ is a regular point. 
%So if we take a sub-box $\hat{B}\subseteq U$ of $B$, then $0$ is a regular value for $F_{|\interior{\hat{B}}}$.%, and we can apply the Regular Value Theorem for $F_{|\interior{\hat{B}}}$. 
%In the following, when we write the condition "$p$ is a regular solution for $F$", we imply "$0$ is a regular value for $F$", by the assumption that for a suited domain of $F$, the two conditions are equivalent.
	
	
	%	Thm 2.2.19  \ \ \url{https://people.math.ethz.ch/~salamon/PREPRINTS/diffgeo.pdf} 
	
	%See Lemma 1 on Section 2 of \cite{Milnor1965TopologyFT}
	

%Let $F: \interior{B}\subseteq \mathbb{R}^m \to \mathbb{R}^n$ be a $C^{\infty}$ function with $m-n = k$ $(\neq 0$) that has a robust zero $\mathbf{p}$. Suppose that $\textbf{0}$ is a regular value for $f$ (i.e. for all $\mathbf{p}\in F^{-1}(\mathbf{0})$ the map $d F_x:T_x\interior{B}\to T_0\mathbb{R}^n $ is surjective, i.e. the Jacobian matrix has maximal degree in every point of the zeroset).



%We can use the following two results: 
%\begin{theorem}[Inverse function]
%	\label{thm:inverseFunction}
%	Let $F:\interior{B}\subseteq \mathbb{R}^{n+k}\to \mathbb{R}^n$ be a $C^1$-function. Given $(\mathbf{a}, \mathbf{ b})=(\mathrm{a_1},\cdots, \mathrm{a_n}, \mathrm{b_1}, \cdots, \mathrm{b_k})$ s.t. $F(\mathbf{a},\mathbf{b})=\mathbf{0}$, if Jacobian matrix has max degree in $(\mathbf{a}, \mathbf{ b})$, then there exists an open $U\subseteq \mathbb{R}^k$ with $\mathbf{a}\in U$ s.t. there exists an unique function $H: U\to \mathbb{R}^n$ s.t.  $H(\mathbf{a})=\mathbf{b}$ and for all $\mathbf{x}\in U$ $F(\mathbf{x},H(\mathbf{x}))=0$.
%\end{theorem}

%\begin{theorem}[Regular values]
	%\label{thm:regvalues2}
	%Let $F$ as in the previous theorem. Then the set $F^{-1}(\mathbf{0})$ is a submanifold of $\mathbb{R}^{n+k}$ of %dimension $k$.
%\end{theorem}

 

We first show that any regular solution is also a robust solution (Lemma \ref{lemma:regulImpliesRobust}), which will be useful to prove the main result of this section, Lemma \ref{thm:FRegsubsetFRob}, which implies  $\FReg \subseteq \FRobI$.  Then, we will show that this inclusion is strict by providing a counter-example in the form of a system of equations that has a solution robust under instantiation but no regular solutions (Lemma \ref{lemma:FRobInotsubsetFReg}).







\begin{lemma}
	\label{lemma:regulImpliesRobust}
	Given a \Cone function $F:B\subseteq\mathbb{R}^n\to \mathbb{R}^n$, if $p$ is a regular solution for $F$, then $p$ is a robust solution for $F$.
\end{lemma}

\begin{proof}
	Assume that $p$ is a regular solution of $F$. 
	Hence the Jacobian of $F$ at $p$ has maximal rank. We prove that $p$ is a robust solution of $F$. 
	So let $\varepsilon>0$ be arbitrary, but fixed
	%Let $\varepsilon'\leq \varepsilon$
	and such that $p$ is the unique solution of $F$ in $\openball{B}{\epsilon}(p)$.
	Such $\varepsilon$ always exists, %since the solution space of $F$ is a submanifold of dimension $n-n = 0$ (by Thm 9.9 \cite{TuIntroManifolds}
        since by the inverse function theorem $F$ maps a neighborhood of $p$ diffeomorphically onto an open set of $\mathbb{R}^n$~\cite[Chapter 1.2]{Milnor1965TopologyFT}.
	%i.e. the solution space is made of disconnected points.
	Hence $0\not\in \partial \openball{B}{\epsilon}(p)$, and, since $p$ is the only solution of $F$ in   $\openball{B}{\epsilon}(p)$, and it is regular, then, by definition,
	$\deg(F, \openball{B}{\epsilon}(p), 0) = |\det(J_F(p))| \neq 0$.	
	%the topological degree of $F$ in $\openball{B}{\epsilon}(p)$ is non-zero
	Let $\delta<\min_{x\in\partial \openball{B}{\epsilon}(p)} |F(x)|$. 
	 %Let $\phi \equiv F=0$. 
	 By Lemma~1~\cite{Franek:12}, every $F'$ with $\dist(F, F')<\delta$ has a zero in $\openball{B}{\epsilon}(p)$. 
	This proves that $p$ is a robust solution for $F$.
\end{proof}


\begin{lemma}
	\label{thm:FRegsubsetFRob}
	%Let $F=0$ be a system of equations with $F:\mathbb{R}^{n+k}\to \mathbb{R}^n$. If $0$ is a regular value for $F$, then there exists $p$ such that at least $k$ coordinates of $p$ are non-periodic rational numbers and $p$ is a robust solution of $F$.
	Let $F:B\subseteq\mathbb{R}^{m}\to \mathbb{R}^n$ (with $m=n+k$) be a \Cone function. If there exists a regular solution $q$ of $F$, then there exists a regular solution $p$ of $F$  in a neighborhood of $q$ such that $p$ is robust under instantiation.
	
	%\
	%a \kfinite{k}, and such that the projection $p_{|n}$ of $p$ over the first $n$ coordinates  
	%is a robust solution for the system of equations $F_{|n}: B_{|n} \subseteq \mathbb{R}^n \to \mathbb{R}^n$ obtained by instantiating the $k$ variables corresponding to the $k$ coordinates of $p$.% (and where $B_{|k}$ is the projection of $B$).
\end{lemma} 

\begin{proof}
	%Thm\ref{thm:regvalues}+Lemma\ref{lemma:submanifFloats}+Lemma\ref{lemma:regulImpliesRobust}.
	
	%\ \\
	If $q$ is a regular solution for $F$, then $J_F(q)$ has maximum rank. 
	Since a rectangular matrix has maximum rank if and only if one of its maximal square sub-matrix has maximum rank,
	 then, without loss of generality, we can reorder the variables so that the square sub-matrix given by the first $n$ columns has maximum rank, 
	 i.e. $\det(J_{F, \projn{x}}(q)) \neq 0$. 
	By the Implicit Function Theorem, there exists a neighborhood $U\subseteq \mathbb{R}^k$ of $(q_{n+1}, \dots, q_{n+k})$ and a \Cone function ${H: U \to \mathbb{R}^n}$ such that $H(q_{n+1}, \dots, q_{n+k})=(q_1 , \dots, q_n)$, and such that, for all $q'\in U$, ${F(H(q'), q')=0}$.
	
	In general, it is not guaranteed that every $(H(q'), q')$ will be a regular point for $F$. 
	However, since the Jacobian $J_F: \mathbb{R}^m \to \mathbb{R}^{m \times n}$%
	, the projection ${\pi_{n \times n}: \mathbb{R}^{m \times n} \to \mathbb{R}^{n \times n}}$
	(that projects a $m\times n$ matrix onto the $n\times n$ sub-matrix of its first $n$ columns)
	and the determinant $\det: \mathbb{R}^{n \times n} \to \mathbb{R}$ are all continuous functions, then the set $U_r \subseteq \mathbb{R}^m$, 
	consisting of all the  points $q'$ for which $\det(J_{F,\projn{x}}(q'))\neq 0$, is open, since  $\mathbb{R} \setminus \{0\}$ is open and, by definition, $U_r = (det \circ \pi_{n \times n} \circ J_F)^{-1}(\mathbb{R} \setminus \{0\})$.
	
	Let us consider the projection of $U_r$ over $\mathbb{R}^k$, i.e. $U_{r_{|k}} = \pi_k(U_r) = \{(q'_{n+1}, \dots, q'_{n+k}) \in \mathbb{R}^k | (q'_1 , \dots, q'_n, q'_{n+1}, \dots, q'_{n+k}) \in U_r  \}$. Since projections are open maps, then $U_{r_{|k}}$ is open.  Since both $U$ and $U_{r_{|k}}$ are neighborhoods of $(q_{n+1}, \dots, q_{n+k})$, then their intersection $U'\defas U \cap U_{r_{|k}}$ is again a neighborhood of $(q_{n+1}, \dots, q_{n+k})$.
	
	
	Now we prove that $U'$ contains at least one  \kfinite{k} $p'$. 
	The set of \kfinites{k} in $\mathbb{R}^k$ is exactly the set of points having coordinates with finite representation.
	Let us call this set $A$.  
	We have that $A$ is dense in $\mathbb{R}^k$, as, for every point ${z=(z_1,\dots , z_k) \in  \mathbb{R}^k}$, $z$ is the limit of the sequence $\{([z_1]_i, \dots , [z_k]_i) \}_{i\in \mathbb{N}} \subseteq A$, where $[z_j]_i$ is the truncation of $z_j$ to the $i$-th digit after the zero. 
	Since $A$ is dense in $\mathbb{R}^k$, then $A$ intersects every non-empty open of $\mathbb{R}^k$. 
	In particular $A \cap U' \neq \emptyset$, hence there exists $p'\in A \cap U'$. 
	%So $p'$ is a $k$-decimal point for which $F(H(p'), p')=0$. 
	
	Let $p\defas(H(p'), p')\in \mathbb{R}^m$. 
	We have that $p$ is a solution for $F$ (since $F(p) = F(H(p'), p')=0$). 
	Moreover, $p$ is also regular, since $p'\in U' \subseteq U_{r_{|k}}$ and hence $(H(p'), p') \in U_r$.
	
	Let $\nu \defas \{x_i \mapsto p_i\}_{i \in [n+1,  n+k]}$. 
	We have that the point $\pn = H(p') \in \mathbb{R}^n$ is a regular point for $\Fv$. Indeed, since $p$ is a regular point, and $J_{F, x_{|n}}(p)$ depends only on the first $n$ coordinates, then $\det(J_{\Fv}(\pn)) \neq 0$.
	
	Since $\pn$ is a regular solution for $\Fv$,  by  Lemma~\ref{lemma:regulImpliesRobust}, $\pn$ is also a robust solution for $\Fv$. Hence $p$ is robust under instantiation.
\end{proof}

Modifying the proof by choosing directly  $p$ as $q$ we get:
\begin{corollary}
	\label{corollary:kfiniteAndRegImpliesRobI}	
	Let $F:B\subseteq\mathbb{R}^{m}\to \mathbb{R}^n$ (with $m=n+k$) be a \Cone function
	  If $q$ is  both a regular solution and a \kfinite{k}, then $q$ is robust under instantiation.
\end{corollary}



Now we show that the converse of Lemma \ref{thm:FRegsubsetFRob} does not hold, i.e. that the existence of a solution robust under instantiation does not imply the existence of a regular solution. Consider the following example:
\begin{example}[Critical solution, but robust under instantiation]
	Let $F: [0, 1]^2 \subseteq \mathbb{R}^2\to \mathbb{R}$ defined by $F(x,y) = (x^2-y^3)$,
	and let $p = (0, 0)$. $J_F(p)=(0,0)$ has non-maximum rank (hence $p$ is a critical solution), but the instantiation $\{x\mapsto 0 \}$ leads to the subsystem $-y^3 = 0$, which is robust.
\end{example}

In this example, we could have chosen a different point, say $p'=(1,1)$, which is both robust under instantiation and regular. 

However, this is not always possible. A system can have a solution robust under instantiation, but no regular solutions. Indeed:



\begin{lemma}
	\label{lemma:FRobInotsubsetFReg}
	$\FRobI \not\subseteq \FReg$
\end{lemma}
\begin{proof}
	%\begin{example}[Robust under instantiation, but with no regular solutions]
	Let ${F:[-1,1]^3\subseteq \mathbb{R}^3\to \mathbb{R}^2}$ defined by 
	\begin{equation*}
		F(x_1, x_2, x_3)=
		\begin{cases}
			x_1^3 & (F_1)\\
			x_2+x_3 & (F_2)
		\end{cases}
	\end{equation*}
	
	\ \\
	The point $(0,0,0)$ is robust under instantiation. Indeed, the instantiation ${\{x_3 \mapsto 0 \}}$ leads to the following system of equations
	\begin{equation*}
		F'(x_1, x_2)=
		\begin{cases}
			x_1^3 & (F'_1)\\
			x_2 & (F'_2)
		\end{cases}
	\end{equation*}
	which has non-zero degree, hence it is robustly sat. % (and so $F\in \FRobI$). 
	It is easy to show that the degree of $F'$ in $[-1,1]^2$ is non-zero. 
	In fact, $F'_1$ depends only on $x_1$ and $F'_2$ only on $x_2$,
	and for both $F'_1$ and $F'_2$ it suffices to apply the Intermediate Value Theorem 
	to prove that $\deg(F'_i, [-1,1], 0) \neq 0$ (for $i=1,2$).
	Since the degree of the Cartesian product is the product of the degrees~\cite[Theorem 7.1.1]{BrouwerDegreeDincaMawhin},
	$\deg(F', [-1,1]^2, 0) = \deg(F'_1, [-1,1], 0) * \deg(F'_2, [-1,1], 0)  \neq 0 $.
	
	
	%	which has non-zero degree, hence it is robustly sat (and so $F\in \FRobI$). To prove that the degree of $F'$ in $[-1,1]^2$ is non-zero without resorting to software such as \textsc{TopDeg}, we can compute the degree manually.
	
	
	So $F\in \FRobI$.	
	However, $F$ does not have any regular solution. 
	Indeed, the first equation implies that for every every solution  $p$ its first coordinate has to be $p_1 = 0$.
	Since, for such a solution $p$, the first row of $J_F(p)$ is everywhere $0$, then the Jacobian cannot have maximum rank. 
	Hence every solution $p$ is not regular, i.e. $F\not\in \FReg$.
	%\end{example}
\end{proof}





\subsection{Robustness preservation after variable instantiation}
\label{subsec:RobPreservationAfterVarInst}

In the previous section, we have proven that, under the  assumption of the existence of a regular solution, there exists a solution that is robust under instantiation.


But what happens if we drop the assumption of regularity? In general, if we don't put any restriction on the functions we are considering, the solution space can be arbitrarily complicated. 
Indeed, for every closed subset $K \subseteq \mathbb{R}^m$, there exists a smooth function $F$ such that $F^{-1}(0)=K$ (\cite[Theorem 2.29]{LeeIntroSmooth}).

One may hope that, by restricting to functions that have a robust solution, we can always find a solution robust under instantiation. In this section, we show that this, unfortunately, does not hold. Consider the following example.
\begin{example}[Robust solution, but non-robust under instantiations]
	\label{ex:robButNonrobUnderInst}
	Let ${F: [-1,1]^2\subseteq \mathbb{R}^2\to \mathbb{R}}$ defined by $F(x,y) = (x^2 - y^2)$,
	and let $p = (0, 0)$. It is easy to show that $p$ is a robust solution. However, whether we instantiate $\{x\mapsto0\}$ or $\{y\mapsto 0\}$, 
	the resulting subfunctions (resp. $F_{|\{x\mapsto0\}}(y)=y^2$ and $F_{|\{y\mapsto0\}}(x)=x^2$)  are not robust.
\end{example}
In this example, we could have chosen another point, for example ${p'=(1,1)}$, which is regular (since $J_F(p')=(2,-2)$), and hence, by Corollary \ref{corollary:kfiniteAndRegImpliesRobI}, robust under instantiation. But in general, this is not always possible. Indeed, we have the following result:

\begin{lemma}
	\label{lemma:FRobnotsubsetFRobI}
	$\FRob \not\subseteq \FRobI$
\end{lemma}
\begin{proof}
%	We will provide a counter-example to $\FRob \subseteq \FRobI$
%\begin{example}[Robust formula, with only \nonRobustUnderInst solutions]
	\label{ex:robNotRobI}
	Let $F: [-1, 1]^4\subseteq \mathbb{R}^4 \to \mathbb{R}^3$ defined by 
	
	\begin{equation*}
		F(x_1, x_2, x_3, x_4)=
		\begin{cases}
			x_1^2+x_2^2-x_3^2-x_4^2 \\
			
			2(x_1 x_4+x_2 x_3) \\
			
			2(x_2 x_4-x_1 x_3)  \\
		\end{cases}
	\end{equation*}
	
	It is easy to show that $p=(0,0,0,0)$ is the only solution of $F$. Moreover, $p$ is robust (see the discussion about Hopf maps in 
	%\cite{RobsatImplementation}
	\cite{FranekHopf})
	, hence $F\in \FRob$. However, no instantiation $\nu_i \defas \{x_i \mapsto 0\}$ is robust. Indeed,  $(0, 0, 0)$ is the only solution of $F_{|\nu_i}$ in $[-1,1]^3$, but $\deg(F_{|\nu_i}, [-1,1]^3, 0) = 0$ (remember that, if a system of equations $F$ has an isolated robust solution in $B$, then $\deg(F, B, 0)\neq 0$). So $F\not\in \FRobI$.
%\end{example}
\end{proof}


%Notation: $I_{x_i = q_i}$ is the function $I(x_1,\dots x_m)\subseteq \mathbb{R}^m\to \mathbb{R}^n$ defined by $I(x_1, \dots, x_m) = x_i - q_i$

Now we show that the converse holds, i.e. that  every solution robust under instantiation is robust:

\begin{lemma}
	\label{lemma:FRobIimpliesFRob}
	Let $F:B\subseteq \mathbb{R}^{n+k}\to \mathbb{R}^n$. If $p$ is a solution robust under instantiation, then $p$ is a robust solution. 
\end{lemma}

\begin{proof}
	% Using variable instantiations
	If $p$ is a solution robust under instantiation, then there exists a set of indices $I$ (w.l.o.g. say $I=\{n+1, \dots, n+k\}$) 
	and a corresponding instantiation $\nu$ such that $\pn$ is a robust solution for $\Fv : B_{|\mathbb{R}^n} \subseteq \mathbb{R}^n\to \mathbb{R}^n$, that is,
	 for every $\epsilon>0$ there is a $\delta$ such that for all $\Fv'$ with $\dist(\Fv,\Fv')<\delta$, 
	there exists a solution $\pn'$ of $\Fv'$ with $\dist(\pn, \pn')<\epsilon$.
	 
	
	To prove that $p$ is a robust solution for $F$ let $\epsilon>0$ be arbitrary, but fixed, and take the corresponding $\delta$ as ensured by robustness under instantiation.
	For every $F'$ with $\dist(F,F')<\delta$,
	we have that also $\dist(\Fv,
	\Fv')<\delta$, hence there exists $\pn'$ that satisfies $\Fv'$ with $\dist(\pn, \pn')<\epsilon$. If $\pn'$ satisfies $\Fv'$, then $p'\defas (\pn', \pnk)$ satisfies $F'$. Since $\dist(p, p') = \dist(\pn, \pn')< \epsilon$,  $p$ satisfies the definition of robust solution for~$F$.
\end{proof}





A straightforward corollary of Lemma \ref{lemma:FRobIimpliesFRob}, is that  $\FRobI \subseteq \FRob$
%
which
concludes the proof that $\FRobI \mysubsetneq \FRob$.  Together with Lemma~\ref{lemma:FRobInotsubsetFReg} and Lemma~\ref{thm:FRegsubsetFRob}, this implies Theorem~\ref{thm:FRobIbounds}.





\subsection{Variable instantiation vs. equation adding}
\label{subsec:InstVarVsAddEq}

Given an under-constrained system of equations, we showed that there are two different ways to reduce to a well-constrained system of equations: decreasing the number of variables (i.e. instantiations) or increasing the number equations. In this section we will show that the class of systems that can be solved via variable instantiation is a subset of the class of systems that can be solved via adding equations.


Recall our definition of $\FRobLEq$ as the set of functions $F: B \subseteq \mathbb{R}^{n+k}\to \mathbb{R}^{n}$ such that there exists a linear function $L: \mathbb{R}^{n+k}\to \mathbb{R}^k $ such that the function ${F_{leq}: B \subseteq \mathbb{R}^{n+k}\to \mathbb{R}^{n+k}}$, defined by $F_{leq}: x \mapsto (F(x), L(x))$,
 has a robust solution.

%The main result of this section will be to prove that $\FRobI \subseteq \FRobLEq$.



Given any partial assignment $\nu \defas \{x_i \mapsto p_i\}_{i \in [n+1, n+k]}$, 
we can consider the function $F_{leq} = (F, L)$, given by $F$ and by the linear function ${L: B\subseteq \mathbb{R}^{n+k}\to \mathbb{R}^k}$
defined by ${L:(x_1,\dots, x_{n+k})\mapsto (x_{n+1} - p_{n+1}, \dots, x_{n+k}-p_{n+k})}$. Equivalently, since $L$ only depends on the last $k$ variables, we can consider it as a function ${L: B_{|\mathbb{R}^k}\subseteq \mathbb{R}^k\to \mathbb{R}^k}$. 

%While it is trivial to show that there is a one-to-one correspondence between the solutions of $F_{|\nu}$ and the solutions of $F_{leq}$, we need to prove that also the robustness of a solution is preserved. 
%This will be shown in the following lemma, that straight-forwardly implies that $\FRobI \subseteq \FRobLEq$.
 
 While it is trivial to show that every solution of $F_{|\nu}$  is also a solution of $F_{leq}$, we need to prove that also the robustness of a solution is preserved. 

\begin{lemma}
	\label{lemma:instVarVsAddEq}
	Given a system of equations $F = 0$ (with $F: B \subseteq \mathbb{R}^{n+k}\to \mathbb{R}^n$),
	a point $p=(p_1, \dots, p_{n+k}) \in B$ and subset of indexes $I:=\{n+1 , n+2, \dots, n+k\}$, let us consider the following statements:
	
	\begin{enumerate}
		\item For the function $\Fv:  B_{|\mathbb{R}^n} \subseteq \mathbb{R}^n\to \mathbb{R}^n$, obtained by instantiating variables via $\nu = \{x_i \mapsto p_i\}_{i\in I}$, the point $\pn \defas (p_1,\dots, p_n)$ is a robust solution 
		\item For the function $F_{leq}:  B \subseteq \mathbb{R}^{n+k}\to \mathbb{R}^{n+k}$, defined by
		$$F_{leq}(x_1, \dots, x_{n+k}) \mapsto (F(x_1, \dots, x_{n+k}), L(x_{n+1}, \dots, x_{n+k}))$$ where
		$L: B_{|\mathbb{R}^k} \subseteq  \mathbb{R}^k\to \mathbb{R}^k$ is defined by \[{L:(x_{n+1},\dots, x_{n+k})\mapsto (x_{n+1} - p_{n+1}, \dots, x_{n+k}-p_{n+k})},\] the point $p$ is a robust solution.
	\end{enumerate}
	\ 
Then, it holds that $1.$ implies $2.$.
\end{lemma}

\begin{proof}
	
	
	To prove that
	1. $\Rightarrow$ 2., it will be useful to consider a third auxiliary condition:
	\begin{enumerate}
		\item[\emph{3.}] 
		\emph{For the function $F_{{|\nu}_{leq}} \defas F_{|\nu} \times L : B \subseteq \mathbb{R}^{n+k}\to \mathbb{R}^{n+k}$ obtained by the Cartesian product between $F_{|\nu}$ and the linear function $L$, the point $p$ is a robust solution
	}
		%\item The system $F_{leq} \defas F \times \{x_i- p_i\}_{i \in I}$ obtained by adding equalities ${\{x_i - p_i = 0\}_{i\in I}}$ has a robust solution
	\end{enumerate}
	and prove first that 
	$1. \Rightarrow 3.$ and then that  
	$3. \Rightarrow 2.$
	%First, consider  $F_{{|\nu}_{leq}} \defas F_{|\nu} \times L$.  
	 %$F_{leq_{|\nu}}$ is the Cartesian product of the two functions of $ F_{|\nu}$ and $L$. 
	 
	 ($1. \Rightarrow 3.$ )
	By Theorem 7.1.1 \cite{BrouwerDegreeDincaMawhin}, the degree of the Cartesian product is the product of the degrees,
	i.e., for every $\Omega = \Omega_1 \times \Omega_2 \subseteq \mathbb{R}^{n}\times \mathbb{R}^k$,
	\begin{equation}
		\deg(F_{{|\nu}_{leq}}, \Omega, 0) = \deg(F_{|\nu}, \Omega_1, 0) * \deg(L, \Omega_2, 0 ) 
	\end{equation}
	Since $J_L(\pnk)$ is the identity matrix, the point $\pnk$ is a regular solution for $L$. Since $\pnk$ is the only solution of $L$, then, for every $\Omega_2 \subseteq \mathbb{R}^{k}$ such that $\pnk \in \Omega_2$, $\deg(L, \Omega_2, 0 ) = \det(J_L(\pnk)) = 1$.
	  
	Thus, for every $\Omega = \Omega_1 \times \Omega_2 \subseteq \mathbb{R}^{n}\times \mathbb{R}^k$ such that $p\in \Omega$, we have that
	\begin{equation}
		\label{eq:Thm711}
		\deg(F_{{|\nu}_{leq}}, \Omega, 0)  = \deg(F_{|\nu}, \Omega_1, 0)
	\end{equation} 


	 If 1. holds, then,  by Thm. 6 \cite{Franek:12}, for every $\epsilon>0$ there exists an open ${\Omega_{1,\epsilon} \subseteq \openball{B}{\epsilon}(\pn)}$ such that $\deg(F_{|\nu}, \Omega_{1,\epsilon}, 0) \neq 0$. 
	 Let 
	 %$\Omega_2 \defas \openball{B}{1}(\pnk)$
	 $\Omega_2$ be any neighborhood of $\pnk$ 
	 s.t. $\Omega_2 \subseteq \openball{B}{\epsilon}(\pnk)$,
	 and let $\Omega_\epsilon \defas \Omega_{1,\epsilon} \times \Omega_2$. 	 
	 By Equation \ref{eq:Thm711}, ${\deg(F_{{|\nu}_{leq}}, \Omega_\epsilon, 0) \neq 0}$. 
	 Hence,
	 for every $\epsilon>0$, $F_{{|\nu}_{leq}}$ is robustly sat in $ \Omega_{1,\epsilon}$. 
	 So we have constructed, for every $\epsilon>0$, a neighborhood $\Omega_{1,\epsilon} \subseteq \openball{B}{\epsilon}(p)_{\epsilon}$ of $p$ in which $F_{{|\nu}_{leq}}$ is robustly sat.  
	 By Proposition \ref{prop:locallyRobustlysat}, this implies that $p$ is a robust solution for $F_{{|\nu}_{leq}}$.
	
	  
	 \ 
	 
	($3. \Rightarrow 2.$ )
	Now we show that if $p$  is a robust solution for $F_{{|\nu}_{leq}}$ then it is a robust solution for    $F_{leq}$.	We will construct a homotopy between $F_{{|\nu}_{leq}}$  and    $F_{leq}$, and then rely on the Homotopy Invariance Property of the topological degree to prove the claim.
	
	Let $S:  \mathbb{R}^{n+k }\times [0,1] \to \mathbb{R}^{n+k }$ be defined by
        \vspace*{-0.32cm}\begin{multline*}
          S:((x_1, \dots, x_n, x_{n+1},\dots x_{n+k}), t) \mapsto\\
              (x_1, \dots, x_n, \ tx_{n+1} + (1-t)p_{n+1},\ \dots \ ,\ tx_{n+k} + (1-t)p_{n+k})
        \end{multline*}          
	
	The map $H: B \times [0,1]  \subseteq   \mathbb{R}^{n+k} \times [0,1]  \to \mathbb{R}^{n+k}$ 
	defined by
	%defined as ${H\defas( F\circ S) \times L}$. 
	$$H:(x_1, \dots, x_{n+k}, t)\mapsto (( F\circ S)(x_1, \dots, x_{n+k}, t) \ , \ L((x_{n+1}, \dots, x_{n+k}))) $$
	 is a homotopy between $F_{{|\nu}_{leq}}$ and $F_{leq}$ since
	 $H(\cdot, 0) \equiv F_{{|\nu}_{leq}}$, $H(\cdot, 1) \equiv F_{leq}$, and $H$ is continuous, being the composition of continuous functions. 
	
	It is easy to see that the functions $F_{leq}$ and $F_{{|\nu}_{leq}}$ have exactly the same solution space. Indeed, every solution of $F_{leq}$ has to satisfy the equations given by $L$, i.e. ${x_{n+1}=p_{n+1}, \dots, x_{n+k}=p_{n+k}}$. 
	By replacing in $F_{leq}$  every $x_i$ with $p_i$, for ${i\in [n+1, n+k]}$, we obtain exactly $F_{{|\nu}_{leq}}$. 
	
	Furthermore, for every $t\in [0,1]$ the solution space of $H(\cdot , t)$ is the same as $H(\cdot , 0) \equiv F_{{|\nu}_{leq}}$. 
	Indeed, for every $t$, a solution of $H(\cdot , t)$ has to satisfy the equations given by $L.$ 
	Then, by replacing  every $x_i$ with $p_i$ for $i\in[n+1, n+k]$, we have that every $tx_i + (1-t)p_i $ is replaced by $tp_i + (1-t)p_i$, which is equal to $p_i$. Thus we reduced  again to  $F_{{|\nu}_{leq}}$.
	
	%Now we can check that, if 3. holds, then $H$ satisfies the conditions required to apply the  Homotopy Invariance Property, 
	%i.e. that  $0\not\in H(\partial B \times [0,1])$.
	
	Now, suppose that 3. holds. Then, for every $\epsilon>0$, there exists $\Omega_\epsilon \subseteq \openball{B}{\epsilon}(p) $ such that $\deg(F_{{|\nu}_{leq}}, \Omega_\epsilon, 0)\neq 0$. 
	%$p$ is the a solution for   $F_{leq}$ (resp. $F_{{|\nu}_{leq}}$) in $\Omega_\epsilon$. 
	This implies that  ${0\not \in F_{{|\nu}_{leq}}(\partial  \Omega_\epsilon)}$, hence  ${0\not \in H(\partial  \Omega_\epsilon, 1)}$.
	 So, by the previous observation, $0\not \in H(\partial  \Omega_\epsilon, t)$ for every $t\in [0,1]$. Hence $0\not \in H(\partial  \Omega_\epsilon, [0,1])$, and we can apply the Homotopy Invariance Property.
	
	The Homotopy Invariance Property of the topological degree states that, if $0\not\in H(\partial  \Omega_\epsilon \times [0,1])$, then $\deg(H(\cdot, 0),  \Omega_\epsilon, 0) = \deg(H(\cdot, 1),  \Omega_\epsilon, 0)$, i.e. ${\deg(F_{{|\nu}_{leq}},  \Omega_\epsilon, 0) = \deg(F_{leq},  \Omega_\epsilon, 0)}$. 
	So we have constructed, for every $\epsilon >0$, a neighborhood $\Omega_\epsilon \subseteq \openball{B}{\epsilon}(p)$ of $p$ in which $F_{leq}$ is robustly sat. By Proposition~\ref{prop:locallyRobustlysat}, this implies that $p$ is a robust solution for  $F_{leq}$.
	
\end{proof}

So robustness of the system obtained by variable instantiation implies robustness of  the system obtained by adding the equalities corresponding to this variable instantiation.
Theorem~\ref{thm:instVersusEq} is a straight-forward consequence.


 


\subsection{Termination}
\label{subsec:termination}
The method discussed in Section \ref{sec:certificate-search} made use of numerical optimization to enumerate the points over which the variable instantiation method is applied. This technique, while practically very efficient---as shown by our experiments---is not guaranteed to terminate, in general. Indeed, even in the bounded case, numerical optimization does not guarantee full coverage of the space.

In this section, we present a variation of our method that uses a different technique for enumerating points, and that is guaranteed to terminate on problems in $\FRobI$. While this variation is not intended to be of practical use, it will serve the purpose of proving Theorem \ref{thm:quasiquasidecidability}.

\ 

Given $F: B \subseteq \mathbb{R}^{n+k} \to \mathbb{R}^n$, if $F \in \FRobI$, by definition we know that there exists a \kfinite{k} $p$ that is robust under instantiation. 
Such a point~$p$ is, in general, not a \kfinite{(n+k)} (indeed,  there are problems for which no solution is a \kfinite{(n+k)}).
Hence no point enumeration technique is guaranteed to find precisely $p$, 
since only \kfinites{(n+k)}  can be expressed explicitly. 
However, this is not an actual limitation. 
Indeed, for our method to succeed, we don't necessarily need to explicitly produce a solution. We just need to find a point sufficiently close to an actual solution, and that shares with the solution $k$ indices, so that, after the instantiation of the corresponding $k$ variables, we end up with a subproblem that is robustly satisfiable. This suffices to produce a certificate.

The following lemma shows that, for every problem in $\FRobI$, it is always possible to find a \kfinite{(n+k)} and a partial assignment such that the resulting subproblem is robustly satisfiable.

\begin{lemma}
	\label{lemma:nkfinite}
	 Let $F : B \subseteq \mathbb{R}^{n+k} \to \mathbb{R}^n$. If $F \in \FRobI$, then there exist a {\kfinite{(n+k)}}  $p'\in B$ and a partial assignment $\nu' \defas \{x_i \to p'_i\}_{i\in I}$ (with $I$ being a set of $k$ indices), such that the function ${F_{|\nu'}: B_{|\mathbb{R}^n} \subseteq \mathbb{R}^n \to \mathbb{R}^n}$ is robustly satisfiable.
 \end{lemma}
\begin{proof}
	$F \in \FRobI$ means that there exists a \kfinite{k} $p$ 
	(w.l.o.g. say the $k$ finitely representable indices are $[n+1, n+k]$) 
	that is robust under instantiation,
	i.e. there exists a partial assignment $\nu\defas \{x_i \mapsto p_i\}_{i\in [n+1, n+k]}$ 
	such that $p_{|\nu}$ is a robust solution for $\Fv$. 
	This implies that $\Fv$ is robustly satisfiable in  $B_{|\mathbb{R}^n}$.
	
	Now, given any $p' \in B$ such that $p'_{n+1} = p_{n+1}, \dots, p'_{n+k} = p_{n+k}$, 
	and, given $\nu'\defas \{x_i \mapsto p'_i\}_{i\in [n+1, n+k]}$, 
	we have that $\nu' \equiv \nu$, hence $\Fv' \equiv F_{|\nu}$, 
	which implies that $F_{|\nu'}$ is robustly satisfiable in  $B_{|\mathbb{R}^n}$. 
	In order to find a $p'$ that respects the statement conditions, first we fix the last $k$ coordinates to be equal to $(p_{n+1},\dots,p_{n+k})$. 
	Then, since the set of \kfinites{n} is dense in $\mathbb{R}^{n}$, and hence intersects $\interior{B_{|\mathbb{R}^n}}$ (being an open), 
	there exists a \kfinite{n} $(p'_1, \dots, p'_n) \in \interior{B_{|\mathbb{R}^n}}$.
	%, 
	%then we construct an infinite sequence of \kfinites{n} in $\mathbb{R}^n$ having limit point $(p_1, \dots, p_n)$ 
	%(this is always possible since the set of \kfinites{n} is dense in $\mathbb{R}^{n}$). 
	%Eventually, the points in the sequence will be contained in $B_{|\mathbb{R}^n}$. 
	If we take such point, and append the last $k$ coordinates previously fixed, 
	we obtain a point $p'\defas (p'_1, \dots, p'_n, p_{n+1}, \dots, p_{n+k})$. Since the last $k$ coordinates of $p'$ coincides with the last $k$ coordinates of $p$, we have that $F_{|\nu'}$ is robustly satisfiable in  $B_{|\mathbb{R}^n}$. 
	Moreover, such $p'$ is a \kfinite{(n+k)}. 
	Indeed, the part consisting of the first $n$ coordinates is $n$-finite by construction, 
	while the second part consisting of the last $k$ coordinates is $k$-finite because $p_{n+1},\dots,p_{n+k}$ is.
\end{proof}


Considering a bounded system of equations and inequalities satisfiable iff it is satisfiable by
a variable assignment the assigns values within the corresponding interval to all variables,
it is straightforward to extend the definitions from Section~\ref{subsec:BackgroundRobustness} analogically from systems of equations to bounded systems of equations and inequalities. Based on this, we can prove the following theorem. 
%	In the statement, we will use the more general notion of robust (un)satisfiability as defined in \cite{Franek:12}.
\begin{theorem}
	\label{thm:quasiquasidecidability}
	There exists a procedure that, given a bounded system of equations and inequalities $F=0 \wedge G \leq 0$,
	\begin{itemize}
        \item always returns the correct answer ``satisfiable'' or ``unsatisfiable'', if it terminates,
        \item always terminates successfully when $F=0 \wedge G \leq 0$ is robustly satisfiable  and  $F\in \FRobI$,
        \item always terminates successfully when $F=0 \wedge G \leq 0$ is robustly unsatisfiable.
	\end{itemize}
\end{theorem}

	\begin{proof}
	 
	We first concentrate on the second point, by showing a procedure that always correctly terminates on problems in $\FRobI$.


	
By Lemma \ref{lemma:nkfinite}, we have that,
for every $F: B \subseteq \mathbb{R}^{n+k}\to \mathbb{R}^n $ such that $F \in \FRobI$, there exists a \kfinite{(n+k)}~$p'$ and a partial assignment $\nu'$ such that $F_{|\nu'}$ is robustly satisfiable.
We can always find such $p'$ and $\nu'$. 
Indeed,
since the set of \kfinites{(n+k)} is countable, we can construct a well-order: 
say, for example, we first take the finite set of points whose  coordinates are represented by at most $1$ digit 
(and sort it by lexicographic order),
then the finite set of points  whose coordinates are represented by at most $2$ digits, and so on
\footnote{Note that this is independent by the Axiom of Choice, which is needed only in the case of uncountable sets.}.
For each such point $p$, and for each instantiation $\nu$ (note that the set of possible instantiations is finite), we can consider the   
resulting subsystem $F_{|\nu'} : B_{|\mathbb{R}^n}\subseteq \mathbb{R}^n \to \mathbb{R}^n$, 
and then apply the box-gridding procedure
discussed in Section \ref{subsec:box}, which---without the stopping criterion---is guaranteed to terminate on robust instances~\cite{Franek:12}. Note that box-gridding method also handles inequalities.



The procedure described so far is not yet guaranteed to converge. 
Indeed, while box-gridding is guaranteed to terminate on robust instances, it could diverge on non-robust instances, thus preventing the general procedure to terminate (either because one point yielded
%yeeted\SRtodo{never heard about this word} 
before $p'$ was non-robust, or because one instantiation tried before $\nu'$ was a non-robust instantiation).

We can overcome this problem by using, instead of depth-first search, a technique called dove-tailing. 
Indeed, we have an infinite sequence of problems (given by the combination of points and instantiations), and, for each, a (possibly infinite) sequence of box-gridding iterations. 
We outline the following iterative procedure:
\begin{itemize}
	\item For $i=1$, we perform the $1$-st box-gridding iteration on the $1$-st problem.
	\item For $i=2$, we perform the $2$-nd box-gridding iteration on the first problem, and then the $1$-st box-gridding iteration on the second problem.
	\item \dots
	\item For $i=N$, we perform the $N$-th box-gridding iteration on the first problem, then the $(N-1)$-th box-gridding iteration on the second problem, \dots, and then the $1$-st box-gridding iteration on the $N$-th problem.	
\end{itemize}
First, we are guaranteed to find the problem given by the point $p'$ and the variable instantiation $\nu'$  after a finite amount of steps. Given such problem, we are guaranteed that also the box-gridding procedure will terminate after finitely many iterations. Hence, also our general procedure is guaranteed to terminate. 
\ 

The third point regarding robustly unsatisfiable problems simply follows by the use of box-gridding on the whole system using the bounds of the given system of equations and inequations as its starting box. Indeed, if the system is robustly unsatisfiable, then this  is guaranteed to terminate with a correct result.

To finalize the proof of the theorem, it suffices to consider the procedure that runs  the two previous procedures in parallel.


\end{proof}

Note that for formulas of the given form (bounded system of equations and inequalities), the procedure described in the proof of Theorem~\ref{thm:quasiquasidecidability} can be seen as an instantiation of the certificate search method presented in Section~\ref{sec:method} that uses exhaustive enumeration on the level of points and instantiations, and directly uses the given bounds as the starting box for box gridding. The theorem shows that 
%under the assumption of robustness, 
under robustness assumptions, 
such an instantiation will always terminate successfully for bounded system of equations and inequalities. However, complete enumeration makes this instantiation hopelessly inefficient in practice, and goal oriented methods, as discussed in the first part of the paper, are necessary for practical efficiency. Also, there is no known way of algorithmically deciding whether a given formula satisfies the robustness precondition that ensures termination of the procedure, and hence Theorem~\ref{thm:quasiquasidecidability} is \emph{not} a decidability result.



%%% Local Variables:
%%% mode: latex
%%% TeX-master: "main"
%%% End:



 \section{Related work}

\paragraph{Cloud \vtpm{}s}
Cloud providers offering \cvm{}s typically provide virtual TPM device that
would serve as a root-of-trust and could also be used for remote
attestation.
%
%
Google cloud only offers plain \sev{} \cvms{} and offers measured boot
attestation via a \vtpm{} managed by the
hypervisor~\cite{vtpm:gcp-shielded-vms}.
%
%
Microsoft Azure cloud relies on azure attestation service for attesting
\cvms{}~\cite{vtpm:azure} that generates a token to decrypt the \vtpm{}
state and the disk, hinting that 
%
Microsoft may have their custom firmware based on \svsm{}
specification~(i.e., inside \vmpl{0}) with a persistent \vtpm{} for attesting
\snp{} VMs.
%
Alibaba cloud offers \vtpm{} support on their elastic compute service
VMs~\cite{vtpm:alibaba}.
%
Amazon AWS provides Nitro TPM, a virtual TPM implementation conforming to
the TPM 2.0 specification as part of their EC2
offering~\cite{vtpm:aws-nitro}.
%
Some of these providers use a qemu-backed \vtpm{} that runs on the host,
which requires trusting the cloud provider.
%
Also, there is very limited public knowledge on how these cloud \vtpm{}s
are designed and the security guarantees of it.
%
In contrast, we plan to publish the source code of \svtpm{} implementation
that is built on top of other standard opensource components~(i.e., Qemu,
Linux, and Keylime).
%
As our \svtpm{} rely only on the hardware-protected isolation environment
offered by the AMD-SP hardware, by bringing their own \svsm{} firmware, a
user can completely eliminate the need for trusting the cloud provider.
%
\paragraph{TEE-based \vtpm{}s}
\cocotpm{} proposes a unified architecture for attestation of
\cvms{} where the hypervisor launches a \cvm{} that acts as a \vtpm{}
manager and handles all the \vtpm{} instances~\cite{cocotpm}.
They require TLS for securing the communication channel between a \cvm{}
and its \vtpm{}.
%
Though the \vtpm{} is running under a TEE, a central \vtpm{} manager
suffers from several attacks ranging from denial of service to colluding
with other \cvm{}s,
%
on the other hand launching a dedicated \cocotpm{} for every \cvm{} results
in wastage of architectural resources as the number of
address space identifiers~(ASIDs) are limited.
%

Several projects rely on running \vtpm{} under isolation provided by other
hardware TEE mechanisms such as Intel SGX~\cite{svtpm, eTPM,
vtpm-for-cloud} and ARM Trustzone~\cite{fTPM}.
%
SvTPM aims to protect against NVRAM replacement, and rollback
attacks~\cite{svtpm} by running the \vtpm{} inside an SGX enclave for
KVM-based VMs, whereas
%
eTPM manages several enclave \vtpm{}s in a Xen environment and relies on a
physical TPM to provide root-of-trust~\cite{eTPM}, similar to Berger et
al.~\cite{vtpm:berger}.
%
In contrast, our \svtpm{} architecture equips each \cvm{} with their own
private \vtpm{} instance by leveraging the \svsm{} architecture that
implements VM privilege levels.
%
Also, by implementing an ephemeral \vtpm{}, we completely eliminate the
classes of attacks that come with state protection.

\paragraph{Trusted execution environments}
Arm introduced confidential compute architecture~(CCA) with their Armv9-A
architecture where the processor provides an isolated hardware execution
environment called \emph{Realms}, for hosting entire VMs in a secure
space~\cite{wp:arm-cca}.
%
Similar to other TEEs~\cite{wp:amd-sev, spec:intel-tdx} they offer
pre-attestation of realms and can do measured boot with their hardware
enforced security~(HES) module specification~\cite{spec:arm-cca-sec-model}
which serves as the root-of-trust~\cite{arm-cca:rss, arm-cca:rss-talk}.

Intel, with their trust domain extensions~(TDX) introduced their own
version of hardware-isolated encrypted virtual machines called trusted
domains~(TDs).
%
Intel TDX relies on an SGX-based quoting enclave called the TD-quoting
enclave to perform remote attestation of trusted domains~\cite{spec:intel-tdx}.
%
However, SGX suffered from numerous vulnerabilities in the
past~\cite{sgx-attacks:survey} where researchers were able to extract the SGX
quoting enclave's attestation keys through micro-architectural side-channel
attacks to forge attestation reports~\cite{sgx-attack:foreshadow}.
%
The attestation keys used by these quoting enclave are long-lived, and when
leaked, affect millions of devices.
%
In our design, we do not have any secrets to guard as the attestation keys
are ephemeral.
	
	

\section{Conclusions}
\label{sec:conclusions}
We introduced a form of satisfiability certificate for \smtnta and formulated the satisfiability verification problem as the problem of searching for such a certificate. We showed how to perform this search in a systematic fashion introducing new and efficient search techniques, and provided a theoretical classification providing insight into the possibilities and restrictions of such an approach. Computational experiments document that the resulting method is able to prove satisfiability of a substantially higher number of benchmark problems than existing methods. This suggests future work on the  integration of such search methods into the deduction machinery of current SMT solvers.

While in the unsatisfiable case~\cite{Barbosa:23}, proof certificates are already produced by some current SMT solvers,  and there has already been a standardization effort concerning proof formats and software infrastructure, this is still a topic for future work in the case of satisfiability of \smtnta. Finally, the extension of such techniques to even richer theories (e.g., reasoning not only about real numbers, but about real-valued functions~\cite{Ratschan:23}), is an interesting topic for future research.


%The approach described in this paper provides a unifying framework over such methods that clearly separates search from verification. 

\section*{Acknowledgments}\label{s:ack}
The work of Stefan Ratschan was supported by
the project 21-09458S of the Czech Science Foundation GA ČR
and institutional support RVO:67985807. 


\bibliographystyle{plain}
\bibliography{refs}


%\input{unused_material}
\end{document}



%%% Local Variables:
%%% mode: latex
%%% TeX-master: t
%%% End:


