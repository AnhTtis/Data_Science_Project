\section{Related Work}
\label{sec:relatedwork}

The computation of certificates for formulas \emph{not} being satisfiable in various first-order theories has been an important research topic of the SAT modulo theory community~\cite{Barbosa:23} over recent years. In the case of satisfiable formulas, this topic has---to our knowledge---been restricted to the \smtnta, since for most other theories used in an SMT context, satisfying assignments have a straight-forward representation.


One strategy for proving satisfiability in \smtnta is to prove a stricter requirement that implies satisfiability, but is easier to check. For example, one can prove that \emph{all} elements of a set of variable assignments satisfy the given formula~\cite{iSAT3}, or that a given variable assignment satisfies the formula for \emph{all possible interpretations} of the involved transcendental functions within some bounds~\cite{incrlin}. Such methods may be quite efficient in proving satisfiability of formulas with inequalities only, since those often have full-dimensional solution sets. However, such methods usually fail to prove satisfiability of equalities, except for special cases with straightforward rational solutions.

% square systems of equalities
Computation of formally verified solutions of square systems of equations is a classical topic in the area of interval analysis~\cite{Rump:10,Neumaier:90,Moore:09}. Such methods usually reduce the problem either to fixpoint theorems such as Brouwer's fixpoint theorem or special cases of the topological degree, for example, Miranda's theorem. Such tests are easier to implement, but less powerful than the topological degree (the former fails to verify equalities with double roots, such as $x^3=0$, and the latter requires the solution sets of the individual equalities to roughly lie normal to the axes of the coordinate system).
% \\
% There exist some alternatives to the topological degree test, such as Miranda's theorem and Borsuk's test, that, albeit easier to check than the topological degree, are less powerful~\cite{Moore:09,Neumaier:90}.

In the area of rigorous global optimization, such techniques are applied~\cite{Hansen:92,Kearfott:98} to conjunctions of equalities and inequalities in a similar way as in this paper, but with a slightly different goal: to compute rigorous upper bounds on the global minimum of an optimization problem. This minimum is often attained at the boundary of the solution set of the given inequalities, whereas satisfiability is typically easier to prove far away from this boundary.

There are several fragments of \nta for which real root isolation methods have appeared~\cite{StrzeboskiExpLogArctan, MCCALLUM201216,StrzebonskiTame,RealRootIsolPolyPower,RootIsolMixedTrigPol, ReductionTranscDecProb}.
However, those fragments only allow univariate functions, they are restricted to certain transcendental functions, and only in certain positions. Moreover, some~\cite{StrzeboskiExpLogArctan,StrzebonskiTame, ReductionTranscDecProb} depend on a currently unproved conjecture (Schanuel's conjecture). Examples of such fragments are exp–log–arctan functions~\cite{StrzeboskiExpLogArctan, MCCALLUM201216}, tame elementary functions~\cite{StrzebonskiTame}, poly-powers~\cite{RealRootIsolPolyPower}, mixed trigonometric-polynomials~\cite{RootIsolMixedTrigPol}, and trigonometric extensions~\cite{ReductionTranscDecProb}. Recently, by leveraging such real root isolation algorithms, decision procedures have been shown for the theory of univariate mixed trigonometric-polynomials~\cite{DecidingMixedTrigPol} and for trigonometric extensions~\cite{ReductionTranscDecProb}.
Unlike these techniques,  our method tackles all of \nta, without any syntactical restrictions.

We are only aware of two approaches that extend verification techniques for square systems of equations to proving satisfiability of quantifier-free non-linear arithmetic~\cite{raSAT,ATVApaper}, one~\cite{raSAT} being restricted to the polynomial case, and the other one also being able to handle transcendental function symbols. Neither approach  is formulated in the form of certificate search. However, both could be interpreted as such, and both could be extended to return a certificate. The present paper actually does this for the second approach~\cite{ATVApaper}, and demonstrates that this does not only ease the independent verification of results, but also allows the systematic design of search techniques that result in significant efficiency improvements. 

An alternative approach is to relax the notation of satisfiability, for example using the notion of $\delta$-satisfiability~\cite{dreal,ksmt2}, that does \emph{not} guarantee that the given formula is satisfiable, but only that the formula is not too far away from a satisfiable one, for a suitable formalization of the notion of ``not too far away''. Another strategy is to return candidate solutions in the form of bounds that guarantee that certain efforts to prove unsatisfiability within those bounds fail~\cite{iSAT3}.


%%% Local Variables:
%%% mode: latex
%%% TeX-master: "main"
%%% End:
