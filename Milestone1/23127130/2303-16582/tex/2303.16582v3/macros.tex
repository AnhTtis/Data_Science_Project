% De-comment the following line to hide margin TODOs
%	\def\hideTODO{}

%\newtheorem{theorem}{Theorem}
%\newtheorem{definition}{Definition}
%\newtheorem{property}{Property}
%\newtheorem{lemma}{Lemma}

%\theoremstyle{example2}






\newcommand{\myexample}{\ \\\textit{Example. }}

\newtheorem{example2}{Example}

\newcommand{\nra}    {\ensuremath{\mathcal{NRA}}\xspace}
\newcommand{\nta}{\ensuremath{\mathcal{N\mkern-1mu T\mkern-4mu A}}\xspace}

\newcommand{\smtnra}{\ensuremath{\text{SMT}(\nra)}\xspace}
\newcommand{\smtnta}{\ensuremath{\text{SMT}(\nta)}\xspace}

\newcommand{\ugotNL}{\textsc{ugotNL}\xspace}
\newcommand{\ugotNLeager}{$\ugotNL_{\text{\textsc{eager}}}$\xspace}

\newcommand{\mathsat}{\textsc{MathSAT}\xspace}
\newcommand{\cvctool}{\textsc{cvc5}\xspace}
\newcommand{\drealtool}{\textsc{dReal}\xspace}
\newcommand{\isattool}{\textsc{iSAT3}\xspace}
\newcommand{\msatUgot}{\textsc{MathSAT+ugotNL}\xspace}



\newcommand{\myparagraph}[1]{\ \\\textbf{#1}}
\newcommand{\myparagraphB}[1]{\emph{\normalsize{#1}}}

\newcommand{\Sat}{\texttt{sat}}
\newcommand{\Unknown}{\texttt{unknown}}
\newcommand{\Unsat}{\texttt{unsat}}
\newcommand{\Vars}[1]{\mathit{vars}_{\mathcal{R}}(#1)}
\newcommand{\LtoO}{\ensuremath{\mathcal{L}2\mathcal{O}}}
\newcommand{\defas}{\ensuremath{\stackrel{\text{\tiny def}}{=}}\xspace}

\newcommand{\Rval}[1]{\ensuremath{\mathrm{#1}}\xspace}
\newcommand{\R}{\ensuremath{\mathbb{R}}\xspace}
\newcommand{\Rvec}[1]{\ensuremath{\mathbf{#1}}\xspace}

\newcommand{\openball}[2]{\ensuremath{\mathcal{#1}_{#2}}}
%\newcommand{\openball}[2]{\ensuremath{{#1}_{#2}}}

\newcommand{\kfinite}[1]{\ensuremath{{#1}}-finite point\xspace}
\newcommand{\kfinites}[1]{\ensuremath{{#1}}-finite points\xspace}

\newcommand{\nonRobustUnderInst}{non-robust under instantiations\xspace}

\newcommand{\FRob}{\mathcal{F}_{Rob}\xspace}

\newcommand{\FRobI}{\mathcal{F}_{RobI}\xspace}
\newcommand{\FReg}{\mathcal{F}_{Reg}\xspace}
\newcommand{\FRobLEq}{\mathcal{F}_{RobLEq}\xspace}

\ifx\hideTODO \undefined

\newcommand{\todoInline}[2][green]{%
	\colorbox{#1}{\parbox{\dimexpr\linewidth-2\fboxsep}{\strut #2\strut}}%
}

%\newcommand\todoInline[1]{\colorbox{green}{#1}}


\newcommand\SRtodo[1]{\todo[backgroundcolor=cyan]{#1}}
\newcommand\ELtodo[1]{\todo[backgroundcolor=yellow,]{#1}}

\else
\newcommand\SRtodo[1]{}
\newcommand\ELtodo[1]{}
\fi


\newcommand{\heuristic}[1]{\texttt{(#1)}\xspace}

\newcommand{\filterOverconstr}{\heuristic{filter-overconstr}}
\newcommand{\sortWrtCost}{\heuristic{sort-literals}}
\newcommand{\checkForcedLiterals}{\heuristic{check-forced-literals}}


\newcommand{\filterOverconstrV}{\heuristic{filter-overconstr-V}}
\newcommand{\filterRankDeficient}{\heuristic{filter-rank-deficient}}
\newcommand{\KearfottOrdering}{\heuristic{Kearfott-ordering}}

\newcommand{\epsInflation}{\heuristic{eps-inflation}}
\newcommand{\boxGridding}{\heuristic{box-gridding}}

\newcommand{\orthogonal}{\heuristic{orthogonal}}
\newcommand{\gaussel}{\heuristic{gauss-elim}}

\newcommand{\expid}[2]{\ ({#1}.{#2}.) \ \ }
\newcommand{\citeexpid}[2]{({#1}.{#2}.)}

\newcommand{\Children}[1]{\mathit{ch}(#1)}

\newcommand{\setOfBoxes}{\beta}

\newcommand{\intervalArithmOperator}{\mathcal{I\!A}}

\newcommand*\interior[1]{#1^{\mathsf{o}}}
\newcommand*\closure[1]{\overline{#1}} 

%\newcommand{\pn}{\ensuremath{p_{[1,n]}}}
%\newcommand{\pnk}{\ensuremath{p_{[n+1,n+k]}}}


\newcommand{\Fv}{\ensuremath{F_{|\nu}}}

\newcommand{\pn}{\projn{p}}
\newcommand{\pnk}{\projnk{p}}

\newcommand{\projn}[1]{\ensuremath{{#1}_{[1,n]}}}
\newcommand{\projnk}[1]{\ensuremath{{#1}_{[n+1,n+k]}}}

\newcommand{\mysubsetneq}{\ensuremath{\subsetneqq}}


\newcommand{\Cone}{\ensuremath{C^1}-}

\newcommand{\e}{\mathrm{e}}
\newcommand{\dist}{\mathrm{d}}




%\newcommand{\pnnk}{\ensuremath{p_{|\mathbb{R}^n}}}

%%% Local Variables:
%%% mode: latex
%%% TeX-master: "main"
%%% End:
