
\section{Method}
\label{sec:method}
Our goal is to find a triple $(\sigma, \nu, \setOfBoxes)$ that is a certificate of satisfiability for a given formula $\phi$. So we have a search problem. In order to make this search as efficient as possible, we want to guide the search toward a triple that indeed turns out to be a certificate, and for which the corresponding conditions are computationally easy to check.

Intuitively, we view the search for a certificate as a hierarchy of nested search problems, where the levels of this hierarchy correspond to the individual components of certificates. We formalize this using a search tree whose nodes on the $i$-th level are labeled with $i$-tuples containing the first $i$ elements of the tuple searched for, starting with the root note that is labeled with the empty tuple $()$. The tree will be spanned by a function $ch$ that assigns to 
each node $(c_1,\dots, c_i)$ of the tree a sequence $\langle x_1,\dots,x_n\rangle$ of possible choices   for the next tuple component. Hence the children of $(c_1,\dots, c_i)$ in the tree are $(c_1,\dots, c_i, x_1),\dots,(c_1,\dots, c_i, x_n)$. We will do depth-first search in the resulting tree, searching for a leaf labeled by a certificate of satisfiability for the input formula $\phi$.

 Based on the observation that on each level of the tree one has the first $i$ components of the tuple available for determining a good sequence of choices, we will add additional information as the first tuple component in the form of a variable assignment $p$ that satisfies the formula
 $\phi$ approximately. Hence we search for a $4$-tuple $(p, \sigma, \nu, \setOfBoxes)$.


It is easy to see that it would be possible to generalize such a search tree to a more fine-grained one, where the individual levels are formed by parts of the choices described above, and where the order of those levels can be arbitrary. For example, it would be possible to first choose an interval for a variable (i.e., part of the box $\beta$), then select a literal from a certain clause (i.e., part of the selection function $\sigma$), and so on. However, in this paper, we keep these levels separated, as discussed above, in order to achieve a clear separation of concerns when exploring design choices at the individual levels.




% Here we use a variable assignment $p$ as auxiliary information, defining a certificate ($\sigma,\nu,\beta$) to be a certificate \emph{based on} a variable assignment $p$ iff\footnote{integrate this into the search criteria mentioned below?}
% \begin{itemize}
	% \item for every variable $v$ instantiated by $\nu$, $\nu(v)=p(v)$, and 
	% \item for every variable $v$ not instantiated by $\nu$, $p(v)\in \beta(v)$.
	% \end{itemize}

% Now we search for a certificate ($\sigma,\nu,\beta$) based on a variable assignment~$p$, by searching for the individual components in the order $p, \sigma, \nu, \beta$. 
% Here
% we use the observation that (1) the criteria for those components are not only satisfied by one unique object, and (2) each component is relevant for the next one. Exploiting this, in each level (including the first one) we search for an object that not only satisfies the corresponding part of its definition, but also optimizes the chances for the further levels to succeed. 

% This hierarchy of search problems allows us to use various form of guidance for search. In general, such guidance can provide several levels of reliability: 
% \begin{itemize}
	% \item Guidance that only estimates the chance of the given choice. Based on this one can then achieve various levels of sophistication:
	%   \begin{itemize}
		%   \item only try choices for which the estimate is better than some threshold, 
		%   \item generate some/all choices and try them in order of their estimate,
		%   \item  explicitly compute a good/optimal choice, then next best choice etc.
		%   \end{itemize}
	% \item Guidance that provides reliable information on whether a choice may fail. Again, one can use this to either filter out choices that certainly do not result in success, or to explicitly compute only choices that the given guidance does not evaluate negatively.
	% \end{itemize}

% In any case, the guidance provided by some lower level in the hierarchy of search problems may also be used in the higher levels.

%%% Local Variables:
%%% mode: latex
%%% TeX-master: "./main.tex"
%%% End:
