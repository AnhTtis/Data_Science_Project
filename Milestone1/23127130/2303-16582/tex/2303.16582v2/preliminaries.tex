
\section{Preliminaries}
\label{sec:preliminaries}
%\paragraph{Satisfiability Modulo Theories.}
We work in the context of \emph{Satisfiability Modulo Theories} (SMT). Our theory of interest is the quantifier-free theory of non-linear real arithmetic augmented with trigonometric and exponential transcendental functions, \smtnta. We assume that the reader is familiar with  standard SMT terminology~\cite{SMTBackground}.
% notions of interpretation, model, satisfiability and boolean abstraction. 

\emph{Notation.} We denote \smtnta-formulas by $\phi, \psi$, clauses by $C_1, C_2$, literals by $l_1, l_2$, real-valued variables by $x_1, x_2,\dots$,
 constants by $a, b$, intervals of real values by $I = [a,b]$, boxes by  $B=I_1 \times \cdots \times I_n$, 
 logical terms with addition, multiplication and transcendental function symbols by $f, g$, and multivariate real functions with $F, G, H$. For any formula $\phi$, we denote by $\Vars{\phi}$ the set of its real-valued variables. When there is no risk of ambiguity we write $f, g$ to also denote  the real-valued functions corresponding to the standard interpretation of the respective terms. We assume that formulas are in Conjunctive Normal Form (CNF) and that their atoms are in the form $f \bowtie 0$, with $\bowtie\;\in \{=, \leq, <\}$. We remove the negation symbol by rewriting every occurrence of $\neg (f = 0)$ as $(f<0 \lor 0< f)$ and distributing $\neg$ over inequalities.


\myparagraph{Points and boxes.} Since we have an order on the real-valued variables $x_1,x_2,\dots$, for any set of variables $V\subseteq \{x_1,x_2,\dots\}$ we can view an assignment $p: V\rightarrow \mathbb{R}$ equivalently as the $|V|$-dimensional point $p\in \mathbb{R}^{|V|}$, and an \textit{interval assignment} $B : V \to \{[a,b] : a,b\in \R\}$ equivalently as the $|V|$-dimensional box $B \subseteq \R^{|V|}$. By abuse of notation, we will use both representations interchangeably, using the type $\mathcal{R}^{V}$ both for assignments in $V\rightarrow \mathbb{R}$ and points in $\mathbb{R}^{|V|}$, and the type $\mathcal{B}^{V}$ both for interval assignments in $V\rightarrow \{[a,b] : a,b\in \R\}$ and corresponding boxes. This
will allow us to apply mathematical notions usually defined on points or boxes to such assignments, as well. Given a point  $p\in \mathcal{R}^{V}$, and a subset $V'\subseteq V$, we denote by $\mathit{proj}_{V'}(p)\in \mathcal{R}^{V'}$ the projection of $p$ to the variables in $V'$, that is, for all $v\in V'$, $\mathit{proj}_{V'}(p)(v):= p(v)$. %Analogically, we also define the projection of boxes. 

\myparagraph{Systems of equations and inequalities.} We say that a formula $\phi$ %with no boolean variables
that contains only conjunctions of atoms in the form $f=0$ and $g\leq 0$  is a \emph{system of equations and inequalities}. If $\phi$ contains only equations (inequalities) then we say it is a \emph{system of equations} (\emph{inequalities}). A system of equations ${f_1 = 0 \land \cdots \land f_n=0}$, where the $f_1,\cdots,f_n$ are terms in the variables $x_1, \cdots, x_m$, can be seen in an equivalent way as the equation $F=0$, where $F$ is the real-valued function ${F := f_1 \times \cdots \times f_n : \R^m \to \R^n}$ and $0$ is a compact way to denote the point $(0,\cdots,0)\in \R^n$. 
Analogously, we can see a system of inequalities ${g_1 \leq 0 \land \cdots \land g_k \leq 0}$ as the inequality $G\leq0$,  where $G$ is the real-valued function ${G := g_1 \times \cdots \times g_k : \R^m \to \R^k}$ and $\leq$ is defined element-wise. We will write $eq(\phi)$ for the function $F$ defined by the equations in the formula $\phi$ and ${ineq(\phi)}$ for the function $G$ defined by the inequalities in~$\phi$. The handling of strict inequalities would be an easy, but technical extension of our method, which we avoid to stream-line the presentation.



\myparagraph{Dulmage-Mendelsohn decomposition.} Given a system of equations~$\phi$, it is possible to construct an associated bipartite graph $\mathcal{G}_\phi$ that
represent important structural properties of the system of equations. This graph
has one vertex per equation, one vertex
per variable, and an edge between a variable $x_i$ and an equation $f_j=0$ iff $x_i$ appears in $f$. The Dulmage–Mendelsohn decomposition~\cite{dulmage_mendelsohn_1958,dulmage_mendelsohn_system_of_equations} is a canonical decomposition from the field of matching theory that partitions the system into three parts: an over-constrained subsystem (more equalities than variables), an under-constrained subsystem (less equalities than variables), and a well-constrained subsystem (as many equalities as variables, and contains no over-constrained subsystem, i.e. it satisfies the Hall property~\cite{Hall:35}). 
\begin{example}
	Let $\phi:= x-tan(y)=0 \land z^2=0 \land w=0 \land sin(w)=0$. Through the DM-decomposition we obtain an under-constrained sub-system $x-tan(y)=0$ (two variables, one equation), a well-constrained sub-system $z^2=0$ (one variable, one equation), and an over-constrained sub-system $w=0 \land sin(w)=0$ (one variable, two equations).
	\end{example}
	


\ \\

\myparagraph{Topological degree.} The notion of the degree of a continuous function (also called the topological degree) comes from differential topology~\cite{Fonseca:95}. 
For a continuous function $F: B\subseteq\mathbb{R}^n\rightarrow \mathbb{R}^n$, such that $0\not\in F(\partial B)$ (where $\partial B$ is the topological boundary of $B$), the degree $deg(F, B, 0)$
is a computable~\cite{Aberth:94,Franek:12b} integer. This integer provides information about the roots of $F$ in $B$, and can be seen as a generalization of the intermediate value theorem to higher-dimensional functions. In analogy to the fact that opposite signs of a continuous function on the endpoints of an interval imply the existence of a zero within the interval, 
$deg(F, B, 0)\neq 0$ implies that $F$ has a root in $B$. The converse is not true, and the existence of a root does not imply nonzero degree in general. Still, 
if a box contains one isolated zero with non-singular Jacobian matrix, then the topological degree is non-zero~\cite{Fonseca:95}. For alternatives to the topological degree test see our discussion of related work.



\myparagraph{Interval Arithmetic.}
The basic algorithmic tool that underlies our approach is floating point interval arithmetic ($\intervalArithmOperator$
~\cite{Rump:10,Neumaier:90,Moore:09} 
which, given a box~$B$ and an $\nta$-term representing a  function $H$, is able to compute an interval $\intervalArithmOperator_H(B)$ that over-approximates the range $\{ H(x) \mid x\in B\}$ of $H$ over $B$. Since this is based on floating point arithmetic, the time needed for computing $\intervalArithmOperator_H(B)$ does not grow with the size of the involved numbers. Moreover conservative rounding guarantees correctness under the presence of round-off errors.
In the paper, we will use interval arithmetic within topological degree computation~\cite{Franek:12b}, and as a tool to prove the validity of inequalities on boxes.



\myparagraph{Robustness.} We say that a formula $\phi$ is robust if there exists some $\epsilon>0$ such that $\phi$ is satisfiable iff every $\epsilon$-perturbation of $\phi$ is satisfiable %\ELtodo{maybe add an informal description of $\epsilon$-perturbation?}
(%intuitively, we can think of an $\epsilon$-perturbation of $\phi$ as a formula obtained from $\phi$ by substituting a term $t$ of $\phi$ with $t'$ such that, given a certain metric, the distance between $t$ and $t'$ is less than $\epsilon$; 
for the precise definition of $\epsilon$-perturbation see \cite{Franek:12}). If $\phi$ is both robust and (un)satisfiable, we say that it is robustly (un)sat.
\\
\indent \emph{Relation between robustness and system of equations}: An over-constrained system of equations is never robustly sat~\cite[Lemma 5]{Franek:12}. It easily follows that a system of equations that contains an over-constrained sub-system (in the sense of the Dulmage-Mendelsohn decomposition) is never robustly sat as well.
\\
\indent \emph{Relation between robustness and topological degree}: Even in the case of an isolated zero, the test for non-zero topological degree can fail if the system is non-robust. For example, the function $F(x) \equiv x^2$ has topological degree $0$ in the interval $[-1,1]$%\ELtodo{add a note that the Jacobian in the solution, $0$, is rank-deficient?}
, although the equality $x^2=0$ has an isolated zero in this interval. However, the zero of $x^2=0$ is not robust: it can vanish under arbitrarily small changes of the function denoted by the left-hand side $x^2$. It can be shown that the topological degree test is able to prove satisfiability in all robust cases for a natural formalization of the notion of robustness~\cite{Franek:12}. We will not provide such a formalization, here, but use robustness as an intuitive measure for the potential success when searching for a certificate. 


\ \\

\myparagraph{Logic-To-Optimization.}  
While symbolic methods usually struggle dealing with \nta, numerical methods, albeit inexact, can handle transcendental functions efficiently. For this reason, an SMT solver can benefit from leveraging numerical techniques. In the Logic-To-Optimization approach~\cite{ATVApaper,xsat}, 
an \smtnta-formula $\phi$ in $m$ variables is translated into a real-valued non-negative function $\LtoO(\phi) \equiv H: \R^{m} \mapsto \R^{\ge 0}$ 
such that---up to a simple translation between Boolean and real values for Boolean variables---each model of $\phi$ is a zero of $H$ (but not vice-versa). When solving a satisfiability problem, one can try to first numerically minimize this function, and then use the obtained numerical (approximate) solution to prove, through exact symbolic methods, that the logical formula has indeed a model. For the precise definition of the operator $\LtoO$ see~\cite[Section 3]{ATVApaper}.





%%% Local Variables:
%%% mode: latex
%%% TeX-master: "./main.tex"
%%% End:

