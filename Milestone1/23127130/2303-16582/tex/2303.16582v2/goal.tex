
\section{Goal}
\label{sec:goal}
Consider an SMT solver that takes as input some formula $\phi$ and as output an element of $\{ \Sat,\Unknown, \Unsat\}$.  How can we gain trust in the correctness of the result of such an SMT solver? One approach would be to ensure that the algorithm itself is correct. Another option is to provide a second algorithm whose output we compare with the original one. Both approaches are, however, very costly, and moreover, the latter approach still may be quite unreliable.

Instead, roughly following McConnell et. al.~\cite{McConnell:11} (see also Figure~\ref{fig:certifying}), we require our solver to return---in addition to its result---some information that makes an independent check of this result easy:

\begin{definition}
	An SMT solver is \emph{certifying} iff for an input formula $\phi$, in addition to an element $r\in \{ \Sat, \Unknown, \Unsat \}$, it returns an object~$w$ (a \emph{certificate}) such that
	\begin{itemize}
		\item $(\phi, r, w)$ satisfies a property $W$ where $W(\phi, r, w)$ implies that $r$ is a correct result for $\phi$, and 
		\item   there is an algorithm (a \emph{certificate checker}) that
		\begin{itemize}
			\item     takes as input a triple $(\phi, r, w)$ and returns $\top$ iff $W(\phi, r, w)$, and that
			\item is simpler than the SMT solver itself.
		\end{itemize}
		
	\end{itemize}
	
      \end{definition}

\begin{figure}[tbh]
\centering
\includegraphics[clip,trim={0cm 3cm 0cm 0cm},width=8cm]{certificate.pdf}
	\caption{Certifying SMT Solver}
	\label{fig:certifying}
\end{figure}




So, for a given formula $\phi$,  one can ensure correctness of the result $(r, w)$ of a certifying SMT solver by using a certificate checker to check the property $W(\phi, r, w)$. Since the certificate checker is simpler than the SMT solver itself, the correctness check is simpler than the computation of the result itself. 

The definition leaves it open, what precisely is meant by  ``simpler''. In general, it could either refer to the run-time of the checker, or to the effort needed for implementing the certificate checker and ensuring its correctness. The former approach is taken in computational complexity theory, the latter in contexts where correctness is the main concern~\cite{McConnell:11}. Indeed, we will later see that our approach succeeds in satisfying both requirements, although we will not use complexity-theoretic measures of run-time, but will measure run-time experimentally. 

The use of such certificates is ongoing research in the unsatisfiable case~\cite{Barbosa:22}. In the satisfiable case, for most theories, one can simply use satisfying assignments (i.e., witnesses) as certificates. Here the property $W$ simply is the property that the given assignment satisfies the formula, which can be checked easily.

For \smtnta, however, the situation is different: Here, no general finite representation of  satisfying assignments is available. Hence one needs to use certificates of a different form. 
We introduce the following definition:
\begin{definition}
Let $\phi$ be a formula in \nta. A \emph{(satisfiability) certificate} for $\phi$ is a triple $(\sigma, \nu, \setOfBoxes)$ such that $W(\phi, \Sat, (\sigma, \nu, \setOfBoxes))$ iff
	\begin{itemize}
	\item $\sigma$ is a function selecting a literal from every clause of $\phi$% such that no two selected Boolean literals contradict each other
		\item $\nu$ is a variable assignment in $\mathcal{R}^{V}$ assigning floating point numbers to a subset $V\subseteq\Vars{\sigma(\phi)}$ (where $\sigma(\phi)$ is a compact way of writing $\bigwedge_{ C \in \phi} \sigma(C)$), s.t. $\sigma(\phi)$
		 contains as many equations as real-valued variables not in $V$.
		\item $\setOfBoxes$ is a finite set of interval assignments in $\mathcal{B}^{\Vars{\phi}\setminus V}$% that assign floating point intervals
		such that their set-theoretic union as boxes
		is again a box $B_\beta$ and, 
		for the system of equations  $F:= eq(\nu(\sigma(\phi)))$ 
		and the system of inequalities $G:= ineq(\nu(\sigma(\phi)))$, it holds that:
	
		\begin{itemize}
			%\item the equalities form a square system, that is $F: \mathbb{R}^k\rightarrow\mathbb{R}^k$, for some $k$,
			\item $0\not\in F(\partial B_\beta)$,
			\item $deg(F, B_\beta, 0)\neq 0$, and
			\item for every $B\in\setOfBoxes$, $\intervalArithmOperator_G(B)\leq 0$.
		\end{itemize}
	\end{itemize}

\end{definition}

\begin{example}
Consider the formula 
\begin{alignat*}{2}
	& \qquad \qquad \qquad \qquad \qquad \phi := C_1 \land C_2 \land C_3 \land C_4  \\
	& C_1 \ \equiv \ cos(y) = 0 \ \lor \ sin(y) = e^x
	&& C_3 \ \equiv \ x-y \leq cos(z) \\ 
	& C_2 \ \equiv \  sin(y)=0 \ \lor \  cos(y) = sin(8x^2-z) 
	&& C_4 \ \equiv \ x+y \geq sin(z) 
      \end{alignat*}      
The following $(\sigma, \nu, \beta)$ is a certificate:
\begin{itemize}
	\item $\sigma := \{ C_1 \mapsto sin(y) = e^x\ ;\  C_2 \mapsto cos(y)=sin(8x^2-z)\ ; \\ C_3 \mapsto C_3\ ;\  C_4 \mapsto C_4\}$ 
	\item $\nu := \{ z \mapsto 0.2 \}$
	\item $\setOfBoxes:=\{B\}$, where $B := \{x \mapsto [-0.1,0.05]\ ;\ y \mapsto [1.4, 1.9] \} $
        \end{itemize}
      \end{example}

      As can be seen in Figure~\ref{fig:example}, the solution sets of $C_1$ and $C_2$ cross at a unique point in $B$, which reflects the fact that the degree of 
      the function $(x,y)\rightarrow (sin(y)-e^x,cos(y)-sin(8x^2-0.2))$ is non-zero. Moreover, the inequalities $C_3$ and $C_4$ hold on all elements of the box.
      \begin{figure}[tb]
\centering
        \includegraphics[width=8cm]{example.pdf}       
        \caption{Solution Sets of Equalities of Example Certificate}
        \label{fig:example}
      \end{figure}
\ 

Due to the properties of the topological degree and of interval arithmetic discussed in the preliminaries, we have:
\begin{property}
	$W(\phi, \Sat, (\sigma, \nu, \setOfBoxes))$ implies that $\phi$ is satisfiable.
\end{property}

Moreover, the topological degree can be computed algorithmically~\cite{Aberth:94,Franek:12b}, and one can easily write a certificate checker based on such an algorithm. Hence such a triple can be used as a certificate for satisfiability. 

In this paper, we will show that in addition to the discussed benefits for correctness, formulating satisfiability checking as the problem of search for such certificates also is beneficial for efficiency of the SMT solver itself. Since we will concentrate on satisfiability, we will simply ignore the case when an SMT solver returns $\Unsat$, so the reader can simply assume that an SMT solver such as the one from Figure~\ref{fig:certifying} only returns an element from the set $\{\Sat, \Unknown\}$.


% For some applications it might make sense to return more information than just ``satisfiable'' or ``unknown'' (e.g., an algorithm-dependent estimate for the probability for future success, or algorithm-independent information about the distance to satisfiability).

%%% Local Variables:
%%% mode: latex
%%% TeX-master: "./main.tex"
%%% End:
