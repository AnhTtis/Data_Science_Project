

\section{Introduction}
SAT modulo theories (SMT) is the problem of checking whether a given quantifier-free first-order formula with both propositional and theory variables is satisfiable in a specific first-order theory. In this paper, we consider the case of \smtnta, non-linear real arithmetic  augmented with trigonometric and exponential transcendental functions. 
 This problem is particularly important in the  verification of hybrid systems and in theorem proving.  Unfortunately, \nta is a very challenging theory. 
 Indeed, it is undecidable \cite{richardson68}, and, moreover, there is no known finite representation of satisfying assignments that could act as a direct certificate of satisfiability. This does not only make it difficult for an SMT-solver to prove satisfiability, but also raises the question of how to verify the result given by an SMT-solver. %(whereas, for linear real arithmetic and non-linear real arithmetic without transcendental functions, a satisfying assignment, having a finite representation, can directly serve as a certificate).
 
 In this paper, we introduce the notion of a satisfiability certificate for \nta. Such a certificate allows independent entities to verify the satisfiability of a given input formula without having to re-do a full check of its satisfiability. More specifically, based on such a certificate, the check of satisfiability is both easier in terms of computational effort and effort needed to implement the checker and to ensure its correctness. The certificate will be based on the notion of topological degree~\cite{Fonseca:95,Aberth:94,Franek:12b}, generalizing the idea that a sign change of a continuous function $f$ implies satisfiability of $f=0$. The basic tool for checking correctness of the certificate is interval arithmetic~\cite{Rump:10,Neumaier:90,Moore:09}.

 The idea to verify satisfiability of \smtnta in such a way, is not new~\cite{ATVApaper}. However, the formulation as the problem of searching for a certificate is. In addition to the possibility of independent verification, such a formulation makes the corresponding search problem explicit. This allows us to introduce new, efficient search heuristics that guide the algorithm toward finding a certificate and prevent the procedure from getting stuck in computation that later turns out to not to lead to success.

We have implemented our method in the tool \ugotNL~\cite{ATVApaper} and present computational experiments  with different heuristics configurations over a wide variety of \nta benchmarks. The experimental results show that this new version of \ugotNL outperforms the previous version, making it---to the best of our knowledge---the most effective solver for proving satisfiability of \nta problems.

It is possible to integrate the resulting method into a conflict-driven clause learning (CDCL) type SMT solver~\cite{ATVApaper}. However, in order to keep the focus of the paper on the concern of certificate search, we ignore this possibility, here.


\myparagraph{Content.} The paper is organized as follows: In Section~\ref{sec:preliminaries} we provide the necessary background. In Section~\ref{sec:goal} we give the formal definitions of \emph{certifying SMT solver} and of \emph{satisfiability certificate} in \smtnta. In Section~\ref{sec:method} we outline our method for searching for a certificate, and in Section~\ref{sec:certificate-search} we illustrate   the heuristics that we introduce in detail. In Section~\ref{sec:experiments} we experimentally evaluate our method. In Section~\ref{sec:relatedwork} we discuss related work. Finally, in Section~\ref{sec:conclusions}, we draw some conclusions.

%%% Local Variables:
%%% mode: latex
%%% TeX-master: "./main.tex"
%%% End: