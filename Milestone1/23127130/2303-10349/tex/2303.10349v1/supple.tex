% Template for ICME 2022 paper; to be used with:
%          spconf.sty  - ICASSP/ICIP/ICME LaTeX style file, and
%          IEEEbib.bst - IEEE bibliography style file.
% --------------------------------------------------------------------------

% Title.
% ------
% \onecolumn
% \section{Supplementary Material}
\newpage
\noindent \Large \textbf{Appendix}
\normalsize
\renewcommand\thesection{\Roman{section}}
\setcounter{section}{0}
\section{Further Demonstration of Cholesky Decomposition}
\label{sec:sec1}

In the paper, the Pyramid Covariance Predictor predicts the Cholesky decomposition of the covariance matrix rather than the covariance itself. The reason we do not directly predict the covariance is that there is a relationship among elements of the covariance matrix. Considering that the covariance matrix is a positive semi-definite matrix, its determinant is non-negative:

\begin{align}
 \label{eq:DET}
 det\left(\boldsymbol{\Sigma_i}\right) = cov\left(\boldsymbol{x}, \boldsymbol{x}\right)cov\left(\boldsymbol{y}, \boldsymbol{y}\right) - cov\left(\boldsymbol{x}, \boldsymbol{y}\right)^2 \geq 0
\end{align}

Therefore, if we directly predict the elements of the matrix by regression, the outputs may be illegal since the outputs of the fully connected layer may not follow Eq.~\ref{eq:DET}. However, the elements of its Cholesky decomposition are independent of each other. So we choose to predict these values by regression.

Also, when using Cholesky decomposition, we hypothesize that the target covariance matrix is a positive definite matrix. This assumption is reasonable since in Eq.~\ref{eq:DET}, $ det\left(\boldsymbol{\Sigma_i}\right) = 0 $ if and only if the correlation coefficient of $ \boldsymbol{X} $ and $ \boldsymbol{Y} $ equals 1 or -1, which is nearly impossible in the reality. Regarding the covariance matrix as a positive definite matrix can simplify the problem.

It has been proven that the Cholesky decomposition of a positive definite matrix always exists. Let $\boldsymbol{A} \in \mathbb{R}^{n \times n}$ be positive definite. Obviously, for $n = 1$, you can take the square root as the Cholesky decomposition. Assuming that its Cholesky decomposition exists for $\boldsymbol{A} \in \mathbb{R}^{(n-1) \times (n-1)}$, then for $\boldsymbol{A} \in \mathbb{R}^{n \times n}$, it can be partitioned as:

\begin{align}
 % \label{eq:part}
 \boldsymbol{A} = \begin{bmatrix} \boldsymbol{\hat{A}} & \alpha \\ \alpha^T & \beta \end{bmatrix}
\end{align}
where $ \boldsymbol{\hat{A}} \in \mathbb{R}^{(n-1) \times (n-1)} $.

Obviously $ \boldsymbol{\hat{A}} $ is positive definite, so it has a Cholesky decomposition $ \boldsymbol{\hat{A}} = \boldsymbol{\hat{L}}\boldsymbol{\hat{L}^T} $. Let:

\begin{gather}
 \label{eq:proof}
 \boldsymbol{L_1} = \begin{bmatrix} \boldsymbol{\hat{L}} & 0 \\ 0 & 1 \end{bmatrix} \\
 \boldsymbol{L_2} = \begin{bmatrix} \boldsymbol{I} & 0 \\ \gamma^T & 1 \end{bmatrix} \\
 \boldsymbol{L_3} = \begin{bmatrix} \boldsymbol{I} & 0 \\ 0 & \lambda \end{bmatrix} \\
 \boldsymbol{L} = \boldsymbol{L_1}\boldsymbol{L_2}\boldsymbol{L_3} = \begin{bmatrix} \boldsymbol{\hat{L}} & 0 \\ \gamma^T & \lambda \end{bmatrix}
\end{gather}
where $ \gamma = \boldsymbol{\hat{L}^{-1}}\alpha $, $ \lambda^2 = \beta - \alpha^T\boldsymbol{A}^{-1}\alpha $. Then we have:

\begin{align}
 % \label{eq:part}
 \boldsymbol{A} = \boldsymbol{L}\boldsymbol{L^T}
\end{align}
where $ \boldsymbol{L} $ is the Cholesky decomposition of $ \boldsymbol{A} $.

Note that the Cholesky decomposition is not always unique since the sign of $ \lambda $ is not fixed. However, with all-positive elements, the Cholesky decomposition is unique. In our method, we force all elements of the predicted Cholesky decomposition to be positive for convenience by:

\begin{align}
 f(x) = \begin{cases}
 x+1, & if \quad x \geq 0 \\
 e^x, & otherwise
 \end{cases}
\end{align}

\setlength{\tabcolsep}{3pt}
\begin{table}[t]
\begin{center}
\caption{Results of comparison experiments on Digital Hand Atlas Dataset}
\label{table1}
\begin{tabular}{@{}ccccc@{}}
\toprule
\multirow{2}{*}{Model} & \multirow{2}{*}{MRE} & \multicolumn{3}{c}{SDR} \\ & & 2mm & 4mm & 10mm \\ \cmidrule(l){1-5}
Lindner et al\cite{lindner2014robust} & 0.85 & 93.67 & 98.95 & 99.94 \\
Štern et al\cite{vstern2016local} & 0.80 & 92.20 & 98.45 & 99.95 \\
Urschler et al\cite{urschler2018integrating} & 0.80 & 92.19 & 98.46 & 99.95 \\
Payer et al\cite{payer2019integrating} & 0.66 & 94.99 & 99.27 & \textbf{99.99} \\
Kang et al\cite{kang2021accurate} & \textbf{0.64} & 96.04 & \textbf{99.66} & 99.98 \\
Ours & \textbf{0.64} & \textbf{96.95} & 99.52 & \textbf{99.99} \\ \bottomrule
\end{tabular}
\end{center}
\end{table}
\setlength{\tabcolsep}{1.4pt}

\setlength{\tabcolsep}{3pt}
\begin{table*}[t]
\begin{center}
\caption{Comparison with or without our covariance predictor on IEEE ISBI Challenge 2015 Dataset.}
\label{table2}
\setlength{\tabcolsep}{2mm}{
\begin{tabular}{@{}ccccccccccc@{}}
\toprule
\multirow{3}{*}{Model} & \multicolumn{5}{c}{Validation Set} & \multicolumn{5}{c}{Test Set}  \\ \cmidrule(l){2-6} \cmidrule(l){7-11} & \multirow{2}{*}{MRE} & \multicolumn{4}{c}{SDR} & \multirow{2}{*}{MRE} & \multicolumn{4}{c}{SDR} \\ & & 2mm & 2.5mm & 3mm & 4mm & & 2mm & 2.5mm & 3mm & 4mm \\ \cmidrule(l){1-11}
Chen et al\cite{chen2019cephalometric} & 1.17 & \textbf{86.67} & \textbf{92.67} & \textbf{95.54} & \textbf{98.53} & 1.48 & 75.05 & 82.84 & 88.53 & 95.05 \\
Chen et al + covariance predictor & \textbf{1.15} & 86.60 & 92.21 & 95.50 & 98.31 & \textbf{1.39} & \textbf{76.00} & \textbf{83.32} & \textbf{89.74} & \textbf{96.32} \\ \bottomrule
\end{tabular}}
\end{center}
\end{table*}
\setlength{\tabcolsep}{1.4pt}

\section{More Results of Comparison Experiments}
\label{sec:sec2}
We conduct more comparison experiments on Digital Hand Atlas Dataset \cite{gertych2007bone}. Digital Hand Atlas Dataset consists of 895 X-ray images
of left hands with an average resolution of $ 1563 \times 2169 $ and 37 landmarks annotated in each image. Following \cite{payer2019integrating,kang2021accurate}, we use the same three-fold cross-validation, where images are divided into approximately 600 training and 300 testing images per fold. We evaluate the performance using MRE and SDR in 2mm, 4mm, and 10mm. The result samples are shown in Fig.~\ref{fig0}.

We select five representative landmark detection methods for comparison. The experimental results shown in Table \ref{table1} demonstrate that our method achieves better performance on MRE and outperforms existing SOTA methods on SDR in 2mm and 10mm radii. For localization evaluated by SDR in 4mm radius, our method also has a near SOTA performance. The extended experiment shows the generalization ability of our method. Due to the effectiveness of uncertainty estimation, our method can not only be used in cephalometric landmark detection, but can also be extended to other medical landmark detection tasks.

We also conduct additional experiments by combining the proposed Pyramid Covariance Predictor with the method of Chen et al \cite{chen2019cephalometric}, and the results are shown in Table \ref{table2}. Benefiting from the proposed Pyramid Covariance Predictor, the method \cite{chen2019cephalometric} achieves comparable performance on validation sets, and outperforms the baseline model on test sets by a clear margin. The results verify the effectiveness of the proposed Pyramid Covariance Predictor.

\section{Further Analysis on Uncertainty Estimation}
\label{sec:sec3}

In our paper, we figure out that the Gaussian distributions of the landmarks should be anisotropic and unlikely to overlap. The landmark predictions of our method reflect that. For reference, we show the ordering of the cephalometric landmarks in Fig.~\ref{fig1}.

We measure the uncertainty of the predicted landmarks. Firstly, for every landmark, we compute all displacement vectors between predicted locations and the ground truth locations of all images in the validation set and test set. Then we show them in one image. For landmarks on that image, we obtain the coordinates of displacement landmarks by adding the ground truth locations and every corresponding displacement vector separately. We show the example image in Fig.~\ref{fig2}, which demonstrates the uncertainty of every landmark. Consistent with what we think, the distributions of displacement landmarks are anisotropic, mainly along the edges. Also, different landmarks have different distribution ranges, which shows that the difficulties in predicting different landmarks are various. 

Note that there are few overlaps among the displacement landmarks. In most of the current heatmap-based methods, since they use a fixed heatmap for every landmark, the distribution ranges may overlap. For example, in our paper, we illustrate the predetermined Gaussian distributions generated by Chen et al \cite{chen2019cephalometric}, which is also utilized in many other heatmap-based methods. There are many overlaps in that image, for example, distributions of landmark 11 and landmark 12. However, the displacement landmarks of landmark 11 and 12 in Fig.~\ref{fig2} show that 
there are no overlaps between them, which indicates the disadvantages of a fixed heatmap.

However, the distributions we predict are not exactly consistent with the distributions of displacement landmarks. There are still some deviations between the directions of our predicted distributions and those of the displacement landmarks. For example, for landmark 2, the distribution of the displacement landmarks is along the nasal root, but our predicted distribution deflects from the nasal root. It indicates that our method can be further optimized and thus achieve better performance.

\begin{figure*}[!t]
\centering
\includegraphics[height=7cm]{subfig0.png}
\caption{The result samples of our method on Digital Hand Atlas Dataset.}
\label{fig0}
\end{figure*}

\begin{figure}[!t]
\centering
\includegraphics[height=5.5cm]{supfig1.png}
\caption{The ordering of the cephalometric landmarks. The left image is the original image of cephalometric tracing \cite{wang2016benchmark}, and the right image is the index label on our illustrated image.}
\label{fig1}
\end{figure}

\begin{figure}[!t]
\centering
\includegraphics[height=8cm]{subfig2.png}
\caption{Sample image of all the displacement landmarks.}
\label{fig2}
\end{figure}

% References should be produced using the bibtex program from suitable
% BiBTeX files (here: strings, refs, manuals). The IEEEbib.bst bibliography
% style file from IEEE produces unsorted bibliography list.
% -------------------------------------------------------------------------

