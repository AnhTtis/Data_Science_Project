% Template for ICME 2022 paper; to be used with:
%          spconf.sty  - ICASSP/ICIP/ICME LaTeX style file, and
%          IEEEbib.bst - IEEE bibliography style file.
% --------------------------------------------------------------------------
\documentclass{article}
\usepackage{spconf,amsmath,epsfig}
\usepackage{graphicx}
\usepackage{amsmath}
\usepackage{amssymb}
\usepackage{amsfonts}
\usepackage{booktabs}

\usepackage{framed,multirow}
\usepackage{latexsym}
\usepackage{verbatim}
\usepackage{color}
\usepackage{subfigure}

\let\OLDthebibliography\thebibliography
\renewcommand\thebibliography[1]{
  \OLDthebibliography{#1}
  \setlength{\parskip}{0pt}
  \setlength{\itemsep}{0pt plus 0.3ex}
}

\pagestyle{empty}


\begin{document}\sloppy

% Example definitions.
% --------------------
\def\x{{\mathbf x}}
\def\L{{\cal L}}


% Title.
% ------
\title{Uncertainty-aware U-Net for Medical Landmark Detection}
%
% Single address.
% ---------------
%\name{Anonymous ICME submission}
\name{Ziyang Ye, Haiyang Yu, and Bin Li}
%Address and e-mail should NOT be added in the submission paper. They should be present only in the camera ready paper. 
\address{\{yezy20, hyyu20, libin\}@fudan.edu.cn\\
Shanghai Key Laboratory of Intelligent Information Processing\\
School of Computer Science, Fudan University}


\maketitle


%
\begin{abstract}
Heatmap-based methods play an important role in anatomical landmark detection. However, most current heatmap-based methods assume that the distributions of all landmarks are the same and the distribution of each landmark is isotropic, which may not be in line with reality. For example, the landmark on the jaw is more likely to be located along the edge and less likely to be located inside or outside the jaw. Manually annotating tends to follow similar rules, resulting in an anisotropic distribution for annotated landmarks, which represents the uncertainty in the annotation. To estimate the uncertainty, we propose a module named Pyramid Covariance Predictor to predict the covariance matrices of the target Gaussian distributions, which determine the distributions of landmarks and represent the uncertainty of landmark annotation. Specifically, the Pyramid Covariance Predictor utilizes the pyramid features extracted by the encoder of the backbone U-Net and predicts the Cholesky decomposition of the covariance matrix of the landmark location distribution. Experimental results show that the proposed Pyramid Covariance Predictor can accurately predict the distributions and improve the performance of anatomical landmark detection.
\end{abstract}

%
\begin{keywords}
uncertainty estimation, anatomical landmark detection, heatmap regression
\end{keywords}
%

\section{Introduction}
\label{sec:Introduction}

Anatomical landmark detection plays an important role in medical image analysis. Due to the advantage of likelihood modeling, researchers often utilize a heatmap to obtain the location of landmarks rather than directly regressing their coordinates \cite{kumar2020luvli}. The heatmap is a probability distribution of landmark location, and as a common practice, researchers usually use a Gaussian likelihood to estimate the distribution. 

% 再解释一下

However, most current heatmap-based methods need predetermined heatmaps with fixed Gaussian kernels as prior knowledge, and this potentially introduces two assumptions into their approaches: 1) The distributions of all landmarks are the same. 2) The distribution of each landmark is isotropic. These hypotheses may not be consistent with reality. Firstly, since there may be differences in the difficulty of locating various landmarks, the distributions of landmarks may be different from each other. For landmarks that are easy to locate, such as those on clear edges, the variances of their distributions may be smaller, and vice versa. Secondly, for a single landmark, its distribution in different directions may also vary. For landmarks that lie on edges, their distributions may be more likely to be along the edge, thus introducing anisotropy \cite{payer2020uncertainty}. Fig.~\ref{fig:fig0} shows an example.

%\begin{figure}[!t]
%\centering
%\includegraphics[height=4cm]{fig10.png}
%\caption{The left image shows the Gaussian distribution of one landmark used in most methods. However, the landmark is more likely to be located along the jaw and less likely to be located out in space or within the jaw, so the distribution illustrated in the right image may be more representative.}
%\label{fig:fig0}
%\end{figure}

\begin{figure}[!t]
\centering
 \subfigure[]{\includegraphics[width=3cm]{fig10a.png}}
 \subfigure[]{\includegraphics[width=3cm]{fig10b.png}}
\caption{The left image shows the Gaussian distribution of one landmark used in most methods. However, the landmark is more likely to be located along the jaw and less likely to be located out in space or within the jaw, so the distribution illustrated in the right image is likely to be more representative.}
\label{fig:fig0}
\end{figure}

% Since the structure of human bodies is complex with various substances, it is difficult to annotate the anatomical landmarks with perfect accuracy.
The aforementioned phenomenon shows the \textit{aleatoric} uncertainty of landmark detection. \textit{Aleatoric} uncertainty represents the noise inherent in the observations and is mainly caused by annotation ambiguities. It is difficult for humans to annotate anatomical landmarks with perfect accuracy. When annotating the same image several times, annotators may give different annotations, and the possibility can be described as a distribution. As mentioned before, different landmarks may have different distributions which are possibly anisotropic. However, previous methods simply utilize a fixed Gaussian kernel to get the heatmap, thus failing to model real distributions of landmarks. Specifically, these methods use a predetermined heatmap with the same variance and optimize the model by minimizing the loss between the predicted heatmap and the predetermined heatmap. Such predetermined heatmaps are not representative enough of the real annotation distributions, which may limit the performance of heatmap-based methods.

% 分布对比图,边缘分布可视化(加experiment中)

% Uncertainty can be divided into two major types \cite{kendall2017uncertainties}: \textit{epistemic} uncertainty is related to model parameters, capturing the lack of knowledge about the model when generating the collected data, and \textit{aleatoric} uncertainty shows the noise inherent in the observations, which can not be reduced even if more data is provided. We focus on the \textit{aleatoric} uncertainty of landmark detection methods since it tends to be ignored by past methods. \textit{Aleatoric} uncertainty of landmark detection is mainly caused by annotation ambiguities. Since the structure of human bodies is complex with various substances, it is difficult to annotate the anatomical landmarks with perfect accuracy. Therefore, the \textit{aleatoric} uncertainty generally exists, representing not only an inter-observer variability between different human annotators but also an intra-observer variability between different images.

% \textit{Aleatoric} uncertainty can be obtained by annotating the same image several times, but generally, each image has only one ground-truth label for each landmark in the existing public datasets. There are also difficulties in collecting a multi-labeled dataset since multiple manual annotations for one landmark in one image are sometimes time-consuming, especially in the medical field. Hence, we need to estimate the uncertainty from the single-labeled images. Fortunately, estimating uncertainty distribution from one ground-truth annotation is not impossible since a heatmap-based landmark detection framework can also be used to model the uncertainty \cite{payer2020uncertainty}.

% 我们的uncertainty和Aleatoric uncertainty关系,怎么对应
% 模型预测的热力图的分布应该符合人工标注关键点的uncertainty
% 如果按照各向同性(之前统一的高斯核)的部分的话,那人为标注的误差会给模型带来负面影响   模型构建的热力图的分布和人工标注的分布存在较大的差异

% the \textit{aleatoric} uncertainty of each landmark
% representing the real distribution of annotation. Our parametric model is based on U-Net \cite{ronneberger2015u}, which is widely used and has proven its effectiveness in heatmap-based landmark detection methods.

To this end, we propose a U-Net-based method to jointly predict the landmarks and their probability distributions. We use a Gaussian distribution to describe the possibility, and let the ground-truth landmark coordinate be the mean of the distribution. To predict the covariance matrix of the target distribution, we develop a Pyramid Covariance Predictor branch (shown in Fig.~\ref{fig:fig1}). Considering differences between low-level and high-level features, the branch processes the two kinds of features separately and fuses them to predict the covariance matrix. The experimental results show that the proposed Pyramid Covariance Predictor can indeed estimate the uncertainty of landmarks and improve the performance of anatomical landmark detection. Our contributions are as follows:

\begin{enumerate}
    \item We propose a method for uncertainty-aware landmark detection that jointly predicts the landmark and the covariance matrix of the target distribution.
    \item We design a module called Pyramid Covariance Predictor to discover the covariance matrix of the heatmap distribution. Visualizations demonstrate that the uncertainty our method estimates is more reasonable than that used in other methods.
    \item The experimental results show that our method achieves overall better performance compared with other methods.
\end{enumerate}

\section{Related Work}
\label{sec:Related Work}

\subsection{Medical Landmark Detection}

Early methods for medical landmark detection were based on a simple classifier, such as Random Forest \cite{ibragimov2015computerized, lindner2015fully}. Recently, deep-learning-based methods \cite{noothout2020deep,zhong2019attention,chen2019cephalometric} have been proven to be effective in medical landmark detection. Existing methods can be classified into three categories \cite{li2020structured}: 1) the coordinate-based approach \cite{noothout2020deep} that directly predicts the location of the landmarks by regression, 2) the graph-based approach \cite{li2020structured} that uses a graph to represent the structure of the landmarks, and 3) the heatmap-based approach.

Heatmap-based methods are the most common. These methods try to obtain the likelihood heatmap for each landmark and use this heatmap (sometimes with offset maps) to predict the locations of landmarks. For instance, in \cite{zhong2019attention}, a two-stage U-Net framework with attention mechanism and heatmap regression is developed to detect landmarks. In \cite{chen2019cephalometric}, an attentive feature pyramid fusion module is proposed to shape enhanced fusion features to improve accuracy. These methods generate the ground-truth heatmap using a symmetric Gaussian distribution with a fixed Gaussian kernel and optimize the model by minimizing the distance between predicted heatmaps and target heatmaps.

\subsection{Uncertainty Estimation}

There are two types of methods in uncertainty estimation: sampling-based and sampling-free. Sampling-based methods \cite{shridhar2019comprehensive,gal2016dropout,ayhan2018test} utilize the existing models and estimate the uncertainty by multiple evaluations. Popular methods, such as Bayesian neural networks \cite{shridhar2019comprehensive} and Monte Carlo dropout \cite{gal2016dropout}, all rely on several predictions by an ensemble of multiple networks or a dropout layer. Running the model on different augmentations\cite{ayhan2018test} is also an effective way.

%Except for sampling multiple models, data augmentation \cite{ayhan2018test} is also utilized in uncertainty estimation. By running the model on different augmentations, one can also obtain the distribution.

Sampling-free methods are modified from previous architectures to compute uncertainty using a single network and the same dataset. For example, in \cite{ovadia2019can}, the authors introduced stochastic variational inference and temperature scaling into the model to evaluate uncertainty. In \cite{sensoy2018evidential}, the uncertainty is estimated by fitting a Dirichlet distribution. In other fields, researchers have also tried to introduce uncertainty into their methods to improve performance, such as face alignment \cite{chen2019face} and body pose estimation \cite{gundavarapu2019structured}.

\begin{figure}[!t]
\centering
\includegraphics[height=5.5cm]{fig11.png}
\caption{Overview of our method. We add a branch to the U-Net architecture and fuse the low-level and high-level features separately. The low-level features are extracted by the first two layers of the encoder, and the high-level features are the output of remaining three layers.}
\label{fig:fig1}
\end{figure}

\section{Methods}
\label{sec:Method}

As shown in Fig.~\ref{fig:fig1}, we use a U-Net architecture as the heatmap predictor to obtain the heatmaps of landmarks. In this paper, we introduce a new branch, called Pyramid Covariance Predictor, to calculate the covariance matrix of the target distribution and estimate the annotating uncertainty. The details of our method are illustrated as follows.

\subsection{Heatmap Predictor}\label{section:HP}

% Although current methods in landmark detection achieve better performance through more complicated architectures, such as stacked or cascaded models, U-Net still shows its effectiveness with a relatively simple structure \cite{mccouat2022contour}. \textbf{Using U-Net to predict the heatmap is also a common method, so we adopt the standard U-Net as the heatmap predictor similarly.} 

We adopt the standard U-Net as the heatmap predictor due to its effectiveness and relatively simple structure. The standard U-Net architecture consists of an encoder and a decoder, where the encoder extracts the multi-scale features of input images and the decoder takes them as input to predict the heatmap. In our method, a VGG19 network \cite{simonyan2014very} pretrained on ImageNet Dataset \cite{krizhevsky2012imagenet} is adopted as the encoder. The decoder consists of consecutive modules. Each module contains an upsampling layer and a multi-scale feature discovery (MSFD) block (note that we omit the upsampling layer in Fig.~\ref{fig:fig1}). The upsampling layer upsamples the output from the last module using bilinear interpolation and combines the upsampled features with the image features from skip connections. The structure of the MSFD block is similar to the context-aware pyramid feature extraction (CPFE) module \cite{zhao2019pyramid}, which aims to fuse the combined features by the following two steps (as shown in Fig.~\ref{fig:fig2}): 1) extract multi-scale features by multiple atrous convolution layers with different dilation rates, and 2) concatenate them and reduce the channel. 

Finally, for $ N $ landmarks, the heatmap predictor outputs the heatmap $ \boldsymbol{\hat{H_i}} $, corresponding to the $i$-th landmark. For further processing, we calculate the predicted coordinate $ \boldsymbol{\hat{x}_i} $ from the heatmap. To utilize information from the whole heatmap and keep the gradient propagation, we use a weighted spatial mean to obtain $\boldsymbol{\hat{x}_i} $:

\begin{align}
 \label{eq:WSM}
 \boldsymbol{\hat{x}_i} = \frac{\Sigma_{\boldsymbol{x}}\delta\left(\boldsymbol{\hat{H_i}}\left(\boldsymbol{x}\right)\right)\boldsymbol{x}}{\Sigma_{\boldsymbol{x}}\delta\left(\boldsymbol{\hat{H_i}}\left(\boldsymbol{x}\right)\right)}
\end{align}
where $ \boldsymbol{\hat{H_i}}\left(\boldsymbol{x}\right) $ is the value at position $ \boldsymbol{x} $ in $ \boldsymbol{\hat{H_i}} $, $ \delta\left(\cdot\right) $ is the activation function.

% heatmap获得坐标

\begin{figure}[!t]
\centering
\includegraphics[height=3cm]{fig12.png}
\caption{The structure of the MSFD block consists of three convolution layers, with the concatenated feature as input. The outputs are then concatenated and processed by a $1 \times 1$ convolution to reduce the channel.}
\label{fig:fig2}
\end{figure}

% 加入SDR

\setlength{\tabcolsep}{3pt}
\begin{table*}[t]
\begin{center}
\caption{Comparison with other SOTA methods on IEEE ISBI Challenge 2015 Dataset.}
\label{table:table1}
\setlength{\tabcolsep}{2mm}{
\begin{tabular}{@{}ccccccccccc@{}}
\toprule

\multirow{3}{*}{Model} & \multicolumn{5}{c}{Validation Set} & \multicolumn{5}{c}{Test Set}  \\ \cmidrule(l){2-6} \cmidrule(l){7-11} & \multirow{2}{*}{MRE} & \multicolumn{4}{c}{SDR} & \multirow{2}{*}{MRE} & \multicolumn{4}{c}{SDR} \\ & & 2mm & 2.5mm & 3mm & 4mm & & 2mm & 2.5mm & 3mm & 4mm \\ \cmidrule(l){1-11}

Ibragimov \textit{et al.} \cite{ibragimov2015computerized} & 1.84 & 71.70 & 77.40 & 81.90  & 88.00 & - & 62.74 & 70.47 & 76.53 & 85.11 \\
Lindner \textit{et al.} \cite{lindner2015fully} & 1.67 & 74.95 & 80.28 & 84.56 & 89.68 & - & 66.11 & 72.00 & 77.63 & 87.42 \\
Arik \textit{et al.} \cite{arik2017fully} & - & 75.37 & 80.91 & 84.32 & 88.25 & - & 67.68 & 74.16 & 79.11 & 84.63 \\
Qian \textit{et al.} \cite{qian2019cephanet} & - & 82.50 & 86.20 & 89.30 & 90.60 & - & 72.40 & 76.15 & 79.65 & 85.90 \\
Chen \textit{et al.} \cite{chen2019cephalometric} & 1.17 & \textbf{86.67} & \textbf{92.67} & 95.54 & 98.53 & \textbf{1.48} & \textbf{75.05} & \textbf{82.84} & 88.53 & 95.05 \\
%Oh et al\cite{oh2020deep} & 1.18 & 86.20 & 91.20 & 94.40 & 97.70 & \textbf{1.45} & 75.89 & 83.36 & 89.26 & \textbf{95.73} \\
Lin \textit{et al.} \cite{lin2021structure} & 1.23 & 85.01 & 91.57 & 94.52 & 97.68 & 1.65 & 72.00 & 81.63 & 87.84 & 94.05 \\
%Li et al \citep{li2020structured} & \textbf{1.04} & \textbf{88.49} & 93.12 & 95.72 & 98.42 & \textbf{1.43} & 76.57 & 83.68 & 88.21 & 94.31 \\
Ours & \textbf{1.16} & 86.25 & 92.18 & \textbf{95.72} & \textbf{98.59} & \textbf{1.48} & 74.26 & 82.11 & \textbf{88.57} & \textbf{95.21} \\ \bottomrule
\end{tabular}}
\end{center}
\end{table*}
\setlength{\tabcolsep}{1.4pt}

\subsection{Pyramid Covariance Predictor}\label{section:CP}
% 点一下功能,呼应
The proposed Pyramid Covariance Predictor utilizes the pyramid features extracted by U-Net to predict the covariance matrix, which determines the distribution of the landmark location. For the multivariate location distribution, there are two approaches to building the target heatmap: Gaussian and Laplacian. Following \cite{payer2020uncertainty}, we typically choose the Gaussian function to represent the distributions of landmarks.

In previous works \cite{zhong2019attention,payer2016regressing}, the target heatmap $ \boldsymbol{H_i} $ of the $i$-th landmark can be described by an isotropic two-dimensional Gaussian function:

\begin{align}
 \label{eq:GF}
 \boldsymbol{H_i}\left(\boldsymbol{x};\sigma_i\right) = \frac{\gamma}{2\pi\sigma_i^2}\exp\left(-\frac{\|\boldsymbol{x} - \boldsymbol{x_i}\|^2_2}{2\sigma_i^2}\right)
\end{align}
where $ \boldsymbol{x_i} $ represents the location of the $i$-th landmark, $ \sigma_i $ is the standard deviation of the Gaussian distribution, and $ \gamma $ is a scaling factor that makes the function numerically stable. The mean of the distribution is set to the ground-truth landmark coordinate $ \boldsymbol{x_i} $ during training.

%However, isotropic Gaussian functions may be too simple to represent the real distribution. An isotropic Gaussian distribution for the landmark indicates that there is the same possibility of the same prediction error in all directions. This assumption may be realistic when the landmark is placed inside, where the surrounding pixels have similar colors. However, for those landmarks that lie on edges, an isotropic Gaussian heatmap may provide unrealistic information. Since the colors of the pixels near the edge change significantly, the landmark may be highly likely to be located along the edge rather than off the edge. Therefore, isotropic Gaussian functions are not representative enough to show the uncertainty of the location of landmarks. To better model the probability distribution, we use an anisotropic Gaussian function to estimate the uncertainty:

However, as aforementioned, isotropic Gaussian functions may be too simple to accurately represent the real distributions of landmarks. To better model the probability distributions, we use an anisotropic Gaussian function to estimate the uncertainty:

\begin{align}
 \label{eq:AGF}
 \boldsymbol{H_i}\left(\boldsymbol{x};\boldsymbol{\Sigma_i}\right) = \frac{\gamma \cdot exp\left(-\frac{1}{2}\left(\boldsymbol{x} - \boldsymbol{x_i}\right)^T\boldsymbol{\Sigma_i}^{-1}\left(\boldsymbol{x} - \boldsymbol{x_i}\right)\right)}{2\pi\sqrt{\left|\boldsymbol{\Sigma_i}\right|}}
\end{align}
Different from Eq.~\ref{eq:GF}, we use a full two-dimensional covariance matrix $ \boldsymbol{\Sigma_i} $ rather than a single value $ \sigma_i $ to represent the heatmap, thus introducing anisotropy into the distribution.

\begin{comment}

However, since the covariance matrix is a positive semi-definite matrix, its determinant is non-negative:

\begin{align}
 \label{eq:DET}
 det\left(\boldsymbol{\Sigma_i}\right) = cov\left(\boldsymbol{x}, \boldsymbol{x}\right)cov\left(\boldsymbol{y}, \boldsymbol{y}\right) - cov\left(\boldsymbol{x}, \boldsymbol{y}\right)^2 \geq 0
\end{align}

Note that $ det\left(\boldsymbol{\Sigma_i}\right) = 0 $ if and only if the correlation coefficient of $ \boldsymbol{X} $ and $ \boldsymbol{Y} $ equals 1 or -1, which is nearly impossible in the reality. Therefore, we hypothesize that the target covariance matrix is a positive definite matrix, which can simplify the following steps.

Eq.~\ref{eq:DET} shows that there is a relationship among elements of the covariance matrix. If we directly predict the elements of the matrix by regression, the outputs may be illegal since all the outputs of the fully connected layer are independent of each other. To solve this problem, we try to capture the degrees of freedom of the covariance matrix. Predicting its Cholesky decomposition is a simple but effective way \cite{kumar2020luvli}. Cholesky decomposition is a useful method in numerical solutions, such as Monte Carlo simulations. It decomposes a Hermitian, positive-definite matrix into the product of a lower-triangular matrix and its conjugate transpose. In our method, the target covariance matrix can be decomposed as:

\end{comment}

%%%%%%%%%%%%%%%%%%%%%%%%%%%%%%%%%%%%%%%%%%%%%%%%%%%%%%%%%%%%%%%%%%%%%%%%%%%%%%%%%%%%%%%%%%%%%%%%%%%%%%%%%%%%%%%%%%%%%%%%%%%%%%%%
%%%%%%%%%%%%%%%%%%%%%%%%%%%%%%%%%%%%%%%%%%%%%%%%%%%%%%%%%%%%%%%%%%%%%%%%%%%%%%%%%%%%%%%%%%%%%%%%%%%%%%%%%%%%%%%%%%%%%%%%%%%%%%%%
%%%%%%%%%%%%%%%%%%%%%%%%%%%%%%%%%%%%%%%%%%%%%%%%%%%%%%%%%%%%%%%%%%%%%%%%%%%%%%%%%%%%%%%%%%%%%%%%%%%%%%%%%%%%%%%%%%%%%%%%%%%%%%%%

We follow \cite{kumar2020luvli} to predict the Cholesky decomposition of the covariance matrix in case of the illegal covariance matrix from direct regression. Details are further demonstrated in the Appendix. According to Cholesky Factorization Theorem \cite{schabauer2010toward}, the target covariance matrix can be decomposed as:

%\textbf{To obtain the covariance matrix, we choose to predict its Cholesky decomposition\cite{kumar2020luvli}. The target covariance matrix can be decomposed as:}

\begin{align}
 \label{eq:CD}
 \boldsymbol{\Sigma_i} = \boldsymbol{C_i}\boldsymbol{C_i}^T
\end{align}
%where $ \boldsymbol{C_i} $ has the form of $ \begin{bmatrix} a_i & 0 \\ b_i & c_i \end{bmatrix} $. It is proved that the Cholesky decomposition of a matrix always exists and is unique. Since $ a_i $, $ b_i $, and $ c_i $ are independent of each other, and they can determine a covariance matrix, we can predict these values by regression.
where $ \boldsymbol{C_i} $ has the form of $ \begin{bmatrix} a_i & 0 \\ b_i & c_i \end{bmatrix} $. 

%这一段的描述不是很清楚 感觉有点乱
In our work, we calculate $ a_i $, $ b_i $, and $ c_i $ by utilizing the multi-scale features extracted by the encoder of U-Net architecture. The multi-scale features can be divided into two categories: low-level features and high-level features \cite{zhao2019pyramid}. Low-level features represent features that show the pattern of the image, such as color, texture, and edges, while high-level features contain more semantic information. Due to the difference between low-level features and high-level features, we process the two types of features separately. Following \cite{zhao2019pyramid}, we choose the features of the first two layers of the encoder as low-level features and choose the features of the remaining three layers as high-level features. For both low-level features and high-level features, we first apply average pooling to generate feature maps with the same resolution. Then, we concatenate them separately into two larger pyramid features. We fuse both features using the self-attention mechanism \cite{vaswani2017attention} since it has shown excellent ability in capturing long-range dependency. The self-attention mechanism is a Query-Key-Value architecture. It computes the scaled dot product of Query and Key and uses the result as a weight on Value. Formally, given an input feature $ X $, to compute its self-attention results, we first embed $ X $ to three different matrices $ \boldsymbol{Q} $, $ \boldsymbol{K} $, $ \boldsymbol{V} $, representing Query, Key, and Value:

\begin{align}
 \boldsymbol{Q} = \boldsymbol{XW_Q}, \quad\boldsymbol{K} = \boldsymbol{XW_K}, \quad\boldsymbol{V} = \boldsymbol{XW_V}
\end{align}
Then the self-attention results can be computed by:

\begin{align}\label{formula:sa}
 \boldsymbol{S} = \sigma\left(\frac{\boldsymbol{QK}^T}{\sqrt{d}}\right)\boldsymbol{V}
\end{align}
where $ d $ is the dimension of $\boldsymbol{K}$, $ \sigma $ is the softmax function.

Finally, we concatenate the output features and predict the Cholesky decomposition through a fully connected layer. Thus, the covariance matrix can be computed using its Cholesky decomposition (shown in Eq. \ref{eq:CD}).

\subsection{Loss Function}\label{section:LF}

The loss function of our method, similar to the negative log-likelihood of the anisotropic Gaussian function, can be expressed as follows:

\begin{align}\label{formula:loss}
 L = \left(\boldsymbol{\hat{x}_i} - \boldsymbol{x_i}\right)^T\boldsymbol{\hat{\Sigma}}_i^{-1}\left(\boldsymbol{\hat{x}_i} - \boldsymbol{x_i}\right) + \alpha log\left|\boldsymbol{\hat{\Sigma}_i}\right|
\end{align}
where $ \boldsymbol{\hat{x}_i} $ and $ \boldsymbol{x_i} $ are the predicted and ground truth locations of the $ i $-th landmark, respectively. $ \boldsymbol{\hat{\Sigma}_i} $ is the predicted covariance matrix.

The first term in Eq.~\ref{formula:loss}  is the squared Mahalanobis distance between the predicted landmark and the ground truth. The second term in Eq.~\ref{formula:loss} is a regularization to prevent the distribution from becoming excessively flattened. $ \alpha $ is a hyperparameter to adjust the weight of the regularization term. In our experiment, we empirically set it to 0.1. 
% 为啥要二分之一
% 权重

\setlength{\tabcolsep}{3pt}
\begin{table*}[t]
\begin{center}
\caption{Results of ablation experiments.}
\label{table:table2}
\begin{tabular}{@{}ccccccccccc@{}}
\toprule
\multirow{3}{*}{Model} & \multicolumn{5}{c}{Validation Set} & \multicolumn{5}{c}{Test Set}  \\ \cmidrule(l){2-6} \cmidrule(l){7-11} & \multirow{2}{*}{MRE} & \multicolumn{4}{c}{SDR} & \multirow{2}{*}{MRE} & \multicolumn{4}{c}{SDR} \\ & & 2mm & 2.5mm & 3mm & 4mm & & 2mm & 2.5mm & 3mm & 4mm \\ \cmidrule(l){1-11}
U-Net & 1.24 & 84.84 & 90.52 & 93.75 & 97.40 & 1.61 & 71.89 & 80.63 & 86.36 & 93.68 \\
Exp-U-Net & 1.23 & 84.59 & 91.64 & 94.87 & 98.28 & \textbf{1.48} & \textbf{75.05} & \textbf{82.84} & \textbf{88.68} & 94.42 \\
Ours & \textbf{1.16} & \textbf{86.25} & \textbf{92.18} & \textbf{95.72} & \textbf{98.59} & \textbf{1.48} & 74.26 & 82.11 & 88.57 & \textbf{95.21} \\ \bottomrule
\end{tabular}
\end{center}
\end{table*}
\setlength{\tabcolsep}{1.4pt}

\begin{figure}[!t]
\centering
\includegraphics[height=7cm]{bigger_fig4.png}
\caption{The result samples of our method. The images in the first row are from the validation set, and the others are from the test set. The indigo and yellow points represent the ground truth landmarks and predicted landmarks, respectively. The red numbers indicate the indices of landmarks.}
% 颜色太接近了  可以搞一个放大的那种,加序号
\label{fig:fig3}
\end{figure}

\section{Experiments}
\label{sec:Experiments}

\subsection{Experimental Setting}\label{section:ES}

\textbf{Datasets.} We conduct experiments on IEEE ISBI Challenge 2015 Dataset \cite{wang2016benchmark}, which is widely used in cephalometric landmark detection tasks. It contains 400 cephalometric radiograph samples with a size of $ 1935 \times 2400 $, each labeled with 19 landmarks by two doctors. We choose the average of two annotations as the ground truth. Following previous methods \cite{chen2019cephalometric,oh2020deep}, we set the sizes of the training set, validation set, and test set to 150, 150, and 100, respectively. The result samples are illustrated in Fig.\ref{fig:fig3}. 

\noindent\textbf{Evaluation.} We use two metrics to evaluate the performance: Mean Radial Error (MRE) and Successful Detection Rate (SDR) in different radii (2mm, 2.5mm, 3mm, and 4mm). These standards are computed as follows:

\begin{gather}
 \text{MRE} = \frac{\sum_{i=1}^{n}\sqrt{\Delta x_i^2 + \Delta y_i^2}}{n} \\
 \text{SDR} = \frac{N_{acc}}{N_{all}} \times 100\%
 % n换个大写字母
\end{gather}
where $ \Delta x_i $ and $ \Delta y_i $ are the absolute differences between the $i$-th ground truth landmark and the corresponding predicted landmark in the x and y axis, respectively. $ N_{acc} $ is the number of successful detections and $ N_{all} $ is the number of detections. A successful detection is defined as the real distance between the two landmarks being lower than the precision radius.

\noindent {\bf Implementation Details.}  For a fair comparison, we resize the cephalometric radiographs to $ 800 \times 640 $ and normalize them for further training. We use PyTorch to build our framework and use Adam \cite{kingma2014adam} as our optimizer. The learning rate is set to 3e-4, and the weight decay is 1e-4.

\subsection{Comparison with Other Methods}\label{section:CWOM}

We select six representative landmark detection methods for comparison. The experimental results shown in Tab. \ref{table:table1} demonstrate that our method achieves better performance on MRE and outperforms existing SOTA methods on SDR in 3mm and 4mm radii. For localization evaluated by SDR in 2mm and 2.5mm radii, our method also achieves comparable results with the SOTA method \cite{chen2019cephalometric}. 

The results show that our method has an overall better result, but compared with Chen \textit{et al.} \cite{chen2019cephalometric}, our method is slightly less effective in terms of more accurate predictions. The method in \cite{chen2019cephalometric} uses not only a heatmap but also two offset maps to predict the landmarks. In their approach, the heatmap only provides a rough prediction, and the offset maps are utilized to precisely locate the position. However, due to the proposed Pyramid Covariance Predictor, our method can improve the performance of the heatmap and achieve comparable results using only heatmaps. Additionally, our method has a better result in larger precision radii. More comparison experiments are shown in the Appendix.
% \textbf{It indicates that our method can make the relatively poor predictions more precise and thus has an overall better result on MRE.}

\subsection{Ablation Studies}\label{section:AS}

In this section, we conduct ablation studies on the Pyramid Covariance Predictor and use the classic U-Net framework as the baseline model. To demonstrate the effectiveness of the proposed Pyramid Covariance Predictor, we also compare our model with the traditional U-Net framework with similar computation and parameter levels (Exp-U-Net). Exp-U-Net is constructed by adding convolution layers.

The experimental results are shown in Tab.~\ref{table:table2}. Our method outperforms the compared methods in all metrics on the validation set. For the test set, our method also achieves better results in MRE and SDR on 4mm radius. Although Exp-U-Net has better performance on the test set in other precision radii, the performance gap between our method and Exp-U-Net gradually narrows with an increase in the precision radius. These results also show that our method can perform better in larger precision radii.

\begin{figure}[!t]
\centering
\includegraphics[height=7cm]{more_fig5.png}
\caption{A comparison between our predicted Gaussian distributions (the lower three images) and the predetermined Gaussian distributions generated by Chen \textit{et al.} \cite{chen2019cephalometric} (the upper three images) is shown. Cyan points represent the landmarks, and yellow ellipses indicate the distribution of each landmark.}
\label{fig:fig4}
\end{figure}

\subsection{Uncertainty Estimation}\label{section:UE}

In our work, we also predict the covariance matrix and obtain the distribution for each landmark. The distributions we predict differ from the predetermined distributions used in most of the current heatmap-based methods, as illustrated in Fig.~\ref{fig:fig4}. We also measure the uncertainty of the predicted landmarks and compare it with the predicted uncertainty; the details are discussed in the Appendix.

The predicted distributions of our method are intuitively in line with the annotation uncertainty. For each ellipse in the right image of Fig.~\ref{fig:fig4}, the direction of the major axis conforms to the direction of the edge to a certain degree. Therefore, the possibility along the normal direction decreases more rapidly than that along the edge, which is in accordance with our cognition. The results show that a heatmap-based landmark detection framework can also be used to model the uncertainty.

In addition, some neighbor ellipses overlap in the left image of Fig.~\ref{fig:fig4}, especially in the areas where landmarks are relatively densely distributed. This indicates that those pixels in the overlapping area may be detected as multiple landmarks with similar probability, which may confuse the model. However, in the generated distributions of our method, each distribution area can be well separated, which further demonstrates the effectiveness of our method.

\section{Conclusion}
\label{sec:Conclusion}

In this paper, we propose the Pyramid Covariance Predictor to discover the \textit{aleatoric} uncertainty of each landmark. The Pyramid Covariance Predictor makes use of multi-scale features and predicts the covariance matrix of the anisotropic Gaussian distribution to better represent the distributions of landmarks. Experimental results show that our method can estimate the uncertainty to some degree and achieve overall better results than other compared methods in cephalometric landmark detection.

% References should be produced using the bibtex program from suitable
% BiBTeX files (here: strings, refs, manuals). The IEEEbib.bst bibliography
% style file from IEEE produces unsorted bibliography list.
% -------------------------------------------------------------------------

\bibliographystyle{IEEEbib}
\bibliography{icme2023template}

In this supplemental material, we provide details for our implementation in Sec.~\ref{SecImple}, dataset pre-processing and text prompt generation in Sec.~\ref{SecData}, baseline implementations in Sec.~\ref{SecBaseline}, additional results in Sec.~\ref{SecAddRes}, and user studies in Sec.~\ref{SecUser}.

\section{Implementations}
\label{SecImple}

\subsection{Shape Auto-Encoder}
\label{SubSecShapeAE}

We adopt a pre-trained shape auto-encoder to extract a set of latent shape codes for CAD models from the 3D-FUTURE~\cite{fu20213dm} dataset. The network architecture of the shape auto-encoder is shown in Fig.~\ref{fig:shapeae}. It is a variational auto-encoder, similar to FoldingNet~\cite{yang2018foldingnet}.
Specifically, a point cloud $\mathbf{P}_{in}$ of size 2,048 is fed into a graph encoder based on PointNet~\cite{qi2017pointnet} with graph convolutions~\cite{wang2019dynamic} to extract a global latent code of dimension 512, which is used to predict the mean $\mathbf{\mu}$ and variance $\mathbf{\sigma}$ of a low-dimensional latent space of size 64.
Subsequently, a compressed latent is sampled from $\mathcal{N}(\mathbf{\mu}, \mathbf{\sigma})$.
%\TODO{maybe stupid question, but what is reparametrization sampling. we should explain that}
Finally, the compressed latent is mapped back to the original space and passed to the FoldingNet decoder to recover a point cloud $\mathbf{P}_{rec}$ of size 2,025.
The used training objective is a weighted combination of Chamfer distance (\ie CD) and KL divergence.
\begin{equation}
    \label{EquaShapeAE}
    L_{vae} = \CD(\mathbf{P}_{in}, \mathbf{P}_{rec}) + \omega_{kl} *\KL(\mathcal{N}(\mathbf{\mu}, \mathbf{\sigma}) || \mathcal{N}(\mathbf{0}, \mathbf{I})) ,
\end{equation}
where $\omega_{kl}$ is set to 0.001.
The latent compression and KL regularization leads to a compact and structured latent space, focusing on global shape structures.
The shape autoencoder is trained on a single RTX 2080 with a batch size of 16 for 1,000 epochs.
The learning rate is initialized to $lr=\expnumber{1}{-4}$ and then gradually decreases with the decay rate of 0.1 in every 400 epochs.
\begin{figure}
    \centering
    \includegraphics[width=\linewidth]{./figs/shapeautoencoder.pdf}
    \caption{\textbf{Shape Auto-encoder.}}
    \label{fig:shapeae}
\end{figure}

\subsection{Shape Code Diffusion}
\label{SubSecShapeDiffu}

We use the extracted latent codes to train shape code diffusion.
While we apply KL regularization, the value range of latent codes is still unbound.
To make it easier to diffuse, we scale the latent codes to $[-1, 1]$ by using the statistical minimum and maximum feature values over the whole set.
During inference, we rescale generated shape codes.

\subsection{Shape Retrieval}
\label{SubSecRetrieval}

During inference, we use shape retrieval as the post-processing procedure to acquire object surface geometries for generated scene graphs.
Concretely, for each graph node, we perform the nearest neighbor search in the 3D-FUTURE~\cite{fu20213dm} dataset to find the CAD model with the same class label, the closest bounding box size, and the closest geometry feature.
Previous works~\cite{wang2021sceneformer, paschalidou2021atiss} only use object semantics and bounding box sizes during shape retrieval, we consider the similarity of geometry descriptors. Thus, our method can retrieve more accurate shape geometries. After the object retrieval, we place the retrieved CAD models into the scene based on the predicted locations, orientation angles, and sizes.  

% \subsection{Loss function}
% \label{SubSecLoss}

\section{Dataset}
\label{SecData}

\paragraph{Preprocessing}
The dataset preprocessing is based on the setting of ATISS~\cite{paschalidou2021atiss}.
We start by filtering out those scenes with problematic object arrangements such as severe object intersections or incorrect object class labels, e.g., beds are misclassified as wardrobes in some scenes.
Then, we remove those scenes with unnatural sizes. The floor size of a natural room is within $6m \times 6m$ and its height is less than $4m$. Subsequently, we ignore scenes that have too few or many objects.
The number of objects in valid bedrooms is between 3 and 13. As for dining and living rooms, the minimum and maximum numbers are set to 3 and 21 respectively. Thus, the number of scene graph nodes is $N=13$ in bedrooms and $N=21$ in dining and living rooms. In addition, we delete scenes that have objects out of pre-defined categories. After pre-processing, we obtained 4,041 bedrooms, 900 dining rooms, and 813 living rooms.

For the semantic class diffusion, we have an additional class of  `empty' to define the existence of an object. Combining with the object categories that appeared in each room type, we have $L=22$ object categories for bedrooms, and
$L=25$ object categories for dining and living rooms in total. The category labels 
are listed as follows.

\begin{python}
# 22 3D-Front bedroom categories
['empty', 'armchair', 'bookshelf', 'cabinet',
'ceiling_lamp', 'chair', 'children_cabinet',
'coffee_table', 'desk', 'double_bed',
'dressing_chair', 'dressing_table', 'kids_bed',
'nightstand', 'pendant_lamp', 'shelf',
'single_bed', 'sofa', 'stool', 'table',
'tv_stand', 'wardrobe']

# 25 3D-Front dining or living room categories
['empty', 'armchair', 'bookshelf', 'cabinet', 
'ceiling_lamp', 'chaise_longue_sofa', 
'chinese_chair', 'coffee_table', 'console_table',  
'corner_side_table',  'desk', 'dining_chair', 
'dining_table', 'l_shaped_sofa', 'lazy_sofa', 
'lounge_chair', 'loveseat_sofa', 
'multi_seat_sofa', 'pendant_lamp', 
'round_end_table', 'shelf', 'stool', 
'tv_stand', 'wardrobe', 'wine_cabinet']
\end{python}

\paragraph{Text Prompt Generation}
We follow the SceneFormer~\cite{wang2021sceneformer} to generate text prompts describing partial scene configurations. Each text prompt contains one to three sentences. We explain the details of text formulation process by using the text prompt 'The room has a dining table, a pendant lamp, and a lounge chair. The pendant lamp is above the dining table. There is a stool to the right of the lounge chair.` as an example. First, we randomly select three objects from a scene, get their class labels, and then count the number of appearances of each selected object category. As such, we can get the first sentence. Then, we find all valid object pairs associated with the selected three objects. An object pair is valid only if the distance between two objects is less than a certain threshold that is set to 1.5 in our method. Next, we calculate the relative orientations and translations, from which we can determine the relationship type of the valid object pair from the candidate pool: 'is above to`, 'is next to`, 'is left of`, 'is right of`, ' surrounding`, 'inside`, 'behind`, 'in front of`, and 'on`. In this way, we can acquire some relation-describing sentences like the second and third sentences in the example. Finally, we randomly sampled zero to two relation-describing sentences.

\section{Baselines}
\label{SecBaseline}

\paragraph{DepthGAN} 
DepthGAN~\cite{yang2021indoor} adopts a generative adversary network to train 3D scene synthesis using both semantic maps and depth images. The generator network is built with 3D convolution layers, which decode a volumetric scene with semantic labels. A differentiable projection layer is applied to project the semantic scene volume into depth images and semantic maps under different views, where a multi-view discriminator is designed to distinguish the synthesized views from ground-truth semantic maps and depth images during the adversarial training.


\paragraph{Sync2Gen} 
Sync2Gen~\cite{yang2021scene} represents a scene arrangement as a sequence of 3D objects characterized by different attributes (e.g., bounding box, class category, shape code). The generative ability of their method relies on a variational auto-encoder network, where they learn objects' relative attributes. Besides, a Bayesian optimization stage is used as a post-processing step to refine object arrangements based on the learned relative attribute priors.

\paragraph{ATISS}
ATISS~\cite{paschalidou2021atiss} considers a scene as an unordered set of objects and then designs a novel autoregressive transformer architecture to model the scene synthesis process. During training, based on the previously known object attributes, ATISS utilizes a permutation-invariant transformer to aggregate their features and  predicts the location, size, orientation, and class category of the next possible object conditioned on the fused feature. 
The original version of ATISS~\cite{paschalidou2021atiss} is conditioned on a 2D room mask from the top-down orthographic projection of the 3D floor plane of a scene. To ensure fair comparisons, we train an unconditional ATISS without using a 2D room mask as input, following the same training strategies and hyperparameters as the original ATISS.


% \section{Evaluation Metrics}
% \label{SecEval}

% \paragraph{Fr{\'e}chet Inception Distance}

% \paragraph{Kernel Inception Distance}

% \paragraph{Scene Classification Accuracy}

% \paragraph{Category KL Divergenece}



\section{Additional Results}
\label{SecAddRes}

\paragraph{Unconditional Scene Synthesis}
\begin{figure*}[!htbp]
	\centering
 	\begin{subfigure}[t]{0.23\textwidth}
		\includegraphics[width=\textwidth]
	{figs/experiments/uncond_gallery/SecondBedroom-35821_15_393.jpg}
 
  	\includegraphics[width=\textwidth]{./figs/experiments/uncond_gallery/LivingRoom-41893_126_445.jpg}
   
		\includegraphics[width=\textwidth]{./figs/experiments/uncond_gallery/LivingDiningRoom-86944_84_103.jpg}

            \includegraphics[width=\textwidth]{./figs/experiments/unconditional/ours/living/LivingRoom-71071_8_074.jpg}
    
	\end{subfigure}
	\rulesep
        %
	\begin{subfigure}[t]{0.23\textwidth}
		\includegraphics[width=\textwidth]
	{figs/experiments/uncond_gallery/SecondBedroom-52584_24_964.jpg}
 
  	\includegraphics[width=\textwidth]{./figs/experiments/uncond_gallery/LivingRoom-50084_76_718.jpg}
   
		\includegraphics[width=\textwidth]{./figs/experiments/uncond_gallery/LivingDiningRoom-99518_174_183.jpg}

            \includegraphics[width=\textwidth]{./figs/experiments/uncond_gallery/LivingDiningRoom-163914_165_492.jpg}

	\end{subfigure}
	\rulesep
        %
        \begin{subfigure}[t]{0.23\textwidth}
		\includegraphics[width=\textwidth]
	{figs/experiments/uncond_gallery/SecondBedroom-86888_75_920.jpg}
 
  	\includegraphics[width=\textwidth]{./figs/experiments/uncond_gallery/LivingRoom-68491_81_650.jpg}
   
		\includegraphics[width=\textwidth]{./figs/experiments/uncond_gallery/LivingDiningRoom-109935_48_096.jpg}

        \includegraphics[width=\textwidth]{./figs/experiments/uncond_gallery/LivingRoom-71071_8_997.jpg}
        
	\end{subfigure}
	\rulesep
         %
	\begin{subfigure}[t]{0.23\textwidth}
		\includegraphics[width=\textwidth]
	{figs/experiments/uncond_gallery/SecondBedroom-258160_63_131.jpg}
 
  	\includegraphics[width=\textwidth]{./figs/experiments/uncond_gallery/LivingRoom-71071_8_485.jpg}
   
		\includegraphics[width=\textwidth]{./figs/experiments/uncond_gallery/LivingDiningRoom-126918_10_090.jpg}

  \includegraphics[width=\textwidth]{./figs/experiments/uncond_gallery/LivingRoom-88425_55_025.jpg}
	\end{subfigure}
	\caption{Diverse and plausible results of unconditional scene synthesis from our method. }
\label{fig:uncond_gallery}
%\vspace{2mm}
\end{figure*}

 In Fig.~\ref{fig:uncond_gallery}, we provide more visualization results of our unconditional scene synthesis model. 

\paragraph{Scene Completion}
\begin{figure*}[t]
    %\vspace{-2mm}
	\centering
	\begin{subfigure}[t]{0.14\textwidth}
            \includegraphics[width=\textwidth]{./figs/experiments/scene_completion_supple/partial/Bedroom-15797_117_075.jpg}
            \includegraphics[width=\textwidth]{./figs/experiments/scene_completion_supple/partial/Bedroom-17102_150_930.jpg}
            \includegraphics[width=\textwidth]{./figs/experiments/scene_completion_supple/partial/LivingDiningRoom-233_45_129.jpg}
            \includegraphics[width=\textwidth]{./figs/experiments/scene_completion_supple/partial/LivingDiningRoom-69704_153_931.jpg}
        \caption{Partial Scenes}
	\end{subfigure}
        \rulesep
        %
        \begin{subfigure}[t]{0.41\textwidth}
    	\includegraphics[width=0.33\textwidth]{./figs/experiments/scene_completion_supple/atiss/Bedroom-15797_075.jpg}%
            \hfill
    	\includegraphics[width=0.33\textwidth]{./figs/experiments/scene_completion_supple/atiss/Bedroom-15797_344.jpg}%
            \hfill
      	\includegraphics[width=0.33\textwidth]{./figs/experiments/scene_completion_supple/atiss/Bedroom-15797_439.jpg} 
       %%%%%%
    	\includegraphics[width=0.33\textwidth]{./figs/experiments/scene_completion_supple/atiss/Bedroom-17102_180.jpg}%
            \hfill
        \includegraphics[width=0.33\textwidth]{./figs/experiments/scene_completion_supple/atiss/Bedroom-17102_713.jpg}%
        \hfill
        \includegraphics[width=0.33\textwidth]{./figs/experiments/scene_completion_supple/atiss/Bedroom-17102_930.jpg}
        %%%%%
        \includegraphics[width=0.33\textwidth]{./figs/experiments/scene_completion_supple/atiss/LivingDiningRoom-233_45_000.jpg}%
        \hfill
        \includegraphics[width=0.33\textwidth]{./figs/experiments/scene_completion_supple/atiss/LivingDiningRoom-233_45_002.jpg}%
        \hfill
        \includegraphics[width=0.33\textwidth]{./figs/experiments/scene_completion_supple/atiss/LivingDiningRoom-233_45_003.jpg}
        %%%%%
        \includegraphics[width=0.33\textwidth]{./figs/experiments/scene_completion_supple/atiss/LivingDiningRoom-69704_153_000.jpg}%
        \hfill
        \includegraphics[width=0.33\textwidth]{./figs/experiments/scene_completion_supple/atiss/LivingDiningRoom-69704_153_001.jpg}%
        \hfill
        \includegraphics[width=0.33\textwidth]{./figs/experiments/scene_completion_supple/atiss/LivingDiningRoom-69704_153_005.jpg}
        \caption{ATISS~\cite{paschalidou2021atiss}}
	\end{subfigure}
        \rulesep
        %
	\begin{subfigure}[t]{0.41\textwidth}
    	\includegraphics[width=0.33\textwidth]{./figs/experiments/scene_completion_supple/ours/Bedroom-15797_117_010.jpg}%
            \hfill
            \includegraphics[width=0.33\textwidth]{./figs/experiments/scene_completion_supple/ours/Bedroom-15797_117_007.jpg}%
    	\hfill
    	\includegraphics[width=0.33\textwidth]{./figs/experiments/scene_completion_supple/ours/Bedroom-15797_117_015.jpg}
        %%%%%%%%%%%%%%%%%%%%%%%
    	\includegraphics[width=0.33\textwidth]{./figs/experiments/scene_completion_supple/ours/Bedroom-17102_150_004.jpg}%
            \hfill
            \includegraphics[width=0.33\textwidth]{./figs/experiments/scene_completion_supple/ours/Bedroom-17102_150_011.jpg}%
    	\hfill
    	\includegraphics[width=0.33\textwidth]{./figs/experiments/scene_completion_supple/ours/Bedroom-17102_150_013_2.jpg}
             %%%%%%%%%%%%%%%%%%%%%%%
    	\includegraphics[width=0.33\textwidth]{./figs/experiments/scene_completion_supple/ours/LivingDiningRoom-233_45_002.jpg}%
            \hfill
            \includegraphics[width=0.33\textwidth]{./figs/experiments/scene_completion_supple/ours/LivingDiningRoom-233_45_012.jpg}%
    	\hfill
    	\includegraphics[width=0.33\textwidth]{./figs/experiments/scene_completion_supple/ours/LivingDiningRoom-233_45_017.jpg}
             %%%%%%%%%%%%%%%%%%%%%%%
    	\includegraphics[width=0.33\textwidth]{./figs/experiments/scene_completion_supple/ours/LivingDiningRoom-69704_016.jpg}%
            \hfill
            \includegraphics[width=0.33\textwidth]{./figs/experiments/scene_completion_supple/ours/LivingDiningRoom-69704_019.jpg}%
    	\hfill
    	\includegraphics[width=0.33\textwidth]{./figs/experiments/scene_completion_supple/ours/LivingDiningRoom-69704_251.jpg}
		\caption{Ours}
	\end{subfigure}
	\caption{\textbf{Scene completion} from partial scenes with only three objects given as inputs. Compared to ATISS, our method produced more diverse completion results with higher fidelity.}
    \label{fig:completion_supple}
    %\vspace{-2mm}
\end{figure*}
We present more qualitative comparisons on the task of scene completion in Fig.~\ref{fig:completion_supple}.

\paragraph{Scene Arrangement}
\begin{figure*}[!ht]
	\centering
	\begin{subfigure}[t]{0.14\textwidth}
            \includegraphics[width=\textwidth]{./figs/experiments/arrangement_supple/noisy/LivingDiningRoom-270_12_012.jpg}
            \includegraphics[width=\textwidth]{./figs/experiments/arrangement_supple/noisy/LivingDiningRoom-13177_88_442.jpg}
            \includegraphics[width=\textwidth]{./figs/experiments/arrangement_supple/noisy/LivingDiningRoom-64631_169_700.jpg}
            \includegraphics[width=\textwidth]{./figs/experiments/arrangement_supple/noisy/LivingDiningRoom-79295_184_184.jpg}
        \caption{Noisy Scenes}
	\end{subfigure}
        \rulesep
        %%%%
        \begin{subfigure}[t]{0.41\textwidth}
    	\includegraphics[width=0.33\textwidth]{./figs/experiments/arrangement_supple/atiss/LivingDiningRoom-270_12_012.jpg}%
            \hfill
    	\includegraphics[width=0.33\textwidth]{./figs/experiments/arrangement_supple/atiss/LivingDiningRoom-270_12_204.jpg}%
            \hfill
      	\includegraphics[width=0.33\textwidth]{./figs/experiments/arrangement_supple/atiss/LivingDiningRoom-270_12_396.jpg} 
            %%%%%%%%%%%%%%%%%%%%%%%
    	\includegraphics[width=0.33\textwidth]{./figs/experiments/arrangement_supple/atiss/LivingDiningRoom-13177_88_088.jpg}%
            \hfill
            \includegraphics[width=0.33\textwidth]{./figs/experiments/arrangement_supple/atiss/LivingDiningRoom-13177_88_796.jpg}%
    	\hfill
            \includegraphics[width=0.33\textwidth]{./figs/experiments/arrangement_supple/atiss/LivingDiningRoom-13177_88_973.jpg}
               %%%%%%%%%%%%%%%%%%%%%%%
    	\includegraphics[width=0.33\textwidth]{./figs/experiments/arrangement_supple/atiss/LivingDiningRoom-64631_169_523.jpg}%
            \hfill
            \includegraphics[width=0.33\textwidth]{./figs/experiments/arrangement_supple/atiss/LivingDiningRoom-64631_169_700.jpg}%
    	\hfill
            \includegraphics[width=0.33\textwidth]{./figs/experiments/arrangement_supple/atiss/LivingDiningRoom-64631_169_877.jpg}
            %%%%%%%%%%%%%%%%%%%%%%%
    	\includegraphics[width=0.33\textwidth]{./figs/experiments/arrangement_supple/atiss/LivingDiningRoom-79295_184_184.jpg}%
            \hfill
            \includegraphics[width=0.33\textwidth]{./figs/experiments/arrangement_supple/atiss/LivingDiningRoom-79295_184_376.jpg}%
    	\hfill
            \includegraphics[width=0.33\textwidth]{./figs/experiments/arrangement_supple/atiss/LivingDiningRoom-79295_184_568.jpg}
        \caption{ATISS~\cite{paschalidou2021atiss}}
	\end{subfigure}
        \rulesep
        %%%
	\begin{subfigure}[t]{0.41\textwidth}
    	\includegraphics[width=0.33\textwidth]{./figs/experiments/arrangement_supple/ours/LivingDiningRoom-270_12_014.jpg}
            \hfill
            \includegraphics[width=0.33\textwidth]{./figs/experiments/arrangement_supple/ours/LivingDiningRoom-270_12_017.jpg}%
    	\hfill
    	\includegraphics[width=0.33\textwidth]{./figs/experiments/arrangement_supple/ours/LivingDiningRoom-270_12_020.jpg}
      %%%%%%%%%%%%%%%%%%%%%%%
    	\includegraphics[width=0.33\textwidth]{./figs/experiments/arrangement_supple/ours/LivingDiningRoom-13177_88_005.jpg}%
            \hfill
            \includegraphics[width=0.33\textwidth]{./figs/experiments/arrangement_supple/ours/LivingDiningRoom-13177_88_021.jpg}%
    	\hfill
    	\includegraphics[width=0.33\textwidth]{./figs/experiments/arrangement_supple/ours/LivingDiningRoom-13177_88_796.jpg}
      %%%%%%%%%%%%%%%%%%%%%%%
      \includegraphics[width=0.33\textwidth]{./figs/experiments/arrangement_supple/ours/LivingDiningRoom-64631_169_017.jpg}%
            \hfill
            \includegraphics[width=0.33\textwidth]{./figs/experiments/arrangement_supple/ours/LivingDiningRoom-64631_169_169.jpg}%
    	\hfill
    	\includegraphics[width=0.33\textwidth]{./figs/experiments/arrangement_supple/ours/LivingDiningRoom-64631_169_346.jpg}
           %%%%%%%%%%%%%%%%%%%%%%%
      \includegraphics[width=0.33\textwidth]{./figs/experiments/arrangement_supple/ours/LivingDiningRoom-79295_184_018.jpg}%
            \hfill
            \includegraphics[width=0.33\textwidth]{./figs/experiments/arrangement_supple/ours/LivingDiningRoom-79295_184_006.jpg}%
    	\hfill
    	\includegraphics[width=0.33\textwidth]{./figs/experiments/arrangement_supple/ours/LivingDiningRoom-79295_184_017.jpg}
		\caption{Ours}
	\end{subfigure}
	\caption{Scene re-arrangements of collections of random objects.  Compared to ATISS, our method generates various object placement options
        with better plausibility.}
    \label{fig:arrangement_supple}
    %\vspace{-4mm}
\end{figure*}

% \begin{figure*}[!ht]
% 	\centering
% 	\begin{subfigure}[t]{0.14\textwidth}
%             % \includegraphics[width=\textwidth]{./figs/experiments/arrangment/noisy/Bedroom-32051_73_000.jpg}
%             \includegraphics[width=\textwidth]{./figs/experiments/arrangment/noisy/LivingDiningRoom-37405_17_004.jpg}
%         \caption{Noisy Scenes}
% 	\end{subfigure}
%         \rulesep
%         \begin{subfigure}[t]{0.41\textwidth}
%     	% \includegraphics[width=0.33\textwidth]{./figs/experiments/arrangment/atiss/Bedroom-32051_73_001.jpg}%
%      %        \hfill
%     	% \includegraphics[width=0.33\textwidth]{./figs/experiments/arrangment/atiss/Bedroom-32051_73_002.jpg}%
%      %        \hfill
%      %  	\includegraphics[width=0.33\textwidth]{./figs/experiments/arrangment/atiss/Bedroom-32051_73_003.jpg} 
%     	\includegraphics[width=0.33\textwidth]{./figs/experiments/arrangment/atiss/LivingDiningRoom-37405_17_005.jpg}%
%             \hfill
%             \includegraphics[width=0.33\textwidth]{./figs/experiments/arrangment/atiss/LivingDiningRoom-37405_17_003.jpg}%
%     	\hfill
%             \includegraphics[width=0.33\textwidth]{./figs/experiments/arrangment/atiss/LivingDiningRoom-37405_17_015.jpg}
%         \caption{ATISS~\cite{paschalidou2021atiss}}
% 	\end{subfigure}
%         \rulesep
% 	\begin{subfigure}[t]{0.41\textwidth}
%     	% \includegraphics[width=0.33\textwidth]{./figs/experiments/arrangment/ours/Bedroom-32051_73_084.jpg}%
%      %        \hfill
%      %        \includegraphics[width=0.33\textwidth]{./figs/experiments/arrangment/ours/Bedroom-32051_73_105.jpg}%
%     	% \hfill
%     	% \includegraphics[width=0.33\textwidth]{./figs/experiments/arrangment/ours/Bedroom-32051_73_150.jpg}
%     	% \includegraphics[width=0.33\textwidth]{./figs/experiments/arrangment/ours/LivingDiningRoom-37405_17_002.jpg}%
%             \hfill
%             \includegraphics[width=0.33\textwidth]{./figs/experiments/arrangment/ours/LivingDiningRoom-37405_17_014.jpg}%
%     	\hfill
%     	\includegraphics[width=0.33\textwidth]{./figs/experiments/arrangment/ours/LivingDiningRoom-37405_17_024.jpg}
% 		\caption{Ours}
% 	\end{subfigure}
% 	\caption{Scene re-arrangements of collections of random objects.  \jiapeng{Compared to ATISS, our diffusion-based method generates various object placement options
%         with better plausibility.}}
%     \label{fig:arrangement}
%     \vspace{-4mm}
% \end{figure*}

\begin{figure*}[!ht]
	\centering
    	\begin{subfigure}[t]{0.23\textwidth}
            \includegraphics[width=\textwidth]{./figs/experiments/arrange_lego/noisy.jpg}
            \caption{Noisy Scenes}
    	\end{subfigure}
        \rulesep
        \begin{subfigure}[t]{0.23\textwidth}
            \includegraphics[width=\textwidth]{./figs/experiments/arrange_lego/atiss.jpg}
            \caption{ATISS~\cite{paschalidou2021atiss}}
	   \end{subfigure}
        \rulesep
    	\begin{subfigure}[t]{0.23\textwidth}
            \includegraphics[width=\textwidth]{./figs/experiments/arrange_lego/lego.jpg}
    		\caption{LEGO~\cite{wei2023lego}}
    	\end{subfigure}
     \rulesep
     \begin{subfigure}[t]{0.23\textwidth}
            \includegraphics[width=\textwidth]{./figs/experiments/arrange_lego/ours.jpg}
    		\caption{Ours}
    	\end{subfigure}
	\caption{Scene re-arrangements of collections of random objects. Compared to ATISS and LEGO, our method generates more favourable object placements with more symmetric pairs.}
    \label{fig:arrangement}
    \vspace{-4mm}
\end{figure*}

\begin{abstract}
In state estimation algorithms that use feature tracks as input, it is customary to assume that the errors in feature track positions are zero-mean Gaussian. Using a combination of calibrated camera intrinsics, ground-truth camera pose, and depth images, it is possible to compute ground-truth positions for feature tracks extracted using an image processing algorithm. We find that feature track errors are not zero-mean Gaussian and that the distribution of errors is conditional on the type of motion, the speed of motion, and the image processing algorithm used to extract the tracks.
\end{abstract}


\section{Introduction}

Many state estimation algorithms assume that measurements are zero-mean Gaussian. This is an explicit assumption in the Kalman Filter and its nonlinear variants \cite{thrun_probabilistic_2005, barrau_invariant_2018} and implicitly built-into the optimization problem of bundle adjustment algorithms \cite{mur-artal_orb-slam:_2015} and outlier-rejection algorithms \cite{civera_1-point_2009}. With extensive calibration with respect to temperature and mechanical alignment, the zero-mean Gaussian assumption is sufficient for the measurements of sensors such as inertial measurement units (IMUs) \cite{vectornav_imu_calibration, tedaldi_robust_2014}, even if it is still not completely true: Extended Kalman Filters (EKFs) that rely on these IMUs are deployed on safety-critical systems actively in use.

Even though several well-known algorithms for Simultaneous Localization and Mapping (SLAM) rely on the often-deployed EKF (e.g. \cite{jones_visual-inertial_2011,Geneva2020ICRA,bloesch_iterated_2017}), SLAM is still an active area of research. The existence of recently-released and actively used research benchmark datasets \cite{hilti_benchmark, tartanair2020iros} indicate that the robotics and computer vision communities still believe that performance of SLAM and an understanding of its failure cases are still insufficient, even after three decades of development \cite{early_slam_tutorial}. This motivates an examination into the fundamental assumptions of SLAM.

This manuscript visits the assumption that feature tracks, the ``measurements" of any indirect visual SLAM algorithm, contain only zero-mean Gaussian error. The covariance of the feature tracks is typically a tuning parameter to for all features at all times. We show that the feature track errors are not zero-mean Gaussian and furthermore, that the errors are conditional on the type of motion, the speed of motion, and the type of feature tracker used to extract the feature tracks. To our knowledge, this is the first study of the mean and covariance of feature tracks \emph{conditional} on the factors that affect them.

The organization of the paper is as follows. Section \ref{sec:feature_track_uq} details the methods. Section \ref{sec:feature_tracker_experiment_details} presents some key figures, and summarizes the error distribution of feature trackers. Section \ref{sec:discussion} ends with some concluding remarks. Additional figures from the experiment are given in the Appendix.



\subsection{Related Work}

\paragraph{Performance of feature detectors and descriptors conditional on nuisances.} The main metric used to evaluate feature detectors is \emph{repeatability} \cite{mikolajczyk_comparison_2005}, or the probability that a feature detector will detect the same feature across multiple images of the same scene under different illuminations and viewpoints. Other metrics are \emph{entropy} \cite{heinly_comparative_2012}, the spread of detected features over an image, and \emph{recall} \cite{aanaes_interesting_2012}, the number of features that are likely ``matchable'' to features in another image of the same scene. On the other hand, the primary metrics used to evaluate feature descriptors are \emph{precision} and \emph{recall}, calculated using pairs of ``matches'' that are found using the descriptor \cite{mikolajczyk_performance_2005}. The evaluation of feature detectors requires multiple images of the same scene. The evaluation of descriptors originally used the same datasets as the evaluation of detectors. To disentangle the problem of detecting features from the evaluation of feature description, two comprehensive datasets of image patches was released in 2017 \cite{balntas_hpatches_2017, maier_ground_2017}. At around the same time, \cite{schonberger_comparative_2017} evaluated both learned and handcrafted feature detectors and descriptors. Of most interest to us are \cite{heinly_comparative_2012}, which used a small dataset containing pure rotation, pure scaling, and illumination changes to evaluate the performance of various detector/descriptor combinations condition on each, \cite{zhao_image_2020}, which extended the datasets used in \cite{heinly_comparative_2012}, and \cite{aanaes_interesting_2012}, which evaluated the performance of feature detectors conditional on change in view angle and lighting condition. Tangentially interesting are \cite{hauagge_image_2012}, which released a dataset of image pairs that are geometrically consistent, but contain large changes in style (e.g. summer vs. winter) and lighting; and \cite{sattler_benchmarking_2018}, which contains groups of image sequences with similar motions, but large outdoor illumination changes.


\paragraph{Learning or Fitting a Covariance Matrix to Feature Tracks.} Early works sought to compute covariance of feature location using information in the RGB image. \cite{kanazawa_we_2001} approximated the covariance with the Hessian of the image centered at the feature point was the covariance of a detected feature -- the idea is that the sharper the curvature given by the Hessian, the more likely a convolutional filter will find the correct location of the feature. \cite{nickels_estimating_2002} contains a sum-of-squared-differences formula for computing feature track covariance. \cite{zeisl_estimation_2009} contains a formula for computing the covariance matrix of SIFT and SURF features. Later on, \cite{sheorey_uncertainty_2015} and \cite{wong_uncertainty_2017} present two methods to model the mean and covariance of Lucas-Kanade feature tracks. With the exception  of \cite{sheorey_uncertainty_2015}, which assumes that the location of a feature track could be a Gaussian Mixture Model, all other models assume that uncertainty is zero-mean Gaussian.



\section{Method}
\label{sec:feature_track_uq}

We wish to characterize the dependence of \textbf{mean error}, \textbf{mean absolute error}, \textbf{covariance}, \textbf{outlier ratio}, and \textbf{feature lifetime} on motion type, speed, tracker type, and when available, lighting. The types of motion investigated are:
\begin{itemize}
\item \textbf{Sideways motion} -- Linear movement with no rotation.
\item \textbf{Fixating motion} -- Moving in a constant radius around a central object. The camera is always pointed directly at the central object, creating some rotation.
\item \textbf{Forwards motion} -- Driving-like motion. The primary change frame-to-frame is scale. Points near the center of an image will stay near the center in subsequent frames.
\item \textbf{AR/VR motion} -- Movement that consists of mostly rotations around a persistent scene.
\end{itemize}
To vary speed, we skip frames at regular intervals from the image sequences. Nominal speed, or a speed of 1.00, means that all frames are used. A speed of 2.00 means that the feature tracker will only see every other frame, and a speed of 3.00 means that the feature tracker will only see one in every three frames. We do not test speeds below 1.00. The exact speeds tested depends on dataset. Finally, we also investigate the effect of two types of feature trackers:
\begin{itemize}
\item \textbf{Lucas-Kanade Sparse Optical Flow} \cite{lucas_iterative_1981}
\item \textbf{Correspondence Tracker} using the SIFT descriptor \cite{lowe_object_1999}. Although computationally expensive, the SIFT descriptor was chosen because of its availability and its performance when used in state estimation tasks \cite{schonberger_comparative_2017}. The descriptor of a feature track is set at the first frame it is detected and never updated.
\end{itemize}

We have chosen \emph{not} to study lens distortion, since this would require multiple similar datasets collected with different cameras. All images in all datasets either have been preprocessed to remove lens distortions, or simulated without lens distortions. Since the Lucas-Kanade tracker is differential, we also choose not to study a differential correspondence tracker that updates the descriptor of a feature track at every frame.



\subsection{Equations}

Consider a feature $i$ that was first detected at time $t^i_0$. If a depth image is available at time $t^i_0$ and $g_{sc}(t^i_0)$ is known, we may fix the feature's position in the spatial frame, $X_s^i$:
\begin{equation}
\begin{aligned}
    X^i_c(t^i_0) &= \pi^{-1}_K(x_p(t^i_0), Z^i_0) \\
    X^i_s &= g_{sc}(t^i_0) \circ X_c(t^i_0) \\
    \label{eq:fixing_Xs}
\end{aligned}
\end{equation}
In the above equation, $Z^i_c(t^i_0)$ is the third coordinate, or depth, of $X^i_c(t^i_0)$. Once, $X_s^i$ is fixed, we can then calculate the \textbf{``ground-truth feature track"} $\bar x_p^i(t)$:
\begin{equation}
    \bar x^i_p(t) = \pi_K(g_{sc}^{-1}(t) \circ X_s^i).
    \label{eq:gt_tracks}
\end{equation}
Some datasets provide a ground-truth point-cloud generated by a single lidar scan rather than a stream of depth images. A lidar scan is a point cloud with $M \sim 10^7$ points in the lidar frame $L$, which is defined as the camera frame at a particular time $t_L$: $\mathbf P_L = \{ P^0_L, P^1_L, \dots, P^M_L \}$. We can calculate the pixel coordinates of each point $j$ in $\mathbf P_L$: 
\begin{equation}
\pi_K(\mathbf P_L) = \{ \pi_K(P^0_L), \pi_K(P^1_L), \dots, \pi_K(P^M_L) \}
\label{eq:laser_scan_proj}
\end{equation}
Feature tracks visible at time $t_L$ can be associated with the nearest point in $\pi_K(\mathbf P_L)$. Suppose the nearest point in $\pi_K(\mathbf P_L)$ to feature $i$ is $P^j_L$. Then, the ground-truth track of feature $i$ is
\begin{equation}
\begin{aligned}
    X^i_s &= g_{sc}(t_L) \circ P^j_L \\
    \bar x^i_p(t) &= \pi_K(g_{sc}^{-1}(t) \circ X^i_s).
    \label{eq:dtu_px_groundtruth}
\end{aligned}
\end{equation}
Once we have a ground-truth feature track for feature $i$, we can calculate the error signal for that feature:
\begin{equation}
    e^i(t) = x_p^i(t) - \bar x_p^i(t)
    \label{eq:px_error_def}
\end{equation}
where $x_p^i(t)$ is the observed track. 


For datasets that provide a ground-truth point cloud at a single frame, the \textbf{mean error at timestep $t$} is
\begin{equation}
    \mu(t) = \frac{1}{M(t)} \sum_{i=1}^{M(t)} e^i(t)
    \label{eq:mean_error_at_time}
\end{equation}
where $M(t)$ is the number of tracked features at time $t$. The \textbf{mean absolute error at timestep $t$}
\begin{equation}
    \kappa(t) = \frac{1}{M(t)} \sum_{i=1}^{M(t)} |e^i(t)|.
    \label{eq:mean_abs_error_at_time}
\end{equation}
Similarly, the \textbf{covariance at timestep $t$} is calculated by
\begin{equation}
    \Sigma(t) = \frac{1}{M(t)-1} \sum_{i=1}^{M(t)} e^i(t) e^i(t)^T.
    \label{eq:cov_at_time}
\end{equation}
It is only possible to compute $\mu(t)$, $\kappa(t)$, and $\Sigma(t)$ for features that are visible at time $t_L$, when the laser scan was acquired.


For datasets that provide a stream of depth images, we use different definitions of mean error, mean absolute error, and covariance. We can also use all features and not just those visible in a particular frame. The \textbf{mean error after $k$ timesteps} is
\begin{equation}
    \nu(k) = \frac{1}{\Psi(k)} \sum_{i=1}^{\Phi(k)} e^i(t^i_0+k\delta_t)
    \label{eq:mean_error_after_timesteps}
\end{equation}
where $\Psi(k)$ is the number of features in the entire dataset tracked for at least $k$ timesteps and $\delta_t$ is the length of each timestep. The \textbf{mean absolute error after $k$ timesteps is:}
\begin{equation}
    \eta(k) = \frac{1}{\Psi(k)} \sum_{i=1}^{\Psi(k)} |e^i(t^i_0+k\delta_t)|
    \label{eq:mean_abs_error_after_timesteps}
\end{equation}
where $\Phi(k)$ is the number of features tracked for at least $k$ timesteps and $\delta_t$ is the length of each timestep. Finally, the \textbf{covariance after $k$ timesteps} is given by
\begin{equation}
    \Phi(k) = \frac{1}{\Psi(k)-1} \sum_{i=1}^{\Psi(k)} e^i(t^i_0+k\delta_t) e^i(t^i_0+k\delta_t)^T.
    \label{eq:cov_after_timesteps}
\end{equation}
When depth data is available at all frames, we define the feature's 3D location at the frame it is first detected and use equations \eqref{eq:mean_error_after_timesteps}, \eqref{eq:mean_abs_error_after_timesteps}, \eqref{eq:cov_after_timesteps}. %

At each frame, a feature tracker will attribute some features in one frame to the features in the previous frame. Let $F(t)$ be the total number of features in the frame at time $t$. The features in each frame will consist of $f_0(t)$ correct attributions, $f_1(t)$ incorrect attributions, and $f_2(t)$ new features, where $f_0(t) + f_1(t) + f_2(t) = F(t)$ and $f_0(t) + f_1(t) \leq F(t-1)$. Outlier rejection algorithms are used to determine $f_0(t)$ and $f_1(t)$ in real-time. The \textbf{outlier ratio} is defined as:
\begin{equation}
\frac{f_1(t)}{F(t-1)}.
\end{equation}

Finally, the \textbf{feature lifetime} of a feature track is the total number of consecutive frames in which it found and successfully attributed. A feature is ``born" at the frame it is first detected and ``dies" if a feature is not found for a single frame.


\section{Experiment Details}
\label{sec:feature_tracker_experiment_details}

\subsection{Feature Tracker Configuration}
\label{sec:feature_tracker_configuration}

We used the feature tracker is the \texttt{Tracker} object integrated with XIVO, our in-house SLAM system. The tracker is configured to use the AGAST corner detector \cite{mair_adaptive_2010}, and to track between 1000 and 1200 features at a time. The AGAST corner detector was chosen for its speed and because it detects a large number of features in most scenes. The feature tracker was configured to track up to 1200 features per scene. We use RANSAC with $p=0.995$ and an error threshold of 3 pixels to reject outliers. More details on the \texttt{Tracker} object and XIVO can be found in Appendix \ref{chapter:about_xivo}.

Since the tracker software was programmed to be part of a larger system and not specifically for these experiments, the implementation of the Correspondence Tracker is not ideal. If a feature is visible in frames 0-5, but is not detected in frame 2, the tracker will drop the feature at frame 2 and initialize a new one in frame 3. This behavior is consistent with the definition of feature lifetime given in the previous section, but is not the ideal implementation for a Correspondence Tracker because there is always a possibility that a corner detector will not find the corner in one frame, or that a descriptor will be just a little too different in one particular frame because of lighting. A more ideal implementation of the Correspondence Tracker would drop frames after a $N_m$ missed frames, where $N_m > 1$ is an experimentally determined number. The definition of feature lifetime would also be changed to accommodate this more complex behavior. As a result of this choice, the distribution of feature lifetimes for the Correspondence Tracker are shorter than they otherwise would be. Furthermore, our experiments will fail to characterize trends that only appear at higher speeds.


\subsection{Dataset-Specific Details}

\paragraph{DTU Point Features Dataset.}

The DTU Point Features Dataset \cite{aanaes_interesting_2012} consists of sixty scenes of fixating motion. In the dataset, one or more objects is placed at the center of stage lit with up to 19 LEDs. A camera is mounted on a robot arm and moved in a precise manner at the stage. At each of 119 fixed locations, the camera acquires an image lit with one of the 19 LEDs, enabling lighting experiments using image-based relighting. The dataset contains a laser scan of the scene at a single frame, called the Key Frame. The original image size is 1600 $\times$ 1200. For speed, we use 800 $\times$ 600 px. grayscale versions of the images instead of the full resolution images.

We make use of the first 49 frames of each scene, or Arc 1 (see Figure \ref{fig:dtu_light_stage}). The Key Frame is Frame 25. We calculate mean error $\mu$, mean absolute error $\kappa$, and covariance $\Sigma$ using equations \eqref{eq:mean_error_at_time}, \eqref{eq:mean_abs_error_at_time}, and \eqref{eq:cov_at_time}. Since 3D data is only available at the Key Frame, calculation of errors and covariances only includes features that exist in Frame 25. Therefore, there is a bias towards longer tracks, as all short tracks that don't exist in Frame 25 are all tossed out. Since the ``ground-truth" position of each feature in 3D is defined by its position in Frame 25, all results will therefore show that Frame 25 has zero covariance and the lowest errors. Statistics on feature lifetime and outlier rejection, however, do include features that do not exist in Frame 25.

To compute the ground-truth location of a feature track, we must associate a feature track to a point in a laser scan point cloud (eq. \eqref{eq:dtu_px_groundtruth}). Since the point cloud does not cover every pixel in the image, associations between features and laser scan points are only made if the pixel value of the laser scan point (eq. \eqref{eq:laser_scan_proj}) is less than 0.25 pixels from the feature.  Associating a pixel to a laser scan point with the incorrect depth measurement will result in a very large calculated means in equation \eqref{eq:mean_error_at_time}. Even with the low 0.25 pixel threshold, this bad association can still happen around edges and corners of objects. So that our analyses do not include very many of these poor depth associations, we throw out feature tracks whose maximum error is greater than the 90th percentile.

Since the DTU Point Features dataset was designed to enable image-based relighting, we also investigated the effects of directional light in addition to speed and the tracker used. We tested the same directional lights as  \cite{aanaes_interesting_2012}. The position of each directional light is shown in Figure \ref{fig:dtu_light_stage}.




\begin{figure}
    \centering
    \includegraphics[width=3.2in]{feature_tracker_uq/annotated_paths_of_interest.png}
    \includegraphics[width=3.2in]{feature_tracker_uq/annotated_light_stage_setup.png}
    \caption{\textbf{An Illustration of the Light Stage Setup in the DTU Point Features Dataset.}  \textbf{Left:} The locations at which images were acquired in the DTU Point Features dataset form three arcs and a linear path. Laser scans of the scenes were collected at the Key Frame (front and center). Frames from Arc 1 (circled in blue) are used for this experiment. \textbf{Right:} Red circles depict the location of 19 physical LEDs used to light the scene, which are spaced out over the scene. At each camera position in the left figure, the authors of the DTU Point Features dataset acquired 19 images. In each image, exactly one of the 19 LEDs is switched on. Acquiring 19 images in each location this way facilitates experiments in lighting using image-based relighting. Diffuse lighting can be simulated by using all 19 photographs from each position equally. More intense directional lighting can be simulated by weighting some LEDs more than others. In our experiments, we vary lighting from back-to-front (BF0-BF7) and left-to-right (LR0-LR9) as the camera follows the motion of Arc 1. Lights LR0 - LR9 and BF0 - BF7 are calculated by using Gaussian-weights on the 19 lights with $\sigma=20$cm; Light LR6 is highlighted in green. Figures are reprinted and annotated with permission.}
    \label{fig:dtu_light_stage}
\end{figure}



\paragraph{KITTI Vision Suite.} The raw data \cite{Geiger2012CVPR} in the KITTI Vision Suite consists RGB, GPS, IMU, and Lidar data captured from a moving vehicle. The motion captured in the images is predominantly forwards. The Lidar data was then processed into a separate benchmark dataset of depth images for single-image depth prediction and depth completion \cite{uhrig_sparsity_2017}. We make use stream \texttt{Image02}. Sequences containing ``still frames" (e.g. significant amount of waiting at a traffic light), are excluded. Excluding sequences containing still frames leaves 28 scenes for our experiments. Although this is fewer scenes than the DTU dataset, it is still more frames because most sequences are longer than 49 frames.

Since 3D data is available at every frame, we define a feature's 3D position using the depth image from the very first frame where it was detected. Therefore, we use mean error $\nu$ (eq. \eqref{eq:mean_error_after_timesteps}), absolute error $\eta$ (eq. \eqref{eq:mean_abs_error_after_timesteps}), and covariance $\Phi$ (eq. \eqref{eq:cov_after_timesteps}). To avoid errors due to bad depth measurements, we throw out the tracks whose maximum L2 error are above the 90th percentile and only calculate $\nu$, $\eta$, and $\Phi$ at timesteps where there are at least 100 features (see Fig. \ref{fig:kitti_avg_feats}).



\paragraph{Simulated Supplementary Data.} For AR/VR motions and sideways motions, we collected simulated RGB-D data in Gazebo. The simulation consisted of a Microsoft Kinect, modified so that RGB and depth data would be co-located, mounted on a Hector quadrotor \cite{hector_quadrotor} in ROS Melodic. The scene consisted of large objects from the Open Source Robotics Foundation's Gazebo Model Library. Images have a resolution of 800 $\times$ 600 pixels. In the subsequent sections, we refer to these datasets as ``Gazebo Linear" and ``Gazebo AR/VR". The AR/VR trajectory used to collect data is shown in Figure \ref{fig:gazebo_arvr_traj}.

In the Gazebo Linear dataset, we throw out tracks whose errors are above the 80-th percentile due to drift that naturally occurs when using the Lucas-Kanade Tracker in an environment containing straight and crisp edges parallel to the direction of motion. More details are given in Figure \ref{fig:gazebo_linear_error_throwout}. In the Gazebo AR/VR dataset, the we throw out tracks whose errors are above the 90-th percentile, as motions are no longer parallel to the straight edges.


\begin{figure}
\centering
\includegraphics[width=0.48\textwidth]{feature_tracker_uq/gazebo_arvr_figs/ARVR_translation_gt.pdf}
\includegraphics[width=0.48\textwidth]{feature_tracker_uq/gazebo_arvr_figs/ARVR_rotation_gt.pdf}
\caption{\textbf{The trajectory generated for the AR/VR scenario.} The commanded trajectory used to collected the AR/VR data was generated from the translation (left) and rotation (right) plotted above. Translation is generated point-to-point using haversines and rotation is generated from slerping.}
\label{fig:gazebo_arvr_traj}
\end{figure}



\subsection{Results}

Overall, we find that mean error, mean absolute error, covariance, feature lifetime, and outlier ratio are all dependent on the type of motion, the tracker used, and the speed. For the DTU Point Features dataset, we found no dependence on the existence of directional light unless the directional light happened to cause tracking failure at high speeds. In Tables \ref{tab:dtu_summary_table} - \ref{tab:gazebo_arvr_summary_table}, we list the exact dependence of mean error, mean absolute error, feature lifetime, covariance, and outlier ratio on each independent variable. Differences in Tables \ref{tab:dtu_summary_table} - \ref{tab:gazebo_arvr_summary_table} lead us to conclude that feature tracks are dependent on motion, tracker, and speed, but not the existence of directional light.

One notable difference between the Lucas-Kanade and Correspondence Trackers is that feature tracks produced by the Lucas-Kanade Tracker drift steadily while the Correspondence Tracker does not. This is because the Lucas-Kanade Tracker is differential, i.e. the characterization of a feature will slightly change frame to frame. For the Correspondence Tracker, this is not true. Therefore, the location of the feature track will drift, and the direction and magnitude of drift is dependent on the direction of motion. With left-to-right fixating motion, drift is positive (see Figure \ref{fig:dtu_diffuse_1.00_meanerror}). With left-to-right linear motion, drift is negative, and also larger (see Figure \ref{fig:gazebo_linear_LK_meanerror}). In AR/VR motion, the direction of drift changes with motion (see Figure \ref{fig:gazebo_arvr_LK_meanerror}). The flipside is that the Lucas-Kanade tracker generates features with a longer lifetime (see Figures \ref{fig:dtu_track_lifetime}, \ref{fig:kitti_feature_lifetime}, \ref{fig:gazebo_linear_feature_lifetime}, \ref{fig:gazebo_arvr_feature_lifetime}). When motion is fixating, the Correspondence Tracker also drifts about the direction of motion (see Figure \ref{fig:dtu_mean_error_sideways}).

Finally, we note that the zero-mean Gaussian assumption holds when motion is predominantly forwards and we are using the Correspondence Tracker (see Figures \ref{fig:kitti_match_meanerror} and \ref{fig:kitti_match_cov}). All figures supporting the assertions in this section are given in the Appendix.






\begin{table}[htp]
    \centering
    \begin{tabular}{p{1in}|p{1.0in}|p{1.0in}|p{2.5in}}
                & \textbf{Tracker} & \textbf{Lighting} & \textbf{Speed} \\
    \hline
    $\mu(t)$ & No (fig. \ref{fig:dtu_diffuse_1.00_meanerror}) & No (figs. \ref{fig:dtu_lighting_mu_LK}, \ref{fig:dtu_lighting_mu_match}) & No (figs. \ref{fig:dtu_match_diffuse_mean_error_varyspeed}, \ref{fig:dtu_LK_mean_varyspeed}) \\
    \hline
    $\kappa(t)$ & Yes (fig. \ref{fig:dtu_diffuse_1.00_MAE_cov}) & No (fig. \ref{fig:dtu_diffuse_1.00_MAE_cov}) & Yes for Correspondence Tracker (fig. \ref{fig:dtu_match_diffuse_MAE_varyspeed}), No for Lucas-Kanade Tracker (fig. \ref{fig:dtu_LK_MAE_varyspeed})\\
    \hline
    $\Sigma(t)$ & Yes (fig. \ref{fig:dtu_diffuse_1.00_MAE_cov}) & No (fig. \ref{fig:dtu_lighting_sigma_LK}, \ref{fig:dtu_lighting_sigma_match}) & Yes for Correspondence Tracker (fig. \ref{fig:dtu_match_diffuse_cov_varyspeed}), No for Lucas-Kanade Tracker (fig. \ref{fig:dtu_LK_cov_varyspeed})\\
    \hline
    Feature Lifetime & Yes (fig. \ref{fig:dtu_track_lifetime}) & No  (fig. \ref{fig:dtu_lighting_feature_lifetimes}) & Yes (fig.  \ref{fig:dtu_active_features}) \\
    \hline
    Outlier Ratio & Yes (figs. \ref{fig:dtu_track_outliers_lights}, \ref{fig:dtu_track_outliers_speed}) & No (fig.  \ref{fig:dtu_track_outliers_lights}) & Yes  (fig. \ref{fig:dtu_track_outliers_speed}) \\
    \end{tabular}
    \caption{\textbf{DTU Point Features Results Summary.} Cells contain whether or not the dependent variables in the left column are affected by the independent variables listed in the top row. Entries also contain figure numbers containing justification. The ``Tracker" and ``Lighting" columns contain references to figures containing plots at nominal speed. Although not indicated in the table, Figures \ref{fig:dtu_speed2.00_percent_outlier} - \ref{fig:dtu_LK_cov_speed12.00} in the Appendix show that the existence of directional lighting continues to not affect outlier ratio, mean error, mean absolute error, and covariance at higher speeds for both the Lucas-Kanade and Correspondence Trackers.}
    \label{tab:dtu_summary_table}
\end{table}



\begin{table}[htp]
    \centering
    \begin{tabular}{p{1in}|p{1.5in}|p{2.50in}}
                & \textbf{Tracker}  & \textbf{Speed} \\
    \hline
    $\nu(t)$ & No (fig. \ref{fig:kitti_1.00_meanerror}) & Yes (figs. \ref{fig:kitti_LK_meanerror}, \ref{fig:kitti_match_meanerror}) \\
    \hline
    $\eta(t)$ & Yes (fig. \ref{fig:kitti_1.00_error_cov}) & No for Correspondence Tracker (fig. \ref{fig:kitti_match_MAE}), Yes for Lucas-Kanade Tracker (figs. \ref{fig:kitti_LK_MAE}) \\
    \hline
    $\Phi(t)$ & Yes (fig. \ref{fig:kitti_1.00_error_cov}) & No for Correspondence Tracker (fig. \ref{fig:kitti_match_cov}), Yes for Lucas-Kanade Tracker (fig. \ref{fig:kitti_LK_cov}) \\ 
    \hline
    Feature Lifetime & Yes (fig. \ref{fig:kitti_feature_lifetime}) & Yes (fig. \ref{fig:kitti_avg_feats}) \\
    \hline
    Outlier Ratio & Yes (fig. \ref{fig:kitti_outlier_ratio}) & Yes (fig. \ref{fig:kitti_outlier_ratio})\\
    \end{tabular}
    \caption{\textbf{KITTI Results Summary.} Cells contain whether or not the dependent variables in the left column are affected by the independent variables listed in the top row. Entries also contain figure numbers containing justification.}
    \label{tab:kitti_summary_table}
\end{table}



\begin{table}[htp]
    \centering
    \begin{tabular}{p{1in}|p{1.5in}|p{2.5in}}
                & \textbf{Tracker}  & \textbf{Speed} \\
    \hline
    $\nu(t)$ & Yes (fig. \ref{fig:gazebo_linear_1.00_meanerror}) & No for Correspondence Tracker (fig. \ref{fig:gazebo_linear_match_meanerror}), Yes for Lucas-Kanade Tracker   (fig. \ref{fig:gazebo_linear_LK_meanerror}) \\
    \hline
    $\eta(t)$ & Yes (fig. \ref{fig:gazebo_linear_1.00_error_cov}) & Yes (figs. \ref{fig:gazebo_linear_LK_MAE}, \ref{fig:gazebo_linear_match_MAE}) \\
    \hline
    $\Phi(t)$ & Yes (fig. \ref{fig:gazebo_linear_1.00_error_cov}) & Yes (figs.  \ref{fig:gazebo_linear_match_cov}, \ref{fig:gazebo_linear_LK_cov}) \\ 
    \hline
    Feature Lifetime & Yes (fig. \ref{fig:gazebo_linear_feature_lifetime}) & Yes (fig. \ref{fig:gazebo_linear_avg_feats}) \\
    \hline
    Outlier Ratio & Yes (fig. \ref{fig:gazebo_linear_outlier_ratio}) &  No for Correspondence Tracker, Yes for Lucas-Kanade Tracker (fig. \ref{fig:gazebo_linear_outlier_ratio})\\
    \end{tabular}
    \caption{\textbf{Gazebo Linear Results Summary.} Cells contain whether or not the dependent variables in the left column are affected by the independent variables listed in the top row. Entries also contain figure numbers containing justification.}
    \label{tab:gazebo_linear_summary_table}
\end{table}


\begin{table}[htp]
    \centering
    \begin{tabular}{p{1in}|p{1.5in}|p{2.5in}}
                & \textbf{Tracker}  & \textbf{Speed} \\
    \hline
    $\nu(t)$ & Yes (fig. \ref{fig:gazebo_arvr_1.00_meanerror}) & No (fig. \ref{fig:gazebo_arvr_LK_meanerror}, \ref{fig:gazebo_arvr_match_meanerror}) \\
    \hline
    $\eta(t)$ & Yes (fig. \ref{fig:gazebo_arvr_1.00_error_cov}) & Yes for Correspondence   Tracker (fig. \ref{fig:gazebo_arvr_match_MAE}), No for Lucas-Kanade  Tracker (fig. \ref{fig:gazebo_arvr_LK_MAE})  \\
    \hline
    $\Phi(t)$ & Yes (fig. \ref{fig:gazebo_arvr_1.00_error_cov}) &  Yes for Correspondence   Tracker (fig. \ref{fig:gazebo_arvr_match_cov}), No for Lucas-Kanade Tracker (fig. \ref{fig:gazebo_arvr_LK_cov})  \\ 
    \hline
    Feature Lifetime & Yes (fig. \ref{fig:gazebo_arvr_feature_lifetime}) & Yes (fig. \ref{fig:gazebo_arvr_avg_feats}) \\
    \hline
    Outlier Ratio & Yes (fig. \ref{fig:gazebo_arvr_outlier_ratio}) & No for Correspondence Tracker, Yes for Lucas-Kanade Tracker (fig. \ref{fig:gazebo_arvr_outlier_ratio})\\
    \end{tabular}
    \caption{\textbf{Gazebo AR/VR Results Summary.} Cells contain whether or not the dependent variables in the left column are affected by the independent variables listed in the top row. Entries also contain figure numbers containing justification.}
    \label{tab:gazebo_arvr_summary_table}
\end{table}


\section{Discussion}
\label{sec:discussion}

Other than the caveat about the Correspondence Tracker noted in Section \ref{sec:feature_tracker_configuration}, the main limitation of this work is that there are more variables we could have tested, but chose not to. Examples of variables we chose not to test are the choice of feature detector and descriptor, and characteristics in the scene. For example, would the Correspondence Tracker have as little drift when moving forwards in an indoor environment and comparing BRIEF descriptors? Testing for conditionality on more variables inevitably leads to an unmanageable experiment, so we chose to lock in the feature detector and descriptor to well-performing available options and let the dataset dictate available scenes. Nevertheless, our work is a first step in characterizing the dependence of mean error, mean absolute error, covariance, feature lifetime, and outlier ratio on motion, tracker, speed, and the existence of directional lighting. The main conclusion is that the common zero-mean Gaussian assumption is rarely true. This conclusion motivates a few areas of future work.

The most immediate direction of future work is to continue to use the Extended Kalman Filter and dynamically adapt filter parameters, such as covariance estimates and the number of tracked features, to the scene. Since feature tracks are not zero-mean, covariance estimates will have to be enlarged so that feature tracks containing the extra bias are not outliers. Machine learning approaches to adapting the covariance already exist \cite{vega-brown_cello_2013, liu_deep_2018}. Since statistical methods are not often desirable in safety critical systems, it is of interest to compare performance when covariance is adjusted by a learned model to when covariance is adjusted by a finite state machine. While this approach is the most immediate, it does not address the fact that it brings no convergence guarantees in a downstream state estimation process and will therefore require extensive testing for each application.

The second area of future work is to adapt existing state estimation algorithms to accommodate feature tracks that are not zero-mean Gaussian. It may not be possible, however, to design a filter that is both computationally tractable, guaranteed to converge, and simple enough to implement on a complex, realistic system. This motivates the study in the next chapter, and the third area of future work.

The third direction of future work is to adjust individual feature tracks \emph{before} they are used by a state estimation algorithm that assumes that measurements are zero-mean Gaussian. This is the approach used for IMUs: errors in IMUs measurements are primarily dependent on temperature and mechanical alignment errors, so IMU measurements are adjusted for temperature and known mechanical misalignments before they are passed to a downstream computer. For feature tracks, the calibration table would be more complex, as it is dependent on speed, motion type, and the type of tracker used.



We visualize additional qualitative comparisons on the task of scene arrangement in Fig.~\ref{fig:arrangement_supple}. Also, the quantitative results are shown in Tab.~\ref{tab:arrange}. 

\paragraph{Text-conditioned Scene Synthesis}
\begin{figure*}[!htbp]
	\centering
	\begin{subfigure}[t]{0.24\textwidth}
            \includegraphics[width=\textwidth]{././figs/experiments/text2scene_supple/LivingDiningRoom-107_text.jpg}
            \vspace{2mm}
            \includegraphics[width=\textwidth]{././figs/experiments/text2scene_supple/LivingDiningRoom-1744_text.jpg}
            \vspace{2mm}
            \includegraphics[width=\textwidth]{././figs/experiments/text2scene_supple/LivingDiningRoom-2106_text.jpg}
            \vspace{2mm}
            \includegraphics[width=\textwidth]{././figs/experiments/text2scene_supple/LivingDiningRoom-3483_text.jpg}
        \caption{Input text}
	\end{subfigure}%
        \hfill
 	\begin{subfigure}[t]{0.215\textwidth}
            \includegraphics[width=\textwidth]{././figs/experiments/text2scene_supple/LivingDiningRoom-107_41_041_gt.jpg}
            \includegraphics[width=\textwidth]{././figs/experiments/text2scene_supple/LivingDiningRoom-1744_61_061_gt.jpg}
            \includegraphics[width=\textwidth]{././figs/experiments/text2scene_supple/LivingDiningRoom-2106_83_083_gt.jpg}
            \includegraphics[width=\textwidth]{././figs/experiments/text2scene_supple/LivingDiningRoom-3483_163_163_gt.jpg}
        \caption{Reference}
	\end{subfigure}%
        \hfill
 	\begin{subfigure}[t]{0.215\textwidth}
            \includegraphics[width=\textwidth]{././figs/experiments/text2scene_supple/LivingDiningRoom-107_41_041_atiss.jpg}
            \includegraphics[width=\textwidth]{././figs/experiments/text2scene_supple/LivingDiningRoom-1744_61_445_atiss.jpg}
            \includegraphics[width=\textwidth]{././figs/experiments/text2scene_supple/LivingDiningRoom-2106_83_467_atiss.jpg}
            \includegraphics[width=\textwidth]{././figs/experiments/text2scene_supple/LivingDiningRoom-3483_163_355_atiss.jpg}
        \caption{ATISS~\cite{paschalidou2021atiss}}
	\end{subfigure}%
        \hfill
 	\begin{subfigure}[t]{0.215\textwidth}
            \includegraphics[width=\textwidth]{././figs/experiments/text2scene_supple/LivingDiningRoom-107_41_018_ours.jpg}
            \includegraphics[width=\textwidth]{././figs/experiments/text2scene_supple/LivingDiningRoom-1744_61_002_ours.jpg}
            \includegraphics[width=\textwidth]{././figs/experiments/text2scene_supple/LivingDiningRoom-2106_83_003_ours.jpg}
            \includegraphics[width=\textwidth]{././figs/experiments/text2scene_supple/LivingDiningRoom-3483_163_010_ours.jpg}
        \caption{Ours}
	\end{subfigure}
	\caption{\textbf{Text-conditioned scene synthesis}. The input text describes only a partial scene configuration. Our method generates more plausible scenes matched with the texts.}
    \label{fig:text2scene_supple}
    %\vspace{-6mm}
\end{figure*}

We provide additional qualitative comparisons on the text-conditioned scene synthesis in Fig.~\ref{fig:text2scene_supple}. 
As observed, in the first and third rows, ATISS has object intersection issues while ours does not. In the second row, our method can correctly generate a corner side table on the left of the armchair. However, ATISS generates a corner side table on the right of the armchair.
 In the fourth row, our method can generate four dining chairs that are consistent with the text description, but ATISS can only generate two dining chairs.
The quantitative results evaluated by FID, KID, and SCA are reported in Tab.~\ref{tab:text}. Our method consistently outperforms ATISS in all used metrics.

\section{User Study}
\label{SecUser}

We conducted a perceptual user study to evaluate the quality of our method against ATISS on the application of text-conditioned scene synthesis.
As shown in Fig.~\ref{fig:user_study}, we provide the visualization of a ground-truth scene used to generate a text prompt as a reference. For each pair of results, a user needs to answer ``which of the generated scene can better match the text prompt?" and ``Which of the generated scene is more reasonable and realistic?".
%needs to decide which of the generated scene can better match the text prompt and judge which of the synthesized scene is more plausibly realistic than the other.
%\TODO{what is plausibly realistic? write down the question that you asked in the study}
We collect the answers of 225 scenes from 45 users and calculate the statistics. 62$\%$ of the user answers prefer our method to ATISS in realism.  55$\%$ of answers think our method is more consistent with the text prompt.

\begin{figure*}[!htbp]
    \centering
    \includegraphics[width=\linewidth]{./figs/experiments/user_study/question_reference.jpg}
    
    \includegraphics[width=\linewidth]{./figs/experiments/user_study/question_match.jpg}

    \includegraphics[width=\linewidth]{./figs/experiments/user_study/question_realism.jpg}

    \caption{\textbf{User Study UI}. Based on the reference scene used to generate text prompts, users are asked which of the synthesized scene is more matched with the text prompt and more realistic. Note that the results from ATISS and our method are randomly shuffled to avoid bias.}
    \label{fig:user_study}
    
\end{figure*}


\end{document}
