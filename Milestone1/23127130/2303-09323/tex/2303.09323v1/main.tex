%%%%%%%%%%%%%%%%%%%%%%%%%%%%%%%%%%%%%%%%%%%%%%%%%%%%%%%%%%%%%%%%%%%%%
%%                                                                 %%
%% Please do not use \input{...} to include other tex files.       %%
%% Submit your LaTeX manuscript as one .tex document.              %%
%%                                                                 %%
%% All additional figures and files should be attached             %%
%% separately and not embedded in the \TeX\ document itself.       %%
%%                                                                 %%
%%%%%%%%%%%%%%%%%%%%%%%%%%%%%%%%%%%%%%%%%%%%%%%%%%%%%%%%%%%%%%%%%%%%%

%%\documentclass[referee,sn-basic]{sn-jnl}% referee option is meant for double line spacing

%%=======================================================%%
%% to print line numbers in the margin use lineno option %%
%%=======================================================%%

%%\documentclass[lineno,sn-basic]{sn-jnl}% Basic Springer Nature Reference Style/Chemistry Reference Style

%%======================================================%%
%% to compile with pdflatex/xelatex use pdflatex option %%
%%======================================================%%

%%\documentclass[pdflatex,sn-basic]{sn-jnl}% Basic Springer Nature Reference Style/Chemistry Reference Style

%%\documentclass[sn-basic]{sn-jnl}% Basic Springer Nature Reference Style/Chemistry Reference Style
\documentclass[sn-mathphys,iicol]{sn-jnl}% Math and Physical Sciences Reference Style
%\documentclass[sn-mathphys]{sn-jnl}% Math and Physical Sciences Reference Style
%%\documentclass[sn-aps]{sn-jnl}% American Physical Society (APS) Reference Style
%%\documentclass[sn-vancouver]{sn-jnl}% Vancouver Reference Style
%%\documentclass[sn-apa]{sn-jnl}% APA Reference Style
%%\documentclass[sn-chicago]{sn-jnl}% Chicago-based Humanities Reference Style
%%\documentclass[sn-standardnature]{sn-jnl}% Standard Nature Portfolio Reference Style
%%\documentclass[default]{sn-jnl}% Default
%%\documentclass[default,iicol]{sn-jnl}% Default with double column layout

%%%% Standard Packages
%%<additional latex packages if required can be included here>
%%%%

%%%%%=============================================================================%%%%
%%%%  Remarks: This template is provided to aid authors with the preparation
%%%%  of original research articles intended for submission to journals published 
%%%%  by Springer Nature. The guidance has been prepared in partnership with 
%%%%  production teams to conform to Springer Nature technical requirements. 
%%%%  Editorial and presentation requirements differ among journal portfolios and 
%%%%  research disciplines. You may find sections in this template are irrelevant 
%%%%  to your work and are empowered to omit any such section if allowed by the 
%%%%  journal you intend to submit to. The submission guidelines and policies 
%%%%  of the journal take precedence. A detailed User Manual is available in the 
%%%%  template package for technical guidance.
%%%%%=============================================================================%%%%

\usepackage{tabularx}
\usepackage{subfig}
\usepackage{float}
\usepackage{multirow}
\usepackage{comment}
\usepackage{amsmath}
\usepackage{fancyref}
\usepackage{booktabs}

%dashed line
\usepackage{array}
\usepackage{arydshln}
\setlength\dashlinedash{0.2pt}
\setlength\dashlinegap{1.5pt}
\setlength\arrayrulewidth{0.3pt}

%Widows & Orphans & Penalties

\widowpenalty500
\clubpenalty500
\clubpenalty=9996
\exhyphenpenalty=50 %for line-breaking at an explicit hyphen
\brokenpenalty=4991
\predisplaypenalty=10000
\postdisplaypenalty=1549
\displaywidowpenalty=1602
\floatingpenalty = 20000




\makeatletter
\renewcommand*\env@matrix[1][\arraystretch]{%
  \edef\arraystretch{#1}%
  \hskip -\arraycolsep
  \let\@ifnextchar\new@ifnextchar
  \array{*\c@MaxMatrixCols c}}
\makeatother


\jyear{2023}%

%% as per the requirement new theorem styles can be included as shown below
\theoremstyle{thmstyleone}%
\newtheorem{theorem}{Theorem}%  meant for continuous numbers
%%\newtheorem{theorem}{Theorem}[section]% meant for sectionwise numbers
%% optional argument [theorem] produces theorem numbering sequence instead of independent numbers for Proposition
\newtheorem{proposition}[theorem]{Proposition}% 
%%\newtheorem{proposition}{Proposition}% to get separate numbers for theorem and proposition etc.

\theoremstyle{thmstyletwo}%
\newtheorem{example}{Example}%
\newtheorem{remark}{Remark}%

\theoremstyle{thmstylethree}%
\newtheorem{definition}{Definition}%

\raggedbottom
%%\unnumbered% uncomment this for unnumbered level heads

\begin{document}

\title{Stock Trend Prediction: A Semantic Segmentation Approach}
%Stock Trend Prediction: Fully Convolutional Network for Semantic Segmentation of Stock Prices Future}% Stock market movement forecast: A Semantic Segmentation Approach}
%
%%=============================================================%%
%% Prefix	-> \pfx{Dr}
%% GivenName	-> \fnm{Joergen W.}
%% Particle	-> \spfx{van der} -> surname prefix
%% FamilyName	-> \sur{Ploeg}
%% Suffix	-> \sfx{IV}
%% NatureName	-> \tanm{Poet Laureate} -> Title after name
%% Degrees	-> \dgr{MSc, PhD}
%% \author*[1,2]{\pfx{Dr} \fnm{Joergen W.} \spfx{van der} \sur{Ploeg} \sfx{IV} \tanm{Poet Laureate} 
%%                 \dgr{MSc, PhD}}\email{iauthor@gmail.com}
%%=============================================================%%

\author[1]{\fnm{Shima} \sur{Nabiee}}\email{snabiee@uci.edu}

\author[1]{\fnm{Nader} \sur{Bagherzadeh}}\email{nader@uci.edu}
%\equalcont{These authors contributed equally to this work.}


\affil[1]{\orgdiv{Department of Electrical Engineering and Computer Science}, \orgname{University of California in Irvine}, \orgaddress{\city{Irvine}, \postcode{92697}, \state{California}, \country{USA}}}


%%==================================%%
%% sample for unstructured abstract %%
%%==================================%%

\abstract{Market financial forecasting is a trending area in deep learning. Deep learning models are capable of tackling the classic challenges in stock market data, such as its extremely complicated dynamics as well as long-term temporal correlation. To capture the temporal relationship among these time-series data, recurrent neural networks are employed. However, it is difficult for recurrent models to learn to keep track of long-term information.  Convolutional Neural Networks (CNN) have been utilized to better capture the dynamics and extract features for both short-term and long-term forecasting. However, semantic segmentation and its well-designed fully convolutional networks have never been studied for time-series dense classification. We present a novel approach to predict long-term daily stock price change trends with fully 2D-convolutional encoder-decoders. We generate input frames with daily open, high, low, and close prices for a time-frame of T days. The aim is to predict future trends by pixel-wise classification of the current price frame. We propose a hierarchical CNN structure to encode multiple price frames to multiscale latent representation in parallel using Atrous Spatial Pyramid Pooling (ASPP) blocks and take that temporal coarse feature stacks into account in the decoding stages. Our hierarchical structure of CNNs make it capable of capturing both long and short-term temporal relationships effectively. The effect of increasing the input time horizon via incrementing parallel encoders has been studied with interesting and substantial changes in the output semantic segmentation maps. We achieve overall accuracy and AUC of \%78.18 and 0.88 for joint trend prediction over the next 20 days, surpassing other semantic segmentation approaches. One variation of the proposed framework results in \%83.19 in accuracy for four prices trends predictions of the next day, and the other variation achieves \%88.14 for the tenth day, the highest in a 20-day time-frame. Finally, we compared our proposed model with several deep learning models that were specifically designed for technical analysis and found that for different output horizons, our proposed model variations outperformed the other models.}

%%================================%%
%% Sample for structured abstract %%
%%================================%%

% \abstract{\textbf{Purpose:} The abstract serves both as a general introduction to the topic and as a brief, non-technical summary of the main results and their implications. The abstract must not include subheadings (unless expressly permitted in the journal's Instructions to Authors), equations or citations. As a guide the abstract should not exceed 200 words. Most journals do not set a hard limit however authors are advised to check the author instructions for the journal they are submitting to.
% 
% \textbf{Methods:} The abstract serves both as a general introduction to the topic and as a brief, non-technical summary of the main results and their implications. The abstract must not include subheadings (unless expressly permitted in the journal's Instructions to Authors), equations or citations. As a guide the abstract should not exceed 200 words. Most journals do not set a hard limit however authors are advised to check the author instructions for the journal they are submitting to.
% 
% \textbf{Results:} The abstract serves both as a general introduction to the topic and as a brief, non-technical summary of the main results and their implications. The abstract must not include subheadings (unless expressly permitted in the journal's Instructions to Authors), equations or citations. As a guide the abstract should not exceed 200 words. Most journals do not set a hard limit however authors are advised to check the author instructions for the journal they are submitting to.
% 
% \textbf{Conclusion:} The abstract serves both as a general introduction to the topic and as a brief, non-technical summary of the main results and their implications. The abstract must not include subheadings (unless expressly permitted in the journal's Instructions to Authors), equations or citations. As a guide the abstract should not exceed 200 words. Most journals do not set a hard limit however authors are advised to check the author instructions for the journal they are submitting to.}

\keywords{Time Series Classification, Fully Convolutional Networks, Semantic Segmentation, Stock Trend Prediction}
%%\pacs[JEL Classification]{D8, H51}

%%\pacs[MSC Classification]{35A01, 65L10, 65L12, 65L20, 65L70}

\maketitle

\section{Introduction}
% Importance and appeal of children's drawings
Children's depictions of the human figure are highly expressive and varied.
As one of the very first subjects children attempt to draw, the representation begins as an almost unintelligible cloud of scribbles. 
As the child grows, their representation of the human figure becomes more developed and is extended to graphically represent many different types of characters: people, animals, and even personified objects (see Figure 1).

Who among us has not wished, either as a child or as an adult, to see such figures come to life and move around on the page?
Sadly, while it is relatively fast to produce a single drawing, creating the sequence of images necessary for animation is a much more tedious endeavor, requiring discipline, skill, patience, and sometimes complicated software.
As a result, most of these figures remain static upon the page.

% We built a system to animate them.
Inspired by the importance and appeal of the drawn human figure, we design and build a system to automatically animate it given an in-the-wild photograph of a child's drawing. 
Our system is fast, intuitive, and robust to much of the variation present in these types of drawings, making it well-suited to allow our target audience--children--to see their own characters coming to life.
The system is comprised of four stages: figure detection, segmentation masking, pose estimation/rigging, and animation. 
We describe each stage and identify common causes of failure in each. 
For object detection and pose estimation, we make use of existing computer vision models designed to detect human figures and joints in photographs; we fine-tune these models for use with children's drawings.
For segmentation, we present a straightforward, image processing-based method that, for animation purposes, is more useful and accurate than segmentation masks obtained from a fine-tuned object detection model.
During the animation step, we take advantage of the \textit{twisted perspective} commonly seen in children’s drawings to retarget motion capture data onto the character in a novel and appealing way.

% We use existing machine learning models. However, given the wide domain gap it's not clear how much fine-tuning data was needed. So we ran some experiments to find out and report it.
While our system leverages existing models and techniques, most are not directly applicable to the task due to the many differences between photographic images and simple pen and paper representations. 
To this end, we couple the presentation of our system with a set of experiments exploring the relationship between fine-tuning training set size and success rates.
We also include a perceptual study validating viewer preference for incorporating \textit{twisted perspective} into the motion retargeting step.

We validate the desirability and appeal of our system by building and publicly releasing a version of it as the \AD Demo \,\cite{animateddrawings}.
Launched in December 2021, this demo has been used by millions of people around the world to animate their children's drawings.
Inspired by this reception, our second contribution is The Amateur Drawings Dataset: \hjs{180,000 drawings and user-accepted annotations collected, with consent, through the demo. See Section \ref{sec:UI} for a description of how the annotations were generated.}
We believe this dataset will be a resource to researchers from various fields seeking to better understand the space of amateur drawings, evaluate new algorithms in this domain, or develop new drawing-based tools in general.

To summarize, our contributions are as follows:
\begin{enumerate}
    \item 
    We explore the problem of automatic sketch-to-animation for children's drawings of human figures and present a framework that achieves this effect. We also present a set of experiments determining the amount of training data necessary to achieve high levels of success and a perceptual study validating the usefulness of our motion retargeting technique.
    \item To encourage additional research in the domain of amateur drawings, we present a first-of-its-kind dataset of 180,000 user-submitted amateur drawings, along with user-accepted bounding box, segmentation mask, and joint location annotations.
\end{enumerate}

Upon acceptance of this paper, we plan to publicly release the Amateur Drawings Dataset, project code, and fine-tuned model weights.


\section{Related Works}
\subsection{Deep Learning in Financial Time-series Trend Prediction}
The scientific community generally utilizes two ways for stock market prediction \cite{puneeth2021comparative}. The first approach is the fundamental analysis, where underlying internal and external factors that affect the value of a stock or a company are used as predictive attributes. These factors include the company’s financial performance, social and political behavior, and economic data \cite{beyaz}. The second one is the technical analysis, where the predictive attributes are mainly historical prices and volumes. This method focuses on an analysis of trends in securities’ prices such as daily opening, high, low, and closing prices. Technical analysis is the most common approach in the literature \cite{nazario2017literature}. Since all the new information, like news and macroeconomic variables, are already represented in stock prices, technical analysts believe market price movements tell everything. Therefore, their strategies are based on the stock prices and technical indicators such as relative strength index (RSI) and moving average \cite{nti2020systematic}.\\

Many machine learning \cite{knn, svm1, svm2, ml1} and deep learning \cite{hu2021survey, jiang2021applications, thakkar2021comprehensive} approaches have been proposed to analyze financial time-series for market prediction. The machine learning approaches usually have limited interpretability, need manual feature selection, and perform weakly for very complex tasks. This encourages the integration of deep learning-based models to enhance stock market predictions. The complex intrinsic patterns of stock price trends can be studied using such models by extracting essential characteristics of highly unstructured financial data. Among the deep learning models, Deep ANN and MLP, RNN, LSTM, and CNN have been the dominant models \cite{sezer2020financial}.\\

Deep Artificial Neural Networks and Multi-Layer Perceptron (MLP) have been shown to have superior performance over traditional models \cite{Prime2020ForecastingTC}. \cite{144} used a deep ANN and open, close, high, and low daily prices of the last 10 days of index data. In addition, MLP and ANN were used for the prediction of index data. \cite{142} created an ensemble network of several deep ANN models for trend prediction.  To better handle temporal data, RNN and LSTM are provided as an enhancement of feed-forward neural networks. \cite{jarrah2019recurrent} applied discrete wavelet transform on stock price time-series followed by a deep RNN to predict the closing price of the next 7 days. \cite{186} compared 3 different recurrent models, namely, vanilla RNN, LSTM, and GRU, to predict the movement of stock prices. \cite{133} used LSTM to predict the trend of stock prices and compared the direction of change classification performance with classical time-series forecasting techniques.\\


%%%%%%%%%%%%%%%%%%%%%%%%%%%%%%%%%%%%%%%%%%%%%%%%%%%%%%%%%%

\subsection{Deep Fully Convolutional Networks}

The idea of dismantling fully connected layers from convolutional layers initially studied in \cite{fcn}, proposing the FCN. The primary objective was to create semantic segmentation map by adapting image classification networks such as AlexNet \cite{alexnet} and VGG \cite{vgg} into fully convolutional networks. The resulting network is considered revolutionary in several aspects. Due to the fully convolutional nature, inference was seen to be considerably faster. More importantly, segmentation maps are allowed to be generated for images with any resolution. And most importantly, they proposed the skip architecture for deep convolutional networks. Deep convolutional networks commonly create feature hierarchies by down-sampling feature maps. Skip connections are used preceding the down-sampling layers to preserve and forward this information to deeper layers allowing information to flow, which would otherwise be lost. This idea to use skip connections in deep convolutional networks eventually evolved into the encoder-decoder structures \cite{segnet, unet} for semantic segmentation. Numerous fully convolutional networks are proposed in this domain, among which we are interested in two types of them that are designed to better capture multi-scale context. \\


A successful encoder-decoder network application is often found in computer vision tasks, such as human pose estimation and object detection. The encoder-decoder network usually consists of two parts: an encoder that gradually reduces the feature maps, while also capturing semantic information, and a decoder that gradually recovers the object details and spatial information. DeconvNet \cite{deconv}, Seg-Net \cite{segnet}, and U-Net \cite{unet} are very well-known examples. They comprise two parts, a contracting path to capture context, and a symmetric expanding path that enables precise localization. However, the highly correlated semantic information, which is provided by the adjacent lower resolution feature map of the encoder, must pass through multiple intermediate layers in order to reach the same decoder stage. This often results in a level of information decay. U-Net overcame this short-coming by utilizing a symmetric skip architecture to provide links between the same-level encoder and decoder stages. Since then, U-Net has been the backbone of many networks for the segmentation of different applications, such as medical images, street view images, satellite images, just to name a few. \\

SegNet utilizes only pooling indices in each stage of the encoder to perform nonlinear up-sampling of the corresponding decoder stage. An issue with encoder-decoder based models is that some of the finer details of an image may be lost due to being part of the resolution that gets lost during encoding. HRNet \cite{hrnet} avoids losing such exact information by connecting the high-to-low resolution convolution streams in a parallel manner and frequently exchanging information between resolutions. They incorporated the multi-scale feature maps in the output node to prevent losing objects at different scales. In fact, many multi-scale models have been proposed. \\


Except using multi-scale features in different stages, many models proposed to also consider multi-scale feature extraction at each stage. Feature Pyramid Network \cite{lin2017feature} is a well-known multiscale analysis model which has been used in different neural network architectures, primarily for object detection but it has also been applied to segmentation. Essentially, the pyramidal hierarchy of deep CNNs is harnessed to create feature pyramids with minimal additional cost. At the same time, to integrate both high- and low- resolution features, the FPN consists of bottom-up and top-down pathways with lateral connections. The spatial pyramid pooling layer adopts parallel convolutional layers with different kernel sizes, then joins the pooled feature maps to fuse feature maps at multiple scales. SPP can effectively increase the extraction range of the backbone features, significantly separate the essential contextual features. PSPNet \cite{zhao2017pyramid} applies SPP at several grid scales including image-level pooling. Even though rich semantic information is encoded in the last feature map, detailed information related to object boundaries is missing due to the pooling or convolutions with striding operations within the network backbone. This could be alleviated by applying the atrous convolution to extract denser feature maps. \\


Deeplab V2 \cite{deeplabv2} proposed Atrous Spatial Pyramid Pooling as the feature extraction modules followed by fully connected CRFs. ASPP combines atrous convolutions with different dilation rates in parallel, which can provide multiscale denser contextual information, a larger receptive field, and more local features. Subsequently, \cite{deeplabv3} proposed to omit the CRFs and proposed a deep convolutional network with cascaded and parallel modules of atrous convolutions. And eventually, DeepLabV3+ \cite{deeplabv3+} upgraded the previous design with an encoder-decoder architecture in conjunction with ASPP blocks or Xception \cite{xception} modules. \\


















\section{Dataset}

\paragraph{Rescaling} In order to build the input price matrixes, a rescaling of the real observations of the time series stock price is needed so that it falls into a small specific interval. Let $X = {x_1,x_2,...,x_n}$ be the considered time series with $n$ components, the rescaling to the interval $[0, 1]$ is achieved by scaling the maximum value of each time series to unit size.
\begin{equation}
    {\tilde{x_i}} =\frac{x_i - min(X)}{max(X) - min(X)}
\end{equation}
Hence, the scaled series is represented by $\tilde{X} = \{\tilde{x_1}, \tilde{x_2}, ..., \tilde{x_n}\}$


\paragraph{Historical Price Image} For a specific stock, the input frames should contain the lowest, opening, highest, and closing prices for T consecutive trading days. Each row has the rescaled historical prices for a day. The following demonstrates the input matrix, or, price frames.\\




\includegraphics[scale=0.22]{in_mat.png} 

\iffalse

\NiceMatrixOptions{
%code-for-first-row = \color{blue} ,
%code-for-last-row = \color{blue} ,
code-for-first-col = \color{black}
%code-for-last-col = \color{blue}
}


\[
\begin{bNiceMatrix}[first-col]%[vlines,first-row,last-row,first-col,last-col]

\textbf{day t-T}     & Open_{t-T}  & Low_{t-T} & High_{t-T} & Close_{t-T}  \\
\Vdots&\Vdots&\Vdots&\Vdots&\Vdots\\
\\
\\
\textbf{day t-1}       & Open_{t-1}  & Low_{t-1} & High_{t-1} & Close_{t-1} \\
\\
\textbf{day t  }       & Open_{t \ \ \ }  & Low_{t \ \ \ } & High_{t \ \ \ } & Close_{t \ \ \ }  \\

\CodeAfter
  \begin{tikzpicture}
  \node [draw=blue, rounded corners=4pt, inner ysep = 2pt, 
       rotate fit=0, fit = (1-1) (1-4) ] {} ;
  \node [draw=blue, rounded corners=4pt, inner ysep = 2pt,
       rotate fit=0, fit = (5-1) (5-4) ] {} ;
  \node [draw=blue, rounded corners=4pt, inner ysep = 2pt,
       rotate fit=0, fit = (7-1) (7-4) ] {} ;
  \end{tikzpicture}
  
 
\end{bNiceMatrix}

\]

\fi









\paragraph{Segmentation Mask} To create the annotated labels, we compare the historical price frame at time frame $n+1$ with the one at time frame $n$. More specifically, each component of the stock price matrix for the next time frame is compared with the one from the last time frame, and each pixel is assigned with zero or one according to the following criteria:

$$
{y^{n+1}_{t,c}} = 
    \begin{cases}
    1, & \qquad {X^{n+T}_{t,c}} > {X^n_{t,c}} \\
    0, & \qquad o.w.
    
    \end{cases}
$$

Where ${y^{n+1}_{t,c}}$ is the stock price trend for the next time frame at row $t$ and column $c$. Resulting matrices are illustrated in Fig. \ref{datamat}.\



\begin{figure}[H]
    \centering
    \includegraphics[width=\linewidth]{matrix2.png}
     \caption{Stock price frames and corresponding output trend segmentation map}
     \label{datamat}
\end{figure}


\section{Methodology}
\section{Experimental setup and data collection}\label{S:SectionIII}

All drying experiments were conducted in a test oven concurrently drying four different filter media, replicating industrial usage. The positions in the oven are weakly coupled and each drying process can be approximated as an independent process. 

\subsection{Drying Procedure and Moisture Content}
The experimental MC was measured using the gravimetric method. 
The drying is split into two phases, namely Drying Phase 1 (DP1) and Drying Phase 2 (DP2). 
DP1 replicates the real-world industrial drying. However, in order to map out the entirety of the drying curve a variation in drying time is induced by extracting the filter media after a predetermined amount of time. 
DP2 lasts 48 hours with an oven temperature of $120 ^{\circ} C$. The purpose of DP2 is to evaporate all MC from the filter media thus enabling the measurement of the solid mass $m_{solid}$ which is used to calculate the experimental MC in the filter media, as seen in (\ref{eq_moisture-content_initial}) and (\ref{eq_moisture-content}).

The mass of the filter media are measured three times during the experiment. The initial (wet) mass, $m_{initial}$, of each filter media is measured before DP1.
$m_{after}$ is measured after DP1, and $m_{solid}$ which is measured after DP2. 

With these measured masses we can now calculate the initial MC as:
\begin{equation} \label{eq_moisture-content_initial}
	MC_{initial} = \frac{m_{initial}-m_{solid}}{m_{solid}}  100 \%,
\end{equation}
and the MC after DP1 as:
\begin{equation} \label{eq_moisture-content}
	MC = \frac{m_{after}-m_{solid}}{m_{solid}}  100 \%.
\end{equation}


\subsection{Dataset}


\begin{figure*}
\centering
\subfloat[]{\includegraphics[height=2.5in]{figures/dataset-mean_blower_blower_dp-histogram_normalised.pdf}\label{fig-dataset-distribution-mean_oven_dp}} %
\subfloat[]{\includegraphics[height=2.5in]{figures/dataset-mean_oven_temperature-histogram_normalised.pdf}\label{fig-dataset-distribution-mean_oven_temperature}} %
\subfloat[]{\includegraphics[height=2.5in]{figures/dataset-initial_mass-histogram_normalised.pdf}\label{fig-dataset-distribution-initialmass}} %
\\
\subfloat[]{\includegraphics[height=2.5in]{figures/dataset-Temperature-vs-MC_normalised.pdf}	\label{fig-dataset-temperature-vs-MC}}
\subfloat[]{\includegraphics[height=2.5in]{figures/dataset-drying_time-vs-MC_normalised.pdf}	\label{fig-dataset-dryingtime-vs-MC}}
\caption{ (a) Distribution of normalized dimensionless mean differential pressure of each filter media during the drying process. (b) Distribution of the normalized dimensionless mean oven temperature of each drying experiment. (c) Distribution of the normalized dimensionless initial mass of filter media. (d). Normalized dimensionless estimated filter media temperature at extraction time as a function of MC. A clear relationship between estimated filter media temperature and MC can be seen. Cluster in upper left corner corresponds to the ICD. (e) MC as a function of normalized dimensionless drying time. Large variance in both drying time and MC can be seen. Cluster in upper left corner corresponds to the ICD.}
\end{figure*}


Automated data collection is used to collect the seven predictor variables that constitute the dataset. The drying time $t_{drying}$, the estimated filter media temperature $\widehat{T}_{filter}$, the oven chamber position $OCP$, the overall mean of the oven input temperature during the drying process of the particular filter media $\widebar{T}_{in}$, the overall mean of the differential pressure across the oven $\widebar{\Delta p}$, the oven temperature at the time of filter extraction $T_{cur}$, and the initial mass of the filter media before drying $m_i$ are collected for each experiment.

Each measurement of the dataset was normalized and scaled such that all values lie in the range of $[0,100]$. The normalized values of a sample $\mathbf{x}$ were calculated using:
\begin{equation}
	\mathbf{z}_k =  \frac{\mathbf{x}_k - min(\mathbf{x}_k^{train})}{max(\mathbf{x}_k^{train})-min(\mathbf{x}_k^{train})} \cdot 100,
\end{equation}
where $\mathbf{z}_k$ is the vector of normalized values of feature $k$, $\mathbf{x}_k$ is the vector of all values of feature $k$, $\mathbf{x}_k^{train}$ is the vector of values of feature $k$ belonging to the training set.

A total of 161 experiments were performed resulting in 322 sets of predictor- and response variable vectors. 161 sets of observations measuring the \textit{initial condition data} (ICD) and 161 sets of predictor- and response variable observations with different drying times. The dataset consists of two classes of datapoints, ICD and \textit{end condition data} (ECD), where the ICD are the sets of observation sampled upon insertion of a filter media into the drying oven, i.e. a drying time of zero minutes. The ICD are information poor, as an equilibrium has not been reached yet and, as an effect, the sensors are sensing the features of the oven and not those of the filter media. The ECD are the sets of observations upon extraction of the filter media from the drying oven, i.e. after the designated drying time for the specific filter media. The ECD are relatively information rich, and regression or estimation can be utilized. The dataset is published in the IEEE DataPort repository and can be found here: https://dx.doi.org/10.21227/hwa2-tp66 \cite{hwa2-tp66-22}.


The features of the dataset can furthermore be classified into two feature types, i.e., status features and oven setting features.



\subsubsection{Oven setting features}
The oven setting features are the features describing the physical environment in which the filter is dried. The oven setting features are the position of the filter media in the oven, the mean oven temperature during the drying time of each specific filter media, the mean oven differential pressure during the specific drying time of the filter media, the current oven temperature, and the initial mass of the filter media before drying begins.

Fig. \ref{fig-dataset-distribution-mean_oven_dp} shows the distribution of the differential pressure over the fan pushing the air into the oven. The differential pressure is correlated with air speed, and thus the mass of air circulating in the oven. As can be seen, one set of 20 drying experiments has been done under other circumstances than the rest of the filter media, and a trained model will need to be able to encompass this deviation in oven setting parameters as well. The outlier data has been included as it will serve to challenge the performance of the produced models.

Fig. \ref{fig-dataset-distribution-mean_oven_temperature} shows the distribution of the mean oven temperature during the drying experiments of each filter media. Here, a binormal distribution can be seen. This is due to the unfortunate deconstruction and reconstruction of the test oven during the multi-month data acquisition period. If the estimation models are able to encompass these different oven setting features, then it only bodes well for the generalizability of the model.

Fig. \ref{fig-dataset-distribution-initialmass} shows the distribution of the initial mass which as can be seen follows a skewed Gaussian distribution.


\subsubsection{Status features}
The status features are the features correlated with the current drying status of the filter i.e., the drying time and the estimated filter media temperature.

Fig. \ref{fig-dataset-temperature-vs-MC} shows the MC as a function of the dimensionless normalised estimated filter media temperature. A clearly dependent relationship between the MC and the estimated filter media temperature can be identified, the lower the temperature the larger the variation in MC as is expected from the behaviour of a typical drying curve. The estimated filter media temperature holds much of the information that the proposed models will be able to utilize in order to make good estimates.

Fig. \ref{fig-dataset-dryingtime-vs-MC} shows a relationship between drying time and MC. There is a large variance along the MC axis, especially for lower drying times. This variance is where the possible gains of utilizing MC estimation can be seen. All low-drying-time or low-MC datapoints represents the possible optimization gains, as early stopping of the drying process can be done if identification of the MC is possible.  


\subsection{Competing Estimation Models}
The proposed ANN-based approach is compared with data-driven models reported as state of the art for different MC estimation applications in the literature. To establish a baseline performance we use semi-empirical thin layer drying models, see Table \ref{tbl:methodology:thin_layer_models}. The thin layer drying models are all fitted using nonlinear least squares in the Matlab curve fitting toolbox \cite{matlabcurvefitting}. 

\begin{table}[]
    \caption{Thin-layer drying models}
    \label{tbl:methodology:thin_layer_models}
    \begin{tabular}{lll}
    Model            & Equation                    & Reference \\ \hline
    Lewis            & $MC =\exp(-kt)$                           &  \cite{lewis1921rate}         \\ 
    Page             & $MC = \exp(-kt^n)$                        &   \cite{page1949factors}        \\ 
    Two term        & $MC = a\exp(-k_1t)+b\cdot\exp(-k_2t)$      &  \cite{madamba1996thin}         \\ 
    Henderson       & $MC = a\exp(-kt)$                         &   \cite{hendersonPabis}        \\ 
    Logarithmic     & $MC = a\exp(-kt)+c$                       &  \cite{yaugciouglu1999drying}         \\ 
    Midilli et al.  & $MC = a\exp(-kt^n)+bt$                    &  \cite{midilli2002new}    \\ \hline
    \end{tabular}
\end{table}


Furthermore, we compare the ANN approach to SVR and RFR as reported by \cite{SaglamC_apple_slices}, and ANFIS as reported by \cite{Amini2021}, and partial least squares (PLS) to act as a baseline for the machine learning models. 
%
All competing models estimate the MC as output. The input for the thin layer drying models is solely drying time. The input for the machine learning models are all the same as that for the ANN. 

All models come in two variations. One trained on the entirety of the data, referred to as With Initial Conditions (WIC), and one trained only on the ECD, referred to as No Initial Conditions (NIC). As postulated earlier, the ICD is relatively information poor, and thus might hamper the estimation performance in the range of interest, the ECD. For practical applications, the quality of the estimates on the ICD can be ignored - as it is a trivial case.


\subsection{Model Performance Validation}
All models are validated using repeated 10 fold cross validation as described by \cite{Burman1989}. Regular 10 fold cross validation was performed by splitting the data into 10 folds, training on all but one fold, and then using the left-out fold for validation. This process was then repeated across all 10 folds, resulting in averages of the estimation error measures as described in (\ref{eq:MSE}), (\ref{eq:MAE}), (\ref{eq:STD}), and (\ref{eq:R2}). 
The data was then shuffled, and the above process was repeated five times. Therefore, all results reported in this section are based on validation data and not training data. Furthermore, all results reported are averages of the five times repeated 10 fold cross validation trials. 



%\section{Experiment Details}
%\input{Experimental Details}

\section{Results}
\section*{Results}
We started by assembling a dataset derived from public hikes. This process included an iterative data cleaning process to remove erroneous/false data, identify and remove breaks (e.g. Fig \ref{Fig2}) to give us a final usable dataset containing 7,636 GPS tracks, with over 1.4 million individual data points and covering almost 88,000 km of travel in the U.K. 

Our curated hike dataset allowed us to create a data-driven model which we can directly compare with existing walking speed algorithms. The model formulation was selected using a small-scale exploratory study which considered data from Scotland (see \nameref{S3_Appendix}). In this exploratory study, multiple different model types were explored which could fit the data, and which matched existing knowledge about walking speeds. Cross-validation methods showed that there was very little difference in performance of the best models, therefore the final model was a Generalised Linear Model (GLM), which was chosen as it was the simplest of those tested (we had no evidence that a more complex model would be superior). This choice also meant that our model was both easy to interpret, and simple to apply to future work.

This final GLM model included all three of the variables suggested by Arnet \cite{Arnet2009ArithmeticalJapan}:

\begin{equation}
    v = exp(a+b\phi+c\theta+d\theta^2)
\end{equation}
where
\begin{quote}
$v = \text{walking speed (km/h)}$\\
$\phi = \text{hill slope angle (degrees)}$\\
$\theta = \text{walking slope angle (degrees)}$
\end{quote}

Terrain obstruction level was included as a factor variable, while we considered the road types as both factor variables and interaction terms. Not all terms had a significant effect on all variables; we therefore created a model with all possible terms, and removed them one at a time (in order of least significance) until all remaining terms were significant to at least 95\% confidence  level (using Wald test). The final values for a, b, c and d are given in Table \ref{tab:2ROUK model variable values} for each of the terrain obstruction levels and road types. The critical gradient for this model is between 14 -- 16 degrees when walking uphill and -16 -- -18 degrees when walking downhill (depending on road and obstruction conditions), which is in line with previous findings. 

Fig \ref{Fig3} shows the predicted walking speeds under different conditions. The importance of including both the hill slope and terrain obstruction variables can be clearly seen when looking at the Off Road Light Obstruction speed predictions. When directly ascending or descending a slope, the walking speed is comparable to walking on a road. However, when traversing a slope while off road, the walking speed is comparable to traversing a slope of double the gradient while on a road or path. Similarly, comparing the walking speed predictions of Off Road Light Obstruction and Off Road Heavy Obstruction reveals that just 10 cm of vegetation (our cutoff point for heavy obstruction) can reduce the walking speed by more than 0.5 km/h.

\begin{table}[!ht]
\begin{adjustwidth}{-0.5in}{0in}
    \centering
    \caption{Final walking speed model variable coefficients}
    \begin{tabular}{|l+c|c|c|c|}
    \hline
    & $a$ & $b$ & $c$  & $d$ \\ 
    \thickhline
    Paved road & 1.580 & -0.00389 & -0.00726 & -0.00218 \\ 
    \hline
    Unpaved road & 1.580 & -0.00389 & -0.00965 & -0.00248 \\
    \hline
    Off-road (obstruction unknown) & 1.536 & -0.00731 & -0.00965 & -0.00187 \\
    \hline
    Off-road (light obstruction) & 1.580 & -0.00731 & -0.00965 & -0.00187 \\ 
    \hline
    Off-road (heavy obstruction) & 1.400 & -0.00731 & -0.00965 & -0.00187 \\ 
    \hline
    \end{tabular}
    \label{tab:2ROUK model variable values}
\end{adjustwidth}
\end{table}

\begin{figure}[!h]
\begin{adjustwidth}{-2.25in}{0in} 
    \includegraphics[width=\linewidth]{Images/Paper/Fig3.eps}
    \captionsetup{width=1\linewidth}
    \caption[width=\textwidth]{{\bf Walking speed predictions under different terrain conditions.}  When: (A) travelling directly up or down hills of varying slope, (B) traversing across hills of varying slope.}
    \label{Fig3}
    \end{adjustwidth}
\end{figure}

Fig \ref{Fig4} compares the Paved Road and Off Road Heavy Obstruction speed predictions from our model against the existing functions from Naismith, Tobler and Campbell et al. When looking at the walking slope, the largest areas of deviation between our model and Naismith's rule occurs when descending a slope, as Naismith's rule does not predict a reduced speed in this scenario. For both Tobler's and Campbell et al.'s functions, the shape of the walking slope component is relatively similar to our new model, with the main distinction being the peak predicted speed on flat ground. None of the existing functions account for the hill slope, which leads to large disparities when predicting the walking speed for slope traversals. A further example of this can be seen in \nameref{S6_Appendix}, which shows the walking speeds for a simulated off-road route which encounters the full range of hill and walking slopes.

\begin{figure}[!h]
\begin{adjustwidth}{-2.25in}{0in} 
    \includegraphics[width=\linewidth]{Images/Paper/Fig4.eps}
    \captionsetup{width=1\linewidth}
    \caption[width=\textwidth]{{\bf Comparison of new model and existing hiking functions.}  Predicted walking speeds of the new model, Naismith's rule, Tobler's function and Campbell et al.'s function when: (A, C, E) travelling directly up or down hills of varying slope, (B, D, F) traversing across hills of varying slope.}
    \label{Fig4}
\end{adjustwidth}
\end{figure}

When comparing the performances of each of the models (Table \ref{tab:2comparison}), the predicted speeds for individual 50 m sections had a lower RMSE and percentage error, and a higher R squared value using our new model than in the existing ones. To isolate the impact of each of the slope variables, we filtered the results to look at the data where a slope was being directly climbed or traversed. Figs \ref{Fig5}A, B and \ref{Fig6}A, B show the RMSE and mean residuals for each of the models, for data which was within 5 degrees of directly climbing (A) or traversing (B) hills of varying slope. From this we can clearly see that Naismith's rule consistently overestimates walking speeds when descending a slope, and underestimates speeds when climbing a slope. When ascending or descending a slope, the RMSE of our GLM is similar to that of Tobler's hiking function. However, one of the main areas where we see an improvement using our model is on slight declines. Tobler's hiking function suggests that walking speed increases on mild descents up to a maximum of 6 km/h. It is clear from Fig \ref{Fig5}A, that Tobler's function overestimates the walking speed in this region. Campbell et al.'s function has a slightly lower RMSE value than our new model on the steepest walking slopes, however it underestimates the walking speeds on flat ground and mild slopes. Previous research has found that most walking takes place on low walking slopes \cite{Proffitt1995PerceivingSlant}, and this is evidenced by our data ($\sim$98\% of our data was from walking slopes of under 10 degrees). Improved walking speed predictions in this region therefore have the greatest impact in real-world situations. Within this region our model consistently has a lower RMSE than the existing functions, and a mean residual error close to 0 km/h. 

\begin{table}[!ht]
\centering
\caption{Comparison of new model against existing methods to calculate walking speeds.}
\begin{tabular}{|l|c|c|c|c|}
\hline
& New Model & Naismith & Tobler & Campbell\\
\hline
Average \% error & 23.68 & 26.36 & 26.17 & 25.33\\
\hline
MSE & 1.20 & 1.61 & 1.53 & 1.58\\
\hline
RMSE & 1.10 & 1.27 & 1.24 & 1.26\\
\hline
R\textsuperscript{2}  & 0.09 & -0.22 & -0.16 & -0.19\\
\hline
\end{tabular}
\label{tab:2comparison}  
\end{table}

\begin{figure}[!h]    
\begin{adjustwidth}{-2.25in}{0in} 
    \includegraphics[width=\linewidth]{Images/Paper/Fig5.eps}
    \captionsetup{width=1\linewidth}
    \caption[width=\textwidth]{{\bf Comparing RMSE values for the new model, Naismith's rule, Tobler's function and Campbell et al.'s function.} When: (A) travelling directly up or down hills of varying slope (all data), (B) traversing across hills of varying slope (all data), (C) travelling directly up or down hills of varying slope (off-road data only), (D) traversing across hills of varying slope (off-road data only). Campbell et al.'s function does not provide off-road speed estimates, so was not included in the off-road data comparisons.}
    \label{Fig5}
\end{adjustwidth}
\end{figure}

\begin{figure}[!h]
    \begin{adjustwidth}{-2.25in}{0in} 
    \includegraphics[width=\linewidth]{Images/Paper/Fig6.eps}
    \captionsetup{width=1\linewidth}
    \caption[width=\textwidth]{{\bf Comparing mean residual values for the new model, Naismith's rule, Tobler's function and Campbell et al.'s function.} When: (A) travelling directly up or down hills of varying slope, (B) traversing across hills of varying slope, (C)  travelling directly up or down hills of varying slope (off-road data only), (D) traversing across hills of varying slope (off-road data only). Campbell et al.'s function does not provide off-road speed estimates, so was not included in the off-road data comparisons.}
    \label{Fig6}
\end{adjustwidth}
\end{figure}

 We also see an improvement in RMSE when using our model to predict speeds for hill traversals (Fig \ref{Fig5}B). We can note from Fig \ref{Fig6}B that both Naismith's rule and Tobler's hiking function consistently overestimate the walking speed when traversing a slope, as they do not take into account the impact that the hill slope has on reducing walking speeds. The performance of Campbell et al's model improves as the hill slope increases, although we suggest this is more due to it underestimating the speed on shallow slopes. We do see that the average error in our model increases as the hill slope increases, but we believe that this is due to limited volumes of data at high hill slopes ($\sim$0.5\% of our data occurs on hill slopes steeper than 40 degrees). 

As well as looking at the overall performance of our new model, we looked to explore how well our model performed in off-road conditions, compared to the off-road adjustments for the existing functions (Naismith's reduced base speed of 4 km/h, and Tobler's correction factor of 0.6). Figs \ref{Fig5}C, D and \ref{Fig6}C, D show the RMSE and mean residuals, only considering data which was recorded in off-road conditions. From Figs \ref{Fig5}C and \ref{Fig6}C it is clear that Tobler's function consistently underestimates the walking speed when off-road. The factor of 0.6 is a larger reduction in walking speed than is observed in practice. As we found when looking at our data as a whole, Naismith's rule underestimates the walking speed when climbing a slope and overestimates when descending a slope. Our new model does not suffer from these problems, with both a lower RMSE and lower absolute mean residual value across all walking slopes. Both of these existing models also consistently underestimate walking speeds when traversing a slope, unlike our new model which has a mean residual of less than 0.4 km/h on slopes of up to 35 degrees. The error in predictions of our new model does increase as the hill slope increases, though the RMSE is generally lower than seen in the existing models. On the steepest hill slopes our model appears to perform less well than the existing ones, though only 0.2\% of our off-road data occurred on a hill slope steeper than 40 degrees. 

Although we have shown an improvement in walking speed predictions over short sections of routes, this did not translate to similar results when looking at predicted walking times for routes as a whole. Our model and all of the existing models which we have explored here had an average percentage error of 13.5\% - 15.5\% when predicting the time taken for a complete route. However, based on the errors seen in Figs \ref{Fig5} and \ref{Fig6}, we believe that this is a result of errors cancelling out over the course of a hike. For example while ascending a hill, Naismith's rule will underestimate the walking speed (and thus overestimate the walking time), but it will then overestimate the walking speed on the subsequent descent, leading to a relatively accurate total time estimate. The results here suggest that Naismith's rule, and other existing functions, are still a good rule of thumb to calculate route times as a whole, but time estimates for individual sections of a route will be less accurate than when using the new model found here.




\section{Conclusion}
%\section{}
%\label{sec:resDir}


\section{Conclusion}
\label{sec:conclusion}
% <>
Since its advent in 1931, Koopman operator theory \cite{koopman:1931} has only recently been actively utilized for solving practical problems, thanks to the introduction of the DMD algorithm in 2008 \cite{schmid:2008}. Since then, a multitude of DMD algorithm variations have risen to prominence and found utility across various fields. A notable feature of our survey paper was reviewing and categorizing the results of over 100 research papers based on both application and algorithm type in smart mobility and vehicle engineering  (see Table~\ref{tab1} and Section~\ref{sec:vehicApp}).  Additionally, this survey paper identified potential research gaps in smart mobility and vehicular engineering applications (Remarks~\ref{remGap1}--\ref{remGap6}). Finally, this review paper discussed theoretical aspects of Koopman operator theory that have been largely neglected by the smart mobility and vehicle engineering community and yet have large potential for contributing to solving open problems in these areas (see Section~\ref{subsec:theorIssue}).

\noindent{\textbf{Future Research Directions.}}	Given the emergence of cyber-threats against connected and autonomous vehicles as well as robotic systems (see, e.g.,~\cite{nekouei2021randomized,mohammadi2022generation}), a future research direction might include utilizing Koopman operator-based algorithms for designing cyber-resilient vehicular and smart mobility applications (see, e.g.,~\cite{taheri2022data} for a related line of research). Another potential research direction is using Koopman operator-based algorithms for predicting the motion of vulnerable road users (VRUs), e.g., pedestrians and cyclists (see, e.g.,~\cite{pool2019context,scholler2020constant}). Finally, rehabilitation robotics and robotic exoskeletons can be the benefactors of the predictive capabilities of Koopman operator-based algorithms for detecting tripping events and/or system  identification in various modes of locomotion (see, e.g.,~\cite{kumar2019extremum,aprigliano2019pre}).



%Fig. 1 depicts the accumulation of such algorithms since 2014, which are particular to vehicle engineering and smart mobility, i.e., the focus of this review. Table 1 summarizes the varieties of relevant algorithms developed in those studies. Furthermore, we have highlighted theoretical issues, whose expansion will have potential applications to the wide research area of smart mobility and vehicle engineering.  

%Although fairly comprehensive, we have found several gaps in this research area. In particular, we could not find any studies related to elevators, robots/vehicles employing crawling, slithering, hopping or peristaltic locomotion, arctic or special-terrain vehicles such as those employing screws or tracks, hovercraft and other amphibious vehicles or subsystems which tolerate flexible environments, classification or guidance systems related to vehicles for drilling or agriculture, or for current-ripple, power-split, battery health monitoring, nuclear propulsion, exoskeletons/prosthetics, personal mobility, motorsports, specialized rovers or similar open problems in emerging areas.  These examples are, of course, not exhaustive.  
%
%The purely data-driven nature of Koopman operators holds the promise of capturing unknown and complex dynamics for reduced-order model generation and system identification, through which the rich machinery of linear control techniques can be utilized. The emergent nature of the smart mobility and vehicular-related applications, where  the Koopman operator  in each particular application needs to be approximated, implies that the development of various Koopman operator approximation  algorithms is expected to grow along with the vehicular problems they aim to solve.  Given the ongoing development of this research area and the many existing open problems in the fields of smart mobility and vehicle engineering, a survey of techniques and open challenges of applying Koopman operator theory to this vibrant area is warranted.  To the best of our knowledge, this survey paper is the \emph{first of its kind} reviewing the applications of Koopman operator theory within a focused research area, namely, smart mobility and vehicle engineering applications. A \emph{notable feature} of our survey paper is reviewing and categorizing the results of over 100 research papers based on both application and algorithm type  (see Tables~\ref{tab1}--~\ref{tab4} and Section~\ref{sec:vehicApp}) that are concerned with the applications of Koopman operator theory to the field of smart mobility and vehicular engineering. Such a \emph{comprehensive and  detailed categorization} will be beneficial to the research practitioners working in the field.  Furthermore, this review paper discusses theoretical aspects of Koopman operator theory that have been largely neglected by the smart mobility and vehicle engineering community and yet have large potential for contributing to solving open problems in these areas. Additionally, our survey paper seeks to \emph{identify gaps} in the smart mobility and vehicle engineering research where new and existing Koopman operator-based methods have the potential to further develop and address unsolved problems  potentially benefiting from the perspectives of nonlinear system identification, control, global linearization, and the predictive powers that Koopman operator theory has to offer (see, e.g., Remarks~\ref{remGap1}--\ref{remGap6}). 


\section*{Declarations}
All authors certify that they have no affiliations with or involvement in any organization or entity with any financial interest or non-financial interest in the subject matter or materials discussed in this manuscript.
\subsection*{Funding}
The authors did not receive support from any organization for the submitted work.
\subsection*{Data availability}
The data that supports the findings of this study is publicly available online at https://www.kaggle.com/datasets/borismarjanovic/price-volume-data-for-all-us-stocks-etfs

%\bibliographystyle{sn-mathphys.bst}
\bibliography{sn-bibliography}% common bib file
%% if required, the content of .bbl file can be included here once bbl is generated
%%\input sn-article.bbl

%% Default %%
%%\input sn-sample-bib.tex%

\end{document}
