
\paragraph{Rescaling} In order to build the input price matrixes, a rescaling of the real observations of the time series stock price is needed so that it falls into a small specific interval. Let $X = {x_1,x_2,...,x_n}$ be the considered time series with $n$ components, the rescaling to the interval $[0, 1]$ is achieved by scaling the maximum value of each time series to unit size.
\begin{equation}
    {\tilde{x_i}} =\frac{x_i - min(X)}{max(X) - min(X)}
\end{equation}
Hence, the scaled series is represented by $\tilde{X} = \{\tilde{x_1}, \tilde{x_2}, ..., \tilde{x_n}\}$


\paragraph{Historical Price Image} For a specific stock, the input frames should contain the lowest, opening, highest, and closing prices for T consecutive trading days. Each row has the rescaled historical prices for a day. The following demonstrates the input matrix, or, price frames.\\




\includegraphics[scale=0.22]{in_mat.png} 

\iffalse

\NiceMatrixOptions{
%code-for-first-row = \color{blue} ,
%code-for-last-row = \color{blue} ,
code-for-first-col = \color{black}
%code-for-last-col = \color{blue}
}


\[
\begin{bNiceMatrix}[first-col]%[vlines,first-row,last-row,first-col,last-col]

\textbf{day t-T}     & Open_{t-T}  & Low_{t-T} & High_{t-T} & Close_{t-T}  \\
\Vdots&\Vdots&\Vdots&\Vdots&\Vdots\\
\\
\\
\textbf{day t-1}       & Open_{t-1}  & Low_{t-1} & High_{t-1} & Close_{t-1} \\
\\
\textbf{day t  }       & Open_{t \ \ \ }  & Low_{t \ \ \ } & High_{t \ \ \ } & Close_{t \ \ \ }  \\

\CodeAfter
  \begin{tikzpicture}
  \node [draw=blue, rounded corners=4pt, inner ysep = 2pt, 
       rotate fit=0, fit = (1-1) (1-4) ] {} ;
  \node [draw=blue, rounded corners=4pt, inner ysep = 2pt,
       rotate fit=0, fit = (5-1) (5-4) ] {} ;
  \node [draw=blue, rounded corners=4pt, inner ysep = 2pt,
       rotate fit=0, fit = (7-1) (7-4) ] {} ;
  \end{tikzpicture}
  
 
\end{bNiceMatrix}

\]

\fi









\paragraph{Segmentation Mask} To create the annotated labels, we compare the historical price frame at time frame $n+1$ with the one at time frame $n$. More specifically, each component of the stock price matrix for the next time frame is compared with the one from the last time frame, and each pixel is assigned with zero or one according to the following criteria:

$$
{y^{n+1}_{t,c}} = 
    \begin{cases}
    1, & \qquad {X^{n+T}_{t,c}} > {X^n_{t,c}} \\
    0, & \qquad o.w.
    
    \end{cases}
$$

Where ${y^{n+1}_{t,c}}$ is the stock price trend for the next time frame at row $t$ and column $c$. Resulting matrices are illustrated in Fig. \ref{datamat}.\



\begin{figure}[H]
    \centering
    \includegraphics[width=\linewidth]{matrix2.png}
     \caption{Stock price frames and corresponding output trend segmentation map}
     \label{datamat}
\end{figure}
