\documentclass[10pt,letterpaper]{article}
\usepackage[top=0.85in,left=2.75in,footskip=0.75in]{geometry}

\usepackage{amsmath,amssymb}
\usepackage{changepage}
\usepackage[utf8x]{inputenc}
\usepackage{textcomp,marvosym}
\usepackage{cite}
\usepackage{nameref,hyperref}
\usepackage[right]{lineno}
\usepackage{microtype}
\DisableLigatures[f]{encoding = *, family = * }
\usepackage[table]{xcolor}
\usepackage{array}
\newcolumntype{+}{!{\vrule width 2pt}}
\newlength\savedwidth
\newcommand\thickcline[1]{
  \cline{#1}%
  \noalign{\vskip\arrayrulewidth}%
  \noalign{\global\arrayrulewidth\savedwidth}%
}
\newcommand\thickhline{\noalign{\global\savedwidth\arrayrulewidth\global\arrayrulewidth 2pt}%
\hline
\noalign{\global\arrayrulewidth\savedwidth}}
\raggedright
\setlength{\parindent}{0.5cm}
\textwidth 5.25in 
\textheight 8.75in
\usepackage[aboveskip=1pt,labelfont=bf,labelsep=period,justification=raggedright,singlelinecheck=off]{caption}
\renewcommand{\figurename}{Fig}
\bibliographystyle{plos2015}
\makeatletter
\renewcommand{\@biblabel}[1]{\quad#1.}
\makeatother
\usepackage{lastpage,fancyhdr,graphicx}
\usepackage{epstopdf}
\pagestyle{fancy}
\fancyhf{}
\rfoot{\thepage/\pageref{LastPage}}
\renewcommand{\headrulewidth}{0pt}
\renewcommand{\footrule}{\hrule height 2pt \vspace{2mm}}
\fancyheadoffset[L]{2.25in}
\fancyfootoffset[L]{2.25in}
\lfoot{\today}



%mypackages
\usepackage{multirow}
\usepackage{todonotes}
\usepackage{multicol}
\usepackage{url}
\def\UrlBreaks{\do\/\do-}

\usepackage{array}
\newcommand{\PreserveBackslash}[1]{\let\temp=\\#1\let\\=\temp}
\newcolumntype{C}[1]{>{\PreserveBackslash\centering}p{#1}}
\newcolumntype{L}[1]{>{\PreserveBackslash\raggedright}p{#1}}

\usepackage{algpseudocode}
\usepackage{algorithm}
\usepackage[normalem]{ulem}

\begin{document}
\vspace*{0.2in}

% Title must be 250 characters or less.
\begin{flushleft}
{\Large
\textbf\newline{Improved prediction of hiking speeds using a data driven approach}
}
\newline
\\
Andrew Wood\textsuperscript{1*},
William Mackaness\textsuperscript{2},
T. Ian Simpson\textsuperscript{1},
J. Douglas Armstrong\textsuperscript{1}
\\

\bigskip
\textbf{1} School of Informatics, University of Edinburgh, Edinburgh, UK
\\
\textbf{2} School of Geoscience, University of Edinburgh, Edinburgh, UK
\\
\bigskip
* andrew.wood@ed.ac.uk
\end{flushleft}

% Please keep the abstract below 300 words
\begin{abstract}
%
Deformable image registration is a fundamental task in medical image analysis and plays a crucial role in a wide range of clinical applications. 
Recently, deep learning-based approaches have been widely studied for deformable medical image registration and achieved promising results. However, existing deep learning image registration techniques do not  theoretically guarantee %diffeomorphic 
topology-preserving transformations. This is a key property to preserve anatomical structures and achieve plausible transformations that can be used in real clinical settings.
%
We propose a novel framework for deformable image registration. Firstly, we introduce a  novel regulariser based on conformal-invariant properties in a nonlinear elasticity setting.
Our regulariser enforces  the deformation field  to be  smooth, invertible and orientation-preserving. %differentiable.  
More importantly, we strictly guarantee topology preservation yielding to a clinical meaningful registration.  Secondly, 
we boost the performance of our regulariser through coordinate MLPs, where one can view the to-be-registered images as continuously differentiable entities. 
We demonstrate, through numerical and visual experiments, that our framework is able to outperform current techniques for image registration.
%
\keywords{Homeomorphic image registration \and Lung CT \and Conformal invariant hyperelastic regularisation.}
%
\end{abstract}
%\section*{Author summary}
% Please keep the Author Summary between 150 and 200 words
% Use first person. PLOS ONE authors please skip this step. 
% Author Summary not valid for PLOS ONE submissions.  

\nolinenumbers
% Importance and appeal of children's drawings
Children's depictions of the human figure are highly expressive and varied.
As one of the very first subjects children attempt to draw, the representation begins as an almost unintelligible cloud of scribbles. 
As the child grows, their representation of the human figure becomes more developed and is extended to graphically represent many different types of characters: people, animals, and even personified objects (see Figure 1).

Who among us has not wished, either as a child or as an adult, to see such figures come to life and move around on the page?
Sadly, while it is relatively fast to produce a single drawing, creating the sequence of images necessary for animation is a much more tedious endeavor, requiring discipline, skill, patience, and sometimes complicated software.
As a result, most of these figures remain static upon the page.

% We built a system to animate them.
Inspired by the importance and appeal of the drawn human figure, we design and build a system to automatically animate it given an in-the-wild photograph of a child's drawing. 
Our system is fast, intuitive, and robust to much of the variation present in these types of drawings, making it well-suited to allow our target audience--children--to see their own characters coming to life.
The system is comprised of four stages: figure detection, segmentation masking, pose estimation/rigging, and animation. 
We describe each stage and identify common causes of failure in each. 
For object detection and pose estimation, we make use of existing computer vision models designed to detect human figures and joints in photographs; we fine-tune these models for use with children's drawings.
For segmentation, we present a straightforward, image processing-based method that, for animation purposes, is more useful and accurate than segmentation masks obtained from a fine-tuned object detection model.
During the animation step, we take advantage of the \textit{twisted perspective} commonly seen in children’s drawings to retarget motion capture data onto the character in a novel and appealing way.

% We use existing machine learning models. However, given the wide domain gap it's not clear how much fine-tuning data was needed. So we ran some experiments to find out and report it.
While our system leverages existing models and techniques, most are not directly applicable to the task due to the many differences between photographic images and simple pen and paper representations. 
To this end, we couple the presentation of our system with a set of experiments exploring the relationship between fine-tuning training set size and success rates.
We also include a perceptual study validating viewer preference for incorporating \textit{twisted perspective} into the motion retargeting step.

We validate the desirability and appeal of our system by building and publicly releasing a version of it as the \AD Demo \,\cite{animateddrawings}.
Launched in December 2021, this demo has been used by millions of people around the world to animate their children's drawings.
Inspired by this reception, our second contribution is The Amateur Drawings Dataset: \hjs{180,000 drawings and user-accepted annotations collected, with consent, through the demo. See Section \ref{sec:UI} for a description of how the annotations were generated.}
We believe this dataset will be a resource to researchers from various fields seeking to better understand the space of amateur drawings, evaluate new algorithms in this domain, or develop new drawing-based tools in general.

To summarize, our contributions are as follows:
\begin{enumerate}
    \item 
    We explore the problem of automatic sketch-to-animation for children's drawings of human figures and present a framework that achieves this effect. We also present a set of experiments determining the amount of training data necessary to achieve high levels of success and a perceptual study validating the usefulness of our motion retargeting technique.
    \item To encourage additional research in the domain of amateur drawings, we present a first-of-its-kind dataset of 180,000 user-submitted amateur drawings, along with user-accepted bounding box, segmentation mask, and joint location annotations.
\end{enumerate}

Upon acceptance of this paper, we plan to publicly release the Amateur Drawings Dataset, project code, and fine-tuned model weights.

\section*{Materials and methods}

\subsection*{Data set, cleaning and key assumptions}
\label{2Data set cleaning and key assumptions}
Full details of the various datasets used in this study are provided in \nameref{S1_Appendix}. Further, a detailed description of the data filtering processes, and choices/assumptions made during data processing are described in \nameref{S2_Appendix}.

In summary, GPS tracks were obtained for hikes in the UK from Hikr.org \cite{Hikr.org2021UnitedReports} and OpenStreetMap (OSM) \cite{OpenStreetMap.org2021Tracks}. Visual inspection of the tracks showed that a large number were not walking tracks, as their speed was implausible, or contained long breaks which could impact the accuracy of a walking speed model. We developed a series of filters to identify and remove significant breaks and tracks or segments which were not hike or walking based. Fig \ref{Fig 2} shows examples of regions where breaks are visible in a GPS track, which had to be filtered prior to analysis.

\begin{figure}[!h]
    \begin{adjustwidth}{-1.75in}{0in} 
\includegraphics{Images/Paper/Fig2.eps}
\captionsetup{width=1\linewidth}
    \caption[width=\textwidth]{{\bf A GPS track where 3 breaks can identified by finding point clusters.} Clusters of points can form on a GPS track when a break is taken during a hike. By identifying these clusters as potential breaks we are able to remove most break periods from the datasets used for our analysis of walking speeds. For full details of these and other data filtering methods see \nameref{S2_Appendix}. Background images reproduced with permission from Bing Maps/Vexcel Imaging\cite{BingMaps,Vexcel}, OpenStreetMap and OpenStreetMap Foundation \cite{OpenStreetMapMAPS}, visualised using QGIS \cite{OpenSourceGeospatialFoundationProjectQGIS.org2020}.}
    \label{Fig 2}
\end{adjustwidth}
\end{figure}

Elevation and slope values were calculated and added to each point using data from the Ordnance Survey Terrain 5 Digital Terrain Map (DTM) \cite{OrdnanceSurvey2020Terrain5}. Hill slope values were found using the quadratic surface method \cite{Zevenbergen1987QuantitativeTopography, Dunn1998TheGIS}. Each data point was then classified as on a paved road, on an unpaved road, or off road, determined by searching a 50 m radius around each point in an OSM Road dataset \cite{OpenStreetMap.org2021Data}. 

Terrain obstruction information was calculated using lidar datasets \cite{LIDARDSMEngland, LIDARDTMEngland, LidarWales}, as the difference in values between a Digital Surface Map (DSM) and Digital Terrain Map (DTM). We had access to lidar data at 2 m resolution covering large areas of England and Wales, but the coverage was not complete. Of our off-road data ($\sim$2,900 km, spread across over 1,200 tracks), over 2,000 km had lidar data available. Exploration of the lidar data (see \nameref{S5_Appendix}) showed that there was a clear drop in walking speeds once more than 10 cm of obstruction was observed, beyond which the speed was relatively constant.  We used this information to classify points into heavy obstruction (\textgreater 10 cm) or light obstruction ($<= 10$ cm) for modelling purposes.

Where the datapoints in the original GPS track were under 50 m in length, they were merged together to minimise the effects of errors in the GPS location values. While doing this, the resulting distance was the sum of all distances in the constituent GPS points, so may be longer than the straight line distance between co-ordinates. Similarly, both hill and walking slope values, as well as obstruction height, were calculated as the weighted average of constituent points, weighted by point duration.

Our final dataset consisted of 7,636 GPS tracks, with over 1.4 million individual data points and almost 88,000 km of travel. Each datapoint contained:
\begin{itemize}
    \setlength\itemsep{0em}
    \item Start coordinate
    \item End coordinate
    \item Start time
    \item Duration
    \item Distance
    \item Speed
    \item Elevation
    \item Walking slope
    \item Hill slope
    \item On-road flag
    \item Paved road flag (if on-road)
    \item Obstruction data available flag (if off-road)
    \item Heavy obstruction flag (if off-road and obstruction data available)
\end{itemize}

\subsection*{Modelling}

\subsubsection*{Model Formulation}

Pilot studies were conducted to identify an appropriate model framework, using tracks within Scotland (see \nameref{S3_Appendix}). 10-fold cross-validation was used to compare the model parameters, looking at R-squared values, root-mean-squared error (RMSE) and mean absolute error. Where multiple models performed equally well, the simplest model was selected for ease of interpretabilty and real-world application. The selected model type was a Generalised Linear Model (GLM). Models were implemented using R-Studio version 1.2.5019 \cite{R}. 

\subsubsection*{Terrain Types}

Each of the three road types (paved road, unpaved road, off-road) was included in the model, both as factor variables, and as interaction terms with each of the slope variables.

Before adding terrain obstruction data to the model, we checked that there was no systematic difference between the walking speeds in regions where we had lidar data, and regions where we did not (see \nameref{S5_Appendix}). Thus our findings in regions where lidar data was available could be extended to those where it was unavailable. Factor variables were then added to the model for each obstruction level (heavy, light or unknown obstruction).

\subsubsection*{Statistical Analysis}

Variables within the model were tested for significance using the Wald test, which allows us to account for correlation between points within the same track (\texttt{coeftest} function within \texttt{lmtest} package in R).

To measure the impact of our model, we compared walking speed predictions of our model against those of Naismith's and Tobler's models. Four different metrics were compared; the average percentage error, mean squared error (MSE), root-mean squared error (RMSE) and R squared value. These were explored when looking at both individual 50 m track sections, as well as predicted walking times for tracks as a whole. Finally, we isolated the off-road track sections in order to assess the improvement of our model at predicting walking speeds for off-path travel.

\subsection*{Code Availability}
\label{0Code}

Documented code written for this study is available online in a Github repository \texttt{\href{https://github.com/AndrewWood94/PhDThesis}{AndrewWood94/PhDThesis}}, and is licensed under the terms of the GNU General Public License v3.0.
% Results and Discussion can be combined.
\section*{Results}
We started by assembling a dataset derived from public hikes. This process included an iterative data cleaning process to remove erroneous/false data, identify and remove breaks (e.g. Fig \ref{Fig2}) to give us a final usable dataset containing 7,636 GPS tracks, with over 1.4 million individual data points and covering almost 88,000 km of travel in the U.K. 

Our curated hike dataset allowed us to create a data-driven model which we can directly compare with existing walking speed algorithms. The model formulation was selected using a small-scale exploratory study which considered data from Scotland (see \nameref{S3_Appendix}). In this exploratory study, multiple different model types were explored which could fit the data, and which matched existing knowledge about walking speeds. Cross-validation methods showed that there was very little difference in performance of the best models, therefore the final model was a Generalised Linear Model (GLM), which was chosen as it was the simplest of those tested (we had no evidence that a more complex model would be superior). This choice also meant that our model was both easy to interpret, and simple to apply to future work.

This final GLM model included all three of the variables suggested by Arnet \cite{Arnet2009ArithmeticalJapan}:

\begin{equation}
    v = exp(a+b\phi+c\theta+d\theta^2)
\end{equation}
where
\begin{quote}
$v = \text{walking speed (km/h)}$\\
$\phi = \text{hill slope angle (degrees)}$\\
$\theta = \text{walking slope angle (degrees)}$
\end{quote}

Terrain obstruction level was included as a factor variable, while we considered the road types as both factor variables and interaction terms. Not all terms had a significant effect on all variables; we therefore created a model with all possible terms, and removed them one at a time (in order of least significance) until all remaining terms were significant to at least 95\% confidence  level (using Wald test). The final values for a, b, c and d are given in Table \ref{tab:2ROUK model variable values} for each of the terrain obstruction levels and road types. The critical gradient for this model is between 14 -- 16 degrees when walking uphill and -16 -- -18 degrees when walking downhill (depending on road and obstruction conditions), which is in line with previous findings. 

Fig \ref{Fig3} shows the predicted walking speeds under different conditions. The importance of including both the hill slope and terrain obstruction variables can be clearly seen when looking at the Off Road Light Obstruction speed predictions. When directly ascending or descending a slope, the walking speed is comparable to walking on a road. However, when traversing a slope while off road, the walking speed is comparable to traversing a slope of double the gradient while on a road or path. Similarly, comparing the walking speed predictions of Off Road Light Obstruction and Off Road Heavy Obstruction reveals that just 10 cm of vegetation (our cutoff point for heavy obstruction) can reduce the walking speed by more than 0.5 km/h.

\begin{table}[!ht]
\begin{adjustwidth}{-0.5in}{0in}
    \centering
    \caption{Final walking speed model variable coefficients}
    \begin{tabular}{|l+c|c|c|c|}
    \hline
    & $a$ & $b$ & $c$  & $d$ \\ 
    \thickhline
    Paved road & 1.580 & -0.00389 & -0.00726 & -0.00218 \\ 
    \hline
    Unpaved road & 1.580 & -0.00389 & -0.00965 & -0.00248 \\
    \hline
    Off-road (obstruction unknown) & 1.536 & -0.00731 & -0.00965 & -0.00187 \\
    \hline
    Off-road (light obstruction) & 1.580 & -0.00731 & -0.00965 & -0.00187 \\ 
    \hline
    Off-road (heavy obstruction) & 1.400 & -0.00731 & -0.00965 & -0.00187 \\ 
    \hline
    \end{tabular}
    \label{tab:2ROUK model variable values}
\end{adjustwidth}
\end{table}

\begin{figure}[!h]
\begin{adjustwidth}{-2.25in}{0in} 
    \includegraphics[width=\linewidth]{Images/Paper/Fig3.eps}
    \captionsetup{width=1\linewidth}
    \caption[width=\textwidth]{{\bf Walking speed predictions under different terrain conditions.}  When: (A) travelling directly up or down hills of varying slope, (B) traversing across hills of varying slope.}
    \label{Fig3}
    \end{adjustwidth}
\end{figure}

Fig \ref{Fig4} compares the Paved Road and Off Road Heavy Obstruction speed predictions from our model against the existing functions from Naismith, Tobler and Campbell et al. When looking at the walking slope, the largest areas of deviation between our model and Naismith's rule occurs when descending a slope, as Naismith's rule does not predict a reduced speed in this scenario. For both Tobler's and Campbell et al.'s functions, the shape of the walking slope component is relatively similar to our new model, with the main distinction being the peak predicted speed on flat ground. None of the existing functions account for the hill slope, which leads to large disparities when predicting the walking speed for slope traversals. A further example of this can be seen in \nameref{S6_Appendix}, which shows the walking speeds for a simulated off-road route which encounters the full range of hill and walking slopes.

\begin{figure}[!h]
\begin{adjustwidth}{-2.25in}{0in} 
    \includegraphics[width=\linewidth]{Images/Paper/Fig4.eps}
    \captionsetup{width=1\linewidth}
    \caption[width=\textwidth]{{\bf Comparison of new model and existing hiking functions.}  Predicted walking speeds of the new model, Naismith's rule, Tobler's function and Campbell et al.'s function when: (A, C, E) travelling directly up or down hills of varying slope, (B, D, F) traversing across hills of varying slope.}
    \label{Fig4}
\end{adjustwidth}
\end{figure}

When comparing the performances of each of the models (Table \ref{tab:2comparison}), the predicted speeds for individual 50 m sections had a lower RMSE and percentage error, and a higher R squared value using our new model than in the existing ones. To isolate the impact of each of the slope variables, we filtered the results to look at the data where a slope was being directly climbed or traversed. Figs \ref{Fig5}A, B and \ref{Fig6}A, B show the RMSE and mean residuals for each of the models, for data which was within 5 degrees of directly climbing (A) or traversing (B) hills of varying slope. From this we can clearly see that Naismith's rule consistently overestimates walking speeds when descending a slope, and underestimates speeds when climbing a slope. When ascending or descending a slope, the RMSE of our GLM is similar to that of Tobler's hiking function. However, one of the main areas where we see an improvement using our model is on slight declines. Tobler's hiking function suggests that walking speed increases on mild descents up to a maximum of 6 km/h. It is clear from Fig \ref{Fig5}A, that Tobler's function overestimates the walking speed in this region. Campbell et al.'s function has a slightly lower RMSE value than our new model on the steepest walking slopes, however it underestimates the walking speeds on flat ground and mild slopes. Previous research has found that most walking takes place on low walking slopes \cite{Proffitt1995PerceivingSlant}, and this is evidenced by our data ($\sim$98\% of our data was from walking slopes of under 10 degrees). Improved walking speed predictions in this region therefore have the greatest impact in real-world situations. Within this region our model consistently has a lower RMSE than the existing functions, and a mean residual error close to 0 km/h. 

\begin{table}[!ht]
\centering
\caption{Comparison of new model against existing methods to calculate walking speeds.}
\begin{tabular}{|l|c|c|c|c|}
\hline
& New Model & Naismith & Tobler & Campbell\\
\hline
Average \% error & 23.68 & 26.36 & 26.17 & 25.33\\
\hline
MSE & 1.20 & 1.61 & 1.53 & 1.58\\
\hline
RMSE & 1.10 & 1.27 & 1.24 & 1.26\\
\hline
R\textsuperscript{2}  & 0.09 & -0.22 & -0.16 & -0.19\\
\hline
\end{tabular}
\label{tab:2comparison}  
\end{table}

\begin{figure}[!h]    
\begin{adjustwidth}{-2.25in}{0in} 
    \includegraphics[width=\linewidth]{Images/Paper/Fig5.eps}
    \captionsetup{width=1\linewidth}
    \caption[width=\textwidth]{{\bf Comparing RMSE values for the new model, Naismith's rule, Tobler's function and Campbell et al.'s function.} When: (A) travelling directly up or down hills of varying slope (all data), (B) traversing across hills of varying slope (all data), (C) travelling directly up or down hills of varying slope (off-road data only), (D) traversing across hills of varying slope (off-road data only). Campbell et al.'s function does not provide off-road speed estimates, so was not included in the off-road data comparisons.}
    \label{Fig5}
\end{adjustwidth}
\end{figure}

\begin{figure}[!h]
    \begin{adjustwidth}{-2.25in}{0in} 
    \includegraphics[width=\linewidth]{Images/Paper/Fig6.eps}
    \captionsetup{width=1\linewidth}
    \caption[width=\textwidth]{{\bf Comparing mean residual values for the new model, Naismith's rule, Tobler's function and Campbell et al.'s function.} When: (A) travelling directly up or down hills of varying slope, (B) traversing across hills of varying slope, (C)  travelling directly up or down hills of varying slope (off-road data only), (D) traversing across hills of varying slope (off-road data only). Campbell et al.'s function does not provide off-road speed estimates, so was not included in the off-road data comparisons.}
    \label{Fig6}
\end{adjustwidth}
\end{figure}

 We also see an improvement in RMSE when using our model to predict speeds for hill traversals (Fig \ref{Fig5}B). We can note from Fig \ref{Fig6}B that both Naismith's rule and Tobler's hiking function consistently overestimate the walking speed when traversing a slope, as they do not take into account the impact that the hill slope has on reducing walking speeds. The performance of Campbell et al's model improves as the hill slope increases, although we suggest this is more due to it underestimating the speed on shallow slopes. We do see that the average error in our model increases as the hill slope increases, but we believe that this is due to limited volumes of data at high hill slopes ($\sim$0.5\% of our data occurs on hill slopes steeper than 40 degrees). 

As well as looking at the overall performance of our new model, we looked to explore how well our model performed in off-road conditions, compared to the off-road adjustments for the existing functions (Naismith's reduced base speed of 4 km/h, and Tobler's correction factor of 0.6). Figs \ref{Fig5}C, D and \ref{Fig6}C, D show the RMSE and mean residuals, only considering data which was recorded in off-road conditions. From Figs \ref{Fig5}C and \ref{Fig6}C it is clear that Tobler's function consistently underestimates the walking speed when off-road. The factor of 0.6 is a larger reduction in walking speed than is observed in practice. As we found when looking at our data as a whole, Naismith's rule underestimates the walking speed when climbing a slope and overestimates when descending a slope. Our new model does not suffer from these problems, with both a lower RMSE and lower absolute mean residual value across all walking slopes. Both of these existing models also consistently underestimate walking speeds when traversing a slope, unlike our new model which has a mean residual of less than 0.4 km/h on slopes of up to 35 degrees. The error in predictions of our new model does increase as the hill slope increases, though the RMSE is generally lower than seen in the existing models. On the steepest hill slopes our model appears to perform less well than the existing ones, though only 0.2\% of our off-road data occurred on a hill slope steeper than 40 degrees. 

Although we have shown an improvement in walking speed predictions over short sections of routes, this did not translate to similar results when looking at predicted walking times for routes as a whole. Our model and all of the existing models which we have explored here had an average percentage error of 13.5\% - 15.5\% when predicting the time taken for a complete route. However, based on the errors seen in Figs \ref{Fig5} and \ref{Fig6}, we believe that this is a result of errors cancelling out over the course of a hike. For example while ascending a hill, Naismith's rule will underestimate the walking speed (and thus overestimate the walking time), but it will then overestimate the walking speed on the subsequent descent, leading to a relatively accurate total time estimate. The results here suggest that Naismith's rule, and other existing functions, are still a good rule of thumb to calculate route times as a whole, but time estimates for individual sections of a route will be less accurate than when using the new model found here.




\section{Discussion and New Perspectives}\label{sec:discuss}
% and Future Directions

In this section, we first discuss challenges and practical considerations, including non-stationarity, heterogeneity, unobserved confounders, subsampling, and expert knowledge.
Then, two new perspectives of temporal causal discovery are provided, which in our opinion will be a promising avenue for future research.

\subsection{Challenges and Practical Considerations}



\textbf{Non-stationarity of data:} We are often faced with non-stationarity in practical scenarios, where the probability distributions of temporal variables conditional on their causes or even the causal relations may change across time, especially for temporal data.
In this condition, causal discovery approaches presuming a fixed causal model may give misleading results. 
Whereas, several types of research have shown that non-stationarity contains information for causal discovery \cite{CD_from_change/conf/uai/TianP01, CD_from_change/peters2016causal, Discussion/Nonstation_hetero/ijcai_ZhangHZGS17, Discussion/Nonstation/state_space_icml_Huang0GG19}.
Thus, it's important to properly tackle the non-stationarity in applications.
Non-stationarity may result from the change of underlying systems and can be seen as a soft intervention \cite{soft_interv/korb2004varieties} done by nature. 
Following this idea, a line of work \cite{Discussion/Nonstation_hetero/ijcai_ZhangHZGS17, Discussion/Nonstation_hetero2/jmlr/Huang0ZRSGS20} leverages a surrogate such as time and domain index to account for nonstationarity where the causal relations are changed, and the CD-NOD framework is proposed. 
Instead of leveraging informative non-stationarity to causal structure learning, another set of research focuses on modeling time-varying relationships \cite{Discussion/Nonstation/pr_GaoY22}. 
Besides, the approach for slowly varying non-stationary process, such as evolutionary spectral and locally stationary processes, is proposed in \cite{Discussion/Nonstation/slowly_varying/du2020causal}.






\textbf{Heterogeneity of data:} In causal discovery for practical applications, the heterogeneity of data lies in two levels: (1) The interacting temporal processes are heterogeneous (having different distributions), for instance, causally related meteorological observations from different stations are influenced by several major weather systems separately \cite{Discussion/heterogeneous/pakdd_BehzadiHP19}. (2) The underlying generating process changes across data sets or different domains \cite{intro/nonts_surveys/glymour2019review}, for instance stock prices from different markets \cite{Discussion/Nonstation_hetero2/jmlr/Huang0ZRSGS20} or individual behaviours in different paradigms \cite{MTS/Attention/icdm_InGRA_ChuWMJZY20}.
For the first condition where the heterogeneity exists among temporal variables, the inferred relations of the traditional causal discovery approaches, which have been designed for specific homogeneous data types, may be inaccurate. As a remedy, several variants of Granger causality, based on methods such as generalized linear models and minimum message length, are proposed in \cite{Applications/anomaly/work2_icdm_BehzadiHP17, Discussion/heterogeneous/pakdd_BehzadiHP19, Discussion/heterogeneous/entropy/Hlavackova-Schindler20}.
For the second condition, a line of work \cite{Discussion/Nonstation_hetero/ijcai_ZhangHZGS17,  Discussion/Nonstation_hetero2/jmlr/Huang0ZRSGS20} leverages the distribution shift from heterogeneity as a soft intervention to assist causal structure learning, which is similar to that in non-stationary data.  
Whereas, another line of causal discovery approaches \cite{MTS/Attention/icdm_InGRA_ChuWMJZY20, Discussion/NewForm/ACD_LoweMSW22} in the second condition focuses on inductively modeling typical structure in heterogeneous data within an end-to-end framework. 



\textbf{Unobserved confounders:}
In practice, we are often met with cases where causal sufficiency is violated, \ie, there exist unobserved confounders. 
This challenging setting may lead to incorrect causal relations~\cite{MTS/FCM/VAR_LINGAM_extend2_icml_GeigerZSGJ15}.
As summarized in Table~\ref{tab:ts_category_overview}, most temporal causal discovery approaches cannot handle unobserved confounders in a straightforward way.
Several constraint-based approaches are designed without causal sufficiency and approaches
Besides, unobserved confounders are modeled by applying a structural bias in~\cite{Discussion/NewForm/ACD_LoweMSW22}.
Several recent studies termed as causal representation learning take a new perspective on unobserved confounders.
It will be detailed in subsection (\ref{subsection:causal_rep}).

\textbf{Subsampling:} In real-world applications, temporal data, especially time series, may be sampled at a rate lower than the rate of the underlying causal process due to the difficulties in data collection.
An ordinary causal discovery algorithm for sub-sampled time series may lead to spurious causal relations and missed ones. 
Several remarks and approaches~\cite{Discussion/subsample/work1, Discussion/subsample/work2_icml_GongZSTG15, Discussion/subsample/work3_nips_rateagnostic_PlisDFC15, Discussion/subsample/uai_subsample_aggr_GongZSGT17, Discussion/subsample/work5_pgm_constraintOPT_HyttinenPJED16, Discussion/subsample/biometrika/tank2019identifiability} are proposed for this issue.

\textbf{Expert knowledge: }Expert knowledge can help the causal discovery process in practice.  % 要强调practical issues.
The approaches of fusing expert knowledge can be categorized into three types~\cite{intro/nonts_surveys/BN21}: (1) \textit{Soft constraints}: the learning process can be influenced by the knowledge~\cite{Discussion/knowledge/ausai/ODonnellNHKAH06}. % (\ie, conditions given with a probability $0<p<1$).
(2) \textit{Hard constraints}: the learnt structure must conform to the enforced requirements (\ie, conditions given with a probability $p=0$ or $p=1$). 
In~\cite{Discussion/knowledge/artmed/AsvatourianLML20}, hard constraints are leveraged in structure learning with a time dependant exposure.
Studies in~\cite{MTS/SB/NTS_NOTEARS} add prior knowledge forbidding the existence of intra-slice dependencies, which is helpful to recover edges that are not explicitly encoded by the prior knowledge.
(3) \textit{Interactive learning}: the human input is leveraged in the learning process~\cite{Discussion/knowledge/ecsqaru/MessaoudLA09, Discussion/knowledge/kdd/MelkasSCMNMP21,https://doi.org/10.48550/arxiv.2206.05420, 9222294}.








\subsection{New perspectives}


\subsubsection{Extension in amortized and supervised paradigms}


In the traditional paradigms, causal discovery methods mostly either treat observational data separately or train a distinct model for each individual. 
These methods do not make full use of the common structure across different samples or supervised information from the datasets whose causal structures are clearly explored, thus suffering from several issues such as the small sample challenge and lack the inductive capability.
Recently, causal discovery is conducted in new paradigms to solve this problem. We can roughly categorize them into two groups: methods based on \textbf{amortized modeling} \cite{MTS/Attention/icdm_InGRA_ChuWMJZY20, Discussion/NewForm/ACD_LoweMSW22}, and methods based on \textbf{supervised learning} \cite{benozzo2017supervised, wang2022meta}.
We introduce them in this subsection, which we believe are a promising avenue for future research. 


In amortized modeling, a global causal discovery framework is trained for individuals with different causal structures. 
As for scenarios with temporal data, these approaches have been detailed in \ref{subsection:NN_Granger} as the deep learning extension of Granger causality with inductive modeling.
InGRA \cite{MTS/Attention/icdm_InGRA_ChuWMJZY20} leverages prototype learning to extract common causal structure while ACD \cite{Discussion/NewForm/ACD_LoweMSW22} proposes an encoder-decoder framework to conduct amortized causal discovery. These methods make full use of information from massive samples and are able to infer causal relations for newly arrived individuals, which are useful in real-world applications such as e-commerce, social network, and neuroimages.

Another line of work has predominately focused on treating the inference process as a black box and learning the mapping from sample data to causal graph structures via supervised learning. Here the label information is causal structure and can be easily accessed in synthetic datasets. 
Earlier work \cite{Discussion/NewForm/RCC/jmlr/Lopez-PazMR15, DBLP:conf/aaai/TonSF21} on learning causal relations by supervised learning is restricted to learning pairwise causal direction where the problem is cast into a classification task to distinguish between $X \to Y$ and $Y \to X$ by using observed samples.
It's later extended to discovery graph structure in \cite{Discussion/NewForm/DAG_EQ/corr/abs-2006-04697,petersen2022causal}.
As the labeled information for training is often originated from synthetic data or real-world datasets which have been explored, the requirement of a supervised approach, in which the distributions of training and test data match or highly overlap, is not guaranteed. In \cite{Discussion/NewForm/ML4S/kdd/00040DJWH022, Discussion/NewForm/CSIvA_DeepMind}, methods such as vicinal graph and meta-learning are leveraged in supervised causal discovery to tackle this `domain shift' issue.  
For the temporal setting, a supervised estimation of Granger causality between time series is proposed in \cite{benozzo2017supervised}. As a recent advance, a method for learning causal discovery is proposed in \cite{wang2022meta} where the learned from large datasets with known causal relations outperform the algorithm in the traditional paradigm when testing on temporal datasets such as fMRI. 
% It's also noted in \cite{wang2022meta} that the causal discovery algorithms in traditional paradigm depart from strong human assumptions about causality. In these approaches (such as constraint-based, score-based and Granger causality), human intuition is implemented in different form. 




% \subsubsection{Extension causal discovery towards causal representation learning (to edit)}
\subsubsection{Extension in causal representation learning}
\label{subsection:causal_rep}
% \subsection{Nonlinear ICA, causal representation learning...}

Extracting the causes of particular phenomena whether explicitly or implicitly from a deep learning black box can be beneficial to the downstream tasks.
The aforementioned causal discovery methods focus on inferring relations between observed variables, or start from the premise that the causal variables are given before hand.
Although some approaches learn causal relations under unobserved variables.
There exist real-world observations (e.g., sensor measurements, image pixels in video) which are not well structured to causal variables to begin with. 
As a generalization of causal discovery from observed variables, there has recently been a growing interest in \textbf{causal representation learning} \cite{CausalRepresentation/nontemp/icml/LocatelloPRSBT20, CausalRepresentation/nontemp/towardsCRL/ScholkopfLBKKGB21, CausalRepresentation/nontemp/CausalVAE/YangLCSHW21}, which aims at learning representation of causal factors in an underlying system.
It estimates latent causal variable graphs from observations.




A line of works in causal representation learning identifies independent factors of variations based on disentanglement and Independent Component Analysis (ICA).
At the heart of this methodology is the postulation of mutually independent latent factors.
It's hard to identify true latent variables, especially in general nonlinear cases.
As a remedy, recent approaches \cite{CausalRepresentation/nontemp/icml/LocatelloPRSBT20, CausalRepresentation/iVAE_nontemp/aistats/KhemakhemKMH20, DBLP:conf/aistats/HyvarinenM17, DBLP:conf/nips/HyvarinenM16} leverage additional information in multiple views, auxiliary variables, or temporal structure, combined with deep learning methods like VAEs and contrastive learning.
A connection between ICA and causality has been recently drawn in \cite{CausalRepresentation/IMA/nontemp/nips/GreseleKSSB21, DBLP:conf/uai/Monti0H19}.
In the context of temporal data, the identifiability of causal variables from temporal sequences is discussed in latent temporal causal process estimation (\textbf{LEAP}) \cite{Discussion/latent/iclr_LEAP_YaoSHS022}. It first provides causal identifiability conditions in a nonparametric, nonstationary setting, and a parametric setting. Then it proposes a learning framework to extract latent causal relations, which extends VAE with a learned causal process network by enforcing the assumed conditions.
The non-stationary noise, modeled by flow-based estimators, can be viewed as a soft intervention to aid identification.
In line with LEAP, subsequent works \cite{TDRL_DBLP:journals/corr/abs-2210-13647} extend the identification theory to a more general case.   % Change to NIPS form citation




Another line of work leverage intervention and data augmentation to help to identify latent causal relations. Under data augmentation, it's demonstrated in \cite{CausalRepresentation/line2/nips/KugelgenSGBSBL21} that common contrastive learning methods can block-identify causal variables that remain unchanged. 
For the temporal setting, \textbf{CITRIS} \cite{CausalRepresentation/CITRIS/icml/LippeMLACG22} is proposed. It's a VAE framework learning causal representation where latent causal factors have possibly been interved on.
By using intervention target information for identification, CITRIS is devoid of suffering from functional or distributional form constraints.
Besides, causal factors in CITRIS are considered as either scalars or potentially multidimensional vectors, which is more practical in complex scenarios. Along this line of work, instantaneous causal relations are extracted in iCITRIS \cite{CausalRepresentation/interv/iCITRIS/abs-2206-06169}.






















%\section{}
%\label{sec:resDir}


\section{Conclusion}
\label{sec:conclusion}
% <>
Since its advent in 1931, Koopman operator theory \cite{koopman:1931} has only recently been actively utilized for solving practical problems, thanks to the introduction of the DMD algorithm in 2008 \cite{schmid:2008}. Since then, a multitude of DMD algorithm variations have risen to prominence and found utility across various fields. A notable feature of our survey paper was reviewing and categorizing the results of over 100 research papers based on both application and algorithm type in smart mobility and vehicle engineering  (see Table~\ref{tab1} and Section~\ref{sec:vehicApp}).  Additionally, this survey paper identified potential research gaps in smart mobility and vehicular engineering applications (Remarks~\ref{remGap1}--\ref{remGap6}). Finally, this review paper discussed theoretical aspects of Koopman operator theory that have been largely neglected by the smart mobility and vehicle engineering community and yet have large potential for contributing to solving open problems in these areas (see Section~\ref{subsec:theorIssue}).

\noindent{\textbf{Future Research Directions.}}	Given the emergence of cyber-threats against connected and autonomous vehicles as well as robotic systems (see, e.g.,~\cite{nekouei2021randomized,mohammadi2022generation}), a future research direction might include utilizing Koopman operator-based algorithms for designing cyber-resilient vehicular and smart mobility applications (see, e.g.,~\cite{taheri2022data} for a related line of research). Another potential research direction is using Koopman operator-based algorithms for predicting the motion of vulnerable road users (VRUs), e.g., pedestrians and cyclists (see, e.g.,~\cite{pool2019context,scholler2020constant}). Finally, rehabilitation robotics and robotic exoskeletons can be the benefactors of the predictive capabilities of Koopman operator-based algorithms for detecting tripping events and/or system  identification in various modes of locomotion (see, e.g.,~\cite{kumar2019extremum,aprigliano2019pre}).



%Fig. 1 depicts the accumulation of such algorithms since 2014, which are particular to vehicle engineering and smart mobility, i.e., the focus of this review. Table 1 summarizes the varieties of relevant algorithms developed in those studies. Furthermore, we have highlighted theoretical issues, whose expansion will have potential applications to the wide research area of smart mobility and vehicle engineering.  

%Although fairly comprehensive, we have found several gaps in this research area. In particular, we could not find any studies related to elevators, robots/vehicles employing crawling, slithering, hopping or peristaltic locomotion, arctic or special-terrain vehicles such as those employing screws or tracks, hovercraft and other amphibious vehicles or subsystems which tolerate flexible environments, classification or guidance systems related to vehicles for drilling or agriculture, or for current-ripple, power-split, battery health monitoring, nuclear propulsion, exoskeletons/prosthetics, personal mobility, motorsports, specialized rovers or similar open problems in emerging areas.  These examples are, of course, not exhaustive.  
%
%The purely data-driven nature of Koopman operators holds the promise of capturing unknown and complex dynamics for reduced-order model generation and system identification, through which the rich machinery of linear control techniques can be utilized. The emergent nature of the smart mobility and vehicular-related applications, where  the Koopman operator  in each particular application needs to be approximated, implies that the development of various Koopman operator approximation  algorithms is expected to grow along with the vehicular problems they aim to solve.  Given the ongoing development of this research area and the many existing open problems in the fields of smart mobility and vehicle engineering, a survey of techniques and open challenges of applying Koopman operator theory to this vibrant area is warranted.  To the best of our knowledge, this survey paper is the \emph{first of its kind} reviewing the applications of Koopman operator theory within a focused research area, namely, smart mobility and vehicle engineering applications. A \emph{notable feature} of our survey paper is reviewing and categorizing the results of over 100 research papers based on both application and algorithm type  (see Tables~\ref{tab1}--~\ref{tab4} and Section~\ref{sec:vehicApp}) that are concerned with the applications of Koopman operator theory to the field of smart mobility and vehicular engineering. Such a \emph{comprehensive and  detailed categorization} will be beneficial to the research practitioners working in the field.  Furthermore, this review paper discusses theoretical aspects of Koopman operator theory that have been largely neglected by the smart mobility and vehicle engineering community and yet have large potential for contributing to solving open problems in these areas. Additionally, our survey paper seeks to \emph{identify gaps} in the smart mobility and vehicle engineering research where new and existing Koopman operator-based methods have the potential to further develop and address unsolved problems  potentially benefiting from the perspectives of nonlinear system identification, control, global linearization, and the predictive powers that Koopman operator theory has to offer (see, e.g., Remarks~\ref{remGap1}--\ref{remGap6}). 


% Include only the SI item label in the paragraph heading. Use the \nameref{label} command to cite SI items in the text.
\section*{Acknowledgments}

Preprocessing of the GPX files made use of the resources provided by the Edinburgh Compute and Data Facility (ECDF) \cite{Eddie2022website}.


\nolinenumbers

\bibliography{Bibliography}

\section*{Supporting information}

\paragraph*{S1 Appendix.}
\label{S1_Appendix}
{\bf Data Sources Table}

\paragraph*{S2 Appendix.}
\label{S2_Appendix}
{\bf Exploratory data modelling study.} 

\paragraph*{S3 Appendix.}
\label{S3_Appendix}
{\bf Exploring the differences between Scotland and the rest of the UK.} 

\paragraph*{S4 Appendix}
\label{S4_Appendix}
{\bf Exploring the impact of terrain obstruction.}
\section*{S1 Supporting Information. Data sources table}

\begin{table}[!h]
\begin{adjustwidth}{-1in}{0in}
\centering
\caption{Summary of data sources used during this work}
\begin{tabular}{| L{0.17\linewidth} | L{0.21\linewidth} | C{0.15\linewidth} | L{0.35\linewidth}|}
\hline
\centering Data Type & \centering Data Source & Download Date & \centering Notes \tabularnewline
\thickhline 
\raggedright Hikr GPS data & Hikr.org & 13-09-2021 & Within the UK data\textsuperscript{1}, only tracks which took place within Scotland\textsuperscript{2} were used for exploratory study (see \nameref{S3_Appendix}).\\
\hline
OpenStreetMap GPS data & OpenStreetMap.org & 05-08-2021 & Accessed using planet.gpx
regional extracts\textsuperscript{3}\\
\hline
Ordnance Survey elevation data & Ordnace Survey Terrain 5 DTM & 06-07-2021 & Accessed using EDINA Digimap Ordnance Survey Service\textsuperscript{2}\\
\hline
OpenStreetMap road data & OpenStreetMap.org & 04-08-2021 & Accessed using planet.osm
regional extracts\textsuperscript{4}
\\ 
\hline
England lidar data & National LIDAR Programme & 16-09-2021 & 2m resolution data was used and accessed using EDINA LIDAR Digimap Service\textsuperscript{2}\\ 
\hline
Wales lidar data & LIDAR terrain and surfaces models Wales & 16-09-2021 & 2m resolution data was used and accessed using EDINA LIDAR Digimap Service\textsuperscript{2}\\ 
\hline
\end{tabular}
\\
\begin{flushleft} 
Note: In regions where lidar data was available as part of both the England and Wales lidar datasets, the data values from England were used.\\
\textsuperscript{1} \url{https://www.hikr.org/region516/ped/?gps=1}\\
\textsuperscript{2} \url{https://www.hikr.org/region518/ped/?gps=1}\\
\textsuperscript{3} \url{http://zverik.openstreetmap.ru/gps/files/extracts/europe/great_britain.tar.xz}\\
\textsuperscript{4} \url{https://digimap.edina.ac.uk}\\
\textsuperscript{5} \url{http://download.geofabrik.de/europe/great-britain.html}\\
\end{flushleft}
\label{}
\end{adjustwidth}
\end{table}


\section*{S2 Appendix. Exploratory data modelling study.}
\label{1Modelling}

Data from Scotland was used for exploratory testing of different modelling approaches. When doing this, an earlier iteration of the break finding and data filtering process was used. Data were processed as described in the main document, with the following exceptions:

\begin{itemize}
    \item GPS track segments had to be fully contained in the reduced `Scotland' OS grid tiles described in the main document.
    \item Track segments where the median speed was greater than 10 km/h were not automatically removed prior to break identification.
    \item  Individual points representing 10 minutes of travel were tagged as breaks (3 minutes was used in later versions).
    \item High speed (\textgreater10 km/h) points occurring immediately following a long (\textgreater3 minute) point were not automatically tagged as breaks.
    \item When merging data into 50 m sections, sections under 50 m in length immediately preceding a break or the end of the segment were also tagged as breaks, rather than combined with the previous section.
    \item After merging the data into 50 m sections, any section with a speed above 10 km/h which was at the start or end of a segment, or next to a break point were only considered to be part of the break in the OSM data, not the Hikr data.
    \item A minimum distance of 250 m of travel was included between breaks. Sections with a distance below this were tagged as a break.
    \item When looking for `key points' to filter out non-walking track sections, the duration required to be tagged as a `key point' was 10 minutes (3 minutes was used in later versions).
    \item Duplicate track segments were not identified or removed.
    \item The fastest and slowest 0.5\% of the merged datapoints were not removed as outliers.
\end{itemize}

Once the data were filtered and processed, we were able to use them to test models for the walking speed. Two different approaches were explored in order to model the data: a generalised linear model (GLM) and a generalised additive model (GAM).

We know that predictions for walking speeds must be non-negative, and two different setups were explored to achieve this: a Gaussian distribution with log link function and a Gamma distribution with inverse link function. The GAM approach was also deployed with both thin plate spline or cubic regression basis functions.
Investigations into different models showed that there was no improvement to model fit beyond cubic terms in a GLM, or 7 knots in each GAM smoothing term, so more complex models than this were not considered for selection.

Both model types were created in R:

\begin{equation*}
    \begin{aligned}
    &\texttt{glm}(v \sim a\phi + b\phi^2 + c\phi^3 + d\theta + e\theta^2 + f\theta^3, \texttt{distribution}) \\
    &\texttt{gam}(v \sim s(\phi,k,b) + s(\theta,k,b), \texttt{distribution})\\      
    \end{aligned}
\end{equation*}

where
\begin{quote}
$v = \text{walking speed}$\\
$\phi = \text{hill slope angle (degrees)}$\\
$\theta = \text{walking slope angle (degrees)}$\\
k = knots used in spline (up to 7)\\
b = basis function\\

\end{quote}

Initially, 10-fold cross-validation was used to compare the model parameters, looking at R-squared values, root-mean-squared error (RMSE) and mean absolute error. All models produced very similar results, with no change in the RMSE to 2 decimal places, although there was a general trend of marginal improvements as the model complexity increased. As no best model could be chosen based on the cross-validation, each was checked in more detail. Firstly, the hill slope component was isolated by investigating the speed when the walking slope was zero (i.e when traversing across a slope). Intuitively, and from experience, this should be a decreasing function; as the slope gets steeper it is harder to traverse, so the walking speed will decrease. Models which failed to predict this were removed under the assumption that the data were overfitted. Following this, the walking slope component was investigated, specifically looking at the walking speed when travelling directly up- or down-hill. By inspection of the data, existing functions, and intuition, this should be modelled as a roughly bell-shaped function with the peak at, or close to, 0 degrees. Any models which predicted an increase in speed as walking slope steepness increased (from a minimum magnitude of 10 degrees) were removed. Secondly, we know from existing work that there exists a critical gradient at a walking slope of around 15 -- 21 degrees, at which it becomes more efficient to zig-zag up a steep hill rather than going directly uphill. Models which failed to predict the critical gradient occurring below 21 degrees when travelling uphill were also removed.

This resulted in 21 model configurations remaining, although it is clear from Fig \ref{fig6} that the speed predictions are very similar in most circumstances. Fig \ref{fig6}A shows that all of the remaining models predict very similar speeds when traversing a slope of up to 40 degrees, after which there is more deviation in predictions. Similarly, when travelling in the slope direction (Fig \ref{fig6}B), all of the models are broadly similar on slopes up to approximately $\pm15$ degrees. More than 96\% of the data is contained within this area, and the relative lack of data outside this region explains the divergent speed predictions. As all of the models provided both very similar R-squared values and very similar predictions over the vast majority of the dataset, we used the following points to make our final selection:
\begin{itemize}
    \item It is easier to apply GLMs than GAMs to future work, as a simple formula to predict the walking speed can be produced for application elsewhere, without needing to recreate the model from the original data.
    \item In general, simpler models are easier to interpret, and we had no clear evidence that a more complex model would perform better.
\end{itemize}

\begin{figure}[!h]
    \includegraphics{Images/Paper/Fig6.eps}
    \caption{{\bf Walking speed predictions from 21 possible models (coloured individually) generated from the Scotland GPS dataset.} (A) Walking speed predictions for traversing across hills of varying slope, overlaid on GPS data where walking slope is below 5 degrees. (B) Walking speed predictions for travelling directly up or down hills of varying slope, overlaid on GPS data where walking slope is within 5 degrees of hill slope.}
    \label{fig6}    
\end{figure}


\include{SectionText/S3_Appendix}
\section*{S4 Supporting Information. Exploring the differences between Scotland and the rest of the UK.}
\label{Scotland VS UK Differences}

When applying the GLM formulation separately to datasets covering Scotland and the rest of the UK (ROUK) (Fig \ref{Fig 7}), we see faster predicted walking speeds in the ROUK model than in the Scotland model when traversing a slope, or when walking uphill. Before building a model on a combined dataset, we wanted to check whether the Scotland dataset was a reasonable subset of the ROUK data.

\begin{figure*}[!h]
    \includegraphics{Images/Paper/Fig7.eps}
    \caption{Comparison of walking speed models produced using data from Scotland and the rest of the UK. Walking speed predictions when: (A) travelling directly up or down hills of varying slope, (B) traversing across hills of varying slope.}
    \label{Fig 7}
\end{figure*}

To take into account the fact that the ROUK datset was much larger than the Scotland dataset (7636 tracks vs 648), we took 100 samples of 650 tracks from the ROUK dataset (to form sample sets of comparable size to the Scotland data) and modelled the walking speed for each one. The results are visualised in Fig \ref{Fig 8}.

\begin{figure*}[!h]
    \includegraphics{Images/Paper/Fig8.eps}
    \caption{Comparison of walking speed models produced using data from Scotland against 100 sampled datasets from the rest of the UK. Walking speed predictions when: (A) travelling directly up or down hills of varying slope, (B) traversing across hills of varying slope.}
    \label{Fig 8}
\end{figure*}

When we look at traversing a hill, it is clear that the two datasets are distinct, as the model for Scotland is outside the range of results seen in the ROUK sample models. Further investigations showed that there were differences between the proportions of paved road, unpaved road and off-road data within each set. To take this into account, we modelled these individually, once again sampling the ROUK dataset 100 times to form sample sets of comparable size to the Scotland dataset (sample sizes of 600 for paved roads and 450 for unpaved and 200 tracks for off-road), and the resulting models can be seen in Fig \ref{Fig 9}.

\begin{figure*}[!h]
    \begin{adjustwidth}{-1.75in}{0in} 
    \includegraphics{Images/Paper/Fig9.eps}
    \captionsetup{width=1\linewidth}
    \caption[width=\textwidth]{Comparison of walking speed models produced using data from Scotland against 100 sampled datasets from the rest of the UK. Walking speed predictions when: (A) travelling on paved roads directly up or down hills of varying slope, (B) traversing on paved roads across hills of varying slope, (C) travelling on unpaved roads directly up or down hills of varying slope, (D) traversing on unpaved roads across hills of varying slope, (E) for travelling off-road directly up or down hills of varying slope, (F) traversing off-road across hills of varying slope.}
    \label{Fig 9}
    \end{adjustwidth}
\end{figure*}

We can clearly see now that our model for paved roads in the Scotland data is comfortably within the range of samples of the ROUK data. It is reasonable to suggest, therefore, that there is no difference in walking on a paved road in Scotland compared to the rest of the UK. However, our unpaved road model and off-raod models for Scotland lie at the extreme edge, or outside of the range of sample models taken from the ROUK unpaved data.

Before finally determining that the two datasets were distinct, we wanted to see if we could find another variable which would account for the lower walking speeds seen in Scotland on both unpaved roads and when off-road. We explored whether this elevation could be responsible for this difference, as higher elevations are likely to have higher exposure, and be more affected by inclement weather, leading to slower walking speeds.

There was a much greater proportion of data at high elevation (\textgreater500 m) in the Scotland dataset than the ROUK dataset on both unpaved roads and when off-road, while a silimar difference was not seen on paved roads (where our models were equal) - Fig \ref{Fig 10}.

\begin{figure*}[!h]
    \begin{adjustwidth}{-1.25in}{0in} 
    \includegraphics{Images/Paper/Fig10.eps}
    \captionsetup{width=1\linewidth}
    \caption{Comparing elevations of track sections between Scotland and the rest of the UK. (A) Scotland paved roads. (B) ROUK paved roads. (C) Scotland unpaved roads. (D) ROUK unpaved roads. (E) Scotland off-road. (F) ROUK off-road.}
    \label{Fig 10}   
    \end{adjustwidth}
\end{figure*}

For this reason we included elevation as a model variable, both as a continuous variable or as a factor variable classifying all points as either high elevation or low elevation (where high elevation consisted of all data \textgreater500 m). However, in both cases we found this to not be a significant factor in the model. Based on the sample data taken, we suggest that the model formulated using the Scotland data is an extreme sample of the ROUK data, where a greater-than-average portion of the data has been sampled from high elevation regions. However, the high elevations themselves are not the cause of the difference between the model coefficients. 


\section*{S5 Supporting Information. Exploring the impact of terrain obstruction.}

To understand how terrain obstruction might impact the walking speed (and thus how it should be incorporated into a model), we conducted an initial exploration into the data. Before doing this, however, we wanted to check that there was not a systematic difference between the walking speeds in regions where we had lidar data, and regions where we did not. If the two regions were not found to be different, then any findings about the effects of terrain obstruction in regions where we had lidar data could also be applied to areas where we didn't have the data. 

When modelling the data for the separate datasets (`obstruction available' vs `no obstruction available'), we see that the models are very similar when ascending or descending a slope (Fig \ref{Fig 11}A). This was not the case when traversing the slope however, as the `no obstruction available' model predicts that hill slope has a greater impact on reducing walking speed than the `obstruction available' data model (Fig \ref{Fig 11}B).

\begin{figure}[!h]
    \includegraphics{Images/Paper/Fig11.eps}
    \caption{{\bf Comparison of off-road walking speed models where obstruction data is, or is not, available.} Walking speed predictions when: (A) travelling directly up or down hills of varying slope, (B) traversing across hills of varying slope.}
    \label{Fig 11} 
\end{figure}

We sampled our larger (`obstruction available') dataset, so that we had a similar number of tracks as in our smaller dataset, and compared models made from more equal volumes of data. When doing this (Fig \ref{Fig 12}), we found that the `no obstruction available' model is within the range of sample models for traversing the slope, albeit at an extreme end. This is likely due to the low volume of data which we had at high hill-slopes. (Only 50 km of data had a hill slope greater than 15 degrees with no lidar data available, and only 130 km with lidar data available). Going forward, we assumed that the regions where we had lidar data were representative of all off-road regions, and so any findings could be applied to both areas.

\begin{figure}[!h]
    \includegraphics{Images/Paper/Fig12.eps}
    \caption{{\bf Comparison of off-road walking speed models produced using a dataset where obstruction data isn't available against 100 sampled datasets where obstruction data is available.} Walking speed predictions when: (A) travelling directly up or down hills of varying slope, (B) traversing across hills of varying slope.}
    \label{Fig 12}   
\end{figure}

To explore the effects of terrain obstruction, we first looked at the range of speeds across the different obstruction values. The data was split into 25 quantiles, and the average walking speed for each was calculated. The results are shown in Fig \ref{Fig 13}A. This shows us two things; firstly the vast majority of our data had very little, or no obstruction (as most of the quantile points occur below 0.5 m of obstruction). Secondly we can see that there is a very steep drop off in walking speed initially, and it then remains relatively constant across obstruction levels. Our initial assumption was that walking would be relatively easy with no, or very little obstruction, and then much slower at obstruction values of approximately 0.5 m - 4 m when it would involve walking through thick vegetation, before getting slightly faster again at higher obstruction values (as you would be walking through a forest and could walk between the trees below the canopy). The data shows this not to be the case, although this may be a result of our data only showing us regions where walking was possible. Due to the crowdsourced nature of our GPS tracks, we had no data showing us the walking speed when in 2 m of thick vegetation, as it is very unlikely that people would have chosen to walk there.

\begin{figure}[!h]
    \includegraphics{Images/Paper/Fig13.eps}
    \caption{{\bf Binned average walking speeds across different levels of obstruction.} (A) Full range. (B) Zoomed range. Each bin contains 1/25th of the datapoints.}
    \label{Fig 13}    
\end{figure}

Fig \ref{Fig 13}B shows a close-up of the steep speed drop off, and we can see that the average speed dropped from approximately 4.8 km/h when there was no obstruction down to about 4 km/h once there was more than 10 cm of obstruction. We used this information to classify all points into heavy obstruction (\textgreater10 cm) or light obstruction ($<=10$ cm). Although the figure suggests a gradual decrease in walking speed between 0 and 10 cm of obstruction, we chose not to model this. Vegetation length is highly variable throughout the year, and it is more practical to classify regions as light or heavy obstruction when discussing walking speeds.
    


\section*{S6 Supporting Information. Comparison of walking speed changes while crossing a simulated off-road terrain region.}

\begin{figure}[!h]
    \begin{adjustwidth}{-1.5in}{0in} 
    \centering
    \includegraphics[width=0.86\linewidth]{Images/Paper/Fig16.eps}
    \captionsetup{width=1\linewidth}
    \caption[width=\textwidth]{{\bf Comparison of walking speed changes while crossing a simulated off-road terrain region.} (A), (B) The simulated route (green) across the terrain. Terrain is coloured by elevation value from low (dark) to high (light). (C) The elevation profile of the route, (D) The walking slope profile of the route, (E) The hill slope profile of the route. (F) Walking speed predictions for different models as the route is traversed. For Naismith's and Tobler's functions, the off-road variants of the models have been used. For the new model, the off-road unknown obstruction coefficients have been used.}
    \label{Fig16}   
\end{adjustwidth}
\end{figure}

\end{document}

