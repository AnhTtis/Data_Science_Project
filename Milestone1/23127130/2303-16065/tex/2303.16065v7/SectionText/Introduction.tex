\section*{Introduction}
\label{0Background}
Knowing how fast people are able to walk between locations is critical information in many situations. In hiking and hillwalking scenarios, this information is vital for safety reasons. If you are leaving in the morning for a hike then it is good practice to provide an estimated return time such that emergency services can be contacted if you get into difficulty and do not return \cite{Mountaineering2023Route}. An inaccurate estimate for how long a route will take could lead to unnecessary callouts, or delay a callout in a situation where every minute is important. Furthermore, in circumstances where a hiker has gone missing, an accurate measure of walking speed can help to restrict a potential search area around a last known location. Finally, when out on a hike there are situations where hikers may be deciding whether to follow a footpath, or take a more direct cross-country route. Accurate estimates of the walking speed and time for both scenarios are required to be able to select the optimal route.

There are a multitude of factors which can impact the walking speed and time predictions for a route \cite{Mountaineering2023Planning}, although these can generally be split into two categories \cite{Miao2023AnalysisEnvironment, Liang2020HowAreas}. The first category covers the individual effects which depend on who precisely is undertaking the walk, and when they are doing it. These effects include group size (larger groups often walk slower), age or fitness of  participants, and weather conditions, as well as the aim of the walk (afternoon stroll vs. specific hike). The second category covers the fixed effects which will affect all individuals who attempt the same route. These include how steep the terrain is and whether the route is paved, along a track or in wild country.

Most of the individual effects cannot be modelled without considerable prior knowledge about the person who is planning a route. Therefore, most existing hiking route planners calculate the walking speed solely based on the terrain, and this is presented as the average time (or time range) it takes to complete a hike. It is then left up to the individual to tune the predicted time for a hike given their knowledge about personal ability and circumstances.

Formulae of varying complexity have been proposed to estimate human walking speed and time along a projected path. A popular early method that is still widely used was put forward by Naismith \cite{Naismith1892CruachMore} which calculates walking time under normal conditions as:
\begin{quote}
``\textit{an hour for every three miles on the map, with an additional hour for every 2,000 feet of ascent.}''
\end{quote}
This approximates to a walking speed of 5 km/h with 10 minutes added on for every 100 m of ascent. This was later adjusted by Aitken \cite{Aitken1977WildernessScotland}, who introduced a reduced base movement speed of 4 km/h on surfaces which are not paths or roads. Naismith's rule is still used today by Scout groups and other casual hikers due to the ease of calculating walking time by hand using a paper map. However, despite the widespread use, Naismith's rule does have a well-known limitation; namely that the predicted speed does not change when descending a hill, regardless of the gradient. 

An alternative hiking function proposed by Tobler \cite{Tobler1993ThreeModelling}, has become more popular in recent research and other situations where speeds do not need to be calculated by hand:
 
\begin{equation*}
        W = 6*exp(-3.5|S + 0.05|),
\end{equation*}
where
\begin{quote}
    W = velocity (km/h)\\
    S = gradient of slope.
\end{quote}

Like Naismith's rule, this gives a speed of 5 km/h on flat ground, with a maximum speed of 6 km/h on a mild descent (around 3 degrees). In a similar manner to Aitken's correction, a factor of 0.6 is applied to the calculated speed for all off-road travel. Tobler's function avoids the issues seen in Naismith's rule when descending slopes, but it predicts a sharp peak in walking speed on mild descents, which may be unrealistic. The formulae discussed here are directly compared in Fig \ref{Fig1}.

\begin{figure}[!h]
    \includegraphics[width=\textwidth]{Images/Paper/Fig1.eps}
    \caption{{\bf Existing functions used to calculate walking speed.} Naismith's rule \cite{Naismith1892CruachMore}, Tobler's hiking function \cite{Tobler1993ThreeModelling} and Campbell et al.'s function \cite{Campbell2022PredictingData} plotted as predicted walking speed in km/h against the slope in the direction of travel (walking slope) in degrees where positive is uphill. For Naismith's function and Tobler's function, on and off-path versions are shown.}
    \label{Fig1}
\end{figure}

Other studies have also looked at providing alternative methods to calculate walking speeds \cite{Irmischer2018MeasuringNavigation, Rees2004Least-costTerrain, Davey1994RunningApplications}, but all continue to use walking slope as the main variable to determine walking speed (with various multiplicative factors applied for off-road travel). 

When exploring speeds of fell-runners, Arnet \cite{Arnet2009ArithmeticalJapan} suggested that movement velocity may be dependent on three factors: obstruction (with different factors applied depending on the kind of obstruction), ascent in the run direction (walking slope) and slope of the terrain (hill slope). The actual values used in Arnet's calculations cannot be directly applied to walking speeds as they were based on orienteering championships where participants were running.

Experience tells us that traversing on a steep hill (while maintaining constant elevation) is more difficult than traversing flat ground. However, the existing methods estimate the same walking speed for both situations. Similarly, high levels of terrain obstruction in off-road areas (such as a thick gorse bush) are much more difficult to walk through than empty fields. The simple multipliers for off-road travel in Aitken’s correction and Tobler’s function do not provide any further distinction between two such regions. 

Wood and Schmidtlein \cite{Wood2012AnisotropicNorthwest}, took all three of Arnet's factors into account, and looked at evacuating citizens in the event of a hurricane. They applied Tobler's function to both the hill slopes and walking slopes, and calculated the terrain obstruction coefficients based on energy usage rather than walking speed (using \cite{Soule1972TerrainPrediction.}). They accepted that these were likely not the correct values, but were unable to find any better alternatives. Campbell, Dennison, and Butler \cite{Campbell2017AMapping} conducted a study using lidar data to explore the effects of ground roughness and vegetation density on firefighter evacuation speeds, but they did not consider the hill slope separately.

All of the studies mentioned above utilised relatively small sample sizes. However, the rise in use of global navigation satellite systems (GNSS), more frequently referred to as GPS tracking, means that a data-driven approach to modelling walking speed is now possible, which provides two main benefits. Firstly, it is possible to access GPS tracks from a wide variety of regions and terrains. Secondly, each track can easily be broken down into individual sections, enabling specific route features to be investigated at much higher spatio-temporal resolution. This has been explored in recent work \cite{Campbell2019UsingRates, Campbell2022PredictingData}, however the crowdsourced nature of these studies meant that data collection was not controlled, and thus that the data could not be assumed to consist wholly of walking or hiking tracks. In \cite{Campbell2019UsingRates}, data from hikes, jogs and runs was processed together, resulting in a very wide range of movement speed estimates. Campbell et al. attempted to overcome this in \cite{Campbell2022PredictingData} by only considering data points with a speed between 0.2 m/s and 5 m/s  (and the resulting model is shown in Fig \ref{Fig1}). However, 5 m/s (18 km/h) is much higher than the maximum predicted speeds from existing methods (such as Naismith's rule), so it is likely some non-walking data remained. Furthermore, applying a blanket 0.2 m/s minimum speed may well overlook valid datapoints recorded by particularly slow individuals, or in especially difficult regions. Finally, although these studies had the benefit of using large sample sizes, they both looked solely at the effect of the walking slope on speed, and did not explore additional variables. 

Here we used a data-driven approach to explore the impact of all three factors discussed by Arnet on walking speeds. These are the walking slope, the hill slope and the terrain obstruction. We aimed to use these factors to develop a model for the walking speed for an average individual. As with the existing methods, this model did not seek to model individual effects, and would still require tuning based on personal ability or conditions.


