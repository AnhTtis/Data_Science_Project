\section*{Materials and methods}

\subsection*{Data set, cleaning and key assumptions}
\label{2Data set cleaning and key assumptions}
Full details of the various datasets used in this study are provided in \nameref{S1_Appendix}. Further, a detailed description of the data filtering processes, and choices/assumptions made during data processing are described in \nameref{S2_Appendix}.

In summary, GPS tracks were obtained for hikes in the UK from Hikr.org \cite{Hikr.org2021UnitedReports} and OpenStreetMap (OSM) \cite{OpenStreetMap.org2021Tracks}. Elevation and walking slope values were calculated and added to every GPS point using data from the Ordnance Survey Terrain 5 Digital Terrain Map (DTM), which provides elevation data at 5 m intervals across the whole of the UK \cite{OrdnanceSurvey2020Terrain5}. Hill slope values were found using the quadratic surface method \cite{Zevenbergen1987QuantitativeTopography, Dunn1998TheGIS}. Each data point was then classified as on a paved road, on an unpaved road, or off road, determined by searching a 50 m radius around each point in an OSM Road dataset \cite{OpenStreetMap.org2021Data}. Paved and unpaved road classification was determined using \cite{OpenStreetMapWikiRoads}, with the unpaved road values being `path', `bridleway' and `track'.

Terrain obstruction information was calculated using lidar datasets \cite{LIDARDSMEngland, LIDARDTMEngland, LidarWales}, as the difference in values between a Digital Surface Map (DSM) and Digital Terrain Map (DTM). This meant that any physical feature which protruded from the ground was regarded as an obstruction. We had access to lidar data at 2 m resolution covering large areas of England and Wales, but the coverage was not complete. Of our off-road data ($\sim$2,900 km, spread across over 1,200 tracks), over 2,000 km had lidar data available. Exploration of the lidar data (see \nameref{S5_Appendix}) showed that there was a clear drop in walking speeds once the height of an obstruction was greater than 10 cm, beyond which the speed was relatively constant. We used this information to classify points into heavy obstruction (\textgreater 10 cm) or light obstruction ($<= 10$ cm) for modelling purposes.

Visual inspection of the tracks showed that a large number contained long breaks which could impact the accuracy of a walking speed model. Fig \ref{Fig2} shows examples of regions where breaks are visible in a GPS track, and the process developed to identify these regions is outlined in Algorithm \ref{BreakfindingROUKAlg}.

\begin{figure}[!h]
    \begin{adjustwidth}{-1.75in}{0in} 
\includegraphics{Images/Paper/Fig2.eps}
\captionsetup{width=1\linewidth}
    \caption[width=\textwidth]{{\bf A GPS track where 3 breaks can be identified by finding point clusters.} Clusters of points can form on a GPS track when a break is taken during a hike. By identifying these clusters as potential breaks we are able to remove most break periods from the datasets used for our analysis of walking speeds. For full details of these and other data filtering methods see \nameref{S2_Appendix}. Background images from OpenStreetMap and OpenStreetMap Foundation \cite{OpenStreetMapMAPS}, visualised using QGIS \cite{OpenSourceGeospatialFoundationProjectQGIS.org2020}.}
    \label{Fig2}
    \end{adjustwidth}
\end{figure}

\begin{algorithm}
\caption{Breakfinding process for a GPX track segment}
\label{BreakfindingROUKAlg}
\begin{algorithmic}[1]
\State Breakpoint\_list $ = \emptyset$
\State Find the median distance ($r_{median}$) and speed ($s_{median}$) of the segment
\For{point ($p_{i}$) in segment}
    \State Calculate travel direction quadrant and point angle
    \State Calculate break likelihood using the point speed and angle
    \If{speed == 0 or distance \textgreater 1 km or duration \textgreater 3 minutes}
        \State Breakpoint\_list += $p_{i}$
    \EndIf
    \If{speed \textgreater 10 km/h and duration($p_{i-1}$) \textgreater 3 minutes}
         \State Breakpoint\_list += $p_{i}$
    \EndIf
\EndFor
\For {point ($p$) in segment}
    \If{Neighbourhood of $p$ is a cluster ($C$)} \Comment{See Defs 1 \& 2, \nameref{S2_Appendix}}
        \For{point ($p_{c}$) in $C$}
            \If{Neighbourhood of $p_{c}$ is a new cluster ($C_{n}$)}
                \State $C = C \cap C_{n}$
            \EndIf
        \EndFor
        \State Remove points at the ends of the cluster with low break likelihood
        \State Add ‘missing’ points to the cluster (to make a continuous run of points) to form a Potential Break (B*)
        \If{less than half the points in B* have low break likelihood and there is travel in opposite quadrants (Q1 \& 3 or Q2 \& 4)}
             \State{Breakpoint\_list += B*}
        \EndIf
    \EndIf
\EndFor
\end{algorithmic}
\end{algorithm}

Where the datapoints in the original GPS track were under 50 m in length, they were merged together to minimise the effects of errors in the GPS location values. While doing this, the resulting distance was the sum of all distances in the constituent GPS points, so may be longer than the straight line distance between co-ordinates. Similarly, both hill and walking slope values, as well as obstruction height, were calculated as the weighted average of constituent points, weighted by point duration.

While the Hikr dataset consisted of tracks which were tagged as a walk or hike, within some of these there were segments where it was clear that the participant was driving to or from the hike location, based on the observed speeds. The OSM data, on the other hand, was not filtered by transport type. There were a large number of tracks which were clearly from faster modes of transport, as their speed was implausible for a hiker. A process to remove these non-walking tracks and segments was created, whereby the known Hikr walking segments were used to create filtering bounds of plausible walking speeds, which could then be applied to the remainder of the dataset. This process is summarised in Algorithm \ref{FilteringROUKAlg}.

\begin{algorithm}
\caption{Filtering process for GPS data from Hikr and OpenStreetMap}
\label{FilteringROUKAlg}
\begin{algorithmic}[1]
    \State Remove duplicate segments (containing sections with identical start location, end location, start time and duration)
    \State Remove all segments with median speed \textgreater 10 km/h
    \State Remove all breaks with duration \textgreater 30 seconds
    \State Remove all breaks containing points with speed \textgreater 10 km/h or distance \textgreater 1 km
    \State Merge remaining points into sections at least 50 m in length.
    \State Recursively remove points with speed \textgreater 10 km/h adjacent to a break, or the end of the track

\\
    \If{Hikr data}
        \If{segment mean speed \textgreater 10 km/h}
            \State remove segment
        \EndIf
        \State Calculate filtering bounds \Comment{Eq (\ref{Q1Hikrmax}) - (\ref{minHikrQ1}), \nameref{S2_Appendix}}
    \Else
        \State Identify Key Points \Comment{see \nameref{S2_Appendix}}
        \State Remove single datapoints between Key Points
        \State Remove points where median speed between consecutive key points \textgreater Eq (\ref{Q1Hikrmax})
        \While{segment length is not consistent}
            \State Remove points with speed \textgreater 10 km/h adjacent to a break, or the end of the track
            \If{ segment median speed \textgreater Eq (\ref{Q1Hikrmax}) \textbf{or}
            segment minimum speed \textgreater Eq (\ref{medHikrmed}) \textbf{or}
            segment upper quartile speed \textgreater Eq (\ref{topHikrmax}) \textbf{or}
            segment upper whisker speed \textless Eq (\ref{minHikrQ1}) \textbf{or} segment duretion \textless 2.5 minutes
            }
                \State Remove segment
            \EndIf
        \EndWhile
    \EndIf
    \\
    \State Combine all segments into a single dataset
    \State Remove the fastest and slowest 0.5\% of the data 
\end{algorithmic}
\end{algorithm}

Following this, a decision was made to remove data from tracks found in Scotland. Lidar data covering the walking tracks was necessary to model the terrain obstruction, and was not sufficiently available in Scotland at the time of the study. Furthermore, analysis showed that that walking speeds in Scotland were at the extreme end of what is seen throughout the rest of the UK (see \nameref{S4_Appendix}). Including this data without also including a corresponding extreme dataset where lidar data is available may result in incorrect modelling. All OSM track segments which took place within Scotland were excluded from further processing. Similarly Hikr tracks which were tagged as taking place in Scotland, and which fully took place in Scotland were excluded.

Our final modelling dataset consisted of 7,636 GPS tracks from England and Wales, with over 1.4 million individual data points and almost 88,000 km of travel. Each datapoint represented approximately 50-100 m of travel, and contained:
\begin{itemize}
    \setlength\itemsep{0em}
    \item Start coordinate
    \item End coordinate
    \item Start time
    \item Duration
    \item Distance
    \item Speed
    \item Elevation
    \item Walking slope
    \item Hill slope
    \item On-road flag
    \item Paved road flag (if on-road)
    \item Obstruction data available flag (if off-road)
    \item Heavy obstruction flag (if off-road and obstruction data available)
\end{itemize}

\subsection*{Modelling}

\subsubsection*{Model Formulation}

Pilot studies were conducted to identify an appropriate model framework, using tracks within Scotland (see \nameref{S3_Appendix}). Generalised linear model (GLM) and generalised additive model (GAM) approaches were explored, and within both we looked at the relationship between the walking and hill slopes, and the walking speed, with a small number of prior assumptions. As it is more challenging to walk on steeper slopes, for both the hill and walking slope components we knew that the walking speed should be a decreasing function of the magnitude of slope (with some allowance for faster walking speeds on mild descents). Models which failed to predict this were removed under the assumption that the data were overfitted. Furthermore, previous work \cite{Davey1994RunningApplications,Kay2012RouteTerrain,Balstrm2002OnTerrain,Llobera2007Zigzagging:Strategies} has identified the existence of a critical gradient; the angle at which it is faster to zig-zag up a hill, rather than ascend directly. This occurs at a walking slope of around 15 -- 21 degrees, so models which failed to predict the critical gradient occurring below 21 degrees were removed.

10-fold cross-validation was used to compare the remaining model parameters, looking at R-squared values, root-mean-squared error (RMSE) and mean absolute error. Where multiple models performed equally well, the simplest model was selected for ease of interpretabilty and real-world application. The selected model type was a Generalised Linear Model (GLM). Models were implemented using R version 3.6.1 \cite{R}. 

\subsubsection*{Terrain Types}

Each of the three road types (paved road, unpaved road, off-road) was included in the model, both as factor variables, and as interaction terms with each of the slope variables.

Before adding terrain obstruction data to the model, we checked that there was no systematic difference between the walking speeds in regions where we had lidar data, and regions where we did not (see \nameref{S5_Appendix}). Thus our findings in regions where lidar data was available could be extended to those where it was unavailable. Factor variables were then added to the model for each obstruction level (heavy, light or unknown obstruction).

\subsubsection*{Statistical Analysis}

Variables within the model were tested for significance using the Wald test, which allows us to account for correlation between points within the same track (\texttt{coeftest} function within \texttt{lmtest} package in R).

To measure the impact of our model, we compared walking speed predictions of our model against those of Naismith's, Tobler's and Campbell et al.'s models. Four different metrics were compared; the average percentage error, mean squared error (MSE), root-mean squared error (RMSE) and R squared value. These were explored when looking at both individual 50 m track sections, as well as predicted walking times for tracks as a whole. Finally, we isolated the off-road track sections in order to assess the improvement of our model at predicting walking speeds for off-path travel.

\subsection*{Code Availability}
\label{0Code}

Documented code written for this study is available online in a Github repository \texttt{\href{https://github.com/AndrewWood94/PhDThesis}{AndrewWood94/PhDThesis}}, and is licensed under the terms of the GNU General Public License v3.0.