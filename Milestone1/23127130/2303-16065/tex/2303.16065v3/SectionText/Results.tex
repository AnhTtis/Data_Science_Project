\section*{Results}
We started by assembling a dataset derived from public hikes. This process included an iterative data cleaning process to remove erroneous/false data, identify and remove breaks (e.g. Fig \ref{fig2}) to give us a final usable dataset containing 7,636 GPS tracks, with over 1.4 million individual data points and covering almost 88,000 km of travel in the U.K. 

Our curated hike dataset allowed us to create a data-driven model which we can directly compare with existing walking speed algorithms. The model formulation was selected using a small-scale exploratory study which considered data from Scotland (see \nameref{S3_Appendix}). In this exploratory study, multiple different model types were explored which could fit the data, and which matched existing knowledge about walking speeds. Cross-validation methods showed that there was very little difference in performance of the best models, therefore the final model was a Generalised Linear Model (GLM), which was chosen as it was the simplest of those tested (we had no evidence that a more complex model would be superior). This choice also meant that our model was both easy to interpret, and simple to apply to future work.

This final GLM model included all three of the variables suggested by Arnet \cite{Arnet2009ArithmeticalJapan}:

\begin{equation}
    v = exp(a+b\phi+c\theta+d\theta^2)
\end{equation}
where
\begin{quote}
$v = \text{walking speed (km/h)}$\\
$\phi = \text{hill slope angle (degrees)}$\\
$\theta = \text{walking slope angle (degrees)}$
\end{quote}

Terrain obstruction level was included as a factor variable, while we considered the road types as both factor variables and interaction terms. Not all terms had a significant effect on all variables; we therefore created a model with all possible terms, and removed them one at a time (in order of least significance) until all remaining terms were significant to at least 95\% confidence  level (using Wald test). The final values for a, b, c and d are given in Table \ref{tab:2ROUK model variable values} for each of the terrain obstruction levels and road types. The critical gradient for this model is between 14 -- 16 degrees when walking uphill and -16 -- -18 degrees when walking downhill (depending on road and obstruction conditions).

\begin{table}[!ht]
\begin{adjustwidth}{-0.5in}{0in}
    \centering
    \caption{Final walking speed model variable coefficients}
    \begin{tabular}{|l+c|c|c|c|}
    \hline
    & intercept & walking slope & walking slope\textsuperscript{2}  & hill slope \\ 
    \thickhline
    Paved road & 1.580 & -0.00726 & -0.00218 & -0.00389 \\ 
    \hline
    Unpaved road & 1.580 & -0.00965 & -0.00248 & -0.00389 \\
    \hline
    Off-road (obstruction unknown) & 1.536 & -0.00965 & -0.00187 & -0.00731 \\
    \hline
    Off-road (light obstruction) & 1.580 & -0.00965 & -0.00187 & -0.00731 \\ 
    \hline
    Off-road (heavy obstruction) & 1.400 & -0.00965 & -0.00187 & -0.00731 \\ 
    \hline
    \end{tabular}
    \label{tab:2ROUK model variable values}
\end{adjustwidth}
\end{table}

We compared the predictions of our model against those of Naismith's and Tobler's models, as shown in Table \ref{tab:2comparison}. Firstly, the predicted speeds for individual 50 m sections had a lower RMSE and percentage error, and a higher R squared value in this model than in other models. However, this did not translate to similar results when looking at predicted walking times. While the average percentage error for predicted time was lower in the new model than existing ones, the RMSE value was substantially higher (103 s vs 22 s). Investigation into the most extreme error values showed us that this was caused by errors in the data, rather than problems with the model. A single 5 km x 5 km DTM tile received from Ordnance Survey contained elevation data which did not match up with neighbouring tiles, leading to apparent steep slopes on the tile borders which caused large differences between the predicted and actual walking times. We confirmed this data was incorrect by comparing elevation values against a paper OS map of the region. No other instances of this were found in the data, and less than 0.5\% of the GPS track segments intersected the affected tile, so we do not believe that the overall walking speed model will have been detrimentally affected. In order to account for these (or other similar) data errors when calculating the RMSE values for predicted walk times, we adjusted our calculations to only look at the middle 99.9\% of the values for each model. This reduced the RMSE value of the new model to 19.5 s, lower than the RMSEs for both Naismith's rule and Tobler's function.

\begin{table}[!ht]
\centering
\caption{Comparison of new model against existing methods to calculate walking speeds.}
\begin{tabular}{|l|c|c|c|}
\hline
& New Model & Naismith & Tobler \\
\hline
Average \% error & 23.68 & 26.36 & 26.17 \\
\hline
MSE & 1.20 & 1.61 & 1.53 \\
\hline
RMSE & 1.10 & 1.27 & 1.24 \\
\hline
R\textsuperscript{2}  & 0.09 & -0.22 & -0.16 \\
\hline
\end{tabular}
\label{tab:2comparison}  
\end{table}

Figs \ref{fig4}A, B and \ref{fig5}A, B show the RMSE and mean residuals for each of the models, looking only at data which was within 5 degrees of directly climbing (A) or traversing (B) hills of varying slope. There are some interesting points to note here. Firstly, Naismith's rule consistently overestimates walking speeds when descending a slope, and underestimates speeds when climbing a slope. This suggests that Naismith's rule is still a good rule of thumb to calculate route times as a whole (as speed over-estimations on descents and under-estimations on ascents ‘cancel each other out’), but particularly steep or difficult sections will likely not be estimated correctly, and time estimates for individual sections of a route will be less accurate than using the new model found here. When ascending or descending a slope, the RMSE of our GLM is similar to that of Tobler's hiking function. However, one of the main areas where we see an improvement using our model is on slight declines. Tobler's hiking function suggests that walking speed increases on mild descents up to a maximum of 6 km/h (seen in Fig \ref{fig1}). It is clear from Fig \ref{fig5}A, that Tobler's function overestimates the walking speed in this region. We know from existing research that most walking takes place on low walking slopes \cite{Proffitt1995PerceivingSlant}, and this is evidenced by our data ($\sim$98\% of our data was from walking slopes of under 10 degrees). The improved walking speed predictions of our model in this region therefore have the greatest impact in real-world situations. We also see an improvement in RMSE when using our model to predict speeds for hill traversals (Fig \ref{fig4}B). We can note from Fig \ref{fig5}B that both Naismith's rule and Tobler's hiking function consistently overestimate the walking speed when traversing a slope, as they do not take into account the impact that the hill slope has on reducing walking speeds. We do see that the average error in our model increases as the hill slope increases, but we believe that this is due to limited volumes of data at high hill slopes ($\sim$0.5\% of our data occurs on hill slopes steeper than 40 degrees). 

\begin{figure}[!h]
    \begin{adjustwidth}{-2.25in}{0in} 
    \includegraphics{Images/Paper/Fig4.eps}
    \captionsetup{width=1\linewidth}
    \caption[width=\textwidth]{{\bf Comparing RMSE values for the new model, Naismith's rule and Tobler's function.}  When: (A) travelling directly up or down hills of varying slope (all data), (B) traversing across hills of varying slope (all data), (C) travelling directly up or down hills of varying slope (off-road data only), (D) traversing across hills of varying slope (off-road data only).}
    \label{fig4}
    \end{adjustwidth}
\end{figure}

\begin{figure}[!h]
    \begin{adjustwidth}{-2.25in}{0in} 
    \includegraphics{Images/Paper/Fig4.eps}
    \captionsetup{width=1\linewidth}
    \caption[width=\textwidth]{{\bf Comparing mean residual values for the new model, Naismith's rule and Tobler's function.} When: (A) travelling directly up or down hills of varying slope, (B) traversing across hills of varying slope, (C)  travelling directly up or down hills of varying slope (off-road data only), (D) traversing across hills of varying slope (off-road data only).}
    \label{fig5}
    \end{adjustwidth}
\end{figure}

As well as looking at the overall performance of our new model, we looked to explore how well our model performed in off-road conditions, compared the existing functions. Note that when doing this we compared our predicted speeds to those from Naismiths's rule with Aitken's correction applied (a reduced base speed of 4 km/h), and Tobler's function with the off-road multiplicative factor of 0.6. Figs \ref{fig4}C, D and \ref{fig5}C, D shows the RMSE and mean residuals for each of the models, only considering data which was recorded in off-road conditions. From Figs \ref{fig4}C and \ref{fig5}C it is clear that Tobler's hiking function consistently underestimates the walking speed when off-road. The factor of 0.6 is a larger reduction in walking speed than is observed in practice. As we found when looking at our data as a whole, Naismith's rule underestimates the walking speed when climbing a slope and overestimates when descending a slope. Our new model does not suffer from these problems, with both a lower RMSE and lower absolute mean residual value across all walking slopes. Both of the existing models also consistently underestimate walking speeds when traversing a slope, unlike our new model which has a mean residual of less than 0.4 km/h on slopes of up to 35 degrees. The error in predictions of our new model does increase as the hill slope increases, though the RMSE is generally lower than seen in the existing models. On the steepest hill slopes our model appears to perform less well than the existing ones, though only 0.2\% of our off-road data occurred on a hill slope steeper than 40 degrees. 

