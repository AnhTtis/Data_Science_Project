\section*{Materials and methods}

\subsection*{Data set, cleaning and key assumptions}
\label{2Data set cleaning and key assumptions}
Full details of the various datasets used in this study are provided in \nameref{S1_Appendix}. Further, a detailed description of the data filtering processes, and choices/assumptions made during data processing are described in \nameref{S2_Appendix}.

In summary, GPS tracks were obtained for hikes in the UK from Hikr.org \cite{Hikr.org2021UnitedReports} and OpenStreetMap (OSM) \cite{OpenStreetMap.org2021Tracks}. Visual inspection of the tracks showed that a large number were not walking tracks, as their speed was implausible, or contained long breaks which could impact the accuracy of a walking speed model. We developed a series of filters to identify and remove significant breaks and tracks or segments which were not hike or walking based. Fig \ref{Fig 2} shows examples of regions where breaks are visible in a GPS track, which had to be filtered prior to analysis.

\begin{figure}[!h]
    \begin{adjustwidth}{-1.75in}{0in} 
\includegraphics{Images/Paper/Fig2.eps}
\captionsetup{width=1\linewidth}
    \caption[width=\textwidth]{{\bf A GPS track where 3 breaks can identified by finding point clusters.} Clusters of points can form on a GPS track when a break is taken during a hike. By identifying these clusters as potential breaks we are able to remove most break periods from the datasets used for our analysis of walking speeds. For full details of these and other data filtering methods see \nameref{S2_Appendix}. Background images reproduced with permission from Bing Maps/Vexcel Imaging\cite{BingMaps,Vexcel}, OpenStreetMap and OpenStreetMap Foundation \cite{OpenStreetMapMAPS}, visualised using QGIS \cite{OpenSourceGeospatialFoundationProjectQGIS.org2020}.}
    \label{Fig 2}
\end{adjustwidth}
\end{figure}

Elevation and slope values were calculated and added to each point using data from the Ordnance Survey Terrain 5 Digital Terrain Map (DTM) \cite{OrdnanceSurvey2020Terrain5}. Hill slope values were found using the quadratic surface method \cite{Zevenbergen1987QuantitativeTopography, Dunn1998TheGIS}. Each data point was then classified as on a paved road, on an unpaved road, or off road, determined by searching a 50 m radius around each point in an OSM Road dataset \cite{OpenStreetMap.org2021Data}. 

Terrain obstruction information was calculated using lidar datasets \cite{LIDARDSMEngland, LIDARDTMEngland, LidarWales}, as the difference in values between a Digital Surface Map (DSM) and Digital Terrain Map (DTM). We had access to lidar data at 2 m resolution covering large areas of England and Wales, but the coverage was not complete. Of our off-road data ($\sim$2,900 km, spread across over 1,200 tracks), over 2,000 km had lidar data available. Exploration of the lidar data (see \nameref{S5_Appendix}) showed that there was a clear drop in walking speeds once more than 10 cm of obstruction was observed, beyond which the speed was relatively constant.  We used this information to classify points into heavy obstruction (\textgreater 10 cm) or light obstruction ($<= 10$ cm) for modelling purposes.

Where the datapoints in the original GPS track were under 50 m in length, they were merged together to minimise the effects of errors in the GPS location values. While doing this, the resulting distance was the sum of all distances in the constituent GPS points, so may be longer than the straight line distance between co-ordinates. Similarly, both hill and walking slope values, as well as obstruction height, were calculated as the weighted average of constituent points, weighted by point duration.

Our final dataset consisted of 7,636 GPS tracks, with over 1.4 million individual data points and almost 88,000 km of travel. Each datapoint contained:
\begin{itemize}
    \setlength\itemsep{0em}
    \item Start coordinate
    \item End coordinate
    \item Start time
    \item Duration
    \item Distance
    \item Speed
    \item Elevation
    \item Walking slope
    \item Hill slope
    \item On-road flag
    \item Paved road flag (if on-road)
    \item Obstruction data available flag (if off-road)
    \item Heavy obstruction flag (if off-road and obstruction data available)
\end{itemize}

\subsection*{Modelling}

\subsubsection*{Model Formulation}

Pilot studies were conducted to identify an appropriate model framework, using tracks within Scotland (see \nameref{S3_Appendix}). 10-fold cross-validation was used to compare the model parameters, looking at R-squared values, root-mean-squared error (RMSE) and mean absolute error. Where multiple models performed equally well, the simplest model was selected for ease of interpretabilty and real-world application. The selected model type was a Generalised Linear Model (GLM). Models were implemented using R-Studio version 1.2.5019 \cite{R}. 

\subsubsection*{Terrain Types}

Each of the three road types (paved road, unpaved road, off-road) was included in the model, both as factor variables, and as interaction terms with each of the slope variables.

Before adding terrain obstruction data to the model, we checked that there was no systematic difference between the walking speeds in regions where we had lidar data, and regions where we did not (see \nameref{S5_Appendix}). Thus our findings in regions where lidar data was available could be extended to those where it was unavailable. Factor variables were then added to the model for each obstruction level (heavy, light or unknown obstruction).

\subsubsection*{Statistical Analysis}

Variables within the model were tested for significance using the Wald test, which allows us to account for correlation between points within the same track (\texttt{coeftest} function within \texttt{lmtest} package in R).

To measure the impact of our model, we compared walking speed predictions of our model against those of Naismith's and Tobler's models. Four different metrics were compared; the average percentage error, mean squared error (MSE), root-mean squared error (RMSE) and R squared value. These were explored when looking at both individual 50 m track sections, as well as predicted walking times for tracks as a whole. Finally, we isolated the off-road track sections in order to assess the improvement of our model at predicting walking speeds for off-path travel.

\subsection*{Code Availability}
\label{0Code}

Documented code written for this study is available online in a Github repository \texttt{\href{https://github.com/AndrewWood94/PhDThesis}{AndrewWood94/PhDThesis}}, and is licensed under the terms of the GNU General Public License v3.0.