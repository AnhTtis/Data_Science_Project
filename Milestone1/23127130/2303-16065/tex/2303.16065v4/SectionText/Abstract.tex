\section*{Abstract}
Hikers and hillwalkers typically use the gradient in the direction of travel (walking slope) as the main variable in established methods for predicting walking time (via the walking speed) along a route. Research into fell-running has suggested further variables which may improve speed algorithms in this context; the gradient of the terrain (hill slope) and the level of terrain obstruction. Recent improvements in data availability, as well as widespread use of GPS tracking now make it possible to explore these variables in a walking speed model at a sufficient scale to test statistical significance. We tested various established models used to predict walking speed against public GPS data from almost 88,000 km of UK walking / hiking tracks. Tracks were filtered to remove breaks and non-walking sections. A new generalised linear model (GLM) was then used to predict walking speeds. Key differences between the GLM and established rules were that the GLM considered the gradient of the terrain (hill slope) irrespective of walking slope, as well as the terrain type and level of terrain obstruction in off-road travel. All of these factors were shown to be highly significant, and this is supported by a lower root-mean-square-error compared to existing functions. We also observed an increase in RMSE between the GLM and established methods as hill slope increases, further supporting the importance of this variable. 