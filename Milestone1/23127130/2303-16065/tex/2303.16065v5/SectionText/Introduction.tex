\section*{Introduction}
\label{0Background}
Knowing how fast people are able to walk between locations is critical information in many situations. In hiking and hillwalking scenarios, this information is vital for safety reasons. If you are leaving in the morning for a hike then it is standard practice to provide an estimated return time such that emergency services can be contacted if you get into difficulty and do not return. An inaccurate estimate for how long a route will take could lead to unnecessary callouts, or delay a callout in a situation where every minute is important. 

There are a multitude of factors which can impact the walking speed, and time predictions for a route, although these can generally be split into two categories. The first category covers the individual effects which depend on who precisely is undertaking the walk, and when they are doing it. These effects include group size (larger groups often walk slower), age or fitness of  participants, and weather conditions, as well as the aim of the walk (afternoon stroll vs. specific hike). The second category covers the fixed effects which will be consistent across all individuals who attempt the same given route. These effects include how steep the terrain is and whether the route is paved, along a road or in wild country.

Most of the individual effects cannot be modelled without considerable prior knowledge about the person who is planning a route. Therefore, most existing hiking route planners calculate the walking speed solely based on the terrain and this is presented as the average time (or time range) it takes to complete a hike. It is then left up to the individual to tune the predicted time for a hike given their knowledge about personal ability and circumstances.

Formulae of varying complexity have been proposed to estimate human walking speed/time along a projected path. A popular early method that is still widely used was put forward by Naismith \cite{Naismith1892CruachMore} which calculates walking time under normal conditions as:
\begin{quote}
``\textit{an hour for every three miles on the map, with an additional hour for every 2,000 feet of ascent.}''
\end{quote}
This approximates to a walking speed of 5 km/h with 10 minutes added on for every 100 m of ascent. Naismith's rule is still widely used today by Scout groups and other casual hikers due to the ease of calculating walking time by hand using a paper map.

Although Naismith's rule is still very widely used it does have a well-known limitation; namely the fact that is does not predict a reduced walking speed regardless of how steep a descent the user is on. As such, a number of updates to Naismith's rule have been proposed over the years, with the aim of improving the accuracy of walking speed predictions. Aitken \cite{Aitken1977WildernessScotland} introduced a reduced base movement speed of 4 km/h on surfaces which are not paths or roads, and Langmuir \cite{Langmuir1984MountaincraftIsles} included extra terms to account for descents. Langmuir put forward that walking time should be calculated as per Naismith's rule (with Aitken's reduced off-path speed), and:
\begin{itemize}
\item 10 minutes should be added per 300 m of descent at an angle greater than 12 degrees.
\item 10 minutes should be subtracted per 300 m of descent at an angle between 5 and 12 degrees.
\end{itemize}

Langmuir's rule does predict lower speeds on steep descents but also suggests a top speed of 12 km/h on shallow descents, which is much faster than will be achieved. Furthermore, it also implies a sudden jump from 12 km/h down to 3 km/h on slightly steeper slopes which is also very unrealistic.

An alternative hiking function proposed by Tobler \cite{Tobler1993ThreeModelling}, has become more popular in recent research and other situations where speeds do not need to be calculated by hand:
 
\begin{equation*}
        W = 6*exp(-3.5|S + 0.05|),
\end{equation*}
where
\begin{quote}
    W = velocity (km/h)\\
    S = gradient of slope.
\end{quote}

Like Naismith's rule, this gives a speed of 5 km/h on flat ground, with a maximum speed of 6 km/h on a mild descent (around 3 degrees). In a similar manner to Aitken's correction, a factor of 0.6 is applied to the calculated speed for all off-road travel. Tobler's function avoids the issues seen in Naismith's and Langmuir's functions, but it predicts a sharp peak in walking speed, which may be unrealistic. The formulae discussed here are directly compared in Fig \ref{Fig 1}.

\begin{figure}[!h]
    \includegraphics{Images/Paper/Fig1.eps}
    \caption{{\bf The most commonly used functions to calculate walking speed.} Naismith's rule, Langmuir's rule and Tobler's hiking function plotted as predicted walking speed in km/h against the slope in the direction of travel (walking slope) in degrees where positive is uphill. For Tobler's function and Langmuir's rule, on and off-path versions are shown.}
    \label{Fig 1}
\end{figure}

Other studies have also looked at providing alternative methods to calculate walking speeds \cite{Irmischer2018MeasuringNavigation, Rees2004Least-costTerrain, Davey1994RunningApplications, Campbell2019UsingRates}, but all continue to use walking slope as the main variable to determine walking speed (with various multiplicative factors applied for off-road travel). 

When exploring speeds of fell-runners, Arnet \cite{Arnet2009ArithmeticalJapan} suggested that movement velocity may be dependent on three factors: obstruction (with different factors applied depending on the kind of obstruction), ascent in the run direction (walking slope) and slope of the terrain (hill slope). The actual values used in Arnet's calculations cannot be directly applied to walking speeds as they were based on orienteering championships which are very different situations.

Experience tells us that traversing on a steep hill (while maintaining constant elevation) is more difficult than traversing flat ground. However, the existing methods estimate the same walking speed for both situations. Similarly, high levels of terrain obstruction in off-road areas (such as a thick gorse bush) are much more difficult to walk through than empty fields. The simple multipliers for off-road travel in Aitken’s correction and Tobler’s function do not provide any further distinction between two such regions. 

Wood and Schmidtlein \cite{Wood2012AnisotropicNorthwest}, took all three of Arnet's factors into account, and looked at evacuating citizens in the event of a hurricane. They applied Tobler's function to both the hill slopes and walking slopes, and calculated the terrain obstruction coefficients based on energy usage rather than walking speed (using \cite{Soule1972TerrainPrediction.}). They accepted that these were likely not the correct values, but they were unable to find any better alternatives. Campbell, Dennison, and Butler \cite{Campbell2017AMapping} conducted a study using lidar data to explore the effects of ground roughness and vegetation density on firefighter evacuation speeds, but they did not consider the hill slope separately.

All of the studies mentioned above were created from relatively small sample sizes. However, the rise in use of GPS tracking in recent years means that a data-driven approach to modelling the walking speed is now possible, which provides a number of benefits. Firstly, it is now possible to access GPS tracks from a wide variety of regions and terrains. Secondly each track can easily be broken down into individual sections, enabling specific route features to be investigated at much higher spatio-temporal resolution. In addition to the accumulation of GPS track data, several other very useful datasets have become publicly available. In 2013, Ordnance Survey (UK) released the OS Terrain 5 DTM \cite{OrdnanceSurvey2020Terrain5}, which provides elevation data at 5 m intervals across the whole of the UK, and can be used to easily calculate hill slopes. Similarly, the National Lidar Program (England, UK) \cite{NationalLIDARProgramme} announced in 2016, and a similar program in Wales \cite{LidarWales} provide a crude but objective measure of terrain obstruction over large regions of the UK.

There are a number of factors which have also been previously observed, and should be taken into consideration when exploring walking speeds. While existing formulae suggest that walking on very steep slopes is achievable, in practice the majority of slopes encountered day-to-day rarely exceed 10 degrees \cite{Proffitt1995PerceivingSlant}. It is therefore most important to provide accurate walking speed predictions in this region, as it will be of greatest practical use. 

A feature noted in fell-running and hillwalking is the existence of an angle at which it is faster to zig-zag up a hill, rather than ascend directly. This was first identified as occurring at approximately 17 - 20 degrees based on treadmill experiments for fell-runners \cite{Davey1994RunningApplications}. This angle was termed the `critical gradient' by Kay \cite{Kay2012RouteTerrain}, who also looked at fell-running data. Kay found the critical gradient to occur between `0.276' and `0.382' (or 15 -- 21 degrees). Evidence also suggested that a critical gradient occurs when walking, and that it occurs at approximately the same point: Balstr\o{}m \cite{Balstrm2002OnTerrain} said that 40\% (approximately 22 degree) slopes are manageable, but hairpins are generally found in paths starting at 30\% (approximately 17 degrees). Furthermore, Llobera and Sluckin \cite{Llobera2007Zigzagging:Strategies} found a critical gradient of `0.287' (approximately 16 degrees), when exploring energy usage rather than walking speeds. Based on this evidence, an accurate model for walking speed should predict the critical gradient to occur in this region.

Here we explored the impact of all three factors discussed by Arnet on walking speeds. These are the walking slope, the hill slope and the terrain obstruction. We aimed to use these factors to develop a data-driven model for the walking speed for an average individual. 


