\section*{Discussion}

We have developed a model for walking speed which is very robust, due the large volume of data (88,000 km) used to build it, and which correlates with the data over a wider range of conditions than commonly used formulae. Data from tracks confirms that each of the walking slope, the hill slope and the terrain type or obstruction are significant factors in determining walking speeds. The model improves on existing methods to predict walking speeds (Figs \ref{Fig5} \& \ref{Fig6}). We have also shown the specific improvement that our new model has on predicting walking speeds in off-road conditions, compared to the simple off-road speed reductions used by existing models.

Our results confirm that Naismith's rule (Fig \ref{Fig1}) is still a good rule-of-thumb to use when estimating the total walking time for a route, especially in situations where the calculation must be done by hand. However, the findings here can be used as an addition to Naismith's rule; it is likely that (under Naismith's rule) the predicted ascent time will be overestimated and the predicted descent time will be underestimated. It is not uncommon for hikers to contact one another when they reach the summit of a hill, and provide an estimated arrival time back at the campsite. Knowing that the descent will likely take longer than estimated by Naismith's rule will result in more accurate arrival estimations being given. Similarly, the knowledge of how the hill slope reduces walking speeds, or that just 10 cm of vegetation can reduce walking speeds by up to 0.6 km/h may well affect route choices made when out on a walk. For example, if a hiker is following a footpath, but can see from their map that the path forms a large curve then they can use our findings to decide whether it will be faster to travel off-road and cut the corner. On flat terrain with heavy levels of obstruction, our model suggests that such a short cut will be faster if the distance covered on the path is more than 15\% longer than the off-road distance. Speed is not the only factor which would affect this decision, as safety and navigability are also important variables, but these results can help people make more informed choices when on a hike.

The benefit of using crowdsourced GPS data to build our model is also a limitation of the approach, as we did not have control over data collection. This meant that models were unable to account for any bias in our data such as group size, ability and composition, or other potential variables such as weather conditions, as factors in determining walking speed (although we would expect the volume of data to cause most of these effects to average out). 

Unlike previous work \cite{Campbell2022PredictingData}, we did not use fixed values to classify breaks and non-walking or hiking tracks. Instead we developed filters based on the attributes of known walking data (see \nameref{S2_Appendix}). The methods used to filter the datasets were blinded to the outcome of the model generation, the choice of filtering methods will have had an impact on the dataset and subsequent model and no ground truth was available against which to test our assumptions. 

Our method of calculating the terrain obstruction value was relatively crude, looking only at the obstruction height at each GPS point. While this did prove to be successful, and we observed a clear difference in walking speeds between areas of light and heavy obstruction (see \nameref{S5_Appendix}), the inaccuracies present within GPS data may have led to some erroneous obstruction measurements, for example in a field sparsely populated with trees. In future, efforts should be made to refine this approach, such as considering the average obstruction level over a wider area around each point.

A further limitation of our data came when we looked to classify points into paved roads, unpaved roads or off-road. A combination of GPS drift and map error means that there is significant uncertainty and so we had to use a search radius around each data point to identify potential roads. We suspect that we were likely overclassifying tracks on roads. While our model appears to be robust to this overclassification (due to the volumes of correctly classified data used), the overclassification left us with a reduced number of off-road datapoints from which to predict off-road travel speeds.

Furthermore, the use of crowdsourced data meant that all of our data came from `walkable' regions by definition. When including the terrain obstruction variable, we were unable to determine if there are levels of terrain obstruction which makes walking impossible. Similarly, the vast majority of the data was collected on shallow hill- and walking slopes, leading to a sparcity of data in steeper areas. While this does mean that we can be very confident about our walking speed predictions in less steep regions (where most walking occurs), it is unclear whether the lack of data on steeper regions is a result of steep slopes being relatively rare, or that they cannot be easily navigated, so hikers chose an alternate path. 
As described above we had to make a number of assumptions regarding data filtering and processing including model selection, and other choices may give different results. To support anyone who wants to challenge or test these assumptions, or try different models, we have made all our code available on Github. Further, all of the data sources used are detailed in \nameref{S1_Appendix} and the filters/assumptions we used to clean the data are fully detailed in \nameref{S2_Appendix}.