\section{Introduction}
\label{sec:intro}

%\fbox{1 page}

Many games that humans can play with each other include rules based on logic about what is allowed. Board games often have different kinds of pieces that the players take turn in moving. One of the more complex games is called Chess. In this paper we model the game of chess using VDM++ and discuss the pros and cons of alternative modelling styles \cite{Fitzgerald&05}. The model was written as an educational example and can be executed to validate whether the rules of a chess game were broken.

The purpose of formal models is to enable formal analysis of desirable properties of the system of interest. However, with each formal model, there are abstraction alternatives and for each of these, it is worthwhile discussing the best paradigms for describing the system in question. In a specification focus is on explaining the key aspects in relation to the purpose of the model and in this context explainability to human beings is much more important than the speed of execution. The explainability of different paradigms will be discussed in this paper using the rules of the game of Chess as an example. 

This paper is structured as follows: After this introduction Section~\ref{sec:background} provides the reader with a brief introduction to the rules of the Chess game. Afterwards, Section~\ref{sec:paradigms} presents reflections about modelling chess using either the object-oriented or the declarative paradigms. Then Section~\ref{sec:coremodel} provides the core of the VDM++ model using the declarative paradigm. After that Section~\ref{sec:PGN} briefly explains how the Portable Game Notation can be incorporated enabling one to incorporate a standard textual format as input for the VDM++ model making it easy to take existing games that have been played into the model. Section~\ref{sec:related} briefly relates the contribution of this paper with related work.
Finally, Section~\ref{sec:concl} provides a few concluding remarks and considers the future directions.