\section{Concluding Remarks and Future Work}
\label{sec:concl}

This article presented a feature-complete model of Chess in VDM++ that can be used as an educational example or as a basis for formal analysis of other topics related to the game of Chess. The model has been added as one of the models that can be imported into the Visual Studio Code version of Overture \cite{Lund&22} so others can also experiment with the entire VDM++ model. 
%Furthermore, the article will be presented as part of the OVT-21 conference where we hope it can rise an interesting discussion.

In general, we find that there are interesting pros and cons when considering which paradigm to use for modelling a system in VDM++. In the case of the game of Chess, we find that the functional paradigm works better than the object-oriented paradigm (OOP), but it also has drawbacks. If OOP is chosen for a given system, one must be aware of the complications it may bring, in particular relating to invariants across object references. The same challenge would also appear for operations with post-conditions because this also could relate to instance variables in other objects before execution of the operation.