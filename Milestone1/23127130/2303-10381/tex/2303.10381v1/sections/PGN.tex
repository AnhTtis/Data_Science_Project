\section{Portable Game Notation}
\label{sec:PGN}

%\fbox{1 page}

The Portable Game Notation~\cite{PortableGameNotation2022} (PGN) is the de-facto standard used for chess annotation on many online chess websites.\footnote{PGN is supported on websites such as \url{chess.com} \url{https://lichess.org/}, and \url{chess24.com}.}
PGN consists of information related to the chess game (e.g. player information) and move text, where the move text describes the actual piece moves of the game. PGN uses letters for the x-axis and numbers for the y-axis as described in \cref{sec:background}. Generally speaking, a move in PGN consists of the \texttt{PieceType} as the first character and the \texttt{Coordinate} as the second and third characters. Furthermore, there is special notation for castling, check, checkmate, and extra information may be added to remove ambiguity.

PGN was added as a way to verify the overall integrity of the VDM model by testing it on some real-world data. Since VDM++ does not include a string manipulation library it was necessary to partly define one. Furthermore, the PGN class has the responsibility of parsing a valid \texttt{String} describing a game of chess through PGN notation to the VDM++ \texttt{Move} representation and vice versa.

\begin{lstlisting}
values
 numerical_chars = "0123456789";
 numerical_char_to_nat : inmap char to nat = 
             {numerical_chars(i) |-> i-1 | i in seq numerical_chars};

 valid_x_chars = "abcdefgh";
 x_char_to_nat1 : inmap char to nat1 = 
              {valid_x_chars(i) |-> i | i in seq valid_x_chars};

 piece_type_to_string : inmap PM`PieceType to String = 
                        {<pawn> |-> "", <rook> |-> "R", <knight> |-> "N",
                         <bishop> |-> "B", <queen> |-> "Q", <king> |-> "K"}

functions

public move_to_pgn_string: BM`Move -> String
move_to_pgn_string(move) ==
    let piece_type = piece_type_to_string(move.from.type),
        x = (inverse x_char_to_nat1)(move.to_.square.x),
        y = (inverse numerical_char_to_nat)(move.to_.square.y) 
     in
        piece_type ^ [x] ^ [y];
\end{lstlisting}

The helper function \texttt{move\_to\_pgn\_string} is used in \texttt{Runner} when parsing a chess game and logging the results to a text file. The \texttt{PieceType} is mapped to a string by applying the type to \texttt{piece\_type\_to\_string}. Similarly, the x- and y-coordinates are found but the inverse mappings are used here, which is possible as the maps are injective. 

This class is currently made in VDM but since VDM is not really meant for parsing strings it would make sense to redo this part in Java as a new library for reading this external format directly into VDM structures. It could even use scanner and parser generators inside if desired but this is mainly an issue about the potential error reporting in case the string provided does not live up to the PGN syntax.

