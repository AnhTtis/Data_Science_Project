\section{The Rules of Playing Chess}
\label{sec:background}

%\fbox{1 page}
%
%\fbox{briefly explaining the rules of chess without writing it up as requirements}

Chess is a two-player turn-based game where one player controls the white pieces and the other controls the black pieces. The game is played on a chessboard that consists of 64 squares (fields) shaped in an 8x8 grid. The columns of the grid are called ``files'' and the rows are called ``ranks''. The squares are coloured alternatively light and dark and a line of squares of the same colour going from one edge to an adjacent edge on the board is called a ``diagonal''. The squares following the horizontal axis on the board are annotated using letters the 'a' -- 'h' and the vertical squares are annotated using the numbers 1 -- 8.

Each player initially controls 16 pieces on the board, as seen on \cref{fig:coremodel:board_initial}, where a piece can be discriminated through three different attributes. The first is the piece type, which determines how a piece is allowed to move. The second is the position which determines which square a piece is placed on. Finally, each piece has a colour indicating which player controls it.

\begin{figure}[hbt]
  \centering
  \includegraphics[width=0.5\textwidth]{figs/chess_board_initial.png}
  \caption{Initial position of the chess pieces where the first row with squares 'a' -- 'e' shows the position of respectively a rook, knight, bishop, queen and king. The second row shows eight white pawns. Image source: \cite{ChessBoardEditor}.}
  \label{fig:coremodel:board_initial}
\end{figure}


There are six types of pieces in the game where each player initially controls eight pawns, two rooks, two knights, two bishops, a queen, and, most importantly, a king.
Each piece type has a unique way of moving around the board. For example the rook can move along either the file or rank on which it stands and the bishop can only move along the diagonals.
A piece can be blocked from moving to a square if there is another piece between the initial and the desired square. However, if an enemy piece occupies the desired square, the player may capture that piece by removing it from the board. When a player can capture a piece, the piece is said to be under attack. The knights have an exception in the moving pattern since they are not blocked by other pieces in their path.

The goal of the game is to put the king of the opposing player in a position where it is impossible to prevent it from being captured in the following turn\footnote{For a full list of rules see
%\url{https://en.wikipedia.org/wiki/Rules_of_chess}
\url{https://www.fide.com/FIDE/handbook/LawsOfChess.pdf}.} % This source is more official

There are certain types of moves in chess that can be considered ``special'' in the regard, that they can only be performed when certain conditions are met. A simple example of such is that a pawn has the option of moving two squares forward from its initial position. Another example is ``castling'' which is the only type of movement involving two pieces. Castling allows the player to move their king two squares towards a rook on the player's first rank, then move the rook to the square the king just passed. However, castling is only allowed if the king and the rook have not been moved in during the game, if the squares between the king and rook are unoccupied and not under attack, and if the king is not under attack (in check).

If a pawn reaches the rank furthest from its initial position it must change its type to one of the
following: Queen, knight, rook or bishop. This type of move is called a ``promotion''. Taking this into account is actually challenging because it means that the type of that piece is changing dynamically.


