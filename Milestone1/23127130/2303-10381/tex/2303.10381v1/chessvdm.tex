\documentclass[a4paper]{llncs}

%\usepackage{amssymb}
\usepackage{graphicx}
%\usepackage{mathtools}
%\usepackage{amssymb}
\usepackage{amsmath}
%\usepackage{stmaryrd}
\usepackage[T1]{fontenc}
\usepackage{hyperref}
\usepackage{varioref}
\usepackage{xspace}
\usepackage{paralist}
\usepackage{fancybox}
\usepackage{xcolor}
\usepackage{calc}
\usepackage{verbatim}
%\usepackage{rt-semantics-defs}
%\usepackage{mdframed}
\def\H#1{\hspace*{#1em}}
\usepackage{relsize}
%\usepackage{skak} % for Chess
%\usepackage{latexsym}
%\usepackage{bsymb}
\usepackage{booktabs}
\usepackage[scaled]{helvet}
%\usepackage{hyphenat}

\setlength\textfloatsep{8.0pt plus 2.0pt minus 2.0pt}
\setlength\intextsep{4.0pt plus 2.0pt minus 2.0pt}
\setlength\floatsep{4.0pt plus 2.0pt minus 2.0pt}
\setlength\abovecaptionskip{4.0pt plus 2.0pt minus 2.0pt}
\setlength\belowcaptionskip{0pt}

%\renewcommand{\paragraph}[1]{\noindent\textbf{\itshape{#1}}\ \ }

%\pagestyle{plain}
\usepackage{times}

\usepackage{listingsVDM}
% *************** Overture list ***************
% use package for VDM language definition
\usepackage{overturelanguagedef}
% Define Overture listing for The VDM language
\lstdefinestyle{overtureLanguageStyle}{basicstyle=\ttfamily,
			frame=trBL,
%			numbers=left,
%			gobble=0,
			showstringspaces=false,
%			linewidth=\textwidth,
			frameround=fttt,
			captionpos=b,
			aboveskip=5mm,
			belowskip=5mm,
			framexleftmargin=0mm,
			framexrightmargin=0mm}

\usepackage{customlangdef}
%
%%\lstdefinestyle{Clang}{
%%			 basicstyle=\ttfamily,
%%			 frame=TRbl,
%%			 numbers=left,
%%			 numberstyle=\tiny,
%%			 gobble=0,
%%			 showstringspaces=false,
%%			 linewidth=\textwidth,
%%			 frameround=ffff,
%%			 captionpos=b,
%%			 aboveskip=5mm,
%%			 belowskip=5mm,
%%			 framexleftmargin=0mm,
%%			 framexrightmargin=0mm
%%}
%%
%%
\newcommand{\keywords}[1]{\par\addvspace\baselineskip
\noindent\keywordname\enspace\ignorespaces#1}
\lstset{basicstyle=\scriptsize,tabsize=2,frame=trBL,frameround=fttt}

\newcommand{\gita}[1]{%
		\textbf{\textcolor{purple}{\scriptsize Gita: #1}} %
	{} %
}

\definecolor{maroon}{rgb}{0.5,0,0}
\definecolor{darkgreen}{rgb}{0,0.5,0}
\definecolor{ao}{rgb}{0.0, 0.5, 0.0}
\definecolor{mycolor1}{rgb}{0.0, 0.53, 0.74}% 
\definecolor{mycolor2}{rgb}{0.21783,0.72504,0.61926}%
\definecolor{mycolor4}{rgb}{0.93, 0.53, 0.18}% 
\definecolor{plum}{rgb}{0.56, 0.27, 0.52}
\definecolor{pinegreen}{rgb}{0.0, 0.47, 0.44}
\definecolor{pthaloblue}{rgb}{0.0, 0.06, 0.54}
\definecolor{saffron}{rgb}{0.96, 0.77, 0.19}


\usepackage[caption=false]{subfig} 
\usepackage{pgfplots}
\pgfplotsset{compat=newest}
\usetikzlibrary{pgfplots.groupplots, pgfplots.units}
%\usepgfplotslibrary{groupplots}
\pgfplotsset{
	every axis label/.append style={font=\normalsize},
	tick label style={font=\small},
	/pgfplots/enlargelimits=false,
    legend style={legend pos=north east, font=\small},
    legend cell align=left,
    xlabel near ticks,
    ylabel near ticks,
	axis on top,
    highlight/.code args={#1:#2}{
        \fill [every highlight] ({axis cs:#1,0}|-{rel axis cs:0,0}) rectangle ({axis cs:#2,0}|-{rel axis cs:0,1});
    },
    /tikz/every highlight/.style={
        on layer=\pgfkeysvalueof{/pgfplots/highlight layer},
        red!10
    },
    /tikz/highlight style/.style={
        /tikz/every highlight/.append style=#1
    },
    highlight layer/.initial=axis background
}%

\lstdefinelanguage{XML}
{
  basicstyle=\ttfamily\scriptsize,
  morestring=[b]",
  moredelim=[s][\bfseries\color{maroon}]{<}{\ },
  moredelim=[s][\bfseries\color{maroon}]{</}{>},
  moredelim=[l][\bfseries\color{maroon}]{/>},
  moredelim=[l][\bfseries\color{maroon}]{>},
  morecomment=[s]{<?}{?>},
  morecomment=[s]{<!--}{-->},
  commentstyle=\color{darkgreen},
  stringstyle=\color{blue},
  identifierstyle=\color{red}
}
\newcommand{\eg}{e.g., }
\newcommand{\ie}{i.e., }

\usetikzlibrary{shapes,fit,arrows,calc,arrows,arrows.meta,positioning}

\lstset{language=VDM++}

\usepackage{morten_preamble}

\begin{document}

\title{Modelling Chess in VDM++}
\titlerunning{Modelling Chess in VDM++}

\author{Morten Haahr Kristensen\inst{1} and Peter Gorm Larsen\inst{1}  }
\institute{DIGIT, Department of Electrical and Computer Engineering, Aarhus University, Denmark, \email{201807664@post.au.dk, pgl@ece.au.dk}
}

\maketitle

\ifnotes %
\textbf{NOTES ENABLED. TURN OFF IN FINAL VERSION.} %
\else %
\fi %

\begin{abstract}
The game of chess is well-known and widely played all over the world. However, the rules for playing it are rather complex since there are different types of pieces and the ways they are allowed to move depend upon the type of the piece. In this paper we discuss alternative paradigms that can be used for modelling the rule of the chess game using VDM++ and show what we believe is the best model. It is also illustrated how this model can be connected to a standard textual notation for the moves in a chess game. This can be used to combine the formal model to a more convenient interface.
\end{abstract}

\section{Introduction}
\label{sec:introduction}
% \begin{itemize}
%     % Diffusion of FL
%     \item {\st{Diffusion of FL}}
%     % Security threats to FL
%     \item {\st{Security threats to FL with particular focus on model poisoning}}
%     % Limitations of existing countermeasures
%     \item {\st{Current countermeasures (e.g., KRUM) and their limitations}}
%     % Proposed method and its advantages
%     \item {\st{Intuitive description of the proposed method and its difference (i.e., advantages) w.r.t. state of the art}}
%     % Main contributions
%     \item {\st{Summary of the main contributions of this work}}
%     % Paper's structure and organization
%     \item {\st{Paper's structure and organization}}
% \end{itemize}

% Diffusion of FL
Recently, {\em federated learning} (FL) has emerged as the leading paradigm for training distributed, large-scale, and privacy-preserving machine learning (ML) systems~\cite{mcmahan2017googleai,mcmahan2017aistats}. 
The core idea of FL is to allow multiple edge clients to collaboratively train a shared, global model without disclosing their local private training data.
%Specifically, an FL system consists of a central server and many edge clients; 
A typical FL round involves the following steps: {\em(i)} the server randomly picks some clients and sends them the current, global model; {\em(ii)} each selected client locally trains its model with its own private data; then, it sends the resulting local model to the server;\footnote{Whenever we refer to global/local model, we mean global/local model {\em parameters}.} {\em(iii)} the server updates the global model by computing an \emph{aggregation function}, usually the average (FedAvg), on the local models received from clients.
% \begin{enumerate}
%     \item[{\em(i)}] the server sends the current, global model to the clients and appoints some of them for training;
%     \item[{\em(ii)}] each selected client locally trains its copy of the global model with its own private data; then, it sends the resulting local model back to the server;\footnote{Whenever we refer to global/local model, we mean global/local model {\em parameters}.}
%     \item[{\em(iii)}] the server updates the global model by computing an \emph{aggregation function} on the local models received from clients (by default, the average, also referred to as FedAvg~\cite{mcmahan2017aistats}).
% \end{enumerate}
This process goes on until the global model converges. %(e.g., after a certain number of rounds or other similar stopping criteria).
%\\
% The advantages of FL over the traditional, centralized learning paradigm are undoubtedly clear in terms of flexibility/scalability (clients can join/disconnect from the FL network dynamically), network communications (only model weights\footnote{We will use \textit{parameters} and \textit{weights} interchangeably.} are exchanged between clients and server), and privacy (each client's private training data is kept local at the client's end and not uploaded to the server).
\\
% Security threats to FL
%However, the growing adoption of FL also raises security concerns~\cite{costa2022covert}, particularly about its confidentiality, integrity, and availability.
Although its advantages over standard ML, FL also raises security concerns~\cite{costa2022covert}. %, particularly about its confidentiality, integrity, and availability~\cite{costa2022covert}.
% OLD, LONG VERSION
% Indeed, some work deals with privacy leakage that may expose the local data of some clients~\cite{melis2019sp}. 
% A large body of work, instead, investigates attacks that usually aim to detriment the predictive accuracy of the learned global model. For instance, \emph{data poisoning} attacks achieve this goal by letting an adversary pollute the training set of some corrupt FL clients with maliciously crafted examples~\cite{jagielski2018sp}.
% Similarly, in \emph{model poisoning} the attacker attempts to tweak the global model weights~\cite{bhagoji2019pmlr} by directly perturbing the local model's weights of some infected FL clients before these are sent to the central server for aggregation, usually via so-called Byzantine attacks. 
% It turns out that Byzantine model poisoning attacks severely impact standard FedAvg; therefore, more robust aggregation functions must be designed to make FL systems secure.
Here, we focus on \emph{untargeted model poisoning} attacks~\cite{bhagoji2019pmlr}, where an adversary attempts to tweak the global model weights %\footnote{We will use the terms \textit{parameters} and \textit{weights} interchangeably.} 
by directly perturbing the local model's parameters of some infected clients before these are sent to the central server for aggregation.
In doing so, the adversary aims to jeopardize the global model \textit{indiscriminately} at inference time.
Such model poisoning attacks severely impact standard FedAvg; therefore, more robust aggregation functions must be designed to secure FL systems.
\\
% In this paper, we focus on designing a novel robust aggregation scheme at the server's end to contrast the effect of Byzantine model poisoning attacks.
%
% Current countermeasures and their limitations
%Several countermeasures have been proposed in the literature to combat model poisoning attacks on FL systems.
% Some methods use simple statistics more robust than plain average to smooth the impact of malicious updates (e.g., Trimmed Mean and FedMedian~\cite{yin2018icml}). 
% Other defenses implement outlier detection techniques to discard malicious updates from the aggregation performed at the server's end. Those are either based on heuristics (e.g., Krum/Multi-Krum~\cite{blanchard2017nips} and Bulyan~\cite{mhamdi2018pmlr}) or data-driven approaches (e.g., K-means clustering~\cite{shen2016acm} or DnC via spectral analysis~\cite{shejwalkar2021ndss}). 
% Finally, some strategies rely on a centralized ``source of trust'' to spot potential malicious updates (e.g., FLTrust~\cite{cao2020fltrust}).
% Several countermeasures have been proposed in the literature to combat model poisoning attacks on FL systems, i.e., to discard possible malicious local updates from the aggregation performed at the server's end. 
% These techniques range from simple statistics more robust than plain average (e.g., Trimmed Mean and FedMedian~\cite{yin2018icml}) to outlier detection heuristics (e.g., Krum/Multi-Krum~\cite{blanchard2017nips} and Bulyan~\cite{mhamdi2018pmlr}) or data-driven approaches (e.g., spectral analysis via K-means clustering~\cite{shen2016acm} or spectral analysis), or methods based on ``source of trust'' (e.g., FLTrust~\cite{cao2020fltrust}).
% OLD, LONG VERSION
%Several countermeasures have been proposed in the literature to combat Byzantine model poisoning attacks on FL systems.
% Descriptive statistics
% For example, Trimmed Mean and FedMedian aggregate local model updates using more robust statistics than standard average~\cite{yin2018icml}.
%
% % Heuristics for outlier detection
% Many existing Byzantine-resilient strategies implement some outlier detection heuristics to discard the model updates sent by potentially malicious clients from the input of the aggregation function.
% One of the most popular heuristics is Krum~\cite{blanchard2017nips}.
% This strategy tries to mitigate the impact of Byzantine attacks by selecting as a global model the local model with the smallest sum of Euclidean distances to {\em all} the other local models.
% Although powerful, Krum requires the server to know (or, at least, estimate) the number of malicious FL clients upfront, which is generally impossible in a realistic attack scenario. %
% Moreover, Krum may become ineffective for complex, high-dimensional model parameter spaces due to the curse of dimensionality.
% Bulyan~\cite{mhamdi2018pmlr} tries to overcome this issue by combining Krum with a variant of Trimmed Mean.
% % Data-driven outlier detection
% Other strategies use data-driven outlier detection techniques -- e.g., via K-means clustering~\cite{shen2016acm} -- to spot potential malicious local model updates. 
% %For instance, Shen et al. propose to cluster local model updates with K-means and thus identify outliers.
%
% % Other techniques
% As far as the server is concerned, any local model received can be from a potential malicious client. 
% FLTrust~\cite{cao2020fltrust} assumes the server acts as a client, i.e., trains a local model on an additional {\em trustworthy} dataset at the server's end and compares it against all the local models from other clients. 
% This way, the server can rely on some ``source of trust'' when discarding potentially malicious clients.
%\\
% Limitations of existing Byzantine-resilient strategies
Unfortunately, existing defense mechanisms either rely on simple heuristics (e.g., Trimmed Mean and FedMedian by~\cite{yin2018icml}) or need strong and unrealistic assumptions to work effectively (e.g., foreknowledge or estimation of the number of malicious clients in the FL system, as for Krum/Multi-Krum~\cite{blanchard2017nips} and Bulyan~\cite{mhamdi2018pmlr}, which, however, cannot exceed a fixed threshold).
Furthermore, outlier detection methods using K-means clustering~\cite{shen2016acm} or spectral analysis like DnC~\cite{shejwalkar2021ndss} do not directly consider the temporal evolution of local model updates received.
Finally, strategies like FLTrust~\cite{cao2020fltrust} require the server to collect its own dataset and act as a proper client, thereby altering the standard FL protocol.
\\
% OLD, LONG VERSION
% Overall, existing Byzantine-resilient strategies are either simple heuristics (e.g., FedMedian) or, if they are more complex, they rely on strong and unrealistic assumptions to work effectively (e.g., knowing the number of malicious clients in the FL system in advance, as for Krum and alike).
% Furthermore, data-driven outlier detection methods do not consider the temporary evolution of local model updates received (e.g., K-means clustering). 
% Finally, strategies like FLTrust requires the server to collect its own dataset and act as a proper client, thereby altering the standard FL protocol.
%
% Description of the proposed method
This work introduces a novel pre-aggregation \textit{filter} robust to untargeted model poisoning attacks. Notably, this filter $(i)$ operates without requiring prior knowledge or constraints on the number of malicious clients and $(ii)$ inherently integrates temporal dependencies. 
The FL server can employ this filter as a preprocessing step before applying \textit{any} aggregation function, be it standard like FedAvg or robust like Krum or Bulyan.
Specifically, we formulate the problem of identifying corrupted updates as a multidimensional (i.e., matrix-valued) time series anomaly detection task. 
The key idea is that legitimate local updates, resulting from well-calibrated iterative procedures like stochastic gradient descent (SGD) with an appropriate learning rate, show \textit{higher predictability} compared to malicious updates. This hypothesis stems from the fact that the sequence of gradients (thus, model parameters) observed during legitimate training exhibit regular patterns, as validated in Section~\ref{subsec:intuition}. %until convergence. 
%This regularity may be more pronounced for smooth convex loss functions, but it can still be captured within an appropriate time window, even for more complex and convoluted loss surfaces. 
%We provide evidence of this claim in Appendix~B, where we show that the average mutual information (i.e., ``predictability''), calculated over pairs of legitimate model updates sent at different FL rounds, is significantly higher than the corresponding computation for a malicious client.
\\
Inspired by the matrix autoregressive (MAR) framework for multidimensional time series forecasting~\cite{chen2021je}, we propose the FLANDERS ({\em \textbf{F}ederated \textbf{L}earning meets \textbf{AN}omaly \textbf{DE}tection for a \textbf{R}obust and \textbf{S}ecure}) filter.
The main advantages of FLANDERS over existing strategies like FLDetector~\cite{zhao2020multivariate} are its resilience to large-scale attacks, where $50\%$ or more FL participants are hostile, and the capability of working under realistic non-iid scenarios.
We attribute such a capability to two key factors: $(i)$ FLANDERS works without knowing a priori the ratio of corrupted clients, and $(ii)$ it embodies temporal dependencies between intra- and inter-client updates, quickly recognizing local model drifts caused by evil players. Below, we summarize our main contributions:

\begin{itemize}
\item[{\em(i)}]
We provide empirical evidence that the sequence of models sent by legitimate clients is more predictable than those of malicious participants performing untargeted model poisoning attacks.
\\
\item[{\em(ii)}] 
We introduce FLANDERS, the first pre-aggregation filter for FL robust to untargeted model poisoning based on multidimensional time series anomaly detection.
\\
\item[{\em(iii)}] 
We integrate FLANDERS into Flower,\footnote{\scriptsize{\url{https://flower.dev/}}} a popular FL simulation framework for reproducibility.
\\
\item[{\em(iv)}] 
We show that FLANDERS improves the robustness of the existing aggregation methods under multiple settings: different datasets, client's data distribution (non-iid), models, and attack scenarios.
\\
\item[{\em(v)}] 
We publicly release all the implementation code of FLANDERS along with our experiments.\footnote{\scriptsize{\url{https://anonymous.4open.science/r/flanders_exp-7EEB}}}
\end{itemize}

% Paper's structure and organization
The remainder of the paper is structured as follows. %some related work and the current state-of-the-art solutions to security issues that FL entails. 
Section~\ref{sec:background} covers background and preliminaries. 
In Section~\ref{sec:related}, we discuss related work.
Section~\ref{sec:problem} and Section~\ref{sec:method} describe the problem formulation and the method proposed. % to tackle it. 
Section~\ref{sec:experiments} gathers experimental results. %, and Section~\ref{sec:limitations} discusses some limitations of this work.
Finally, we conclude in Section~\ref{sec:conclusion}.
 %discusses the limitations of this work and draws future research directions.
%reports conclusions and draws perspectives for future research directions.

%%%%%%% OLD %%%%%%%
%to overcome the resilience of Byzantine failures in distributed Stochastic Gradient Descent computations. 
% The strength of Krum is its time complexity, which is linear in the gradient dimension. 
% However, the robustness of the approach is guaranteed for gradient-based learning applications only when the majority of the clients are not compromised. 
% Besides, the aggregation mechanism of Krum, as well as that of similar methods, is robust from a coarse-grained perspective and does not provide solutions to errors and perturbations that may occur at inference time.
%A related approach to~\cite{blanchard2017nips} is the work of Su et al.~\cite{su2016dc}. Here, the authors propose an iterated approximate agreement to tackle a multi-layer scenario attacked by Byzantine agents. 
%However, the method works efficiently on the sole discrete context and it is inapplicable to continuous state environments.
%\gabri{Maybe, we should just talk about the main limitations of existing countermeasures without digging into their details (or, we can just mention Krum as this is the most popular one). I will move the description of all these methods to the Related Work section.}

\section{Background on Network Calculus}
\label{sec: background}


\begin{figure*}[tbh]
\centering
\begin{subfigure}[b]{0.3\textwidth}
    \centering
    \includegraphics[width=\linewidth]{images/in-out.png}
    \caption{Arrival and departure data and their relation with delay $d(t)$ and backlog $b(t)$. For a FIFO system, the delay is the horizontal distance between $R(t)$ and $R^*(t)$ but some other multiplexing techniques may shift the data to a later priority, causing a longer delay.}
    \label{fig: data in-out}
\end{subfigure}
\hfill
\begin{subfigure}[b]{0.35\textwidth}
    \centering
    \includegraphics[width=\linewidth]{images/arrival-service.png}
    \caption{Characteristics of an arrival curve and a service curve. From any point of observation, the arriving data never exceeds its arrival curve; the departure data is also never less than the service curve with respect to the data arrival.}
    \label{fig: arrival-service curves}
\end{subfigure}
\hfill
\begin{subfigure}[b]{0.33\textwidth}
    \centering
    \includegraphics[width=\linewidth]{images/bound.png}
    \caption{Delay and backlog bounds of a system. Backlog is the maximum vertical distance between $\alpha(t)$ and $\beta(t)$; FIFO delay is their maximum horizontal distance; but for arbitrary multiplexing, the delay guarantee is when the system clears its buffer, thus it's the intersection of $\alpha(t)$ and $\beta(t)$.}
    \label{fig: system bounds}
\end{subfigure}
\caption{Network calculus framework. We let $R(t)$ and $R^*(t)$ be the arrival and departure data flow of a system; $\alpha(t)$ be the piecewise linear concave arrival curve and $\beta(t)$ be the piecewise linear convex service curve of a system.}
% \hossein{Better to show piece-wise linear concave arrival curve and piece-wise linear convex service curve instead of token-bucket and rate-latency.}}
\end{figure*}

We recall some of the network calculus essentials for a better understanding of the framework used in Saihu. In the following context, we use the following notation: $\mbb{R}^+$ is the set of non-negative real numbers; $[x]_+$ denotes $\max(0, x)$

The data flow is by convention modeled as a left-continuous wide-sense increasing function $R(t): \mbb{R}^+ \mapsto \mbb{R}^+$ with respect to time $t$~\cite{ncbook2001leboudec}. 

A system $\mcal{S}$ receives arrival data described as a cumulative function $R(t)$ and delivers departure data as another cumulative function $R^*(t)$. Figure~\ref{fig: data in-out} illustrates such a system $\mcal{S}$. The benefit of representing a system like this is that we can observe system backlog and delay with such a model. 

\begin{definition}[Backlog and Delay~\cite{ncbook2001leboudec}]
    The backlog of a system at time~$t$ is
    \begin{equation}
        b(t) = R(t) - R^*(t)
    \end{equation}
    
    The virtual delay of a FIFO system at time $t$ is
    \begin{equation}
        d_{FIFO}(t) = \inf \lbp \tau \geq 0 : R(t) \leq R^*(t+\tau) \rbp
    \end{equation}
\end{definition}



The backlog of a system can be viewed as the vertical distance between $R$ and $R^*$. The FIFO (\textit{First-in First-out}) delay is the horizontal distance between $R$ and $R^*$. One may obtain other delay values if the multiplexing technique is not FIFO.

% \begin{figure}
%     \centering
%     \includegraphics[width=0.9\linewidth]{images/in-out.png}
%     \caption{In/out data flow; delay and backlog}
%     \label{fig: data in-out}
% \end{figure}

Since we are interested in the system guarantee instead of a single instance of data flow, we would like to have general bounds to the arrival and departure data flows. Therefore, we define \textit{arrival curve} and \textit{service curve} as the bounds of arrival and departure data flows.

\begin{definition}[Arrival Curve~\cite{ncbook2001leboudec}]
    Given a wide-sense increasing function $\alpha: \mbb{R}^+ \mapsto \mbb{R}^+$, we say that a flow $R(t)$ is $\alpha$-constrained if and only if for all $s \leq t$:
    \begin{equation}
        R(t) - R(s) \leq \alpha(t-s)
    \end{equation}
    We say $R(t)$ has $\alpha$ as an arrival curve.
\end{definition}

\begin{definition}[Service Curve~\cite{ncbook2001leboudec}]
    Given a wide-sense increasing function $\beta: \mbb{R}^+ \mapsto \mbb{R}^+$ and $\beta(0) = 0$. A system $\mcal{S}$ having $R(t)$ and $R^*(t)$ as its arrival and departure flows. We say $\mcal{S}$ offers a service curve $\beta$ if and only if
    \begin{equation}
        R^*(t) \geq (R \otimes \beta)(t) =: \inf_{s \leq t} \lbp R(s) + \beta(t-s) \rbp
    \end{equation}
    where $\otimes$ denotes the min-plus convolution
\end{definition}

Figure~\ref{fig: arrival-service curves} illustrates the arrival and service curves. Any segment of arrival flow $R(t)$ is constrained by arrival curve $\alpha$ and the output curve $R^*(t)$ is always no less than the curve $R\otimes\beta$. As a result, an arrival curve upper bounds the incoming traffic, and a service curve lower bounds the outgoing traffic.

% \begin{figure}
%     \centering
%     \includegraphics[width=\linewidth]{images/arrival-service.png}
%     \caption{Arrival/Service curve}
%     \label{fig: arrival-service curves}
% \end{figure}

We consider 2 special types of curves throughout this paper, \textit{token-bucket} (or sometimes called \textit{leaky-bucket}) curve and \textit{rate-Latency} curve.

\begin{definition}[Token-bucket and Rate-latency~\cite{ncbook2001leboudec}]
    A token-bucket curve $\gamma_{r,b}$ with arrival rate $r$ and burst $b$ is defined as
    \begin{equation}
        \gamma_{r,b}(t) = b + rt
    \end{equation}

    A rate-latency curve $\beta_{R,T}$ with service rate $R$ and latency $T$ is defined as
    \begin{equation}
        \beta_{R,T}(t) = R \lb t - T \rb_+
    \end{equation}
\end{definition}

A token-bucket curve is determined by a burst $b$ and an arrival rate~$r$. Burst represents the maximum possible data volume that can arrive simultaneously, and arrival rate represents the maximum long-term data rate~\cite{bouillard2022tradeoff}.
A rate-latency curve is determined by a latency~$T$ and a service rate~$R$. Latency represents the time a server needs before starting to process the incoming data, and service rate represents the minimum rate to process data after the initial latency.

With the help of arrival and service curves, we can derive delay and backlog bounds for a system $\mcal{S}$ illustrated in Figure~\ref{fig: system bounds}. Suppose a system $\mcal{S}$ has arrival curve $\alpha$ and service curve~$\beta$, its worst-case backlog $b^*$ is the maximum vertical distance between~$\alpha$ and~$\beta$. Similarly, depending on the multiplexing technique applied to the system, its worst-case delay bound $d^*$ is the maximum horizontal distance between $\alpha$ and $\beta$ if $\mcal{S}$ is a FIFO system. If we don't have any information about its multiplexing technique, referred to as arbitrary multiplexing, the best we can say is that when $\alpha$ and $\beta$ intersect each other, where all data has been delivered out of the system. Consequently, the worst-case delay bound for arbitrary multiplexing is the time required for $\mcal{S}$ to clear its buffer.

% \begin{figure}
%     \centering
%     \includegraphics[width=\linewidth]{images/bound.png}
%     \caption{System delay/backlog bounds}
%     \label{fig: system bounds}
% \end{figure}

While a service curve captures the slowest possible output speed of a system, a link's transmission capacity limits the speed as well. Hence, we model this phenomenon using a \textit{greedy shaper} with a sub-additive function $\sigma: \mbb{R}^+ \mapsto \mbb{R}^+$ concatenated with a server. We consider a concatenation as shown in Figure \ref{fig: system}. By convention we assume $\sigma(0) = 0$ and $\beta(t) \leq \sigma(t), \forall t \in \mbb{R}^+$, meaning that the buffer is cleared at the beginning and the service never exceed its physical limitation. With the above definition, such greedy shaper conserves the service provided by the system due to theorem \ref{thm: shaping}.

\begin{figure}[thb]
    \centering
    \includegraphics[width=0.7\linewidth]{images/system.png}
    \caption{Shaping of departure data. A flow that has an arrival curve $\alpha$ feeds into a server with an arrival data flow $R(t)$. The server having service curve $\beta$ takes $R(t)$ and gives a departure data flow $R^*(t)$ to a shaper with shaping function $\sigma$. The shaper takes $R^*(t)$ and shape the data flow as another departure $D(t)$.}
    \label{fig: system}
\end{figure}


\begin{theorem}[Shaping conserves service \cite{ncbook2001leboudec}]
\label{thm: shaping}
Following the system shown in Figure \ref{fig: system}, we have
\begin{equation}
     D = R^* \otimes \sigma \geq \lp R \otimes \beta \rp \otimes \sigma = R \otimes \lp \beta \otimes \sigma \rp = R \otimes \beta
\end{equation}
\end{theorem}

In the following context, we model the shaping function $\sigma$ as a token-bucket curve $\gamma_{C,L}$ with transmission capacity $C$ and the packet size $L$ to capture the link capacity and packetization~\cite{bouillard2022tradeoff}.


\section{Alternative Paradigms for Modelling the Game of Chess}
\label{sec:paradigms}
%\fbox{1 page}

The different VDM dialects generally encourage following a Functional Paradigm (FP), but the VDM++ dialect introduces the possibility of using Object-Oriented Paradigm (OOP) for structuring the models. In an OOP setting, there is typically a need to have instance variables inside classes to represent state, and to access and adjust these there is typically a need for operations that need to use the imperative paradigm with assignments to such instance variables. The question of whether or not to use such features arise.

When determining a paradigm to follow, one must consider if it is easier to encapsulate the moving parts or to minimise them. In the case of the game of Chess both OOP and FP paradigms may look appealing as many of the rules of chess generally are stateless and therefore without moving parts. However, the special rules in particular introduce statefulness to the game. This includes example moves such as en passant and castling, where the validity of the moves depends on previous moves.

\subsection{Considering the Functional and Object-Oriented Paradigms}

Two architectures following OOP and FP were considered. The first follows a typical OOP structure with a base class \texttt{Piece} that defines basic methods for determining possible moves. Each piece type then has a subclass implementation defining its unique movement pattern and potential attributes. The special moves would be modelled through \texttt{Board\`{}possible\_moves} as the \texttt{Board} class knows the state of the game. A simplified class diagram of such an architecture can be seen in \cref{fig:par:op_arch}.

\begin{figure}[hbt]
    \centering
    \includegraphics[width=0.5\textwidth]{figs/arch_general_oop.pdf}
    \caption{UML class diagram of the basis for an object-oriented architecture for chess.}
    \label{fig:par:op_arch}
\end{figure}

The second architecture was written following the FP where only immutable variables were used. In VDM++ this meant defining all data structures as composite types. The benefit of this is that one only has values in the specification which means that the model is stateless. When writing VDM++ in a functional style, one should consider the classes as modules that encapsulate functionalities together in a namespace (it would thus look more like a VDM-SL model but we have kept it as a VDM++ model in order to ease the comparison). An example of such an architecture can be seen in \cref{fig:par:fp_arch}. In this architecture, all pieces share a common data structure containing their colour, position and type. 

\begin{figure}[hbt]
    \centering
    \includegraphics[width=0.5\textwidth]{figs/arch_general_fp.pdf}
    \caption{UML class diagram of the basis for a functional-style architecture for Chess.}
    \label{fig:par:fp_arch}
\end{figure}

One benefit of following the FP architecture relates to reasoning about the model. Generally speaking, it can be more difficult to reason about imperative models as their functionality may depend on a global model state. In order to formally verify the behaviour of a function inside an imperative model, one must consider all the methods that can also manipulate the global state. In contrast, a purely functional model consists of referentially transparent functions. 

Another benefit relates to testing and verification of the model. When states are introduced to a model the complexity increases significantly as there are more moving parts and thereby more test combinations to be written if the model is to be tested exhaustively. In practice, this means that a model following the FP requires fewer tests to be written as one does not need to consider all the states the model may appear in.

However, modelling Chess through an FP architecture also has some limitations, in particular related to the stateful aspects of the game e.g., when modelling castling one needs knowledge of whether the involved king and rook have moved. To determine this with FP one must know the previous moves made in the game and determine if the pieces were involved. With OOP one can simply introduce a boolean attribute on the rooks and kings indicating whether they have moved. While this is also possible following the FP, it implicates all the pieces as inheritance is not available.

At this point, it should be clear that there are benefits to both types of architecture. One should consider whether it is more important to minimise or encapsulate the moving parts when writing the architecture, as this decision may greatly impact the difficulty of the implementation. Finally, one must also consider how exhaustively the model is to be tested as a stateless model should contain less testing combinatorics compared to a stateful model.

An example of how FP can be more elegant than OOP can be found when moving pieces. It is to be assumed that a \texttt{Move} is modelled as a composite type of two \texttt{Piece}s, one indicating which \texttt{Piece} is being moved and the other indicating where it is moved to\footnote{The latter must be a \texttt{Piece} over a \texttt{Coordinate} to account for promotion.}, and that the \texttt{BoardState} is a \texttt{set of Piece}s.
In the OOP case, one would need to find the \texttt{Piece} in the current \texttt{BoardState} with similar attributes as \texttt{Move.from\_}, check if a \texttt{Piece} is occupying the same square as \texttt{Move.to\_}, potentially remove that \texttt{Piece}, and update the found \texttt{Piece} to match the new \texttt{Coordinate}. In the case of promotion, one would need to add a new \texttt{Piece} of the promoted type and remove the original.
When following FP one can simply update the \texttt{BoardState} by filtering out the \texttt{Piece}s with the \texttt{Coordinate}s in the \texttt{Move} and making a union with \texttt{Move.to\_}.

\subsection{Invariants on Compound Types in VDM++}

During the early stages of development where the OOP architecture was used, an issue within the VDM tool ``VDMJ'' \cite{Battle09} lead to an interesting discussion, that at its core relates to the choice of paradigm. The issue relates to the model snippet seen below where a \texttt{Piece} is moved on to a given \texttt{Coordinate}. The operation updates the state of the \texttt{Board} by first removing the captured \texttt{Piece} and then updating the position of the moved \texttt{Piece}. Since class instances are reference types in VDM++ the latter can be done through the assignment operator directly on the \texttt{Piece} as the \texttt{board\_state} has a reference to the instance. The \texttt{board\_state} is a \texttt{set of Piece} with an invariant stating that the positions of the \texttt{Piece}s inside the set must be unique.

\begin{lstlisting}
public move: Piece * Piece`Coordinate ==> ()
move(piece, coord) == (
    let current_coords = state_to_coords_set(board_state) in
        if coord in set current_coords then
            let dead_piece = {p | p in set board_state & p.position = coord} in
                board_state := board_state \ dead_piece;
    piece.position := coord
)
pre piece in set board_state and dead_piece in set board_state;
\end{lstlisting}

The logic of the model seemed sound but in practice, the interpreter would report an invariant violation when a \texttt{Piece} was captured. When debugging the operation it was shown that the invariant was correctly checked when updating the position of the \texttt{Piece}, but it was checked against an old \texttt{board\_state} still containing the \texttt{dead\_piece}, which caused the invariant to be violated.
In short, the issue was caused by an ``invariant listener'' on the \texttt{piece} object that was not correctly updated when the \texttt{board\_state} was modified\footnote{A link to the issue can be found here:\\\url{https://github.com/overturetool/vdm-vscode/issues/197}}. While the exact issue is not a concern of this paper, the complexity of having invariants on compound types containing references is an interesting topic that showcases some of the issues mutability brings. 

Since VDM++ objects are references they bring aliasing to the language, i.e., an object like \texttt{piece} can be accessed and modified in several places in the specification. If such an object reference is also a member of a compound type instance with an invariant, e.g., \texttt{board\_state}, the invariant for the compound type must be checked whenever the object is modified. The invariant must also be checked when the compound type instance itself is modified.
Furthermore, an object like \texttt{piece} could potentially be placed in multiple instances of different compound types with different invariants, which would mean modifying \texttt{piece} would result in several invariants being checked. In the case of Chess this might not be an issue as the data structures are relatively flat but in complex industrial cases it may not be the case. The reason for this issue showing up is that in order to make the VDM interpreter efficient while still ensuring that invariants are not violated it only tests the invariants whenever changes are made where the invariants are used. In order for this to work in the presence of aliasing this essentially requires checking the transitive closure of instances connected. This is obviously not efficient so the VDM interpreter ignores such invariants going across instances and thus this is a challenge for the OOP model presented here. 

Had the specification been written using immutable datatypes like composite types, the invariant issue would not have been a concern, as the \texttt{BoardState} would then be a set containing immutable values instead of references. This means that the invariant would only need to be checked when the \texttt{board\_state} was changed (or following an entirely FP, when a new \texttt{BoardState} was created).

While the writer of a specification should not typically concern themselves with the details of the tools, we believe this example makes a strong argument of why following the OOP may introduce unnecessary complexity to a specification. If one was to reason about a specification following the OOP with an instance of a compound type, one would need to consider all the places where the instance was modified but also all the places where the members of the instance could be modified. All of this is without considering more complex OOP concepts such as inheritance that only strengthens the point.



\section{Overview of the FP VDM Model of Chess}
\label{sec:coremodel}

%\fbox{9 pages}

The model described in this section follows the FP architecture as shown in \cref{sec:paradigms}, where a bottom-up approach will be taken of the most interesting parts of the model.

\subsection{PieceModule}
The PieceModule (\texttt{PM}) class has the responsibility of defining the types necessary to describe a Chess piece and providing the functions necessary to describe their basic movement patterns.

The piece type can be modelled as a union of quote types with the following options: \texttt{<pawn>|<rook>|<knight>|<bishop>|<queen>|<king>}. A composite type is used to define the \texttt{Coordinate}s which consists of two \texttt{nat1} to describe the x- and y-coordinates of a piece. The convention of annotating the x-axis through letters is abstracted away to make the two axes consistent. An invariant is put on \texttt{Coordinate} to ensure that the position is legal, i.e., the values are less than nine.

A data structure for the \texttt{Piece}s could now be established as a composite type containing the attributes \texttt{type : PieceType}, \texttt{square : Coordinate}, and \texttt{colour : Colour}.

The simple moves, i.e., excluding special moves, can be found for a given \texttt{Piece} using the \texttt{type\_based\_moves} function that takes \texttt{Piece * ObstacleSet} as parameters and returns a \texttt{set of Coordinates} containing the valid \texttt{Coordinate}s that the \texttt{Piece} can move to.
An \texttt{ObstacleSet} is needed to filter out the moves that are invalid due to the \texttt{Piece}'s path being blocked by an \texttt{Obstacle}. An \texttt{Obstacle} contains a \texttt{Coordinate} and \texttt{Colour} where the \texttt{Colour} is needed to indicate if the \texttt{Obstacle} is capturable or not.
Alternatively, one could have modelled the function without the \texttt{ObstacleSet} by returning the collection of legal \texttt{Coordinate}s assuming the board was empty since this would decouple the \texttt{PieceModule} further from the state of the board. However, it would require the \texttt{BoardModule} to be more closely coupled to the \texttt{PieceType} as it would need a special case for handling the movements of a pawn, as the pawn moves forward but attacks diagonally.

The \texttt{type\_based\_moves} function is written using a cases expression to pattern-match the \texttt{PieceType} to a function defining the specific movement pattern of the piece. The movement pattern of the knight and king can be modelled using a common helper function, \texttt{possible\_move\_direction}, as seen in the snippet below. Among the \texttt{Piece} and \texttt{ObstacleSet}, a pair of ints are provided as parameters to indicate the direction where the possible moves are to be considered. A new \texttt{[Coordinate]} is generated based on the \texttt{dir} through \texttt{coordinate\_factory} that returns \texttt{nil} if the provided inputs result in an invalid \texttt{Coordinate}. \texttt{possible\_move\_direction} evaluates to \texttt{nil} if the generated \texttt{Coordinate} is invalid or occupied by a friendly piece. Otherwise, the new \texttt{Coordinate} is returned.

\begin{lstlisting}
possible_move_direction: Piece * ObstacleSet * (int * int) -> [Coordinate]
possible_move_direction(p, os, dir) ==
let new_c = coordinate_factory(p.square.x + dir.#1, p.square.y + dir.#2) 
in
  if (new_c = nil) or 
     exists piece in set os &
      (piece.square = new_c) and (piece.colour = p.colour) 
  then nil
  else new_c;
\end{lstlisting}

Having now defined \texttt{possible\_move\_direction} the movement pattern of the knight can be defined as below.

\begin{lstlisting}
knight_move_pattern : Piece * ObstacleSet -> set of Coordinate
knight_move_pattern(p, os) ==
  {possible_move_direction(p, os, mk_(1, 2)), -- 2Up1Right
   possible_move_direction(p, os, mk_(-1, 2)), -- 2Up1Left
   possible_move_direction(p, os, mk_(1, -2)), -- 2Down1Right
   possible_move_direction(p, os, mk_(-1, -2)), -- 2Down1Left
   possible_move_direction(p, os, mk_(2, 1)), -- 1Up2Right
   possible_move_direction(p, os, mk_(-2, 1)), -- 1Up2Left
   possible_move_direction(p, os, mk_(2, -1)), -- 1Down2Right
   possible_move_direction(p, os, mk_(-2, -1)) -- 1Down2Left
    } \ {nil} 
pre p.type = <knight>;
\end{lstlisting}

Similarly, the other piece types rook, bishop and queen has been modelled using the \texttt{possible\_moves\_direction} as seen in the snippet below. This time recursion is needed as these can continue in a direction they are blocked or can capture a piece.
Once again a new \texttt{[Coordinate]} is generated.
The first conditional is identical to the one in \texttt{possible\_move\_direction} but the empty set is returned in this case. A check is then made to determine if an opponent is on the square that is being evaluated, where the recursion is terminated and a set containing the \texttt{Coordinate} is returned. At last, the recursive case is defined where a union between the set containing the \texttt{Coordinate} and the result of recursively calling the function in the same direction is returned.

\begin{lstlisting}
possible_moves_direction: Piece * ObstacleSet * (int * int) -> set of Coordinate
possible_moves_direction(p, os, dir) ==
  let new_c = coordinate_factory(p.square.x + dir.#1, p.square.y + dir.#2) 
  in
     if (new_c = nil) or 
        exists piece in set os &
                (piece.square = new_c) and (piece.colour = p.colour) 
    then {}
    elseif exists piece in set os &
                (piece.square = new_c) and (piece.colour = opposite_color(p.colour))
     then {new_c}
     else {new_c} union 
            possible_moves_direction(mk_Piece(p.type, new_c, p.colour), os, dir);
\end{lstlisting}

\noindent In principle this recursive function should have a proper \texttt{measure} ensuring the termination but it is not straightforward.

The movement pattern of the queen can then be defined as seen below.

\begin{lstlisting}
queen_move_pattern : Piece * ObstacleSet -> set of Coordinate
queen_move_pattern(p, os) ==
  dunion {possible_moves_direction(p, os, mk_(0, 1)),
          possible_moves_direction(p, os, mk_(0, -1)),
          possible_moves_direction(p, os, mk_(1, 0)),
          possible_moves_direction(p, os, mk_(-1, 0)),
          possible_moves_direction(p, os, mk_(1, 1)),
          possible_moves_direction(p, os, mk_(-1, -1)),
          possible_moves_direction(p, os, mk_(-1, 1)),
          possible_moves_direction(p, os, mk_(1, -1))}
pre p.type = <queen>;
\end{lstlisting}

\subsection{BoardModule}
The BoardModule (\texttt{BM}) class has the responsibility of defining and updating the state of a chessboard. Furthermore, it determines whether or not the special rule moves are possible.

The first type defined in the BM class is the composite type \texttt{Move}. Initially, \texttt{Move} was modelled as a product type consisting of a \texttt{Piece * Coordinate}\footnote{\texttt{Move} was changed from a product type to a composite type as the intent is clearer when the fields are named compared to referencing them through ''.\#1'' and ''.\#2''.}. This structure made sense up until the point where promotion was implemented since promotion allows for the \texttt{PieceType} to be changed, which could not be captured with the old definition.
Instead, \texttt{Move} was modelled with the attributes \texttt{from\_} and \texttt{to\_} that are both of type \texttt{Piece}. An invariant was placed upon \texttt{Move} that states the following: \texttt{m.from\_.colour = m.to\_.colour and m.from\_.square <> m.to\_.square} \\since a \texttt{Move} cannot change the colour and must update the position of the \texttt{Piece}.

It was then possible to define \texttt{History} as a sequence of \texttt{Move}s. A sequence was chosen since the ordering of the moves matters and there might be duplicates if a player moves a piece back and forth.

A \texttt{BoardState} could then be defined as a \texttt{set1 of Piece}, where a set was chosen since the ordering does not matter and duplicates are not allowed as that would indicate two pieces of the same colour and type being placed on the same square. To restrict two \texttt{Piece}s from sharing a position the following invariant was written: \texttt{forall p1, p2 in set b \& p1 <> p2 => p1.square <> p2.square}.
Finally, a \texttt{Board} type was introduced as a composite type containing a \texttt{BoardState} and a \texttt{History}.

The function \texttt{possible\_moves} is responsible for finding the set of valid moves for a \texttt{Piece} where both the simple- and special movement patterns are considered. The function finds the set of simple type-based moves, the set of stateful special moves, and the set of illegal stateful moves. It then evaluates to the union between the former two with the set difference of the latter. The logic can essentially be boiled down to ``find the entire set of possible moves and remove the impossible ones''. Here a set comprehension is used to convert the \texttt{Coordinate}s from \texttt{simple\_moves} to \texttt{Move}s through the helper function \texttt{piece\_coord\_to\_move}.

\begin{lstlisting}
public possible_moves : Board * PM`Piece -> set of Move
possible_moves (board, piece) == (
  let state_p_moves = stateful_possible_moves(board, piece),
      state_imp_moves = stateful_impossible_moves(board, piece),
      simple_moves = PM`type_based_moves(piece,
          PM`pieces_to_obstacles(board.board_state)) in
          ({piece_coord_to_move(piece, c) | c in set simple_moves} union
              state_p_moves) \ state_imp_moves
)
pre piece in set board.board_state;
\end{lstlisting}

The function \texttt{stateful\_impossible\_moves} yields the \texttt{set of Move} containing \texttt{Move}s that result in the player losing, as the rules of Chess disallow such a move from being performed. Thus it contains the moves that put the player in check and if the player already is in check it filters out moves that do not put them out of check.
Furthermore, if the \texttt{PieceType} is \texttt{<pawn>} and promotion is possible then it also contains the \texttt{Move} where the pawn moves to a square on the last rank without changing the \texttt{PieceType}, as it is illegal to not promote the \texttt{Piece}.

\texttt{stateful\_possible\_moves} is a dispatcher that considers the special rules of the pieces, i.e., castling, promotion, en passant and the option for a pawn to move two squares on its first move.

\begin{lstlisting}
stateful_possible_moves: Board * PM`Piece -> set of Move
stateful_possible_moves(board, piece) == (
  cases piece.type:
      <pawn> -> dunion {
          pawn_move_two(board.board_state, piece),
          en_passant(board, piece),
          pawn_promotion(board.board_state, piece)},
      <king> -> castling_possible(board, piece),
      others -> {}
  end
);
\end{lstlisting}

An example of how a special move can be implemented is seen in the snippet below where promotion is modelled. First, the local definitions \texttt{last\_y} and \texttt{promotable} \texttt{\_types} are defined. A set containing the \texttt{promotion\_squares} is constructed, which is simply the set containing the \texttt{Coordinate}s that the pawn could normally move to that are also placed on the last rank. The polymorphic helper function \texttt{sets\_combine\_} \texttt{tuple} is finally used to generate a set of tuples with combinations of \texttt{promotable\_types} and \texttt{promotion\_squares}, which can be used to return the set of promotion \texttt{Move}s.

\begin{lstlisting}
pawn_promotion: BoardState * PM`Piece -> set of Move
pawn_promotion(board_state, pawn) == (
    let last_y = if pawn.color = <white> then 8 else 1,
        promotable_types = {<knight>, <bishop>, <rook>, <queen>} in
        let promotion_squares = {coord | coord in set
          PM`type_based_moves(pawn, PM`pieces_to_obstacles(board_state))
            & coord.y = last_y} in
            {mk_Move(pawn, mk_PM`Piece(t_c_tuple.#1, t_c_tuple.#2, pawn.color)) |
              t_c_tuple in set sets_combine_tuple[PM`PieceType, PM`Coordinate]
              (promotable_types, promotion_squares)}
)
pre pawn.type = <pawn> and pawn in set board_state;
\end{lstlisting}

So far the functions have focused on how the valid moves could be determined. Performing a move is done similarly through the function \texttt{move} where it is necessary to have different behaviour for castling and en passant as the former changes the \texttt{PieceType} and the latter captures a \texttt{Piece} without moving to the square. The other types of moves can be modelled through \texttt{move\_other}. The precondition specifies that the \texttt{Move} must be valid and the postcondition specifies that the returned \texttt{Board} has a different \texttt{BoardState} and a longer \texttt{History}.

\begin{lstlisting}
public move: Board * Move -> Board
move(board, mov) == 
  if mov.from_.type = <king> and iss_castling(board, mov) 
  then move_castling(board, mov)
  elseif (mov.from_.type = <pawn> and iss_en_passant(board, mov)) 
  then move_en_passant(board, mov)
  else move_other(board, mov)
pre mov in set possible_moves(board, mov.from_)
  and mov.from_ in set board.board_state
post len board.history < len RESULT.history
  and board.board_state <> RESULT.board_state;
\end{lstlisting}

The definition of \texttt{move\_other} is seen below. First, the potentially captured piece is found which is defined in the local definition \texttt{dead\_piece}. Due to the invariant on \texttt{BoardState} it is guaranteed to contain a single \texttt{Piece} or be the empty set. The new \texttt{BoardState} can then be defined in \texttt{new\_state} as the previous state without the \texttt{dead\_piece} and with an updated version of the moved \texttt{Piece}. Finally, the new \texttt{Board} is returned.

\begin{lstlisting}
move_other: Board * Move -> Board
move_other(board, mov) == (
  let dead_piece = {p | p in set board.board_state & p.square = mov.to_.square} in
      let new_state = (board.board_state \
        (dead_piece union {mov.from_})) union {mov.to_} in
          mk_Board(new_state, [mov] ^ board.history)
)
pre pre_move(board, mov)
post post_move(board, mov, RESULT);
\end{lstlisting}

Additionally, a helper function \texttt{default\_board} can be made that defines a board with the initial position as seen in \cref{fig:coremodel:board_initial} and an empty \texttt{History}. This is used as the starting point for new games.

\begin{lstlisting}
public default_board : () -> Board
default_board() ==
(
    let board_state : BoardState = dunion {
      {mk_PM`Piece(<pawn>, mk_PM`Coordinate(x, 2), <white>) | x in set {1,...,8}},
      {mk_PM`Piece(<pawn>, mk_PM`Coordinate(x, 7), <black>) | x in set {1,...,8}}
      -- Repeat for other PieceTypes
    } in
        mk_Board(board_state, [])
);
\end{lstlisting}

\subsection{GameModule}
The GameModule (\texttt{GM}) class has the responsibility of controlling who has the current turn and declaring the game-winner. The module defines the optional union of quote types \texttt{Winner} with the values \texttt{[PM`Color | <remis>]} where \texttt{nil} indicates that the game is ongoing. Furthermore, the module defines the composite type \texttt{Game} which contains a \texttt{Board} and a \texttt{PM`Color} indicating the turn.

The function \texttt{move} is used to perform a \texttt{Move} on the \texttt{Board} and potentially determine the winner. First, the \texttt{Move} is performed through \texttt{BM`move} and saved in the local definition \texttt{new\_board}. Then it is checked whether the opponent has any valid \texttt{Move}s for the next turn. If not then the \texttt{Game} is either won by the player or ended in remis, depending on whether the opponent is in check.

\begin{lstlisting}
public move : Game * BM`Move -> (Game * Winner)
move(game, mov) == (
  let new_board = BM`move(game.board, mov),
    opposite_c = PM`opposite_color(game.turn) in
    if forall p in set new_board.board_state &
      p.color = opposite_c => BM`possible_moves(new_board, p) = {} then
      if BM`in_check(new_board.board_state, opposite_c) then
        mk_(game, game.turn)
      else
        mk_(game, <remis>)
    else
      mk_(mk_Game(new_board, opposite_c), nil)
)
pre mov.from.color = game.turn and
  mov in set BM`possible_moves(game.board, mov.from)
post len game.board.history < len RESULT.#1.board.history
    and game.board.board_state <> RESULT.#1.board.board_state;
\end{lstlisting}

\subsection{Runner}
The last part of the model consists of a class, \texttt{Runner}, that reads the contents of a PGN file (\cref{sec:PGN}), converts it to \texttt{Move}s through the PGN module, and iteratively performs the moves.

From a \texttt{Runner} perspective it could also potentially be interesting to create a graphical rendering of the Chess board itself. This could potentially be achieved using either VDMPad \cite{Oda&15a} or ViennaTalk \cite{Oda&16a}. In an Overture context this kind of visualisation was also enabled when it was based on Eclipse but this has not yet been fully incorporated in the VSC version \cite{Nielsen&12}.

\section{Portable Game Notation}
\label{sec:PGN}

%\fbox{1 page}

The Portable Game Notation~\cite{PortableGameNotation2022} (PGN) is the de-facto standard used for chess annotation on many online chess websites.\footnote{PGN is supported on websites such as \url{chess.com} \url{https://lichess.org/}, and \url{chess24.com}.}
PGN consists of information related to the chess game (e.g. player information) and move text, where the move text describes the actual piece moves of the game. PGN uses letters for the x-axis and numbers for the y-axis as described in \cref{sec:background}. Generally speaking, a move in PGN consists of the \texttt{PieceType} as the first character and the \texttt{Coordinate} as the second and third characters. Furthermore, there is special notation for castling, check, checkmate, and extra information may be added to remove ambiguity.

PGN was added as a way to verify the overall integrity of the VDM model by testing it on some real-world data. Since VDM++ does not include a string manipulation library it was necessary to partly define one. Furthermore, the PGN class has the responsibility of parsing a valid \texttt{String} describing a game of chess through PGN notation to the VDM++ \texttt{Move} representation and vice versa.

\begin{lstlisting}
values
 numerical_chars = "0123456789";
 numerical_char_to_nat : inmap char to nat = 
             {numerical_chars(i) |-> i-1 | i in seq numerical_chars};

 valid_x_chars = "abcdefgh";
 x_char_to_nat1 : inmap char to nat1 = 
              {valid_x_chars(i) |-> i | i in seq valid_x_chars};

 piece_type_to_string : inmap PM`PieceType to String = 
                        {<pawn> |-> "", <rook> |-> "R", <knight> |-> "N",
                         <bishop> |-> "B", <queen> |-> "Q", <king> |-> "K"}

functions

public move_to_pgn_string: BM`Move -> String
move_to_pgn_string(move) ==
    let piece_type = piece_type_to_string(move.from.type),
        x = (inverse x_char_to_nat1)(move.to_.square.x),
        y = (inverse numerical_char_to_nat)(move.to_.square.y) 
     in
        piece_type ^ [x] ^ [y];
\end{lstlisting}

The helper function \texttt{move\_to\_pgn\_string} is used in \texttt{Runner} when parsing a chess game and logging the results to a text file. The \texttt{PieceType} is mapped to a string by applying the type to \texttt{piece\_type\_to\_string}. Similarly, the x- and y-coordinates are found but the inverse mappings are used here, which is possible as the maps are injective. 

This class is currently made in VDM but since VDM is not really meant for parsing strings it would make sense to redo this part in Java as a new library for reading this external format directly into VDM structures. It could even use scanner and parser generators inside if desired but this is mainly an issue about the potential error reporting in case the string provided does not live up to the PGN syntax.



\section{Related work}
\noindent \textbf{Video foundation models.}
With sufficient computational power and an abundant source of data, there have been attempts to build a single large-scale foundation model that can be adapted to diverse downstream tasks.
Along with the success of foundations models in the natural language processing domain~\cite{brown2020language,chen2021evaluating,devlin2019bert} and in computer vision~\cite{bertasius2021space,jia2021scaling,radford2021learning}, video data has become another data type of interest, as it has grown in scale due to numerous internet video-sharing platforms.
Accordingly, several methods to train a video foundation model have been proposed.
Due to the innate multi-modality of video data, \textit{i.e.}, a combination of visual $\cdot$ vocal $\cdot$ textual context, most works have centered around the variations of the cross-modal attention mechanism \cite{akbari2021vatt,bertasius2021space,gabeur2020multi,luo2020univl,neimark2021video,tan2021look,wei2020multi,yang2021taco}.
In addition, as most video data lack proper labels or descriptions, contrastive learning methods were studied to learn meaningful feature representations or enhance video-text alignment in a self-supervised manner \cite{akbari2021vatt,kuang2021video,luo2020univl,yang2021taco}.

More specifically, MERLOT \cite{zellers2021merlot} proposed a multi-modal representation learning method for visual commonsense reasoning, which also performed well in twelve video reasoning tasks.
VATT \cite{akbari2021vatt} introduced a multi-modal learning method via contrastive learning. 
The pre-trained model performed well in a variety of vision tasks from image classification to video action recognition and zero-shot video retrieval.
Another representative work, UniVL \cite{luo2020univl} proposed a straightforward pre-training method with auxiliary loss functions. 
After fine-tuning on a specific task, the pre-trained model performed outstandingly in a wide range of tasks of text-to-video retrieval, action segmentation, action step localization, video sentiment analysis, and video captioning.
Other foundation models for multiple video tasks include \cite{li2020hero,sun2019learning,sun2019videobert,zhu2020actbert,fu2021violet,wang2022all}. 

\noindent \textbf{Auxiliary learning.}
In order to enhance the performance of one or a multitude of primary tasks, auxiliary learning methods can be incorporated.
\cite{ruder2017overview} introduced Multi-task learning (MTL) to the deep neural networks by training a single model with multiple task losses to assist learning on the main task.
Such a method is generally adapted to pre-train the foundation models in the self-supervised manner~\cite{li2020hero,sun2019learning,sun2019videobert,zhu2020actbert,fu2021violet,wang2022all}.
However, these various pretext task losses used in the pre-training phase are ignored in the fine-tuning phase, and only the primary task loss is minimized.

Recently, meta-learning methods have been introduced for auxiliary learning.
\cite{liu2019self,navon2020auxiliary,shu2019meta} proposed a meta-learning method in which the model learns auxiliary tasks to generalize well to unseen data. 
In these settings, a separate subset of data is held out as the primary task, while the others are used as auxiliary tasks that aid the primary task's performance.
Similar methods were adopted for computer vision tasks such as semantic segmentation \cite{xu2021leveraging}.
Other domain applications include navigation tasks with reinforcement learning \cite{ye2021auxiliary}, or self-supervised learning methods on graph data \cite{hwang2020self}.

\section{Conclusions}
We consider the phase-extraction problem, and we showed that, given a unitary $U = e^{i\pi H}$ and its inverse $U^{\dag}$, we could implement a block-encoding of $\phi(H)$ for some smooth function $\phi(x)$. The word `smooth' here means existence and continuity of the derivatives: the higher the number of continuous derivatives that a function has, the faster its Fourier sum (and thus the Laurent polynomial on the eigenphases) uniformly converges to that function. We are confident this can have many more applications beyond what is shown in this work. It is also worth remarking that Jackson showed that the convergence rate of a Fourier series is almost-optimal, in the sense that no trigonometric (or, equivalently, complex exponential) series can approximate the desired function faster, up to that $\log d$ factor~\cite[p.\ 21]{jacksonTheoryApproximation1930a}. Also remember that `smoothing' a function, i.e., replacing its derivative with a continuous function, does not give faster convergence for free in general, as its derivative will become steep in the points where we smooth out discontinuities, and this translates to a high Lipschitz constant: a~clear example is given by Eq.~\ref{eq:lipschitz-constant-recurrence-solution}, but in that case, fortunately, nothing depends on the size of the input $N$, and thus does not influence the asymptotic query complexity of Algorithm~\ref{alg:prop-sampling-qsp}, although the constant factor can become large even for $p = 20$. From a theoretical point of view, this work shows that, for any $\eta > 0$, there is an algorithm with query complexity 
$$\Tilde{\bigO}\left(\frac{1}{\bar{c}^{\frac{1}{2} + \eta}} \frac{1}{\epsilon^\eta} \right)$$
solving the proportional-sampling problem. This statement seems to suggest there exists an algorithm which directly solves the problem with $\eta = 0$, and an open question would be to find such algorithm.


It is also interesting to remark that Theorems~\ref{thm:haah-construction},~\ref{thm:haah-completion} indeed allow the construction for any $\phi$, even complex-valued, provided that its absolute value is reciprocal.

One could think that, in Section~\ref{sec:prop-sampling}, instead of using the linear function in the phase-extraction subroutine, we could approximate the square root and then apply the transformation directly on $e^{i \pi c(x)}$. However, in the case of proportional sampling this would be inconvenient, as the derivative of the square root function has a discontinuity with an infinite jump around 0, and we could not choose a constant $\delta$ if we had values of the oracle that are too close to $0$.

\paragraph{Acknowledgements}
We would like to thank the anonymous reviewers for valuable feedback on the original version of this paper.

\bibliographystyle{splncs04}
\bibliography{references.bib}
%\fbox{1 page for references}

\end{document}
