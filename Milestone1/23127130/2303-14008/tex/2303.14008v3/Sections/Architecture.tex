\section{Architecting Scalable and Reconfigurable Quantum Processors}
%\textcolor{red}{BSC, UPV}
NISQ devices are publicly accessible (e.g., IBM Q experience, Qinspire) and users can already run small instances of quantum algorithms. However, these quantum processors suffer from several constraints, such as limited connectivity 
among the qubits. This requires the quantum compiler to modify the quantum algorithm, described as a quantum circuit, to realize it on a given quantum chip. 
This transformation process is known as mapping and might compromise the successful execution of the algorithm \cite{almudever2020realizing}. 

Exact approaches for small-size 
quantum circuits and approximate mapping solutions using heuristics for larger quantum circuits have been developed for specific single-core NISQ 
devices \cite{lao2021timing,murali2019full}. Recently, the first compiler techniques for mapping and scheduling quantum algorithms onto connectivity-simplified multi-core quantum 
architectures have been proposed \cite{baker2020time,rodrigo2021double} as distributed or modular architectural approaches are becoming more prominent for scaling-up quantum hardware \cite{vandersypen2017interfacing, chow2021quantum, laracuente2022short,brown2016co}. These 
mapping solutions all focus on multi-Qcore architectures that assume all-to-all intra- and inter-core connectivity with unlimited and ideal communication resources and a fixed and invariant interconnection network. In this work it is proposed to
build on top of these quantum hardware advances and define scalable multi-Qcore architectures, including the communication perspective and the required compilation techniques, connecting today's experiments with future practical implementations. We will also perform on-line dynamic optimizations of the Qcores transfers for an efficient execution of the algorithms leveraging the reconfigurability provided by the wireless control plane.


\begin{comment}
\begin{figure}[t!]
	\centering
	\includegraphics[width=0.8\linewidth]{fig/Mapping.png}
 	\caption{Mapping a quantum algorithm based on its qubit interaction graph into a multi-Qcore architecture.}
 	\label{fig:comms}
\end{figure}
\end{comment}