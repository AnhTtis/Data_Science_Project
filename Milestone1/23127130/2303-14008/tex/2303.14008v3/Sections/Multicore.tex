\section{Multi-core quantum computing architectures enabled by a hybrid interconnection network}

\label{SecII}

%\textcolor{red}{UPV, state-of-the-art on multi-core quantum computers}


One of the main challenges for scaling up quantum computers is the wiring between the quantum processor and control electronics \cite{xue2021cmos}. Existing quantum computers physically span several temperature levels, with the qubits typically operating at sub-1-K temperatures to ensure their coherence, while the classical control electronics are operating at room temperature. While this thermal gap can be readily bridged by a few wires for the small quantum processors available today ($<$100 qubits), it would be impractical to wire thousands or millions of qubits in future quantum computers due to wiring size and reliability issues. Recent advances are closing this vertical gap by pushing the qubits to operate at temperatures near 4K \cite{petit2020universal}, where cryogenic refrigerators can dissipate the heat generated by the classical control electronics, and by developing cryogenic analog/RF control circuits that can be potentially co-integrated with the qubits at the same temperature level \cite{xue2021cmos}. Additionally, recent research advocates for also moving digital circuits to cryogenic temperatures \cite{schriek2020cryo}, an approach that further alleviates the wiring bottleneck. 

%as it avoids the need to have a room-temperature host computer mapping the algorithm to the quantum processor and controlling its execution. 

Although the full-cryogenic computer resulting from closing the vertical gap would be more compact than today’s prototypes, scaling issues remain in the horizontal dimension. Specifically, scaling this integrated scheme to a very large number of qubits as a single array (Qcore) is deemed unfeasible due to the practical bottleneck in the wiring between the qubits and their interface electronics \cite{vandersypen2017interfacing}. Moreover, other issues arise when integrating a high amount of qubits on a single chip, such as crosstalk, limited yield and qubit addressability. Instead, the multi-Qcore vision proposed here assumes $N_c$ quantum chips, with each chip hosting a Qcore composed of a moderately sized array of $N_q$ qubits. Thus, the Qcores can be spaced to fit the local electronics next to each core, thereby improving qubit addressability and reliability. Connecting the Qcores via quantum links allows the ensemble to remain coherent, thus maintaining the computational power of $N_c \times N_q$ entangled qubits, i.e. $O(2^{N_cN_q})$, rather than the incoherent addition of the Qcores, $O(N_c2^{N_q})$. The realization and integration of such quantum links at the chip scale is a fundamental challenge that has not been addressed yet.

\begin{comment}
\begin{figure}[t!]
	\centering
	\includegraphics[width=\linewidth]{fig/Qchannel.png}
 	\caption{Communication among quantum cores through a quantum cavity channel and a classical mm-wave channel.}
 	\label{fig:Qlink}
\end{figure}
\end{comment}

%The first key element to enable the vision is an expansive quantum cavity channel as enabler of the quantum-coherent interconnect fabric. By this approach, the chips containing the Qcores are connected to the cavity by means of a flip-chip assembly configuration, as shown in Fig.~\ref{fig:Qlink}. The quantum channel proposed here, instead of using optical photon-based entangled pairs that are prevalent in large-scale systems, operates through entangled microwave in a substrate dielectric waveguide \cite{kannan2020generating}. The underlying principle is that qubits, when excited at a proper frequency, can serve as high-quality quantum emitters \cite{giounanlis2019photon}. The silicon qubits are strongly coupled to the cavity waveguide such that their quantum information can be transmitted as standing-wave photons through the waveguide. These microwave photons emitted from multiple sources into the waveguide will interact with each other. If properly controlled and synchronized, the quantum interference between the photons can generate spatially entangled traveling photons. This enables high quality communications featuring high entanglement rates between the quantum cores. %In the proposed approach, we will demonstrate this quantum link using quantum-dot (QD)-based spin/position hybrid qubits, showing that (i) an absorbed microwave photon can facilitate tunneling of electrons between QDs by means of pumping energy into the system, and (ii) the reverse process is possible as well: a microwave photon emission due to the tunneling of particles in a quantum transport operation where double QDs are coupled through a cavity.


The first key element to enable the vision is an expansive quantum channel as enabler of the quantum interconnect fabric. The other key element in the vision (in hybrid co-existence) is a multi-chip interconnection network that, with the exchange of classical data, coordinates the emitter and receiver of the quantum state transfer, and  enables the sharing of control signals and data across the Qcores. Without such an interconnection network, all data would have to go through the vertical connections to the host computer and back, which quickly becomes a roadblock. One could implement the inter-Qcore network through wired interconnects, yet this approach has drawbacks such as the difficulty of routing physical wires through the already densely wired system, or the high latency of system-wide transfers, which are critical for control. In this position paper, instead, we propose to realize the classical side of communications by means of a compact, high-bandwidth, highly reconfigurable wireless in-package network. The main idea is to integrate on-chip cryogenic antennas with RF cryogenic transceivers (developed with the same technology and techniques already employed in existing high-frequency qubit control circuits) so as to implement a wireless network within the quantum computing package. The resulting network is naturally system-wide and broadcast, allowing for agile reconfiguration of the underlying architecture. Moreover, although designing integrated circuits beyond their originally intended temperature operating range presents several challenges, the cryogenic environment is expected to opportunistically boost the antenna efficiency, reduce the thermal noise, and improve the transistor speed \cite{xue2021cmos,8036394}.

