\section{Design Space Exploration for Architecture/Network/Circuit/link Co-Design}

\label{SecvII}

Although quantum computers already exist in the form of intermediate-scale quantum processors, there is no consensus in the quantum computing community on what benchmarks and metrics to use to compare them and to assess their performance. As a first attempt, IBM proposed Quantum Volume (QV) and Circuit Layer Operations Per Second (CLOPS) as metrics to measure three key quantum computing performance attributes: quality, speed and scale \cite{wack2021quality}. Also, different sets of quantum benchmarks for assessing their overall performance have been introduced \cite{blume2020volumetric, mills2021application}. In addition, the higher layers of full-stack systems, consisting of both quantum software and classical control hardware, have been so far designed and provided solutions following a bottom-up approach; that is, they have been developed for a specific quantum device. This approach is appropriate for a single rigid design, which could be functional, but it is unclear how close it is to the optimal design in terms of overall performance and scalability, and how such performance holds in different specifications. This is particularly critical in the harsh constraints of quantum computing. In addition, a long-standing perceived need from the quantum community identifies the lack of design guidelines from top architectural layers down to physical ones or even cross-layer co-design \cite{almudever2021structured, tomesh2021quantum}. 

In this work, we propose to employ structured Design Space Exploration (DSE) methodologies to perform a cross-layer co-design of the full-stack quantum system, including both communication and computation, allowing for both top-down and bottom-up optimizations across layers (see Figure \ref{fig:DSE}). To this purpose, models will be developed that describe (i) the computational part, from the distributed architecture down to the single Qcore and qubit impairments, and (ii) the classical and quantum communication parts from the perspective of performance (e.g. throughput, fidelity) and efficiency (e.g. power). Finally, we will create and profile a set of large-scale benchmarks able to stress the multi-Qcore platform and define different performance metrics of the overall architecture. 


\begin{figure}[t!]
	\centering
	\includegraphics[width=0.8\linewidth]{fig/DSE.png}
 	\caption{Exploration framework for quantum computing architectures.}
 	\label{fig:DSE}
\end{figure}
