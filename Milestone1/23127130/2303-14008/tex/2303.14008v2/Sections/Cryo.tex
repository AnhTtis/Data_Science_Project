
\section{Cryo-CMOS RF transceivers}
%\textcolor{red}{EPFL, TU Delft}

Nanometer-scale CMOS electronics operating at cryogenic temperatures (cryo-CMOS) is widely accepted as the preferred IC technology for quantum-processor interfaces, thanks to its capability to operate down to (at least) 30 
mK, its unmatched very large scale of integration (VLSI), and the mature design automation infrastructure, all necessary to handle millions of qubits \cite{8036394, 8329135}. Several cryo-CMOS individual circuit blocks have been demonstrated, as required for qubit control and co-integration. A broad range of cryo-CMOS 
interface circuits for the control and readout of qubits has been recently shown, including complex microwave transmitters operating at 4 K %(Fig.~\ref{fig:Horse}) 
providing $<$100 MHz data bandwidth/qubit for $<$-16 dBm output power 
at a carrier frequency $<$20 GHz \cite{xue2021cmos, 9209175}, RF receivers \cite{prabowo202113} and high-speed baseband data converters \cite{9365927}. Nevertheless, the lowest reported operating temperature for a wireless transceiver is 170 K \cite{4384375}, and no deep-cryogenic transceiver has been demonstrated. 

In this work, it is proposed to implement the first-ever deep-cryogenic wireless transceiver, operating down to 4 K. To miniaturize the on-chip antenna, the operating frequency is expected to increase $>3\times$ beyond the state-of-the-art for cryo-CMOS, while 
the data bandwidth and the transmitter’s output power will both be extended by $>20\times$
beyond the state of the art to realize a robust 
communication between the qubit tiles. We will explore using the waveguide proposed in section II for classical data transmission at 60 GHz by exciting the waveguide with an on-chip antenna and by minimizing the crosstalk to the quantum  channels. 

\begin{comment}
\begin{figure}[t!]
	\centering
		\includegraphics[width=0.5\linewidth]{fig/Horse.png}
 	\caption{Horse Ridge cryo-CMOS SoC with 4x 22-nm FinFET microwave drivers for 128 qubits.}
 	\label{fig:Horse}
\end{figure}

\end{comment}