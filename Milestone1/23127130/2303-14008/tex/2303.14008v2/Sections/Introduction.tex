\section{Proposed vision}

Today’s tremendous interdisciplinary efforts towards building a quantum computer is aimed at a machine capable of tackling problems beyond the reach of any classical computer. The so-called quantum advantage, a term referring to a quantum computer performing a specific computation that is intractable for a classical computer, has been recently claimed with state-of-the-art Noisy Intermediate-Scale Quantum (NISQ) computers consisting of several tens of quantum bits (qubits) \cite{arute2019quantum}. Nevertheless, it is widely recognized that addressing any real-world 
problem will require upscaling to thousands or even millions of qubits \cite{Preskill_2018}. Scaling quantum computers to such a large number of qubits is a major challenge due to, among others, the confluence of (i) technology factors confining the qubits to low fidelity, (ii) the need for cryogenic temperatures to reach practical coherence times, (iii) the dense integration of digital/RF control circuits, which are needed on a per-qubit basis \cite{staszewski2021cryo}, and (iv) the manifold architectural and algorithmic implications of managing noisy and short-lived 
qubits. The scaling-up race is fiercely complex and mostly revolves around the qubit technological aspects, but the bottleneck will very soon shift to the architectural problems stemming from the need to densely pack together the qubits and their classical electronics interfaces. 
%This position paper, therefore focuses upon the grand challenge of scalability in quantum computers from an architectural standpoint. Such a fundamental issue will be stalling the progress of quantum computing in the coming years, thus preventing it from unleashing its enormous potential in scientific and technical fields such as chemistry, material science, and artificial intelligence.

\begin{figure}[t!]
	\centering
		\includegraphics[width=1\columnwidth]{fig/Architecture_vision4.png}
 	\caption{The multi-Qcore architectural vision.}
 	\label{fig:vision}
\end{figure}


In analogy to early von Neumann computers, the first approach to the scaling of quantum computers has been to monolithically and densely integrate qubits within a single large array on the same silicon substrate, next to their classical control electronics. However, such a solution would impose an impractically high density of interconnects wiring the electronics at the array boundary to every single qubit or, in case of sharing of control lines, impractical requirements on qubit uniformity, increased noise (crosstalk) and limited yield. A viable architectural alternative proposed by leading quantum computing experts is to split the quantum processor into smaller cores (named here as Qcores) to be sparsely placed \cite{vandersypen2017interfacing, chow2021quantum, laracuente2022short}. This approach is by no means trivial, though, as it implies the development of (i) appropriate architectural means to seamlessly manage multiple quantum cores, i.e. a multi-Qcore architecture, (ii) an interconnect that supports the quantum and digital communication needs of such architecture, and (iii) a compact integrated quantum-coherent shared medium (i.e preserving quantum state coherence) to realize the quantum side of the interconnect. These key aspects are crucially missing and require a leap beyond the current technology to be realized.

In this context, this position paper envisions creating post-NISQ massive quantum processors through the development of scalable architectures with multiple Qcores interconnected via quantum links within the cryogenic package (see Figure \ref{fig:vision}). This is only possible by providing a quantum-coherent alternative to the rigid, static and wire-dense interconnects that are nowadays commonplace in quantum computers. Specifically, an ideal multi-Qcore interconnect fabric should not only support many simultaneous quantum state transfers across the Qcores, but also include a complementary control fabric capable of orchestrating such quantum state exchanges. If realized, this would unleash the architectural scalability potential of quantum computers by virtue of a balanced adaptive trade-off between communication and computation functions.

To realize this vision, this paper proposes an all-RF solution to the problem of building an integrated, scalable, and agile network spanning both quantum state and classical data transfers. This solution takes the form of multi-chip quantum cavity links compactly co-existing with wireless communication networks at cryogenic temperatures, both built on the solid grounds of cryogenic RF technology already developed for qubit interfacing. This approach encompasses solid experimental foundations of this interconnect fabric, but also demonstrating its paradigm-shifting architectural implications with a cross-cutting approach. In Section \ref{SecII}, this vision is further articulated, and in the subsequent sections the related multidisciplinary fields are discussed, with a final Section \ref{SecvII} devoted to a proposed design space exploration methodology across layers to assess the feasibility of this vision.

