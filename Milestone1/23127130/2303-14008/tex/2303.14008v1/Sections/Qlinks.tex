
\section{Quantum-coherent high-Q planar cavity links}
%\textcolor{red}{Equa1, U. of Siegen}

Seamless micro-integration of cryogenic quantum communication links with quantum computational functions is the key enabler for the scaling and pervasive use of quantum information technology platforms \cite{awschalom2021development}. Optical photons have widely been used for high-fidelity remote entanglement and quantum state transfer \cite{ritter2012elementary}. However, optical quantum state transfer is highly probabilistic due to inefficiencies in photon coupling and transfer \cite{Monroe_2014}, placing fundamental limits on its use for inter-Qcore communication rates. In contrast, microwave cavities and circuits can combine low loss with strong coupling and are therefore well suited for on-demand high-rate quantum transfer, and thus to scale-up quantum computational architectures in a modular fashion. To date, a limited amount of microwave quantum communication solutions has been demonstrated, concentrating mainly in quantum register coupling \cite{axline2018demand}. Switchable quantum cavity channels to directly and resonantly entangle qubits from different Qcores are proposed to be implemented in the microwave and mm-wave frequency bands, and their CMOS-compatible heterogeneous integration into multi-Qcore solutions will be modeled, technological integration route developed, experimental demonstrator built-up and performance and limitations quantified. 

The qubits will be strongly coupled to a substrate dielectric waveguide \cite{giounanlis2019photon}, allowing direct, fast and tightly coupled transfer of quantum 
information from qubits to microwave photons and vice versa. Depending on the quality factor of the quantum cavity channel, the substrate material will be a hollow waveguide or sapphire which can provide low loss ($\tan \delta \thicksim 5\cdot 10^{-7}$) and %, together with a superconducting cladding, 
quality factors exceeding $Q>10^6$. 
These features 
are essential for the efficient transmission and storage of microwave photons in the 
cavity. However, the high Q of the cavity leads to an extremely narrow-bandwidth frequency 
response, which requires tuning to and of the qubits’ resonance frequencies. This will be 
achieved by using a tunable coupler realized as a resonator with a tunable center frequency and tuning the qubit frequencies in resonance with the cavity waveguide. The planned dielectric waveguide can accommodate multiple modes, which can be used to selectively couple or decouple specific qubits in a multi-Qcore array, enabling frequency-multiplex quantum architectures. In this direction, two fundamental state-of-the-art challenges are addressed: (i) nanoscopic integration via plasmonic cavity approaches \cite{hosseininejad2017study} in order to provide a robust and monolithic integration technology with potential for tightly integrated wireless multi-Qcore communications, and (ii) low-loss, high quality-factor, low-noise coupling between neighboring Qcores providing inter-Qcore entanglement for massive quantum computation scalability. Microwave ($\thicksim$ 12 GHz) and mm-wave technologies are adressed to explore the advantages and limitations of alternative channel configurations.


