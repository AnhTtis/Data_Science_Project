\section{Communication Intranet within a Quantum Computer}
%\textcolor{red}add reference to Figure 3}

For an appropriate architectural integration of all-RF interconnects, the quantum photon and RF propagation within a quantum package needs to be characterized, and adequate communication 
protocols must be developed. On the one hand, existing works in channel modeling are mostly simulation-based studies for conventional computing chips at room temperature, at the mm-wave band and in the frequency domain only~\cite{abadal_ieeeaccess20}. In contrast, for quantum computers at cryogenic temperatures, this field is at its infancy and the only relevant work is at microwave frequencies and not oriented to communications~\cite{huang_prxquantum21}. Hence, there are no models capable of capturing the particularities of the channels within quantum computer packages and cavities, which differ substantially from those in conventional computers, and at cryogenic temperature. We propose to explore 
this uncharted territory and provide a complete channel model in the time and frequency domains. 

Since it appears to be the first time that multi-chip wireless/quantum interconnects are proposed for quantum computers, adequate protocols are missing. Protocols for quantum communications are currently limited 
to large-scale Quantum Internet proposals, which are grounded on the distribution of entangled pairs with repeaters at kilometer distances~\cite{dahlberg_sigdc19}. These assumptions do not apply to the proposed vision scenario, which is based on cavity-enabled quantum coupling at chip-scale distance and shows tight interplay with computation and high latency criticality. On the wireless transmission of classical data at the chip scale, the existing research has focused on theoretical wireless on-chip networks for conventional processors with tens of transmitters. MAC protocols are 
generally variants of token passing, seeking simplicity and performance~\cite{abadal_ieeecm18}; whereas, at the network layer, wireless links are generally considered fixed unicast connections between distant cores~\cite{mansoor_tmscs15}. None of these works has therefore considered the unique problem addressed in this position paper, namely the need for protocols to manage the access to potentially hundreds of latency-critical, computing-driven quantum links at cryogenic temperatures, with the implications that this has on the design of protocols for the non-quantum wireless links. We propose to develop a protocol stack covering both the quantum and non-quantum 
planes and capable of systematically leveraging architectural information to maximize the system performance.

\begin{comment}


\begin{figure}[t!]
	\centering
	\includegraphics[width=0.8\linewidth]{fig/Communication.png}
 	\caption{Schematic of a conventional chip package and its field distribution.}
 	\label{fig:comms}
\end{figure}

\end{comment}