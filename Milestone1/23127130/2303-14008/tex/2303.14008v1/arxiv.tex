\documentclass[conference]{IEEEtran}
\IEEEoverridecommandlockouts
% The preceding line is only needed to identify funding in the first footnote. If that is unneeded, please comment it out.
\usepackage{cite}
\usepackage{amsmath,amssymb,amsfonts}
\usepackage{algorithmic}
\usepackage{graphicx}
\usepackage{textcomp}
\usepackage{xcolor}
\usepackage{url}
\usepackage{tabularx}
\usepackage{subfig}
\usepackage{comment}


\def\tablename{Table}
\usepackage{etoolbox}
\makeatletter
\patchcmd{\@makecaption}
  {\scshape}
  {}
  {}
  {}
\makeatletter
\patchcmd{\@makecaption}
  {\\}
  {.\ }
  {}
  {}
\makeatother
\def\tablename{Table}
\def\BibTeX{{\rm B\kern-.05em{\sc i\kern-.025em b}\kern-.08em
    T\kern-.1667em\lower.7ex\hbox{E}\kern-.125emX}}
\begin{document}

\title{Scalable multi-chip quantum architectures enabled by 
cryogenic hybrid wireless/quantum-coherent network-in-package\\
%{\footnotesize \textsuperscript{*}Note: Sub-titles are not captured in Xplore and
%should not be used}
%\thanks{Identify applicable funding agency here. If none, delete this.}
}

%\title{Scalable multi-chip quantum computing architectures \\
%{\footnotesize \textsuperscript{*}Note: Sub-titles are not captured in Xplore and
%should not be used}
%\thanks{Identify applicable funding agency here. If none, delete this.}
%}
 
\author{\IEEEauthorblockN{Eduard Alarc\'on\IEEEauthorrefmark{1},
 Sergi Abadal\IEEEauthorrefmark{1}, Fabio Sebastiano\IEEEauthorrefmark{2}, Massoud Babaie\IEEEauthorrefmark{2}, Eduardo Charbon\IEEEauthorrefmark{3}, Peter Haring Bol\'ivar\IEEEauthorrefmark{4}, \\
 Maurizio Palesi\IEEEauthorrefmark{5}, Elena Blokhina\IEEEauthorrefmark{6}, Dirk Leipold\IEEEauthorrefmark{6}, Bogdan Staszewski\IEEEauthorrefmark{7}, Artur Garcia-S\'aez\IEEEauthorrefmark{8} and Carmen G. Almudever\IEEEauthorrefmark{9}}
 
 
\IEEEauthorblockA{\IEEEauthorrefmark{1}\textit{Technical University of Catalunya, BarcelonaTech, Spain}}
\IEEEauthorblockA{\IEEEauthorrefmark{2}\textit{Delft University of Technology, The Netherlands}}
\IEEEauthorblockA{\IEEEauthorrefmark{3}\textit{École polytechnique fédérale de Lausanne, Switzerland}}
\IEEEauthorblockA{\IEEEauthorrefmark{4}\textit{University of Siegen, Germany}}
\IEEEauthorblockA{\IEEEauthorrefmark{5}\textit{University of Catania, Italy}}
\IEEEauthorblockA{\IEEEauthorrefmark{6}\textit{Equal 1, Ireland}}
\IEEEauthorblockA{\IEEEauthorrefmark{7}\textit{University College Dublin, Ireland}}
\IEEEauthorblockA{\IEEEauthorrefmark{8}\textit{Barcelona Supercomputing Center, Spain}}
\IEEEauthorblockA{\IEEEauthorrefmark{9}\textit{Technical University of Valencia, Spain}}}


\begin{comment}
\author{\IEEEauthorblockN{1\textsuperscript{st} Given Name Surname}
\IEEEauthorblockA{\textit{dept. name of organization (of Aff.)} \\
\textit{name of organization (of Aff.)}\\
City, Country \\
email address or ORCID}
\and
\IEEEauthorblockN{2\textsuperscript{nd} Given Name Surname}
\IEEEauthorblockA{\textit{dept. name of organization (of Aff.)} \\
\textit{name of organization (of Aff.)}\\
City, Country \\
email address or ORCID}
\and
\IEEEauthorblockN{3\textsuperscript{rd} Given Name Surname}
\IEEEauthorblockA{\textit{dept. name of organization (of Aff.)} \\
\textit{name of organization (of Aff.)}\\
City, Country \\
email address or ORCID}
\and
\IEEEauthorblockN{4\textsuperscript{th} Given Name Surname}
\IEEEauthorblockA{\textit{dept. name of organization (of Aff.)} \\
\textit{name of organization (of Aff.)}\\
City, Country \\
email address or ORCID}
\and
\IEEEauthorblockN{5\textsuperscript{th} Given Name Surname}
\IEEEauthorblockA{\textit{dept. name of organization (of Aff.)} \\
\textit{name of organization (of Aff.)}\\
City, Country \\
email address or ORCID}
\and
\IEEEauthorblockN{6\textsuperscript{th} Given Name Surname}
\IEEEauthorblockA{\textit{dept. name of organization (of Aff.)} \\
\textit{name of organization (of Aff.)}\\
City, Country \\
email address or ORCID}
}

\end{comment}



\maketitle

\begin{abstract}

The grand challenge of scaling up quantum computers requires a full-stack architectural standpoint. In this position paper, we will present the vision of a new generation of scalable quantum computing architectures featuring distributed quantum cores (Qcores) interconnected via quantum-coherent qubit state transfer links and orchestrated via an integrated wireless interconnect.


\end{abstract}

\begin{IEEEkeywords}

Scalability quantum computing systems, full-stack architecture design.
\end{IEEEkeywords}




% Importance and appeal of children's drawings
Children's depictions of the human figure are highly expressive and varied.
As one of the very first subjects children attempt to draw, the representation begins as an almost unintelligible cloud of scribbles. 
As the child grows, their representation of the human figure becomes more developed and is extended to graphically represent many different types of characters: people, animals, and even personified objects (see Figure 1).

Who among us has not wished, either as a child or as an adult, to see such figures come to life and move around on the page?
Sadly, while it is relatively fast to produce a single drawing, creating the sequence of images necessary for animation is a much more tedious endeavor, requiring discipline, skill, patience, and sometimes complicated software.
As a result, most of these figures remain static upon the page.

% We built a system to animate them.
Inspired by the importance and appeal of the drawn human figure, we design and build a system to automatically animate it given an in-the-wild photograph of a child's drawing. 
Our system is fast, intuitive, and robust to much of the variation present in these types of drawings, making it well-suited to allow our target audience--children--to see their own characters coming to life.
The system is comprised of four stages: figure detection, segmentation masking, pose estimation/rigging, and animation. 
We describe each stage and identify common causes of failure in each. 
For object detection and pose estimation, we make use of existing computer vision models designed to detect human figures and joints in photographs; we fine-tune these models for use with children's drawings.
For segmentation, we present a straightforward, image processing-based method that, for animation purposes, is more useful and accurate than segmentation masks obtained from a fine-tuned object detection model.
During the animation step, we take advantage of the \textit{twisted perspective} commonly seen in children’s drawings to retarget motion capture data onto the character in a novel and appealing way.

% We use existing machine learning models. However, given the wide domain gap it's not clear how much fine-tuning data was needed. So we ran some experiments to find out and report it.
While our system leverages existing models and techniques, most are not directly applicable to the task due to the many differences between photographic images and simple pen and paper representations. 
To this end, we couple the presentation of our system with a set of experiments exploring the relationship between fine-tuning training set size and success rates.
We also include a perceptual study validating viewer preference for incorporating \textit{twisted perspective} into the motion retargeting step.

We validate the desirability and appeal of our system by building and publicly releasing a version of it as the \AD Demo \,\cite{animateddrawings}.
Launched in December 2021, this demo has been used by millions of people around the world to animate their children's drawings.
Inspired by this reception, our second contribution is The Amateur Drawings Dataset: \hjs{180,000 drawings and user-accepted annotations collected, with consent, through the demo. See Section \ref{sec:UI} for a description of how the annotations were generated.}
We believe this dataset will be a resource to researchers from various fields seeking to better understand the space of amateur drawings, evaluate new algorithms in this domain, or develop new drawing-based tools in general.

To summarize, our contributions are as follows:
\begin{enumerate}
    \item 
    We explore the problem of automatic sketch-to-animation for children's drawings of human figures and present a framework that achieves this effect. We also present a set of experiments determining the amount of training data necessary to achieve high levels of success and a perceptual study validating the usefulness of our motion retargeting technique.
    \item To encourage additional research in the domain of amateur drawings, we present a first-of-its-kind dataset of 180,000 user-submitted amateur drawings, along with user-accepted bounding box, segmentation mask, and joint location annotations.
\end{enumerate}

Upon acceptance of this paper, we plan to publicly release the Amateur Drawings Dataset, project code, and fine-tuned model weights.

\section{Multi-core quantum computing architectures enabled by a hybrid interconnection network}

\label{SecII}

%\textcolor{red}{UPV, state-of-the-art on multi-core quantum computers}


One of the main challenges for scaling up quantum computers is the wiring between the quantum processor and control electronics \cite{xue2021cmos}. Existing quantum computers physically span several temperature levels, with the qubits typically operating at sub-1-K temperatures to ensure their coherence, while the classical control electronics are operating at room temperature. While this thermal gap can be readily bridged by a few wires for the small quantum processors available today ($<$100 qubits), it would be impractical to wire thousands or millions of qubits in future quantum computers due to wiring size and reliability issues. Recent advances are closing this vertical gap by pushing the qubits to operate at temperatures near 4K \cite{petit2020universal}, where cryogenic refrigerators can dissipate the heat generated by the classical control electronics, and by developing cryogenic analog/RF control circuits that can be potentially co-integrated with the qubits at the same temperature level \cite{xue2021cmos}. Additionally, recent research advocates for also moving digital circuits to cryogenic temperatures \cite{schriek2020cryo}, an approach that further alleviates the wiring bottleneck. 

%as it avoids the need to have a room-temperature host computer mapping the algorithm to the quantum processor and controlling its execution. 

Although the full-cryogenic computer resulting from closing the vertical gap would be more compact than today’s prototypes, scaling issues remain in the horizontal dimension. Specifically, scaling this integrated scheme to a very large number of qubits as a single array (Qcore) is deemed unfeasible due to the practical bottleneck in the wiring between the qubits and their interface electronics \cite{vandersypen2017interfacing}. Moreover, other issues arise when integrating a high amount of qubits on a single chip, such as crosstalk, limited yield and qubit addressability. Instead, the multi-Qcore vision proposed here assumes $N_c$ quantum chips, with each chip hosting a Qcore composed of a moderately sized array of $N_q$ qubits. Thus, the Qcores can be spaced to fit the local electronics next to each core, thereby improving qubit addressability and reliability. Connecting the Qcores via quantum links allows the ensemble to remain coherent, thus maintaining the computational power of $N_c \times N_q$ entangled qubits, i.e. $O(2^{N_cN_q})$, rather than the incoherent addition of the Qcores, $O(N_c2^{N_q})$. The realization and integration of such quantum links at the chip scale is a fundamental challenge that has not been addressed yet.

\begin{comment}
\begin{figure}[t!]
	\centering
	\includegraphics[width=\linewidth]{fig/Qchannel.png}
 	\caption{Communication among quantum cores through a quantum cavity channel and a classical mm-wave channel.}
 	\label{fig:Qlink}
\end{figure}
\end{comment}

%The first key element to enable the vision is an expansive quantum cavity channel as enabler of the quantum-coherent interconnect fabric. By this approach, the chips containing the Qcores are connected to the cavity by means of a flip-chip assembly configuration, as shown in Fig.~\ref{fig:Qlink}. The quantum channel proposed here, instead of using optical photon-based entangled pairs that are prevalent in large-scale systems, operates through entangled microwave in a substrate dielectric waveguide \cite{kannan2020generating}. The underlying principle is that qubits, when excited at a proper frequency, can serve as high-quality quantum emitters \cite{giounanlis2019photon}. The silicon qubits are strongly coupled to the cavity waveguide such that their quantum information can be transmitted as standing-wave photons through the waveguide. These microwave photons emitted from multiple sources into the waveguide will interact with each other. If properly controlled and synchronized, the quantum interference between the photons can generate spatially entangled traveling photons. This enables high quality communications featuring high entanglement rates between the quantum cores. %In the proposed approach, we will demonstrate this quantum link using quantum-dot (QD)-based spin/position hybrid qubits, showing that (i) an absorbed microwave photon can facilitate tunneling of electrons between QDs by means of pumping energy into the system, and (ii) the reverse process is possible as well: a microwave photon emission due to the tunneling of particles in a quantum transport operation where double QDs are coupled through a cavity.


The first key element to enable the vision is an expansive quantum channel as enabler of the quantum interconnect fabric. The other key element in the vision (in hybrid co-existence) is a multi-chip interconnection network that, with the exchange of classical data, coordinates the emitter and receiver of the quantum state transfer, and  enables the sharing of control signals and data across the Qcores. Without such an interconnection network, all data would have to go through the vertical connections to the host computer and back, which quickly becomes a roadblock. One could implement the inter-Qcore network through wired interconnects, yet this approach has drawbacks such as the difficulty of routing physical wires through the already densely wired system, or the high latency of system-wide transfers, which are critical for control. In this position paper, instead, we propose to realize the classical side of communications by means of a compact, high-bandwidth, highly reconfigurable wireless in-package network. The main idea is to integrate on-chip cryogenic antennas with RF cryogenic transceivers (developed with the same technology and techniques already employed in existing high-frequency qubit control circuits) so as to implement a wireless network within the quantum computing package. The resulting network is naturally system-wide and broadcast, allowing for agile reconfiguration of the underlying architecture. Moreover, although designing integrated circuits beyond their originally intended temperature operating range presents several challenges, the cryogenic environment is expected to opportunistically boost the antenna efficiency, reduce the thermal noise, and improve the transistor speed \cite{xue2021cmos,8036394}.



\section{Quantum-coherent high-Q planar cavity links}
%\textcolor{red}{Equa1, U. of Siegen}

Seamless micro-integration of cryogenic quantum communication links with quantum computational functions is the key enabler for the scaling and pervasive use of quantum information technology platforms \cite{awschalom2021development}. Optical photons have widely been used for high-fidelity remote entanglement and quantum state transfer \cite{ritter2012elementary}. However, optical quantum state transfer is highly probabilistic due to inefficiencies in photon coupling and transfer \cite{Monroe_2014}, placing fundamental limits on its use for inter-Qcore communication rates. In contrast, microwave cavities and circuits can combine low loss with strong coupling and are therefore well suited for on-demand high-rate quantum transfer, and thus to scale-up quantum computational architectures in a modular fashion. To date, a limited amount of microwave quantum communication solutions has been demonstrated, concentrating mainly in quantum register coupling \cite{axline2018demand}. Switchable quantum cavity channels to directly and resonantly entangle qubits from different Qcores are proposed to be implemented in the microwave and mm-wave frequency bands, and their CMOS-compatible heterogeneous integration into multi-Qcore solutions will be modeled, technological integration route developed, experimental demonstrator built-up and performance and limitations quantified. 

The qubits will be strongly coupled to a substrate dielectric waveguide \cite{giounanlis2019photon}, allowing direct, fast and tightly coupled transfer of quantum 
information from qubits to microwave photons and vice versa. Depending on the quality factor of the quantum cavity channel, the substrate material will be a hollow waveguide or sapphire which can provide low loss ($\tan \delta \thicksim 5\cdot 10^{-7}$) and %, together with a superconducting cladding, 
quality factors exceeding $Q>10^6$. 
These features 
are essential for the efficient transmission and storage of microwave photons in the 
cavity. However, the high Q of the cavity leads to an extremely narrow-bandwidth frequency 
response, which requires tuning to and of the qubits’ resonance frequencies. This will be 
achieved by using a tunable coupler realized as a resonator with a tunable center frequency and tuning the qubit frequencies in resonance with the cavity waveguide. The planned dielectric waveguide can accommodate multiple modes, which can be used to selectively couple or decouple specific qubits in a multi-Qcore array, enabling frequency-multiplex quantum architectures. In this direction, two fundamental state-of-the-art challenges are addressed: (i) nanoscopic integration via plasmonic cavity approaches \cite{hosseininejad2017study} in order to provide a robust and monolithic integration technology with potential for tightly integrated wireless multi-Qcore communications, and (ii) low-loss, high quality-factor, low-noise coupling between neighboring Qcores providing inter-Qcore entanglement for massive quantum computation scalability. Microwave ($\thicksim$ 12 GHz) and mm-wave technologies are adressed to explore the advantages and limitations of alternative channel configurations.




\section{Cryo-CMOS RF transceivers}
%\textcolor{red}{EPFL, TU Delft}

Nanometer-scale CMOS electronics operating at cryogenic temperatures (cryo-CMOS) is widely accepted as the preferred IC technology for quantum-processor interfaces, thanks to its capability to operate down to (at least) 30 
mK, its unmatched very large scale of integration (VLSI), and the mature design automation infrastructure, all necessary to handle millions of qubits \cite{8036394, 8329135}. Several cryo-CMOS individual circuit blocks have been demonstrated, as required for qubit control and co-integration. A broad range of cryo-CMOS 
interface circuits for the control and readout of qubits has been recently shown, including complex microwave transmitters operating at 4 K %(Fig.~\ref{fig:Horse}) 
providing $<$100 MHz data bandwidth/qubit for $<$-16 dBm output power 
at a carrier frequency $<$20 GHz \cite{xue2021cmos, 9209175}, RF receivers \cite{prabowo202113} and high-speed baseband data converters \cite{9365927}. Nevertheless, the lowest reported operating temperature for a wireless transceiver is 170 K \cite{4384375}, and no deep-cryogenic transceiver has been demonstrated. 

In this work, it is proposed to implement the first-ever deep-cryogenic wireless transceiver, operating down to 4 K. To miniaturize the on-chip antenna, the operating frequency is expected to increase $>3\times$ beyond the state-of-the-art for cryo-CMOS, while 
the data bandwidth and the transmitter’s output power will both be extended by $>20\times$
beyond the state of the art to realize a robust 
communication between the qubit tiles. We will explore using the waveguide proposed in section II for classical data transmission at 60 GHz by exciting the waveguide with an on-chip antenna and by minimizing the crosstalk to the quantum  channels. 

\begin{comment}
\begin{figure}[t!]
	\centering
		\includegraphics[width=0.5\linewidth]{fig/Horse.png}
 	\caption{Horse Ridge cryo-CMOS SoC with 4x 22-nm FinFET microwave drivers for 128 qubits.}
 	\label{fig:Horse}
\end{figure}

\end{comment}
\section{Communication Intranet within a Quantum Computer}
%\textcolor{red}add reference to Figure 3}

For an appropriate architectural integration of all-RF interconnects, the quantum photon and RF propagation within a quantum package needs to be characterized, and adequate communication 
protocols must be developed. On the one hand, existing works in channel modeling are mostly simulation-based studies for conventional computing chips at room temperature, at the mm-wave band and in the frequency domain only~\cite{abadal_ieeeaccess20}. In contrast, for quantum computers at cryogenic temperatures, this field is at its infancy and the only relevant work is at microwave frequencies and not oriented to communications~\cite{huang_prxquantum21}. Hence, there are no models capable of capturing the particularities of the channels within quantum computer packages and cavities, which differ substantially from those in conventional computers, and at cryogenic temperature. We propose to explore 
this uncharted territory and provide a complete channel model in the time and frequency domains. 

Since it appears to be the first time that multi-chip wireless/quantum interconnects are proposed for quantum computers, adequate protocols are missing. Protocols for quantum communications are currently limited 
to large-scale Quantum Internet proposals, which are grounded on the distribution of entangled pairs with repeaters at kilometer distances~\cite{dahlberg_sigdc19}. These assumptions do not apply to the proposed vision scenario, which is based on cavity-enabled quantum coupling at chip-scale distance and shows tight interplay with computation and high latency criticality. On the wireless transmission of classical data at the chip scale, the existing research has focused on theoretical wireless on-chip networks for conventional processors with tens of transmitters. MAC protocols are 
generally variants of token passing, seeking simplicity and performance~\cite{abadal_ieeecm18}; whereas, at the network layer, wireless links are generally considered fixed unicast connections between distant cores~\cite{mansoor_tmscs15}. None of these works has therefore considered the unique problem addressed in this position paper, namely the need for protocols to manage the access to potentially hundreds of latency-critical, computing-driven quantum links at cryogenic temperatures, with the implications that this has on the design of protocols for the non-quantum wireless links. We propose to develop a protocol stack covering both the quantum and non-quantum 
planes and capable of systematically leveraging architectural information to maximize the system performance.

\begin{comment}


\begin{figure}[t!]
	\centering
	\includegraphics[width=0.8\linewidth]{fig/Communication.png}
 	\caption{Schematic of a conventional chip package and its field distribution.}
 	\label{fig:comms}
\end{figure}

\end{comment}
\section{Architecting Scalable and Reconfigurable Quantum Processors}
%\textcolor{red}{BSC, UPV}
NISQ devices are publicly accessible (e.g., IBM Q experience, Qinspire) and users can already run small instances of quantum algorithms. However, these quantum processors suffer from several constraints, such as limited connectivity 
among the qubits. This requires the quantum compiler to modify the quantum algorithm, described as a quantum circuit, to realize it on a given quantum chip. 
This transformation process is known as mapping and might compromise the successful execution of the algorithm \cite{almudever2020realizing}. 

Exact approaches for small-size 
quantum circuits and approximate mapping solutions using heuristics for larger quantum circuits have been developed for specific single-core NISQ 
devices \cite{lao2021timing,murali2019full}. Recently, the first compiler techniques for mapping and scheduling quantum algorithms onto connectivity-simplified multi-core quantum 
architectures have been proposed \cite{baker2020time,rodrigo2021double} as distributed or modular architectural approaches are becoming more prominent for scaling-up quantum hardware \cite{vandersypen2017interfacing, chow2021quantum, laracuente2022short,brown2016co}. These 
mapping solutions all focus on multi-Qcore architectures that assume all-to-all intra- and inter-core connectivity with unlimited and ideal communication resources and a fixed and invariant interconnection network. In this work it is proposed to
build on top of these quantum hardware advances and define scalable multi-Qcore architectures, including the communication perspective and the required compilation techniques, connecting today's experiments with future practical implementations. We will also perform on-line dynamic optimizations of the Qcores transfers for an efficient execution of the algorithms leveraging the reconfigurability provided by the wireless control plane.


\begin{comment}
\begin{figure}[t!]
	\centering
	\includegraphics[width=0.8\linewidth]{fig/Mapping.png}
 	\caption{Mapping a quantum algorithm based on its qubit interaction graph into a multi-Qcore architecture.}
 	\label{fig:comms}
\end{figure}
\end{comment}
\section{Design Space Exploration for Architecture/Network/Circuit/link Co-Design}

\label{SecvII}

Although quantum computers already exist in the form of intermediate-scale quantum processors, there is no consensus in the quantum computing community on what benchmarks and metrics to use to compare them and to assess their performance. As a first attempt, IBM proposed Quantum Volume (QV) and Circuit Layer Operations Per Second (CLOPS) as metrics to measure three key quantum computing performance attributes: quality, speed and scale \cite{wack2021quality}. Also, different sets of quantum benchmarks for assessing their overall performance have been introduced \cite{blume2020volumetric, mills2021application}. In addition, the higher layers of full-stack systems, consisting of both quantum software and classical control hardware, have been so far designed and provided solutions following a bottom-up approach; that is, they have been developed for a specific quantum device. This approach is appropriate for a single rigid design, which could be functional, but it is unclear how close it is to the optimal design in terms of overall performance and scalability, and how such performance holds in different specifications. This is particularly critical in the harsh constraints of quantum computing. In addition, a long-standing perceived need from the quantum community identifies the lack of design guidelines from top architectural layers down to physical ones or even cross-layer co-design \cite{almudever2021structured, tomesh2021quantum}. 

In this work, we propose to employ structured Design Space Exploration (DSE) methodologies to perform a cross-layer co-design of the full-stack quantum system, including both communication and computation, allowing for both top-down and bottom-up optimizations across layers (see Figure \ref{fig:DSE}). To this purpose, models will be developed that describe (i) the computational part, from the distributed architecture down to the single Qcore and qubit impairments, and (ii) the classical and quantum communication parts from the perspective of performance (e.g. throughput, fidelity) and efficiency (e.g. power). Finally, we will create and profile a set of large-scale benchmarks able to stress the multi-Qcore platform and define different performance metrics of the overall architecture. 


\begin{figure}[t!]
	\centering
	\includegraphics[width=0.8\linewidth]{fig/DSE.png}
 	\caption{Exploration framework for quantum computing architectures.}
 	\label{fig:DSE}
\end{figure}












\section{Conclusions}
%\section{}
%\label{sec:resDir}


\section{Conclusion}
\label{sec:conclusion}
% <>
Since its advent in 1931, Koopman operator theory \cite{koopman:1931} has only recently been actively utilized for solving practical problems, thanks to the introduction of the DMD algorithm in 2008 \cite{schmid:2008}. Since then, a multitude of DMD algorithm variations have risen to prominence and found utility across various fields. A notable feature of our survey paper was reviewing and categorizing the results of over 100 research papers based on both application and algorithm type in smart mobility and vehicle engineering  (see Table~\ref{tab1} and Section~\ref{sec:vehicApp}).  Additionally, this survey paper identified potential research gaps in smart mobility and vehicular engineering applications (Remarks~\ref{remGap1}--\ref{remGap6}). Finally, this review paper discussed theoretical aspects of Koopman operator theory that have been largely neglected by the smart mobility and vehicle engineering community and yet have large potential for contributing to solving open problems in these areas (see Section~\ref{subsec:theorIssue}).

\noindent{\textbf{Future Research Directions.}}	Given the emergence of cyber-threats against connected and autonomous vehicles as well as robotic systems (see, e.g.,~\cite{nekouei2021randomized,mohammadi2022generation}), a future research direction might include utilizing Koopman operator-based algorithms for designing cyber-resilient vehicular and smart mobility applications (see, e.g.,~\cite{taheri2022data} for a related line of research). Another potential research direction is using Koopman operator-based algorithms for predicting the motion of vulnerable road users (VRUs), e.g., pedestrians and cyclists (see, e.g.,~\cite{pool2019context,scholler2020constant}). Finally, rehabilitation robotics and robotic exoskeletons can be the benefactors of the predictive capabilities of Koopman operator-based algorithms for detecting tripping events and/or system  identification in various modes of locomotion (see, e.g.,~\cite{kumar2019extremum,aprigliano2019pre}).



%Fig. 1 depicts the accumulation of such algorithms since 2014, which are particular to vehicle engineering and smart mobility, i.e., the focus of this review. Table 1 summarizes the varieties of relevant algorithms developed in those studies. Furthermore, we have highlighted theoretical issues, whose expansion will have potential applications to the wide research area of smart mobility and vehicle engineering.  

%Although fairly comprehensive, we have found several gaps in this research area. In particular, we could not find any studies related to elevators, robots/vehicles employing crawling, slithering, hopping or peristaltic locomotion, arctic or special-terrain vehicles such as those employing screws or tracks, hovercraft and other amphibious vehicles or subsystems which tolerate flexible environments, classification or guidance systems related to vehicles for drilling or agriculture, or for current-ripple, power-split, battery health monitoring, nuclear propulsion, exoskeletons/prosthetics, personal mobility, motorsports, specialized rovers or similar open problems in emerging areas.  These examples are, of course, not exhaustive.  
%
%The purely data-driven nature of Koopman operators holds the promise of capturing unknown and complex dynamics for reduced-order model generation and system identification, through which the rich machinery of linear control techniques can be utilized. The emergent nature of the smart mobility and vehicular-related applications, where  the Koopman operator  in each particular application needs to be approximated, implies that the development of various Koopman operator approximation  algorithms is expected to grow along with the vehicular problems they aim to solve.  Given the ongoing development of this research area and the many existing open problems in the fields of smart mobility and vehicle engineering, a survey of techniques and open challenges of applying Koopman operator theory to this vibrant area is warranted.  To the best of our knowledge, this survey paper is the \emph{first of its kind} reviewing the applications of Koopman operator theory within a focused research area, namely, smart mobility and vehicle engineering applications. A \emph{notable feature} of our survey paper is reviewing and categorizing the results of over 100 research papers based on both application and algorithm type  (see Tables~\ref{tab1}--~\ref{tab4} and Section~\ref{sec:vehicApp}) that are concerned with the applications of Koopman operator theory to the field of smart mobility and vehicular engineering. Such a \emph{comprehensive and  detailed categorization} will be beneficial to the research practitioners working in the field.  Furthermore, this review paper discusses theoretical aspects of Koopman operator theory that have been largely neglected by the smart mobility and vehicle engineering community and yet have large potential for contributing to solving open problems in these areas. Additionally, our survey paper seeks to \emph{identify gaps} in the smart mobility and vehicle engineering research where new and existing Koopman operator-based methods have the potential to further develop and address unsolved problems  potentially benefiting from the perspectives of nonlinear system identification, control, global linearization, and the predictive powers that Koopman operator theory has to offer (see, e.g., Remarks~\ref{remGap1}--\ref{remGap6}). 

This position paper addresses from a full-stack architectural perspective the challenge of scaling up quantum computers.  A vision is presented of a new generation of scalable quantum computing architectures in which distributed quantum cores consider quantum-coherent qubit state transfer links in the data plane to interconnect, and a wireless medium orchestrator. The scientific and technology needs at each architectural layer are critically discussed, and a cross-layer design-space exploration framework is proposed to assess the feasibility of the proposed architectures.

\section*{Acknowledgments}
All authors acknowledge support from the EU, grant HORIZON-ERC-101042080 and grant HORIZON-EIC-2022-PATHFINDEROPEN-01-101099697, QUADRATURE.


\begin{thebibliography}{10}

\bibitem{arute2019quantum}
F.~Arute, K.~Arya, R.~Babbush, D.~Bacon, J.~C. Bardin, R.~Barends, R.~Biswas,
  S.~Boixo, F.~G. Brandao, D.~A. Buell, {\em et~al.}, ``Quantum supremacy using
  a programmable superconducting processor,'' {\em Nature}, vol.~574, no.~7779,
  pp.~505--510, 2019.

\bibitem{Preskill_2018}
J.~Preskill, ``Quantum computing in the {NISQ} era and beyond,'' {\em Quantum},
  vol.~2, p.~79, aug 2018.

\bibitem{staszewski2021cryo}
R.~B. Staszewski, I.~Bashir, E.~Blokhina, and D.~Leipold, ``{Cryo-CMOS} for
  quantum system on-chip integration: Quantum computing as the development
  driver,'' {\em IEEE Solid-State Circuits Magazine}, vol.~13, no.~2,
  pp.~46--53, 2021.

\bibitem{vandersypen2017interfacing}
L.~Vandersypen, H.~Bluhm, J.~Clarke, A.~Dzurak, R.~Ishihara, A.~Morello,
  D.~Reilly, L.~Schreiber, and M.~Veldhorst, ``Interfacing spin qubits in
  quantum dots and donors—hot, dense, and coherent,'' {\em npj Quantum
  Information}, vol.~3, no.~1, pp.~1--10, 2017.

\bibitem{chow2021quantum}
J.~M. Chow, ``Quantum intranet,'' {\em IET Quantum Communication}, vol.~2,
  no.~1, pp.~26--27, 2021.

\bibitem{laracuente2022short}
N.~LaRacuente, K.~N. Smith, P.~Imany, K.~L. Silverman, and F.~T. Chong,
  ``Short-range microwave networks to scale superconducting quantum
  computation,'' {\em arXiv preprint arXiv:2201.08825}, 2022.

\bibitem{xue2021cmos}
X.~Xue, B.~Patra, J.~P. van Dijk, N.~Samkharadze, S.~Subramanian, A.~Corna,
  B.~Paquelet~Wuetz, C.~Jeon, F.~Sheikh, E.~Juarez-Hernandez, {\em et~al.},
  ``{CMOS-based} cryogenic control of silicon quantum circuits,'' {\em Nature},
  vol.~593, no.~7858, pp.~205--210, 2021.

\bibitem{petit2020universal}
L.~Petit, H.~Eenink, M.~Russ, W.~Lawrie, N.~Hendrickx, S.~Philips, J.~Clarke,
  L.~Vandersypen, and M.~Veldhorst, ``Universal quantum logic in hot silicon
  qubits,'' {\em Nature}, vol.~580, no.~7803, pp.~355--359, 2020.

\bibitem{schriek2020cryo}
E.~Schriek, F.~Sebastiano, and E.~Charbon, ``A {cryo-CMOS} digital cell library
  for quantum computing applications,'' {\em IEEE Solid-State Circuits
  Letters}, vol.~3, pp.~310--313, 2020.

\bibitem{8036394}
B.~Patra, R.~M. Incandela, J.~P.~G. van Dijk, H.~A.~R. Homulle, L.~Song,
  M.~Shahmohammadi, R.~B. Staszewski, A.~Vladimirescu, M.~Babaie,
  F.~Sebastiano, and E.~Charbon, ``{Cryo-CMOS} circuits and systems for quantum
  computing applications,'' {\em IEEE Journal of Solid-State Circuits},
  vol.~53, no.~1, pp.~309--321, 2018.

\bibitem{awschalom2021development}
D.~Awschalom, K.~K. Berggren, H.~Bernien, S.~Bhave, L.~D. Carr, P.~Davids,
  S.~E. Economou, D.~Englund, A.~Faraon, M.~Fejer, {\em et~al.}, ``Development
  of quantum interconnects (quics) for next-generation information
  technologies,'' {\em PRX Quantum}, vol.~2, no.~1, p.~017002, 2021.

\bibitem{ritter2012elementary}
S.~Ritter, C.~N{\"o}lleke, C.~Hahn, A.~Reiserer, A.~Neuzner, M.~Uphoff,
  M.~M{\"u}cke, E.~Figueroa, J.~Bochmann, and G.~Rempe, ``An elementary quantum
  network of single atoms in optical cavities,'' {\em Nature}, vol.~484,
  no.~7393, pp.~195--200, 2012.

\bibitem{Monroe_2014}
C.~Monroe, R.~Raussendorf, A.~Ruthven, K.~R. Brown, P.~Maunz, L.-M. Duan, and
  J.~Kim, ``Large-scale modular quantum-computer architecture with atomic
  memory and photonic interconnects,'' {\em Physical Review A}, vol.~89, feb
  2014.

\bibitem{axline2018demand}
C.~J. Axline, L.~D. Burkhart, W.~Pfaff, M.~Zhang, K.~Chou, P.~Campagne-Ibarcq,
  P.~Reinhold, L.~Frunzio, S.~Girvin, L.~Jiang, {\em et~al.}, ``On-demand
  quantum state transfer and entanglement between remote microwave cavity
  memories,'' {\em Nature Physics}, vol.~14, no.~7, pp.~705--710, 2018.

\bibitem{giounanlis2019photon}
P.~Giounanlis, E.~Blokhina, D.~Leipold, and R.~B. Staszewski, ``Photon enhanced
  interaction and entanglement in semiconductor position-based qubits,'' {\em
  Applied Sciences}, vol.~9, no.~21, p.~4534, 2019.

\bibitem{hosseininejad2017study}
S.~E. Hosseininejad, E.~Alarcon, N.~Komjani, S.~Abadal, M.~C. Lemme, P.~H.
  Bol{\'\i}var, and A.~Cabellos-Aparicio, ``Study of hybrid and pure plasmonic
  terahertz antennas based on graphene guided-wave structures,'' {\em Nano
  communication networks}, vol.~12, pp.~34--42, 2017.

\bibitem{8329135}
R.~M. Incandela, L.~Song, H.~Homulle, E.~Charbon, A.~Vladimirescu, and
  F.~Sebastiano, ``Characterization and compact modeling of nanometer {CMOS}
  transistors at deep-cryogenic temperatures,'' {\em IEEE Journal of the
  Electron Devices Society}, vol.~6, pp.~996--1006, 2018.

\bibitem{9209175}
J.~P.~G. Van~Dijk, B.~Patra, S.~Subramanian, X.~Xue, N.~Samkharadze, A.~Corna,
  C.~Jeon, F.~Sheikh, E.~Juarez-Hernandez, B.~P. Esparza, H.~Rampurawala, B.~R.
  Carlton, S.~Ravikumar, C.~Nieva, S.~Kim, H.-J. Lee, A.~Sammak, G.~Scappucci,
  M.~Veldhorst, L.~M.~K. Vandersypen, E.~Charbon, S.~Pellerano, M.~Babaie, and
  F.~Sebastiano, ``A scalable {Cryo-CMOS} controller for the wideband
  frequency-multiplexed control of spin qubits and transmons,'' {\em IEEE
  Journal of Solid-State Circuits}, vol.~55, no.~11, pp.~2930--2946, 2020.

\bibitem{prabowo202113}
B.~Prabowo, G.~Zheng, M.~Mehrpoo, B.~Patra, P.~Harvey-Collard, J.~Dijkema,
  A.~Sammak, G.~Scappucci, E.~Charbon, F.~Sebastiano, {\em et~al.}, ``A
  {6-to-8GHz} 0.17 mw/qubit {cryo-CMOS} receiver for multiple spin qubit
  readout in 40nm {CMOS} technology,'' in {\em 2021 IEEE International
  Solid-State Circuits Conference (ISSCC)}, vol.~64, pp.~212--214, IEEE, 2021.

\bibitem{9365927}
G.~Kiene, A.~Catania, R.~Overwater, P.~Bruschi, E.~Charbon, M.~Babaie, and
  F.~Sebastiano, ``A {1GS/s} 6-to-8b 0.5mw/qubit {Cryo-CMOS SAR ADC} for
  quantum computing in 40nm {CMOS},'' in {\em 2021 IEEE International Solid-
  State Circuits Conference (ISSCC)}, vol.~64, pp.~214--216, 2021.

\bibitem{4384375}
W.~Kuhn, N.~E. Lay, E.~Grigorian, D.~Nobbe, I.~Kuperman, J.~Jeon, K.~Wong,
  Y.~Tugnawat, and X.~He, ``A microtransceiver for {UHF} proximity links
  including mars surface-to-orbit applications,'' {\em Proceedings of the
  IEEE}, vol.~95, no.~10, pp.~2019--2044, 2007.

\bibitem{abadal_ieeeaccess20}
S.~Abadal, C.~Han, and J.~M. Jornet, ``Wave propagation and channel modeling in
  chip-scale wireless communications: A survey from millimeter-wave to
  terahertz and optics,'' {\em IEEE Access}, vol.~8, pp.~278--293, 2020.

\bibitem{huang_prxquantum21}
S.~Huang, B.~Lienhard, G.~Calusine, A.~Veps\"al\"ainen, J.~Braum\"uller, D.~K.
  Kim, A.~J. Melville, B.~M. Niedzielski, J.~L. Yoder, B.~Kannan, T.~P.
  Orlando, S.~Gustavsson, and W.~D. Oliver, ``Microwave package design for
  superconducting quantum processors,'' {\em PRX Quantum}, vol.~2, Apr 2021.

\bibitem{dahlberg_sigdc19}
A.~Dahlberg, M.~Skrzypczyk, T.~Coopmans, L.~Wubben, F.~Rozpundefineddek,
  M.~Pompili, A.~Stolk, P.~Pawe\l{}czak, R.~Knegjens, J.~de~Oliveira~Filho,
  R.~Hanson, and S.~Wehner, ``A link layer protocol for quantum networks,'' in
  {\em Proceedings of the ACM Special Interest Group on Data Communication},
  SIGCOMM '19, (New York, NY, USA), pp.~159--173, Association for Computing
  Machinery, 2019.

\bibitem{abadal_ieeecm18}
S.~Abadal, A.~Mestres, J.~Torrellas, E.~Alarcon, and A.~Cabellos-Aparicio,
  ``Medium access control in wireless network-on-chip: A context analysis,''
  {\em IEEE Communications Magazine}, vol.~56, no.~6, pp.~172--178, 2018.

\bibitem{mansoor_tmscs15}
N.~Mansoor, P.~J.~S. Iruthayaraj, and A.~Ganguly, ``Design methodology for a
  robust and energy-efficient millimeter-wave wireless network-on-chip,'' {\em
  IEEE Transactions on Multi-Scale Computing Systems}, vol.~1, no.~1,
  pp.~33--45, 2015.

\bibitem{almudever2020realizing}
C.~G. Almudever, L.~Lao, R.~Wille, and G.~G. Guerreschi, ``Realizing quantum
  algorithms on real quantum computing devices,'' in {\em 2020 Design,
  Automation \& Test in Europe Conference \& Exhibition (DATE)}, pp.~864--872,
  IEEE, 2020.

\bibitem{lao2021timing}
L.~Lao, H.~Van~Someren, I.~Ashraf, and C.~G. Almudever, ``Timing and
  resource-aware mapping of quantum circuits to superconducting processors,''
  {\em IEEE Transactions on Computer-Aided Design of Integrated Circuits and
  Systems}, vol.~41, no.~2, pp.~359--371, 2021.

\bibitem{murali2019full}
P.~Murali, N.~M. Linke, M.~Martonosi, A.~J. Abhari, N.~H. Nguyen, and C.~H.
  Alderete, ``Full-stack, real-system quantum computer studies: Architectural
  comparisons and design insights,'' in {\em 2019 ACM/IEEE 46th Annual
  International Symposium on Computer Architecture (ISCA)}, pp.~527--540, IEEE,
  2019.

\bibitem{baker2020time}
J.~M. Baker, C.~Duckering, A.~Hoover, and F.~T. Chong, ``Time-sliced quantum
  circuit partitioning for modular architectures,'' in {\em Proceedings of the
  17th ACM International Conference on Computing Frontiers}, pp.~98--107, 2020.

\bibitem{rodrigo2021double}
S.~Rodrigo, S.~Abadal, E.~Alarc{\'o}n, M.~Bandic, H.~Van~Someren, and C.~G.
  Almud{\'e}ver, ``On double full-stack communication-enabled architectures for
  multicore quantum computers,'' {\em IEEE micro}, vol.~41, no.~5, pp.~48--56,
  2021.

\bibitem{brown2016co}
K.~R. Brown, J.~Kim, and C.~Monroe, ``Co-designing a scalable quantum computer
  with trapped atomic ions,'' {\em npj Quantum Information}, vol.~2, no.~1,
  pp.~1--10, 2016.

\bibitem{wack2021quality}
A.~Wack, H.~Paik, A.~Javadi-Abhari, P.~Jurcevic, I.~Faro, J.~M. Gambetta, and
  B.~R. Johnson, ``Quality, speed, and scale: three key attributes to measure
  the performance of near-term quantum computers,'' {\em arXiv preprint
  arXiv:2110.14108}, 2021.

\bibitem{blume2020volumetric}
R.~Blume-Kohout and K.~C. Young, ``A volumetric framework for quantum computer
  benchmarks,'' {\em Quantum}, vol.~4, p.~362, 2020.

\bibitem{mills2021application}
D.~Mills, S.~Sivarajah, T.~L. Scholten, and R.~Duncan, ``Application-motivated,
  holistic benchmarking of a full quantum computing stack,'' {\em Quantum},
  vol.~5, p.~415, 2021.

\bibitem{almudever2021structured}
C.~G. Almudever and E.~Alarcon, ``Structured optimized architecting of
  full-stack quantum systems in the nisq era,'' in {\em 2021 Design, Automation
  \& Test in Europe Conference \& Exhibition (DATE)}, pp.~762--767, IEEE, 2021.

\bibitem{tomesh2021quantum}
T.~Tomesh and M.~Martonosi, ``Quantum codesign,'' {\em IEEE Micro}, vol.~41,
  no.~5, pp.~33--40, 2021.

\end{thebibliography}

\end{document}
