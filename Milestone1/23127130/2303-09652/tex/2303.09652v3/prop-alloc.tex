\documentclass[11pt]{article}

\setlength{\textwidth}{6.5in} \setlength{\textheight}{8.5in}\hoffset=-0.1in \voffset=-0.5in

\usepackage{subcaption,booktabs}

\usepackage{authblk}
\usepackage[english]{babel}
\usepackage[utf8]{inputenc}
\usepackage[T1]{fontenc}
\usepackage{amsmath,amsfonts,amsthm,amssymb}
\usepackage{mathtools}
\usepackage{graphicx}
\usepackage[table]{xcolor}
\usepackage{multirow}
\usepackage{xurl}
\usepackage{hyperref}
\usepackage{fullpage}
\usepackage[numbers]{natbib}
\hypersetup{
    colorlinks=true,
    linkcolor=red,
    filecolor=magenta,      
    urlcolor=blue,
    linkbordercolor = white
}

% Reducing the bibliography spacing
%\setlength{\bibsep}{0.0pt}
\usepackage{enumitem}
%\setlength{\itemsep}{0em}

\newtheorem{theorem}{Theorem}
\newtheorem{proposition}[theorem]{Proposition}
\newtheorem{example}{Example}%
\newtheorem{remark}{Remark}%
\newtheorem{definition}{Definition}%

\newcommand{\sym}[1]{{\sf #1}}

% Algorithms
%\usepackage[margin=3cm]{geometry}
\usepackage{algorithm2e}
\RestyleAlgo{ruled}
%% This is needed if you want to add comments in
%% your algorithm with \Comment
\SetKwComment{Comment}{/* }{ */}

% Colours to mark portions of text
\newcommand{\red}[1]{\textcolor{red}{#1}}
\newcommand{\blue}[1]{\textcolor{blue}{#1}}

\title{On Using Proportional Representation Methods as Alternatives to Pro-Rata Based Order Matching Algorithms in Stock Exchanges}
%\titlerunning{Proportional Representation Based Order Matching}
\author[1]{Sanjay Bhattacherjee}
%\affil[1]{School of Computing, University of Kent, CT2 7NF, United Kingdom, email: s.bhattacherjee@kent.ac.uk}
\affil[1]{
	Institute of Cyber Security for Society and School of Computing \authorcr 
	Keynes College, University of Kent \authorcr 
	CT2 7NP, United Kingdom \authorcr 
	email: s.bhattacherjee@kent.ac.uk \authorcr
	ORCID: 0000-0002-3367-6192}
\author[2]{Palash Sarkar\thanks{Corresponding author.}}
\affil[2]{
	Indian Statistical Institute \authorcr 
	203, B.T. Road, Kolkata \authorcr
	India 700108 \authorcr 
	email: palash@isical.ac.in \authorcr
	ORCID: 0000-0002-5346-2650}

\date{\today}

\begin{document}

\maketitle


\begin{abstract}
	The main observation of this short note is that methods for determining proportional representation in electoral systems may be suitable as alternatives to 
	the pro-rata order matching algorithm used in stock exchanges. Our simulation studies provide strong evidence that the Jefferson/D'Hondt 
	and the Webster/Saint-Lagu\"{e} proportional representation methods provide order allocations which are closer to proportionality than the order allocations 
	obtained from the pro-rata algorithm. \\
	{\bf Keywords:} order matching algorithm, pro-rata algorithm, proportional representation, Jefferson/D'Hondt method, Webster/Saint-Lagu\"{e} method. \\
	{\bf JEL codes:} D49.
\end{abstract}

%Suppose there are n resting orders of sizes S_1 to S_n and total size S. Let p_i=S_i/S. Suppose the incoming order is of size K. His method performs K independent trials of the following random %experiment. Pick a number in 1 to n, where number i has probability p_i of being chosen; assign one unit of the incoming order to resting order number i. The rationale behind this is that on the average, %the assignment of orders will be the ideal distribution.

\newpage


\section{Introduction\label{sec:intro}}
Financial instruments are traded on stock exchanges. Traders place orders to buy and sell such instruments. 
An order specifies, among other things, the quantity or size, i.e. the number of units to be purchased or sold, and the price at which the order is to be executed.
The quoted price is required to be a multiple of a unit of price called tick. A stock exchange maintains an order book which records for each financial instrument
the corresponding list of orders. The orders quoting the same buying or selling price are placed at the same price level in the order book. 
%The number of price 
%levels of a financial instrument is called its book depth. The difference between the highest bid for buying and the lowest ask for selling is called the bid-ask spread 
%of the instrument.

Trading in a stock exchange occurs by executing orders. Buy orders for an instrument are matched with corresponding sell orders. 
Matching happens in units of the instrument. The quantity specified in an order may not be equal to the quantity of the counter-party order that it is matched with. An 
order is filled when all its units have been matched with one or more counter-party matching orders. Unmatched or partially filled orders rest in the 
order book waiting for new orders to be matched with.
Orders can be of different types. %See~\cite{CS2020} for an overview of various kinds of orders. 
A common and important type of order is a limit order which specify both the quantity and the price at which the order is to be executed.

Let us consider the order book entries of a financial instrument with $p$ price levels $L_1, \ldots, L_p$. Let the number of buy and sell limit orders at price $L_i$ be denoted as 
$b_i$ and $s_i$ respectively. We note that in the resting state of the order book for the instrument, either $b_i=0$ or $s_i=0$ for a price $L_i$. Otherwise, 
the opposing orders will be matched until there are no more buy or sell orders to be matched (i.e., either $b_i=0$ or $s_i=0$ or both). A new incoming order at price 
$L_i$ could be of the same type as the resting orders in which case it will be added to the list of resting orders or it could be a counter-party order in which case it 
will be matched with the resting orders.

Let us assume that at price level $L$ there are $n$ resting orders and the quantities of the orders are given by a vector $\mathbf{T} = (T_1, \ldots, T_n)$, 
where $T_i$ is a positive integer which represents the quantity of the $i$-th order. As discussed above, these $n$ orders are either all buy orders or all sell orders. 
Let $T = T_1 + \cdots + T_n$ be the total quantity of these orders. A new incoming counter-party order of quantity $S$ at 
price $L$ will be matched with the resting orders in $\mathbf{T}$. As a result, some or all of the resting orders may be executed. If $S \geq T$, the resting orders 
are all filled and can be fully executed. However, if $S < T$, not all resting orders can be completely filled. In this case, an order matching algorithm is required
to allocate portions of the incoming counter-party order to the $n$ resting orders. Henceforth, we will assume that the condition $S<T$ holds.

Formally, an order matching algorithm $\mathcal{M}(n, \mathbf{T}, S)$ takes as input the number $n$ of resting orders, the vector $\mathbf{T} = (T_1, \ldots, T_n)$ of the 
resting order quantities, and an incoming counter-party order of quantity $S$ with $0<S<T$. It 
outputs $\mathbf{S} = (S_1, \ldots, S_n)$ so that $S_i$ quantity of $T_i$ may be executed. In other words, $0\leq S_i \leq T_i$ and $S = S_1 + \ldots + S_n$. 

Two of the most common order matching algorithms are the price-time priority (also called first-come-first-served (FCFS) or
first-in-first-out (FIFO)) and the pro-rata methods (see for example~\cite{CMEMatchingAlgorithms,Pr11,CS2020,He22}). Both methods aim to achieve some kind of fariness
in the allocation of orders. In this work we focus on the pro-rata method.

The idea behind the pro-rata method is to distribute $S$ to the $n$ resting orders more or less in proportion to their fractions of the total order. 
So ideally the $i$-th resting order would receive $ST_i/T$ portion of the incoming counter-party order. This, however, has a problem. Financial instruments are
traded in indivisible atomic units. So if $ST_i/T$ is not an integer, then this amount of the order cannot be executed. The pro-rata order matching algorithm
adopts a two-step approach to this problem. In the first step, the algorithm assigns an amount $S_i^{\prime}=\lfloor ST_i/T\rfloor$ of the incoming counter-party order 
to the $i$-th resting 
order\footnote{For a real number $x$, $\lfloor x\rfloor$ denotes the greatest integer not greater 
than $x$, and $\lceil x\rceil$ denotes the least integer not less than $x$.}$^,$\footnote{Sometimes a simple modification to the first step is used. For 
example,~\cite{CMEMatchingAlgorithms} adopts the strategy in which if some $S_i^{\prime}$ turns out to be $1$, then it is instead set to 0.}.
This strategy consumes $S^{\prime}=S_1^{\prime}+\cdots+S_n^{\prime}$ units of the incoming counter-party order. In the second step, the remaining 
$S-S^{\prime}$ units are distributed to the resting orders based upon some strategy which could be the FCFS
strategy\footnote{With the second step as FCFS, the overall method is a hybrid of pro-rata and FCFS. Other hybrid combinations of
FCFS and pro-rata have been proposed. See for example~\cite{BCS15}}.

In this note, we raise two questions.
\begin{enumerate}
	\item How well does the pro-rata order matching algorithm achieve its goal of distributing the incoming order to the resting orders in proportion to their
		fractions of the total order?
	\item Are there other algorithms which perform better than the pro-rata algorithm in achieving proportionality?
\end{enumerate}
To answer the above questions, we need a measure to assess the performance of an order matching algorithm in achieving proportionality. For $n$ resting
orders, with $\mathbf{T}=(T_1,\ldots,T_n)$ the vector of quantities of the resting orders, and $S$ the size of the incoming order, the ideal allocation, or
the ideal proportional distribution is given by the vector 
\begin{eqnarray}\label{eqn-ideal}
\mathbf{I}=(ST_1/T,ST_2/T,\ldots,ST_n/T). 
\end{eqnarray}
Suppose $\mathcal{A}$ is an order matching algorithm which on input $n$, $\mathbf{T}$ and $S$
produces the allocation vector $\mathbf{S}_{\mathcal{A}}=(S_1,\ldots,S_n)$ as output, where $S_i\leq T_i$ for $i=1,\ldots,n$, and $S_1+\cdots+S_n=S$. 
The distance between the vectors $\mathbf{I}$ and $\mathbf{S}_{\mathcal{A}}$ is a measure of the performance of the algorithm $\mathcal{A}$ in achieving
proportionality. The closer $\mathbf{S}_{\mathcal{A}}$ is to $\mathbf{I}$, the better is the performance of $\mathcal{A}$ in achieving proportionality. 

We consider two standard measures of distance between two vectors, namely the $L_1$ and the $L_2$ metrics defined as follows.
\begin{eqnarray*}
	L_1(\mathbf{S}_{\mathcal{A}},\mathbf{I}) = \sum_{i=1}^n \mid S_i - ST_i/T \mid, & &  L_2(\mathbf{S}_{\mathcal{A}},\mathbf{I}) = \sum_{i=1}^n (S_i - ST_i/T)^2.
\end{eqnarray*}
Using these two metrics, we can quantifiably answer the first question posed above. The metrics also provide a method to address the second question.
For an order matching algorithm $\mathcal{A}$, let $\ell_{1,\mathcal{A}}$ and $\ell_{2,\mathcal{A}}$ denote
$L_1(\mathbf{S}_{\mathcal{A}},\mathbf{I})$ and $L_2(\mathbf{S}_{\mathcal{A}},\mathbf{I})$ respectively. For two order matching algorithms $\mathcal{A}_1$
and $\mathcal{A}_2$, we say that $\mathcal{A}_1$ is $L_1$-better (resp. $L_2$-better) than $\mathcal{A}_2$ if 
$\ell_{1,\mathcal{A}_1}<\ell_{1,\mathcal{A}_2}$ (resp. $\ell_{2,\mathcal{A}_1}<\ell_{2,\mathcal{A}_2}$). In other words, $\mathcal{A}_1$ is 
$L_1$-better (resp. $L_2$-better) than $\mathcal{A}_2$, if its output is closer to the ideal allocation with respect to the $L_1$ (resp. $L_2$) metric.
Using the above terminology, we can rephrase the second question as follows. 
\begin{quote}
Is there an order matching algorithm $\mathcal{A}$ which is better than
the pro-rata order matching algorithm with respect to either or both of the $L_1$ and $L_2$ metrics?
\end{quote}

\subsection{Seat Distribution in Electoral Systems \label{subsec-elec} }
To answer the above question, we visit the literature on proportional representation in electoral systems which is far removed from the stock exchanges and
more generally the financial world. 
Proportional representation is the most common kind of electoral system where the seats are not contested individually. Instead, the total number of seats is allocated 
to the contesting parties in proportion to the number of votes they have won in the election.  Let us consider an election contested by $n$ parties over $K$ seats that are 
distributed using a proportional representation method. Let $V_j$ denote the number of votes won by party $j \in [1, \ldots, n]$ in the election. The electoral output 
is denoted by the vector $\mathbf{V} = (V_1, \ldots, V_n)$, and the total number of votes cast in the election is $V = V_1 + \cdots + V_n$. Suppose the total number
of seats to be distributed among the parties is $K$. A proportional representation method determines the seat allocation vector 
$\mathbf{K}=(K_1,\ldots,K_n)$ where $K_i$ is the number of seats allocated to the $i$-th party, and $K_1+\cdots+K_n=K$. Typically, the total number of seats $K$ is much 
smaller than the total number of votes $V$, i.e. $K<V$. Further, it is reasonable to assume that in practice the number of seats allocated to the 
$i$-th party is at most the number of votes received by the party, i.e. $K_i\leq V_i$. 

Formally, a proportional representation method is an algorithm $\mathcal{A}(n, \mathbf{V}, K)$ which takes as input the number $n$ of parties, the vote count vector 
$\mathbf{V} = (V_1, \ldots, V_n)$, and the number $K$ of seats to be distributed, where $0<K<V$. It outputs the seat allocation vector $\mathbf{K} = (K_1, \ldots, K_n)$
such that $0\leq K_i\leq V_i$ and $K_1+\cdots+K_n=K$. 

From the above description, it becomes clear that the goal of allocating an incoming counter-party order to resting orders in proportion to the sizes of the resting orders 
is the same as the goal of assigning a fixed number of seats to several contesting parties in proportion to the number of votes obtained by these parties. The correspondence becomes 
clear by identifying the size $T_i$ of the $i$-th order with the number of votes $V_i$
received by the $i$-th party, the size $S$ of the incoming counter-party order with the total number of seats $K$, and the quantity $S_i$ of the $i$-th order
that is filled with the number of seats $K_i$ alloted to the $i$-th party. Having identified this correspondence, any algorithm for proportional representation of seats
in an electoral system becomes a potential candidate for use as an order matching algorithm by a stock exchange for proportional fulfillment of orders. The 
identification of proportional representation methods as possible substitutes for the pro-rata order matching algorithm is the key observation of the present note.
Not all proportional representation methods, however, are suitable for use as order matching algorithms. Some methods may require certain conditions to be 
applied which cannot be expected to hold in the context of order matching. We point out some such examples in Section~\ref{sec:order-matching}.

There is a large literature on electoral systems in general and proportional representation methods in particular. We refer the reader to~\cite{Herron2018,Norris2004}
for elaborate discussions on these topics. A number of proportional representation methods have been proposed in the context of electoral systems. 
These can be divided into two types, the highest averages method and the largest remainder method. 
The most
well known of the highest averages method are the Jefferson/D'Hondt (JD) and the Webster/Sainte-Lagu\"{e} (WS) methods. Both of these methods continue to be of active interest.
See for example~\cite{HM08,Me19,FSS20,GF17}. Among the largest remainder method, the two well known methods are the Hare and the Droop methods. In fact, the
Hare method has the attractive property that it minimises the $L_1$-distance to the ideal allocation. However, the largest remainder methods suffer from 
certain paradoxes (see Section~\ref{sec:order-matching}), which make them unsuitable for use as order matching algorithms. 

In the present context, both JD and the WX methods are well suited to be used as order matching algorithms in stock exchanges. We report simulation studies comparing
the pro-rata, the JD and the WS methods. Such studies show that both the JD and the WS methods are both $L_1$ and $L_2$-better than the pro-rata method. Among the
three methods, the allocation determined by the WS method turns out to be the closest to the ideal allocation in an overwhelming number of cases for both $L_1$ and $L_2$ metrics.
This provides sufficient evidence to seriously consider the adoption of Webster/Sainte-Lagu\"{e} method based order matching by stock exchanges.

\subsection{Procedural Fairness \label{subsec-fair} }
A recent paper by Hersch~\cite{He22} investigated the issue of procedural fairness of order allocation methods. In the paper, it was argued that
both the FIFO and the pro-rata are fair in principle, but not in practice. It was pointed out that the main disadvantage of pro-rata is the requirement
of the second step ``requiring exchanges to introduce secondary matching rules that can be gamed''. 

An alternative method called the random selection for service (RSS) method was proposed. 
Given the vector $\mathbf{T}=(T_1,\ldots,T_n)$ of resting orders, the RSS method defines a probability distribution $\pi$ over $\{1,\ldots,n\}$, where 
$\pi$ associates probability $T_i/T$ to $i$, for $i=1,\ldots,n$, 
Suppose the incoming counter-party order consists of $S$ units. Allocation is done by repeating the following procedure $S$ times:
an independent random $i$ is drawn from $\{1,\ldots,n\}$ following the probability distribution $\pi$ and
one unit is alloted to the $i$-th resting order. At the end of the procedure, let $S_i$ be the number of units alloted to the $i$-th resting order so that the
final allotment is $\mathbf{S}=(S_1,\ldots,S_n)$ satisfying $S=S_1+\cdots+S_n$.
It was argued by Hersch~\cite{He22} that the RSS method is fair in both principle and practice. Below we revisit this method and point out a crucial difference between principle
and practice.

Given $\mathbf{T}=(T_1,\ldots,T_n)$ and $S$, suppose the RSS method is executed $\Gamma$ times and for $\gamma=1,\ldots,\Gamma$, let
the allotment of $\gamma$-th execution be $\mathbf{S}_{\gamma}=(S_{\gamma,1},\ldots,S_{\gamma,n})$. For $i=1,\ldots,n$, let 
$\widehat{S}_i = (S_{1,i}+\cdots+S_{\Gamma,i})/\Gamma$, i.e. $\widehat{S}_i$ is the average allotment to the $i$-th resting order computed over all the
$\Gamma$ trials. The law of large numbers assures us that asymptotically, i.e. as $\Gamma$ goes to infinity, the average allotment $\widehat{S}_i$
tends to $ST_i/T$ which is equal to the $i$-th component of the ideal allocation vector $\mathbf{I}$ (see~\eqref{eqn-ideal}). So in principle, Hersch~\cite{He22} implicitly
considers achieving allocation close to the ideal allocation vector $\mathbf{I}$ to be procedurally fair. To this extent, Hersch's objective and ours coincide.
Additionally, our use of the $L_1$ and $L_2$ metrics to measure deviation from the ideal allocation vector can be considered to be a quantification of procedural fairness.
As such it expands the theoretical framework for studying procedural fairness of the order allocation methods.

From a practical point of view, however, the RSS method has a significant shortcoming. The law of large numbers applies in an asymptotic contex, i.e. as
$\Gamma$ goes to infinity. In practice, given $\mathbf{T}=(T_1,\ldots,T_n)$ and $S$, a stock exchange will execute the RSS method exactly once to obtain a single allocation 
vector $\mathbf{S}=(S_1,\ldots,S_n)$. In other words, in practice the value of $\Gamma$ will be 1. 
The law of large numbers does not say anything about the value obtain in a single execution. In particular,
the $S_i$'s obtained after a single execution of RSS can be any value in the set $\{0,\ldots,S\}$. Considering a particular example with $n=2$, $\mathbf{T}=(10,90)$ and
$S=10$, Hersch~\cite{He22} provides probabilities that the $S_i$'s can take certain values: the probability that $S_1\geq 1$ (resp. $S_1=10$) is about 
0.88 (resp. $7\times 10^{-6}$); the probability that $S_2=20$ (resp. $S_2\geq 10$) is about 0.12 (resp. approaches 1). These probabilities, however, do not
enlighten us about the concrete values of the $S_i$'s after a single execution. In particular, the probability that $S_2\geq 10$ approaches 1 suggests an
asymptotic approach, where the frequentist view of probability is taken. To interpret such probabilities, one again needs to consider a large number $\Gamma$
of trials of the RSS method and consider the average allocation over all the $\Gamma$ trials. 

To test the practical efficacy of the RSS method, we have run experiments with the method. It turns out that the allocation vector obtained by the RSS method
has a very large deviation from the ideal allocation vector in terms of both the $L_1$ and the $L_2$ metrics. Particular examples are provided in Section~\ref{sec-sim-res}.
By the above explanation, this observation is not surprising.

As mentioned earlier, according to Hersch, the main disadvantage of the pro-rata method is the use of secondary matching rules in the second step of the method
which leads to the possibility of gaming. The proportional representation based order allocation method that we introduce does not require any such secondary
matching rules which can be gamed. 
So our proposal overcomes the disadvantage of the pro-rata method pointed out by Hersch. In terms of procedural fairness as measured by
distance to the ideal allocation vector, our simulation studies show that proportional representation based order allocation outperforms the pro-rata method.

\section{Proportional Representation Methods \label{sec:order-matching}}

There are many different proportional representation methods (see~\cite{Herron2018,Norris2004}). Below we describe two well known families of proportional representation 
methods, namely the highest averages or the divisor method, and the highest remainder method. 

\paragraph{Highest averages method.}
Recall the setting described in Section~\ref{subsec-elec}, where $V_i$ is the number of votes received by the $i$-th party and $K$ is the total number of available seats. 
The goal is to determine $K_i$ which is the number of seats alloted to the $i$-th party.
Let $f: \mathbb{Z}^{+} \cup \{0\} \to \mathbb{R}$ be a function from the non-negative integers to the reals. Let $V=V_1+\cdots+V_n$ and $v_i=V_i/V$. 
In the highest averages method, seats are allotted iteratively. The seat distribution algorithm goes through $K$ iterations and in each iteration exactly one seat 
is alloted to one of the parties. Initially, the algorithm sets $K_1=K_2=\cdots=K_n=0$. For $k$ from $1$ to $K$, in the $k$-th iteration
the algorithm determines $j=\arg\max\{v_i/f(K_i): i=1,\ldots,n\}$ and increments $K_j$ by one. After $K$ iterations, the final values of $K_1,\ldots,K_n$ are the numbers of
seats alloted to the various parties. Various different methods arise from the different definitions of $f$. The definitions of $f$ for the well known
Jefferson/D'Hondt and the Webster/Sainte-Lagu\"{e} methods are shown in Table~\ref{tab:highest-averages-methods}.

\begin{table}[!htb]
    \centering
    \begin{tabular}{l|l}
    \hline
        Method name & $f(t)$ \\
    \hline
%	    Adam (Ad) & $\lceil t \rceil$ \\
%	    Danish (Da) & $t/(t+(1/3))$ \\
%	    Dean (De) & ${t(t+1)}/{(t+0.5)}$ \\
%	    Huntington/Hill (HH) & $\sqrt{t(t+1)}$ \\
	    Jefferson/D'Hondt (JD) & $(t+1)$ \\
	    Webster/Sainte-Lagu\"{e} (WS) & $(t+0.5)$ \\
    \hline
    \end{tabular}
	\caption{The functions used for two important methods of proportional representation.  \label{tab:highest-averages-methods}}
\end{table}

Apart from the JD and the WS methods, there are a number of other proportional representation methods, such as Dean's, Adam's, Huntington/Hill and the Danish methods.
(See~\cite{Wiki-division} for a compact description of these methods.)
These four methods require a positivity constraint to be satisfied which is not required in either the JD or the WS methods. They initially allocate one seat to each of the
contesting parties (i.e., they start with $K_1=\cdots=K_n=1$ instead of starting with $K_1=\cdots=K_n=0$) and then employ the highest averages method described above. 
As a result, at the end of the allocation each party has at least one seat. The 
definitions of the function $f$ for these four methods are different from the definitions of the function corresponding to the JD and the WS methods. Importantly, in 
these four methods, the function $f$ satisfies the condition $f(0)=0$. 
Consequently, the function cannot be evaluated unless the present number of seats allocated to a party is at least 1. This constraint is not present for
the JD and the WS methods. Note that the constraint of allocating at least one seat to each party requires the number of seats to be at least as large as the
number of parties.

In the context of order matching, the feature of assigning at least one seat to each party will translate to assiging at least one unit of the 
incoming counter-party order to each of the resting orders. This requires the size of the incoming counter-party order to be at least the number of resting orders,
i.e. $S\geq n$. Such a condition cannot be imposed in general, since there is no control over the size $S$ of the incoming counter-party order. More generally,
the principle of assigning at least one unit of the incoming order to each of the resting orders does not appear to have any justification in the context
of stock exchanges. On the contrary,
some stock exchanges follow the rule that if the order unit determined by the pro-rata method is 1, then this is instead set to 0~\cite{CMEMatchingAlgorithms}.
The principle of at least one unit for each resting order may not be welcome by such exchanges. Due to this reason as well as the unimplementable constraint
of $S\geq n$, in this paper we do not consider the proportional representation methods which follow the principle of assigning at least one seat to each party.

\paragraph{Largest remainder method.}
The method uses a parameter called the quota $Q$. The seat allocation is done in two phases. In the first phase, the $i$-th party is allocated $\lfloor V_i/Q\rfloor$ 
seats. Let $R_i=V_i/Q - \lfloor V_i/Q\rfloor$ be the remainder corresponding to the $i$-th party. Suppose after the first phase $k$ seats remain unallocated. 
In the second phase, the parties with the $k$ largest remainders are each allocated one seat. Let $K_i$ be the number of seats allocated to the $i$-th party at the
end of the second round. The method ensures that $K_1+\cdots+K_n=K$. 
Various methods arise by choosing an appropriate value of the quota $Q$. The Hare quota chooses $Q$ to be equal to $V/K$, while the Droop quota chooses
$Q$ to be equal to $1 + \lfloor V/(1+K) \rfloor$. Other choices for $Q$ have been 
proposed\footnote{\url{https://en.wikipedia.org/wiki/Largest_remainder_method}}. 

The largest remainder method satisfies the quota rule, i.e.  the condition $\lfloor K V_i/V\rfloor \leq K_i \leq \lceil K V_i/V\rceil$.
It is known that the largest remainder method with the Hare quota minimises the Loosemore-Hanby index\footnote{\url{https://en.wikipedia.org/wiki/Loosemore\%E2\%80\%93Hanby_index}}
which is equivalent to minimising the $L_1$-distance from the ideal proportional seat allocation $(KV_1/V,KV_2/V,\ldots,KV_n/V)$. 

\paragraph{Paradoxes.}
Balinski and Young~\cite{BY01} identified several paradoxes of proportional representation. Consider two possible vector of votes and the total number of seats
with $(V_1,\ldots,V_n)$ and $K$ be one of the scenarios and $(V_1^{\prime},\ldots,V_n^{\prime})$ and $K^{\prime}$. Let $(K_1,\ldots,K_n)$ and
$(K_1^{\prime},\ldots,K_n^{\prime})$ be the corresponding seat allocation vectors. 
\begin{itemize}
	\item The Alabama paradox is the following: $V_i=V_i^{\prime}$ for $i=1,\ldots,n$
		and $K^{\prime}>K$, but there is a $j\in \{1,\ldots,n\}$ such that $K_j>K_j^{\prime}$. 
		In other words, the votes polled remain the same and the total number of seats has gone up, but the number of seats allocated to a party has gone down.
	\item The population paradox is the following: $K=K^{\prime}$, $V_i^{\prime}\geq V_i$, and for some $j,k\in\{1,\ldots,n\}$, 
		$V_j^{\prime}-V_j > V_k^{\prime}-V_k$, but $K_j^{\prime}<K$. In other words,
		the number of seats remains the same and the additional votes polled by one party is more than another, but the former party is allocted fewer seats.
\end{itemize}
The Balinski-Young theorem states that any method of apportionment which satisfies the quota rule will necessarily suffer from either the Alabama or the population
paradox. 

In the context of order matching, both the Alabama and the population paradoxes are problematic. The Alabama paradox translates to the following.
The quantities of the resting orders remain the same and an increase occurs in the incoming counter-party order, yet the allocation to a particular resting order goes down. 
The population paradox translates to the following. The size of the incoming counter-party order remains the same, and the size of a particular resting order increases
by an amount which is more than the increase in the size of another resting order, but the units allocated to the former resting order goes down.
Both of these are counterintuitive and hard to explain to the stakeholders of a stock exchange. Due to these paradoxes, the largest remainder method with Hare quota
is unsuitable for use as order matching algorithm even though it minimises the distance to the ideal allocation vector.

The highest averages methods are free of these paradoxes. As a consequence, by the Balinski-Young theorem, they violate the quota rule. It is, however, known that
the while the WS method in principle violates the quota rule, it does so rarely. 

Based on the above discussion, we consider only the JD and the WS method in our simulation studies.

\section{Simulation and Results \label{sec-sim-res}}
%We have implemented the pro-rata, the JD and the WS methods.

Given $n$, $\mathbf{T}=(T_1,\ldots,T_n)$ and $S$, the pro-rata allocation takes place in two steps. In the first step, the vector 
$$\mathbf{S}_{\mathcal{P}}^{\prime}=(S_1^{\prime},S_2^{\prime},\ldots,S_n^{\prime})=(\lfloor ST_1/T \rfloor, \lfloor ST_2/T \rfloor, \ldots, \lfloor  ST_n/T\rfloor)$$ 
is computed. In the second step, the remaining $S-S^{\prime}$ (where $S^{\prime}=S_1^{\prime}+S_2^{\prime}+\cdots+S_n^{\prime}$) quantity of the incoming order 
is allocated using the first-come-first-served strategy. In our implementation, we have used a modified version of the first-come-first-served-strategy where we 
have prioritised smaller orders as follows: allocate one unit to all orders in the first-come-first-served manner which got zero allocation in the first step, 
next allocate one unit to all orders in the first-come-first-served manner which was alloted one unit in the first step, and so on until all the $S-S^{\prime}$ units left
over from the first step are exhausted. The second step increases the allocation to any resting order by at most one unit.
%and we also check and ensure that alloted number of units to the $i$-th resting order is not greater than $ST_i/T$. 
The rationale for using the modified first-come-first-served strategy is to provide some benefit to smaller orders. 
%Since $S-S^{\prime}$ is at most $n$, the overall effect of this modified strategy has a negligible effect on the values of $\ell_{1,\mathcal{P}}$ and $\ell_{2,\mathcal{P}}$, 
%where $\mathcal{P}$ denotes the pro-rata method. 

The pro-rata method clearly takes $O(n)$ time. The highest averages method (of which the JD and the WS methods are special cases)
described in Section~\ref{sec:order-matching} can be implemented in time $O(n+K\log n)$. (By the identification of the size $S$ of the incoming order with the number $K$
of available seats, we have $O(n+K\log n)=O(n+S\log n)$.)
We briefly discuss how this can be done. Note that $O(n)$ time is required to initialise the allocation vector $K=(K_1,\ldots,K_n)$ to the all-zero vector. Next 
a max-heap data structure~\cite{AHU78} is built on the $n$ values $f_1=v_1/f(K_1), f_2=v_2/f(K_2),\ldots, f_n=v_n/f(K_n)$. This takes $O(n)$ time. The heap data structure 
stores the maximum of the $f_i$'s on the top. In each of the $K$ iterations,
exactly one $K_i$ is incremented, so exactly one $f_i$ is modified and the other $f_j$'s remain unchanged. The heap data structure is updated so that the new maximum
gets to the top. This can be done in $O(\log n)$ time and makes the new maximum available for the next iteration. So the $K$ iterations of the highest averages
method take $O(K\log n)$ time. 
%Below we present simulation results which show that the JD and the WS algorithms achieve better proportionality. The trade-off is that both of these algorithms
%take more time than the pro-rata method.

To compare the performances of the different algorithms, we have performed simulation studies. The input to an order matching algorithm is the number of resting
orders $n$, the vector $\mathbf{T}=(T_1,\ldots,T_n)$, where $T_i$ is the size of the $i$-th order, and the size $S$ of the incoming counter-party order. 
In our simulations, we have randomly generated the values $T_1,\ldots,T_n$ using the following strategy. Fix two non-negative integers $m$ and $M$ with $m<M$. 
Let $\mu$ and $\sigma$ be positive real numbers which specify the normal ${\mathcal N}(\mu,\sigma)$ distribution. For each $i$ in $1$ to $n$, the following
procedure is performed: draw a sample from ${\mathcal N}(\mu,\sigma)$ and round to the nearest integer, repeat until the rounded value is in the range $[m,M]$; 
once the rounded value satisfies the range check, set $T_i$ to be equal to this rounded value.
After $n$ iterations, we obtain the random vector $\mathbf{T}=(T_1,\ldots,T_n)$ which is a simulated distribution of the resting orders. 
Note that all the samples are drawn {\em independently} from ${\mathcal N}(\mu,\sigma)$. Our rationale for
choosing the normal distribution is that in the absence of any other information, the sizes of the orders may be assumed to follow the normal distribution. If, on the
other hand, additional information is available, then it is possible to change the normal distribution to another distribution without affecting the rest of the
simulation. 

In Table~\ref{tab-ex}, we provide some examples of the order allocation vector $\mathbf{S}_{\mathcal{A}}$, where 
$\mathcal{A}$ is one of $\mathcal{P}$ (denoting the pro-rata method), RSS, JD, or WS method. 
The ideal allocation vector is $\mathbf{I}=(ST_1/T,\ldots,ST_n/T)$. 
%For the pro-rata method, we provide both the vectors $\mathbf{S}_{\mathcal{P}}^{\prime}$ (the output of the first step of the pro-rata method) and 
%$\mathbf{S}_{\mathcal{P}}$ (the final output of the pro-rata method). 
From the examples, we observe that for the RSS method, the $L_1$ and $L_2$ distances from the ideal allocation vector $\mathbf{I}$ are much larger than these
distances from the other methods. This is as expected (see Section~\ref{subsec-fair}) and highlights the impracticability of the RSS method. While the table
provides only three examples, we have obtained many other examples and the observation that the order allocation vector produced by the RSS method
is much farther away from the ideal allocation vector compared to the other methods holds in all the examples. In view of this, we do not consider the
RSS method any further in our simulation studies.

Among the three methods, i.e. the pro-rata method, the JD and the WS methods, note that in all the cases, the JD and the WS methods are both $L_1$-better and 
$L_2$-better than the pro-rata method. Comparing
$\ell_{1,\mathcal{P}}$ with $\ell_{1,{\rm JD}}$ and $\ell_{1,{\rm WS}}$ and $\ell_{2,\mathcal{P}}$ with $\ell_{2,{\rm JD}}$ and $\ell_{2,{\rm WS}}$, we find significant
difference in these values. So these examples suggest that the JD and the WS methods are significantly better than the pro-rata method with respect to the 
$L_1$ and $L_2$ metrics.

\begin{table}
\centering
{\scriptsize
	\begin{tabular}{|l|l|r|r|r|r|r|r|r|r|r|r|r|r|}
		\cline{13-14}
		\multicolumn{12}{c|}{} & $\ell_{1,\mathcal{A}}$ & $\ell_{2,\mathcal{A}}$ \\ \hline
		\multirow{5}{*}{Ex~1}
		& $\mathbf{T}$ & 209 & 727 & 746 & 808 & 995 & 204 & 598 & 773 & 979 & 899 & - & - \\ \cline{2-14}
		& $\mathbf{I}$ & 3.01 & 10.48 & 10.75 & 11.65 & 14.34 & 2.94 & 8.62 & 11.14 & 14.11 & 12.96 & - & - \\ \cline{2-14}
		%& $\mathbf{S}_{\mathcal{P}}^{\prime}$ & 3 & 10 & 10 & 11 & 14 & 2 & 8 & 11 & 14 & 12 & - & -\\ \cline{2-14}
		& $\mathbf{S}_{\mathcal{P}}$ & 4 & 11 & 11 & 11 & 14 & 3 & 9 & 11 & 14 & 12 & 4.39 & 2.94 \\ \cline{2-14}
		& $\mathbf{S}_{{\rm RSS}}$ & 4 & 7 & 17 & 11 & 18 & 3 & 5 & 10 & 15 & 10 & 23.69 & 89.86 \\ \cline{2-14}
		& $\mathbf{S}_{{\rm JD}}$ & 3 & 10 & 11 & 12 & 14 & 3 & 9 & 11 & 14 & 13 & 2.17 & 0.71 \\ \cline{2-14}
		& $\mathbf{S}_{{\rm WS}}$ & 3 & 10 & 11 & 12 & 14 & 3 & 9 & 11 & 14 & 13 & 2.17 & 0.71 \\ \hline
		\multirow{5}{*}{Ex~2}
		& $\mathbf{T}$ & 1 & 655 & 307 & 138 & 647 & 48 & 625 & 382 & 95 & 424 & - & - \\ \cline{2-14}
		& $\mathbf{I}$ & 0.03 & 19.72 & 9.24 & 4.15 & 19.48 & 1.44 & 18.81 & 11.50 & 2.86 & 12.76 & - & - \\ \cline{2-14}
		%& $\mathbf{S}_{\mathcal{P}}^{\prime}$ & 0 & 19 & 9 & 4 & 19 & 1 & 18 & 11 & 2 & 12 & - & -\\ \cline{2-14}
		& $\mathbf{S}_{\mathcal{P}}$ & 1 & 19 & 10 & 5 & 19 & 2 & 18 & 11 & 3 & 12 & 6.54 & 4.79 \\ \cline{2-14}
		& $\mathbf{S}_{{\rm RSS}}$ & 0 & 21 & 15 & 2 & 21 & 1 & 17 & 8 & 4 & 11 & 19.41 & 61.91 \\ \cline{2-14}
		& $\mathbf{S}_{{\rm JD}}$ & 0 & 20 & 9 & 4 & 20 & 1 & 19 & 12 & 2 & 13 & 3.46 & 1.72 \\ \cline{2-14}
		& $\mathbf{S}_{{\rm WS}}$ & 0 & 20 & 9 & 4 & 19 & 1 & 19 & 12 & 3 & 13 & 2.69 & 0.95 \\ \hline
		\multirow{5}{*}{Ex~3}
		& $\mathbf{T}$ & 268 & 806 & 409 & 420 & 869 & 659 & 189 & 317 & 286 & 721 & - & - \\ \cline{2-14}
		& $\mathbf{I}$ & 5.42 & 16.30 & 8.27 & 8.50 & 17.58 & 13.33 & 3.82 & 6.41 & 5.78 & 14.58 & - & - \\ \cline{2-14}
		%& $\mathbf{S}_{\mathcal{P}}^{\prime}$ & 5 & 16 & 8 & 8 & 17 & 13 & 3 & 6 & 5 & 14 & - & -\\ \cline{2-14}
		& $\mathbf{S}_{\mathcal{P}}$ & 6 & 16 & 9 & 8 & 17 & 13 & 4 & 7 & 6 & 14 & 4.57 & 2.41 \\ \cline{2-14}
		& $\mathbf{S}_{{\rm RSS}}$ & 2 & 15 & 8 & 4 & 12 & 17 & 2 & 6 & 9 & 25 & 34.61 & 200.59 \\ \cline{2-14}
		& $\mathbf{S}_{{\rm JD}}$ & 5 & 17 & 8 & 8 & 18 & 13 & 4 & 6 & 6 & 15 & 3.86 & 1.69 \\ \cline{2-14}
		& $\mathbf{S}_{{\rm WS}}$ & 5 & 16 & 8 & 9 & 18 & 13 & 4 & 6 & 6 & 15 & 3.47 & 1.31 \\ \hline
	\end{tabular}
	\caption{Examples of simulation runs with $n=10$, $S=100$, $m=1$, $M=1000$, $\mu=500$ and $\sigma=400$. Here $\mathcal{P}$ is the pro-rata method. \label{tab-ex}}
}
\end{table}

A few examples do not provide sufficient evidence. It is required to consider many more examples. On the other hand, when there are a large number of examples,
it is not possible to visually inspect all such examples. So we have used a program to perform the comparison for the various simulation studies.
Since $\mathbf{T}$ is determined by $n$, $\mu$ and $\sigma$, the parameters for the simulations are the different values of $n$, $\mu$ and $\sigma$ as well as $S$. 
For a specific set of values of $n$, $\mu$, $\sigma$ and $S$, we have performed $N$ iterations of the simulation. In each iteration, we have computed 
the ideal allocation vector $\mathbf{I}=(ST_1/T,\ldots,ST_n/T)$, and the order allocation vector
$\mathbf{S}_{\mathcal{A}}=(S_1,\ldots,S_n)$ produced by the order matching algorithm $\mathcal{A}$, where $\mathcal{A}$ is one of pro-rata, the JD or the WS algorithms.
%The ideal proportional allocation vector in each iteration is $\mathbf{I}=(ST_1/T,\ldots,ST_n/T)$ and the output of algorithm $\mathcal{A}$ is 
%given by the vector $\mathbf{S}_{\mathcal{A}}=(S_1,S_2,\ldots,S_n)$. 
Next we computed the $L_1$ and $L_2$ distances of $\mathbf{S}_{\mathcal{A}}$ from $\mathbf{I}$ given by $\ell_{1,\mathcal{A}}$ and $\ell_{2,\mathcal{A}}$. 
In each of the $N$ iterations, we have compared the JD and the WS methods with the pro-rata method. After $N$ iterations, aggregate statistics are determined
for the particular simulation. Further details are given below.

In our experiments, we have taken the number of iterations $N$ to be 10000. The parameters of the various simulation runs are given in Table~\ref{tab-sim-param}.
To obtain an idea of the comparison between the different algorithms, we have considered a number of variations in the parameters. The value of $n$ has been chosen
to be as low as 10 to a moderate value of 100, while the value of $S$ has been taken to be as small as 30 to as large as 3000. While we report results for the
values of parameters shown in Table~\ref{tab-sim-param}, we have also experimented with various other values. The results in all such cases turned out to be very similar 
to the results that we report here. 

Table~\ref{tab-res} provides a summary of the results that we obtained from the simulations.
The columns of the table list the pro-rata method along with the JD and the WS methods.
The rows correspond to the various simulation runs whose parameters are given in Table~\ref{tab-sim-param}. For a row starting with $L_1$, all entries in the
corresponding row are with respect to the $L_1$ metric. Similarly for a row starting with $L_2$, all entries in the corresponding row are
with respect to the $L_2$ metric. Each entry in the table is a pair of numbers. 
Suppose $(x_1,x_2)$ appears in the column headed by algorithm $\mathcal{A}$ in a row corresponding to simulation number $s$ for the $L_1$ metric.
The value $x_1$ is the percentage of times that $\ell_{1,\mathcal{A}}$ came out to be lower than $\ell_{1,\mathcal{P}}$ (where $\mathcal{P}$ denotes
the pro-rata method) in simulation number $s$, while
the value $x_2$ is the percentage of times that $\ell_{1,\mathcal{A}}$ came out to be the minimum among all the three methods.
For example, the pair $(100.00,99.31)$ appearing under the column headed by WS in row labelled Sim~1 and starting with $L_1$ indicates that
the WS method is $L_1$-better than the pro-rata method in 100\% of the cases (i.e., in all the iterations of Sim~1); further, with respect to the $L_1$ metric, 
in 99.31\% of the iterations in Sim~1, the WS method provides the closest approximation to the ideal allocation among all the three methods. 
A similar interpretation holds for a pair of values appearing in a row starting with $L_2$, with the only change being that the $L_1$ metric is
replaced by the $L_2$ metric. Note that for each pair under the column headed pro-rata, the first entry is a `-', since it is not meaningful to compare 
pro-rata method with itself. Also, it is possible that in a particular iteration, the distances of two of the methods to the ideal are both minimum; so
the sum of the percentages of cases for which the different methods are minimum can be greater than 100. 

The simulation results bring out two important issues.
\begin{enumerate}
	\item Both the JD and the WS methods are better than the pro-rata method for an overwhelming number of cases.
	\item Among the three algorithms, the WS method provides the closest approximation to the ideal allocation in most of the cases.
\end{enumerate}
Consequently, the simulations provide sufficient evidence for stock exchanges to seriously consider the adoption of the Webster/Saint-Lagu\"{e} based order matching 
algorithm as a replacement of the pro-rata order matching algorithm.


\begin{table}
\begin{subtable}{0.5\textwidth}
	\centering
	{\scriptsize
		\begin{tabular}{|l|r|r|r|r|r|r|}
			\cline{2-7}
			\multicolumn{1}{c|}{} & \multicolumn{1}{c|}{$n$} & \multicolumn{1}{c|}{$m$} & 
			\multicolumn{1}{c|}{$M$} & \multicolumn{1}{c|}{$\mu$} & \multicolumn{1}{c|}{$\sigma$} & \multicolumn{1}{c|}{$S$} \\ \hline
			Sim~1 & 20 & 1 & 1000 & 500 & 400 & 50 \\ \hline 
			Sim~2 & 200 & 1 & 1000 & 500 & 400 & 30 \\ \hline 
			Sim~3 & 10 & 1 & 10000 & 5000 & 3000 & 300 \\ \hline 
			Sim~4 & 100 & 1 & 10000 & 5000 & 3000 & 300 \\ \hline 
			Sim~5 & 100 & 1 & 10000 & 5000 & 3000 & 3000 \\ \hline 
		\end{tabular}
		}
		\caption{Parameters of the various simulation runs. \label{tab-sim-param} }
\end{subtable}
\begin{subtable}{0.5\textwidth}
	\centering
	{\scriptsize
		\begin{tabular}{|l|l|r|r|r|}
			\cline{3-5} 
			\multicolumn{2}{c|}{} & \multicolumn{1}{c|}{pro-rata} & \multicolumn{1}{c|}{JD} & \multicolumn{1}{c|}{WS} \\ \hline
			\multirow{2}{*}{Sim~1} & 
			  $L_1$ & (-, 0.00) & (96.08, 4.79) & (100.00, 99.31) \\ \cline{2-5}
			& $L_2$ & (-, 0.01) & (95.99, 0.01) & ( 99.99, 99.30) \\ \hline
			\multirow{2}{*}{Sim~2} & 
			  $L_1$ & (-, 0.00) & (100.00, 100.00) & (100.00, 100.00) \\ \cline{2-5}
			& $L_2$ & (-, 0.00) & (100.00, 100.00) & (100.00, 100.00) \\ \hline
			\multirow{2}{*}{Sim~3} & 
			  $L_1$ & (-, 0.42) & (91.11, 25.58) & (99.65, 97.09) \\ \cline{2-5}
			& $L_2$ & (-, 0.46) & (90.89, 25.57) & (99.57, 97.02) \\ \hline
			\multirow{2}{*}{Sim~4} & 
			  $L_1$ & (-, 0.00) & (100.00, 0.00) & (100.00, 100.00) \\ \cline{2-5}
			& $L_2$ & (-, 0.00) & (100.00, 0.00) & (100.00, 100.00) \\ \hline
			\multirow{2}{*}{Sim~5} & 
			  $L_1$ & (-, 0.00) & (100.00, 0.00) & (100.00, 100.00) \\ \cline{2-5}
			& $L_2$ & (-, 0.00) & (100.00, 0.00) & (100.00, 100.00) \\ \hline
		\end{tabular}
		}
		\caption{Summary of simulation results. \label{tab-res}}
\end{subtable}
	\caption{Simulation parameters and summary. \label{tab-sim} }
\end{table}



%\bibliographystyle{alpha}
%\bibliography{prop-alloc}
\documentclass[11pt]{article}

\setlength{\textwidth}{6.5in} \setlength{\textheight}{8.5in}\hoffset=-0.1in \voffset=-0.5in

\usepackage{subcaption,booktabs}

\usepackage{authblk}
\usepackage[english]{babel}
\usepackage[utf8]{inputenc}
\usepackage[T1]{fontenc}
\usepackage{amsmath,amsfonts,amsthm,amssymb}
\usepackage{mathtools}
\usepackage{graphicx}
\usepackage[table]{xcolor}
\usepackage{multirow}
\usepackage{xurl}
\usepackage{hyperref}
\usepackage{fullpage}
\usepackage[numbers]{natbib}
\hypersetup{
    colorlinks=true,
    linkcolor=red,
    filecolor=magenta,      
    urlcolor=blue,
    linkbordercolor = white
}

% Reducing the bibliography spacing
%\setlength{\bibsep}{0.0pt}
\usepackage{enumitem}
%\setlength{\itemsep}{0em}

\newtheorem{theorem}{Theorem}
\newtheorem{proposition}[theorem]{Proposition}
\newtheorem{example}{Example}%
\newtheorem{remark}{Remark}%
\newtheorem{definition}{Definition}%

\newcommand{\sym}[1]{{\sf #1}}

% Algorithms
%\usepackage[margin=3cm]{geometry}
\usepackage{algorithm2e}
\RestyleAlgo{ruled}
%% This is needed if you want to add comments in
%% your algorithm with \Comment
\SetKwComment{Comment}{/* }{ */}

% Colours to mark portions of text
\newcommand{\red}[1]{\textcolor{red}{#1}}
\newcommand{\blue}[1]{\textcolor{blue}{#1}}

\title{On Using Proportional Representation Methods as Alternatives to Pro-Rata Based Order Matching Algorithms in Stock Exchanges}
%\titlerunning{Proportional Representation Based Order Matching}
\author[1]{Sanjay Bhattacherjee}
%\affil[1]{School of Computing, University of Kent, CT2 7NF, United Kingdom, email: s.bhattacherjee@kent.ac.uk}
\affil[1]{
	Institute of Cyber Security for Society and School of Computing \authorcr 
	Keynes College, University of Kent \authorcr 
	CT2 7NP, United Kingdom \authorcr 
	email: s.bhattacherjee@kent.ac.uk \authorcr
	ORCID: 0000-0002-3367-6192}
\author[2]{Palash Sarkar\thanks{Corresponding author.}}
\affil[2]{
	Indian Statistical Institute \authorcr 
	203, B.T. Road, Kolkata \authorcr
	India 700108 \authorcr 
	email: palash@isical.ac.in \authorcr
	ORCID: 0000-0002-5346-2650}

\date{\today}

\begin{document}

\maketitle


\begin{abstract}
	The main observation of this short note is that methods for determining proportional representation in electoral systems may be suitable as alternatives to 
	the pro-rata order matching algorithm used in stock exchanges. Our simulation studies provide strong evidence that the Jefferson/D'Hondt 
	and the Webster/Saint-Lagu\"{e} proportional representation methods provide order allocations which are closer to proportionality than the order allocations 
	obtained from the pro-rata algorithm. \\
	{\bf Keywords:} order matching algorithm, pro-rata algorithm, proportional representation, Jefferson/D'Hondt method, Webster/Saint-Lagu\"{e} method. \\
	{\bf JEL codes:} D49.
\end{abstract}

%Suppose there are n resting orders of sizes S_1 to S_n and total size S. Let p_i=S_i/S. Suppose the incoming order is of size K. His method performs K independent trials of the following random %experiment. Pick a number in 1 to n, where number i has probability p_i of being chosen; assign one unit of the incoming order to resting order number i. The rationale behind this is that on the average, %the assignment of orders will be the ideal distribution.

\newpage


\section{Introduction\label{sec:intro}}
Financial instruments are traded on stock exchanges. Traders place orders to buy and sell such instruments. 
An order specifies, among other things, the quantity or size, i.e. the number of units to be purchased or sold, and the price at which the order is to be executed.
The quoted price is required to be a multiple of a unit of price called tick. A stock exchange maintains an order book which records for each financial instrument
the corresponding list of orders. The orders quoting the same buying or selling price are placed at the same price level in the order book. 
%The number of price 
%levels of a financial instrument is called its book depth. The difference between the highest bid for buying and the lowest ask for selling is called the bid-ask spread 
%of the instrument.

Trading in a stock exchange occurs by executing orders. Buy orders for an instrument are matched with corresponding sell orders. 
Matching happens in units of the instrument. The quantity specified in an order may not be equal to the quantity of the counter-party order that it is matched with. An 
order is filled when all its units have been matched with one or more counter-party matching orders. Unmatched or partially filled orders rest in the 
order book waiting for new orders to be matched with.
Orders can be of different types. %See~\cite{CS2020} for an overview of various kinds of orders. 
A common and important type of order is a limit order which specify both the quantity and the price at which the order is to be executed.

Let us consider the order book entries of a financial instrument with $p$ price levels $L_1, \ldots, L_p$. Let the number of buy and sell limit orders at price $L_i$ be denoted as 
$b_i$ and $s_i$ respectively. We note that in the resting state of the order book for the instrument, either $b_i=0$ or $s_i=0$ for a price $L_i$. Otherwise, 
the opposing orders will be matched until there are no more buy or sell orders to be matched (i.e., either $b_i=0$ or $s_i=0$ or both). A new incoming order at price 
$L_i$ could be of the same type as the resting orders in which case it will be added to the list of resting orders or it could be a counter-party order in which case it 
will be matched with the resting orders.

Let us assume that at price level $L$ there are $n$ resting orders and the quantities of the orders are given by a vector $\mathbf{T} = (T_1, \ldots, T_n)$, 
where $T_i$ is a positive integer which represents the quantity of the $i$-th order. As discussed above, these $n$ orders are either all buy orders or all sell orders. 
Let $T = T_1 + \cdots + T_n$ be the total quantity of these orders. A new incoming counter-party order of quantity $S$ at 
price $L$ will be matched with the resting orders in $\mathbf{T}$. As a result, some or all of the resting orders may be executed. If $S \geq T$, the resting orders 
are all filled and can be fully executed. However, if $S < T$, not all resting orders can be completely filled. In this case, an order matching algorithm is required
to allocate portions of the incoming counter-party order to the $n$ resting orders. Henceforth, we will assume that the condition $S<T$ holds.

Formally, an order matching algorithm $\mathcal{M}(n, \mathbf{T}, S)$ takes as input the number $n$ of resting orders, the vector $\mathbf{T} = (T_1, \ldots, T_n)$ of the 
resting order quantities, and an incoming counter-party order of quantity $S$ with $0<S<T$. It 
outputs $\mathbf{S} = (S_1, \ldots, S_n)$ so that $S_i$ quantity of $T_i$ may be executed. In other words, $0\leq S_i \leq T_i$ and $S = S_1 + \ldots + S_n$. 

Two of the most common order matching algorithms are the price-time priority (also called first-come-first-served (FCFS) or
first-in-first-out (FIFO)) and the pro-rata methods (see for example~\cite{CMEMatchingAlgorithms,Pr11,CS2020,He22}). Both methods aim to achieve some kind of fariness
in the allocation of orders. In this work we focus on the pro-rata method.

The idea behind the pro-rata method is to distribute $S$ to the $n$ resting orders more or less in proportion to their fractions of the total order. 
So ideally the $i$-th resting order would receive $ST_i/T$ portion of the incoming counter-party order. This, however, has a problem. Financial instruments are
traded in indivisible atomic units. So if $ST_i/T$ is not an integer, then this amount of the order cannot be executed. The pro-rata order matching algorithm
adopts a two-step approach to this problem. In the first step, the algorithm assigns an amount $S_i^{\prime}=\lfloor ST_i/T\rfloor$ of the incoming counter-party order 
to the $i$-th resting 
order\footnote{For a real number $x$, $\lfloor x\rfloor$ denotes the greatest integer not greater 
than $x$, and $\lceil x\rceil$ denotes the least integer not less than $x$.}$^,$\footnote{Sometimes a simple modification to the first step is used. For 
example,~\cite{CMEMatchingAlgorithms} adopts the strategy in which if some $S_i^{\prime}$ turns out to be $1$, then it is instead set to 0.}.
This strategy consumes $S^{\prime}=S_1^{\prime}+\cdots+S_n^{\prime}$ units of the incoming counter-party order. In the second step, the remaining 
$S-S^{\prime}$ units are distributed to the resting orders based upon some strategy which could be the FCFS
strategy\footnote{With the second step as FCFS, the overall method is a hybrid of pro-rata and FCFS. Other hybrid combinations of
FCFS and pro-rata have been proposed. See for example~\cite{BCS15}}.

In this note, we raise two questions.
\begin{enumerate}
	\item How well does the pro-rata order matching algorithm achieve its goal of distributing the incoming order to the resting orders in proportion to their
		fractions of the total order?
	\item Are there other algorithms which perform better than the pro-rata algorithm in achieving proportionality?
\end{enumerate}
To answer the above questions, we need a measure to assess the performance of an order matching algorithm in achieving proportionality. For $n$ resting
orders, with $\mathbf{T}=(T_1,\ldots,T_n)$ the vector of quantities of the resting orders, and $S$ the size of the incoming order, the ideal allocation, or
the ideal proportional distribution is given by the vector 
\begin{eqnarray}\label{eqn-ideal}
\mathbf{I}=(ST_1/T,ST_2/T,\ldots,ST_n/T). 
\end{eqnarray}
Suppose $\mathcal{A}$ is an order matching algorithm which on input $n$, $\mathbf{T}$ and $S$
produces the allocation vector $\mathbf{S}_{\mathcal{A}}=(S_1,\ldots,S_n)$ as output, where $S_i\leq T_i$ for $i=1,\ldots,n$, and $S_1+\cdots+S_n=S$. 
The distance between the vectors $\mathbf{I}$ and $\mathbf{S}_{\mathcal{A}}$ is a measure of the performance of the algorithm $\mathcal{A}$ in achieving
proportionality. The closer $\mathbf{S}_{\mathcal{A}}$ is to $\mathbf{I}$, the better is the performance of $\mathcal{A}$ in achieving proportionality. 

We consider two standard measures of distance between two vectors, namely the $L_1$ and the $L_2$ metrics defined as follows.
\begin{eqnarray*}
	L_1(\mathbf{S}_{\mathcal{A}},\mathbf{I}) = \sum_{i=1}^n \mid S_i - ST_i/T \mid, & &  L_2(\mathbf{S}_{\mathcal{A}},\mathbf{I}) = \sum_{i=1}^n (S_i - ST_i/T)^2.
\end{eqnarray*}
Using these two metrics, we can quantifiably answer the first question posed above. The metrics also provide a method to address the second question.
For an order matching algorithm $\mathcal{A}$, let $\ell_{1,\mathcal{A}}$ and $\ell_{2,\mathcal{A}}$ denote
$L_1(\mathbf{S}_{\mathcal{A}},\mathbf{I})$ and $L_2(\mathbf{S}_{\mathcal{A}},\mathbf{I})$ respectively. For two order matching algorithms $\mathcal{A}_1$
and $\mathcal{A}_2$, we say that $\mathcal{A}_1$ is $L_1$-better (resp. $L_2$-better) than $\mathcal{A}_2$ if 
$\ell_{1,\mathcal{A}_1}<\ell_{1,\mathcal{A}_2}$ (resp. $\ell_{2,\mathcal{A}_1}<\ell_{2,\mathcal{A}_2}$). In other words, $\mathcal{A}_1$ is 
$L_1$-better (resp. $L_2$-better) than $\mathcal{A}_2$, if its output is closer to the ideal allocation with respect to the $L_1$ (resp. $L_2$) metric.
Using the above terminology, we can rephrase the second question as follows. 
\begin{quote}
Is there an order matching algorithm $\mathcal{A}$ which is better than
the pro-rata order matching algorithm with respect to either or both of the $L_1$ and $L_2$ metrics?
\end{quote}

\subsection{Seat Distribution in Electoral Systems \label{subsec-elec} }
To answer the above question, we visit the literature on proportional representation in electoral systems which is far removed from the stock exchanges and
more generally the financial world. 
Proportional representation is the most common kind of electoral system where the seats are not contested individually. Instead, the total number of seats is allocated 
to the contesting parties in proportion to the number of votes they have won in the election.  Let us consider an election contested by $n$ parties over $K$ seats that are 
distributed using a proportional representation method. Let $V_j$ denote the number of votes won by party $j \in [1, \ldots, n]$ in the election. The electoral output 
is denoted by the vector $\mathbf{V} = (V_1, \ldots, V_n)$, and the total number of votes cast in the election is $V = V_1 + \cdots + V_n$. Suppose the total number
of seats to be distributed among the parties is $K$. A proportional representation method determines the seat allocation vector 
$\mathbf{K}=(K_1,\ldots,K_n)$ where $K_i$ is the number of seats allocated to the $i$-th party, and $K_1+\cdots+K_n=K$. Typically, the total number of seats $K$ is much 
smaller than the total number of votes $V$, i.e. $K<V$. Further, it is reasonable to assume that in practice the number of seats allocated to the 
$i$-th party is at most the number of votes received by the party, i.e. $K_i\leq V_i$. 

Formally, a proportional representation method is an algorithm $\mathcal{A}(n, \mathbf{V}, K)$ which takes as input the number $n$ of parties, the vote count vector 
$\mathbf{V} = (V_1, \ldots, V_n)$, and the number $K$ of seats to be distributed, where $0<K<V$. It outputs the seat allocation vector $\mathbf{K} = (K_1, \ldots, K_n)$
such that $0\leq K_i\leq V_i$ and $K_1+\cdots+K_n=K$. 

From the above description, it becomes clear that the goal of allocating an incoming counter-party order to resting orders in proportion to the sizes of the resting orders 
is the same as the goal of assigning a fixed number of seats to several contesting parties in proportion to the number of votes obtained by these parties. The correspondence becomes 
clear by identifying the size $T_i$ of the $i$-th order with the number of votes $V_i$
received by the $i$-th party, the size $S$ of the incoming counter-party order with the total number of seats $K$, and the quantity $S_i$ of the $i$-th order
that is filled with the number of seats $K_i$ alloted to the $i$-th party. Having identified this correspondence, any algorithm for proportional representation of seats
in an electoral system becomes a potential candidate for use as an order matching algorithm by a stock exchange for proportional fulfillment of orders. The 
identification of proportional representation methods as possible substitutes for the pro-rata order matching algorithm is the key observation of the present note.
Not all proportional representation methods, however, are suitable for use as order matching algorithms. Some methods may require certain conditions to be 
applied which cannot be expected to hold in the context of order matching. We point out some such examples in Section~\ref{sec:order-matching}.

There is a large literature on electoral systems in general and proportional representation methods in particular. We refer the reader to~\cite{Herron2018,Norris2004}
for elaborate discussions on these topics. A number of proportional representation methods have been proposed in the context of electoral systems. 
These can be divided into two types, the highest averages method and the largest remainder method. 
The most
well known of the highest averages method are the Jefferson/D'Hondt (JD) and the Webster/Sainte-Lagu\"{e} (WS) methods. Both of these methods continue to be of active interest.
See for example~\cite{HM08,Me19,FSS20,GF17}. Among the largest remainder method, the two well known methods are the Hare and the Droop methods. In fact, the
Hare method has the attractive property that it minimises the $L_1$-distance to the ideal allocation. However, the largest remainder methods suffer from 
certain paradoxes (see Section~\ref{sec:order-matching}), which make them unsuitable for use as order matching algorithms. 

In the present context, both JD and the WX methods are well suited to be used as order matching algorithms in stock exchanges. We report simulation studies comparing
the pro-rata, the JD and the WS methods. Such studies show that both the JD and the WS methods are both $L_1$ and $L_2$-better than the pro-rata method. Among the
three methods, the allocation determined by the WS method turns out to be the closest to the ideal allocation in an overwhelming number of cases for both $L_1$ and $L_2$ metrics.
This provides sufficient evidence to seriously consider the adoption of Webster/Sainte-Lagu\"{e} method based order matching by stock exchanges.

\subsection{Procedural Fairness \label{subsec-fair} }
A recent paper by Hersch~\cite{He22} investigated the issue of procedural fairness of order allocation methods. In the paper, it was argued that
both the FIFO and the pro-rata are fair in principle, but not in practice. It was pointed out that the main disadvantage of pro-rata is the requirement
of the second step ``requiring exchanges to introduce secondary matching rules that can be gamed''. 

An alternative method called the random selection for service (RSS) method was proposed. 
Given the vector $\mathbf{T}=(T_1,\ldots,T_n)$ of resting orders, the RSS method defines a probability distribution $\pi$ over $\{1,\ldots,n\}$, where 
$\pi$ associates probability $T_i/T$ to $i$, for $i=1,\ldots,n$, 
Suppose the incoming counter-party order consists of $S$ units. Allocation is done by repeating the following procedure $S$ times:
an independent random $i$ is drawn from $\{1,\ldots,n\}$ following the probability distribution $\pi$ and
one unit is alloted to the $i$-th resting order. At the end of the procedure, let $S_i$ be the number of units alloted to the $i$-th resting order so that the
final allotment is $\mathbf{S}=(S_1,\ldots,S_n)$ satisfying $S=S_1+\cdots+S_n$.
It was argued by Hersch~\cite{He22} that the RSS method is fair in both principle and practice. Below we revisit this method and point out a crucial difference between principle
and practice.

Given $\mathbf{T}=(T_1,\ldots,T_n)$ and $S$, suppose the RSS method is executed $\Gamma$ times and for $\gamma=1,\ldots,\Gamma$, let
the allotment of $\gamma$-th execution be $\mathbf{S}_{\gamma}=(S_{\gamma,1},\ldots,S_{\gamma,n})$. For $i=1,\ldots,n$, let 
$\widehat{S}_i = (S_{1,i}+\cdots+S_{\Gamma,i})/\Gamma$, i.e. $\widehat{S}_i$ is the average allotment to the $i$-th resting order computed over all the
$\Gamma$ trials. The law of large numbers assures us that asymptotically, i.e. as $\Gamma$ goes to infinity, the average allotment $\widehat{S}_i$
tends to $ST_i/T$ which is equal to the $i$-th component of the ideal allocation vector $\mathbf{I}$ (see~\eqref{eqn-ideal}). So in principle, Hersch~\cite{He22} implicitly
considers achieving allocation close to the ideal allocation vector $\mathbf{I}$ to be procedurally fair. To this extent, Hersch's objective and ours coincide.
Additionally, our use of the $L_1$ and $L_2$ metrics to measure deviation from the ideal allocation vector can be considered to be a quantification of procedural fairness.
As such it expands the theoretical framework for studying procedural fairness of the order allocation methods.

From a practical point of view, however, the RSS method has a significant shortcoming. The law of large numbers applies in an asymptotic contex, i.e. as
$\Gamma$ goes to infinity. In practice, given $\mathbf{T}=(T_1,\ldots,T_n)$ and $S$, a stock exchange will execute the RSS method exactly once to obtain a single allocation 
vector $\mathbf{S}=(S_1,\ldots,S_n)$. In other words, in practice the value of $\Gamma$ will be 1. 
The law of large numbers does not say anything about the value obtain in a single execution. In particular,
the $S_i$'s obtained after a single execution of RSS can be any value in the set $\{0,\ldots,S\}$. Considering a particular example with $n=2$, $\mathbf{T}=(10,90)$ and
$S=10$, Hersch~\cite{He22} provides probabilities that the $S_i$'s can take certain values: the probability that $S_1\geq 1$ (resp. $S_1=10$) is about 
0.88 (resp. $7\times 10^{-6}$); the probability that $S_2=20$ (resp. $S_2\geq 10$) is about 0.12 (resp. approaches 1). These probabilities, however, do not
enlighten us about the concrete values of the $S_i$'s after a single execution. In particular, the probability that $S_2\geq 10$ approaches 1 suggests an
asymptotic approach, where the frequentist view of probability is taken. To interpret such probabilities, one again needs to consider a large number $\Gamma$
of trials of the RSS method and consider the average allocation over all the $\Gamma$ trials. 

To test the practical efficacy of the RSS method, we have run experiments with the method. It turns out that the allocation vector obtained by the RSS method
has a very large deviation from the ideal allocation vector in terms of both the $L_1$ and the $L_2$ metrics. Particular examples are provided in Section~\ref{sec-sim-res}.
By the above explanation, this observation is not surprising.

As mentioned earlier, according to Hersch, the main disadvantage of the pro-rata method is the use of secondary matching rules in the second step of the method
which leads to the possibility of gaming. The proportional representation based order allocation method that we introduce does not require any such secondary
matching rules which can be gamed. 
So our proposal overcomes the disadvantage of the pro-rata method pointed out by Hersch. In terms of procedural fairness as measured by
distance to the ideal allocation vector, our simulation studies show that proportional representation based order allocation outperforms the pro-rata method.

\section{Proportional Representation Methods \label{sec:order-matching}}

There are many different proportional representation methods (see~\cite{Herron2018,Norris2004}). Below we describe two well known families of proportional representation 
methods, namely the highest averages or the divisor method, and the highest remainder method. 

\paragraph{Highest averages method.}
Recall the setting described in Section~\ref{subsec-elec}, where $V_i$ is the number of votes received by the $i$-th party and $K$ is the total number of available seats. 
The goal is to determine $K_i$ which is the number of seats alloted to the $i$-th party.
Let $f: \mathbb{Z}^{+} \cup \{0\} \to \mathbb{R}$ be a function from the non-negative integers to the reals. Let $V=V_1+\cdots+V_n$ and $v_i=V_i/V$. 
In the highest averages method, seats are allotted iteratively. The seat distribution algorithm goes through $K$ iterations and in each iteration exactly one seat 
is alloted to one of the parties. Initially, the algorithm sets $K_1=K_2=\cdots=K_n=0$. For $k$ from $1$ to $K$, in the $k$-th iteration
the algorithm determines $j=\arg\max\{v_i/f(K_i): i=1,\ldots,n\}$ and increments $K_j$ by one. After $K$ iterations, the final values of $K_1,\ldots,K_n$ are the numbers of
seats alloted to the various parties. Various different methods arise from the different definitions of $f$. The definitions of $f$ for the well known
Jefferson/D'Hondt and the Webster/Sainte-Lagu\"{e} methods are shown in Table~\ref{tab:highest-averages-methods}.

\begin{table}[!htb]
    \centering
    \begin{tabular}{l|l}
    \hline
        Method name & $f(t)$ \\
    \hline
%	    Adam (Ad) & $\lceil t \rceil$ \\
%	    Danish (Da) & $t/(t+(1/3))$ \\
%	    Dean (De) & ${t(t+1)}/{(t+0.5)}$ \\
%	    Huntington/Hill (HH) & $\sqrt{t(t+1)}$ \\
	    Jefferson/D'Hondt (JD) & $(t+1)$ \\
	    Webster/Sainte-Lagu\"{e} (WS) & $(t+0.5)$ \\
    \hline
    \end{tabular}
	\caption{The functions used for two important methods of proportional representation.  \label{tab:highest-averages-methods}}
\end{table}

Apart from the JD and the WS methods, there are a number of other proportional representation methods, such as Dean's, Adam's, Huntington/Hill and the Danish methods.
(See~\cite{Wiki-division} for a compact description of these methods.)
These four methods require a positivity constraint to be satisfied which is not required in either the JD or the WS methods. They initially allocate one seat to each of the
contesting parties (i.e., they start with $K_1=\cdots=K_n=1$ instead of starting with $K_1=\cdots=K_n=0$) and then employ the highest averages method described above. 
As a result, at the end of the allocation each party has at least one seat. The 
definitions of the function $f$ for these four methods are different from the definitions of the function corresponding to the JD and the WS methods. Importantly, in 
these four methods, the function $f$ satisfies the condition $f(0)=0$. 
Consequently, the function cannot be evaluated unless the present number of seats allocated to a party is at least 1. This constraint is not present for
the JD and the WS methods. Note that the constraint of allocating at least one seat to each party requires the number of seats to be at least as large as the
number of parties.

In the context of order matching, the feature of assigning at least one seat to each party will translate to assiging at least one unit of the 
incoming counter-party order to each of the resting orders. This requires the size of the incoming counter-party order to be at least the number of resting orders,
i.e. $S\geq n$. Such a condition cannot be imposed in general, since there is no control over the size $S$ of the incoming counter-party order. More generally,
the principle of assigning at least one unit of the incoming order to each of the resting orders does not appear to have any justification in the context
of stock exchanges. On the contrary,
some stock exchanges follow the rule that if the order unit determined by the pro-rata method is 1, then this is instead set to 0~\cite{CMEMatchingAlgorithms}.
The principle of at least one unit for each resting order may not be welcome by such exchanges. Due to this reason as well as the unimplementable constraint
of $S\geq n$, in this paper we do not consider the proportional representation methods which follow the principle of assigning at least one seat to each party.

\paragraph{Largest remainder method.}
The method uses a parameter called the quota $Q$. The seat allocation is done in two phases. In the first phase, the $i$-th party is allocated $\lfloor V_i/Q\rfloor$ 
seats. Let $R_i=V_i/Q - \lfloor V_i/Q\rfloor$ be the remainder corresponding to the $i$-th party. Suppose after the first phase $k$ seats remain unallocated. 
In the second phase, the parties with the $k$ largest remainders are each allocated one seat. Let $K_i$ be the number of seats allocated to the $i$-th party at the
end of the second round. The method ensures that $K_1+\cdots+K_n=K$. 
Various methods arise by choosing an appropriate value of the quota $Q$. The Hare quota chooses $Q$ to be equal to $V/K$, while the Droop quota chooses
$Q$ to be equal to $1 + \lfloor V/(1+K) \rfloor$. Other choices for $Q$ have been 
proposed\footnote{\url{https://en.wikipedia.org/wiki/Largest_remainder_method}}. 

The largest remainder method satisfies the quota rule, i.e.  the condition $\lfloor K V_i/V\rfloor \leq K_i \leq \lceil K V_i/V\rceil$.
It is known that the largest remainder method with the Hare quota minimises the Loosemore-Hanby index\footnote{\url{https://en.wikipedia.org/wiki/Loosemore\%E2\%80\%93Hanby_index}}
which is equivalent to minimising the $L_1$-distance from the ideal proportional seat allocation $(KV_1/V,KV_2/V,\ldots,KV_n/V)$. 

\paragraph{Paradoxes.}
Balinski and Young~\cite{BY01} identified several paradoxes of proportional representation. Consider two possible vector of votes and the total number of seats
with $(V_1,\ldots,V_n)$ and $K$ be one of the scenarios and $(V_1^{\prime},\ldots,V_n^{\prime})$ and $K^{\prime}$. Let $(K_1,\ldots,K_n)$ and
$(K_1^{\prime},\ldots,K_n^{\prime})$ be the corresponding seat allocation vectors. 
\begin{itemize}
	\item The Alabama paradox is the following: $V_i=V_i^{\prime}$ for $i=1,\ldots,n$
		and $K^{\prime}>K$, but there is a $j\in \{1,\ldots,n\}$ such that $K_j>K_j^{\prime}$. 
		In other words, the votes polled remain the same and the total number of seats has gone up, but the number of seats allocated to a party has gone down.
	\item The population paradox is the following: $K=K^{\prime}$, $V_i^{\prime}\geq V_i$, and for some $j,k\in\{1,\ldots,n\}$, 
		$V_j^{\prime}-V_j > V_k^{\prime}-V_k$, but $K_j^{\prime}<K$. In other words,
		the number of seats remains the same and the additional votes polled by one party is more than another, but the former party is allocted fewer seats.
\end{itemize}
The Balinski-Young theorem states that any method of apportionment which satisfies the quota rule will necessarily suffer from either the Alabama or the population
paradox. 

In the context of order matching, both the Alabama and the population paradoxes are problematic. The Alabama paradox translates to the following.
The quantities of the resting orders remain the same and an increase occurs in the incoming counter-party order, yet the allocation to a particular resting order goes down. 
The population paradox translates to the following. The size of the incoming counter-party order remains the same, and the size of a particular resting order increases
by an amount which is more than the increase in the size of another resting order, but the units allocated to the former resting order goes down.
Both of these are counterintuitive and hard to explain to the stakeholders of a stock exchange. Due to these paradoxes, the largest remainder method with Hare quota
is unsuitable for use as order matching algorithm even though it minimises the distance to the ideal allocation vector.

The highest averages methods are free of these paradoxes. As a consequence, by the Balinski-Young theorem, they violate the quota rule. It is, however, known that
the while the WS method in principle violates the quota rule, it does so rarely. 

Based on the above discussion, we consider only the JD and the WS method in our simulation studies.

\section{Simulation and Results \label{sec-sim-res}}
%We have implemented the pro-rata, the JD and the WS methods.

Given $n$, $\mathbf{T}=(T_1,\ldots,T_n)$ and $S$, the pro-rata allocation takes place in two steps. In the first step, the vector 
$$\mathbf{S}_{\mathcal{P}}^{\prime}=(S_1^{\prime},S_2^{\prime},\ldots,S_n^{\prime})=(\lfloor ST_1/T \rfloor, \lfloor ST_2/T \rfloor, \ldots, \lfloor  ST_n/T\rfloor)$$ 
is computed. In the second step, the remaining $S-S^{\prime}$ (where $S^{\prime}=S_1^{\prime}+S_2^{\prime}+\cdots+S_n^{\prime}$) quantity of the incoming order 
is allocated using the first-come-first-served strategy. In our implementation, we have used a modified version of the first-come-first-served-strategy where we 
have prioritised smaller orders as follows: allocate one unit to all orders in the first-come-first-served manner which got zero allocation in the first step, 
next allocate one unit to all orders in the first-come-first-served manner which was alloted one unit in the first step, and so on until all the $S-S^{\prime}$ units left
over from the first step are exhausted. The second step increases the allocation to any resting order by at most one unit.
%and we also check and ensure that alloted number of units to the $i$-th resting order is not greater than $ST_i/T$. 
The rationale for using the modified first-come-first-served strategy is to provide some benefit to smaller orders. 
%Since $S-S^{\prime}$ is at most $n$, the overall effect of this modified strategy has a negligible effect on the values of $\ell_{1,\mathcal{P}}$ and $\ell_{2,\mathcal{P}}$, 
%where $\mathcal{P}$ denotes the pro-rata method. 

The pro-rata method clearly takes $O(n)$ time. The highest averages method (of which the JD and the WS methods are special cases)
described in Section~\ref{sec:order-matching} can be implemented in time $O(n+K\log n)$. (By the identification of the size $S$ of the incoming order with the number $K$
of available seats, we have $O(n+K\log n)=O(n+S\log n)$.)
We briefly discuss how this can be done. Note that $O(n)$ time is required to initialise the allocation vector $K=(K_1,\ldots,K_n)$ to the all-zero vector. Next 
a max-heap data structure~\cite{AHU78} is built on the $n$ values $f_1=v_1/f(K_1), f_2=v_2/f(K_2),\ldots, f_n=v_n/f(K_n)$. This takes $O(n)$ time. The heap data structure 
stores the maximum of the $f_i$'s on the top. In each of the $K$ iterations,
exactly one $K_i$ is incremented, so exactly one $f_i$ is modified and the other $f_j$'s remain unchanged. The heap data structure is updated so that the new maximum
gets to the top. This can be done in $O(\log n)$ time and makes the new maximum available for the next iteration. So the $K$ iterations of the highest averages
method take $O(K\log n)$ time. 
%Below we present simulation results which show that the JD and the WS algorithms achieve better proportionality. The trade-off is that both of these algorithms
%take more time than the pro-rata method.

To compare the performances of the different algorithms, we have performed simulation studies. The input to an order matching algorithm is the number of resting
orders $n$, the vector $\mathbf{T}=(T_1,\ldots,T_n)$, where $T_i$ is the size of the $i$-th order, and the size $S$ of the incoming counter-party order. 
In our simulations, we have randomly generated the values $T_1,\ldots,T_n$ using the following strategy. Fix two non-negative integers $m$ and $M$ with $m<M$. 
Let $\mu$ and $\sigma$ be positive real numbers which specify the normal ${\mathcal N}(\mu,\sigma)$ distribution. For each $i$ in $1$ to $n$, the following
procedure is performed: draw a sample from ${\mathcal N}(\mu,\sigma)$ and round to the nearest integer, repeat until the rounded value is in the range $[m,M]$; 
once the rounded value satisfies the range check, set $T_i$ to be equal to this rounded value.
After $n$ iterations, we obtain the random vector $\mathbf{T}=(T_1,\ldots,T_n)$ which is a simulated distribution of the resting orders. 
Note that all the samples are drawn {\em independently} from ${\mathcal N}(\mu,\sigma)$. Our rationale for
choosing the normal distribution is that in the absence of any other information, the sizes of the orders may be assumed to follow the normal distribution. If, on the
other hand, additional information is available, then it is possible to change the normal distribution to another distribution without affecting the rest of the
simulation. 

In Table~\ref{tab-ex}, we provide some examples of the order allocation vector $\mathbf{S}_{\mathcal{A}}$, where 
$\mathcal{A}$ is one of $\mathcal{P}$ (denoting the pro-rata method), RSS, JD, or WS method. 
The ideal allocation vector is $\mathbf{I}=(ST_1/T,\ldots,ST_n/T)$. 
%For the pro-rata method, we provide both the vectors $\mathbf{S}_{\mathcal{P}}^{\prime}$ (the output of the first step of the pro-rata method) and 
%$\mathbf{S}_{\mathcal{P}}$ (the final output of the pro-rata method). 
From the examples, we observe that for the RSS method, the $L_1$ and $L_2$ distances from the ideal allocation vector $\mathbf{I}$ are much larger than these
distances from the other methods. This is as expected (see Section~\ref{subsec-fair}) and highlights the impracticability of the RSS method. While the table
provides only three examples, we have obtained many other examples and the observation that the order allocation vector produced by the RSS method
is much farther away from the ideal allocation vector compared to the other methods holds in all the examples. In view of this, we do not consider the
RSS method any further in our simulation studies.

Among the three methods, i.e. the pro-rata method, the JD and the WS methods, note that in all the cases, the JD and the WS methods are both $L_1$-better and 
$L_2$-better than the pro-rata method. Comparing
$\ell_{1,\mathcal{P}}$ with $\ell_{1,{\rm JD}}$ and $\ell_{1,{\rm WS}}$ and $\ell_{2,\mathcal{P}}$ with $\ell_{2,{\rm JD}}$ and $\ell_{2,{\rm WS}}$, we find significant
difference in these values. So these examples suggest that the JD and the WS methods are significantly better than the pro-rata method with respect to the 
$L_1$ and $L_2$ metrics.

\begin{table}
\centering
{\scriptsize
	\begin{tabular}{|l|l|r|r|r|r|r|r|r|r|r|r|r|r|}
		\cline{13-14}
		\multicolumn{12}{c|}{} & $\ell_{1,\mathcal{A}}$ & $\ell_{2,\mathcal{A}}$ \\ \hline
		\multirow{5}{*}{Ex~1}
		& $\mathbf{T}$ & 209 & 727 & 746 & 808 & 995 & 204 & 598 & 773 & 979 & 899 & - & - \\ \cline{2-14}
		& $\mathbf{I}$ & 3.01 & 10.48 & 10.75 & 11.65 & 14.34 & 2.94 & 8.62 & 11.14 & 14.11 & 12.96 & - & - \\ \cline{2-14}
		%& $\mathbf{S}_{\mathcal{P}}^{\prime}$ & 3 & 10 & 10 & 11 & 14 & 2 & 8 & 11 & 14 & 12 & - & -\\ \cline{2-14}
		& $\mathbf{S}_{\mathcal{P}}$ & 4 & 11 & 11 & 11 & 14 & 3 & 9 & 11 & 14 & 12 & 4.39 & 2.94 \\ \cline{2-14}
		& $\mathbf{S}_{{\rm RSS}}$ & 4 & 7 & 17 & 11 & 18 & 3 & 5 & 10 & 15 & 10 & 23.69 & 89.86 \\ \cline{2-14}
		& $\mathbf{S}_{{\rm JD}}$ & 3 & 10 & 11 & 12 & 14 & 3 & 9 & 11 & 14 & 13 & 2.17 & 0.71 \\ \cline{2-14}
		& $\mathbf{S}_{{\rm WS}}$ & 3 & 10 & 11 & 12 & 14 & 3 & 9 & 11 & 14 & 13 & 2.17 & 0.71 \\ \hline
		\multirow{5}{*}{Ex~2}
		& $\mathbf{T}$ & 1 & 655 & 307 & 138 & 647 & 48 & 625 & 382 & 95 & 424 & - & - \\ \cline{2-14}
		& $\mathbf{I}$ & 0.03 & 19.72 & 9.24 & 4.15 & 19.48 & 1.44 & 18.81 & 11.50 & 2.86 & 12.76 & - & - \\ \cline{2-14}
		%& $\mathbf{S}_{\mathcal{P}}^{\prime}$ & 0 & 19 & 9 & 4 & 19 & 1 & 18 & 11 & 2 & 12 & - & -\\ \cline{2-14}
		& $\mathbf{S}_{\mathcal{P}}$ & 1 & 19 & 10 & 5 & 19 & 2 & 18 & 11 & 3 & 12 & 6.54 & 4.79 \\ \cline{2-14}
		& $\mathbf{S}_{{\rm RSS}}$ & 0 & 21 & 15 & 2 & 21 & 1 & 17 & 8 & 4 & 11 & 19.41 & 61.91 \\ \cline{2-14}
		& $\mathbf{S}_{{\rm JD}}$ & 0 & 20 & 9 & 4 & 20 & 1 & 19 & 12 & 2 & 13 & 3.46 & 1.72 \\ \cline{2-14}
		& $\mathbf{S}_{{\rm WS}}$ & 0 & 20 & 9 & 4 & 19 & 1 & 19 & 12 & 3 & 13 & 2.69 & 0.95 \\ \hline
		\multirow{5}{*}{Ex~3}
		& $\mathbf{T}$ & 268 & 806 & 409 & 420 & 869 & 659 & 189 & 317 & 286 & 721 & - & - \\ \cline{2-14}
		& $\mathbf{I}$ & 5.42 & 16.30 & 8.27 & 8.50 & 17.58 & 13.33 & 3.82 & 6.41 & 5.78 & 14.58 & - & - \\ \cline{2-14}
		%& $\mathbf{S}_{\mathcal{P}}^{\prime}$ & 5 & 16 & 8 & 8 & 17 & 13 & 3 & 6 & 5 & 14 & - & -\\ \cline{2-14}
		& $\mathbf{S}_{\mathcal{P}}$ & 6 & 16 & 9 & 8 & 17 & 13 & 4 & 7 & 6 & 14 & 4.57 & 2.41 \\ \cline{2-14}
		& $\mathbf{S}_{{\rm RSS}}$ & 2 & 15 & 8 & 4 & 12 & 17 & 2 & 6 & 9 & 25 & 34.61 & 200.59 \\ \cline{2-14}
		& $\mathbf{S}_{{\rm JD}}$ & 5 & 17 & 8 & 8 & 18 & 13 & 4 & 6 & 6 & 15 & 3.86 & 1.69 \\ \cline{2-14}
		& $\mathbf{S}_{{\rm WS}}$ & 5 & 16 & 8 & 9 & 18 & 13 & 4 & 6 & 6 & 15 & 3.47 & 1.31 \\ \hline
	\end{tabular}
	\caption{Examples of simulation runs with $n=10$, $S=100$, $m=1$, $M=1000$, $\mu=500$ and $\sigma=400$. Here $\mathcal{P}$ is the pro-rata method. \label{tab-ex}}
}
\end{table}

A few examples do not provide sufficient evidence. It is required to consider many more examples. On the other hand, when there are a large number of examples,
it is not possible to visually inspect all such examples. So we have used a program to perform the comparison for the various simulation studies.
Since $\mathbf{T}$ is determined by $n$, $\mu$ and $\sigma$, the parameters for the simulations are the different values of $n$, $\mu$ and $\sigma$ as well as $S$. 
For a specific set of values of $n$, $\mu$, $\sigma$ and $S$, we have performed $N$ iterations of the simulation. In each iteration, we have computed 
the ideal allocation vector $\mathbf{I}=(ST_1/T,\ldots,ST_n/T)$, and the order allocation vector
$\mathbf{S}_{\mathcal{A}}=(S_1,\ldots,S_n)$ produced by the order matching algorithm $\mathcal{A}$, where $\mathcal{A}$ is one of pro-rata, the JD or the WS algorithms.
%The ideal proportional allocation vector in each iteration is $\mathbf{I}=(ST_1/T,\ldots,ST_n/T)$ and the output of algorithm $\mathcal{A}$ is 
%given by the vector $\mathbf{S}_{\mathcal{A}}=(S_1,S_2,\ldots,S_n)$. 
Next we computed the $L_1$ and $L_2$ distances of $\mathbf{S}_{\mathcal{A}}$ from $\mathbf{I}$ given by $\ell_{1,\mathcal{A}}$ and $\ell_{2,\mathcal{A}}$. 
In each of the $N$ iterations, we have compared the JD and the WS methods with the pro-rata method. After $N$ iterations, aggregate statistics are determined
for the particular simulation. Further details are given below.

In our experiments, we have taken the number of iterations $N$ to be 10000. The parameters of the various simulation runs are given in Table~\ref{tab-sim-param}.
To obtain an idea of the comparison between the different algorithms, we have considered a number of variations in the parameters. The value of $n$ has been chosen
to be as low as 10 to a moderate value of 100, while the value of $S$ has been taken to be as small as 30 to as large as 3000. While we report results for the
values of parameters shown in Table~\ref{tab-sim-param}, we have also experimented with various other values. The results in all such cases turned out to be very similar 
to the results that we report here. 

Table~\ref{tab-res} provides a summary of the results that we obtained from the simulations.
The columns of the table list the pro-rata method along with the JD and the WS methods.
The rows correspond to the various simulation runs whose parameters are given in Table~\ref{tab-sim-param}. For a row starting with $L_1$, all entries in the
corresponding row are with respect to the $L_1$ metric. Similarly for a row starting with $L_2$, all entries in the corresponding row are
with respect to the $L_2$ metric. Each entry in the table is a pair of numbers. 
Suppose $(x_1,x_2)$ appears in the column headed by algorithm $\mathcal{A}$ in a row corresponding to simulation number $s$ for the $L_1$ metric.
The value $x_1$ is the percentage of times that $\ell_{1,\mathcal{A}}$ came out to be lower than $\ell_{1,\mathcal{P}}$ (where $\mathcal{P}$ denotes
the pro-rata method) in simulation number $s$, while
the value $x_2$ is the percentage of times that $\ell_{1,\mathcal{A}}$ came out to be the minimum among all the three methods.
For example, the pair $(100.00,99.31)$ appearing under the column headed by WS in row labelled Sim~1 and starting with $L_1$ indicates that
the WS method is $L_1$-better than the pro-rata method in 100\% of the cases (i.e., in all the iterations of Sim~1); further, with respect to the $L_1$ metric, 
in 99.31\% of the iterations in Sim~1, the WS method provides the closest approximation to the ideal allocation among all the three methods. 
A similar interpretation holds for a pair of values appearing in a row starting with $L_2$, with the only change being that the $L_1$ metric is
replaced by the $L_2$ metric. Note that for each pair under the column headed pro-rata, the first entry is a `-', since it is not meaningful to compare 
pro-rata method with itself. Also, it is possible that in a particular iteration, the distances of two of the methods to the ideal are both minimum; so
the sum of the percentages of cases for which the different methods are minimum can be greater than 100. 

The simulation results bring out two important issues.
\begin{enumerate}
	\item Both the JD and the WS methods are better than the pro-rata method for an overwhelming number of cases.
	\item Among the three algorithms, the WS method provides the closest approximation to the ideal allocation in most of the cases.
\end{enumerate}
Consequently, the simulations provide sufficient evidence for stock exchanges to seriously consider the adoption of the Webster/Saint-Lagu\"{e} based order matching 
algorithm as a replacement of the pro-rata order matching algorithm.


\begin{table}
\begin{subtable}{0.5\textwidth}
	\centering
	{\scriptsize
		\begin{tabular}{|l|r|r|r|r|r|r|}
			\cline{2-7}
			\multicolumn{1}{c|}{} & \multicolumn{1}{c|}{$n$} & \multicolumn{1}{c|}{$m$} & 
			\multicolumn{1}{c|}{$M$} & \multicolumn{1}{c|}{$\mu$} & \multicolumn{1}{c|}{$\sigma$} & \multicolumn{1}{c|}{$S$} \\ \hline
			Sim~1 & 20 & 1 & 1000 & 500 & 400 & 50 \\ \hline 
			Sim~2 & 200 & 1 & 1000 & 500 & 400 & 30 \\ \hline 
			Sim~3 & 10 & 1 & 10000 & 5000 & 3000 & 300 \\ \hline 
			Sim~4 & 100 & 1 & 10000 & 5000 & 3000 & 300 \\ \hline 
			Sim~5 & 100 & 1 & 10000 & 5000 & 3000 & 3000 \\ \hline 
		\end{tabular}
		}
		\caption{Parameters of the various simulation runs. \label{tab-sim-param} }
\end{subtable}
\begin{subtable}{0.5\textwidth}
	\centering
	{\scriptsize
		\begin{tabular}{|l|l|r|r|r|}
			\cline{3-5} 
			\multicolumn{2}{c|}{} & \multicolumn{1}{c|}{pro-rata} & \multicolumn{1}{c|}{JD} & \multicolumn{1}{c|}{WS} \\ \hline
			\multirow{2}{*}{Sim~1} & 
			  $L_1$ & (-, 0.00) & (96.08, 4.79) & (100.00, 99.31) \\ \cline{2-5}
			& $L_2$ & (-, 0.01) & (95.99, 0.01) & ( 99.99, 99.30) \\ \hline
			\multirow{2}{*}{Sim~2} & 
			  $L_1$ & (-, 0.00) & (100.00, 100.00) & (100.00, 100.00) \\ \cline{2-5}
			& $L_2$ & (-, 0.00) & (100.00, 100.00) & (100.00, 100.00) \\ \hline
			\multirow{2}{*}{Sim~3} & 
			  $L_1$ & (-, 0.42) & (91.11, 25.58) & (99.65, 97.09) \\ \cline{2-5}
			& $L_2$ & (-, 0.46) & (90.89, 25.57) & (99.57, 97.02) \\ \hline
			\multirow{2}{*}{Sim~4} & 
			  $L_1$ & (-, 0.00) & (100.00, 0.00) & (100.00, 100.00) \\ \cline{2-5}
			& $L_2$ & (-, 0.00) & (100.00, 0.00) & (100.00, 100.00) \\ \hline
			\multirow{2}{*}{Sim~5} & 
			  $L_1$ & (-, 0.00) & (100.00, 0.00) & (100.00, 100.00) \\ \cline{2-5}
			& $L_2$ & (-, 0.00) & (100.00, 0.00) & (100.00, 100.00) \\ \hline
		\end{tabular}
		}
		\caption{Summary of simulation results. \label{tab-res}}
\end{subtable}
	\caption{Simulation parameters and summary. \label{tab-sim} }
\end{table}



%\bibliographystyle{alpha}
%\bibliography{prop-alloc}
\documentclass[11pt]{article}

\setlength{\textwidth}{6.5in} \setlength{\textheight}{8.5in}\hoffset=-0.1in \voffset=-0.5in

\usepackage{subcaption,booktabs}

\usepackage{authblk}
\usepackage[english]{babel}
\usepackage[utf8]{inputenc}
\usepackage[T1]{fontenc}
\usepackage{amsmath,amsfonts,amsthm,amssymb}
\usepackage{mathtools}
\usepackage{graphicx}
\usepackage[table]{xcolor}
\usepackage{multirow}
\usepackage{xurl}
\usepackage{hyperref}
\usepackage{fullpage}
\usepackage[numbers]{natbib}
\hypersetup{
    colorlinks=true,
    linkcolor=red,
    filecolor=magenta,      
    urlcolor=blue,
    linkbordercolor = white
}

% Reducing the bibliography spacing
%\setlength{\bibsep}{0.0pt}
\usepackage{enumitem}
%\setlength{\itemsep}{0em}

\newtheorem{theorem}{Theorem}
\newtheorem{proposition}[theorem]{Proposition}
\newtheorem{example}{Example}%
\newtheorem{remark}{Remark}%
\newtheorem{definition}{Definition}%

\newcommand{\sym}[1]{{\sf #1}}

% Algorithms
%\usepackage[margin=3cm]{geometry}
\usepackage{algorithm2e}
\RestyleAlgo{ruled}
%% This is needed if you want to add comments in
%% your algorithm with \Comment
\SetKwComment{Comment}{/* }{ */}

% Colours to mark portions of text
\newcommand{\red}[1]{\textcolor{red}{#1}}
\newcommand{\blue}[1]{\textcolor{blue}{#1}}

\title{On Using Proportional Representation Methods as Alternatives to Pro-Rata Based Order Matching Algorithms in Stock Exchanges}
%\titlerunning{Proportional Representation Based Order Matching}
\author[1]{Sanjay Bhattacherjee}
%\affil[1]{School of Computing, University of Kent, CT2 7NF, United Kingdom, email: s.bhattacherjee@kent.ac.uk}
\affil[1]{
	Institute of Cyber Security for Society and School of Computing \authorcr 
	Keynes College, University of Kent \authorcr 
	CT2 7NP, United Kingdom \authorcr 
	email: s.bhattacherjee@kent.ac.uk \authorcr
	ORCID: 0000-0002-3367-6192}
\author[2]{Palash Sarkar\thanks{Corresponding author.}}
\affil[2]{
	Indian Statistical Institute \authorcr 
	203, B.T. Road, Kolkata \authorcr
	India 700108 \authorcr 
	email: palash@isical.ac.in \authorcr
	ORCID: 0000-0002-5346-2650}

\date{\today}

\begin{document}

\maketitle


\begin{abstract}
	The main observation of this short note is that methods for determining proportional representation in electoral systems may be suitable as alternatives to 
	the pro-rata order matching algorithm used in stock exchanges. Our simulation studies provide strong evidence that the Jefferson/D'Hondt 
	and the Webster/Saint-Lagu\"{e} proportional representation methods provide order allocations which are closer to proportionality than the order allocations 
	obtained from the pro-rata algorithm. \\
	{\bf Keywords:} order matching algorithm, pro-rata algorithm, proportional representation, Jefferson/D'Hondt method, Webster/Saint-Lagu\"{e} method. \\
	{\bf JEL codes:} D49.
\end{abstract}

%Suppose there are n resting orders of sizes S_1 to S_n and total size S. Let p_i=S_i/S. Suppose the incoming order is of size K. His method performs K independent trials of the following random %experiment. Pick a number in 1 to n, where number i has probability p_i of being chosen; assign one unit of the incoming order to resting order number i. The rationale behind this is that on the average, %the assignment of orders will be the ideal distribution.

\newpage


\section{Introduction\label{sec:intro}}
Financial instruments are traded on stock exchanges. Traders place orders to buy and sell such instruments. 
An order specifies, among other things, the quantity or size, i.e. the number of units to be purchased or sold, and the price at which the order is to be executed.
The quoted price is required to be a multiple of a unit of price called tick. A stock exchange maintains an order book which records for each financial instrument
the corresponding list of orders. The orders quoting the same buying or selling price are placed at the same price level in the order book. 
%The number of price 
%levels of a financial instrument is called its book depth. The difference between the highest bid for buying and the lowest ask for selling is called the bid-ask spread 
%of the instrument.

Trading in a stock exchange occurs by executing orders. Buy orders for an instrument are matched with corresponding sell orders. 
Matching happens in units of the instrument. The quantity specified in an order may not be equal to the quantity of the counter-party order that it is matched with. An 
order is filled when all its units have been matched with one or more counter-party matching orders. Unmatched or partially filled orders rest in the 
order book waiting for new orders to be matched with.
Orders can be of different types. %See~\cite{CS2020} for an overview of various kinds of orders. 
A common and important type of order is a limit order which specify both the quantity and the price at which the order is to be executed.

Let us consider the order book entries of a financial instrument with $p$ price levels $L_1, \ldots, L_p$. Let the number of buy and sell limit orders at price $L_i$ be denoted as 
$b_i$ and $s_i$ respectively. We note that in the resting state of the order book for the instrument, either $b_i=0$ or $s_i=0$ for a price $L_i$. Otherwise, 
the opposing orders will be matched until there are no more buy or sell orders to be matched (i.e., either $b_i=0$ or $s_i=0$ or both). A new incoming order at price 
$L_i$ could be of the same type as the resting orders in which case it will be added to the list of resting orders or it could be a counter-party order in which case it 
will be matched with the resting orders.

Let us assume that at price level $L$ there are $n$ resting orders and the quantities of the orders are given by a vector $\mathbf{T} = (T_1, \ldots, T_n)$, 
where $T_i$ is a positive integer which represents the quantity of the $i$-th order. As discussed above, these $n$ orders are either all buy orders or all sell orders. 
Let $T = T_1 + \cdots + T_n$ be the total quantity of these orders. A new incoming counter-party order of quantity $S$ at 
price $L$ will be matched with the resting orders in $\mathbf{T}$. As a result, some or all of the resting orders may be executed. If $S \geq T$, the resting orders 
are all filled and can be fully executed. However, if $S < T$, not all resting orders can be completely filled. In this case, an order matching algorithm is required
to allocate portions of the incoming counter-party order to the $n$ resting orders. Henceforth, we will assume that the condition $S<T$ holds.

Formally, an order matching algorithm $\mathcal{M}(n, \mathbf{T}, S)$ takes as input the number $n$ of resting orders, the vector $\mathbf{T} = (T_1, \ldots, T_n)$ of the 
resting order quantities, and an incoming counter-party order of quantity $S$ with $0<S<T$. It 
outputs $\mathbf{S} = (S_1, \ldots, S_n)$ so that $S_i$ quantity of $T_i$ may be executed. In other words, $0\leq S_i \leq T_i$ and $S = S_1 + \ldots + S_n$. 

Two of the most common order matching algorithms are the price-time priority (also called first-come-first-served (FCFS) or
first-in-first-out (FIFO)) and the pro-rata methods (see for example~\cite{CMEMatchingAlgorithms,Pr11,CS2020,He22}). Both methods aim to achieve some kind of fariness
in the allocation of orders. In this work we focus on the pro-rata method.

The idea behind the pro-rata method is to distribute $S$ to the $n$ resting orders more or less in proportion to their fractions of the total order. 
So ideally the $i$-th resting order would receive $ST_i/T$ portion of the incoming counter-party order. This, however, has a problem. Financial instruments are
traded in indivisible atomic units. So if $ST_i/T$ is not an integer, then this amount of the order cannot be executed. The pro-rata order matching algorithm
adopts a two-step approach to this problem. In the first step, the algorithm assigns an amount $S_i^{\prime}=\lfloor ST_i/T\rfloor$ of the incoming counter-party order 
to the $i$-th resting 
order\footnote{For a real number $x$, $\lfloor x\rfloor$ denotes the greatest integer not greater 
than $x$, and $\lceil x\rceil$ denotes the least integer not less than $x$.}$^,$\footnote{Sometimes a simple modification to the first step is used. For 
example,~\cite{CMEMatchingAlgorithms} adopts the strategy in which if some $S_i^{\prime}$ turns out to be $1$, then it is instead set to 0.}.
This strategy consumes $S^{\prime}=S_1^{\prime}+\cdots+S_n^{\prime}$ units of the incoming counter-party order. In the second step, the remaining 
$S-S^{\prime}$ units are distributed to the resting orders based upon some strategy which could be the FCFS
strategy\footnote{With the second step as FCFS, the overall method is a hybrid of pro-rata and FCFS. Other hybrid combinations of
FCFS and pro-rata have been proposed. See for example~\cite{BCS15}}.

In this note, we raise two questions.
\begin{enumerate}
	\item How well does the pro-rata order matching algorithm achieve its goal of distributing the incoming order to the resting orders in proportion to their
		fractions of the total order?
	\item Are there other algorithms which perform better than the pro-rata algorithm in achieving proportionality?
\end{enumerate}
To answer the above questions, we need a measure to assess the performance of an order matching algorithm in achieving proportionality. For $n$ resting
orders, with $\mathbf{T}=(T_1,\ldots,T_n)$ the vector of quantities of the resting orders, and $S$ the size of the incoming order, the ideal allocation, or
the ideal proportional distribution is given by the vector 
\begin{eqnarray}\label{eqn-ideal}
\mathbf{I}=(ST_1/T,ST_2/T,\ldots,ST_n/T). 
\end{eqnarray}
Suppose $\mathcal{A}$ is an order matching algorithm which on input $n$, $\mathbf{T}$ and $S$
produces the allocation vector $\mathbf{S}_{\mathcal{A}}=(S_1,\ldots,S_n)$ as output, where $S_i\leq T_i$ for $i=1,\ldots,n$, and $S_1+\cdots+S_n=S$. 
The distance between the vectors $\mathbf{I}$ and $\mathbf{S}_{\mathcal{A}}$ is a measure of the performance of the algorithm $\mathcal{A}$ in achieving
proportionality. The closer $\mathbf{S}_{\mathcal{A}}$ is to $\mathbf{I}$, the better is the performance of $\mathcal{A}$ in achieving proportionality. 

We consider two standard measures of distance between two vectors, namely the $L_1$ and the $L_2$ metrics defined as follows.
\begin{eqnarray*}
	L_1(\mathbf{S}_{\mathcal{A}},\mathbf{I}) = \sum_{i=1}^n \mid S_i - ST_i/T \mid, & &  L_2(\mathbf{S}_{\mathcal{A}},\mathbf{I}) = \sum_{i=1}^n (S_i - ST_i/T)^2.
\end{eqnarray*}
Using these two metrics, we can quantifiably answer the first question posed above. The metrics also provide a method to address the second question.
For an order matching algorithm $\mathcal{A}$, let $\ell_{1,\mathcal{A}}$ and $\ell_{2,\mathcal{A}}$ denote
$L_1(\mathbf{S}_{\mathcal{A}},\mathbf{I})$ and $L_2(\mathbf{S}_{\mathcal{A}},\mathbf{I})$ respectively. For two order matching algorithms $\mathcal{A}_1$
and $\mathcal{A}_2$, we say that $\mathcal{A}_1$ is $L_1$-better (resp. $L_2$-better) than $\mathcal{A}_2$ if 
$\ell_{1,\mathcal{A}_1}<\ell_{1,\mathcal{A}_2}$ (resp. $\ell_{2,\mathcal{A}_1}<\ell_{2,\mathcal{A}_2}$). In other words, $\mathcal{A}_1$ is 
$L_1$-better (resp. $L_2$-better) than $\mathcal{A}_2$, if its output is closer to the ideal allocation with respect to the $L_1$ (resp. $L_2$) metric.
Using the above terminology, we can rephrase the second question as follows. 
\begin{quote}
Is there an order matching algorithm $\mathcal{A}$ which is better than
the pro-rata order matching algorithm with respect to either or both of the $L_1$ and $L_2$ metrics?
\end{quote}

\subsection{Seat Distribution in Electoral Systems \label{subsec-elec} }
To answer the above question, we visit the literature on proportional representation in electoral systems which is far removed from the stock exchanges and
more generally the financial world. 
Proportional representation is the most common kind of electoral system where the seats are not contested individually. Instead, the total number of seats is allocated 
to the contesting parties in proportion to the number of votes they have won in the election.  Let us consider an election contested by $n$ parties over $K$ seats that are 
distributed using a proportional representation method. Let $V_j$ denote the number of votes won by party $j \in [1, \ldots, n]$ in the election. The electoral output 
is denoted by the vector $\mathbf{V} = (V_1, \ldots, V_n)$, and the total number of votes cast in the election is $V = V_1 + \cdots + V_n$. Suppose the total number
of seats to be distributed among the parties is $K$. A proportional representation method determines the seat allocation vector 
$\mathbf{K}=(K_1,\ldots,K_n)$ where $K_i$ is the number of seats allocated to the $i$-th party, and $K_1+\cdots+K_n=K$. Typically, the total number of seats $K$ is much 
smaller than the total number of votes $V$, i.e. $K<V$. Further, it is reasonable to assume that in practice the number of seats allocated to the 
$i$-th party is at most the number of votes received by the party, i.e. $K_i\leq V_i$. 

Formally, a proportional representation method is an algorithm $\mathcal{A}(n, \mathbf{V}, K)$ which takes as input the number $n$ of parties, the vote count vector 
$\mathbf{V} = (V_1, \ldots, V_n)$, and the number $K$ of seats to be distributed, where $0<K<V$. It outputs the seat allocation vector $\mathbf{K} = (K_1, \ldots, K_n)$
such that $0\leq K_i\leq V_i$ and $K_1+\cdots+K_n=K$. 

From the above description, it becomes clear that the goal of allocating an incoming counter-party order to resting orders in proportion to the sizes of the resting orders 
is the same as the goal of assigning a fixed number of seats to several contesting parties in proportion to the number of votes obtained by these parties. The correspondence becomes 
clear by identifying the size $T_i$ of the $i$-th order with the number of votes $V_i$
received by the $i$-th party, the size $S$ of the incoming counter-party order with the total number of seats $K$, and the quantity $S_i$ of the $i$-th order
that is filled with the number of seats $K_i$ alloted to the $i$-th party. Having identified this correspondence, any algorithm for proportional representation of seats
in an electoral system becomes a potential candidate for use as an order matching algorithm by a stock exchange for proportional fulfillment of orders. The 
identification of proportional representation methods as possible substitutes for the pro-rata order matching algorithm is the key observation of the present note.
Not all proportional representation methods, however, are suitable for use as order matching algorithms. Some methods may require certain conditions to be 
applied which cannot be expected to hold in the context of order matching. We point out some such examples in Section~\ref{sec:order-matching}.

There is a large literature on electoral systems in general and proportional representation methods in particular. We refer the reader to~\cite{Herron2018,Norris2004}
for elaborate discussions on these topics. A number of proportional representation methods have been proposed in the context of electoral systems. 
These can be divided into two types, the highest averages method and the largest remainder method. 
The most
well known of the highest averages method are the Jefferson/D'Hondt (JD) and the Webster/Sainte-Lagu\"{e} (WS) methods. Both of these methods continue to be of active interest.
See for example~\cite{HM08,Me19,FSS20,GF17}. Among the largest remainder method, the two well known methods are the Hare and the Droop methods. In fact, the
Hare method has the attractive property that it minimises the $L_1$-distance to the ideal allocation. However, the largest remainder methods suffer from 
certain paradoxes (see Section~\ref{sec:order-matching}), which make them unsuitable for use as order matching algorithms. 

In the present context, both JD and the WX methods are well suited to be used as order matching algorithms in stock exchanges. We report simulation studies comparing
the pro-rata, the JD and the WS methods. Such studies show that both the JD and the WS methods are both $L_1$ and $L_2$-better than the pro-rata method. Among the
three methods, the allocation determined by the WS method turns out to be the closest to the ideal allocation in an overwhelming number of cases for both $L_1$ and $L_2$ metrics.
This provides sufficient evidence to seriously consider the adoption of Webster/Sainte-Lagu\"{e} method based order matching by stock exchanges.

\subsection{Procedural Fairness \label{subsec-fair} }
A recent paper by Hersch~\cite{He22} investigated the issue of procedural fairness of order allocation methods. In the paper, it was argued that
both the FIFO and the pro-rata are fair in principle, but not in practice. It was pointed out that the main disadvantage of pro-rata is the requirement
of the second step ``requiring exchanges to introduce secondary matching rules that can be gamed''. 

An alternative method called the random selection for service (RSS) method was proposed. 
Given the vector $\mathbf{T}=(T_1,\ldots,T_n)$ of resting orders, the RSS method defines a probability distribution $\pi$ over $\{1,\ldots,n\}$, where 
$\pi$ associates probability $T_i/T$ to $i$, for $i=1,\ldots,n$, 
Suppose the incoming counter-party order consists of $S$ units. Allocation is done by repeating the following procedure $S$ times:
an independent random $i$ is drawn from $\{1,\ldots,n\}$ following the probability distribution $\pi$ and
one unit is alloted to the $i$-th resting order. At the end of the procedure, let $S_i$ be the number of units alloted to the $i$-th resting order so that the
final allotment is $\mathbf{S}=(S_1,\ldots,S_n)$ satisfying $S=S_1+\cdots+S_n$.
It was argued by Hersch~\cite{He22} that the RSS method is fair in both principle and practice. Below we revisit this method and point out a crucial difference between principle
and practice.

Given $\mathbf{T}=(T_1,\ldots,T_n)$ and $S$, suppose the RSS method is executed $\Gamma$ times and for $\gamma=1,\ldots,\Gamma$, let
the allotment of $\gamma$-th execution be $\mathbf{S}_{\gamma}=(S_{\gamma,1},\ldots,S_{\gamma,n})$. For $i=1,\ldots,n$, let 
$\widehat{S}_i = (S_{1,i}+\cdots+S_{\Gamma,i})/\Gamma$, i.e. $\widehat{S}_i$ is the average allotment to the $i$-th resting order computed over all the
$\Gamma$ trials. The law of large numbers assures us that asymptotically, i.e. as $\Gamma$ goes to infinity, the average allotment $\widehat{S}_i$
tends to $ST_i/T$ which is equal to the $i$-th component of the ideal allocation vector $\mathbf{I}$ (see~\eqref{eqn-ideal}). So in principle, Hersch~\cite{He22} implicitly
considers achieving allocation close to the ideal allocation vector $\mathbf{I}$ to be procedurally fair. To this extent, Hersch's objective and ours coincide.
Additionally, our use of the $L_1$ and $L_2$ metrics to measure deviation from the ideal allocation vector can be considered to be a quantification of procedural fairness.
As such it expands the theoretical framework for studying procedural fairness of the order allocation methods.

From a practical point of view, however, the RSS method has a significant shortcoming. The law of large numbers applies in an asymptotic contex, i.e. as
$\Gamma$ goes to infinity. In practice, given $\mathbf{T}=(T_1,\ldots,T_n)$ and $S$, a stock exchange will execute the RSS method exactly once to obtain a single allocation 
vector $\mathbf{S}=(S_1,\ldots,S_n)$. In other words, in practice the value of $\Gamma$ will be 1. 
The law of large numbers does not say anything about the value obtain in a single execution. In particular,
the $S_i$'s obtained after a single execution of RSS can be any value in the set $\{0,\ldots,S\}$. Considering a particular example with $n=2$, $\mathbf{T}=(10,90)$ and
$S=10$, Hersch~\cite{He22} provides probabilities that the $S_i$'s can take certain values: the probability that $S_1\geq 1$ (resp. $S_1=10$) is about 
0.88 (resp. $7\times 10^{-6}$); the probability that $S_2=20$ (resp. $S_2\geq 10$) is about 0.12 (resp. approaches 1). These probabilities, however, do not
enlighten us about the concrete values of the $S_i$'s after a single execution. In particular, the probability that $S_2\geq 10$ approaches 1 suggests an
asymptotic approach, where the frequentist view of probability is taken. To interpret such probabilities, one again needs to consider a large number $\Gamma$
of trials of the RSS method and consider the average allocation over all the $\Gamma$ trials. 

To test the practical efficacy of the RSS method, we have run experiments with the method. It turns out that the allocation vector obtained by the RSS method
has a very large deviation from the ideal allocation vector in terms of both the $L_1$ and the $L_2$ metrics. Particular examples are provided in Section~\ref{sec-sim-res}.
By the above explanation, this observation is not surprising.

As mentioned earlier, according to Hersch, the main disadvantage of the pro-rata method is the use of secondary matching rules in the second step of the method
which leads to the possibility of gaming. The proportional representation based order allocation method that we introduce does not require any such secondary
matching rules which can be gamed. 
So our proposal overcomes the disadvantage of the pro-rata method pointed out by Hersch. In terms of procedural fairness as measured by
distance to the ideal allocation vector, our simulation studies show that proportional representation based order allocation outperforms the pro-rata method.

\section{Proportional Representation Methods \label{sec:order-matching}}

There are many different proportional representation methods (see~\cite{Herron2018,Norris2004}). Below we describe two well known families of proportional representation 
methods, namely the highest averages or the divisor method, and the highest remainder method. 

\paragraph{Highest averages method.}
Recall the setting described in Section~\ref{subsec-elec}, where $V_i$ is the number of votes received by the $i$-th party and $K$ is the total number of available seats. 
The goal is to determine $K_i$ which is the number of seats alloted to the $i$-th party.
Let $f: \mathbb{Z}^{+} \cup \{0\} \to \mathbb{R}$ be a function from the non-negative integers to the reals. Let $V=V_1+\cdots+V_n$ and $v_i=V_i/V$. 
In the highest averages method, seats are allotted iteratively. The seat distribution algorithm goes through $K$ iterations and in each iteration exactly one seat 
is alloted to one of the parties. Initially, the algorithm sets $K_1=K_2=\cdots=K_n=0$. For $k$ from $1$ to $K$, in the $k$-th iteration
the algorithm determines $j=\arg\max\{v_i/f(K_i): i=1,\ldots,n\}$ and increments $K_j$ by one. After $K$ iterations, the final values of $K_1,\ldots,K_n$ are the numbers of
seats alloted to the various parties. Various different methods arise from the different definitions of $f$. The definitions of $f$ for the well known
Jefferson/D'Hondt and the Webster/Sainte-Lagu\"{e} methods are shown in Table~\ref{tab:highest-averages-methods}.

\begin{table}[!htb]
    \centering
    \begin{tabular}{l|l}
    \hline
        Method name & $f(t)$ \\
    \hline
%	    Adam (Ad) & $\lceil t \rceil$ \\
%	    Danish (Da) & $t/(t+(1/3))$ \\
%	    Dean (De) & ${t(t+1)}/{(t+0.5)}$ \\
%	    Huntington/Hill (HH) & $\sqrt{t(t+1)}$ \\
	    Jefferson/D'Hondt (JD) & $(t+1)$ \\
	    Webster/Sainte-Lagu\"{e} (WS) & $(t+0.5)$ \\
    \hline
    \end{tabular}
	\caption{The functions used for two important methods of proportional representation.  \label{tab:highest-averages-methods}}
\end{table}

Apart from the JD and the WS methods, there are a number of other proportional representation methods, such as Dean's, Adam's, Huntington/Hill and the Danish methods.
(See~\cite{Wiki-division} for a compact description of these methods.)
These four methods require a positivity constraint to be satisfied which is not required in either the JD or the WS methods. They initially allocate one seat to each of the
contesting parties (i.e., they start with $K_1=\cdots=K_n=1$ instead of starting with $K_1=\cdots=K_n=0$) and then employ the highest averages method described above. 
As a result, at the end of the allocation each party has at least one seat. The 
definitions of the function $f$ for these four methods are different from the definitions of the function corresponding to the JD and the WS methods. Importantly, in 
these four methods, the function $f$ satisfies the condition $f(0)=0$. 
Consequently, the function cannot be evaluated unless the present number of seats allocated to a party is at least 1. This constraint is not present for
the JD and the WS methods. Note that the constraint of allocating at least one seat to each party requires the number of seats to be at least as large as the
number of parties.

In the context of order matching, the feature of assigning at least one seat to each party will translate to assiging at least one unit of the 
incoming counter-party order to each of the resting orders. This requires the size of the incoming counter-party order to be at least the number of resting orders,
i.e. $S\geq n$. Such a condition cannot be imposed in general, since there is no control over the size $S$ of the incoming counter-party order. More generally,
the principle of assigning at least one unit of the incoming order to each of the resting orders does not appear to have any justification in the context
of stock exchanges. On the contrary,
some stock exchanges follow the rule that if the order unit determined by the pro-rata method is 1, then this is instead set to 0~\cite{CMEMatchingAlgorithms}.
The principle of at least one unit for each resting order may not be welcome by such exchanges. Due to this reason as well as the unimplementable constraint
of $S\geq n$, in this paper we do not consider the proportional representation methods which follow the principle of assigning at least one seat to each party.

\paragraph{Largest remainder method.}
The method uses a parameter called the quota $Q$. The seat allocation is done in two phases. In the first phase, the $i$-th party is allocated $\lfloor V_i/Q\rfloor$ 
seats. Let $R_i=V_i/Q - \lfloor V_i/Q\rfloor$ be the remainder corresponding to the $i$-th party. Suppose after the first phase $k$ seats remain unallocated. 
In the second phase, the parties with the $k$ largest remainders are each allocated one seat. Let $K_i$ be the number of seats allocated to the $i$-th party at the
end of the second round. The method ensures that $K_1+\cdots+K_n=K$. 
Various methods arise by choosing an appropriate value of the quota $Q$. The Hare quota chooses $Q$ to be equal to $V/K$, while the Droop quota chooses
$Q$ to be equal to $1 + \lfloor V/(1+K) \rfloor$. Other choices for $Q$ have been 
proposed\footnote{\url{https://en.wikipedia.org/wiki/Largest_remainder_method}}. 

The largest remainder method satisfies the quota rule, i.e.  the condition $\lfloor K V_i/V\rfloor \leq K_i \leq \lceil K V_i/V\rceil$.
It is known that the largest remainder method with the Hare quota minimises the Loosemore-Hanby index\footnote{\url{https://en.wikipedia.org/wiki/Loosemore\%E2\%80\%93Hanby_index}}
which is equivalent to minimising the $L_1$-distance from the ideal proportional seat allocation $(KV_1/V,KV_2/V,\ldots,KV_n/V)$. 

\paragraph{Paradoxes.}
Balinski and Young~\cite{BY01} identified several paradoxes of proportional representation. Consider two possible vector of votes and the total number of seats
with $(V_1,\ldots,V_n)$ and $K$ be one of the scenarios and $(V_1^{\prime},\ldots,V_n^{\prime})$ and $K^{\prime}$. Let $(K_1,\ldots,K_n)$ and
$(K_1^{\prime},\ldots,K_n^{\prime})$ be the corresponding seat allocation vectors. 
\begin{itemize}
	\item The Alabama paradox is the following: $V_i=V_i^{\prime}$ for $i=1,\ldots,n$
		and $K^{\prime}>K$, but there is a $j\in \{1,\ldots,n\}$ such that $K_j>K_j^{\prime}$. 
		In other words, the votes polled remain the same and the total number of seats has gone up, but the number of seats allocated to a party has gone down.
	\item The population paradox is the following: $K=K^{\prime}$, $V_i^{\prime}\geq V_i$, and for some $j,k\in\{1,\ldots,n\}$, 
		$V_j^{\prime}-V_j > V_k^{\prime}-V_k$, but $K_j^{\prime}<K$. In other words,
		the number of seats remains the same and the additional votes polled by one party is more than another, but the former party is allocted fewer seats.
\end{itemize}
The Balinski-Young theorem states that any method of apportionment which satisfies the quota rule will necessarily suffer from either the Alabama or the population
paradox. 

In the context of order matching, both the Alabama and the population paradoxes are problematic. The Alabama paradox translates to the following.
The quantities of the resting orders remain the same and an increase occurs in the incoming counter-party order, yet the allocation to a particular resting order goes down. 
The population paradox translates to the following. The size of the incoming counter-party order remains the same, and the size of a particular resting order increases
by an amount which is more than the increase in the size of another resting order, but the units allocated to the former resting order goes down.
Both of these are counterintuitive and hard to explain to the stakeholders of a stock exchange. Due to these paradoxes, the largest remainder method with Hare quota
is unsuitable for use as order matching algorithm even though it minimises the distance to the ideal allocation vector.

The highest averages methods are free of these paradoxes. As a consequence, by the Balinski-Young theorem, they violate the quota rule. It is, however, known that
the while the WS method in principle violates the quota rule, it does so rarely. 

Based on the above discussion, we consider only the JD and the WS method in our simulation studies.

\section{Simulation and Results \label{sec-sim-res}}
%We have implemented the pro-rata, the JD and the WS methods.

Given $n$, $\mathbf{T}=(T_1,\ldots,T_n)$ and $S$, the pro-rata allocation takes place in two steps. In the first step, the vector 
$$\mathbf{S}_{\mathcal{P}}^{\prime}=(S_1^{\prime},S_2^{\prime},\ldots,S_n^{\prime})=(\lfloor ST_1/T \rfloor, \lfloor ST_2/T \rfloor, \ldots, \lfloor  ST_n/T\rfloor)$$ 
is computed. In the second step, the remaining $S-S^{\prime}$ (where $S^{\prime}=S_1^{\prime}+S_2^{\prime}+\cdots+S_n^{\prime}$) quantity of the incoming order 
is allocated using the first-come-first-served strategy. In our implementation, we have used a modified version of the first-come-first-served-strategy where we 
have prioritised smaller orders as follows: allocate one unit to all orders in the first-come-first-served manner which got zero allocation in the first step, 
next allocate one unit to all orders in the first-come-first-served manner which was alloted one unit in the first step, and so on until all the $S-S^{\prime}$ units left
over from the first step are exhausted. The second step increases the allocation to any resting order by at most one unit.
%and we also check and ensure that alloted number of units to the $i$-th resting order is not greater than $ST_i/T$. 
The rationale for using the modified first-come-first-served strategy is to provide some benefit to smaller orders. 
%Since $S-S^{\prime}$ is at most $n$, the overall effect of this modified strategy has a negligible effect on the values of $\ell_{1,\mathcal{P}}$ and $\ell_{2,\mathcal{P}}$, 
%where $\mathcal{P}$ denotes the pro-rata method. 

The pro-rata method clearly takes $O(n)$ time. The highest averages method (of which the JD and the WS methods are special cases)
described in Section~\ref{sec:order-matching} can be implemented in time $O(n+K\log n)$. (By the identification of the size $S$ of the incoming order with the number $K$
of available seats, we have $O(n+K\log n)=O(n+S\log n)$.)
We briefly discuss how this can be done. Note that $O(n)$ time is required to initialise the allocation vector $K=(K_1,\ldots,K_n)$ to the all-zero vector. Next 
a max-heap data structure~\cite{AHU78} is built on the $n$ values $f_1=v_1/f(K_1), f_2=v_2/f(K_2),\ldots, f_n=v_n/f(K_n)$. This takes $O(n)$ time. The heap data structure 
stores the maximum of the $f_i$'s on the top. In each of the $K$ iterations,
exactly one $K_i$ is incremented, so exactly one $f_i$ is modified and the other $f_j$'s remain unchanged. The heap data structure is updated so that the new maximum
gets to the top. This can be done in $O(\log n)$ time and makes the new maximum available for the next iteration. So the $K$ iterations of the highest averages
method take $O(K\log n)$ time. 
%Below we present simulation results which show that the JD and the WS algorithms achieve better proportionality. The trade-off is that both of these algorithms
%take more time than the pro-rata method.

To compare the performances of the different algorithms, we have performed simulation studies. The input to an order matching algorithm is the number of resting
orders $n$, the vector $\mathbf{T}=(T_1,\ldots,T_n)$, where $T_i$ is the size of the $i$-th order, and the size $S$ of the incoming counter-party order. 
In our simulations, we have randomly generated the values $T_1,\ldots,T_n$ using the following strategy. Fix two non-negative integers $m$ and $M$ with $m<M$. 
Let $\mu$ and $\sigma$ be positive real numbers which specify the normal ${\mathcal N}(\mu,\sigma)$ distribution. For each $i$ in $1$ to $n$, the following
procedure is performed: draw a sample from ${\mathcal N}(\mu,\sigma)$ and round to the nearest integer, repeat until the rounded value is in the range $[m,M]$; 
once the rounded value satisfies the range check, set $T_i$ to be equal to this rounded value.
After $n$ iterations, we obtain the random vector $\mathbf{T}=(T_1,\ldots,T_n)$ which is a simulated distribution of the resting orders. 
Note that all the samples are drawn {\em independently} from ${\mathcal N}(\mu,\sigma)$. Our rationale for
choosing the normal distribution is that in the absence of any other information, the sizes of the orders may be assumed to follow the normal distribution. If, on the
other hand, additional information is available, then it is possible to change the normal distribution to another distribution without affecting the rest of the
simulation. 

In Table~\ref{tab-ex}, we provide some examples of the order allocation vector $\mathbf{S}_{\mathcal{A}}$, where 
$\mathcal{A}$ is one of $\mathcal{P}$ (denoting the pro-rata method), RSS, JD, or WS method. 
The ideal allocation vector is $\mathbf{I}=(ST_1/T,\ldots,ST_n/T)$. 
%For the pro-rata method, we provide both the vectors $\mathbf{S}_{\mathcal{P}}^{\prime}$ (the output of the first step of the pro-rata method) and 
%$\mathbf{S}_{\mathcal{P}}$ (the final output of the pro-rata method). 
From the examples, we observe that for the RSS method, the $L_1$ and $L_2$ distances from the ideal allocation vector $\mathbf{I}$ are much larger than these
distances from the other methods. This is as expected (see Section~\ref{subsec-fair}) and highlights the impracticability of the RSS method. While the table
provides only three examples, we have obtained many other examples and the observation that the order allocation vector produced by the RSS method
is much farther away from the ideal allocation vector compared to the other methods holds in all the examples. In view of this, we do not consider the
RSS method any further in our simulation studies.

Among the three methods, i.e. the pro-rata method, the JD and the WS methods, note that in all the cases, the JD and the WS methods are both $L_1$-better and 
$L_2$-better than the pro-rata method. Comparing
$\ell_{1,\mathcal{P}}$ with $\ell_{1,{\rm JD}}$ and $\ell_{1,{\rm WS}}$ and $\ell_{2,\mathcal{P}}$ with $\ell_{2,{\rm JD}}$ and $\ell_{2,{\rm WS}}$, we find significant
difference in these values. So these examples suggest that the JD and the WS methods are significantly better than the pro-rata method with respect to the 
$L_1$ and $L_2$ metrics.

\begin{table}
\centering
{\scriptsize
	\begin{tabular}{|l|l|r|r|r|r|r|r|r|r|r|r|r|r|}
		\cline{13-14}
		\multicolumn{12}{c|}{} & $\ell_{1,\mathcal{A}}$ & $\ell_{2,\mathcal{A}}$ \\ \hline
		\multirow{5}{*}{Ex~1}
		& $\mathbf{T}$ & 209 & 727 & 746 & 808 & 995 & 204 & 598 & 773 & 979 & 899 & - & - \\ \cline{2-14}
		& $\mathbf{I}$ & 3.01 & 10.48 & 10.75 & 11.65 & 14.34 & 2.94 & 8.62 & 11.14 & 14.11 & 12.96 & - & - \\ \cline{2-14}
		%& $\mathbf{S}_{\mathcal{P}}^{\prime}$ & 3 & 10 & 10 & 11 & 14 & 2 & 8 & 11 & 14 & 12 & - & -\\ \cline{2-14}
		& $\mathbf{S}_{\mathcal{P}}$ & 4 & 11 & 11 & 11 & 14 & 3 & 9 & 11 & 14 & 12 & 4.39 & 2.94 \\ \cline{2-14}
		& $\mathbf{S}_{{\rm RSS}}$ & 4 & 7 & 17 & 11 & 18 & 3 & 5 & 10 & 15 & 10 & 23.69 & 89.86 \\ \cline{2-14}
		& $\mathbf{S}_{{\rm JD}}$ & 3 & 10 & 11 & 12 & 14 & 3 & 9 & 11 & 14 & 13 & 2.17 & 0.71 \\ \cline{2-14}
		& $\mathbf{S}_{{\rm WS}}$ & 3 & 10 & 11 & 12 & 14 & 3 & 9 & 11 & 14 & 13 & 2.17 & 0.71 \\ \hline
		\multirow{5}{*}{Ex~2}
		& $\mathbf{T}$ & 1 & 655 & 307 & 138 & 647 & 48 & 625 & 382 & 95 & 424 & - & - \\ \cline{2-14}
		& $\mathbf{I}$ & 0.03 & 19.72 & 9.24 & 4.15 & 19.48 & 1.44 & 18.81 & 11.50 & 2.86 & 12.76 & - & - \\ \cline{2-14}
		%& $\mathbf{S}_{\mathcal{P}}^{\prime}$ & 0 & 19 & 9 & 4 & 19 & 1 & 18 & 11 & 2 & 12 & - & -\\ \cline{2-14}
		& $\mathbf{S}_{\mathcal{P}}$ & 1 & 19 & 10 & 5 & 19 & 2 & 18 & 11 & 3 & 12 & 6.54 & 4.79 \\ \cline{2-14}
		& $\mathbf{S}_{{\rm RSS}}$ & 0 & 21 & 15 & 2 & 21 & 1 & 17 & 8 & 4 & 11 & 19.41 & 61.91 \\ \cline{2-14}
		& $\mathbf{S}_{{\rm JD}}$ & 0 & 20 & 9 & 4 & 20 & 1 & 19 & 12 & 2 & 13 & 3.46 & 1.72 \\ \cline{2-14}
		& $\mathbf{S}_{{\rm WS}}$ & 0 & 20 & 9 & 4 & 19 & 1 & 19 & 12 & 3 & 13 & 2.69 & 0.95 \\ \hline
		\multirow{5}{*}{Ex~3}
		& $\mathbf{T}$ & 268 & 806 & 409 & 420 & 869 & 659 & 189 & 317 & 286 & 721 & - & - \\ \cline{2-14}
		& $\mathbf{I}$ & 5.42 & 16.30 & 8.27 & 8.50 & 17.58 & 13.33 & 3.82 & 6.41 & 5.78 & 14.58 & - & - \\ \cline{2-14}
		%& $\mathbf{S}_{\mathcal{P}}^{\prime}$ & 5 & 16 & 8 & 8 & 17 & 13 & 3 & 6 & 5 & 14 & - & -\\ \cline{2-14}
		& $\mathbf{S}_{\mathcal{P}}$ & 6 & 16 & 9 & 8 & 17 & 13 & 4 & 7 & 6 & 14 & 4.57 & 2.41 \\ \cline{2-14}
		& $\mathbf{S}_{{\rm RSS}}$ & 2 & 15 & 8 & 4 & 12 & 17 & 2 & 6 & 9 & 25 & 34.61 & 200.59 \\ \cline{2-14}
		& $\mathbf{S}_{{\rm JD}}$ & 5 & 17 & 8 & 8 & 18 & 13 & 4 & 6 & 6 & 15 & 3.86 & 1.69 \\ \cline{2-14}
		& $\mathbf{S}_{{\rm WS}}$ & 5 & 16 & 8 & 9 & 18 & 13 & 4 & 6 & 6 & 15 & 3.47 & 1.31 \\ \hline
	\end{tabular}
	\caption{Examples of simulation runs with $n=10$, $S=100$, $m=1$, $M=1000$, $\mu=500$ and $\sigma=400$. Here $\mathcal{P}$ is the pro-rata method. \label{tab-ex}}
}
\end{table}

A few examples do not provide sufficient evidence. It is required to consider many more examples. On the other hand, when there are a large number of examples,
it is not possible to visually inspect all such examples. So we have used a program to perform the comparison for the various simulation studies.
Since $\mathbf{T}$ is determined by $n$, $\mu$ and $\sigma$, the parameters for the simulations are the different values of $n$, $\mu$ and $\sigma$ as well as $S$. 
For a specific set of values of $n$, $\mu$, $\sigma$ and $S$, we have performed $N$ iterations of the simulation. In each iteration, we have computed 
the ideal allocation vector $\mathbf{I}=(ST_1/T,\ldots,ST_n/T)$, and the order allocation vector
$\mathbf{S}_{\mathcal{A}}=(S_1,\ldots,S_n)$ produced by the order matching algorithm $\mathcal{A}$, where $\mathcal{A}$ is one of pro-rata, the JD or the WS algorithms.
%The ideal proportional allocation vector in each iteration is $\mathbf{I}=(ST_1/T,\ldots,ST_n/T)$ and the output of algorithm $\mathcal{A}$ is 
%given by the vector $\mathbf{S}_{\mathcal{A}}=(S_1,S_2,\ldots,S_n)$. 
Next we computed the $L_1$ and $L_2$ distances of $\mathbf{S}_{\mathcal{A}}$ from $\mathbf{I}$ given by $\ell_{1,\mathcal{A}}$ and $\ell_{2,\mathcal{A}}$. 
In each of the $N$ iterations, we have compared the JD and the WS methods with the pro-rata method. After $N$ iterations, aggregate statistics are determined
for the particular simulation. Further details are given below.

In our experiments, we have taken the number of iterations $N$ to be 10000. The parameters of the various simulation runs are given in Table~\ref{tab-sim-param}.
To obtain an idea of the comparison between the different algorithms, we have considered a number of variations in the parameters. The value of $n$ has been chosen
to be as low as 10 to a moderate value of 100, while the value of $S$ has been taken to be as small as 30 to as large as 3000. While we report results for the
values of parameters shown in Table~\ref{tab-sim-param}, we have also experimented with various other values. The results in all such cases turned out to be very similar 
to the results that we report here. 

Table~\ref{tab-res} provides a summary of the results that we obtained from the simulations.
The columns of the table list the pro-rata method along with the JD and the WS methods.
The rows correspond to the various simulation runs whose parameters are given in Table~\ref{tab-sim-param}. For a row starting with $L_1$, all entries in the
corresponding row are with respect to the $L_1$ metric. Similarly for a row starting with $L_2$, all entries in the corresponding row are
with respect to the $L_2$ metric. Each entry in the table is a pair of numbers. 
Suppose $(x_1,x_2)$ appears in the column headed by algorithm $\mathcal{A}$ in a row corresponding to simulation number $s$ for the $L_1$ metric.
The value $x_1$ is the percentage of times that $\ell_{1,\mathcal{A}}$ came out to be lower than $\ell_{1,\mathcal{P}}$ (where $\mathcal{P}$ denotes
the pro-rata method) in simulation number $s$, while
the value $x_2$ is the percentage of times that $\ell_{1,\mathcal{A}}$ came out to be the minimum among all the three methods.
For example, the pair $(100.00,99.31)$ appearing under the column headed by WS in row labelled Sim~1 and starting with $L_1$ indicates that
the WS method is $L_1$-better than the pro-rata method in 100\% of the cases (i.e., in all the iterations of Sim~1); further, with respect to the $L_1$ metric, 
in 99.31\% of the iterations in Sim~1, the WS method provides the closest approximation to the ideal allocation among all the three methods. 
A similar interpretation holds for a pair of values appearing in a row starting with $L_2$, with the only change being that the $L_1$ metric is
replaced by the $L_2$ metric. Note that for each pair under the column headed pro-rata, the first entry is a `-', since it is not meaningful to compare 
pro-rata method with itself. Also, it is possible that in a particular iteration, the distances of two of the methods to the ideal are both minimum; so
the sum of the percentages of cases for which the different methods are minimum can be greater than 100. 

The simulation results bring out two important issues.
\begin{enumerate}
	\item Both the JD and the WS methods are better than the pro-rata method for an overwhelming number of cases.
	\item Among the three algorithms, the WS method provides the closest approximation to the ideal allocation in most of the cases.
\end{enumerate}
Consequently, the simulations provide sufficient evidence for stock exchanges to seriously consider the adoption of the Webster/Saint-Lagu\"{e} based order matching 
algorithm as a replacement of the pro-rata order matching algorithm.


\begin{table}
\begin{subtable}{0.5\textwidth}
	\centering
	{\scriptsize
		\begin{tabular}{|l|r|r|r|r|r|r|}
			\cline{2-7}
			\multicolumn{1}{c|}{} & \multicolumn{1}{c|}{$n$} & \multicolumn{1}{c|}{$m$} & 
			\multicolumn{1}{c|}{$M$} & \multicolumn{1}{c|}{$\mu$} & \multicolumn{1}{c|}{$\sigma$} & \multicolumn{1}{c|}{$S$} \\ \hline
			Sim~1 & 20 & 1 & 1000 & 500 & 400 & 50 \\ \hline 
			Sim~2 & 200 & 1 & 1000 & 500 & 400 & 30 \\ \hline 
			Sim~3 & 10 & 1 & 10000 & 5000 & 3000 & 300 \\ \hline 
			Sim~4 & 100 & 1 & 10000 & 5000 & 3000 & 300 \\ \hline 
			Sim~5 & 100 & 1 & 10000 & 5000 & 3000 & 3000 \\ \hline 
		\end{tabular}
		}
		\caption{Parameters of the various simulation runs. \label{tab-sim-param} }
\end{subtable}
\begin{subtable}{0.5\textwidth}
	\centering
	{\scriptsize
		\begin{tabular}{|l|l|r|r|r|}
			\cline{3-5} 
			\multicolumn{2}{c|}{} & \multicolumn{1}{c|}{pro-rata} & \multicolumn{1}{c|}{JD} & \multicolumn{1}{c|}{WS} \\ \hline
			\multirow{2}{*}{Sim~1} & 
			  $L_1$ & (-, 0.00) & (96.08, 4.79) & (100.00, 99.31) \\ \cline{2-5}
			& $L_2$ & (-, 0.01) & (95.99, 0.01) & ( 99.99, 99.30) \\ \hline
			\multirow{2}{*}{Sim~2} & 
			  $L_1$ & (-, 0.00) & (100.00, 100.00) & (100.00, 100.00) \\ \cline{2-5}
			& $L_2$ & (-, 0.00) & (100.00, 100.00) & (100.00, 100.00) \\ \hline
			\multirow{2}{*}{Sim~3} & 
			  $L_1$ & (-, 0.42) & (91.11, 25.58) & (99.65, 97.09) \\ \cline{2-5}
			& $L_2$ & (-, 0.46) & (90.89, 25.57) & (99.57, 97.02) \\ \hline
			\multirow{2}{*}{Sim~4} & 
			  $L_1$ & (-, 0.00) & (100.00, 0.00) & (100.00, 100.00) \\ \cline{2-5}
			& $L_2$ & (-, 0.00) & (100.00, 0.00) & (100.00, 100.00) \\ \hline
			\multirow{2}{*}{Sim~5} & 
			  $L_1$ & (-, 0.00) & (100.00, 0.00) & (100.00, 100.00) \\ \cline{2-5}
			& $L_2$ & (-, 0.00) & (100.00, 0.00) & (100.00, 100.00) \\ \hline
		\end{tabular}
		}
		\caption{Summary of simulation results. \label{tab-res}}
\end{subtable}
	\caption{Simulation parameters and summary. \label{tab-sim} }
\end{table}



%\bibliographystyle{alpha}
%\bibliography{prop-alloc}
\documentclass[11pt]{article}

\setlength{\textwidth}{6.5in} \setlength{\textheight}{8.5in}\hoffset=-0.1in \voffset=-0.5in

\usepackage{subcaption,booktabs}

\usepackage{authblk}
\usepackage[english]{babel}
\usepackage[utf8]{inputenc}
\usepackage[T1]{fontenc}
\usepackage{amsmath,amsfonts,amsthm,amssymb}
\usepackage{mathtools}
\usepackage{graphicx}
\usepackage[table]{xcolor}
\usepackage{multirow}
\usepackage{xurl}
\usepackage{hyperref}
\usepackage{fullpage}
\usepackage[numbers]{natbib}
\hypersetup{
    colorlinks=true,
    linkcolor=red,
    filecolor=magenta,      
    urlcolor=blue,
    linkbordercolor = white
}

% Reducing the bibliography spacing
%\setlength{\bibsep}{0.0pt}
\usepackage{enumitem}
%\setlength{\itemsep}{0em}

\newtheorem{theorem}{Theorem}
\newtheorem{proposition}[theorem]{Proposition}
\newtheorem{example}{Example}%
\newtheorem{remark}{Remark}%
\newtheorem{definition}{Definition}%

\newcommand{\sym}[1]{{\sf #1}}

% Algorithms
%\usepackage[margin=3cm]{geometry}
\usepackage{algorithm2e}
\RestyleAlgo{ruled}
%% This is needed if you want to add comments in
%% your algorithm with \Comment
\SetKwComment{Comment}{/* }{ */}

% Colours to mark portions of text
\newcommand{\red}[1]{\textcolor{red}{#1}}
\newcommand{\blue}[1]{\textcolor{blue}{#1}}

\title{On Using Proportional Representation Methods as Alternatives to Pro-Rata Based Order Matching Algorithms in Stock Exchanges}
%\titlerunning{Proportional Representation Based Order Matching}
\author[1]{Sanjay Bhattacherjee}
%\affil[1]{School of Computing, University of Kent, CT2 7NF, United Kingdom, email: s.bhattacherjee@kent.ac.uk}
\affil[1]{
	Institute of Cyber Security for Society and School of Computing \authorcr 
	Keynes College, University of Kent \authorcr 
	CT2 7NP, United Kingdom \authorcr 
	email: s.bhattacherjee@kent.ac.uk \authorcr
	ORCID: 0000-0002-3367-6192}
\author[2]{Palash Sarkar\thanks{Corresponding author.}}
\affil[2]{
	Indian Statistical Institute \authorcr 
	203, B.T. Road, Kolkata \authorcr
	India 700108 \authorcr 
	email: palash@isical.ac.in \authorcr
	ORCID: 0000-0002-5346-2650}

\date{\today}

\begin{document}

\maketitle


\begin{abstract}
	The main observation of this short note is that methods for determining proportional representation in electoral systems may be suitable as alternatives to 
	the pro-rata order matching algorithm used in stock exchanges. Our simulation studies provide strong evidence that the Jefferson/D'Hondt 
	and the Webster/Saint-Lagu\"{e} proportional representation methods provide order allocations which are closer to proportionality than the order allocations 
	obtained from the pro-rata algorithm. \\
	{\bf Keywords:} order matching algorithm, pro-rata algorithm, proportional representation, Jefferson/D'Hondt method, Webster/Saint-Lagu\"{e} method. \\
	{\bf JEL codes:} D49.
\end{abstract}

%Suppose there are n resting orders of sizes S_1 to S_n and total size S. Let p_i=S_i/S. Suppose the incoming order is of size K. His method performs K independent trials of the following random %experiment. Pick a number in 1 to n, where number i has probability p_i of being chosen; assign one unit of the incoming order to resting order number i. The rationale behind this is that on the average, %the assignment of orders will be the ideal distribution.

\newpage


\section{Introduction\label{sec:intro}}
Financial instruments are traded on stock exchanges. Traders place orders to buy and sell such instruments. 
An order specifies, among other things, the quantity or size, i.e. the number of units to be purchased or sold, and the price at which the order is to be executed.
The quoted price is required to be a multiple of a unit of price called tick. A stock exchange maintains an order book which records for each financial instrument
the corresponding list of orders. The orders quoting the same buying or selling price are placed at the same price level in the order book. 
%The number of price 
%levels of a financial instrument is called its book depth. The difference between the highest bid for buying and the lowest ask for selling is called the bid-ask spread 
%of the instrument.

Trading in a stock exchange occurs by executing orders. Buy orders for an instrument are matched with corresponding sell orders. 
Matching happens in units of the instrument. The quantity specified in an order may not be equal to the quantity of the counter-party order that it is matched with. An 
order is filled when all its units have been matched with one or more counter-party matching orders. Unmatched or partially filled orders rest in the 
order book waiting for new orders to be matched with.
Orders can be of different types. %See~\cite{CS2020} for an overview of various kinds of orders. 
A common and important type of order is a limit order which specify both the quantity and the price at which the order is to be executed.

Let us consider the order book entries of a financial instrument with $p$ price levels $L_1, \ldots, L_p$. Let the number of buy and sell limit orders at price $L_i$ be denoted as 
$b_i$ and $s_i$ respectively. We note that in the resting state of the order book for the instrument, either $b_i=0$ or $s_i=0$ for a price $L_i$. Otherwise, 
the opposing orders will be matched until there are no more buy or sell orders to be matched (i.e., either $b_i=0$ or $s_i=0$ or both). A new incoming order at price 
$L_i$ could be of the same type as the resting orders in which case it will be added to the list of resting orders or it could be a counter-party order in which case it 
will be matched with the resting orders.

Let us assume that at price level $L$ there are $n$ resting orders and the quantities of the orders are given by a vector $\mathbf{T} = (T_1, \ldots, T_n)$, 
where $T_i$ is a positive integer which represents the quantity of the $i$-th order. As discussed above, these $n$ orders are either all buy orders or all sell orders. 
Let $T = T_1 + \cdots + T_n$ be the total quantity of these orders. A new incoming counter-party order of quantity $S$ at 
price $L$ will be matched with the resting orders in $\mathbf{T}$. As a result, some or all of the resting orders may be executed. If $S \geq T$, the resting orders 
are all filled and can be fully executed. However, if $S < T$, not all resting orders can be completely filled. In this case, an order matching algorithm is required
to allocate portions of the incoming counter-party order to the $n$ resting orders. Henceforth, we will assume that the condition $S<T$ holds.

Formally, an order matching algorithm $\mathcal{M}(n, \mathbf{T}, S)$ takes as input the number $n$ of resting orders, the vector $\mathbf{T} = (T_1, \ldots, T_n)$ of the 
resting order quantities, and an incoming counter-party order of quantity $S$ with $0<S<T$. It 
outputs $\mathbf{S} = (S_1, \ldots, S_n)$ so that $S_i$ quantity of $T_i$ may be executed. In other words, $0\leq S_i \leq T_i$ and $S = S_1 + \ldots + S_n$. 

Two of the most common order matching algorithms are the price-time priority (also called first-come-first-served (FCFS) or
first-in-first-out (FIFO)) and the pro-rata methods (see for example~\cite{CMEMatchingAlgorithms,Pr11,CS2020,He22}). Both methods aim to achieve some kind of fariness
in the allocation of orders. In this work we focus on the pro-rata method.

The idea behind the pro-rata method is to distribute $S$ to the $n$ resting orders more or less in proportion to their fractions of the total order. 
So ideally the $i$-th resting order would receive $ST_i/T$ portion of the incoming counter-party order. This, however, has a problem. Financial instruments are
traded in indivisible atomic units. So if $ST_i/T$ is not an integer, then this amount of the order cannot be executed. The pro-rata order matching algorithm
adopts a two-step approach to this problem. In the first step, the algorithm assigns an amount $S_i^{\prime}=\lfloor ST_i/T\rfloor$ of the incoming counter-party order 
to the $i$-th resting 
order\footnote{For a real number $x$, $\lfloor x\rfloor$ denotes the greatest integer not greater 
than $x$, and $\lceil x\rceil$ denotes the least integer not less than $x$.}$^,$\footnote{Sometimes a simple modification to the first step is used. For 
example,~\cite{CMEMatchingAlgorithms} adopts the strategy in which if some $S_i^{\prime}$ turns out to be $1$, then it is instead set to 0.}.
This strategy consumes $S^{\prime}=S_1^{\prime}+\cdots+S_n^{\prime}$ units of the incoming counter-party order. In the second step, the remaining 
$S-S^{\prime}$ units are distributed to the resting orders based upon some strategy which could be the FCFS
strategy\footnote{With the second step as FCFS, the overall method is a hybrid of pro-rata and FCFS. Other hybrid combinations of
FCFS and pro-rata have been proposed. See for example~\cite{BCS15}}.

In this note, we raise two questions.
\begin{enumerate}
	\item How well does the pro-rata order matching algorithm achieve its goal of distributing the incoming order to the resting orders in proportion to their
		fractions of the total order?
	\item Are there other algorithms which perform better than the pro-rata algorithm in achieving proportionality?
\end{enumerate}
To answer the above questions, we need a measure to assess the performance of an order matching algorithm in achieving proportionality. For $n$ resting
orders, with $\mathbf{T}=(T_1,\ldots,T_n)$ the vector of quantities of the resting orders, and $S$ the size of the incoming order, the ideal allocation, or
the ideal proportional distribution is given by the vector 
\begin{eqnarray}\label{eqn-ideal}
\mathbf{I}=(ST_1/T,ST_2/T,\ldots,ST_n/T). 
\end{eqnarray}
Suppose $\mathcal{A}$ is an order matching algorithm which on input $n$, $\mathbf{T}$ and $S$
produces the allocation vector $\mathbf{S}_{\mathcal{A}}=(S_1,\ldots,S_n)$ as output, where $S_i\leq T_i$ for $i=1,\ldots,n$, and $S_1+\cdots+S_n=S$. 
The distance between the vectors $\mathbf{I}$ and $\mathbf{S}_{\mathcal{A}}$ is a measure of the performance of the algorithm $\mathcal{A}$ in achieving
proportionality. The closer $\mathbf{S}_{\mathcal{A}}$ is to $\mathbf{I}$, the better is the performance of $\mathcal{A}$ in achieving proportionality. 

We consider two standard measures of distance between two vectors, namely the $L_1$ and the $L_2$ metrics defined as follows.
\begin{eqnarray*}
	L_1(\mathbf{S}_{\mathcal{A}},\mathbf{I}) = \sum_{i=1}^n \mid S_i - ST_i/T \mid, & &  L_2(\mathbf{S}_{\mathcal{A}},\mathbf{I}) = \sum_{i=1}^n (S_i - ST_i/T)^2.
\end{eqnarray*}
Using these two metrics, we can quantifiably answer the first question posed above. The metrics also provide a method to address the second question.
For an order matching algorithm $\mathcal{A}$, let $\ell_{1,\mathcal{A}}$ and $\ell_{2,\mathcal{A}}$ denote
$L_1(\mathbf{S}_{\mathcal{A}},\mathbf{I})$ and $L_2(\mathbf{S}_{\mathcal{A}},\mathbf{I})$ respectively. For two order matching algorithms $\mathcal{A}_1$
and $\mathcal{A}_2$, we say that $\mathcal{A}_1$ is $L_1$-better (resp. $L_2$-better) than $\mathcal{A}_2$ if 
$\ell_{1,\mathcal{A}_1}<\ell_{1,\mathcal{A}_2}$ (resp. $\ell_{2,\mathcal{A}_1}<\ell_{2,\mathcal{A}_2}$). In other words, $\mathcal{A}_1$ is 
$L_1$-better (resp. $L_2$-better) than $\mathcal{A}_2$, if its output is closer to the ideal allocation with respect to the $L_1$ (resp. $L_2$) metric.
Using the above terminology, we can rephrase the second question as follows. 
\begin{quote}
Is there an order matching algorithm $\mathcal{A}$ which is better than
the pro-rata order matching algorithm with respect to either or both of the $L_1$ and $L_2$ metrics?
\end{quote}

\subsection{Seat Distribution in Electoral Systems \label{subsec-elec} }
To answer the above question, we visit the literature on proportional representation in electoral systems which is far removed from the stock exchanges and
more generally the financial world. 
Proportional representation is the most common kind of electoral system where the seats are not contested individually. Instead, the total number of seats is allocated 
to the contesting parties in proportion to the number of votes they have won in the election.  Let us consider an election contested by $n$ parties over $K$ seats that are 
distributed using a proportional representation method. Let $V_j$ denote the number of votes won by party $j \in [1, \ldots, n]$ in the election. The electoral output 
is denoted by the vector $\mathbf{V} = (V_1, \ldots, V_n)$, and the total number of votes cast in the election is $V = V_1 + \cdots + V_n$. Suppose the total number
of seats to be distributed among the parties is $K$. A proportional representation method determines the seat allocation vector 
$\mathbf{K}=(K_1,\ldots,K_n)$ where $K_i$ is the number of seats allocated to the $i$-th party, and $K_1+\cdots+K_n=K$. Typically, the total number of seats $K$ is much 
smaller than the total number of votes $V$, i.e. $K<V$. Further, it is reasonable to assume that in practice the number of seats allocated to the 
$i$-th party is at most the number of votes received by the party, i.e. $K_i\leq V_i$. 

Formally, a proportional representation method is an algorithm $\mathcal{A}(n, \mathbf{V}, K)$ which takes as input the number $n$ of parties, the vote count vector 
$\mathbf{V} = (V_1, \ldots, V_n)$, and the number $K$ of seats to be distributed, where $0<K<V$. It outputs the seat allocation vector $\mathbf{K} = (K_1, \ldots, K_n)$
such that $0\leq K_i\leq V_i$ and $K_1+\cdots+K_n=K$. 

From the above description, it becomes clear that the goal of allocating an incoming counter-party order to resting orders in proportion to the sizes of the resting orders 
is the same as the goal of assigning a fixed number of seats to several contesting parties in proportion to the number of votes obtained by these parties. The correspondence becomes 
clear by identifying the size $T_i$ of the $i$-th order with the number of votes $V_i$
received by the $i$-th party, the size $S$ of the incoming counter-party order with the total number of seats $K$, and the quantity $S_i$ of the $i$-th order
that is filled with the number of seats $K_i$ alloted to the $i$-th party. Having identified this correspondence, any algorithm for proportional representation of seats
in an electoral system becomes a potential candidate for use as an order matching algorithm by a stock exchange for proportional fulfillment of orders. The 
identification of proportional representation methods as possible substitutes for the pro-rata order matching algorithm is the key observation of the present note.
Not all proportional representation methods, however, are suitable for use as order matching algorithms. Some methods may require certain conditions to be 
applied which cannot be expected to hold in the context of order matching. We point out some such examples in Section~\ref{sec:order-matching}.

There is a large literature on electoral systems in general and proportional representation methods in particular. We refer the reader to~\cite{Herron2018,Norris2004}
for elaborate discussions on these topics. A number of proportional representation methods have been proposed in the context of electoral systems. 
These can be divided into two types, the highest averages method and the largest remainder method. 
The most
well known of the highest averages method are the Jefferson/D'Hondt (JD) and the Webster/Sainte-Lagu\"{e} (WS) methods. Both of these methods continue to be of active interest.
See for example~\cite{HM08,Me19,FSS20,GF17}. Among the largest remainder method, the two well known methods are the Hare and the Droop methods. In fact, the
Hare method has the attractive property that it minimises the $L_1$-distance to the ideal allocation. However, the largest remainder methods suffer from 
certain paradoxes (see Section~\ref{sec:order-matching}), which make them unsuitable for use as order matching algorithms. 

In the present context, both JD and the WX methods are well suited to be used as order matching algorithms in stock exchanges. We report simulation studies comparing
the pro-rata, the JD and the WS methods. Such studies show that both the JD and the WS methods are both $L_1$ and $L_2$-better than the pro-rata method. Among the
three methods, the allocation determined by the WS method turns out to be the closest to the ideal allocation in an overwhelming number of cases for both $L_1$ and $L_2$ metrics.
This provides sufficient evidence to seriously consider the adoption of Webster/Sainte-Lagu\"{e} method based order matching by stock exchanges.

\subsection{Procedural Fairness \label{subsec-fair} }
A recent paper by Hersch~\cite{He22} investigated the issue of procedural fairness of order allocation methods. In the paper, it was argued that
both the FIFO and the pro-rata are fair in principle, but not in practice. It was pointed out that the main disadvantage of pro-rata is the requirement
of the second step ``requiring exchanges to introduce secondary matching rules that can be gamed''. 

An alternative method called the random selection for service (RSS) method was proposed. 
Given the vector $\mathbf{T}=(T_1,\ldots,T_n)$ of resting orders, the RSS method defines a probability distribution $\pi$ over $\{1,\ldots,n\}$, where 
$\pi$ associates probability $T_i/T$ to $i$, for $i=1,\ldots,n$, 
Suppose the incoming counter-party order consists of $S$ units. Allocation is done by repeating the following procedure $S$ times:
an independent random $i$ is drawn from $\{1,\ldots,n\}$ following the probability distribution $\pi$ and
one unit is alloted to the $i$-th resting order. At the end of the procedure, let $S_i$ be the number of units alloted to the $i$-th resting order so that the
final allotment is $\mathbf{S}=(S_1,\ldots,S_n)$ satisfying $S=S_1+\cdots+S_n$.
It was argued by Hersch~\cite{He22} that the RSS method is fair in both principle and practice. Below we revisit this method and point out a crucial difference between principle
and practice.

Given $\mathbf{T}=(T_1,\ldots,T_n)$ and $S$, suppose the RSS method is executed $\Gamma$ times and for $\gamma=1,\ldots,\Gamma$, let
the allotment of $\gamma$-th execution be $\mathbf{S}_{\gamma}=(S_{\gamma,1},\ldots,S_{\gamma,n})$. For $i=1,\ldots,n$, let 
$\widehat{S}_i = (S_{1,i}+\cdots+S_{\Gamma,i})/\Gamma$, i.e. $\widehat{S}_i$ is the average allotment to the $i$-th resting order computed over all the
$\Gamma$ trials. The law of large numbers assures us that asymptotically, i.e. as $\Gamma$ goes to infinity, the average allotment $\widehat{S}_i$
tends to $ST_i/T$ which is equal to the $i$-th component of the ideal allocation vector $\mathbf{I}$ (see~\eqref{eqn-ideal}). So in principle, Hersch~\cite{He22} implicitly
considers achieving allocation close to the ideal allocation vector $\mathbf{I}$ to be procedurally fair. To this extent, Hersch's objective and ours coincide.
Additionally, our use of the $L_1$ and $L_2$ metrics to measure deviation from the ideal allocation vector can be considered to be a quantification of procedural fairness.
As such it expands the theoretical framework for studying procedural fairness of the order allocation methods.

From a practical point of view, however, the RSS method has a significant shortcoming. The law of large numbers applies in an asymptotic contex, i.e. as
$\Gamma$ goes to infinity. In practice, given $\mathbf{T}=(T_1,\ldots,T_n)$ and $S$, a stock exchange will execute the RSS method exactly once to obtain a single allocation 
vector $\mathbf{S}=(S_1,\ldots,S_n)$. In other words, in practice the value of $\Gamma$ will be 1. 
The law of large numbers does not say anything about the value obtain in a single execution. In particular,
the $S_i$'s obtained after a single execution of RSS can be any value in the set $\{0,\ldots,S\}$. Considering a particular example with $n=2$, $\mathbf{T}=(10,90)$ and
$S=10$, Hersch~\cite{He22} provides probabilities that the $S_i$'s can take certain values: the probability that $S_1\geq 1$ (resp. $S_1=10$) is about 
0.88 (resp. $7\times 10^{-6}$); the probability that $S_2=20$ (resp. $S_2\geq 10$) is about 0.12 (resp. approaches 1). These probabilities, however, do not
enlighten us about the concrete values of the $S_i$'s after a single execution. In particular, the probability that $S_2\geq 10$ approaches 1 suggests an
asymptotic approach, where the frequentist view of probability is taken. To interpret such probabilities, one again needs to consider a large number $\Gamma$
of trials of the RSS method and consider the average allocation over all the $\Gamma$ trials. 

To test the practical efficacy of the RSS method, we have run experiments with the method. It turns out that the allocation vector obtained by the RSS method
has a very large deviation from the ideal allocation vector in terms of both the $L_1$ and the $L_2$ metrics. Particular examples are provided in Section~\ref{sec-sim-res}.
By the above explanation, this observation is not surprising.

As mentioned earlier, according to Hersch, the main disadvantage of the pro-rata method is the use of secondary matching rules in the second step of the method
which leads to the possibility of gaming. The proportional representation based order allocation method that we introduce does not require any such secondary
matching rules which can be gamed. 
So our proposal overcomes the disadvantage of the pro-rata method pointed out by Hersch. In terms of procedural fairness as measured by
distance to the ideal allocation vector, our simulation studies show that proportional representation based order allocation outperforms the pro-rata method.

\section{Proportional Representation Methods \label{sec:order-matching}}

There are many different proportional representation methods (see~\cite{Herron2018,Norris2004}). Below we describe two well known families of proportional representation 
methods, namely the highest averages or the divisor method, and the highest remainder method. 

\paragraph{Highest averages method.}
Recall the setting described in Section~\ref{subsec-elec}, where $V_i$ is the number of votes received by the $i$-th party and $K$ is the total number of available seats. 
The goal is to determine $K_i$ which is the number of seats alloted to the $i$-th party.
Let $f: \mathbb{Z}^{+} \cup \{0\} \to \mathbb{R}$ be a function from the non-negative integers to the reals. Let $V=V_1+\cdots+V_n$ and $v_i=V_i/V$. 
In the highest averages method, seats are allotted iteratively. The seat distribution algorithm goes through $K$ iterations and in each iteration exactly one seat 
is alloted to one of the parties. Initially, the algorithm sets $K_1=K_2=\cdots=K_n=0$. For $k$ from $1$ to $K$, in the $k$-th iteration
the algorithm determines $j=\arg\max\{v_i/f(K_i): i=1,\ldots,n\}$ and increments $K_j$ by one. After $K$ iterations, the final values of $K_1,\ldots,K_n$ are the numbers of
seats alloted to the various parties. Various different methods arise from the different definitions of $f$. The definitions of $f$ for the well known
Jefferson/D'Hondt and the Webster/Sainte-Lagu\"{e} methods are shown in Table~\ref{tab:highest-averages-methods}.

\begin{table}[!htb]
    \centering
    \begin{tabular}{l|l}
    \hline
        Method name & $f(t)$ \\
    \hline
%	    Adam (Ad) & $\lceil t \rceil$ \\
%	    Danish (Da) & $t/(t+(1/3))$ \\
%	    Dean (De) & ${t(t+1)}/{(t+0.5)}$ \\
%	    Huntington/Hill (HH) & $\sqrt{t(t+1)}$ \\
	    Jefferson/D'Hondt (JD) & $(t+1)$ \\
	    Webster/Sainte-Lagu\"{e} (WS) & $(t+0.5)$ \\
    \hline
    \end{tabular}
	\caption{The functions used for two important methods of proportional representation.  \label{tab:highest-averages-methods}}
\end{table}

Apart from the JD and the WS methods, there are a number of other proportional representation methods, such as Dean's, Adam's, Huntington/Hill and the Danish methods.
(See~\cite{Wiki-division} for a compact description of these methods.)
These four methods require a positivity constraint to be satisfied which is not required in either the JD or the WS methods. They initially allocate one seat to each of the
contesting parties (i.e., they start with $K_1=\cdots=K_n=1$ instead of starting with $K_1=\cdots=K_n=0$) and then employ the highest averages method described above. 
As a result, at the end of the allocation each party has at least one seat. The 
definitions of the function $f$ for these four methods are different from the definitions of the function corresponding to the JD and the WS methods. Importantly, in 
these four methods, the function $f$ satisfies the condition $f(0)=0$. 
Consequently, the function cannot be evaluated unless the present number of seats allocated to a party is at least 1. This constraint is not present for
the JD and the WS methods. Note that the constraint of allocating at least one seat to each party requires the number of seats to be at least as large as the
number of parties.

In the context of order matching, the feature of assigning at least one seat to each party will translate to assiging at least one unit of the 
incoming counter-party order to each of the resting orders. This requires the size of the incoming counter-party order to be at least the number of resting orders,
i.e. $S\geq n$. Such a condition cannot be imposed in general, since there is no control over the size $S$ of the incoming counter-party order. More generally,
the principle of assigning at least one unit of the incoming order to each of the resting orders does not appear to have any justification in the context
of stock exchanges. On the contrary,
some stock exchanges follow the rule that if the order unit determined by the pro-rata method is 1, then this is instead set to 0~\cite{CMEMatchingAlgorithms}.
The principle of at least one unit for each resting order may not be welcome by such exchanges. Due to this reason as well as the unimplementable constraint
of $S\geq n$, in this paper we do not consider the proportional representation methods which follow the principle of assigning at least one seat to each party.

\paragraph{Largest remainder method.}
The method uses a parameter called the quota $Q$. The seat allocation is done in two phases. In the first phase, the $i$-th party is allocated $\lfloor V_i/Q\rfloor$ 
seats. Let $R_i=V_i/Q - \lfloor V_i/Q\rfloor$ be the remainder corresponding to the $i$-th party. Suppose after the first phase $k$ seats remain unallocated. 
In the second phase, the parties with the $k$ largest remainders are each allocated one seat. Let $K_i$ be the number of seats allocated to the $i$-th party at the
end of the second round. The method ensures that $K_1+\cdots+K_n=K$. 
Various methods arise by choosing an appropriate value of the quota $Q$. The Hare quota chooses $Q$ to be equal to $V/K$, while the Droop quota chooses
$Q$ to be equal to $1 + \lfloor V/(1+K) \rfloor$. Other choices for $Q$ have been 
proposed\footnote{\url{https://en.wikipedia.org/wiki/Largest_remainder_method}}. 

The largest remainder method satisfies the quota rule, i.e.  the condition $\lfloor K V_i/V\rfloor \leq K_i \leq \lceil K V_i/V\rceil$.
It is known that the largest remainder method with the Hare quota minimises the Loosemore-Hanby index\footnote{\url{https://en.wikipedia.org/wiki/Loosemore\%E2\%80\%93Hanby_index}}
which is equivalent to minimising the $L_1$-distance from the ideal proportional seat allocation $(KV_1/V,KV_2/V,\ldots,KV_n/V)$. 

\paragraph{Paradoxes.}
Balinski and Young~\cite{BY01} identified several paradoxes of proportional representation. Consider two possible vector of votes and the total number of seats
with $(V_1,\ldots,V_n)$ and $K$ be one of the scenarios and $(V_1^{\prime},\ldots,V_n^{\prime})$ and $K^{\prime}$. Let $(K_1,\ldots,K_n)$ and
$(K_1^{\prime},\ldots,K_n^{\prime})$ be the corresponding seat allocation vectors. 
\begin{itemize}
	\item The Alabama paradox is the following: $V_i=V_i^{\prime}$ for $i=1,\ldots,n$
		and $K^{\prime}>K$, but there is a $j\in \{1,\ldots,n\}$ such that $K_j>K_j^{\prime}$. 
		In other words, the votes polled remain the same and the total number of seats has gone up, but the number of seats allocated to a party has gone down.
	\item The population paradox is the following: $K=K^{\prime}$, $V_i^{\prime}\geq V_i$, and for some $j,k\in\{1,\ldots,n\}$, 
		$V_j^{\prime}-V_j > V_k^{\prime}-V_k$, but $K_j^{\prime}<K$. In other words,
		the number of seats remains the same and the additional votes polled by one party is more than another, but the former party is allocted fewer seats.
\end{itemize}
The Balinski-Young theorem states that any method of apportionment which satisfies the quota rule will necessarily suffer from either the Alabama or the population
paradox. 

In the context of order matching, both the Alabama and the population paradoxes are problematic. The Alabama paradox translates to the following.
The quantities of the resting orders remain the same and an increase occurs in the incoming counter-party order, yet the allocation to a particular resting order goes down. 
The population paradox translates to the following. The size of the incoming counter-party order remains the same, and the size of a particular resting order increases
by an amount which is more than the increase in the size of another resting order, but the units allocated to the former resting order goes down.
Both of these are counterintuitive and hard to explain to the stakeholders of a stock exchange. Due to these paradoxes, the largest remainder method with Hare quota
is unsuitable for use as order matching algorithm even though it minimises the distance to the ideal allocation vector.

The highest averages methods are free of these paradoxes. As a consequence, by the Balinski-Young theorem, they violate the quota rule. It is, however, known that
the while the WS method in principle violates the quota rule, it does so rarely. 

Based on the above discussion, we consider only the JD and the WS method in our simulation studies.

\section{Simulation and Results \label{sec-sim-res}}
%We have implemented the pro-rata, the JD and the WS methods.

Given $n$, $\mathbf{T}=(T_1,\ldots,T_n)$ and $S$, the pro-rata allocation takes place in two steps. In the first step, the vector 
$$\mathbf{S}_{\mathcal{P}}^{\prime}=(S_1^{\prime},S_2^{\prime},\ldots,S_n^{\prime})=(\lfloor ST_1/T \rfloor, \lfloor ST_2/T \rfloor, \ldots, \lfloor  ST_n/T\rfloor)$$ 
is computed. In the second step, the remaining $S-S^{\prime}$ (where $S^{\prime}=S_1^{\prime}+S_2^{\prime}+\cdots+S_n^{\prime}$) quantity of the incoming order 
is allocated using the first-come-first-served strategy. In our implementation, we have used a modified version of the first-come-first-served-strategy where we 
have prioritised smaller orders as follows: allocate one unit to all orders in the first-come-first-served manner which got zero allocation in the first step, 
next allocate one unit to all orders in the first-come-first-served manner which was alloted one unit in the first step, and so on until all the $S-S^{\prime}$ units left
over from the first step are exhausted. The second step increases the allocation to any resting order by at most one unit.
%and we also check and ensure that alloted number of units to the $i$-th resting order is not greater than $ST_i/T$. 
The rationale for using the modified first-come-first-served strategy is to provide some benefit to smaller orders. 
%Since $S-S^{\prime}$ is at most $n$, the overall effect of this modified strategy has a negligible effect on the values of $\ell_{1,\mathcal{P}}$ and $\ell_{2,\mathcal{P}}$, 
%where $\mathcal{P}$ denotes the pro-rata method. 

The pro-rata method clearly takes $O(n)$ time. The highest averages method (of which the JD and the WS methods are special cases)
described in Section~\ref{sec:order-matching} can be implemented in time $O(n+K\log n)$. (By the identification of the size $S$ of the incoming order with the number $K$
of available seats, we have $O(n+K\log n)=O(n+S\log n)$.)
We briefly discuss how this can be done. Note that $O(n)$ time is required to initialise the allocation vector $K=(K_1,\ldots,K_n)$ to the all-zero vector. Next 
a max-heap data structure~\cite{AHU78} is built on the $n$ values $f_1=v_1/f(K_1), f_2=v_2/f(K_2),\ldots, f_n=v_n/f(K_n)$. This takes $O(n)$ time. The heap data structure 
stores the maximum of the $f_i$'s on the top. In each of the $K$ iterations,
exactly one $K_i$ is incremented, so exactly one $f_i$ is modified and the other $f_j$'s remain unchanged. The heap data structure is updated so that the new maximum
gets to the top. This can be done in $O(\log n)$ time and makes the new maximum available for the next iteration. So the $K$ iterations of the highest averages
method take $O(K\log n)$ time. 
%Below we present simulation results which show that the JD and the WS algorithms achieve better proportionality. The trade-off is that both of these algorithms
%take more time than the pro-rata method.

To compare the performances of the different algorithms, we have performed simulation studies. The input to an order matching algorithm is the number of resting
orders $n$, the vector $\mathbf{T}=(T_1,\ldots,T_n)$, where $T_i$ is the size of the $i$-th order, and the size $S$ of the incoming counter-party order. 
In our simulations, we have randomly generated the values $T_1,\ldots,T_n$ using the following strategy. Fix two non-negative integers $m$ and $M$ with $m<M$. 
Let $\mu$ and $\sigma$ be positive real numbers which specify the normal ${\mathcal N}(\mu,\sigma)$ distribution. For each $i$ in $1$ to $n$, the following
procedure is performed: draw a sample from ${\mathcal N}(\mu,\sigma)$ and round to the nearest integer, repeat until the rounded value is in the range $[m,M]$; 
once the rounded value satisfies the range check, set $T_i$ to be equal to this rounded value.
After $n$ iterations, we obtain the random vector $\mathbf{T}=(T_1,\ldots,T_n)$ which is a simulated distribution of the resting orders. 
Note that all the samples are drawn {\em independently} from ${\mathcal N}(\mu,\sigma)$. Our rationale for
choosing the normal distribution is that in the absence of any other information, the sizes of the orders may be assumed to follow the normal distribution. If, on the
other hand, additional information is available, then it is possible to change the normal distribution to another distribution without affecting the rest of the
simulation. 

In Table~\ref{tab-ex}, we provide some examples of the order allocation vector $\mathbf{S}_{\mathcal{A}}$, where 
$\mathcal{A}$ is one of $\mathcal{P}$ (denoting the pro-rata method), RSS, JD, or WS method. 
The ideal allocation vector is $\mathbf{I}=(ST_1/T,\ldots,ST_n/T)$. 
%For the pro-rata method, we provide both the vectors $\mathbf{S}_{\mathcal{P}}^{\prime}$ (the output of the first step of the pro-rata method) and 
%$\mathbf{S}_{\mathcal{P}}$ (the final output of the pro-rata method). 
From the examples, we observe that for the RSS method, the $L_1$ and $L_2$ distances from the ideal allocation vector $\mathbf{I}$ are much larger than these
distances from the other methods. This is as expected (see Section~\ref{subsec-fair}) and highlights the impracticability of the RSS method. While the table
provides only three examples, we have obtained many other examples and the observation that the order allocation vector produced by the RSS method
is much farther away from the ideal allocation vector compared to the other methods holds in all the examples. In view of this, we do not consider the
RSS method any further in our simulation studies.

Among the three methods, i.e. the pro-rata method, the JD and the WS methods, note that in all the cases, the JD and the WS methods are both $L_1$-better and 
$L_2$-better than the pro-rata method. Comparing
$\ell_{1,\mathcal{P}}$ with $\ell_{1,{\rm JD}}$ and $\ell_{1,{\rm WS}}$ and $\ell_{2,\mathcal{P}}$ with $\ell_{2,{\rm JD}}$ and $\ell_{2,{\rm WS}}$, we find significant
difference in these values. So these examples suggest that the JD and the WS methods are significantly better than the pro-rata method with respect to the 
$L_1$ and $L_2$ metrics.

\begin{table}
\centering
{\scriptsize
	\begin{tabular}{|l|l|r|r|r|r|r|r|r|r|r|r|r|r|}
		\cline{13-14}
		\multicolumn{12}{c|}{} & $\ell_{1,\mathcal{A}}$ & $\ell_{2,\mathcal{A}}$ \\ \hline
		\multirow{5}{*}{Ex~1}
		& $\mathbf{T}$ & 209 & 727 & 746 & 808 & 995 & 204 & 598 & 773 & 979 & 899 & - & - \\ \cline{2-14}
		& $\mathbf{I}$ & 3.01 & 10.48 & 10.75 & 11.65 & 14.34 & 2.94 & 8.62 & 11.14 & 14.11 & 12.96 & - & - \\ \cline{2-14}
		%& $\mathbf{S}_{\mathcal{P}}^{\prime}$ & 3 & 10 & 10 & 11 & 14 & 2 & 8 & 11 & 14 & 12 & - & -\\ \cline{2-14}
		& $\mathbf{S}_{\mathcal{P}}$ & 4 & 11 & 11 & 11 & 14 & 3 & 9 & 11 & 14 & 12 & 4.39 & 2.94 \\ \cline{2-14}
		& $\mathbf{S}_{{\rm RSS}}$ & 4 & 7 & 17 & 11 & 18 & 3 & 5 & 10 & 15 & 10 & 23.69 & 89.86 \\ \cline{2-14}
		& $\mathbf{S}_{{\rm JD}}$ & 3 & 10 & 11 & 12 & 14 & 3 & 9 & 11 & 14 & 13 & 2.17 & 0.71 \\ \cline{2-14}
		& $\mathbf{S}_{{\rm WS}}$ & 3 & 10 & 11 & 12 & 14 & 3 & 9 & 11 & 14 & 13 & 2.17 & 0.71 \\ \hline
		\multirow{5}{*}{Ex~2}
		& $\mathbf{T}$ & 1 & 655 & 307 & 138 & 647 & 48 & 625 & 382 & 95 & 424 & - & - \\ \cline{2-14}
		& $\mathbf{I}$ & 0.03 & 19.72 & 9.24 & 4.15 & 19.48 & 1.44 & 18.81 & 11.50 & 2.86 & 12.76 & - & - \\ \cline{2-14}
		%& $\mathbf{S}_{\mathcal{P}}^{\prime}$ & 0 & 19 & 9 & 4 & 19 & 1 & 18 & 11 & 2 & 12 & - & -\\ \cline{2-14}
		& $\mathbf{S}_{\mathcal{P}}$ & 1 & 19 & 10 & 5 & 19 & 2 & 18 & 11 & 3 & 12 & 6.54 & 4.79 \\ \cline{2-14}
		& $\mathbf{S}_{{\rm RSS}}$ & 0 & 21 & 15 & 2 & 21 & 1 & 17 & 8 & 4 & 11 & 19.41 & 61.91 \\ \cline{2-14}
		& $\mathbf{S}_{{\rm JD}}$ & 0 & 20 & 9 & 4 & 20 & 1 & 19 & 12 & 2 & 13 & 3.46 & 1.72 \\ \cline{2-14}
		& $\mathbf{S}_{{\rm WS}}$ & 0 & 20 & 9 & 4 & 19 & 1 & 19 & 12 & 3 & 13 & 2.69 & 0.95 \\ \hline
		\multirow{5}{*}{Ex~3}
		& $\mathbf{T}$ & 268 & 806 & 409 & 420 & 869 & 659 & 189 & 317 & 286 & 721 & - & - \\ \cline{2-14}
		& $\mathbf{I}$ & 5.42 & 16.30 & 8.27 & 8.50 & 17.58 & 13.33 & 3.82 & 6.41 & 5.78 & 14.58 & - & - \\ \cline{2-14}
		%& $\mathbf{S}_{\mathcal{P}}^{\prime}$ & 5 & 16 & 8 & 8 & 17 & 13 & 3 & 6 & 5 & 14 & - & -\\ \cline{2-14}
		& $\mathbf{S}_{\mathcal{P}}$ & 6 & 16 & 9 & 8 & 17 & 13 & 4 & 7 & 6 & 14 & 4.57 & 2.41 \\ \cline{2-14}
		& $\mathbf{S}_{{\rm RSS}}$ & 2 & 15 & 8 & 4 & 12 & 17 & 2 & 6 & 9 & 25 & 34.61 & 200.59 \\ \cline{2-14}
		& $\mathbf{S}_{{\rm JD}}$ & 5 & 17 & 8 & 8 & 18 & 13 & 4 & 6 & 6 & 15 & 3.86 & 1.69 \\ \cline{2-14}
		& $\mathbf{S}_{{\rm WS}}$ & 5 & 16 & 8 & 9 & 18 & 13 & 4 & 6 & 6 & 15 & 3.47 & 1.31 \\ \hline
	\end{tabular}
	\caption{Examples of simulation runs with $n=10$, $S=100$, $m=1$, $M=1000$, $\mu=500$ and $\sigma=400$. Here $\mathcal{P}$ is the pro-rata method. \label{tab-ex}}
}
\end{table}

A few examples do not provide sufficient evidence. It is required to consider many more examples. On the other hand, when there are a large number of examples,
it is not possible to visually inspect all such examples. So we have used a program to perform the comparison for the various simulation studies.
Since $\mathbf{T}$ is determined by $n$, $\mu$ and $\sigma$, the parameters for the simulations are the different values of $n$, $\mu$ and $\sigma$ as well as $S$. 
For a specific set of values of $n$, $\mu$, $\sigma$ and $S$, we have performed $N$ iterations of the simulation. In each iteration, we have computed 
the ideal allocation vector $\mathbf{I}=(ST_1/T,\ldots,ST_n/T)$, and the order allocation vector
$\mathbf{S}_{\mathcal{A}}=(S_1,\ldots,S_n)$ produced by the order matching algorithm $\mathcal{A}$, where $\mathcal{A}$ is one of pro-rata, the JD or the WS algorithms.
%The ideal proportional allocation vector in each iteration is $\mathbf{I}=(ST_1/T,\ldots,ST_n/T)$ and the output of algorithm $\mathcal{A}$ is 
%given by the vector $\mathbf{S}_{\mathcal{A}}=(S_1,S_2,\ldots,S_n)$. 
Next we computed the $L_1$ and $L_2$ distances of $\mathbf{S}_{\mathcal{A}}$ from $\mathbf{I}$ given by $\ell_{1,\mathcal{A}}$ and $\ell_{2,\mathcal{A}}$. 
In each of the $N$ iterations, we have compared the JD and the WS methods with the pro-rata method. After $N$ iterations, aggregate statistics are determined
for the particular simulation. Further details are given below.

In our experiments, we have taken the number of iterations $N$ to be 10000. The parameters of the various simulation runs are given in Table~\ref{tab-sim-param}.
To obtain an idea of the comparison between the different algorithms, we have considered a number of variations in the parameters. The value of $n$ has been chosen
to be as low as 10 to a moderate value of 100, while the value of $S$ has been taken to be as small as 30 to as large as 3000. While we report results for the
values of parameters shown in Table~\ref{tab-sim-param}, we have also experimented with various other values. The results in all such cases turned out to be very similar 
to the results that we report here. 

Table~\ref{tab-res} provides a summary of the results that we obtained from the simulations.
The columns of the table list the pro-rata method along with the JD and the WS methods.
The rows correspond to the various simulation runs whose parameters are given in Table~\ref{tab-sim-param}. For a row starting with $L_1$, all entries in the
corresponding row are with respect to the $L_1$ metric. Similarly for a row starting with $L_2$, all entries in the corresponding row are
with respect to the $L_2$ metric. Each entry in the table is a pair of numbers. 
Suppose $(x_1,x_2)$ appears in the column headed by algorithm $\mathcal{A}$ in a row corresponding to simulation number $s$ for the $L_1$ metric.
The value $x_1$ is the percentage of times that $\ell_{1,\mathcal{A}}$ came out to be lower than $\ell_{1,\mathcal{P}}$ (where $\mathcal{P}$ denotes
the pro-rata method) in simulation number $s$, while
the value $x_2$ is the percentage of times that $\ell_{1,\mathcal{A}}$ came out to be the minimum among all the three methods.
For example, the pair $(100.00,99.31)$ appearing under the column headed by WS in row labelled Sim~1 and starting with $L_1$ indicates that
the WS method is $L_1$-better than the pro-rata method in 100\% of the cases (i.e., in all the iterations of Sim~1); further, with respect to the $L_1$ metric, 
in 99.31\% of the iterations in Sim~1, the WS method provides the closest approximation to the ideal allocation among all the three methods. 
A similar interpretation holds for a pair of values appearing in a row starting with $L_2$, with the only change being that the $L_1$ metric is
replaced by the $L_2$ metric. Note that for each pair under the column headed pro-rata, the first entry is a `-', since it is not meaningful to compare 
pro-rata method with itself. Also, it is possible that in a particular iteration, the distances of two of the methods to the ideal are both minimum; so
the sum of the percentages of cases for which the different methods are minimum can be greater than 100. 

The simulation results bring out two important issues.
\begin{enumerate}
	\item Both the JD and the WS methods are better than the pro-rata method for an overwhelming number of cases.
	\item Among the three algorithms, the WS method provides the closest approximation to the ideal allocation in most of the cases.
\end{enumerate}
Consequently, the simulations provide sufficient evidence for stock exchanges to seriously consider the adoption of the Webster/Saint-Lagu\"{e} based order matching 
algorithm as a replacement of the pro-rata order matching algorithm.


\begin{table}
\begin{subtable}{0.5\textwidth}
	\centering
	{\scriptsize
		\begin{tabular}{|l|r|r|r|r|r|r|}
			\cline{2-7}
			\multicolumn{1}{c|}{} & \multicolumn{1}{c|}{$n$} & \multicolumn{1}{c|}{$m$} & 
			\multicolumn{1}{c|}{$M$} & \multicolumn{1}{c|}{$\mu$} & \multicolumn{1}{c|}{$\sigma$} & \multicolumn{1}{c|}{$S$} \\ \hline
			Sim~1 & 20 & 1 & 1000 & 500 & 400 & 50 \\ \hline 
			Sim~2 & 200 & 1 & 1000 & 500 & 400 & 30 \\ \hline 
			Sim~3 & 10 & 1 & 10000 & 5000 & 3000 & 300 \\ \hline 
			Sim~4 & 100 & 1 & 10000 & 5000 & 3000 & 300 \\ \hline 
			Sim~5 & 100 & 1 & 10000 & 5000 & 3000 & 3000 \\ \hline 
		\end{tabular}
		}
		\caption{Parameters of the various simulation runs. \label{tab-sim-param} }
\end{subtable}
\begin{subtable}{0.5\textwidth}
	\centering
	{\scriptsize
		\begin{tabular}{|l|l|r|r|r|}
			\cline{3-5} 
			\multicolumn{2}{c|}{} & \multicolumn{1}{c|}{pro-rata} & \multicolumn{1}{c|}{JD} & \multicolumn{1}{c|}{WS} \\ \hline
			\multirow{2}{*}{Sim~1} & 
			  $L_1$ & (-, 0.00) & (96.08, 4.79) & (100.00, 99.31) \\ \cline{2-5}
			& $L_2$ & (-, 0.01) & (95.99, 0.01) & ( 99.99, 99.30) \\ \hline
			\multirow{2}{*}{Sim~2} & 
			  $L_1$ & (-, 0.00) & (100.00, 100.00) & (100.00, 100.00) \\ \cline{2-5}
			& $L_2$ & (-, 0.00) & (100.00, 100.00) & (100.00, 100.00) \\ \hline
			\multirow{2}{*}{Sim~3} & 
			  $L_1$ & (-, 0.42) & (91.11, 25.58) & (99.65, 97.09) \\ \cline{2-5}
			& $L_2$ & (-, 0.46) & (90.89, 25.57) & (99.57, 97.02) \\ \hline
			\multirow{2}{*}{Sim~4} & 
			  $L_1$ & (-, 0.00) & (100.00, 0.00) & (100.00, 100.00) \\ \cline{2-5}
			& $L_2$ & (-, 0.00) & (100.00, 0.00) & (100.00, 100.00) \\ \hline
			\multirow{2}{*}{Sim~5} & 
			  $L_1$ & (-, 0.00) & (100.00, 0.00) & (100.00, 100.00) \\ \cline{2-5}
			& $L_2$ & (-, 0.00) & (100.00, 0.00) & (100.00, 100.00) \\ \hline
		\end{tabular}
		}
		\caption{Summary of simulation results. \label{tab-res}}
\end{subtable}
	\caption{Simulation parameters and summary. \label{tab-sim} }
\end{table}



%\bibliographystyle{alpha}
%\bibliography{prop-alloc}
\input{prop-alloc.bbl}

\end{document}

\newpage 

\section{Statements \& Declarations}

%The following statements must be included in your submitted manuscript under the heading 'Statements and Declarations'. This should be placed after the References section. Please note that submissions that do not include required statements will be returned as incomplete.

\subsection{Funding}
The authors declare that no funds, grants, or other support were received during the preparation of this manuscript.

%Please describe any sources of funding that have supported the work. The statement should include details of any grants received (please give the name of the funding agency and grant number).

%Example statements:

\subsection{Competing Interests}
The authors have no relevant financial or non-financial interests to disclose.

%Authors are required to disclose financial or non-financial interests that are directly or indirectly related to the work submitted for publication. Interests within the last 3 years of beginning the work (conducting the research and preparing the work for submission) should be reported. Interests outside the 3-year time frame must be disclosed if they could reasonably be perceived as influencing the submitted work.

\subsection{Author Contributions}
All authors contributed to the study conception and design. 
%
%Authors are encouraged to include a statement that specifies the contribution of every author to the research and preparation of the manuscript.
%
%Example statement:
%
%“All authors contributed to the study conception and design. Material preparation, data collection and analysis were performed by [full name], [full name] and [full name]. The first draft of the manuscript was written by [full name] and all authors commented on previous versions of the manuscript. All authors read and approved the final manuscript.”
%
%Please refer to the “Authorship Principles ” section below for more information on how to complete this section.


\end{document}


%On going through the above paper, another measure of disproportionality occurred to me. I am describing this below, but it is not for consideration in the context of order matching. Let V_i and K_i be the vote counts and seats. Then the ideal proportion is V_i/V and the actual proportion is K_i/K. I think a basic requirement is that V_i/V \geq K_i/K. Define a_i = V_i/V - K_i/K. Then one may consider the inequality in the vector (a_1,...a_n) as a measure of equitable distribution of the votes to the seats. One may say that it is desirable to obtain an allocation which minimises this inequality. The above paper mentions the use of inequality measures in the context of disproportionality, but in a different manner.


\end{document}

\newpage 

\section{Statements \& Declarations}

%The following statements must be included in your submitted manuscript under the heading 'Statements and Declarations'. This should be placed after the References section. Please note that submissions that do not include required statements will be returned as incomplete.

\subsection{Funding}
The authors declare that no funds, grants, or other support were received during the preparation of this manuscript.

%Please describe any sources of funding that have supported the work. The statement should include details of any grants received (please give the name of the funding agency and grant number).

%Example statements:

\subsection{Competing Interests}
The authors have no relevant financial or non-financial interests to disclose.

%Authors are required to disclose financial or non-financial interests that are directly or indirectly related to the work submitted for publication. Interests within the last 3 years of beginning the work (conducting the research and preparing the work for submission) should be reported. Interests outside the 3-year time frame must be disclosed if they could reasonably be perceived as influencing the submitted work.

\subsection{Author Contributions}
All authors contributed to the study conception and design. 
%
%Authors are encouraged to include a statement that specifies the contribution of every author to the research and preparation of the manuscript.
%
%Example statement:
%
%“All authors contributed to the study conception and design. Material preparation, data collection and analysis were performed by [full name], [full name] and [full name]. The first draft of the manuscript was written by [full name] and all authors commented on previous versions of the manuscript. All authors read and approved the final manuscript.”
%
%Please refer to the “Authorship Principles ” section below for more information on how to complete this section.


\end{document}


%On going through the above paper, another measure of disproportionality occurred to me. I am describing this below, but it is not for consideration in the context of order matching. Let V_i and K_i be the vote counts and seats. Then the ideal proportion is V_i/V and the actual proportion is K_i/K. I think a basic requirement is that V_i/V \geq K_i/K. Define a_i = V_i/V - K_i/K. Then one may consider the inequality in the vector (a_1,...a_n) as a measure of equitable distribution of the votes to the seats. One may say that it is desirable to obtain an allocation which minimises this inequality. The above paper mentions the use of inequality measures in the context of disproportionality, but in a different manner.


\end{document}

\newpage 

\section{Statements \& Declarations}

%The following statements must be included in your submitted manuscript under the heading 'Statements and Declarations'. This should be placed after the References section. Please note that submissions that do not include required statements will be returned as incomplete.

\subsection{Funding}
The authors declare that no funds, grants, or other support were received during the preparation of this manuscript.

%Please describe any sources of funding that have supported the work. The statement should include details of any grants received (please give the name of the funding agency and grant number).

%Example statements:

\subsection{Competing Interests}
The authors have no relevant financial or non-financial interests to disclose.

%Authors are required to disclose financial or non-financial interests that are directly or indirectly related to the work submitted for publication. Interests within the last 3 years of beginning the work (conducting the research and preparing the work for submission) should be reported. Interests outside the 3-year time frame must be disclosed if they could reasonably be perceived as influencing the submitted work.

\subsection{Author Contributions}
All authors contributed to the study conception and design. 
%
%Authors are encouraged to include a statement that specifies the contribution of every author to the research and preparation of the manuscript.
%
%Example statement:
%
%“All authors contributed to the study conception and design. Material preparation, data collection and analysis were performed by [full name], [full name] and [full name]. The first draft of the manuscript was written by [full name] and all authors commented on previous versions of the manuscript. All authors read and approved the final manuscript.”
%
%Please refer to the “Authorship Principles ” section below for more information on how to complete this section.


\end{document}


%On going through the above paper, another measure of disproportionality occurred to me. I am describing this below, but it is not for consideration in the context of order matching. Let V_i and K_i be the vote counts and seats. Then the ideal proportion is V_i/V and the actual proportion is K_i/K. I think a basic requirement is that V_i/V \geq K_i/K. Define a_i = V_i/V - K_i/K. Then one may consider the inequality in the vector (a_1,...a_n) as a measure of equitable distribution of the votes to the seats. One may say that it is desirable to obtain an allocation which minimises this inequality. The above paper mentions the use of inequality measures in the context of disproportionality, but in a different manner.


\end{document}

\newpage 

\section{Statements \& Declarations}

%The following statements must be included in your submitted manuscript under the heading 'Statements and Declarations'. This should be placed after the References section. Please note that submissions that do not include required statements will be returned as incomplete.

\subsection{Funding}
The authors declare that no funds, grants, or other support were received during the preparation of this manuscript.

%Please describe any sources of funding that have supported the work. The statement should include details of any grants received (please give the name of the funding agency and grant number).

%Example statements:

\subsection{Competing Interests}
The authors have no relevant financial or non-financial interests to disclose.

%Authors are required to disclose financial or non-financial interests that are directly or indirectly related to the work submitted for publication. Interests within the last 3 years of beginning the work (conducting the research and preparing the work for submission) should be reported. Interests outside the 3-year time frame must be disclosed if they could reasonably be perceived as influencing the submitted work.

\subsection{Author Contributions}
All authors contributed to the study conception and design. 
%
%Authors are encouraged to include a statement that specifies the contribution of every author to the research and preparation of the manuscript.
%
%Example statement:
%
%“All authors contributed to the study conception and design. Material preparation, data collection and analysis were performed by [full name], [full name] and [full name]. The first draft of the manuscript was written by [full name] and all authors commented on previous versions of the manuscript. All authors read and approved the final manuscript.”
%
%Please refer to the “Authorship Principles ” section below for more information on how to complete this section.


\end{document}


%On going through the above paper, another measure of disproportionality occurred to me. I am describing this below, but it is not for consideration in the context of order matching. Let V_i and K_i be the vote counts and seats. Then the ideal proportion is V_i/V and the actual proportion is K_i/K. I think a basic requirement is that V_i/V \geq K_i/K. Define a_i = V_i/V - K_i/K. Then one may consider the inequality in the vector (a_1,...a_n) as a measure of equitable distribution of the votes to the seats. One may say that it is desirable to obtain an allocation which minimises this inequality. The above paper mentions the use of inequality measures in the context of disproportionality, but in a different manner.
