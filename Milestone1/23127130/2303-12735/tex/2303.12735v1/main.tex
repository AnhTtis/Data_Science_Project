\documentclass{article}

% Template for ICASSP-2021 paper; to be used with:
%          spconf.sty  - ICASSP/ICIP LaTeX style file, and
%          IEEEbib.bst - IEEE bibliography style file.
% --------------------------------------------------------------------------

\usepackage{spconf,amsmath,amsfonts,amssymb,graphicx}
\DeclareMathOperator{\tr}{tr}
\DeclareMathOperator{\card}{card}
\DeclareMathOperator{\cov}{cov}
\DeclareMathOperator{\diag}{diag}
\DeclareMathOperator*{\minimize}{\text{minimize}}
\DeclareMathOperator*{\maximize}{\text{maximize}}
\usepackage{adjustbox} 
\newcommand{\bbox}{\text{bbox}}
\newcommand{\alphapck}{\alpha_\bbox}
\newcommand{\kcycle}{\text{k-CyPCK}}
\newcommand{\cycle}{\text{-CyPCK}}

\newcommand{\I}{\mathbf{I}}
\newcommand{\Ia}{\I^\text{a}}
\newcommand{\Ib}{\I^\text{b}}
\newcommand{\Iatob}{\I^\text{a $\rightarrow$ b}}
\newcommand{\F}{\mathbf{F}}
\newcommand{\Fa}{\F^\text{a}}
\newcommand{\Fb}{\F^\text{b}}
\newcommand{\f}{\mathbf{f}}
\newcommand{\fa}{\f^\text{a}}
\newcommand{\fb}{\f^\text{b}}
\newcommand{\p}{\mathbf{p}}
\newcommand{\pa}{\p^\text{a}}
\newcommand{\pb}{\p^\text{b}}
\newcommand{\A}{\boldsymbol{\Phi}_\text{align}}
\newcommand{\G}{\mathbf{G}}
\newcommand{\C}{\mathbf{C}}
\newcommand{\Ca}{\C^\text{a}}
\newcommand{\Cb}{\C^\text{b}}
\newcommand{\cc}{\mathbf{c}}
\newcommand{\cca}{\cc^\text{a}}
\newcommand{\ccb}{\cc^\text{b}}
\newcommand{\Irec}{\I_\text{Recon}}
\newcommand{\M}{\mathbf{M}}
\newcommand{\Mrec}{\M_\text{Recon}}
\newcommand{\loss}{\mathcal{L}}
\newcommand{\T}{\mathcal{T}}
\newcommand{\W}{\mathcal{W}}
\newcommand{\Id}{\mathcal{I}}

% \usepackage{threeparttable}
\usepackage{booktabs}
\usepackage{wrapfig}
\usepackage{array}
\usepackage{tablefootnote}
\usepackage{subcaption}
\usepackage{pifont}
\usepackage{colortbl}
\usepackage{hyperref} 
\usepackage{bm}

\DeclareMathOperator*{\st}{\text{subject to}}
\DeclareMathAlphabet\mathbfcal{OMS}{cmsy}{b}{n}
% Example definitions.
% --------------------
% \def\x{{\mathbf x}}
% \def\L{{\cal L}}

\usepackage{color}
\newcommand{\SL}[1]{\textcolor{red}{SL: #1}}
\newcommand{\hui}[1]{\textcolor{blue}{Hui: #1}}
\newcommand{\JH}[1]{\textcolor{cyan}{Jinghan: #1}}
\newcommand{\yuguang}[1]{\textcolor{green}{yuguang: #1}}
\newcommand{\modl}{\textsc{MoDL}}
\newcommand{\us}{\textsc{SMUG}}
\newcommand{\usold}{\textsc{SMUGv0}}
\usepackage{cite}
\usepackage{tcolorbox}


% Title.
% ------

\title{SMUG: Towards Robust MRI Reconstruction by Smoothed Unrolling}
%
% Single address.
% ---------------
\name{Hui Li$^1$ \, Jinghan Jia$^2$ \, Shijun Liang$^2$ \, Yuguang Yao$^2$ 
\, Saiprasad Ravishankar$^{2}$ \,  Sijia Liu$^2$}
%\thanks{Thanks to XYZ agency for funding.}
\address{$^1$Huazhong University of Science and Technology, China
\\
%$^2$
%Dept. of Computer Science Engineering, 
$^2$Michigan State University, East Lansing, MI, USA 
% \\ $^3$Dept. of Biomedical Engineering, Michigan State University, East Lansing, MI\\ $^4$Dept. of Computational Mathematics, Science and Engineering, Michigan State University, East Lansing, MI.
% \\
% $^3$Dept. CMSE,  Michigan State University, USA
\thanks{$^1$The work is done during remote internship at MSU.}
}
%
% For example:
% ------------
%\address{School\\
%	Department\\
%	Address}
%
% Two addresses (uncomment and modify for two-address case).
% ----------------------------------------------------------
%\twoauthors
%  {A. Author-one, B. Author-two\sthanks{Thanks to XYZ agency for funding.}}
%	{School A-B\\
%	Department A-B\\
%	Address A-B}
%  {C. Author-three, D. Author-four\sthanks{The fourth author performed the work
%	while at ...}}
%	{School C-D\\
%	Department C-D\\
%	Address C-D}
%
\begin{document}
\ninept
%
\maketitle
%


% \newcommand{\argmin}{\operatornamewithlimits{arg\,min}}
% \newcommand{\argmax}{\operatornamewithlimits{arg\,max}}
% \def\NoNumber#1{{\def\alglinenumber##1{}\State #1}\addtocounter{ALG@line}{-1}}


%\newcolumntype{M{>{\centering\arraybackslash}m{\dimexpr.25\linewidth-2\tabcolsep}}


\newcommand{\Def}[0]{\mathrel{\mathop:}=}


\newcommand{\Blue}[1]{\textcolor[rgb]{0.00,0.00,1.00}{#1}}
\newcommand{\tiger}[1]{\Blue{#1}}
 
\newcommand{\bx}{\mathbf{x}}
\newcommand{\by}{\mathbf{y}}
\newcommand{\bz}{\mathbf{z}}
\newcommand{\bX}{\mathbf{X}}
\newcommand{\bh}{\mathbf{h}}
\newcommand{\bu}{\mathbf{u}}
\newcommand{\bg}{\mathbf{g}}
\newcommand{\din}{\mathcal D^{\texttt{tr}}}
\newcommand{\dint}{\tilde{\mathcal D}^{\texttt{tr}}}
\newcommand{\dout}{\mathcal D^{\texttt{val}}}
\newcommand{\doutt}{\tilde{\mathcal D}^{\texttt{val}}}
\newcommand{\btheta}{\boldsymbol{\theta}}
\newcommand{\bphi}{\boldsymbol{\phi}}
\newcommand{\bdelta}{\boldsymbol{\delta}}
\newcommand{\grad}{\nabla}
\newcommand{\egrad}{\widehat{\nabla}}
\newcommand{\task}{\mathcal{T}}
\newcommand{\rnn}{\text{{\tt RNN}}\xspace}
\newcommand{\EI}{\text{EI}}
\newcommand{\lopt}{\text{LO}}
\newcommand{\V}{\mathbb{V}}
\newcommand{\indep}{\perp \!\!\! \perp}
\newcommand{\pphi}[1]{\frac{\partial #1}{\partial \bphi}}


%\renewcommand{\baselinestretch}{0.92}

\newcommand{\pre}{\textsc{p}}
\newcommand{\ft}{\textsc{f}}
\newcommand{\adv}{\textsc{a}}
\newcommand{\sta}{\textsc{n}}
\newcommand{\super}{\textsc{s}}
\newcommand{\self}{\textsc{ss}}
\newcommand{\Sp}{\textit{supervised}}
\newcommand{\Ss}{\textit{self-supervised}}

\newcommand{\cL}{\mathcal{L}}
\newcommand{\cD}{\mathcal{D}}

\begin{abstract}
% \JH{Should we mention robustness against unrolling steps, sampling rate?}
Although deep learning ({DL}) has gained much popularity for accelerated magnetic resonance imaging ({MRI}), recent studies have shown that DL-based MRI reconstruction models could be over-sensitive to  tiny input perturbations (that are called `adversarial perturbations'), which cause unstable, low-quality reconstructed images.
This raises the question of how to design robust DL methods for  MRI reconstruction. 
To address this problem, we propose  a novel image reconstruction framework, termed \textsc{\underline{Sm}oothed \underline{U}nrollin\underline{g}} ({\us}), which advances a deep unrolling-based MRI reconstruction model using a randomized smoothing (RS)-based robust learning operation. RS, which improves the tolerance of a model against input noises, has been widely used in the design of adversarial defense for image classification. Yet, we find that  the conventional design that applies RS to  the entire DL process is ineffective for  MRI reconstruction. We show that {\us} addresses the above issue by customizing the RS operation based on the unrolling  architecture  of the DL-based MRI reconstruction model. Compared to   the vanilla RS approach and several variants of {\us},  we show that 
{\us} improves the robustness of MRI reconstruction with respect to a diverse set of  perturbation sources, including perturbations to input measurements, different measurement sampling rates, and different unrolling steps. Code for {\us} will be available at \texttt{\url{https://github.com/LGM70/SMUG}}.
%Although RS has been widely used in adversarial defense for image classification, 
%the problem of MRI reconstruction imposes two challenges: 
% We find that   the tradition method of mounting RS to the end-to-end DL process (like RS for robust image classification) become ineffective to the deep unrolling-based MRI reconstruction. Thus, a careful design is needed to integrate RS into MRI reconstruction. We show that 

% tiny  perturbations to the input image and t transformations, or changes in the number of samples will lead to the instabilities of DL-based MRI reconstruction models. 

% In this paper, we revisit randomized smoothing (RS) on the DL-based MRI reconstruction models; inspired by the special architecture of deep unrolling-based MRI reconstruction models, we propose a novel framework, called \textsc{\underline{Sm}ooth \underline{U}nrollin\underline{g}} ({\us}), that can integrate RS and deep unrolling-based MRI reconstruction models naturally to enhance the robustness of DL-based MRI reconstruction models further compared to the origin randomized smoothing method. We also conduct extensive experiments to demonstrate the effectiveness of our proposed methods.

%{inspired by and extended from the celebrated randomized smoothing (RS) technique in adversarial learning %{\hui{adversarial defense?}}.} 
%{Nonetheless, unlike conventional RS-based robustification techniques for classification networks,we work with deep unrolling-based reconstruction networks (\textit{e.g.}, {\modl}) and we demonstrate that {\us} with unrolling loss training scheme improve the robustness of MODL}

\end{abstract}
%
\begin{keywords}
Magnetic resonance imaging (MRI), machine learning, deep unrolling, adversarial robustness, randomized smoothing.
\end{keywords}
%
\section{Introduction}
\label{sec:intro}
%\SL{[1) Review the importance of DL-based image reconstruction; 2) Despite its high-accuracy, suffers the vulnerability to adversarial attacks. [Need attack definition]. However, nearly all of work towards robust image reconstruction focused on prediction-evasion attacks. 3) Then define and introduce `train-time poisoning attacks'. Why is challenging in image reconstruction. How you address and contributions.]}

%\SL{[Need more references/related work.]}

% \noindent \textbf{Relevant works.}


%\SL{[Needs to rewrite Introduction. (1) One paragraph for DL in MRI reconstruction, and introduction of MoDL (this is our focused model), (2) one paragraph for lack of robustness of DL-based MRI reconstruction, (3) one paragraph for randomized smoothing (e.g., in image classification), and the most relevant work to ours. (4) state your contributions.]}

Magnetic resonance imaging (MRI) is a widely used imaging modality in clinical practice that is used to image both anatomical structures and physiological functions. However, the data collection in MRI is sequential and slow. Thus, many methods\cite{lustig2008compressed,yang2010fast,Aggarwal2019MoDL:Problems} have been developed to provide accurate image reconstructions from limited (rapidly collected) data.

Recently, deep learning (DL) has become a powerful tool   to solve  image reconstruction and inverse problems in general ~\cite{Schlemper2019Sigma-net:Reconstruction,Ravishankar2018DeepReconstruction,Aggarwal2019MoDL:Problems,Schlemper2018AReconstruction}.  In this paper, we focus on the application of DL to MRI reconstruction.
%This paper focuses on the application of DL to magnetic resonance imaging (MRI) reconstruction.
% such as X-ray computed tomography and magnetic resonance imaging (MRI). 
% With the advent of deep learning-based ({DL-based}) approaches in computer vision tasks, convolutional neural networks have gained popularity in image denoising, classification for medical image diagnosis, and accelerated magnetic resonance imaging. 
%\SL{[The two types are unclear.]}
%The DL-based MRI reconstruction approaches can be roughly divided into two types. 
Among DL-based methods, image or sensor domain denoising networks are well-known. The most prevalent deep neural networks include the U-Net~\cite{Unet} and variants~\cite{han2018framing, lee2018deep} that are adapted to correct the artifacts in MRI reconstructions from undersampled data. 
Hybrid-domain methods that combine neural networks together with imaging physics such as forward models have become quite popular. 
One such state-of-the-art algorithm is the unrolled network scheme, MoDL~\cite{Aggarwal2019MoDL:Problems} that mimics an iterative algorithm to solve the regularized inverse problem in MRI reconstruction. Its variants have achieved top performance in recent open data-driven competitions. 
% This method consists of multiple iterations or blocks, as ``unrolled'' implies. 
 
However, many studies \cite{antun2020instabilities,zhang2021instabilities, gilton2021deep} have demonstrated that DL-based MRI reconstruction models suffer from a lack of robustness. It has been shown that DL-based models are vulnerable to tiny input perturbations \cite{antun2020instabilities, zhang2021instabilities}, changes in measurement sampling rate \cite{antun2020instabilities}, and changes in the number of iterations of the model~\cite{gilton2021deep}. In these scenarios, the reconstructed images generated by DL-based models are of poor quality, which may lead to false diagnoses and adverse clinical consequences.

Although many defense methods~\cite{madry2017towards,zhang2019theoretically,cohen2019certified,salman2020denoised} were proposed to address the lack of robustness of DL models on the image classification task, the approaches of robustifying DL-based MRI reconstruction models are under-developed due to their regression-based learning objectives.
%\SL{[This should be the most relevant work to ours!]}
Randomized smoothing (RS) 
% is the technique that used to improve and enhance the convergence rates of algorithms for non smooth convex optimization problems. 
and its variants~\cite{cohen2019certified, salman2020denoised,zhang2022robustify} are quite popular adversarial defense methods in image classification. Different from conventional defense methods\cite{madry2017towards,zhang2019theoretically} which generate empirical robustness and are prone to fail against stronger attacks, RS  guarantees the model's robustness within a small sphere around the input image \cite{cohen2019certified}, which is vital for medical applications like MRI. A recent preliminary work attempted to apply RS to DL-based MRI reconstruction in an end-to-end (E2E) manner~\cite{wolfmaking}.  



%{However, Deep learning techniques for MRI reconstruction are imperfect due to a significant robustness issue that MRI is based on sampling the Fourier transform.} 
% This finding indicates that it is difficult to generalize DL~\cite{antun2020instabilities,MRI_adv_train}.
%{Recently, there have been plenty of methods,including Adversarial Training, TRADES and Randomized Smoothing ~\cite{madry17, zhang2019theoretically,cohen2019certified,salman2020denoised,ryou2020fast,niu2020limitations,salman2019provably,zhao2020maximum}} 
%{was used to address the adversarial robustness issue, but they mainly focus on the image classification tasks.} 
%{For the reconstruction area, Nearly all of existing work focuses on generating and defending such {\it test-time reconstruction evasion attacks}   \cite{antun2020instabilities,raj2020improving,choi2019evaluating,bungert2020robust}. %However, few work studies \textit{train-time poisoning attack} in the context of image reconstruction. \hui{we are not of this type either.}}
%{Besides the adversarial robustness training, the random smoothing technique also has been used for improving and analyzing convergence rates of algorithms for non smooth convex optimization problems. All of the recent works \cite{salman2020denoised,cohen2019certified} is also focus on the classification problem.} 

%2. Robustness of DL-based models in MRI reconstruction

%instable \textit{w.r.t.} adversarial perturbations generated by PGD, \textit{w.r.t.} undersampling rate

%3. Adversarial defense method

%on image classification

%Adversarial training,
%TRADES,
%Randomized Smoothing and its variants

%on MRI reconstruction



% Image reconstruction is an inverse problem that constitutes the recovery of an image from finite indirect measurements and is used in many applications, including critical ones like medical imaging (eg: MRI, CT). Deep learning (DL) provides novel methods which try to solve the problem more accurately and with much lesser samples compared to classical methods \cite{schlemper2017deep,zhu2018image,strack2018ai}. This could potentially make MRI scans, for instance, much efficient and effective \cite{zbontar2018fastmri}.

% Adversarial perturbations are tiny changes in the data input that cause a large degradation in the performance of pre-trained networks \cite{Goodfellow2015explaining,carlini2017towards}. While these DL-based methods have achieved state-of-the-art (SOTA) performance on image reconstruction \cite{Muckley_2021}, these networks have  been also found to be very unstable and prone to adversarial perturbations \cite{antun2020instabilities,raj2020improving},  leading to large degradation in reconstruction accuracy at testing time. 
% Nearly all of   existing work focuses on generating and defending such {\it test-time reconstruction evasion attacks}   \cite{antun2020instabilities,raj2020improving,choi2019evaluating,bungert2020robust}. However, few work studies \textit{train-time poisoning attack} in the context of image reconstruction.  

% \noindent \textbf{Approaches.} 
Given the advantages of RS and deep unrolling-based (hybrid domain) image reconstructors, we propose a novel approach dubbed \textsc{\underline{Sm}oothed \underline{U}nrollin\underline{g}} ({\us}) to mitigate the lack of robustness of DL-based MRI reconstruction models by systematically integrating RS into {\modl}~\cite{Aggarwal2019MoDL:Problems} architectures.
%To mitigate the problems mentioned above, we propose {\us}, which integrates RS \cite{cohen2019certified} with {\modl} \cite{aggarwal2018modl}. 
Instead of inefficient conventional \ref{eq: denoised smoothing mri} \cite{wolfmaking}, we apply  RS  in every unrolling step and on intermediate unrolled denoisers in {\modl}. We follow the `pre-training + fine-tuning' technique \cite{zoph2020rethinking,salman2020denoised}, adopting a mean square error (MSE) loss for pre-training and proposing an unrolling stability (UStab) loss along with the vanilla {\modl} reconstruction loss for fine-tuning.
Different from the existing art, our \textbf{contributions} are summarized as follows.\\
% \noindent $\bullet$ We show that {\modl} suffers from three types of instabilities upon adversarial perturbations, different undersampling rates, and different unrolling steps. 
\noindent $\bullet$ \ We propose {\us} that systematically integrates RS with {\modl} using an deep unrolled architecture.\\
\noindent $\bullet$ \  We study in detail where to apply RS in the unrolled architecture for better performance and propose a novel unrolling loss to improve training efficiency.\\
\noindent $\bullet$ We compared our methods with two related baselines: vanilla {\modl}~\cite{Aggarwal2019MoDL:Problems} and \ref{eq: denoised smoothing mri} \cite{wolfmaking}. Extensive experiments demonstrate the significant effectiveness of our proposed method on the major types of instabilities of {\modl}. 



% \section{Preliminaries and Problem Statement}
% \label{sec: preliminaries}

% In this section, we provide a brief background on MRI reconstruction and randomized smoothing. We will then motivate the lack of robustness   of MRI reconstruction and the problem of our interest. 


% \vspace*{1mm}
% \noindent \textbf{Setup of MRI reconstruction.} MRI reconstruction is an 
%  ill-posed inverse problem \cite{compress}, which aims to reconstruct the original signal from its sparse observations. 
%  In the context of multi-coil MRI reconstruction, let $\bx\in\mathbb{C}^{q}$ and $\mathbf y_c \in  \mathbb{C}^p$ (with $p < q$) denote the image to be recovered and its measurements associated with the $c$th  magnetic coil. We can cast the problem of MRI reconstruction  as 
% %  can be described roughly as a model-fitting method to reconstruct   image with a data-consistency term. This is typically accomplished by adding a regularization term that enforces the sparsity-inducing prior on $\by$.
% % The regularizer imposes additional constraints on the desired type of solution, resulting in an eventual solution that is more stable. 
% % For multi-coil MRI reconstruction of an image 
% % $\bx\in\mathbb{C}^{q}$, the optimization problem can be mathematically framed as
% {
% %\vspace*{-3mm}
% \begin{align}
%     \hat{\bx}=\underset{\bx}{\arg\min} ~\sum_{c=1}^{N_c}\|A_c \bx - \by_c \|^{2}_2 + \lambda \mathcal{R}(\bx),
%     \label{eq:inv_pro}
% \end{align}
% }%
% where $ N_c$ is the number of magnetic coils, $\mathcal{R}(\cdot)$ is a regularization function (\textit{e.g.}, $\ell_1$ norm to impose a sparse prior), and $\lambda > 0$ is a regularization parameter.  
% In \eqref{eq:inv_pro},  $A_c = M \mathcal{F} S_c$ denotes   the forward system   operation,
% where $M \in \{0,1\}^{p\times q}$ is a given sampling mask, 
% $\mathcal{F}\in \mathbb{C}^{q\times q}$ is 
% the $2D$ Fourier transformation, and $S_c \in 
% \mathbb{C}^{q\times q}$ is a complex-valued position-dependent sensitivity map. 

% % And the   regularizer $\mathcal{R}(\cdot)$  in \eqref{eq:inv_pro} is used to restrict the 
% % solutions to the space of desirable
% % images, and $\lambda$ is the regularization parameter controlling the trade-off between the residual norm and regularity.

% % Choices for the regularizer in MRI reconstruction can vary from the $\ell_1$ penalty on wavelet coefficients~\cite{wave} or the most widely used CS method is total variation~\cite{totalv} denoising which enforces piecewise constant images by uniformly penalizing image gradients.

% \subsection{MoDL and  the lack of robustness}
% \label{sec: modl}


% {\modl} relies on a formulation similar to \eqref{eq:inv_pro}, where the hand-crafted image regularization penalties in \eqref{eq:inv_pro} are replaced with learned priors. {\modl} has 
% shown promise for MRI reconstruction
% by combining a denoising network with a data-consistency ($\mathcal{DC}$) module in each stage
% \cite{aggarwal2018modl}.
% In the {\modl} model, Problem~\eqref{eq:inv_pro} is following:
    
% {
% \vspace*{-3mm}
% \begin{align}
%     \hat{\bx}=\underset{\bx}{\arg\min} ~  \sum_{c=1}^{N_c} \|A_c \bx - \by_c\|_2^2 + \lambda \| \bx - \cD (\bm{\theta};\bx) \|_2^2,
% \label{eq:altmin}    
% \end{align}
% }%
% %where $\nu \geq 0$ weights the data-fidelity term and $\mu \geq 0$ weights the proximity of $x$ to $z$ above. 
% The problem can be solved by alternating between updating $\bx$ and $\cD (\bm{\theta};\bx)$.
% The {\modl} scheme performs the update of $\bx$ (\textit{i.e.}, $\mathcal{DC}$) by Conjugate Gradient descent, and the update of $\cD (\bm{\theta};\bx)$ involving with a CNN denoiser ($\cD (\cdot)$) that is applied to the updated $\bx$.


% \SL{[This is an important part to motivate our work. You need to define adversarial perturbations, undersampling rates, and unrolling steps, and associate them with math notations.]}
% {\modl} is unstable against adversarial perturbations, different undersampling rates
% % (When the sampling factor equals four,it will only collect $8\%$ of central k-space lines)
% , and different unrolling steps
% % ( $k$ for the number of alternative step between CNN denoiser update and Data consistence update) 
% with training setup. 
% The adversarial perturbation refers to a small, almost invisible to naked eyes, perturbation $\delta$ added to the input image $\bx_0$, which is constructed to deteriorate the performance of the model, as
% \begin{equation}
%     % \delta=\underset{\delta\sim\mathcal{S}}{\arg\max}~\cL(f(\bx+\delta), f(\bx))
%     \underset{||\delta||_\infty\le r}{\arg\max}~\cL(f(\bx_0+\delta), f(\bx_0))
% \end{equation}
% % where $\mathcal{S}$ refers to a certain sphere around the data point $\bx$ with some distance measure.
% where $f$ is the reconstruction model, and $r$ is the distance measure of the perturbation. Different undersampling rates refer to 
% the new undersampling mask $M$ with different density is used to obtain the k-space measurements $\mathbf{y}_c$ from the ground truth in \eqref{eq:altmin} in the evaluation stage. And different unrolling steps mean a different number of iterations is used to solve \eqref{eq:altmin} in the evaluation stage.
% In Fig. \ref{fig: weakness}, the reconstructed images blurs when adversarial perturbations are added to the undersampled input, a denser undersampling mask is used, and a larger number of unrolling steps is adopted in {\modl}.
 
% \begin{figure}[htb]
% %\vspace*{-3mm}
% %    \centering
% %    \begin{subfigure}{.10\textwidth}
% %        \centering
% %        \includegraphics[width=\textwidth]{Figures/output_vanilla_vis.pdf}
% %        \caption{\footnotesize{clean data
% %        }}
% %    \end{subfigure}
% %    \begin{subfigure}{.10\textwidth}
% %        \centering
% %        \includegraphics[width=\textwidth]{Figures/visualization/vanilla_MoDL_eps0.5_255.pdf}
% %        \caption{\footnotesize{adversarial perturbations
% %        }}
% %    \end{subfigure}
% %    \begin{subfigure}{.10\textwidth}
% %        \centering
% %        \includegraphics[width=\textwidth]{Figures/undersampling_vis.pdf}
% %        \caption{\footnotesize{undersampling rate}}
% %    \end{subfigure}
% %    \begin{subfigure}{.10\textwidth}
% %        \centering
% %        \includegraphics[width=\textwidth]{Figures/unrolling_steps_vis.pdf}
% %        \caption{\footnotesize{number of unrolling steps
% %        }}
% %    \end{subfigure}
    
% %    \vspace*{-1em}
% %    \caption{\footnotesize{{\modl}'s %instabiliites \textit{w.r.t.} adversarial perturbations, undersampling rate and number of unrolling steps}}
% %    \label{fig: weakness}
% %    \vspace*{-2mm}
% \begin{tabular}[b]{cccc}
%         \includegraphics[width=.2\linewidth, trim=70 10 70 10]{Figures/visualization/vanilla_MoDL_recon.pdf}
%         &
%         \includegraphics[width=.2\linewidth, trim=70 10 70 10]{Figures/visualization/vanilla_MoDL_eps0.5_255.pdf}
%         &
%         \includegraphics[width=.2\linewidth, trim=70 10 70 10]{Figures/visualization/undersampling_2.0_vis.pdf}
%         &
%         \includegraphics[width=.2\linewidth, trim=70 10 70 10]{Figures/visualization/unrolling_steps_16_clean_vis.pdf}
%         \\[-0pt]
%         \scriptsize{(a)} 
%         & 
%         \scriptsize{(b)} 
%         &  
%         \scriptsize{(c)} 
%         &
%         \scriptsize{(d)}

% \end{tabular}
% \vspace*{-5mm}
% \caption{\footnotesize{{\modl}'s instabilities against adversarial perturbations, different undersampling rates, and different unrolling steps. (a) the reconstruction image from the clean image, (b) the reconstruction image from the adversarial perturbed image with the perturbation $||\delta||_{\infty}\le0.002$, (c) the reconstruction image when evaluated using a mask with the undersampling rate being 2 \hui{50\%?} (4 \hui{25\%?} in training), (d) the reconstruction image when evaluated using 16 unrolling steps (8 in training).}}
% \label{fig: weakness}
% \vspace*{-4.5mm}
% \end{figure}

% \subsection{Randomized smoothing} %and denoised smoothing}


% Randomized smoothing \cite{cohen2019certified} uses a model's 
% \SL{[why just classifier? makes it more general.]}
% robustness to random noise to create a new model robust to adversarial perturbations. In classfication case, given a base classifier mapping from image domain $\mathbb R^d$ to classes $\mathcal Y$, the smoothed classifier $g$ 
% %Given the CNN network by augmenting the data with noise ${x_{aliased}^{i},y^{i}}$
% %\begin{equation}
% %    f_{rs} = \frac{1}{K} \sum_{k=1}^{K} 
% %    f_{\theta_{\text{noise}}}(x+\epsilon^{k}),\quad \epsilon\sim\mathcal N(0, \sigma^2I)
% %\end{equation}
%  returns the label which the base classifier $f$ is most likely to return when image $\bx$ is perturbed by Gaussian noise, as
% is
% {
% \begin{align}
%     g(\bx)=\underset{c\in\mathcal Y}{\arg\max}~ \mathbb P(f(\bx+\epsilon)=c),\quad \epsilon\sim\mathcal N(0, \sigma^2I)
%     \label{eq: randomized smoothing}
% \end{align}
% }%
% where the base classifier $f$ is trained with Gaussian noise augmented data, and Monte Carlo sampling is used to estimate the probability.

% % Proved by Neyman-Pearson lemma, the smoothed classifier $g$ is certified to be robust to adversarial perturbations within a small $l_2$ radius compared to $\sigma$. This $l_2$ radius is related with the robustness of base classifier $f$ towards Gaussian noise. Consequently, in randomized smoothing, the base classifier is trained with Gaussian noise augmented data.

% To avoid retraining the off-the-shelf model with Gaussian noise augmented data, paper \cite{salman2020denoised} proposed Denoised Smoothing where a denoiser $\cD (\cdot)$ is appended to the classifier $f$ to form a new base classifier $f \circ \cD$ robust to Gaussian noise. Then the smoothed classifier takes the majority vote over $f \circ \cD$, as
% {
% \begin{align}
%     g(\bx)=\underset{c\in\mathcal Y}{\arg\max}~ \mathbb P(f(\cD (\bm{\theta};\bx + \epsilon))=c),\quad \epsilon\sim\mathcal N(0, \sigma^2I)
%     \label{eq: denoised smoothing}
% \end{align}
% }

% Noting that \eqref{eq: randomized smoothing} and \eqref{eq: denoised smoothing} cannot be directly used in MRI reconstruction tasks due to their regression-based learning objective, we convert \eqref{eq: denoised smoothing} into 
% {
% \begin{align}
%     g(\bx)=\mathbb{E}_{\epsilon\sim\mathcal{N}(0, \sigma^2I)} f(\cD (\bm{\theta};\bx + \epsilon))
%     \label{eq: denoised smoothing mri}
% \end{align}
% }
% to get the smooth prediction.
% %In denoised smoothing, the classifier $f$ is trained on clean data, and no data augmentation is needed. However, the denoiser $\cD (\cdot)$ is custom-trained to remove the Gaussian noise. They proposed two training objectives, mean square error (MSE) and stability (Stab). The latter takes the downstream classification task into account compared to the traditional MSE one. These two losses are defined as
% %{\begin{align}
% %    \cL_\mathrm{MSE}&=\mathbb{E}_{\epsilon\sim\mathcal N(0, \sigma^2I)}||\cD(\bm{\theta};\bx+\epsilon)-\bx||_2^2
% %    \label{eq: MSE loss}
% %    \\
% %    \cL_\mathrm{Stab}&=\mathbb{E}_{\epsilon\sim\mathcal N(0, \sigma^2I)}||f(\cD(\bm{\theta};\bx+\epsilon))-f(\bx)||_2^2
%     \label{eq: Stab loss}
% %\end{align}}%

% %MSE + Stab training scheme is used in denoised smoothing for efficiency and accuracy.

% \SL{[why do we need to mention denoised smoothing? I did not get the point.]}

%%
\section{Preliminaries and Problem Statement}
\label{sec: preliminaries}

In this section, we provide a brief background on MRI reconstruction and randomized smoothing. We will then motivate the lack of robustness   of MRI reconstruction and the problem of our interest. 


\vspace*{1mm}
\noindent \textbf{Setup of MRI reconstruction.} MRI reconstruction is an 
 ill-posed inverse problem \cite{compress}, which aims to reconstruct the original signal from its sparse observations. 
 In the context of multi-coil MRI reconstruction, let $\bx\in\mathbb{C}^{q}$ and $\mathbf y_c \in  \mathbb{C}^p$ (with $p < q$) denote the image to be recovered and its measurements associated with the $c$th  magnetic coil. We can cast the problem of MRI reconstruction  as 
%  can be described roughly as a model-fitting method to reconstruct   image with a data-consistency term. This is typically accomplished by adding a regularization term that enforces the sparsity-inducing prior on $\by$.
% The regularizer imposes additional constraints on the desired type of solution, resulting in an eventual solution that is more stable. 
% For multi-coil MRI reconstruction of an image 
% $\bx\in\mathbb{C}^{q}$, the optimization problem can be mathematically framed as
{
%\vspace*{-3mm}
\begin{align}
    \hat{\bx}=\underset{\bx}{\arg\min} ~\sum_{c=1}^{N_c}\|A_c \bx - \by_c \|^{2}_2 + \lambda \mathcal{R}(\bx),
    \label{eq:inv_pro}
\end{align}
}%
where $ N_c$ is the number of magnetic coils, $\mathcal{R}(\cdot)$ is a regularization function (\textit{e.g.}, $\ell_1$ norm to impose a sparse prior), and $\lambda > 0$ is a regularization parameter.  
In \eqref{eq:inv_pro},  $A_c = M \mathcal{F} S_c$ denotes   the forward system   operation,
where $M \in \{0,1\}^{p\times q}$ is a given sampling mask, 
$\mathcal{F}\in \mathbb{C}^{q\times q}$ is 
the $2D$ Fourier transformation, and $S_c \in 
\mathbb{C}^{q\times q}$ is a complex-valued position-dependent sensitivity map. 

% And the   regularizer $\mathcal{R}(\cdot)$  in \eqref{eq:inv_pro} is used to restrict the 
% solutions to the space of desirable
% images, and $\lambda$ is the regularization parameter controlling the trade-off between the residual norm and regularity.

% Choices for the regularizer in MRI reconstruction can vary from the $\ell_1$ penalty on wavelet coefficients~\cite{wave} or the most widely used CS method is total variation~\cite{totalv} denoising which enforces piecewise constant images by uniformly penalizing image gradients.

\subsection{MoDL and  the lack of robustness}
\label{sec: modl}


{\modl} relies on a formulation similar to \eqref{eq:inv_pro}, where the hand-crafted image regularization penalties in \eqref{eq:inv_pro} are replaced with learned priors. {\modl} has 
shown promise for MRI reconstruction
by combining a denoising network with a data-consistency ($\mathcal{DC}$) module in each stage
\cite{aggarwal2018modl}.
In the {\modl} model, Problem~\eqref{eq:inv_pro} is following:
    
{
\vspace*{-3mm}
\begin{align}
    \hat{\bx}=\underset{\bx}{\arg\min} ~  \sum_{c=1}^{N_c} \|A_c \bx - \by_c\|_2^2 + \lambda \| \bx - \cD (\bm{\theta};\bx) \|_2^2,
\label{eq:altmin}    
\end{align}
}%
%where $\nu \geq 0$ weights the data-fidelity term and $\mu \geq 0$ weights the proximity of $x$ to $z$ above. 
The problem can be solved by alternating between updating $\bx$ and $\cD (\bm{\theta};\bx)$.
The {\modl} scheme performs the update of $\bx$ (\textit{i.e.}, $\mathcal{DC}$) by Conjugate Gradient descent, and the update of $\cD (\bm{\theta};\bx)$ involving with a CNN denoiser ($\cD (\cdot)$) that is applied to the updated $\bx$.


\SL{[This is an important part to motivate our work. You need to define adversarial perturbations, undersampling rates, and unrolling steps, and associate them with math notations.]}
{\modl} is unstable against adversarial perturbations, different undersampling rates
% (When the sampling factor equals four,it will only collect $8\%$ of central k-space lines)
, and different unrolling steps
% ( $k$ for the number of alternative step between CNN denoiser update and Data consistence update) 
with training setup. 
The adversarial perturbation refers to a small, almost invisible to naked eyes, perturbation $\delta$ added to the input image $\bx_0$, which is constructed to deteriorate the performance of the model, as
\begin{equation}
    % \delta=\underset{\delta\sim\mathcal{S}}{\arg\max}~\cL(f(\bx+\delta), f(\bx))
    \underset{||\delta||_\infty\le r}{\arg\max}~\cL(f(\bx_0+\delta), f(\bx_0))
\end{equation}
% where $\mathcal{S}$ refers to a certain sphere around the data point $\bx$ with some distance measure.
where $f$ is the reconstruction model, and $r$ is the distance measure of the perturbation. Different undersampling rates refer to 
the new undersampling mask $M$ with different density is used to obtain the k-space measurements $\mathbf{y}_c$ from the ground truth in \eqref{eq:altmin} in the evaluation stage. And different unrolling steps mean a different number of iterations is used to solve \eqref{eq:altmin} in the evaluation stage.
In Fig. \ref{fig: weakness}, the reconstructed images blurs when adversarial perturbations are added to the undersampled input, a denser undersampling mask is used, and a larger number of unrolling steps is adopted in {\modl}.
 
\begin{figure}[htb]
%\vspace*{-3mm}
%    \centering
%    \begin{subfigure}{.10\textwidth}
%        \centering
%        \includegraphics[width=\textwidth]{Figures/output_vanilla_vis.pdf}
%        \caption{\footnotesize{clean data
%        }}
%    \end{subfigure}
%    \begin{subfigure}{.10\textwidth}
%        \centering
%        \includegraphics[width=\textwidth]{Figures/visualization/vanilla_MoDL_eps0.5_255.pdf}
%        \caption{\footnotesize{adversarial perturbations
%        }}
%    \end{subfigure}
%    \begin{subfigure}{.10\textwidth}
%        \centering
%        \includegraphics[width=\textwidth]{Figures/undersampling_vis.pdf}
%        \caption{\footnotesize{undersampling rate}}
%    \end{subfigure}
%    \begin{subfigure}{.10\textwidth}
%        \centering
%        \includegraphics[width=\textwidth]{Figures/unrolling_steps_vis.pdf}
%        \caption{\footnotesize{number of unrolling steps
%        }}
%    \end{subfigure}
    
%    \vspace*{-1em}
%    \caption{\footnotesize{{\modl}'s %instabiliites \textit{w.r.t.} adversarial perturbations, undersampling rate and number of unrolling steps}}
%    \label{fig: weakness}
%    \vspace*{-2mm}
\begin{tabular}[b]{cccc}
        \includegraphics[width=.2\linewidth, trim=70 10 70 10]{Figures/visualization/vanilla_MoDL_recon.pdf}
        &
        \includegraphics[width=.2\linewidth, trim=70 10 70 10]{Figures/visualization/vanilla_MoDL_eps0.5_255.pdf}
        &
        \includegraphics[width=.2\linewidth, trim=70 10 70 10]{Figures/visualization/undersampling_2.0_vis.pdf}
        &
        \includegraphics[width=.2\linewidth, trim=70 10 70 10]{Figures/visualization/unrolling_steps_16_clean_vis.pdf}
        \\[-0pt]
        \scriptsize{(a)} 
        & 
        \scriptsize{(b)} 
        &  
        \scriptsize{(c)} 
        &
        \scriptsize{(d)}

\end{tabular}
\vspace*{-5mm}
\caption{\footnotesize{{\modl}'s instabilities against adversarial perturbations, different undersampling rates, and different unrolling steps. (a) the reconstruction image from the clean image, (b) the reconstruction image from the adversarial perturbed image with the perturbation $||\delta||_{\infty}\le0.002$, (c) the reconstruction image when evaluated using a mask with the undersampling rate being 2 \hui{50\%?} (4 \hui{25\%?} in training), (d) the reconstruction image when evaluated using 16 unrolling steps (8 in training).}}
\label{fig: weakness}
\vspace*{-4.5mm}
\end{figure}

\subsection{Randomized smoothing} %and denoised smoothing}


Randomized smoothing \cite{cohen2019certified} uses a model's 
\SL{[why just classifier? makes it more general.]}
robustness to random noise to create a new model robust to adversarial perturbations. In classfication case, given a base classifier mapping from image domain $\mathbb R^d$ to classes $\mathcal Y$, the smoothed classifier $g$ 
%Given the CNN network by augmenting the data with noise ${x_{aliased}^{i},y^{i}}$
%\begin{equation}
%    f_{rs} = \frac{1}{K} \sum_{k=1}^{K} 
%    f_{\theta_{\text{noise}}}(x+\epsilon^{k}),\quad \epsilon\sim\mathcal N(0, \sigma^2I)
%\end{equation}
 returns the label which the base classifier $f$ is most likely to return when image $\bx$ is perturbed by Gaussian noise, as
is
{
\begin{align}
    g(\bx)=\underset{c\in\mathcal Y}{\arg\max}~ \mathbb P(f(\bx+\epsilon)=c),\quad \epsilon\sim\mathcal N(0, \sigma^2I)
    \label{eq: randomized smoothing}
\end{align}
}%
where the base classifier $f$ is trained with Gaussian noise augmented data, and Monte Carlo sampling is used to estimate the probability.

% Proved by Neyman-Pearson lemma, the smoothed classifier $g$ is certified to be robust to adversarial perturbations within a small $l_2$ radius compared to $\sigma$. This $l_2$ radius is related with the robustness of base classifier $f$ towards Gaussian noise. Consequently, in randomized smoothing, the base classifier is trained with Gaussian noise augmented data.

To avoid retraining the off-the-shelf model with Gaussian noise augmented data, paper \cite{salman2020denoised} proposed Denoised Smoothing where a denoiser $\cD (\cdot)$ is appended to the classifier $f$ to form a new base classifier $f \circ \cD$ robust to Gaussian noise. Then the smoothed classifier takes the majority vote over $f \circ \cD$, as
{
\begin{align}
    g(\bx)=\underset{c\in\mathcal Y}{\arg\max}~ \mathbb P(f(\cD (\bm{\theta};\bx + \epsilon))=c),\quad \epsilon\sim\mathcal N(0, \sigma^2I)
    \label{eq: denoised smoothing}
\end{align}
}

Noting that \eqref{eq: randomized smoothing} and \eqref{eq: denoised smoothing} cannot be directly used in MRI reconstruction tasks due to their regression-based learning objective, we convert \eqref{eq: denoised smoothing} into 
{
\begin{align}
    g(\bx)=\mathbb{E}_{\epsilon\sim\mathcal{N}(0, \sigma^2I)} f(\cD (\bm{\theta};\bx + \epsilon))
    \label{eq: denoised smoothing mri}
\end{align}
}
to get the smooth prediction.
%In denoised smoothing, the classifier $f$ is trained on clean data, and no data augmentation is needed. However, the denoiser $\cD (\cdot)$ is custom-trained to remove the Gaussian noise. They proposed two training objectives, mean square error (MSE) and stability (Stab). The latter takes the downstream classification task into account compared to the traditional MSE one. These two losses are defined as
%{\begin{align}
%    \cL_\mathrm{MSE}&=\mathbb{E}_{\epsilon\sim\mathcal N(0, \sigma^2I)}||\cD(\bm{\theta};\bx+\epsilon)-\bx||_2^2
%    \label{eq: MSE loss}
%    \\
%    \cL_\mathrm{Stab}&=\mathbb{E}_{\epsilon\sim\mathcal N(0, \sigma^2I)}||f(\cD(\bm{\theta};\bx+\epsilon))-f(\bx)||_2^2
    \label{eq: Stab loss}
%\end{align}}%

%MSE + Stab training scheme is used in denoised smoothing for efficiency and accuracy.

\SL{[why do we need to mention denoised smoothing? I did not get the point.]}
%\section{Preliminaries and Problem Statement}
\label{sec: preliminaries}

In this section, we provide a brief background on MRI reconstruction and randomized smoothing. We will then motivate the lack of robustness   of MRI reconstruction and the problem of our interest. 


\vspace*{1mm}
\noindent \textbf{Setup of MRI reconstruction.} MRI reconstruction is an 
 ill-posed inverse problem \cite{compress}, which aims to reconstruct the original signal from its sparse observations. 
 In the context of multi-coil MRI reconstruction, let $\bx\in\mathbb{C}^{q}$ and $\mathbf y_c \in  \mathbb{C}^p$ (with $p < q$) denote the image to be recovered and its measurements associated with the $c$th  magnetic coil. We can cast the problem of MRI reconstruction  as 
%  can be described roughly as a model-fitting method to reconstruct   image with a data-consistency term. This is typically accomplished by adding a regularization term that enforces the sparsity-inducing prior on $\by$.
% The regularizer imposes additional constraints on the desired type of solution, resulting in an eventual solution that is more stable. 
% For multi-coil MRI reconstruction of an image 
% $\bx\in\mathbb{C}^{q}$, the optimization problem can be mathematically framed as
{
%\vspace*{-3mm}
\begin{align}
    \hat{\bx}=\underset{\bx}{\arg\min} ~\sum_{c=1}^{N_c}\|A_c \bx - \by_c \|^{2}_2 + \lambda \mathcal{R}(\bx),
    \label{eq:inv_pro}
\end{align}
}%
where $ N_c$ is the number of magnetic coils, $\mathcal{R}(\cdot)$ is a regularization function (\textit{e.g.}, $\ell_1$ norm to impose a sparse prior), and $\lambda > 0$ is a regularization parameter.  
In \eqref{eq:inv_pro},  $A_c = M \mathcal{F} S_c$ denotes   the forward system   operation,
where $M \in \{0,1\}^{p\times q}$ is a given sampling mask, 
$\mathcal{F}\in \mathbb{C}^{q\times q}$ is 
the $2D$ Fourier transformation, and $S_c \in 
\mathbb{C}^{q\times q}$ is a complex-valued position-dependent sensitivity map. 

% And the   regularizer $\mathcal{R}(\cdot)$  in \eqref{eq:inv_pro} is used to restrict the 
% solutions to the space of desirable
% images, and $\lambda$ is the regularization parameter controlling the trade-off between the residual norm and regularity.

% Choices for the regularizer in MRI reconstruction can vary from the $\ell_1$ penalty on wavelet coefficients~\cite{wave} or the most widely used CS method is total variation~\cite{totalv} denoising which enforces piecewise constant images by uniformly penalizing image gradients.

\subsection{MoDL and  the lack of robustness}
\label{sec: modl}


{\modl} relies on a formulation similar to \eqref{eq:inv_pro}, where the hand-crafted image regularization penalties in \eqref{eq:inv_pro} are replaced with learned priors. {\modl} has 
shown promise for MRI reconstruction
by combining a denoising network with a data-consistency ($\mathcal{DC}$) module in each stage
\cite{aggarwal2018modl}.
In the {\modl} model, Problem~\eqref{eq:inv_pro} is following:
    
{
\vspace*{-3mm}
\begin{align}
    \hat{\bx}=\underset{\bx}{\arg\min} ~  \sum_{c=1}^{N_c} \|A_c \bx - \by_c\|_2^2 + \lambda \| \bx - \cD (\bm{\theta};\bx) \|_2^2,
\label{eq:altmin}    
\end{align}
}%
%where $\nu \geq 0$ weights the data-fidelity term and $\mu \geq 0$ weights the proximity of $x$ to $z$ above. 
The problem can be solved by alternating between updating $\bx$ and $\cD (\bm{\theta};\bx)$.
The {\modl} scheme performs the update of $\bx$ (\textit{i.e.}, $\mathcal{DC}$) by Conjugate Gradient descent, and the update of $\cD (\bm{\theta};\bx)$ involving with a CNN denoiser ($\cD (\cdot)$) that is applied to the updated $\bx$.


\SL{[This is an important part to motivate our work. You need to define adversarial perturbations, undersampling rates, and unrolling steps, and associate them with math notations.]}
{\modl} is unstable against adversarial perturbations, different undersampling rates
% (When the sampling factor equals four,it will only collect $8\%$ of central k-space lines)
, and different unrolling steps
% ( $k$ for the number of alternative step between CNN denoiser update and Data consistence update) 
with training setup. 
The adversarial perturbation refers to a small, almost invisible to naked eyes, perturbation $\delta$ added to the input image $\bx_0$, which is constructed to deteriorate the performance of the model, as
\begin{equation}
    % \delta=\underset{\delta\sim\mathcal{S}}{\arg\max}~\cL(f(\bx+\delta), f(\bx))
    \underset{||\delta||_\infty\le r}{\arg\max}~\cL(f(\bx_0+\delta), f(\bx_0))
\end{equation}
% where $\mathcal{S}$ refers to a certain sphere around the data point $\bx$ with some distance measure.
where $f$ is the reconstruction model, and $r$ is the distance measure of the perturbation. Different undersampling rates refer to 
the new undersampling mask $M$ with different density is used to obtain the k-space measurements $\mathbf{y}_c$ from the ground truth in \eqref{eq:altmin} in the evaluation stage. And different unrolling steps mean a different number of iterations is used to solve \eqref{eq:altmin} in the evaluation stage.
In Fig. \ref{fig: weakness}, the reconstructed images blurs when adversarial perturbations are added to the undersampled input, a denser undersampling mask is used, and a larger number of unrolling steps is adopted in {\modl}.
 
\begin{figure}[htb]
%\vspace*{-3mm}
%    \centering
%    \begin{subfigure}{.10\textwidth}
%        \centering
%        \includegraphics[width=\textwidth]{Figures/output_vanilla_vis.pdf}
%        \caption{\footnotesize{clean data
%        }}
%    \end{subfigure}
%    \begin{subfigure}{.10\textwidth}
%        \centering
%        \includegraphics[width=\textwidth]{Figures/visualization/vanilla_MoDL_eps0.5_255.pdf}
%        \caption{\footnotesize{adversarial perturbations
%        }}
%    \end{subfigure}
%    \begin{subfigure}{.10\textwidth}
%        \centering
%        \includegraphics[width=\textwidth]{Figures/undersampling_vis.pdf}
%        \caption{\footnotesize{undersampling rate}}
%    \end{subfigure}
%    \begin{subfigure}{.10\textwidth}
%        \centering
%        \includegraphics[width=\textwidth]{Figures/unrolling_steps_vis.pdf}
%        \caption{\footnotesize{number of unrolling steps
%        }}
%    \end{subfigure}
    
%    \vspace*{-1em}
%    \caption{\footnotesize{{\modl}'s %instabiliites \textit{w.r.t.} adversarial perturbations, undersampling rate and number of unrolling steps}}
%    \label{fig: weakness}
%    \vspace*{-2mm}
\begin{tabular}[b]{cccc}
        \includegraphics[width=.2\linewidth, trim=70 10 70 10]{Figures/visualization/vanilla_MoDL_recon.pdf}
        &
        \includegraphics[width=.2\linewidth, trim=70 10 70 10]{Figures/visualization/vanilla_MoDL_eps0.5_255.pdf}
        &
        \includegraphics[width=.2\linewidth, trim=70 10 70 10]{Figures/visualization/undersampling_2.0_vis.pdf}
        &
        \includegraphics[width=.2\linewidth, trim=70 10 70 10]{Figures/visualization/unrolling_steps_16_clean_vis.pdf}
        \\[-0pt]
        \scriptsize{(a)} 
        & 
        \scriptsize{(b)} 
        &  
        \scriptsize{(c)} 
        &
        \scriptsize{(d)}

\end{tabular}
\vspace*{-5mm}
\caption{\footnotesize{{\modl}'s instabilities against adversarial perturbations, different undersampling rates, and different unrolling steps. (a) the reconstruction image from the clean image, (b) the reconstruction image from the adversarial perturbed image with the perturbation $||\delta||_{\infty}\le0.002$, (c) the reconstruction image when evaluated using a mask with the undersampling rate being 2 \hui{50\%?} (4 \hui{25\%?} in training), (d) the reconstruction image when evaluated using 16 unrolling steps (8 in training).}}
\label{fig: weakness}
\vspace*{-4.5mm}
\end{figure}

\subsection{Randomized smoothing} %and denoised smoothing}


Randomized smoothing \cite{cohen2019certified} uses a model's 
\SL{[why just classifier? makes it more general.]}
robustness to random noise to create a new model robust to adversarial perturbations. In classfication case, given a base classifier mapping from image domain $\mathbb R^d$ to classes $\mathcal Y$, the smoothed classifier $g$ 
%Given the CNN network by augmenting the data with noise ${x_{aliased}^{i},y^{i}}$
%\begin{equation}
%    f_{rs} = \frac{1}{K} \sum_{k=1}^{K} 
%    f_{\theta_{\text{noise}}}(x+\epsilon^{k}),\quad \epsilon\sim\mathcal N(0, \sigma^2I)
%\end{equation}
 returns the label which the base classifier $f$ is most likely to return when image $\bx$ is perturbed by Gaussian noise, as
is
{
\begin{align}
    g(\bx)=\underset{c\in\mathcal Y}{\arg\max}~ \mathbb P(f(\bx+\epsilon)=c),\quad \epsilon\sim\mathcal N(0, \sigma^2I)
    \label{eq: randomized smoothing}
\end{align}
}%
where the base classifier $f$ is trained with Gaussian noise augmented data, and Monte Carlo sampling is used to estimate the probability.

% Proved by Neyman-Pearson lemma, the smoothed classifier $g$ is certified to be robust to adversarial perturbations within a small $l_2$ radius compared to $\sigma$. This $l_2$ radius is related with the robustness of base classifier $f$ towards Gaussian noise. Consequently, in randomized smoothing, the base classifier is trained with Gaussian noise augmented data.

To avoid retraining the off-the-shelf model with Gaussian noise augmented data, paper \cite{salman2020denoised} proposed Denoised Smoothing where a denoiser $\cD (\cdot)$ is appended to the classifier $f$ to form a new base classifier $f \circ \cD$ robust to Gaussian noise. Then the smoothed classifier takes the majority vote over $f \circ \cD$, as
{
\begin{align}
    g(\bx)=\underset{c\in\mathcal Y}{\arg\max}~ \mathbb P(f(\cD (\bm{\theta};\bx + \epsilon))=c),\quad \epsilon\sim\mathcal N(0, \sigma^2I)
    \label{eq: denoised smoothing}
\end{align}
}

Noting that \eqref{eq: randomized smoothing} and \eqref{eq: denoised smoothing} cannot be directly used in MRI reconstruction tasks due to their regression-based learning objective, we convert \eqref{eq: denoised smoothing} into 
{
\begin{align}
    g(\bx)=\mathbb{E}_{\epsilon\sim\mathcal{N}(0, \sigma^2I)} f(\cD (\bm{\theta};\bx + \epsilon))
    \label{eq: denoised smoothing mri}
\end{align}
}
to get the smooth prediction.
%In denoised smoothing, the classifier $f$ is trained on clean data, and no data augmentation is needed. However, the denoiser $\cD (\cdot)$ is custom-trained to remove the Gaussian noise. They proposed two training objectives, mean square error (MSE) and stability (Stab). The latter takes the downstream classification task into account compared to the traditional MSE one. These two losses are defined as
%{\begin{align}
%    \cL_\mathrm{MSE}&=\mathbb{E}_{\epsilon\sim\mathcal N(0, \sigma^2I)}||\cD(\bm{\theta};\bx+\epsilon)-\bx||_2^2
%    \label{eq: MSE loss}
%    \\
%    \cL_\mathrm{Stab}&=\mathbb{E}_{\epsilon\sim\mathcal N(0, \sigma^2I)}||f(\cD(\bm{\theta};\bx+\epsilon))-f(\bx)||_2^2
    \label{eq: Stab loss}
%\end{align}}%

%MSE + Stab training scheme is used in denoised smoothing for efficiency and accuracy.

\SL{[why do we need to mention denoised smoothing? I did not get the point.]}
%%
\section{{\us}: SMoothed UnrollinG}
\label{sec: approach}

In this section, we describe our proposed {\us} method, in which we integrate {\modl} \cite{aggarwal2018modl} with randomized smoothing \cite{cohen2019certified}, to solve the instability problems mentioned in Sec. \ref{sec: preliminaries}. We first explicitly describe the architecture of our {\us} method in Sec. \ref{sec: unrolling}. Later, we elaborate on our training scheme in Sec. \ref{sec: how to smoothing}.

\subsection{Unrolling: key in smoothing on MoDL}
\label{sec: unrolling}

Following the denoised smoothing \cite{salman2020denoised}, we use a custom-trained denoiser to make the base model, \textit{i.e.} {\modl}, robust to Gaussian noise before applying randomized smoothing to generate robustness against adversarial perturbations. But instead of introducing a new denoiser, we exploit the denoiser in {\modl} for model simplicity and overall time efficiency. Compared to appending a new denoiser, our approach of reusing the existing denoiser in {\modl} is capable of removing Gaussian noises in every unrolling step, which enhances the Gaussian noise robustness of the base model and makes it more potential for randomized smoothing.

We then apply randomized smoothing \cite{cohen2019certified} to {\modl}. 
We present two smoothing architectures here: 1) {\us} on the whole ({\us} ($\cD + \mathcal{DC}$)), and 2) {\us}. {\us} is proposed to integrate randomized smoothing with {\modl} naturally. The differences between these architectures and \ref{eq: denoised smoothing mri} mainly focus on where to add the Gaussian noises and where to take the majority vote, \textit{i.e.} estimate the expectation in MRI reconstruction. Their comparison can also be seen in Fig. \ref{fig: model-arch}.
In \ref{eq: denoised smoothing mri}, the Gaussian noises $\epsilon$ are added on the input of the entire {\modl}, $\bx_0$, and the expectation \textit{w.r.t.} $\epsilon$ is estimated on the output of the entire {\modl}, $\bx_K$. In {\us}, however, the Gaussian noises $\epsilon$ are added on the input of \textit{every} unrolling step in {\modl}, $\bx_k$, and the expectations are estimated in \textit{every} unrolling step on the output of the whole unrolling step, \textit{i.e.} the output of the $\mathcal{DC}$, for {\us} on the whole, and the output of the denoiser for {\us}. 

\begin{figure}[htb]
% \vspace*{-3mm}
    \centering
    \begin{subfigure}{.48\textwidth}
        \centering
        \includegraphics[width=\textwidth]{Figures/E2E_Smoothing.pdf}
        \caption{\footnotesize{\ref{eq: denoised smoothing mri}
        }}
    \end{subfigure}
    
    \begin{subfigure}{.48\textwidth}
        \centering
        \includegraphics[width=\textwidth]{Figures/SMUG_whole.pdf}
        \caption{\footnotesize{{\us} on the whole
        }}
    \end{subfigure}

    \begin{subfigure}{.48\textwidth}
        \centering
        \includegraphics[width=\textwidth]{Figures/SMUG.pdf}
        \caption{\footnotesize{\us}}
    \end{subfigure}
    
    \vspace*{-1em}
    \caption{\footnotesize{The architecture of (a) \ref{eq: denoised smoothing mri}, (b) {\us} on the whole, and (c) {\us}.}}
    \label{fig: model-arch}
    % \vspace*{-4mm}
\end{figure}

% To be congruent with the presence of the denoiser in every unrolling step, we adopt an unrolling version of randomized smoothing, which we call {\us}. {\us} exploits the full potential of the unrolling presence of the denoiser in Gaussian noise robustness, thus obtaining high robustness for {\modl}. 

More concretely, our method of {\us} can be formularized as the following iterative algorithm, 
\begin{subequations}
\label{eq: unrolling smoothing}
\begin{align}
    \bz_k &= \mathbb{E}_{\epsilon\sim\mathcal{N}(0,\sigma^2I)} \cD(\bm{\theta}; \bx_k + \epsilon)
    \\
    \bx_{k+1} &=\underset{\bx}{\arg\min} \sum_{c=1}^{N_c}||A_c \bx - \by_c||_2^2 + \lambda ||\bx-\bz_k||^2
    \label{eq: unrolling smoothing optimization}
\end{align}
\end{subequations}
which is initialized with $\bx_0=A^H\by$, where $A^H$ transforms k-space data to the image domain. $\bz_k$ is estimated through the Monte Carlo sampling. In every unrolling step (\textit{i.e.} iteration), we first generate a number of Gaussian noises $\epsilon \sim \mathcal N(0, \sigma I^2)$, and add it to $\bx_k$. Then, we estimate the expectation of the denoised version of these images.
The denoiser $\cD (\cdot)$ is different from that in {\modl}, because the denoiser here is custom-trained to remove the Gaussian noises as well as alias artifacts and noises obtained in the undersampling and the upstream image processing. 
Finally, the conjugate gradient method is used to solve the optimization problem \eqref{eq: unrolling smoothing optimization}.


We evaluate the effectiveness of the three architectures towards adversarial robustness, and the results are shown in Table \ref{tab: exp_smoothing} and Fig. \ref{fig:archi_PSNR}. Experiment shows that {\us} has the highest robustness and comparable accuracy with other models. This implies the effectiveness of our proposed {\us}. We suspect that {\us} exploits the full potential of the presence of denoiser in each unrolling step compared to \ref{eq: denoised smoothing mri}, and avoids the noises introduced by the conjugate gradient method compared to {\us} on the whole.

\subsection{Unrolling loss}
\label{sec: how to smoothing}

The effectiveness of {\us} highly depends on the training scheme. We propose \textbf{unrolling loss} for better performance, as
{
\begin{align}
    \cL_{\mathrm{Unrolling}}&=\lambda'||\bx_K-\bx_\mathrm{label}||_2^2
    \notag
    \\
    +\sum_{k=0}^{K-1}&~\mathbb{E}_{\epsilon\sim\mathcal{N}(0,\sigma^2I)}||\cD(\bm{\theta}; \bx_k+\epsilon)-\cD(\bm{\theta}; \bx_\mathrm{label})||^2_2
    \label{eq: unrolling loss}
\end{align}
}%
where $\bx_K$ is the output of the $K$-th and the final unrolling step, \textit{i.e.} the final reconstructed image, $\bx_\mathrm{label}$ is the ground truth image. Inspired by TRADES \cite{zhang2019theoretically}, our \textbf{unrolling loss} contains a robustness regularizer besides a residual norm for accuracy, and a hyper-parameter $\lambda'$ is introduced to balance the trade-off between accuracy and robustness. When $\lambda'=0$, the \textbf{unrolling loss} exploits the full potential of our {\us} architecture towards robustness; when $\lambda'$ approaches $+\infty$, the \textbf{unrolling loss} penalties the model's inaccuracy directly. The expectation here is, again, estimated through the Monte Carlo sampling using the same noises as the forwarding propagation.

Moreover, we do not only rely on \textbf{unrolling loss} to train {\us}. In general, we adopt MSE + Unrolling Stab scheme following MSE + Stab scheme in denoised smoothing \cite{salman2020denoised}. First, the denoiser is pre-trained on the ground truth image to remove added Gaussian noises. We use an MSE loss as denoised smoothing, but $\bx$ here is the ground truth image.
This pre-training only takes a small amount of time compared to the following fine-tuning. However, the pre-training can find a good starting point for fine-tuning, which leads to remarkably higher robustness of the final model.
Then, the denoiser is fine-tuned through a stability loss. Conventionally, an end-to-end (E2E) stability loss is used, as
\begin{equation}
    \cL_{\mathrm{E2E}} = ~\mathbb{E}_{\epsilon\sim\mathcal{N}(0,\sigma^2I)}||f(\bm{\theta}; \bx, \epsilon) - f_\mathrm{base}(\bx)||_2^2%,\quad\epsilon\sim\mathcal{N}(0, \sigma I^2)
\end{equation}
where $f$ is the smoothed model, including \ref{eq: denoised smoothing mri}, and $f_\mathrm{base}$ is the base model, \textit{i.e.} the trained vanilla {\modl}. However, we adopt \textbf{unrolling loss} for fine-tuning. We study the effectiveness of our training scheme in Sec. \ref{sec: experiment}.

% We, instead, adopt an unrolling form of stability loss to exploit the unrolling architecture of {\us}, which we call \textit{unrolling loss}, as
% \begin{align}
%     \cL_{\mathrm{Unrolling}}&=\lambda'||\bx_K-\bx_\mathrm{label}||_2^2
%     \notag
%     \\
%     +\sum_{k=0}^{K-1}&~\mathbb{E}_{\epsilon\sim\mathcal{N}(0,\sigma^2I)}||\cD(\bm{\theta}; \bx_k+\epsilon)-\cD(\bm{\theta}; \bx_\mathrm{label})||^2_2
%     \label{eq: unrolling loss}
% \end{align}
% where $\bx_K$ is the output of the $K$-th and the final unrolling step, \textit{i.e.} the final reconstructed image, $\bx_\mathrm{label}$ is the ground truth image. Inspired by TRADES \cite{zhang2019theoretically}, our unrolling loss contains a robustness regularizer besides one for accuracy in our unrolling loss, and $\lambda'$ is a hyper-parameter that balances the trade-off between accuracy and robustness. When $\lambda'=0$, the unrolling loss exploits the full potential of our {\us} architecture towards robustness; when $\lambda'$ approaches $+\infty$, the unrolling loss penalties the model's inaccuracy directly. The expectation here is, again, estimated through the Monte Carlo sampling using the same noises as the forwarding propagation.

We emphasize that $\cD(\bm{\theta}; \bx_\mathrm{label})$ in the unrolling loss as the reference image is carefully selected. Because of the convergence and added Gaussian noises, the denoiser $\cD(\cdot)$ and its input image $\bx_n + \epsilon$ are both unstable in the training process. Consequently, the choice for the reference image in the unrolling loss is significant.
We select $\cD(\bm{\theta};\bx_\mathrm{label})$ as the reference image, in which $\cD(\cdot)$ countacts the instability of the denoiser, and $\bx_\mathrm{label}$ makes the training more aggressive, promoting training efficiency.

% \vspace{-3mm}

%%%

%%%
\section{Experiments}
\label{sec: experiment}

% \SL{[A summary paragraph to summarize your key metrics and results. E.g., In this section, we evaluate the effectiveness of our proposal from the following aspects xxxx. With a detailed comparison with the heuristics-based poisoning strategy, we find that xxx. ]}
%In this section, we evaluate different smoothing architectures and different training schemes discussed in Sec. \ref{sec: unrolling} and Sec. \ref{sec: how to smoothing} respectively. And we demonstrate the effectiveness of our proposed {\us} in three aspects of robustness mentioned in Sec. \ref{sec: preliminaries}.
% Result show that our proposed {\us} outperforms all other baselines in these three aspects.

% \vspace{-3mm}
\subsection{Experiment setup} 
%\vspace*{1mm}
\noindent \textbf{Models \& datasets.} 
The studied RS-baked {\modl} architectures are shown %in Fig.\,\ref{fig: model-arch} architecture is shown 
in \textbf{Figs.\,\ref{fig: RS-E2E}} and \textbf{\ref{fig: model-arch}}.
In experiments, we set  the total number of unrolling steps to $N = 8$, and set the denoising regularization parameter   $\lambda = 1$ in vanilla {\modl}. %\SL{[is my understanding  on $\lambda$ correct?]}\hui{[yes]}. 
%The number of unrolling steps in {\modl} $K$ and hyper-parameter $\lambda$ are set to 8 and 1 respectively to reach high reconstruction quality and time efficiency as well. 
For the denoising network $\mathcal{D}_{\btheta}$, we  use the Deep Iterative Down-Up Network (DIDN) \cite{yu2019deep} with three down-up blocks and 64 channels. We adopt 
the conjugate gradient method \cite{Aggarwal2019MoDL:Problems}  with tolerance $1e^{-6}$ to implement the DC block. 
We conduct our experiments on 
the \texttt{fastMRI} dataset \cite{zbontar2018fastmri}. 
The observed data $\by$ are obtained with $15$ coils and are  cropped to the resolution of $320\times320$ for MRI reconstruction. To implement the observation model, we adopt a Cartesian mask  at $4\times$ acceleration (\textit{i.e.}, $25\%$ sampling rate). 
The coil sensitivity maps for all cases were obtained using the BART toolbox \cite{tamir2016generalized}.
% % \JH{add references}
% %, 
% and all the coil sensitivity maps were estimated from undersampled (center of k-space) data to stimulate the realistic representation of real MRI experiments.

% \vspace*{-4mm}
\vspace*{1mm}
\noindent \textbf{Training \& evaluation.}
% Do we need to use the past tense?
We use 304 images for training, 32 images for validation, and 64 images for testing (that are unseen during training). 
%\SL{And we use xxx testing images selected from xxx?}
At \textbf{training time}, the batch size is set to $2$   trained on   two GPUs. 
%For every batch of images, the same noises $\epsilon$ were used in all unrolling steps.
We use the the Adam optimizer to train studied MRI reconstruction models with the momentum parameters $(0.5, 0.999)$.
%The models were trained using the Adam optimizer with parameter $\beta\text{s}=[0.5,0.999]$.
The number of epochs is set to $60$ with a linearly decaying learning rate from $10^{-4}$ to $0$ after epoch 20. The stability parameter $\lambda_\ell$ 
% \hui{$\lambda$ here}
in \eqref{eq: finetune_loss} is tuned so that  the standard accuracy  of the learned model is comparable to the vanilla {\modl}.
In RS, we set the standard deviation of Gaussian noise as 
$\sigma = 0.01$, and use $10$ Monte Carlo samplings to implement the smoothing operation. 
% \vspace*{-4mm}
%\paragraph*{Evaluation setup.}
% \SL{[mention baselines here and metrics to evaluate the performance]}
At \textbf{testing time}, 
we evaluate our methods on clean data, random noise-injected data and adversarial examples generated by 10-step PGD attack \cite{antun2020instabilities} of $\ell_\infty$-norm radius $\epsilon = 0.004$. The quality of reconstructed images is measured using  
peak signal-to-noise ratio (PSNR) and structure similarity (SSIM).
In addition to adversarial robustness, we also
evaluate the performance of our methods at the presence of another two perturbation sources (\textit{i.e.}, altered sampling rate and unrolling step number at testing time), as shown in \textbf{Fig.\,\ref{fig: robustness}}. 
% \hui{
% }
 
% against three   perturbation sources as s in Sec. \ref{sec: preliminaries} by changing the number of unrolling steps and the undersampling rate of masks.
 \begin{table}[htb]
\centering
% \vspace*{-1em}
\vspace*{-3mm}
\caption{\footnotesize{
% Clean accuracy and 
Accuracy performance  of different smoothing architectures (\ref{eq: denoised smoothing mri}, {\usold}, {\us}), together with the vanilla  {\modl}. Here `Clean Accuracy', `Noise Accuracy', and `Robust Accuracy' refer to PSNR/SSIM evaluated on benign data, random noise-injected data, and PGD attack-enabled adversarial data, respectively.
%with different training schemes, where \textsc{US} represents {\us}. 
%The noise accuracy and robust accuracy are evaluated with Gaussian noise $||\epsilon||_2 \le 0.004$ and adversarial perturbation $||\delta||_2 \le 0.004$ respectively.
$\uparrow$ signifies that a higher   number indicates a better reconstruction accuracy. The result $a${\tiny{$\pm b$}} represents mean $a$ and standard deviation $b$ over {64} testing images.
The relative performance is reported with respect to that of vanilla {\modl}.
% \hui{, in the format mean\tiny{$\pm$std}.} 
% The best performance is highlighted in \textbf{bold}.
}}
\label{tab: exp_smoothing}
\vspace*{-2mm}
\resizebox{0.48\textwidth}{!}{%
\begin{tabular}{c|cc|cc|cc}
\toprule[1pt]
\midrule
Models 
& \multicolumn{2}{c}{Clean Accuracy} 
& \multicolumn{2}{c}{Noise Accuracy} 
& \multicolumn{2}{c}{Robust Accuracy} 
\\
Metrics 
& PSNR \textcolor{red}{$\uparrow$} & SSIM \textcolor{red}{$\uparrow$}
& PSNR \textcolor{red}{$\uparrow$} & SSIM \textcolor{red}{$\uparrow$}
& PSNR \textcolor{red}{$\uparrow$} & SSIM \textcolor{red}{$\uparrow$}
\\
\midrule
Vanilla {\modl}
& 29.73\footnotesize{$\pm$3.27}
& 0.900\footnotesize{$\pm$0.07}
& 28.70\footnotesize{$\pm$2.77}
& 0.874\footnotesize{$\pm$0.07}
& 22.91\footnotesize{$\pm$2.42}
& 0.729\footnotesize{$\pm$0.07}
\\
\midrule
%\ref{eq: denoised smoothing mri}
RS-E2E
& \textbf{+0.09}\footnotesize{$\pm$3.24}
% & \textbf{+0.0016}
& \textbf{+0.002}\footnotesize{$\pm$0.07}
& +0.38\footnotesize{$\pm$2.90}
% & +0.0101
& +0.010\footnotesize{$\pm$0.07}
& +0.78\footnotesize{$\pm$2.70}
& \textbf{+0.034}\footnotesize{$\pm$0.08}
\\
{\usold}  %($\cD + \mathcal{DC}$)}
& -1.01\footnotesize{$\pm$3.07}
& -0.014\footnotesize{$\pm$0.08}
& -0.09\footnotesize{$\pm$2.99}
% & +0.0079
& +0.008\footnotesize{$\pm$0.08}
& +3.08\footnotesize{$\pm$2.42}
& -0.014\footnotesize{$\pm$0.11}
\\
\rowcolor[gray]{.8}
{\us} (ours)
& -0.34\footnotesize{$\pm$3.06}
% & -0.0057
& -0.006\footnotesize{$\pm$0.08}
& \textbf{+0.53}\footnotesize{$\pm$2.98}
& \textbf{+0.016}\footnotesize{$\pm$0.08}
& \textbf{+3.87}\footnotesize{$\pm$2.28}
% & +0.0082
& +0.008\footnotesize{$\pm$0.11}
\\
\midrule
\bottomrule[1pt]
\end{tabular}%
}
\vspace*{-4mm}
\end{table}

\subsection{Experiment results}
   
%\SL{[You can also give paragraph title, like above, to organize your experiment findings.]}
   
%\SL{In Figure/Table, we present xxxxx. As we can see, xxx.  This implies that xxx}

%\SL{In Figure/Table, we show xxxx. We observe that xxx.}


%\begin{figure}[htb]
\begin{wrapfigure}{r}{43mm}
\vspace*{-5mm}
   % \begin{minipage}{0.23\textwidth}
    \centering
  \hspace*{-3mm}  \includegraphics[width=0.25\textwidth]{Figures/smoothing_archi_PSNR.pdf}
    \vspace*{-3mm}
    \caption{\footnotesize{PSNR  of   baseline methods and proposed {\us} versus perturbation strength $\epsilon$ used in   PGD attack-generated adversarial examples at testing time. The case of $\epsilon =0$ corresponds to clean accuracy. 
    % \SL{[update figure legend, {\usold}]}
    %(\textit{i.e.} PGD $\epsilon = 0$) and adversarial examples generated by 10-step PGD. \ref{eq: denoised smoothing mri} is trained using MSE + Stab scheme, and {\us} is trained using MSE + Unrolling Stab scheme.
    }}
    \label{fig:archi_PSNR}
  %  \end{figure}
  \vspace*{-5mm}
  \end{wrapfigure}
\textbf{Table\,\ref{tab: exp_smoothing}} shows PSNR and SSIM values for different smoothing architectures with different training schemes, along with vanilla {\modl} as a baseline, evaluated on clean and adversarial test datasets. We present the PSNR results for these models under different scales of adversarial perturbations (\textit{i.e.}, attack strength $\epsilon$) in \textbf{Fig.\,\ref{fig:archi_PSNR}}.
We observe that our method {\us} outperforms all other models in robustness, consistent with the visualization of reconstructed images in \textbf{Fig.\,\ref{fig: vis}}. Also, {\us} yields a promising clean accuracy performance, which is better  than {\usold} and   comparable to the vanilla {\modl} model. This shows the effectiveness of our proposed method for improving robustness while preserving clean accuracy \emph{(i.e., without the perturbations)}.   
%overall architecture, {\us} with unrolling loss training scheme, in providing adversarial robustness for {\modl}. 
%This discovery indicates integrating randomized smoothing with complicated models, compared to the single neural network architecture in the image classification task. All we need is to convert the denoiser into a smoothed one and leave the rest of the model unchanged, \textit{i.e.} adding Gaussian noises before the denoiser and estimating the expectation of outputs of the denoiser \textit{w.r.t.} that noise. Moreover, a specific loss function should be designed to match the architecture of the model to reach high robustness.



% \vspace*{-3mm}
% \begin{wrapfigure}{l}{.22\textwidth}
%     \centering
%     \includegraphics[width=.22\textwidth]{Figures/smoothing_archi_PSNR.pdf}
%     \caption{\footnotesize{PSNR results of different smoothing architectures evaluated on clean data ($\epsilon$=0) and adversarial data generated by 10-step PGD.}}
%     \label{fig:archi_PSNR}
% \end{wrapfigure}
% \begin{wrapfigure}{r}{.22\textwidth}
%     \centering
%     \includegraphics[width=.22\textwidth]{Figures/reference_image_PSNR.pdf}
%     \caption{\footnotesize{PSNR results of {\us}, trained using different reference image in unrolling loss, evaluated on clean data ($\epsilon$=0) and adversarial data generated by 10-step PGD.}}
%     \label{fig:reference_PSNR}
% \end{wrapfigure}

% \begin{figure}[htb]
%     \begin{minipage}{0.23\textwidth}
%     \centering
%     \includegraphics[width=\textwidth]{Figures/smoothing_archi_PSNR.pdf}
%     \vspace*{-6mm}
%     \caption{\footnotesize{PSNR results of different smoothing architectures evaluated on clean data (\textit{i.e.} PGD $\epsilon = 0$) and adversarial examples generated by 10-step PGD. \ref{eq: denoised smoothing mri} is trained using MSE + Stab scheme, and {\us} is trained using MSE + Unrolling Stab scheme.}}
%     \label{fig:archi_PSNR}
% \end{minipage}\hfill
% \begin{minipage}{0.23\textwidth}

%     \centering
%     \includegraphics[width=\textwidth]{Figures/reference_image_PSNR.pdf}
%     \vspace*{-6mm}
%     \caption{\footnotesize{PSNR results of {\us}, trained using different reference images in the unrolling loss, evaluated on clean data (\textit{i.e.} PGD $\epsilon = 0$) and adversarial examples generated by 10-step PGD. $\cD_{\mathrm{base}}$ represents the denoiser in the base model, \textit{i.e.} trained vanilla MoDL.}}
%     \label{fig:reference_PSNR}
%     \end{minipage}\hfill
% \vspace*{-4mm}
% \end{figure}

\begin{figure}[htb]

\begin{tabular}[b]{cccc}
        \includegraphics[width=.21\linewidth, trim=70 10 70 10]{Figures/visualization/ground_truth.pdf}
        &\hspace{-0.2cm}
        \includegraphics[width=.21\linewidth, trim=70 10 70 10]{Figures/visualization/vanilla_MoDL_eps0.5_255.pdf}
        &\hspace{-0.2cm}
        \includegraphics[width=.21\linewidth, trim=70 10 70 10]{Figures/visualization/E2E_smoothing_E2E_loss_eps0.5_255.pdf}

        &\hspace{-0.2cm}
        \includegraphics[width=.21\linewidth, trim=70 10 70 10]{Figures/visualization/unrolling_smoothing_unrolling_loss_eps0.5_255.pdf}
        \\[-0pt]
        \scriptsize{(a) Ground Truth} 
        &\hspace{-0.2cm} 
        \scriptsize{(b) Vanilla {\modl}} 
        &\hspace{-0.2cm} 
        \scriptsize{(c) RS-E2E}
        &\hspace{-0.2cm} 
        \scriptsize{(d) \textsc{SMUG}} 
        \\ 
\end{tabular}
\vspace*{-4mm}
\caption{\footnotesize{Visualization of   ground-truth   and reconstructed images using different methods, evaluated on PGD attack-generated adversarial inputs of perturbation strength $\epsilon = 0.002$.
%with perturbation strength $\epsilon = 0.002$.
%, and {\us}. 
%\ref{eq: denoised smoothing mri} is trained using the E2E loss and {\us} is trained using the unrolling loss.
}}
\label{fig: vis}
\vspace*{-2mm}
\end{figure}


% \begin{table}[htb]
% \centering
% % \vspace*{-1em}
% \caption{Clean accuracy and robust accuracy of {\us} with different reference image in loss function, where $\cD_{\theta}$ is the denoier in {\us} and $\cD_{\theta_0}$ is the denoier in vanilla {\modl}. The robust accuracy is tested with adversarial perturbation $||\delta||_2 \le 0.5/255$.
% % The best performance is highlighted in \textbf{bold}.
% }
% \label{tab: exp_unrolling_loss}
% \resizebox{0.48\textwidth}{!}{%
% \begin{tabular}{c|ccc|ccc}
% \toprule[1pt]
% \midrule
% Reference image & \multicolumn{3}{c}{Clean Accuracy} & \multicolumn{3}{c}{Robust Accuracy} \\
% Metrics & RMSE $\downarrow$ & PSNR $\uparrow$ & SSIM $\uparrow$ & RMSE $\downarrow$ & PSNR $\uparrow$ & SSIM $\uparrow$ \\
% \midrule
% $\bx_n$
% & 0.0256
% & 30.2319
% & 0.9058
% & 0.0347
% & 27.3069
% & 0.8017
% \\
% $\bx_\text{label}$
% & 0.026
% & 30.0599
% & 0.904
% & 0.0428 
% & 25.4393
% & 0.7793
% \\
% $\cD_{\theta_0}(\bx_n)$
% & 0.0259 
% & 30.0624 
% & 0.9038 
% & 0.0339 
% & 27.6306
% & 0.8681
% \\
% $\cD_{\theta_0}(\bx_\text{label})$
% & 0.0262 
% & 29.9721 
% & 0.9034 
% & 0.0385 
% & 26.4517 
% & 0.831
% \\
% $\cD_{\theta}(\bx_n)$
% & 0.0269 
% & 29.7071 
% & 0.897 
% & 0.0319 
% & 28.1294 
% & \textbf{0.8751}
% \\
% \rowcolor[gray]{.8}
% $\cD_{\theta}(\bx_\text{label})$
% & 0.0278 
% & 29.3913 
% & 0.8939 
% & \textbf{0.0308}
% & \textbf{28.4137} 
% & 0.8467
% \\
% \midrule
% \bottomrule[1pt]
% \end{tabular}%
% }
% \vspace*{-3mm}
% \end{table}

Next, we evaluate the  effectiveness  of MRI reconstruction methods when facing sampling rate and unrolling step perturbations at testing time.  In other words, there exists a test-time shift for the training  setup of MRI reconstruction.  
In \textbf{Fig.\,\ref{fig: robustness}}, we present the evaluation results of {\us}, with two baselines, vanilla {\modl} and \ref{eq: denoised smoothing mri}, on different unrolling steps and sampling rates. Note that   these models are trained with the number of unrolling steps $K=8$ and sampling masks with the $4\times$ {acceleration (\textit{i.e.}, 25\% sampling rate).}
% undersampling factor. 
As we can see, {\us} achieves a remarkable improvement in robustness against different sampling rates and unrolling steps, which {\modl} and \ref{eq: denoised smoothing mri} fail to achieve.
Although we do not intentionally design our method to mitigate {\modl}'s instabilities against perturbed sampling rate and unrolling step number, {\us} still provides improved PSNRs over  other baselines. 
We credit the improvement to the close relationships between these two instabilities with adversarial robustness. 
%For instability against the undersampling rates, we notice that masks of different undersampling rates are similar to adversarial perturbed masks. And for instability against the number of unrolling steps, we argue that {\us} alleviates the instability with the adversarial robustness of \textit{each} unrolling step, thereby preventing degradation through steps.
% Furthermore, \textbf{Fig.\,\ref{fig: smug weakness}} shows examples of  reconstructed images associated with \textbf{Fig.\,\ref{fig: robustness}}.
%And the visualization of these is shown in Fig. \ref{fig: weakness} for {\modl} and for {\us}.

\begin{figure}[htb]
\vspace*{-3mm}
    \centering
\begin{tabular}{cc}
  \hspace*{-2mm}  \includegraphics[width=0.24\textwidth]{Figures/robustness/undersampling_rate_PSNR_new.pdf}   &    \hspace*{-5mm} \includegraphics[width=0.24\textwidth]{Figures/robustness/unrolling_steps_PSNR.pdf}
\end{tabular}
    % \begin{subfigure}{.23\textwidth}
    %     \centering
    %     \includegraphics[width=\textwidth]{Figures/robustness/undersampling_rate_PSNR.pdf}
    %     \caption{\footnotesize{Undersampling Rate
    %     }}
    % \end{subfigure}
    % \begin{subfigure}{.23\textwidth}
    %     \centering
    %     \includegraphics[width=\textwidth]{Figures/robustness/unrolling_steps_PSNR.pdf}
    %     \caption{\footnotesize{Unrolling Step
    %     }}
    % \end{subfigure}
    \vspace*{-5mm}
    \caption{\footnotesize{
    %\hui{
    PSNR results of different MRI reconstruction methods versus
    different measurement sampling rates ($4\times$ acceleration \textit{i.e.}, $25\%$ sampling rate at training; Left plot) and
    unrolling steps (8 at training; Right plot). 
  %  }
    % \SL{caption is wrong.}
    %The robust accuracy is evaluated with adversarial perturbation $||\delta||_2 \le 0.002$. %{It's worth noting that the higher the undersampling rate is, the sparser the mask is.}
    %\hui{x-axis needs to be changed.}
    }}
    \label{fig: robustness}
    \vspace*{-2mm}
\end{figure}

% \begin{figure}[htb]
% % \vspace{-3mm}
% \centering
% \begin{tabular}[b]{ccc}
%         \includegraphics[width=.28\linewidth, trim=70 10 70 10]{Figures/visualization/SMUG_recon.pdf}
%         &
%         \includegraphics[width=.28\linewidth, trim=70 10 70 10]{Figures/visualization/SMUG_undersampling_2.0_vis.pdf}
%         &
%         \includegraphics[width=.28\linewidth, trim=70 10 70 10]{Figures/visualization/SMUG_unrolling_steps_16_clean_vis.pdf}
%         \\[-0pt]
%         \scriptsize{(a)} 
%         & 
%         \scriptsize{(b)} 
%         &  
%         \scriptsize{(c)}
% \end{tabular}
% \vspace*{-2mm}
% \caption{\footnotesize{Visualization of reconstruction results of {\us} for different undersampling rates and   unrolling steps. (a)  Reconstructed image from the benign measurement; (b) Reconstructed image when   using $2\times$ acceleration (corresponding to Fig.\,\ref{fig: robustness}-Left); (c)  Reconstructed image   when   using 16 unrolling steps (corresponding to Fig.\,\ref{fig: robustness}-Right).
% \SL{[caption is wrong, and consider to remove it.]}
% }}
% \label{fig: smug weakness}
% \end{figure}

% \begin{table}[htb]
% \centering
% % \vspace*{-1em}
% \caption{Clean accuracy and robust accuracy of {\us} with different $\sigma$. The robust accuracy is tested with adversarial perturbation $||\delta||_2 \le 0.5/255$.
% % The best performance is highlighted in \textbf{bold}.
% }
% \label{tab: ablation_sigma}
% \resizebox{0.48\textwidth}{!}{%
% \begin{tabular}{c|ccc|ccc}
% \toprule[1pt]
% \midrule
% $\sigma$ & \multicolumn{3}{c}{Clean Accuracy} & \multicolumn{3}{c}{Robust Accuracy} \\
% Metrics & RMSE $\downarrow$ & PSNR $\uparrow$ & SSIM $\uparrow$ & RMSE $\downarrow$ & PSNR $\uparrow$ & SSIM $\uparrow$ \\
% \midrule
% 0.001 
% & 0.0274 
% & 29.5286 
% & 0.8943 
% & 0.0302 
% & 28.464 
% & 0.7916
% \\
% 0.005 & 0.0282 & 29.2658 & 0.8929 & 0.0311 & 28.3198 & 0.8594
% \\
% 0.01 & 0.0278 & 29.3913 & 0.8939 & 0.0308 & 28.4137 & 0.8467
% \\
% 0.05 & 0.0325 & 28.0298 & 0.874 & 0.0333 & 27.7624 & 0.8499
% \\
% 0.1 & 0.0322 & 28.0921 & 0.8764 & 0.0325 & 28.0224 & 0.8746
% \\
% \midrule
% \bottomrule[1pt]
% \end{tabular}%
% }
% \vspace*{-3mm}
% \end{table}

% \begin{table}[htb]
% \centering
% % \vspace*{-1em}
% \caption{Clean accuracy and robust accuracy of {\us} with different Monte Carlo sampling number $n$. The robust accuracy is tested with adversarial perturbation $||\delta||_2 \le 0.5/255$.
% % The best performance is highlighted in \textbf{bold}.
% }
% \label{tab: ablation_n}
% \resizebox{0.48\textwidth}{!}{%
% \begin{tabular}{c|ccc|ccc}
% \toprule[1pt]
% \midrule
% $n$ & \multicolumn{3}{c}{Clean Accuracy} & \multicolumn{3}{c}{Robust Accuracy} \\
% Metrics & RMSE $\downarrow$ & PSNR $\uparrow$ & SSIM $\uparrow$ & RMSE $\downarrow$ & PSNR $\uparrow$ & SSIM $\uparrow$ \\
% \midrule
% 3 & 0.0306 & 28.5343 & 0.8815 & 0.0324 & 27.9376 & 0.8235
% \\
% 5 & 0.0292 & 28.9498 & 0.8886 & 0.0315 & 28.1755 & 0.8301
% \\
% 10 & 0.0278 & 29.3913 & 0.8939 & 0.0308 & 28.4137 & 0.8467
% \\
% \midrule
% \bottomrule[1pt]
% \end{tabular}%
% }
% \vspace*{-3mm}
% \end{table}

%\subsection{Ablation study}



\begin{wrapfigure}{r}{43mm}
\vspace*{-6mm}
\centering
  \hspace*{-3mm}  \includegraphics[width=0.25\textwidth]{Figures/reference_image_PSNR.pdf}
    \vspace*{-4mm}
    \caption{\footnotesize{PSNR vs. adversarial attack strength ($\epsilon)$ of {\us} for different configurations of   UStab loss \eqref{eq: unrolling loss}. %$\mathcal{D}_{{\btheta}_{\textsc{MoDL}}}$ represents the denoiser in vanilla {\modl} \eqref{eq:altmin}. The case of $\epsilon =0$ corresponds to clean accuracy. 
    }}
    \label{fig: reference_PSNR}
  %  \end{figure}
  \vspace*{-5mm}
\end{wrapfigure}
We conduct additional experiments showing the importance of integrating    target image denoising into {\us}'s training pipeline in \eqref{eq: unrolling loss}. \textbf{Fig.\,\ref{fig: reference_PSNR}} shows PSNR versus perturbation strength ($\epsilon$) when using different alternatives to $\mathcal D_{\btheta} (\mathbf t)$ in~\eqref{eq: unrolling loss}, including 
$\mathbf t$ (the original target image), $\mathcal D_{\btheta}(\mathbf x_n)$ (denoised   output of each unrolling step), and their variants when using the fixed, vanilla {\modl}'s denoiser $\mathcal D_{\btheta_\text{\modl}}$ instead.
% we 
% on different reference images in UStab loss \eqref{eq: unrolling loss} instead of $\mathcal D_{\btheta}(\mathbf t)$, whose results are shown in Fig. \ref{fig: reference_PSNR}. 
As we can see, the performance of {\us} varies when the UStab loss \eqref{eq: unrolling loss} is configured differently. The proposed $\mathcal D_{\btheta}(\mathbf t)$ outperforms the other baselines. A possible reason is that it infuses supervision of target images in an adaptive, denoising-friendly manner, \textit{i.e.}, taking influence of $\mathcal D_{\btheta}$ into consideration.
%We suspect $\mathcal D_{\btheta}(\mathbf t)$ countacts the influence of the instability of $\cD_{\btheta}$ in the training process and makes training aggressive by having $\mathbf{t}$ in it.


% Below is an example of how to insert images. Delete the ``\vspace'' line,
% uncomment the preceding line ``\centerline...'' and replace ``imageX.ps''
% with a suitable PostScript file name.
% -------------------------------------------------------------------------


% To start a new column (but not a new page) and help balance the last-page
% column length use \vfill\pagebreak.
% -------------------------------------------------------------------------
%\vfill
%\pagebreak
\vspace*{-4mm}


% \section{Experiments}
% \label{sec: experiment}

% % \SL{[A summary paragraph to summarize your key metrics and results. E.g., In this section, we evaluate the effectiveness of our proposal from the following aspects xxxx. With a detailed comparison with the heuristics-based poisoning strategy, we find that xxx. ]}
% %In this section, we evaluate different smoothing architectures and different training schemes discussed in Sec. \ref{sec: unrolling} and Sec. \ref{sec: how to smoothing} respectively. And we demonstrate the effectiveness of our proposed {\us} in three aspects of robustness mentioned in Sec. \ref{sec: preliminaries}.
% % Result show that our proposed {\us} outperforms all other baselines in these three aspects.

% % \vspace{-3mm}
% \subsection{Experiment setup} 
% %\vspace*{1mm}
% \noindent \textbf{Datasets \& model architectures.} 
% Following the 
% Overall model architecture has been shown in Fig. \ref{fig: model-arch}. The number of unrolling steps in {\modl} $K$ and hyper-parameter $\lambda$ are set to 8 and 1 respectively to reach high reconstruction quality and time efficiency as well. We use Deep Iterative Down-Up Network (DIDN) \cite{yu2019deep} with three down-up blocks and 64 channels as denoiser $\cD(\cdot)$, and the conjugate gradient method with tolerance $1e^{-6}$ to solve the optimization problem in $\mathcal{DC}$ block. We use part of \texttt{fastMRI} \cite{zbontar2018fastmri} multi-coil knee validation dataset for evaluation. The k-space data $\by$ has 15 coils and is cropped to $320\times320$ for reconstruction. We adopt a Cartesian mask  at 4x acceleration (25.0\% sampling). Also, the coil sensitivity maps for all cases are obtained using the BART toolbox \cite{tamir2016generalized}
% % \JH{add references}
% , and all the coil sensitivity maps are estimated from undersampled data to stimulate the realistic representation of real MRI experiments.

% % \vspace*{-4mm}
% \vspace*{1mm}
% \noindent \textbf{Training \& evaluation.}
% % Do we need to use the past tense?
% We used 304 images for training and 32 images for validation. We used a batch size of 2 which were simultaneously trained on 2 GPUs. For every batch of images, the same noises $\epsilon$ were used in all unrolling steps. The models were trained using the Adam optimizer with parameter $\beta\text{s}=[0.5,0.999]$. The number of epochs is set to 60 with a linearly decaying learning rate from $1e^{-4}$ to 0 after epoch 20. The hyper-parameter $\lambda'$ is selected, if possible, to generate similar accuracy in all scenarios.
% $\sigma$ used in Gaussian noise sampling is set to 0.01 to balance accuracy with robustness, and Monte Carlo sampling number $n$ is set to 10 due to the memory limitation of our GPUs.

% % \vspace*{-4mm}
% %\paragraph*{Evaluation setup.}
% % \SL{[mention baselines here and metrics to evaluate the performance]}

% We evaluate our methods on clean images and adversarial examples generated by 10-step projected gradient descent (PGD) attack \cite{antun2020instabilities} with 64 test images. The quality of reconstructed images is measured using metrics %root mean square error (RMSE) ,
% peak signal-to-noise ratio (PSNR), and structure similarity (SSIM)
% % \JH{full name for them}
% . We also evaluate our methods on the other two types of robustness mentioned in Sec. \ref{sec: preliminaries} by changing the number of unrolling steps and the undersampling rate of masks.
 
% \subsection{Experiment results}
   
% %\SL{[You can also give paragraph title, like above, to organize your experiment findings.]}
   
% %\SL{In Figure/Table, we present xxxxx. As we can see, xxx.  This implies that xxx}

% %\SL{In Figure/Table, we show xxxx. We observe that xxx.}

% Table \ref{tab: exp_smoothing} shows PSNR, and SSIM of different smoothing architectures with different training schemes, along with vanilla {\modl} as a baseline, evaluated on clean and adversarial test dataset. And we present the PSNR results of these models under different scales of adversarial perturbations in Fig. \ref{fig:archi_PSNR}.
% Our method {\us} with unrolling loss outperforms all other models in robustness, consistent with visualization in Fig. \ref{fig: vis}. Also, {\us} shows  better, clean accuracy with other models. This shows the effectiveness of our overall architecture, {\us} with unrolling loss training scheme, in generating adversarial robustness on {\modl}. 
% %This discovery indicates integrating randomized smoothing with complicated models, compared to the single neural network architecture in the image classification task. All we need is to convert the denoiser into a smoothed one and leave the rest of the model unchanged, \textit{i.e.} adding Gaussian noises before the denoiser and estimating the expectation of outputs of the denoiser \textit{w.r.t.} that noise. Moreover, a specific loss function should be designed to match the architecture of the model to reach high robustness.


% \begin{table}[htb]
% \centering
% % \vspace*{-1em}
% % \vspace*{-3mm}
% \caption{\footnotesize{
% % Clean accuracy and 
% Robust accuracy of smoothing architectures with different training schemes, where \textsc{US} represents {\us}. The noise accuracy and robust accuracy are evaluated with Gaussian noise $||\epsilon||_2 \le 0.004$ and adversarial perturbation $||\delta||_2 \le 0.004$ respectively.
% The relative performance is shown with respect to vanilla {\modl}.
% % The best performance is highlighted in \textbf{bold}.
% }}
% \label{tab: exp_smoothing}
% \vspace*{-2mm}
% \resizebox{0.48\textwidth}{!}{%
% \begin{tabular}{c|cc|cc|cc}
% \toprule[1pt]
% \midrule
% Models 
% & \multicolumn{2}{c}{Clean Accuracy} 
% & \multicolumn{2}{c}{Noise Accuracy} 
% & \multicolumn{2}{c}{Robust Accuracy} 
% \\
% Metrics 
% & PSNR \textcolor{red}{$\uparrow$} & SSIM \textcolor{red}{$\uparrow$}
% & PSNR \textcolor{red}{$\uparrow$} & SSIM \textcolor{red}{$\uparrow$}
% & PSNR \textcolor{red}{$\uparrow$} & SSIM \textcolor{red}{$\uparrow$}
% \\
% \midrule
% Vanilla {\modl}
% & 29.7366 
% & 0.8996 
% & 28.7057
% & 0.8735
% & 22.9133
% & 0.7286
% \\
% \midrule
% \ref{eq: denoised smoothing mri}
% & \textbf{+0.0915 }
% & \textbf{+0.0016 }
% & +0.3865
% & +0.0101
% & +0.7869
% & \textbf{+0.0337}
% \\
% {{\us} ($\cD + \mathcal{DC}$)}
% & -1.013
% & -0.0141
% & -0.0945
% & +0.0079
% & +3.0829
% & -0.0141
% \\
% \rowcolor[gray]{.8}
% {\us} (ours)
% & -0.3453 
% & -0.0057
% & \textbf{+0.5376}
% & \textbf{+0.0162}
% & \textbf{+3.8723}
% & +0.0082
% \\
% \midrule
% \bottomrule[1pt]
% \end{tabular}%
% }
% \vspace*{-4mm}
% \end{table}

% % \vspace*{-3mm}
% % \begin{wrapfigure}{l}{.22\textwidth}
% %     \centering
% %     \includegraphics[width=.22\textwidth]{Figures/smoothing_archi_PSNR.pdf}
% %     \caption{\footnotesize{PSNR results of different smoothing architectures evaluated on clean data ($\epsilon$=0) and adversarial data generated by 10-step PGD.}}
% %     \label{fig:archi_PSNR}
% % \end{wrapfigure}
% % \begin{wrapfigure}{r}{.22\textwidth}
% %     \centering
% %     \includegraphics[width=.22\textwidth]{Figures/reference_image_PSNR.pdf}
% %     \caption{\footnotesize{PSNR results of {\us}, trained using different reference image in unrolling loss, evaluated on clean data ($\epsilon$=0) and adversarial data generated by 10-step PGD.}}
% %     \label{fig:reference_PSNR}
% % \end{wrapfigure}

% \begin{figure}
%     \begin{minipage}{0.23\textwidth}
%     \centering
%     \includegraphics[width=\textwidth]{Figures/smoothing_archi_PSNR.pdf}
%     \vspace*{-6mm}
%     \caption{\footnotesize{PSNR results of different smoothing architectures evaluated on clean data (\textit{i.e.} PGD $\epsilon = 0$) and adversarial examples generated by 10-step PGD. \ref{eq: denoised smoothing mri} is trained using MSE + Stab scheme, and {\us} is trained using MSE + Unrolling Stab scheme.}}
%     \label{fig:archi_PSNR}
% \end{minipage}\hfill
% \begin{minipage}{0.23\textwidth}

%     \centering
%     \includegraphics[width=\textwidth]{Figures/reference_image_PSNR.pdf}
%     \vspace*{-6mm}
%     \caption{\footnotesize{PSNR results of {\us}, trained using different reference images in the unrolling loss, evaluated on clean data (\textit{i.e.} PGD $\epsilon = 0$) and adversarial examples generated by 10-step PGD. $\cD_{\mathrm{base}}$ represents the denoiser in the base model, \textit{i.e.} trained vanilla MoDL.}}
%     \label{fig:reference_PSNR}
%     \end{minipage}\hfill
% \vspace*{-4mm}
% \end{figure}

% \begin{figure}[htb]

% \begin{tabular}[b]{cccc}
%         \includegraphics[width=.21\linewidth, trim=70 10 70 10]{Figures/visualization/ground_truth.pdf}
%         &\hspace{-0.2cm}
%         \includegraphics[width=.21\linewidth, trim=70 10 70 10]{Figures/visualization/vanilla_MoDL_eps0.5_255.pdf}
%         &\hspace{-0.2cm}
%         \includegraphics[width=.21\linewidth, trim=70 10 70 10]{Figures/visualization/E2E_smoothing_E2E_loss_eps0.5_255.pdf}

%         &\hspace{-0.2cm}
%         \includegraphics[width=.21\linewidth, trim=70 10 70 10]{Figures/visualization/unrolling_smoothing_unrolling_loss_eps0.5_255.pdf}
%         \\[-0pt]
%         \scriptsize{(a) Ground Truth} 
%         &\hspace{-0.2cm} 
%         \scriptsize{(b) Vanilla {\modl}} 
%         &\hspace{-0.2cm} 
%         \scriptsize{(c) RS-E2E}
%         &\hspace{-0.2cm} 
%         \scriptsize{(d) \textsc{SMUG}} 
%         \\ 
% \end{tabular}
% \vspace*{-4mm}
% \caption{\footnotesize{Visualization of the ground truth image and reconstruction results of different model architectures when evaluated on adversarial examples with perturbation $||\delta||_2 \le 0.002$
% , and {\us}. \ref{eq: denoised smoothing mri} is trained using the E2E loss and {\us} is trained using the unrolling loss.}}
% \label{fig: vis}
% \vspace*{-2mm}
% \end{figure}

% We also conduct experiments on different reference images in unrolling loss, whose results are shown in Fig. \ref{fig:reference_PSNR}. We find that the performance of {\us} varies using different reference images in the loss function, and those containing $\cD(\cdot)$ get better results. This conforms with our speculation on the instability of $\cD(\cdot)$ in the training process and substantiates the benefit of using $\cD(\bm{\theta};\bx_\mathrm{label})$ in loss function as a reference image.

% % \begin{table}[htb]
% % \centering
% % % \vspace*{-1em}
% % \caption{Clean accuracy and robust accuracy of {\us} with different reference image in loss function, where $\cD_{\theta}$ is the denoier in {\us} and $\cD_{\theta_0}$ is the denoier in vanilla {\modl}. The robust accuracy is tested with adversarial perturbation $||\delta||_2 \le 0.5/255$.
% % % The best performance is highlighted in \textbf{bold}.
% % }
% % \label{tab: exp_unrolling_loss}
% % \resizebox{0.48\textwidth}{!}{%
% % \begin{tabular}{c|ccc|ccc}
% % \toprule[1pt]
% % \midrule
% % Reference image & \multicolumn{3}{c}{Clean Accuracy} & \multicolumn{3}{c}{Robust Accuracy} \\
% % Metrics & RMSE $\downarrow$ & PSNR $\uparrow$ & SSIM $\uparrow$ & RMSE $\downarrow$ & PSNR $\uparrow$ & SSIM $\uparrow$ \\
% % \midrule
% % $\bx_n$
% % & 0.0256
% % & 30.2319
% % & 0.9058
% % & 0.0347
% % & 27.3069
% % & 0.8017
% % \\
% % $\bx_\text{label}$
% % & 0.026
% % & 30.0599
% % & 0.904
% % & 0.0428 
% % & 25.4393
% % & 0.7793
% % \\
% % $\cD_{\theta_0}(\bx_n)$
% % & 0.0259 
% % & 30.0624 
% % & 0.9038 
% % & 0.0339 
% % & 27.6306
% % & 0.8681
% % \\
% % $\cD_{\theta_0}(\bx_\text{label})$
% % & 0.0262 
% % & 29.9721 
% % & 0.9034 
% % & 0.0385 
% % & 26.4517 
% % & 0.831
% % \\
% % $\cD_{\theta}(\bx_n)$
% % & 0.0269 
% % & 29.7071 
% % & 0.897 
% % & 0.0319 
% % & 28.1294 
% % & \textbf{0.8751}
% % \\
% % \rowcolor[gray]{.8}
% % $\cD_{\theta}(\bx_\text{label})$
% % & 0.0278 
% % & 29.3913 
% % & 0.8939 
% % & \textbf{0.0308}
% % & \textbf{28.4137} 
% % & 0.8467
% % \\
% % \midrule
% % \bottomrule[1pt]
% % \end{tabular}%
% % }
% % \vspace*{-3mm}
% % \end{table}

% In Fig \ref{fig: robustness}, we present the evaluation results of {\us}, trained by unrolling loss, with two baselines, vanilla {\modl} and \ref{eq: denoised smoothing mri}, on different unrolling steps and undersampling rates, noting these models are trained with the number of unrolling steps $K=8$ and masks with the 4x undersampling factor . And the visualization of these is shown in Fig. \ref{fig: weakness} for {\modl} and Fig. \ref{fig: smug weakness} for {\us}.
% As we can see, {\us} achieves a remarkable improvement in robustness against different undersampling rates and unrolling steps, which \ref{eq: denoised smoothing mri} fails to achieve.
% Although we do not intentionally design our method to mitigate {\modl}'s instabilities against undersampling rate and unrolling steps, {\us} still gets better performance compared to other baselines. We accredit the improvement to the close relationships between these two instabilities with the adversarial robustness. %For instability against the undersampling rates, we notice that masks of different undersampling rates are similar to adversarial perturbed masks. And for instability against the number of unrolling steps, we argue that {\us} alleviates the instability with the adversarial robustness of \textit{each} unrolling step, thereby preventing degradation through steps.

% \begin{figure}[htb]
% % \vspace*{-3mm}
%     \centering

%     \begin{subfigure}{.23\textwidth}
%         \centering
%         \includegraphics[width=\textwidth]{Figures/robustness/undersampling_rate_PSNR.pdf}
%         \caption{\footnotesize{Undersampling Rate
%         }}
%     \end{subfigure}
%     \begin{subfigure}{.23\textwidth}
%         \centering
%         \includegraphics[width=\textwidth]{Figures/robustness/unrolling_steps_PSNR.pdf}
%         \caption{\footnotesize{Unrolling Step
%         }}
%     \end{subfigure}

%     \vspace*{-2mm}
%     \caption{\footnotesize{PSNR results of different smoothing architectures with 
%     different unrolling steps (trained at 8)
%     and measurement sampling rates (trained at 4 \hui{25\%?}). The robust accuracy is evaluated with adversarial perturbation $||\delta||_2 \le 0.002$. %{It's worth noting that the higher the undersampling rate is, the sparser the mask is.}
%     \hui{x-axis needs to be changed.}
%     }}
%     \label{fig: robustness}
%     \vspace*{-2mm}
% \end{figure}

% \begin{figure}[htb]
% % \vspace{-3mm}
% \centering
% \begin{tabular}[b]{ccc}
%         \includegraphics[width=.2\linewidth, trim=70 10 70 10]{Figures/visualization/SMUG_recon.pdf}
%         &
%         \includegraphics[width=.2\linewidth, trim=70 10 70 10]{Figures/visualization/SMUG_undersampling_2.0_vis.pdf}
%         &
%         \includegraphics[width=.2\linewidth, trim=70 10 70 10]{Figures/visualization/SMUG_unrolling_steps_16_clean_vis.pdf}
%         \\[-0pt]
%         \scriptsize{(a)} 
%         & 
%         \scriptsize{(b)} 
%         &  
%         \scriptsize{(c)}

% \end{tabular}
% \vspace*{-2mm}
% \caption{\footnotesize{Visualization of reconstruction results of {\us} on different undersampling rates and different unrolling steps. (a) the reconstruction image from the clean image, (b) the reconstruction image when evaluated using a mask with the undersampling rate being 2 (4 in training), (c) the reconstruction image when evaluated using 16 unrolling steps (8 in training).}}
% \label{fig: smug weakness}
% \end{figure}

% % \subsection{Ablation Study}

% % \begin{table}[htb]
% % \centering
% % % \vspace*{-1em}
% % \caption{Clean accuracy and robust accuracy of {\us} with different $\sigma$. The robust accuracy is tested with adversarial perturbation $||\delta||_2 \le 0.5/255$.
% % % The best performance is highlighted in \textbf{bold}.
% % }
% % \label{tab: ablation_sigma}
% % \resizebox{0.48\textwidth}{!}{%
% % \begin{tabular}{c|ccc|ccc}
% % \toprule[1pt]
% % \midrule
% % $\sigma$ & \multicolumn{3}{c}{Clean Accuracy} & \multicolumn{3}{c}{Robust Accuracy} \\
% % Metrics & RMSE $\downarrow$ & PSNR $\uparrow$ & SSIM $\uparrow$ & RMSE $\downarrow$ & PSNR $\uparrow$ & SSIM $\uparrow$ \\
% % \midrule
% % 0.001 
% % & 0.0274 
% % & 29.5286 
% % & 0.8943 
% % & 0.0302 
% % & 28.464 
% % & 0.7916
% % \\
% % 0.005 & 0.0282 & 29.2658 & 0.8929 & 0.0311 & 28.3198 & 0.8594
% % \\
% % 0.01 & 0.0278 & 29.3913 & 0.8939 & 0.0308 & 28.4137 & 0.8467
% % \\
% % 0.05 & 0.0325 & 28.0298 & 0.874 & 0.0333 & 27.7624 & 0.8499
% % \\
% % 0.1 & 0.0322 & 28.0921 & 0.8764 & 0.0325 & 28.0224 & 0.8746
% % \\
% % \midrule
% % \bottomrule[1pt]
% % \end{tabular}%
% % }
% % \vspace*{-3mm}
% % \end{table}

% % \begin{table}[htb]
% % \centering
% % % \vspace*{-1em}
% % \caption{Clean accuracy and robust accuracy of {\us} with different Monte Carlo sampling number $n$. The robust accuracy is tested with adversarial perturbation $||\delta||_2 \le 0.5/255$.
% % % The best performance is highlighted in \textbf{bold}.
% % }
% % \label{tab: ablation_n}
% % \resizebox{0.48\textwidth}{!}{%
% % \begin{tabular}{c|ccc|ccc}
% % \toprule[1pt]
% % \midrule
% % $n$ & \multicolumn{3}{c}{Clean Accuracy} & \multicolumn{3}{c}{Robust Accuracy} \\
% % Metrics & RMSE $\downarrow$ & PSNR $\uparrow$ & SSIM $\uparrow$ & RMSE $\downarrow$ & PSNR $\uparrow$ & SSIM $\uparrow$ \\
% % \midrule
% % 3 & 0.0306 & 28.5343 & 0.8815 & 0.0324 & 27.9376 & 0.8235
% % \\
% % 5 & 0.0292 & 28.9498 & 0.8886 & 0.0315 & 28.1755 & 0.8301
% % \\
% % 10 & 0.0278 & 29.3913 & 0.8939 & 0.0308 & 28.4137 & 0.8467
% % \\
% % \midrule
% % \bottomrule[1pt]
% % \end{tabular}%
% % }
% % \vspace*{-3mm}
% % \end{table}

% % Below is an example of how to insert images. Delete the ``\vspace'' line,
% % uncomment the preceding line ``\centerline...'' and replace ``imageX.ps''
% % with a suitable PostScript file name.
% % -------------------------------------------------------------------------


% % To start a new column (but not a new page) and help balance the last-page
% % column length use \vfill\pagebreak.
% % -------------------------------------------------------------------------
% %\vfill
% %\pagebreak
% \vspace*{-4mm}
\section{Conclusion}
%\vspace*{-2mm}
In this work, we proposed a scheme for improving robustness of DL-based MRI reconstruction. We showed deep unrolled reconstruction's ({\modl}'s) weaknesses in robustness against adversarial perturbations, sampling rates, and unrolling steps. To improve the robustness of {\modl}, we proposed {\us} with a novel unrolled smoothing loss.
 Compared to the vanilla
{\modl} approach and several variants of {\us}, we empirically showed that
%Our experiments demonstrated the effectiveness of
our approach is effective and can   significantly improve the robustness of {\modl} against a diverse set of external perturbations. In the future, we will study the problem of certified robustness  and derive the certification bound of adversarial perturbations using  randomized smoothing.

%all three types of robustness.

% List and number all bibliographical references at the end of the
% paper. The references can be numbered in alphabetic order or in
% order of appearance in the document. When referring to them in
% the text, type the corresponding reference number in square
% brackets as shown at the end of this sentence \cite{C2}. An
% additional final page (the fifth page, in most cases) is
% allowed, but must contain only references to the prior
% literature.

% References should be produced using the bibtex program from suitable
% BiBTeX files (here: strings, refs, manuals). The IEEEbib.bst bibliography
% style file from IEEE produces unsorted bibliography list.
% -------------------------------------------------------------------------
\clearpage
\newpage 
\bibliographystyle{IEEEbib}
\bibliography{reference,refs_adv}

\end{document}
