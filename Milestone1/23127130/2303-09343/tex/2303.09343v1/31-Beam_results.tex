To assess the validity of our method, we first consider a toy problem involving a
square section beam with 304 hexahedal elements.
The network is trained using 20,000 pairs $(\gv,\uv_\gv)$, computed using
a Neo-Hookean material law with a Young modulus $E=4,500\ \mathrm{Pa}$
and a Poisson ratio $\nu = 0.49$.

We create 10,000 additional synthetic deformations of the beam, distinct from the
training dataset, using the SOFA finite element framework \cite{faure:hal-00681539}.
\Cref{fig:beams_examples} shows three examples of synthetic deformations, along with
the sampled point clouds.
Generated deformations include bending (\Cref{fig:beam-bending}), torsion (\Cref{fig:beam-torsion})
or a combination of them (\Cref{fig:beam-both}).
For each deformation, we sample the deformed surface to create a point cloud.
We then apply our algorithm with a relative tolerance of $10^{-4}$ on the objective gradient norm. We computed some statistics regarding the performance of our method over a series of 10,000 different scenarios and obtained the following results: mean registration error: $6 \times 10^{-5} \pm 6.15 \times 10^{-5}$, mean computation time: $48$ ms $\pm 19$ ms and mean number of iterations: 27 $\pm$ 11. 
% \begin{figure}[H]
%     \centering
%     \includegraphics[width=0.5\columnwidth]{images/Transparent_rest_beam.png}
%     \caption{Beam used in this section (blue) attached to the grey wall which represents Dirichlet's boundary conditions}
%     \label{fig:rest_beam}
% \end{figure}
% \vspace{-4.0ex}
\begin{figure}[H]
    \centering
    \begin{subfigure}[b]{0.32\linewidth}
        \centering
        \includegraphics[width=\linewidth]{Beam_b.png}
        \caption{Reg. error: $5.9 \times 10^{-5}$,
        time: 0.07 s, iterations: 13}
        \label[figure]{fig:beam-bending}
    \end{subfigure}
\hfill
    \begin{subfigure}[b]{0.32\linewidth}
        \centering
        \includegraphics[width=\linewidth]{Beam_bt.png}
        \caption{Reg. error: $6.6 \times 10^{-5}$ m, time: 0.09 s, iterations: 15}
        \label[figure]{fig:beam-both}
    \end{subfigure}
\hfill
    \begin{subfigure}[b]{0.32\linewidth}
        \centering
        \includegraphics[width=\linewidth]{Beam_t.png}
        \caption{Reg. error: $3.4 \times 10^{-5}$,
        time: 0.115 s, iterations: 19}
        \label[figure]{fig:beam-torsion}
    \end{subfigure}
\caption{Deformations from the test dataset. 
         The red dots represent the target point clouds, and the color map represents the Von Mises stress error of the neural network prediction.}
\label[figure]{fig:beams_examples}
\end{figure}

% \begin{table}[H]
%     \centering
%     \begin{tabular}{c|c|c|c|c}
%         Measurement & Mean & Std. & Minimum & Maximum \\
%         \hline
%         \hline
%         Registration & $5.99 \times 10^{-5}$ & $6.15 \times 10^{-5}$ & $1.30 \times 10^{-7}$ & $6.20 \times 10^{-4}$ \\
%         \hline
%         Computation time & 48 ms & 19 ms & 2 ms & 210 ms\\
%         \hline
%         Number of iterations & $27$ & $11$ & $0$ & $122$
%     \end{tabular}
%     \caption{This table gathers statistics for 10,000 test cases and presents registration errors, number of iterations and computation times (in ms).}
%     \label{tab:beam-stats}
% \end{table}
Using a FEM solver, each sample of the test dataset took between 1 and 2 seconds to compute. This is mostly due to the complexity of the deformations as shown in \Cref{fig:beams_examples}. Such displacement fields require numerous costly Newton-Raphson iterations to reach equilibrium.
The neural network provides physical deformations in less than a millisecond regardless of the complexity of the force or resulting deformation, which highly improves the computation time of the method. From our analysis, the time repartition of the different tasks in the algorithm is pretty consistent, even with denser meshes. Network predictions and loss function evaluations represent $10\%$ to $15\%$ each, gradient computations represent up to the last $80\%$ of the whole optimization process.
This allows us to reach an average registration error of $5.37 \times 10^{-5}$ in less time than it takes to compute a single simulation of the problem using a classic FEM solver.
% Such error represents an excellent fit of the mesh in the point cloud as shown in \Cref{fig:beams_examples}.

Due to the beam shape symmetry, some point clouds may be compatible with several
deformed configurations, resulting in wrong displacement fields returned by the procedure.
However, our procedure achieved a satisfying surface matching in each case.
These results on a toy scenario prove that our algorithm provides fast and accurate registrations.

In the next section, we apply our method in the field of augmented surgery with the partial surface registration of a liver and show that with no additional computation our approach produces with satisfying accuracy the forces that generate such displacements.
