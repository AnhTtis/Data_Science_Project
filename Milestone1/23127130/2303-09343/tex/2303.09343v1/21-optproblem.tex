To model the registration problem, we use the optimal control formulation introduced  in \citet{mestdagh2022}.
The deformable object is represented by a tetrahedral mesh, endowed with a hyperelastic model.
In its reference configuration, the elastic object occupies the domain $\Omega_0$,
whose boundary is $\dO_0$.
When a displacement field $\uv$ is applied to $\Omega_0$, the deformed domain is
denoted by $\Omega_\uv$, and its boundary is denoted by $\partial \Omega_\uv$ as shown in \Cref{wrap-fig:toy_demo}.
Applying a surface force distribution $\gv$ onto the object boundary results in
the elastic displacement $\uv_\gv$, solution to the static equilibrium equation
\begin{equation}\label{eq:hyperelastic}
    \Fv(\uv_\gv) = \gv,
\end{equation}

where $\Fv$ is the residual from the hyperelastic model.
Displacements are discretized using continuous piecewise linear finite element functions
so that the system state is fully known through the displacement of mesh nodes, stored in~$\uv$.
Note that $\gv$ contains the nodal forces that apply on the mesh vertices.
As we only consider surface loadings, nodal forces are zero for nodes
inside the domain.
Finally, the observed data are represented by a point cloud $\Gamma = \{y_1, \dots, y_m\}$.
\begin{figure}[ht]
\centering
\includegraphics[width=0.6\linewidth]{gui_full.png}
\caption{Schematic of the problem which we are trying to optimize for.}\label{wrap-fig:toy_demo}
\vspace{-5ex}
\end{figure} 
We compute a nodal force distribution that achieves the matching between $\dO_{\uv_\gv}$ and $\Gamma$ by solving the optimization problem
\begin{align}\label{eq:opt-problem}
    &\min_{\gv\in G}\quad \Phi(\gv) + \tfrac{\alpha}{2}\|\gv\|^2\\
    &\qquad\text{where}\qquad \Phi(\gv) = J(\uv_\gv),
\end{align}
where, $\alpha > 0$ is a regularization parameter, $G$ denotes the set of admissible nodal forces distributions, and $J$ is the least-square term
\begin{equation}\label{eq:functional}
    J(\uv) = \tfrac{1}{2m} \sum_{j=1}^m d^2(y_j, \dO_\uv).
\end{equation}

Here, $d(y, \dO_\uv) = \min_{x\in \partial \Omega_\uv} \|y - x\|$ denotes the distance
between $y\in\Gamma$ and $\dO_\uv$.
The functional $J$ measures the discrepancy between $\dO_\uv$ and $\Gamma$, and it
evaluates to zero whenever every point $y\in\Gamma$ is matched by $\partial\Omega_\uv$.

A wide range of displacement fields $\uv$ are minimizers of problem \eqref{eq:opt-problem},
but most of them have no physical meaning.
Defining a set of admissible controls $G$ is critical to generate only displacements that
are consistent with a certain physical scenario.
The set $B$ decides, among others, on which vertices nodal forces may apply,
but also which magnitude they are allowed to take.
Selecting zones where surface forces apply is useful to obtain
physically plausible solutions.

