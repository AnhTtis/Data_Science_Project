We now turn to another test case involving a more complex domain.
The setting is similar to \cite[Sect. 3.2]{mestdagh2022}.
In this context, a patient-specific liver mesh is generated from tomographic images and the objective is to provide augmented reality by registering, in real-time, the mesh to the deformed organ. During the surgery, only a partial point cloud of the visible liver surface can be obtained. The contact zones with the surgical instruments can also be estimated.
\begin{figure}
    \centering
    \includegraphics[width=0.75\linewidth]{Liver.png}
    \caption{Mesh of the liver used in this section. Composed of 3,046 vertices and 10,703 tetrahedral elements which represents a challenge compared to the one used in \Cref{subsec:31}}
    \label{fig:Liver_at_rest}
\end{figure}
In our case, the liver mesh contains 3,046 vertices and 10,703 tetrahedral elements. Homogeneous Dirichlet conditions are applied at zones where ligaments hold the liver, and at the hepatic vein entry.
Like previously, we use a Neo-Hookean constitutive law with $E=4,500\ \mathrm{Pa}$ and $\nu = 0.49$, and the network is trained on 20,000 force/displacement pairs.
We create 5 series of synthetic deformations by applying a variable local force, distributed on a few nodes, on the liver mesh boundary.
For each series, 50 incremental displacements are generated, along with the corresponding point clouds.
The network-based registration algorithm is used to update the displacement field and forces between two frames.
We also run a standard adjoint method involving the Newton algorithm, to compare with our approach.
As the same mesh is used for data generation and reconstruction, the Newton-based reconstruction is expected to perform well.
\subsection{Liver partial surface matching for augmented surgery}

In this subsection, we present two relevant metrics: target registration error and computation times.
In augmented surgery, applications such as robot-aided surgery or holographic lenses require accurate calibrations that rely on registration. One of the most common metrics in registration tasks is the target registration error (TRE), which is the distance between corresponding markers not used in the registration process.
In our case we work on the synthetic deformation of a liver, thus, the markers will be the nodes of the deformed mesh. 
\begin{figure}[h]
    \centering
    \begin{subfigure}[b]{0.48\linewidth}
        \centering
        \includegraphics[width=\linewidth]{average_TRE.pdf}
         \label[figure]{fig:liver-tre}
    \end{subfigure}
    \begin{subfigure}[b]{0.48\linewidth}
        \centering
        \includegraphics[width=\linewidth]{times.pdf}
         \label[figure]{fig:liver-times}
    \end{subfigure}
\caption{Average target registration error and computation times of each sequence.}
\label[figure]{fig:tre_times}
\end{figure}
The 5 scenarios present similar results with TRE between $3.5 \ mm$ and $0.5 \ mm$. Such errors are entirely acceptable and preserve the physical properties of the registered mesh. We point out that the average TRE for the classic method is around $0.1 \ mm$ which shows the impact of the network approximations.

Due to the non-linearity introduced by the Neo-Hookean material used to simulate the liver we need multiple iterations to converge toward the target point cloud. Considering the complexity of the mesh, computing a single iteration of the algorithm using a classical solver takes multiple seconds which leads to an average of 14 minutes per frame.
Our proposed algorithm uses a neural network to improve the computation speed of both the hyper-elastic and adjoint problems. The hyper-elastic problem takes around 4 to 5 milliseconds to compute while the adjoint problem takes around $11 \ ms$. This leads to great improvement in convergence speed as seen in \Cref{fig:liver-times} where on average we reduce the computation time by a factor of 6000.

\subsection{Force estimation for robotic surgery}
In the context of liver computer-assisted surgery, the objective is to estimate a force distribution supported by a small zone on the liver boundary.
Such a local force is for instance applied when a robotic instrument manipulates the organ.
In this case, it is critical to estimate the net force magnitude applied by the instrument, to avoid damaging the liver.
\begin{figure}[ht]
    \centering
    % \begin{subfigure}[b]{\linewidth}
    %     \centering
    %     \includegraphics[width=\linewidth]{images/distributions-00.png}
    %     \caption{Frame 1}
    % \end{subfigure}
    \begin{subfigure}[b]{\linewidth}
        \centering
        \includegraphics[width=\linewidth]{distributions-25.png}
        \caption{Frame 26}
    \end{subfigure}
    \begin{subfigure}[b]{\linewidth}
        \centering
        \includegraphics[width=\linewidth]{distributions-49.png}
        \caption{Frame 50}
    \end{subfigure}
\caption{Synthetic liver deformations and force distributions (left), reconstructed deformations
         and forces using the Newton method (middle) and the network (right) for test case 3.}
\label[figure]{fig:liver-compare}
\end{figure}
To represent the uncertainty about the position of the instruments the reconstructed forces are allowed to be nonzero on a larger support than the original distribution.
\Cref{fig:liver-compare} shows the reference and reconstructed deformations and nodal forces for three frames of the same series.
While the Newton-based reconstruction looks similar to the reference one, network-based
nodal forces are much noisier.
This is mostly due to the network providing only an approximation of the hyperelastic model.
\begin{wrapfigure}{r}{6.0cm}
    \centering
        \centering
        \includegraphics[width=\linewidth]{errs.pdf}
        \caption{Force estimation error of the 5 sequences using our method, in red the average force reconstruction error with the classical method.}
    \label{fig:force_erros_seq}
    \vspace{-4ex}
\end{wrapfigure}
The great improvement in speed comes at the cost of precision. As shown in \Cref{fig:liver-compare} the neural network provides noisy force reconstructions. This is mostly due to prediction errors since the ANN only approximates solutions. These errors also propagate through the backward pass (adjoint problem), thus, accumulate in the final solution.
Although the force estimation is noisy for most cases it remains acceptable as displayed in \Cref{fig:force_erros_seq}. The red dotted line corresponds to the average error obtained with the classical adjoint method (10.04 \%). While we are not reaching such value, some sequences such as 1 and 3 provide good reconstructions. The difference in errors between scenarios is mostly due to training force distribution. This problem can be corrected by simply adding more data to the dataset thus providing better coverage of the force and deformation space.

These results show that this algorithm can produce fast and accurate registration at the expense of force reconstruction accuracy. This also shows that the force estimation is not directly correlated to registration accuracy. For example sequence 1 has the worst TRE but a good force reconstruction compared to sequence 4.

% \Cref{fig:liver-compare} shows the reference and reconstructed deformations and nodal forces
% for three frames of the same series.
% While the Newton-based reconstruction looks similar to the reference one, network-based
% nodal forces are much noisier.
% This is mostly due to the network providing only an approximation of the hyperelastic model.
% \begin{figure}[ht]
%     \centering
%     \begin{subfigure}[b]{\linewidth}
%         \centering
%         \includegraphics[width=\linewidth]{images/distributions-00.png}
%         \caption{Frame 1}
%     \end{subfigure}
%     \begin{subfigure}[b]{\linewidth}
%         \centering
%         \includegraphics[width=\linewidth]{images/distributions-25.png}    
%         \caption{Frame 26}
%     \end{subfigure}
%     \begin{subfigure}[b]{\linewidth}
%         \centering
%         \includegraphics[width=\linewidth]{images/distributions-49.png}
%         \caption{Frame 50}
%     \end{subfigure}
% \caption{Synthetic liver deformations and force distributions (left), reconstructed deformations
%          and forces using the Newton method (middle) and the network (right) for test case 3.}
% \label[figure]{fig:liver-compare}
% \end{figure}

% We will now discuss three relevant metrics: target registration error, computation times, and finally the force estimation error.
% In augmented surgery applications such as robot-aided surgery or holographic lenses require precise calibrations that rely on registration. One of the most common metrics in registration tasks is the target registration error (TRE), which is the distance between corresponding markers not used in the registration process.
% In our case we work on the synthetic deformation of a liver, thus, the markers will be the nodes of the deformed mesh. 
% The TRE over the 5 sequences is presented in \Cref{fig:TRE_seq}. 
% \begin{figure}[ht]
%     \centering
%     \includegraphics[width=0.55\linewidth]{images/average_TRE.pdf}
%     \caption{Average TRE in mm of the 5 sequences}
%     \label{fig:TRE_seq}
% \end{figure}

% The 5 scenarios present similar results with TRE between $3.5 \ mm$ and $0.5 \ mm$. Such errors are entirely acceptable and preserve the physical properties of the registered mesh. We point out that the average TRE for the classic method is around $0.1 \ mm$ which shows the impact of the network approximations.

% Due to the non-linearity introduced by the Neo-Hookean material used to simulate the liver we need multiple iterations to converge toward the target point cloud. Considering the complexity of the mesh, computing a single iteration of the algorithm using a classical solver takes multiple seconds which leads to an average of 14 minutes per registration.
% \begin{figure}[ht]
%     \centering
%         \centering
%         \includegraphics[width=0.55\linewidth]{images/times.pdf}
%         \caption{Computation times in ms of each sequence using our method. }
%         \label[figure]{fig:time_NN_seq}
% \end{figure}
% Our proposed algorithm uses a neural network to improve the computation speed of both the hyper-elastic and adjoint problems. The hyper-elastic problem takes around 4 to 5 milliseconds to compute while the adjoint problem takes around $11 \ ms$. This leads to great improvement in convergence speed as seen in \Cref{fig:time_NN_seq}. In total, we reduce the computation time by a factor of at least 2500. 


% The great improvement in speed comes at the cost of precision. As shown in \Cref{fig:liver-compare} the neural network provides noisy force reconstructions. This is mostly due to prediction errors since the ANN only approximates solutions. These errors also propagate through the backward pass (adjoint problem), thus, accumulate in the final solution.
% Although the force estimation is noisy for most cases it remains acceptable as displayed in \Cref{fig:force_erros_seq}. The red dotted line corresponds to the average error obtained with the classical adjoint method (10.04 \%). While we are not reaching such value some sequences such as 1 and 3 provide good reconstruction. 
% \begin{figure}[H]
%     \centering
%         \centering
%         \includegraphics[width=0.55\linewidth]{images/errs.pdf}
%         \caption{Force estimation error of the 5 sequences using our method, in red the average force reconstruction error with the classical method}
%     \label{fig:force_erros_seq}
% \end{figure}
% These results show that this algorithm can produce fast and accurate registration at the cost of a loss of force reconstruction accuracy. This also shows that the force estimation is not directly correlated to registration accuracy. For example sequence 1 has the worst TRE but a good force reconstruction compared to sequence 4.






