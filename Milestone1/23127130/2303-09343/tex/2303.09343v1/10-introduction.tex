
We consider an elastic shape-matching problem between a deformable solid
and a point cloud.
Namely, an elastic solid in its reference configuration is represented by a
tridimensional mesh, while the point cloud represents a part of the solid boundary
in a deformed configuration.
The objective of the procedure is not only to deform the mesh so that
its boundary matches the point cloud, but also to estimate the displacement
field inside the object.

This situation also arises in computer-assisted liver surgery, where augmented reality
is used to help the medical staff navigate the operation scene \citep{haouchine2015}.
% Minimally-invasive interventions are challenging for surgeons, as they manipulate organs
% through the patient's skin using instruments inserted in small incisions, while visual
% feedback is provided thanks to a laparoscopic camera inserted through one of the incisions.
% The augmented reality system provides a virtual view of the patient, along with its
% internal structures, including blood vessels and tumors, to guide the surgeon's gesture.
% The liver deformation is updated in real-time using an elastic shape-matching procedure.
% In this case, a liver mesh is extracted from pre-operative tomographic
% images, while the point cloud is obtained by processing images from the laparoscopic camera.
Most methods for intra-operative organ shape-matching revolve around a biomechanical model to
describe how the liver is deformed when forces are applied to its boundary.
Sometimes, a deformation is created by applying forces \citep{plantefeve2016}
or constraints \citep{peterlik2018,malti2015linear} to enforce surface correspondence.
Other approaches prefer to solve an inverse problem, where the final displacement
minimizes a cost functional among a range of admissible displacements
\citep{heiselman2020}.
However, while living tissues are known to exhibit a highly nonlinear behavior
\citep{marchesseau2017}, using hyperelastic models in the context of real-time
shape matching is prohibited due to high computational costs.
For this reason, the aforementioned methods either fall back to linear elasticity
\citep{heiselman2020} or to the linear co-rotational model \citep{plantefeve2016}.
In this paper, we perform real-time hyperelastic shape matching by predicting nonlinear
displacement fields using a neural network.
The network is included in an adjoint-like method, where the backward chain is
executed automatically using automatic differentiation.

Neural networks are used to predict solutions to partial differential equations,
in compressible aerodynamics~\citep{renganathan2021}, structural
optimization~\citep{white2019} or astrophysics~\citep{khan2021}.
Here we work at a small scale, but try to obtain real-time simulations using complex models.
Also, the medical image processing literature is full of networks that perform shape-matching
in one step \citep{pfeiffer2020}.
However, the range of available displacement fields is limited by the training dataset of the network, and thus less robust to unexpected deformations.
On the other hand, assigning a very generic task to the network results in a very flexible method, where details of the physical model, including the range of forces that can be applied to the liver and the zones where they apply may be chosen after the training.
Therefore, our shape-matching approach provides a good compromise between the speed of learning-based methods with the flexibility of standard simulations.
We want to mention that for the rest of this article due to how the method is formulated we interchangeably use the terms "shape-matching" and "registration".

We start by presenting the method split into three parts. First, the optimization problem; second, the used neural network and finally, the adjoint method computed using an automatic differentiation framework.

We then present the results considering a toy problem involving a square section beam and a more realistic one involving a liver.