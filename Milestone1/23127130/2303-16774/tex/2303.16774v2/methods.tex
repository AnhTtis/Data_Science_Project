% Materials and methods
\section{Methods}\label{sec:methods}

\subsection{Structural Balance Theory}\label{sec:methods:sbt}

Here we review the main assumptions and results of SBT.

Let $iRj$ denote a relation between entities $i$ and $j$ and $i\neg{}Rj$ its opposite. 
Positive and negative relations will be denoted by $iLj$ and $i\neg{}Lj$ respectively. 
Now, the assumption of SBT is that relations should be symmetric ($iRj \land jR'i \Rightarrow R=R'$),
and that positively linked nodes should agree with respect to  valences of their other ties 
($iLj \land iRk \land jR'k \Rightarrow R = R'$), while negatively linked should disagree 
($i\neg{}Lj \land iRk \land jR'k \Rightarrow R = \neg{}R'$). Moreover, this consistency property 
should extend to indirect preferences implied by (anti)transitivity over longer chains
(e.g.~$iLj \land j\neg{}Lk \land kRm \land iR'm \Rightarrow R'=\neg{}R$). 
It is also assumed that such balanced relations are easier/cheaper to maintain and therefore 
should be more stable in time and generally more frequent than unbalanced
ties~\cite{heiderAttitudesCognitiveOrganization1946,cartwrightStructuralBalanceGeneralization1956}.

The core results of SBT, due to~\citet{cartwrightStructuralBalanceGeneralization1956},
translate the above requirements to specific properties of signed graphs allowing
for measurement and practical applications of the notion of structural balance.
Below we rehash these results.

\begin{definition}[Strong balance property]\label{def:strong-balance}
    A signed graph is balanced if and only if every semicycle it contains is positive
    (the product over all edge signs is positive).
\end{definition}

\begin{theorem}[Strong structure theorem]\label{thm:strong-structure-I}
    A signed graph is balanced if and only if its vertices can be partitioned
    into two subsets such that positive edges are contained within the subsets and negative
    ones go between them.
\end{theorem}

The above results concerned with $2$-clusterability 
were later generalized  by~\citet{davisClusteringStructuralBalance1967}, who provided necessary and
sufficient conditions for $k$-clusterability (where $k \geq 2$ is an unknown integer).

\begin{definition}[Weak balance property]\label{def:weak-balance}
    A signed graph is $k$-balanced if and only if no semicycle contains exactly one negative edge.
\end{definition}

\begin{theorem}[Weak structure theorem]\label{thm:weak-structure-I}
    A signed graph is $k$-balanced if and only if its vertices can be partitioned into $k$
    subsets such that positive edges are contained within the subsets and negative ones go between them.
\end{theorem}


\subsection{Analytic functions of real symmetric matrices}\label{sec:methods:analytic}

Let $\mathbf{X} \in \mathbb{R}^{n \times n}$ be a real symmetric matrix and 
$f: \mathbb{R}^{n \times n} \to \mathbb{R}^{n \times n}$ an analytic function defined over
the field of real square matrices. 
Then, $f\left(\mathbf{X}\right) = \mathbf{Q}f\left(\mathbf{\Lambda}\right)\mathbf{Q}^\top$,
where $\mathbf{\Lambda}$ is a real diagonal matrix with eigenvalues of $\mathbf{X}$ 
($\lambda_1 \geq \lambda_2 \geq \ldots \geq \lambda_n$)
and the columns of $\mathbf{Q}$ are the corresponding eigenvectors. 
This implies that:
\begin{align}
    f\left(\mathbf{X}\right)_{ij} &= \sum_{r=1}^n \mathbf{Q}_{ir}f(\lambda_r)\mathbf{Q}_{jr}
    \label{eq:fXij}\\
    \tr{}f\left(\mathbf{X}\right) &= \sum_{r=1}^n f(\lambda_r)
    \label{eq:trfX}
\end{align}

In particular, we will be interested in matrix exponentials and finite truncations of their
power series:
\begin{equation}\label{eq:texp}
    \exp\kk(\beta\mathbf{X})
    = \mathbf{Q}\left(\sum_{k=k_0}^{k_1}\frac{\beta^k}{k!}\mathbf{\Lambda}^k\right)\mathbf{Q}^\top
    \xrightarrow[k_0 \to 0]{k_1 \to \infty} \exp(\beta\mathbf{X})
\end{equation}
where $\beta$ is a free parameter that can be used for controlling the rate of growth/decay
of the weights attached to different powers of $\mathbf{X}$, and for $\beta = 1$ the standard 
exponential is recovered.


\subsection{Weak balance}\label{sec:methods:weak-balance}

Following Ref.~\cite{kirkleyBalanceSignedNetworks2019} we define non-negative matrices $\mathbf{P}$
and $\mathbf{N}$ corresponding to positive and negative parts of a signed semiadjacency matrix such that:
\begin{equation}\label{eq:pos-neg-matrices}
    \mathbf{S} = \mathbf{P} - \mathbf{N}
\end{equation}

Weak balance is defined in terms of the extent to which a network is free of semicycles with exactly
one negative edge. This single negative link can be placed anywhere along a semipath starting at node $i$.
Hence, we first define a matrix counting weakly unbalanced semiwalks of length $k$ between nodes $i$ and $j$
in a signed graph $G$ as:
\begin{equation}\label{eq:Vk}
\begin{split}
    \mathbf{V}_k(G) 
    &= \sum_{l=1}^k\mathbf{P}^{l-1}\mathbf{NP}^{k-l} \\
    &= \sum_{l=1}^k\mathbf{Q\Lambda}^{l-1}\mathbf{Q}^\top\mathbf{NQ\Lambda}^{k-l}\mathbf{Q}^\top \\
    &= \mathbf{Q}\left[    
        \left(\sum_{l=1}^k\mathbf{L}(k,l)\right) \otimes \mathbf{M}
    \right]\mathbf{Q}^\top
\end{split}
\end{equation}
where $\mathbf{Q\Lambda{}Q}^\top$ is the eigendecomposition of $\mathbf{P}$,
$\mathbf{M} = \mathbf{Q}^\top\mathbf{NQ}$ and $\mathbf{L}(k,l)_{ij} = \lambda_i^{l-1}\lambda_j^{k-l}$.
Moreover, we used the fact that 
$\mathbf{\Lambda}^{l-1}\mathbf{M\Lambda}^{k-l} = \mathbf{L}(k,l) \otimes \mathbf{M}$.

Now, a matrix with weighted sums of counts of semiwalks
of lengths $k_0, \ldots, k_1$ joining nodes $i$ and $j$ is given by:
\begin{align}
    \mathbf{V}\kk(G,\beta) 
    &= \sum_{k=k_0}^{k_1}\frac{\beta^k}{k!}\mathbf{V}_k(G)
     = \mathbf{Q}\mathbf{U}\kk(G,\beta)\mathbf{Q}^\top
    \label{eq:V} \\ 
    \mathbf{U}\kk(G,\beta)
    &= \left[
        \sum_{k=k_0}^{k_1}\frac{\beta^k}{k!}\sum_{l=1}^k\mathbf{L}(k,l)
    \right] \otimes \mathbf{M}
    \label{eq:U}
\end{align}
Clearly, once $\mathbf{M}$ is known, $\mathbf{U}\kk(G, \beta)$ can be
computed roughly in $\mathcal{O}(k_1^2m^2)$ time, where $m$ is the number of eigenvalues.

Next, we can use Eq.~\eqref{eq:V} to calculate the overall weighted sums of counts of unbalanced 
closed semiwalks from appropriate traces:
\begin{align}
    \tr\mathbf{V}\kk(G,\beta)
    &= \sum_{k=k_0}^{k_1}\frac{\beta^k}{k!}\tr\mathbf{V}_k(G)
    \label{eq:weakly-unbalanced-walks} \\
    \tr\mathbf{V}_k(G)
    &= k\sum_{i=1}^m\lambda_i^{k-1}\mathbf{M}_{ii}
    \label{eq:weakly-unbalanced-walks-k}
\end{align}
where we used the fact that trace is invariant under cyclic permutations and $\mathbf{Q}$ is orthonormal.
The weighted sum of counts of closed semiwalks at a node $i$ is similarly given by the diagonal elements
$\mathbf{V}\kk(G,\beta)_{ii}$.

Now, Eqs.~\eqref{eq:W} and \eqref{eq:weakly-unbalanced-walks}
can be used to define the measure of the overall weak balance:
\begin{equation}\label{eq:dob-weak}
    B_W(G, \beta) 
    = 1 - \frac{\mu_W}{\mu_B + \mu_U}
    = 1 - \frac{\tr\mathbf{V}\kk(G,\beta)}{\tr\mathbf{W}(|G|,\beta)}
\end{equation}
with the local node-level DoB following the same pattern, but using node-specific counts of closed
semiwalks. Moreover, weak DoB at a particular length $k$ is given by:
\begin{equation}\label{eq:weak-dob-k}
    B_W(G,k) 
    = 1 - \frac{\tr\mathbf{V}_k(G)}{\tr|\mathbf{S}|^k}
\end{equation}

Similarly, following the arguments presented in Sec.~\ref{sec:results:pairwise}, 
weak pairwise DoB is defined as:
\begin{equation}\label{eq:weak-dob-pairwise}
    B_W(G,\beta, i, j) = 1 - \frac{\mathbf{V}\kk(G,\beta)_{ij}}{\mathbf{W}(|G|,\beta)_{ij}}
\end{equation}

This weak notion of pairwise balance is based on the fact that two nodes
are jointly weakly balanced if and only if it holds that if they are connected by a path with only 
positive edges, then there is no path with exactly one negative edge connecting them.
This leads to an analogous alternative weak structure theorem, which justifies
the notion of weak pairwise DoB (see SI, Sec.~S2, for the proof).

\begin{theorem}[Weak structure theorem (alternative)]\label{thm:weak-structure-II}
    A signed graph is weakly balanced if and only if for any pair of nodes it holds that
    if there is an all positive path between them then there are no paths with exactly one
    negative edge joining them.
\end{theorem}

Last but not least, the matrix series defined in Eq. \eqref{eq:V} used for counting 
unbalanced closed (semi)walks always converges, so it is well-defined. Note that:
\begin{equation}\label{eq:weakly-unbalanced-walks-convergence}
    0 \leq
    \sum_{k=k_0}^{k_1}\frac{\beta^k}{k!}\sum_{l=1}^k
    \tr\mathbf{P}^{l-1}\mathbf{N}\mathbf{P}^{k-l}
    \leq
    \sum_{k=0}^\infty\frac{\beta^k}{k!}\tr\left(\mathbf{P} + \mathbf{N}\right)^k
    = \tr\exp{(\beta\mathbf{|S|})}
\end{equation}
where it is known that the rightmost matrix exponential and its trace always converge, 
so the middle part of the inequality must converge too.

\subsection{Frustration index}\label{sec:methods:findex}

Very conveniently, SBT provides an objective measure of the quality of a network clustering solution.
Since perfect balance implies that negative edges go only between clusters and positive ones are contained
within them, a natural clustering quality measure is Frustration Index, which, 
following Ref.~\cite{doreianPartitioningApproachStructural1996}, 
can be defined in terms of the relative weighted count of edges inconsistent with the SBT hypothesis:
\begin{equation}\label{eq:findex}
    F(G,\mathbf{B}) = \frac{%
        \mathbb{1}^\top\left[
        (\mathbf{BB}^\top) \otimes \mathbf{N} + 
        (\mathbb{1}\mathbb{1}^\top-\mathbf{BB}^\top) \otimes \mathbf{P}
        \right]\mathbb{1}
    }{%
        \mathbb{1}^\top|\mathbf{S}|\mathbb{1}
    } 
\end{equation}
where $\mathbb{1}$ is a vector of ones of an appropriate length and
$\mathbf{B} \in \mathbb{R}^{n \times b}$ is a block-partition matrix such that 
$\mathbf{B}_{ij} = 1$ if node $i$ belongs to the block $j$ and is $0$ otherwise.

\subsection{Hierarchical clustering with pairwise DoB measures}\label{sec:methods:hclust}

Here we will use the following naive yet effective clustering procedure for signed networks
based on pairwise DoB measures (see Secs. \ref{sec:results:pairwise} and \ref{sec:methods:weak-balance}).
Let $\mathbf{D}^S_{ij} = 1 - B_S(G,\beta_{\max}, i, j)$ and $\mathbf{D}^W_{ij} = 1- B_W(G,\beta_{\max}, i, j)$
be pairwise dissimilarity matrices based on the notions of strong and weak balance respectively and
let $N_b$ be the maximum number of clusters one is willing to consider. Then, for $b = 2, \ldots, N_b$:
\begin{enumerate}
    \item Run Hierarchical Clustering (HC)~\cite{hastieElementsStatisticalLearning2008} algorithm
    for $b$ clusters using $\mathbf{D}^S$ as input and calculate frustration index according to 
    Eq.~\eqref{eq:findex} for the obtained block-partition matrix $\mathbf{B}$.
    \item Run HC for $b$ clusters using $\mathbf{D}^W$ as input and calculate the corresponding
    frustration index.
    \item Store the lower of the two frustration indices and its corresponding block partition.
\end{enumerate}
Finally, choose the partition with the lowest frustration index.

\subsection{Accuracy, efficiency and numerical stability}\label{sec:methods:performance}

All computations of SWB can be implemented in a computationally efficient and accurate manner
using approximations based on $m$ leading eigenvalues and eigenvectors from both ends of the spectrum. 
Leading eigenpairs can be found very efficiently using modern linear algebra routines such as implicitly
restarted Arnoldi method~\cite{lehoucqARPACKUsersGuide1998,sorensenDeflationTechniquesImplicitly1996}.
Moreover, numerical stability can be guaranteed by conducting all computations in the log-space 
and using log-sum-exp trick (to avoid overflow when counting closed walks). 
This requires a bit of extra care as some eigenvalues may be non-positive. 
However, zero eigenvalues can be ignored altogether and the calculations can be done over the field of 
complex numbers, where the logarithm of any number with non-zero modulus is well-defined,
and cast back to real values only at the very end. As a result, SWB methods can be remarkably
efficient, even when applied to large systems. Sec.~S3 in SI presents
an empirical analysis of the efficiency of our implementation using three large real-world networks.

A more in-depth discussion of implementation details is beyond the scope of this paper, but we invite
the interested reader to study our source code (see: Data and code availability).

\subsubsection{Accuracy of SWB approximations}\label{sec:methods:accuracy}

SWB approach is based on three different approximations. First, it approximates semicycles with
closed semiwalks. This is a fundamental design decision ensuring high computational efficiency,
but it comes at a price of introducing an error, of which impact is hard to assess. A proper analysis
of this problem would require a detailed comparison with methods based on counting proper cycles,
such as~\cite{giscardEvaluatingBalanceSocial2017}, which is beyond the scope of this paper.
However, there are good reasons to believe that the error does not
distort the results in a significant way. This is indicated by the high quality and meaningfulness
of the empirical results discussed in Secs. \ref{sec:results:monks} and \ref{sec:results:congress}.
Moreover, by satisfying the Locality Principle,
SWB ensures that DoB measures are driven primarily by patterns found in short closed walks, which
coincide with cycles much more often than long walks.

The second approximation happens when truncating power series to include only the terms of orders
$k_0, \ldots, k_1$. However, this approximation introduces no significant error by design,
as LP ensures that higher order terms have very low and monotonically decreasing contributions to the 
overall DoB calculations. Thus, as long as enough terms are included, and typically about a dozen is enough,
the truncation introduces no noticeable error. In principle, the lowest number of terms necessary
for attaining a given cumulative contribution score can be determined easily by inspecting the contribution
profile. However, here we used a simple rule-of-thumb and in all cases, unless specified otherwise,
used $k_1 = 30$, which is typically more than enough (see Fig.~\ref{app:fig:accuracy}A).

\begin{figure}[htb!]
\centering
\includegraphics[width=\textwidth]{figs/accuracy}
\caption{
    Effects of SWB approximations assessed using bill co-sponsorship network
    from the U.S. Senate during 114th Congress ($|V| = 100$, $|E| = 3696$)
    as well as its randomized counterparts based on Erdős–Rényi model and configuration
    model~\cite{newmanNetworks2018}.
    \textbf{(A)}~Contribution profiles are clealy almost identical for the original and
    randomized networks. Crucially, at $\beta = \beta_{\max}$ almost all the cumulative
    contribution score is driven by leading low order terms, meaning that higher order terms
    (roughly $k > 10$) can be safely omitted.
    \textbf{(B)}~Errors of DoB measures based on leading eigenvalues calculated relative
    to values obtained using full spectra. Errors are typically low even for $m = 1$
    and in all cases quickly decrease as $m$ increases. Moreover, in most of the cases
    they are lower for the real network, which is consistent with the fact that
    errors should be lower for networks with heterogeneous distributions of eigenvalues.
}
\label{app:fig:accuracy}
\end{figure}

The last approximation happens when only $m$ leading eigenpairs from the both ends of the spectrum
are used. This allows for solving the corresponding eigenproblems and running other downstream 
calculations much faster. Moreover, it is clear from Eqs. \eqref{eq:fXij} and \eqref{eq:trfX}
that, as long as $|f(x)|$ is decreasing for $x < 0$ and increasing for $x > 0$,
the true value can be accurately approximated using leading eigenvalues/eigenvectors from both ends
of the spectrum, especially when the distribution of eigenvalues is highly heterogeneous,
that is, the cumulative absolute sum is dominated by just a few leading eigenvalues. 
Fig.~\ref{app:fig:accuracy}B provides an empirical support for this claim.


\subsection{Network datasets}\label{sec:methods:datasets}

\subsubsection{New Guinea Highlands tribes}\label{sec:methods:datasets:tribes}

An undirected unweighted signed network of friendships among tribes of Gahuku-Gama alliance structure 
of the Eastern Central Highlands region in New Guinea~\cite{readCulturesCentralHighlands1954}.
Edge sign indicates either friendship or enmity. 
Accessed from: \url{https://networks.skewed.de/net/new_guinea_tribes}

\subsubsection{Epinions trust network}\label{sec:methods:datasets:epinions}

This is a who-trust-whom online social network (directed, unweighted and signed) 
of a a general consumer review site \texttt{Epinions.com}. 
Members of the site can decide whether to \enquote{trust} each other. All the trust relationships interact
and form the Web of Trust which is then combined with review ratings to determine which reviews are shown
to the user~\cite{richardsonTrustManagementSemantic2003}.
Accessed from: \url{https://snap.stanford.edu/data/soc-Epinions1.html}.

\subsubsection{Wikipedia adminship vote}\label{sec:methods:datasets:wikipedia}

A directed unweighted signed network of votes on Request for Adminship (RfA) elections from a 2008 
snapshot of Wikipedia~\cite{leskovecGovernanceSocialMedia2010}.
Nodes represent editors, and a directed edge $(i,j)$ indicates that editor $i$ voted on editor $j$. 
Edge sign indicates the direction of the vote: positive = for, and negative = against. Edges are timestamped.
Accessed from: \url{https://networks.skewed.de/net/elec}.

\subsubsection{Slashdot Zoo network}\label{sec:methods:datasets:slashdot}

A directed unweighted signed network of interactions among users on Slashdot (\texttt{slashdot.org}), 
a technology news website~\cite{kunegisSlashdotZooMining2009}. 
Users name each other as friends (positive tie) or foe (negative tie). 
The friend label increases the scores of post, and the foe label decreases the score.
Accessed from: \url{https://networks.skewed.de/net/slashdot\_zoo}.

\subsubsection{Sampson's Monastery dataset}\label{sec:methods:datasets:monks}

Time series of 5 signed directed weighted networks measuring positive and negative relations between
postulants and novices in a New England monastery in 1960's~\cite{sampsonNovitiatePeriodChange1968}.
We used a version of the dataset studied in Ref.~\cite{doreianPartitioningApproachStructural1996}
in which edges have weights between -3 and 3 corresponding to the ranking of the least and most
(dis)liked/(dis)esteemed colleagues.
Accessed from: \url{http://vlado.fmf.uni-lj.si/pub/networks/data/esna/sampson.htm}.

\subsubsection{Co-sponsorship relations in the U.S. Congress}\label{sec:methods:datasets:congress}

Series of directed unweighted signed networks inferred from the data on bill co-sponsorships 
in both chambers of the U.S. Congress (House of Representatives and Senate) 
using Stochastic Degree Sequence Model~\cite{nealBackboneBipartiteProjections2014,nealSignTimesWeak2020}.
The data covers the period from 1973 (93rd Congress) to 2016 (114th Congress). Edges are signed, 
indicating the presence of a significant tendency to co-sponsor, or tendency to not co-sponsor, bills.
See SI, Sec.~S4, for a table with descriptive statistics.
Accessed from: \url{https://figshare.com/articles/dataset/A_Sign_of_the_Times/8096429}.