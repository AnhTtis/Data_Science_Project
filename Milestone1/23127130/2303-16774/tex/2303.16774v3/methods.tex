% Materials and methods
\section{Methods}\label{sec:methods}

\rev{
\subsection{Overview of Structural Balance Theory}\label{sec:methods:sbt}

Here we state the main definitions and theorems of SBT concerned with bi-clusterability as formulated 
by Cartwright and Harrary~\cite{cartwrightStructuralBalanceGeneralization1956}.
We use the general formulation based on semipaths and semicycles, so the theorems are
applicable to both undirected and directed graphs. Thus, we first define semipaths and semicycles.

\begin{definition}[Semipath]\label{def:semipath}
    A semipath is a walk in which each (directed) edge can be traversed both ways
    but only once and each node is visited exactly once. 
\end{definition}

\begin{definition}[Semicycle]\label{def:semicycle}
    A semipath starting and ending at the same node
    (which in this case is allowed to appear twice).
\end{definition}

\begin{corollary}
    Notions of paths/cycles and semipaths/semicycles are equivalent in undirected graphs,
    since an undirected edge is treated in this context as two directed edges
    pointing in opposite directions.
\end{corollary}

\begin{definition}[Strong balance property]\label{def:strong-balance}
    A signed graph is balanced if and only if every semicycle it contains is positive
    (the product over all edge signs is positive).
\end{definition}

\begin{theorem}[Strong structure theorem]\label{thm:strong-structure-I}
    A signed graph is balanced if and only if its vertices can be partitioned
    into two subsets such that positive edges connect vertices from the same subset 
    and negative ones link vertices from different subsets.
\end{theorem}

The above results were later generalized  by~\citet{davisClusteringStructuralBalance1967}, 
who provided necessary and sufficient conditions for $b$-clusterability 
(where $b \geq 2$ is an unknown integer).

\begin{definition}[Weak balance property]\label{def:weak-balance}
    A signed graph is weakly balanced if and only if no semicycle contains exactly one negative edge.
\end{definition}

\begin{theorem}[Weak structure theorem]\label{thm:weak-structure-I}
    A signed graph is weakly balanced if and only if its vertices can be partitioned into $b$
    subsets such that positive edges connect vertices from the same subset 
    and negative ones link vertices from different subsets.
\end{theorem}
}

\rev{
\subsection{Weak balance}\label{sec:methods:weak-balance}

Following Ref.~\cite{kirkleyBalanceSignedNetworks2019} we define non-negative matrices 
$\mathbf{P}(\mathbf{A})$ and $\mathbf{N}(\mathbf{A})$ corresponding to positive and negative parts 
of signed adjacency matrix such that 
$\mathbf{A} = \mathbf{P} - \mathbf{N}$ and $|\mathbf{A}| = \mathbf{P} + \mathbf{N}$.
In what follows we will use the simpler notation without the explicit dependence on $\mathbf{A}$,
but it is important to remember that $\mathbf{P}$ and $\mathbf{N}$ are functions of $\mathbf{A}$.

Weak balance is defined in terms of the extent to which a network is free of cycles with exactly
one negative edge. This single negative link can be placed anywhere along a path starting at node $i$.
Hence, we first define a matrix counting weakly unbalanced walks of length $k$ between nodes $i$ and $j$
in a signed graph $G$ as:
\begin{equation}\label{eq:Vk}
\begin{split}
    \mathbf{V}_k(\mathbf{A}) 
    &= \sum_{l=1}^k\mathbf{P}^{l-1}\mathbf{NP}^{k-l} \\
    &= \sum_{l=1}^k\mathbf{Q\Lambda}^{l-1}\mathbf{Q}^\top\mathbf{NQ\Lambda}^{k-l}\mathbf{Q}^\top \\
    &= \mathbf{Q}\left[    
        \left(\sum_{l=1}^k\mathbf{L}(k,l)\right) \odot \mathbf{M}
    \right]\mathbf{Q}^\top
\end{split}
\end{equation}
where $\mathbf{Q\Lambda{}Q}^\top$ is the eigendecomposition of $\mathbf{P}$,
$\mathbf{M} = \mathbf{Q}^\top\mathbf{NQ}$ and $\mathbf{L}(k,l)_{ij} = \lambda_i^{l-1}\lambda_j^{k-l}$.
Moreover, we used the fact that 
$\mathbf{\Lambda}^{l-1}\mathbf{M\Lambda}^{k-l} = \mathbf{L}(k,l) \odot \mathbf{M}$.

Now, a matrix with weighted sums of counts of walks
of lengths $k = k_{\min}, \ldots, k_{\max}$ joining nodes $i$ and $j$ is given by:
\begin{equation}\label{eq:V}
\begin{split}
    \mathbf{V}(\mathbf{A}, \beta) 
    &= \sum_k\frac{\beta^k}{k!}\mathbf{V}_k(\mathbf{A}) \\
    &= \mathbf{Q}\left\{\left[
        \sum_k\frac{\beta^k}{k!}\sum_{l=1}^k\mathbf{L}(k,l)
    \right] \odot \mathbf{M}\right\}\mathbf{Q}^\top
\end{split}
\end{equation}

Next, we can use Eq.~\eqref{eq:V} to calculate the overall weighted sums of counts of unbalanced 
closed walks from appropriate traces:
\begin{align}
    \tr\mathbf{V}(\mathbf{A}, \beta)
    &= \sum_k\frac{\beta^k}{k!}\tr\mathbf{V}_k(\mathbf{A})
    \label{eq:weakly-unbalanced-walks} \\
    \tr\mathbf{V}_k(\mathbf{A})
    &= k\sum_{i=1}^m\lambda_i^{k-1}\mathbf{M}_{ii}
    \label{eq:weakly-unbalanced-walks-k}
\end{align}
where we used the fact that trace is invariant under cyclic permutations and $\mathbf{Q}$ is orthonormal.
The weighted sum of counts of closed walks at a node $i$ is similarly given by the diagonal elements,
$\mathbf{V}(\mathbf{A}, \beta)_{ii}$.

Now, Eqs.~\eqref{eq:W} and \eqref{eq:weakly-unbalanced-walks}
can be used to define the measure of the overall weak balance:
\begin{equation}\label{eq:dob-weak}
    W(\beta) 
    = 1 - \frac{\mu_W}{\mu_+ + \mu_-}
    = 1 - \frac{\tr\mathbf{V}(\mathbf{A}, \beta)}{\tr\mathbf{W}(|\mathbf{A}|,\beta)}
\end{equation}
where $\mu_W$ is the sum of weighted counts of weakly unbalanced closed walks.
Weak pairwise cohesion scores are given by ratios of individual matrix elements:
\begin{equation}\label{eq:weak-cohesion}
    w_{ij}(\beta) 
    = 1 - \frac{\mathbf{V}(\mathbf{A}, \beta; k_{\min} = 2)_{ij}}%
    {\mathbf{W}(|\mathbf{A}|,\beta, k_{\min} = 2)_{ij}}
\end{equation}
with local (node-level) weak DoB given by the diagonal elements, $w_{ii}(\beta; k_{\min} = 3)$.
Similarly, weak $k$-balance is given by considering only closed walks of a particular length $k$:
\begin{equation}\label{eq:weak-dob-k}
    W_k = 1 - \frac{\tr\mathbf{V}_k(\mathbf{A})}{\tr|\mathbf{A}|^k}
\end{equation}
Importantly, as in the case of strong balance, global weak DoB can be expressed as a weighted
average of weak $k$-balance with weights given by the corresponding contribution scores
(see SI, Sec.~\ref{app:sec:weak-dob-average}, for the proof).

Last but not least, the trace of the matrix series defined in Eq. \eqref{eq:V} used for counting 
unbalanced closed walks always converges, so it is well-defined. Note that:
\begin{equation}\label{eq:weakly-unbalanced-walks-convergence}
    0 \leq
    \sum_k\frac{\beta^k}{k!}\sum_{l=1}^k
    \tr\mathbf{P}^{l-1}\mathbf{N}\mathbf{P}^{k-l}
    \leq
    \sum_{k=0}^\infty\frac{\beta^k}{k!}\tr\left(\mathbf{P} + \mathbf{N}\right)^k
    = \tr{}e^{\beta|\mathbf{A}|}
\end{equation}
where it is known that the rightmost matrix exponential and its trace always converge, 
so the middle part of the inequality must converge too.
}

\rev{
\subsection{Directed measures}\label{sec:methods:directed}

Here we extend all the previously defined measures to directed signed networks.
To do so, we first note that the structure theorems of SBT in their most general form
are formulated in terms of semipaths and semicycles (Sec.~\ref{sec:methods:sbt}).
Thus, it is natural to extend our approach to directed networks by simply using semiwalks
instead of ordinary walks.

\begin{definition}[Semiwalk]\label{def:semiwalk}
    A semiwalk is a sequence of adjacent edges such that for every two consecutive edges 
    $(i, j)$ and $(k, l)$ it holds that $k \in \{i, j\}$ or $l \in \{i, j\}$.
\end{definition}

More intuitively, semiwalks are just ordinary walks ignoring edge
directions~\cite{wassermanSocialNetworkAnalysis1994}, or walks on an undirected multigraph derived
from a given directed graph by making all edges bidirectional. Thus, semiwalks between all pairs of nodes
in a graph $G$ are counted by powers of its semiadjacency matrix, which is defined as the symmetric part 
of the adjacency matrix:
\begin{equation}\label{eq:semiadjacency}
    \mathbf{S}(\mathbf{A}) = \frac{1}{2}\left(\mathbf{A}+ \mathbf{A}^\top\right)
\end{equation}
Note that $\mathbf{S}$ is symmetric and $\mathbf{S}(\mathbf{A}) = \mathbf{A}$ when $\mathbf{A}$ is symmetric,
which jointly means that $\mathbf{S}[\mathbf{S}(\mathbf{A})] = \mathbf{S}$, so the semiadjacency operator
is idempotent. In what follows, we will use a simpler notation without the explicit dependence on $\mathbf{A}$
and we will use $\mathbf{S}$ to denote $\mathbf{S}(\mathbf{A})$ and $|\mathbf{S}|$ to denote
$\mathbf{S}(|\mathbf{A}|)$.

\begin{figure}[ht!]
\centering
\includegraphics[width=\textwidth]{semiwalks}
\caption{
Semiwalks in directed signed networks. Positive and negative semiwalks passing through symmetric
dyads with opposite edge signs cancel each other out.
}
\label{fig:semiwalks}
\end{figure}

Importantly, $\mathbf{S}$ is not a lossless representation of the adjacency matrix
of the undirected multigraph underlying a given directed signed network, but it is lossy in a way 
which does not affect any balance-related calculations. 
Firstly, reciprocal edges with opposite signs cancel each other out in $\mathbf{S}(\mathbf{A})$. 
However, this does not affect the difference between counts of positive and negative semiwalks,
$\mu_+ - \mu_-$, since each symmetric dyad with opposite edge signs will be included in the same
number of positive and negative semiwalks between $i$ and $j$ (Fig.~\ref{fig:semiwalks}).
Secondly, the $1/2$ factor means that $\mathbf{S}$ approximates the adjacency matrix of the 
multigraph divided by 2,
but, again, this does not matter as in our approach edge weights are reweighted by the $\beta$ parameter,
which sets the average edge weight, anyway. The gain from using the $1/2$ factor is that $\mathbf{S}$ is
idempotent and equal to $\mathbf{A}$ for undirected graphs.

As a result, directed balance measures are obtained simply by substituting $\mathbf{A}$
with $\mathbf{S}$ and $|\mathbf{A}|$ with $|\mathbf{S}|$ in all the formulas. However, to account
for the fact that 2-cycles in directed signed networks are not trivial
(i.e.~they may be both balanced and unbalanced), an additional correction
is needed. As explained in Sec.~\ref{sec:results:pre:approx}, asymmetric dyads do not span any
2-semicycles, while symmetric ones do. Thus, in the case of directed networks one needs to apply
corrections to Eqs. \eqref{eq:W} and \eqref{eq:V} to count proper 2-semicycles:
\begin{align}
    \vec{\mathbf{W}}(\mathbf{A}, \beta) 
    &= \frac{\beta^2}{2}\mathbf{A}^2 + \mathbf{W}(\mathbf{S}, \beta)
    \label{eq:W-dir} \\
    \vec{\mathbf{V}}(\mathbf{A}, \beta)
    &= \frac{\beta^2}{2}\left(\mathbf{PN} + \mathbf{NP}\right) + \mathbf{V}(\mathbf{S}, \beta)
    \label{eq:V-dir}
\end{align}
where both $\mathbf{W}$ and $\mathbf{V}$ still use $k_{\min} = 3$.
}

\subsection{Hierarchical clustering with pairwise DoB measures}\label{sec:methods:hclust}

Here we will use the following naive, yet effective, clustering procedure for signed networks
based on pairwise cohesion measures (see Secs. \ref{sec:results:pairwise} and \ref{sec:methods:weak-balance}).
Let $\mathbf{D}^S_{ij} = 1 - b_{ij}(\beta_{\max})$ and $\mathbf{D}^W_{ij} = 1- w_{ij}(\beta_{\max})$
be pairwise dissimilarity matrices
\rev{
(so $\mathbf{D}^S_{ii} = \mathbf{D}^W_{ii} \coloneqq 0$)
}
based on the notions of strong and weak balance respectively, 
and let $N_b$ be the maximum number of clusters one is willing to consider. Then, for $b = 1, \ldots, N_b$:
\begin{enumerate}
    \item Run Hierarchical Clustering (HC)~\cite{hastieElementsStatisticalLearning2008} algorithm
    for $b$ clusters using $\mathbf{D}^S$ as input and calculate frustration index according to 
    Eq.~\eqref{eq:findex} for the obtained block-partition matrix $\mathbf{B}$.
    \item Run HC for $b$ clusters using $\mathbf{D}^W$ as input and calculate the corresponding
    frustration index.
    \item Store the lower of the two frustration indices and its corresponding block partition.
\end{enumerate}
Finally, choose the partition with the lowest frustration index.


\rev{
\subsection{Accuracy of semiwalk-based approximations}\label{sec:methods:perf:walks}

\begin{figure}[ht!]
\centering
\includegraphics[width=\textwidth]{figs/semiwalks-accuracy}
\caption{
    Accuracy of semiwalk-based approximations relative to cycle-based DoB 
    estimates~\cite{giscardEvaluatingBalanceSocial2017} as measured by Pearson correlation
    $r$ and relative error $\bar{\epsilon} = \mean{|\frac{x - y}{y}|}$
    \textbf{(A)}~\enquote{Grayscale} ($k$-balance) measures based on semiwalks (MSB) and proper cycles
    in small networks studied in this paper
    (see Secs. \ref{sec:methods:datasets:tribes} and \ref{sec:methods:datasets:monks}). 
    In this case DoB values are reported for all possible cycle lengts and both MSB and cycle-based
    estimates are exact.
    \textbf{(B)}~Pearson correlations and relative errors for different cycle lengths calculated
    over co-sponsorship networks from the U.S. congress (Sec.~\ref{sec:methods:datasets:congress}).
    In this case MSB approximations are based on $m = 10$ leading eigenvalues and cycle-based
    estimates are approximated using sampling based on 10000 samples. Only cycles of lenght
    up to 15 were considered.
}
\label{fig:semiwalks-accuracy}
\end{figure}

MSB approach approximates semicycles with closed semiwalks. 
This is a fundamental design decision ensuring high computational efficiency,
but it comes at the price of introducing a discrepancy relative to cycle-based methods.
Here we present a comparison of $k$-balance methods provided by MSB and cycle-based approach from
Ref.~\cite{giscardEvaluatingBalanceSocial2017} based on several small and mid-sized networks.
The results indicate a strong similarity between the walk-based and the cycle-based DoB estimates
(Fig.~\ref{fig:semiwalks-accuracy}). 
Thus, it seems that the error introduced by walk-based approximations relative to cycle-based estimates
is typically small.
This should not come as a surprise as, thanks to Locality Principle, our MSB approach
ensures that DoB measures are driven primarily by patterns found in short closed walks, 
which coincide with cycles much more often than long walks
(e.g.~closed walks of length 3 are equivalent to 3-cycles).
}


\subsection{Numerical approximations and efficiency}\label{sec:methods:perf}

All computations of MSB can be implemented in a computationally efficient and accurate manner
using approximations based on $m$ leading eigenvalues and eigenvectors from both ends of the spectrum. 
Leading eigenpairs can be found very efficiently using modern linear algebra routines such as implicitly
restarted Arnoldi method~\cite{lehoucqARPACKUsersGuide1998,sorensenDeflationTechniquesImplicitly1996}.
Moreover, numerical stability can be guaranteed by conducting all computations in the log-space 
and using log-sum-exp trick (to avoid overflow when counting closed walks). 
This requires a bit of extra care as some eigenvalues may be non-positive. 
However, zero eigenvalues can be ignored altogether,
\rev{
since no measure defined here depends on the zeroth powers of adjacency matrices,
}
so the calculations can be done over the field of complex numbers, 
where the logarithm of any number with non-zero modulus is well-defined,
and cast back to real values only at the very end. As a result, MSB methods can be remarkably
efficient, even when applied to very large systems. 
\rev{
Secs. \ref{app:sec:numerical} and \ref{app:sec:efficiency} in the SI presents
empirical analyses of accuracy and efficiency of our implementation. Sec.~\ref{app:sec:analytic}
discusses the theoretical basis for approximations based on leading eigenvalues and eigenvectors. 
}

A more in-depth discussion of implementation details is beyond the scope of this paper, but we invite
the interested reader to study our source code (see: Data and code availability).


\subsection{Network datasets}\label{sec:methods:datasets}

\subsubsection{New Guinea Highlands tribes}\label{sec:methods:datasets:tribes}

An undirected unweighted signed network of friendships among tribes of Gahuku-Gama alliance structure 
of the Eastern Central Highlands region in New Guinea~\cite{readCulturesCentralHighlands1954}.
Edge sign indicates either friendship or enmity. 
Accessed from: \url{https://networks.skewed.de/net/new_guinea_tribes}

\subsubsection{Epinions trust network}\label{sec:methods:datasets:epinions}

This is a who-trust-whom online social network (directed, unweighted and signed) 
of a a general consumer review site \texttt{Epinions.com}. 
Members of the site can decide whether to \enquote{trust} each other. All the trust relationships interact
and form the Web of Trust which is then combined with review ratings to determine which reviews are shown
to the user~\cite{richardsonTrustManagementSemantic2003}.
Accessed from: \url{https://snap.stanford.edu/data/soc-Epinions1.html}.

\subsubsection{Wikipedia adminship vote}\label{sec:methods:datasets:wikipedia}

A directed unweighted signed network of votes on Request for Adminship (RfA) elections from a 2008 
snapshot of Wikipedia~\cite{leskovecGovernanceSocialMedia2010}.
Nodes represent editors, and a directed edge $(i,j)$ indicates that editor $i$ voted on editor $j$. 
Edge sign indicates the direction of the vote: positive = for, and negative = against. Edges are timestamped.
Accessed from: \url{https://networks.skewed.de/net/elec}.

\subsubsection{Slashdot Zoo network}\label{sec:methods:datasets:slashdot}

A directed unweighted signed network of interactions among users on Slashdot (\texttt{slashdot.org}), 
a technology news website~\cite{kunegisSlashdotZooMining2009}. 
Users name each other as friends (positive tie) or foe (negative tie). 
The friend label increases the scores of post, and the foe label decreases the score.
Accessed from: \url{https://networks.skewed.de/net/slashdot\_zoo}.

\subsubsection{Sampson's Monastery dataset}\label{sec:methods:datasets:monks}

Time series of 5 signed directed weighted networks measuring positive and negative relations between
postulants and novices in a New England monastery in 1960's~\cite{sampsonNovitiatePeriodChange1968}.
We used a version of the dataset studied in Ref.~\cite{doreianPartitioningApproachStructural1996}
in which edges have weights between -3 and 3 corresponding to the ranking of the least and most
(dis)liked/(dis)esteemed colleagues.
Accessed from: \url{http://vlado.fmf.uni-lj.si/pub/networks/data/esna/sampson.htm}.

\subsubsection{Co-sponsorship relations in the U.S. Congress}\label{sec:methods:datasets:congress}

Series of \rev{undirected} unweighted signed networks inferred from the data on bill co-sponsorships 
in both chambers of the U.S. Congress (House of Representatives and Senate) 
using Stochastic Degree Sequence Model~\cite{nealBackboneBipartiteProjections2014,nealSignTimesWeak2020}.
The data covers the period from 1973 (93rd Congress) to 2016 (114th Congress). Edges are signed, 
indicating the presence of a significant tendency to co-sponsor, or tendency to not co-sponsor, bills.
See SI, Sec.~\ref{app:sec:congress}, for a table with descriptive statistics.
Accessed from: \url{https://figshare.com/articles/dataset/A_Sign_of_the_Times/8096429}.