\section{Introduction}

Networks are used in many branches of science and engineering for modeling
complex relational systems.
Depending on properties of systems they represent, they may be undirected (ties are bidirectional)
or directed and weighted (ties have weights which usually indicate strength)
or unweighted~\cite{newmanNetworks2018}.
A particular variant are signed networks, in which ties are either positive or negative and so can be used to model valenced relations such as liking and disliking or alliances and 
war~\cite{doreianPartitioningApproachStructural1996,teixeiraEmergenceSocialBalance2017,estradaWalkbasedMeasureBalance2014,kirkleyBalanceSignedNetworks2019}. 
Signed networks are commonly used for representing systems capable of polarization, or clustering into 
groups with positive in-group and negative out-group ties. As a result, they have long been
important to social scientists interested in polarization and differentiation processes
inherent to formation of groups, attitudes and 
opinions~\cite{wassermanSocialNetworkAnalysis1994,doreianPartitioningApproachStructural1996,denooyLiteraryPlaygroundLiterary1999,schweighoferWeightedBalanceModel2020,nealSignTimesWeak2020}.
However, signed networks are also used in other disciplines for modeling diverse phenomena such as brain activation~\cite{saberiTopologicalImpactNegative2021}, ecological interactions~\cite{saizEvidenceStructuralBalance2017}, and financial 
time series~\cite{ferreiraLossStructuralBalance2021}. Moreover, it is often not only the signs that matter, but also the weights indicating intensities
of particular relations. Therefore, principled methods for analyzing signed networks, 
possibly with weights, are important for many applications.

Since signed networks represent valenced relations, a fundamental question concerns 
the degree to which positive and negative ties are consistent with respect to notions
of (anti)transitivity, and whether these microscopic patterns give rise to a polarized macroscopic
organization into mutually antagonistic clusters. 
Both problems are studied in Structural Balance Theory (SBT)~\cite{cartwrightStructuralBalanceGeneralization1956,hararyStructuralModelsIntroduction1965},
which originated from Gestalt psychology and the work of Fritz Heider~\cite{heiderAttitudesCognitiveOrganization1946}. Heider proposed that positive relations should be transitive (a friend of my friend is my friend)
and negative relations antitransitive (an enemy of my enemy is my friend), i.e.~valences of relations
should agree between positively linked agents and be opposite for negatively linked ones.
These considerations were later formalized and generalized in graph-theoretic terms and used to
demonstrate that (anti)transitivity of (negative) positive relations is directly linked to the properties 
of semicycles, and as a result to clustering and polarization. Namely, polarized macrostates
clustered in two antagonistic groups require that all semicycles are positive; that is, the 
products of the signs of their edges are positive 
(strong balance property; Fig.~\ref{fig:examples})~\cite{cartwrightStructuralBalanceGeneralization1956}.
Systems clustered into $k \geq 2$ antagonistic groups require that there are no semicycles with exactly
one negative edge (weak balance property)~\cite{davisClusteringStructuralBalance1967}. 
See Methods, Sec.~\ref{sec:methods:sbt}, for an overview of the main terms and results of SBT. 
A more detailed discussion can be found 
in~\cite{cartwrightStructuralBalanceGeneralization1956,davisClusteringStructuralBalance1967}.

\begin{figure}[htb!]
\centering
\includegraphics[width=.9\textwidth]{examples}
\caption{
    Example of high and low polarization in a social system. Two networks of bill co-sponsorship 
    in the U.S. Senate. Positive ties (blue) link senators, who tended to promote the same bills together more often than by chance, and negative ties (red) correspond to those who collaborated less 
    often than what would be expected by chance. See Sec.~\ref{sec:results:congress} 
    for the data description and a detailed analysis.
    \textbf{(A)}~96th Congress (1979-81, Carter administration) was a period of low polarization
    with frequent positive between-party and negative within-party links. This translated to 
    approximately equal frequencies of balanced (positive product of edge signs) and 
    unbalanced (negative product of edge signs) cycles.
    \textbf{(B)}~114th Congress (2015-17, Obama administration) featured high polarization
    with in-group ties being almost exclusively positive and out-group ties negative.
    This induced a distribution skewed significantly towards balanced (positive) cycles.
}
\label{fig:examples}
\end{figure}

SBT specifies strict requirements for signed networks to be balanced 
(partitioned in antagonistic groups), but real-world systems are rarely organized
neatly enough to satisfy them completely. This is why a lot of work in SBT is concerned with measures
of the degree of balance (DoB), or partial balance~\cite{arefMeasuringPartialBalance2018}, 
which can be seen as indicators of \enquote{distance} from the perfectly balanced state.
Such measures are typically directly or indirectly related to the relative frequencies of positive
and negative semicycles (or semicycles with exactly one negative edge, in the case of weak balance).

Applying SBT in practice is complicated by the fact that enumerating and counting 
(semi)cycles is computationally expensive, especially for large graphs. 
This problem can be partially alleviated via novel algorithms and sampling
methods~\cite{giscardEvaluatingBalanceSocial2017}, but exact solutions will always
remain prohibitively expensive due to the nature of the problem. Thus, several approximations
have been proposed which can roughly be divided into two families of local and global
measures. Local measures attain efficiency by focusing only on cycles of particular,
usually short, lengths, such as 3-cycles (triads). They can be fast, but provide only
a limited description of the real structure of a network. Thus, we argue that global measures 
are preferable.

Several global approaches have been proposed. Some ignore the problem
of counting cycles entirely, and instead search for partitions minimizing 
frustration~\cite{facchettiComputingGlobalStructural2011}
(the number of edges incompatible with SBT assumptions, see Methods, Sec.~\ref{sec:methods:findex}),
but they suffer from similar computational constraints due to their combinatorial nature.
Others leverage spectral properties of signed graphs and are therefore computationally efficient,
but measure only strong balance properties and quantify DoB using the smallest eigenvalue
of the signed Laplacian matrix~\cite{kunegisSpectralAnalysisSigned2010}, 
which is not normalized and can be difficult to compare between networks.
The last major approach is based on approximating cycle counts via counts of closed walks which can
be calculated, or at least approximated, very efficiently with standard linear
algebra~\cite{estradaWalkbasedMeasureBalance2014,singhMeasuringBalanceSigned2017,kirkleyBalanceSignedNetworks2019}. Moreover, it can produce both local and global measures as well as
capture strong and weak balance properties. On the other hand, walk-based
approximations can be misleading as they may combine patterns found at very different cycle
lengths~\cite{giscardEvaluatingBalanceSocial2017}. It is not immediately clear how such multiscalar information should be aggregated into a single numeric
DoB measure. Crucially, most of the existing walk-based 
approaches~\cite{estradaWalkbasedMeasureBalance2014,singhMeasuringBalanceSigned2017,giscardEvaluatingBalanceSocial2017,arefMeasuringPartialBalance2018,kirkleyBalanceSignedNetworks2019}
are defined in terms of closed walks which approximate ordinary cycles, while SBT theorems are concerned
with semicycles which should be approximated with semiwalks. This is a subtle yet important difference,
which may significantly affect results produced by walk-based approximations for directed networks.

Here we propose Multiscale Semiwalk Balance (MSB): an approach applicable to
(un)directed, (un)weighted signed networks without self-loops and multilinks.
It is multiscale as it provides both local measures approximating DoB at particular cycle lengths and global indicators aggregating local measures across multiple scales in a principled manner.
Namely, it accounts for the fact that counts of closed walks grow geometrically, where longer 
walks typically outnumber shorter very quickly and produce DoB estimates biased towards long-walk patterns. This violates what we call the Locality Principle (LP), which states that short semicycles matter
more than long ones. This was observed already
by~\citet{cartwrightStructuralBalanceGeneralization1956},
and later justified empirically by demonstrating that it is easier for people to memorize
valences of ties in shorter cycles~\cite{zajoncStructuralBalanceReciprocity1965}.
More recently, analyses based on counting simple cycles demonstrated that real networks often 
have a relatively low cycle length threshold after which DoB measures quickly decrease, 
indicating that structural balance is driven by structures at smaller
scales~\cite{giscardEvaluatingBalanceSocial2017}.

Our work builds on the Walk Balance (WB) approach proposed by 
Estrada and Benzi~\cite{estradaWalkbasedMeasureBalance2014},
for which it was noticed that it tends to underestimate DoB, especially in large
networks~\cite{singhMeasuringBalanceSigned2017,giscardEvaluatingBalanceSocial2017}.
We show that this is caused by too much weight being placed on long cycles and can be fixed
by introducing a formal resolution parameter. Namely, we demonstrate how the inverse temperature
parameter, $\beta$, considered briefly by the authors, can be reinterpreted and used to determine the
range of cycle lengths across which DoB measures should be aggregated.
It also allows our MSB approach to be applicable and meaningful in the context of
weighted signed networks.
Additionally, we generalize the WB approach to capture both strong and weak balance, as well as define DoB measures
not only at the level of entire graphs but also for particular nodes and pairs of nodes to enable the development of effective SBT-aware clustering methods.
Furthermore, by using semiwalk-based approximations our methods are more directly linked to SBT theorems
and therefore meaningful for directed signed networks.

We demonstrate the utility of our approach 
in two case studies of polarization in social systems. The first is a re-analysis
of the famous Sampson's Monks dataset~\cite{sampsonNovitiatePeriodChange1968},
in which we show that the commonly accepted \enquote{ground truth} partition is not SBT-optimal
by finding a better one, which also sheds some additional light on the underlying social dynamics. 
In the second study we use our methods to provide strong evidence for increasing
polarization in the U.S. Congress based on bill co-sponsorship data~\cite{nealSignTimesWeak2020}.

\subsection{Notation}\label{sec:notation}

Here we consider weighted graphs $G = (V, E, \omega)$ with $|V|$ vertices and $|E|$
edges and no self-loops or multilinks, where $V, E$ are vertex and edge sets respectively, 
and $\omega: E \to \mathbb{R}$ is a function assigning weights to edges. 
The weights can be negative, so the above definition captures all 
(un)signed, (un)weighted and (un)directed graphs.
An unsigned graph induced by an arbitrary graph $G$ will be denoted by 
$|G| = (V, E, |\omega|)$.

The adjacency matrix of a graph $G$ is given by a square $n \times n$ matrix $\mathbf{A}(G)$
such that $\mathbf{A}_{ij} = \omega(i, j)$ if $(i, j) \in E$ or otherwise $\mathbf{A}_{ij} = 0$. 
Whenever possible without introducing ambiguity, we will drop the explicit dependence on $G$ 
and prefer a simpler notation, $\mathbf{A}$. 
Matrix trace will be denoted by $\tr\mathbf{A}$ and Hadamard (elementwise) matrix product by $\otimes$.

\section{Results}\label{sec:results}

\subsection{Multiscale Semiwalk Balance approach}\label{sec:results:swb}

Here we introduce the Multiscale Semiwalk Balance (MSB) approach to SBT. 
We first develop it without considering the role of edge weights, which, 
as discussed below, appear in our approach naturally also in the context of unweighted networks. 
Once the core framework is established, we show that
it automatically extends to weighted graphs in a meaningful way.
The main text focuses on strong balance, while the analogous measures of weak balance are discussed
in Methods, Sec.~\ref{sec:methods:weak-balance}.

\subsubsection{Semiwalks and semiadjacency matrix}\label{sec:results:semiwalks}

We start by defining semicycles and semiwalks~\cite{wassermanSocialNetworkAnalysis1994}:
\begin{definition}[Semicycle]\label{def:semicycle}
    A semicycle is a closed walk in which each directed edge can be traversed both ways but
    only once and each intermediate node is visited exactly once. 
\end{definition}
\begin{corollary}
    A directed cycle is also a semicycle and a semicycle of length $2$
    is also a proper cycle.
\end{corollary}
\begin{definition}[Semiwalk]\label{def:semiwalk}
    A semiwalk is a sequence of edges such that for every two consecutive edges $(i, j)$ and $(k, l)$
    it holds that $k \in \{i, j\}$ or $l \in \{i, j\}$.
\end{definition}

Following the main idea behind all walk-based approaches, we will approximate
semicycles with closed semiwalks. Note that the above definition implies that semiwalks are equivalent
to ordinary walks on an undirected multigraph induced by a directed graph after edge directions 
are ignored. Fig.~\ref{fig:signed-semicycles} shows examples of the connection between cycles and semicycles.

\begin{figure}[ht!]
\centering
\includegraphics[width=\textwidth]{figs/semicycles}
\caption{
    Relationship between signed cycles and semicycles.
    \textbf{(A)}~Dyads in undirected networks do not generate any semicycles,
    as this would imply crossing the same edge twice. Moreover, closed 2-walks in undirected
    signed networks are always trivially balanced, since $1^2 = (-1)^2 = 1$.
    \textbf{(B)}~A single directed edge does not generate any semicycles, as by definition
    when making a cycle it can be crossed only once. On the other hand, fully-connected dyads
    generate proper semicycles which are balanced when the relation is symmetric and unbalanced
    otherwise.
    \textbf{(C)}~General relationship between directed cycles and semicycles. Edges in semicycles
    are treated as if they were bidirectional, so the problem is equivalent
    to considering cycles in an undirected multigraph in which each pair of nodes may be connected
    by one or two edges, and each edge can be traversed only once.
}
\label{fig:signed-semicycles}
\end{figure}

To count efficiently, we introduce a semiadjacency matrix operator
defined as the symmetric part of an adjacency matrix:
\begin{equation}\label{eq:semiadjacency}
    \mathbf{S}(G) = \frac{1}{2}\left(\mathbf{A}(G) + \mathbf{A}(G)^\top\right)
\end{equation}

Note that the 1/2 factor in front ensures that $\mathbf{S}(\mathbf{X}) = \mathbf{X}$ 
for any symmetric matrix $\mathbf{X}$. Moreover, $\mathbf{S}(\mathbf{X})$ is symmetric by definition.
Thus, the operator is idempotent, that is, $\mathbf{S}(\mathbf{S}(\mathbf{X})) = \mathbf{S}(\mathbf{X})$.
Importantly, $\mathbf{S}(G)$ represents an undirected weighted simple graph instead of
an undirected unweighted multigraph. Moreover, when $G$ is signed, $\mathbf{S}_{ij} = 0$
whenever $\omega(i, j) = -\omega(j, i)$. However, this is not a problem for counting signed semiwalks
because a pair of edges $(i, j)$ and $(j, i)$ with opposite signs would generate the same number
of semiwalks with opposite signs, which would eventually cancel each other out during DoB calculations.
Moreover, the fact that $\mathbf{S}(G)$ is weighted is also of no consequence, because,
as we show later, in our approach all graphs are treated as if they are weighted.
Lastly, $\mathbf{S}(G)$ is always symmetric. Thus, by the spectral theorem~\cite{nicaBriefIntroductionSpectral2018}, 
it is always diagonalizable and has real eigenvalues and eigenvectors.

Semiadjacency matrices can be used to count semiwalks in the same way
adjacency matrices can, so $\mathbf{S}^k$ counts semiwalks of length $k$ joining all pairs of nodes. 
In particular, following Eq.~\eqref{eq:texp} (Methods,~\ref{sec:methods:analytic}),
a matrix with weighted counts of semiwalks of lengths $k_0, k_0 + 1, \ldots, k_1 - 1, k_1$
between nodes $i$ and $j$ in an unsigned graph $|G|$ is given by a truncated matrix exponential,
$\exp\kk\left[\beta\mathbf{S}(|G|)\right]$.

Crucially, SBT is concerned with properties of semicycles, and semiwalks of lengths 0 and 1 cannot
be cycles. Moreover, in undirected networks cycles of length 2 are always trivially balanced, 
so 2-cycles should not be included when measuring structural balance and hence $k_0 = 3$ in this case. 
On the other hand, $\mathbf{S}$ is symmetric, so it always
produces trivially balanced 2-semicycles. Thus, 2-semicycle counts in directed networks should be 
determined using the raw adjacency matrix, $\mathbf{A}$, since 2-cycles and 2-semicycles are equivalent.
Thus, the weighted count of semiwalks in an (un)directed unsigned graph $|G|$ is given by:
\begin{equation}\label{eq:W}
    \mathbf{W}(|G|, \beta) = \begin{cases}
        \exp_3^{k_1}\left[\mathbf{S}(|G|, \beta)\right],
        \quad&\text{if $|G|$ is undirected} \\
        \frac{\beta^2}{2}\mathbf{A}(|G|)^2 + \exp_3^{k_1}\left[\mathbf{S}(|G|, \beta)\right],
        \quad&\text{otherwise}
    \end{cases}
\end{equation}
where $\beta$ is a free parameter used for controlling the amount of weight put on short and long
closed semiwalks. In Sec.~\ref{sec:results:lp} we show a principled way for finding
an appropriate value of $\beta$ informed by the Locality Principle.

Furthermore, there are no semicycles longer than the number of nodes, $n$, so $k_1 \leq n$,
and, since higher order terms in the exponential power series die off very quickly, typically
$k_1 \ll n$ approximates $k_1 = n$ very well.

\subsubsection{Strong balance}\label{sec:results:strong-balance}

Following Estrada and Benzi ~\cite{estradaWalkbasedMeasureBalance2014}, we note that if $|G|$
is induced by a signed graph $G$, then elements of $\mathbf{W}$ give weighted counts
of balanced (positive) and unbalanced (negative) walks together,
$\mathbf{W}(|G|,\beta)_{ij} = \mu_{ij}^+ + \mu_{ij}^-$. For the original
signed graph $G$ one gets the difference between weighted counts of balanced and unbalanced walks,
$\mathbf{W}(G,\beta)_{ij} = \mu_{ij}^+ - \mu_{ij}^-$.

Counts of closed semiwalks are given by the diagonal elements, so the overall counts are
given by appropriate matrix traces. Thus, to measure structural balance in a signed network one can
use the Balance Index~\cite{estradaWalkbasedMeasureBalance2014}, 
or the ratio of the difference between weighted counts of balanced and unbalanced walks to the 
count of all walks irrespective of signs: 
\begin{equation}\label{eq:bindex-strong}
    K_S(G, \beta) 
    = \frac{\mu^+ - \mu^-}{\mu^+ + \mu^-}
    =\frac{\tr\mathbf{W}(G,\beta)}{\tr\mathbf{W}(|G|,\beta)}
\end{equation}

A conceptually simpler measure is Degree of Balance (DoB), proposed already 
by~\citet{cartwrightStructuralBalanceGeneralization1956}, which represents the proportion of balanced walks:
\begin{equation}\label{eq:dob-strong}
    B_S(G, \beta)
    = \frac{\mu^+}{\mu^+ + \mu^-}
    = \frac{1}{2}\left[K(G, \beta) + 1\right]
\end{equation}

Following~\cite{estradaWalkbasedMeasureBalance2014} again, we can define node-level measures
simply by calculating diagonals instead of traces:
\begin{align}
    K_S(G, \beta, i) 
    &= \frac{\mathbf{W}(G, \beta)_{ii}}{\mathbf{W}(|G|, \beta)_{ii}} \label{eq:bindex-strong-node} \\
    B_S(G, \beta, i)
    &= \frac{1}{2}\left[K_S(G, \beta, i) + 1\right] \label{eq:dob-strong-node}
\end{align}

Moreover, measures of balance at a particular semicycle length $k$ can also be easily defined:
\begin{align}
    K_S(G, k) 
    &= \frac{\tr\mathbf{S}(G)^k}{\tr\mathbf{S}(|G|)^k} \label{eq:bindex-strong-k} \\
    B_S(G, k)
    &= \frac{1}{2}(K_S(G,k) - 1) \label{eq:dob-strong-k}
\end{align}

Note that this measure does not depend on $\beta$, since, even if it did, 
the same weighting factor would have to appear in both the numerator and denominator. 
This shows that $\beta$ indeed controls only the amount of weight put on different cycle
lengths, but does not influence the degree of balance at particular lengths.

\subsubsection{Contribution profiles and the Locality Principle}\label{sec:results:lp}

Using truncated exponentials defined in Eq.~\eqref{eq:texp}, one can asses the
contribution of closed semiwalks of length $k$ to the total weighted sum of semiwalk counts for lengths 
$k_0, \ldots, k_1$:
\begin{equation}\label{eq:contribution}
    C(G, \beta, k)
    = \frac{\beta^k}{k!} 
      \times \frac{\tr\mathbf{S}(|G|)^k}{\tr\mathbf{W}(|G|, \beta)}
\end{equation}

In other words, Eq.~\eqref{eq:contribution} measures the ratio of the weighted sum of closed semiwalks
of length $k$ to the total weighted sum of semiwalks over a specified range of lengths. 
It is normalized by construction, so $C(\ldots) \in [0, 1]$ and 
$\sum_kC(\ldots, k) = 1$.

The contribution score clearly depends on $\beta$, which can be used for 
controlling the influence of different length scales on the overall calculations. 
This is a crucial feature of our approach as it provides a straightforward operationalization
of the Locality Principle: shorter cycles should generally matter more than longer cycles.
In more technical terms, the contribution score can be used to identify a range of reasonable values for $\beta$. Namely, it should be that $\beta \in (0, \beta_{\max}]$, where:
\begin{equation}\label{eq:beta-max}
    \beta_{\max} \coloneqq \max{\beta} 
    \quad\text{s.t.}\quad
    C(G, \beta, k_0) > \ldots > C(G, \beta, k_1) 
\end{equation}
Finally, following the parsimony principle, we choose the weakest LP assumption possible and set
$\beta = \beta_{\max}$. Thus, $\beta$ plays the role of a resolution parameter setting the range
of cycle lengths across which DoB should be assessed. Moreover, it can be shown that $\beta_{\max}$
always  exists and indeed controls the weights put on closed (semi)walks of different lengths
(see sec.~\ref{app:sec:beta-max} of Supplementary Information (SI) for the proof).

Our results also explain why the original WB approach~\cite{estradaWalkbasedMeasureBalance2014}
underestimates DoB in large networks. 
Namely, it does so because without determining the characteristic scale of a network by tuning $\beta$,
the contribution profile may peak over very long cycles (Fig.~\ref{fig:contrib}).
Crucially, this problem is likely to affect any other walk-based methods, which do not use a well-tuned
resolution parameter. Moreover, without a measure akin to Eq.~\eqref{eq:contribution}, 
it is hard to know for sure whether a given method will produce correct results for a given network.

\begin{figure}[htb!]
\centering
\includegraphics[width=\textwidth]{contrib}
\caption{%
    Contribution (top) and local balance (bottom) profiles in four real networks studied by 
    Estrada and Benzi~\cite{estradaWalkbasedMeasureBalance2014}
    (see Methods, Sec.~\ref{sec:methods:datasets}, for dataset descriptions)
    based on WB ($\beta = 1$) and MSB ($\beta_{\max}$) approaches.
    Approximations based on $m = 10$ leading eigenvalues from both ends of the spectrum were used.
    Evidently, WB places most of the weight on very long cycles 
    ($k \approx 100$) in large networks, 
    which clearly violates LP. As a result, it produces much lower
    DoB estimates than MSB, since products of signs over very long closed walks are arguably
    mostly random. Only in the case of Epinions network WB produces an estimate close to the one 
    given by MSB. However, as balance measures at particular cycle lengths show, this happens only
    because of the very particular structure of the network resulting in high DoB at cycle 
    lengths of approximately 100. Moreover, this seems to be a statistical artifact which disappears 
    almost completely when balance is assessed based on semiwalks (MSB) instead of ordinary walks (WB).
}
\label{fig:contrib}
\end{figure}

\subsubsection{The Locality Principle and psychosocial constraints}\label{sec:results:lp:constraints}

The Locality Principle is justified also by a long history of social and psychological research. 
Indeed, it represents one of the three core principles of social influence
theory~\cite{latane1981psychology}, and has been extensively observed in empirical studies of social
networks~\cite{marsden1993network}. Not only has it been well-documented that geographical proximity 
is an important predictor of social
influence~\cite{festingerSocialPressuresInformal1950,meyners2017role,bell2007neighborhood}, 
but social proximity (friends vs. strangers) has also been shown to be a key factor across 
numerous areas of research \cite{ibarra1993power,dewan2017popularity,ma2015latent}. 

Recently, \cite{tunccgencc2021social} analyzed a global dataset and found that close
social connections had the strongest effect on people to follow social distancing guidelines amidst
the COVID-19 pandemic. Studies of social networks have demonstrated various domains of influence that are
affected by social proximity, including attitudes in an online music community~\cite{dewan2017popularity},
online purchasing behavior \cite{ma2015latent}, and information sharing on Twitter~\cite{zhang2013social}.
Thus, we argue that the Locality Principle is a central psychosocial factor constraining the flow of information
and influence in social networks, and therefore should be accounted for in operationalizations of SBT.

\subsubsection[Communicability and \textbeta{} as inverse temperature]{%
    Communicability and $\beta$ as inverse temperature
}\label{sec:results:communicability}

We already established that $\beta$ controls the impact that shorter and longer walks have on assessing
structural balance. In line with previous models of network centrality based on statistical
physics~\cite{estradaCommunicabilityComplexNetworks2008,koponenCharacterisingHeavytailedNetworks2021},
$\beta$ can be considered an inverse temperature parameter, which quantifies how much energy 
a random walker exploring a given network spends, or disperses, while traversing its links and 
nodes~\cite{estradaPhysicsCommunicabilityComplex2012,ghavasieh2021unraveling}. 
Here, both WB and MSB are closely related to the notion of communicability~\cite{estradaCommunicabilityComplexNetworks2008,estradaPhysicsCommunicabilityComplex2012},
which is defined as the exponential of the adjacency matrix rescaled by the inverse temperature, $\exp{\beta\mathbf{A}}$. Truncation notwithstanding, this definition coincides with our Eq.~\eqref{eq:W}. Given the nature of communicability as the energy spent or the information exchanged by a network
of quantum harmonic oscillators to sync, $\beta$ can be interpreted as the inverse temperature of the system~\cite{estradaPhysicsCommunicabilityComplex2012}. As the \enquote{temperature} 
approaches infinity, i.e. when $\beta \to 0$, less and less information flows between nodes and
only the shortest scales matter. In fact, one can check that the density matrix $\mathbb{\rho}$
--- relative to the entanglement of such exchange \cite{ghavasieh2021unraveling} --- 
scales as $\mathbb{\rho}_{\beta \rightarrow 0} \propto \mathbf{I} + \beta\mathbf{A}$, 
which means that only short-range connections influence information exchange. 
In the extreme case $\beta=0$, no information flows between nodes and this (lack of) exchange
becomes trivially independent of the network topology. 
However, there is another limit in which network topology does not influence the exchange of information, but for rather different reasons. When the temperature decreases and 
$\beta \to \infty$, the information flow becomes so high as to become independent of the network topology,
so in this regime any distinction between shorter and longer scales is lost~\cite{estradaPhysicsCommunicabilityComplex2012}.
In between these two extremes, there is a finite value $\beta = \beta_{\max}$
(see Eq.~\ref{eq:beta-max}) that describes the characteristic length of a system by determining the
maximal correlation range, which is still consistent with the Locality Principle.
As a result, LP provides a useful heuristic allowing for treating $\beta$ as a resolution parameter
determining the most appropriate scale, or rather a weighted average of scales or cycle lengths, 
for evaluating degree of balance. Without this physical interpretation and parameter tuning, 
(semi)walk-based estimates of DoB may not be meaningful.

\subsubsection[Weighted measures and \textbeta{} as average edge weight]{%
    Weighted measures and $\beta$ as average edge weight
}\label{sec:results:weighted}

$\beta$ can also be interpreted in terms of an average edge weight. Any unweighted network can be seen as a weighted network with uniform absolute edge weights of 1. Note that in this case the absolute product over a closed semiwalk of any length is always equal to 1, so every semiwalk is considered equal, 
and it is only $\beta$ that controls the characteristic length of the system by re-scaling edge weights,
which induces nonuniform semiwalk weights (through $\beta^k$ scaling). 
Thus, an arguably natural way to handle non-unitary weights is
to re-scale them, so the average absolute weight is equal to 1:
\begin{equation}\label{eq:weighting-scheme}
    w'_{ij} = \frac{mw_{ij}}{\sum_{kl} |w_{kl}|}
\end{equation}
where $m = |E|$ is the number of edges.

This retains the interpretation of $\beta$ in terms of an average edge weight and ensures that in a
network with a completely trivial topology (i.e.~Erdős–Rényi random graph with randomly and independently
assigned signs and absolute weights) the expected value of a semiwalk weight (i.e. the product of the
corresponding edge weights) gets fixed to 1 when $\beta = 1$.

\subsubsection{Pairwise measures and clustering}\label{sec:results:pairwise}

MSB approach can be also used to define pairwise measures of DoB \enquote{between}
any two nodes $i$ and $j$. Such a pairwise notion of DoB makes sense because it is directly
connected to yet another SBT theorem~\cite{cartwrightStructuralBalanceGeneralization1956}.

\begin{theorem}[Strong structure theorem (alternative)]\label{thm:strong-structure-II}
    A signed graph is strongly balanced if and only if all semipaths joining the same pair of nodes
    have the same sign.
\end{theorem}

This view of SBT makes it explicit that structural balance is determined not only by properties
of semicycles but also semipaths (from which the semicycles are formed). 
Thus, it is justified to define pairwise balance measures,
which, as it turns out, follow the same logic as before. Namely, strong pairwise Balance Index and DoB 
can be defined simply as:
\begin{align}
    K_S(G,\beta, i, j) &= \frac{\mathbf{W}(G,\beta)_{ij}}{\mathbf{W}(|G|,\beta)_{ij}}
    \label{eq:strong-bindex-pairwise} \\
    B_S(G,\beta, i, j) &= \frac{1}{2}\left(K_S(G,\beta, i, j)-1\right)
    \label{eq:strong-dob-pairwise}
\end{align}

% The above formulation makes sense, because two nodes are \enquote{jointly balanced}, and likely to belong
% to the same cluster, if and only if they are connected by many positive semipaths and few negative ones.

Pairwise DoB measures are important because they allow developing SBT-aware clustering methods.
We leave a detailed study of this idea for future work. However, in what follows we use pairwise
measures together with standard 
agglomerative hierarchical clustering~\cite{hastieElementsStatisticalLearning2008}
(see Methods, Sec.~\ref{sec:methods:hclust}, for details)
to show that MSB approach produces meaningful results such as detecting network partitions 
of objectively higher quality than \enquote{ground truth} solutions frequently assumed in the 
SBT literature.

\subsection{Re-analysis of Sampson's Monastery dataset}\label{sec:results:monks}

Sampson's Monastery study~\cite{sampsonNovitiatePeriodChange1968} produced one of the most famous network
datasets studied in Social Network Analysis (SNA) in general, and SBT in particular. It describes
the evolution of the social structure in a group of postulants and novices in a monastery 
in New England in 1960's. Namely, a network of liking (positive) and disliking (negative) relations was
measured at five points in time (see Methods, Sec.~\ref{sec:methods:datasets:monks}, for details). 
The dataset is particularly valuable because, as the study had been conducted, the group went through a major conflict, which eventually lead to either resignation or expulsion of the
majority of the members of the congregation. Moreover, Sampson identified a partition into three groups,
which later have been independently validated with analytic SBT-motivated
clustering methods~\cite{doreianPartitioningApproachStructural1996},
and therefore is commonly recognized as the \enquote{ground truth} solution.

The most important events happened at times $t=2,3,4$, which correspond to a period of differentiation
and polarization~\cite{doreianPartitioningApproachStructural1996} that eventually lead to an open
conflict and disintegration of the group. At $t=2$ twelve new members joined the monastery,
while some older members left after $t=1$, so the new group consisted of 18 men in total.
This perturbation lead to an emergence of two competing groups (Loyal Opposition and Young Turks)
as well as a group of peripheral members, who were not fully accepted by the rest (Outcasts).
The network at time $t=4$ depicts the structure just before the open conflict and disintegration.
At $t = 5$ only 7 members remained in the monastery, and those who stayed
(they are marked with red labels on Fig.~\ref{fig:monks}C, $t=4$)
belonged almost exclusively to the Loyal Opposition, which clearly \enquote{won} the conflict.

\begin{figure}[ht!]
\centering
\includegraphics[width=\textwidth]{monks}
\caption{
Re-analysis of Sampson's Monastery networks using SWB approach.
Full spectra were used in computations (exact results).
\textbf{(A)}~Signed sociograms at times $t=2, 3, 4$. Left side colors denote block membership according
to the \enquote{ground truth} partition and right side colors correspond to SWB partitions.
Positive ties are blue and negative are red. Individuals of which \enquote{ground truth} and SWB block
memberships differ (Amand, Basil and John Bosco) as well as the leaders of Young Turks 
(John Bosco and Gregory) are labelled. 
Network layout was determined with Kamada-Kawai algorithm using only positive ties with weights
(distances) on cross-block ties rescaled by the factor of 5.
\textbf{(B)}~Time series of strong and weak SWB DoB measures as well as strong DoB
based on WB approach of \citet{estradaWalkbasedMeasureBalance2014}, which is equivalent to SWB approach
with $\beta=1$ using ordinary adjacency matrix.
\textbf{(C)}~Weak local balance expressed as $z$-scores relative to the overall distribution.
Points are sized proportionally to local contributions and ordered first by block membership and
then by balance scores. Members who remained at the monastery after the
culmination of the conflict ($t=5$) are marked with red labels on the subplot for $t=4$.
\textbf{(D)}~Time series of frustration indices (see Methods, Sec.~\ref{sec:methods:findex})
according to partitions obtained with SWB and WB ($\beta=1$) approaches as well as the 
\enquote{ground truth} solution (which is defined only for times $t=2,3,4$).
}
\label{fig:monks}
\end{figure}

Here we use MSB approach to demonstrate that the \enquote{ground truth} partition 
is not SBT-optimal, or maximally consistent with Theorem~\ref{thm:weak-structure-I}.
This can be measured objectively in terms of frustration index, $F(G,\mathbf{B})$ 
(Methods, Sec.~\ref{sec:methods:findex}). We show that there is a partition with
lower frustration and that it has a meaningful interpretation leading to a deeper understanding
of the social dynamics at play.

Fig.~\ref{fig:monks}A shows both the \enquote{ground truth} and the MSB network partitions 
for times $t=2,3,4$ (see Methods, Sec.~\ref{sec:methods:hclust}).
They differ only in a few details, which are, nonetheless, very informative about the unfolding dynamics.
Firstly, according to the \enquote{ground truth} partition, Basil was a member of the Outcasts.
However, MSB analysis indicates that initially ($t=2$) he interacted mostly with the Young Turks
and only later was rejected and became one of the Outcasts. Secondly, Amand, a member of the 
Loyal Opposition according to the \enquote{ground truth}, was consistently identified as one of the
Outcasts by our MSB clustering procedure. Most importantly, according to MSB, John Bosco, 
who was considered one of the two leaders of the Young Turks (the second one was Gregory), became one of
the Outcasts just before the disintegration of the monastery ($t=4$). This says a lot about
why the Young Turks \enquote{lost} the competition against the Loyal Opposition, of which core constituted
most of the group that remained at the monastery. 

As evident in Fig.~\ref{fig:monks}C,
local weak balance scores of John Bosco were consistently low and at time $t=4$ also Gregory, the second
leader, attained low local balance. 
This was largely driven by the tension in their personal relationship 
(at $t=4$ the Gregory$\to$John Bosco tie is positive and John Bosco$\to$Gregory is negative),
which then propagated through the entire group 
(note that both of them had high local contribution scores, Fig.~\ref{fig:monks}C)
leading, probably, to its decomposition. 
As Fig~\ref{fig:monks}A shows, over time John Bosco established more positive connections with Outcasts
and developed negative feelings towards Gregory. At the same time, the core of Loyal Opposition 
strengthened internal connections and became very cohesive at time $t=4$, as indicated by high weak 
local balance scores of most of the individuals with red labels on Fig.~\ref{fig:monks}.

Last but not least, MSB measures of DoB are clearly high during the evolution of the
conflict ($t=2,3,4$), with the maximum at $t=4$, while analogous WP measures, 
which are not based on LP, yielded markedly lower DoB values
that cannot be readily interpreted as indicative of a conflict, as they are not much greater than
$1/2$ (which can be expected for a random assignment of edge signs). Similarly, frustration values
obtained with MSB clustering are consistently lower than those of \enquote{ground truth} partition,
and at times $t=1,2,3,4$ also lower than the ones obtained using WB.

Thus, the analysis indicates that MSB may be used to produce high quality, meaningful results,
which can be even better than popular \enquote{ground truth} solutions.
Moreover, through a combination of global and local measures an in-depth understanding of the underlaying
dynamics may be achieved.

\subsection{Polarization in the U.S. Congress}\label{sec:results:congress}

It is often claimed that political life in contemporary democracies have polarized significantly
over the last few decades. Arguably, this debate is particularly relevant for the U.S.,
because of its largely two party political system, for which the notion of (bi)polarization is 
particularly well-defined. 
Such a hypothesis is also supported by a lot of empirical evidence
(cf.~\cite{nealSignTimesWeak2020,hohmannQuantifyingIdeologicalPolarization2023} and references therein).

\begin{figure}[thb!]
\centering
\includegraphics[width=\textwidth]{congress}
\caption{
    Polarization in the U.S. Congress between 93th and 114th Congress (1973-2016).
    Panels are divided into regions corresponding to subsequent White House administrations
    with colors denoting Republican (red) and Democratic (blue) presidents.
    Approximations based on $m=10$ leading eigenpairs from the both ends of the spectrum were used.
    Starting from the top,
    \textbf{(1st~panel)} shows strong DoB time series based on MSB approach, $B_S(\beta_{\max})$ 
    and WB of \citet{estradaWalkbasedMeasureBalance2014}, $B_S(1)$.
    \textbf{(2nd~panel)} presents values of frustration index for optimal partitions into 2 clusters, 
    $F(G, \mathbf{2})$, general partitions minimizing $F(G, \mathbf{B})$,
    and partitions based on partisan affiliations, $F(G, \mathbf{P})$.
    \textbf{(3rd~panel)} quantifies similarity between party-based partitions and optimal bipartitions,
    $\text{AMI}(\mathbf{2}, \mathbf{P})$, as well as optimal partitions into $k$ clusters,
    $\text{AMI}(\mathbf{B}, \mathbf{P})$, using Adjusted Mutual Information
    (AMI) score~\cite{vinhInformationTheoreticMeasures2010}. The closer values are to 1, the better
    is the match between two clustering solutions.
    \textbf{(4th~panel)} shows the number of clusters in the solution minimizing $F(G, \mathbf{B})$
    (left $y$-axis), as well as the fraction of nodes within the two largest clusters (right $y$-axis).
}
\label{fig:congress}
\end{figure}

Here we use MSB approach to study polarization in both chambers of the U.S. Congress 
based on patterns of bill co-sponsorship between 1973 and 2016
(93rd to 114th Congress)~\cite{nealSignTimesWeak2020}.
The dataset consists of two sequences of signed networks
(see Methods, Sec.~\ref{sec:methods:datasets:congress} for details) 
inferred from co-sponsorship data, where positive ties indicate statistically significant 
tendency of two representatives/senators to promote the same bills and negative ties the opposite
tendency to not work on the same projects. 

Our analysis indicates that polarization increased markedly in both the House of Representatives
(Fig.~\ref{fig:congress}A) and the Senate (Fig.~\ref{fig:congress}B). This is evident in the steadily
increasing strong DoB values meaning that co-sponsorship networks became easier to bipartition in time.
Interestingly, and consistently with our previous analysis
of the importance of Locality Principle, WB approach yielded almost exclusively very low DoB values,
and thus would not capture the true trend. This is, of course, the consequence of the violation of LP.

In both chambers frustration index values clearly converge (Fig.~\ref{fig:congress}, 2nd panels) 
meaning that optimal bipartitions and optimal clusterings (in $k$ groups) based on MSB approach,
as well as partitions following partisan affiliations are becoming more and more consistent with the
SBT theorems and therefore also similar. This is evident in the time series of
the similarity between MSB and partisan partitions (Fig.~\ref{fig:congress}, 3rd panels).
Moreover, even in $k$-clusterings with $k$ large, most of the nodes tend to belong to the two 
largest clusters, indicating, again, an increasingly bipolar structure organized along the party lines.


\section{Discussion}\label{sec:discussion}

Polarization is often considered a salient, and perhaps worrying, feature of contemporary 
societies~\cite{schweighoferWeightedBalanceModel2020,nealSignTimesWeak2020,arefDetectingCoalitionsOptimally2020,hohmannQuantifyingIdeologicalPolarization2023}.
It can result in a sharp divergence of popular beliefs or attitudes (ideological polarization)
as well as in-group favouritism and out-group hostility 
(affective polarization)~\cite{hohmannQuantifyingIdeologicalPolarization2023}.
Crucially, the latter implies clustering of social networks into 2 or more groups
with primarily positive in-group and negative out-group ties. This structural aspect of polarization
is studied in Structural Balance Theory (SBT), which links it to properties
of semicycles in signed networks and provides strict criteria for measuring 
polarization~\cite{cartwrightStructuralBalanceGeneralization1956,davisClusteringStructuralBalance1967}.
 
Here we introduced Multiscale Semiwalk Balance (MSB) approach for measuring both strong and weak
degree of balance (DoB), which is applicable to any kind of (simple) signed networks, including directed
and weighted ones. MSB is computationally efficient by approximating semicycles with semiwalks,
which can be counted using standard linear algebra. Crucially, it is explicitly multiscale as it provides
both local measures, assessing DoB for cycles of particular length, as well as global measures aggregating
local information  in a principled manner based on the Locality Principle (LP). The resolution of an
analysis is controlled by  a single inverse temperature parameter, $\beta$, which can be tuned based on
first principles to capture the characteristic scale of a network at which its DoB should be assessed.
This is a crucial feature of our framework, as even though many other approaches apply some decaying
weights to longer cycles, typically the decay rate is fixed or controlled by a free parameter with no
principled way of selecting an appropriate 
value~\cite{estradaWalkbasedMeasureBalance2014,singhMeasuringBalanceSigned2017,giscardEvaluatingBalanceSocial2017,arefMeasuringPartialBalance2018,kirkleyBalanceSignedNetworks2019}.

% \subsection{Locality principle improves DoB measures}\label{sec:discussion:lp}

Real networks rarely fully satisfy the requirements of the structure theorems of SBT,
so in practice measures of partial balance, or degree of balance, are 
used~\cite{cartwrightStructuralBalanceGeneralization1956,arefMeasuringPartialBalance2018}. 
Such measures are often based on comparing relative frequencies of balanced and unbalanced cycles,
but it is not immediately clear how cycles of different lengths should be weighted to produce
meaningful results. MSB solves this problem by following the Locality Principle (LP), which motivates
a principled weighting scheme for aggregating relative counts across different lengths.
Our results show that this allows for a very effective analysis of signed networks, 
including finding clearly interpretable partitions of objectively higher quality 
(measured with frustration index) than well-known \enquote{ground truth} solutions
(Sec.~\ref{sec:results:monks}). We also demonstrated that MSB provides useful tools for studying polarization
in social systemss by providing strong evidence for increasing polarization in both chambers
of the U.S. Congress based on bill co-sponsorship data (Sec.~\ref{sec:results:congress}).

% \subsection{The importance of multiscale approaches}\label{sec:discussion:multiscale}

LP is justified not only by its usefulness as a heuristic guiding DoB methods,
but also by a long history of social and psychological
research~\cite{zajoncStructuralBalanceReciprocity1965,latane1981psychology,marsden1993network,ibarra1993power,dewan2017popularity,ma2015latent}. 
This stresses the importance of multiscale approaches to SBT and network science more generally. 
By linking structural balance to
communicability~\cite{estradaCommunicabilityComplexNetworks2008,estradaPhysicsCommunicabilityComplex2012},
our results suggest that, perhaps, other network descriptors defined
in terms of walks, or powers of adjacency matrices, can be informed by Locality Principle. 
Note that contribution scores defined in Eq.~\eqref{eq:contribution}, and used for operationalizing LP,
can be calculated for any, also unsigned, network. Thus, LP is a heuristic for determining
communicability, or the characteristic intensity and length of internode correlations, of a network,
and this determines the appropriate weighting scheme for aggregating walk-based measures across multiple
length scales. More generally, our results contribute also to research on the importance 
of local structures in 
networks~\cite{miloNetworkMotifsSimple2002,mattssonFunctionalStructureProduction2021,talagaStructuralMeasuresSimilarity2022}.

% \subsection{Principled DoB measures for directed and weighted signed networks}

Unlike many other approaches to
SBT~\cite{estradaWalkbasedMeasureBalance2014,singhMeasuringBalanceSigned2017,kirkleyBalanceSignedNetworks2019}, 
MSB is formulated explicitly in terms of semicycles and semiwalks.
This connects it more directly to the structure
theorems~\cite{cartwrightStructuralBalanceGeneralization1956,davisClusteringStructuralBalance1967}, 
and as a result facilitates meaningful analyses of directed networks. 
Moreover, by linking structural balance to communicability~\cite{estradaCommunicabilityComplexNetworks2008} 
and LP, we proposed a principled interpretation of edge weights in signed networks.
Thus, our work may be used to significantly extend the range of applicability of SBT,
facilitating research on structural balance and polarization in a wide variety of systems,
irrespective of whether they are (un)directed or (un)weighted.

% \subsection{Limitations}\label{sec:discussion:limitations}

Our methods, of course, are not defined for multilayer or higher-order
networks~\cite{newmanNetworks2018}.
This stems from the fact that it is not yet entirely clear how to generalize SBT to such systems
in a way that would retain the core features of the theory, e.g.~reproduce the structure theorems
(although some promising attempts have been made~\cite{panStructuralBalanceMultiplex2018}).
Moreover, in both case studies we analyzed dynamics, or evolution,
of social structure. As long as one can assume that the structure does not change significantly 
on short time scales, it is reasonable to analyze \enquote{snapshot} networks
aggregating relations over longer time intervals. This is exactly what we did.
Nonetheless, generalizations of SBT to other dynamical frameworks, such as relational event
model~\cite{stadtfeldInteractionsActorsTime2017}
or stream graphs~\cite{citraro2022delta}, 
remains an interesting problem.

Secondly, our approach attains computational efficiency by approximating semicycles with semiwalks, 
a strategy which can sometimes produce less accurate results than exact (but much less efficient) 
methods based on counting proper cycles~\cite{giscardEvaluatingBalanceSocial2017}. However, thanks to LP
our measures are driven primarily by patterns found in relatively short closed walks, which by definition
are more similar to cycles than long walks (closed walks of lengths 2 and 3 are equivalent 
to 2- and 3-cycles). Moreover,  the accuracy of our approach could be improved by reformulating it 
using non-backtracking (Hashimoto) matrices~\cite{torresNonbacktrackingCyclesLength2019},
which would further decrease the discrepancy between closed walks and cycles. 
Such an extension may be a promising direction for future work.
