% Appendices
\subsection[Proof of the existence of \textbeta-max and related results]{%
    Proof of the existence of $\beta_{\max}$ and related results
}\label{app:sec:beta-max}

Using Eqs.~\eqref{eq:bindex-strong-k} and \eqref{eq:contribution} one can derive 
Eq.~\eqref{eq:bindex-strong} as a weighted average of local balance:
\begin{equation}\label{app:eq:beta-proof}
\begin{split}
    K_S(G, \beta)
    &= \frac{\tr\mathbf{W}\kk(G, \beta)}{\tr\mathbf{W}\kk(|G|, \beta)} \\
    &= \frac{1}{\tr\mathbf{W}\kk(|G|, \beta)}
    \sum_{k=k_0}^{k_1}\frac{\beta^k}{k!}\tr\mathbf{S}^k \\
    &= \sum_{k=k_0}^{k_1}\frac{\beta^k}{k!} 
    \times \frac{\tr|\mathbf{S}|^k}{\tr\mathbf{W}\kk(|G|, \beta)}
    \times \frac{\tr\mathbf{S}^k}{\tr|\mathbf{S}|^k} \\
    &= \sum_{k=k_0}^{k_1}C_{k_0}^{k_1}(G, \beta, k)K_S(G, k)
\end{split}
\end{equation}

This result shows that global Balance Index, $K_S(G, \beta)$, is just a weighted average of 
local Balance Index values, $K_S(G, k)$, weighted by the corresponding contributions
of semiwalks of different lengths. Now, contribution values are proportional to powers of 
$\beta^k$, namely, $C(G, \beta, k) \propto (\beta^k/k!)\tr\mathbf{|S|}^k$,
so in the limit as $\beta \to 0$ only the leading term $C(G, \beta, k_0)K_S(G, k_0)$ survives. 
Moreover, the same argument applies to any comparison between terms $k$ and $k+1$ meaning that the 
locality principle can be always satisfied by choosing a sufficiently small $\beta_{\max}$.
Hence, $\beta_{\max}$ always exists. In particular, with sufficiently
small $\beta$ contribution arbitrarily close to $1$ can be always assigned to the length $k = k_0$.

To strengthen the above argument we will now derive an analytic formula for approximating $\beta_{\max}$,
which is quite accurate for graphs with large enough spectral gap, that is, the difference between largest 
and second largest eigenvalue. First, let us note that for the Locality Principle to hold, for all
$k = k_0, \ldots, k_1-1$ it must be true that:
\begin{equation}\label{app:eq:beta-max-approx-1}
\begin{split}
    &C\kk(G, \beta, k) \geq C\kk(G, \beta, k+1) \\
    \Longrightarrow\quad&
    \frac{\beta^k}{k!}\tr\mathbf{|S|}^k \geq \frac{\beta^{k+1}}{(k+1)!}\tr\mathbf{|S|}^{k+1} \\
    \Longrightarrow\quad&
    \sum_{i=1}^n \lambda_i^k \geq \frac{\beta}{k+1}\sum_{i=1}^n \lambda_i^{k+1}
\end{split}
\end{equation}
where $\lambda_1 \geq \lambda_2 \geq \ldots \geq \lambda_n$ are eigenvalues of 
$\mathbf{|S|}$ sorted in decreasing order.
Now, since $\mathbf{|S|}$ is symmetric all its eigenvalues are real. Moreover by extended Perron-Frobenius
theorem~\cite{nicaBriefIntroductionSpectral2018} $\lambda_1$ is always positive. Thus, the inequality in
Eq.~\eqref{app:eq:beta-max-approx-1}, provided that the spectral gap $\mathbf{|S|}$ is large enough, 
can be simplified to:
\begin{equation}\label{app:eq:beta-max-approx-2}
    \lambda_1^k \geq \frac{\beta}{k+1}\lambda_1^{k+1}
    \quad\Longrightarrow\quad
    \beta \leq (k+1)\lambda_1^{-1}
\end{equation}
which gives the conditions for finding $\beta_{\max}$. Now, the last step is to realize that the
above inequality holds for all $k$ if and only if it holds for $k = k_0$. Hence:
\begin{equation}\label{app:eq:beta-max-approx-3}
    \beta_{\max} \approx (k_0+1)\lambda_1^{-1}
\end{equation}
which serves as yet another evidence for the existence of $\beta_{\max}$.


\subsection{Proof of Theorem~\ref{thm:weak-structure-II}}\label{app:sec:weak-pairwise-balance}

We first prove necessity of the stated fact using a contradiction argument. Let $G$ be a weakly balanced
graph with $i$ and $j$ being a pair of nodes such that there is a semipath $p_{i \to j}$ with no negative
edges between them and another semipath $q_{j \to i}$ with exactly one negative edge. Then, there is
a cycle $(p_{i \to j}, q_{j \to i})$ with exactly one negative edge, so $G$ is not weakly balanced
by definition, which leads to a contradiction.

Now we show sufficiency. Let $i$ and $j$ be any two nodes connected by at least one all positive
semipath. By hypothesis, such a pair cannot be simultaneously connected by a semipath with exactly
one negative edge. As a result, $G$ does not contain any semicycles with exactly one negative edge
and is therefore weakly balanced.


\subsection{Computational complexity}\label{app:sec:efficiency}

Below are computation times for global, local and node-wise DoB measures for the three large
networks studied in this paper (Epinions, Slashdot and Wikipedia). Performance was assessed
using a laptop with AMD Ryzen 9 5900HX CPU and 16Gb of RAM. As evident in Fig.~\ref{app:fig:efficiency},
all running times were arguably short. Global DoB and balance profiles were calculated in about
1 second or much less. Node-wise measures (for all nodes) were calculated in no more than 
16 seconds (in the case of the largest network). All results include both the time needed for solving 
the eigenproblem(s), which can be cached and re-used in multiple computations, as well as any downstream
computations using eigenvalues and eigenvectors. Furthermore, in all cases computation times seem to 
scale with respect to $m$ in a very similar fashion with an average slope coefficient (in log-log scale)
of about $0.61$. This indicates that, at least for relatively low values of $m$, SWB computation times
are only moderately (sub-linearly) affected when increasing the number of used leading eigenpairs.

\begin{figure}[htb!]
\centering
\includegraphics[width=.85\textwidth]{figs/efficiency}
\caption{
Running times of global, local and node-wise DoB measures (both strong and weak).
Lines correspond to median times (over 10 repetitions) and bounds to 2nd and 9th deciles.
}
\label{app:fig:efficiency}
\end{figure}

\newpage
\subsection{Descriptive statistics for the U.S. Congress co-sponsorship networks}
\label{app:sec:congress}

\begin{table}[ht!]
\footnotesize
\begin{threeparttable}
\begin{tabular}{ll|rrrrrrrrr}
\toprule
Chamber & Congress & $f_R$ & $f_D$ & $|V|$ & $|E|$ & $f_+$ & $B_S$ & $B_W$ & $\bar{d}$ & $d_{\text{cv}}$ \\
\midrule
\multirow[t]{21}{*}{House} & 93 & 0.43 & 0.56 & 446 & 18083 & 0.56 & 0.71 & 0.94 & 81.09 & 0.73 \\
 & 94 & 0.33 & 0.67 & 445 & 19503 & 0.58 & 0.71 & 0.95 & 87.65 & 0.70 \\
 & 95 & 0.34 & 0.66 & 444 & 21133 & 0.59 & 0.68 & 0.96 & 95.19 & 0.60 \\
 & 96 & 0.38 & 0.62 & 442 & 51081 & 0.84 & 0.55 & 1.00 & 231.14 & 0.33 \\
 & 97 & 0.45 & 0.54 & 447 & 49364 & 0.82 & 0.57 & 1.00 & 220.87 & 0.35 \\
 & 98 & 0.39 & 0.61 & 444 & 48721 & 0.75 & 0.59 & 0.99 & 219.46 & 0.31 \\
 & 99 & 0.42 & 0.57 & 443 & 49764 & 0.72 & 0.61 & 0.98 & 224.67 & 0.29 \\
 & 100 & 0.42 & 0.58 & 446 & 50688 & 0.73 & 0.62 & 0.99 & 227.30 & 0.29 \\
 & 101 & 0.42 & 0.58 & 449 & 56231 & 0.70 & 0.62 & 0.98 & 250.47 & 0.25 \\
 & 102 & 0.40 & 0.60 & 447 & 58067 & 0.68 & 0.63 & 0.98 & 259.81 & 0.25 \\
 & 103 & 0.42 & 0.58 & 446 & 59092 & 0.68 & 0.70 & 0.98 & 264.99 & 0.23 \\
 & 104 & 0.53 & 0.46 & 445 & 62154 & 0.72 & 0.78 & 1.00 & 279.34 & 0.25 \\
 & 105 & 0.52 & 0.48 & 449 & 66701 & 0.69 & 0.80 & 1.00 & 297.11 & 0.23 \\
 & 106 & 0.51 & 0.49 & 442 & 63652 & 0.67 & 0.83 & 0.99 & 288.02 & 0.24 \\
 & 107 & 0.51 & 0.48 & 447 & 63851 & 0.68 & 0.84 & 0.99 & 285.69 & 0.24 \\
 & 108 & 0.52 & 0.47 & 444 & 66277 & 0.67 & 0.84 & 0.99 & 298.55 & 0.24 \\
 & 109 & 0.53 & 0.47 & 445 & 66700 & 0.68 & 0.82 & 1.00 & 299.78 & 0.24 \\
 & 110 & 0.46 & 0.54 & 452 & 70923 & 0.67 & 0.82 & 0.99 & 313.82 & 0.20 \\
 & 111 & 0.41 & 0.59 & 451 & 70160 & 0.68 & 0.77 & 0.99 & 311.13 & 0.21 \\
 & 112 & 0.54 & 0.45 & 450 & 77872 & 0.66 & 0.82 & 1.00 & 346.10 & 0.18 \\
 & 113 & 0.53 & 0.47 & 447 & 75771 & 0.64 & 0.86 & 1.00 & 339.02 & 0.18 \\
 & 114 & 0.56 & 0.44 & 446 & 75180 & 0.65 & 0.86 & 1.00 & 337.13 & 0.19 \\
 \midrule
\multirow[t]{21}{*}{Senate} & 93 & 0.42 & 0.56 & 101 & 2439 & 0.76 & 0.62 & 0.99 & 48.30 & 0.35 \\
 & 94 & 0.37 & 0.61 & 100 & 2432 & 0.79 & 0.63 & 1.00 & 48.64 & 0.35 \\
 & 95 & 0.37 & 0.62 & 104 & 2336 & 0.81 & 0.57 & 1.00 & 44.92 & 0.39 \\
 & 96 & 0.41 & 0.58 & 101 & 2275 & 0.82 & 0.55 & 1.00 & 45.05 & 0.39 \\
 & 97 & 0.52 & 0.47 & 101 & 2073 & 0.79 & 0.60 & 0.99 & 41.05 & 0.37 \\
 & 98 & 0.53 & 0.47 & 101 & 2194 & 0.76 & 0.58 & 0.99 & 43.45 & 0.33 \\
 & 99 & 0.52 & 0.48 & 101 & 2177 & 0.75 & 0.61 & 0.99 & 43.11 & 0.33 \\
 & 100 & 0.46 & 0.54 & 101 & 2143 & 0.72 & 0.60 & 0.99 & 42.44 & 0.36 \\
 & 101 & 0.44 & 0.54 & 101 & 2445 & 0.68 & 0.63 & 0.98 & 48.42 & 0.31 \\
 & 102 & 0.42 & 0.56 & 102 & 2479 & 0.71 & 0.64 & 0.99 & 48.61 & 0.31 \\
 & 103 & 0.44 & 0.54 & 101 & 2257 & 0.72 & 0.70 & 0.99 & 44.69 & 0.35 \\
 & 104 & 0.52 & 0.46 & 102 & 2324 & 0.74 & 0.81 & 1.00 & 45.57 & 0.38 \\
 & 105 & 0.53 & 0.45 & 100 & 3002 & 0.70 & 0.81 & 1.00 & 60.04 & 0.27 \\
 & 106 & 0.53 & 0.45 & 102 & 2930 & 0.72 & 0.78 & 0.99 & 57.45 & 0.25 \\
 & 107 & 0.48 & 0.50 & 101 & 2522 & 0.73 & 0.70 & 0.99 & 49.94 & 0.30 \\
 & 108 & 0.50 & 0.48 & 100 & 2387 & 0.74 & 0.79 & 0.99 & 47.74 & 0.28 \\
 & 109 & 0.53 & 0.45 & 101 & 2823 & 0.73 & 0.82 & 0.99 & 55.90 & 0.25 \\
 & 110 & 0.49 & 0.49 & 102 & 2779 & 0.70 & 0.85 & 0.99 & 54.49 & 0.30 \\
 & 111 & 0.39 & 0.59 & 109 & 3645 & 0.74 & 0.68 & 1.00 & 66.88 & 0.27 \\
 & 112 & 0.48 & 0.50 & 101 & 3914 & 0.69 & 0.78 & 1.00 & 77.50 & 0.20 \\
 & 113 & 0.44 & 0.54 & 105 & 3932 & 0.65 & 0.86 & 1.00 & 74.90 & 0.24 \\
 & 114 & 0.54 & 0.44 & 100 & 3696 & 0.61 & 0.96 & 1.00 & 73.92 & 0.20 \\
 \bottomrule
\end{tabular}
\begin{tablenotes}
    \footnotesize
    \item $f_R, f_D$ -- fraction of Republicans/Democrats 
    (may not sum up to 1 due to the presence of other parties and/or independents)
    \item $|V|, |E|$ -- number of nodes/edges
    \item $f_+$ -- fraction of positive edges
    \item $B_S, B_W$ -- strong/weak degree of balance
    \item $\bar{d}, d_{\text{cv}}$ -- average degree and coefficient of variation of degree distribution
\end{tablenotes}
\end{threeparttable}
\end{table}
