 
\newpage
\label{sec:supplementary}
\renewcommand{\thepage}{S\arabic{page}}  
\renewcommand{\thesection}{S\arabic{section}}    
\renewcommand{\thetable}{S\arabic{table}}   
\renewcommand{\thefigure}{S\arabic{figure}}
% \newfloat{algorithm}{htbp}{loa}[figure]
% \floatname{algorithm}{Algorithm}
% \renewcommand{\thealgorithm}{S\arabic{algorithm}}
\section{Method}
\subsection{Sub-datasets identification and creation}
In the following we provide a more detailed explanation for the sub-datasets generation process.\\
{\bf Generation of Sub-datasets}:
As an initial step we train a base model on the entire dataset. We denote the initial predicted patch $\hat{P}_{init}$ and the patch in the ground-truth RGB image as $P$, both of size $H_p \times W_p \times 3$. Then, we collect different types of patches by using a variety of metrics and create a pool of sub-datasets.
Each sub-dataset consists of the 5 percent patch subset which produces the largest metric error. The following custom metrics are used for this stage:
\begin{enumerate}
    \item \textit{Zipper metric} - this metric $d_{zip}$ is used to find zipper artifacts near edges in the image. First we calculate a mask $M_{NE}$ to identify non-edge areas as follows
    \begin{equation}
        \label{eq:MNE}
        M_{NE} := | C( \hat{P}_{init} ) - C\left( P \right) | < \varepsilon_1,
    \end{equation}
    where $C$ is some kind of edge detector, \eg Canny, and $\varepsilon_1$ is a predefined threshold.
    Then we are able to calculate
    \begin{equation}
        \label{eq:HZ}
        d_{zip} := \| | \, | \nabla \hat{P}_{init} | - \left| \nabla P \right| |  \odot M_{NE} \|_1,
    \end{equation}
    where $ \left| \nabla \right| = \left|\frac{\partial}{\partial x} \right| + \left|\frac{\partial}{\partial y} \right|$, and $\odot$ is the element-wise product.
    
    \item \textit{Grid metric} - this metric $d_{grid}$ is used to find grid-like artifacts in flat areas as follows
    \begin{multline}
        \label{eq:HG}
        d_{grid} := \| 2 \cdot | \nabla \hat{P}_{init} | \, \odot \\
        ( \sigma ( \frac{\alpha}{\left| \nabla P \right|}  ) -0.5 ) \odot ( | \nabla \hat{P}_{init} | > \varepsilon_2 ) \|_1,
    \end{multline}
    where $\sigma$ is the Sigmoid function, $\alpha > 0$ is a parameter that scales the values of the Sigmoid, and $\varepsilon_2$ is another predefined threshold.

    \item $L_1$ \textit{distance} - a simple photometric distance.
    \begin{equation}
    \label{eq:L1}
        d_{L_1}: = \| \hat{P}_{init} -  P  \|_1     
    \end{equation}
    

    \item \textit{Perceptual metric} \cite{ledig2017photo} - this metric $ d_{VGG/i.j} $ is used for capturing the perceptual distance between the source and predicted patches.
    \begin{equation}
	\label{eq:vgg}
	    d_{VGG/i.j} := \| \phi_{i,j}(\hat{P}_{init}) - \phi_{i,j}(P) \|^2_2,
	\end{equation}
    where $\phi_{i,j}$ indicates the feature map obtained by the j-th convolution (after activation) before the i-th maxpooling layer within the pre-trained VGG19 network \cite{simonyan2014very}.
    
    \item \textit{Edge distance} - This metric $d_{edge}$ is used to find distortions around edges
    \begin{equation}
	\label{eq:edge}
	    d_{edge} := \| C( \hat{P}_{init} ) - C\left( P \right) \|_1.
	\end{equation}
    
\end{enumerate}

Note, some sub-datasets are generated by changing the threshold of the metric such as in \Cref{eq:HG}. \\
\subsection{Selection and elimination of sub-datasets}
Our method for selecting sub-datasets is presented in the pseudo code \Cref{alg:IdentifyingDesiredDataCategories}. We outline the method for selecting specific sub-datasets among the collected pool of sub-datasets. The method involves two steps: In step 1 we re-train the base model that was previously trained on the entire dataset. The re-training is performed on one of the sub-datasets. We calculate the validation loss curves on all the sub-datasets, as well as the validation loss curve over the entire dataset. We then check if the validation loss curve on the trained sub-dataset is negatively correlated with the validation loss curve on the entire dataset. If so, the sub-dataset is added to a dictionary called "category bank". In step 2, the algorithm checks the correlation between the validation loss curves for each pair of sub-datasets in the bank. If the correlation is positive, the algorithm combines the two sub-datasets into a single sub-dataset. We use Spearman's rank correlation coefficient to measure the correlation between the validation loss curves. The algorithm also includes a threshold parameter ($Th_1$) to determine when a correlation is considered significant and another threshold parameter ($Th_2$) to determine when two sub-datasets should be merged.


% Pseudo code  \Cref{alg:IdentifyingDesiredDataCategories}. This pseudo code outlines the method for selecting specific sub-categories among the mined sub-categories. The method involves two steps: In step 1, we use a pre-trained model to train each mined sub-category and calculate its validation loss curve, as well as that of the original dataset. We then check if the validation loss curve for a given sub-category is negatively correlated with the validation loss curve for the original dataset. If so, the sub-category is added to a dictionary called "bank categories". In step 2, the algorithm checks the correlation between the validation loss curves for each pair of sub-categories in the bank. If the correlation is positive, the algorithm combines the two sub-categories into a single sub-category.


\begin{algorithm}
\small 
\caption{Identifying Sub-DataSets}
\label{alg:IdentifyingDesiredDataCategories}
\SetAlgoLined
\DontPrintSemicolon 

% \textbf{Input}: Pre-trained model trained on all dataset, mined data categories.

% \textbf{Output}: Non correlate data categories.

\KwIn{ \textbf{generalModel} as pre-trained model trained on general dataset \\ \textbf{allSubCategories} as the mined sub-categories}
\KwOut{Non correlated sub-categories}

\medskip
 
\textbf{Step 1. Inverse-correlation:} \\ 
     allValiCurves = $\{ \}$, CateogryBank = $\{ \}$ \\ 

    \For{ subCategory  {\normalfont in} allSubCategories}
        {   
        \begin{enumerate}
            \item Load \textit{generalModel} into  \textit{model}
            \item Train \textit{model} on \textit{subCategory}
            \item Calculate the validation loss curve for all mined categories, including general dataset validation curve.
            \item \textit{valiCorr} $\gets$ \textbf{corr}(\textit{genlVali}, \textit{SubCatVali})
            \item     \If{  valiCorr $< Th_1 $   }
            { 
                \tcp*{Negative correlation response}
                \hspace{3mm} CateogryBank $\gets$ sub category; \\
                \hspace{3mm} allValiCurves[subCategory]  $ \gets $ all categories validation errors
            }
        \end{enumerate} }
        
   allValiCurves output:
    \begin{tabular}{l l}
    Sub-category1: & all validation curves \footnotesize (trained on \\ sub-category1)\\ 
        ... & ... \\
    Sub-categoryN: & all validation curves \footnotesize (trained on \\ sub-categoryN) \\ 
    \end{tabular} \\ ~ \\ 

\textbf{Step2. Inter-correlation between sub-categories:}
 
\For{subCategory  {\normalfont in}  CategoryBank}
    {
    \begin{enumerate}
        \item  \textit{valiCurvesSubCategory} $\gets$  \textit{allValiCurves}[\textit{subCategory}] 
        % \item  ValSubCategory - validation curve for sub-category trained over sub-category.  
        % \item Get all validation curves over all categories during\\the training over sub-category. 'ValidationCategories[\textbf{SubCategory}]'

        \item  \textit{selfValiCurv} $\gets$  \textit{ValiCurvesSubCategory}[\textit{subCatagorey}]
        
       \item \For{otherValiCurv {\normalfont in} valiCurvesSubCategory}
        {
            \vspace{1mm}  \textit{valiCorr} $\gets$ \textbf{corr}( \textit{selfValiCurv},  \textit{otherValiCurv})
            
            \If{  valiCorr $> Th_2 $ \\ }
            {
               \hspace{-5mm} \tcp*{Positive correlation response}
                
                \textbf{merge}  \textit{otherCategory} with  \textit{subCategory} in \textit{CategoryBank}
            }
        }
    \end{enumerate} 
    } 
  \textbf{Return}  CategoryBank.
\end{algorithm}  



\section{Experiments}
\subsection{SDAT method settings:}
\textbf{Learning rate scheduler:}
Every training phase on a chosen sub-dataset the LR is initialized to $\frac{Base ~ LR}{10}$ and increases till the end of the phase to $10 \times Base ~ LR$. See \cref{S:LR_SCHEDULER}.
\begin{figure}[ht!]
    \begin{center}
        \includegraphics[height=5cm, width=0.85\linewidth]{sec/S_LR_SCHEDULE.png }
    \end{center}  
        \caption{The lr scheduler used in SDAT. $t_{c_i}$ and $t_{end}$ - denotes the starting and ending time index respectively for training over sub-dataset $c_i$.}
	\label{S:LR_SCHEDULER}     
\end{figure}   

\subsection{Models Architecture}
\label{S:model_arch}
In the following we provide a more detailed explanation of the implemented architectures used in the experiments. \\
\textbf{CNN- architecture:}
In the experiments section we provided a range of CNN networks with a varying number of parameters, each architecture consisted of two modules with three Inverted Linear bottlenecks (ILB)\cite{sandler2018mobilenetv2}. By adjusting ILB's "Expansion Factor" hyper parameter, we obtain networks with different number of parameters. The first module operates on the input and produces 12 output channels. Subsequently a pixel shuffle layer transform the output into 3 channels with increased spatial resolution. Then, the second module receives the 3 output channels to produce the final result. \\ 
\textbf{GRL variants architecture:}
We used the GRL-S variant, this architecture receives an RGB image as input, produced by a Matlab demosaic algorithm. As a results the GRL-S model focuses on artifacts removal solution. In order to match our training protocol to be consistent with the other trained networks, as to provide the network a Bayer image as input, we replaced the initial Matlab demosaic with our CNN 16K parameters variant.
\subsection{Qualtitative results}
The following contains supplementary qualitative examples that illustrate the effectiveness of our approach. We compare the outcomes of our SDAT against EDT method.\label{sub:Visual results}\\  
\textbf{EDT in compare to SDAT:} 
We demonstrate a qualitative evaluation  using a 16K parameters model over chosen sub-datasets. We compare between the obtained results using EDT to our SDAT. See Figures \ref{fig:edges} to \ref{fig:zipper}.

\begin{figure*}[ht!]
    \begin{center}
        \includegraphics[height=10cm, width=0.85\linewidth]{S_EDGES_REAL_METRIC.png }
    \end{center}  
        \caption{Results for the 16k CNN, using two training methods - EDT and SDAT over sub-dataset obtained by edge metric.}
	\label{fig:edges}   
\end{figure*}            

\begin{figure*}[ht!]
    \begin{center}
        \includegraphics[height=10cm, width=0.85\linewidth]{S_L1_METRIC.png }
    \end{center}  
        \caption{Results for the 16k CNN, using two training methods - EDT and SDAT over sub-dataset obtained by L1 metric.}
	\label{fig:l1}   
\end{figure*} 
   
\begin{figure*}[ht!]
    \begin{center}
        \includegraphics[height=10cm, width=0.85\linewidth]{S_GRID_METRIC.png }
    \end{center}  
        \caption{Results for the 16k CNN, using two training methods - EDT and SDAT over sub-dataset obtained by Grids metric.}
	\label{fig:grids}   
\end{figure*} 

\begin{figure*}[ht!]
    \begin{center}
        \includegraphics[height=10cm, width=0.85\linewidth]{S_PERCEP_METRIC.png }
    \end{center}  
        \caption{Results for the 16k CNN, using two training methods - EDT and SDAT over sub-dataset obtained by Perceptual metric.}
	\label{fig:perceptual}     
\end{figure*} 

\begin{figure*}[ht!]
    \begin{center}
        \includegraphics[height=10cm, width=0.85\linewidth]{S_ZIPPER_METRIC.png}
    \end{center}  
        \caption{Results for the 16k CNN, using two training methods - EDT and SDAT over sub-dataset obtained by Zipper metric.} 
	\label{fig:zipper}   
\end{figure*}  

\begin{figure*}[h!]
    \begin{center}
        \includegraphics[height=9cm, width=1\linewidth]{kodak_1.png}
    \end{center}    
        \caption{Qualitative result using three models with increasing number of parameters: 16K, 84k and 900k over image 1 in Kodak dataset  \cite{li2008image} .}
	\label{fig:kodak_1}   
\end{figure*}    

 
\begin{figure*}[ht!]
    \begin{center}
        \includegraphics[height=9.5cm, width=1.05\linewidth]{Kodak19.png}
    \end{center}  
        \caption{Qualitative result using three models with increasing number of parameters: 16K, 84k and 900k  over image 19 in Kodak dataset \cite{li2008image} .  }
	\label{fig:kodak_19}   
\end{figure*}   

\iffalse

\begin{figure*}[ht!]
  \centering
  \subfloat[] {\label{fig:subfig1}
    \includegraphics[height=10cm, width=1.05\linewidth]{urban_86.png}
    }
    \hspace{0.4\linewidth}
   \subfloat[] {\label{fig:subfig2}
    \includegraphics[height=10cm, width=1.05\linewidth]{urban_28.png}
  }
  \caption{Qualitative comparison of our method compared to other top methods RNAN and RSTCANet. The RNAN model has 9M parameters, while RSTCANet-B , RSTCANet-S and RSTCANET-L have 0.9M , 3.1M and  7.1M respectively.}
  \label{fig:merged}
\end{figure*}
\fi


\textbf{Qualitative results over different models:}
In the following, we present the outcomes of our training method utilizing three models with varying parameter counts, namely 16k, 84k, and 900k parameters. To construct the 16k and 84k parameter models, we utilized our CNN variants as explained in \cref{S:model_arch}, while for the 900k parameter model, we implemented the RSTCANetB architecture \cite{xing2022residual}. We demonstrated over chosen examples from Kodak dataset \cite{li2008image}. See in Figures \ref{fig:kodak_1} and \ref{fig:kodak_19}.
\iffalse
\subsection{Our method compare to top demosaicing solutions}
Qualitative comparison of our method to other leading demosaicing methods, RNAN \cite{zhang2019residual} and RSTCANet over chosen images from Urban100 \cite{huang2015single}. RNAN comprises 9M parameters,  RSTCANet consists of three different model sizes  B, S, and L  with 0.9M, 3.1M, and 7.1M parameters, respectively. We use our suggested training scheme to train the B model.  Our results, as shown in \Cref{fig:merged}, demonstrate that our method outperforms the RSTCANet-B method, which has the same architecture, and also outperforms RSTCANet-S and RNAN, despite having significantly fewer parameters. When compared to RSTCANet-L, our RSTCANet-B produces similar results, with a slight advantage towards RSTCANet-L in some cases.
\fi
%%%% -*-BibTeX-*-
%%% Do NOT edit. File created by BibTeX with style
%%% ACM-Reference-Format-Journals [18-Jan-2012].

\begin{thebibliography}{52}

%%% ====================================================================
%%% NOTE TO THE USER: you can override these defaults by providing
%%% customized versions of any of these macros before the \bibliography
%%% command.  Each of them MUST provide its own final punctuation,
%%% except for \shownote{}, \showDOI{}, and \showURL{}.  The latter two
%%% do not use final punctuation, in order to avoid confusing it with
%%% the Web address.
%%%
%%% To suppress output of a particular field, define its macro to expand
%%% to an empty string, or better, \unskip, like this:
%%%
%%% \newcommand{\showDOI}[1]{\unskip}   % LaTeX syntax
%%%
%%% \def \showDOI #1{\unskip}           % plain TeX syntax
%%%
%%% ====================================================================

\ifx \showCODEN    \undefined \def \showCODEN     #1{\unskip}     \fi
\ifx \showDOI      \undefined \def \showDOI       #1{#1}\fi
\ifx \showISBNx    \undefined \def \showISBNx     #1{\unskip}     \fi
\ifx \showISBNxiii \undefined \def \showISBNxiii  #1{\unskip}     \fi
\ifx \showISSN     \undefined \def \showISSN      #1{\unskip}     \fi
\ifx \showLCCN     \undefined \def \showLCCN      #1{\unskip}     \fi
\ifx \shownote     \undefined \def \shownote      #1{#1}          \fi
\ifx \showarticletitle \undefined \def \showarticletitle #1{#1}   \fi
\ifx \showURL      \undefined \def \showURL       {\relax}        \fi
% The following commands are used for tagged output and should be
% invisible to TeX
\providecommand\bibfield[2]{#2}
\providecommand\bibinfo[2]{#2}
\providecommand\natexlab[1]{#1}
\providecommand\showeprint[2][]{arXiv:#2}

\bibitem[\protect\citeauthoryear{Albrecht and Stone}{Albrecht and
  Stone}{2017}]%
        {Albrecht2017ReasoningAH}
\bibfield{author}{\bibinfo{person}{Stefano~V. Albrecht} {and}
  \bibinfo{person}{P. Stone}.} \bibinfo{year}{2017}\natexlab{}.
\newblock \showarticletitle{Reasoning about Hypothetical Agent Behaviours and
  their Parameters}. In \bibinfo{booktitle}{\emph{AAMAS}}.
\newblock


\bibitem[\protect\citeauthoryear{Andrejczuk, Berger, Rodriguez-Aguilar, Sierra,
  and Mar{\'\i}n-Puchades}{Andrejczuk et~al\mbox{.}}{2018}]%
        {andrejczuk2018composition}
\bibfield{author}{\bibinfo{person}{Ewa Andrejczuk}, \bibinfo{person}{Rita
  Berger}, \bibinfo{person}{Juan~A Rodriguez-Aguilar}, \bibinfo{person}{Carles
  Sierra}, {and} \bibinfo{person}{V{\'\i}ctor Mar{\'\i}n-Puchades}.}
  \bibinfo{year}{2018}\natexlab{}.
\newblock \showarticletitle{The composition and formation of effective teams:
  computer science meets organizational psychology}.
\newblock \bibinfo{journal}{\emph{The Knowledge Engineering Review}}
  \bibinfo{volume}{33} (\bibinfo{year}{2018}), \bibinfo{pages}{e17}.
\newblock


\bibitem[\protect\citeauthoryear{Arjona-Medina, Gillhofer, Widrich,
  Unterthiner, Brandstetter, and Hochreiter}{Arjona-Medina
  et~al\mbox{.}}{2019}]%
        {arjona2019rudder}
\bibfield{author}{\bibinfo{person}{Jose~A Arjona-Medina},
  \bibinfo{person}{Michael Gillhofer}, \bibinfo{person}{Michael Widrich},
  \bibinfo{person}{Thomas Unterthiner}, \bibinfo{person}{Johannes
  Brandstetter}, {and} \bibinfo{person}{Sepp Hochreiter}.}
  \bibinfo{year}{2019}\natexlab{}.
\newblock \showarticletitle{Rudder: Return decomposition for delayed rewards}.
\newblock \bibinfo{journal}{\emph{NeurIPS}}  \bibinfo{volume}{32}
  (\bibinfo{year}{2019}).
\newblock


\bibitem[\protect\citeauthoryear{Beal, Changder, Norman, and Ramchurn}{Beal
  et~al\mbox{.}}{2020}]%
        {beal2020learning}
\bibfield{author}{\bibinfo{person}{Ryan Beal}, \bibinfo{person}{Narayan
  Changder}, \bibinfo{person}{Timothy Norman}, {and} \bibinfo{person}{Sarvapali
  Ramchurn}.} \bibinfo{year}{2020}\natexlab{}.
\newblock \showarticletitle{Learning the value of teamwork to form efficient
  teams}. In \bibinfo{booktitle}{\emph{Proceedings of the AAAI Conference on
  Artificial Intelligence}}, Vol.~\bibinfo{volume}{34}.
  \bibinfo{pages}{7063--7070}.
\newblock


\bibitem[\protect\citeauthoryear{Beetz, Hoyningen-Huene, Bandouch,
  Kirchlechner, Gedikli, and Maldonado}{Beetz et~al\mbox{.}}{2006}]%
        {beetz2006camera}
\bibfield{author}{\bibinfo{person}{Michael Beetz}, \bibinfo{person}{Nico~v
  Hoyningen-Huene}, \bibinfo{person}{Jan Bandouch}, \bibinfo{person}{Bernhard
  Kirchlechner}, \bibinfo{person}{Suat Gedikli}, {and} \bibinfo{person}{Alexis
  Maldonado}.} \bibinfo{year}{2006}\natexlab{}.
\newblock \showarticletitle{Camera-based observation of football games for
  analyzing multi-agent activities}. In \bibinfo{booktitle}{\emph{Proceedings
  of the fifth international joint conference on Autonomous agents and
  multiagent systems}}. \bibinfo{pages}{42--49}.
\newblock


\bibitem[\protect\citeauthoryear{Bialkowski, Lucey, Carr, Yue, Sridharan, and
  Matthews}{Bialkowski et~al\mbox{.}}{2014}]%
        {bialkowski2014large}
\bibfield{author}{\bibinfo{person}{Alina Bialkowski}, \bibinfo{person}{Patrick
  Lucey}, \bibinfo{person}{Peter Carr}, \bibinfo{person}{Yisong Yue},
  \bibinfo{person}{Sridha Sridharan}, {and} \bibinfo{person}{Iain Matthews}.}
  \bibinfo{year}{2014}\natexlab{}.
\newblock \showarticletitle{Large-scale analysis of soccer matches using
  spatiotemporal tracking data}. In \bibinfo{booktitle}{\emph{2014 IEEE
  international conference on data mining}}. IEEE, \bibinfo{pages}{725--730}.
\newblock


\bibitem[\protect\citeauthoryear{Bouveret and Lang}{Bouveret and Lang}{2014}]%
        {bouveret2014manipulating}
\bibfield{author}{\bibinfo{person}{Sylvain Bouveret} {and}
  \bibinfo{person}{J{\'e}r{\^o}me Lang}.} \bibinfo{year}{2014}\natexlab{}.
\newblock \showarticletitle{Manipulating picking sequences.}. In
  \bibinfo{booktitle}{\emph{ECAI}}, Vol.~\bibinfo{volume}{14}.
  \bibinfo{pages}{141--146}.
\newblock


\bibitem[\protect\citeauthoryear{Brams and Straffin~Jr}{Brams and
  Straffin~Jr}{1979}]%
        {brams1979prisoners}
\bibfield{author}{\bibinfo{person}{Steven~J Brams} {and}
  \bibinfo{person}{Philip~D Straffin~Jr}.} \bibinfo{year}{1979}\natexlab{}.
\newblock \showarticletitle{Prisoners' dilemma and professional sports drafts}.
\newblock \bibinfo{journal}{\emph{The American Mathematical Monthly}}
  \bibinfo{volume}{86}, \bibinfo{number}{2} (\bibinfo{year}{1979}),
  \bibinfo{pages}{80--88}.
\newblock


\bibitem[\protect\citeauthoryear{Bransen and Van~Haaren}{Bransen and
  Van~Haaren}{2020}]%
        {bransen2020player}
\bibfield{author}{\bibinfo{person}{Lotte Bransen} {and} \bibinfo{person}{Jan
  Van~Haaren}.} \bibinfo{year}{2020}\natexlab{}.
\newblock \showarticletitle{Player chemistry: Striving for a perfectly balanced
  soccer team}.
\newblock \bibinfo{journal}{\emph{Sports Analytics Conference}}
  (\bibinfo{year}{2020}).
\newblock


\bibitem[\protect\citeauthoryear{Dafoe, Bachrach, Hadfield, Horvitz, Larson,
  and Graepel}{Dafoe et~al\mbox{.}}{2021}]%
        {DafoeNature2021}
\bibfield{author}{\bibinfo{person}{Allan Dafoe}, \bibinfo{person}{Yoram
  Bachrach}, \bibinfo{person}{Gillian Hadfield}, \bibinfo{person}{Eric
  Horvitz}, \bibinfo{person}{Kate Larson}, {and} \bibinfo{person}{Thore
  Graepel}.} \bibinfo{year}{2021}\natexlab{}.
\newblock \showarticletitle{Cooperative {AI}: machines must learn to find
  common ground}.
\newblock \bibinfo{journal}{\emph{Nature}}  \bibinfo{volume}{593}
  (\bibinfo{year}{2021}), \bibinfo{pages}{33--36}.
\newblock


\bibitem[\protect\citeauthoryear{Derks and Peters}{Derks and Peters}{1993}]%
        {derks1993shapley}
\bibfield{author}{\bibinfo{person}{Jean Derks} {and} \bibinfo{person}{Hans
  Peters}.} \bibinfo{year}{1993}\natexlab{}.
\newblock \showarticletitle{A Shapley value for games with restricted
  coalitions}.
\newblock \bibinfo{journal}{\emph{International Journal of Game Theory}}
  \bibinfo{volume}{21}, \bibinfo{number}{4} (\bibinfo{year}{1993}),
  \bibinfo{pages}{351--360}.
\newblock


\bibitem[\protect\citeauthoryear{Durugkar, Liebman, and Stone}{Durugkar
  et~al\mbox{.}}{2020}]%
        {Durugkar2020BalancingIP}
\bibfield{author}{\bibinfo{person}{Ishan Durugkar}, \bibinfo{person}{E.
  Liebman}, {and} \bibinfo{person}{P. Stone}.} \bibinfo{year}{2020}\natexlab{}.
\newblock \showarticletitle{Balancing Individual Preferences and Shared
  Objectives in Multiagent Reinforcement Learning}. In
  \bibinfo{booktitle}{\emph{IJCAI}}.
\newblock


\bibitem[\protect\citeauthoryear{Elitzur}{Elitzur}{2020}]%
        {elitzur2020data}
\bibfield{author}{\bibinfo{person}{Ramy Elitzur}.}
  \bibinfo{year}{2020}\natexlab{}.
\newblock \showarticletitle{Data analytics effects in major league baseball}.
\newblock \bibinfo{journal}{\emph{Omega}}  \bibinfo{volume}{90}
  (\bibinfo{year}{2020}), \bibinfo{pages}{102001}.
\newblock


\bibitem[\protect\citeauthoryear{Ellis}{Ellis}{1983}]%
        {ellis1983similarities}
\bibfield{author}{\bibinfo{person}{M Ellis}.} \bibinfo{year}{1983}\natexlab{}.
\newblock \showarticletitle{Similarities and differences in games: A system for
  classification}. In \bibinfo{booktitle}{\emph{International association for
  physical education in higher education Conference}}.
\newblock


\bibitem[\protect\citeauthoryear{Fern{\'a}ndez, Bornn, and
  Cervone}{Fern{\'a}ndez et~al\mbox{.}}{2021}]%
        {fernandez2021framework}
\bibfield{author}{\bibinfo{person}{Javier Fern{\'a}ndez}, \bibinfo{person}{Luke
  Bornn}, {and} \bibinfo{person}{Daniel Cervone}.}
  \bibinfo{year}{2021}\natexlab{}.
\newblock \showarticletitle{A framework for the fine-grained evaluation of the
  instantaneous expected value of soccer possessions}.
\newblock \bibinfo{journal}{\emph{Machine Learning}} \bibinfo{volume}{110},
  \bibinfo{number}{6} (\bibinfo{year}{2021}), \bibinfo{pages}{1389--1427}.
\newblock


\bibitem[\protect\citeauthoryear{Fisac, Bronstein, Stefansson, Sadigh, Sastry,
  and Dragan}{Fisac et~al\mbox{.}}{2019}]%
        {fisac2019hierarchical}
\bibfield{author}{\bibinfo{person}{Jaime~F Fisac}, \bibinfo{person}{Eli
  Bronstein}, \bibinfo{person}{Elis Stefansson}, \bibinfo{person}{Dorsa
  Sadigh}, \bibinfo{person}{S~Shankar Sastry}, {and} \bibinfo{person}{Anca~D
  Dragan}.} \bibinfo{year}{2019}\natexlab{}.
\newblock \showarticletitle{Hierarchical game-theoretic planning for autonomous
  vehicles}. In \bibinfo{booktitle}{\emph{ICRA}}. IEEE,
  \bibinfo{pages}{9590--9596}.
\newblock


\bibitem[\protect\citeauthoryear{Garner, Humphrey, and Simkins}{Garner
  et~al\mbox{.}}{2016}]%
        {garner2016business}
\bibfield{author}{\bibinfo{person}{Jacqueline Garner},
  \bibinfo{person}{Phillip~R Humphrey}, {and} \bibinfo{person}{Betty Simkins}.}
  \bibinfo{year}{2016}\natexlab{}.
\newblock \showarticletitle{The business of sport and the sport of business: A
  review of the compensation literature in finance and sports}.
\newblock \bibinfo{journal}{\emph{International Review of Financial Analysis}}
  \bibinfo{volume}{47} (\bibinfo{year}{2016}), \bibinfo{pages}{197--204}.
\newblock


\bibitem[\protect\citeauthoryear{Goes, Kempe, Meerhoff, and Lemmink}{Goes
  et~al\mbox{.}}{2019}]%
        {goes2019not}
\bibfield{author}{\bibinfo{person}{Floris~R Goes}, \bibinfo{person}{Matthias
  Kempe}, \bibinfo{person}{Laurentius~A Meerhoff}, {and}
  \bibinfo{person}{Koen~APM Lemmink}.} \bibinfo{year}{2019}\natexlab{}.
\newblock \showarticletitle{Not every pass can be an assist: a data-driven
  model to measure pass effectiveness in professional soccer matches}.
\newblock \bibinfo{journal}{\emph{Big data}} \bibinfo{volume}{7},
  \bibinfo{number}{1} (\bibinfo{year}{2019}), \bibinfo{pages}{57--70}.
\newblock


\bibitem[\protect\citeauthoryear{Hu, Xie, Liang, and Chang}{Hu
  et~al\mbox{.}}{2022}]%
        {hu2022policy}
\bibfield{author}{\bibinfo{person}{Siyi Hu}, \bibinfo{person}{Chuanlong Xie},
  \bibinfo{person}{Xiaodan Liang}, {and} \bibinfo{person}{Xiaojun Chang}.}
  \bibinfo{year}{2022}\natexlab{}.
\newblock \showarticletitle{Policy diagnosis via measuring role diversity in
  cooperative multi-agent {RL}}. In \bibinfo{booktitle}{\emph{ICML}}.
  \bibinfo{pages}{9041--9071}.
\newblock


\bibitem[\protect\citeauthoryear{Le, Yue, Carr, and Lucey}{Le
  et~al\mbox{.}}{2017}]%
        {le2017coordinated}
\bibfield{author}{\bibinfo{person}{Hoang~M Le}, \bibinfo{person}{Yisong Yue},
  \bibinfo{person}{Peter Carr}, {and} \bibinfo{person}{Patrick Lucey}.}
  \bibinfo{year}{2017}\natexlab{}.
\newblock \showarticletitle{Coordinated multi-agent imitation learning}. In
  \bibinfo{booktitle}{\emph{International Conference on Machine Learning}}.
  PMLR, \bibinfo{pages}{1995--2003}.
\newblock


\bibitem[\protect\citeauthoryear{Ledezma, Aler, Sanchis, and Borrajo}{Ledezma
  et~al\mbox{.}}{2009}]%
        {ledezma2009ombo}
\bibfield{author}{\bibinfo{person}{Agapito Ledezma}, \bibinfo{person}{Ricardo
  Aler}, \bibinfo{person}{Araceli Sanchis}, {and} \bibinfo{person}{Daniel
  Borrajo}.} \bibinfo{year}{2009}\natexlab{}.
\newblock \showarticletitle{OMBO: An opponent modeling approach}.
\newblock \bibinfo{journal}{\emph{{AI} Communications}} \bibinfo{volume}{22},
  \bibinfo{number}{1} (\bibinfo{year}{2009}), \bibinfo{pages}{21--35}.
\newblock


\bibitem[\protect\citeauthoryear{Lewis}{Lewis}{2004}]%
        {lewis2004moneyball}
\bibfield{author}{\bibinfo{person}{Michael Lewis}.}
  \bibinfo{year}{2004}\natexlab{}.
\newblock \bibinfo{booktitle}{\emph{Moneyball: The art of winning an unfair
  game}}.
\newblock \bibinfo{publisher}{WW Norton \& Company}.
\newblock


\bibitem[\protect\citeauthoryear{Liemhetcharat and Luo}{Liemhetcharat and
  Luo}{2015}]%
        {liemhetcharat2015applying}
\bibfield{author}{\bibinfo{person}{Somchaya Liemhetcharat} {and}
  \bibinfo{person}{Yicheng Luo}.} \bibinfo{year}{2015}\natexlab{}.
\newblock \showarticletitle{Applying the Synergy Graph Model to Human
  Basketball.}. In \bibinfo{booktitle}{\emph{AAMAS}}.
  \bibinfo{pages}{1695--1696}.
\newblock


\bibitem[\protect\citeauthoryear{Liu, Schulte, Poupart, Rudd, and Javan}{Liu
  et~al\mbox{.}}{2020}]%
        {liu2020learning}
\bibfield{author}{\bibinfo{person}{Guiliang Liu}, \bibinfo{person}{Oliver
  Schulte}, \bibinfo{person}{Pascal Poupart}, \bibinfo{person}{Mike Rudd},
  {and} \bibinfo{person}{Mehrsan Javan}.} \bibinfo{year}{2020}\natexlab{}.
\newblock \showarticletitle{Learning agent representations for ice hockey}.
\newblock \bibinfo{journal}{\emph{Advances in Neural Information Processing
  Systems}}  \bibinfo{volume}{33} (\bibinfo{year}{2020}),
  \bibinfo{pages}{18704--18715}.
\newblock


\bibitem[\protect\citeauthoryear{Ljung, Carlsson, and Lambrix}{Ljung
  et~al\mbox{.}}{2018}]%
        {Ljung2018PlayerPV}
\bibfield{author}{\bibinfo{person}{Dennis Ljung}, \bibinfo{person}{Niklas
  Carlsson}, {and} \bibinfo{person}{P. Lambrix}.}
  \bibinfo{year}{2018}\natexlab{}.
\newblock \showarticletitle{Player Pairs Valuation in Ice Hockey}. In
  \bibinfo{booktitle}{\emph{MLSA@PKDD/ECML}}.
\newblock


\bibitem[\protect\citeauthoryear{Lucey, Bialkowski, Carr, Foote, and
  Matthews}{Lucey et~al\mbox{.}}{2012}]%
        {lucey2012characterizing}
\bibfield{author}{\bibinfo{person}{Patrick Lucey}, \bibinfo{person}{Alina
  Bialkowski}, \bibinfo{person}{Peter Carr}, \bibinfo{person}{Eric Foote},
  {and} \bibinfo{person}{Iain Matthews}.} \bibinfo{year}{2012}\natexlab{}.
\newblock \showarticletitle{Characterizing multi-agent team behavior from
  partial team tracings: Evidence from the english premier league}. In
  \bibinfo{booktitle}{\emph{Proceedings of the AAAI Conference on Artificial
  Intelligence}}, Vol.~\bibinfo{volume}{26}. \bibinfo{pages}{1387--1393}.
\newblock


\bibitem[\protect\citeauthoryear{Pourmehr and Dadkhah}{Pourmehr and
  Dadkhah}{2011}]%
        {pourmehr2011overview}
\bibfield{author}{\bibinfo{person}{Shokoofeh Pourmehr} {and}
  \bibinfo{person}{Chitra Dadkhah}.} \bibinfo{year}{2011}\natexlab{}.
\newblock \showarticletitle{An overview on opponent modeling in RoboCup soccer
  simulation 2D}.
\newblock \bibinfo{journal}{\emph{Robot Soccer World Cup}}
  (\bibinfo{year}{2011}), \bibinfo{pages}{402--414}.
\newblock


\bibitem[\protect\citeauthoryear{Raabe, Nabben, and Memmert}{Raabe
  et~al\mbox{.}}{2022}]%
        {raabe2022graph}
\bibfield{author}{\bibinfo{person}{Dominik Raabe}, \bibinfo{person}{Reinhard
  Nabben}, {and} \bibinfo{person}{Daniel Memmert}.}
  \bibinfo{year}{2022}\natexlab{}.
\newblock \showarticletitle{Graph representations for the analysis of
  multi-agent spatiotemporal sports data}.
\newblock \bibinfo{journal}{\emph{Applied Intelligence}}
  (\bibinfo{year}{2022}), \bibinfo{pages}{1--21}.
\newblock


\bibitem[\protect\citeauthoryear{Radke, Brecht, and Radke}{Radke
  et~al\mbox{.}}{2022a}]%
        {radke2022identifying}
\bibfield{author}{\bibinfo{person}{David Radke}, \bibinfo{person}{Tim Brecht},
  {and} \bibinfo{person}{Daniel Radke}.} \bibinfo{year}{2022}\natexlab{a}.
\newblock \showarticletitle{Identifying Completed Pass Types and Improving
  Passing Lane Models}. In \bibinfo{booktitle}{\emph{Link{\"o}ping Hockey
  Analytics Conference}}. \bibinfo{pages}{71--86}.
\newblock


\bibitem[\protect\citeauthoryear{Radke, Larson, and Brecht}{Radke
  et~al\mbox{.}}{2022b}]%
        {Radke2022Exploring}
\bibfield{author}{\bibinfo{person}{David Radke}, \bibinfo{person}{Kate Larson},
  {and} \bibinfo{person}{Tim Brecht}.} \bibinfo{year}{2022}\natexlab{b}.
\newblock \showarticletitle{Exploring the Benefits of Teams in Multiagent
  Learning}. In \bibinfo{booktitle}{\emph{IJCAI}}.
\newblock


\bibitem[\protect\citeauthoryear{Radke, Larson, and Brecht}{Radke
  et~al\mbox{.}}{2022c}]%
        {radke2022importance}
\bibfield{author}{\bibinfo{person}{David Radke}, \bibinfo{person}{Kate Larson},
  {and} \bibinfo{person}{Tim Brecht}.} \bibinfo{year}{2022}\natexlab{c}.
\newblock \showarticletitle{The Importance of Credo in Multiagent Learning}.
\newblock \bibinfo{journal}{\emph{ALA Workshop at AAMAS}}
  (\bibinfo{year}{2022}).
\newblock


\bibitem[\protect\citeauthoryear{Radke, Radke, Brecht, and Pawelczyk}{Radke
  et~al\mbox{.}}{2021}]%
        {Radke2021Passing}
\bibfield{author}{\bibinfo{person}{D.~T. Radke}, \bibinfo{person}{D.~L. Radke},
  \bibinfo{person}{T. Brecht}, {and} \bibinfo{person}{A. Pawelczyk}.}
  \bibinfo{year}{2021}\natexlab{}.
\newblock \showarticletitle{Passing and Pressure Metrics in Ice Hockey}.
\newblock \bibinfo{journal}{\emph{Workshop of AI for Sports Analytics}}
  (\bibinfo{year}{2021}).
\newblock


\bibitem[\protect\citeauthoryear{Rahimian and Toka}{Rahimian and Toka}{2022}]%
        {rahimian2022optical}
\bibfield{author}{\bibinfo{person}{Pegah Rahimian} {and}
  \bibinfo{person}{Laszlo Toka}.} \bibinfo{year}{2022}\natexlab{}.
\newblock \showarticletitle{Optical tracking in team sports}.
\newblock \bibinfo{journal}{\emph{Journal of Quantitative Analysis in Sports}}
  \bibinfo{volume}{18}, \bibinfo{number}{1} (\bibinfo{year}{2022}),
  \bibinfo{pages}{35--57}.
\newblock


\bibitem[\protect\citeauthoryear{Rahwan, Michalak, Wooldridge, and
  Jennings}{Rahwan et~al\mbox{.}}{2015}]%
        {rahwan2015coalition}
\bibfield{author}{\bibinfo{person}{Talal Rahwan}, \bibinfo{person}{Tomasz~P
  Michalak}, \bibinfo{person}{Michael Wooldridge}, {and}
  \bibinfo{person}{Nicholas~R Jennings}.} \bibinfo{year}{2015}\natexlab{}.
\newblock \showarticletitle{Coalition structure generation: A survey}.
\newblock \bibinfo{journal}{\emph{Artificial Intelligence}}
  \bibinfo{volume}{229} (\bibinfo{year}{2015}), \bibinfo{pages}{139--174}.
\newblock


\bibitem[\protect\citeauthoryear{Rashid, Samvelyan, Schroeder, Farquhar,
  Foerster, and Whiteson}{Rashid et~al\mbox{.}}{2018}]%
        {rashid2018qmix}
\bibfield{author}{\bibinfo{person}{Tabish Rashid}, \bibinfo{person}{Mikayel
  Samvelyan}, \bibinfo{person}{Christian Schroeder}, \bibinfo{person}{Gregory
  Farquhar}, \bibinfo{person}{Jakob Foerster}, {and} \bibinfo{person}{Shimon
  Whiteson}.} \bibinfo{year}{2018}\natexlab{}.
\newblock \showarticletitle{Qmix: Monotonic value function factorisation for
  deep multi-agent reinforcement learning}. In
  \bibinfo{booktitle}{\emph{ICML}}. \bibinfo{pages}{4295--4304}.
\newblock


\bibitem[\protect\citeauthoryear{Rein and Memmert}{Rein and Memmert}{2016}]%
        {rein2016big}
\bibfield{author}{\bibinfo{person}{Robert Rein} {and} \bibinfo{person}{Daniel
  Memmert}.} \bibinfo{year}{2016}\natexlab{}.
\newblock \showarticletitle{Big data and tactical analysis in elite soccer:
  future challenges and opportunities for sports science}.
\newblock \bibinfo{journal}{\emph{SpringerPlus}} \bibinfo{volume}{5},
  \bibinfo{number}{1} (\bibinfo{year}{2016}), \bibinfo{pages}{1--13}.
\newblock


\bibitem[\protect\citeauthoryear{Ritchie, Harell, and Shreeves}{Ritchie
  et~al\mbox{.}}{2022}]%
        {ritchie2022pass}
\bibfield{author}{\bibinfo{person}{Robyn Ritchie}, \bibinfo{person}{Alon
  Harell}, {and} \bibinfo{person}{Phillip Shreeves}.}
  \bibinfo{year}{2022}\natexlab{}.
\newblock \showarticletitle{Pass Evaluation in Women's Olympic Ice Hockey}. In
  \bibinfo{booktitle}{\emph{Proceedings of the 5th International ACM Workshop
  on Multimedia Content Analysis in Sports}}. \bibinfo{pages}{65--73}.
\newblock


\bibitem[\protect\citeauthoryear{Sampaio, McGarry, Calleja-Gonz{\'a}lez,
  Jim{\'e}nez~S{\'a}iz, Schelling i~del Alc{\'a}zar, and Balciunas}{Sampaio
  et~al\mbox{.}}{2015}]%
        {sampaio2015exploring}
\bibfield{author}{\bibinfo{person}{Jaime Sampaio}, \bibinfo{person}{Tim
  McGarry}, \bibinfo{person}{Julio Calleja-Gonz{\'a}lez},
  \bibinfo{person}{Sergio Jim{\'e}nez~S{\'a}iz}, \bibinfo{person}{Xavi
  Schelling i~del Alc{\'a}zar}, {and} \bibinfo{person}{Mindaugas Balciunas}.}
  \bibinfo{year}{2015}\natexlab{}.
\newblock \showarticletitle{Exploring game performance in the National
  Basketball Association using player tracking data}.
\newblock \bibinfo{journal}{\emph{PloS one}} \bibinfo{volume}{10},
  \bibinfo{number}{7} (\bibinfo{year}{2015}), \bibinfo{pages}{e0132894}.
\newblock


\bibitem[\protect\citeauthoryear{Santos, Santos, Pacheco, and Levin}{Santos
  et~al\mbox{.}}{2021}]%
        {Santos2021SocialNI}
\bibfield{author}{\bibinfo{person}{F. Santos}, \bibinfo{person}{F.~C. Santos},
  \bibinfo{person}{J. Pacheco}, {and} \bibinfo{person}{S. Levin}.}
  \bibinfo{year}{2021}\natexlab{}.
\newblock \showarticletitle{Social Network Interventions to Prevent
  Reciprocity-driven Polarization}. In \bibinfo{booktitle}{\emph{AAMAS}}.
\newblock


\bibitem[\protect\citeauthoryear{Schr{\"o}der, Hoey, and Rogers}{Schr{\"o}der
  et~al\mbox{.}}{2016}]%
        {schroder2016modeling}
\bibfield{author}{\bibinfo{person}{Tobias Schr{\"o}der}, \bibinfo{person}{Jesse
  Hoey}, {and} \bibinfo{person}{Kimberly~B Rogers}.}
  \bibinfo{year}{2016}\natexlab{}.
\newblock \showarticletitle{Modeling dynamic identities and uncertainty in
  social interactions: Bayesian affect control theory}.
\newblock \bibinfo{journal}{\emph{American Sociological Review}}
  \bibinfo{volume}{81}, \bibinfo{number}{4} (\bibinfo{year}{2016}),
  \bibinfo{pages}{828--855}.
\newblock


\bibitem[\protect\citeauthoryear{Schuckers}{Schuckers}{2011}]%
        {schuckers2011s}
\bibfield{author}{\bibinfo{person}{Michael~E Schuckers}.}
  \bibinfo{year}{2011}\natexlab{}.
\newblock \showarticletitle{What's An NHL Draft Pick Worth? A Value Pick Chart
  for the National Hockey League}.
\newblock \bibinfo{journal}{\emph{Statistical Sports Consulting}}
  (\bibinfo{year}{2011}).
\newblock


\bibitem[\protect\citeauthoryear{Schulte, Khademi, Gholami, Zhao, Javan, and
  Desaulniers}{Schulte et~al\mbox{.}}{2017}]%
        {schulte2017markov}
\bibfield{author}{\bibinfo{person}{Oliver Schulte}, \bibinfo{person}{Mahmoud
  Khademi}, \bibinfo{person}{Sajjad Gholami}, \bibinfo{person}{Zeyu Zhao},
  \bibinfo{person}{Mehrsan Javan}, {and} \bibinfo{person}{Philippe
  Desaulniers}.} \bibinfo{year}{2017}\natexlab{}.
\newblock \showarticletitle{A Markov Game model for valuing actions, locations,
  and team performance in ice hockey}.
\newblock \bibinfo{journal}{\emph{Data Mining and Knowledge Discovery}}
  \bibinfo{volume}{31}, \bibinfo{number}{6} (\bibinfo{year}{2017}),
  \bibinfo{pages}{1735--1757}.
\newblock


\bibitem[\protect\citeauthoryear{Schwind, Demirovic, Inoue, and
  Lagniez}{Schwind et~al\mbox{.}}{2021}]%
        {schwind2021partial}
\bibfield{author}{\bibinfo{person}{Nicolas Schwind}, \bibinfo{person}{Emir
  Demirovic}, \bibinfo{person}{Katsumi Inoue}, {and}
  \bibinfo{person}{Jean-Marie Lagniez}.} \bibinfo{year}{2021}\natexlab{}.
\newblock \showarticletitle{Partial Robustness in Team Formation: Bridging the
  Gap between Robustness and Resilience.}. In
  \bibinfo{booktitle}{\emph{AAMAS}}, Vol.~\bibinfo{volume}{21}.
  \bibinfo{pages}{20th}.
\newblock


\bibitem[\protect\citeauthoryear{Simon}{Simon}{1990}]%
        {simon1990bounded}
\bibfield{author}{\bibinfo{person}{Herbert~A Simon}.}
  \bibinfo{year}{1990}\natexlab{}.
\newblock \showarticletitle{Bounded rationality}.
\newblock In \bibinfo{booktitle}{\emph{Utility and probability}}.
  \bibinfo{publisher}{Springer}, \bibinfo{pages}{15--18}.
\newblock


\bibitem[\protect\citeauthoryear{Spearman}{Spearman}{2018}]%
        {spearman2018beyond}
\bibfield{author}{\bibinfo{person}{William Spearman}.}
  \bibinfo{year}{2018}\natexlab{}.
\newblock \showarticletitle{Beyond expected goals}. In
  \bibinfo{booktitle}{\emph{Proceedings of the 12th MIT sloan sports analytics
  conference}}. \bibinfo{pages}{1--17}.
\newblock


\bibitem[\protect\citeauthoryear{Stone, Riley, and Veloso}{Stone
  et~al\mbox{.}}{2000}]%
        {stone2000defining}
\bibfield{author}{\bibinfo{person}{Peter Stone}, \bibinfo{person}{Patrick
  Riley}, {and} \bibinfo{person}{Manuela Veloso}.}
  \bibinfo{year}{2000}\natexlab{}.
\newblock \showarticletitle{Defining and using ideal teammate and opponent
  agent models}. In \bibinfo{booktitle}{\emph{AAAI/IAAI}}.
  \bibinfo{pages}{1040--1045}.
\newblock


\bibitem[\protect\citeauthoryear{Tuyls, Omidshafiei, Muller, Wang, Connor,
  Hennes, Graham, Spearman, Waskett, Steel, et~al\mbox{.}}{Tuyls
  et~al\mbox{.}}{2021}]%
        {tuyls2021game}
\bibfield{author}{\bibinfo{person}{Karl Tuyls}, \bibinfo{person}{Shayegan
  Omidshafiei}, \bibinfo{person}{Paul Muller}, \bibinfo{person}{Zhe Wang},
  \bibinfo{person}{Jerome Connor}, \bibinfo{person}{Daniel Hennes},
  \bibinfo{person}{Ian Graham}, \bibinfo{person}{William Spearman},
  \bibinfo{person}{Tim Waskett}, \bibinfo{person}{Dafydd Steel},
  {et~al\mbox{.}}} \bibinfo{year}{2021}\natexlab{}.
\newblock \showarticletitle{Game Plan: What {AI} can do for Football, and What
  Football can do for AI}.
\newblock \bibinfo{journal}{\emph{Journal of Artificial Intelligence Research}}
   \bibinfo{volume}{71} (\bibinfo{year}{2021}), \bibinfo{pages}{41--88}.
\newblock


\bibitem[\protect\citeauthoryear{Van Der~Hoek, Jamroga, and Wooldridge}{Van
  Der~Hoek et~al\mbox{.}}{2005}]%
        {van2005logic}
\bibfield{author}{\bibinfo{person}{Wiebe Van Der~Hoek},
  \bibinfo{person}{Wojciech Jamroga}, {and} \bibinfo{person}{Michael
  Wooldridge}.} \bibinfo{year}{2005}\natexlab{}.
\newblock \showarticletitle{A logic for strategic reasoning}. In
  \bibinfo{booktitle}{\emph{Proceedings of the fourth international joint
  conference on Autonomous agents and multiagent systems}}.
  \bibinfo{pages}{157--164}.
\newblock


\bibitem[\protect\citeauthoryear{Vats, Fani, Clausi, and Zelek}{Vats
  et~al\mbox{.}}{2022}]%
        {vats2022evaluating}
\bibfield{author}{\bibinfo{person}{Kanav Vats}, \bibinfo{person}{Mehrnaz Fani},
  \bibinfo{person}{David~A Clausi}, {and} \bibinfo{person}{John~S Zelek}.}
  \bibinfo{year}{2022}\natexlab{}.
\newblock \showarticletitle{Evaluating deep tracking models for player tracking
  in broadcast ice hockey video}.
\newblock \bibinfo{journal}{\emph{arXiv preprint arXiv:2205.10949}}
  (\bibinfo{year}{2022}).
\newblock


\bibitem[\protect\citeauthoryear{Visser, Dr{\"u}cker, H{\"u}bner, Schmidt, and
  Weland}{Visser et~al\mbox{.}}{2000}]%
        {visser2000recognizing}
\bibfield{author}{\bibinfo{person}{Ubbo Visser}, \bibinfo{person}{Christian
  Dr{\"u}cker}, \bibinfo{person}{Sebastian H{\"u}bner}, \bibinfo{person}{Esko
  Schmidt}, {and} \bibinfo{person}{Hans-Georg Weland}.}
  \bibinfo{year}{2000}\natexlab{}.
\newblock \showarticletitle{Recognizing formations in opponent teams}. In
  \bibinfo{booktitle}{\emph{Robot Soccer World Cup}}. Springer,
  \bibinfo{pages}{391--396}.
\newblock


\bibitem[\protect\citeauthoryear{Williamson and Cox}{Williamson and
  Cox}{2014}]%
        {williamson2014distributed}
\bibfield{author}{\bibinfo{person}{Kellie Williamson} {and}
  \bibinfo{person}{Rochelle Cox}.} \bibinfo{year}{2014}\natexlab{}.
\newblock \showarticletitle{Distributed cognition in sports teams: Explaining
  successful and expert performance}.
\newblock \bibinfo{journal}{\emph{Educational Philosophy and Theory}}
  \bibinfo{volume}{46}, \bibinfo{number}{6} (\bibinfo{year}{2014}),
  \bibinfo{pages}{640--654}.
\newblock


\bibitem[\protect\citeauthoryear{Yan, Kroer, and Peysakhovich}{Yan
  et~al\mbox{.}}{2020}]%
        {Yan2020EvaluatingAR}
\bibfield{author}{\bibinfo{person}{Tom Yan}, \bibinfo{person}{Christian Kroer},
  {and} \bibinfo{person}{A. Peysakhovich}.} \bibinfo{year}{2020}\natexlab{}.
\newblock \showarticletitle{Evaluating and Rewarding Teamwork Using Cooperative
  Game Abstractions}.
\newblock \bibinfo{journal}{\emph{NeurIPS}} (\bibinfo{year}{2020}).
\newblock


\end{thebibliography}


