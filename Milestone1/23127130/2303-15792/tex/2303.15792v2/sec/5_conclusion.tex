\section{Conclusion} 
In this paper, we addressed the challenge of dataset inductive bias for the image demosaic task and proposed SDAT, a novel training method to improve the performance of a model. The method involves two main steps: defining and identifying sub-datasets beneficial to model convergence and an alternating training scheme. 
SDAT demonstrated an improved performance for image demosaicing over standard training methods and achieved state-of-the-art results on three highly popular benchmarks. Our suggestion also effectively utilized the model's capacity, surpassing recent relevant works on low-capacity demosaicing models across all benchmarks using fewer parameters. This result is crucial for edge devices. The method's success also demonstrates the importance of considering the dataset structure in optimizing a model's training process. Our findings suggest that our proposed method can be applied to other image restoration tasks and can be used as a trigger for further research in this field.
  

%  In this paper, we addressed the challenge of inductive bias in image demosaic task and proposed a novel training method to improve the generalization of a model. Our method involves two main steps: defining and identifying sub-categories beneficial to model convergence and a custom training scheme. 
% Our method demonstrated an improved performance over standard training methods and achieved state-of-the-art results on several benchmarks using smaller models. Our method also effectively utilized the model's capacity, surpassing recent relevant works across all benchmarks using fewer parameters. This result is crucial for edge devices, where the trend is towards low-capacity models.

% \section{Conclusion}  
% In conclusion, our proposed method effectively addresses the challenge of inductive bias in image demosaicing task. Our method's success demonstrates the importance of considering the dataset structure in optimizing a model's training process. Our findings suggest that our proposed method can be applied to other image restoration tasks and can be used as a trigger for further research in this field.
