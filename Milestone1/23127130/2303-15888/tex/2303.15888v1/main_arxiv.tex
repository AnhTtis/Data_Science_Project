%%%%%%%% ICML 2022 EXAMPLE LATEX SUBMISSION FILE %%%%%%%%%%%%%%%%%

\documentclass[nohyperref]{article}

% Recommended, but optional, packages for figures and better typesetting:
\usepackage{microtype}

% hyperref makes hyperlinks in the resulting PDF.
% If your build breaks (sometimes temporarily if a hyperlink spans a page)
% please comment out the following usepackage line and replace
% \usepackage{icml2022} with \usepackage[nohyperref]{icml2022} above.
\usepackage{hyperref}


% Attempt to make hyperref and algorithmic work together better:
\newcommand{\theHalgorithm}{\arabic{algorithm}}

% Use the following line for the initial blind version submitted for review:
% \usepackage{icml2023}

% If accepted, instead use the following line for the camera-ready submission:
\usepackage[accepted]{arxiv}

% For theorems and such
\usepackage{amsmath}
\usepackage{amssymb}
\usepackage{mathtools}
\usepackage{amsthm}

% if you use cleveref..
\usepackage[capitalize,noabbrev]{cleveref}

% user packages
% \usepackage{tikz}
\usepackage{graphicx}
\usepackage{amsmath}
\usepackage{amssymb}
\usepackage{booktabs}
 
\usepackage[utf8]{inputenc} % allow utf-8 input
\usepackage[T1]{fontenc}    % use 8-bit T1 fonts
\usepackage{url}            % simple URL typesetting
\usepackage{amsfonts}       % blackboard math symbols
\usepackage{booktabs} % for professional tables
\usepackage{nicefrac}       % compact symbols for 1/2, etc.
\usepackage{microtype}      % microtypography
\usepackage[export]{adjustbox}
\usepackage{bm}
\usepackage{xcolor}
\usepackage{amsmath}
\usepackage{common}
\usepackage{caption}
\usepackage{subcaption}
\usepackage{multirow}
% \usepackage{algorithm}
% \usepackage{algpseudocode}
\usepackage{nicefrac}
\usepackage{graphics}
\usepackage{xspace}
\usepackage{makecell}

% CUSTOM COMMANDS
\newcommand{\antinline}[1]{{\color{blue}ANT: #1}}
\newcommand{\JW}[1]{{\color{red}JW: #1}}
\newcommand{\rev}[1]{{\color{blue}#1}}

% MATH COMMANDS
\newcommand{\exalgo}{{\cal{A}}}
\newcommand{\lexp}{{\mathit{e}}}
\newcommand{\data}{{\cal{D}}}
\newcommand{\classexp}{{\cal{Y}}}
\newcommand{\Natural}{\mathbb{N}}

\newcommand{\pcl}{\theta^{CL}}
\newcommand{\pexp}{\theta^{SC}}
\newcommand{\hexp}{\bm{h}^{SC}}

\newcommand{\xood}{\bm{x}_{ood}}
\newcommand{\yood}{y_{ood}}
\newcommand{\Dood}{\data_{ood}}

\newcommand{\fexp}{f^{SC}}
\newcommand{\fcl}{f^{CL}}
\newcommand{\fexpert}{f^{SC}}

\newcommand{\ce}{{\cal{L}}^{CE}}
\newcommand{\kd}{{\cal{L}}^{KD}}
\newcommand{\kdce}{{\cal{L}}^{KD-CE}}
\newcommand{\loss}{{\cal{L}}}

% \newcommand{\mustd}[2]{$#1 \scriptstyle{ \pm  #2}$}
\newcommand{\emptycell}{ \multicolumn{1}{c}{--} }

%%%%%%%%%%%%%%%%%%%%%%%%%%%%%%%%
% THEOREMS
%%%%%%%%%%%%%%%%%%%%%%%%%%%%%%%%
\theoremstyle{plain}
\newtheorem{theorem}{Theorem}[section]
\newtheorem{proposition}[theorem]{Proposition}
\newtheorem{lemma}[theorem]{Lemma}
\newtheorem{corollary}[theorem]{Corollary}
\theoremstyle{definition}
\newtheorem{definition}[theorem]{Definition}
\newtheorem{assumption}[theorem]{Assumption}
\theoremstyle{remark}
\newtheorem{remark}[theorem]{Remark}

% Todonotes is useful during development; simply uncomment the next line
%    and comment out the line below the next line to turn off comments
%\usepackage[disable,textsize=tiny]{todonotes}
% \usepackage[textsize=tiny]{todonotes}


% The \icmltitle you define below is probably too long as a header.
% Therefore, a short form for the running title is supplied here:
\icmltitlerunning{Projected Latent Distillation for Data-Agnostic Consolidation in Distributed Continual Learning}

\begin{document}

\twocolumn[
\icmltitle{Projected Latent Distillation for Data-Agnostic Consolidation in \\ Distributed Continual Learning}
%Projected Latent Distillation for Continual Learning with Self-Centered Devices
% Achieving Forward Transfer by Conintually Training Self-Centered Devices
% COntinual learning in with Self-Centered Devices
%
%\icmltitle{Submission and Formatting Instructions for \\
%           International Conference on Machine Learning (ICML 2022)}

% It is OKAY to include author information, even for blind
% submissions: the style file will automatically remove it for you
% unless you've provided the [accepted] option to the icml2022
% package.

% List of affiliations: The first argument should be a (short)
% identifier you will use later to specify author affiliations
% Academic affiliations should list Department, University, City, Region, Country
% Industry affiliations should list Company, City, Region, Country

% You can specify symbols, otherwise they are numbered in order.
% Ideally, you should not use this facility. Affiliations will be numbered
% in order of appearance and this is the preferred way.
\icmlsetsymbol{equal}{*}

\begin{icmlauthorlist}
\icmlauthor{Antonio Carta}{unipi}
\icmlauthor{Andrea Cossu}{sns}
\icmlauthor{Vincenzo Lomonaco}{unipi}
\icmlauthor{Davide Bacciu}{unipi}
\icmlauthor{Joost van de Weijer}{cvc}
%\icmlauthor{}{sch}
%\icmlauthor{}{sch}
\end{icmlauthorlist}

\icmlaffiliation{unipi}{Department of Computer Science, University of Pisa, Pisa, Italy}
\icmlaffiliation{sns}{Scuola Normale Superiore, Pisa, Italy}
\icmlaffiliation{cvc}{Computer Vision Center, Barcelona, Spain}

\icmlcorrespondingauthor{Antonio Carta}{antonio.carta@unipi.it}

% You may vide any keywords that you
% find helpful for describing your paper; these are used to populate
% the "keywords" metadata in the PDF but will not be shown in the document
\icmlkeywords{Continual Learning; Lifelong Learning; Communication-efficient learning}
\vskip 0.3in
]

% this must go after the closing bracket ] following \twocolumn[ ...

% This command actually creates the footnote in the first column
% listing the affiliations and the copyright notice.
% The command takes one argument, which is text to display at the start of the footnote.
% The \icmlEqualContribution command is standard text for equal contribution.
% Remove it (just {}) if you do not need this facility.

\printAffiliationsAndNotice{}  % leave blank if no need to mention equal contribution
% \printAffiliationsAndNotice{\icmlEqualContribution} % otherwise use the standard text.

\begin{abstract}
    Distributed learning on the edge often comprises 
    \emph{self-centered devices} (SCD) which learn local tasks independently and are unwilling to contribute to the performance of other SDCs. \emph{How do we achieve forward transfer at zero cost for the single SCDs?} We formalize this problem as a \emph{Distributed Continual Learning} scenario, where SCD adapt to local tasks and a CL model consolidates the knowledge from the resulting stream of models without looking at the SCD's private data. 
    Unfortunately, current CL methods are not directly applicable to this scenario.
    We propose Data-Agnostic Consolidation (DAC), a novel double knowledge distillation method that consolidates the stream of SC models without using the original data. DAC performs distillation in the latent space via a novel Projected Latent Distillation loss. Experimental results show that DAC enables forward transfer between SCDs and reaches state-of-the-art accuracy on Split CIFAR100, CORe50 and Split TinyImageNet, both in reharsal-free and distributed CL scenarios. Somewhat surprisingly, even a single out-of-distribution image is sufficient as the only source of data during consolidation.
\end{abstract}
    

%%%%%%%%%%%%%%%%%%%%%%%%%%%%
% Introduction
%%%%%%%%%%%%%%%%%%%%%%%%%%%%

% \section{Introduction}

The increasing complexity of source code poses a key challenge to the reliability of large-scale software systems. Software bugs in these systems can lead to safety issues~\cite{bug_safety} for users around the world as well as cause non-negligible financial losses~\cite{bug_loss}. As such, developers have to spend a large amount of time and effort on bug fixing. Consequently, \aprfull (\apr), designed to automatically generate patches to fix software bugs, has attracted wide attention from both academia and industry~\cite{long2016prophet, legoues2012genprog, long2015spr, lou2020can, tufano2018empstudy}. 


To achieve \apr, one popular approach is known as Generate-and-Validate (G\&V)~\cite{qi2015gv, ghanbari2019prapr, lou2020can, le2016hdrepair, legoues2012genprog, wen2018capgen, hua2018sketchfix, martinez2016astor, koyuncu2020fixminder, liu2019tbar, liu2019avatar}, which is typically based on the following pipeline: First, fault localization techniques~\cite{wong2016fl, abreu2007ochiai, zhang2013injecting, papadakis2015metallaxis, li2019deepfl, li2017transforming} are applied to determine the suspicious locations in programs where bugs are likely to exist. Then, the buggy locations are used by the \apr tools to generate a list of patches that replace buggy lines with correct lines. Afterward, each patch is validated against the original test suite to identify any \emph{plausible patches} (i.e., passing all tests in the test suite). Finally, to determine the \emph{correct patches}, developers examine the list of plausible patches to see if any of them can correctly fix the bug. 

Traditional \apr tools can mainly be categorized into heuristic-based~\cite{legoues2012genprog, le2016hdrepair, wen2018capgen}, constraint-based~\cite{mechtaev2016angelix, le2017s3, demacro2014nopol, long2015spr} and \template~\cite{ghanbari2019prapr, hua2018sketchfix, martinez2016astor, liu2019tbar, liu2019avatar}. Among these traditional tools, \template \apr tools~\cite{ghanbari2019prapr, liu2019tbar, benton2020effectiveness} have been able to achieve state-of-the-art results. \Template \apr tools typically leverage pre-defined templates (e.g., adding a nullness check) for bug fixing. However, since these fix templates are typically handcrafted, the number and types of bugs they are able to fix can be limited. 



To address the limitations of traditional \apr, researchers have proposed various \learning \apr tools~\cite{li2020dlfix, chen2018sequencer, jiang2021cure, lutellier2020coconut, zhu2021recoder, ye2022rewardrepair} based on the \nmtfull (\nmt) architecture~\cite{sutskever2014mt} where the input is the buggy code snippets and the goal is to translate the buggy code snippets into a fixed version. To accomplish this, \learning \apr tools require supervised training datasets with pairs of both buggy and fixed code snippets in order to learn how to perform this translation step. These training data are usually obtained by mining historical bug fixes using heuristics/keywords~\cite{dallmeier2007benchmark}, which can be imprecise for identifying bug-fixing commits; even the actual bug-fixing commits can include irrelevant code changes, leading to further pollution in the dataset~\cite{xia2022alpharepair}.
% 
Moreover, it can be hard for such \apr tools to generalize and fix bug types unseen during training. 



To better leverage recent advances in \plmfull{s} (\plm{s}), researchers~\cite{xia2022alpharepair, xia2023repairstudy, kolak2022patch, prenner2021codexws} have directly applied \plm{s} to generate patches without bug-fixing datasets. These \llm-based \apr tools work by either directly generating a complete code function~\cite{prenner2021codexws, xia2023repairstudy} or predict/infill the correct code snippet given its surrounding context~\cite{xia2022alpharepair, xia2023repairstudy}. By directly using \llm{s} that are pre-trained on billions of open-source code snippets, \llm-based \apr tools can achieve state-of-the-art performance on many repair datasets~\cite{xia2022alpharepair}. 


% 
%
%

Traditional \apr tools have long used the insight of the \emph{plastic surgery hypothesis}~\cite{barr2014plastic} where it states that the code ingredients to fix a bug already exist within the same project. Traditional \apr tools have manually designed pattern-~\cite{ghanbari2019prapr, saha2017elixir} or heuristic-based~\cite{jiang2018simfix, legoues2012genprog} approaches to finding and using such relevant code ingredients to generate fixes for bugs. However, the plastic surgery hypothesis has been largely ignored in \llm-based \apr. In fact, \llm provides a unique opportunity to fully automate the plastic surgery hypothesis idea via fine-tuning (learning project-specific information via model updates from the buggy project) and prompting (directly providing relevant code ingredients to the model), and make it directly applicable to different languages (since the \llm{s} are typically multi-lingual).%
Moreover, despite the intensive manual efforts involved, traditional \apr tools still cannot fully leverage project-specific information due to large search space for leveraging/composing existing code ingredients. In contrast, the project-specific information can effectively leveraged by \llm{s} due to their power in code understanding/vectorization, e.g., even partial/imprecise information may still guide \llm{s} in correct patch generation!
 To this end, we ask the question: \emph{How useful is the plastic surgery hypothesis in the era of \plm{s}}?








\mypara{Our Work.} To answer the question, we present \ourtech{\xspace} -- a \llm-based approach that automatically utilizes the plastic surgery hypothesis by systematically combining multiple fine-tuning and prompting strategies for \apr. \ourtech fine-tunes \plm{s} using two novel domain-specific training strategies: \textbf{\epfinetune} -- we fine-tune using the original buggy project by aggressively masking out a high percentage of tokens, which allows \plm to learn project-specific code tokens and programming styles; and \textbf{\rofinetune} -- which only masks out a single continuous code sequence per training sample, allowing the model to get used to the final \csapr task of predicting a single continuous code sequence. Furthermore, we directly leverage the ability for \plm{s} to understand natural language instructions and introduce a novel prompting strategy, \textbf{\idprompting}, which uses information retrieval and static analysis to obtain a list of relevant identifiers for the buggy lines. While such relevant identifiers are critical for fixing some difficult bugs, they may not be seen by the \llm during inference due to limited context window size. Through the use of prompting, we directly tell the model to use these extracted identifiers (relevant code ingredients) to generate the correct code. Finally, to perform repair, we combine all four model variants (including the base model, both fine-tuned models and the base model with prompting) for the final repair.





While our insight of leveraging the plastic surgery hypothesis for \llm-based \apr is generalizable across different types of \plm{s}, to implement \ourtech, we choose a recent \plm{\xspace}, \ctfive~\cite{wang2021codet5}, which is pre-trained on millions of open-source code snippets. \ctfive is an encoder-decoder model trained using \mspfull (\msp) objective where a percentage of tokens are masked out and each continuous masked token sequence is referred to as a masked span. Also, although we only extract relevant identifiers from the current buggy project (since this paper focuses on the plastic surgery hypothesis), our work can be easily extended to obtain other code information (such as relevant statements or functions) from other sources, such as  the massive pre-training corpora~\cite{husain2020codesearchnet} or historical bug-fixing datasets~\cite{jiang2019infer}, which can provide more coding knowledge for \llm{s}. Besides, although we mainly focus on using traditional string comparison algorithms for information retrieval in this paper, these techniques can be easily replaced by other frequency-based retrieval~\cite{robertson2009probabilistic} and neural search (or embedding-based search)~\cite{reimers2019sentence}.
  In summary, this paper makes the following contributions:


%


\begin{itemize}[noitemsep, leftmargin=*, topsep=0pt]
    \item \textbf{Dimension.} This paper is the first to revisit the important plastic surgery hypothesis in the era of \llm{s}. It opens up a new dimension for \llm-based \apr to incorporate previously neglected information from the buggy project itself to boost \apr performance. Furthermore, it demonstrates the promising future of retrieval-based prompting for modern \llm-based \apr.
    \item \textbf{Implementation.} We implement \ourtech based on the recent \ctfive model. We augment the model using two novel fine-tuning strategies: \epfinetune and \rofinetune, along with a novel prompting strategy based on information retrieval and static analysis: \idprompting. We combine the patches generated by all four models together and perform patch ranking to speed up \apr.% 
    \item \textbf{Evaluation Study.} We conduct an extensive evaluation against state-of-the-art \apr tools. On the widely studied \dfj 1.2 and 2.0 datasets~\cite{just2014dfj}, \ourtech is able to achieve the new state-of-the-art results of 89 and 44 correct bug fixes (15 and 8 more than best baseline) respectively.  Furthermore, we perform a broad ablation study to justify our design. \ourtech demonstrates for the first time that the plastic surgery hypothesis can substantially boost \llm-based \apr and advance state-of-the-art \apr, while being fully automated and general. Moreover, even partial/imprecise code ingredients may still effectively guide \llm{s} for \apr!
\end{itemize}



\section{Introduction} \label{sec.intro}

We consider the problem of minimizing a function over a network. In this setting, each node of the network has a portion of the global objective function and the edges represent neighbor nodes that can exchange information, i.e., communicate. The goal is to collectively minimize a finite sum of functions where each component is only known to one of the $n$ nodes (or agents) of the network. Such problems arise in many application areas such as machine learning \cite{forero2010consensus,tsianos2012consensus}, sensor networks \cite{baingana2014proximal, predd2007distributed}, multi-agent coordination \cite{cao2012overview, zhou2011multirobot} and signal processing \cite{combettes2011proximal}. The problem, known as a \emph{decentralized optimization} problem, can be represented as follows:
\begin{align}		\label{eq:prob}
	\min_{x\in \mathbb{R}^d}\quad f(x) = \frac{1}{n} \sum_{i=1}^n f_i(x),
\end{align}
where $f: \mathbb{R}^d \rightarrow \mathbb{R}$ is the global objective function, $f_i: \mathbb{R}^d \rightarrow \mathbb{R}$ for each $i\in \{1,2,...,n \}$ is the local objective function known only to node $i$ and $x\in \mathbb{R}^d$ is the decision variable.

To decouple the computation across different nodes, \eqref{eq:prob} is often reformulated as 
\begin{equation}\label{eq:cons_prob}
\begin{aligned}	
	\min_{x_i \in \mathbb{R}^d}&\quad \frac{1}{n} \sum_{i=1}^n f_i(x_i)\\
    \text{s.t.} &\quad  x_i = x_j, \quad \forall \,\, (i, j) \in \mathcal{E},
\end{aligned}
\end{equation}
where $x_i \in \mathbb{R}^d$ for each node $i\in \{1,2,...,n \}$ is a local copy of the decision variable, and  $\mathcal{E}$ denotes the set of edges of the network; see e.g., \cite{bertsekas2015parallel,nedic2009distributed}. If the underlying network is connected, the \emph{consensus} constraint ensures that all local copies are equal, and, thus, problems \eqref{eq:prob} and \eqref{eq:cons_prob} are equivalent. For compactness, we express problem \eqref{eq:cons_prob} as
\begin{equation}\label{eq:cons_prob1}
\begin{aligned}		
	\min_{x_i \in \mathbb{R}^d}&\quad \textbf{f} (\textbf{x}) = \frac{1}{n} \sum_{i=1}^n f_i(x_i)\\
	\text{s.t.} & \quad (\textbf{W}\otimes I_d)\textbf{x} = \textbf{x}, 
\end{aligned}
\end{equation}
where $\textbf{x} \in \mathbb{R}^{nd}$ is a concatenation of local copies $x_i$, $\textbf{W} \in \mathbb{R}^{n \times n}$ is a matrix that captures the connectivity of the underlying network, $I_d \in \mathbb{R}^{d \times d}$ is the identity matrix of dimension $d$, and the operator $\otimes$ denotes the Kronecker product,  $\textbf{W}\otimes I_d \in \mathbb{R}^{nd \times nd}$. The matrix $\textbf{W}$, known as the \emph{mixing} matrix, is a symmetric, doubly-stochastic matrix with $w_{ii}>0$ and $w_{ij}>0$ ($i\neq j$) if and only if $(i, j) \in \mathcal{E}$ in the underlying network. This matrix ensures that $(\textbf{W}\otimes I_d) \textbf{x}=\textbf{x}$ if and only if $x_i=x_j \,\, \forall \,\, (i, j) \in \mathcal{E}$ in the connected network, thus, % making 
\eqref{eq:cons_prob} and \eqref{eq:cons_prob1} are equivalent.% problems.

In this paper, we focus on gradient tracking methods. These first-order methods update and communicate the local decision variables, and also maintain, update and communicate an additional auxiliary variable that estimates (tracks) the gradient of the global objective function.
%\rb{These first-order methods update and communicate the local decision variables, and an additional auxiliary variable that estimates (tracks) the gradient of the global objective function.  }
%average gradient across all the nodes.} 
%These methods maintain an auxiliary variable ($y_i$) that estimates the average gradient across the network in addition to the decision variable ($x_i$) at each agent (node) $i$. 
We refer to the information shared by the methods as the communication strategy. When applied to the same decentralized setting, the theoretical convergence guarantees and practical implementations of gradient tracking methods with different communication strategies can vary significantly. %When applied to the same decentralized setting, the theoretical convergence guarantees and practical implementations of the methods vary with respect to the communication strategy. 
We propose an algorithmic framework that unifies communication strategies in gradient tracking methods and that allows for a direct theoretical and empirical comparison. The framework recovers popular gradient tracking methods as special cases.

The update form of gradient tracking methods can be generalized and decomposed as: $(1)$ one \emph{computation step} of calculating the local gradients, and $(2)$ one \emph{communication step} of sharing information based on the communication strategy. The %In practice, the 
complexity (cost) of these two steps can vary significantly across applications. For example, a large-scale machine learning problem solved on a cluster of computers with shared memory access has a higher cost of computation than communication \cite{tsianos2012consensus}. On the other hand, optimally allocating channels over a wireless sensor network requires economic usage of communications due to limited battery power \cite{magnusson2017bandwidth}.
The subject of developing algorithms (and convergence guarantees) that balance these costs has received significant attention in recent years; see e.g.,~\cite{chen2012fast,berahas2018balancing,9479747,berahas2019nested,sayed2014diffusion,zhang2018communication} and the references therein. In this paper, we follow the approach used in~\cite{berahas2018balancing} and 
explicitly decompose the two steps.
As a result, our algorithms are endowed with flexibility in terms of the number of communication and computation steps performed at each iteration. We show the benefits of this flexibility theoretically and empirically.

\subsection{Literature Review} \label{sec.lit}

Decentralized Gradient Descent (DGD) \cite{bertsekas2015parallel, nedic2009distributed}, a primal first-order method, is considered the prototypical method for solving~\eqref{eq:prob}. 
At each iteration nodes perform local computations and communicate local decision variable to neighbors. 
Gradient tracking methods, e.g., EXTRA \cite{shi2015extra}, SONATA \cite{sun2022distributed}, NEXT \cite{di2016next}, DIGing \cite{nedic2017achieving}, Aug-DGM \cite{xu2015augmented}, have emerged as popular alternatives due to their superior theoretical guarantees and empirical performance. 
They maintain, update and communicate an additional auxiliary variable that tracks the average gradient (additional communication cost compared to DGD). These methods are usually applied to smooth convex functions over undirected networks; however, they are also applicable to various other settings such as time varying networks \cite{nedic2017achieving}, uncoordinated step sizes \cite{nedic2017geometrically, xu2015augmented}, directed networks \cite{nedic2017achieving, pu2020push}, nonconvex functions \cite{di2016next, sun2022distributed}  
and stochastic gradients \cite{pu2021distributed}. Our algorithmic framework generalizes and extends current gradient tracking methodologies, allowing for a unified analysis and direct comparison of popular methods. Notably, our framework differs significantly from existing works that aim to unify gradient tracking methods. In \cite{sundararajan2017robust} and \cite{zhang2019computational}, semi-definite programming is used for this purpose. In \cite{alghunaim2020decentralized} and \cite{xu2021distributed}, the authors introduce unifying frameworks, similar to those proposed in this paper, for comparing different communication strategies. %have introduced unifying frameworks, much like ours, for comparing different communication strategies. 
However, our framework is simpler and allows for the exact specification of communication and computation steps at each iteration within the network. Furthermore, our proposed framework can accommodate a wider range of communication strategies than those discussed in \cite{sundararajan2017robust,zhang2019computational, alghunaim2020decentralized, xu2021distributed}. As a result of this increased algorithmic flexibility, our framework %This flexibility 
makes it possible to perform comprehensive comparisons among popular gradient tracking methods.

%\rb{Furthermore, our framework is significantly different from other existing frameworks that unify different communications strategies. } 
% In \cite{sundararajan2017robust,zhang2019computational}, semi-definite programming is used to unify communication strategies in gradient tracking methods. 
% \sg{In \cite{alghunaim2020decentralized} and \cite{xu2021distributed}, the authors also introduce unifying frameworks similar to ours to perform comparisons among communication strategies.
% \rb{However, o}ur framework is simpler, more intuitive in terms of exact specification of communication of quantities within the network and allows for more general communication strategies than those discussed in \cite{sundararajan2017robust,zhang2019computational, alghunaim2020decentralized, xu2021distributed}. \rb{Moreover, w}e also extend our framework to specify exact composition for communication and computation steps at each iteration and perform comparisons among gradient tracking methods with this flexibility.}


Another class of popular methods is  
primal-dual methods \cite{arjevani2020ideal, jakovetic2014linear, ling2015dlm, shi2014linear, wei20131, mansoori2021flexpd, mancino2021decentralized}. Of these methods, Flex-PD \cite{mansoori2021flexpd} and ADAPD \cite{mancino2021decentralized} allow for flexibility with respect to the number of communication and computation steps. That said, Flex-PD \cite{mansoori2021flexpd} does not show improved performance with the employment of the flexibility and ADAPD \cite{mancino2021decentralized} does not allow for a balance between communication and computation. In \cite{nguyen2022performance}, the authors propose LU-GT, an algorithm that has similarities to our framework in terms of executing multiple local computation steps within gradient tracking methods. Despite the common motivation and similarities, there are several distinct and notable differences. The LU-GT algorithm has two step size hyper-parameters, whereas our approach has only one. 
% Furthermore, our analysis results in less pessimistic step size conditions and more favorable convergence rates. 
Furthermore, our analysis results in less pessimistic step size conditions. 
It is worth noting that modifying LU-GT to align with our framework by setting the second step size to one is not possible due to the required conditions imposed in \cite{nguyen2022performance}. 
Moreover, our framework also provides a unifying foundation encompassing all popular gradient tracking methods. 
% Finally, our framework provides a unifying foundation encompassing all gradient tracking methods.

% \rb{In \cite{nguyen2022performance}, the authors introduced LU-GT, a method similar to ours in performing multiple local computation steps within gradient tracking methods. Although there are some similarities, our approach differs significantly. While LU-GT requires two hyper-parameter step sizes, our approach requires only one. Furthermore, our analysis results in more favorable convergence conditions on the stepsizes. We also note that modifying LU-GT to align with our framework by setting the second step size to one is not feasible due to their outlined convergence conditions. Additionally, our framework provides a unifying foundation encompassing all gradient tracking methods.} 
% \sg{In \cite{nguyen2022performance}, the authors present LU-GT, an algorithm that closely resembles our approach in terms of executing multiple local computation steps within gradient tracking methods. Despite a shared motivation, our work exhibits notable distinctions. In contrast to LU-GT, which necessitates two hyper-parameter step sizes, our algorithm relies on just one. Moreover, our analytical findings yield less pessimistic convergence conditions. It's worth noting that modifying LU-GT to align with our framework by setting the second step size to one is not feasible due to the convergence conditions outlined in their work. Additionally, our framework provides a unifying foundation encompassing all gradient tracking methods.}
Finally, algorithms that consider the consensus constraint as a proximal operator have been proposed. These algorithms aim to reduce communication load on distributed systems via a randomization scheme but are primarily designed for fully connected networks (all pairs of nodes are connected). 
Examples of such methods include, but are not limited to, Scaffnew \cite{mishchenko2022proxskip}, FedAvg \cite{li2019convergence}, Scaffold \cite{karimireddy2020scaffold}, Local-SGD \cite{gorbunov2021local} and FedLin \cite{mitra2021linear}.


\subsection{Contributions} \label{sec.contri}
We summarize our main contributions as follows:
\begin{enumerate}
	\item We propose a gradient tracking algorithmic framework (\texttt{GTA}) that unifies communication strategies in gradient tracking methods and provides flexibility in the number of communication and computation steps performed at each iteration. The framework recovers as special cases popular gradient tracking methods, i.e., ~\texttt{GTA-1} \cite{shi2015extra, nedic2017achieving}, \texttt{GTA-2} \cite{di2016next, sun2022distributed} and \texttt{GTA-3} \cite{nedic2017geometrically, xu2015augmented}; see \cref{tab: Algorithm Def}.
    \item We establish the conditions required, on the communication strategy and the step size parameter, that %to
    guarantee a global linear rate of convergence for \texttt{GTA} with multiple communication and multiple computation steps. 
    We also compare the relative performance of the special case gradient tracking algorithms, and illustrate the theoretical advantages of \texttt{GTA-3} over \texttt{GTA-2} (and \texttt{GTA-2} over \texttt{GTA-1}), a direct comparison not established in prior literature. 
    \item We show that the rate of convergence improves 
    with increasing the number of communication steps, and the extent of improvement depends on the communication strategy. 
    The improvements are much more profound in \texttt{GTA-3} as compared to \texttt{GTA-2} and \texttt{GTA-1}. 
    \item We illustrate the empirical performance of the proposed \texttt{GTA} framework on quadratic and binary classification logistic regression problems. We show the effect and benefits of %performing 
    multiple communication and/or computation steps per iteration on the performance of the %special case 
    algorithms.
\end{enumerate}

\subsection{Notation} \label{sec.notation}
Our proposed algorithmic framework is iterative and works with inner and outer loops. The variables $x_{i, k, j} \in \mathbb{R}^d$ and $y_{i, k, j} \in \mathbb{R}^d$ denote the local copies %copy 
of the decision variable and the auxiliary variable, respectively, of node $i$, in outer iteration $k$ and inner iteration $j$. The averages of all local decision variables and local auxiliary variables are denoted by $\bar{x}_{k, j} = \frac{1}{n} \sum_{i=1}^n x_{i, k, j}$ and $\bar{y}_{k, j} = \frac{1}{n} \sum_{i=1}^n y_{i, k, j}$, respectively. Boldface lowercase letters represent concatenated vectors of local copies
\begin{align*}
    \xmbf_{k, j} = 
    \begin{bmatrix}
        x_{1, k, j}\\
        x_{2, k, j}\\
        \vdots \\
        x_{n, k, j}
    \end{bmatrix} \in \mathbb{R}^{nd}\mbox{,} \quad
    \ymbf_{k, j} = 
    \begin{bmatrix}
        y_{1, k, j}\\
        y_{2, k, j}\\
        \vdots \\
        y_{n, k, j}
    \end{bmatrix} \in \mathbb{R}^{nd}
    \mbox{,} \quad
      \nabla \fmbf(\xmbf_{k, j}) = 
    \begin{bmatrix}
        \nabla f_1(x_{1, k, j})\\
        \nabla f_2(x_{2, k, j})\\
        \vdots \\
        \nabla f_n(x_{n, k, j})
    \end{bmatrix} \in \mathbb{R}^{nd}.%,
\end{align*}
The concatenated vector of the average of decision variables ($\Bar{x}_{k, j}$) and auxiliary variables ($\Bar{y}_{k, j}$) repeated $n$ times is denoted by $\xbb_{k, j}$ and $\ybb_{k, j}$, respectively. 
The $n$ dimensional vector of all ones is denoted by $1_n$ and the identity matrix of dimension $n$ is denoted by $I_n$. The spectral radius of square matrix $A$ is $\rho(A)$. Matrix inequalities are defined component wise. 
The Kronecker product of any two matrices $A \in \mathbb{R}^{n \times n}$ and $B \in \mathbb{R}^{d \times d}$ is represented using the operator $\otimes$ and denoted as $A \otimes B \in \mathbb{R}^{nd \times nd}$.

\subsection{Paper Organization} In \cref{sec.methods}, we describe our proposed gradient tracking algorithmic framework (\texttt{GTA}). In \cref{sec.theory}, we provide theoretical convergence guarantees for the proposed algorithmic framework for multiple communication steps and a single computation step at each iteration (\cref{sec.mult comms}) and multiple communication and computation steps at each iteration (\cref{sec.mult grads}). In \cref{sec.full graph res}, we consider the special case 
of fully connected networks. Numerical experiments on quadratic and binary classification logistic regression problems 
are presented in \cref{sec.num_exp}. Finally, we provide concluding remarks in \cref{sec.conc}.

%%%%%%%%%%%%%%%%%%%%%%%%%%%%
% Methods
%%%%%%%%%%%%%%%%%%%%%%%%%%%%
%\section{PoseRAC Model}
\label{sec4}

\begin{figure*}[t]
\centering
\includegraphics[width=1.0\textwidth]{figure5.pdf}
\caption{Overview of our proposed PoseRAC. For a input video, the repetitive count can be obtained through Pose Estimation, Transformer Encoder, Pose Mapping and Action-trigger, where only the Encoder and the Pose Mapping need to be trained. We use Triplet Margin Loss to train the Encoder while Binary Cross Entropy Loss to train both the Encoder and the Pose Mapping. In addition to achieving the state-of-the-art performance so far, the biggest highlight of our PoseRAC is that it is lightweight enough to be easily trained on a CPU.}
\label{fig5}
\end{figure*}

Given a video $V={\{x_i\}}^{T}_{1}\in \mathbb{R}^{C\times H\times W\times T}$ with $T$ RGB frames, repetitive action counting model aims to predict a certain value $Y$, which is the number of repetitive actions. In this section, we will introduce our PoseRAC in detail.

\subsection{Model Overview}

As shown in Figure \ref{fig5}, PoseRAC consists of four parts. 

\begin{itemize}

\item The first is a state-of-the-art and lightweight Pose Estimation Network~($\S\ref{first}$), which is used to estimate the poses represented by lots of human pose key points from each frame of the original video sequence. 

\item The second is a simple Transformer Encoder~($\S\ref{second}$) to embed the key points of poses into high-level feature space, where the same class have similar distances, while the distances of different classes are far apart.

\item The third is a Pose Mapping Module~($\S\ref{third}$), where the unique mapping relationship between the salient poses and the action classes can be learned. Each pose can be mapped to the action class with the highest probability after the previous encoding.

\item The fourth part is a lightweight Action-trigger Module~($\S\ref{fourth}$). When we get the salient action classification results of all frames of the entire video sequence, we can use this module to calculate the repetition count in a short time.

\end{itemize}

\subsection{Pose Estimation Network}
\label{first}
Our model first converts the video sequence into a sequence of human pose key points, which can be defined as: 
\begin{equation}
\begin{split}
&V={\{x_i\}}^{T}_{1}\in \mathbb{R}^{C\times H\times W\times T}\\
&V\xrightarrow{\mathrm{Pose Estimation}} P={\{p_i\}}^{T}_{1}\in \mathbb{R}^{D\times K\times T}
\end{split}
\label{eq1}
\end{equation}
where each $x_i$ represents a single RGB frame, and each $p_i$ represents the key points of each frame. To express the key points of each frame, we use $D\times K$ sequence, which includes two parts, one ($K$) is the number of key points to fully represent the current pose, the other ($D$) is the dimension of each key point, generally three, which are the two coordinates of the planes and the depth estimation.

Here we use state-of-the-art pose estimation models such as Vitpose\cite{xu2022vitpose} and BlazePose\cite{bazarevsky2020blazepose}. The pose estimation algorithms themselves are not designed by us, but we introduce pose information into the action counting task, which is a novel design not explored by previous work.

Moreover, our pose-level poses estimation processes the primitive information of video, which is similar to the feature extraction network in all video-level algorithms such as I3D\cite{carreira2017quo}, VideoSwinTransformer\cite{liu2022video}, and TSN\cite{wang2016temporal}. But the difference is that the result of video-level incorporates all information, while pose-level only produces core information, which greatly improves the performance. Additionally, using pose information can contribute to the lightweight of model. For instance, for a 1024-frame video, video-level feature extraction with an output dimension of 512 would produce a data volume of $1024\times 512=524288$, while using pose information with 33 key points produces a data volume of only $1024\times 33 \times 3=101376$.

\subsection{Encoding Poses with Transformer}
\label{second}
Here we specify our data representation for the Transformer Encoder, which requires input batch size, sequence length, and embedding dimensions. In our pose-level approach, each frame is a batch, the number of key points in each frame is the sequence length, and the feature dimension of each key point is the embedding dimension.

First we get the pose of each frame ${p_i}\in \mathbb{R}^{D\times K}$ through the Pose Estimation Network, where $i\in {1, 2, \dots, T}$ is the frame index, $K$ is the number of key points, and $D$ is the dimension of each key point. We further define $p_i = {\{k_j\}}^{K}_{1}$ to represent each key point, where $k_j\in \mathbb{R}^D$, and we embed it to obtain richer information. Our embedding projection $\mathrm{\bf{E}}$ is a simple MLP network with ReLU as the activation function. These calculations can be defined as:
\begin{equation}
\begin{split}
\mathrm{\bf{Z}}^0 = [\mathrm{\bf{E}}(k_1), \mathrm{\bf{E}}(k_2), \dots, \mathrm{\bf{E}}(k_K)]^T
\end{split}
\end{equation}
where $\mathrm{\bf{E}}(k_j)\in \mathbb{R}^{D^{\prime}}$ is the embedding feature. Then the next Transformer takes $\mathrm{\bf{Z}}^0$ as input and encodes it with self-attention. Given $\mathrm{\bf{Z}}^0\in \mathbb{R}^{K\times D^{\prime}}$ with $K$ key point features, each of which is $D^{\prime}$-dimensional, $\mathrm{\bf{Z}}^0$ is projected using $\mathrm{\bf{W}}_Q\in \mathbb{R}^{D^{\prime}\times D_q}$, $\mathrm{\bf{W}}_K\in \mathbb{R}^{D^{\prime}\times D_k}$, $\mathrm{\bf{W}}_V\in \mathbb{R}^{D^{\prime}\times D_v}$, where $D_k=D_q$, to extract feature representations query($\mathrm{\bf{Q}}$), key($\mathrm{\bf{K}}$) and value($\mathrm{\bf{V}}$), which can be defined as:
\begin{equation}
\begin{split}
&\mathrm{\bf{Q}}=\mathrm{\bf{Z}}^0\times \mathrm{\bf{W}}_Q\\
&\mathrm{\bf{K}}=\mathrm{\bf{Z}}^0\times \mathrm{\bf{W}}_K\\
&\mathrm{\bf{V}}=\mathrm{\bf{Z}}^0\times \mathrm{\bf{W}}_V
\end{split}
\end{equation}
and the output of self-attention can be computed as:
\begin{equation}
\begin{split}
\mathrm{\bf{Attn}}=\mathrm{Softmax}(\frac{\mathrm{\bf{Q}}\mathrm{\bf{K}}^T}{\sqrt{D_q}})\mathrm{\bf{V}}
\end{split}
\end{equation}
where $\mathrm{\bf{Attn}}\in \mathbb{R}^{K\times D^{\prime}}$. Also, we use common multi-head self-attention (MHSA) to make several self-attention operations calculate in parallel.

Now we introduce the overall architecture of Transformer Encoder, which has $L$ layers with each layer consisting of MHSA and MLP blocks. Also, LayerNorm and Residual Connection are applied before and after every MHSA or MLP block, respectively. Because the number of key points of each frame is  a bit less, so our encoder does not include the downsampling module that other models may have. The overall process can be defined as:
\begin{equation}
\begin{split}
&\mathrm{\bf{\hat{Z}}}^l = \mathrm{MHSA}(\mathrm{LN}(\mathrm{\bf{Z}}^{l-1})) + \mathrm{\bf{Z}}^{l-1}\\
&\mathrm{\bf{Z}}^l = \mathrm{MLP}(\mathrm{LN}(\mathrm{\bf{\hat{Z}}}^l)) + \mathrm{\bf{\hat{Z}}}^l
\end{split}
\end{equation}
where $\mathrm{\bf{Z}}^{l-1}$, $\mathrm{\bf{\hat{Z}}}^l$, $\mathrm{\bf{Z}}^l\in \mathbb{R}^{K\times D^{\prime}}$.


\subsection{Pose Mapping}
\label{third}
Taking the Encoder output $\mathrm{\bf{Z}}^L\in \mathbb{R}^{K\times D^{\prime}}$ as input, Pose Mapping module outputs probability scores $\mathrm{\bf{S}}\in \mathbb{R}^{C}$ of the current frame over all action classes. We perform binary classification after Sigmoid activation for each class, with the two salient poses of each class represented by the same bit data. To realize such a module, we use a very lightweight MLP network, which avoids the complexity. First, the two dimensions $K$ and $D^{\prime}$ of $\mathrm{\bf{Z}}^L$ are flattened into $\mathbb{R}^{KD^{\prime}}$, and then it passes through an MLP module, where the output channels is set to $C$, which can be defined as:
\begin{equation}
\begin{split}
\mathrm{\bf{S}} = \sigma(\mathrm{MLP}(\mathrm{Flatten}(\mathrm{\bf{Z}}^L)))
\end{split}
\end{equation}
where $\sigma$ represents the Sigmoid activation function.

With such Pose Mapping, we can obtain the scores of single frame. It should be noted that we extract the poses of all frames, and use the convenience of matrix operations to obtain scores in parallel, which is actually consistent with the idea of mini batch. So at last, we combine the scores of all frames to get the video score matrix $\mathrm{\bf{\hat{S}}}\in \mathbb{R}^{C\times T}$, where $T$ represents the number of frames in the current video. 


\subsection{Action-trigger Module}
\label{fourth}
We use the lightweight Action-trigger Module to obtain the final output $Y$, the repetitive action count, which has a time complexity of $\mathcal{O}(n)$. First, we get the scores $S_c\in \mathbb{R}^T$ of a given action class from $\mathrm{\bf{\hat{S}}}$. Then, we scan all frames and use the action-trigger mechanism to count when the two salient poses of the action class occur sequentially. We set upper and lower bounds to distinguish the scores of the two salient poses, which cluster non-salient poses in the middle and easily classify the salient poses to the two ends.

\subsection{Losses and Metric Learning}

The modules need to be trained are Embedding, Transformer Encoder and Pose Mapping, and because we perform binary classification for each class, so we use the Binary Cross Entropy Loss, which can be defined as follows:
\begin{gather}
\mathcal{L}_{bce} = -\frac{1}{N}\sum\limits_{i=1}^{N}(\frac{1}{C}\sum\limits_{j=1}^{C}loss(i,j))  \\
 loss(i,j)=y_{ij}\log p_{ij} + (1-y_{ij})\log(1-p_{ij})
\end{gather}
where $N$ represents the batch size (in our method, each frame is a batch), $C$ represents the number of classes, $y$ and $p$ are the labels and our predictions, respectively.

Moreover, we use Metric Learning to improve our Encoder and introduce the Pose Triplet Loss. Given a pose, Encoder produces higher-level features $\mathrm{\bf{Z}}^L$, which should be more representative. As shown in Figure \ref{fig5}, we achieve this with Triplet Margin Loss function, which selects anchors, same class positive samples, and different classes negative samples in a batch. It can be expressed as:
\begin{equation}
\begin{split}
\mathcal{L}_{tri} = \mathrm{max}(\mathrm{CS}(a,p)-\mathrm{CS}(a,n)+\mathrm{margin},0)
\end{split}
\end{equation}
where $a$, $p$, $d$ are anchors, positive and negative samples, and $\mathrm{CS}$ represents the Cosine Similarity to measure the distance between features. We pay more attention to hard samples, where the distances between anchors and negative samples are even smaller than those of positive samples. After Metric Learning, the poses of each action can be distinguishable, which cluster in the high-level space.

At last, our overall training combines these two losses:
\begin{equation}
\begin{split}
\mathcal{L} = \mathcal{L}_{bce} + \alpha\mathcal{L}_{tri}
\end{split}
\end{equation}
where $\alpha$ is the weight factor to control the two losses in the same numeric scale.
\subsection{Implementation Details}

\noindent{\bf Training.} We use the \emph{RepCount-pose} and \emph{UCFRep-pose} dataset we created to train our model. Only the frames with salient poses are inputted into the network instead of the entire video to speed up the fitting.

\noindent{\bf Inference.} During inference, the entire video sequence is inputted into the model. The poses of all frames pass through the Encoder and Pose Mapping, and then enter the Action-trigger Module to output the repetitive count.
% \section{Gradient Tracking Algorithmic Framework}\label{sec.methods}
\rb{In this section, we first describe our framework to unify gradient tracking algorithms and to incorporate flexibility in the number of communication steps. We then develop an algorithm that incorporates flexibility in the number of computation steps based on this framework.
The iterate update form of gradient tracking methods with multiple communication steps is given by  
\begin{equation}\label{eq : general form}
\begin{aligned}
    \xmbf_{k+1} & = \Zmbf_1^{n_c} \xmbf_k - \alpha \Zmbf_2^{n_c} \ymbf_k \\ 
    \ymbf_{k+1} & = \Zmbf_3^{n_c} \ymbf_k + \Zmbf_4^{n_c} (\nabla \fmbf(\xmbf_{k+1}) - \nabla \fmbf(\xmbf_{k})), 
\end{aligned}
\end{equation}
where $\Wmbf_1$, $\Wmbf_2$, $\Wmbf_3$ and $\Wmbf_4$ are communication matrices, $\Zmbf_i = \Wmbf_i \otimes I_d, \,\, \forall \,\, i = 1, 2, 3, 4$, and $n_c$ denote number of communication steps. A communication matrix, $\Umbf \in \mathbb{R}^{n \times n}$, is a symmetric, doubly stochastic matrix that respects 
the connectivity of the network, i.e. $u_{ii} > 0$ and $u_{ij} = 0$ if $(i,j) \notin \mathcal{E}$. Thus, \rb{a communication matrix} represents a network topology consisting of all nodes and a subset of the edges \rb{(or all edges)} present in the network. %(possibly all edges). 
Moreover, performing multiple communication steps ($n_c$) improves the consensus of the communicated vector across the network and represented in a compact form by taking power $n_c$ of $\Zmbf_i \,\, \forall \,\, i = 1, 2, 3, 4$ as in \cite{berahas2018balancing}.   
%Also, performing multiple communication steps ($n_c$), which improves consensus across the network, is compactly represented by taking power $n_c$ of $\Zmbf_i \,\, \forall \,\, i = 1, 2, 3, 4$ as in \cite{berahas2018balancing}.
%Performing multiple communication steps can improve the consensus across the network and 


The four communication matrices, $\Wmbf_1$, $\Wmbf_2$, $\Wmbf_3$ and $\Wmbf_4$, represent four (possibly different) network topologies over which the corresponding vectors are communicated. Due to these communication matrices, the iterate update form given in \eqref{eq : general form} can represent all ATC and CTA communication strategies based algorithms.% seen in literature using \eqref{eq : general form}. 
Some of these algorithms have been shown in \cref{tab: Algorithm Def} where $\Wmbf$ is the mixing matrix for the connected network.

%This allows one to represent all ATC and CTA based communication strategies seen in literature using \eqref{eq : general form}. Some of these cases have been shown in \cref{tab: Algorithm Def} where $\Wmbf$ is the mixing matrix for the connected network. The iterate update form \eqref{eq : general form} also represents \emph{heterogeneous} communication strategies, where different pieces of information are shared across a different subset of network edges. Such strategies are useful in several decentralized settings such as networks with bandwidth limitations where the amount of data transfer on each edge is limited XXX. Such network settings can be represented using appropriate choice of communication matrices used in \eqref{eq : general form}. In \cref{sec.theory} we establish conditions on the communication matrices to ensure linear convergence of the iterates generated by GTA framework. Specifically,} ensuring $\Wmbf_1$ and $\Wmbf_3$ to represent a connected (possibly different) network is sufficient to guarantee linear convergence to the solution.

The iterate update form \eqref{eq : general form} also represents \emph{heterogeneous} communication strategies, where different pieces of information are shared across a different subset of network edges. These communication strategies are useful in several decentralized settings such as networks with bandwidth limitations where the amount of data transfer on each edge is limited XXX. Such network settings can be represented using appropriate choice of communication matrices used in \eqref{eq : general form}. In \cref{sec.theory} we establish conditions on the communication matrices to ensure linear convergence of the iterates generated by GTA framework. Specifically, ensuring $\Wmbf_1$ and $\Wmbf_3$ to represent a connected (possibly different) network is sufficient to guarantee linear convergence to the solution.

We can extend \eqref{eq : general form} to incorporate multiple computation steps. We present \texttt{GTA} (\cref{alg : Deterministic}), a deterministic algorithm that performs $n_c \geq 1$ communication steps and $n_g \geq 1$ computation steps each iteration. If $\ymbf_k$ is a good estimate of the average gradient across the network, performing multiple computation steps using only local information will drive the local iterate closer to the global optimal solution without distorting the consensus error too much.
A balance between the number of communication and computation steps is required to achieve overall efficiency and \cref{alg : Deterministic} allows for such flexibility in these steps using the parameters $n_g$ and $n_c$.

\bremark 
We make the following remarks about \cref{alg : Deterministic}. 

\begin{itemize}
    \item \textbf{Inner and Outer Loops} - Lines 2--8 form the outer loop. The algorithm performs $n_c$ communication steps each outer iteration in lines 7--8. These multiple communications are theoretically represented by taking powers of the communication matrices. The algorithm performs $n_g$ local gradient computations each outer iteration with one computation in line 8 $(\nabla \fmbf(\xmbf_{k+1, 1}))$ and $n_g - 1$ computations in the inner loop (lines 4-6, $(\nabla \fmbf(\xmbf_{k, j+1}))$). The inner loop is \rb{only} executed if more than one computation is to be performed every outer iteration (line 3).
    
    \item \textbf{Step size} ($\alpha$) - The algorithm employs a constant step size \rb{that depends on the choice of $n_g$ and $n_c$}. Increase in communication steps per iteration ($n_c$) allows employing a larger stepsize. Increase in the computation steps per iteration ($n_g$) requires the stepsize to be reduced at a polynomial rate.
    
    \item \textbf{Communication Strategy} - The communication matrices employed in lines 7--8 allow for various communication strategies to be employed. If $\Wmbf_1$ and $\Wmbf_2$ represent connected networks, the algorithm can show linear rate of convergence for any composition of communication and computation steps.
\end{itemize} 
\eremark
}



\section{Gradient Tracking Algorithmic Framework}\label{sec.methods}
\rb{In this section, we first describe our framework to unify gradient tracking algorithms. We then develop an algorithm that incorporates flexibility in the number of communication and computation steps based on this framework. }
%\rb{In this section, we describe our framework to unify gradient tracking algorithms and to incorporate flexibility in the number of communication and computation steps.} 
%In this section, we first describe our framework to unify gradient tracking algorithms. We then describe our algorithm for flexibility in number of communication and computation steps. 
\rb{The iterate update form given in \eqref{eq : general form} unifies different communication strategies used in gradient tracking algorithms,} 
%The iterate form we use to unify different communication strategies in gradient tracking algorithms is given in \eqref{eq : general form}, 
\begin{equation}\label{eq : general form}
\begin{aligned}
    \xmbf_{k+1} & = \Zmbf_1 \xmbf_k - \alpha \Zmbf_2 \ymbf_k \\ 
    \ymbf_{k+1} & = \Zmbf_3 \ymbf_k + \Zmbf_4 (\nabla \fmbf(\xmbf_{k+1}) - \nabla \fmbf(\xmbf_{k})), 
\end{aligned}
\end{equation}
% \begin{align}\label{eq : general form}
%     \xmbf_{k+1} & = \Zmbf_1 \xmbf_k - \alpha \Zmbf_2 \ymbf_k \\ 
%     \ymbf_{k+1} & = \Zmbf_3 \ymbf_k + \Zmbf_4 (\nabla \fmbf(\xmbf_{k+1}) - \nabla \fmbf(\xmbf_{k})), \nonumber
% \end{align}
where $\Wmbf_1$, $\Wmbf_2$, $\Wmbf_3$ and $\Wmbf_4$ are communication matrices, and $\Zmbf_i = \Wmbf_i \otimes I_d, \,\, \forall \,\, i = 1, 2, 3, 4$. 
A communication matrix, $\Umbf \in \mathbb{R}^{n \times n}$, is a symmetric, doubly stochastic matrix that respects 
the connectivity of the network, i.e. $u_{ii} > 0$ and $u_{ij} = 0$ if $(i,j) \notin \mathcal{E}$. Thus, \rb{a communication matrix} represents a network topology consisting of all nodes and a subset of the edges \rb{(or all edges)} present in the network. %(possibly all edges). 
The four communication matrices, $\Wmbf_1$, $\Wmbf_2$, $\Wmbf_3$ and $\Wmbf_4$, represent four (possibly different) network topologies over which the corresponding vectors are communicated. 
This allows one to represent all ATC and CTA based communication strategies seen in literature using \eqref{eq : general form}. Some of these cases have been shown in \cref{tab: Algorithm Def} where $\Wmbf$ is the mixing matrix for the connected network.

\rb{The iterate update form \eqref{eq : general form} also represents \emph{heterogeneous} communication strategies, where different pieces of information are shared across a different subset of network edges. Such strategies are useful in several decentralized settings such as networks with bandwidth limitations where the amount of data transfer on each edge is limited XXX. Such network settings can be represented using appropriate choice of communication matrices used in \eqref{eq : general form}. In \cref{sec.theory} we establish conditions on the communication matrices to ensure linear convergence of the iterates generated by GTA framework. Specifically,} ensuring $\Wmbf_1$ and $\Wmbf_3$ to represent a connected (possibly different) network is sufficient to guarantee linear convergence to the solution.
%Our generalization can also represent heterogeneous communication strategies. In this case, different pieces of information are shared across a different subset of network edges. 
%Such a system can be represented using the communication matrices. This might be useful in decentralised systems with bandwidth limitations to reduce load on some network edges. We find that ensuring $\Wmbf_1$ and $\Wmbf_3$ represent a connected (possibly different) network is sufficient to ensure linear convergence to the solution.

\begin{algorithm}[H]
    \caption{\texttt{GTA}: Gradient Tracking Algorithm}
    \textbf{Inputs:} initial point $\xmbf_{0, 1} \in \R{nd}$, step size $\alpha$, computations $n_g$, communications $n_c$
    \begin{algorithmic}[1]
        \State $\textbf{y}_{0, 1} \gets \nabla \textbf{f}(\textbf{x}_{0, 1})$
        \For{$k \gets 0, 1, 2$ ... }    
            \If{$n_g > 1$}
                \For{$j \gets 1, 2$ ... $, n_g-1$}
                    \State $\textbf{x}_{k, j+1} \gets \textbf{x}_{k, j} - \alpha \,\textbf{y}_{k, j}$
                    \State $\textbf{y}_{k, j+1} \gets \textbf{y}_{k, j} + \nabla \textbf{f}(\textbf{x}_{k, j+1})  - \nabla \textbf{f}(\textbf{x}_{k, j})$
                \EndFor
            \EndIf
            
            \State $\textbf{x}_{k+1, 1} \gets \textbf{Z}_1^{n_c} \textbf{x}_{k, n_g} - \alpha \, \textbf{Z}_2^{n_c} \textbf{y}_{k, n_g}$
            \State $\textbf{y}_{k+1, 1} \gets \textbf{Z}_3^{n_c} \textbf{y}_{k, n_g} + \textbf{Z}_4^{n_c}(\nabla \textbf{f}(\textbf{x}_{k+1, 1})  - \nabla \textbf{f}(\textbf{x}_{k, n_g}))$
        \EndFor
    \end{algorithmic}
    \label{alg : Deterministic}
\end{algorithm}

Based on \eqref{eq : general form}, we present \texttt{GTA} (\cref{alg : Deterministic}), a deterministic algorithm that performs $n_c \geq 1$ communication steps and $n_g \geq 1$ computation steps each iteration. 
% It first performs $n_g - 1$ computation steps done in an inner loop without communication. Then it performs one computation with $n_c$ communication steps. The multiple communications are represented by taking power $n_c$ of $\Zmbf_i \,\, \forall \,\, i = 1, 2, 3, 4$ as in \cite{berahas2018balancing}.
\rb{Performing multiple communication steps improves consensus across the network. Also, if $\ymbf_k$ is a good estimate of the average gradient across the network, performing multiple computation steps using only local information will drive the local iterate closer to the global optimal solution without distorting the consensus error too much. A balance between these two steps is required to achieve overall efficiency and \cref{alg : Deterministic} allows for such flexibility in these steps by choosing $n_g$ and $n_c$ appropriately.}
%The intuition behind the method is that the multiple communication steps would provide improved consensus across the network. Also, if $\ymbf_k$ is a good estimate of the average gradient across the network, performing multiple computation steps updating it with only local information will drive the local iterate closer to the global optimal solution without distorting the consensus error too much.

\begin{table}\centering
\caption{Special cases of Gradient Tracking Algorithms (GTA).}\label{tab: Algorithm Def}
\begin{tabular}{l*{4}{>{\centering\arraybackslash}p{0.8cm}}c}\toprule
\multirow{2}{*}{Method} &\multicolumn{4}{c}{Communication Matrices} & Algorithms in literature\\\cmidrule{2-5}
&$\Wmbf_1$&$\Wmbf_2$&$\Wmbf_3$&$\Wmbf_4$& $(n_c = n_g = 1)$\\\midrule
\texttt{GTA-1} &$\Wmbf$ &$I_n$ &$\Wmbf$ &$I_n$& EXTRA\footnotemark[4] \cite{shi2015extra}, DIGing \cite{nedic2017achieving} \\\hdashline
\texttt{GTA-2} &$\Wmbf$ &$\Wmbf$ &$\Wmbf$ &$I_n$ & SONATA \cite{sun2022distributed}, NEXT \cite{di2016next}, \cite{pu2020push} \\\hdashline
\texttt{GTA-3} &$\Wmbf$ &$\Wmbf$ &$\Wmbf$ &$\Wmbf$ & Aug-DGM \cite{xu2015augmented}, ATC-DIGing \cite{nedic2017geometrically}\\ 
\bottomrule
\end{tabular}
\end{table}

\bremark 
We make the following remarks about \cref{alg : Deterministic}. 

\begin{itemize}
    \item \textbf{Inner and Outer Loops} - Lines 2--8 form the outer loop. The algorithm performs $n_c$ communication steps each outer iteration in lines 7--8. These multiple communications are theoretically represented by taking powers of the communication matrices. The algorithm performs $n_g$ local gradient computations each outer iteration with one computation in line 8 $(\nabla \fmbf(\xmbf_{k+1, 1}))$ and $n_g - 1$ computations in the inner loop (lines 4-6, $(\nabla \fmbf(\xmbf_{k, j+1}))$). The inner loop is \rb{only} executed if more than one computation is to be performed every outer iteration (line 3).
    
    \item \textbf{Step size} ($\alpha$) - The algorithm employs a constant step size \rb{that depends on the choice of $n_g$ and $n_c$}. Increase in communication steps per iteration ($n_c$) allows employing a larger stepsize. Increase in the computation steps per iteration ($n_g$) requires the stepsize to be reduced at a polynomial rate.
    
    \item \textbf{Communication Strategy} - The communication matrices employed in lines 7--8 allow for various communication strategies to be employed. If $\Wmbf_1$ and $\Wmbf_2$ represent connected networks, the algorithm can show linear rate of convergence for any composition of communication and computation steps.
\end{itemize} 
\eremark

% \texttt{GTA} converges to the solution at a linear rate employing a constant stepsize given $\Wmbf_1$ and $\Wmbf_3$ represent connected networks. Increasing $n_c$ (communication steps) allows a larger stepsize to be employed providing accelerated convergence. Increase in $n_g$ (computation steps) in \texttt{GTA} requires the stepsize to be reduced at a polynomial rate. While the effect of $n_g$ on convergence rate is not clear, we empirically find it brings acceleration to convergence. 

We analyse \texttt{GTA} and provide results for some popular communication strategies as special cases summarised in \cref{tab: Algorithm Def}. The choice of the communication matrices, $\textbf{W}_1, \textbf{W}_2, \textbf{W}_3$ and $\textbf{W}_4$, hence the communication strategy, impacts both convergence of algorithm (discussed in \cref{sec.theory}) and practical implementation. Consider instances \texttt{GTA-1}, \texttt{GTA-2} and \texttt{GTA-3} from \cref{tab: Algorithm Def} when $n_g=1$. Computing local gradients and communications can be performed in parallel in \texttt{GTA-1} and \texttt{GTA-2}. \texttt{GTA-3} requires performing communication and computation sequentially. Depending on properties of the problem and system, such trade-offs can create significant impacts. 


\sg{Removed empirical experiments to explain how ATC and CTA differ as there is already so much theory and literature on it in the paper, felt redundant, let me know? The figures are available in Figures/Method Explanation}

\begin{comment}
To provide intuition on how these instances differ, we run an experiment of minimizing quadratic functions over a 16 node network where each node is initialized at the global minimizer and the gradient track $\textbf{y}_0$ is initialized as the local gradient at each node. We run this experiment over two networks, a fully connected network, where all nodes are connected with equal weights and a cyclic network where each node is connected to exactly 2 nodes with 50 communication and 5 computation steps. We measure the optimization error ($\|\Bar{x}_k - x^*\|$) and the consensus error ($\|x_k - 1\Bar{x}_k^T\|$) summarised in figure~\cref{fig : Method Differences Display}. In the fully connected network, no agent moves away from the global optimizer in DIGing\_2K and DIGing\_3K while they do so in DIGing\_K to find their way back. In the cyclic network, due to lack of communication the iterates will move away but we can see how DIGing\_3K performs better in both optimization and consensus error than DIGing\_2K and DIGing\_K.

The above discussion highly favours DIGing\_3K but the choice of communication structure also impacts the implementation of these algorithms. The communication and first local gradient calculation can be parallilized in DIGing\_K and DIGing\_2K while are required to be performed sequentially for DIGing\_3K which has the potential to highly impact practical performance of certain distributed system problems. Thus the mixing matrices should be chosen to reflect the specifics of the task and the system being used.


\begin{figure}[H]
\centering
    
    \begin{subfigure}{0.45\textwidth}
        \centering
        \includegraphics[width=\textwidth]{Figures/Method Explanation/Full1_error_opt_cost_iter.png}
    \end{subfigure}
    \begin{subfigure}{0.45\textwidth}
        \centering
        \includegraphics[width=\textwidth]{Figures/Method Explanation/Full1_error_con_cost_iter.png}
    \end{subfigure}
    
\caption{Optimization and Consensus error for instances of Algorithm~\cref{alg : Deterministic} when the initial point for every agent is the global optimizer}
\label{fig : Method Differences Display}
\end{figure}
\end{comment}

\footnotetext[4]{\texttt{GTA-1} with $n_c = n_g = 1$ is a special case of EXTRA when the first mixing matrix in EXTRA is set to $(2\Wmbf - I_n)$ and the second is set to $\Wmbf^2$.}
\section{Gradient Tracking Algorithmic Framework}\label{sec.methods}
In this section, we describe our algorithmic framework (\texttt{GTA}) that unifies gradient tracking methods. We then extend the framework to allow for flexibility in the number of communication and  computation steps performed at every iteration. Finally, we make remarks about the algorithmic framework and implementation, and then discuss popular gradient tracking methods as special cases of our proposed framework.

The 
iterate update form (for all $k\geq0$) for the decision variable $\xmbf \in \mathbb{R}^{nd}$ and the auxiliary variable $\ymbf \in \mathbb{R}^{nd}$ that we propose to unify gradient tracking methods is  
\begin{equation}\label{eq: general_form}
\begin{aligned}
    \xmbf_{k+1, 1} & = \Zmbf_1 \xmbf_{k, 1} - \alpha \Zmbf_2 \ymbf_{k,1} \\ 
    \ymbf_{k+1, 1} & = \Zmbf_3 \ymbf_{k, 1} + \Zmbf_4 (\nabla \fmbf(\xmbf_{k+1, 1}) - \nabla \fmbf(\xmbf_{k, 1})), 
\end{aligned}
\end{equation}
where $\alpha>0$ is the constant step size, $\Zmbf_i = \Wmbf_i \otimes I_d \in \mathbb{R}^{nd \times nd}$ for $i = 1, 2, 3, 4$ and $\Wmbf_i \in \mathbb{R}^{n \times n}$ are communication matrices. A communication matrix $\Umbf \in \mathbb{R}^{n \times n}$ is a symmetric, doubly stochastic matrix that respects the connectivity of the network, i.e., $u_{ii} > 0$ and $u_{ij}  \geq  0$ ($i \neq j$) if $(i,j) \in \mathcal{E}$ and $u_{ij} = 0$ ($i \neq j$) if $(i,j) \notin \mathcal{E}$. 
The communication matrices, $\Wmbf_i$ for $i = 1, 2, 3, 4$, represent four (possibly different) network topologies consisting of all the nodes and (possibly different) subsets of the edges of the network over which the corresponding vectors are communicated.
The update form given in \eqref{eq: general_form} generalizes many popular gradient tracking methods for different choices of the communication matrices; see \cref{tab: Algorithm Def}. 
While the  methodology has %shares 
similarities to \cite{xu2021distributed, alghunaim2020decentralized}, our framework allows for the exact specification of the communication quantities within the network and does not %while not imposing 
impose any interdependent conditions among the communication matrices $\Wmbf_i$ for $i = 1, 2, 3, 4$.
In \eqref{eq: general_form} one communication and one computation step is performed at every iteration and so the inner iteration index is always $1$. 
We include this subscript for consistency with the presentation of the algorithm and analysis with multiple communication and computation steps.

\begin{table}[H]\centering
\caption{Special cases of Gradient Tracking Algorithm (\texttt{GTA}).  
}\label{tab: Algorithm Def}
\begin{tabular}{l*{4}{>{\centering\arraybackslash}p{0.8cm}}c}\toprule
\multirow{2}{*}{Method} &\multicolumn{4}{c}{Communication Matrices} & Algorithms in literature\\\cmidrule{2-5}
&$\Wmbf_1$&$\Wmbf_2$&$\Wmbf_3$&$\Wmbf_4$& $(n_c = n_g = 1)$\\\midrule
\texttt{GTA-1} &$\Wmbf$ &$I_n$ &$\Wmbf$ &$I_n$& DIGing \cite{nedic2017achieving}, EXTRA  
\cite{shi2015extra},  \\\hdashline
\texttt{GTA-2} &$\Wmbf$ &$\Wmbf$ &$\Wmbf$ &$I_n$ & SONATA \cite{sun2022distributed}, NEXT \cite{di2016next,pu2020push} \\\hdashline
\texttt{GTA-3} &$\Wmbf$ &$\Wmbf$ &$\Wmbf$ &$\Wmbf$ & Aug-DGM \cite{xu2015augmented}, ATC-DIGing \cite{nedic2017geometrically}\\ 
\bottomrule
\end{tabular}

Note: $\Wmbf$ is a mixing matrix.
\end{table}

We incorporate multiple communications in \eqref{eq: general_form} by replacing $\Zmbf_i$ with $\Zmbf_i^{n_c} = \Wmbf_i^{n_c} \otimes I_d$ for $i=1, 2, 3, 4$, where $n_c \geq 1$ is the number of communication steps at each iteration. 
Taking the communication matrices to the $n_c$ power represents performing $n_c$ communication (consensus) steps at every iteration. We further extend \eqref{eq: general_form} to incorporate multiple computation steps at each iteration. That is, the algorithm performs multiple local updates before communicating information with local neighbors. Our full algorithmic framework with flexibility in the number of communication and computation steps, i.e., $n_c \geq 1$ and $n_g \geq 1$, is given in \cref{alg : Deterministic}. A balance between the number of communication and computation steps is required to achieve overall efficiency for different applications, and \texttt{GTA} allows for such flexibility in these steps via the parameters $n_g$ and $n_c$.

\begin{algorithm}[H]
    \caption{\texttt{GTA}: Gradient Tracking Algorithm}
    \textbf{Inputs:} initial point $\xmbf_{0, 1} \in \R{nd}$, step size $\alpha >0$, computations $n_g \geq 1$, 
    
    communications $n_c \geq 1$.
    \begin{algorithmic}[1]
        \State $\textbf{y}_{0, 1} \gets \nabla \textbf{f}(\textbf{x}_{0, 1})$
        \For{$k \gets 0, 1, 2$ ... }    
            \If{$n_g > 1$}
                \For{$j \gets 1, 2$ ... $, n_g-1$}
                    \State $\textbf{x}_{k, j+1} \gets \textbf{x}_{k, j} - \alpha \,\textbf{y}_{k, j}$
                    \State $\textbf{y}_{k, j+1} \gets \textbf{y}_{k, j} + \nabla \textbf{f}(\textbf{x}_{k, j+1})  - \nabla \textbf{f}(\textbf{x}_{k, j})$
                \EndFor
            \EndIf
            
            \State $\textbf{x}_{k+1, 1} \gets \textbf{Z}_1^{n_c} \textbf{x}_{k, n_g} - \alpha \, \textbf{Z}_2^{n_c} \textbf{y}_{k, n_g}$
            \State $\textbf{y}_{k+1, 1} \gets \textbf{Z}_3^{n_c} \textbf{y}_{k, n_g} + \textbf{Z}_4^{n_c}(\nabla \textbf{f}(\textbf{x}_{k+1, 1})  - \nabla \textbf{f}(\textbf{x}_{k, n_g}))$
        \EndFor
    \end{algorithmic}
    \label{alg : Deterministic}
\end{algorithm}
\bremark 
We make the following remarks about \cref{alg : Deterministic}. 
\begin{itemize}
    \item \textbf{Communications and Computations:} The number of communication and computation steps are dictated by $n_c$ and $n_g$, respectively. 
    By performing multiple communication steps, the goal is to improve consensus across the local decision variables. By performing multiple computation steps, the goal is for individual nodes to make more progress on their local objective functions. 
    \item \textbf{Inner and Outer Loops:} Lines 2--8 form the outer loop and Lines 4--6 form the inner loop. The algorithm performs $n_c$ communication steps each outer iteration (Lines 7 and 8). The algorithm performs $n_g$ local (gradient) computations at each outer iteration; $n_g-1$ computations in the inner loop (Line 6, $\nabla \fmbf(\xmbf_{k, j+1})$) and one computation in the outer loop (Line 8, $\nabla \fmbf(\xmbf_{k+1, 1})$). 
    The inner loop is only executed if more than one computation, i.e., $n_g>1$,  is to be performed every outer iteration (Line 3). By default, we refer to outer iterations when we say iterations unless otherwise specified.
    \item \textbf{Step size ($\alpha>0$):} The algorithm employs a constant step size that depends on the problem parameters, the choices of $n_c$ and $n_g$, and the communication strategy, i.e., $\Wmbf_i$ for $i = 1, 2, 3, 4$. 
\end{itemize} 
\eremark

We analyze \texttt{GTA} and provide results for several popular communication strategies as special cases; summarized in \cref{tab: Algorithm Def}. The choice of the communication matrices ($\Wmbf_i$ for $i = 1, 2, 3, 4$), or equivalently the communication strategy, impact both the convergence of the algorithm and practical implementation. Notice that all methods in \cref{tab: Algorithm Def} require that $\Wmbf_1$ and $\Wmbf_3$ are mixing matrices. Our theoretical results recover this for the general framework. Consider \texttt{GTA-1}, \texttt{GTA-2} and \texttt{GTA-3} defined in \cref{tab: Algorithm Def} with $n_g=1$. In \texttt{GTA-1} and \texttt{GTA-2}, computing local gradients and communications can be performed in parallel because the local gradients need not be communicated ($\Wmbf_4 = I_n$). On the other hand, in \texttt{GTA-3}, these steps need to be performed sequentially. Such trade-offs can create significant impact depending on the problem setting and system.

As mentioned above, the communication matrices ($\Wmbf_i$ for $i = 1, 2, 3, 4$) in \texttt{GTA} need not be the same. That is,  different information can be exchanged on subsets of the edges of the network. This allows for a flexibility in the communication strategy that current gradient tracking methodologies do not possess. 
Such strategies can be useful in applying gradient tracking methods to decentralized settings with networks with bandwidth limitations, e.g., optimization problems in cyberphysical systems with battery powered wireless sensors \cite{magnusson2017bandwidth}. 



%%%%%%%%%%%%%%%%%%%%%%%%%%%%
% Theoretical Results
%%%%%%%%%%%%%%%%%%%%%%%%%%%%

\section{Convergence Analysis}\label{sec.theory}

%%%%%%%%%%%%%%%%%%%%%%%%%%%%%%%%%%%%%%%%%%%%%
% Prequals to the theory analysis
%%%%%%%%%%%%%%%%%%%%%%%%%%%%%%%%%%%%%%%%%%%%%
In this section, we present %prove 
theoretical convergence guarantees for our proposed algorithmic framework (\texttt{GTA}). 
The analysis is divided into three subsections. 
In \cref{sec.mult comms}, we analyze the effect of multiple communications, i.e., $n_c \geq 1$ (and $n_g = 1$), on \texttt{GTA} and the three special cases \texttt{GTA-1}, \texttt{GTA-2} and \texttt{GTA-3}. 
While these results are a special case of the results presented in \cref{sec.mult grads}, we present these results first as they are simpler to derive, easier to follow and allow us to gain intuition about the effect of the number of communications. 
We then look at the effect of multiple computations in conjunction with multiple communications, i.e., $n_c \geq 1$ and $n_g \geq 1$, in \cref{sec.mult grads} by extending the analysis from \cref{sec.mult comms}. In \cref{sec.full graph res}, we analyze \texttt{GTA-2} and \texttt{GTA-3} for fully connected networks; this special case is not captured by the analysis in the previous subsections. 

We make the following assumption on the functions. 

\bassumption    \label{asum.convex and smooth}
    The global objective function $f: \mathbb{R}^d \rightarrow \mathbb{R}$ is $\mu$-strongly convex. Each component function $f_i: \mathbb{R}^d \rightarrow \mathbb{R}$ $($for $i \in \{ 1,2,\dots,n\}$$)$ has L-Lipschitz continuous gradients. That is, for all $z, z' \in \mathbb{R}^d$
    \begin{align*}
        &f(z')  \geq f(z) +  \langle \nabla f(z), z' - z \rangle + \tfrac{\mu}{2} \|z' - z\|_2^2, \\
        &\|\nabla f_i(z) - \nabla f_i(z')\|_2 \leq L\|z - z'\|_2,    \qquad \qquad \quad \forall \; i = 1, \dots, n.
    \end{align*}
\eassumption
By \cref{asum.convex and smooth}, the global minimizer of~\eqref{eq:prob} is unique, and we denote it by $x^*$.

For notational convenience, we define
\begin{align*}
   \beta^{n_c} = \left\|\Wmbf^{n_c} - \tfrac{1_n1_n^T}{n}\right\|_2, \quad \beta_i^{n_c} = \left\|\Wmbf_i^{n_c} - \tfrac{1_n1_n^T}{n}\right\|_2, \qquad \forall \; i = 1, 2, 3, 4, 
\end{align*}
% \begin{align*}
%    \left\|\Wmbf^{n_c} - \tfrac{1_n1_n^T}{n}\right\|_2 = \beta^{n_c}, \quad \left\|\Wmbf_i^{n_c} - \tfrac{1_n1_n^T}{n}\right\|_2 = \beta_i^{n_c}, \qquad \forall \,\,\, i = 1, 2, 3, 4,    
% \end{align*}
where $\beta \in [0, 1)$ because $\Wmbf$ is a mixing matrix for a connected network and $\beta_i \in [0,1]$ because $\Wmbf_i$ for $ i = 1, 2, 3, 4$ are symmetric, doubly stochastic matrices. Using the definitions of $\Zmbf^{n_c} = \Wmbf^{n_c}\otimes I_{d}$ and $\Zmbf_i^{n_c} = \Wmbf_i^{n_c}\otimes I_{d}$ for $ i = 1, 2, 3, 4$, it follows that
\begin{align}   \label{eq : beta and Z}
    \|\Zmbf^{n_c} - \Imbf\|_2 = \beta^{n_c}, \quad \|\Zmbf_i^{n_c} - \Imbf\|_2 = \beta_i^{n_c},  \qquad \forall \,\, i = 1, 2, 3, 4.
\end{align}
We also define,
\begin{align}\label{eq : derivative terms define}
   h_{k, j} = \frac{1}{n} \sum_{i = 1}^n \nabla f_i(x_{i, k, j}), \quad \hbar_{k, j} = \frac{1}{n} \sum_{i = 1}^n \nabla f_i(\xbar_{k, j}), \quad \mbox{and} \quad \Imbf = \frac{1_n1_n^T}{n} \otimes I_d .
\end{align}
where $x_{i,k,j}$, denotes the local copy of the $i$th node, at outer iteration $k$ and inner iteration $j$. 
In the analysis, for all $k\geq 0$, we consider the following error vector
\begin{align*} %\label{eq : error vector def}
    r_k = \begin{bmatrix}
        \|\xbar_{k,1} - x^*\|_2\\
        \|\xmbf_{k,1} - \Bar{\xmbf}_{k,1}\|_2\\
        \|\ymbf_{k,1} - \Bar{\ymbf}_{k,1}\|_2\\
    \end{bmatrix}. %\quad \forall \,\, k \geq 0,
\end{align*}
The error vector $r_k$ combines the optimization error, $\|\xbar_{k,1} - x^*\|_2$, and consensus errors, $\|\xmbf_{k,1} - \Bar{\xmbf}_{k,1}\|_2$ and $\|\ymbf_{k,1} - \Bar{\ymbf}_{k,1}\|_2$ where $\xmbf_{k, 1}$ and $\ymbf_{k, 1}$ are the first iterates of outer iteration $k$. We establish general technical lemmas that quantify the relation between $r_{k+1}$ and $r_k$ for each case of the presented algorithm.  

%%%%%%%%%%%%%%%%%%%%%%%%%%%%%%%%%%%%%%%%%%%%%
% Deterministic Analysis for g = 0
%%%%%%%%%%%%%%%%%%%%%%%%%%%%%%%%%%%%%%%%%%%%%

\subsection{\texttt{GTA} with multiple communication \texorpdfstring{($n_c \geq 1, n_g=1$)}{Lg}} \label{sec.mult comms} In this section, we analyse \texttt{GTA} when only one computation step is performed per iteration. % i.e., $n_g = 1$. We use this special case to introduce the analysis as it is simpler to follow, and extend it to the case where $n_g \geq 1$ in \cref{sec.mult grads}. 
In this setting ($n_g=1$), the inner loop (Lines 4--6 in \cref{alg : Deterministic}) is never executed. Thus, the inner iteration counter in \texttt{GTA} can be ignored and the iteration simplifies to 
\begin{equation}   \label{eq : g=1 general form}
\begin{aligned}
    \xmbf_{k+1} & = \Zmbf_1^{n_c} \xmbf_k - \alpha \Zmbf_2^{n_c} \ymbf_k,   \\ 
    \ymbf_{k+1} & = \Zmbf_3^{n_c} \ymbf_k + \Zmbf_4^{n_c} (\nabla \fmbf(\xmbf_{k+1}) - \nabla \fmbf(\xmbf_{k})).
\end{aligned}
\end{equation}
We note that throughout this subsection we drop the subscript related to the inner iteration $j$, i.e., $x_{i,k,j}$ is denoted as $x_{i,k}$ (since $j=1$), and similar with other quantities. 
We first establish the progression of the error vector $r_k$ as a linear system for \eqref{eq : g=1 general form}. Then, we provide the step size conditions and convergence rates for \eqref{eq : g=1 general form} and the instances in \cref{tab: Algorithm Def} when $n_g = 1$.

\begin{lemma}\label{lem:lyapunov g = 1}
    Suppose \cref{asum.convex and smooth} holds and the number of gradient steps in each outer iteration of \cref{alg : Deterministic} is set to one (i.e., $n_g=1$). If $\alpha \leq \tfrac{1}{L}$, then for all $k\geq 0$,
    \begin{align*}
      r_{k+1} \leq A(n_c) r_k, 
    \end{align*}
    % where
    \begin{align} \label{eq : g = 1 general A}
    \mbox{where} \quad A(n_c) = \begin{bmatrix}
        1 - \alpha \mu & \tfrac{\alpha L}{\sqrt{n}} & 0\\
        0 & \beta_1^{n_c} & \alpha\beta_2^{n_c}\\
        \sqrt{n}\alpha \beta_4^{n_c} L^2 & \beta_4^{n_c}L(\|\Zmbf_1^{n_c}-I_{nd}\|_2 + \alpha L) & \beta_3^{n_c} + \alpha \beta_4^{n_c} L\\
        \end{bmatrix}.
    \end{align}
\end{lemma}
\begin{proof}
If $n_g=1$, using \eqref{eq : g=1 general form}, the average iterates can be expressed as
\begin{align*}
    \Bar{x}_{k+1} & = \Bar{x}_k - \alpha \Bar{y}_k,   \\
    \Bar{y}_{k+1} & = \Bar{y}_k + h_{k+1} - h_{k},
\end{align*} 
where $h_k$ is defined in~\eqref{eq : derivative terms define}.
Taking the telescopic sum of $\Bar{y}_{i+1}$ from $i=0$ to $k-1$ with $\bar{y}_0 = h_0$, it follows that
\begin{align*}
    \Bar{y}_{k} & = h_k. \numberthis \label{eq:y_bar_telescope}
\end{align*}

We first consider the optimization error on the average iterates. That is, 
%Let's first consider the optimization error on the average iterates. That is, 
\begin{align*}
    \|\Bar{x}_{k+1} - x^*\|_2 & = \left\|\Bar{x}_k - \alpha \Bar{y}_k + \alpha \hbar_k - \alpha\hbar_k - x^*\right\|_2  \\
    & \leq \left\|\Bar{x}_k- \alpha\hbar_k - x^*\right\|_2 + \alpha \left\|\Bar{y}_k - \hbar_k\right\|_2  \\
    % & \leq (1-\alpha \mu) \|\Bar{x}_k - x^*\|_2 + \alpha \left\| \Bar{y}_k - \hbar_k\right\|_2   \\
    &\leq  (1-\alpha \mu) \|\Bar{x}_k - x^*\|_2 + \alpha \left\|h_k - \hbar_k\right\|_2 \\
    &= (1-\alpha \mu) \|\Bar{x}_k - x^*\|_2 + \tfrac{\alpha}{n}\left\|\sum_{i=1}^n \nabla f_i(x_{i, k}) - \nabla f_i(\Bar{x}_{k})\right\|_2 \\
    & \leq (1-\alpha \mu) \|\Bar{x}_k - x^*\|_2 + \tfrac{\alpha L}{n} \sum_{i=1}^n  \| x_{i, k} - \Bar{x}_{k}\|_2    \\
    & \leq (1-\alpha \mu) \|\Bar{x}_k - x^*\|_2 + \tfrac{\alpha L}{\sqrt{n}}  \| \xmbf_{k} - \Bar{\xmbf}_{k}\|_2    \numberthis \label{eq : g = 1 opt bound}
\end{align*}
where the first inequality is due to the triangle inequality, the second inequality is obtained by performing one gradient descent iteration on function $f$ under \cref{asum.convex and smooth} at the average iterate $\bar{x}_k$ with $\alpha \leq \tfrac{1}{L}$ \cite[Theorem 2.1.14]{nesterov1998introductory} and substituting using \eqref{eq:y_bar_telescope},  the equality is due to \eqref{eq : derivative terms define}, the second to last inequality follows by \cref{asum.convex and smooth}, and the last inequality is due to  $\sum_{i=1}^n \|x_{i, k} - \xbar_k\|_2 \leq \sqrt{n}\|\xmbf_{k} - \Bar{\xmbf}_{k}\|_2$.
% where in the first inequality is using triangle inequality. In the second inequality, we used the bound obtained by performing one gradient descent iteration on function $f$ under \cref{asum.convex and smooth} at the average iterate $\bar{x}_k$ with $\alpha \leq \frac{1}{L}$ \cite[Theorem 2.1.14]{nesterov1998introductory}. Then we subsitutue $\ybar_k$ using \eqref{eq:y_bar_telescope}. The second last inequality is from \cref{asum.convex and smooth} and the last inequality is using $\sum_{i=1}^n \|x_{i, k} - \xbar_k\|_2 \leq \sqrt{n}\|\xmbf_{k} - \Bar{\xmbf}_{k}\|_2$.  

Next, we consider the consensus error in $\xmbf_k$,
%consider the consensus error in $\xmbf_k$.
\begin{align*}
    \xmbf_{k+1} - \Bar{\xmbf}_{k+1} &= \Zmbf_1^{n_c}\xmbf_k - \Bar{\xmbf}_{k} - \alpha \Zmbf_2^{n_c}\ymbf_k + \alpha \Bar{\ymbf}_k \\
    & = \Zmbf_1^{n_c}\xmbf_k - \Zmbf_1^{n_c}\Bar{\xmbf}_{k} - \alpha \Zmbf_2^{n_c}\ymbf_k + \alpha \Zmbf_2^{n_c}\Bar{\ymbf}_k  - \Imbf (\xmbf_k - \Bar{\xmbf}_k) + \Imbf (\ymbf_k - \Bar{\ymbf}_k)\\
    & = \left(\Zmbf_1^{n_c} - \Imbf\right)(\xmbf_k - \Bar{\xmbf}_k) - \alpha\left(\Zmbf_2^{n_c} - \Imbf\right)(\ymbf_k - \Bar{\ymbf}_k). 
\end{align*}
where the second equality follows from adding $- \Imbf (\xmbf_k - \Bar{\xmbf}_k) = 0$ and $\Imbf (\ymbf_k - \Bar{\ymbf}_k) = 0$. By the triangle inequality and~\eqref{eq : beta and Z}, 
% Therefore using triangle inequality, 
\begin{equation}\label{eq : g = 1 x con error}
\begin{aligned}
    \|\xmbf_{k+1} - \Bar{\xmbf}_{k+1}\|_2 & \leq \left\|\Zmbf_1^{n_c} - \Imbf\right\|_2 \|\xmbf_k - \Bar{\xmbf}_k\|_2 + \alpha \left\|\Zmbf_2^{n_c} - \Imbf\right\|_2 \|\ymbf_k - \Bar{\ymbf}_k\|_2   \\ 
    & = \beta_1^{n_c} \|\xmbf_k - \Bar{\xmbf}_k\|_2 + \alpha \beta_2^{n_c} \|\ymbf_k - \Bar{\ymbf}_k\|_2.  
\end{aligned}
\end{equation}

Finally, we consider the consensus error in $\ymbf_k$. By the triangle inequality and~\eqref{eq : beta and Z},
\begin{equation}\label{eq: y_bar_bnd_twoterm}
\begin{aligned}
    &\left\|\ymbf_{k+1} - \Bar{\ymbf}_{k+1}\right\|_2 \\
    =& \left\|\Zmbf_3^{n_c}\ymbf_k - \Bar{\ymbf}_{k} + \Zmbf_4^{n_c} (\nabla \fmbf(\xmbf_{k+1}) - \nabla \fmbf(\xmbf_{k})) - \Imbf (\nabla \fmbf(\xmbf_{k+1}) - \nabla \fmbf(\xmbf_{k}))\right\|_2 \\
     \leq& \left\|\left(\Zmbf_3^{n_c} - \Imbf\right)(\ymbf_k - \Bar{\ymbf}_k)\right\|_2 + \left\|\left(\Zmbf_4^{n_c} - \Imbf\right)(\nabla \fmbf(\xmbf_{k+1}) - \nabla \fmbf(\xmbf_{k}))\right\|_2 \\
    \leq& \beta_3^{n_c}\|\ymbf_k - \Bar{\ymbf}_k\|_2 + \beta_4^{n_c} \left\|\nabla \fmbf(\xmbf_{k+1}) - \nabla \fmbf(\xmbf_{k}) \right\|_2.
\end{aligned}
\end{equation}
%where the first inequality is using triangle inequality and we substitute \eqref{eq : beta and Z} at the end. 
The last term in \eqref{eq: y_bar_bnd_twoterm} can be bounded as follows,
\begin{align*}
\left\|\nabla \fmbf(\xmbf_{k+1}) - \nabla \fmbf(\xmbf_{k}) \right\|_2 
&\leq L\|\xmbf_{k+1} - \xmbf_{k}\|_2 \\
&= L\|\Zmbf_1^{n_c}\xmbf_{k} - \alpha \Zmbf_2^{n_c}\ymbf_k - \xmbf_{k}\|_2   \\
% &= L\|(\Zmbf_1^{n_c}-I_{nd})\xmbf_{k} - (\Zmbf_1^{n_c} - I_{nd})\Bar{\xmbf}_k  - \alpha \Zmbf_2^{n_c}\ymbf_k\|_2 \\
&= L\|(\Zmbf_1^{n_c}-I_{nd})(\xmbf_{k} - \Bar{\xmbf}_k) - \alpha \Zmbf_2^{n_c}\ymbf_k\|_2 \\
&\leq L\|\Zmbf_1^{n_c}-I_{nd}\|_2\|\xmbf_{k} - \Bar{\xmbf}_k\|_2 + \alpha L\|\Zmbf_2^{n_c}\|_2\|\ymbf_k + \Bar{\ymbf}_k - \Bar{\ymbf}_k\|_2   \\
&\leq L\|\Zmbf_1^{n_c}-I_{nd}\|_2\|\xmbf_{k} - \Bar{\xmbf}_k\|_2 + \alpha L\|\ymbf_k - \Bar{\ymbf}_k\|_2 + \alpha L\left\|\Bar{\ymbf}_k\right\|_2,\numberthis \label{eq: y_bar_bnd_twoterm33}
\end{align*}
where the first inequality is due to \cref{asum.convex and smooth}, the first equality is due to iterate update form \eqref{eq : g=1 general form}, the second equality is by adding $-(\Zmbf_1^{n_c} - I_{nd})\xbb_{k} = 0$ and the last two inequalities are applications of the triangle inequality.
Next we bound the term $\left\|\Bar{\ymbf}_k\right\|_2$. By \eqref{eq:y_bar_telescope}, \cref{asum.convex and smooth} and $\sum_{i=1}^n \|x_{i, k} - \xbar_k\|_2 \leq \sqrt{n}\|\xmbf_{k} - \Bar{\xmbf}_{k}\|_2$, %it follows that,
\begin{align*}
\left\|\Bar{\ymbf}_k\right\|_2 & \leq \sqrt{n} \|\bar{y}_k\|_2 \\
&= \sqrt{n} \|h_k\|_2 \\
&\leq \sqrt{n}\left\|\tfrac{1}{n} \sum_{i = 1}^n \nabla f_i(x_{i, k}) -  \tfrac{1}{n} \sum_{i = 1}^n \nabla f_i(\bar{x}_{k})\right\|_2 + \sqrt{n}\left\|\tfrac{1}{n} \sum_{i = 1}^n \nabla f_i(\bar{x}_{k})\right\|_2  \\
&= \tfrac{1}{\sqrt{n}}\left\| \sum_{i = 1}^n \nabla f_i(x_{i, k}) -  \sum_{i = 1}^n \nabla f_i(\bar{x}_{k})\right\|_2 + \tfrac{1}{\sqrt{n}} \left\| \sum_{i = 1}^n \nabla f_i(\bar{x}_{k}) - \sum_{i = 1}^n \nabla f_i(x^*)\right\|_2 \\
% &\leq \frac{1}{\sqrt{n}}\left\|\nabla f(x_{k}) - \nabla f(\bar{x}_k)\right\|_2 + \frac{1}{\sqrt{n}}\left\|\nabla f(\bar{x}_k) - \nabla f((x^*)^T)\right\|_2 \\
&\leq L\left\|\xmbf_k - \bar{\xmbf}_k\right\|_2 + \sqrt{n} L \| \Bar{x}_{k} - x^*\|_2. \numberthis \label{eq : y_bar_bound}
\end{align*}
Thus, by \eqref{eq: y_bar_bnd_twoterm}, \eqref{eq: y_bar_bnd_twoterm33} and \eqref{eq : y_bar_bound}, it follows that
\begin{equation}\label{eq.y_result}
\begin{aligned}
    \left\|\ymbf_{k+1} - \Bar{\ymbf}_{k+1}\right\|_2 
    % &\leq \beta_3^{n_c}\|\ymbf_k - \Bar{\ymbf}_k\|_2 + \beta_4^{n_c} \left\|\nabla \fmbf(\xmbf_{k+1}) - \nabla \fmbf(\xmbf_{k}) \right\|_2 \\
    % &\leq \beta_3^{n_c}\|\ymbf_k - \Bar{\ymbf}_k\|_2 \\
    % & \quad + \beta_4^{n_c} \left( L\|\Zmbf_1^{n_c}-I_{nd}\|_2\|\xmbf_{k} - \Bar{\xmbf}_k\|_2 + \alpha L\|\ymbf_k - \Bar{\ymbf}_k\|_2 + \alpha L\left\|\Bar{\ymbf}_k\right\|_2\right)\\
    % &\leq \beta_3^{n_c}\|\ymbf_k - \Bar{\ymbf}_k\|_2 \\
    % & \quad + \beta_4^{n_c} \left( L\|\Zmbf_1^{n_c}-I_{nd}\|_2\|\xmbf_{k} - \Bar{\xmbf}_k\|_2 + \alpha L\|\ymbf_k - \Bar{\ymbf}_k\|_2 + \alpha L\left( L\left\|\xmbf_k - \bar{\xmbf}_k\right\|_2 + L\sqrt{n} \| \Bar{x}_{k} - x^*\|_2\right) \right)\\
    &\leq \beta_4^{n_c}\sqrt{n}\alpha L^2 \| \Bar{x}_{k} - x^*\|_2 + \beta_4^{n_c}L\left( \|\Zmbf_1^{n_c}-I_{nd}\|_2 + \alpha L\right) \|\xmbf_{k} - \Bar{\xmbf}_k\|_2\\
    % & \quad + \beta_4^{n_c}L\left( \|\Zmbf_1^{n_c}-I_{nd}\|_2 + \alpha L\right) \|\xmbf_{k} - \Bar{\xmbf}_k\|_2 \\
    & \qquad +  \left( \beta_3^{n_c} + \beta_4^{n_c}\alpha L\right) \|\ymbf_k - \Bar{\ymbf}_k\|_2.
\end{aligned}
\end{equation}
Combining \eqref{eq : g = 1 opt bound}, \eqref{eq : g = 1 x con error} and \eqref{eq.y_result} concludes the proof.
% Substituting the above bound in \eqref{eq: y_bar_bnd_twoterm}, we get, 
% \begin{align*}
%     \left\|\ymbf_{k+1} - \Bar{\ymbf}_{k+1}\right\| &\leq  (\beta_3^{n_c} + \alpha \beta_4^{n_c} L) \|\ymbf_k - \Bar{\ymbf}_k\|_2 + \beta_4^{n_c}L\|\Zmbf_1^{n_c}-I_{nd}\|_2\|\xmbf_{k} - \Bar{\xmbf}_k\|_2 + \alpha L \beta_4^{n_c}\left\|\Bar{\ymbf}_k\right\|_2 \numberthis \label{eq:y_bar_bnd_threeterm}
% \end{align*}
% Now, let's analyze the last term on the right hand side in the above inequality.
% \begin{align*}
% \left\|\Bar{\ymbf}_k\right\|_2 & \leq \sqrt{n} \|\bar{y}_k\|_2 = \sqrt{n} \|h_k\|_2 \\
% &\leq \sqrt{n}\left\|\frac{1}{n} \sum_{i = 1}^n \nabla f_i(x_{i, k}) -  \frac{1}{n} \sum_{i = 1}^n \nabla f_i(\bar{x}_{k})\right\|_2 + \sqrt{n}\left\|\frac{1}{n} \sum_{i = 1}^n \nabla f_i(\bar{x}_{k})\right\|_2  \\
% &\leq \frac{1}{\sqrt{n}}\left\| \sum_{i = 1}^n \nabla f_i(x_{i, k}) -  \sum_{i = 1}^n \nabla f_i(\bar{x}_{k})\right\|_2 + \frac{1}{\sqrt{n}} \left\| \sum_{i = 1}^n \nabla f_i(\bar{x}_{k}) - \sum_{i = 1}^n \nabla f_i(x^*)\right\|_2 \\
% % &\leq \frac{1}{\sqrt{n}}\left\|\nabla f(x_{k}) - \nabla f(\bar{x}_k)\right\|_2 + \frac{1}{\sqrt{n}}\left\|\nabla f(\bar{x}_k) - \nabla f((x^*)^T)\right\|_2 \\
% &\leq L\left\|\xmbf_k - \bar{\xmbf}_k\right\|_2 + L\sqrt{n} \| \Bar{x}_{k} - x^*\|_2 \numberthis \label{eq : y_bar_bound}
% \end{align*}
% The first equality is using \eqref{eq:y_bar_telescope}. The last inequality is due to \cref{asum.convex and smooth} and $\sum_{i=1}^n \|x_{i, k} - \xbar_k\|_2 \leq \sqrt{n}\|\xmbf_{k} - \Bar{\xmbf}_{k}\|_2$.
% Substituting the above bound into \eqref{eq:y_bar_bnd_threeterm} gives the desired bound on the consensus error in $\ymbf_k$. Combining \eqref{eq : g = 1 opt bound}, \eqref{eq : g = 1 x con error} and \eqref{eq:y_bar_bnd_threeterm} gives the desired form in \eqref{eq : g = 1 general A}
\end{proof}

Using \cref{lem:lyapunov g = 1}, we now provide the explicit form for $A(n_c)$ in order to establish the progression of the error vector $r_k$ for the special cases defined in \cref{tab: Algorithm Def}.
% to establish the progression of error vector $r_k$ for the special cases defined in \cref{tab: Algorithm Def} defined over the mixing matrix $\Wmbf$.

\begin{corollary} \label{col. A special cases}
Suppose the conditions of \cref{lem:lyapunov g = 1} are satisfied. Then, the matrices $A(n_c)$ for the methods described in \cref{tab: Algorithm Def} are defined as:%$A(n_c)$ for the methods given in \cref{tab: Algorithm Def} is as follows: 
\begin{align*}
    \mbox{\texttt{GTA-1}:} & \quad A_1({n_c}) = \begin{bmatrix}
            1 - \alpha \mu & \tfrac{\alpha L}{\sqrt{n}} & 0\\
            0 & \beta^{n_c} & \alpha\\
            \sqrt{n}\alpha L^2 & L(2 + \alpha L) & \beta^{n_c} + \alpha L\\
        \end{bmatrix},\\
    \mbox{\texttt{GTA-2}:} & \quad A_2({n_c})  = \begin{bmatrix}
            1 - \alpha \mu & \tfrac{\alpha L}{\sqrt{n}} & 0\\
            0 & \beta^{n_c} & \alpha \beta^{n_c}\\
            \sqrt{n}\alpha L^2 & L(2 + \alpha L) & \beta^{n_c} + \alpha L\\
        \end{bmatrix},   \numberthis \label{eq : g = 1 algos A}\\
    \mbox{\texttt{GTA-3}:} & \quad A_3({n_c}) = \begin{bmatrix}
            1 - \alpha \mu & \tfrac{\alpha L}{\sqrt{n}} & 0\\
            0 & \beta^{n_c} & \alpha \beta^{n_c}\\
            \beta^{n_c} \sqrt{n}\alpha L^2 & \beta^{n_c}L(2 + \alpha L) & \beta^{n_c}(1 + \alpha L)\\
        \end{bmatrix}.
\end{align*}
\end{corollary}
\begin{proof}
Substituting the matrix values for each method in \eqref{eq : g = 1 general A} and using $\|\Zmbf_1^{n_c} - I_{nd}\|_2 \leq 2$ gives the desired result. 
\end{proof}
The convergence properties of \texttt{GTA} when $n_g=1$ can be analyzed using the spectral radius of the matrix $A({n_c})$. We now qualitatively establish the effect of $n_c$ on $\rho(A({n_c}))$, the spectral radius of the matrix $A({n_c})$, and the relative ordering between $\rho(A_1({n_c}))$,  $ \rho(A_2({n_c}))$ and $ \rho(A_3({n_c}))$.

\btheorem   \label{th.incr rates}
Suppose \cref{asum.convex and smooth} holds and the number of gradient steps in each outer iteration of  \cref{alg : Deterministic} is set to one (i.e., $n_g=1$). If $\alpha \leq \tfrac{1}{L}$, then as ${n_c}$ increases, $\rho(A({n_c}))$ decreases where $A(n_c)$ is defined in \eqref{eq : g = 1 general A}. Thus, as ${n_c}$ increases, $\rho(A_i({n_c}))$ decreases, for $i=1,2,3$  defined in \eqref{eq : g = 1 algos A}. Moreover, if all three methods described in \cref{tab: Algorithm Def} ({\texttt{GTA-1}}, \texttt{GTA-2} and \texttt{GTA-3}) employ the same step size, 
\begin{align*}
\rho(A_1({n_c})) \geq \rho(A_2({n_c})) \geq \rho(A_3({n_c})),
\end{align*}
where the matrices $A_1({n_c})$, $A_2({n_c})$ and $A_3({n_c})$ are defined in \eqref{eq : g = 1 algos A}.
% Moreover, as ${n_c}$ increases, $\rho(A_i({n_c}))$ decreases, for $i=1,2,3$. 
\etheorem
\bproof
Note that $A(n_c) \geq 0$ and $A(n_c) \geq A(n_c + 1) \geq 0$.  By \cite[Corollary 8.1.19]{horn2012matrix}, it follows that $\rho(A(n_c)) \geq \rho(A(n_c + 1))$. The same argument is applicable for $A_1({n_c})$, $A_2({n_c})$ and $A_3({n_c})$. Now, observe that $A_1({n_c}) \geq A_2({n_c}) \geq A_3({n_c}) \geq 0$ when the same step size is employed. Thus, again by \cite[Corollary 8.1.19]{horn2012matrix}, it follows that $\rho(A_1({n_c})) \geq \rho(A_2({n_c})) \geq \rho(A_3({n_c}))$.
\eproof

% step size conditions g = 1
%%%%%%%%%%%%%%%%%%%%%%%%%%%%%%%%%%%%%%%%%%%%%
We now derive  conditions for establishing a  linear rate of convergence to the solution for \cref{alg : Deterministic} when $n_g = 1$ in terms of network parameters ($\beta_1, \beta_2, \beta_3, \beta_4$) and objective function parameters ($L$, $\mu$, $\kappa = \tfrac{L}{\mu}$).

\btheorem \label{th. general g=1 step cond}
 Suppose \cref{asum.convex and smooth} holds and the number of gradient steps at each outer iteration of \cref{alg : Deterministic} is set to one (i.e., $n_g=1$). If the matrix $A({n_c})$ is irreducible, $\beta_1, \beta_3 < 1$ and 
 \begin{align} \label{eq : g = 1 gen step cond}
    \alpha < \min \left\{\tfrac{1}{L}, \tfrac{1 - \beta_3^{n_c}}{L\beta_4^{n_c}} , \tfrac{(1 - \beta_1^{n_c} + 2\beta_2^{n_c})}{2\beta_2^{n_c}\kappa(L + \mu)} \left(\sqrt{1 + \tfrac{4(1-\beta_1^{n_c})(1-\beta_3^{n_c})\beta_2^{n_c}(\kappa+1)}{\beta_4^{n_c}(1 - \beta_1^{n_c} + 2\beta_2^{n_c})^2}} - 1\right)\right\},
 \end{align}
%  and $\beta_1, \beta_3 < 1$, then there exists a sequence $\epsilon_k\geq 0$ such that,
% \begin{align*}
%     \|r_{k}\|_2 \leq (\rho(A({n_c})) + \epsilon_k)^k \|r_0\|_2 \quad \mbox{and} \quad \lim_{k \rightarrow \infty} \epsilon_k = 0
% \end{align*}
% for all $k\geq 0$, where $\rho(A({n_c})) < 1$.
then, for all $\epsilon > 0$ there exists a constant $C_\epsilon>0$ such that, for all $k\geq 0$,
\begin{align*}
    \|r_{k}\|_2 \leq C_\epsilon(\rho(A({n_c})) + \epsilon)^k \|r_0\|_2, \quad \text{where } \; \rho(A({n_c})) < 1.
\end{align*}
%where $\rho(A({n_c})) < 1$.
\etheorem

\bproof
% Using \cite[Lemma 5]{pu2021distributed}, 
Following \cite[Lemma 5]{pu2021distributed}, derived from the Perron-Forbenius Theorem \cite[Theorem 8.4.4]{horn2012matrix} for a $3\times3$ matrix, when the matrix $A({n_c})$ is non-negative and irreducible, it is sufficient to show that the diagonal elements of $A({n_c})$ are less than one and that $\det(I_3 - A({n_c})) > 0$ in order to guarantee $\rho(A({n_c})) < 1$. We upper bound $\|\Zmbf_1^{n_c} - I_{nd}\|_2 \leq 2$ in $A(n_c)$ for the results.

Let us first consider the diagonal elements of the matrix $A({n_c})$. The first element is, $1 - \alpha \mu \leq 1 -\tfrac{\mu}{L} < 1$ by \eqref{eq : g = 1 gen step cond}. The second element is $\beta_1^{n_c} < 1$ as $\beta_1 < 1$. Finally, the third element is $\beta_3^{n_c} + \alpha\beta_4^{n_c}L < \beta_3^{n_c} + \tfrac{1 - \beta_3^{n_c}}{\beta_4^{n_c}L}\beta_4^{n_c}L = 1$ due to \eqref{eq : g = 1 gen step cond} and $\beta_3 < 1$.

Next, let us consider
\begin{align*}
&\det(I_3 - A({n_c}))\\
= &-\alpha(\alpha^2L^2\beta_2^{n_c}\beta_4^{n_c}\left(L + \mu \right) + \alpha\mu L\beta_4^{n_c}\left(1 - \beta_1^{n_c} + 2\beta_2^{n_c}\right) - \mu\left(1 - \beta_1^{n_c}\right)\left(1 - \beta_3^{n_c}\right)) \\
=&-L^2\beta_2^{n_c}\beta_4^{n_c}(L + \mu)\alpha(\alpha - \alpha_l)(\alpha - \alpha_u), 
\end{align*}
% where $\alpha_l = \alpha_1 + \alpha_2$, $\alpha_u = \alpha_1 - \alpha_2$, and 
% \begin{align*}
% \alpha_1 = \tfrac{-(1 - \beta_1^{n_c} + 2\beta_2^{n_c})}{2\beta_2^{n_c}\kappa(L + \mu)} \quad \text{and} \quad 
% \alpha_2 = -\alpha_1\sqrt{1 + \tfrac{4(1-\beta_1^{n_c})(1-\beta_3^{n_c})\beta_2^{n_c}(\kappa+1)}{\beta_4^{n_c}(1 - \beta_1^{n_c} + 2\beta_2^{n_c})^2}}. 
% \end{align*}
% Observe that $\alpha_l < 0 < \alpha_u$ since $\alpha_1 < 0$ and $\alpha_2 > |\alpha_1|$. From \eqref{eq : g = 1 gen step cond}, we have $0<\alpha < \alpha_2$. Therefore, $\det(I_3 - A({n_c})) > 0$, which combined with the fact that the diagonal elements of the matrix are less than 1, implies $\rho(A(n_c)) < 1$.
where $\alpha_l = \alpha_1 - \alpha_2$, $\alpha_u = \alpha_1 + \alpha_2$, and 
\begin{align*}
\alpha_1 = \tfrac{-(1 - \beta_1^{n_c} + 2\beta_2^{n_c})}{2\beta_2^{n_c}\kappa(L + \mu)} \quad \text{and} \quad 
\alpha_2 = -\alpha_1\sqrt{1 + \tfrac{4(1-\beta_1^{n_c})(1-\beta_3^{n_c})\beta_2^{n_c}(\kappa+1)}{\beta_4^{n_c}(1 - \beta_1^{n_c} + 2\beta_2^{n_c})^2}}. 
\end{align*}
Observe that $\alpha_l < 0 < \alpha_u$ and $\alpha_2 > |\alpha_1|$.
%Observe that $\alpha_l < 0 < \alpha_u$ since $\alpha_1 < 0$ and $\alpha_2 > |\alpha_1|$. 
From \eqref{eq : g = 1 gen step cond}, we have $0<\alpha < \alpha_u$. Therefore, $\det(I_3 - A({n_c})) > 0$, which combined with the fact that the diagonal elements of the matrix are less than 1, implies $\rho(A(n_c)) < 1$.

% \vrb{Now, let us consider 
% \begin{align*}
% &
% \det(I_3 - A({n_c}))\\
% &~~= -\alpha(\alpha^2L^2\beta_2^{n_c}\beta_4^{n_c}\left(L + \mu \right) + \alpha\mu L\beta_4^{n_c}\left(1 - \beta_1^{n_c} + 2\beta_2^{n_c}\right) - \mu\left(1 - \beta_1^{n_c}\right)\left(1 - \beta_3^{n_c}\right)) \\
% &~~=-\alpha(\alpha - \alpha_l)(\alpha - \alpha_u) 
% \end{align*}
% where $\alpha_l = \alpha_1 + \alpha_2$, $\alpha_u = \alpha_1 - \alpha_2$, and 
% \begin{align*}
% \alpha_1 &= \tfrac{-(1 - \beta_1^{n_c} + 2\beta_2^{n_c})}{2\beta_2^{n_c}\kappa(L + \mu)} \\
% \alpha_2 &= -\alpha_1\sqrt{1 + \tfrac{4(1-\beta_1^{n_c})(1-\beta_3^{n_c})\beta_2^{n_c}(\kappa+1)}{\beta_4^{n_c}(1 - \beta_1^{n_c} + 2\beta_2^{n_c})^2}}. 
% \end{align*}

% Observe that $\alpha_l < 0 < \alpha_u$ since $\alpha_1 < 0$ and $\alpha_2 > |\alpha_1|$. From \eqref{eq : g = 1 gen step cond}, we have $0<\alpha < \alpha_2$. Therefore, $\det(I_3 - A({n_c})) > 0$ which implies $\rho(A(n_c)) < 1$. 
% }

% Now, let us consider the condition $\det(I_3 - A({n_c})) > 0$.
% \begin{align*}
% % &\det(I_3 - A({n_c})) \\
% &\alpha \mu \left((1-\beta_1^{n_c})(1-\beta_3^{n_c} - \alpha L\beta_4^{n_c}) - \alpha \beta_2^{n_c}\beta_4^{n_c}L(2 + \alpha L)\right) + \frac{\alpha L}{\sqrt{n}}\left(-\alpha^2 L^2\beta_2^{n_c}\beta_4^{n_c} \sqrt{n}\right) > 0\\
% &-\alpha \left(\alpha^2\left(L^3\beta_2^{n_c}\beta_4^{n_c} + \mu L^2\beta_2^{n_c}\beta_4^{n_c}\right) + \alpha\left(L\mu\beta_4^{n_c}(1- \beta_1^{n_c}) + 2L\mu\beta_2^{n_c}\beta_4^{n_c}\right) \right. > 0\\
% & \quad \left.- \mu\left(1 - \beta_1^{n_c}\right)\left(1 - \beta_3^{n_c}\right)\right)
% \end{align*}
% % Setting $\det(I_3 - A({n_c})) > 0$ yields the following quadratic inequality in $\alpha$
% As $\alpha > 0$, the condition reduces to the following quadratic inequality
% \begin{align*}
% \alpha^2L^2\beta_2^{n_c}\beta_4^{n_c}\left(L + \mu \right) + \alpha\mu L\beta_4^{n_c}\left(1 - \beta_1^{n_c} + 2\beta_2^{n_c}\right) - \mu\left(1 - \beta_1^{n_c}\right)\left(1 - \beta_3^{n_c}\right) < 0,
% \end{align*}
% % where the roots, after minor simplification and defining $\kappa = \frac{L}{\mu}$, are given as, 
% % \begin{align*}
% %    \frac{-(1 - \beta_1^{n_c} + 2\beta_2^{n_c})}{2\beta_2^{n_c}\kappa(L + \mu)} \pm \frac{\sqrt{(1-\beta_1^{n_c} + 2\beta_2^{n_c})^2 + 4(\kappa + 1)\left(1 - \beta_1^{n_c}\right)\left(1 - \beta_3^{n_c}\right)\beta_2^{n_c}\beta_4^{-{n_c}}}}{2\beta_2^{n_c}\kappa(L + \mu)}.
% % \end{align*}
% % Therefore, both the roots are real as all the terms inside the square root are positive. Thus, the quadratic inequality with $\alpha > 0$ is satisfied when
% with roots $\alpha_1$ and $\alpha_2$ such that $\alpha_1 < 0 < \alpha_2$. Thus, the condition requires $\alpha < \alpha_2$ which is satisfied by \eqref{eq : g = 1 gen step cond} as
% \begin{align*}
% \alpha < \frac{-(1 - \beta_1^{n_c} + 2\beta_2^{n_c})}{2\beta_2^{n_c}\kappa(L + \mu)} + \frac{(1 - \beta_1^{n_c} + 2\beta_2^{n_c})}{2\beta_2^{n_c}\kappa(L + \mu)}\sqrt{1 + \frac{4(1-\beta_1^{n_c})(1-\beta_3^{n_c})\beta_2^{n_c}(\kappa+1)}{\beta_4^{n_c}(1 - \beta_1^{n_c} + 2\beta_2^{n_c})^2}}.
% \end{align*}
% % as this upper bound is the larger positive root of the quadratic. Observe that the bounding quantity is positive, thus there exists a positive step size that can satisfy \eqref{eq : g = 1 gen step cond}. 
% % Now, we observe that,
% % \begin{align*}
% %     \|r_{k}\|_2 &\leq \|A({n_c})^k\|_2\|r_0\|_2,
% % \end{align*}
% % and using \cite[Theorem 5.6.12]{horn2012matrix}, we can bound the matrix norm with a spectral norm using some sequence $\epsilon_k \geq 0$ with $\lim_{k \rightarrow \infty} \epsilon_k = 0$.
% Thus given \eqref{eq : g = 1 gen step cond}, $\rho(A(n_c)) < 1$. 

% Finally, we bound the norm of error vector $\|r_k\|_2$ by telescoping $r_{i+1} \leq A(n_c) r_{i}$ from $i = 0$ to $k-1$ and triangle inequality as
% \begin{align*}
%     \|r_{k}\|_2 &\leq \|A({n_c})^k\|_2\|r_0\|_2.
% \end{align*}
% From \cite[Corollary 5.6.13]{horn2012matrix}, we can bound $\|A({n_c})^k\|_2 \leq C(\rho(A(n_c)) + \epsilon)^k$ where $\epsilon > 0$ and $C$ is a constant matrix depending on $A(n_c)$ and $\epsilon$. Using, \cite[Theorem 5.6.12]{horn2012matrix}, $\lim_{k \rightarrow \infty} A(n_c)^k = 0$ as $\rho(A(n_c)) < 1$. Thus, we can bound the matrix norm with a spectral radius using some sequence $\epsilon_k \geq 0$ with $\lim_{k \rightarrow \infty} \epsilon_k = 0$ and obtain the desired result.

Finally, we bound the norm of error vector $\|r_k\|_2$ by telescoping $r_{i+1} \leq A(n_c) r_{i}$ from $i = 0$ to $k-1$ and triangle inequality as
\begin{align*}
    \|r_{k}\|_2 &\leq \|A({n_c})^k\|_2\|r_0\|_2.
\end{align*}
From \cite[Corollary 5.6.13]{horn2012matrix}, we can bound $\|A({n_c})^k\|_2 \leq C_{\epsilon}(\rho(A(n_c)) + \epsilon)^k$ where $\epsilon > 0$ and $C_{\epsilon}$ is a positive constant that depends on $A(n_c)$ and $\epsilon$.
\eproof

The only constraint \cref{th. general g=1 step cond} imposes on the system (network) is $\beta_1, \beta_3 < 1$. This implies that the communication matrices $\Wmbf_1$ and $\Wmbf_3$ must represent connected networks (not necessarily the same network). Properties of $\Wmbf_2$ and $\Wmbf_4$ change the step size requirement but are not part of the sufficient conditions for convergence. \cref{th. general g=1 step cond} also does not require any relation among $\Wmbf_1$, $\Wmbf_2$, $\Wmbf_3$ and $\Wmbf_4$. This allows for more flexibility than the structures considered in the literature. %This shows a greater generalization than ATC and CTA structures discussed in literature. 
The variables can be communicated along different connections within the network. %One drawback of %gradient tracking methods is the requirement \rb{to %communicate two} %of communication of 2 
%vectors instead of one, resulting in higher communication bandwidth. This limitation can be overcome using our general structure by limiting communication of variables on certain edges as long as \cref{th. general g=1 step cond} conditions are met. 
We note that if $A(n_c)$ is a reducible matrix, the analysis for the progression of $r_k$ can be further simplified from \cref{lem:lyapunov g = 1}. For example, when 
% \asb{In this case,}
$\Wmbf = \tfrac{1_n1_n^T}{n}$, i.e., $\beta = 0$, in \texttt{GTA-2} and \texttt{GTA-3}. 
The analysis for these cases is presented in \cref{sec.full graph res}. 

The next result establishes step size conditions that guarantee a linear rate of convergence for the three special cases (\texttt{GTA-1}, \texttt{GTA-2} and \texttt{GTA-3}).
% We next establish the step size conditions for linear rate of convergence for our special cases \texttt{GTA-1}, \texttt{GTA-2} amd \texttt{GTA-3}.

\bcorollary \label{col. g=1 step cond}
Suppose \cref{asum.convex and smooth} holds, $\Wmbf \neq \tfrac{1_n1_n^T}{n}$, and the number of gradient steps at each outer iteration of \cref{alg : Deterministic} is set to one (i.e., $n_g=1$). If the following step size conditions hold for the methods described in \cref{tab: Algorithm Def},
\begin{align*}
    \mbox{\texttt{GTA-1}:} & \quad \alpha < \min \left\{ \tfrac{1 - \beta^{n_c}}{L} , \tfrac{(3 - \beta^{n_c})}{2\kappa(L + \mu)}\left(\sqrt{1 + 4(\kappa + 1)\left( \tfrac{1 - \beta^{n_c}}{3 - \beta^{n_c}} \right)^2} - 1\right) \right\},\\
    \mbox{\texttt{GTA-2}:} & \quad \alpha < \min \left\{\tfrac{1 - \beta^{n_c}}{L} , \tfrac{(1+\beta^{n_c})}{2\kappa(L + \mu)\beta^{n_c}} \left( \sqrt{1 + 4(\kappa + 1)\beta^{n_c}\left(\tfrac{1 - \beta^{n_c}}{1+\beta^{n_c}}\right)^2} - 1\right)  \right\}, \\
    \mbox{\texttt{GTA-3}:} & \quad \alpha < \min \left\{\tfrac{1}{L}, \tfrac{1 - \beta^{n_c}}{L\beta^{n_c}} , \tfrac{(1 + \beta^{n_c})}{2\kappa(L + \mu)\beta^{n_c}} \left(\sqrt{1 + 4(\kappa + 1)\left(\tfrac{1 - \beta^{n_c}}{1 + \beta^{n_c}}\right)^2} - 1\right)\right\},
\end{align*}
% \begin{align*}
%     \intertext{For \texttt{GTA-1}}       
%     &\alpha < \min \left\{ \tfrac{1 - \beta^{n_c}}{L} , \tfrac{(3 - \beta^{n_c})}{2\kappa(L + \mu)}\left[\sqrt{1 + 4(\kappa + 1)\left[ \tfrac{1 - \beta^{n_c}}{3 - \beta^{n_c}} \right]^2} - 1\right] \right\} 
%     \intertext{For \texttt{GTA-2}}   
%     &\alpha < \min \left\{\tfrac{1 - \beta^{n_c}}{L} , \tfrac{(1+\beta^{n_c})}{2\kappa(L + \mu)\beta^{n_c}} \left[ \sqrt{1 + 4(\kappa + 1)\beta^{n_c}\left[\tfrac{1 - \beta^{n_c}}{1+\beta^{n_c}}\right]^2} - 1\right]  \right\}  %\numberthis \label{eq : g = 1 step cond}
%     \intertext{For \texttt{GTA-3}}    
%     &\alpha < \min \left\{\tfrac{1}{L}, \tfrac{1 - \beta^{n_c}}{L\beta^{n_c}} , \tfrac{(1 + \beta^{n_c})}{2\kappa(L + \mu)\beta^{n_c}} \left[\sqrt{1 + 4(\kappa + 1)\left[\tfrac{1 - \beta^{n_c}}{1 + \beta^{n_c}}\right]^2} - 1\right]\right\},   
% \end{align*}
% then there exists a sequence $\epsilon_k\geq 0$ such that,
% \begin{align*}
%     \|r_{k}\|_2 \leq (\rho(A_i({n_c})) + \epsilon_k)^k \|r_0\|_2 \quad \mbox{and} \quad \lim_{k \rightarrow \infty} \epsilon_k = 0
% \end{align*}
% for all $k\geq 0$ where $\rho(A_i({n_c})) < 1 \,\, \forall \,\, i = 1, 2, 3$.
then, for all $\epsilon > 0$ there exist constants $C_{i,\epsilon} > 0$ such that, for all $k\geq 0$,
\begin{align*}
    \|r_{k}\|_2 \leq C_{i,\epsilon}(\rho(A_i({n_c})) + \epsilon)^k \|r_0\|_2, \; \text{where } \; \rho(A_i({n_c})) < 1, \text{ for } \;  i=1, 2, 3.
\end{align*}
 %where $\rho(A_i({n_c})) < 1$ for all $i=1, 2, 3$.}
\ecorollary
\bproof
The conditions given in \cref{th. general g=1 step cond} are satisfied for all three methods. That is, the matrices are irreducible as $\Wmbf \neq \tfrac{1_n1_n^T}{n}$, i.e., $\beta > 0$ and $\beta_1, \beta_3 < 1$ in all the three methods as $\beta < 1$ because $\Wmbf$ is mixing matrix of a connected network. Thus, we can use \eqref{eq : g = 1 gen step cond} to derive the conditions on the step size for each of the methods. Substituting the values for $\beta_1,\beta_2,\beta_3$ and $\beta_4$ for each method yields the desired result. We should note that in \texttt{GTA-1} and  \texttt{GTA-2}, we  ignore the term $\tfrac{1}{L}$ since $\tfrac{1}{L} > \tfrac{1-\beta^{n_c}}{L}$. 
\eproof

\cref{col. g=1 step cond} shows how the communication strategy affects the step size when $n_g = 1$. Among the three methods, \texttt{GTA-3} allows for the largest step size, even having the possibility to use the step size $\tfrac{1}{L}$ if sufficiently large number of communications are performed (high $n_c$) and depending on $\beta$. Among \texttt{GTA-1} and \texttt{GTA-2}, \texttt{GTA-2} allows for a larger step size. While these share the same first term in the $\min$ bound, the presence of the $\beta^{n_c}$ factor in the denominator of the second term in \texttt{GTA-2} makes the bound larger than \texttt{GTA-1} and possibly allowing for a larger step size.

% While $\beta_1 = \beta_3 = \beta$, as $\beta_2$ is decreased from $1$ to $\beta$ from \texttt{GTA-1} to \texttt{GTA-2}, the step size condition improves. As $\beta_4$ is now changed from $1$ to $\beta$ from \texttt{GTA-2} to \texttt{GTA-3} , the condition improves even more, now allowing the possibility of $\tfrac{1}{L}$ gradient descent step size condition if $\beta$ is small enough or enough communications are performed ($n_c$ increased). Among


\cref{th. general g=1 step cond} states that there exists a step size such that \texttt{GTA} converges at a linear rate when $n_g = 1$. We now proceed to analyze the convergence rate \texttt{GTA} when $n_g = 1$. Before that, we provide a technical lemma that shows that the largest eigenvalue of the matrix $A(n_c)$ is a positive real number. 

\begin{lemma}\label{lem. g=1 spec norm}
Suppose \cref{asum.convex and smooth} holds, the number of gradient steps at each outer iteration of \cref{alg : Deterministic} is set to one (i.e., $n_g=1$) and $\alpha \leq \tfrac{1}{L}$. If the matrix $A({n_c})$ defined in  \eqref{eq : g = 1 general A} is irreducible, then, the spectral radius of $A({n_c})$ is the largest eigenvalue of $A({n_c})$ and is a positive real number. Consequently, if $\Wmbf \neq \tfrac{1_n1_n^T}{n}$, the spectral radius of matrices $A_1({n_c})$, $A_2({n_c})$, $A_3({n_c})$ defined in \eqref{eq : g = 1 algos A} are also positive real numbers and equal to their largest eigenvalues, respectively.
\end{lemma}

\begin{proof}
The statement about the matrix $A({n_c})$ follows from the Perron-Forbenius Theorem \cite[Theorem 8.4.4]{horn2012matrix}, and the fact that the matrix is %observing that the matrix $A({n_c})$ is 
non-negative and irreducible. Using similar arguments, the statement about the matrices $A_1({n_c})$, $A_2({n_c})$ and $A_3({n_c})$ follows as these matrices are irreducible when $\Wmbf \neq \tfrac{1_n1_n^T}{n}$, i.e., $\beta > 0$.
\end{proof}

% rate bounds g = 0
%%%%%%%%%%%%%%%%%%%%%%%%%%%%%%%%%%%%%%%%%%%%%
The next theorem provides an upper bound on the convergence rate of \texttt{GTA} for sufficiently small constant step sizes.%$ $\alpha$. %To make the bounds simpler, we will employ $\beta_1=\beta_3$.
\btheorem  \label{th. general g=1 rate bound} 
    Suppose \cref{asum.convex and smooth} holds and the number of gradient steps at each outer iteration of \cref{alg : Deterministic} is set to one (i.e., $n_g=1$). If the matrix $A({n_c})$ is irreducible and $\alpha \leq \tfrac{1}{L}$, then, %and $\beta_1=\beta_3 = \beta$,  then, 
    \begin{align*}
        \rho(A({n_c})) & \leq \,\lambda_u = \max\left\{1 - \tfrac{\alpha\mu}{2}, \hat{\lambda} + \sqrt{2\alpha L \kappa \beta_2^{n_c}\beta_4^{n_c}}\right\}, \numberthis \label{eq : general g = 1 rate}
    \end{align*}
    where $\hat{\lambda} = \tfrac{\beta_1^{n_c} + \beta_3^{n_c} + L\alpha\beta_4^{n_c}  + \sqrt{\left(\beta_1^{n_c} - \beta_3^{n_c} - L\alpha\beta_4^{n_c}\right)^2 + 4\beta_2^{n_c}\beta_4^{n_c} L^2\alpha^2 + 8L\alpha\beta_2^{n_c}\beta_4^{n_c}}}{2}$.
    % FINAL VERSION OF EQUATION
    % \begin{align*}
    % \hat{\lambda} = \tfrac{\beta_1^{n_c} + \beta_3^{n_c} + L\alpha\beta_4^{n_c}  + \sqrt{\left(\beta_1^{n_c} - \beta_3^{n_c} - L\alpha\beta_4^{n_c}\right)^2 + 4\beta_2^{n_c}\beta_4^{n_c} L^2\alpha^2 + 8L\alpha\beta_2^{n_c}\beta_4^{n_c}}}{2}.
    % \end{align*}
% 
    % \begin{align*}
    % \hat{\lambda} = \tfrac{\beta_1^{n_c} + \beta_3^{n_c} + L\alpha\beta_4^{n_c}  + \sqrt{\left(\beta_1^{n_c} - \beta_3^{n_c}\right)^2 + L^2\alpha^2(\beta_4^{n_c})^2 - 2L\alpha\beta_4^{n_c} (\beta_1^{n_c} - \beta_3^{n_c}) + 4\beta_2^{n_c}\beta_4^{n_c} L^2\alpha^2 + 8L\alpha\beta_2^{n_c}\beta_4^{n_c}}}{2}.
    % \end{align*}
    % \begin{align*}
    % \hat{\lambda} = \tfrac{\beta_1^{n_c} + \beta_3^{n_c} + L\alpha\beta_4^{n_c}  + \sqrt{\left(\beta_1^{n_c} - \beta_3^{n_c}\right)^2 + L^2\alpha^2\left( (\beta_4^{n_c})^2 + 4\beta_2^{n_c}\beta_4^{n_c}\right) + L\alpha \left( 8\beta_2^{n_c}\beta_4^{n_c}  - 2\beta_4^{n_c} (\beta_1^{n_c} - \beta_3^{n_c}) \right)  }}{2}.
    % \end{align*}
\etheorem

\bproof
%Since the matrix $A({n_c})$ is non-negative and irreducible, using \cref{lem. g=1 spec norm}, we know that the spectral norm is equal to the largest eigenvalue which is a positive real number. 
Using \cref{lem. g=1 spec norm}, we know that the spectral radius of $A(n_c)$ is equal to the largest eigenvalue which is a positive real number.
Following a similar approach to \cite{qu2017harnessing}, we prove $\lambda_u$ is an upper bound on the largest eigenvalue by showing the characteristic equation is non-negative at $\lambda_u$ and strictly increasing for all values greater than $\lambda_u$. Consider 
 \begin{align*}
 g(\lambda) 
= &\det(\lambda I_3 - A({n_c})) \\
= &(\lambda - 1 + \alpha\mu)\left((\lambda - \beta_1^{n_c})(\lambda - \beta_3^{n_c} - \alpha L\beta_4^{n_c}) - \alpha L(2 + \alpha L)\beta_2^{n_c}\beta_4^{n_c}\right) 
\\
& \quad - \alpha^3 L^3 \beta_2^{n_c}\beta_4^{n_c} \\
=& (\lambda - 1 + \alpha\mu) q(\lambda) - \alpha^3 L^3 \beta_2^{n_c}\beta_4^{n_c},
% =& (\lambda - 1 + \alpha\mu) \left(\lambda^2 -\lambda(\beta_1^{n_c} + \beta_3^{n_c} + L\alpha\beta_4^{n_c})  + \beta_1^{n_c}\beta_3^{n_c} + L\alpha\beta_4^{n_c}(\beta_1^{n_c} - 2\beta_2^{n_c} - L\alpha\beta_2^{n_c})\right) \\
% &\qquad - \alpha^3 L^3 \beta_2^{n_c}\beta_4^{n_c}
 \end{align*}
where $q(\lambda) = \lambda^2 -\lambda(\beta_1^{n_c} + \beta_3^{n_c} + L\alpha\beta_4^{n_c})  + \beta_1^{n_c}\beta_3^{n_c} + L\alpha\beta_4^{n_c}(\beta_1^{n_c} - 2\beta_2^{n_c} - L\alpha\beta_2^{n_c})$.
Let the roots of the quadratic function $q(\lambda)$ be denoted as $\lambda_1$ and $\lambda_2$. Then, we have,  %can be upper bounded as follows.
\begin{align*}
\max\{\lambda_1, \lambda_2\} 
=& \tfrac{\beta_1^{n_c} + \beta_3^{n_c} + L\alpha\beta_4^{n_c}  + \sqrt{\left(\beta_1^{n_c} + \beta_3^{n_c} + L\alpha\beta_4^{n_c}\right)^2 - 4\left(\beta_1^{n_c}\beta_3^{n_c} + L\alpha\beta_4^{n_c}(\beta_1^{n_c} - 2\beta_2^{n_c} - L\alpha\beta_2^{n_c})\right)}}{2}\\
% =& \frac{\beta_1^{n_c} + \beta_3^{n_c} + L\alpha\beta_4^{n_c}  + \sqrt{\left(\beta_1^{n_c} - \beta_3^{n_c}\right)^2 + L^2\alpha^2(\beta_4^{n_c})^2 - 2L\alpha\beta_4^{n_c} (\beta_1^{n_c} - \beta_3^{n_c}) + 4\beta_2^{n_c}\beta_4^{n_c} L^2\alpha^2 + 8L\alpha\beta_2^{n_c}\beta_4^{n_c}}}{2} \\
=& \tfrac{\beta_1^{n_c} + \beta_3^{n_c} + L\alpha\beta_4^{n_c}  + \sqrt{\left(\beta_1^{n_c} - \beta_3^{n_c} - L\alpha\beta_4^{n_c}\right)^2 + 4\beta_2^{n_c}\beta_4^{n_c} L^2\alpha^2 + 8L\alpha\beta_2^{n_c}\beta_4^{n_c}}}{2} 
\end{align*}
Thus, for any 
$\lambda \geq \max\left\{1 - \alpha \mu, \hat{\lambda}\right\}$, the function $g(\lambda)$ is increasing and is lower bounded by $(\lambda - 1 + \alpha\mu)(\lambda - \hat{\lambda})^2 - \alpha^3 L^3 \beta_2^{n_c}\beta_4^{n_c}$. 
% Now, let
% \begin{align*}
% \lambda_u =  \max\left\{1 - \frac{\alpha\mu}{2}, \hat{\lambda} + \sqrt{\frac{2\alpha L^2 \beta_2^{n_c}\beta_4^{n_c}}{\mu}}\right\}.
% \end{align*}
By $\lambda_u \geq \max\left\{1 - \alpha \mu, \hat{\lambda}\right\}$, 
\begin{align*}
g(\lambda_u) &\geq (\lambda - 1 + \alpha\mu)(\lambda - \hat{\lambda})^2 - \alpha^3 L^3 \beta_2^{n_c}\beta_4^{n_c} \\
&\geq \left(1 - \tfrac{\alpha\mu}{2} -1 + \alpha\mu\right)(\lambda - \hat{\lambda})^2 - \alpha^3 L^3 \beta_2^{n_c}\beta_4^{n_c} \\
&\geq \tfrac{\alpha \mu}{2}\left(\tfrac{2\alpha L^2\beta_2^{n_c}\beta_4^{n_c}}{\mu}\right)  - \alpha^3 L^3 \beta_2^{n_c}\beta_4^{n_c} \\
&=\alpha^2L^2\beta_2^{n_c}\beta_4^{n_c}(1 - \alpha L) \geq 0,
\end{align*}
% where the second and third inequalities are due to definition of $\lambda_u$ and the last inequality is due to $\alpha L \leq 1$. 
where the second and third inequalities are due to the definition of $\lambda_u$ and the final quantity is non-negative since $\alpha \leq \tfrac{1}{L}$. Therefore, by the above arguments, we conclude that $\rho(A({n_c})) \leq \lambda_u$ which completes the proof. 
\eproof

\cref{th. general g=1 rate bound} is derived independent of the conditions in \cref{th. general g=1 step cond}. 
%It places the same conditions on the system as \cref{th. general g=1 step cond}. 
When $\rho(A(n_c)) < 1$ is imposed using \cref{th. general g=1 rate bound}, $\beta_1,\beta_3 < 1$ is %recognized as 
a necessary condition for convergence. We show this by constructing a lower bound on $\lambda_u$, $\lambda_u \geq \hat{\lambda} \geq \tfrac{\beta_1^{n_c} + \beta_3^{n_c}+ \left|\beta_1^{n_c} - \beta_3^{n_c}\right|}{2}$.
% \begin{align*}
%     \lambda_u & \geq \hat{\lambda} 
%     % \\
%     % & \geq \tfrac{\beta_1^{n_c} + \beta_3^{n_c} + L\alpha\beta_4^{n_c}  + \sqrt{\left(\beta_1^{n_c} - \beta_3^{n_c} - L\alpha\beta_4^{n_c}\right)^2 + 4\beta_2^{n_c}\beta_4^{n_c} L^2\alpha^2 + 8L\alpha\beta_2^{n_c}\beta_4^{n_c}}}{2} \\
%     % & \geq \tfrac{\beta_1^{n_c} + \beta_3^{n_c} + L\alpha\beta_4^{n_c}  + \sqrt{\left(\beta_1^{n_c} - \beta_3^{n_c} - L\alpha\beta_4^{n_c}\right)^2}}{2} \\ 
%     % & \geq \tfrac{\beta_1^{n_c} + \beta_3^{n_c} + L\alpha\beta_4^{n_c}  + \left|\beta_1^{n_c} - \beta_3^{n_c} - L\alpha\beta_4^{n_c}\right|}{2} \\
%     % & 
%     \geq \tfrac{\beta_1^{n_c} + \beta_3^{n_c}+ \left|\beta_1^{n_c} - \beta_3^{n_c}\right|}{2}.
% \end{align*}
% where the second inequality is obtained by removing the last two positive terms in the sqaure root and the last inequality is due to traingle inequality.
For convergence we require $\lambda_u < 1$, i.e., $\tfrac{\beta_1 + \beta_3 + |\beta_1 - \beta_3|}{2} < 1$, which implies $\beta_1,\beta_3 < 1$ as $\beta_1, \beta_3 \in [0,1]$. Thus, again we require $\Wmbf_1$ and $\Wmbf_3$ to represent a connected network. The step size condition in \cref{th. general g=1 step cond} is $\mathcal{O}(L^{-1}\kappa^{-0.5})$ while \cref{th. general g=1 rate bound} requires $\mathcal{O}(L^{-1}\kappa^{-1})$, which is more pessimistic. That  said,  the precise and interpretable characterization of the convergence rate in \cref{th. general g=1 rate bound}  allows us to better differentiate amongst the communication strategies and the effect of $n_c$. 
% But \cref{th. general g=1 rate bound} gives more interpretable results for the convergence rate and allows us to differentiate better among communication strategies and effect of $n_c$.

\bcorollary  \label{col. g=1 rate bound}
Suppose \cref{asum.convex and smooth} holds, $\Wmbf \neq \tfrac{1_n1_n^T}{n}$, and the number of gradient steps at each outer iteration of \cref{alg : Deterministic} is set to one (i.e., $n_g=1$). If $\alpha \leq \tfrac{1}{L}$, then, the spectral radii for the methods described in \cref{tab: Algorithm Def} satisfy
\begin{align*}
    \mbox{\texttt{GTA-1}:} & \quad \rho(A_1({n_c}))   \leq \max\left\{1 - \tfrac{\alpha\mu}{2}, \beta^{n_c} + \sqrt{\alpha L} \left(2.5 + \sqrt{\kappa}\right)\right\},\\
    \mbox{\texttt{GTA-2}:} & \quad \rho(A_2({n_c}))  \leq \max\left\{1 - \tfrac{\alpha\mu}{2}, \beta^{n_c} + \sqrt{\alpha L} \left(2.5 + \sqrt{\kappa \beta^{n_c}}\right)\right\}, \\%\numberthis \label{eq : g = 1 rate} \\
    \mbox{\texttt{GTA-3}:} & \quad \rho(A_3({n_c}))  \leq \max\left\{1 - \tfrac{\alpha\mu}{2}, \beta^{n_c}\left(1 + \sqrt{\alpha L} \left(2.5 + \sqrt{\kappa}\right)\right)\right\}.
\end{align*}
    % \begin{align*}
    %     \intertext{For \texttt{GTA-1}}       
    %         \rho(A_1({n_c}))  & < \max\left\{1 - \frac{\alpha\mu}{2}, \beta^{n_c} + \sqrt{\alpha L} \left(2.5 + \sqrt{\kappa}\right)\right\}
    %     \intertext{For \texttt{GTA-2}}   
    %         \rho(A_2({n_c})) & < \max\left\{1 - \frac{\alpha\mu}{2}, \beta^{n_c} + \sqrt{\alpha L} \left(2.5 + \sqrt{\kappa \beta^{n_c}}\right)\right\} \numberthis \label{eq : g = 1 rate}
    %     \intertext{For \texttt{GTA-3}}    
    %         \rho(A_3({n_c})) & < \max\left\{1 - \frac{\alpha\mu}{2}, \beta^{n_c}\left(1 + \sqrt{\alpha L} \left(2.5 + \sqrt{\kappa}\right)\right)\right\} 
    % \end{align*}
\ecorollary
\bproof
The conditions in \cref{th. general g=1 rate bound} are satisfied due to \cref{lem. g=1 spec norm}. Thus, we can plug in the values for $\beta_i$ ($i=1,2,3,4$) for each method to get an upper bound on the spectral radii. The upper bound $\lambda_u$ for \texttt{GTA-1} can be simplified as
\begin{align*}
\hat{\lambda} + \sqrt{\tfrac{2\alpha L^2 \beta_2^{n_c}\beta_4^{n_c}}{\mu}} &= \tfrac{2\beta^{n_c} + L\alpha  + \sqrt{5L^2\alpha^2 + 8L\alpha}}{2} + \sqrt{\tfrac{2\alpha L^2}{\mu}} \\
&= \beta^{n_c} + \tfrac{\sqrt{\alpha L}}{2}\left(\sqrt{\alpha L} + 2\sqrt{\kappa} + \sqrt{8 + 5L\alpha} \right) \\
&\leq \beta^{n_c} + \sqrt{\alpha L} \left(2.5 + \sqrt{\kappa}\right)
\end{align*}
where the last inequality is due to $\alpha \leq \tfrac{1}{L}$. Following the same approach, $\lambda_u$ for \texttt{GTA-2} can be simplified as
\begin{align*}
\hat{\lambda} + \sqrt{\tfrac{2\alpha L^2 \beta_2^{n_c}\beta_4^{n_c}}{\mu}} &= \tfrac{2\beta^{n_c} + L\alpha  + \sqrt{L^2\alpha^2 + 4L^2\alpha^2\beta^{n_c} + 8L\alpha\beta^{n_c}}}{2} + \sqrt{\tfrac{2\alpha L^2\beta^{n_c}}{\mu}} \\
&= \beta^{n_c} + \tfrac{\sqrt{\alpha L}}{2}\left(\sqrt{\alpha L} + 2\sqrt{\kappa \beta^{n_c}} + \sqrt{8\beta^{n_c} + 4L\alpha\beta^{n_c} + L\alpha} \right) \\
&\leq \beta^{n_c} + \sqrt{\alpha L} \left(2.5 + \sqrt{\kappa \beta^{n_c}}\right)
\end{align*}
where the last inequality uses $\alpha \leq \tfrac{1}{L}$ and $\beta < 1$. Finally, the upper bound $\lambda_u$ for \texttt{GTA-3} is
\begin{align*}
\hat{\lambda} + \sqrt{\tfrac{2\alpha L^2 \beta_2^{n_c}\beta_4^{n_c}}{\mu}} &= \tfrac{2\beta^{n_c} + L\alpha\beta^{n_c}  + \sqrt{ 5L^2\alpha^2(\beta^{n_c})^2 + 8L\alpha(\beta^{n_c})^2}}{2} + \sqrt{\tfrac{2\alpha L^2(\beta^{n_c})^2}{\mu}} \\
&= \beta^{n_c}\left(1  + \tfrac{\sqrt{\alpha L}}{2}\left(\sqrt{\alpha L} + 2\sqrt{\kappa} + \sqrt{8 + 5L\alpha} \right)\right) \\
&\leq \beta^{n_c}\left(1 + \sqrt{\alpha L} \left(2.5 + \sqrt{\kappa}\right)\right)
\end{align*}
where the last inequality is due to $\alpha \leq \tfrac{1}{L}$ and $\beta < 1$. 

\eproof

%Using \cref{col. g=1 rate bound}, we can conclude that in \texttt{GTA-1}, \texttt{GTA-2} and \texttt{GTA-3}, the convergence rate improves with increased communication ($n_c$) when $n_g = 1$. The effect is more profound from \texttt{GTA-1} to \texttt{GTA-2}. In \texttt{GTA-1}, increasing $n_c$ only effects the constant term while $\sqrt{L\kappa}$ still dominates. In \texttt{GTA-2}, the dominating term is also impacted. From \texttt{GTA-2} to \texttt{GTA-3}, a similar difference is observed. In \texttt{GTA-3}, increasing $n_c$ impacts the entire convergence rate expression. Thus, a sufficient increase in $n_c$ can achieve gradient descent performance in \texttt{GTA-3} when $n_g = 1$. Based on the properties of the objective, an even higher increase in $n_c$ when $n_g = 1$ can achieve gradient descent performance in \texttt{GTA-2} as well while it does not seem possible in \texttt{GTA-1}.

\cref{col. g=1 rate bound} characterizes the effect of  multiple communication steps (when $n_g = 1$) on the convergence rates of \texttt{GTA-1}, \texttt{GTA-2} and \texttt{GTA-3}.
%shows how performing multiple communications (when $n_g = 1$) impacts the convergence rate differently across the communication strategies in \texttt{GTA-1}, \texttt{GTA-2} and \texttt{GTA-3}. 
First, the convergence rate improves with increased communications (increase in $n_c$) when $n_g = 1$ for all methods. The improvement is strongest in \texttt{GTA-3} as increasing $n_c$ drives the second term in the $\max$ bound to zero. Thus, if a sufficient number of communication steps are performed in \texttt{GTA-3}, the method can achieve convergence rates similar to those of gradient descent, i.e., $(1 - \tfrac{\alpha\mu}{2})$. %achieves gradient descent performance with convergence rate $(1 - \tfrac{\alpha\mu}{2})$. 
The improvement is less apparent in \texttt{GTA-2} and the weakest in \texttt{GTA-1}. With an increase in $n_c$, the dominating term in the max bound, i.e., $\sqrt{\alpha L\kappa}$, remains unchanged in \texttt{GTA-1} and changes to $\sqrt{\alpha L\kappa \beta^{n_c}}$ in \texttt{GTA-2} which is affected by the number of communication steps $n_c$. %That dominating term changes to $\sqrt{\alpha L\kappa \beta^{n_c}}$ in \texttt{GTA-2}, now reducing with increase in $n_c$.}

%%%%%%%%%%%%%%%%%%%%%%%%%%%%%%%%%%%%%%%%%%%%%
% Deterministic Analysis for g > 1
%%%%%%%%%%%%%%%%%%%%%%%%%%%%%%%%%%%%%%%%%%%%%

\subsection{\texttt{GTA} with multiple communication and computation \texorpdfstring{($n_c \geq 1, n_g\geq 1$)}{Lg}} \label{sec.mult grads}
In this section, we analyze \texttt{GTA} when multiple computation and communication steps are performed every iteration. We extend the analysis from \cref{sec.mult comms}; the case $n_g = 1$ is a special case of the analysis in this section. The subscript for the inner iteration counter is re-introduced in this section as we consider cases with $n_g > 1$ and the inner loop (Lines 4--6 in \cref{alg : Deterministic}) is executed. We first provide a technical lemma that bounds the errors 
due to the execution of the inner loop. We use this result to extend \cref{lem:lyapunov g = 1} and establish the progression of the error vector $r_k$ when multiple communication and computation steps are performed. Finally, we provide the conditions for linear convergence of \cref{alg : Deterministic} with any composition of communication and computation steps.

\begin{lemma} \label{lem : inner loop deviations}
Suppose \cref{asum.convex and smooth} holds and $\alpha \leq \frac{1}{n_gL}$ in \cref{alg : Deterministic}. Then, for all $k\geq0$ and $1 \leq j \leq n_g$
\begin{align}%   \label{eq : iterate deviation bound}
    \|\Bar{\ymbf}_{k,1}\|_2 &\leq \|\ymbf_{k, 1} -  \bar{\ymbf}_{k, 1}\|_2 + L\left\|\xmbf_{k,1} - \bar{\xmbf}_{k,1}\right\|_2 + L\sqrt{n} \| \Bar{x}_{k,1} - x^*\|_2\label{eq:y_bound} \\
    \|\xmbf_{k, j} - \xmbf_{k, 1}\|_2 &\leq 2\alpha (j - 1 ) \|\ymbf_{k, 1}\|_2, \label{eq:iterate_deviation}\\
    \|\xmbf_{k, j} - \xbb_{k, j}\|_2 &\leq 2 \alpha (j - 1)\|\ymbf_{k, 1}\|_2 + \|\xmbf_{k, 1} - \xbb_{k, 1}\|_2, \label{eq:consensus_error_deviation}
\end{align}
% \begin{align*}%   \label{eq : iterate deviation bound}
%     \|\xmbf_{k, j} - \xmbf_{k, 1}\|_2 &\leq 2\alpha (j - 1 )\left[ \|\ymbf_{k, 1} - \bar{\ymbf}_{k, 1}\|_2 + L\|\xmbf_{k, 1} - \bar{\xmbf}_{k, 1}\|_2 + \sqrt{n}L\|\Bar{x}_{k, 1} - x^*\|_2 \right]\\
%     \|\xmbf_{k, j} - \xbb_{k, j}\|_2 &\leq 2 \alpha (j - 1)\left[ \|\ymbf_{k, 1} - \ybb_{k, 1}\|_2 + \sqrt{n}L \|\bar{x}_{k, 1} - x^*\|_2\right] \\
%     & \quad + (2\alpha (j - 1)L + 1)\|\xmbf_{k, 1} - \xbb_{k, 1}\|_2. 
% \end{align*}
\end{lemma}
\bproof
% \asb{Summing $\ymbf_{k, i}$~\eqref{} from $i=2$ to $j$, 
Taking a telescopic sum of $\ymbf_{k, i+1} = \ymbf_{k, i} + \nabla \fmbf(\xmbf_{k, i+1}) - \nabla \fmbf(\xmbf_{k, i})$, the inner loop update, from $i=1$ to $j-1$ we get
\begin{align}
    \ymbf_{k, j} & = \ymbf_{k, 1} + \nabla \fmbf(\xmbf_{k, j}) - \nabla \fmbf(\xmbf_{k, 1}). \label{eq: v telescope}
\end{align}
Using~\eqref{eq: v telescope},  $\ymbf_{k+1, 1}$ can be expressed as
\begin{equation}\label{eq : y general telescope sum}
\begin{aligned}
    \ymbf_{k+1, 1} &= \Zmbf_3^{n_c}\left(\ymbf_{k, 1} + \nabla \fmbf(\xmbf_{k, n_g}) - \nabla \fmbf(\xmbf_{k, 1})\right) + \Zmbf_4^{n_c}\left(\nabla \fmbf(\xmbf_{k+1, 1}) - \nabla \fmbf(\xmbf_{k, n_g})\right)   \\  
    &= \Zmbf_3^{n_c}\ymbf_{k, 1} + \Zmbf_4^{n_c}\nabla \fmbf(\xmbf_{k+1, 1}) - \Zmbf_3^{n_c}\nabla \fmbf(\xmbf_{k, 1}) + \Zmbf_3^{n_c}\nabla \fmbf(\xmbf_{k, n_g}) - \Zmbf_4^{n_c} \nabla \fmbf(\xmbf_{k, n_g})
\end{aligned}
\end{equation}
Taking the component-wise average across all nodes in \eqref{eq: v telescope} and \eqref{eq : y general telescope sum} and using~\eqref{eq : derivative terms define}, it follows that
\begin{align}
    \Bar{y}_{k, j} & = \Bar{y}_{k, 1} + h_{k, j} - h_{k, 1}, \label{eq: g > 1 y_j_telescope}\\
    \Bar{y}_{k+1, 1} & = \Bar{y}_{k, 1} + h_{k+1, 1} - h_{k, 1}.  \label{eq : g > 1 y bar telescope}
\end{align}
% Following similar logic as that used to derive~\eqref{eq:y_bar_telescope}, $\Bar{y}_{k, 1} = h_{k, 1}$, from which it follows by~\eqref{eq: g > 1 y_j_telescope} that 
% \begin{align}   \label{eq: v_bar_telescope}
%     \bar{y}_{k, j} &= \bar{y}_{k, 1} + h_{k, j} - h_{k, 1} = h_{k, j}.
% \end{align}
Performing a similar telescopic sum as \eqref{eq:y_bar_telescope} with \eqref{eq : g > 1 y bar telescope}, we obtain $\Bar{y}_{k, 1} = h_{k, 1}$. Thus, substituting $\Bar{y}_{k, 1} = h_{k, 1}$ in \eqref{eq: g > 1 y_j_telescope} yields 
%a generalization of \eqref{eq:y_bar_telescope}, i.e., 
\begin{align}   \label{eq: v_bar_telescope}
    \bar{y}_{k, j} &= \bar{y}_{k, 1} + h_{k, j} - h_{k, 1} = h_{k, j}.
\end{align}
By the triangle inequality, $\|\ymbf_{k, 1}\|_2 \leq \|\ymbf_{k, 1} -  \bar{\ymbf}_{k, 1}\|_2   + \| \bar{\ymbf}_{k, 1} \|_2$, where $\|\bar{\ymbf}_{k, 1}\|_2$ can be bounded by a similar procedure to \eqref{eq : y_bar_bound} due to $\Bar{y}_{k, 1} = h_{k, 1}$ to yield \eqref{eq:y_bound}.
% By $\Bar{y}_{k, 1} = h_{k, 1}$, we can also perform a procedure similar to \eqref{eq : y_bar_bound} to obtain a bound on $\|\ybb_{k,1}\|_2$ as
% \begin{align}   \label{eq : y_bar_bnd g > 1}
% \left\|\Bar{\ymbf}_{k,1}\right\|_2 &\leq L\left\|\xmbf_{k,1} - \bar{\xmbf}_{k,1}\right\|_2 + L\sqrt{n} \| \Bar{x}_{k,1} - x^*\|_2.
% \end{align}
% Now, by triangle inequality and \eqref{eq : y_bar_bnd g > 1},
% \begin{equation} \label{eq: g>1_y_bound}
%     \begin{aligned}
%         \|\ymbf_{k, 1}\|_2 &\leq \|\ymbf_{k, 1} -  \bar{\ymbf}_{k, 1}\|_2   + \| \bar{\ymbf}_{k, 1} \|_2 \\
%         &\leq \|\ymbf_{k, 1} -  \bar{\ymbf}_{k, 1}\|_2 + L\left\|\xmbf_{k,1} - \bar{\xmbf}_{k,1}\right\|_2 + L\sqrt{n} \| \Bar{x}_{k,1} - x^*\|_2.
%     \end{aligned}
% \end{equation}
% Taking the telescopic sum of $\ymbf_{k, i}$ from $i=2$ to $j$, it follows that
% \begin{align}
%     \ymbf_{k, j} & = \ymbf_{k, 1} + \nabla \fmbf(\xmbf_{k, j}) - \nabla \fmbf(\xmbf_{j, 1}). \label{eq: v telescope}
% \end{align}
% Taking the component wise average across all nodes in \eqref{eq: v telescope} and using \eqref{eq : derivative terms define}, we get 
% \begin{align}
%     \Bar{y}_{k, j} & = \Bar{y}_{k, 1} + h_{k, j} - h_{k, 1}.  \label{eq: g > 1 y_j_telescope}
% \end{align}
% Using \eqref{eq: v telescope}, $\ymbf_{k+1, 1}$ can be expanded as
% \begin{align*}
%     \ymbf_{k+1, 1} &= \Zmbf_3^{n_c}\left(\ymbf_{k, 1} + \nabla \fmbf(\xmbf_{k, n_g}) - \nabla \fmbf(\xmbf_{k, 1})\right) + \Zmbf_4^{n_c}\left(\nabla \fmbf(\xmbf_{k+1, 1}) - \nabla \fmbf(\xmbf_{k, n_g})\right)   \numberthis \label{eq : y general telescope sum}\\  
%     &= \Zmbf_3^{n_c}\ymbf_{k, 1} + \Zmbf_4^{n_c}\nabla \fmbf(\xmbf_{k+1, 1}) - \Zmbf_3^{n_c}\nabla \fmbf(\xmbf_{k, 1}) + \Zmbf_3^{n_c}\nabla \fmbf(\xmbf_{k, n_g}) - \Zmbf_4^{n_c} \nabla \fmbf(\xmbf_{k, n_g})
% \end{align*}
% Taking the component wise average across all nodes in \eqref{eq : y general telescope sum} and using \eqref{eq : derivative terms define} yields
% \begin{align}
%     \Bar{y}_{k+1, 1} & = \Bar{y}_{k, 1} + h_{k+1, 1} - h_{k, 1}  \label{eq : g > 1 y bar telescope}.
% \end{align}
% Thus, performing a similar telescopic sum as
% \eqref{eq:y_bar_telescope} using \eqref{eq : g > 1 y bar telescope}, we obtain $\Bar{y}_{k, 1} = h_{k, 1}$. Using this in \eqref{eq: g > 1 y_j_telescope} yields a generalization of \eqref{eq:y_bar_telescope}, i.e.
% % Thus, the telescopic sum of $\Bar{y}_{k, i}$ from $i = 2$ to $j$ yields a generalization of \eqref{eq:y_bar_telescope}, i.e.
% \begin{align}   \label{eq: v_bar_telescope}
%     \bar{y}_{k, j} &= \bar{y}_{k, 1} + h_{k, j} - h_{k, 1} = h_{k, j}.
% \end{align}
% So following a procedure similar to \eqref{eq : y_bar_bound}, we can produce a bound for $\|\ybb_{k,1}\|_2$ as 
% \begin{align}   \label{eq : y_bar_bnd g > 1}
% \left\|\Bar{\ymbf}_{k,1}\right\|_2 &\leq L\left\|\xmbf_{k,1} - \bar{\xmbf}_{k,1}\right\|_2 + L\sqrt{n} \| \Bar{x}_{k,1} - x^*\|_2
% \end{align}

Now, taking the telescopic sum of the inner loop update $\xmbf_{k, i} = \xmbf_{k, i-1} - \alpha \ymbf_{k, i-1}$ from $i=2$ to $j$ yields $\xmbf_{k, j} = \xmbf_{k, 1} - \alpha \sum_{i=1}^{j-1}\ymbf_{k, i}$. The sum $\sum_{i = 1}^{j-1} \ymbf_{k, i}$ is evaluated using \eqref{eq: v telescope} as
\begin{equation}\label{eq: sum_y_k_j}
\begin{aligned}   
    \sum_{i = 1}^{j -1} \ymbf_{k, j} & = \ymbf_{k, 1} + \sum_{i = 2}^{j-1} \ymbf_{k, i} + \nabla \fmbf(\xmbf_{k, i}) - \nabla \fmbf(\xmbf_{k, 1}) \\
    &= (j - 1) \ymbf_{k, 1} + \sum_{i = 2}^{j - 1} \nabla \fmbf(\xmbf_{k, i}) - \nabla \fmbf(\xmbf_{k, 1}).
\end{aligned}
\end{equation}
By the triangle inequality and \cref{asum.convex and smooth}, it follows that
% \begin{align*}
% \xmbf_{k, j} &= \xmbf_{k, 1} - \alpha \sum_{i=1}^{j-1}\ymbf_{k, i} = \xmbf_{k, 1} - \alpha \sum_{i=1}^{j-1}\left[ \ymbf_{k, 1} + \nabla \fmbf(\xmbf_{k, i}) - \nabla \fmbf(\xmbf_{k, 1}) \right]\\
% &= \xmbf_{k, 1} - \alpha(j-1)\ymbf_{k, 1}  - \alpha  \sum_{i=1}^{j-1} \left[ \nabla \fmbf(\xmbf_{k, i}) - \nabla \fmbf(\xmbf_{k, 1}) \right].
% \end{align*}
% Therefore, for any $j \geq 2$, we have, 
\begin{align*}
\|\xmbf_{k, j} - \xmbf_{k, 1}  \|_2 &\leq \alpha(j-1)\|\ymbf_{k, 1}\|_2 + \alpha  \sum_{i=1}^{j-1}\| \nabla \fmbf(\xmbf_{k, i}) - \nabla \fmbf(\xmbf_{k, 1})\|_2 \\
&\leq  \alpha(j-1)\|\ymbf_{k, 1}\|_2 + \alpha L \sum_{i=1}^{j-1}\| \xmbf_{k, i} - \xmbf_{k, 1}\|_2.
\end{align*}
Now we apply induction to show \eqref{eq:iterate_deviation} using the above inequality.
% Now we apply induction to show $\|\xmbf_{k, j} - \xmbf_{k, 1}  \|_2 \leq 2 \alpha (j-1) \|\ymbf_{k, 1}\|_2$ for all $j=1,\dots,n_g$ using the above inequality.
\begin{align*}
    \mbox{For $j = 1$,} \qquad \|\xmbf_{k, 1} - \xmbf_{k, 1}\|_2 &= 0 =  2 \alpha (1 - 1) \|\ymbf_{k, 1}\|_2. \\
    \mbox{For $j \geq 2$,} \qquad \|\xmbf_{k, j} - \xmbf_{k, 1}\|_2 &\leq \alpha(j-1)\|\ymbf_{k, 1}\|_2 + \alpha L \sum_{i=1}^{j-1}\| \xmbf_{k, i} - \xmbf_{k, 1}\|_2 \\
    &\leq \alpha(j-1)\|\ymbf_{k, 1}\|_2 + 2 \alpha^2 L  \sum_{i=1}^{j-1} (i-1) \|\ymbf_{k, 1}\|_2 \\
    &=  \alpha(j-1)\|\ymbf_{k, 1}\|_2 + 2 \alpha^2 L  \|\ymbf_{k, 1}\|_2 \tfrac{(j-2)(j-1)}{2} \\
    &= \alpha(j-1) \left(1 + \alpha L (j-2) \right)\|\ymbf_{k, 1}\|_2  \\
    %&\leq \alpha(j-1) \left(1 + \tfrac{(n_g-2)}{n_g} \right)\|\ymbf_{k, 1}\|_2  \\
    %& = \alpha(j-1)\|\ymbf_{k, 1}\|_2 \left(1 + \frac{n_g-2}{n_g + 2} \right) \\
    &\leq 2 \alpha(j-1)\|\ymbf_{k, 1}\|_2,
\end{align*}
where the first equality uses the sum of $j-1$ natural numbers and the second to last inequality is due to $\alpha L \leq \frac{1}{n_g}$ and $j \leq n_g$. 
% \begin{align*}
% \|\xmbf_{k, j} - \xmbf_{k, 1}\|_2 &\leq 2 \alpha (j-1) \|\ymbf_{k, 1}\|_2 \\
% \|\xmbf_{k, j} - \xmbf_{k, 1}\|_2 &\leq 2 \alpha (j-1) \left(\|\ymbf_{k, 1} -  \bar{\ymbf}_{k, 1}\|_2   + \| \bar{\ymbf}_{k, 1} \|_2\right).
% \end{align*}
% The first result follows by substituting \eqref{eq : y_bar_bnd g > 1} in the above.
% The first result follows by substituting \eqref{eq: g>1_y_bound} in the induction result.

By \eqref{eq:iterate_deviation}, the triangle inequality and $\|I_{nd} - \Imbf\|_2 = 1$, it follows that
\begin{align*}
    \|\xmbf_{k, j} - \xbb_{k, j}\|_2 &\leq \|\xmbf_{k, j} - \xmbf_{k, 1} + \xbb_{k,1} -  \xbb_{k, j}\|_2 + \|\xmbf_{k, 1} - \xbb_{k, 1}\|_2    \\
    &\leq \|(I_{nd} - \Imbf)(\xmbf_{k, j} - \xmbf_{k, 1})\|_2 + \|\xmbf_{k, 1} - \xbb_{k, 1}\|_2    \\
    &\leq \|\xmbf_{k, j} - \xmbf_{k, 1}\|_2 + \|\xmbf_{k, 1} - \xbb_{k, 1}\|_2.    
\end{align*}
% where the last inequality is due to $\|I_{nd} - \Imbf\|_2 = 1$.
\eproof% and use \eqref{eq : iterate deviation bound} in the end.
% \eproof
% \asb{The first result XXX}

The two bounds, \eqref{eq:iterate_deviation} and \eqref{eq:consensus_error_deviation}, in \cref{lem : inner loop deviations} bound the deviation of the local decision variables from the start of the outer iteration, $\|\xmbf_{k, j} - \xmbf_{k, 1}\|_2$, and the consensus error, $\|\xmbf_{k, j} - \xbb_{k, j}\|_2$,  in inner iteration $j$, respectively. Combined with \eqref{eq:y_bound}, these quantities are bounded as an $\mathcal{O}(\alpha j)$ multiple of the components of the error vector $r_k$. This property has two implications; $(1)$ if one performs more inner iterations, i.e., increases $n_g$, the constant step size $\alpha$ needs to be reduced to reduce these quantities, $(2)$ if an outer iteration is the optimal solution, the inner loop does not introduce any deviations in the iterates and maintains optimality.

We now establish the progression of error vector $r_k$ under multiple communication and computation steps being performed every iteration in \cref{alg : Deterministic}.

\begin{lemma}\label{lem:lyapunov g > 1}
Suppose \cref{asum.convex and smooth} holds and $\alpha \leq \frac{1}{n_g L}$ in \cref{alg : Deterministic}. Then, for all $k\geq 0$,
\begin{align*}
  r_{k+1} \leq B(n_c, n_g) r_k, \quad \text{where $\; B(n_c, n_g) = A(n_c, n_g) + \alpha L (n_g - 1) E(n_c, n_g)$}, \quad     
\end{align*}
% where $B(n_c, n_g) = A(n_c, n_g) + \alpha L (n_g - 1) E(n_c, n_g)$,
\begin{equation}\label{eq : g > 1 general A}
\begin{aligned}
A(n_c, n_g) &= \begin{bmatrix}
    (1 - \alpha \mu)^{n_g}& \frac{\kappa}{\sqrt{n}}(1 - (1 - \alpha\mu)^{n_g}) & 0\\
    0 & \beta_1^{n_c} & \alpha\left((n_g-1)\beta_1^{n_c} + \beta_2^{n_c}\right)\\
    \sqrt{n}\alpha \beta_4^{n_c} L^2 & \beta_4^{n_c}L(\|\Zmbf_1^{n_c}-I_{nd}\|_2 + \alpha L) & \beta_3^{n_c} + \alpha \beta_4^{n_c} L
    \end{bmatrix} ,\\
E(n_c, n_g) &= \begin{bmatrix}
    \alpha L n_g & \frac{\alpha L n_g}{\sqrt{n}} & \frac{\alpha n_g}{\sqrt{n}}\\
    \sqrt{n}\alpha L \delta_1(n_c, n_g) & \alpha L \delta_1(n_c, n_g) & \alpha \delta_1(n_c, n_g)\\
    \sqrt{n} L\delta_2(n_c, n_g) & L \delta_2(n_c, n_g)& \delta_2(n_c, n_g)
    \end{bmatrix},
\end{aligned}
\end{equation}
and
\begin{equation}\label{eq:delta_errors}
\begin{aligned}
    \delta_1(n_c, n_g) &= 2\beta_2^{n_c} + \beta_1^{n_c}(n_g - 2) ,\\
    \delta_2(n_c, n_g) &= 2 \left( \beta_4^{n_c} \|\Zmbf_1^{n_c} - I_{nd}\|_2 + \tfrac{\beta_4^{n_c}}{n_g} + \beta_3^{n_c} \right).
\end{aligned}
\end{equation}
\end{lemma}
\bproof We first consider the optimization error of the average iterates $\xbar_{k, 1}$. Similar to \eqref{eq : g = 1 opt bound}, we  bound the  optimization error as %for $\xbar_{k, j}$, the average iterate in the inner iterations as
\begin{align*}
    % \|\Bar{x}_{k, j+1} - x^*\|_2 & = \left\|\Bar{x}_{k, j} - \alpha \Bar{y}_{k, j} + \alpha \hbar_{k, j} - \alpha \hbar_{k, j} - x^*\right\|_2  \\
    % & \leq (1-\alpha \mu) \|\Bar{x}_{k, j} - x^*\|_2 + \alpha \left\| \Bar{y}_{k, j} - \hbar_{k, j}\right\|_2   \\
    % &=  (1-\alpha \mu) \|\Bar{x}_{k, j} - x^*\|_2 + \alpha \left\|h_{k, j} - \hbar_{k, j}\right\|_2 \\
    % &\leq (1-\alpha \mu) \|\Bar{u}_j - x^*\|_2 + \alpha \frac{\|1\|}{n}\left\|\nabla f(u_{j}) - \nabla f(1\Bar{u}_{j}^T)\right\|_2 \\
    \|\Bar{x}_{k, j+1} - x^*\|_2 & \leq (1-\alpha \mu) \|\Bar{x}_{k, j} - x^*\|_2 + \tfrac{\alpha L}{\sqrt{n}} \| \xmbf_{k, j} - \xbb_{k, j}\|_2 \quad  \forall \,\, 1 \leq j \leq n_g - 1,
\end{align*}
where the above holds by using \eqref{eq: v_bar_telescope} (the generalization of \eqref{eq:y_bar_telescope}) and the error bound of gradient descent from \cite[Theorem 2.1.14]{nesterov1998introductory}.  %due to \eqref{eq: v_bar_telescope} as a generalization of \eqref{eq:y_bar_telescope} and the error bound of gradient descent from \cite[Theorem 2.1.14]{nesterov1998introductory}. 
Next, we bound the optimization error in $\xbar_{k+1, 1}$ with respect to $\xmbf_{k, n_g}$ in a similar manner as, %The optimization error for $\xbar_{k+1, 1}$ can be bound  w.r.t. $\xmbf_{k, n_g}$, the last inner iterate, in a similar way as
\begin{align*}
    \|\Bar{x}_{k+1, 1} - x^*\|_2 & \leq (1-\alpha \mu) \|\Bar{x}_{k, n_g} - x^*\|_2 + \tfrac{\alpha L}{\sqrt{n}} \| \xmbf_{k, n_g} - \xbb_{k, n_g}\|_2.
\end{align*}
% where in the first inequality, we used the bound that is obtained by performing one gradient descent iteration on the strongly convex function $f$ (\cref{asum.convex and smooth}) at the average iterate $\bar{x}_{k, j}$ with $\alpha < \frac{1}{L}$ \cite[Theorem 2.1.14]{nesterov1998introductory}, the second equality is due to \eqref{eq: v_bar_telescope}, and the last inequality is due to \cref{asum.convex and smooth}. 
% using the error bound of gradient descent from \cite[Theorem 2.1.14]{nesterov1998introductory} and due to \eqref{eq: v_bar_telescope}.
Recursively applying the above two bounds, by \eqref{eq:consensus_error_deviation} it follows that,
% We join the two bounds above by telescoping as follows,
% \begin{align*}
%     \|\Bar{x}_{k, n_g} - x^*\|_2 & \leq (1-\alpha \mu)^{n_g-1} \|\Bar{x}_{k, 1} - x^*\|_2 + \frac{\alpha L}{\sqrt{n}} \sum_{j = 1}^{n_g-1}(1 - \alpha\mu)^{n_g - j - 1}\| \xmbf_{k, j} - \xbb_{k, j}\|_2.
% \end{align*}
% We obtain the optimization error at $x_{k+1}$ the same way as
% \begin{align*}
%     \|\Bar{x}_{k+1, 1} - x^*\|_2 & \leq (1-\alpha \mu) \|\Bar{x}_{k, n_g} - x^*\|_2 + \frac{\alpha L}{\sqrt{n}} \| \xmbf_{k, n_g} - \xbb_{k, n_g}\|_2.
% \end{align*}
% We combine the above 2 equations to get
\begin{align*}
    \|\Bar{x}_{k+1, 1} - x^*\|_2 & 
    % \leq (1-\alpha \mu)^2 \|\Bar{x}_{k, n_g - 1} - x^*\|_2 + \tfrac{(1-\alpha \mu) \alpha L}{\sqrt{n}} \| \xmbf_{k, n_g - 1} - \xbb_{k, n_g - 1}\|_2 \\
    % & \quad+ \tfrac{\alpha L}{\sqrt{n}} \| \xmbf_{k, n_g} - \xbb_{k, n_g}\|_2 \\
    % & \leq (1-\alpha \mu)^{n_g} \|\Bar{x}_{k, 1} - x^*\|_2 + \tfrac{\alpha L}{\sqrt{n}} \sum_{j = 1}^{n_g}(1 - \alpha\mu)^{n_g - j}\| \xmbf_{k, j} - \xbb_{k, j}\|_2 \\
    \leq (1-\alpha \mu)^{n_g} \|\Bar{x}_{k, 1} - x^*\|_2  + \tfrac{\alpha L}{\sqrt{n}} \sum_{j = 1}^{n_g}(1 - \alpha\mu)^{n_g - j} \|\xmbf_{k,1} - \xbb_{k, 1}\|_2 \\
    & \quad + \tfrac{2\alpha^2 L}{\sqrt{n}} \sum_{j = 1}^{n_g}(1 - \alpha\mu)^{n_g - j} (j-1) \|\ymbf_{k, 1}\|_2\\
    & \leq (1-\alpha \mu)^{n_g} \|\Bar{x}_{k, 1} - x^*\|_2  + \tfrac{\kappa}{\sqrt{n}} \left[1 - (1 - \alpha\mu)^{n_g}\right] \|\xmbf_{k, 1} - \xbb_{k, 1}\|_2 \\
    & \quad + \tfrac{\alpha^2 L}{\sqrt{n}} n_g(n_g - 1) \|\ymbf_{k, 1}\|_2,
\end{align*}
% Old full version
% \begin{align*}
%     \|\Bar{x}_{k+1, 1} - x^*\|_2 & \leq (1-\alpha \mu)^2 \|\Bar{x}_{k, n_g - 1} - x^*\|_2 + \tfrac{(1-\alpha \mu) \alpha L}{\sqrt{n}} \| \xmbf_{k, n_g - 1} - \xbb_{k, n_g - 1}\|_2 \\
%     & \quad+ \tfrac{\alpha L}{\sqrt{n}} \| \xmbf_{k, n_g} - \xbb_{k, n_g}\|_2 \\
%     & \leq (1-\alpha \mu)^{n_g} \|\Bar{x}_{k, 1} - x^*\|_2 + \tfrac{\alpha L}{\sqrt{n}} \sum_{j = 1}^{n_g}(1 - \alpha\mu)^{n_g - j}\| \xmbf_{k, j} - \xbb_{k, j}\|_2 \\
%     & \leq (1-\alpha \mu)^{n_g} \|\Bar{x}_{k, 1} - x^*\|_2  + \tfrac{\alpha L}{\sqrt{n}} \sum_{j = 1}^{n_g}(1 - \alpha\mu)^{n_g - j} \|\xmbf_{k,1} - \xbb_{k, 1}\|_2 \\
%     & \quad + \tfrac{2\alpha^2 L}{\sqrt{n}} \sum_{j = 1}^{n_g}(1 - \alpha\mu)^{n_g - j} (j-1) \|\ymbf_{k, 1}\|_2. 
% \end{align*}
%where the last inequality follows from \eqref{eq:consensus_error_deviation}. 
where the last inequality is due to the fact that $(1 - \alpha\mu)^{n_g - j} \leq 1 \,\forall j = 1, 2, ..., n_g$ due to $\alpha \leq \frac{1}{L n_g}$, the coefficient of the second term is the sum of a geometric progression, and the coefficient of the third term is the sum of the first $n_g-1$ natural numbers. 
% In the above bound, the second term coefficient is the sum of a geometric progression. In the third term coefficient, we bound $(1 - \alpha\mu)^{n_g - j} \leq 1 \,\forall j = 1, 2, ..., n_g$ due to $\alpha \leq \frac{1}{L n_g}$ and calculate the resulting coefficient as the sum of the first $n_g - 1$ natural numbers. This yields
% \begin{align*}
    % \|\Bar{x}_{k+1, 1} - x^*\|_2 
    % & \leq (1-\alpha \mu)^{n_g} \|\Bar{x}_{k, 1} - x^*\|_2  + \frac{\alpha L}{\sqrt{n}} \sum_{j = 1}^{n_g}(1 - \alpha\mu)^{n_g - j} \|\xmbf_{k,1} - \xbb_{k, 1}\|_2 \\
    % & \quad + \frac{\alpha L}{\sqrt{n}} \sum_{j = 2}^{n_g}(1 - \alpha\mu)^{n_g - j}\|\xmbf_{k, j} - \xmbf_{k, 1}\|_2\ \\
    % & \leq (1-\alpha \mu)^{n_g} \|\Bar{x}_{k, 1} - x^*\|_2  + \frac{\alpha L}{\sqrt{n}} \left[\frac{1 - (1 - \alpha\mu)^{n_g}}{1 - (1 - \alpha\mu)}\right] \|\xmbf_{k, 1} - \xbb_{k, 1}\|_2 \\
    % & \quad + \frac{2\alpha^2 L}{\sqrt{n}} \sum_{j = 1}^{n_g}(1 - \alpha\mu)^{n_g - j} (j-1) \|\ymbf_{k, 1}\|_2.
% \end{align*}
% We bound $(1 - \alpha\mu)^{n_g - j} \leq 1 \,\,\forall j = 1, 2, ..., n_g$ as $\alpha \leq \frac{1}{L}$ to get
% \begin{align*}
%     \|\Bar{x}_{k+1, 1} - x^*\|_2 
%     & \leq (1-\alpha \mu)^{n_g} \|\Bar{x}_{k, 1} - x^*\|_2  + \tfrac{\kappa}{\sqrt{n}} \left[1 - (1 - \alpha\mu)^{n_g}\right] \|\xmbf_{k, 1} - \xbb_{k, 1}\|_2 \\
%     & \quad + \tfrac{\alpha^2 L}{\sqrt{n}} n_g(n_g - 1) \|\ymbf_{k, 1}\|_2
% \end{align*} 
By \eqref{eq:y_bound}, we obtain the desired bound on the optimization error.

Next, we consider the consensus error in $\xmbf_{k, 1}$,
\begin{align*}
     &\xmbf_{k + 1, 1} - \xbb_{k+1, 1} \\
    =& \left(I_{nd} - \Imbf\right)\xmbf_{k+1, 1} = \left(I_{nd} - \Imbf \right)(\Zmbf_1^{n_c} \xmbf_{k, n_g} - \alpha \Zmbf_2^{n_c}\ymbf_{k, n_g})    \\
    =& \left(I_{nd} - \Imbf \right)\left(\Zmbf_1^{n_c}\left(\xmbf_{k, 1} - \alpha\sum_{j = 1}^{n_g - 1}\ymbf_{k, j}\right) - \alpha \Zmbf_2^{n_c}\ymbf_{k, n_g}\right) \\
    =& \left(I_{nd} - \Imbf\right)\left(\Zmbf_1^{n_c}\xmbf_{k, 1} - \alpha \Zmbf_1^{n_c}\left((n_g-1) \ymbf_{k, 1} + \sum_{j = 2}^{n_g-1} \nabla \fmbf(\xmbf_{k, j}) - \nabla \fmbf(\xmbf_{k, 1})\right)\right) \\
    & \; - \alpha \left(I_{nd} - \Imbf\right) \left(\Zmbf_2^{n_c}( \ymbf_{k, 1} + \nabla \fmbf(\xmbf_{k, n_g}) - \nabla \fmbf(\xmbf_{k, 1}))\right)\\
    =& \left(\Zmbf_1^{n_c} - \Imbf\right)(\xmbf_{k, 1} - \xbb_{k, 1}) -\alpha \left((n_g-1)\left(\Zmbf_1^{n_c} - \Imbf\right) + \left(\Zmbf_2^{n_c} - \Imbf\right)\right) (\ymbf_{k, 1} - \ybb_{k, 1})  \\
    & \; - \alpha\left(\Zmbf_2^{n_c} - \Imbf\right)\left(\nabla \fmbf(\xmbf_{k, n_g}) - \nabla \fmbf(\xmbf_{k, 1})\right) - \alpha\left(\Zmbf_1^{n_c} - \Imbf\right)\left(\sum_{j = 2}^{n_g-1} \nabla \fmbf(\xmbf_{k, j}) - \nabla \fmbf(\xmbf_{k, 1})\right) 
\end{align*}
where the second equality is a telescopic sum of the inner loop update ($\xmbf_{k, i} = \xmbf_{k, i-1} - \alpha \ymbf_{k, i-1}$) from $i=2$ to $n_g$ and the third equality is due to \eqref{eq: sum_y_k_j}. By the triangle inequality, \cref{asum.convex and smooth} and \eqref{eq : beta and Z},
\begin{align*}
    \|\xmbf_{k + 1, 1} - \xbb_{k+1, 1}\|_2 &\leq \beta_1^{n_c}\|\xmbf_{k, 1} - \xbb_{k, 1}\|_2 + \alpha \left((n_g-1)\beta_1^{n_c} + \beta_2^{n_c}\right) \|\ymbf_{k, 1} - \ybb_{k, 1}\|_2 \\
    & \quad + \alpha\beta_2^{n_c} L\|\xmbf_{k, n_g} - \xmbf_{k, 1}\|_2 + \alpha\beta_1^{n_c} L \sum_{j = 2}^{n_g-1} \|\xmbf_{k, j} - \xmbf_{k, 1}\|_2.
\end{align*}
Adding $\alpha\beta_1^{n_c} L \|\xmbf_{k, 1} - \xmbf_{k, 1}\|_2 = 0$ to the right hand side and \eqref{eq:iterate_deviation}, it follows,
\begin{align*}
    \|\xmbf_{k + 1, 1} - \xbb_{k+1, 1}\|_2 &\leq \beta_1^{n_c}\|\xmbf_{k, 1} - \xbb_{k, 1}\|_2 + \alpha \left((n_g-1)\beta_1^{n_c} + \beta_2^{n_c}\right) \|\ymbf_{k, 1} - \ybb_{k, 1}\|_2 \\
    & \quad + \alpha^2\beta_2^{n_c} L (2(n_g - 1)) \|y_{k, 1}\|_2 + \alpha^2\beta_1^{n_c} L \sum_{j = 1}^{n_g-1} 2(j - 1) \|y_{k, 1}\|_2 \\
    &= \beta_1^{n_c}\|\xmbf_{k, 1} - \xbb_{k, 1}\|_2 + \alpha \left((n_g-1)\beta_1^{n_c} + \beta_2^{n_c}\right) \|\ymbf_{k, 1} - \ybb_{k, 1}\|_2 \\
    & \quad + 2\alpha^2 L (n_g - 1) \left( \beta_2^{n_c} + \beta_1^{n_c}\tfrac{(n_g - 2)}{2}  \right)\|\ymbf_{k, 1}\|_2.
\end{align*}
The desired bound for the consensus error in $\xmbf_{k, 1}$ follows by using \eqref{eq:y_bound}.
% Substituting \eqref{eq:y_bound} in the above gives the desired bound for consensus error in $\xmbf_{k, 1}$. 

Finally, we consider the consensus error in $\ymbf_{k, 1}$. By \eqref{eq : y general telescope sum},
\begin{align*}
    \ymbf_{k + 1, 1} - \ybb_{k+1, 1} &= \left(I_{nd} - \Imbf\right)\ymbf_{k+1, 1} \\
    &= \left(I_{nd} - \Imbf\right)(\Zmbf_3^{n_c}\ymbf_{k, 1} + \Zmbf_4^{n_c}\nabla \fmbf(\xmbf_{k+1, 1}) - \Zmbf_3^{n_c}\nabla \fmbf(\xmbf_{k, 1})) \\
    &\quad + \left(I_{nd} - \Imbf\right)(\Zmbf_3^{n_c}\nabla \fmbf(\xmbf_{k, n_g}) - \Zmbf_4^{n_c} \nabla \fmbf(\xmbf_{k, n_g}))\\
    & = \left(\Zmbf_3^{n_c} - \Imbf\right)(\ymbf_{k, 1} - \ybb_{k, 1}) + \left(\Zmbf_4^{n_c} - \Imbf\right)(\nabla \fmbf(\xmbf_{k+1, 1}) - \nabla \fmbf(\xmbf_{k, n_g})) \\
    & \quad + \left(\Zmbf_3^{n_c} - \Imbf\right)(\nabla \fmbf(\xmbf_{k, n_g}) - \nabla \fmbf(\xmbf_{k, 1}) )
\end{align*}
By \cref{asum.convex and smooth} and \eqref{eq : beta and Z},
\begin{align*}
    &\|\ymbf_{k + 1, 1} - \ybb_{k+1, 1}\|_2 \\
    &\leq \beta_3^{n_c}\|\ymbf_{k, 1} - \ybb_{k, 1}\|_2 + \beta_4^{n_c} L\|\xmbf_{k+1, 1} - \xmbf_{k, n_g}\|_2 + \beta_3^{n_c} L\|\xmbf_{k, n_g} - \xmbf_{k, 1}\|_2 \\
    &= \beta_3^{n_c}\|\ymbf_{k, 1} - \ybb_{k, 1}\|_2 + \beta_4^{n_c} L\|(\Zmbf_1^{n_c} - I_{nd})(\xmbf_{k, n_g} - \xbb_{k, n_g}) - \alpha \Zmbf_2^{n_c} \ymbf_{k, n_g}\|_2   \\
    & \quad+ \beta_3^{n_c} L\|\xmbf_{k, n_g} - \xmbf_{k, 1}\|_2\\
    &\leq \beta_3^{n_c}\|\ymbf_{k, 1} - \ybb_{k, 1}\|_2 + \beta_4^{n_c} L\|\Zmbf_1^{n_c} - I_{nd}\|_2\|\xmbf_{k, n_g} - \xbb_{k, n_g}\|_2 + \alpha \beta_4^{n_c} L \|\Zmbf_2^{n_c}\|_2\| \ymbf_{k, n_g}\|_2 \\
    & \quad  + \beta_3^{n_c} L\|\xmbf_{k, n_g} - \xmbf_{k, 1}\|_2 \\
    &= \beta_3^{n_c}\|\ymbf_{k, 1} - \ybb_{k, 1}\|_2 + \beta_4^{n_c} L\|\Zmbf_1^{n_c} - I_{nd}\|_2\|\xmbf_{k, n_g} - \xbb_{k, n_g}\|_2 \\
    & \quad + \alpha \beta_4^{n_c} L \| \ymbf_{k, 1} + \nabla \fmbf(\xmbf_{k, n_g}) - \nabla \fmbf(\xmbf_{k, 1})\|_2 + \beta_3^{n_c} L\|\xmbf_{k, n_g} - \xmbf_{k, 1}\|_2 \\
    &\leq \beta_3^{n_c}\|\ymbf_{k, 1} - \ybb_{k, 1}\|_2 + \beta_4^{n_c} L\|\Zmbf_1^{n_c} - I_{nd}\|_2\|\xmbf_{k, n_g} - \xbb_{k, n_g}\|_2 \\
    & \quad + \alpha \beta_4^{n_c} L \| \ymbf_{k, 1}\|_2 + \alpha \beta_4^{n_c} L^2 \|\xmbf_{k, n_g} - \xmbf_{k, 1}\|_2 + \beta_3^{n_c} L\|\xmbf_{k, n_g} - \xmbf_{k, 1}\|_2
\end{align*}
where the first equality follows from $\xmbf_{k+1, 1} = \Zmbf_1^{n_c} \xmbf_{k, n_g} - \alpha\Zmbf_2^{n_c}\ymbf_{k, n_g}$ and $- (\Zmbf_1^{n_c} - I_{nd})\xbb_{k, n_g} = 0$, the second inequality is by the triangle inequality, the second equality follows by %substitutes $\ymbf_{k, n_g}$ by 
\eqref{eq: v telescope}, and the last inequality is an application of triangle inequality and \cref{asum.convex and smooth}. By \eqref{eq:iterate_deviation}, \eqref{eq:consensus_error_deviation} and $\alpha L \leq \frac{1}{n_g}$, it follows,
\begin{align*}
    &\|\ymbf_{k + 1, 1} - \ybb_{k+1, 1}\|_2 \\
    \leq &\beta_3^{n_c}\|\ymbf_{k, 1} - \ybb_{k, 1}\|_2 + \beta_4^{n_c} L\|\Zmbf_1^{n_c} - I_{nd}\|_2 \|\xmbf_{k, 1} - \xbb_{k, 1}\|_2 \\
    &  + \left( \alpha \beta_4^{n_c} L + 2\alpha (n_g-1) L \left( \beta_4^{n_c} \|\Zmbf_1^{n_c} - I_{nd}\|_2 + \tfrac{\beta_4^{n_c}}{n_g}  + \beta_3^{n_c} \right) \right) \|\ymbf_{k, 1}\|_2.
\end{align*}
Substituting \eqref{eq:y_bound} yields the desired bound for the consensus error in $\ymbf_{k,1}$.
\eproof

\cref{lem:lyapunov g > 1} quantifies the progression of error vector $r_k$ using the matrix $B(n_c, n_g)$, similar to \cref{lem:lyapunov g = 1} but now allowing for multiple computation steps. Notice that when $n_g = 1$, \cref{lem:lyapunov g > 1} reduces to \cref{lem:lyapunov g = 1}, making it a special case of this analysis. We split the matrix $B(n_c, n_g)$ into the matrices $A(n_c, n_g)$ and $E(n_c, n_g)$. The latter matrix is characterized by the terms $\delta_1(n_c, n_g)$ and $\delta_2(n_c, n_g)$. 
We now define the explicit form of $B(n_c, n_g)$ %in terms of the above mentioned components 
for the methods defined in \cref{tab: Algorithm Def}.

\bcorollary \label{col. B special cases}
Suppose the conditions of \cref{lem:lyapunov g > 1} are satisfied. Then, the matrices $A(n_c, n_g)$ for the methods described in \cref{tab: Algorithm Def} are defined as:
\begin{equation} \label{eq : g > 1 algos A}
    \begin{aligned}
        \mbox{\texttt{GTA-1}:} & \quad A_1({n_c, n_g}) = \begin{bmatrix}
        (1 - \alpha \mu)^{n_g}& \frac{\kappa}{\sqrt{n}}(1 - (1 - \alpha\mu)^{n_g}) & 0\\
        0 & \beta^{n_c} & \alpha\left((n_g-1)\beta^{n_c} + 1\right)\\
        \sqrt{n}\alpha L^2 & L(2 + \alpha L) & \beta^{n_c} + \alpha L
    \end{bmatrix},\\
    \mbox{\texttt{GTA-2}:} & \quad A_2({n_c, n_g})  = \begin{bmatrix}
        (1 - \alpha \mu)^{n_g}& \frac{\kappa}{\sqrt{n}}(1 - (1 - \alpha\mu)^{n_g}) & 0\\
        0 & \beta^{n_c} & \alpha\beta^{n_c}n_g\\
        \sqrt{n}\alpha L^2 & L(2 + \alpha L) & \beta^{n_c} + \alpha  L
    \end{bmatrix}, \\
    \mbox{\texttt{GTA-3}:} & \quad A_3({n_c, n_g}) = \begin{bmatrix}
        (1 - \alpha \mu)^{n_g}& \frac{\kappa}{\sqrt{n}}(1 - (1 - \alpha\mu)^{n_g}) & 0\\
        0 & \beta^{n_c} & \alpha\beta^{n_c}n_g\\
        \sqrt{n}\alpha \beta^{n_c} L^2 & \beta^{n_c}L(2 + \alpha L) & \beta^{n_c}(1 + \alpha L)
    \end{bmatrix}.
    \end{aligned}
\end{equation}
The matrix $E(n_c, n_g)$ for the methods described in \cref{tab: Algorithm Def} is defined using the error terms $(\delta_1(n_c, n_g)$ and $\delta_2(n_c, n_g))$. 
The error terms for the methods described in \cref{tab: Algorithm Def} are defined in \cref{tab: n_g > 1 error terms special cases}.
\begin{table}[H]\centering
\caption{Error terms ($\delta_1(n_c, n_g)$ and $\delta_2(n_c, n_g)$) for \texttt{GTA-1}, \texttt{GTA-2} and \texttt{GTA-3}}\label{tab: n_g > 1 error terms special cases}
\begin{tabular}{lcc}\toprule
Method & $\delta_1(n_c, n_g)$ & $\delta_2(n_c, n_g)$ \vspace{2pt} \\ \hline \\[-8pt]
\texttt{GTA-1} & $2 + \beta^{n_c}(n_g - 2)$ & $2 \left( 2 + \tfrac{1}{n_g} + \beta^{n_c} \right)$ \vspace{2pt} \\ \hdashline \\[-10pt]
\texttt{GTA-2} & $n_g\beta^{n_c}$ & $2 \left( 2 + \tfrac{1}{n_g} + \beta^{n_c} \right)$ \vspace{2pt} \\ \hdashline \\[-10pt]
\texttt{GTA-3} & $ n_g\beta^{n_c} $ & $2 \beta^{n_c} \left( 3 + \tfrac{1}{n_g} \right)$ \vspace{2pt} \\
\bottomrule
\end{tabular}
\end{table}
\ecorollary
\bproof
Substituting the matrix values for each method in \eqref{eq : g > 1 general A} and bounding $\|\Zmbf_1^{n_c} - I_{nd}\|_2 \leq 2$ gives the desired result.
\eproof

\cref{col. B special cases} presents the explicit form of the matrices $B_i(n_c, n_g) = A_i(n_c, n_g) + \alpha L (n_g - 1) E_i(n_c, n_g)$ for $i = 1, 2, 3$, for each of the methods in \cref{tab: Algorithm Def}. The convergence properties of \texttt{GTA} can be analysed using the spectral radius of $B(n_c, n_g)$. We now qualitatively establish the effect of the number of communication steps $n_c$ on $\rho(B(n_c, n_g))$ and a relative ordering for $\rho(B_1(n_c, n_g))$, $\rho(B_2(n_c, n_g))$ and $\rho(B_3(n_c, n_g))$.

% old version
%By \cref{tab: n_g > 1 error terms special cases}, we define matrices $E_1(n_c, n_g), E_2(n_c, n_g), E_3(n_c, n_g)$ for \texttt{GTA-1}, \texttt{GTA-2} and \texttt{GTA-3} respectively. Correspondingly we define $B_i(n_c, n_g) = A_i(n_c, n_g) + \alpha L (n_g - 1) E(n_c, n_g)$ for $i = 1, 2, 3$. The convergence properties of \texttt{GTA} can be analysed using the spectral radius of $B(n_c, n_g)$. We now qualitatively establish the effect of $n_c$ on $\rho(B(n_c, n_g))$ and a relative ordering for $\rho(B_1(n_c, n_g))$, $\rho(B_2(n_c, n_g))$ and $\rho(B_3(n_c, n_g))$.
% \begin{lemma}\label{lem. g>1 spec norm}
% Suppose \cref{asum.convex and smooth} holds, and $\alpha < \frac{1}{n_g L}$. If the matrix $B({n_c, n_g})$ defined in  \cref{lem:lyapunov g > 1} is irreducible then the spectral norm of $B({n_c, n_g})$ is the largest eigen value of $B({n_c, n_g})$ and is a positive real number. Consequently, if $\Wmbf \neq \frac{1_n1_n^T}{n}$, i.e., the network is not fully connected, the spectral norm of matrices $B_1({n_c, n_g})$, $B_2({n_c, n_g})$, $B_3({n_c, n_g})$ defined using \cref{col. B special cases} are also positive real numbers and equal to their largest eigenvalues respectively.
% \end{lemma}

% \begin{proof}
% The statement about general matrix $B({n_c, n_g})$ directly follows from the Perron-Forbenius Theorem (\cite[Theorem 8.4.4]{horn2012matrix}), and observing that the matrix $B({n_c, n_g})$ is nonnegative and irreducible. Consequently, the statement about $B_1({n_c, n_g})$, $B_2({n_c, n_g})$, $B_3({n_c, n_g})$ also follows as $A_1({n_c, n_g})$, $A_2({n_c, n_g})$, $A_3({n_c, n_g})$ matrices are irreducible when $\Wmbf \neq \frac{1_n1_n^T}{n}$, i.e., $\beta > 0$.
% \end{proof}

\btheorem   \label{th.incr rates g > 1}
Suppose \cref{asum.convex and smooth} holds. If $\alpha \leq \frac{1}{L n_g}$ in \cref{alg : Deterministic}, then
as $n_c$ increases, $\rho(B(n_c, n_g))$ decreases where $B(n_c, n_g)$ is defined in \cref{lem:lyapunov g > 1}. Thus, as $n_c$ increases, $\rho(B_i(n_c, n_g))$ decreases for all $i =1, 2, 3$ defined in \cref{col. B special cases}. Moreover, if all three methods defined in \cref{tab: Algorithm Def} (\texttt{GTA-1}, \texttt{GTA-2} and \texttt{GTA-3}) employ the same stepsize,
\begin{align*}
\rho(B_1({n_c, n_g})) \geq \rho(B_2({n_c, n_g})) \geq \rho(B_3({n_c, n_g})).
\end{align*}
\etheorem

\bproof
Note that $A(n_c, n_g) \geq 0$ and $E(n_c, n_g) \geq 0$, thus $B(n_c, n_g) \geq 0$. Also, $A(n_c, n_g) \geq A(n_c + 1, n_g) $, $\delta_1(n_c, n_g) \geq \delta_1(n_c + 1, n_g)$, $\delta_2(n_c, n_g) \geq \delta_2(n_c + 1, n_g)$, thus $E(n_c, n_g) \geq E(n_c + 1, n_g)$ and $B(n_c, n_g) \geq B(n_c + 1, n_g)$.  By \cite[Corollary 8.1.19]{horn2012matrix}, it follows that $\rho(A(n_c, n_g)) \geq \rho(A(n_c + 1, n_g))$, $\rho(E(n_c, n_g)) \geq \rho(E(n_c + 1, n_g))$ and $\rho(B(n_c, n_g)) \geq \rho(B(n_c + 1, n_g))$. The same argument is applicable for $B_1({n_c, n_g})$, $B_2({n_c, n_g})$ and $B_3({n_c, n_g})$. Now, observe that $B_1({n_c, n_g}) \geq B_2({n_c, n_g}) \geq B_3({n_c, n_g}) \geq 0$ when the same step size is employed. Thus, again by \cite[Corollary 8.1.19]{horn2012matrix}, it follows that $\rho(B_1({n_c, n_g})) \geq \rho(B_2({n_c, n_g})) \geq \rho(B_3({n_c, n_g}))$.
\eproof

The effect of the number of computation steps $n_g$ on $\rho(B({n_c, n_g}))$ is not clear as the effect of the number of communication steps $n_c$. Increasing $n_g$ increases all elements of the matrix $\alpha L (n_g - 1)E(n_c, n_g)$, while $(1 - \alpha \mu)^{n_g}$ in the matrix $A(n_c, n_g)$ decreases since $\alpha \leq \frac{1}{Ln_g}$. Thus, the effect of $n_g$ on $\rho(B(n_c, n_g))$ is not monotonic. 

We now derive conditions for establishing a 
linear rate of convergence for \cref{alg : Deterministic} with multiple communication and computation steps every iteration in terms of network parameters ($\beta_1, \beta_2, \beta_3, \beta_4$) and objective function parameters ($L, \mu, \kappa = \frac{L}{\mu}$).

\btheorem  \label{th. alpha bound g > 1}
Suppose \cref{asum.convex and smooth} holds and a finite number of computation steps are performed at each outer iteration of \cref{alg : Deterministic} (i.e., $1 \leq n_g < \infty$). If the matrix $B(n_c, n_g)$ is irreducible, $\beta_1, \beta_3 < 1$ and
\begin{align} \label{eq : alpha mult grads gen}
    \alpha < \min \left\{\tfrac{1}{n_g L}, \tfrac{\mu}{(2L^2 + \mu^2)(n_g - 1)}, \tfrac{1}{2L} \sqrt{\tfrac{3(1 - \beta_1^{n_c})}{\delta_1(n_c, n_g) (n_g - 1)}}, \tfrac{3(1 - \beta_3^{n_c})}{4L(\beta_4^{n_c} + \delta_2(n_c, n_g)(n_g - 1))}, \tfrac{ - b_2 + \sqrt{b_2^2 + 4b_1b_3}}{2b_1}\right\}
 \end{align}
where
\begin{align*}
    % b_1 &= \tfrac{\mu L^2 n_g}{2} \left((n_g-1) \left( \beta_1^{n_c} + \delta_1(n_c, n_g)\right) + \beta_2^{n_c} \right)\left(\beta_4^{n_c} +  (n_g - 1)\delta_2(n_c, n_g)\right) \\
    % &+L^2 (n_g - 1)\delta_1(n_c, n_g) \left( L n_g +  (n_g - 1)\right) \left(\tfrac{1 - \beta_3^{n_c}}{4}\right) \\
    % &+ L^3 (n_g - 1)^2 \delta_1(n_c, n_g)\left(3\beta_4^{n_c} + (n_g - 1)\delta_2(n_c, n_g)\right) \\
    % &+ L^2( \beta_4^{n_c} + (n_g - 1)\delta_2(n_c, n_g)) \left(L n_g + (n_g - 1)\right) \left((n_g-1) \left( \beta_1^{n_c} + \delta_1(n_c, n_g)\right) + \beta_2^{n_c} \right) \\
    % & +L^3 n_g (n_g - 1) (\beta_4^{n_c} + (n_g - 1)\delta_2(n_c, n_g)) \left(\tfrac{1 - \beta_1^{n_c}}{4}\right) \\
    b_1 =& \tfrac{\mu L^2 n_g}{2} \left[(n_g-1) \left( \beta_1^{n_c} + \delta_1(n_c, n_g)\right) + \beta_2^{n_c} \right]\left[\beta_4^{n_c} +  (n_g - 1)\delta_2(n_c, n_g)\right] \\
    &+L^3 n_g (n_g - 1) \left[ \delta_1(n_c, n_g)  \left(\tfrac{1 - \beta_3^{n_c}}{4}\right) + (\beta_4^{n_c} + (n_g - 1)\delta_2(n_c, n_g)) \left(\tfrac{1 - \beta_1^{n_c}}{4}\right) \right]\\
    &+ L^2 (n_g - 1)^2 \left[ L\delta_1(n_c, n_g)\left(3\beta_4^{n_c} + (n_g - 1)\delta_2(n_c, n_g)\right) +  \delta_1(n_c, n_g) \left(\tfrac{1 - \beta_3^{n_c}}{4}\right)\right] \\
    &+ L^2[ \beta_4^{n_c} + (n_g - 1)\delta_2(n_c, n_g)] \left[L n_g + (n_g - 1)\right] \left[(n_g-1) \left( \beta_1^{n_c} + \delta_1(n_c, n_g)\right) + \beta_2^{n_c} \right] \\
    % b_1 =& \sg{\tfrac{\mu L^2 n_g}{2} \left[(n_g-1) \left( \beta_1^{n_c} + \delta_1(n_c, n_g)\right) + \beta_2^{n_c} \right]\left[\beta_4^{n_c} +  (n_g - 1)\delta_2(n_c, n_g)\right]} \\
    % &\sg{+L^3 n_g (n_g - 1) \left[ \delta_1(n_c, n_g)  \left(\tfrac{1 - \beta_3^{n_c}}{4}\right) + \beta_4^{n_c}  \left(\tfrac{1 - \beta_1^{n_c}}{4}\right) \right] }\\
    % &\sg{+ L^2 (n_g - 1)^2 \left[ L\delta_1(n_c, n_g)\left(3\beta_4^{n_c} + (n_g - 1)\delta_2(n_c, n_g)\right) +  \delta_1(n_c, n_g) \left(\tfrac{1 - \beta_3^{n_c}}{4}\right)  +L n_g \delta_2(n_c, n_g) \left(\tfrac{1 - \beta_1^{n_c}}{4}\right)\right]} \\
    % &\sg{+ L^2[ \beta_4^{n_c} + (n_g - 1)\delta_2(n_c, n_g)] \left[L n_g + (n_g - 1)\right] \left[(n_g-1) \left( \beta_1^{n_c} + \delta_1(n_c, n_g)\right) + \beta_2^{n_c} \right]} \\
    % & +L^3 n_g (n_g+L^3 n_g (n_g - 1) (\beta_4^{n_c} + (n_g - 1)\delta_2(n_c, n_g)) \left(\tfrac{1 - \beta_1^{n_c}}{4}\right) - 1) (\beta_4^{n_c} + (n_g - 1)\delta_2(n_c, n_g)) \left(\tfrac{1 - \beta_1^{n_c}}{4}\right) \\
    b_2  = & \mu n_g \beta_4^{n_c}L \left((n_g-1)(\beta_1^{n_c} + \delta_1(n_c, n_g)) + \beta_2^{n_c} \right), \text{ and }\; b_3 = \tfrac{\mu n_g}{2} \left( \tfrac{1 - \beta_1^{n_c}}{4}\right) \left(\tfrac{1 - \beta_3^{n_c}}{4}\right)
    % b_2 & =  \mu n_g \beta_4^{n_c}L \left((n_g-1)\beta_1^{n_c} + \beta_2^{n_c} + (n_g - 1) \delta_1(n_c, n_g)\right) \\
    % b_3 & = \tfrac{\mu n_g}{2} \left( \tfrac{1 - \beta_1^{n_c}}{4}\right) \left(\tfrac{1 - \beta_3^{n_c}}{4}\right)
\end{align*}
and $\delta_1(n_c, n_g)$ and $\delta_2(n_c, n_g)$ are defined in \eqref{eq:delta_errors}, 
% then there exists a sequence $\epsilon_k\geq 0$ such that,
% \begin{align*}
%     \|r_{k}\|_2 \leq (\rho(B(n_c, n_g)) + \epsilon_k)^k \|r_0\|_2 \quad \mbox{and} \quad \lim_{k \rightarrow \infty} \epsilon_k = 0
% \end{align*}
% for all $k \geq 0$, where $\rho(B({n_c, n_g})) < 1$.
then, for all $\epsilon > 0$ there exists a constant $C_{\epsilon} > 0$ such that, for all $k\geq 0$,
\begin{align*}
    \|r_{k}\|_2 \leq C_{\epsilon}(\rho(B({n_c, n_g})) + \epsilon)^k \|r_0\|_2, \quad \text{where } \; \rho(B({n_c, n_g})) < 1.
\end{align*} 
\etheorem
\bproof
By the binomial expansion of $(1-\alpha\mu)^{n_g}$ and the condition that $\alpha \leq \tfrac{1}{L n_g}$, it follows that $1 - \alpha \mu n_g \leq (1 - \alpha \mu)^{n_g} \leq 1 - \alpha \mu n_g + \alpha^2 \mu^2 \tfrac{n_g (n_g-1)}{2} $. Following a similar approach to \cite[Theorem 2]{pu2020push}, since the step size satisfies \eqref{eq : alpha mult grads gen}, the first, second and third diagonal terms of $B(n_c, n_g)$
% \sg{Upon taylor expanding $(1-\alpha\mu)^{n_g}$, under $\alpha \leq \tfrac{1}{L n_g}$, the series is decreasing in magnitude with alternating signs. Thus, ignoring higher-order terms we can bound $1 - \alpha \mu n_g \leq (1 - \alpha \mu)^{n_g} \leq 1 - \alpha \mu n_g + \alpha^2 \mu^2 \tfrac{n_g (n_g-1)}{2} $. Following a similar approach to \cite[Theorem 2]{pu2020push}, since the step size satisfies \eqref{eq : alpha mult grads gen}, the first}, second and third diagonal terms of $B(n_c, n_g)$
% Following an approach similar to \cite{pu2020push}, as $\alpha$ satisfies the inequalities
% \begin{align*}
%     \alpha < \tfrac{\mu}{2L^2(n_g - 1)}, \quad \alpha < \tfrac{1}{2L} \sqrt{\frac{3(1 - \beta_1^{n_c})}{\delta_1(n_c, n_g) (n_g - 1)}}, \quad \alpha <  \tfrac{3(1 - \beta_3^{n_c})}{4L(\beta_4^{n_c} + \delta_2(n_c, n_g)(n_g - 1))},
% \end{align*}
% the first, second and third diagonal terms of $B(n_c, n_g)$ 
can be upper bounded as
\begin{align*}
    % (1 - \alpha \mu)^{n_g} + \alpha^2 L^2 n_g (n_g - 1)  \asb{\approx} 1 - \alpha \mu n_g + \alpha^2 L^2 n_g (n_g - 1)& < 1 - \tfrac{\alpha \mu n_g}{2}, \\ 
    (1 - \alpha \mu)^{n_g} + \alpha^2 L^2 n_g (n_g - 1)  \leq 1 - \alpha \mu n_g + \alpha^2 (L^2 + \tfrac{\mu^2}{2}) n_g (n_g - 1)& < 1 - \tfrac{\alpha \mu n_g}{2}, \\ 
    \beta_1^{n_c} + \alpha^2 L^2 (n_g - 1) \delta_1(n_c, n_g) & < \tfrac{3 + \beta_1^{n_c}}{4}, \\
    \beta_3^{n_c} + \alpha \beta_4^{n_c} L + \alpha L (n_g - 1) \delta_2(n_c, n_g) & < \tfrac{3 + \beta_3^{n_c}}{4}.
\end{align*}
With the above bounds, $(1-\alpha \mu)^{n_g} \geq 1 - \alpha\mu n_g$ and $\|\Zmbf_1^{n_c} - I_{nd}\| \leq 2$, we construct the $3\times 3$ matrix $\Tilde{B}(n_c, n_g)$ that has entries $\tilde{b}_{ij}$ defined as follows:
% $B(n_c, n_g) = A(n_c, n_g) + \alpha L (n_g - 1) E(n_c, n_g)$
% \begin{align*}
% A(n_c, n_g) &= \begin{bmatrix}
%     (1 - \alpha \mu)^{n_g} + & \frac{\kappa}{\sqrt{n}}(1 - (1 - \alpha\mu)^{n_g}) & 0\\
%     0 & \beta_1^{n_c} & \alpha\left((n_g-1)\beta_1^{n_c} + \beta_2^{n_c}\right)\\
%     \sqrt{n}\alpha \beta_4^{n_c} L^2 & \beta_4^{n_c}L(\|\Zmbf_1^{n_c}-I_{nd}\|_2 + \alpha L) & \beta_3^{n_c} + \alpha \beta_4^{n_c} L
%     \end{bmatrix},\\
% E(n_c, n_g) &= \begin{bmatrix}
%     \alpha L \gamma(n_g)& \frac{\alpha L}{\sqrt{n}} \gamma(n_g) & \frac{\alpha}{\sqrt{n}}\gamma(n_g)\\
%     \sqrt{n}\alpha L \phi(n_c, n_g) & \alpha L \phi(n_c, n_g) & \alpha \phi(n_c, n_g)\\
%     \sqrt{n} L\delta(n_c, n_g) & L \delta(n_c, n_g)& \delta(n_c, n_g)
%     \end{bmatrix}
% \end{align*}
% and
% \begin{align*}
%     &\gamma(n_g) =  n_g, \\
%     &\phi(n_c, n_g) = 2\beta_2^{n_c} + \beta_1^{n_c}(n_g - 2) ,\\
%     &\delta(n_c, n_g) = 2 \left[ \beta_4^{n_c} \|\Zmbf_1^{n_c} - I_{nd}\|_2 + \frac{\beta_4^{n_c}}{n_g} + \beta_3^{n_c} \right].
% \end{align*}
% \begin{align*}
% \Tilde{B}(n_c, n_g) = 
% \begin{bmatrix}
%     \Tilde{b}_{11} & \Tilde{b}_{12} & \Tilde{b}_{13}\\
%     \Tilde{b}_{21} & \Tilde{b}_{22} & \Tilde{b}_{23}\\
%     \Tilde{b}_{31} & \Tilde{b}_{32} & \Tilde{b}_{33}
%     \end{bmatrix}
% \end{align*}
% where 
% \begin{align*}
% &\Tilde{B}(n_c, n_g) = \\
% &\begin{bmatrix}
%     (1 - \frac{\alpha \mu n_g}{2})& \frac{\alpha L n_g}{\sqrt{n}} + \frac{\alpha^2 L^2 n_g(n_g - 1)}{\sqrt{n}} & \frac{\alpha^2 L n_g (n_g - 1)}{\sqrt{n}}\\
%     \sqrt{n}\alpha^2 L^2 (n_g - 1) \delta_1(n_c, n_g) & \frac{3 + \beta_1^{n_c}}{4} & \alpha\left((n_g-1)\beta_1^{n_c} + \beta_2^{n_c}\right) + \alpha^2 L (n_g - 1) \delta_1(n_c, n_g)\\
%     \sqrt{n}\alpha \beta_4^{n_c} L^2 + \sqrt{n}\alpha L^2 (n_g - 1)\delta_2(n_c, n_g) & \beta_4^{n_c}L(2 + \alpha L) + \alpha L^2 (n_g - 1) \delta_2(n_c, n_g) & \frac{3 + \beta_3^{n_c}}{4}
%     \end{bmatrix}
% \end{align*}
% \begin{align*}
%     &\Tilde{b}_{11} = 1 - \tfrac{\alpha \mu n_g}{2}, \quad \Tilde{b}_{12} = \tfrac{\alpha L n_g}{\sqrt{n}} + \tfrac{\alpha^2 L^2 n_g(n_g - 1)}{\sqrt{n}},\quad \Tilde{b}_{13} = \tfrac{\alpha^2 L n_g (n_g - 1)}{\sqrt{n}}\\
%     &\Tilde{b}_{21} = \sqrt{n}\alpha^2 L^2 (n_g - 1) \delta_1(n_c, n_g), \quad \Tilde{b}_{22} = \tfrac{3 + \beta_1^{n_c}}{4},\\
%     & \Tilde{b}_{23} = \alpha\left((n_g-1)\beta_1^{n_c} + \beta_2^{n_c}\right) + \alpha^2 L (n_g - 1) \delta_1(n_c, n_g),\\
%     &\Tilde{b}_{31} = \sqrt{n}\alpha \beta_4^{n_c} L^2 + \sqrt{n}\alpha L^2 (n_g - 1)\delta_2(n_c, n_g), \\
%     &\Tilde{b}_{32} = \beta_4^{n_c}L(2 + \alpha L) + \alpha L^2 (n_g - 1) \delta_2(n_c, n_g), \quad \Tilde{b}_{33} = \tfrac{3 + \beta_3^{n_c}}{4},
% \end{align*}
\begin{align*}
    &\Tilde{b}_{11} = 1 - \tfrac{\alpha \mu n_g}{2}, \quad \Tilde{b}_{12} = \tfrac{\alpha L n_g}{\sqrt{n}} \left(1 + \alpha L (n_g - 1)\right),\quad \Tilde{b}_{13} = \tfrac{\alpha^2 L n_g (n_g - 1)}{\sqrt{n}}\\
    &\Tilde{b}_{21} = \sqrt{n}\alpha^2 L^2 (n_g - 1) \delta_1(n_c, n_g), \quad \Tilde{b}_{22} = \tfrac{3 + \beta_1^{n_c}}{4},\\
    & \Tilde{b}_{23} = \alpha\left((n_g-1) (\beta_1^{n_c}  + \alpha L \delta_1(n_c, n_g)) + \beta_2^{n_c}\right), \\
    &\Tilde{b}_{31} = \sqrt{n}\alpha L^2 \left(\beta_4^{n_c} + (n_g - 1)\delta_2(n_c, n_g)\right), \\
    &\Tilde{b}_{32} = \beta_4^{n_c}L(2 + \alpha L) + \alpha L^2 (n_g - 1) \delta_2(n_c, n_g), \quad \Tilde{b}_{33} = \tfrac{3 + \beta_3^{n_c}}{4},
\end{align*}
such that $0 \leq B(n_c, n_g) \leq \Tilde{B}(n_c, n_g)$ and by \cite[Corollary 8.1.19]{horn2012matrix}, $\rho(B(n_c, n_g)) \leq \rho(\Tilde{B}(n_c, n_g))$. Following \cite[Lemma 5]{pu2021distributed} derived from the Perron-Forbenius Theorem \cite[Theorem 8.4.4]{horn2012matrix} for a $3\times3$ matrix, when the matrix $\Tilde{B}(n_c, n_g)$ is nonnegative and irreducible, it is sufficient to show that the diagonal elements of $\Tilde{B}(n_c, n_g)$ are less than one and $\det(I_3 - \Tilde{B}(n_c, n_g)) > 0$ in order to guarantee $\rho(\Tilde{B}(n_c, n_g)) < 1$ which suffices to show $\rho(B(n_c, n_g)) < 1$. 

% \pagebreak
Consider the diagonal elements of the matrix $\Tilde{B}(n_c, n_g)$. The first element is $1 - \frac{\alpha \mu n_g}{2} \leq 1 - \frac{\mu}{2L} < 1$ by \eqref{eq : alpha mult grads gen}. The second element is $\frac{3 + \beta_1^{n_c}}{4} < 1$ as $\beta_1 < 1$. Finally the third element is $\frac{3 + \beta_3^{n_c}}{4} < 1$ as $\beta_3 < 1$. Next, let us consider,
\begin{align*}
    &\det(I_3 - \Tilde{B}(n_c, n_g)) \\
    =& \tfrac{\alpha \mu n_g}{2} \left( \tfrac{1 - \beta_1^{n_c}}{4}\right) \left(\tfrac{1 - \beta_3^{n_c}}{4}\right) - \alpha^3 L^3 n_g (n_g - 1)\left[\beta_4^{n_c}  \left(\tfrac{1 - \beta_1^{n_c}}{4}\right) + \delta_1(n_c, n_g)  \left(\tfrac{1 - \beta_3^{n_c}}{4}\right)\right]\\
    &- \tfrac{\alpha^2 \mu L n_g}{2} \left[(n_g-1)(\beta_1^{n_c}+ \alpha L \delta_1(n_c, n_g)) + \beta_2^{n_c} \right]\left[\beta_4^{n_c}(2 + \alpha L) + \alpha L (n_g - 1)\delta_2(n_c, n_g)\right] \\
    &-\alpha^4 L^4 n_g (n_g - 1)^2\delta_1(n_c, n_g) \left[2\beta_4^{n_c} + \alpha L \left( \beta_4^{n_c} + (n_g - 1)\delta_2(n_c, n_g)\right) \right] \\
    & - \alpha^3 L^3 n_g\left(1 + \alpha (n_g - 1)\right)\left[ \beta_4^{n_c}  +(n_g - 1)\delta_2(n_c, n_g)\right] \left[(n_g-1) (\beta_1^{n_c} + \alpha L \delta_1(n_c, n_g)) + \beta_2^{n_c} \right] \\
    & - \alpha^3 L^3 n_g (n_g - 1)^2\left[ \delta_2(n_c, n_g)  \left(\tfrac{1 - \beta_1^{n_c}}{4}\right) + \alpha \delta_1(n_c, n_g)  \left(\tfrac{1 - \beta_3^{n_c}}{4}\right)\right]\\  
    % &= \tfrac{\alpha \mu n_g}{2} \left( \tfrac{1 - \beta_1^{n_c}}{4}\right) \left(\tfrac{1 - \beta_3^{n_c}}{4}\right) \\
    % &- \tfrac{\alpha \mu n_g}{2} \left(\alpha\left((n_g-1)\beta_1^{n_c} + \beta_2^{n_c}\right) + \alpha^2 L (n_g - 1) \delta_1(n_c, n_g)\right)\left(\beta_4^{n_c}L(2 + \alpha L) + \alpha L^2 (n_g - 1)\delta_2(n_c, n_g)\right) \\
    % &-\alpha^2 L^2 (n_g - 1)\delta_1(n_c, n_g) \left(\alpha L n_g + \alpha^2 L n_g (n_g - 1)\right) \left(\tfrac{1 - \beta_3^{n_c}}{4}\right) \\
    % &- \alpha^2 L^2 (n_g - 1)\delta_1(n_c, n_g)   \left(\alpha^2 L n_g(n_g - 1)\right)\left(\beta_4^{n_c}L(2 + \alpha L) + \alpha L^2 (n_g - 1)\delta_2(n_c, n_g)\right) \\
    % & - (\alpha \beta_4^{n_c} L^2 +\alpha L^2(n_g - 1)\delta_2(n_c, n_g)) \left(\alpha L n_g + \alpha^2 L n_g (n_g - 1)\right) \left(\alpha\left((n_g-1)\beta_1^{n_c} + \beta_2^{n_c}\right) + \alpha^2 L (n_g - 1) \delta_1(n_c, n_g)\right) \\
    % & - (\alpha \beta_4^{n_c} L^2 + \alpha L^2(n_g - 1)\delta_2(n_c, n_g))  \left(\alpha^2 L n_g (n_g - 1) \right)\left(\tfrac{1 - \beta_1^{n_c}}{4}\right)\\  
    \geq& \alpha(- b_1 \alpha^2 - b_2 \alpha + b_3) = -b_1 \alpha(\alpha - \alpha_l)(\alpha - \alpha_u)
\end{align*}
% \begin{align*}
%     &\det(I_3 - \Tilde{B}(n_c, n_g))\\
%     &= \tfrac{\alpha \mu n_g}{2} \left(1 - \tfrac{3 + \beta_1^{n_c}}{4}\right) \left(1 - \tfrac{3 + \beta_3^{n_c}}{4}\right) \\
%     &- \tfrac{\alpha \mu n_g}{2} \left(\alpha\left((n_g-1)\beta_1^{n_c} + \beta_2^{n_c}\right) + \alpha^2 L (n_g - 1) \delta_1(n_c, n_g)\right)\left(\beta_4^{n_c}L(2 + \alpha L) + \alpha L^2 (n_g - 1)\delta_2(n_c, n_g)\right) \\
%     &- \sqrt{n}\alpha^2 L^2 (n_g - 1)\delta_1(n_c, n_g) \left(\tfrac{\alpha L n_g}{\sqrt{n}} + \tfrac{\alpha^2 L n_g (n_g - 1)}{\sqrt{n}}\right) \left(1 - \tfrac{3 + \beta_3^{n_c}}{4}\right) \\
%     &- \sqrt{n}\alpha^2 L^2 (n_g - 1)\delta_1(n_c, n_g)   \left(\tfrac{\alpha^2 L n_g(n_g - 1)}{\sqrt{n}}\right)\left(\beta_4^{n_c}L(2 + \alpha L) + \alpha L^2 (n_g - 1)\delta_2(n_c, n_g)\right) \\
%     & - (\sqrt{n}\alpha \beta_4^{n_c} L^2 +\sqrt{n}\alpha L^2(n_g - 1)\delta_2(n_c, n_g)) \left(\tfrac{\alpha L n_g}{\sqrt{n}} + \tfrac{\alpha^2 L n_g (n_g - 1)}{\sqrt{n}}\right) \left(\alpha\left((n_g-1)\beta_1^{n_c} + \beta_2^{n_c}\right) + \alpha^2 L (n_g - 1) \delta_1(n_c, n_g)\right) \\
%     & - (\sqrt{n}\alpha \beta_4^{n_c} L^2 +\sqrt{n}\alpha L^2(n_g - 1)\delta_2(n_c, n_g))  \left(\tfrac{\alpha^2 L n_g (n_g - 1)}{\sqrt{n}}\right)\left(1 - \tfrac{3 + \beta_1^{n_c}}{4}\right) \\
%     \\ 
%     &= \tfrac{\alpha \mu n_g}{2} \left( \tfrac{1 - \beta_1^{n_c}}{4}\right) \left(\tfrac{1 - \beta_3^{n_c}}{4}\right) \\
%     &- \tfrac{\alpha \mu n_g}{2} \left(\alpha\left((n_g-1)\beta_1^{n_c} + \beta_2^{n_c}\right) + \alpha^2 L (n_g - 1) \delta_1(n_c, n_g)\right)\left(\beta_4^{n_c}L(2 + \alpha L) + \alpha L^2 (n_g - 1)\delta_2(n_c, n_g)\right) \\
%     &-\alpha^2 L^2 (n_g - 1)\delta_1(n_c, n_g) \left(\alpha L n_g + \alpha^2 L n_g (n_g - 1)\right) \left(\tfrac{1 - \beta_3^{n_c}}{4}\right) \\
%     &- \alpha^2 L^2 (n_g - 1)\delta_1(n_c, n_g)   \left(\alpha^2 L n_g(n_g - 1)\right)\left(\beta_4^{n_c}L(2 + \alpha L) + \alpha L^2 (n_g - 1)\delta_2(n_c, n_g)\right) \\
%     & - (\alpha \beta_4^{n_c} L^2 +\alpha L^2(n_g - 1)\delta_2(n_c, n_g)) \left(\alpha L n_g + \alpha^2 L n_g (n_g - 1)\right) \left(\alpha\left((n_g-1)\beta_1^{n_c} + \beta_2^{n_c}\right) + \alpha^2 L (n_g - 1) \delta_1(n_c, n_g)\right) \\
%     & - (\alpha \beta_4^{n_c} L^2 + \alpha L^2(n_g - 1)\delta_2(n_c, n_g))  \left(\alpha^2 L n_g (n_g - 1) \right)\left(\tfrac{1 - \beta_1^{n_c}}{4}\right)\\  
%     \\ 
%     &\geq \tfrac{\alpha \mu n_g}{2} \left( \tfrac{1 - \beta_1^{n_c}}{4}\right) \left(\tfrac{1 - \beta_3^{n_c}}{4}\right) \\
%     &- \tfrac{\alpha^2 \mu n_g}{2} \left((n_g-1)\beta_1^{n_c} + \beta_2^{n_c} + (n_g - 1) \delta_1(n_c, n_g)\right)\left(\beta_4^{n_c}L(2 + \alpha L) + \alpha L^2 (n_g - 1)\delta_2(n_c, n_g)\right) \\
%     &-\alpha^2 L^2 (n_g - 1)\delta_1(n_c, n_g) \left(\alpha L n_g + \alpha (n_g - 1)\right) \left(\tfrac{1 - \beta_3^{n_c}}{4}\right) \\
%     &- \alpha^2 L^2 (n_g - 1)\delta_1(n_c, n_g)   \left(\alpha(n_g - 1)\right)\left(3\beta_4^{n_c}L + L (n_g - 1)\delta_2(n_c, n_g)\right) \\
%     & - \alpha^2L^2( \beta_4^{n_c} + (n_g - 1)\delta_2(n_c, n_g)) \left(\alpha L n_g + \alpha (n_g - 1)\right) \left((n_g-1)\beta_1^{n_c} + \beta_2^{n_c} +  (n_g - 1) \delta_1(n_c, n_g)\right) \\
%     & -\alpha L^2 (\beta_4^{n_c} + (n_g - 1)\delta_2(n_c, n_g))  \left(\alpha^2 L n_g (n_g - 1) \right)\left(\tfrac{1 - \beta_1^{n_c}}{4}\right) \\
%     \\
%     &= \alpha(- b_1 \alpha^2 - b_2 \alpha + b_3) = - \alpha(\alpha - \alpha_l)(\alpha - \alpha_u)
% \end{align*}
where the inequality is due to $\alpha L n_g \leq 1$ and thus $\alpha L \leq 1$ as $n_g \geq 1$, and 
\begin{align*}
\alpha_l = \tfrac{-b_2 - \sqrt{b_2^2 + 4b_1 b_3}}{2b_1} \quad \text{and} \quad 
\alpha_u = \tfrac{-b_2 + \sqrt{b_2^2 + 4b_1 b_3}}{2b_1}. 
\end{align*}
Observe that $\alpha_l < 0 < \alpha_u$ since $b_1, b_2, b_3 \geq 0$. From \eqref{eq : alpha mult grads gen}, we have $0<\alpha < \alpha_u$. Therefore, $\det(I_3 - \Tilde{B}({n_c, n_g})) > 0$, which combined with the fact that the diagonal elements of the matrix are less than 1, implies $\rho(B(n_c, n_g)) \leq \rho(\Tilde{B}(n_c, n_g)) < 1$.

% Now, we observe that,
% \begin{align*}
%     \|r_{k}\|_2 &\leq \|B({n_c, n_g})^k\|_2\|r_0\|_2,
% \end{align*}
% and using \cite[Theorem 5.6.12]{horn2012matrix}, we can bound the matrix norm with a spectral radius using some sequence $\epsilon_k \geq 0$ with $\lim_{k \rightarrow \infty} \epsilon_k = 0$.
Finally, we bound the norm of error vector $\|r_k\|_2$ by telescoping $r_{i+1} \leq B(n_c, n_g) r_{i}$ from $i = 0$ to $k-1$ and triangle inequality as
\begin{align*}
    \|r_{k}\|_2 &\leq \|B(n_c, n_g)^k\|_2\|r_0\|_2.
\end{align*}
From \cite[Corollary 5.6.13]{horn2012matrix}, we can bound $\|B(n_c, n_g)^k\|_2 \leq C_{\epsilon}(\rho(B(n_c, n_g)) + \epsilon)^k$ where $\epsilon > 0$ and $C_{\epsilon}$ is a positive constant depending on $B(n_c, n_g)$ and $\epsilon$.
\eproof

Similar to \cref{th. general g=1 rate bound}, the only constraint \cref{th. alpha bound g > 1} imposes on the system is $\beta_1, \beta_3 < 1$. This implies the communication matrices $\Wmbf_1$ and $\Wmbf_3$ must represent connected networks (not necessarily the same network) even when multiple communication and multiple computation steps are performed. \cref{th. alpha bound g > 1} does not impose any restrictions on the relation among $\Wmbf_1$, $\Wmbf_2$, $\Wmbf_3$ and $\Wmbf_4$. Thus, it allows for more flexibility than the structures considered in the literature even when multiple communication and multiple computation steps are performed. %Variables can be communicated along different connections within the network while the agents perform multiple computation steps using only local information. 
% One drawback of gradient tracking methods is the requirement of communication of 2 vectors instead of one requiring higher communication bandwidth. This limitation can be overcome using our general structure by limiting communication of variables on certain edges as long as \cref{th. alpha bound g > 1} conditions are met. 
\cref{th. alpha bound g > 1} uses a relaxation of the original matrix $B(n_c, n_g)$ to provide a more pessimistic step size condition than required. But observe, when $n_g = 1$, \eqref{eq : alpha mult grads gen} recovers the $\mathcal{O}(L^{-1}\kappa^{-0.5})$ step size condition of \cref{th. general g=1 step cond}, suggesting it might not be very pessimistic.

Based on \cref{th. alpha bound g > 1}, the step size conditions for methods described in \cref{tab: Algorithm Def} can be derived. We omit these conditions as they are complex and do not offer any additional insights. %These conditions have been omitted as they are highly complex and do not offer any \asb{additional} insights. 
We also omit the counterpart to \cref{th. general g=1 rate bound} as the matrix $B(n_c, n_g)$ is now a dense matrix, thus any such bounds are again highly complex and do not offer strong insights into the effects of communication and computation on the convergence rate. If $B(n_c, n_g)$ is a reducible matrix, the analysis for the progression of $r_k$ can be further simplified from \cref{lem:lyapunov g > 1}. The analysis for this case is presented in \cref{sec.full graph res} with the example of \texttt{GTA-2} and \texttt{GTA-3} when $\Wmbf = \frac{1_n1_n^T}{n}$, i.e., $\beta= 0$. 



%%%%%%%%%%%%%%%%%%%%%%%%%%%%%%%%%%%%%%%%%%%%%
% Fully connected Graph Analysis
%%%%%%%%%%%%%%%%%%%%%%%%%%%%%%%%%%%%%%%%%%%%%

\subsection{Fully connected network} \label{sec.full graph res}

In this section, we analyse the methods defined in \cref{tab: Algorithm Def} under a fully connected network. While showing linear convergence of \texttt{GTA} in \cref{th. alpha bound g > 1}, we assume $B(n_c, n_g)$ is an irreducible matrix. When the network is fully connected, i.e., $\Wmbf = \tfrac{1_n1_n^T}{n}$ and $\beta = 0$, the assumption doesn't hold for \texttt{GTA-2} and \texttt{GTA-3} as the matrices  $B_2(n_c, n_g)$ and $B_3(n_c, n_g)$ defined by \cref{col. B special cases} are reducible. For \texttt{GTA-1}, such an issue does not arise as $A_1(n_c, n_g)$ defined in \cref{col. B special cases} is irreducible for all $\beta \in [0,1]$. Thus, we now present sufficient conditions for linear rate of convergence and the convergence rate for \texttt{GTA-2} and \texttt{GTA-3} under a fully connected network.

% \asb{Discuss GTA-1}

\btheorem \label{th. fully connected}
Suppose \cref{asum.convex and smooth} holds, $\Wmbf = \frac{1_n1_n^T}{n}$ and a finite number of computation steps are performed each outer iteration of \texttt{GTA-3} defined in \cref{tab: Algorithm Def} (i.e., $1 \leq n_g < \infty$). If $\alpha < \min \left\{\frac{\mu}{(2L^2 + \mu^2)(n_g-1)}, \frac{1}{L n_g}\right\}$, then for all $k \geq 0$,
\begin{align*}
    \|\xbar_{k+1, 1} - x^*\|_2 & \leq \left(
    (1 - \alpha \mu)^{n_g} + \alpha^2 L^2 n_g(n_g - 1) \right)  \|\xbar_{k, 1} - x^*\|_2.
\end{align*}
%where $(1 - \alpha \mu)^{n_g} + \alpha^2 L^2 n_g(n_g - 1) < 1$. 
Moreover, suppose the number of computation steps performed each outer iteration of \texttt{GTA-2} and \texttt{GTA-3} defined in \cref{tab: Algorithm Def} is set to one (i.e., $n_g =1$). If $\alpha \leq \frac{1}{L}$, then for both the methods, for all $k \geq 0$,
\begin{align*}
    \|\xbar_{k+1, 1} - x^*\|_2 & \leq 
    (1 - \alpha \mu) \|\xbar_{k, 1} - x^*\|_2.
\end{align*}
% where $1 - \alpha\mu < 1$.
\etheorem
\bproof
When we substitute $\beta = 0$ in \cref{col. B special cases} as $\alpha < \frac{1}{n_g L}$, the matrices $B_2(n_c, n_g)$ and $B_3(n_c, n_g)$ now have rows of zeros that make them reducible. Thus, we reduce these matrices by ignoring the error terms corresponding to the row of zeros. This yields the following systems for the progression of errors in these methods,
\begin{align}
    &\mbox{\texttt{GTA-2: }}
     \Tilde{r}_{k+1} \label{eq : g > 1 reduced 2}    
    \leq \begin{bmatrix}
    (1 - \alpha \mu)^{n_g} + \alpha^2 L^2 n_g(n_g - 1) & \frac{\alpha^2 L n_g(n_g - 1)}{\sqrt{n}}\\
    \sqrt{n}\alpha L^2 \Tilde{\delta}(n_c, n_g) & \alpha L\Tilde{\delta}(n_c, n_g)\\
    \end{bmatrix} \Tilde{r}_{k}, \\
    &\mbox{\texttt{GTA-3: }}
         \|\xbar_{k+1, 1} - x^*\|_2  \leq \left(
    (1 - \alpha \mu)^{n_g} + \alpha^2 L^2 n_g(n_g - 1) \right)
        \|\xbar_{k, 1} - x^*\|_2, \label{eq : g > 1 reduced 3} 
\end{align}
where $\Tilde{\delta}(n_c, n_g) = 1 + 2(n_g - 1)\left(2 + \tfrac{1}{n_g}\right)$ and $\Tilde{r}_{k} = \begin{bmatrix}
        \|\xbar_{k, 1} - x^*\|_2\\
        \|\ymbf_{k, 1} - \Bar{\ymbf}_{k, 1}\|_2\\
    \end{bmatrix}$. \\
    By $\alpha < \frac{\mu}{(2L^2 + \mu^2)(n_g-1)}$ and $(1-\alpha\mu)^{n_g} \leq 1 - \alpha\mu n_g + \alpha^2\mu^2 \tfrac{n_g(n_g-1)}{2}$ from \cref{th. alpha bound g > 1},
\begin{align*}
    (1 - \alpha \mu)^{n_g} + \alpha^2 L^2 n_g(n_g - 1) \leq 1 - \alpha \mu n_g + \alpha^2 \left(L^2 + \tfrac{\mu^2}{2}\right) n_g(n_g - 1) < 1,
\end{align*}
and thus the result for \texttt{GTA-3} follows. When the number of computation steps performed each outer iteration is set to one, i.e., $n_g = 1$, 
the result for \texttt{GTA-3} follows by substituting $n_g=1$ in \eqref{eq : g > 1 reduced 3}, where $1 - \alpha\mu < 1$ as $\alpha \leq \frac{1}{L}$. Substituting $n_g=1$ in \eqref{eq : g > 1 reduced 2} for \texttt{GTA-2} yields,  $\Tilde{r}_{k+1} \leq \begin{bmatrix}
    1 - \alpha \mu & 0\\
    \sqrt{n}\alpha L^2 & \alpha L\\
    \end{bmatrix}
        \Tilde{r}_{k}$, 
% \begin{align*}
%     &\mbox{\texttt{GTA-2:}}
%     &\Tilde{r}_{k+1} &\leq \begin{bmatrix}
%     1 - \alpha \mu & 0\\
%     \sqrt{n}\alpha L^2 & \alpha L\\
%     \end{bmatrix}
%         \Tilde{r}_{k}, %\label{eq : g = 1 reduced 2} 
% \end{align*}
where the bound on optimization error is independent of the consensus error in $\ymbf_{k, 1}$.
Thus, we obtain $\|\xbar_{k+1, 1} - x^*\|_2 \leq \left(1 - \alpha \mu \right) \|\xbar_{k, 1} - x^*\|_2$ for \texttt{GTA-2}.
\eproof

By \cref{th. fully connected}  if the network is fully connected and a single computation step is performed, i.e., $n_g = 1$, \texttt{GTA-2} and \texttt{GTA-3} display gradient descent performance. For \texttt{GTA-2}, when the network is fully connected and $n_g > 1$, the convergence rate can be expressed as the spectral radius of the $2\times2$ matrix in \eqref{eq : g > 1 reduced 2}. %It is computable but results in a convoluted and non interpretable expression, thus we omit it.

 
%%%%%%%%%%%%%%%%%%%%%%%%%%%%
% Numerical Experiments
%%%%%%%%%%%%%%%%%%%%%%%%%%%%


\section{Numerical Experiments}\label{sec.num_exp}

\setcounter{footnote}{1} 
In this section, we 
illustrate the empirical performance of the methods defined in \cref{tab: Algorithm Def} using Python implementations\footnote{Our code will be made publicly available upon publication of the manuscript. Github repository: \url{https://github.com/Shagun-G/Gradient-Tracking-Algorithmic-Framework}. Moreover, additional extensive numerical results can be found in the same repository.}. 
The aim of this section is to show, over multiple problems, that different communication strategies and the balance between communication and computation steps 
can substantially effect the algorithm's performance. Specifically, we establish the relative performance of the methods defined in \cref{tab: Algorithm Def} and illustrate the benefits of the flexibility in terms of communication and computation steps.

We present results on two problems: $(1)$ a synthetic strongly convex quadratic problem (\cref{sec : quads}); and, $(2)$ binary classification logistic regression problems over the mushroom and australian datasets \cite{Dua:2019} (\cref{sec : logistic}). We 
investigated 
two  
network structures (different mixing matrix $\Wmbf$) with $n = 16$ nodes: %. 
$(1)$ a connected cyclic network ($\beta = 0.992$) where all nodes have two neighbours; and, $(2)$ a connected star network ($\beta = 0.95$) where all nodes are connected to a single central node. %All chosen 
Both networks have low connectivity (i.e., high $\beta$). We should note that the performance of \cref{alg : Deterministic} with multiple communication steps is equivalent to the performance over a network with higher connectivity (i.e., lower $\beta$).

The methods defined in \cref{tab: Algorithm Def} 
are denoted 
as \texttt{GTA}$-i(n_c, n_g)$, $i = 1, 2, 3$, where $n_c$ and $n_g$ are the number of communication and computation steps, respectively. We %test them over 
tested 5 values of $n_c$ and $n_g$ for each of the methods; %. That is,  
$n_c \in \{1, 5, 10, 50, 100\}$ and $n_g \in \{ 1, 5, 20, 50, 100\}$. We compared the  performance of %with 
popular gradient tracking methods, which are special cases of our generalized framework. %, as shown in \cref{tab: Algorithm Def}}.  
The step sizes were tuned over the set $\{2^{-t} | t = 0, 1, 2, .., 20\}$ for all algorithms and problems, and the initial iterates for all algorithms, problems and nodes were set to the zero vector (i.e., $\xmbf_k = \mathbf{0}$). 
The performance of the methods was measured %using
in terms of the optimization error ($\|\bar{x}_k - x^*\|_2$) and the consensus error ($\|\xmbf_k - \xbb_k\|_2$). We do not report %track 
the consensus error in the auxiliary variable $\ymbf_k$ ($\|\ymbf_k - \ybb_k\|_2$) as this measure does not provide any significant additional insights about the performance of the algorithms.  
The optimal solution $x^*$ for quadratic problem was obtained analytically and for the logistic regression problems was obtained by
running gradient descent in the centralized setting to high accuracy, i.e., $\|\nabla f(x^*)\|_2 \leq 10^{-12}$. 

\subsection{Quadratic Problems}\label{sec : quads}

We first consider quadratic problems
\begin{align*} 
    f(x) = \frac{1}{n} \sum_{i=1}^n \frac{1}{2}x^TQ_ix + b_i^Tx,
\end{align*}
where $Q_i \in \mathbb{R}^{10 \times 10}$, $Q_i \succ 0$ and $b_i \in \mathbb{R}^{10}$ is the local information at each node $i \in \{1, 2, .., n\}$, and $n=16$. Each local problem is strongly convex and was generated using the procedure described in \cite{mokhtari2016network}, with global condition number $\kappa \approx 10^4$. 

\cref{fig : Quadratic cyclic,fig : Quadratic Star} show the performance of \texttt{GTA-1}, \texttt{GTA-2} and \texttt{GTA-3} over a cyclic network and a star network, respectively. Our first observation, from the iteration plots in both the figures, is that the optimization error and consensus error converge at a linear rate for all methods, matching the theoretical results of \cref{sec.theory}. Moreover, improvements in the rates of convergence of all methods are observed as a result of the flexibility in terms of the number of communication and computation steps. Specifically, the consensus error is improved (and on par optimization error) when multiple communication steps with single computation step are performed (see \texttt{GTA-i}($1$, 1) vs. \texttt{GTA-i}($n_c$, 1) lines), and the optimization error is improved (and on par consensus error) when multiple computation steps with same number of communication steps are performed (see \texttt{GTA-i}($n_c$, 1) vs. \texttt{GTA-i}($n_c$, $n_g$) lines). 
These observations match the theory presented in Section~\ref{sec.mult grads}. 
That being said, %However 
these improvements come at a higher cost in terms of total communication or computation steps, respectively, and an optimal choice of $(n_c, n_g)$ depends on the exact cost structure that combines the complexity of both these steps; see e.g., \cite{berahas2018balancing}. Finally, we also observe that \texttt{GTA-2} and \texttt{GTA-3} outperform \texttt{GTA-1} in terms of optimization error and achieve similar consensus error. The performance of \texttt{GTA-2} and \texttt{GTA-3} is very similar for this problem, we suspect the reason for this behavior is due to the large $\beta$ and the high condition number ($\kappa \approx 10^4$) that dominate the rate constant; see \cref{col. g=1 rate bound}. 

\begin{figure}[h]
\centering
\includegraphics[width=\textwidth]{Figures/Quad_cyclic.pdf}  
\caption{Optimization Error ($\|\xbar_k - x^*\|_2$) and Consensus Error ($\|\xmbf_k - \xbb_k\|_2$) of \texttt{GTA-1}, \texttt{GTA-2} and \texttt{GTA-3} with respect to number of iterations, communications and gradient evaluations for a synthetic quadratic problem ($n = 16$, $d = 10$, $\kappa = 10^4$) over a cyclic network ($\beta =  0.992$).}
\label{fig : Quadratic cyclic}
\end{figure}

\begin{figure}[h]
\centering
\includegraphics[width=\textwidth]{Figures/Quad_star.pdf}    
\caption{Optimization Error ($\|\xbar_k - x^*\|_2$) and Consensus Error ($\|\xmbf_k - \xbb_k\|_2$) of \texttt{GTA-1}, \texttt{GTA-2} and \texttt{GTA-3} with respect to number of iterations, communications and gradient evaluations for a synthetic quadratic problem ($n = 16$, $d = 10$, $\kappa = 10^4$) over star network($\beta =  0.95$).}
\label{fig : Quadratic Star}
\end{figure}

\subsection{Binary Classification Logistic Regression}\label{sec : logistic}
Next, we consider $\ell_2$-regularized binary classification logistic regression problems of the form
\begin{align*} %\label{eq : logistic problem}
    f(x) &= \frac{1}{n} \sum_{i=1}^n \frac{1}{n_i}\log(1 + e^{-b_i^TA_ix}) + \frac{1}{n_i}\|x\|_2^2, 
\end{align*}
where each node $i \in \{1, 2, .., n\}$ has a portion of data samples $A_i \in \mathbb{R}^{n_i \times d}$ and corresponding labels $b_i \in \{0, 1\}^{n_i}$. Experiments were performed over the mushroom dataset ($n = 16$, $d = 117$, $\sum_{i=1}^n n_i = 8124$) and the australian dataset ($n = 16$, $d = 41$, $\sum_{i=1}^n n_i = 690$) \cite{Dua:2019}. 

\cref{fig : Mushroom Cyclic,fig : Austrailian Star} show the performance of \texttt{GTA-1}, \texttt{GTA-2} and \texttt{GTA-3} over a cyclic network ($\beta =  0.992$) for the mushroom dataset and a star network for the australian dataset ($\beta =  0.95$), respectively. Similar observations  to those made for the quadratic problem with respect to the effect of performing multiple communication and computation steps can also be made for these problems. 
Additionally, we observe that \texttt{GTA-3} outperforms \texttt{GTA-2} on these problems. 
We should note that although \texttt{GTA-3} performs the best within these experiments, it also brings certain implementation constraints; see \cref{sec.methods}.

\begin{figure}[]
\centering
\includegraphics[width=\textwidth]{Figures/Mushroom_cyclic.pdf} 
\caption{Optimization Error ($\|\xbar_k - x^*\|_2$) and Consensus Error ($\|\xmbf_k - \xbb_k\|_2$) of \texttt{GTA-1}, \texttt{GTA-2} and \texttt{GTA-3} with respect to number of iterations, communications and gradient evaluations for binary logistic regression on Mushroom dataset ($n = 16$, $d = 117$, $\sum_{i=1}^n n_i = 8124$) over cyclic network ($\beta =  0.992$).}
\label{fig : Mushroom Cyclic}
\end{figure}

\begin{figure}[]
\centering
\includegraphics[width=\textwidth]{Figures/Austrailian_star.pdf}  
\caption{Optimization Error ($\|\xbar_k - x^*\|_2$) and Consensus Error ($\|\xmbf_k - \xbb_k\|_2$) of \texttt{GTA-1}, \texttt{GTA-2} and \texttt{GTA-3} with respect to number of iterations, communications and gradient evaluations for binary logistic regression on Australian dataset ($n = 16$, $d = 41$, $\sum_{i=1}^n n_i = 690$) over star network ($\beta = 0.95$).
}
\label{fig : Austrailian Star}
\end{figure}
  
%%%%%%%%%%%%%%%%%%%%%%%%%%%%
% Conclusion
%%%%%%%%%%%%%%%%%%%%%%%%%%%%


\section{Final Remarks}\label{sec.conc}

In this paper, we have proposed a framework that unifies and generalizes communication strategies in gradient tracking methods with flexibility in the number of communication and computation steps performed at every iteration. We have established convergence guarantees for the proposed gradient tracking framework. Specifically, we have shown linear convergence for the general framework and the special cases of gradient tracking methods. Moreover, we have shown the positive influence of performing multiple communication steps at every iteration on the convergence rate and provide results that allow for the direct comparison of popular gradient tracking methods. Our experiments on quadratic and logistic regression problems illustrate the effects of different communication strategies and the benefits of the flexibility in terms of iterations and number of communication and computation steps. The advantages of the proposed framework can be further realized when the actual cost, i.e., a combination of the complexity of both communication and computation steps that is application specific, is considered. 
Finally, the algorithmic framework can be extended to other interesting settings such as nonconvex problems, stochastic local information, asynchronous updates, and higher-order approaches.

\bibliography{biblio}
\bibliographystyle{icml2023}


%%%%%%%%%%%%%%%%%%%%%%%%%%%%%%%%%%%%%%%%%%%%%%%%%%%%%%%%%%%%%%%%%%%%%%%%%%%%%%%
%%%%%%%%%%%%%%%%%%%%%%%%%%%%%%%%%%%%%%%%%%%%%%%%%%%%%%%%%%%%%%%%%%%%%%%%%%%%%%%
% APPENDIX
%%%%%%%%%%%%%%%%%%%%%%%%%%%%%%%%%%%%%%%%%%%%%%%%%%%%%%%%%%%%%%%%%%%%%%%%%%%%%%%
%%%%%%%%%%%%%%%%%%%%%%%%%%%%%%%%%%%%%%%%%%%%%%%%%%%%%%%%%%%%%%%%%%%%%%%%%%%%%%%
\newpage
\appendix
\onecolumn
% \section{You \emph{can} have an appendix here.}
\section{Appendix for Proofs}

\paragraph{Proof of Theorem \ref{thm:main}.}

\begin{proof}
\label{proof:main}
Our proof has two steps. In Step 1, we will show that SimCLR is equivalent to minimizing the cross entropy loss defined in Eqn.~(\ref{eqn:cross-entropy}). 
In Step 2, we will show  that minimizing the cross-entropy loss 
is equivalent to spectral clustering on $\bfpi$. 
Combining the two steps together, we have proved our theorem. 

\textbf{Step 1: } SimCLR is equivalent to minimizing the cross entropy loss.

The cross-entropy loss takes expectation over 
$\bfW_\bfX\sim \mathbb{P}(\cdot ; \bfpi)$, 
which means $\bfW_\bfX$ has exactly one non-zero entry in each row $i$. By Lemma~\ref{lem:multinomial}, we know every row $i$ of $\bfW_\bfX$ is independent of other rows. Moreover, 
$\bfW_{\bfX,i}\sim \mathcal{M}(1, \bfpi_i/\sum_j \bfpi_{i,j})=\mathcal{M}(1, \bfpi_i)$, because $\bfpi_i$ itself is a probability distribution.
Similarly, we know $\bfW_\bfZ$ also has the row-independent property by sampling over $\mathbb{P}(\cdot;\bfK_\bfZ)$.
Therefore, by Lemma~\ref{lem:cross_split}, we know Eqn.~(\ref{eqn:cross-entropy}) is equivalent to:
\[
 -\sum_{i=1}^n \mathbb{E}_{\bfW_{\bfX,i}}[\log \mathbb{P}(\bfW_{\bfZ,i}=\bfW_{\bfX,i};\bfK_\bfZ)],
\]

This expression takes expectation over $\bfW_{\bfX,i}$ for the given row $i$. Notice that 
$\bfW_{\bfX,i}$ has exactly one non-zero entry, which equals $1$ (same for $\bfW_{\bfZ,i}$). 
As a result
we expand the above expression to be:
\begin{equation}
 -\sum_{i=1}^n \sum_{j\neq i} \Pr(\bfW_{\bfX,i,j}=1)\log \Pr(\bfW_{\bfZ,i,j}=1).
\label{eqn:detailed-expansion}    
\end{equation}


By Lemma~\ref{lem:multinomial}, $\Pr(\bfW_{\bfZ,i,j}=1)=\bfK_{\bfZ,i,j}/\|\bfK_{\bfZ,i}\|_1$ for $j\neq i$. Recall that $\bfK_\bfZ=(k(\bfZ_i-\bfZ_j))_{(i,j)\in[n]^2}$, which means 
$\bfK_{\bfZ,i,j}/\|\bfK_{\bfZ,i}\|_1=\frac{\exp(-\|\bfZ_i-\bfZ_j\|^2/{2\tau})}{\sum_{k\neq i}
\exp(-\|\bfZ_i-\bfZ_k\|^2/{2\tau})
}$ for $j\neq i$, when $k$ is the Gaussian kernel with variance $\tau$. 

Notice that $\bfZ_i=f(\bfX_i)$, so we know
\begin{equation}
-\log \Pr(\bfW_{\bfZ,i,j}=1)=
-\log \frac{\exp(-\|f(\bfX_i)-f(\bfX_j)\|^2/{2\tau})}{\sum_{k\neq i}
\exp(-\|f(\bfX_i)-f(\bfX_k)\|^2/{2\tau}),
}
\label{eqn:infonce-equivalence}    
\end{equation}


The right hand side is exactly the InfoNCE loss defined in Eqn.~(\ref{eqn:infonce}).
Inserting Eqn.~(\ref{eqn:infonce-equivalence}) into Eqn.~(\ref{eqn:detailed-expansion}), we get the SimCLR algorithm, which first samples augmentation pairs $(i,j)$ with $\Pr(\bfW_{\bfX,i,j}=1)$ for each row $i$, and then optimize the InfoNCE loss. 

\textbf{Step 2: } minimizing the cross entropy loss 
is equivalent to spectral clustering on $\bfpi$.


By Lemma~\ref{lem:convert_to_spectral}, we may further convert the loss to 
\begin{equation}
\label{eqn:main-theorem-repul-attr}
\min_{\bfZ}
-\sum_{(i,j)\in [n]^2} \mathbf{P}_{i,j}
\log k (\bfZ_i-\bfZ_j)+\log \mathbf{R}(\bfZ).
\end{equation}
Since $k$ is the Gaussian kernel, this reduces to \[
\min_\bfZ \mathrm{tr}(\bfZ^\top \mathbf{L}(\bfpi) \bfZ)
+\log \mathbf{R}(\bfZ),
\]

where we use the fact that $\mathbb{E}_{\bfW_\bfX\sim \mathbb{P}(\cdot; \bfpi)}[\mathbf{L}(\bfW_\bfX)]
=\mathbf{L}(\bfpi)
$, because the Laplacian operator is linear and $
\mathbb{E}_{\bfW_\bfX\sim \mathbb{P}(\cdot; \bfpi)}(\bfW_\bfX)=\bfpi
$.
\end{proof}

\paragraph{Proof of Theorem \ref{thm:clip}.}
\begin{proof}
Since $\bfW_\bfX\sim \mathbb{P}(\cdot;\bfpi_{\mathbf{A}, \mathbf{B}})$, we know 
$\bfW_\bfX$ has exactly one non-zero entry in each row, denoting the pair that got sampled. 
A notable difference compared to the previous proof is we now have $n_\mathcal{A}+n_\mathcal{B}$ objects in our graph. CLIP deals with this by taking a mini-batch of size $2N$, 
such that $n_\mathcal{A}=n_\mathcal{B}=N$, and adding the $2N$ InfoNCE losses together. We label the objects in $\mathcal{A}$ as $[n_\mathcal{A}]$, and the objects in $\mathcal{B}$ as $\{n_\mathcal{A}+1, \cdots, n_\mathcal{A}+n_\mathcal{B}\}$. 

Notice that $\bfpi_{\mathbf{A}, \mathbf{B}}$ is a bipartite graph, so the edges of objects in $\mathcal{A}$ will only connect to object in $\mathcal{B}$ and vice versa. We can define the similarity matrix in $\cZ$ as $\bfK_\bfZ$, 
where $\bfK_\bfZ(i, j+n_\mathcal{A})=\bfK_\bfZ(j+n_\mathcal{A},i)= k(\bfZ_i-\bfZ_j)$ for $i\in [n_\mathcal{A}], j\in [n_\mathcal{B}]$, and otherwise we set $\bfK_\bfZ(i,j)=0$. 
The rest is same as the previous proof. 
\end{proof}

\paragraph{Proof of Theorem \ref{thm:exponential}.}

\begin{proof}
\label{proof:exponential}
Since the objective function consists of a linear term combined with an entropy regularization, which is a strongly concave function, the maximization problem is a convex optimization problem. Owing to the implicit constraints provided by the entropy function, the problem is equivalent to having only the equality constraint. We then introduce the Lagrangian multiplier $\lambda$ and obtain the following relaxed problem:

$$
\widetilde{E}(\boldsymbol{\alpha})=\psi_{1}-\sum_{i=1}^n \alpha_{i} \psi_{i}+\tau \sum_{i=1}^n \alpha_{i}\log \alpha_{i}+\lambda\left(\boldsymbol{\alpha}^{\top} \mathbf{1}_n-1\right).
$$

As the relaxed problem is unconstrained, taking the derivative with respect to $\alpha_{i}$ yields

$$
\frac{\partial \widetilde{E}(\boldsymbol{\alpha})}{\partial \alpha_{i}}=-\psi_{i}+\tau\left(\log \alpha_{i}+\alpha_{i} \frac{1}{\alpha_{i}}\right)+\lambda=0.
$$

Solving the above equation implies that $\alpha_{i}$ takes the form
$
\alpha_{i}=\exp \left(\frac{1}{\tau} \psi_{i}\right) \exp \left(\frac{-\lambda}{\tau}-1\right).
$ Since $\alpha_{i}$ lies on the probability simplex, the optimal $\alpha_{i}$ is explicitly given by
$
\alpha^{*}_{i}=\frac{\exp \left(\frac{1}{\tau} \psi_{i}\right)}{\sum_{i^{\prime}=1}^n \exp \left(\frac{1}{\tau} \psi_{i^{\prime}}\right)} .
$ Substituting the optimal point into the objective function, we obtain
$$
\begin{aligned}
E\left(\boldsymbol{\alpha}^*\right)  &=\psi_1-\sum_{i=1}^n \frac{\exp \left(\frac{1}{\tau} \psi_{i}\right)}{\sum_{i^{\prime}=1}^n \exp \left(\frac{1}{\tau} \psi_{i^{\prime}}\right)} \psi_{i}+\tau \sum_{i=1}^n \frac{\exp \left(\frac{1}{\tau} \psi_{i}\right)}{\sum_{i^{\prime}=1}^n \exp \left(\frac{1}{\tau} \psi_{i^{\prime}}\right)}\log \frac{\exp \left(\frac{1}{\tau} \psi_{i}\right)}{\sum_{i^{\prime}=1}^n \exp \left(\frac{1}{\tau} \psi_{i^{\prime}}\right)} \\
& =\psi_1 - \tau \log \left(\sum_{i=1}^n \exp \left(\frac{1}{\tau} \psi_{i}\right)\right).
\end{aligned}
$$
Thus, the Lagrangian dual function is given by
\begin{equation*}
-E\left(\boldsymbol{\alpha}^*\right)= -\tau \log \frac{\exp \left(\frac{1}{\tau} \psi_{1}\right)}{\sum_{i=1}^n \exp \left(\frac{1}{\tau} \psi_{i}\right)}.\qedhere
\end{equation*}
\end{proof}



\section{More on Experiments} \label{section: experiment_details}

\paragraph{CIFAR-10 and CIFAR-100} CIFAR-10 ~\citep{krizhevsky2009learning} and CIFAR-100 ~\citep{krizhevsky2009learning} are well-known classic image classification datasets. Both CIFAR-10 and CIFAR-100 contain a total of 60k $32 \times 32$ labeled images of different classes, with 50k for training and 10k for testing. CIFAR-10 is similar to CIFAR-100, except there are 10 different classes in CIFAR-10 and 100 classes in CIFAR-100.

\paragraph{TinyImageNet} TinyImageNet ~\citep{le2015tiny} is a subset of ImageNet ~\citep{deng2009imagenet}. There are 200 different object classes in TinyImageNet, with 500 training images, 50 validation images, and 50 test images for each class. All the images in TinyImageNet are colored and labeled with a size of $64 \times 64$.

\textbf{Pseudo-code.} Algorithm \ref{alg:Training Procedure} presents the pseudo-code for our empirical training procedure.

\begin{algorithm}[!htbp]
\caption{Training Procedure}
\label{alg:Training Procedure}
\begin{algorithmic}[1]
\REQUIRE trainable encoder network $f$, batch size $N$, augmentation strategy \textit{aug}, loss function $L$ with hyperparameters \textit{args}
\FOR {sampled minibatch ${x_i}_{i=1}^N$}
\FORALL{$i \in { 1, ..., N }$}
\STATE draw two augmentations $t_i = \textit{aug}\left(x_i\right) $, $t_i' = \textit{aug}\left(x_i\right) $
\STATE $z_i = f\left(t_i\right)$, $z_i' = f\left(t_i'\right)$
\ENDFOR
\STATE compute loss $\mathcal{L} = L(N, z, z', \textit{args})$
\STATE update encoder network $f$ to minimize $\mathcal{L}$
\ENDFOR
\STATE \textbf{Return} encoder network $f$
\end{algorithmic}
\end{algorithm}

We also provide the pseudo-code for our core loss function used in the training procedure in Algorithm \ref{alg:Core loss}. The pseudo-code is almost identical to SimCLR's loss function, with the exception of an extra parameter $\gamma$.

\begin{algorithm}[!htbp]
\caption{Core loss function $\mathcal{C}$}
\label{alg:Core loss}
\begin{algorithmic}[1]
\REQUIRE batch size $N$, two encoded minibatches $z_1, z_2$, $\gamma$, temperature $\tau$
\STATE $z = \textit{concat}\left(z_1, z_2\right)$
\FOR {$i \in {1, ..., 2N }, j \in {1, ..., 2N}$ }
\STATE $s_{i,j} = \Vert z_i - z_j \Vert_2^{\gamma}$
\ENDFOR
\STATE \textbf{define} $l(i, j)$ \textbf{as} $l(i, j) = - \log \frac{exp\left(s_{i,j}/\tau \right)}{\sum_{k=1}^{2N} \mathbf{1}{[k \ne i]} exp\left(s{i, j} / \tau \right)} $
\STATE \textbf{Return} $\frac{1}{2N} \sum_{k=1}^N\left[l(i, i+N) + l(i+N, i)\right]$
\end{algorithmic}
\end{algorithm}

Utilizing the core loss function $\mathcal{C}$, we can define all kernel loss functions used in our experiments in Table \ref{table: loss definition}. For all $z_i \in z$ with even dimensions $n$, we define $z_{L_i} = z_i\left[0:n/2\right]$ and $z_{R_i} = z_i\left[n/2:n\right]$.

\begin{table}[ht]
\centering
\begin{tabular}{{@{}l|l@{}}}
Kernel  &  Loss function \\ \midrule
Laplacian & $\mathcal{C}\left(N, z, z', \gamma=1, \tau\right)$\\ \midrule
Sum       & $\lambda * \mathcal{C}\left(N, z, z', \gamma=1, \tau_1\right) + (1-\lambda) * \mathcal{C}\left(N, z, z', \gamma=2, \tau_2\right)$  \\ \midrule
Concatenation Sum&$\lambda * \mathcal{C}\left(N, z_L, z'_L, \gamma=1, \tau_1\right) + (1-\lambda) * \mathcal{C}\left(N, z_R, z'_R, \gamma=2, \tau_2\right)$\\ \midrule
$\gamma = 0.5$ & $\mathcal{C}\left(N, z, z', \gamma=0.5, \tau\right)$          \\ 

\end{tabular}

\caption{Definition of kernel loss functions in our experiments}
\label {table: loss definition}
\end{table}

\textbf{Baselines.} We reproduce the SimCLR algorithm using PyTorch Lightning~\citep{PytorchLightning}.

\textbf{Encoder details.}
The encoder $f$ consists of a backbone network and a projection network. We employ ResNet50~\citep{ResNet} as the backbone and a 2-layer MLP (connected by a batch normalization~\citep{ioffe2015batch} layer and a ReLU \cite{nair2010rectified} layer) with hidden dimensions 2048 and output dimensions 128 (or 256 in the concatenation kernel case).

\textbf{Encoder hyperparameter tuning.}
For each encoder training case, we randomly sample 500 hyperparameter groups (sample details are shown in Table \ref{table: Hyperparameter sample}) and train these samples simultaneously using Ray Tune ~\citep{RayTune}, with the ASHA scheduler~\citep{li2018massively}. Ultimately, the hyperparameter group that maximizes the online validation accuracy (integrated in PyTorch Lightning) within 5000 validation steps is chosen for the given encoder training case.

\begin{table}[ht]
\centering

\begin{tabular}{@{}l|l|l@{}}
\midrule
Hyperparameter  & Sample Range & Sample Strategy \\ \midrule
start learning rate & $\left[10^{-2}, 10\right]$ & log uniform \\ \midrule
$\lambda$       & $\left[0, 1\right]$ & uniform \\ \midrule
$\tau$, $\tau_1$, $\tau_2$ & $\left[0, 1\right]$ & log uniform \\ \midrule
\end{tabular}

\caption{Hyperparameters sample strategy}
\label {table: Hyperparameter sample}
\end{table}

\textbf{Encoder training.} 
We train each encoder using the LARS optimizer~\citep{LARSOptimizer}, LambdaLR Scheduler in PyTorch, momentum 0.9, weight decay $10^{-6}$, batch size 256, and the aforementioned hyperparameters for 400 epochs on a single A-100 GPU.

\textbf{Image transformation.} The image transformation strategy, including augmentation, is identical to the default transformation strategy provided by PyTorch Lightning.

\textbf{Linear evaluation.}
The linear head is trained using the SGD optimizer with a cosine learning rate scheduler, batch size 64, and weight decay $10^{-6}$ for 100 epochs. The learning rate starts at $0.3$ and ends at $0$.

\textbf{Moco Experiments.} We also tested our method based on MoCo~\citep{he2019moco}. The results are summarized in Table \ref{tab:results-moco}. Here we choose ResNet18~\citep{ResNet} as the backbone and set a temperature of $0.1$ as default. For our simple sum kernel, we set $\lambda=0.8$. The results show that our method outperforms the original MoCo method.

\begin{table}[thb]
\centering
\caption{MoCo Experiment Results on CIFAR-10 and CIFAR-100.}
\label{tab:results-moco}
\resizebox{\textwidth}{!}{%
\begin{tabular}{@{}c|ccc|ccc@{}}
\toprule
\multirow{3}{*}{Method} & \multicolumn{3}{c|}{CIFAR-10} & \multicolumn{3}{c}{CIFAR-100} \\ \cmidrule(lr){2-4} \cmidrule(lr){5-7} 
                        & 200 epochs & 400 epochs    & 1000 epochs   & 200 epochs & 400 epochs & 1000 epochs         \\ \midrule
MoCo (repro.)         & $76.41 \pm 0.12$    & $80.01 \pm 0.15$          & $84.45 \pm 0.08$    & $\mathbf{47.02 \pm 0.11}$ & $52.50 \pm 0.07$ & $57.62 \pm 0.15$            \\
\midrule
Laplacian Kernel        & ${78.09 \pm 0.10}$    & $\mathbf{83.85 \pm 0.09}$          & $\mathbf{88.34 \pm 0.16}$    & $46.12 \pm 0.22$   & $53.44 \pm 0.17$ & $59.10 \pm 0.14$        \\
Simple Sum Kernel & $\mathbf{78.12 \pm 0.15}$   & $83.23 \pm 0.18$ & $87.50 \pm 0.20$ & $46.65 \pm 0.06$ & $\mathbf{53.62 \pm 0.19}$ & $\mathbf{59.83 \pm 0.12}$\\
\bottomrule
\end{tabular}
}
\end{table}



\section{More Experiments on Synthetic Data}


Consider a scenario with $n$ clusters, each containing $k$ vertices. Let the probability of vertices $u$ and $v$ from the same cluster belonging to $\bfpi$ be $p$. Conversely, for vertices $u$ and $v$ from different clusters, let the probability of belonging to $\pi$ be $q$. We generate the graph $\bfpi$ randomly, based on $p$ and $q$. We experiment with values of $k=100$ and $n=6$ for ease of visualization, embedding all points in a two-dimensional space. Each vertex's initial position originates from a normal distribution. In each iteration, we sample a subgraph of $\bfpi$ uniformly, ensuring each vertex has an out-degree of $1$. We then optimize the corresponding vectors using InfoNCE loss with an SGD optimizer and iterate until convergence. Our experimental setup consists of an SGD learning rate of $1$, an InfoNCE loss temperature of $0.5$, and a batch size of $50$. We evaluate two scenarios with different $p$ and $q$ values: $p=1$, $q=0$, and $p=0.75$, $q=0.2$. The results of these experiments are visualized in Figure \ref{fig:vis-spectral-cluster}. The obtained embeddings exhibit the hallmark pattern of spectral clustering of graph $\bfpi$.

\begin{figure}[!tb]
\centering
\subfigure{
\includegraphics[width=1\textwidth]{Figures/cluster_pi.png}
\label{fig:vis-cluster}
}
\subfigure{
\includegraphics[width=1\textwidth]{Figures/noised_cluster_pi.png}
\label{fig:vis-noised-cluster}
}
\caption{Visualizations of the optimization process using InfoNCE Loss on the vectors corresponding to $\bfpi$. Points of identical color belong to the same cluster within $\bfpi$. To showcase the internal structure of $\bfpi$, we randomly select 10 vertices from each cluster to display the edge distribution of $\bfpi$.}
\label{fig:vis-spectral-cluster}
\end{figure}



% You can have as much text here as you want. The main body must be at most $8$ pages long.
% For the final version, one more page can be added.
%   If you want, you can use an appendix like this one, even using the one-column format.
%%%%%%%%%%%%%%%%%%%%%%%%%%%%%%%%%%%%%%%%%%%%%%%%%%%%%%%%%%%%%%%%%%%%%%%%%%%%%%%
%%%%%%%%%%%%%%%%%%%%%%%%%%%%%%%%%%%%%%%%%%%%%%%%%%%%%%%%%%%%%%%%%%%%%%%%%%%%%%%


\end{document}


% This document was modified from the file originally made available by
% Pat Langley and Andrea Danyluk for ICML-2K. This version was created
% by Iain Murray in 2018, and modified by Alexandre Bouchard in
% 2019 and 2021 and by Csaba Szepesvari, Gang Niu and Sivan Sabato in 2022. 
% Previous contributors include Dan Roy, Lise Getoor and Tobias
% Scheffer, which was slightly modified from the 2010 version by
% Thorsten Joachims & Johannes Fuernkranz, slightly modified from the
% 2009 version by Kiri Wagstaff and Sam Roweis's 2008 version, which is
% slightly modified from Prasad Tadepalli's 2007 version which is a
% lightly changed version of the previous year's version by Andrew
% Moore, which was in turn edited from those of Kristian Kersting and
% Codrina Lauth. Alex Smola contributed to the algorithmic style files.
