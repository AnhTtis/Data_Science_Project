% \vspace{-2mm}


\begin{table}[t]
	\begin{center}
	\resizebox{0.48\textwidth}{!}{
		\begin{centering}
		\begin{tabular}{cccc|ccc} % {|c|c|c|c|c|c|c|c|c|c|c|c|c|}
\toprule
%Method & Rep. & Proxy & Bed & Bookshelf & Chair & Desk & Sofa & Table & toilet & Bathhub & Dresser & Night Stand\tabularnewline
~&PC. & Image & Depth  & Avg.\_IN& Avg.\_OUT & Avg.\_OBJ\tabularnewline
\midrule
% \hline 

(\romannumeral1)&$\checkmark$ &  &  & 61.3  & 28.8 & 39.4 \tabularnewline
% \hline 
(\romannumeral2)& & $\checkmark$ &  & 64.2  & 41.1 & -  \tabularnewline

(\romannumeral3)&&  & $\checkmark$  & 56.9  & 23.9 & 39.0 \tabularnewline
\midrule
% \hline 
(f)&$\checkmark$ & $\checkmark$ &  & 68.7  & \textcolor{blue}{43.9} &  - \tabularnewline
% \hline 
(g)&$\checkmark$ &  & $\checkmark$ & 64.8  & 30.4 & \textcolor{blue}{43.2}  \tabularnewline
(h)&$\checkmark$ & $\checkmark$ &  $\checkmark$  & \textcolor{blue}{69.6}  & 42.3 & - \tabularnewline
\bottomrule
		\end{tabular}
	\end{centering}
		}
		\end{center}
					\vspace{-5mm}
	\caption{Analysis on the representation ensembling schemes. 
	}\label{tab-ablation-ensemble}
 \vspace{-3mm}
\end{table}
\section{Limitation}
%As described in our paper, we are the first to explore the contrastive language-3D pretraining with the data source from free realistic open world. Though our method enable zero-shot localization and recognition with proposed triplet proxy generation and learned transferable 3D representation, we can not provide the accurate tight bounding box for open-world 3D objects as a common detector. We believe our framework can facilitate the development of open-world 3D detector, by introducing our recognition ability on a general 3D detector or directly conduct detector training based on our present 3D proxies.
 As a pilot work for the language-3D pretraining problem, though CLIP$^2$ enables zero-shot localization and recognition with proposed triplet proxy generation and learned transferable 3D representation, it can not provide the accurate tight bounding box for open-world 3D objects as a common detector does. We believe CLIP$^2$ can facilitate the development of open-world 3D detectors by introducing the recognition ability to general 3D detectors or providing presented 3D proxies to enable further training of 3D detectors.
%  \vspace{-2mm}
\section{Conclusion}
In this paper, we present a novel contrastive language-image-point cloud pretraining framework, CLIP$^2$, which consists of a triplet proxy collection scheme and a cross-modal contrastive learning mechanism. Based on the observation that realistic scenarios contain a massive amount of open-world objects, we innovatively propose to collect triplet proxies from realistic scenes as pretraining data. We then conduct cross-modal contrastive alignment across language, image and point cloud feature space to learn transferable 3D representation. The zero-shot transfer results on various indoor and outdoor benchmarks validate the ability of  CLIP$^2$ for 3D open-world understanding.

\paragraph{Acknowledgements} We gratefully acknowledge the support of MindSpore\footnote{\url{https://www.mindspore.cn/}}, CANN (Compute
Architecture for Neural Networks) and Ascend AI Processor in this work.
