\chapter{Conclusion and Future Work}\label{chapter_5}

We propose a deep-learning based approach to predict drug–target binding affinity using only sequences of proteins and drugs. We use Convolutional Neural Networks (CNN) along with the residual skip connections (ResDTA) to learn representations from the raw sequence data of proteins and drugs and also use a combined stream in our network to build an overall robust representation and finally used fully connected layers in the affinity prediction task. We compare the performance of the proposed model with recent studies that used 1D string representation of the protein and drugs. We experimented on the KIBA dataset~\cite{KIBA}.\\


Our experiments show that using residual skip connections works much better than attention mechanism. It creates robust representation in each of the stream of the network. We also show that using the combined stream taking input from the last convolution layer from each of the stream creates a combined representation which reduce the necessity to use additional information along with the input data. The results show that the prediction that our model is producing has some statistical significance and it is not generating prediction by accident or by fluke. Our result also shows an improvement from the current state-of-the-art work AttentionDTA~\cite{attentiondta} using 1D string representation for protein and drugs. (From CI score 0.882 to CI score 0.885).\\



The major contribution of this study is the presentation of a novel deep learning-based model for drug–target affinity prediction that uses only character representations of proteins and drugs. By simply using raw sequence information for both drugs and targets, we were able to achieve better performance than the baseline methods.\\


In future to further improve our result we can use natural language processing (NLP) models for protein embeddings (e.g. ProteinBERT~\cite{brandes2022proteinbert}). These models help to represent protein sequence in an appropriate embedding which is better than any other methods and also these embeddings are able to create much better representation if we passed it in our model’s protein stream. Besides, we can use generative model like autoencoders to generate the molecular graph from the SMILES string and use those graphs along with graph convolutional network to create a robust representation of the drug. Therefore, we would not need any additional domain knowledge for using molecular graph. So, incorporating all these along with our model we can definitely improve our result further in our future work.\\


A large percentage of proteins remains untargeted, either due to bias in the drug discovery field for a select group of proteins or due to their undruggability, and this untapped pool of proteins has gained interest with protein deorphanizing efforts~\cite{edwards2011too, friedman2001greedy, fedorov2010targeted, o2016ligand}. The methodology can then be extended to predict the affinity of known compounds/targets to novel targets/drugs as well as to the prediction of the affinity of novel drug–target pairs.\\


\endinput

