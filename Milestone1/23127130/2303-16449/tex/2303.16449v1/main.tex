\documentclass[onecolumn,twoside]{article}


\ifdefined\siggraph
\usepackage{times}
\fi

%\usepackage{parskip}
\usepackage{color}
\usepackage{ifthen}
\usepackage{float}
\usepackage{alltt}
\usepackage{newlfont} % for Box
\usepackage{array}
\usepackage{wrapfig}
\usepackage{booktabs}
\usepackage{multirow}
\usepackage{amsfonts}
\usepackage{dsfont}
\usepackage[linesnumbered,ruled,vlined]{algorithm2e}
 
%%% Coloring the comment as blue
\newcommand\mycommfont[1]{\footnotesize\ttfamily\textcolor{blue}{#1}}
\SetCommentSty{mycommfont}
 


% -------- revision commands --------------------------
\newcommand{\rem}[1]{{\color{red}#1}}
\newcommand{\add}[1]{{\color{blue}#1}}

\definecolor{comment}{HTML}{9326ff}
\newcommand{\com}[1]{{\color{comment}[#1]}}
\definecolor{commentH}{HTML}{EE82EE}
\newcommand{\comH}[1]{{\color{commentH}[#1]}}
\definecolor{commentJ}{HTML}{23ba4c}
\newcommand{\comJ}[1]{{\color{commentJ}[#1]}}
\newcommand{\phdagger}{{\phantom{\dagger}}}

% -------- math and braket definitions ----------------
\newcommand{\tr}{\mathrm{Tr}}
\renewcommand{\Im}{\mathrm{Im}}
\renewcommand{\Re}{\mathrm{Re}}
\def\bra#1{\langle{#1}|}
\def\ket#1{|{#1}\rangle}
\def\braket#1{\langle{#1}\rangle}
\newcommand{\ketbra}[2]{\ket{#1}\!\bra{#2}}
% ----------------------------------------------
% --- colors
\definecolor{lightterminal}{HTML}{f7f7ff}
\definecolor{energy_orange}{HTML}{ffdbc9}
\definecolor{light_blue}{HTML}{f2f5ff}

%New colors defined below
\definecolor{darkterminal}{HTML}{8792f5} %  --> subtle
\definecolor{quasiblack}{HTML}{545880} %  --> hard
\definecolor{softblue}{HTML}{4759ff} %
\definecolor{deep_purple}{HTML}{6f00ff}
\definecolor{boxcolor}{HTML}{e5e3fa}

% ------------------ rounded rectangle--------------------
\newcommand{\round}[2]{\tikz[baseline=-4pt]\node [draw=#2,fill=#2,semithick ,rectangle, inner sep=3pt, rounded corners=1pt] (X) {\footnotesize{\transparent{1}\texttt{#1}}};}
\newcommand{\python}{\round{{\color{blue}python}}{boxcolor}\hspace{5pt}}

% --- package for code display
\usepackage{verbatim}
% --- Minted for code highlights
%\usepackage[finalizecache,cachedir=minted-cache]{minted}
\usepackage[frozencache,cachedir=minted-cache]{minted}

\usepackage{etoolbox,xpatch}

\makeatletter
\AtBeginEnvironment{minted}{\dontdofcolorbox}
\def\dontdofcolorbox{\renewcommand\fcolorbox[4][]{##4}}
\xpatchcmd{\inputminted}{\minted@fvset}{\minted@fvset\dontdofcolorbox}{}{}
\xpatchcmd{\mintinline}{\minted@fvset}{\minted@fvset\dontdofcolorbox}{}{} % see https://tex.stackexchange.com/a/401250/
\makeatother

%\usemintedstyle{rainbow_dash}
\usemintedstyle{vs}
\renewcommand\theFancyVerbLine{\color{darkterminal}\footnotesize\arabic{FancyVerbLine}}

% --- \ref in colour for code
\newcommand{\coderef}[1]{{\hypersetup{linkcolor=white}\ref{#1}}}

% minted styles for Python and MATLAB and Mathematica
\setminted[python]{xleftmargin=20pt,baselinestretch=1.2,linenos}
\setminted[MATLAB]{xleftmargin=20pt,baselinestretch=1.2,linenos}
\setminted[Mathematica]{xleftmargin=20pt,baselinestretch=1.2,linenos}

\newtcbox{\circled}[1][lightterminal]{on line,
    colback=#1, colframe=#1, boxsep=0pt, boxrule=0pt, size=small, arc=2pt}
    
% ---- code box
\newtcolorbox[auto counter, number within = section, number freestyle={\noexpand\thesection.\noexpand\arabic{\tcbcounter}}]{code_box}[4]{breakable,center, fonttitle = \bfseries\sffamily, coltitle = white, colback = lightterminal, fontlower = \footnotesize, arc=1mm, width=\textwidth, frame style = softblue, pad at break=3mm, boxrule=0.5pt,
break at=-\baselineskip/0pt, enhanced jigsaw,
height fixed for=middle, before upper={\parindent7pt\noindent}, toprule=0pt, boxsep = 5pt, left=4pt, right=1.5pt,
title={\small\textcolor{energy_orange}{Script{ }\thetcbcounter:}{ } #2 {\hspace{5pt}} \href{#1}{\includegraphics[width=2.5mm]{Packages/link.pdf}} \hfill \circled{\textcolor{quasiblack}{\texttt{#4}}}}, label=#3}

\newtcolorbox{eqbox}{center, breakable, colback = lightterminal, arc=2mm, colframe = softblue,  boxrule=0.8pt, width = \textwidth}

% --- Code command
\newcommand{\code}[5]{
\begin{code_box}{#1}{#2}{#3}{#4}
\footnotesize
\inputminted{#4}{#5}
\end{code_box}
}

\newtcolorbox{codeline}[1]{colback=lightterminal, colframe=softblue,fonttitle=\bfseries,colbacktitle=softblue, enhanced, attach boxed title to top left={yshift=-7.1mm, xshift = 1.6mm},title={\footnotesize\texttt{#1}},boxrule=0.5pt}


\def\iconYes{\faCheck}
\def\iconNo{\faTimes}
\def\iconPartially{\faCircleThin}
\def\iconWarn{\faExclamation}
\def\iconBeer{\faBeer}
\def\iconAward{\faTrophy}
\def\iconQuote{\faQuoteLeft}
\def\iconInteraction{\faStreetView}

\definecolor{thesisGreen}{RGB}{46, 204, 64}
\definecolor{thesisRed}{RGB}{255, 65, 54}


\def\acceptedPaper{\color{white}\circled{green}{\iconBeer}}
\usepackage{csquotes}

\usepackage{times}
\renewcommand\ttdefault{cmtt}

\makeatletter
\renewcommand{\maketitle}{\bgroup\setlength{\parindent}{0pt}
\begin{flushleft}
  \textbf{\@title}

  \@author
\end{flushleft}\egroup
}
\makeatother

\renewenvironment{abstract}
 {\par\noindent\textsf{\textbf{\abstractname}}\ \ignorespaces\vspace{10pt} \\}
 {\par\medskip}

\title{\sffamily{\bfseries{\LARGE{A Tutorial on Quantum Master Equations:\vspace{5pt} \\ {\Large Tips and tricks for quantum optics, quantum computing and beyond}}}}\vspace{10pt}}

%\usepackage[symbol]{footmisc}
\makeatletter
\def\@xfootnote[#1]{%
  \protected@xdef\@thefnmark{#1}%
  \@footnotemark\@footnotetext}
\makeatother

\author[1,2]{Francesco Campaioli\footnote[$\dagger$]{\href{francesco.campaioli@rmit.edu.au}{francesco.campaioli@rmit.edu.au}}}

\author[1]{Jared H. Cole\footnote[*]{\href{jared.cole@rmit.edu.au}{jared.cole@rmit.edu.au}}}

\author[1]{Harini Hapuarachchi\footnote[$\ddagger$]{\href{harini.hapuarachchi@rmit.edu.au}{harini.hapuarachchi@rmit.edu.au}}}

\affil[1]{\small{\textit{Chemical and Quantum Physics, and ARC Centre of Excellence in Exciton
Science,}
\textit{School of Science, RMIT University, Melbourne 3000, Australia}}}

\affil[2]{\small{\textit{Dipartimento di Fisica e Astronomia “G. Galilei,” Università degli Studi di Padova, I-35131 Padua, Italy,}

\textit{Padua Quantum Technologies Research Center, Università degli Studi di Padova, I-35131 Padua, Italy}}
}




\begin{document}

\date{}

\maketitle
%\makeatletter

\thispagestyle{empty}

\begin{abstract}
Quantum master equations are an invaluable tool to model the dynamics of a plethora of microscopic systems, ranging from quantum optics and quantum information processing, to energy and charge transport, electronic and nuclear spin resonance, photochemistry, and more. This tutorial offers a concise and pedagogical introduction to quantum master equations, accessible to a broad, cross-disciplinary audience. The reader is guided through the basics of quantum dynamics with hands-on examples that build up in complexity. The tutorial covers essential methods like the Lindblad master equation, Redfield relaxation, and Floquet theory, as well as techniques like Suzuki-Trotter expansion and numerical approaches for sparse solvers. These methods are illustrated with code snippets implemented in \texttt{python} and other languages, which can be used as a starting point for generalisation and more sophisticated implementations. 
\end{abstract}

\hypersetup{linkcolor = black}
\tableofcontents
\hypersetup{linkcolor = blue}

% ----------- Introduction
\section{Introduction}
\label{sec:introduction}
% \begin{itemize}
%     % Diffusion of FL
%     \item {\st{Diffusion of FL}}
%     % Security threats to FL
%     \item {\st{Security threats to FL with particular focus on model poisoning}}
%     % Limitations of existing countermeasures
%     \item {\st{Current countermeasures (e.g., KRUM) and their limitations}}
%     % Proposed method and its advantages
%     \item {\st{Intuitive description of the proposed method and its difference (i.e., advantages) w.r.t. state of the art}}
%     % Main contributions
%     \item {\st{Summary of the main contributions of this work}}
%     % Paper's structure and organization
%     \item {\st{Paper's structure and organization}}
% \end{itemize}

% Diffusion of FL
Recently, {\em federated learning} (FL) has emerged as the leading paradigm for training distributed, large-scale, and privacy-preserving machine learning (ML) systems~\cite{mcmahan2017googleai,mcmahan2017aistats}. 
The core idea of FL is to allow multiple edge clients to collaboratively train a shared, global model without disclosing their local private training data.
%Specifically, an FL system consists of a central server and many edge clients; 
A typical FL round involves the following steps: {\em(i)} the server randomly picks some clients and sends them the current, global model; {\em(ii)} each selected client locally trains its model with its own private data; then, it sends the resulting local model to the server;\footnote{Whenever we refer to global/local model, we mean global/local model {\em parameters}.} {\em(iii)} the server updates the global model by computing an \emph{aggregation function}, usually the average (FedAvg), on the local models received from clients.
% \begin{enumerate}
%     \item[{\em(i)}] the server sends the current, global model to the clients and appoints some of them for training;
%     \item[{\em(ii)}] each selected client locally trains its copy of the global model with its own private data; then, it sends the resulting local model back to the server;\footnote{Whenever we refer to global/local model, we mean global/local model {\em parameters}.}
%     \item[{\em(iii)}] the server updates the global model by computing an \emph{aggregation function} on the local models received from clients (by default, the average, also referred to as FedAvg~\cite{mcmahan2017aistats}).
% \end{enumerate}
This process goes on until the global model converges. %(e.g., after a certain number of rounds or other similar stopping criteria).
%\\
% The advantages of FL over the traditional, centralized learning paradigm are undoubtedly clear in terms of flexibility/scalability (clients can join/disconnect from the FL network dynamically), network communications (only model weights\footnote{We will use \textit{parameters} and \textit{weights} interchangeably.} are exchanged between clients and server), and privacy (each client's private training data is kept local at the client's end and not uploaded to the server).
\\
% Security threats to FL
%However, the growing adoption of FL also raises security concerns~\cite{costa2022covert}, particularly about its confidentiality, integrity, and availability.
Although its advantages over standard ML, FL also raises security concerns~\cite{costa2022covert}. %, particularly about its confidentiality, integrity, and availability~\cite{costa2022covert}.
% OLD, LONG VERSION
% Indeed, some work deals with privacy leakage that may expose the local data of some clients~\cite{melis2019sp}. 
% A large body of work, instead, investigates attacks that usually aim to detriment the predictive accuracy of the learned global model. For instance, \emph{data poisoning} attacks achieve this goal by letting an adversary pollute the training set of some corrupt FL clients with maliciously crafted examples~\cite{jagielski2018sp}.
% Similarly, in \emph{model poisoning} the attacker attempts to tweak the global model weights~\cite{bhagoji2019pmlr} by directly perturbing the local model's weights of some infected FL clients before these are sent to the central server for aggregation, usually via so-called Byzantine attacks. 
% It turns out that Byzantine model poisoning attacks severely impact standard FedAvg; therefore, more robust aggregation functions must be designed to make FL systems secure.
Here, we focus on \emph{untargeted model poisoning} attacks~\cite{bhagoji2019pmlr}, where an adversary attempts to tweak the global model weights %\footnote{We will use the terms \textit{parameters} and \textit{weights} interchangeably.} 
by directly perturbing the local model's parameters of some infected clients before these are sent to the central server for aggregation.
In doing so, the adversary aims to jeopardize the global model \textit{indiscriminately} at inference time.
Such model poisoning attacks severely impact standard FedAvg; therefore, more robust aggregation functions must be designed to secure FL systems.
\\
% In this paper, we focus on designing a novel robust aggregation scheme at the server's end to contrast the effect of Byzantine model poisoning attacks.
%
% Current countermeasures and their limitations
%Several countermeasures have been proposed in the literature to combat model poisoning attacks on FL systems.
% Some methods use simple statistics more robust than plain average to smooth the impact of malicious updates (e.g., Trimmed Mean and FedMedian~\cite{yin2018icml}). 
% Other defenses implement outlier detection techniques to discard malicious updates from the aggregation performed at the server's end. Those are either based on heuristics (e.g., Krum/Multi-Krum~\cite{blanchard2017nips} and Bulyan~\cite{mhamdi2018pmlr}) or data-driven approaches (e.g., K-means clustering~\cite{shen2016acm} or DnC via spectral analysis~\cite{shejwalkar2021ndss}). 
% Finally, some strategies rely on a centralized ``source of trust'' to spot potential malicious updates (e.g., FLTrust~\cite{cao2020fltrust}).
% Several countermeasures have been proposed in the literature to combat model poisoning attacks on FL systems, i.e., to discard possible malicious local updates from the aggregation performed at the server's end. 
% These techniques range from simple statistics more robust than plain average (e.g., Trimmed Mean and FedMedian~\cite{yin2018icml}) to outlier detection heuristics (e.g., Krum/Multi-Krum~\cite{blanchard2017nips} and Bulyan~\cite{mhamdi2018pmlr}) or data-driven approaches (e.g., spectral analysis via K-means clustering~\cite{shen2016acm} or spectral analysis), or methods based on ``source of trust'' (e.g., FLTrust~\cite{cao2020fltrust}).
% OLD, LONG VERSION
%Several countermeasures have been proposed in the literature to combat Byzantine model poisoning attacks on FL systems.
% Descriptive statistics
% For example, Trimmed Mean and FedMedian aggregate local model updates using more robust statistics than standard average~\cite{yin2018icml}.
%
% % Heuristics for outlier detection
% Many existing Byzantine-resilient strategies implement some outlier detection heuristics to discard the model updates sent by potentially malicious clients from the input of the aggregation function.
% One of the most popular heuristics is Krum~\cite{blanchard2017nips}.
% This strategy tries to mitigate the impact of Byzantine attacks by selecting as a global model the local model with the smallest sum of Euclidean distances to {\em all} the other local models.
% Although powerful, Krum requires the server to know (or, at least, estimate) the number of malicious FL clients upfront, which is generally impossible in a realistic attack scenario. %
% Moreover, Krum may become ineffective for complex, high-dimensional model parameter spaces due to the curse of dimensionality.
% Bulyan~\cite{mhamdi2018pmlr} tries to overcome this issue by combining Krum with a variant of Trimmed Mean.
% % Data-driven outlier detection
% Other strategies use data-driven outlier detection techniques -- e.g., via K-means clustering~\cite{shen2016acm} -- to spot potential malicious local model updates. 
% %For instance, Shen et al. propose to cluster local model updates with K-means and thus identify outliers.
%
% % Other techniques
% As far as the server is concerned, any local model received can be from a potential malicious client. 
% FLTrust~\cite{cao2020fltrust} assumes the server acts as a client, i.e., trains a local model on an additional {\em trustworthy} dataset at the server's end and compares it against all the local models from other clients. 
% This way, the server can rely on some ``source of trust'' when discarding potentially malicious clients.
%\\
% Limitations of existing Byzantine-resilient strategies
Unfortunately, existing defense mechanisms either rely on simple heuristics (e.g., Trimmed Mean and FedMedian by~\cite{yin2018icml}) or need strong and unrealistic assumptions to work effectively (e.g., foreknowledge or estimation of the number of malicious clients in the FL system, as for Krum/Multi-Krum~\cite{blanchard2017nips} and Bulyan~\cite{mhamdi2018pmlr}, which, however, cannot exceed a fixed threshold).
Furthermore, outlier detection methods using K-means clustering~\cite{shen2016acm} or spectral analysis like DnC~\cite{shejwalkar2021ndss} do not directly consider the temporal evolution of local model updates received.
Finally, strategies like FLTrust~\cite{cao2020fltrust} require the server to collect its own dataset and act as a proper client, thereby altering the standard FL protocol.
\\
% OLD, LONG VERSION
% Overall, existing Byzantine-resilient strategies are either simple heuristics (e.g., FedMedian) or, if they are more complex, they rely on strong and unrealistic assumptions to work effectively (e.g., knowing the number of malicious clients in the FL system in advance, as for Krum and alike).
% Furthermore, data-driven outlier detection methods do not consider the temporary evolution of local model updates received (e.g., K-means clustering). 
% Finally, strategies like FLTrust requires the server to collect its own dataset and act as a proper client, thereby altering the standard FL protocol.
%
% Description of the proposed method
This work introduces a novel pre-aggregation \textit{filter} robust to untargeted model poisoning attacks. Notably, this filter $(i)$ operates without requiring prior knowledge or constraints on the number of malicious clients and $(ii)$ inherently integrates temporal dependencies. 
The FL server can employ this filter as a preprocessing step before applying \textit{any} aggregation function, be it standard like FedAvg or robust like Krum or Bulyan.
Specifically, we formulate the problem of identifying corrupted updates as a multidimensional (i.e., matrix-valued) time series anomaly detection task. 
The key idea is that legitimate local updates, resulting from well-calibrated iterative procedures like stochastic gradient descent (SGD) with an appropriate learning rate, show \textit{higher predictability} compared to malicious updates. This hypothesis stems from the fact that the sequence of gradients (thus, model parameters) observed during legitimate training exhibit regular patterns, as validated in Section~\ref{subsec:intuition}. %until convergence. 
%This regularity may be more pronounced for smooth convex loss functions, but it can still be captured within an appropriate time window, even for more complex and convoluted loss surfaces. 
%We provide evidence of this claim in Appendix~B, where we show that the average mutual information (i.e., ``predictability''), calculated over pairs of legitimate model updates sent at different FL rounds, is significantly higher than the corresponding computation for a malicious client.
\\
Inspired by the matrix autoregressive (MAR) framework for multidimensional time series forecasting~\cite{chen2021je}, we propose the FLANDERS ({\em \textbf{F}ederated \textbf{L}earning meets \textbf{AN}omaly \textbf{DE}tection for a \textbf{R}obust and \textbf{S}ecure}) filter.
The main advantages of FLANDERS over existing strategies like FLDetector~\cite{zhao2020multivariate} are its resilience to large-scale attacks, where $50\%$ or more FL participants are hostile, and the capability of working under realistic non-iid scenarios.
We attribute such a capability to two key factors: $(i)$ FLANDERS works without knowing a priori the ratio of corrupted clients, and $(ii)$ it embodies temporal dependencies between intra- and inter-client updates, quickly recognizing local model drifts caused by evil players. Below, we summarize our main contributions:

\begin{itemize}
\item[{\em(i)}]
We provide empirical evidence that the sequence of models sent by legitimate clients is more predictable than those of malicious participants performing untargeted model poisoning attacks.
\\
\item[{\em(ii)}] 
We introduce FLANDERS, the first pre-aggregation filter for FL robust to untargeted model poisoning based on multidimensional time series anomaly detection.
\\
\item[{\em(iii)}] 
We integrate FLANDERS into Flower,\footnote{\scriptsize{\url{https://flower.dev/}}} a popular FL simulation framework for reproducibility.
\\
\item[{\em(iv)}] 
We show that FLANDERS improves the robustness of the existing aggregation methods under multiple settings: different datasets, client's data distribution (non-iid), models, and attack scenarios.
\\
\item[{\em(v)}] 
We publicly release all the implementation code of FLANDERS along with our experiments.\footnote{\scriptsize{\url{https://anonymous.4open.science/r/flanders_exp-7EEB}}}
\end{itemize}

% Paper's structure and organization
The remainder of the paper is structured as follows. %some related work and the current state-of-the-art solutions to security issues that FL entails. 
Section~\ref{sec:background} covers background and preliminaries. 
In Section~\ref{sec:related}, we discuss related work.
Section~\ref{sec:problem} and Section~\ref{sec:method} describe the problem formulation and the method proposed. % to tackle it. 
Section~\ref{sec:experiments} gathers experimental results. %, and Section~\ref{sec:limitations} discusses some limitations of this work.
Finally, we conclude in Section~\ref{sec:conclusion}.
 %discusses the limitations of this work and draws future research directions.
%reports conclusions and draws perspectives for future research directions.

%%%%%%% OLD %%%%%%%
%to overcome the resilience of Byzantine failures in distributed Stochastic Gradient Descent computations. 
% The strength of Krum is its time complexity, which is linear in the gradient dimension. 
% However, the robustness of the approach is guaranteed for gradient-based learning applications only when the majority of the clients are not compromised. 
% Besides, the aggregation mechanism of Krum, as well as that of similar methods, is robust from a coarse-grained perspective and does not provide solutions to errors and perturbations that may occur at inference time.
%A related approach to~\cite{blanchard2017nips} is the work of Su et al.~\cite{su2016dc}. Here, the authors propose an iterated approximate agreement to tackle a multi-layer scenario attacked by Byzantine agents. 
%However, the method works efficiently on the sole discrete context and it is inapplicable to continuous state environments.
%\gabri{Maybe, we should just talk about the main limitations of existing countermeasures without digging into their details (or, we can just mention Krum as this is the most popular one). I will move the description of all these methods to the Related Work section.}

% ----------- Density operators
\section{Density Operators} 
\label{s:density_operators}
In this section we briefly review the mathematical description of the \textit{state} of a quantum system, focusing on the numerical implementation of state vectors and density operators. We assume that the reader is familiar with the \textit{postulates of Quantum Mechanics}, Hilbert spaces, expectation values, time evolution, and composite systems, which can be reviewed in any of these textbooks~\cite{Cohen1978,Scully1997,Gardiner2000,Bransden2000,Breuer2002,Schlosshauer2007,Wiseman2009,Weiss2012,,Nielsen2010}. 

\subsection{Pure states}
\label{ss:pure_states}
Let us consider a $d$-dimensional quantum system with Hilbert space $\mathcal{H}$. Let $\mathcal{B}:= \{ |\phi_1\rangle, |\phi_2\rangle, ..., |\phi_d\rangle\}$ be an orthonormal basis for $\mathcal{H}$, so that $\langle \phi_i|\phi_j\rangle = \delta_{ij}$. For example, $\mathcal{B}$ could be given by the orthonormal eigenstates of a hermitian operator such as some Hamiltonian $H$. 
Any state of the system can be expressed as a \emph{coherent superposition} with complex coefficients $c_i\in\mathbb{C}$,
\begin{equation}
\label{eq:coherent_superposition}
	|\psi\rangle = c_1|\phi_1\rangle+c_2|\phi_2\rangle+...+ c_d|\phi_d\rangle = \sum_{j=1}^d c_j\ket{\phi_j},
\end{equation}
where the coefficient $c_j$ are such that $\braket{\psi|\psi} = \sum_{j=1}^d |c_j|^2= 1$, according to the Born interpretation of the wavefunction~\cite{Mcmahon2013}.
The square of the coefficients in Eq.~\eqref{eq:coherent_superposition}, $|c_j|^2$, represents the probability of finding the system in the eigenstate $|\phi_j\rangle$ upon measurement in the considered basis $\mathcal{B}$. See Ref.~\cite{Cohen1978} for a review of projective measurement and Ref.~\cite{Nielsen2010} for the generalisation to positive operator valued measures (POVMs). 

Unit vectors like $\ket{\psi}$ are called \emph{pure states}. A pure state contains all the available physical information about the system, such as the expectation value of an observable $\mathcal{A}$ associated with hermitian operator $A$,
\begin{equation}
    \label{eq:pure_expectation_value}
    \braket{A} = \braket{\psi|A|\psi}.
\end{equation}
The following \texttt{python} script uses methods from the \href{https://numpy.org/}{\texttt{numpy}} library to implement state vectors and operators, and calculates the expectation value of some observable.
% ------ code: pure states and expectation values ------
\code{https://github.com/frnq/qme/blob/main/python/pure_states_expect.py}{Pure states and expectation values}{code:pure_states_expect}{python}{Scripts/python/pure_states_expect.txt}

\subsection{Mixed states: Proper and improper mixtures}
\label{ss:mixed_states}

There are two important scenarios where pure states are no longer sufficient to describe the state of a system. First, in experimental settings, we often lack the knowledge of the exact pure state $|\psi\rangle$ of our system. Instead, we may know that the system is in any of the pure orthonormal states $\{\ket{\psi_j}\}$ with some probabilities $\{ p_j \}$. In other words, our knowledge of the system is represented by a \emph{statistical mixture of pure states}, described by the set $\lbrace|\psi_j\rangle, p_j\rbrace$. 
In such case, when more than one $p_j$ is non-zero, the system is said to be in a \emph{mixed state}. This is sometimes referred to as a \textit{proper} mixture~\cite{Masillo2009}.

Second, when studying the dynamics of composite systems, pure states are no longer the most general description of a state. This is because the marginal state of any \textit{entangled} state cannot be represented as a pure state, and instead, needs to be represented as a statistical mixture over the basis elements of the considered subsystem~\cite{Nielsen2010}, as discussed in Sec.~\ref{ss:composite systems}. This is sometimes referred to as an \textit{improper mixture}~\cite{Masillo2009}. See Refs.~\cite{Cohen1978,Nielsen2010} for more on composite systems, and Refs.~\cite{Mintert2005,Bengtsson2006,Modi2012} for an in-depth analysis of entanglement and other quantum correlations. 

\subsection{Definition and properties of the density operator}
\label{ss:properties_density_operator}
Whether we are dealing with proper or improper mixtures of states, we can represent the set $\{\ket{\psi_j},p_j\}$ using a linear operator on the Hilbert space,
\begin{equation}
\label{eq:density_operator}
	{\rho} = \sum_{j=1}^{d}p_j \ketbra{\psi_j}{\psi_j},
\end{equation}
known as the density operator~\cite{Nielsen2010}, where $\ketbra{\psi_j}{\psi_j}$ is the outer product of $\ket{\psi_j}$ with itself, that is, the vector product of $\ket{\psi_j}$ with its dual $\bra{\psi_j}$. The coefficients $p_j > 0$ are such that $\sum_j p_j = 1$, since they represent probabilities (also known as \textit{convex combination}).  Density operators have three fundamental properties,
\begin{enumerate}
    \item \textbf{Hermitian:} ${\rho} = {\rho}^\dagger$. This implies that $\rho$ has only real eigenvalues.
    \item \textbf{Positive\footnote{Or, more specifically, \textit{positive semi-definite}.}:} $\rho > 0$. That is, $\rho$ eigenvalues $p_j\in[0,1]$ are not negative.
    \item \textbf{Normalised:} $\tr\rho= 1$, which can also be stated as $\sum_j p_j = 1$, i.e., the sum of its eigenvalues (probabilities) must add up to 1.
\end{enumerate}

Density operator can represent both pure and mixed states, and can be expressed in any basis $\mathcal{B} = \{\ket{\phi_i}\}_{i=1}^d$ of the Hilbert space $\mathcal{H}$ as
\begin{equation}\label{Eq:Density_Matrix_in_basis}
	{\rho} = \sum_{i,j=1}^{d}\rho_{ij}\ketbra{\phi_i}{\phi_j} = \begin{pmatrix}
	\rho_{11} & \rho_{12} &\dots &\rho_{1d}\\
	\rho_{21} & \rho_{22} &\dots &\rho_{2d}\\
	\vdots & \vdots & \ddots & \vdots\\
	\rho_{d1} & \rho_{d2} &\dots &\rho_{dd}
	\end{pmatrix},
\end{equation}
where $\rho_{ij}$ is the associated matrix element with row $i$ and column $j$.
The diagonal elements $\rho_{ii}$ of the density matrix are known as \emph{populations} and they denote the probabilities of finding the system in the respective basis states $|\phi_i\rangle$. The off-diagonal elements $\rho_{ij}$ are known as \emph{coherences}, and provide information about the coherent superposition of the basis states $|\phi_i\rangle$ and $|\phi_j\rangle$~\cite{Manzano2020}. 

Similarly to state vectors, density operators encode all the available information that can be extracted from the considered system. For example, the expectation value of some observable $\mathcal{A}$ associated with hermitian operator $A$ can be calculated as,
\begin{equation}
    \label{eq:expectation_mixed}
    \braket{A} = \tr[A\rho].
\end{equation}
The following \texttt{python} script provides an implementation of a density operator and the evaluation of the expectation value of some observable. There, a system with dimension $d=3$ is in a mixed state defined by state vectors $\{\ket{\psi_1},\ket{\psi_2},\ket{\psi_3}\}$ with probabilities $\{0.1,0.3,0.6\}$, represented by the density operator $\rho = 0.1\ketbra{\psi_1}{\psi_1}+0.3\ketbra{\psi_2}{\psi_2}+0.6\ketbra{\psi_3}{\psi_3}$.
% ------ code: mixed states and expectation values ------
\code{https://github.com/frnq/qme/blob/main/python/mixed_states_expect.py}{Mixed states and expectation values}{code:mixed_states_expect}{python}{Scripts/python/mixed_states_expect.txt}

\subsection{Composite systems}
\label{ss:composite systems}

Composite systems consist of two or more (interacting) quantum systems, whose Hilbert space is given by the tensor product of the individual Hilbert subspaces, $\mathcal{H} = \bigotimes_{i} \mathcal{H}_i$~\cite{Nielsen2010}. For example, a composite system might be given by a pair of interacting two-level systems (\textit{qubits}, in quantum information theory), or by a system $\mathrm{S}$ interacting with some large environment $\mathrm{E}$.

\subsubsection{Tensor product and partial trace}
\label{sss:tensor_product_partial}
Any state $\rho$ of a composite system can be represented using a basis $\mathcal{B}$ constructed using the \textit{tensor product} of the basis elements of each subsystems' basis $\mathcal{B}_\alpha = \{ \ket{\phi_i}_\alpha \}_{i=1}^{d_\alpha}$. For example, a bipartite system can be expressed in the following basis,
\begin{equation}
    \mathcal{B} = \Big\{ \ket{\phi_i}_1\otimes\ket{\phi_j}_2 \Big\}_{i,j}.
\end{equation} 
In \texttt{python}, the tensor product can be implemented with \texttt{numpy} using the Kroneker product \href{https://numpy.org/doc/stable/reference/generated/numpy.kron.html}{\texttt{kron}}. 
\begin{codeline}{python}
\footnotesize
    \begin{minted}[linenos=false]{python}
        psi = numpy.kron(psi1,psi2)
    \end{minted}
\end{codeline}
\noindent
Similar implementations are available in \texttt{Mathematica} and \texttt{MATLAB}, with \href{https://reference.wolfram.com/language/ref/KroneckerProduct.html}{\texttt{KroneckerProduct}} and \href{https://au.mathworks.com/help/matlab/ref/kron.html}{\texttt{kron}}, respectively.

When taking expectation values for composite systems, it may be useful to focus only on the \textit{marginal state} of one of the subsystems. For example, the marginal state $\rho_1$ of subsystem $1$ is obtained from the total state $\rho$ by \textit{tracing over} the degrees of freedom associated with the rest of the Hilbert space (here, subsystem $2$),
\begin{equation}
    \label{eq:partial_trace}
    \rho_1 = \tr_2 [\rho].
\end{equation}
The linear operator $\tr_i[\cdot]$ is called \textit{partial trace}, and its definition can be found in Ref.~\cite{Breuer2002}. For the case of bipartite systems with dimensions $d_1$ and $d_2$, the partial trace can be implemented in \texttt{python} using \texttt{numpy}.
\begin{codeline}{python}
\footnotesize
    \begin{minted}[linenos=false]{python}
        rho1 = np.trace(rho.reshape(d1,d2,d1,d2), axis1=0, axis2=2)
        rho2 = np.trace(rho.reshape(d1,d2,d1,d2), axis1=1, axis2=3)
    \end{minted}
\end{codeline}

For example, let us consider the following bipartite pure state 
\begin{equation}
    \label{eq:bipartite_pure}
    \ket{\psi(\theta)} = \cos(\theta)\ket{00}+\sin(\theta)\ket{11},
\end{equation}
where $\ket{00}=\ket{0}_1\otimes\ket{0}_2$, $\ket{11}=\ket{1}_1\otimes\ket{1}_2$, and its associated density operator is given by $\rho(\theta) = \ketbra{\psi(\theta)}{\psi(\theta)}$. The state $\rho(\theta)$ is separable for $\theta = 0,\pi/2$, and entangled otherwise, being maximally entangled\footnote{The state $\ket{\psi(\pi/4)}$ is the $\Phi_+$ Bell state~\cite{Nielsen2010}.} for $\theta = \pi/4$. 
As a result, for $\theta\neq k\pi/2$ the partial state of each subsystem $\rho_i(\theta) = \tr_j[\rho(\theta)]$ is not pure, and is therefore an improper mixture.

To measure the degree of mixedness of a density operator we can use the \textit{purity} $\mathcal{P}$,
\begin{equation}
    \label{eq:purity}
    \mathcal{P}[\rho] = \tr[\rho^2] = \sum_{j=1}^d p_j^2,
\end{equation}
which is bounded between 1, for pure states $\rho = \ketbra{\psi}{\psi}$, and $1/d$, for maximally mixed states $\rho = \mathbb{1}/d$. For more on purity, entropy, measures of distinguishability, and other information-theoretic figures of merit see Refs.~\cite{Bengtsson2006,Nielsen2010}.

The following \texttt{python} script calculates the marginal state of the first subsystem, $\rho_1(\theta) = \tr_2\rho(\theta)$, showing that its purity $\mathcal{P}[\rho_1(\theta)]<1$ for $\theta \neq k\pi/2$. Notice that $\rho_1(\theta)$ is maximally mixed when $\rho(\theta)$ is maximally entangled, i.e., $\tr\rho_1(\pi/4)=1/2$, as shown in Fig.~\ref{fig:partial_trace}. A powerful implementation of the tensor product and the partial trace (\href{https://qutip.org/docs/3.1.0/guide/guide-tensor.html}{\texttt{ptrace}}) for any type of composite system is available in \href{https://qutip.org/}{\texttt{QuTiP}}, as shown in the script~\ref{code:tensor_partial}.
% ----- code for block-matrix -----
\code{https://github.com/frnq/qme/blob/main/python/partial_trace.py}{Partial trace and purity of entangled states}{code:partial_trace}{python}{Scripts/python/partial_trace.txt}

\begin{figure}[h]
    \centering
    \includegraphics{Figures/partial_trace.pdf}
    \caption{Purity of the marginal state $\rho_1(\theta) = \tr_2\rho(\theta)$, calculated using script~\ref{code:partial_trace}. The state $\rho_1(\theta)$ is maximally mixed for $\theta = \pi/4$, since $\rho(\pi/4)$ is maximally entangled. This is an example of an improper mixture.}
    \label{fig:partial_trace}
\end{figure}

\subsubsection{Direct sum}
\label{sss:direct_sum}

Sometimes, it is useful to compose systems given by the \textit{addition} of different Hilbert spaces together. For example, when studying a pair of interacting systems with Hilbert space $\mathcal{H}_a = \mathcal{H}_1\otimes\mathcal{H}_2$ and dimension $d_a$, it might be convenient to add some states $\{\ket{\phi_i}_b\}_{i=1}^{d_b}$ to the picture, perhaps representing the result of some transitions that are modelled phenomenologically.
In these cases the total Hilbert space is given by
\begin{equation}
    \label{eq:direct_sum}
    \mathcal{H} = \mathcal{H}_a \oplus \mathcal{H}_b.
\end{equation} 
Numerically, a basis for this space can be constructed, from the bases of each individual subsystem, using a block matrix structure,
\begin{equation}
    M = \begin{pmatrix} 
        M_a & \bm{0} \\
        \bm{0}^\mathrm{T} & M_b
    \end{pmatrix},
\end{equation}
where $M_a$ and $M_b$ are $d_a \times d_a$ and $d_b \times d_b$ matrices, respectively, and $\bm{0}$ is a $d_a\times d_b$ matrix. The above structure can be implemented in \texttt{python} using the following script. For more information on tensor products, direct sums, and irreducible representations, see Ref.~\cite{Cohen1978}.

% ----- code for block-matrix -----
\code{https://github.com/frnq/qme/blob/main/python/block_matrix.py}{Composing a block-matrix operator}{code:block_matrix}{python}{Scripts/python/block_matrix.txt}

\subsection{Schr\"odinger and von Neumann equations}
\label{ss:schrodinger_von_neumann}

When studying the dynamics of quantum systems using the density operator representation, Schr\"odinger's equation~\eqref{eq:schroedinger} becomes,
\begin{equation}
    \label{eq:von_neumann_equation}
    \dot{\rho}(t) = - \frac{i}{\hbar}[H,\rho(t)], 
\end{equation}
known as the \textit{von Neumann}\footnote{Or \textit{Liouville-von Neumann} equation.} equation, where $H$ is the Hamiltonian of the system (which can be time-dependent), $\dot{\rho} = \partial_t\rho$, and $[\cdot,\cdot]$ is the commutator~\cite{Breuer2002}. In general, the solution to this equation is given by some unitary operator $U(t;t_0)$ that propagates the state of the system from some initial time $t_0$ to some time $t$,
\begin{equation}
    \label{eq:solution}
    \rho(t) = U(t;t_0) \rho(t_0) U(t;t_0)^\dagger,
\end{equation}
where $\dagger$ is the conjugate transpose (\textit{adjoint)}. If $H$ is time-independent the solution is given by $U(t;t_0) = \exp[-i H (t-t_0)/\hbar]$ and can be reduced to $U(\tau) = \exp[-i H \tau/\hbar]$ for all $t,t_0$ such that $\tau = t-t_0$. See Ref.~\cite{Joachain1975,Breuer2002} for more on the solution $U$ for time-dependent Hamiltonian using time-ordering operators and the Dyson series.

\subsubsection{Open quantum systems}
\label{sss:open_quantum_systems}
The focus of this tutorial is the dynamics of systems that interact with their surrounding environment. These can be seen as composed of a system of interest $\mathrm{S}$ and an environment $\mathrm{E}$ that is usually large, uncontrollable, or not experimentally accessible~\cite{Breuer2002}. The dynamics of the full composite system $\mathrm{S}$-$\mathrm{E}$ (or \textit{universe}) follows equation Eq.~\eqref{eq:von_neumann_equation} with Hamiltonian
\begin{equation}
    \label{eq:open_system_hamiltonian}
    H = H_\mathrm{S} + H_\mathrm{E} + H_\mathrm{int},
\end{equation}
where $H_\mathrm{int}$ represents the interaction between the system with Hamiltonian $H_\mathrm{S}$ and the environment with Hamiltonian $H_\mathrm{E}$.

If the solution $U(t;t_0)$ is known, the dynamics of the system $\mathrm{S}$ can be drawn from the state of the universe $\rho$ by tracing over the environment's degrees of freedom,
\begin{equation}
    \rho_\mathrm{S}(t) = \tr_\mathrm{E} \big[ \rho(t) \big].
\end{equation}
However, finding $U$ for large composite systems is often a difficult problem, both numerically and analytically. Instead, we may seek to obtain a prescription for the dynamics of the system's state by performing the partial trace of Eq.~\eqref{eq:von_neumann_equation}, to obtain
\begin{equation}
\label{eq:reduced_von_neumann_equation}
    \dot{\rho}_\mathrm{S}(t) = -\frac{i}{\hbar}\tr_\mathrm{E}\big\{ [H,\rho(t)] \big\}.
\end{equation}
Eq.~\eqref{eq:reduced_von_neumann_equation} provides the starting point for the derivation of density operator master equations such as those reviewed in Secs.~\ref{s:doma} and~\ref{s:bloch-redfield}.

% ----------- Master equations
\section{Density operator master equations}
\label{s:doma}

Density operator master equations are a powerful tool to study the dynamics of quantum systems that interact weakly with their surrounding environment. Originally developed in the field of quantum optics to study light-matter interactions~\cite{Carmichael1999}, they are used to simulate a variety of quantum mechanical phenomena, such as noise models for quantum information processing~\cite{Nielsen2010}, transient emission and absorption spectra of optically active materials~\cite{Buchheit2016}, and electronic and nuclear spin resonance experiments~\cite{Goldfarb2018}.

The power of master equations resides in the choice of ignoring the environment's dynamics, often uncontrollable and inaccessible. By neglecting the environment's degrees of freedom, we can limit the scaling of the computational requirements to a polynomial of $d = \mathrm{dim}\mathcal{H}_\mathrm{S}$, where $\mathcal{H}_\mathrm{S}$ is the system's Hilbert space. In this section we introduce quantum master equations and focus on their numerical implementation and solution, providing direction for further readings. 

\subsection{Introduction to Lindblad master equation}
The paradigmatic example of a density operator master equation is the Gorini-Kossakowski-Sudarshan-Lindblad (GKSL) master equation~\cite{Milz2017}, often known as the \textit{Lindblad} master equation,
\begin{eqbox}
    \begin{equation}
        \label{eq:lindblad_master_equation}
        	\dot{{\rho}}(t) = -\frac{i}{\hbar}[H, {\rho(t)}] + \sum_{k}\gamma_k\bigg(L^\phdagger_k{\rho(t)}L_k^\dagger -\frac{1}{2}\Big\{ L_k^\dagger L_k^\phdagger, {\rho(t)} \Big\}\bigg),
    \end{equation}
\end{eqbox}
\noindent
where $\rho$ is the system's density operator\footnote{From now on we will drop the subscript $\mathrm{S}$ from the system's density operator, unless specified otherwise.}, $H$ is the system Hamiltonian, and $\{L_k\}$ are the \textit{Lindblad} operators\footnote{Also known as \textit{collapse} operators or \textit{jump} operators.} representing some non-unitary processes like relaxation or decoherence that occur at some rates $\{\gamma_k\}$. The operators $[.,.]$ and $\lbrace.,.\rbrace$ denote the commutator and anti-commutator of the operands. Note that, from now on $H$ will represent the system's Hamiltonian, unless specified otherwise.

Like the Hamiltonian generates coherent dynamics, the Lindblad operators\footnote{Formally, the Lindblad operators are dimensionless linear combinations of the basis operators in Liouville space~\cite{Breuer2002}, and therefore the index $k$ in the sum of Eq.~\eqref{eq:lindblad_master_equation} can be limited to $d^2-1$.} generate incoherent transitions in the space of states. Unlike the Hamiltonian, they do not need to be hermitian. For example, a decay transition from some excited state $\ket{e}$ to some ground state $\ket{g}$ is mediated by the Lindblad operator
\begin{equation}
    \label{eq:lindblad_op_example}
    L_\downarrow = \ketbra{g}{e}.
\end{equation}
Indeed, when we apply $L_\downarrow$ to $\ket{e}$ we obtain $\ket{g} = L_\downarrow\ket{e}$. Note that $L_\downarrow^\dagger = \ketbra{e}{g} \neq L_\downarrow$.

Eq.~\eqref{eq:lindblad_master_equation} is used to approximate the evolution of the density operator of a system $\mathrm{S}$ with Hamiltonian $H$ that is weakly coupled to a Markovian (memory-less) environment~\cite{Breuer2002}. The Lindblad master equation is the general form for a completely positive and trace-preserving (CPTP) Markovian and time-homogeneous map for the evolution of the system's density operator $\rho$~\cite{Breuer2002}. More on the motivation for the requirements of CPTP and Markovianity can be found in Refs.~\cite{Breuer2002,Milz2017}. Derivations of Eq.~\eqref{eq:lindblad_master_equation} can be found in Refs.~\cite{Breuer2002,Lidar2019,Manzano2020}. 

\subsection{The Liouville superoperator}
\label{ss:superoperator}

When solving Eq.~\eqref{eq:lindblad_master_equation}, it is convenient to express the master equation in a vector notation, 
\begin{equation}
\label{eq:superop_form}
\dot{\bm{\rho}} = \mathcal{L} \bm{\rho},
\end{equation}
known as \textit{superoperator} or \textit{Liouville} form, 
where $\bm{\rho} = \mathrm{vec}({\rho})$ is the \textit{vectorised} form of ${\rho}$, and $\mathcal{L}$ is the superoperator associated with the generator $\dot{\rho}$ of Eq.~\eqref{eq:lindblad_master_equation}. The matrix associated with the density operator ${\rho}$ can be reshaped into a vector in many equivalent ways\footnote{Column and row ordering are common choices.} resulting in different superoperators. Any reshaping is valid, as long as one keeps track of the ordering in the elements of the superoperator. The following is a \texttt{python} script that illustrates a reshaping via the \texttt{numpy} method \texttt{reshape}:

% ----- code for density matrix reshaping -----
\code{https://github.com/frnq/qme/blob/main/python/vectorising_rho.py}{Vectorising a density matrix}{code:vectorising_rho}{python}{Scripts/python/vectorising_rho.txt}
\noindent 
Similar methods are available in \texttt{Mathematica} and \texttt{MATLAB}. A robust implementation of the reshaping is implemented in \texttt{QuTiP} with the methods \texttt{operator\_to\_vector} and \texttt{vector\_to\_operator}.

\subsubsection{Constructing the Liouville superoperator}
\label{sss:constructing_superoperator}
While the superoperator $\mathcal{L}$ can be constructed ``by hand'' for small systems, it is advisable to have a systematic approach to compile it from some Hamiltonian $H$ and some Lindblad operators $\{L_k\}$. Two common ways are to either follow an index prescription for the superoperator tensor $\rho_{ab} = \sum_{cd}\mathcal{L}_{abcd}\rho_{cd}$, or to use the following linear algebra identity for the column-ordered form of $\mathrm{vec}(\rho)$~\cite{Barnett1990, Byron2012}:
\begin{equation}
\label{eq:row_ordered_identity}
	\mathrm{vec}({A}{X}{B}) = ({B}^\mathrm{T}\otimes{A})\mathrm{vec}({X}).
\end{equation}
To take advantage of the latter, we proceed inserting the identity operator $\mathbb{1}$ into Eq.~\eqref{eq:lindblad_master_equation}
\begin{equation}
\label{eq:lindblad_with_idenity}
\mathbb{1}\dot{{\rho}}\mathbb{1} = -\frac{i}{\hbar}\Big(H{\rho}\mathbb{1} - \mathbb{1}{\rho}H\Big) + \sum_k\gamma_k\bigg(L_k^\phdagger{\rho}L_k^\dagger -\frac{1}{2}\Big(L_k^\dagger L_k^\phdagger{\rho}\mathbb{1} + \mathbb{1}{\rho}L_k^\dagger L_k^\phdagger\Big)\bigg),
\end{equation}
from which the superoperator can be easily constructed using the tensor product structure discussed in Sec.~\ref{ss:composite systems}, and implemented with the \texttt{kron} method in \texttt{python} and \texttt{MATLAB} or the  \texttt{KroneckerProduct} function in \texttt{Mathematica}. Combining Eqs.~\eqref{eq:superop_form},~\eqref{eq:row_ordered_identity} and~\eqref{eq:lindblad_with_idenity}
we obtain
\begin{equation}
\label{eq:liouville_ordered}
\mathcal{L} = -\frac{i}{\hbar}\Big(\mathbb{1} \otimes H - H^\mathrm{T}\otimes\mathbb{1}\Big) + \sum_k\gamma_k \bigg( L_k^*\otimes L_k^\phdagger - \frac{1}{2} \Big(\mathbb{1}\otimes L_k^\dagger L_k^\phdagger + L_k^\mathrm{T} L_k^*\otimes\mathbb{1}\Big)\bigg).
\end{equation}
As discussed in the next sections, the power and advantage of the superoperator form consists in offering a direct pathway to solving Eq.~\eqref{eq:lindblad_master_equation}, based on the solution of a system of linear ordinary differential equations. The following \texttt{python} script implements Eq.~\eqref{eq:liouville_ordered} using \texttt{numpy} arrays. It is worth noting that the rates $\gamma_k$ are here embedded into the Lindblad operators via $L_k \to L'_k = \sqrt{\gamma_k} L_k$ for a simpler implementation. 
% ----- code for superoperators -----
\code{https://github.com/frnq/qme/blob/main/python/superoperator.py}{Constructing the superoperator}{code:superoperator_python}{python}{Scripts/python/superoperator.txt}
\noindent 

\subsection{Steady-state solution}
\label{ss:staeady_state}

Before looking at the dynamics $\rho(t)$ of the density operator, let us go through some methods to obtain the steady-state solution of Eqs.~\eqref{eq:lindblad_master_equation} and~\eqref{eq:superop_form}.

\subsubsection{Using the null space of Liouville superoperator}
Once we have expressed a linear master equation in the superoperator form, we can use the matrix $\mathcal{L}$ to study the behaviour of the system. Of immediate interest is the steady state solution ($\dot{{\rho}} = 0$) which is often measured directly in experiments. To find any steady state solutions we solve for the \textit{null space} of $\mathcal{L}$~\cite{Axler1997}, which is the subspace of all vectors $\bm{\rho}$ that satisfy the equation
\begin{equation}
\label{eq:null_space_equation}
	\mathcal{L}\bm{\rho}=0.
\end{equation}
Numerically, this can be done using the \texttt{NullSpace} function in \texttt{Mathematica}, the \texttt{null} function in \texttt{MATLAB}, or the \texttt{null\_space} method in the \texttt{numpy} library \texttt{scipy}. An analytic solution can also be sought with this approach with \texttt{Mathematica}, or \texttt{SymPy} in \texttt{python}. If there is a unique solution, solving for the null space will provide the corresponding steady state density matrix vector $\bm{\rho}(\infty)$ up to a constant factor, the value of which is given by the original normalization condition $\text{Tr}({\rho})=1$.
 
If there are multiple solutions, solving for the null space will give linearly independent vectors. In such case, the steady state depends on the initial state of the system. For example, let us consider a two-level system Hamiltonian\footnote{In Sec.~\ref{ss:two_level_atom_with_field} we outline how to obtain (\ref{eq:tl_hamiltonian}) for a two-level atom interacting with an electric field. } $H$ with energy splitting $\Delta$ and coupling $\Omega$, and Lindblad operators $L_\downarrow$ and $L_0$ associated with spontaneous relaxation and dephasing, respectively,
\begin{equation}
\label{eq:tl_hamiltonian}
	{H} = \hbar\begin{pmatrix}
	0 & \Omega\\
	\Omega & \Delta
	\end{pmatrix}, \;\;\;\ L_\downarrow = \sqrt{\gamma_\downarrow} \begin{pmatrix}
    0 & 1\\
    0 & 0
    \end{pmatrix}, 
    \;\;\;\ L_0 = \sqrt{\gamma_0} \begin{pmatrix}
    1 & 0\\
    0 & 1
    \end{pmatrix},
\end{equation}
where $\gamma_\downarrow$ and $\gamma_0$ are the rates associated with relaxation and dephasing.
In the following script, we construct $\mathcal{L}$ in \texttt{python} and solve for its null space for the case of (i) relaxation and no-driving limit $\Omega,\gamma_0=0$, and (ii) dephasing and driving, with no relaxation, $\gamma_\downarrow = 0$.

\code{https://github.com/frnq/qme/blob/main/python/null_space.py}{Solving for the null space of superoperator {\normalfont\textsf{(requires script~\ref*{code:superoperator_python})}}}{code:null_space}{python}{Scripts/python/null_space.txt}
\noindent
In the limit of relaxation and no-driving, there is a unique steady state $\bm{\rho}(\infty) = \begin{pmatrix} 1,0,0,0 \end{pmatrix}^\textrm{T}$, which is the ground state of the system, as expected for a two-state system undergoing spontaneous relaxation with no driving field. Instead, for the case of dephasing and driving, the \texttt{null} function returns two vectors that span the two-dimensional linear subspace associated with the null space of $\mathcal{L}$. In this case, the specific steady state depends on the choice of initial state. See Sec.~\ref{a:examples} for a \texttt{MATLAB} implementation of the method used in script~\ref{code:null_space}.

\subsubsection{Algebraic solution}
The steady state solution ${\rho}(\infty)$ for both linear and non-linear\footnote{A non-linear generator is such that $\mathcal{L}(\rho)$ depends on the state of the system. See Ref.~\cite{Breuer2002} for more on non-linear density operator master equations.} generators can be obtained by solving Eq.~\eqref{eq:lindblad_master_equation} for $\dot{{\rho}} = 0$ algebraically (or symbolically). In \texttt{python}, this can be done using the \texttt{solve} method of the \texttt{SymPy} library, as demonstrated in the script below for the case of $\Omega=0, \gamma_0 = 0$, with respect to Eq.~\eqref{eq:tl_hamiltonian}.
% -------- code: symbolic solution for the null space --------
\code{https://github.com/frnq/qme/blob/main/python/symbolic_solution.py}{Symbolic steady-state solution}{code:symbolic_matrix_solving_python}{python}{Scripts/python/symbolic_solution.txt}

\noindent
Algebraic solutions can also be sought in \texttt{MATLAB} with \texttt{solve}, or in \texttt{Mathematica}, using the \texttt{Solve} method. These provide a more straightforward approach to solving symbolic matrix equations. A \texttt{MATLAB} implementation of script~\ref{code:symbolic_matrix_solving_python} can be found in the Appendix in script~\ref{code:symbolic_matrix_solving}. 

\subsection{Solving the dynamics of the system}
\label{ss:solving_dynamics}

Let us now discuss how to solve Eq.~\eqref{eq:lindblad_master_equation} in order to obtain the state of the system $\rho(t)$ at any time $t$ from a given initial condition $\rho_0 = \rho(t_0)$. Let us represent the solution with the dynamical map $\rho(t) = \Lambda(t;t_0)[\rho_0]$. For linear, time-independent generators $\mathcal{L}$, the solution to Eq.~\eqref{eq:superop_form} can be obtained by calculating the following matrix exponential~\cite{Breuer2002},
\begin{equation}
    \label{eq:matrix_exp_superoperator}
    \bm{\rho}(t) = \exp\big[\mathcal{L} (t-t_0)\big]\bm{\rho}(t_0).
\end{equation}
The operator $P(t;t_0) = \exp\big[\mathcal{L} (t-t_0)\big]$ is called the propagator of the evolution. From the propagator, we can obtain the solution $\rho(t)$ by reshaping $\bm{\rho}(t)$ as described earlier in this section. See Ref.~\cite{Lidar2019} for details on how to obtain the dynamical map $\Lambda$ from $P$ using, for example, a Kraus operators representation. Eq.~\eqref{eq:matrix_exp_superoperator} is implemented in \texttt{python} using \texttt{scipy} in the following script, with the result shown in Fig.~\ref{fig:super_matrix_exp}.
% ------- code: super matrix exponential  --------
\code{https://github.com/frnq/qme/blob/main/python/super_matrix_exp.py}{Propagator using matrix exponential {\normalfont\textsf{(requires script~\ref*{code:superoperator_python})}} }{code:super_matrix_exp}{python}{Scripts/python/super_matrix_exp.txt}
\noindent
See script~\ref{code:solution_matlab_ode} for an implementation of Eq.~\eqref{eq:matrix_exp_superoperator} using \texttt{MATLAB}.
\begin{figure}
    \centering
    \includegraphics{Figures/propagation.pdf}
    \caption{Propagation using matrix exponential (\texttt{expm} from \texttt{numpy}), obtained using script~\ref{code:super_matrix_exp}. The propagated state $\rho(t)$ is obtained using Eq.~\eqref{eq:matrix_exp_superoperator}, and compared to the solution obtained using a finite-difference method (\texttt{mesolve} from \texttt{QuTiP}), implemented in script~\ref{code:mesolve_example}.}
    \label{fig:super_matrix_exp}
\end{figure}

It is worth pointing out that the approach used in script~\ref{code:super_matrix_exp} is by no mean optimised, and calculates a new propagator for every time step in the considered time domain. When working with evenly-spaced time steps we can reduce the computational cost by exploiting the composition rule of dynamical semigroups, as discussed in Sec.~\ref{ss:semigroup}. In \texttt{QuTiP}, instead, the solution is obtained using the sophisticated and powerful \texttt{mesolve} method, which by default uses \texttt{scipy}'s numerical integration library \href{https://docs.scipy.org/doc/scipy/tutorial/integrate.html}{\texttt{integrate}}. 

\subsubsection{Singular value decomposition of the Liouville superoperator}\label{sss:td:superop}

The superoperator $\mathcal{L}$ is generally a complex, non-Hermitian matrix. For this reason a spectral decomposition of $\mathcal{L}$ is not always guaranteed, that is, $\mathcal{L}$ may not admit the diagonal representation $\mathcal{L} = V D V^{-1}$. However, $\mathcal{L}$ always admits a singular value decomposition\footnote{The generalisation of the eigenvalue decomposition.} (SVD), and therefore can be represented in terms of its left and right-singular vectors, $\bm{L}_k$ and $\bm{R}_k$, respectively, and the set of complex singular values $\{\lambda_k\}$, that abide by the following relationships~\cite{Sacha2020},
\begin{align}
    \mathcal{L} \bm{R}_k &= \lambda_k \bm{R}_k, \label{eq:right_eig_vec}\\
    \bm{L}^\dagger_k \mathcal{L} &= \lambda_k \bm{L}^\dagger_k.  
    \label{eq:left_eig_vec}
\end{align}
Notice that both $\bm{R}_k$ and $\bm{L}_k$ are column vectors, hence $\bm{L}^\dagger_k$ is a row vector. Each left and right-singular vectors can be normalized via,
\begin{align}
    \hat{\bm{R}}_k &= \bm{R}_k\big/\sqrt{\bm{L}^\dagger_k\bm{R}_k}
    \label{eq:Norm_r}\\
    \hat{\bm{L}}^\dagger_k &= \bm{L}^\dagger_k\big/\sqrt{\bm{L}^\dagger_k\bm{R}_k}.
    \label{eq:norm_Ldag}
\end{align}
The normalized singular vector pairs then follow the usual orthonormalisation condition~\cite{Sacha2020},
\begin{equation}
\label{eq:bi-orthonormality}
    \hat{\bm{L}}^\dagger_i  \hat{\bm{R}}_j^\phdagger = \delta_{ij}. 
\end{equation}
 
The solution of Eq.~\eqref{eq:superop_form} for a system with time independent Liouville superoperator $\mathcal{L}$ can now be expressed as follows,
\begin{equation}
\label{eq:solution_svd}
    \bm{\rho}(t) = \sum_{k=1}^{d^2}  \hat{\bm{L}}^\dagger_k\bm{\rho}(t_0) \hat{\bm{R}}_k e^{\lambda_k (t-t_0)}
\end{equation}
where $d$ is the dimension of the Hilbert space. 

The advantage of expressing the time evolution in form of Eqs.~\eqref{eq:matrix_exp_superoperator} and~\eqref{eq:solution_svd} is that it is exact (when the singular values are found exactly) for all times and therefore does not depend on the step size or other operational details of the integration routine used to solve the differential equation. In the following \texttt{python} code, we use \texttt{scipy} to obtain the temporal solutions for a system with,
\begin{equation}
\label{eq:singular_value_problem}
	H = \hbar\begin{pmatrix}
	0 & \Omega\\
	\Omega & 0
	\end{pmatrix},
	\quad \quad
    L = \begin{pmatrix}
    0 & 1\\
    1 & 0
    \end{pmatrix}, \quad \textrm{and}\quad
    \bm{\rho}(0) = \begin{pmatrix}0, 0, 0, 1\end{pmatrix}^\textrm{T}
\end{equation}
using the singular value decomposition of $\mathcal{L}$.

% ------- code: python solution ---------
\code{https://github.com/frnq/qme/blob/main/python/dynamics_norm_svd.py}{Solution using normalized singular vectors {\normalfont\textsf{(requires script~\ref*{code:superoperator_python})}}}{code:temporal_python_solution}{python}{Scripts/python/dynamics_norm_svd.txt}

\noindent
The solution is shown in Fig.~\ref{fig:temporal_python_solution}. A \texttt{MATLAB} implementation of this method can be found in the Appendix in script~\ref{code:temporal_solution}.

\begin{figure}
    \centering
\includegraphics{Figures/solution_singluar_vectors.pdf}
    \caption{Dynamics of the state $\rho(t)$ for Eq.~\eqref{eq:singular_value_problem}, solving the Lindblad master equation using the normalised superoperator singular vectors, obtained with script~\ref{code:temporal_python_solution}. }
    \label{fig:temporal_python_solution}
\end{figure}

\subsubsection{Time-dependent generators}
\label{sss:time-dependent_generators}
If the Hamiltonian or the decoherence terms depend on time, Eqs.~\eqref{eq:superop_form} is generalised to
\begin{equation}
    \label{eq:time-dependent_generator}
    \dot{\bm{\rho}} = \mathcal{L}(t) \bm{\rho},
\end{equation}
where the Liouville generator $\mathcal{L}(t)$ now explicitly depends on time. In this case the solution of Eq.~\eqref{eq:matrix_exp_superoperator} is not valid. The general solution of Eq.~\eqref{eq:time-dependent_generator} is given by
\begin{equation}
    \label{eq:time-ordered_solution}
    \bm{\rho}(t) = \mathcal{T}\big\{ \exp[\textstyle\int_0^t ds \mathcal{L}(s)] \big\} \bm{\rho}(t_0),
\end{equation}
where $\mathcal{T}$ is the time-ordering operator, analogue to the Dyson series for time-dependent Hamiltonians and wavefunction propagation~\cite{Breuer2002,Schnell2020}. Eq.~\eqref{eq:time-ordered_solution} can be approximated, for instance, by means of a sequence of step-wise time-independent generators, before resorting to other means like numerical integration. 

If the generator $\mathcal{L}(t)$ is approximately \emph{piecewise time-independent}, then Eq.~\eqref{eq:matrix_exp_superoperator} can be applied to each time slice, using the result of the previous slice to provide the input state for the next slice. This scenario is common in many optical and spin resonance experiments. For example, it can be used to
compute the effect of applying a laser pulse resonant with an atomic transition, to then observing the behaviour of the system while the pulse is on and immediately after it has been turned off. 

For example, let us consider a system with Hamiltonian $H = H_0 + v(t) H_1$, where $H_0 = \omega_0\sigma_z/2$, $H_1 = \omega_0\sigma_x/2$, $v(t) = \cos(\omega t)$, and a Lindblad dephasing operator $J = \ketbra{g}{g} =(\mathbb{1}-\sigma_z)/2$, with dephasing rate $\gamma$. The generator $\mathcal{L}(t) = \mathcal{L}_0+\mathcal{L}_1(t)$ can be split into a time-independent part $\mathcal{L}_0$, associated with $H_0$ and $L_0$, and a time-dependent part $\mathcal{L}_1(t)$. To reduce the computational cost when propagating this system, we can update the propagator by updating only the time-dependent part.
The following \texttt{python} script generalises the solution of Eq.~\eqref{eq:matrix_exp_superoperator} to the case of time-dependent generators, by updating the superoperator at each time $t$; the solution is shown in Fig.~\ref{fig:td_propagation}. Note that for this approach to be accurate, the time step $\delta t$ has to be sufficiently small so that $v(t+\delta t)\approx v(t)+\mathcal{O}(\delta t^2)$. For rapidly varying time-dependent Hamiltonians other methods are required. If $H(t)$ is periodic, a solution can be found using an effective time-independent Hamiltonian, obtained using Floquet theory, as discussed in Sec.~\ref{s:oscillatory}.

% ------- code: time-dependent generator ---------
\code{https://github.com/frnq/qme/blob/main/python/time_dependent_generator.py}{Solution of time-dependent generator {\normalfont\textsf{(requires script~\ref*{code:superoperator_python})}}}{code:time_dependent_generator}{python}{Scripts/python/time_dependent_generator.txt}
\begin{figure}
    \centering
    \includegraphics{Figures/td_propagation.pdf}
    \caption{Solution of time-dependent generator using piecewise time-independent propagator, for the system considered in  script~\ref{code:time_dependent_generator}. The evolution is generated by a time-dependent Hamiltonian $H(t) = \omega_0\sigma_z/2 + v(t)\omega_0\sigma_x/2$, with $v(t)=\cos(\omega t)$, and a time-independent Lindblad dephasing operator $J = (\mathbb{1}-\sigma_z)/2$, associated with rate $\gamma$. The solution is obtained for $\omega_0 = 1$, $\omega =  3$ and $\gamma = 0.3$.}
    \label{fig:td_propagation}
\end{figure}

Note that in practice, especially when using theoretical system parameters, it is often possible to get exact cancellations which may have no physical grounding but can result in degenerate eigenvectors. While there are mathematical techniques which deal with these situations, it is often easier to just add an infinitesimal (numerically of order machine precision) imaginary term $i\varepsilon$, $\varepsilon\ll1$, to each element of the matrix. This can remove the degeneracy, even if the term is made sufficiently small to have no perceivable effect on the resulting calculations. 


\subsubsection{Propagation via semigroup composition}
\label{ss:semigroup}

The dynamical maps generated by a linear Markovian quantum master equation like Eq.~\eqref{eq:lindblad_master_equation} are a family of single-parameter maps $\Lambda_t$ that have the following composition property,
\begin{equation}
    \label{eq:quantum_dynamical_semigroup}
    \Lambda_s \circ \Lambda_t = \Lambda_{s+t}, \;\;\;\; t,s \geq 0,
\end{equation}
and, therefore, are known as a \textit{quantum dynamical semigroup} (QDS).
The above can also be expressed as $\Lambda_s[\Lambda_t[\rho]]=\Lambda_{s+t}[\rho]$. For more on QDS see Ref.~\cite{Breuer2002}. Eq.~\eqref{eq:quantum_dynamical_semigroup} can be expressed in the superoperator form as
\begin{equation}
    \label{eq:superoperator_semigroup}
    P(s)P(t) = P(s+t), \;\;\;\; t,s \geq 0,
\end{equation}
which follows directly from the properties of the exponential and the fact that $[\mathcal{L}s,\mathcal{L}t]=0$. Note that the above does not generally hold for time-dependent $\mathcal{L}(t)$ and non-linear generators $\mathcal{L}(\rho(t))$.

When propagating a system in time over an evenly-spaced time set $\{k\delta t\}_{k=1}^{m}$ we can exploit the composition rule of dynamical semigroups to vastly reduce the computational cost of propagation. Instead of calculating a new propagator $P(t_k)$ for each time step $t_k = t_0 +k\delta t$, we can calculate a single propagator $P_1 = P(\delta t)$ and obtain all the others using
\begin{equation}
    \label{eq:propagator_semigroup}
    P(t_k) = \prod_{j=1}^{k} P_1 = P_1^k.
\end{equation}
The following \texttt{python} script implements Eq.~\eqref{eq:propagator_semigroup}, and the results are shown in Fig.~\ref{fig:propagation_semigroup}.
% -------- code: semigroup propagator ---------
\code{https://github.com/frnq/qme/blob/main/python/semigroup_propagator.py}{Propagation using dynamical semigroup composition {\normalfont\textsf{(requires script~\ref*{code:superoperator_python})}}}{code:semigroup_propagator}{python}{Scripts/python/semigroup_propagator.txt}

This approach is particularly useful when propagating for very long times or when using large time-steps, in which cases \texttt{scipy}'s \texttt{integrate} methods usually tend to accumulate large numerical errors. When propagating over several orders of magnitude, it may be convenient to break each timescale into evenly-spaced time sets to resolve the details of different dynamical transients. For example, this is useful when looking at dynamics from the femtosecond to the nanosecond timescales.
\begin{figure}[h]
    \centering
    \includegraphics{Figures/propagation_semigroup.pdf}
    \caption{Propagation using semigroup decomposition, obtained using script~\ref{code:semigroup_propagator}, compared to that obtained by computing a new propagator for each time-step.}
    \label{fig:propagation_semigroup}
\end{figure}

\subsubsection{Baker–Campbell–Hausdorff \& Zassenhaus formula}
\label{ss:zassenhaus_BCH}

Hamiltonians and superoperators are often sums of two or more terms, such as $W = U+V$. As briefly noted in Sec.~\ref{ss:semigroup}, when the terms commute with each other $[U,V]=0$, the solution can be obtained from the composition of individual terms. For example, let $\mathcal{L} = \mathcal{L}_1 + \mathcal{L}_2$, with $[\mathcal{L}_1 ,\mathcal{L}_2] = 0$, then
\begin{equation}
    \label{eq:composition_commuting}
    P(t) = \exp(\mathcal{L} t) = \exp(\mathcal{L}_1 t)\exp(\mathcal{L}_2 t).
\end{equation}
Instead, when considering pairs of non-commuting operators $[X,Y]\neq 0$, we have $\exp(X+Y)\neq\exp(X)\exp(Y) = \exp(Z)$. The solution to the latter equation for $Z$ is known as the Baker-Campbell-Hausdorff (BCH) formula~\cite{Rossmann2002}, and reads,
\begin{equation}
    \label{eq:BCH}
    Z = X + Y + \frac{1}{2}[X,Y] + \frac{1}{12}\bigg( [X,[X,Y]] + [Y,[Y,X]]\bigg) + \cdots.
\end{equation}
The BCH solution finds application when used in the Zassenhaus formula, which allows us to decompose a matrix exponential $\exp[(X+Y)t]$, where $t$ is a scalar parameter, in terms of a product series,
\begin{equation}
    \label{eq:zassenhaus}
    \exp[(X+Y)t] = \exp[Xt]\exp[Yt]\exp\bigg[-\frac{1}{2}[X,Y]t^2\bigg]\exp\bigg[\frac{1}{3}\Big([Y,[X,Y]]+\frac{1}{2}[X,[X,Y]]\Big)t^3\bigg]\cdots.
\end{equation}
The formula becomes useful when the product series can be truncated or approximated to a certain set of terms. This is for example particularly useful when the generator is time-dependent $\mathcal{L}_t$ and $[\mathcal{L}_t,\mathcal{L}_s]\neq 0$: By choosing a sufficiently small time step $\delta t$ such that $s = t+\delta t$, the series of Eq.~\eqref{eq:zassenhaus} can be truncated to terms in $\mathcal{O}(\delta t^m)$ for some $m>1$, as discussed in the next section.

\subsubsection{Suzuki-Trotter expansion}
\label{ss:suzuki-trotter}

A consequence of the Zassenhaus formula is that, for \textit{small} time steps $\delta t$, Eq.~\eqref{eq:zassenhaus} can be truncated to the first order in $\delta t$ with errors of the order of $O(\delta t^2)$
\begin{equation}
    \label{eq:limit_zassenhaus}
    \exp[(X+Y)\delta t] = \exp[X\delta t]\exp[Y\delta t] + O(\delta t^2).
\end{equation}
This can be used to obtain the solution for long times using the product series,
\begin{equation}
    \exp[(X+Y)\delta t] = \lim_{n \to \infty} \bigg[ \exp\bigg(X \frac{t}{n}\bigg)\exp\bigg(Y\frac{t}{n}\bigg) \bigg]^n,
\end{equation}
also known as \textit{Suzuki–Trotter expansion} or \textit{Lie product formula}~\cite{Rossmann2002}. This approach is particularly useful when studying the dynamics of interacting many body systems or time-dependent generators. The following \texttt{python} script uses the Suzuki–Trotter expansion to propagate a system by separating the contribution of the two non-commuting superoperators. The results are shown in Fig.~\ref{fig:suzuki_trotter}.
% -------- code: suzuki-trotter ---------
\code{https://github.com/frnq/qme/blob/main/python/suzuki.py}{Propagation using Suzuki-Trotter expansion {\normalfont\textsf{(requires script~\ref*{code:superoperator_python})}}}{code:suzuki}{python}{Scripts/python/suzuki.txt}

\begin{figure}[h]
    \centering
    \includegraphics{Figures/suzuki_trotter.pdf}
    \caption{Propagation using Suzuki-Trotter expansion for different amounts $m = 50,100,1000$ of time steps, obtained using script~\ref{code:suzuki}.}
    \label{fig:suzuki_trotter}
\end{figure}

\subsubsection{Numerical solution with finite-difference methods}

While the matrix exponential is a powerful tool to obtain exact solution of Eq.~\eqref{eq:superop_form}, it may be less computationally expensive to compromise some precision in favor of less demanding time and memory requirements. Not only finite-difference methods can prove efficient at solving density operator master equations, but they can also be used to solve the dynamics of non-linear and time-dependent generators.
In this case, the approach consists in solving the set of coupled differential equations obtained by element-wise comparison of the left and right hand sides of Eq.~\eqref{eq:lindblad_master_equation}. 

The following script is a continuation of script~\ref{code:temporal_python_solution}, and solves the dynamics of the same two-level system using the 4-5th order Runge-Kutta differential equation method. The method is implemented using the initial-value problem solver \texttt{solve\_ivp} from \texttt{scipy.integrate} library for \texttt{python}. The solution is shown in Fig.~\ref{fig:solution_rk45}. A \texttt{MATLAB} implementation of the same code can be found in script~\ref{code:solution_matlab_ode} in the Appendix.

% ------- code: runge kutta method ---------
\code{https://github.com/frnq/qme/blob/main/python/dynamics_finite_diff.py}{Solution with finite-difference method {\normalfont \textsf{(requires script~\ref*{code:temporal_python_solution})}}}{code:solution_python_ode}{python}{Scripts/python/dynamics_finite_diff.txt}

\begin{figure}[h]
    \centering
    \includegraphics{Figures/solution_rk45.pdf}
    \caption{Propagation using finite-difference approach, based on the 4-5th order Runge-Kutta method. The solution is obtained using script~\ref{code:solution_python_ode}, and compared to that obtained using script~\ref{code:temporal_python_solution}, based on the singular value decomposition.}
    \label{fig:solution_rk45}
\end{figure}

\subsubsection{Solution using the stochastic wavefunction method}
\label{ss:swfm}

Since the amount of complex floating point numbers required to represent superoperators like $\mathcal{L}$ and $P$ scales as $d^4$, memory may become an issue for large systems. To circumvent this problem we can propagate a density operator using the stochastic wavefunction method~\cite{Carmichael1999}, also known as \textit{Monte Carlo wavefunction} method or \textit{master equation unravelling}. Originally developed for quantum optics, the method is an adaptation of the kinetic Monte Carlo method~\cite{Bortz1975} to the solution of Eq.~\eqref{eq:lindblad_master_equation}. 

Instead of propagating a density operator solving Eq.~\eqref{eq:superop_form}, the method provides a procedure to propagate a state vector $\ket{\psi_0}$ under the influence of some generator $\mathcal{L}$, by sampling a sufficiently large amount $N$ of stochastic trajectories $\Psi_j = \{\ket{\psi_j(t)}\}$, to then obtain the time-evolved density operator $\rho(t)$ by averaging over them,
\begin{equation}
    \label{eq:MCWF_rho_average}
    \rho(t) = \sum_{j=1}^{N} \ketbra{\psi_j(t)}{\psi_j(t)}.
\end{equation}

Let $H$ be the Hamiltonian of the system, and $\{L_k\}_{k=1}^{M}$ a collection of Hermitian\footnote{The method can be implemented with non-Hermitian Lindblad operators too, upon some adaptations to avoid division-by-zero errors in the normalisation steps.} Lindblad operators. In the simplest form of the method, each trajectory $\Psi_j$ is sampled according to the following steps:
\begin{enumerate}
    \item The probabilities associated with any of the $k$ incoherent transitions mediated by the $L_k$ jump operators is calculated, 
    \begin{equation}
        \label{eq:MCWF_prob_array}
        \delta p_k = \delta t \braket{\psi(t)|L_k^\dagger L_k^\phdagger|\psi(t)}\geq0,
    \end{equation}
    with $\delta p = \sum_{k=1}^{M}\delta p_k$.
    \item A uniform random number $u\in(0,1]$ is sampled. 
        \begin{enumerate}
            \item If $\delta p < u$, then no jump occurs and the state $\ket{\psi(t)}$ at time $t$ is evolved by means of the non-Hermitian effective Hamiltonian $H_\textrm{eff} = H - i\hbar \sum_{k=1}^M L_k^\dagger L_k^\phdagger /2$,
            \begin{equation}
                \label{eq:MCWF_non-hermitian_prop}
                \ket{\widetilde{\psi}(t+\delta t)} = \bigg(1-\frac{i}{\hbar} H_\textrm{eff}^\dagger \delta t\bigg)\ket{\psi(t)},
            \end{equation}
            where $\ket{\widetilde{\psi}}$ indicates that the state vector may not be normalised. 
            \item If $\delta p\geq u$, a jump occurs. A new uniform random number $u'\in(0,1]$ is sampled. The event that occurs is chosen finding the first $k$ such that $Q_k > u'$, where $Q_k = \sum_{j=1}^{k}\delta p_j/\delta p$. The state is propagated to be
            \begin{equation}
                \label{eq:MCWF_jump occurs}
                \ket{\widetilde{\psi}(t+\delta t)} = L_k \ket{\psi(t)}.
            \end{equation} 
        \end{enumerate}
    \item The state is normalised $\ket{\widetilde{\psi}(t+\delta t)}\to\ket{\psi(t+\delta t)} = \ket{\widetilde{\psi}(t+\delta t)}/\sqrt{\braket{\widetilde{\psi}(t+\delta t)|\widetilde{\psi}(t+\delta t)}}$.
\end{enumerate}
Note that in this approach no superoperator is assembled and no matrix exponential is calculated. Furthermore, since the trajectories $\Psi_j$ are completely independent of each other, this method can be trivially parallelised by running $N$ trajectories over $N$ different processing nodes to cut down the computational time by a factor of $N$.

A \texttt{python} implementation is presented in the script below, for a two-level system with $H = \sigma_z$ and Lindblad operators $\{ \sigma_z/2, \sigma_x/5\}$, with initial state $\ket{\psi}= (\ket{0}+\ket{1})/\sqrt{2}$ in the $\sigma_z$ basis. The results are shown in Fig.~\ref{fig:mcwf}. Note that the time-step $\delta t$ can be chosen to be a fraction of some operator norm of the Hamiltonian, such that $\delta t \ll \| H\|_\mathrm{op}^{-1}$. An equivalent \texttt{Mathematica} implementation can be found in script~\ref{code:mcwf} in the Appendix. A robust implementation of the stochastic wavefunction method is also available in \texttt{QuTiP}.
% -------- code: mcwf ---------
\code{https://github.com/frnq/qme/blob/main/python/mcwf.py}{Propagation using stochastic wavefunction method}{code:mcwf_python}{python}{Scripts/python/mcwf.txt}

\begin{figure}
    \centering
    \includegraphics{Figures/mcwf.pdf}
    \caption{Propagation with stochastic wavefunction method with $N=10,100,1000$ trajectories, using script~\ref{code:mcwf_python} with \texttt{sample} $=N$. The stochastic wavefunction solution approaches the exact one in the limit of large $N$. Here, the solution is compared to that obtained with \texttt{QuTiP}'s finite-difference method \texttt{mesolve}.}
    \label{fig:mcwf}
\end{figure}

\subsubsection{Sparse solvers}
\label{ss:sparse_solvers}
When dealing with very large systems, it is worth thinking about sparseness of superoperator and states, since finding its singular value decomposition may become prohibitively expensive.
A number of different techniques can be used to treat sparse and large superoperators, such as
\begin{itemize}
    \item using methods for sparse arrays (\texttt{SparseArray} in \texttt{Mathematica}), such as null-space solvers. A library of linear algebra methods for sparse arrays for \texttt{MATLAB} is available at Ref.~\cite{Davis2006}; 
    \item using Krylov subspace methods to solve for $\mathrm{exp}(\mathcal{L}t)\bm{\rho}_0$ directly~\cite{Vo2017}; Packages \href{https://github.com/weinbe58/expokitpy}{\texttt{expokitpy}} and \href{https://krypy.readthedocs.io/en/latest/#}{\texttt{KryPy}}~\cite{Gaul2014} offer Krylov method implementations for \texttt{python}.
    \item taking the action of the exponential on a given sparse initial state. In \texttt{Mathematica} this can be done with \texttt{MatrixExp} as follows,
    \begin{codeline}{Mathematica}
        \footnotesize
            \begin{minted}[linenos=false]{Mathematica}
                Pt = MatrixExp[M t, psi]
            \end{minted}
    \end{codeline}
    \item using the Arnoldi method~\cite{knap2011emission}. This can be done in \texttt{Mathematica} using the \texttt{Eigensystem} function in combination with \texttt{"Arnoldi"},
        \begin{codeline}{Mathematica}
        \footnotesize
            \begin{minted}[linenos=false]{Mathematica}
                {evals,evecs} = Eigensystem[M t, k, Method -> "Arnoldi"] 
            \end{minted}
        \end{codeline}
    where \texttt{k} represents the index of the eigenvalue (or singular value) to be calculated.
\end{itemize}
However, sometimes the simplest option may be to implement a finite difference method like Runge-Kutta with sparse linear algebra, as it is often just as fast as more sophisticated methods.

\subsection{Correlation functions}
\label{ss:correlation_functions}
Correlation functions measure the relationship between microscopic quantities across time, space and other observables. In statistical mechanics, they are used to calculate the ensemble properties of stochastic processes, and determine the degree of order or randomness in a system. For example, the effect of atmospheric turbulence on the propagation of light beams can be modelled from the correlation functions $C(t,t',\bm{r},\bm{r}') = \braket{n(t,\bm{r})n(t',\bm{r}')}$ of the refractive index $n(t,\bm{r})$~\cite{Paterson2005}. Similarly, the magnetic properties of materials can be inferred from the spatial correlation functions between spins~\cite{Gutzwiller1963}.

In quantum stochastic processes, correlation functions are used to determine the magnitude of decoherence and relaxation processes, as we will discuss in depth in Sec.~\ref{s:bloch-redfield}. The macroscopic properties of a variety of systems can be indeed calculated from the correlation functions of their microscopic features. Of particular importance are, emission and absorption spectra in light-matter interaction (see Sec.~\ref{sss:emiss_abs}), noise power spectra and relaxation rates, bunching and anti-bunching statistics of photons~\cite{Hennrich2005}, electrons~\cite{Emary2012} and other particles. Here, we will examine the basics of correlation functions and show how these can be calculated from the master equation governing the evolution of the density operator. We will then apply these results to calculate the emission spectrum in a simple example of a two-level system interacting with the electromagnetic field. 

\subsubsection{Quantum regression theorem}\label{sss:quantum_regression_theorem}
Linear systems are amply studied in physics because of their simplicity and exact solvability. The equations of motion of the averages of the operators of such systems are often linear, as for the case of Eq.~\ref{eq:superop_form}. 
For these systems, it can be shown that the averages of their two-time correlation functions obey exactly the same equations of motion. This result, first derived by \emph{Lax}, is known as the \emph{quantum regression theorem}~\cite{ficek2005quantum,gardiner2004quantum}, and it provides a method for calculating any two-time correlation function $\braket{A(t)B(t')}$, i.e., involving any two observables at different points in time, for a system whose dynamics are prescribed by a quantum master equation $\dot{\rho} = \mathcal{L}_t[\rho]$~\cite{ficek2005quantum}. 

Suppose that for a certain set of operators $\{A_i\}$, the linear master equation~\eqref{eq:superop_form} yields the following closed system of linear ordinary differential equations to their averages~\cite{Breuer2002},
\begin{equation} \label{Eq:QRT1}
    \frac{d}{d t}\langle A_i(t)\rangle = \sum_j G_{ij}\langle A_j(t) \rangle,
\end{equation}
for some coefficients $G_{ij}$.
Then, their two-point correlation functions
\begin{equation}
    \label{eq:corre_functions}
    \langle A_i(t+\tau)A_l(t)\rangle  = \mathrm{Tr}\big[ A_i \Lambda(t+\tau;t)[A_l\rho(t)]\big],
\end{equation}
where $\Lambda(t;t_0)$ is the dynamical map from time $t_0$ to time $t$, associated with the the master equation $\dot{\rho} = \mathcal{L}_t[\rho]$,
observe the same dynamics, 
\begin{equation} \label{Eq:QRT2}
	\frac{d}{d t}\langle A_i(t+\tau) A_l(t)\rangle = \sum G_{ij} \langle A_j(t+\tau)A_l(t) \rangle.
\end{equation}
Note how the right-hand side of (\ref{eq:corre_functions}) corresponds to the average of $A_i$ at time $t+\tau$ with the choice of initial density operator $\rho\to A_l\rho(t)$~\cite{gardiner2004quantum}.

Any two-time correlation function $\langle A(t+\tau)B(t)\rangle$ can then be simplified using (\ref{eq:corre_functions}) as \cite{Johansson2012},
\begin{align}\label{Eq:TT_corr_simp}
	\langle A(t+\tau)B(t)\rangle &= \mathrm{Tr}[A \Lambda(t+\tau;t)[B\rho(t)]\rbrace], \\ 
 &= \mathrm{Tr}[A \Lambda(t+\tau;t)[B \Lambda(t;0)[  \rho(0)]]].
\end{align}
When calculating $\langle A(t+\tau)B(t)\rangle$ numerically, we can first obtain $\rho(t) = \Lambda(t;0)[\rho(0)]$ with $\rho(0)$ as the initial state. We then propagate $B\rho(t)$ using the dynamical map, to obtain $\Lambda(t + \tau,t)[B\rho(t)]$, and conclude by taking the trace of the resulting operators. If we are interested in steady-state properties, the two-time correlation functions simplify further. By replacing $\rho(0)$ with $\rho(\infty) = \lim_{t\to\infty}\Lambda(t;0)[\rho(0)]$, we can calculate 
$\braket{A(t+\tau)B(t)}$ as
\begin{align}
    \label{eq:steady_state}
    \braket{A(t+\tau)B(t)} &= \tr[A\Lambda(t+\tau;t)[B\rho(\infty)]], \\
    & = \tr[A\Lambda(\tau;0)[B\rho(\infty)]], \\
    & = \braket{A(\tau)B(0)}.
\end{align}

\subsubsection{Emission and absorption spectra}\label{sss:emiss_abs}

Emission and absorption spectra of an optical material can be calculated from the two-time correlation functions of the transition operators associated with the emission and absorption of photons, respectively. For example, an atomic medium given by an ensemble of non-interacting $d$-level systems that interact with the electromagnetic field, will emit light when excited. Its spectrally-resolved intensity is proportional to its emission spectrum $E(\omega)$, which measures the likelihood of transition between eigenstates $\ket{\phi_i}\to\ket{\phi_j}$ with energy difference $\omega$. In first-order perturbation theory, $E(\omega)$ can be calculated using the Fermi golden rule~\cite{Breuer2002}. Line-broadening effects caused by decoherence and relaxation processes can be calculated in second-order perturbation theory using two-point correlation functions and the quantum regression theorem. 

Let us consider a generic two-level emitter with Hamiltonian $H = \Omega\sigma_z/2$ to illustrate how the emission spectrum is calculated. The system can emit a photon via the transition operator $\sigma_- = (\sigma_x-i\sigma_y)/2$ and absorb a photon via its Hermitian conjugate $\sigma_-^\dagger = \sigma_+ = (\sigma_x+i\sigma_y)/2$. Let the system be in a stationary state $\rho(\infty)$. Then, its emission spectrum is calculated from the correlation function of the transition operators $\braket{\sigma_-^\dagger(\tau)\sigma_-(0)}$ as~\cite{Breuer2002},
\begin{align}\label{Eq:TLA_emission_spectrum}
	E(\omega) & \propto \mathcal{F}(\omega)[\braket{\sigma_-^\dagger(\tau)\sigma_-^{\phdagger}(0)}], \\
 & = \int_{-\infty}^{\infty}d\tau e^{-i\omega\tau}\langle{\sigma^\dagger_-}(\tau){\sigma_-^{\phdagger}}(0)\rangle \\ 
 \label{eq:real_form}
 &= 2\;\mathrm{Re}\left\lbrace\int_{0}^{\infty}d\tau e^{-i\omega\tau}\langle{\sigma^\dagger_-}(\tau){\sigma_-^{\phdagger}}(0)\rangle\right\rbrace,
\end{align}
where $\mathcal{F}(\omega)$ is the Fourier transform.
Eq.~(\ref{eq:real_form}) follows from decomposing the limits of the Fourier transform in Eq.~\eqref{Eq:TLA_emission_spectrum} at $t=0$, followed by the use of relation $\langle{\sigma^\dagger_-}(-\tau_+){\sigma_-^{\phdagger}}(0)\rangle  =   \langle{\sigma^\dagger_-}(\tau_+){\sigma_-^{\phdagger}}(0)\rangle^*$, where $\tau_-$ denotes $\tau<0$ and $\tau_+$ denotes $\tau \geq0$~\cite{Breuer2002}.
The generalisation to the emission spectra of a multi-level emitters is obtained by generalisation of Eq.~(\ref{Eq:TLA_emission_spectrum}) as discussed in Ref.~\cite{Hapuarachchi2022}. The emission spectrum $E(\omega)$ is calculated as a sum of all the contributions from the possible transitions $\ket{i}\to\ket{j}$ between the eigenstates of the system with $i>j$, modelled by the operators $\sigma_{ij} = \ketbra{\phi_j}{\phi_i}$,
\begin{equation}
    \label{eq:d-level_emission}
    E(\omega) \propto \sum_{i>j} \mathcal{F}(\omega)[\braket{J_{ij}^\dagger(\tau) J_{ij}^{\phdagger}(0)}],
\end{equation}
with $J_{ij} = \sqrt{\gamma_{ij}}\sigma_{ij}$, for some rates $\gamma_{ij}$.

If the emitter is illuminated by a tunable probe field with angular frequency $\omega_\mathrm{p}$, whose amplitude is assumed to be weak as to not significantly perturb the atom's Hamiltonian, the steady-state probe absorption spectrum can be obtained as follows~\cite{zhou1996absorption, tanas1998analytical, xu2013frontiers},
\begin{equation}\label{Eq:TLA_absorption_spectrum}
	A(\nu) \propto \mathrm{Re}\left\lbrace \int_{0}^{\infty} d\tau e^{i\nu\tau}\langle [\sigma_+^\dagger(\tau),\sigma_+^{\phdagger}(0)]\rangle  \right\rbrace,
\end{equation}
where $\nu = \omega_\mathrm{p}-\omega$ is the detuning of the probe beam relative to the driving laser.

We refer the reader to references \cite{Carmichael1999, meystre2007elements, nation2011qutip} for further details on correlation functions and spectra. We have included the step-by-step implementation of an example of two-level system emission spectrum using (\ref{Eq:TLA_emission_spectrum}) below. The resulting time-domain emission correlation and spectrum are depicted in Fig.\ \ref{Fig:Correlation_spectrum}, alongside the corresponding \texttt{QuTiP} version of the same calculation. 

%%%%%%%% FIGURE %%%%%%%%% 
\begin{figure*}[h!]
	\includegraphics[width=\textwidth]{Figures/correlations.pdf}
	\centering
	\caption{ (\textit{Left}) Real and imaginary part of the steady-state correlation function $C(t) = \braket{\sigma_+(\tau)\sigma_-^{\phdagger}(0)}_{ss}$ for a two-level system $H=\Omega\sigma_z/2$, with Rabi frequency $\Omega$ and decay rate $\Gamma = \Omega/10$. (\textit{Right}) Emission spectrum $E(\omega)$ of the considered two-level system associated with transition operator $\sigma_-^{\phdagger}=\ketbra{e}{g}=\sigma_+^\dagger$. The peaks coincide with the transition frequencies $\omega_{ij} = \omega_i - \omega_j$, associated with transitions $\ket{i}\to\ket{j}$ as shown by the labels. The emission spectrum calculated using the semigroup composition rule is compared with the one obtained using \texttt{QuTiP}, using script~\ref{code:emission_qutip}.}
 \label{Fig:Correlation_spectrum}
\end{figure*}
%%%%%%%% FIGURE %%%%%%%%%


% -------- code: Corr ---------
\code{https://github.com/frnq/qme/blob/main/python/emission_spectrum.py}{Emission spectrum of a two-level atom}{code:emission_spectrum}{python}{Scripts/python/emission_spectrum.txt}

% ----------- Bloch-Redfield Theory
\section{Bloch-Redfield theory}
\label{s:bloch-redfield}

In the previous section we discussed how to implement the Lindblad master equation from a \textit{phenomenological} model of decoherence and relaxation. However, it is sometimes necessary to start from a microscopic description---i.e., the system and environment Hamiltonian---to obtain a master equation for the density operator of the system. When the system interacts weakly with its environment, this can be achieved using Bloch-Redfield theory~\cite{Cohen1992,Breuer2002}. 
This theory is useful when we lack a model for decoherence and relaxation, but we know the nature of the system-environment interactions that drive such processes. As a result, the theory provides a powerful approach to determine the temperature dependence of dephasing and thermalisation rates directly from \textit{first principles}. 

\subsection{Bloch-Redfield master equation}
\label{ss:bloch-redfield_me}

Let us consider a system $\mathrm{S}$, with dimension $\mathrm{dim}_\mathrm{S}$, that interacts with its environment $\mathrm{E}$ according to the following general Hamiltonian
\begin{align}
    \label{eq:BR_microscopic_hamiltonian}
        H &= H_\mathrm{S}+H_\mathrm{E}+H_\mathrm{int} \\
          &= H_\mathrm{S}+H_\mathrm{E} +\sum_\alpha A_\alpha\otimes B_\alpha,
\end{align}
where the coupling operators $A_\alpha$ ($B_\alpha$) are Hermitian and act on the system (environment) such that $H_\mathrm{int}$ is a small perturbation of the unperturbed Hamiltonian $H_0 = H_\mathrm{S}+H_\mathrm{E}$.
Then, under the conditions~\ref{C:BR1}---\ref{C:BR4} discussed in Sec.~\ref{ss:bloch-redfield_approximations}, 
the dynamics of the system's density operator $\rho$ in the eigenbasis $\{\ket{\omega_a}\}$ of $H_\mathrm{S}$~\footnote{$H_\mathrm{S}\ket{\omega_a} = \hbar\omega_a\ket{\omega_a}$.} is prescribed by the Bloch-Redfield master equation,
\begin{eqbox}
    \begin{equation}
        \label{eq:BR_standard_form}
        \dot{\rho}_{ab}(t) = -i\omega_{ab}\rho_{ab}(t)+\sum_{c,d}R_{abcd}\rho_{cd}(t),
    \end{equation}
\end{eqbox}
\noindent
where $\omega_{ab} = \omega_a - \omega_b$ are the frequencies associated with transitions $\ket{\omega_b}\to\ket{\omega_a}$. The Bloch-Redfield tensor $R_{abcd}$ is prescribed by the following expression, where $\delta_{ij}$ is the Kronecker delta,
\begin{equation}
    \label{eq:BR_tensor}
    \begin{split}
    R_{abcd} = -\frac{1}{2\hbar^2}\sum_{\alpha,\beta}\bigg\{ &\delta_{bd}\sum_{n=1}^{\mathrm{dim}_\mathrm{S}} 
    A_{an}^{(\alpha)}A_{nc}^{(\beta)} S_{\alpha\beta}(\omega_{cn}) -  
    A_{ac}^{(\alpha)}A_{db}^{(\beta)} S_{\alpha\beta}(\omega_{ca}) + \\ & \delta_{ac}\sum_{n=1}^{\mathrm{dim}_\mathrm{S}} 
    A_{dn}^{(\alpha)}A_{nb}^{(\beta)} S_{\alpha\beta}(\omega_{dn}) -  
    A_{ac}^{(\alpha)}A_{db}^{(\beta)} S_{\alpha\beta}(\omega_{db})
    \bigg\}.
    \end{split}
\end{equation}
In Eq.~\eqref{eq:BR_tensor}, $A_{ab}^{(\alpha)} = \braket{\omega_a|A_\alpha|\omega_b}$ are the elements of the coupling operators $A_\alpha$ in the eigenbasis of the system Hamiltonian, while $S_{\alpha\beta}(\omega)$ corresponds to the noise-power spectrum of the environment coupling operators~\cite{Jones2017,Gardiner2000},
\begin{equation}
    \label{eq:BR_noise_power_spectra}
    S_{\alpha\beta}(\omega) = \int_{-\infty}^{\infty} d\tau e^{i\omega t}\: \tr\Big[B_\alpha(\tau)B_\beta(0)\rho_\mathrm{E}\Big],
\end{equation}
taken assuming $\rho_\mathrm{E}$ to be some steady state of the environment.

\subsubsection{Thermal relaxation and detailed balance condition}
\label{sss:bloch-redfield_detailed_balance}
When using BR theory it is common to consider environments in thermal equilibrium at inverse temperature $\beta = 1/k_B T$. For example, the environment may be assumed to be in a Bose-Einstein distribution,
\begin{equation}
    \label{eq:BR_gibbs_state}
    G_\beta(H_\mathrm{E}) = \frac{\exp(-\beta H_\mathrm{E})}{\mathcal{Z}},
\end{equation}
with $\mathcal{Z} = \tr[\exp(-\beta H_\mathrm{E})]$, and to be invariant under future evolutions (Gibbs state)~\cite{Breuer2002}.
An out-of-equilibrium density operator that evolves under the dynamics prescribed by Eq.~\eqref{eq:BR_standard_form} with $\rho_\mathrm{E} =G_\beta(H_\mathrm{E})$ will relax towards thermal equilibrium (exchanging energy with the environment). Indeed, the steady state of Eq.~\eqref{eq:BR_standard_form} is itself a Gibbs state $G_\beta(H_\mathrm{S})$ at thermal equilibrium with inverse temperature $\beta$.

The condition for this to occur is known as \textit{detailed balance}, and can be expressed in terms of the ratio between the rates $k_{a\to b}$ associated with transitions $\ket{\omega_a}\to\ket{\omega_b}$ separated by energy $\omega_{ba} = \omega_b -\omega_a$.
\begin{equation}
    \label{eq:BR_detailed_balance}
    \frac{k_{a\to b}}{k_{b\to a}} = \exp(-\beta\omega_{ba}).
\end{equation}
The detailed balance condition implies that the equilibrium populations of the eigenstates of the system follow the Boltzmann distribution $p_a\propto \exp(-\beta \omega_a)$. In terms of noise-power spectra, the detailed balance condition becomes $S_{\alpha\beta}(-\omega)/S_{\alpha\beta}(\omega) = \exp(-\beta\omega)$.

\subsubsection{Example: Spin-boson}
\label{ss:bloch-redfield_spin-boson}
Before discussing the approximation required to derive the BR master equation, let us implement BR theory for the simple and ubiquitous spin-boson model. We consider a two-level system coupled with a large ensemble of \textit{uncorrelated} harmonic oscillators at thermal equilibrium (bosonic bath) 
\begin{equation}
    \label{eq:BR_spin-boson}
    H = \frac{\epsilon_0}{2} \sigma_z+\frac{\Delta}{2} \sigma_z + \sum_k \hbar \omega_k b^\dagger_k b^\phdagger_k + \sigma_z\otimes\sum_k g_k \big(b^\dagger_k + b^\phdagger_k\big),
\end{equation}
where $g_k$ is the strength of the coupling between $\sigma_z$ and some mode $\omega_k$. 

First, we calculate the correlation functions $C_{kk'}(t)$ for the bath operators $B_k = g_k \big(b^\dagger_k + b^\phdagger_k\big)$
\begin{align}
    \label{eq:BR_bath_correlations}
    C_{kk'}(t) &= \delta_{kk'}\tr\big[B_k(t)B_{k'}(0) G_\beta(H_\mathrm{E}) \big], \\
                          &= \frac{g_k^2}{1-\exp(-\beta\omega_{k})}\bigg( e^{-i\omega_k t}+e^{i\omega_k t -\beta\omega_k}\bigg),
\end{align}
where we used the fact that the modes are uncorrelated ($\delta_{kk'}$) and assumed the bath to be in thermal equilibrium  $\rho_\mathrm{E} = G_\beta(H_\mathrm{E})$ at inverse temperature $\beta$, as in Eq.~\eqref{eq:BR_gibbs_state}.

To treat the contribution of a large ensemble of modes, we replace sum over the coupling strength $g_k$ with an integral over some spectral density $J(\omega)$ that well approximates the bath:
\begin{equation}
    \label{eq:spectral_sum_to_integral}
    \sum_k g_k^2 \to \int_0^\infty d\omega J(\omega).
\end{equation}
A common choice is the Ohmic spectral density $J(\omega) = \eta \omega e^{-\omega/\omega_c}$, which is characterised by a cut-off frequency $\omega_c$ and a dimensionless parameter $\eta$, from which we obtain the noise-power~\cite{Lidar2019},
\begin{align}
    \label{eq:BR_NPS}
    S(\omega) &= \int_{-\infty}^{\infty} dt e^{i\omega t} \sum_k C_{kk}(t) \\
              &\approx \int_{-\infty}^{\infty} dt e^{i\omega t} \int_0^{\infty} d\omega' J(\omega') \frac{\big( e^{-i\omega_k t}+e^{i\omega_k t -\beta\omega_k}\big)}{1-\exp(-\beta\omega_{k})} \\
              & = \frac{2\pi\eta\omega \exp(-|\omega|/\omega_c)}{1-\exp(-\beta \omega)}.
\end{align}

We now possess all the elements required to compose the BR tensor of Eq.~\eqref{eq:BR_tensor}. Note that we only have one system coupling operator $A = \sigma_z$, associated with a single noise-power spectrum $S(\omega)$. The following is a \texttt{python} implementation of the Bloch-Redfield tensor, which can then be used to propagate the state of the system using one of the methods discussed in Sec.~\ref{s:doma}. Note that to simplify the solution of Eq.~\eqref{eq:superop_form}, the unitary part of the generator has been absorbed into the tensor $R$,
\begin{equation}
    \label{eq:BR_with_unitary}
    R_{abcd} \to R'_{abcd} =-i\omega_{ac}\delta_{ac}\delta_{bd} + R_{abcd},
\end{equation}
and that system coupling operators are considered to be mutually uncorrelated, $S_{\alpha\beta}=\delta_{\alpha\beta}S_{\alpha\alpha}$.
% --------- code for BR tensor -------------
\code{https://github.com/frnq/qme/blob/main/python/bloch_redfield_tensor.py}{Bloch-Redfield tensor and spin-boson relaxation}{code:BR_spin-boson}{python}{Scripts/python/bloch_redfield_tensor.txt}

\subsection{Approximations for Bloch-Redfield master equation}
\label{ss:bloch-redfield_approximations}
While the Lindblad master equation is guaranteed to be completely positive and trace-preserving\footnote{See Sec.~\ref{s:density_operators} for definition and properties and CPTP maps.}, care must be taken when using BR theory. First, 
the following approximations have to be respected to obtain Eq.~\eqref{eq:BR_standard_form} from the reduced-state von Neumann equation~\cite{Cohen1992,Breuer2002}, as discussed in Sec.~\ref{ss:schrodinger_von_neumann}:

\begin{enumerate}[label=\textbf{C.\arabic*}]
    \item \label{C:BR1} \textbf{Weak coupling approximation:} The interaction $H_\mathrm{int}$ is a small perturbation of the unperturbed Hamiltonian $H_0 = H_\mathrm{S}+H_\mathrm{E}$;
    \item \label{C:BR2} \textbf{Born approximation:} The system-environment density operator is factorised at all times, $\rho_\mathrm{int}(t) = \rho_\mathrm{S}(t)\otimes\rho_\mathrm{E}$, with $\rho_\mathrm{E}$ being some steady state of the environment (justified also by~\ref{C:BR1});
    \item \label{C:BR3} \textbf{Markov approximation:} The bath correlation functions $g_{\alpha\beta}(\tau) = \tr\big[B_\alpha(\tau)B_\beta(0)\rho_\mathrm{E}\big]$ have a short correlation time scale $\tau_\mathrm{E}$, $g_{\alpha\beta}(\tau) \approx 0$ for $\tau \gg \tau_\mathrm{E}$.
    \item \label{C:BR4} \textbf{Rotating wave approximation:} All the contributions from the rapidly oscillating terms, i.e., with characteristic frequency $| \omega_{ab} - \omega_{cd}|\geq \tau^{-1}_\mathrm{E}$, are neglected as they approximately average to zero.
\end{enumerate}

Second, the BR master equation does not, in principle, guarantee positivity of the density operator. That is, when propagating the system in time $\rho(t) = \Lambda_t[\rho_0]$, the populations of $\rho$ may become negative for some time $t>0$~\cite{Whitney2008}. For this reason, when propagating a density operator numerically, it is advisable to check its positivity. The following \texttt{python} script can be used to test positivity, hermitianity and normalisation condition of a density operator. The function \texttt{is\_state(rho)} returns 1 if a \texttt{rho} is a density operator, and a value $s<1$ if \texttt{rho} deviates from the conditions of positivity, hermitianity and normalisation, where $1-s$ is a measure of such deviation.
% ----------- code for density operator test --------------
\code{https://github.com/frnq/qme/blob/main/python/density_operator_test.py}{Is this operator still a state?}{code:density_operator_test}{python}{Scripts/python/density_operator_test.txt}

\subsection{Lindblad form of the Bloch-Redfield master equation}
\label{ss:bloch-redfield_lindblad}
Under certain conditions, it is possible to write the BR master equation in the Lindblad form of Eq.~\eqref{eq:lindblad_master_equation},
\begin{equation}
\label{eq:BR_lindblad_form}
    \dot{\rho}(t) = -\frac{i}{\hbar}[H_\mathrm{S},\rho(t)]+\sum_{\alpha\beta}\sum_{\omega} S_{\alpha\beta}(\omega) \bigg(A_\alpha^\phdagger(\omega)\rho(t) A_\beta^\dagger(\omega) -\frac{1}{2}\Big\{A_\alpha^\dagger(\omega)A_\beta^\phdagger(\omega),\rho(t)\Big\}\bigg),
\end{equation}
where $A_\alpha(\omega) = \sum_{\omega = \omega_b-\omega_a} A_{ab}^{(\alpha)}\ketbra{\omega_a}{\omega_b}$ are the coupling operators in the frequency domain, such that the sum over $\omega$ only needs to be carried out over the transition (Bohr) frequencies $\omega = \omega_b -\omega_a$, as in Eq.~\eqref{eq:BR_tensor}~\cite{Breuer2002}.

This form is useful, for example, to systematically compile the BR tensor from a list of system coupling operators $A_\alpha$ and noise-power spectra $S_{\alpha\alpha}$, or even to compose the full Liouville superoperator associated with the dynamics of Eq.~\eqref{eq:BR_lindblad_form}. 

\subsubsection{Example: Network with random energies and couplings}
\label{ss:bloch-redfield_random_sites}
Let us consider a system consisting of $N$ states $\ket{k}$ with energies $\varepsilon_k$, that interact via couplings $v_{jk}$, with associated Hamiltonian
\begin{equation}
    \label{eq:BR_random_network}
    H_\mathrm{S} = \sum_k \varepsilon_k \ketbra{k}{k} + \sum_{j<k} \bigg(v_{jk}\ketbra{j}{k}+h.c.\bigg).
\end{equation}
Let us assume that each state $\ket{k}$ couples with a local environment of uncorrelated bosonic modes characterised by some noise power spectrum $S_{k}(\omega)$. This type of system-environment model is typically used to model the transport of charge carriers (electrons, holes) or coupled electron-hole pairs (\textit{excitons}) in disordered organic semiconductors~\cite{Jang2018}. In the following \texttt{python} script we study the dynamics of an instance of such random quantum network using Bloch-Redfield theory, with the results shown in Fig.~\ref{fig:random_network}. The BR tensor is calculated using the general method introduced in script~\ref{code:BR_spin-boson}, while the propagator is calculated adaptively for different time scales. A robust and efficient method for the calculation of the Bloch-Redfield tensor is implemented in the \texttt{bloch\_redfield\_tensor} function of \texttt{QuTiP}'s module \texttt{bloch\_redfield}.
% ----------- code for random network --------------
\code{https://github.com/frnq/qme/blob/main/python/random_quantum_network.py}{Random quantum network \normalfont{\textsf{(requires script~\ref*{code:BR_spin-boson})}}}{code:random_quantum_network}{python}{Scripts/python/random_quantum_network.txt}

\subsection{Computational resources for Bloch-Redfield master equation}
\label{ss:bloch-redfield_resources}

Markovian master equations like Lindblad and Bloch-Redfield are generally numerically inexpensive when compared to methods involving memory kernels or environmental degrees of freedom~\cite{Breuer2002,Strathearn2018}. Nevertheless, as the size of the system increases, solving density operator master equations can become computationally demanding~\cite{Kondov2001}. Therefore, when implementing BR theory numerically it is important to keep track of the required computational resources.

\subsubsection{Memory requirements}
\label{sss:bloch-redfield_memory_requirements}
Let $d = \mathrm{dim}\mathcal{H}_\mathrm{S}$ be the dimension of the Hilbert space associated with system's Hamiltonian $H_\mathrm{S}$. For any density operator master equation, the amount of complex floating point (FP) numbers required to store the density operator scales with $d^2$, with the coherences (off-diagonal elements) taking up the majority of this memory requirement. Analogously, the memory requirements to store the Liouville superoperator associated with Eqs.~\eqref{eq:lindblad_master_equation} and~\eqref{eq:BR_standard_form} scale as $d^4$. When memory becomes an issue, it is possible to use \textit{stochastic wave function} methods to limit the memory scaling to that of the system dimension ($d$) for the state, and that of the Hamiltonian ($d^2$) for the propagation, as discussed in Sec.~\ref{ss:swfm}.

\subsubsection{Operations requirements}
\label{sss:bloch-redfield_operations_requirements}
There are three main computationally demanding tasks encountered when solving any density operator master equation numerically in Liouville space: 
\begin{itemize}
    \item Constructing the generator of the evolution $\mathcal{L}$, associated with $\dot{\bm{\rho}} = \mathcal{L} \bm{\rho}$;
    \item Computing the propagator $P_t = \exp[ \mathcal{L} t]$;
    \item Propagating the state $\bm{\rho}_t = P_t \bm{\rho}_0$.
\end{itemize}
As discussed in Sec.~\ref{ss:solving_dynamics}, there is an array of approaches to reduce the expense of these tasks, depending on the type of problem. \vspace{5pt}

\noindent
\textbf{\textsf{Propagation}} --- Starting from the bottom, propagating the state in Liouville space involves a matrix multiplication $P \bm{\rho}$ between a $d^2$-vector $\bm{\rho} = \mathrm{vec}(\rho)$ and a $d^2\times d^2$ operator $P$. Without any optimisation, the number of floating point operations required scales with $d^4$~\cite{Kondov2001}. \vspace{5pt}

\noindent
\textbf{\textsf{Matrix exponential}} --- The number of operations required to compute the propagator depends on the method used to calculate the exponential of the matrix associated with $\mathcal{L}$. For example, \href{https://scipy.org/}{\texttt{scipy}}'s implementation (\href{https://docs.scipy.org/doc/scipy/reference/generated/scipy.linalg.expm.html}{\texttt{scipy.linalg.expm}}) uses the Pad\'e method to approximate the matrix exponential (see Refs.~\cite{Moler2003,AlMohy2010} for details on the amount of operations required). This is generally a demanding task, for Lindblad and BR master equations alike: Some approaches to mitigate the computational costs associated with this task are discussed in Sec.~\ref{ss:solving_dynamics}.\vspace{5pt}

\noindent
\textbf{\textsf{Redfield tensor}} --- However, when it comes to constructing the generator of the evolution, calculating the Bloch-Redfield tensor $R$ becomes substantially more demanding than the bare Lindblad generator $\mathcal{L}$.
In essence, this is because each system coupling operator $A_\alpha$ may contribute to any of the $d^2$ transitions $\ketbra{\omega_a}{\omega_b}$ in the eigenbasis of $H_\mathrm{S}$. Therefore, when constructing a Redfield tensor from $m$ coupling operators $A_\alpha$ we may need to perform a number of operations that scales with $m^2 \times d^2$. In constrast, to construct a Lindblad superoperator $\mathcal{L}$ from $m$ \textit{jump} operators $L_k$ we only need a number of operations that scales with $m$. See Ref.~\cite{Kondov2001} for further information on the computational resources required for BR theory, and the efficiency of different numerical implementations.

\subsection{Pauli master equation}
\label{ss:bloch-redfield_pauli}
The computational cost of BR master equations reduces dramatically under some special circumstances. When the system's Hamiltonian $H_\mathrm{S}$ is non-degenerate, the equations of motion for the populations $p_a(t)$ of the eigenstates $\ket{\omega_a}$ are closed and decoupled from the equations of motion for the coherences~\cite{Breuer2002}. The result is a system of linear ordinary differential equations to the populations, known as the Pauli master equation (PME):
\begin{equation}
\label{eq:BR_pauli}
    \dot{p}_a(t) = \sum_b \big[W_{ab}p_b(t)-W_{ba}p_a(t)\big],
\end{equation}
where the matrix elements $W_{ab} = \sum_{\alpha\beta}A_{ba}^{(\alpha)}A_{ab}^{(\beta)}S_{\alpha\beta}(\omega_{ba})$ represent the transition rates between eigenstates $a$ and $b$. 

The Pauli equation~\eqref{eq:BR_pauli} can be written in the vector form $\dot{\bm{p}}(t) = W \bm{p}(t)$ and solved analytically or numerically using the matrix exponential $\bm{p}(t) = \exp[W t]\bm{p}(0)$. Since the population vector $\bm{p}$ is $d$-dimensional, the computational resources required to implement the PME scale with $d^2$. Pauli master equations find applications in scenarios where dephasing happens over a much shorter time scale than thermal relaxation. As an example, room-temperature exciton transport properties have been studied using this approach in Ref.~\cite{Davidson2020,Davidson2022}. The following script implements the PME associated with the problem set up in script~\ref{code:random_quantum_network}. The results are shown in Fig.~\ref{fig:random_network}.

% ----------- code for PME --------------
\code{https://github.com/frnq/qme/blob/main/python/pauli_master_equation.py}{Pauli master equation {\normalfont \textsf{(requires script~\ref*{code:random_quantum_network})}}}{code:pauli_master_equation}{python}{Scripts/python/pauli_master_equation.txt}

\begin{figure}
    \centering
    \includegraphics{Figures/random_network_pauli.pdf}
    \caption{Solution of Bloch-Redfield master equation and associated Pauli master equations for the random quantum network of scripts~\ref{code:random_quantum_network} and~\ref{code:pauli_master_equation}.}
    \label{fig:random_network}
\end{figure}

% --------- Oscillatory terms and Floquet theory
\section{The prethermalized expectation value problem}
\label{sec:pre}


\subsection{Time evolution on digital quantum computers\label{sec:equilibration}}
The time evolution under a Hamiltonian $H$ can be reproduced on a digital quantum computer using the Suzuki--Trotter decomposition. In its simplest, first-order form, the decomposition approximates the time-evolution unitary $U(\tau) = e^{- i H \tau}$ by
\begin{eqnarray}\label{def:Utrot}
	U_{\mathrm{Trotter}}(\tau) = \prod_{j=1}^{\Gamma} e^{-i H_j\tau}.
\end{eqnarray}
where $H = \sum_{j=1}^{\Gamma} H_j$. Each $H_j$ is a sum of mutually commuting local terms, such that $e^{-i H_j \tau}$ can be efficiently implemented using local gates. The smaller the Trotter step $\tau$, the more accurate the Trotter decomposition. For the $p$-th order Trotter decomposition \cite{Hatano2005}, which generalizes the previous simple formula, the error of $U_\mathrm{Trotter}(\tau)$ with respect to the desired unitary $U(\tau)$ is bounded from above by $\varO(N\tau^{p+1})$, where $N$ is the system size~\cite{childs2021}. The dependence on $N$ can be eliminated if all quantities of interest are local observables. According to the Lieb--Robinson bound, only a light cone with a radius proportional to the total evolution time $T$ is relevant~\cite{Lieb1972}. Therefore, the system size $N$ can be replaced with the size of the light cone $\sim T^d$ before it reaches the edges of the system, where $d$ is the spatial dimension. We hence require that the Trotter step $\tau$ be less than $\varO(\max\left\{T^{-(d+1)/p}, (NT)^{-1/p}\right\})$ for the Trotterized time evolution of local observables to converge to the continuous evolution under $H$.

We can now define the following computational problem.
\begin{problem}[The Trotter time-average problem]
    \label{def:time_average}
    Given a unitary $U_{\mathrm{Trotter}}(\tau)$, a state $\ket{\psi}$, a local observable $A$ and a time $t=m \tau$ for positive integer $m$, and a small positive constant $\epsilon$, 
    compute the time-averaged observable
\begin{eqnarray} \label{eq:AFloq}
    \braket{A}_t = \frac{1}{m + 1} \sum_{n = 0}^{m}  \braket{ \psi |  U^{\dagger}_{\mathrm{Trotter}}(\tau)^{n} A
    U_{\mathrm{Trotter}}(\tau)^{n} | \psi}
\end{eqnarray}
within additive error $\epsilon \Vert A\Vert$, where $\Vert \cdot \Vert$ is the operator norm.
\end{problem}
Note that the Trotterization is not uniquely defined by the Hamiltonian and $U_\mathrm{Trotter}$ must be specified explicitly. The cost of solving this problem on a classical computer generically scales exponentially with either the number of Trotter steps $m$ or the system size $N$~\footnote{For example, a state vector simulation scales linearly in the number of Trotter steps but exponentially with the system size. While a tensor network simulation scales polynomially in system size but exponentially with the number of Trotter steps.}, whereas on a fault-tolerant quantum computer, the effort increases at most polynomially with both. The hardness of the problem is further supported by the fact that it becomes BQP-complete at times $t = \mathrm{poly}(n)$ if the Trotter error is negligible~\cite{janzing2005}. In section~\ref{sec:mitigation}, we present evidence that the problem is solvable on noisy quantum computers up to a maximum number of Trotter steps, which is independent of system size. We then show in section~\ref{sec:implementation} that noisy quantum devices may reach a classically intractable regime with realistic noise parameters, even when taken into account the overhead of our error mitigation strategy.

\subsection{Prethermalization}

Problem~\ref{def:time_average} is not only interesting from the perspective of dynamics but it can also yield insight into equilibrium properties. In condensed matter or statistical physics, one would typically describe a system in equilibrium in terms of its temperature, or in case of the microcanonical ensemble, its internal energy. Under ETH, %Problem~\ref{def:time_average} gives a way to probe 
the microcanonical ensemble at the mean energy of the state $\ket{\psi}$ can be approximated by solving Problem~\ref{def:time_average}.

More precisely, in the limit of continuous time evolution, the long-time average of an observable is described by the diagonal ensemble. For a given initial state $\ket{\psi}$ and an observable $A$,
\begin{eqnarray}
    \begin{aligned}
        \lim_{T\to\infty}\frac{1}{T} \int_0^{T}  \braket{\psi (t)| A | \psi (t)} \dInt t
        = \sum_{k} |\braket{k | \psi}|^2 \braket{k | A | k},
    \end{aligned}
\end{eqnarray}
where $H = \sum_{k} E_k \ket{k}\bra{k}$ is the spectral decomposition of a non-degenerate Hamiltonian~\footnote{In the case of degenerate Hamiltonian spectrum, one can still diagonalize the observable projected onto each subspace of Hamiltonian eigenvalue to define the diagonal ensemble as long time average}. Assuming ETH, the expectation value $\braket{k | A | k}$ is a smooth function of the energy $E_k$ up to a small, state-dependent correction~\cite{Srednicki1999}. The diagonal ensemble is then equivalent to the microcanonical ensemble at energy $\braket{\psi | H | \psi}$ provided the energy variance of $\ket{\psi}$ is sufficiently small. For observables that are an average of an extensive number of local terms, e.g., the total magnetization per site, we expect the microcanonical ensemble to vary significantly only on an extensive energy scale. It is thus possible to estimate expectation values in the microcanonical ensemble from the diagonal ensemble of states whose width in energy is subextensive. Product states satisfy this condition as their widths in energy are (under weak assumptions) proportional to $\sqrt{N}$~\cite{Hartmann2004}.

The above discussion shows that it is possible to probe the microcanonical ensemble by solving problem~\ref{def:time_average} with product initial states at different mean energies. This is, however, challenging with current quantum devices for two reasons. First, the maximum number of Trotter steps $T/\tau$ is limited by the maximum circuit depth in the presence of noise, while the total time $T$ required to reach equilibrium may be large. Therefore, noisy quantum devices are usually unable to reach long enough times with bounded Trotter error.  Secondly, the finite calibration precision renders it challenging to get high relative precision in the angle of rotation for gates that are very close to the identity, bounding from below the size of $\tau$. 

We will now argue that it is nevertheless possible to study equilibrium phenomena. Using larger, experimentally feasible Trotter steps can be viewed as applying a periodic Floquet drive. The system can be described by the Floquet Hamiltonian $H_F$, which is implicitly defined by
\begin{eqnarray}
	U_{\mathrm{Trotter}}(\tau) = e^{-i H_F \tau}.
\end{eqnarray}
The Floquet Hamiltonian is not unique as its eigenvalues are only defined modulo $\omega=2 \pi / \tau$, the effective driving frequency. For large $\tau$, (small $\omega$), i.e., outside the Trotter limit, the Floquet Hamiltonian is highly non-local and will cause a generic initial state to heat up to infinite temperature~\cite{lazarides2014,dalessio2014}. Despite this, it is possible to observe (approximate) equilibration if the heating time scale is much greater than the equilibration time scale. This is known as Floquet prethermalization~\cite{kuwahara2016, Fleckenstein2020, Morningstar2021}. Fortunately for our purposes, Floquet prethermalization is relatively easy to access because Floquet heating occurs on a time scale $t_F \propto e^{\varO(\omega / kJ)}$, where $k$ is the interaction range and $J$ is the local energy scale, assuming $\omega \gtrsim kJ $. We highlight the favorable exponential dependence of $t_F$ on $\omega / k J$ and the fact that $k J$ is independent of the system size.

For times much less than $t_F$, the system evolves approximately according to an effective Hamiltonian which is close to, but not the same as, the original Hamiltonian $H$. More precisely, the effective Hamiltonian is local and it is given by the $n_0$-th order Magnus expansion~\cite{Magnus1954, Blanes2009} of the Floquet Hamiltonian, where $n_0 = \varO(\omega / kJ)$ (see Appendix~\ref{sec:magnus} for details). Observables start to equilibrate under the effective Hamiltonian before eventually heating up. If the equilibration time $t_0$ is much shorter than $t_F$, then there exists a prethermal plateau $t_0 \le t \ll t_F$, during which the expectation value of the observable is approximately constant. We provide a formal definition of a plateau in Appendix \ref{sec:def}. 

The above observations motivate the definition of the PEVP: 
\begin{problem}[Prethermalized expectation value problem]
    \label{prob:PEVP}
    Given a unitary $U_{\mathrm{Trotter}}(\tau)$, a state $\ket{\psi}$, and a local observable $A$, assume that a prethermal plateau exists between times $t_1$ to $t_2$, such that $\max_{t \in [t_1, t_2)} \langle A \rangle_t - \min_{t \in [t_1, t_2)} \langle A \rangle_t \leq \epsilon \Vert A\Vert$ for some positive constant $\epsilon$. Find the value of $\braket{A}_t$ to within additive error $2 \epsilon \Vert A\Vert$ for any $t \in [t_1, t_2)$ .
\end{problem}
This problem reduces to solving Problem~\ref{def:time_average} at time $t = t_1$. In the following sections, we show using the example of the two-dimensional XY model that the prethermal plateau is indeed accessible and that the properties of the effective Hamiltonian closely resemble those of the initial Hamiltonian. We further demonstrate that the PEVP can be solved on a noisy quantum device with realistic parameters up to system sizes for which classical simulation of the dynamics is intractable.


\subsection{PEVP with the XY model\label{sec:xy}}
We focus on the two-dimensional quantum XY model on a square lattice for the remainder of this work. We emphasize, however, that the approach can be readily applied to many other models. The Hamiltonian of the XY model is given by
\begin{eqnarray}
	H_{\mathrm{XY}} = - J \sum_{\braket{ij}} \left( S^x_i S^x_j + S^y_i S^y_j \right),
\end{eqnarray} 
where $J$ is the interaction strenth, $S_i^\alpha$ ($\alpha \in \{ x, y, z\}$) are spin-1/2 operators on site $i$, and the sum runs over all pairs of nearest neighbors. The model is convenient for digital quantum computers as its two-site interaction generates a partial iSWAP gate,
\begin{eqnarray}
    e^{-i J\left( S^x_i S^x_j + S^y_i S^y_j \right) \tau} = \text{iSWAP}^{ - J \tau / \pi}_{ij}.
\end{eqnarray}
A single Trotter step in a first-order decomposition consists of applying a partial iSWAP gate to each nearest-neighbor pair of qubits. As non-overlapping gates can be performed in parallel, these operations can be carried out in a circuit whose depth is equal to the number of nearest neighbors (4 in the case of the square lattice).

The XY model in two dimensions can be solved with quantum Monte Carlo algorithms~\cite{Loh1985, Ding1992} and thus serves as a good benchmark to our method. It is known to undergo the Kosterlitz--Thouless (KT) transition~\cite{Kosterlitz1973, Ding1992} at nonzero temperature. This phase transition can be characterized by the mean-squared in-plane magnetization per site,
\begin{eqnarray}
	m_{x}^2 + m_{y}^2 = 4\cdot \frac{ \left( \sum_i S^x_i\right)^2 + \left( \sum_i S^y_i\right)^2 }{N^2},
\end{eqnarray}
which is an approximation to the in-plane susceptibility~\cite{Ding1992}. The mean-squared magnetization can be written as the sum of two-site correlators, which decay exponentially with the distance between the two sites at high temperature. Hence, $m_x^2 + m_y^2$ decreases with the system size as $1/N$ in the thermodynamic limit. Below the critical temperature, the system exhibits quasi long-range order. The mean-squared magnetization decays only as $1/N^{1/8}$ and its value remains non-negligible for moderately large systems ~\cite{Ding1992}.

\begin{figure}
    \centering
    \xincludegraphics[width=0.45\textwidth, label=(a)]{figs/02a_longTimeEvol_XY_Lx4_Ly4_hx0.0_theta8.0.pdf}
    \xincludegraphics[width=0.48\textwidth, label=(b)]    {figs/02b_Lx4_Ly4_Jz0.00_p0_r20_alphaPi8_delta0.5.pdf}
    \caption{\textbf{(a)}~Prethermal plateau of the 2D XY model for system size $N = 4 \times 4$. The initial state is $\ket{X+}$. The colored lines show the time averages of the mean-squared in-plane magnetization for different Trotter step sizes $\tau$, corresponding to different driving frequencies $\omega = 2 \pi / \tau$. The large circles stand for the starting and end points of the plateaus according to Definition~\ref{def:plateau} with tolerance $\epsilon = 0.05$ and a maximum value of $t_2 J$ of $10^3$. The black dashed line represents the value in the diagonal ensemble of the initial Hamiltonian. \textbf{(b)}~Comparison of the value at the prethermal plateau with the values in the microcanonical and diagonal ensemble values of the initial XY Hamiltonian and in the diagonal ensemble of the first-order Magnus expansion. The system size is $4 \times 4$. The driving frequency is as $\omega = 8J$ and the plateau value is taken from the time average at $t = 20 / J$, which is on the prethermal plateau for all computed initial states with tolerance $\epsilon = 0.05$. For the microcanonical ensemble, we average over an energy window of width $\delta = 0.5J$ in the $m_z = 0$ subspace (see Appendix~\ref{sec:def}).}
    \label{fig:prethermal}
\end{figure}

In analogy to the long-time average that gives rise to the diagonal ensemble, we probe the prethermal plateaus using the Floquet time average as in Definition~\ref{def:time_average}, where the Trotterization is shown in the appendix in Fig.~\ref{fig:magnus}a. We explore this quantity using exact diagonalization on a square lattice with $N = 4 \times 4$ spins and open boundary conditions. Figure~\ref{fig:prethermal}a shows the values of the mean-squared in-plane magnetization for the initial state $\ket{\psi} = \ket{X+} = \left[ \frac{1}{\sqrt{2}}\left( \ket{0} + \ket{1} \right)\right]^{\otimes N}$. The different colors indicate the Trotter step size $\tau$ or, equivalently, the driving frequency $\omega = 2\pi / \tau$. The initial state is close to the ground state of the XY Hamiltonian. We therefore expect the in-plane magnetization to remain high in the prethermal plateau, provided the effective Hamiltonian does not differ too much from the XY model.

We indeed observe prethermal plateaus for large driving frequencies ($\omega \ge 8J$), and these last for $t > 10^3 / J$ when $\omega \ge 9J$. The plateau values approach the diagonal ensemble value (black dashed line) with increasing driving frequencies. They deviate only slightly due to the correction in the Magnus expansion, which will be discussed later in this subsection. This confirms that the dynamics with fast Floquet drive are similar to the dynamics of the original Hamiltonian in this prethermal regime. By contrast, no plateaus are observed at low driving frequencies, where the time average of the mean-squared magnetization quickly drops to expected value at infinite temperature, $2 / N$. 

We may perform the same analysis for different initial states. We choose product states in which the spins on the two sublattices of the square lattice are in the respective states $\ket{\theta, 0}$ and $\ket{\pi - \theta, \phi}$, where $\ket{\theta, \phi} = \cos(\theta / 2) \ket{0} + \sin(\theta /2 ) e^{i\phi} \ket{1}$ parametrizes an arbitrary state of a qubit (spin-1/2). This choice of states allows us to cover a wide range of the spectrum while ensuring that the total magnetization in the $z$ direction vanishes. The latter constraint is convenient because the Hamiltonian conserves the total $z$-magnetization, $m_z = \sum_{i = 1} ^{N} \sigma^z_i / N$. Thermalization therefore occurs in the eigenspaces of $m_z$. Low-energy product states however are not eigenstates of $m_z$. By choosing the expectation value of $m_z$ to be zero, we maximize the overlap of the product state with the sectors of low $z$-magnetization, for which we expect similar equilibration dynamics.


We find that all product states of the above form exhibit prethermal plateaus at similar driving frequencies and evolution times. We evaluate the prethermal values of the in-plane magnetization by performing the Floquet time average up to time $t = 20/J$ with driving frequency $\omega = 8 J$. The result is shown for various initial states as a function of their mean energy in Fig.~\ref{fig:prethermal}b. For comparison, we also show the diagonal and microcanonical ensemble values of the initial XY model, as well as the diagonal ensemble one of the first-order Magnus expansion of Floquet Hamiltonian, given by
\begin{eqnarray}
	\begin{aligned}
		H_{\mathrm{Magnus}}^{(1)} =  & \frac{1}{\tau} \int_{0}^{\tau} \dInt t_1 H(t_1)            \\
		& + \frac{1}{2i \tau} \int_{0}^{\tau} \dInt t_1 \int_{0}^{t_1} \dInt t_2 \left[H(t_1), H(t_2)\right].
	\end{aligned}
\end{eqnarray}
Here, $H(t)$ is the piecewise constant Hamiltonian corresponding to the different terms of the Trotter expansion Eq.~(\ref{def:Utrot}): 
\begin{eqnarray}
    H(t) = \Gamma H_j \text{ for } (j-1)\tau / \Gamma \le t < j \tau / \Gamma,
\end{eqnarray}
where $1 \le j \le \Gamma$. Definitions of the different ensembles and higher orders of the Magnus expansion can be found in App.~\ref{sec:def} and App.~\ref{sec:magnus}, respectively.

The values at the prethermal plateau are close to those of the diagonal ensemble $H_\mathrm{Magnus}^{(1)}$, indicating that the first-order truncation already serves as a good approximation for Floquet Hamiltonian in the prethermal regime. In Appendix~\ref{sec:magnus}, we show that the higher orders lead to no significant improvement for $\omega = 8J$. The thermal equilibrium values of the initial XY Hamiltonian, in both the diagonal and the microcanonical ensemble, deviate slightly from the Floquet values. Nevertheless, the comparison indicates that the prethermal properties of the Floquet system can reveal nontrivial thermal properties of the XY Hamitlonian.


% --------- Further readings
We provide some comments on the growth conditions which constituted the majority of our analysis in sections \ref{sec:Hmixing} and \ref{sec:Hsigma}. In the simplest cases of Lemma \ref{lemma:unstableGrowth}, growth was established in an analogous fashion to the old one-step expansion condition (\ref{eq:oldOneStepExpansion}), finding the relevant Jacobians $M_j$ and checking that their expansion factors $K(M_j)$ satisfy
\begin{equation}
    \label{eq:discussionOneStep}
    \sum_j \frac{1}{K(M_j)} <1.
\end{equation}
For the more complicated cases, the inductive method used to establish growth near the accumulation points in Lemma \ref{lemma:unstableGrowth} and the weakened one-step expansion condition (\ref{eq:oneStep}) both address the same fundamental issue: the splitting of unstable curves by singularities into an unbounded number of small components. They circumvent this obstacle in rather different ways, however. While (\ref{eq:oneStep}) generalises (\ref{eq:discussionOneStep}) to ensure an growth of unstable curves `on average' (see \cite{chernov_statistical_2009} for a precise statement), our inductive method is a more direct adaptation of (\ref{eq:discussionOneStep}), using it to generate contradictory geometric conditions which a hypothetical non-growing unstable curve must satisfy. It may be possible to prove Theorem \ref{sec:Hmixing} using (\ref{eq:oneStep}) as the basis for growth. Since we required (\ref{eq:oneStep}) anyway for proving Theorem \ref{thm:HsigmaExp}, this could potentially condense our analysis, but only to a minor extent. A convenience of the method used in section \ref{sec:Hmixing} is that, by way of the `simple intersection' property, it naturally gives geometric information on the images of manifolds, useful for proving the property \textbf{(M)} of Theorem \ref{thm:katok-strelcyn}.

We expect that essentially analogous analysis can be applied to establish mixing properties in a wide class of piecewise linear non-uniformly hyperbolic maps, including those (like the OTM) which sit on the boundary of ergodicity and beyond. While we have relied on the precise partition structure of $H_\sigma$, its fundamental feature (self-similar sequences of elements $A^k$, sharing boundaries with its neighbours $A^{k-1},A^{k+1}$ and accumulating onto some point $p$) is quite typical to return map systems. See, for example, those of various stadium billiards \cite{chernov_chaotic_2006,chernov_improved_2008,chernov_statistical_2009} and LTMs \cite{springham_polynomial_2014}. Indeed, the same method can be used to prove the Bernoulli property for non-monotonic LTMs \cite{myers_hill_mixing_2022}, where monotonicity of the manifold images cannot be assumed and the classical argument \cite{sturman_mathematical_2006} fails. The OTM is the pointwise limit of these maps as the boundary shrinks to null measure. It further has utility in proving growth conditions for maps which are uniformly hyperbolic but possess regions $A_j$ where the hyperbolicity is very weak, signified by $K(M_j) \approx 1$, so that (\ref{eq:discussionOneStep}) fails. Typically this leads to suboptimal bounds on mixing windows, see e.g. \cite{wojtkowski_model_1981,przytycki_ergodicity_1983,myers_hill_family_2022}. The map $H_{(\eta,\eta)}$ for $\eta \approx 1/2$ is another example, possessing weak hyperbolicity over $A_2, A_3$. Letting $\varepsilon = |\eta-1/2|>0$, there is an upper bound $N = N(\varepsilon)$ on escape times from the intersections $A_2\cap \sigma, A_3 \cap \sigma$. The growth lemma then follows by applying the inductive step roughly $N$ times and can be established for arbitrarily small $\varepsilon$, opening the door to establishing optimal mixing windows.

The above gives two examples of piecewise linear perturbations to $H$ where mixing with respect to Lebesgue is preserved and our methods can be applied. Nonlinear perturbations to the shear profiles complicate the analysis in several ways. Firstly as the map's Jacobians takes on a broader range of values, cone invariance becomes an increasingly harder condition to establish. Cones must be widened, giving looser bounds on expansion factors, which may already be weak due to new regions of weaker stretching. This, together with the change from polygonal to curvilinear return time partition elements and nonlinear local manifolds, adds some complexity to showing growth conditions. This does not rule out certain (small) nonlinear perturbations however. There is some leeway in the inequalities which govern cone invariance and growth of local manifolds, the latter of which is not too dissimilar from the piecewise linear setting (see Lemmas \ref{lemma:piecewiseApprox}, \ref{lemma:componentLength}). Certain small perturbations would not alter the \emph{topological} structure of the return time partition, i.e. which elements share boundaries, the key information needed for setting up the induction. Finally while the partition elements would no longer be polygonal, only coarse geometric information is required for verifying each inductive step. Following the above, a potential perturbation could be to replace the linear portions of each shear by a cubic, perturbing the tent profile
\[  f(t) = \begin{cases} 2t & 0 \leq t \leq 1/2, \\ 2(1-t) & 1/2 \leq t \leq 1 ,\end{cases} \]
of the OTM shears to
\[  f_a(t) = \begin{cases} \frac{1}{8} t \left(16 - a + 6at - 8at^{2} \right) & 0 \leq t \leq 1/2, \\ \frac{1}{8}\left(1-t\right)\left( 16 - a + 6a\left(1-t\right) - 8a\left(1-t\right)^{2}\right)  & 1/2 \leq t \leq 1, \end{cases}   \]
for $a>0$. For small enough $a$ the gradient range $f'(t)$ is restricted to small neighbourhoods of $\{ 2, -2\}$ and the escape time partition retains a similar structure. We illustrate this in Figure \ref{fig:perturbations}, showing escapes from the square $S_3$ under the map $G \circ F$, equivalent to escapes from the perturbed $A_3$ under the $G \circ F$, but with a cleaner geometry for comparison. When $a$ is too large the analogy to the OTM breaks down. At $a=16$ the map is twice differentiable everywhere and features a new source of slowed mixing, the Jacobian is the identity at the corner points $x,y \in \{  0, 1/2 \}$ giving locally parabolic behaviour (visible in the escape time partition). 

\begin{figure}
    \centering
    \includegraphics[width=0.24 \linewidth]{0.png}
    \includegraphics[width=0.24 \linewidth]{4.png}
    \includegraphics[width=0.24 \linewidth]{8.png}
    \includegraphics[width=0.24 \linewidth]{16.png}
    \caption{Partition of escape times from $S_3$ under the mapping $F \circ G$ for $a= 0,4,8,16$. }
    \label{fig:perturbations}
\end{figure}

% ---------- Acknowledgments
% Acknowledgments---Will not appear in anonymized version
\midlacknowledgments{We would like to thank Dr. Birgit Stierstorfer (Non Clinical Drug Safety, Boehringer Ingelheim, Biberach, Germany) and Dr. Charlotte Lempp (Drug Discovery Sciences, Boehringer Ingelheim, Biberach, Germany)
for their help with the visual evaluation of the artificially generated images and Dr. Martin Lenter (Drug
Discovery Sciences, Boehringer Ingelheim, Biberach, Germany) for helpful discussions and project support. We would also
like to thank Dr. Lina Humbeck (Medicinal Chemistry,
Boehringer Ingelheim, Biberach, Germany) for her comments,
which helped to improve the quality of the paper.}

% ---------- Bibliography
%\printbibliography
%\bibliographystyle{unsrt}
%\bibliography{library}
% This must be in the first 5 lines to tell arXiv to use pdfLaTeX, which is strongly recommended.
\pdfoutput=1
% In particular, the hyperref package requires pdfLaTeX in order to break URLs across lines.

\documentclass[11pt]{article}

% Remove the "review" option to generate the final version.
%\usepackage[review]{ACL2023}
\usepackage{ACL2023}

% Standard package includes
\usepackage{times}
\usepackage{latexsym}

% For proper rendering and hyphenation of words containing Latin characters (including in bib files)
\usepackage[T1]{fontenc}
% For Vietnamese characters
% \usepackage[T5]{fontenc}
% See https://www.latex-project.org/help/documentation/encguide.pdf for other character sets

% This assumes your files are encoded as UTF8
\usepackage[utf8]{inputenc}

% This is not strictly necessary, and may be commented out.
% However, it will improve the layout of the manuscript,
% and will typically save some space.
\usepackage{microtype}

% This is also not strictly necessary, and may be commented out.
% However, it will improve the aesthetics of text in
% the typewriter font.
\usepackage{inconsolata}


% If the title and author information does not fit in the area allocated, uncomment the following
%
%\setlength\titlebox{10cm}
%
% and set <dim> to something 5cm or larger.

%%%%%%%%%%%%%%%%%%%%%%%%%%%%%%%%%%
\usepackage{graphicx}
\usepackage{amsfonts}
\usepackage{amsmath}
\usepackage{bigdelim}
\usepackage{diagbox}
\usepackage{amsthm}
\usepackage{makecell}
\usepackage{mathtools}
\usepackage{booktabs}
\usepackage[shortlabels]{enumitem}
\graphicspath{ {figs/} }

\theoremstyle{remark}
\newtheorem*{question}{Question}

\newcommand{\tk}[1]{\textcolor{blue}{{#1}}}
\newcommand{\sy}[1]{\textcolor{red}{{#1}}}
\newcommand{\mg}[1]{\textcolor{purple}{{#1}}}
\newcommand{\lh}[1]{\textcolor{green}{{#1}}}
\newcommand{\lc}[1]{\textcolor{green}{{#1}}}

% Rounded color box
\definecolor{light_blue}{HTML}{cfdfff}
\usepackage[most]{tcolorbox}
\tcbset{on line, 
        boxsep=1pt, left=0pt,right=0pt,top=0pt,bottom=0pt,
        colframe=white,colback=light_blue,  
        highlight math style={enhanced}
        }

\newcommand{\quash}[1]{}  %Anything in \quash is ignored
\newcommand{\gpt}{\textsc{GPT-2}}
\newcommand{\bert}{\textsc{BERT}}
\newcommand{\bertlarge}{\textsc{BERT-large}}
\newcommand{\mask}{\texttt{[MASK]}}
\newcommand{\cls}{\texttt{[CLS]}}
\newcommand{\sep}{\texttt{[SEP]}}
\newcommand{\mat}{\texttt{mat}}
\newcommand{\id}{\texttt{id}}
\newcommand{\matl}{\texttt{mat}_{\ell \rightarrow \ell'}}
\newcommand{\matattnl}{\texttt{mat\_attn}_{\ell \rightarrow \ell'}}
\newcommand{\matffl}{\texttt{mat\_ffn}_{\ell \rightarrow \ell'}}
\newcommand{\matlnl}{\texttt{mat\_ln1\_ln2}_{\ell \rightarrow \ell'}}
\newcommand{\idl}{\texttt{id}_{\ell \rightarrow \ell'}}
\newcommand{\matlL}{\texttt{mat}_{\ell \rightarrow L}}
\newcommand{\matattnlL}{\texttt{mat\_attn}_{\ell \rightarrow L}}
\newcommand{\matfflL}{\texttt{mat\_ffn}_{\ell \rightarrow L}}
\newcommand{\matlnlL}{\texttt{mat\_ln1\_ln2}_{\ell \rightarrow L}}
\newcommand{\idlL}{\texttt{id}_{\ell \rightarrow L}}

\definecolor{blue(munsell)}{rgb}{0.0, 0.5, 0.69}
%%%%%%%%%%%%%%%%%%%%%%%%%%%%%%%%%%

\title{Jump to Conclusions: Short-Cutting Transformers\\With Linear Transformations}

% Author information can be set in various styles:
% For several authors from the same institution:
% \author{Author 1 \and ... \and Author n \\
%         Address line \\ ... \\ Address line}
% if the names do not fit well on one line use
%         Author 1 \\ {\bf Author 2} \\ ... \\ {\bf Author n} \\
% For authors from different institutions:
% \author{Author 1 \\ Address line \\  ... \\ Address line
%         \And  ... \And
%         Author n \\ Address line \\ ... \\ Address line}
% To start a seperate ``row'' of authors use \AND, as in
% \author{Author 1 \\ Address line \\  ... \\ Address line
%         \AND
%         Author 2 \\ Address line \\ ... \\ Address line \And
%         Author 3 \\ Address line \\ ... \\ Address line}

\author{Alexander Yom Din$^{1}$ ~~~~~ Taelin Karidi$^{1}$ ~~~~~ Leshem Choshen$^{1}$ ~~~~~
Mor Geva$^{2}$ 
\vspace{0.2cm} \\
$^1$Hebrew University of Jerusalem ~~~ $^2$Google Research \\
\small{\texttt{\{alexander.yomdin, taelin.karidi, leshem.choshen\}@mail.huji.ac.il}}, \small{\texttt{pipek@google.com}}}

\quash{
\author{Alexander Yom Din \\
  Hebrew University of Jerusalem \\ \texttt{alexander.yomdin@mail.huji.ac.il} \\\And
  Taelin Karidi \\
  Hebrew University of Jerusalem \\
  \texttt{taelin.karidi@mail.huji.ac.il} \\\And
  Leshem Choshen \\
  Hebrew University of Jerusalem \\ \texttt{leshem.choshen@mail.huji.ac.il} \\\And
  Mor Geva \\
  Google Research \\
  \texttt{pipek@google.com} \\}
}

\begin{document}
\maketitle



\begin{abstract}
% \vspace{-1em}
The diffusion-based generative models have achieved remarkable success in text-based image generation. However, since it contains enormous randomness in generation progress, it is still challenging to apply such models for real-world visual content editing, especially in videos. 
In this paper, we propose \texttt{FateZero}, a zero-shot text-based editing method on real-world videos without per-prompt training or use-specific mask. 
\RM{Specifically, different from a pipeline of two independent inversion and then generation stages, we find the intermediate attention maps during inversions store better structure and motion information. We thus reform them to temporally casual attention and replace them in the generation progress. To further reduce the unnecessary semantic leakage of source video and enhance the editing quality, we then remix the temporally casual attentions via the cross-attention features of the source prompt as the mask.}
To edit videos consistently, we propose several techniques based on the pre-trained models. Firstly, in contrast to the straightforward DDIM inversion technique, our approach captures intermediate attention maps during inversion, which effectively retain both structural and motion information. These maps are directly fused in the editing process rather than generated during denoising. To further minimize semantic leakage of the source video, we then fuse self-attentions with a blending mask obtained by cross-attention features from the source prompt. Furthermore, we have implemented a reform of the self-attention mechanism in denoising UNet by introducing spatial-temporal attention to ensure frame consistency.
Yet succinct, our method is the first one to show the ability of zero-shot text-driven video style and local attribute editing from the trained text-to-image model. We also have a better zero-shot shape-aware editing ability based on the text-to-video model~\cite{tuneavideo}. \RM{Besides video, our unified method also achieves state-of-the-art performance in zero-shot image editing.\chenyang{Need exp or remove the zero-shot image}} Extensive experiments demonstrate our superior temporal consistency and editing capability than previous works.
% The code will be released.
% \chenyang{emphasize: our observation at inversion time} \xiaodong{replacing the bold part to the actual pipeline: \textbf{Specifically, we work on replacing and mixing the attention maps between the inversion and generation since the self-attention map keeps the structure of the original natural image and the cross-attention is semantic-related, after remixing, we replace them in the corresponding generation steps for denoising.}}
% \footnote{Since there is no general video diffusion model is publicly available, we use one-shot video generation method~(Tune-A-Video~\cite{tuneavideo}) as the pretrained video diffusion model for zero-shot video editing\xiaodong{can be removed if we actually zero-shot on video}.}.
\end{abstract}
\section{Introduction}

The ability to reason about plans is critical for performing long-horizon tasks \citep{erol1996hierarchical, sohn2018hierarchical, sharma-etal-2022-skill}, compositional generalization \citep{corona-etal-2021-modular} and generalization to unseen tasks and environments \citep{shridhar2020alfred}.
Consider a simple long-horizon planning scenario where a robot is tasked with preparing a meal and serving it on the table. 
This presents a non-trivial planning problem since the agent needs to understand the sequence of operations required to perform the task and search for the relevant objects in the unfamiliar environment by interacting with various objects. %



Large language models have been recently shown to possess commonsense knowledge about the world such as object affordances and physical dynamics \citep{ouyang2022training,chowdhery2022palm}.
Early approaches considered text based environments and fine-tuned PLMs to predict actions given the history of past observations and actions \citep{jansen-2020-visually,micheli-fleuret-2021-language,yao-etal-2020-keep}.
Recent work has used this ability to reason about plans from text instructions in simulated household environments with simplifying assumptions such as text-only environment observations or feedback \citep{huang2022language,ahn2022can,li2022pre,logeswaran-etal-2022-shot}.


We focus on \emph{visually grounded planning} with PLMs --- the ability to adapt plans based on interaction and visual feedback from the environment.
While PLMs have strong planning commonsense priors, predictions from a PLM may not be directly realizable in the environment since the observation and action spaces are unknown.
This requires \emph{grounding} the PLM in the environment and adapting it to observe visual feedback, which is highly non-trivial.
Some prior works assume the availability of a pre-trained affordance function \citep{ahn2022can} or a success detector \citep{mirchandani2021ella}.
Notably, SayCan \citep{ahn2022can} completely decouples the PLM from observation information by selecting actions that have both high affordability (through a pre-trained affordance model) and high PLM likelihood.
Although this partially addresses the grounding problem, the use of visual feedback for action affordance alone is limited.
Often an agent must choose one of many affordable actions using information from observations.
For example, a driving agent should re-navigate and possibly turn around when encountering a ``road closed'' sign, but both turning around and driving forward are indistinguishable to SayCan because they are both affordable and the PLM is blind to observations.

Another workaround explored in prior work is translating the information in the visual observations to text using a pre-trained captioning system \citep{shridhar2021alfworld,huang2022language}.
However, it can be difficult to faithfully describe an image in words and information is lost in this inherently noisy process, which limits the information available to the planner.



Recent work shows that PLMs can be adapted for various natural language tasks by inserting tunable embeddings or soft prompts at the input of the PLM (also called prompt tuning or prefix tuning)~\citep{li-liang-2021-prefix,lester-etal-2021-power}.
This approach also extends to multi-modal understanding tasks such as image captioning \citep{mokady2021clipcap} and VQA \citep{tsimpoukelli2021multimodal} where images are encoded as soft prompts and finetuned for the target task.
Transformer based architectures have also been successfully applied to offline Reinforcement Learning in recent work \citep{chen2021decision,janner2021offline,li2022pre,reid2022can}.

Taking inspiration from these works, we propose the simple approach of embedding visual observations (`visual prompts') and \textit{directly inserting them as PLM input embeddings}.
The visual encoder and PLM are jointly trained for the target task, an approach we call \textbf{\oursfull}~(\ours).
By teaching the PLM to use observations for planning in an end to end manner, we remove the dependency on external data such as captions and affordability information that was used in prior work.
We show that this simple approach performs better than prior PLM-based planning approaches on two embodied planning benchmarks based on ALFWorld~\citep{shridhar2021alfworld} and Virtualhome~\cite{puig2018virtualhome}.



\section{Related Work}

%Here we summarize prior work on transfer learning and property inference.

%\shortsection{Transfer Learning}
%%Transfer learning reuses features learned by pre-trained models for new tasks, with the pretext that inherent similarities in the generic features will be useful for the downstream tasks and hence reducing their cost of downstream training. Specifically, the downstream model trainer will use a pre-trained upstream model as the starting point for the downstream training, with inclusion of (or replacement with) the task-specific classification layer/module. The downstream model is then trained by either updating all layers of the model (including ones reused from upstream model) or freezing some earlier layers of the reused parts as the ``feature extractor'' and only updating the rest. The latter approach is more popular as the reused feature extractors can already learn useful feature representations and the training cost is also much lower and affordable for individuals with limited computational resources. We study the vulnerability of the latter transfer learning approach in this paper. 


%\shortsection{Transfer Learning} 
Several works have demonstrated risks associated with transfer learning across a variety of attack goals. Wang et al.~\cite{wang2018great} and Yao et al.~\cite{yao2019latent} consider manipulating the upstream model such that the fine-tuned downstream models contain backdoors, misclassifying test inputs that contain predefined backdoor triggers. These transfer manipulations are tailored to their particular attack goals and cannot be applied for the property inference goal considered in this paper. Zou et al.~\cite{zou2020privacy} study the threat of membership inference attacks on transfer learning, but with normally trained upstream models.  
%\dnote{its clear that the goals are different for these attacks, but how similar are the methods?} \ynote{similarity of the methods? more details about the methods? do not know what is expected here}
%In contrast, we investigate the possibility of boosting the effectiveness of property inference by manipulating the upstream model training. % Schuster et al.~\cite{schuster2020humpty} show that the attacker can modify the corpus on which the word embedding is trained such that the downstream NLP models which use that embedding will behave abnormally.

%\shortsection{Property Inference}
The risk of property inference was introduced by Ateniese et al.~\cite{ateniese2015hacking}, % introduces the threat of inferring properties of the training data from pre-trained models, 
and several subsequent works have developed property inference (also known as distribution inference) attacks~\cite{Wang2022GroupPI, suri2022formalizing, Jurez2022BlackBoxAF, Hartmann2022DistributionIR}.
% Ganju et al.~\cite{ganju2018property} and Suri and Evans~\cite{suri2022formalizing} 
These works study property inference against normally trained models, and they launch attacks using a variety of black-box and white-box attacks. All the white-box attacks use meta-classifiers, which take the permutation-invariant representation~\cite{ganju2018property} of the model parameters as the features. We use the state-of-the-art white-box attack~\cite{suri2022formalizing} in our experiments.
%We will use the state-of-the-art white-box method proposed by Ganju et al.~\cite{ganju2018property} and later extended by suri et al.~\cite{suri2022formalizing} in this paper.
%\dnote{do we use these attacks?} 
Melis et al.~\cite{melis2019exploiting} and Zhang et al.~\cite{zhang2021leakage} focus on property inference in distributed training scenarios. In their settings, the attacker is a participant in the global model training and conducts property inference using meta-classifiers that are trained on model outputs or gradients. Similarly, Suri et al.~\cite{suri2022subject} focus on federated learning settings where the attacker is a participant (or the central server) that utilizes black-box attacks for inferring membership of data from particular subjects. %\dnote{if we use black-box attacks, explain which ones, or how ours are related to previous ones} 
For our experiments, We improve the black-box meta-classifier proposed by Zhang et al.~\cite{zhang2021leakage} using the ``query tuning'' technique in Xu et al.~\cite{xu2019detecting}. 

The closest works to ours are Chase et al.~\cite{saeed} and Chaudhari et al.~\cite{Chaudhari2022SNAPEE}, which both consider a scenario where the attacker can manipulate some of the training data of the model to induce a model that significantly increases property inference risk.
% \dnote{it enables precise property inference attacks?}.
These works assume an adversary with the ability to poison the victim's training data, while the adversary in our scenario has no access to the victim's training data, and therefore, their methods are not applicable.
% \dnote{example how different from ours, and why the methods are not applicable}
%Thus, their methods are not applicable to our transfer learning scenario.
%Their methods rely on inducing certain behavior correlated with the properties to be inferred, and thus are not applicable to our transfer learning scenario. \anote{Still a bit unclear why that is the case.}
%
There are also works similar to ours that leverage ``adversarial initializations'' for attack purposes.
% \cite{grosse2019adversarial, boenisch2021curious, wen2022fishing, fowl2021robbing}.
Grosse et al.~\cite{grosse2019adversarial} focus on scenarios where the attacker can control the parameter initialization of a model, and demonstrate that the attacker can use special initializations to damage the performance of the trained model. %This attack is orthogonal to ours.
Other works \cite{boenisch2021curious, wen2022fishing, fowl2021robbing} show that the malicious central server in a federated learning protocol can reconstruct some training samples via falsifying the global model in some training rounds and then analyzing the submitted gradients. These kinds of attacks do not apply to our transfer-learning scenario since the attacker cannot access the downstream gradients, and can only manipulate the upstream training.

\iffalse %%%%%%%%%%%%%%%%%%%%%%%%%%%%%%%%

In this section, we provide the background and also the summary of prior attacks on transfer learning (Section~\ref{sec:transfer_learning}) and property inference (Section~\ref{sec:property_inference}). Then, we introduce the closely related manipulation attacks against machine learning models to boost different privacy risks in Section~\ref{sec:active_inference_attacks}.

%\anote{Do we really need a dedicated section for this? It's barely 2 paragraphs right now.}

%\dnote{the most closely related work to ours are works that attempt to amplify inference attacks by poisoning models, the two most relevant I know of are \url{https://www.computer.org/csdl/proceedings-article/sp/2022/131600b569/1CIO8nmuota} and \url{https://arxiv.org/abs/2204.00032}, but need to look thoroughly for others. We should definitely be describing this and relating it to our work, probably in the introduction. Most of what is here is Background, but should be clear what this section is for (not muddling background and related work)}

\subsection{Transfer Learning} \label{sec:transfer_learning}
Transfer learning reuses features learned by pre-trained models for new tasks, with the pretext that inherent similarities in generic features can be useful for downstream tasks, thus reducing the cost of downstream training. Specifically, the downstream model trainer uses a pre-trained upstream model as the starting point for downstream training, with the inclusion (or replacement) of task-specific classification layers/modules. The downstream model is then trained by either updating all layers of the model (including ones reused from the upstream model) or freezing some earlier layers of the reused parts as the ``feature extractor'' and only updating the rest. The latter approach is more popular as the reused feature extractors can already learn useful feature representations and the training cost is also much lower and affordable for individuals with limited computational resources. We study the vulnerability of the latter transfer learning approach in this paper. 
%mainly in two ways:  1) all the layers (including ones reused from ) and tune the full model; the other one is to freeze some earlier layers of the model as the feature extractor and only tune the rest later layers. The second update strategy could achieve better efficiency since the frozen layers can already produce meaningful feature representations~\cite{wang2018great,yao2019latent}, and we will study the transfer learning using this strategy. 

Recently, various attacks have been proposed for the transfer learning setting, but with different attack goals from ours. Wang et al.~\cite{wang2018great} generate adversarial examples against black-box student models that transfer knowledge from publicly available teacher models without repeated queries. Yao et al.~\cite{yao2019latent} propose to manipulate the upstream model such that the downstream models derived from the upstream model contain backdoors, which would misclassify test inputs that contain some predefined backdoor triggers. Zou et al.~\cite{zou2020privacy} study the threat of membership inference attacks on transfer learning and the upstream models are trained normally. In contrast, we investigate the possibility of boosting the effectiveness of property inference by manipulating the upstream model training. Schuster et al.~\cite{schuster2020humpty} show that the attacker can modify the corpus on which the word embedding is trained such that the downstream NLP models which use that embedding will behave abnormally.

%This additionally allows model trainers to achieve satisfactory performance with limited training samples, leading to reduced computational costs. The most common approach reuses parameters in the earlier layers of the pre-trained model, either by fixing them as the feature extractor or just using them for initialization, to conduct downstream training.

\subsection{Property Inference} \label{sec:property_inference}

\shortsection{Property Inference Attacks} In property inference attacks, the adversary aims to infer some sensitive properties of some data, given a model trained on it. For example, the adversary may be interested in sensitive properties like the presence of people of a specific race in the dataset~\cite{ateniese2015hacking, melis2019exploiting}), or even be curious about the 
the statistics of the training set (e.g, the ratio of people with a specific gender~\cite{saeed, ganju2018property, suri2022formalizing, zhang2021leakage}).


Ateniese et al.~\cite{ateniese2015hacking} were the first to identify the threat of inferring properties of the training data from pre-trained models. Ganju et al.~\cite{ganju2018property} and Suri and Evans~\cite{suri2022formalizing} 
study property inference against normally trained models, and they launch attacks using white-box meta-classifiers, which utilize the permutation-invariance representation~\cite{ganju2018property} of the model parameters, while other works focus on distributed training~\cite{zhang2021leakage} where the attacker is a participant in the global model training and conducts property inference using meta-classifiers trained on model outputs. Similarly, Suri et al.~\cite{suri2022subject} focus on federated learning, where the attacker is a participant (or the central server) that utilizes black-box attacks for inferring membership of data from particular subjects. Chase et al.~\cite{saeed} propose an active property inference attack for data poisoning scenarios, which we will cover and compare to in Section~\ref{sec:active_inference_attacks}.

%The closest work to ours are by Chase et al.~\cite{saeed} and Tramer et al.~\cite{tramer2022truth}. In their work, the attacker can manipulate some of the training data of the model such that a model trained (from scratch) on the poisoned data has an increased inference risk. However, their methods are not applicable to the transfer learning scenario. 
%In this work, we will focus on the property inference in transfer learning scenarios in which the attacker releases the upstream model and infer sensitive properties of the downstream models tuned from that upstream model.
% 

\shortsection{Defenses}
Defending against property inference attacks is an open problem. There are no studies in the current literature on active adversaries, and only a couple on passive ones. Ma et. al.~\cite{ma2021nosnoop} propose a defense against property inference attacks on data batches in the  collaborative learning setting. However, adversaries in the transfer-learning setting do not have access to batch-wise gradients of the downstream trainer. Chen and Ohrimenko~\cite{chen2022protecting} utilize mechanisms that add carefully-crafted noise to features to provide theoretical guarantees against inference adversaries, but focus on query-based access to the underlying dataset, not a machine learning model trained on it. These existing defenses thus do not apply to our threat model.

%propose a framework that reduces property inference to Boolean functions of individual members, posing the ratio of members satisfying the given function in a dataset as the property. These property inference attacks have since then been proposed as distribution inference attacks~\cite{suri2022formalizing}, presenting such attacks as inferring properties of the distributions used to sample datasets, differentiating them from exact inference attacks like dataset inference~\cite{maini2021dataset}. Nearly all property inference attacks use meta-classifiers to perform inference: training models on versions of datasets with and without the target property, followed by training a meta-classifier on top of these classifiers's model representations. These representations can take several forms: using model weights themselves with permutation-invariance~\cite{ganju2018property}, or model activations or logits for a generated set of query points~\cite{xu2019detecting}. However, the capability of such approaches is limited: the most that these attacks have been shown to work is medium-sized convolutional networks on the CelebA dataset~\cite{suri2022formalizing}.


\subsection{Active Privacy Attacks} \label{sec:active_inference_attacks}
% Perhaps the closely related works to ours as ones that proactively enhance the effectiveness of privacy attacks by manipulating the model training process in certain ways~\cite{saeed, melis2019exploiting, nasr2019comprehensive, tramer2022truth}. 
%shown that the adversary can, by using proactive ways, achieve stronger attacks that infer private information from deep learning systems~\cite{nasr2019comprehensive, melis2019exploiting, tramer2022truth, saeed}. In this section, we introduce the ones that are close to ours.

In the decentralized federated learning training, by submitting specially crafted gradients to the central server, malicious agents can increase membership inference risk~\cite{nasr2019comprehensive} and property inference risks~\cite{melis2019exploiting} of other benign agents' training data. However, these attacks do not apply to transfer learning scenario, as the attacker cannot control model gradients of downstream training. In the centralized setting, researchers propose attacks to poison the victim's training data such that the impacts of attribute inference and membership inference~\cite{tramer2022truth} and property inference~\cite{saeed} attacks are amplified on the poisoned model.
The ability to poison the victim's data is a threat model orthogonal to ours, since we have no access to the victim's downstream data. While there is scope to combine such approaches for stronger attacks (albeit with stronger access assumptions), we choose to focus on the scenario with no read/write access to the victim's data.

\fi %%%%%%%%%%%%%%%%%%%%%%%%%%%%%%%%

\section{Linear Shortcut Across Blocks}
\label{sec:layer_jump}

To use a hidden representation from layer $\ell<L$ as a final representation, we propose to cast it using linear regression, while skipping the computation in-between these layers. More generally, this approach can be applied to cast any $\ell$-th hidden representation to any subsequent layer $\ell'>\ell$.


\subsection{Method}
\label{subsec:methodology_linear_shortcut}

Given a source layer $\ell$ and a target layer $\ell'$ such that $0 \leq \ell < \ell' \leq L$, our goal is to learn a mapping
%$A_{\ell', \ell} \in \mathbb{R}^{d_h \times d_h}$
from hidden representations at layer $\ell$ to those at layer $\ell'$. To this end, we first collect a set of corresponding hidden representation pairs $(h^\ell, h^{\ell'})$. Concretely, we run a set $\mathcal{T}$ of input sequences through the model, and for each input $s$, we extract the hidden representations $h_{i_s}^{\ell}, h_{i_s}^{\ell'}$, where $i_s$ is a random position in $s$.
Next, we learn a matrix $A_{\ell', \ell} \in \mathbb{R}^{d_h \times d_h}$ by fitting linear regression over $\mathcal{T}$, i.e., $A_{\ell', \ell}$ is a numerical minimizer for:
$$ A \mapsto \sum_{s \in \mathcal{T}} || A \cdot h_{i_s}^\ell - h_{i_s}^{\ell'} ||^2,$$ 
and define the mapping of a representation $h$ from layer $\ell$ to layer $\ell'$ as:
\begin{equation}
\label{eq:linear_jump}
    \matl{} (h) \coloneqq A_{\ell', \ell} \cdot h.
\end{equation}


\subsection{Baseline}
\label{subsec:baseline}

We evaluate 
% our method against 
the prevalent approach of ``reading'' hidden representations directly, without any transformation. 
Namely, the propagation of a hidden representation from layer $\ell$ to layer $\ell'$ is given by the identity function, dubbed \id{}:

$$ \idl{} (h) \coloneqq h.$$

% Notably, 
This baseline 
assumes that representations at different layers operate in the same linear space.

\subsection{Quality of Fit}
\label{subsec:experiments_r2}

We first evaluate our method by measuring how well the learned linear mappings approximate the representations at the target layer. To this end, we calculate the (coordinate-averaged) $r^2$-score of our mapping's outputs with respect to the representations obtained from a full inference pass, and compare to the same for the \id{} baseline.


\paragraph{Models.}

We use \gpt{} \cite{radford2019language}, a decoder-only auto-regressive LM, with $L = 48$, $d_h = 1600$, and \bert{} \cite{devlin-etal-2019-bert}, an encoder-only model trained with masked language modeling, with $L=24$, $d_h=1024$.
% \footnote{\label{footnote:hf}We use models and data from Huggingface \cite{wolf-etal-2020-transformers,lhoest-etal-2021-datasets}.}
%For masked token prediction, we use a masked LM head pre-trained for our \bert{} model.

% \footnote{Specifically, we use the Huggingface Transformers \cite{wolf-etal-2020-transformers} implementations of all these models.}

%\sy{We use \gpt{} \cite{radford2019language}, a decoder-only auto-regressive LM, coming in four scales; $\texttt{gpt2}$ ($L = 12$, $d_h = 768$), $\texttt{gpt2-medium}$ ($L = 24$, $d_h = 1024$), $\texttt{gpt2-large}$ ($L = 36$, $d_h = 1280$) and $\texttt{gpt2-xl}$ ($L = 48$, $d_h = 1600$). Also, we use \bert{} \cite{devlin-etal-2019-bert}, an encoder-only model trained with masked language modeling, coming in two scales;  \texttt{bert-base-uncased} ($L=12$, $d_h=768$) and \texttt{bert-large-uncased} ($L=24$, $d_h=1024$). For masked token prediction, we use masked LM heads pre-trained for our models. Specifically, we use the Huggingface Transformers \cite{wolf-etal-2020-transformers} implementations of all these models. The plots presented in this section are for $48$-layered \gpt{} and $24$-layered \bert{}.}

%\sy{We use \gpt{} \cite{radford2019language}, a decoder-only auto-regressive LM, in the Huggingface \cite{wolf-etal-2020-transformers} implementation\footnote{\url{https://huggingface.co/gpt2}}, coming in four scales; $\texttt{gpt2}$ ($L = 12$, $d_h = 768$), $\texttt{gpt2-medium}$ ($L = 24$, $d_h = 1024$), $\texttt{gpt2-large}$ ($L = 36$, $d_h = 1280$) and $\texttt{gpt2-xl}$ ($L = 48$, $d_h = 1600$). Also, we use \bert{} \cite{devlin-etal-2019-bert}, an encoder-only model trained with masked language modeling, in the Hugginface implementation, coming in two scales;  \texttt{bert-base-uncased}\footnote{\url{https://huggingface.co/bert-base-uncased}} ($L=12$, $d_h=768$) and \texttt{bert-large-uncased}\footnote{\url{https://huggingface.co/bert-large-uncased}} ($L=24$, $d_h=1024$). For masked token prediction, we use the \texttt{BertForMaskedLM} heads from Huggingface, pretrained for these models. The plots presented in this section are for $48$-layered \gpt{} and $24$-layered \bert{}.}

\paragraph{Data.}
We sample random sentences from Wikipedia,
% \footref{footnote:hf} 
collecting 9,000 (resp. 3,000) sentences for the training set $\mathcal{T}$ (resp. validation set $\mathcal{V}$).\footnote{We use sentences rather than full documents to simplify the analysis.}
%\sy{We use two data sources to evaluate our method. One is Wikiepdia \cite{lhoest-etal-2021-datasets}\footnote{\url{https://huggingface.co/datasets/wikipedia}}; we use \texttt{spaCy}\footnote{\url{https://spacy.io/}} to divide documents into sentences\footnote{We use sentences rather than full documents to simplify the analysis.}\footnote{We pick randomly a Wikipedia document and then pick randomly a sentence ending in a newline character in it. \sy{[maybe this footnote is not needed?]}}, collecting 9,000 (resp. 3,000) random sentences for the training set $\mathcal{T}$ (resp. validation set $\mathcal{V}$). The second is a news article sentences dataset, the 10K English 2020 news sentences corpus
% \footnote{\url{https://downloads.wortschatz-leipzig.de/corpora/eng_news_2020_10K.tar.gz}} from the Leipzig Corpora Collection \cite{goldhahn-etal-2012-building}, which we randomly divide into a training set $\mathcal{T}$ consisting of 9,000 examples and a validation set $\mathcal{V}$ consisting of 1,000 examples.
% We truncate sentences to the maximal token length allowed by the model \mg{do we ever need to truncate? a sentence has about 10 words and the max. input len is thousands} \sy{[I surely did not need to in Leipzig, but discovered (via a transformers runtime warning) that I do need to for some (probably a minority) of the Wikipedia sentences. This probably has to do with that it is not really ``sentences" necessarily, for example, I noticed that it has some listings or something like that (bulleted items)... So some minority might get very long I guess...]}.
For each example $s$, we select a random position $i_s$ and extract the hidden representations $h_{i_s}^{\ell}$ at that position from all the layers.
For \bert{}, we first replace the input token at position $i_s$ with a \mask{} token, as our motivation is interpreting predictions, which are obtained via masked tokens in \bert{} (see \S\ref{subsec:BERT}).
Thus, in this case, the hidden representations we consider
%in the case of \bert{}
are of \mask{} tokens only.
%As we observed highly similar results for the two data sources across all our experiments, throughout the paper we will mainly report results for Wikipedia (except for \S\ref{sec:robustness}, where we cross-validate).


\begin{figure}[t]
\includegraphics[scale=0.2]{figs/r2_scores_48.pdf}
% \includegraphics[width=\columnwidth]{figs/r2_scores_48.pdf}
\caption{The coordinate-averaged $r^2$-score of $\matl{}$ (left) and $\idl{}$ (right) (\gpt{}).}
\label{fig:r2_scores}
\end{figure}


\begin{figure}[t]
\setlength{\belowcaptionskip}{-10pt}
\includegraphics[scale=0.2]{figs/bertmask_r2_scores_24.pdf}
% \includegraphics[width=\columnwidth]{figs/bertmask_r2_scores_24.pdf}
\caption{The coordinate-averaged $r^2$-score of $\matl{}$ (left) and $\idl{}$ (right) (\bert{}).}
\label{fig:bertmask_r2_scores}
\end{figure}



\paragraph{Evaluation.}
For every pair of layers $\ell, \ell'$, such that $0 \leq \ell < \ell' \leq L$, we use the training set $\mathcal{T}$ to fit linear regression as described in \S\ref{subsec:methodology_linear_shortcut}, and obtain a mapping $\matl{}$. 
Next, we evaluate the quality of $\matl{}$ as well as of $\idl{}$ using the $r^2$-coefficient, uniformly averaged over all coordinates. Concretely, we compute the $r^2$-coefficient of each of the predicted representations $\matl{} (h_{i_s}^{\ell})$ and $\idl{} (h_{i_s}^{\ell})$ versus the true representations $h_{i_s}^{\ell'}$
over all $s \in \mathcal{V}$.
%as we vary $s \in \mathcal{V}$.
%for every $s \in \mathcal{V}$.



\paragraph{Results.}
Results for \gpt{} and \bert{} are presented in Figs.~\ref{fig:r2_scores} and~\ref{fig:bertmask_r2_scores}, respectively.
In both models, \mat{} consistently yields better approximations than \id{}, as it obtains higher $r^2$-scores (in blue) across the network. 
This gap between \mat{} and \id{} is especially evident in \bert{}, where \id{} completely fails to map the representations between most layers, suggesting that hidden representations are modified  substantially by every transformer block.
Overall, this highlights the shortcoming of existing practices to inspect representations in the same linear space, and the gains from using our method to approximate future layers.
% in the network.
\section{Linear Shortcut for Language Modeling}
\label{sec:prediction}

We saw that our method approximates future hidden representations substantially better than a naive propagation. 
In this section, we will show that this improvement also translates to better predictive abilities from earlier layers. Specifically, we will use our method to estimate how often intermediate representations encode the final prediction, in the context of two fundamental LM tasks; next token prediction and masked token prediction.

\paragraph{Evaluation Metrics.}
Let $h, h' \in \mathbb{R}^{d_h}$ be a final representation and a substitute final representation obtained by some mapping, and denote by $\delta (h), \delta (h') \in \mathbb{R}^{d_v}$ their corresponding output probability distributions (obtained through projection to the output vocabulary -- see details below). 
We measure the prediction quality of $h'$ with respect to $h$ using two metrics:
\begin{itemize}
[leftmargin=*,topsep=1pt,parsep=1pt]
    \item \textbf{Precision@$k$} ($\uparrow$ is better): This checks whether the token with the highest probability according to $\delta(h')$ appears in the top-$k$ tokens according to $\delta(h)$. Namely, we sort $\delta(h)$ and assign a score of $1$ if $\arg\max(\delta(h'))$ appears in the top-$k$ tokens by $\delta(h)$, and $0$ otherwise.
    
    \item \textbf{Surprisal} ($\downarrow$ is better): We measure the minus log-probability according to $\delta(h)$, of the highest-probability token according to $\delta(h')$. Intuitively, low values mean that the model sees the substitute result as probable and hence not surprising.
\end{itemize}

\noindent We report the average Precision@$k$ and Surprisal over the validation set $\mathcal{V}$.



\subsection{Next Token Prediction}
\label{subsec:next_token_prediction_task}

Auto-regressive LMs output for every position a probability distribution over the vocabulary for the next token. Specifically, the output distribution for every position $i$ is given by $\delta (h_i^L)$, where:
\begin{equation}\label{eq:output_distribution}
    \delta (h) = \texttt{softmax} ( E^\top \cdot h) \in \mathbb{R}^{d_v}
\end{equation}
For some LMs, including \gpt{}, a layer normalization $\texttt{ln\_f}$ is applied to the final layer representation before this conversion (i.e., computing $\delta (\texttt{ln\_f}(h))$ rather than $\delta (h)$).

Recall that our goal is to measure how well this distribution can be estimated from intermediate representations, i.e. estimating $\delta (h_i^L)$ from $\delta (h_i^\ell)$ where $\ell<L$. To this end, we first run examples from the validation set through the model, while extracting for each example $s$ the hidden representation of a random position $i_s$ at every layer. Next, we apply our mappings $\matlL{}$ and the $\idlL{}$ baseline to cast the hidden representations of every layer $\ell$ to final layer substitutes (see \S\ref{sec:layer_jump}). Last, for each layer, we convert its corresponding final-layer substitute to an output distribution (Eq.~\ref{eq:output_distribution}) and compute the average Precision@$k$ (for $k=1,5,10$) and Surprisal scores with respect to the final output distribution, over the validation set.

\paragraph{Results.}
Figs.~\ref{fig:pre} and~\ref{fig:surp} show the average Precision@$k$ and Surprisal scores per layer in $48$-layered \gpt{}, respectively (the plots for the other \gpt{} models are presented in \S\ref{sec:app_scale}). Across all layers, \mat{} outperforms \id{} in terms of both scores, often by a large margin (e.g. till layer $44$ the Precision@$1$ achieved by \mat{} is bigger than that of $\id{}$ by more than $0.2$). 
This shows that linear mappings enable not just better estimation of final layer representations, but also of the predictions they induce. Moreover, the relatively high Precision@$k$ scores of \mat{} in early layers ($0.62$-$0.82$ for $k=10$, $0.52$-$0.74$ for $k=5$, and $0.28$-$0.45$ for $k=1$) suggest that early representations already encode a good estimation of the final prediction. Also, the substantially lower Surprisal scores of \mat{} compared to \id{} imply that our method allows for a more representative reading into the layer-wise prediction-formation of the model than allowed through direct projection to the vocabulary.

\begin{figure}[t]
\centering
\includegraphics[scale=0.4]{figs/pre_48.pdf}
\caption{Precision@$k$ ($k = 1,5, 10$) of $\matlL{}$ and $\idlL{}$ for next token prediction in $48$-layered \gpt{}.}
\label{fig:pre}
\end{figure}

\begin{figure}[t]
\centering
\includegraphics[scale=0.35]{figs/surp_48.pdf}
\caption{Surprisal for $\matlL$ and the baseline $\idlL{}$ ($48$-layered \gpt{} next token prediction task). A 95\% confidence interval surrounds the lines.}
\label{fig:surp}
\end{figure}

\subsection{Masked Token Prediction}
\label{subsec:BERT}

We now conduct the same experiment for the task of masked language modeling, where the model predicts a probability distribution of a masked token in the input rather than the token that follows the input. Unlike next token prediction, where the output distribution is computed from representations of varying input tokens, in masked token prediction the output is always obtained from representations of the same input token (i.e. \texttt{[MASK]}).

For this experiment, we use \bert{}, on top of which we use a pretrained masked language model head $\delta$; given a token sequence $s$, a \mask{} token inside it and its final representation $h$, $\delta (h) \in \mathbb{R}^{d_v}$
 is a probability distribution over tokens giving the model's assessment
 of the likelihood of tokens to be fitting in place of the \mask{} token in $s$.


\begin{figure}[t]
\centering
\includegraphics[scale=0.4]{figs/bertmask_pre_24.pdf}
\caption{Precision@$k$ ($k = 1,5, 10$) for  $\matlL{}$ and the baseline $\idlL{}$ ($24$-layered \bert{} masked token prediction task).}
\label{fig:bertmask_pre}
\end{figure}

\begin{figure}[t]
\centering
\includegraphics[scale=0.35]{figs/bertmask_surp_24.pdf}
\caption{Surprisal for $\matlL{}$ and the baseline $\idlL{}$ ($24$-layered \bert{} masked token prediction task). A 95\% confidence interval surrounds the lines.}
\label{fig:bertmask_surp}
\end{figure}

\paragraph{Results.}
Figs.~\ref{fig:bertmask_pre} and~\ref{fig:bertmask_surp} present the average Precision@$k$ and Surprisal scores per layer in $24$-layered \bert{} (the plots for the $12$-layered \bert{} model are presented in \S\ref{sec:app_scale}), overall showing trends similar to those observed for next token prediction in \gpt{} (\S\ref{subsec:next_token_prediction_task}). This is despite the differences between the two tasks and the considerable architectural differences between \bert{} and \gpt{}.
Notably, the superiority of \mat{} over \id{} in this setting is even more prominent; 
while \mat{}'s precision is between $0.2-0.6$ in the first ten layers (Fig.~\ref{fig:bertmask_pre}), \id{}'s precision for all values of $k$ is close to zero, again strongly indicating that our method allows for better reading into early layer hidden representations. 
More generally, \mat{} improves the Precision@$1$ of \id{} by more than $17\%$ at most layers, and unveils that a substantial amount of predictions ($>25\%$ starting from layer $3$) appear already in the very first layers.
Interestingly, the (rough) divide between the first half of layers and last half of layers for $\id{}$ in Figs.~\ref{fig:bertmask_pre},~\ref{fig:bertmask_surp} seems to align with the two-hump shape of the blue region for $\mat{}$ in Fig.~\ref{fig:bertmask_r2_scores}.

\paragraph{Analysis.}
We manually compare the predictions of our mapping $\matlL{}$ with $\idlL{}$, for a $24$-layered \bert{} model.  Concretely, we select 50 random sentences from the Leipzig dataset. Next, for each layer $\ell$, we manually analyze how many of the top-$5$ tokens according to $\matlL{}$ and $\idlL{}$ fit into context. We consider a token to fit into context if it is grammatically plausible within the sentence (see Tab.~\ref{tab:manual} for concrete examples).
In the resulting $1250$ instances (i.e. $50$ sentences $\times$ $25$ representations), we observe a substantially higher plausibility rate of $85.36\%$ for \mat{} compared to $52.8\%$ for \id{}. In fact, only in less than $4.3\%$ of the instances there are more plausible tokens among the top-$5$ tokens according to \id{} than among the top-$5$ tokens according to \mat{}, further supporting the Surprisal results above.

\begin{table*}
\footnotesize
\setlength{\belowcaptionskip}{-15pt}
\begin{tabular}{p{0.3\linewidth}ccccc}
& $\texttt{id}_{4 \rightarrow 24}$ & $\texttt{mat}_{4 \rightarrow 24}$ & $\texttt{id}_{12 \rightarrow 24}$ & $\texttt{mat}_{12 \rightarrow 24}$ & $\texttt{id}_{24 \rightarrow 24}$ \\ \midrule
\multirow{5}{=}{aldridge had shoulder surgery in \mask{}.} & fellowship & \tcbox{time} & cyclist & \tcbox{2009} & \tcbox{september} \\
& employment & \tcbox{it} & emergencies & \tcbox{2008} & \tcbox{november} \\
& agreement & her & seniors & \tcbox{2010} & \tcbox{december} \\
& \#\#ostal & them & cycling & \tcbox{2006} & \tcbox{august} \\
& \#\#com & work & \tcbox{pennsylvania} & \tcbox{2007} & \tcbox{july} \\ \midrule
\multirow{5}{=}{on your next view you will be asked to \mask{} continue reading.} & \#\#com & be & be & be & \tcbox{please} \\
& accreditation & get & undergo & \tcbox{please} & \tcbox{simply} \\ 
& $	\copyright$ & go & spartans & help & \tcbox{also} \\ 
& fellowship & \tcbox{help} & seniors & \tcbox{simply} & \tcbox{again} \\ 
& summer & have & * & say & \tcbox{immediately} \\ \bottomrule
\end{tabular}
\caption{Examples of top-$5$ predictions at layers $4$, $12$ and $24$, under the mappings $\matlL{}$ and $\idlL{}$, for a $24$-layered \bert{} model. Grammatically plausible predictions (according to a human annotator) are marked in \tcbox{blue}. Note that at layer $24$ the predictions of $\matlL{}$ and $\idlL{}$ are the same (by definition).} 
\label{tab:manual}
\end{table*}

\section{Implication to Early Exiting}
\label{sec:applications}

%The fact that it is often possible to approximate
The possibility of approximating
the final prediction already in the early layers has important implications for efficiency; applying our linear mapping instead of executing transformer blocks of quadratic time complexity, could save a substantial portion of the computation. In this section, we demonstrate this in the context of early exiting.

When 
% performing transformer model inference under 
using an early exit strategy \cite{schwartz-etal-2020-right, xin-etal-2020-deebert, schuster2022confident}, one aims at deciding dynamically at which layer to stop the computation and ``read'' the prediction from the hidden representation of that layer.
More precisely, under a confidence measure paradigm, one decides to stop the computation for a position $i$ at layer $\ell$ based on a confidence criterion, that is derived from casting the hidden representation $h_i^\ell$ as a final-layer representation and converting it to an output probability distribution. Specifically, following \citet{schuster2022confident}, a decision to exit is made if the difference between the highest and the second highest probabilities is bigger than $$ 0.9 \cdot \lambda + 0.1 \cdot {\rm exp} (-4 i / N),$$
where $N$ is the average length of the input until position $i_s$ for $s \in \mathcal{V}$, and $\lambda$ is a hyper-parameter.

\begin{figure}[t]
\setlength{\belowcaptionskip}{-10pt}
\centering
\includegraphics[width=\columnwidth]{figs/ee_gpt2bert.pdf}
\caption{Precision@$1$ with early exit and ``fixed exit'', applied to the $24$-layer \gpt{} for next token prediction (left) and the $24$-layer \bert{} for masked token prediction (right). Varying the confidence parameter $\lambda$, the $x$-coordinate is the average number of layers processed before an early exit decision is reached.}
\label{fig:ee_gpt2bert}
\end{figure}

\quash{
\begin{figure}[t]
\setlength{\belowcaptionskip}{-10pt}
\centering
\includegraphics[scale=0.35]{figs/ee_pre1_24.pdf}
\caption{Precision@$1$ for the various early exit methods, and previous ``fixed exit'' methods for comparison ($24$-layer \gpt{} next token prediction task). Varying the confidence parameter $\lambda$, the $x$-coordinate is the average number of layers processed before an early exit decision is reached.}
\label{fig:ee_pre1}
\end{figure}
}

\paragraph{Experiment.}
We assess the utility of our mapping $\matlL{}$ for early exit as a plug-and-play replacement for $\idlL{}$, through which intermediate representations are cast into final-layer representations.
We use \gpt{} for the next token prediction and \bert{} for masked token prediction (both with 24 layers).
We run each of the models over the validation set examples, while varying the confidence parameter $\lambda$ and using either $\idlL{}$ or $\matlL{}$ for casting intermediate representations.
Furthermore, we compare these early exit variants to the ``fixed exit'' strategy from \S\ref{sec:prediction}, where the computation is stopped after a pre-defined number of layers rather than relying on a dynamic decision.
We evaluate each variant in terms of both prediction's accuracy, using the Precision@$1$ metric (see \S\ref{sec:prediction}), and efficiency, measured as the average number of transformer layers processed during inference.


\paragraph{Results.}
%Figs.~\ref{fig:ee_pre1} and~\ref{fig:bertmask_ee_pre1}
Fig.~\ref{fig:ee_gpt2bert}
plots the average Precision@$1$ score against the average number of layers processed, for $24$-layer \gpt{} and $24$-layer \bert{}. For both models, under an early exit strategy our mapping \mat{} again provides a substantial improvement over \id{}.
For example, aiming at $95\%$ average precision, \mat{} saves $\sim3.3$ ($13.8$\%) layers in \gpt{} compared to only $\sim1.4$ ($5.9$\%) layers by \id{}, and $\sim4.8$ ($20$\%) layers in \bert{} versus $\sim3.5$ ($14.6$\%) layers by \id{}.
These results highlight the potential gains prominent early exit methods can obtain by using our method.
Notably, in both models and for each of the mapping methods, early exit obtains better results than fixed layer exit, as expected. 

\quash{
\begin{figure}[t]
\setlength{\belowcaptionskip}{-10pt}
\centering
\includegraphics[scale=0.35]{figs/bertmask_ee_pre1_24.pdf}
\caption{Precision@$1$ for the various early exit methods, and previous ``fixed exit'' methods for comparison ($24$-layer \bert{} masked token prediction task). Varying the confidence parameter $\lambda$, the $x$-coordinate is the average number of layers processed before an early exit decision is reached.}
\label{fig:bertmask_ee_pre1}
\end{figure}
}
\section{Linear Shortcut Across Sub-Modules}
\label{sec:submodules}

% Our experiments show that
% , despite the commonly-applied simplification by interpretability works, transformer layers do not operate in the same linear space and 
% there is a major gap in approximating future representations using an identity mapping (\S\ref{sec:layer_jump}, \S\ref{sec:prediction}).
% Here, 
In this section, we investigate whether discrepancies across layers result from specific sub-modules or are a general behaviour of all sub-modules in the network.  
This is done by extending our approach to test how well particular components in transformer blocks can be linearly approximated. 


\paragraph{Method.}

Consider \gpt{} for definiteness, then:
% we have 
$$ \texttt{b}_{\ell} = \texttt{b}_{\ell}^{\texttt{ffn}} \circ \texttt{b}_{\ell}^{\texttt{attn}}$$ 
% with
\begin{equation}\label{eq:attn} \texttt{b}^{\texttt{attn}}_{\ell} (H) = \texttt{attn}_{\ell} (\texttt{ln1}_{\ell} (H)) + H,\end{equation} 
where $\texttt{attn}_{\ell}$ is
%a multi-head self-attention
a MHSA
layer and \texttt{ln1} is a layer normalization (LN), and 
$$ \texttt{b}^{\texttt{ffn}}_{\ell} (H) = \texttt{ffn}_{\ell} (\texttt{ln2}_{\ell} (H)) + H,$$  
where $\texttt{ffn}_{\ell}$ is
%a feed-forward network
an FFN
layer and $\texttt{ln2}$ is a LN.
\quash{
Given a block $\texttt{b}_\ell$ and one of its sub-modules $\texttt{ln1}_\ell, \ \texttt{attn}_\ell, \ \texttt{ln2}_\ell$, or $\texttt{ffn}_\ell$, we fit linear regression approximating the output of the sub-module given its input and then use it in order to define mappings, as we now describe.
}
Given a block $\texttt{b}_\ell$ and one of its sub-modules $\texttt{ln1}_\ell, \ \texttt{attn}_\ell, \ \texttt{ln2}_\ell$, or $\texttt{ffn}_\ell$, we fit linear regression approximating the output of the sub-module given its input, and then use it to define mappings $\matattnl{}$, $\matlnl{}$ and $\matffl{}$.
%We provide the definition of $\matattnl{}$ below, and that of the other two in App. \ref{sec:app_submodule_skip_description}.
We provide the formal definitions of these mappings in App. \ref{sec:app_submodule_skip_description}.
\iffalse
\paragraph{$\matattnl{}$.}
%Illustrating this on $\texttt{attn}_\ell$ for definiteness,
For an input $s$, let $v^\ell_{i_s}$ be the vector at position $i_s$ in the output of $\texttt{attn}_\ell (\texttt{ln1}_\ell (H^{\ell - 1}))$. We denote by $A_\ell^{\texttt{attn}} \in \mathbb{R}^{d_h \times d_h}$ the matrix numerically minimizing 
$$ A \mapsto \sum_{s \in \mathcal{T}} || A \cdot \texttt{ln1}_\ell (h^{\ell-1}_{i_s}) - v^\ell_{i_s}||^2,$$
and define an attention sub-module replacement (Eq.~\ref{eq:attn}) by $$
\texttt{b}^{\overline{\texttt{attn}}}_\ell (h) \coloneqq A_{\ell}^{\texttt{attn}} \cdot \texttt{ln1}_\ell (h) + h. $$
We then define a mapping between two layers ${\ell \rightarrow \ell'}$ by:
$$ \matattnl{} (h) \coloneqq $$
$$ \texttt{b}^{\texttt{ffn}}_{\ell'} ( \texttt{b}^{\overline{\texttt{attn}}}_{\ell'} ( \ldots (\texttt{b}^{\texttt{ffn}}_{\ell+1} ( \texttt{b}^{\overline{\texttt{attn}}}_{\ell+1} (h)))\ldots)).$$ 
Namely, when applying each $\ell''$-th block, $\ell < \ell'' \leq \ell'$, we replace its attention sub-module $\texttt{attn}_{\ell''}$ by its linear approximation.
%In an analogous way, we consider the mappings $\matffl{}$ and $\matlnl{}$, where in the latter we perform the linear shortcut both for \texttt{ln1} and for \texttt{ln2} (see~\S\ref{sec:app_submodule_skip_description} for precise descriptions).
Importantly, unlike the original attention module, the approximation $\texttt{b}^{\overline{\texttt{attn}}}_\ell$ operates on each position independently, and therefore applying $\matattnl{}$ disables any contextualization between the layers $\ell$ and $\ell'$. Note that this is not the case for $\matffl{}$ and $\matlnl{}$, which retain the self-attention sub-modules and operate contextually.
\fi

\paragraph{Evaluation.}


We analyze the $24$-layered \gpt{}, and proceed completely analogously to \S\ref{subsec:next_token_prediction_task}, evaluating the Precision@$1$ and Surprisal metrics for the mappings $\matattnlL{}$, $\matfflL{}$ and $\matlnlL{}$.

\begin{figure}[t]
\setlength{\belowcaptionskip}{-0pt}
\centering
%\includegraphics[scale=0.2]
\includegraphics[width=\columnwidth]{figs/parts_presurp_24.pdf}
\caption{Precision@$1$ and Surprisal for the various sub-module linear mappings, and $\matlL{}$ for comparison ($24$-layer \gpt{} next token prediction task). A 95\% confidence interval surrounds the Surprisal lines.}
\label{fig:parts_presurp}
\end{figure}

\quash{
\begin{figure}[t]
\centering
\includegraphics[scale=0.4]{figs/parts_pre1_24.pdf}
\caption{Precision@$1$ for the various sub-module linear shortcut mappings, and the mapping $\matlL{}$ for comparison (\gpt{} next token prediction task).}
\label{fig:parts_pre1}
\end{figure}

\begin{figure}[t]
\centering
\includegraphics[scale=0.35]{figs/parts_surp_24.pdf}
\caption{Surprisal for the various sub-module linear shortcut mappings, and the mapping $\matlL{}$ for comparison (\gpt{} next token prediction task). A 95\% confidence interval surrounds the lines.}
\label{fig:parts_surp}
\end{figure}
}

\paragraph{Results.}
Fig.~\ref{fig:parts_presurp} shows the average Precision@$1$ and Surprisal scores per layer.
From a certain layer (\textasciitilde$7$), all sub-module mappings achieve better results than the full-block mapping $\matlL{}$. Thus, it is not just the cumulative effect of all the sub-modules in the transformer block that is amenable to linear approximation, but also individual sub-modules can be linearly approximated. 
Furthermore, the linear approximation of attention sub-modules is less harmful than that of the FFN or LN sub-modules. 
% Hypothetically, 
A possible reason is that the linear replacement of FFN or LN ``erodes'' the self-attention computation after a few layers. 
Moreover, the good performance of $\matattnlL{}$ suggests that contextualization often exhausts itself in early layers; speculatively, it is only in more delicate cases that the self-attention of late layers adds important information. Last, remark the sharp ascent of the scores for layer normalization in layers $5$-$8$, for which we do not currently see a particular reason. To conclude, we see that the possibility of linear approximation permeates
%the various
transformer components.


\section{Related Work}

Recently, there was a lot of interest in utilizing intermediate representations in transformer-based LMs, both for interpretability and for efficiency.

In the direction of interpretability, one seeks to understand the prediction construction process of the model \cite{tenney-etal-2019-bert, voita-etal-2019-bottom}.

More recent works use mechanistic interpretability and view the inference pass as a residual stream of information \cite{dar2022analyzing,geva-etal-2022-transformer}. Additionally, there are works on probing, attempting to understand what features are stored in the hidden representations \cite{adi2017finegrained, conneau-etal-2018-cram,liu-etal-2019-linguistic}. Our work is different in that it attempts to convert intermediate representations into a final-layer form, which is interpretable by design.

In the direction of efficiency, there is the thread of work on early exit, where computation is cut at a dynamically-decided earlier stage \cite{schwartz-etal-2020-right,xin-etal-2020-deebert,schuster2022confident}. Other works utilize a fixed early stage network to parallelize inference \citep{leviathan2022fast, chen2023accelerating}. However, intermediate representations are directly propagated in these works, which we show is substantially worse than our approach. Moreover, our method requires training considerably less parameters than methods such as \citet{schuster-etal-2021-consistent}, that learn a different output softmax for each intermediate layer.  

More broadly, skipping transformer layers and analyzing the linearity properties of transformer components have been discussed in prior works \cite{Zhao2021of,mickus-etal-2022-dissect,wang-etal-2022-skipbert,lamparth2023analyzing}.


\section{Conclusion and Future Work}

We present a simple and effective method for enhancing utilization of hidden representations in transformer-based LMs, that uses 
pre-fitted context-free and token-uniform linear mappings.
Through a series of experiments on different data sources, model architectures and scales, we show that our method consistently outperforms the prevalent practice of interpreting representations in the final-layer space of the model, yielding better approximations of succeeding representations and the predictions they induce, thus allowing a more faithful interpretation of the model's prediction-formation.
We demonstrate the practicality of our method for improving computation efficiency, saving a substantial amount of compute on top of prominent early exiting approaches. 
Also, by extending our method to sub-modules, 
% more specifically the attention sub-modules, 
we observe that replacing a part of the transformer inference by a non-contextual linear computation often results in a small deterioration of the prediction.
This opens new research directions for improving model efficiency,
% and parallelizability.
% including breaking the computation into several parallelizable tasks.
including breaking the computation into parallel tasks.

\section*{Limitations}

Although we see in this work that there is more linear structure to transformer inference than could be explained solely by the residual connection, we do not elucidate a reason for that. We also do not try to formulate formal criteria according to which to judge, in principle, the quality of ways of short-cutting transformer inference in-between layers. In addition, our experiments cover only English data.


%\section*{Ethics Statement}
%Scientific work published at ACL 2023 must comply with the ACL Ethics Policy.\footnote{\url{https://www.aclweb.org/portal/content/acl-code-ethics}} We encourage all authors to include an explicit ethics statement on the broader impact of the work, or other ethical considerations after the conclusion but before the references. The ethics statement will not count toward the page limit (8 pages for long, 4 pages for short papers).

\section*{Acknowledgements}

We thank Tal Schuster for constructive comments.

% Entries for the entire Anthology, followed by custom entries
\bibliography{anthology,custom}
\bibliographystyle{acl_natbib}

\appendix

\section{Descriptions of $\matattn{}$, $\matff{}$ and $\matln{}$}
\label{sec:app_submodule_skip_description}

Here we detail the definitions of the mappings $\matattnl{}$, $\matffl{}$ and $\matlnl{}$ utilized in \S\ref{sec:submodules}.

\paragraph{Description of $\matattnl{}$.}
%Illustrating this on $\texttt{attn}_\ell$ for definiteness,
For an input $s$, let $v^\ell_{i_s}$ be the vector at position $i_s$ in the output of $\texttt{attn}_\ell (\texttt{ln1}_\ell (H^{\ell - 1}))$. We denote by $A_\ell^{\texttt{attn}} \in \mathbb{R}^{d_h \times d_h}$ the matrix numerically minimizing 
$$ A \mapsto \sum_{s \in \mathcal{T}} || A \cdot \texttt{ln1}_\ell (h^{\ell-1}_{i_s}) - v^\ell_{i_s}||^2,$$
and define an attention sub-module replacement (Eq.~\ref{eq:attn}) by $$
\texttt{b}^{\overline{\texttt{attn}}}_\ell (h) \coloneqq A_{\ell}^{\texttt{attn}} \cdot \texttt{ln1}_\ell (h) + h. $$
We then define a mapping between two layers ${\ell \rightarrow \ell'}$ by:
$$ \matattnl{} (h) \coloneqq $$
$$ \texttt{b}^{\texttt{ffn}}_{\ell'} ( \texttt{b}^{\overline{\texttt{attn}}}_{\ell'} ( \ldots (\texttt{b}^{\texttt{ffn}}_{\ell+1} ( \texttt{b}^{\overline{\texttt{attn}}}_{\ell+1} (h)))\ldots)).$$ 
Namely, when applying each $\ell''$-th block, $\ell < \ell'' \leq \ell'$, we replace its attention sub-module $\texttt{attn}_{\ell''}$ by its linear approximation.
%In an analogous way, we consider the mappings $\matffl{}$ and $\matlnl{}$, where in the latter we perform the linear shortcut both for \texttt{ln1} and for \texttt{ln2} (see~\S\ref{sec:app_submodule_skip_description} for precise descriptions).
Importantly, unlike the original attention module, the approximation $\texttt{b}^{\overline{\texttt{attn}}}_\ell$ operates on each position independently, and therefore applying $\matattnl{}$ disables any contextualization between the layers $\ell$ and $\ell'$. Note that this is not the case for $\matffl{}$ and $\matlnl{}$, which retain the self-attention sub-modules and operate contextually.

\paragraph{Description of $\matffl{}$.}
Let $v^\ell_{i_s}$ be the vector at position $i_s$ in the output of $\texttt{ln2}_{\ell} (\texttt{b}_\ell^{\texttt{attn}} (H^{\ell - 1}))$, for a given input $s$. We denote by $A_\ell^{\texttt{ffn}} \in \mathbb{R}^{d_h \times d_h}$ the matrix numerically minimizing 
$$ A \mapsto \sum_{s \in \mathcal{T}} || A \cdot v^{\ell}_{i_s} - \texttt{ffn}_{\ell} (v^\ell_{i_s})||^2,$$
and define a replacement of the feed-forward sub-module $\texttt{b}_{\ell}^{\texttt{ffn}}$ by $$ \texttt{b}^{\overline{\texttt{ffn}}}_\ell (H) \coloneqq A_{\ell}^{\texttt{ffn}} \cdot \texttt{ln2}_\ell (H) + H.$$
We then define a mapping between two layers ${\ell \rightarrow \ell'}$ by:
$$ \matffl{} (H) \coloneqq $$
$$ \texttt{b}^{\overline{\texttt{ffn}}}_{\ell'} ( \texttt{b}^{\texttt{attn}}_{\ell'} ( \ldots (\texttt{b}^{\overline{\texttt{ffn}}}_{\ell+1} ( \texttt{b}^{\texttt{attn}}_{\ell+1} (H))\ldots)).$$

\paragraph{Description of $\matlnl{}$.}
Let $v^\ell_{i_s}$ be the vector at position $i_s$ in the output of $\texttt{b}^{\texttt{attn}}_{\ell} (H^{\ell - 1})$, for a given input $s$. We denote by $A_\ell^{\texttt{ln1}} \in \mathbb{R}^{d_h \times d_h}$ the matrix numerically minimizing 
$$ A \mapsto \sum_{s \in \mathcal{T}} || A \cdot h^{\ell}_{i_s} - \texttt{ln1}_{\ell} (h^\ell_{i_s})||^2$$ and we denote by $A_\ell^{\texttt{ln2}} \in \mathbb{R}^{d_h \times d_h}$ the matrix numerically minimizing $$ A \mapsto \sum_{s \in \mathcal{T}} || A \cdot v^{\ell}_{i_s} - \texttt{ln2}_{\ell} (v^\ell_{i_s})||^2.$$ We define a replacement of the block $\texttt{b}^{\texttt{attn}}_{\ell}$ by \begin{equation} \texttt{b}^{\overline{\texttt{ln1}}}_\ell (H) \coloneqq \texttt{attn}_{\ell} (A_{\ell}^{\texttt{ln1}} \cdot H) + H\end{equation} and we define a replacement of the block $\texttt{b}^{\texttt{ffn}}_{\ell}$ by \begin{equation} \texttt{b}^{\overline{\texttt{ln2}}}_\ell (H) \coloneqq \texttt{ffn}_{\ell} (A_{\ell}^{\texttt{ln2}} \cdot H) + H.\end{equation}
We then define a mapping between two layers ${\ell \rightarrow \ell'}$ by:
$$ \matlnl{} (H) \coloneqq $$
$$ \texttt{b}^{\overline{\texttt{ln2}}}_{\ell'} ( \texttt{b}^{\overline{\texttt{ln1}}}_{\ell'} ( \ldots (\texttt{b}^{\overline{\texttt{ln2}}}_{\ell+1} ( \texttt{b}^{\overline{\texttt{ln1}}}_{\ell+1} (H))\ldots)).$$


\end{document}


% ---------- Appendix
\newpage
\clearpage

\appendix

\noindent
{\bfseries\sffamily\LARGE{Appendix}}

\renewcommand{\theequation}{B-\arabic{equation}}
\setcounter{equation}{0}  % reset counter 

\section{Examples in Mathematica, MATLAB and QuTiP}
\label{a:examples}

% ------ code: tensor product and partial trace ------
\code{https://github.com/frnq/qme/blob/main/python/qutip/tensor_partial.py}{Tensor product and partial trace using QuTiP}{code:tensor_partial}{python}{Scripts/python/qutip/tensor_partial.txt}

The following script implements a symbolic steady-state solution using \texttt{MATLAB}. 
% -------- code: symbolic solution for the null space --------
\code{https://github.com/frnq/qme/blob/main/matlab/symbolic_solution.m}{Symbolic steady-state solution}{code:symbolic_matrix_solving}{MATLAB}{Scripts/matlab/symbolic_solution.txt}

Alternatively, we could solve the set of coupled algebraic equations obtained by element-wise comparison of the left and right hand sides of Eq.~\eqref{eq:lindblad_master_equation}, by replacing the last two code lines above with the following.
% ------- code: solving algebraically --------
\code{https://github.com/frnq/qme/blob/main/matlab/steady_state_algebraic.m}{Solving element-wise algebraic equations {\normalfont \textsf{(requires script~\ref*{code:symbolic_matrix_solving})}} }{code:element-wise_algebraic}{MATLAB}{Scripts/matlab/steady_state_algebraic.txt}

%------ code for null space solution -----
\code{https://github.com/frnq/qme/blob/main/matlab/null_space_matlab.m}{Solving for the null space of superoperator}{code:null_space_MATLAB}{MATLAB}{Scripts/matlab/null_space.txt}

% ------- code: qutip mesolve example ---------
\code{https://github.com/frnq/qme/blob/main/python/qutip/mesolve_example.py}{Solution using QuTiP {\normalfont\textsf{(requires scripts~\ref*{code:super_matrix_exp})}}}{code:mesolve_example}{python}{Scripts/python/qutip/mesolve_example.txt}

% ------- code: matlab solution ---------
\code{https://github.com/frnq/qme/blob/main/matlab/dynamics_norm_svd.m}{Solution using normalized singular vectors}{code:temporal_solution}{MATLAB}{Scripts/matlab/dynamics_norm_svd.txt}

Alternatively, we could obtain the same solution using non-normalized singular vectors as
\begin{equation}
    \label{eq:non_normalised_svd_solution}
    \bm{\rho}(t) = \sum^{d^2}_{k=1}b_k\bm{R}_k e^{\lambda_k t}
\end{equation}
where the coefficients $b_k$ are found by performing row reduction on the following augmented matrix formed with $\bm{R}_k$'s and $\bm{\rho}(0)$ as columns,
\begin{equation}\label{Eq:Row_reduction_mat}
    R = \left(\;\bm{R}_1 \;\; \bm{R}_2 \;\; \hdots \;\; \bm{R}_{d^2} \;\big|\; \bm{\rho}(0)\; \right).
\end{equation}
When the above matrix is in row echelon form, the right hand column will give the values of $b_k$'s. The following continuation of the earlier code implements the alternative method and the resulting excited state population.
%variation can be found in Fig.\ \ref{Fig:Temporal_sol}.
% -------- code: echelon form ---------
\code{https://github.com/frnq/qme/blob/main/matlab/dynamics_non_norm_svd.m}{Solution using non-normalized singular vectors  {\normalfont \textsf{(requires script~\ref*{code:temporal_solution})}}}{code:non-normalised}{MATLAB}{Scripts/matlab/dynamics_non_norm_svd.txt}

% ------- code: runge kutta method ---------
\code{https://github.com/frnq/qme/blob/main/matlab/dynamics_finite_diff.m}{Solution with finite-difference methods {\normalfont \textsf{(requires script~\ref*{code:temporal_solution})}}}{code:solution_matlab_ode}{MATLAB}{Scripts/matlab/dynamics_finite_diff.txt}

%%%%%%%% FIGURE %%%%%%%%%
\begin{figure}  %% [H] to force the figure here
	\includegraphics[width=0.6\columnwidth]{Figures/example_2.png}
	\centering
	\caption{Temporal evolution of the excited state population ($\rho_{22}$) obtained using the normalized superoperator eigenvectors (Norm-eig), non-normalized eigenvector (reduced) row-echelon-form (RREF), and numerical differential equation solving (ode45).\label{Fig:Temporal_sol}} 
\end{figure}

% -------- code: mcwf ---------
\code{https://github.com/frnq/qme/blob/main/mathematica/mcwf.nb}{Propagation using stochastic wavefunction method}{code:mcwf}{Mathematica}{Scripts/mathematica/mcwf.txt}

% ------- code: qutip emission spectrum ---------
\code{https://github.com/frnq/qme/blob/main/python/qutip/emission_spectrum.py}{Emission spectrum using QuTiP {\normalfont\textsf{(requires scripts~\ref*{code:superoperator_python} and~\ref*{code:emission_spectrum})}}}{code:emission_qutip}{python}{Scripts/python/qutip/emission_spectrum.txt}

The following script~\ref{code:floquet_plot_python} can be used to obtain the plots shown in Fig.~\ref{fig:floquet}.
% -------- code: floquet_plot ---------
\code{https://github.com/frnq/qme/blob/main/python/floquet_plot.py}{Transition probability of two-level atom under strong driving}{code:floquet_plot_python}{python}{Scripts/python/floquet_plot.txt}

% -------- code: floquet ---------
\code{https://github.com/frnq/qme/blob/main/matlab/floquet_rates.m}{Function for the time-evolution of a two-level atom under strong driving}{code:floquet_wave_vector}{MATLAB}{Scripts/matlab/floquet_rates.txt}

% -------- code: floquet plott ---------
\code{https://github.com/frnq/qme/blob/main/matlab/floquet_plot.m}{Time-evolution of a two-level atom under strong driving}{code:floquet_wave_vector_plot}{MATLAB}{Scripts/matlab/floquet_plot.txt}

\section{Software requirements}
\label{a:software_requirements}

The \texttt{python} scripts in the \textit{main text} have been tested using \texttt{Python 3.9.6}, and require the libraries \texttt{numpy}, \texttt{scipy} and \texttt{matplotlib}. The \texttt{python} scripts in the \textit{appendix} also require the libraries \texttt{qutip}, \texttt{sympy} and \texttt{tqdm}. The \texttt{Mathematica} (\texttt{MATLAB}) scripts in the appendix were tested using version \texttt{13.2} (\texttt{R2022a}), and do not require any additional library.

\end{document}