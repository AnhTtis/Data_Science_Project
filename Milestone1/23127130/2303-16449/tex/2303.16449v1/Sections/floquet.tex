%\section{Oscillatory terms in a Hamiltonian, the rotating frame and Floquet theory}
\section{Periodically driven systems and Floquet theory}
\label{s:oscillatory}

Up until this point, all the Hamiltonians considered are constant, piecewise constant or vary slowly enough that they can be considered piecewise constant. Now we consider the common situation where some part of the Hamiltonian is periodically oscillating in time
\begin{equation}
    \label{eq:periodically_drive}
    H(t) = H_0 + \sin(\omega t + \phi_0) H_1,
\end{equation}
where $H_0$ and $H_1$ are two (generally non-commuting) time-independent Hamiltonians, $\omega$ is some oscillation frequency and $\phi_0\in\mathbb{R}$ is some initial phase. A very common example is a two-level system interacting with an oscillating electric or magnetic field, which is encountered experimentally when driving transitions with a laser or microwave field. However, the approach detailed here is very general and applies to any harmonically oscillating Hamiltonian whose frequency $\omega$ and overtones $k\omega$ ($k\in\mathbb{Z}$) is near resonant with a transition $\ket{E_n}\to\ket{E_m}$ between eigenstates $\ket{E_n}$ with energy $E_n$ of the considered internal Hamiltonian $H_0$,
\begin{equation}
    \label{eq:near-resonant}
    k\:\hbar \omega \approx |E_m - E_n|.
\end{equation}

\subsection{Two-level system interacting with an electric field}
\label{ss:two_level_atom_with_field}

Let's consider a single two-level system (TLS) subjected to an oscillating electric field $\bm{E}(t)$ of wavelength $\lambda$. If the atom is much smaller than $\lambda$, the field would appear spatially constant in the region occupied by the atom. This enables us to write the field as a function of time, 
\begin{equation}
    \label{eq:field}
    \bm{E}(t) = \Big(E_0 e^{-i\omega t} + E_0^* e^{i\omega t}\Big)\hat{z},
\end{equation}
assuming that $\bm{E}$ is oriented along the $\hat{z}$ direction, where $\omega$ is the angular frequency of the incoming radiation.

The total system Hamiltonian $H = H_\mathrm{S} + H_\mathrm{int}$ is the sum of the TLS Hamiltonian,
\begin{equation}
    \label{eq:TLS}
    H_\mathrm{S} = \hbar\frac{\omega_0}{2}\sigma_z,
\end{equation}
with eigenstates $\ket{g}$ and $\ket{e}$,
and the atom-field dipolar interaction Hamiltonian ${H}_\mathrm{int}$~\cite{steck2007quantum, artuso2012optical},
\begin{equation}\label{Eq:TLA_total_Hamiltonian}
H_\text{int} = - \bm{d}\cdot\bm{E},
\end{equation}
where $\bm{d}$ is the transition dipole moment operator of the atom. 
Assuming that the field predominantly interacts with only one electron in the atom, we write $\bm{d}$ in terms of the electron position $\bm{r}_\text{e}$ as $\bm{d} = -e \bm{r}_\text{e}$, where $e$ is the elementary charge. Using a parity argument, it can be shown that the diagonal matrix elements of $\bm{d}$ vanish, i.e., $\langle g|\bm{d}| g\rangle = \langle e|\bm{d}| e\rangle = 0$.
As a result the dipole operator reads
\begin{equation}
    \bm{d} = \braket{e|\bm{d}|g}\ketbra{e}{g} + \braket{e|\bm{d}|g}^*\ketbra{g}{e},
\end{equation}
from which we define the Rabi frequency $\Omega$ of the TLS, and its associated \textit{counter-rotating} frequency $\widetilde{\Omega}$,
\begin{equation}
    \label{Eq:TLA_Rabi}
    \Omega = \braket{g|\bm{d}\cdot \hat{z}|e}\frac{E_0}{\hbar}, \;\;\;\;\widetilde{\Omega} = \braket{e|\bm{d}\cdot \hat{z}|g}\frac{E_0^*}{\hbar}.
\end{equation}
The interaction Hamiltonian then reads 
\begin{equation}
    H_\mathrm{int} = -\hbar\Big(\Omega e^{-i\omega t} + \widetilde{\Omega}e^{i\omega t}\Big)\ketbra{e}{g} -\hbar\Big(\widetilde{\Omega}^* e^{-i\omega t} + \Omega^*e^{i\omega t}\Big)\ketbra{g}{e}.
\end{equation}

\subsubsection{The rotating-wave approximation}
\label{sss:rwa}
Let us now write the full Hamiltonian $H$ in the interaction picture $\widetilde{H} = U_0^\dagger  H U_0$, with $U_0 = \exp(-i H_\mathrm{S} t/\hbar)$,
\begin{equation}
    \label{eq:full_interaction_picture}
        \widetilde{H} = H_\mathrm{S} -\hbar\Big(\Omega e^{-i\Delta \omega t} + \widetilde{\Omega}e^{i(\omega+\omega_0)t}\Big)\ketbra{e}{g} -\hbar\Big(\widetilde{\Omega}^* e^{-i(\omega+\omega_0) t} + \Omega^*e^{i\Delta \omega t}\Big)\ketbra{g}{e}
\end{equation}
with $\Delta \omega = \omega-\omega_0$. If the driving field is close to resonance with the energy splitting of the two-level system, i.e., $\omega\approx\omega_0$, the two time scales involved in the dynamics are separated from each other,
\begin{equation}
    \label{eq:rwa_time-scales}
    \Delta \omega\ll \omega + \omega_0.
\end{equation}
The rapidly oscillating terms in $\omega+\omega_0$, associated with the counter-rotating frequency $\widetilde{\Omega}$, quickly average to zero over the time scale of the Rabi frequency $\Omega$.
As a result the rotating wave approximation (RWA) of the Hamiltonian $H$ in the original frame reads
\begin{equation}
\label{eq:rwa_full}
    H^\mathrm{RWA} = H_\mathrm{S} -\hbar\Big(\Omega e^{-i\omega t}\ketbra{e}{g} + \Omega^* e^{i\omega t}\ketbra{g}{e}\Big).
\end{equation}

%Verify yourself that using a negative or positive coefficient for $\Omega$ result in the same population evolution. \comH{Do we want these type of asides throughout? Worth keeping the style the same, irrespective of what we decide.}

\subsubsection{Time-independent Hamiltonian in the rotating frame}
\label{sss:time-independent_rotating_frame}
The Hamiltonian of Eq.~\eqref{eq:rwa_full} can be written in the rotating frame of the driving field, via the transformation generated by the time-dependent unitary $V_\omega = \exp(i H_\mathrm{S} t/\hbar) = \exp(i\omega\sigma_z/2 t)$~ \cite{steck2007quantum},
\begin{equation}\label{Eq:Rotating_frame_transformation}
H^\mathrm{RWA}\to{H}^\mathrm{RWA}_\omega = {V_\omega}{H}{V_\omega}^\dagger + i\hbar\dot{{V}_\omega}{V_\omega}^\dagger.
\end{equation} 
In this frame the Hamiltonian reads
\begin{equation}\label{Eq:H_RF}
\begin{split}
{H}^\mathrm{RWA}_\omega &= \hbar\frac{\Delta \omega}{2}\sigma_z + \hbar\mathrm{Re}[\Omega]\sigma_x+\hbar\mathrm{Im}[\Omega]\sigma_y, \\
    & = \frac{\hbar}{2}\begin{pmatrix}
            \Delta\omega & 2\Omega^* \\
            2\Omega^* & -\Delta\omega 
            \end{pmatrix}.
\end{split}
\end{equation}
This is now a time-independent Hamiltonian in the rotating frame of the driving field, and can be treated with the methods introduced in previous sections. Typically, the decoherence operators are not oscillatory and are also time-independent in this frame, which means solving the master equation also proceeds as above. 

\subsection{Floquet theory and Schr\"odinger evolution}
The RWA is strictly only valid when the Rabi freqency $\Omega$ is small compared to the transition frequency $\omega_0$. When this is not the case, for example in the limit of strong driving inducing multi-photon processes, more sophisticated techniques are required~\cite{Scala2007}.

A common approach to treating strong driving beyond the RWA is using Floquet theory. In this approach, the evolution of a system undergoing periodic variation is expressed in a Fourier series in terms of the oscillation frequency. The Floquet theorem states that a set of time-dependent differential equations whose coefficients vary periodically will have solutions with the same periodicity. This is the temporal equivalent of Bloch's theorem in space, with the solution expressed in terms of quasi-energies instead of quasi-momenta.

In the context of quantum systems, Floquet theory provides a method for finding solutions to the time-dependent Schr\"odinger equation due to the influence of a time-periodic Hamiltonian. The Floquet treatment of the two-level system problem under strong driving was treated by Shirley~\cite{Shirley1965}. However, the approach is of general validity and invaluable in a variety of time-dependent problems, such as analogue quantum simulation~\cite{Kyriienko2018}, quantum information processing~\cite{Bomantara2018}, heat engines and laser cooling~\cite{Restrepo2018}, quantum optimal control~\cite{Bartels2013,Castro2022}, and time crystals~\cite{Else2016,Sacha2018}.

\subsubsection{Floquet modes and quasi-energies}
\label{sss:floquet_modes_quasi_energies}

Let us consider the time-dependent Schr\"odinger equation for a periodic Hamiltonian $H(t) = H(t+n T)$, for all $n\in\mathbb{Z}$,
\begin{equation}
\label{eq:shroedinger_time-dep}
    i \hbar \frac{d}{dt} \ket{\psi(t)} = H(t) \ket{\psi(t)}.
\end{equation}
The Floquet theorem states that the general solution has the form
\begin{equation}
    \label{eq:floquet_theorem}
    \ket{\psi(t)} = \sum_{\alpha}e^{-i\epsilon_\alpha t/\hbar}\ket{\phi_\alpha(t)},
\end{equation}
where $\ket{\phi_\alpha(t)} = \ket{\phi_\alpha(t+n T)}$ are some periodic functions, known as \textit{Floquet modes}, and $\epsilon_\alpha$ are the associated quasi-energies, constant in time and uniquely defined up to multiples of $\omega = 2\pi/T$~\cite{Shirley1965}. By plugging Eq.~\eqref{eq:floquet_theorem} back into Eq.~\eqref{eq:shroedinger_time-dep}, we can recast the problem as an eigenvalue problem to the quasi-energies for the operator $\mathrm{H}(t):=H(t) - i\hbar d_t$,
\begin{equation}
    \label{eq:quasi_energy_eigen}
    \mathrm{H}(t) \ket{\phi_\alpha(t)} = \epsilon_\alpha\ket{\phi_\alpha(t)}.
\end{equation}
This equation can be solved numerically or analytically in order to find the quasi-energies and the Floquet modes. An alternative approach to finding the solution is to solve the eigenvalue problem posed by the propagator $U(t+nT;t)$~\cite{Creffield2003},
\begin{equation}
    \label{eq:alternative_floquet_solution}
    U(t+nT;t)\ket{\phi_\alpha (t)} = e^{-i\epsilon_\alpha T/\hbar}\ket{\phi_\alpha(t)},
\end{equation}
with is then solved for $\eta_\alpha = \exp(-i\varepsilon_\alpha T/\hbar)$, to find $\varepsilon_\alpha = -\hbar \arg(\eta_\alpha)/T$. This approach is implemented in \texttt{QuTiP} with the \texttt{floquet\_modes} method.

\subsubsection{The Floquet Hamiltonian and Fourier analysis}
\label{sss:floquet_fourier}

Thanks to their shared periodicity we can express both the Hamiltonian and the Floquet modes as Fourier series,
\begin{equation}
    \ket{\phi_\alpha(t)} = \sum_n e^{-i \omega n t} \ket{\alpha,n}, \quad  H(t) = \sum_n e^{-i \omega n t} H_n,
\end{equation}
where we have implicitly introduced the Fourier components $\ket{\alpha,n}$ and $H_n$ of the Floquet modes and of the Hamiltonian, respectively, 
\begin{equation}
    \label{eq:Floquet_fourier_components}
    \ket{\alpha,n} = \frac{1}{T}\int_0^T dt e^{i\omega n t} \ket{\phi_\alpha(t)}, \quad
    H_n = \frac{1}{T}\int_0^T dt e^{i \omega n t} H(t).
\end{equation} 
This allows us to define a \textit{Floquet Hamiltonian}, $H_F$, whose components are given by
\begin{equation}
    \bra{\alpha, n} H_F \ket{\beta, m} = H_{n-m}^{(\alpha,\beta)} + n \omega \delta_{\alpha \beta} \delta_{n m},
\end{equation}
which can be used to calculate transition probabilities $P_{\alpha\to\beta}(t)$ between the modes $\alpha\to\beta$, as discussed in the next section.

\subsubsection{Transition probabilities from Floquet Theory}
\label{sss:Floquet_transition_probabilities}

Let us consider a simple sinusoidal variation in the Hamiltonian, such that $H$ has a finite Fourier series
\begin{align}
\label{eq:example_sinusoidal}
    H(t) &= \sum_{n=-1}^{1} e^{-i\omega n t} H_n, \\
         &= H_0 + \widetilde{H}_1 \cos(\omega t),
\end{align}
with $\widetilde{H}_1 := H_{-1} e^{-i\omega t} + H_1 e^{i\omega t}$. Then, the Floquet Hamiltonian has the general structure
\begin{equation}
    H_F = \left( \begin{array}{ccccc}
    H_0 - 2 \hbar \omega & H_1 & 0 & 0 & 0 \\
    H_{-1} & H_0 - \hbar \omega & H_1 & 0 & 0 \\
    0 & H_{-1} & H_0 & H_1 & 0 \\
    0 & 0 & H_{-1} & H_0 + \hbar \omega & H_1 \\
    0 & 0 & 0 & H_{-1} & H_0 + 2 \hbar \omega \\
    \end{array}
    \right)
\end{equation}
where the size of the matrix is limited by the number of harmonics included in the Fourier expansion. If we then diagonalise $H_F$, the time dependent wavefunction can be written in terms of the eigenvectors $\ket{\lambda}$ and corresponding eigenvalues $\lambda$ of the Floquet Hamiltonian
\begin{equation}
    \label{eq:floquet_eigenvalue}
    H_F \ket{\lambda} = \lambda \ket{\lambda}.
\end{equation}
The time-dependent wavefunction $\ket{\psi(t)} = U(t;t_0)\ket{\psi(t_0)}$ is then expressed in terms of the propagator $U(t;t_0)$, whose elements can be written as
\begin{equation}
    U_{\beta \alpha}(t;t_0) = \sum_n \bra{\beta, n}\exp[-i H_F (t-t_0)/\hbar]\ket{\alpha, 0} e^{i n \omega t} = \sum_n \sum_{\lambda} \braket{\beta, n|\lambda}\braket{\lambda|\alpha, 0} e^{-i\lambda (t-t_0)/\hbar} e^{i n \omega t}.
\end{equation}
The probability at time $t$ of a given transition $\alpha \rightarrow \beta$ between Floquet modes with quasi-energies $\epsilon_\alpha$, $\epsilon_\beta$ can then be computed directly,
\begin{equation}
    P_{\alpha \rightarrow \beta}(t-t_0) = \sum_k|\bra{\beta k}\exp[-i H_F (t-t_0)/\hbar]\ket{\alpha 0}|^2.
\end{equation}

In addition, because the time evolution is given by the Floquet components, the time-averaged probability $\overline{P}_{\alpha\to\beta}$ can be evaluated as
\begin{equation}
\label{eq:time-averaged}
    \overline{P}_{\alpha \rightarrow \beta} = \sum_k \sum_{\lambda}|\braket{\beta k|\lambda}\braket{\lambda|\alpha 0}|^2.    
\end{equation}
This equation is implemented in the following \texttt{python} script for a system given by a two-level system interacting with a quantised electromagnetic field mode $a^\dagger$ with frequency $\omega$,
\begin{equation}
    \label{eq:floquet_example}
    H = H_\mathrm{S} + V\sigma_z (a^\dagger e^{-i\omega t}+a e^{i\omega t}) + \hbar \omega a^\dagger a,
\end{equation}
under different driving strengths $V$, as shown in Fig.~\ref{fig:floquet}. The size of the Floquet Hamiltonian scales with both the number of states and the number of modes included in the Floquet expansion. The relative magnitude of $||H_1||$ to $||H_0||$ controls how many modes need to be included. In practice, this can be determined by increasing the number of modes until the result converges. It is worth noting that this method can be computationally costly due to the size of the Floquet Hamiltonian. However, if convergence can be achieved, the method is \textit{exact} and therefore can be used to compute the effects of strong driving, multi-photon transitions and other effects beyond the rotating wave approximation.
% -------- code: mcwf ---------
\code{https://github.com/frnq/qme/blob/81c4f16e45c7cf80ddee873814ee9d27f13d325d/python/floquet_rates_python.py}{Transition probability with Floquet theory}{code:floquet_wave_vector_python}{python}{Scripts/python/floquet_rates.txt}

\begin{figure}
    \centering
    \includegraphics[width=0.80\textwidth]{Figures/floquet.pdf}
    \caption{Absorption probability associated with the transition $\ket{g}\to \ket{e}$ in a two level system with Hamiltonian $H_0 = \delta\sigma_z/2+\varepsilon\sigma_x$ and eigenvalues $\lambda_g,\lambda_e$. The system is strongly driven via the interaction $H_{int} = V\sigma_z/2$ with a cavity mode of frequency $\omega$. The vertical dashed lines correspond to the $n$-photon transitions, which are enabled as the interaction strength $V$ increases. The figures is generated using script~\ref{code:floquet_plot_python}.}
    \label{fig:floquet}
\end{figure}

\subsubsection{Extension of Floquet theory to decoherence processes}
\label{sss:extension_to_decoherence}

While the extension of Shirley's approach to model decoherence is less well established, there have been a number of different approaches, depending on how the expansion in Floquet components is introduced to the master equation~\cite{Breuer2000,Wu2010,Schnell2020,Mori2023} as well as other approaches to including beyond-rotating wave physics into a master equation treatment~\cite{Scala2007,Werlang2008,Majenz2013,Muller2017,Kohler2018}.

One approach, which is also relatively simple to code, was introduced by Bain and Dumont~\cite{Bain2010} to model higher order corrections in magic angle spinning NMR experiments. In their approach they use the Liouville form introduced in section~\ref{ss:superoperator}, and express a periodic superoperator $\mathcal{L}_t$ as a Floquet expansion, resulting in a \textit{Floquet superoperator} $\mathcal{L}_F$ that generates the dynamics in an effective time-independent Markovian master equation, in analogy with the Floquet Hamiltonian in the Shirley approach. However, it is important to notice the existence of a time-independent Floquet superoperator $\mathcal{L}_F$ is not always guaranteed, as shown in Ref.~\cite{Schnell2020}. In fact, depending on the choice of $\mathcal{L}_t$, the evolution might be described by an equivalent non-Markovian master equation that is homogeneous in time but not time-local. Although more computationally demanding than the standard Floquet approach, this extension to decoherence processes is quite general and can be applied to master equations with oscillatory Hamiltonian components fairly easily~\cite{Bushev2010,Schoen:2020}. 
 
