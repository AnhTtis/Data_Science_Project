\section{Density Operators} 
\label{s:density_operators}
In this section we briefly review the mathematical description of the \textit{state} of a quantum system, focusing on the numerical implementation of state vectors and density operators. We assume that the reader is familiar with the \textit{postulates of Quantum Mechanics}, Hilbert spaces, expectation values, time evolution, and composite systems, which can be reviewed in any of these textbooks~\cite{Cohen1978,Scully1997,Gardiner2000,Bransden2000,Breuer2002,Schlosshauer2007,Wiseman2009,Weiss2012,,Nielsen2010}. 

\subsection{Pure states}
\label{ss:pure_states}
Let us consider a $d$-dimensional quantum system with Hilbert space $\mathcal{H}$. Let $\mathcal{B}:= \{ |\phi_1\rangle, |\phi_2\rangle, ..., |\phi_d\rangle\}$ be an orthonormal basis for $\mathcal{H}$, so that $\langle \phi_i|\phi_j\rangle = \delta_{ij}$. For example, $\mathcal{B}$ could be given by the orthonormal eigenstates of a hermitian operator such as some Hamiltonian $H$. 
Any state of the system can be expressed as a \emph{coherent superposition} with complex coefficients $c_i\in\mathbb{C}$,
\begin{equation}
\label{eq:coherent_superposition}
	|\psi\rangle = c_1|\phi_1\rangle+c_2|\phi_2\rangle+...+ c_d|\phi_d\rangle = \sum_{j=1}^d c_j\ket{\phi_j},
\end{equation}
where the coefficient $c_j$ are such that $\braket{\psi|\psi} = \sum_{j=1}^d |c_j|^2= 1$, according to the Born interpretation of the wavefunction~\cite{Mcmahon2013}.
The square of the coefficients in Eq.~\eqref{eq:coherent_superposition}, $|c_j|^2$, represents the probability of finding the system in the eigenstate $|\phi_j\rangle$ upon measurement in the considered basis $\mathcal{B}$. See Ref.~\cite{Cohen1978} for a review of projective measurement and Ref.~\cite{Nielsen2010} for the generalisation to positive operator valued measures (POVMs). 

Unit vectors like $\ket{\psi}$ are called \emph{pure states}. A pure state contains all the available physical information about the system, such as the expectation value of an observable $\mathcal{A}$ associated with hermitian operator $A$,
\begin{equation}
    \label{eq:pure_expectation_value}
    \braket{A} = \braket{\psi|A|\psi}.
\end{equation}
The following \texttt{python} script uses methods from the \href{https://numpy.org/}{\texttt{numpy}} library to implement state vectors and operators, and calculates the expectation value of some observable.
% ------ code: pure states and expectation values ------
\code{https://github.com/frnq/qme/blob/main/python/pure_states_expect.py}{Pure states and expectation values}{code:pure_states_expect}{python}{Scripts/python/pure_states_expect.txt}

\subsection{Mixed states: Proper and improper mixtures}
\label{ss:mixed_states}

There are two important scenarios where pure states are no longer sufficient to describe the state of a system. First, in experimental settings, we often lack the knowledge of the exact pure state $|\psi\rangle$ of our system. Instead, we may know that the system is in any of the pure orthonormal states $\{\ket{\psi_j}\}$ with some probabilities $\{ p_j \}$. In other words, our knowledge of the system is represented by a \emph{statistical mixture of pure states}, described by the set $\lbrace|\psi_j\rangle, p_j\rbrace$. 
In such case, when more than one $p_j$ is non-zero, the system is said to be in a \emph{mixed state}. This is sometimes referred to as a \textit{proper} mixture~\cite{Masillo2009}.

Second, when studying the dynamics of composite systems, pure states are no longer the most general description of a state. This is because the marginal state of any \textit{entangled} state cannot be represented as a pure state, and instead, needs to be represented as a statistical mixture over the basis elements of the considered subsystem~\cite{Nielsen2010}, as discussed in Sec.~\ref{ss:composite systems}. This is sometimes referred to as an \textit{improper mixture}~\cite{Masillo2009}. See Refs.~\cite{Cohen1978,Nielsen2010} for more on composite systems, and Refs.~\cite{Mintert2005,Bengtsson2006,Modi2012} for an in-depth analysis of entanglement and other quantum correlations. 

\subsection{Definition and properties of the density operator}
\label{ss:properties_density_operator}
Whether we are dealing with proper or improper mixtures of states, we can represent the set $\{\ket{\psi_j},p_j\}$ using a linear operator on the Hilbert space,
\begin{equation}
\label{eq:density_operator}
	{\rho} = \sum_{j=1}^{d}p_j \ketbra{\psi_j}{\psi_j},
\end{equation}
known as the density operator~\cite{Nielsen2010}, where $\ketbra{\psi_j}{\psi_j}$ is the outer product of $\ket{\psi_j}$ with itself, that is, the vector product of $\ket{\psi_j}$ with its dual $\bra{\psi_j}$. The coefficients $p_j > 0$ are such that $\sum_j p_j = 1$, since they represent probabilities (also known as \textit{convex combination}).  Density operators have three fundamental properties,
\begin{enumerate}
    \item \textbf{Hermitian:} ${\rho} = {\rho}^\dagger$. This implies that $\rho$ has only real eigenvalues.
    \item \textbf{Positive\footnote{Or, more specifically, \textit{positive semi-definite}.}:} $\rho > 0$. That is, $\rho$ eigenvalues $p_j\in[0,1]$ are not negative.
    \item \textbf{Normalised:} $\tr\rho= 1$, which can also be stated as $\sum_j p_j = 1$, i.e., the sum of its eigenvalues (probabilities) must add up to 1.
\end{enumerate}

Density operator can represent both pure and mixed states, and can be expressed in any basis $\mathcal{B} = \{\ket{\phi_i}\}_{i=1}^d$ of the Hilbert space $\mathcal{H}$ as
\begin{equation}\label{Eq:Density_Matrix_in_basis}
	{\rho} = \sum_{i,j=1}^{d}\rho_{ij}\ketbra{\phi_i}{\phi_j} = \begin{pmatrix}
	\rho_{11} & \rho_{12} &\dots &\rho_{1d}\\
	\rho_{21} & \rho_{22} &\dots &\rho_{2d}\\
	\vdots & \vdots & \ddots & \vdots\\
	\rho_{d1} & \rho_{d2} &\dots &\rho_{dd}
	\end{pmatrix},
\end{equation}
where $\rho_{ij}$ is the associated matrix element with row $i$ and column $j$.
The diagonal elements $\rho_{ii}$ of the density matrix are known as \emph{populations} and they denote the probabilities of finding the system in the respective basis states $|\phi_i\rangle$. The off-diagonal elements $\rho_{ij}$ are known as \emph{coherences}, and provide information about the coherent superposition of the basis states $|\phi_i\rangle$ and $|\phi_j\rangle$~\cite{Manzano2020}. 

Similarly to state vectors, density operators encode all the available information that can be extracted from the considered system. For example, the expectation value of some observable $\mathcal{A}$ associated with hermitian operator $A$ can be calculated as,
\begin{equation}
    \label{eq:expectation_mixed}
    \braket{A} = \tr[A\rho].
\end{equation}
The following \texttt{python} script provides an implementation of a density operator and the evaluation of the expectation value of some observable. There, a system with dimension $d=3$ is in a mixed state defined by state vectors $\{\ket{\psi_1},\ket{\psi_2},\ket{\psi_3}\}$ with probabilities $\{0.1,0.3,0.6\}$, represented by the density operator $\rho = 0.1\ketbra{\psi_1}{\psi_1}+0.3\ketbra{\psi_2}{\psi_2}+0.6\ketbra{\psi_3}{\psi_3}$.
% ------ code: mixed states and expectation values ------
\code{https://github.com/frnq/qme/blob/main/python/mixed_states_expect.py}{Mixed states and expectation values}{code:mixed_states_expect}{python}{Scripts/python/mixed_states_expect.txt}

\subsection{Composite systems}
\label{ss:composite systems}

Composite systems consist of two or more (interacting) quantum systems, whose Hilbert space is given by the tensor product of the individual Hilbert subspaces, $\mathcal{H} = \bigotimes_{i} \mathcal{H}_i$~\cite{Nielsen2010}. For example, a composite system might be given by a pair of interacting two-level systems (\textit{qubits}, in quantum information theory), or by a system $\mathrm{S}$ interacting with some large environment $\mathrm{E}$.

\subsubsection{Tensor product and partial trace}
\label{sss:tensor_product_partial}
Any state $\rho$ of a composite system can be represented using a basis $\mathcal{B}$ constructed using the \textit{tensor product} of the basis elements of each subsystems' basis $\mathcal{B}_\alpha = \{ \ket{\phi_i}_\alpha \}_{i=1}^{d_\alpha}$. For example, a bipartite system can be expressed in the following basis,
\begin{equation}
    \mathcal{B} = \Big\{ \ket{\phi_i}_1\otimes\ket{\phi_j}_2 \Big\}_{i,j}.
\end{equation} 
In \texttt{python}, the tensor product can be implemented with \texttt{numpy} using the Kroneker product \href{https://numpy.org/doc/stable/reference/generated/numpy.kron.html}{\texttt{kron}}. 
\begin{codeline}{python}
\footnotesize
    \begin{minted}[linenos=false]{python}
        psi = numpy.kron(psi1,psi2)
    \end{minted}
\end{codeline}
\noindent
Similar implementations are available in \texttt{Mathematica} and \texttt{MATLAB}, with \href{https://reference.wolfram.com/language/ref/KroneckerProduct.html}{\texttt{KroneckerProduct}} and \href{https://au.mathworks.com/help/matlab/ref/kron.html}{\texttt{kron}}, respectively.

When taking expectation values for composite systems, it may be useful to focus only on the \textit{marginal state} of one of the subsystems. For example, the marginal state $\rho_1$ of subsystem $1$ is obtained from the total state $\rho$ by \textit{tracing over} the degrees of freedom associated with the rest of the Hilbert space (here, subsystem $2$),
\begin{equation}
    \label{eq:partial_trace}
    \rho_1 = \tr_2 [\rho].
\end{equation}
The linear operator $\tr_i[\cdot]$ is called \textit{partial trace}, and its definition can be found in Ref.~\cite{Breuer2002}. For the case of bipartite systems with dimensions $d_1$ and $d_2$, the partial trace can be implemented in \texttt{python} using \texttt{numpy}.
\begin{codeline}{python}
\footnotesize
    \begin{minted}[linenos=false]{python}
        rho1 = np.trace(rho.reshape(d1,d2,d1,d2), axis1=0, axis2=2)
        rho2 = np.trace(rho.reshape(d1,d2,d1,d2), axis1=1, axis2=3)
    \end{minted}
\end{codeline}

For example, let us consider the following bipartite pure state 
\begin{equation}
    \label{eq:bipartite_pure}
    \ket{\psi(\theta)} = \cos(\theta)\ket{00}+\sin(\theta)\ket{11},
\end{equation}
where $\ket{00}=\ket{0}_1\otimes\ket{0}_2$, $\ket{11}=\ket{1}_1\otimes\ket{1}_2$, and its associated density operator is given by $\rho(\theta) = \ketbra{\psi(\theta)}{\psi(\theta)}$. The state $\rho(\theta)$ is separable for $\theta = 0,\pi/2$, and entangled otherwise, being maximally entangled\footnote{The state $\ket{\psi(\pi/4)}$ is the $\Phi_+$ Bell state~\cite{Nielsen2010}.} for $\theta = \pi/4$. 
As a result, for $\theta\neq k\pi/2$ the partial state of each subsystem $\rho_i(\theta) = \tr_j[\rho(\theta)]$ is not pure, and is therefore an improper mixture.

To measure the degree of mixedness of a density operator we can use the \textit{purity} $\mathcal{P}$,
\begin{equation}
    \label{eq:purity}
    \mathcal{P}[\rho] = \tr[\rho^2] = \sum_{j=1}^d p_j^2,
\end{equation}
which is bounded between 1, for pure states $\rho = \ketbra{\psi}{\psi}$, and $1/d$, for maximally mixed states $\rho = \mathbb{1}/d$. For more on purity, entropy, measures of distinguishability, and other information-theoretic figures of merit see Refs.~\cite{Bengtsson2006,Nielsen2010}.

The following \texttt{python} script calculates the marginal state of the first subsystem, $\rho_1(\theta) = \tr_2\rho(\theta)$, showing that its purity $\mathcal{P}[\rho_1(\theta)]<1$ for $\theta \neq k\pi/2$. Notice that $\rho_1(\theta)$ is maximally mixed when $\rho(\theta)$ is maximally entangled, i.e., $\tr\rho_1(\pi/4)=1/2$, as shown in Fig.~\ref{fig:partial_trace}. A powerful implementation of the tensor product and the partial trace (\href{https://qutip.org/docs/3.1.0/guide/guide-tensor.html}{\texttt{ptrace}}) for any type of composite system is available in \href{https://qutip.org/}{\texttt{QuTiP}}, as shown in the script~\ref{code:tensor_partial}.
% ----- code for block-matrix -----
\code{https://github.com/frnq/qme/blob/main/python/partial_trace.py}{Partial trace and purity of entangled states}{code:partial_trace}{python}{Scripts/python/partial_trace.txt}

\begin{figure}[h]
    \centering
    \includegraphics{Figures/partial_trace.pdf}
    \caption{Purity of the marginal state $\rho_1(\theta) = \tr_2\rho(\theta)$, calculated using script~\ref{code:partial_trace}. The state $\rho_1(\theta)$ is maximally mixed for $\theta = \pi/4$, since $\rho(\pi/4)$ is maximally entangled. This is an example of an improper mixture.}
    \label{fig:partial_trace}
\end{figure}

\subsubsection{Direct sum}
\label{sss:direct_sum}

Sometimes, it is useful to compose systems given by the \textit{addition} of different Hilbert spaces together. For example, when studying a pair of interacting systems with Hilbert space $\mathcal{H}_a = \mathcal{H}_1\otimes\mathcal{H}_2$ and dimension $d_a$, it might be convenient to add some states $\{\ket{\phi_i}_b\}_{i=1}^{d_b}$ to the picture, perhaps representing the result of some transitions that are modelled phenomenologically.
In these cases the total Hilbert space is given by
\begin{equation}
    \label{eq:direct_sum}
    \mathcal{H} = \mathcal{H}_a \oplus \mathcal{H}_b.
\end{equation} 
Numerically, a basis for this space can be constructed, from the bases of each individual subsystem, using a block matrix structure,
\begin{equation}
    M = \begin{pmatrix} 
        M_a & \bm{0} \\
        \bm{0}^\mathrm{T} & M_b
    \end{pmatrix},
\end{equation}
where $M_a$ and $M_b$ are $d_a \times d_a$ and $d_b \times d_b$ matrices, respectively, and $\bm{0}$ is a $d_a\times d_b$ matrix. The above structure can be implemented in \texttt{python} using the following script. For more information on tensor products, direct sums, and irreducible representations, see Ref.~\cite{Cohen1978}.

% ----- code for block-matrix -----
\code{https://github.com/frnq/qme/blob/main/python/block_matrix.py}{Composing a block-matrix operator}{code:block_matrix}{python}{Scripts/python/block_matrix.txt}

\subsection{Schr\"odinger and von Neumann equations}
\label{ss:schrodinger_von_neumann}

When studying the dynamics of quantum systems using the density operator representation, Schr\"odinger's equation~\eqref{eq:schroedinger} becomes,
\begin{equation}
    \label{eq:von_neumann_equation}
    \dot{\rho}(t) = - \frac{i}{\hbar}[H,\rho(t)], 
\end{equation}
known as the \textit{von Neumann}\footnote{Or \textit{Liouville-von Neumann} equation.} equation, where $H$ is the Hamiltonian of the system (which can be time-dependent), $\dot{\rho} = \partial_t\rho$, and $[\cdot,\cdot]$ is the commutator~\cite{Breuer2002}. In general, the solution to this equation is given by some unitary operator $U(t;t_0)$ that propagates the state of the system from some initial time $t_0$ to some time $t$,
\begin{equation}
    \label{eq:solution}
    \rho(t) = U(t;t_0) \rho(t_0) U(t;t_0)^\dagger,
\end{equation}
where $\dagger$ is the conjugate transpose (\textit{adjoint)}. If $H$ is time-independent the solution is given by $U(t;t_0) = \exp[-i H (t-t_0)/\hbar]$ and can be reduced to $U(\tau) = \exp[-i H \tau/\hbar]$ for all $t,t_0$ such that $\tau = t-t_0$. See Ref.~\cite{Joachain1975,Breuer2002} for more on the solution $U$ for time-dependent Hamiltonian using time-ordering operators and the Dyson series.

\subsubsection{Open quantum systems}
\label{sss:open_quantum_systems}
The focus of this tutorial is the dynamics of systems that interact with their surrounding environment. These can be seen as composed of a system of interest $\mathrm{S}$ and an environment $\mathrm{E}$ that is usually large, uncontrollable, or not experimentally accessible~\cite{Breuer2002}. The dynamics of the full composite system $\mathrm{S}$-$\mathrm{E}$ (or \textit{universe}) follows equation Eq.~\eqref{eq:von_neumann_equation} with Hamiltonian
\begin{equation}
    \label{eq:open_system_hamiltonian}
    H = H_\mathrm{S} + H_\mathrm{E} + H_\mathrm{int},
\end{equation}
where $H_\mathrm{int}$ represents the interaction between the system with Hamiltonian $H_\mathrm{S}$ and the environment with Hamiltonian $H_\mathrm{E}$.

If the solution $U(t;t_0)$ is known, the dynamics of the system $\mathrm{S}$ can be drawn from the state of the universe $\rho$ by tracing over the environment's degrees of freedom,
\begin{equation}
    \rho_\mathrm{S}(t) = \tr_\mathrm{E} \big[ \rho(t) \big].
\end{equation}
However, finding $U$ for large composite systems is often a difficult problem, both numerically and analytically. Instead, we may seek to obtain a prescription for the dynamics of the system's state by performing the partial trace of Eq.~\eqref{eq:von_neumann_equation}, to obtain
\begin{equation}
\label{eq:reduced_von_neumann_equation}
    \dot{\rho}_\mathrm{S}(t) = -\frac{i}{\hbar}\tr_\mathrm{E}\big\{ [H,\rho(t)] \big\}.
\end{equation}
Eq.~\eqref{eq:reduced_von_neumann_equation} provides the starting point for the derivation of density operator master equations such as those reviewed in Secs.~\ref{s:doma} and~\ref{s:bloch-redfield}.