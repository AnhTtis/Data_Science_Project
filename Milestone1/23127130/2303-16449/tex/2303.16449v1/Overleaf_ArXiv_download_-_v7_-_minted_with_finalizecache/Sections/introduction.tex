\section{Introduction}
\label{s:introduction}

\thispagestyle{empty}

Master equations are differential equations used to model the dynamics of systems that can be described as a probabilistic combination of some states. For example, the concentration dynamics of a chemical reaction $ x\rightleftharpoons y$, where some reactants $x$ lead to some products $y$, can be described by the differential equations,
\begin{equation}
    \label{eq:synthesis_example}
    \begin{cases}
    &\dot{p}_x = k_{y\to x} p_{y} -k_{x\to y} p_x , \\
    &\dot{p}_y = k_{x\to y} p_x - k_{y\to x} p_{y},
    \end{cases}
\end{equation}
where $p_i$ represent the concentrations of species $i=x,y$, with $\dot{p}_i = dp_i/dt$ being their time derivative, and $k_{i\to j}$ the transition rates from species $i$ to $j$. This equation can be easily solved to obtain the transient and steady state concentration of the reactants and products, as a function of their initial concentrations and transition rates. In a reaction like the one modelled in
Eq.~\eqref{eq:synthesis_example}, the total concentration is conserved, since $\dot{p}_\mathrm{tot} := \dot{p}_x+\dot{p}_y = 0$. Then, by recasting the problem in terms of relative concentrations $p_i \to p_i/p_\mathrm{tot}$, we can interpret $p_i$ as \textit{the probability of being in state $i$}. We can generalise this idea to formulate master equations as first-order differential equations to the vector of probabilities $\bm{p} = (p_1,\cdots,p_n)$ of being in one of the $n$ states of some system of interest. As a result, the dynamics of the states probabilities are prescribed by the master equation
\begin{equation}
    \label{eq:general_master_equation}
    \dot{\bm{p}} = \bm{F}(\bm{p},t),
\end{equation}
with $\bm{F}$ often being a linear function of $\bm{p}$ represented by some generating matrix $A$, as in $\dot{\bm{p}} = A \bm{p}$.

However, when dealing with quantum systems we must take into account that coherent superpositions of states participate in the evolution, as prescribed by Schr\"odinger's equation
\begin{equation}
    \label{eq:schroedinger}
    \frac{d}{dt}\ket{\psi(t)} = -\frac{i}{\hbar} H\ket{\psi(t)},
\end{equation}
where $H$ is the Hamiltonian of the system, and $\ket{\psi(t)} = \sum_{j=1}^n c_j(t) \ket{\phi_j}$ is its state at time $t$, expressed as a coherent superposition of the eigenstates $\mathcal{B}_H = \{\ket{\phi_i},\cdots,\ket{\phi_n}\}$ of the Hamiltonian, via the normalised complex coefficients $c_j(t)$ satisfying $\sum_i |c_i(t)|^2 = 1$.
In this case, a vector of probabilities $\bm{p}$, with $p_i = |c_i|^2$, is no longer sufficient to completely describe the dynamics of the system, since different phases of $c_i$ will lead to different solutions. Master equations for the dynamics of quantum systems can then be expressed by employing another representation of the state of the system, known as the density operator $\rho$. As discussed in details in Sec.~\ref{s:density_operators}, the density operator contains all the information regarding the probabilities (known as \textit{populations}) of being in each state $i$, given by $p_i = \braket{\phi_i|\rho|\phi_i}$, as well as the phases (known as \textit{coherences}) $\varphi_{ij} = \braket{\phi_i|\rho|\phi_j}$ associated with the coherent superpositions between basis states $\ket{\phi_i}$ and $\ket{\phi_j}$. Quantum master equations are then formulated by generalisation of Eq.~\eqref{eq:general_master_equation}, as first-order differential equations to the density operator,
\begin{equation}
    \label{eq:general_quantum_master_equation}
    \dot{\rho} = \mathcal{F}(\rho,t).
\end{equation}

In this tutorial we will primarily cover a specific type of linear quantum master equations (QMEs), that respect a set of requirements for the evolution of the density operator, as discussed in Sec.~\ref{s:doma}. QMEs, initially developed in quantum optics to study light-matter interactions~\cite{Carmichael1999}, have been adopted in a multitude of settings, across different disciplines and fields, such as photochemistry~\cite{Atkins1974,Iwasaki2001,Forecast2023}, energy and charge 
transport~\cite{Plenio2008,Mohseni2008,Lee2015}, high-precision magnetometry~\cite{Betzholz2014,Jeske2017,Hapuarachchi2022},
electronic~\cite{Nakano2016,Norambuena2020,Kobori2020,Collins2022} and nuclear spin resonance~\cite{Redfield1955,Hendrickson1973,Jeener1998}, quantum information processing~\cite{Sarandy2005,Verstraete2009,Keck2017,Campaioli2022}, thermodynamics in the quantum regime~\cite{Uzdin2015,Farina2019,Gherardini2020,}, and are certainly not limited to these settings. One of they key aspects of QMEs is that they provide a coarse-grained stochastic description of the effect of unknown and uncontrollable agents on a system of interest~\cite{Breuer2002}, leading to a computationally inexpensive ensemble-averaged picture of the dynamics of quantum systems. QMEs can be phenomenological~\cite{Genkin2008} or derived, using first principles~\cite{Breuer2002}, from a microscopic model of the system-environment interactions, as done in Sec.~\ref{s:bloch-redfield}. They can be used to derive qualitative trends~\cite{Albers1971} or make quantitatively accurate predictions~\cite{Hapuarachchi2022}. They are just as suitable for the derivation of analytical results~\cite{Atkins1974} as they are for the numerical simulation of complex systems with a large number of degrees of freedom~\cite{Campaioli2021}. For these reasons, QMEs have become a standard approach to model the dynamics of quantum systems, and a starting point for the formulation of more sophisticated descriptions.

Quantum master equations are now more accessible than ever, thanks to the many dedicated libraries and software packages, such as \href{https://qutip.org/}{\texttt{QuTiP}}~\cite{Johansson2012}, \href{https://github.com/USCqserver/HOQSTTutorials.jl}{\texttt{HOQST}}~\cite{Chen2022},  \href{https://spindynamics.org}{\texttt{Spinach}}~\cite{Hogben2011}, and \href{https://github.com/jevonlongdell/qotoolbox}{\texttt{qotoolbox}}, to name a few. These resources offer an invaluable platform for the quick implementation of models and their systematic exploration. Indeed, they have established themselves as a staple tool on the workbench of a vast community of researchers. Pedagogical tutorials and documentations of these libraries are just as precious as the software itself, offering an accessible starting point and a pathway for rapid progression. Nevertheless, when directing newcomers from different research areas to QMEs, an obstacle is often presented by the vast and technical library of resources like textbooks and notes, written for a specialised audience, which may not be ideal for cross-disciplinary readers. To bridge this gap, this tutorial provides the reader with a concise introduction to quantum master equations, with a pedagogical, hands-on approach, in the style of an interactive lesson or a workshop. The aim is to provide a handbook for third-year students joining the research group, master students ready to implement models, and PhD students and cross-disciplinary researchers looking to consolidate and expand their expertise. 

In this tutorial we cover essential theories, like the Lindblad master equation, Bloch-Redfield theory and Floquet theory, as well as numerical techniques for their solutions, such as the stochastic wavefunction method, the Suzuki-Trotter expansion, and numerical approaches for sparse matrices.
We illustrate these methods using scripts implemented in \texttt{python}. Building up in complexity, these examples aim to provide a deeper understanding of the methods implemented behind the curtains in libraries like \texttt{QuTiP} and \texttt{qotoolbox}, and can be used as a starting point for generalisations. 
%All the scripts included in the tutorial are available on \href{https://github.com/frnq/qme}{\texttt{GitHub}}. 
Versions of these scripts in \texttt{MATLAB} and \texttt{Mathematica} can be found in Appendix~\ref{a:examples}. 