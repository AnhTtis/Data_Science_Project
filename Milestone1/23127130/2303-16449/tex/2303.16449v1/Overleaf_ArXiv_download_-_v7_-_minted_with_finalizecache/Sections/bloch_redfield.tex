\section{Bloch-Redfield theory}
\label{s:bloch-redfield}

In the previous section we discussed how to implement the Lindblad master equation from a \textit{phenomenological} model of decoherence and relaxation. However, it is sometimes necessary to start from a microscopic description---i.e., the system and environment Hamiltonian---to obtain a master equation for the density operator of the system. When the system interacts weakly with its environment, this can be achieved using Bloch-Redfield theory~\cite{Cohen1992,Breuer2002}. 
This theory is useful when we lack a model for decoherence and relaxation, but we know the nature of the system-environment interactions that drive such processes. As a result, the theory provides a powerful approach to determine the temperature dependence of dephasing and thermalisation rates directly from \textit{first principles}. 

\subsection{Bloch-Redfield master equation}
\label{ss:bloch-redfield_me}

Let us consider a system $\mathrm{S}$, with dimension $\mathrm{dim}_\mathrm{S}$, that interacts with its environment $\mathrm{E}$ according to the following general Hamiltonian
\begin{align}
    \label{eq:BR_microscopic_hamiltonian}
        H &= H_\mathrm{S}+H_\mathrm{E}+H_\mathrm{int} \\
          &= H_\mathrm{S}+H_\mathrm{E} +\sum_\alpha A_\alpha\otimes B_\alpha,
\end{align}
where the coupling operators $A_\alpha$ ($B_\alpha$) are Hermitian and act on the system (environment) such that $H_\mathrm{int}$ is a small perturbation of the unperturbed Hamiltonian $H_0 = H_\mathrm{S}+H_\mathrm{E}$.
Then, under the conditions~\ref{C:BR1}---\ref{C:BR4} discussed in Sec.~\ref{ss:bloch-redfield_approximations}, 
the dynamics of the system's density operator $\rho$ in the eigenbasis $\{\ket{\omega_a}\}$ of $H_\mathrm{S}$~\footnote{$H_\mathrm{S}\ket{\omega_a} = \hbar\omega_a\ket{\omega_a}$.} is prescribed by the Bloch-Redfield master equation,
\begin{eqbox}
    \begin{equation}
        \label{eq:BR_standard_form}
        \dot{\rho}_{ab}(t) = -i\omega_{ab}\rho_{ab}(t)+\sum_{c,d}R_{abcd}\rho_{cd}(t),
    \end{equation}
\end{eqbox}
\noindent
where $\omega_{ab} = \omega_a - \omega_b$ are the frequencies associated with transitions $\ket{\omega_b}\to\ket{\omega_a}$. The Bloch-Redfield tensor $R_{abcd}$ is prescribed by the following expression, where $\delta_{ij}$ is the Kronecker delta,
\begin{equation}
    \label{eq:BR_tensor}
    \begin{split}
    R_{abcd} = -\frac{1}{2\hbar^2}\sum_{\alpha,\beta}\bigg\{ &\delta_{bd}\sum_{n=1}^{\mathrm{dim}_\mathrm{S}} 
    A_{an}^{(\alpha)}A_{nc}^{(\beta)} S_{\alpha\beta}(\omega_{cn}) -  
    A_{ac}^{(\alpha)}A_{db}^{(\beta)} S_{\alpha\beta}(\omega_{ca}) + \\ & \delta_{ac}\sum_{n=1}^{\mathrm{dim}_\mathrm{S}} 
    A_{dn}^{(\alpha)}A_{nb}^{(\beta)} S_{\alpha\beta}(\omega_{dn}) -  
    A_{ac}^{(\alpha)}A_{db}^{(\beta)} S_{\alpha\beta}(\omega_{db})
    \bigg\}.
    \end{split}
\end{equation}
In Eq.~\eqref{eq:BR_tensor}, $A_{ab}^{(\alpha)} = \braket{\omega_a|A_\alpha|\omega_b}$ are the elements of the coupling operators $A_\alpha$ in the eigenbasis of the system Hamiltonian, while $S_{\alpha\beta}(\omega)$ corresponds to the noise-power spectrum of the environment coupling operators~\cite{Jones2017,Gardiner2000},
\begin{equation}
    \label{eq:BR_noise_power_spectra}
    S_{\alpha\beta}(\omega) = \int_{-\infty}^{\infty} d\tau e^{i\omega t}\: \tr\Big[B_\alpha(\tau)B_\beta(0)\rho_\mathrm{E}\Big],
\end{equation}
taken assuming $\rho_\mathrm{E}$ to be some steady state of the environment.

\subsubsection{Thermal relaxation and detailed balance condition}
\label{sss:bloch-redfield_detailed_balance}
When using BR theory it is common to consider environments in thermal equilibrium at inverse temperature $\beta = 1/k_B T$. For example, the environment may be assumed to be in a Bose-Einstein distribution,
\begin{equation}
    \label{eq:BR_gibbs_state}
    G_\beta(H_\mathrm{E}) = \frac{\exp(-\beta H_\mathrm{E})}{\mathcal{Z}},
\end{equation}
with $\mathcal{Z} = \tr[\exp(-\beta H_\mathrm{E})]$, and to be invariant under future evolutions (Gibbs state)~\cite{Breuer2002}.
An out-of-equilibrium density operator that evolves under the dynamics prescribed by Eq.~\eqref{eq:BR_standard_form} with $\rho_\mathrm{E} =G_\beta(H_\mathrm{E})$ will relax towards thermal equilibrium (exchanging energy with the environment). Indeed, the steady state of Eq.~\eqref{eq:BR_standard_form} is itself a Gibbs state $G_\beta(H_\mathrm{S})$ at thermal equilibrium with inverse temperature $\beta$.

The condition for this to occur is known as \textit{detailed balance}, and can be expressed in terms of the ratio between the rates $k_{a\to b}$ associated with transitions $\ket{\omega_a}\to\ket{\omega_b}$ separated by energy $\omega_{ba} = \omega_b -\omega_a$.
\begin{equation}
    \label{eq:BR_detailed_balance}
    \frac{k_{a\to b}}{k_{b\to a}} = \exp(-\beta\omega_{ba}).
\end{equation}
The detailed balance condition implies that the equilibrium populations of the eigenstates of the system follow the Boltzmann distribution $p_a\propto \exp(-\beta \omega_a)$. In terms of noise-power spectra, the detailed balance condition becomes $S_{\alpha\beta}(-\omega)/S_{\alpha\beta}(\omega) = \exp(-\beta\omega)$.

\subsubsection{Example: Spin-boson}
\label{ss:bloch-redfield_spin-boson}
Before discussing the approximation required to derive the BR master equation, let us implement BR theory for the simple and ubiquitous spin-boson model. We consider a two-level system coupled with a large ensemble of \textit{uncorrelated} harmonic oscillators at thermal equilibrium (bosonic bath) 
\begin{equation}
    \label{eq:BR_spin-boson}
    H = \frac{\epsilon_0}{2} \sigma_z+\frac{\Delta}{2} \sigma_z + \sum_k \hbar \omega_k b^\dagger_k b^\phdagger_k + \sigma_z\otimes\sum_k g_k \big(b^\dagger_k + b^\phdagger_k\big),
\end{equation}
where $g_k$ is the strength of the coupling between $\sigma_z$ and some mode $\omega_k$. 

First, we calculate the correlation functions $C_{kk'}(t)$ for the bath operators $B_k = g_k \big(b^\dagger_k + b^\phdagger_k\big)$
\begin{align}
    \label{eq:BR_bath_correlations}
    C_{kk'}(t) &= \delta_{kk'}\tr\big[B_k(t)B_{k'}(0) G_\beta(H_\mathrm{E}) \big], \\
                          &= \frac{g_k^2}{1-\exp(-\beta\omega_{k})}\bigg( e^{-i\omega_k t}+e^{i\omega_k t -\beta\omega_k}\bigg),
\end{align}
where we used the fact that the modes are uncorrelated ($\delta_{kk'}$) and assumed the bath to be in thermal equilibrium  $\rho_\mathrm{E} = G_\beta(H_\mathrm{E})$ at inverse temperature $\beta$, as in Eq.~\eqref{eq:BR_gibbs_state}.

To treat the contribution of a large ensemble of modes, we replace sum over the coupling strength $g_k$ with an integral over some spectral density $J(\omega)$ that well approximates the bath:
\begin{equation}
    \label{eq:spectral_sum_to_integral}
    \sum_k g_k^2 \to \int_0^\infty d\omega J(\omega).
\end{equation}
A common choice is the Ohmic spectral density $J(\omega) = \eta \omega e^{-\omega/\omega_c}$, which is characterised by a cut-off frequency $\omega_c$ and a dimensionless parameter $\eta$, from which we obtain the noise-power~\cite{Lidar2019},
\begin{align}
    \label{eq:BR_NPS}
    S(\omega) &= \int_{-\infty}^{\infty} dt e^{i\omega t} \sum_k C_{kk}(t) \\
              &\approx \int_{-\infty}^{\infty} dt e^{i\omega t} \int_0^{\infty} d\omega' J(\omega') \frac{\big( e^{-i\omega_k t}+e^{i\omega_k t -\beta\omega_k}\big)}{1-\exp(-\beta\omega_{k})} \\
              & = \frac{2\pi\eta\omega \exp(-|\omega|/\omega_c)}{1-\exp(-\beta \omega)}.
\end{align}

We now possess all the elements required to compose the BR tensor of Eq.~\eqref{eq:BR_tensor}. Note that we only have one system coupling operator $A = \sigma_z$, associated with a single noise-power spectrum $S(\omega)$. The following is a \texttt{python} implementation of the Bloch-Redfield tensor, which can then be used to propagate the state of the system using one of the methods discussed in Sec.~\ref{s:doma}. Note that to simplify the solution of Eq.~\eqref{eq:superop_form}, the unitary part of the generator has been absorbed into the tensor $R$,
\begin{equation}
    \label{eq:BR_with_unitary}
    R_{abcd} \to R'_{abcd} =-i\omega_{ac}\delta_{ac}\delta_{bd} + R_{abcd},
\end{equation}
and that system coupling operators are considered to be mutually uncorrelated, $S_{\alpha\beta}=\delta_{\alpha\beta}S_{\alpha\alpha}$.
% --------- code for BR tensor -------------
\code{https://github.com/frnq/qme/blob/main/python/bloch_redfield_tensor.py}{Bloch-Redfield tensor and spin-boson relaxation}{code:BR_spin-boson}{python}{Scripts/python/bloch_redfield_tensor.txt}

\subsection{Approximations for Bloch-Redfield master equation}
\label{ss:bloch-redfield_approximations}
While the Lindblad master equation is guaranteed to be completely positive and trace-preserving\footnote{See Sec.~\ref{s:density_operators} for definition and properties and CPTP maps.}, care must be taken when using BR theory. First, 
the following approximations have to be respected to obtain Eq.~\eqref{eq:BR_standard_form} from the reduced-state von Neumann equation~\cite{Cohen1992,Breuer2002}, as discussed in Sec.~\ref{ss:schrodinger_von_neumann}:

\begin{enumerate}[label=\textbf{C.\arabic*}]
    \item \label{C:BR1} \textbf{Weak coupling approximation:} The interaction $H_\mathrm{int}$ is a small perturbation of the unperturbed Hamiltonian $H_0 = H_\mathrm{S}+H_\mathrm{E}$;
    \item \label{C:BR2} \textbf{Born approximation:} The system-environment density operator is factorised at all times, $\rho_\mathrm{int}(t) = \rho_\mathrm{S}(t)\otimes\rho_\mathrm{E}$, with $\rho_\mathrm{E}$ being some steady state of the environment (justified also by~\ref{C:BR1});
    \item \label{C:BR3} \textbf{Markov approximation:} The bath correlation functions $g_{\alpha\beta}(\tau) = \tr\big[B_\alpha(\tau)B_\beta(0)\rho_\mathrm{E}\big]$ have a short correlation time scale $\tau_\mathrm{E}$, $g_{\alpha\beta}(\tau) \approx 0$ for $\tau \gg \tau_\mathrm{E}$.
    \item \label{C:BR4} \textbf{Rotating wave approximation:} All the contributions from the rapidly oscillating terms, i.e., with characteristic frequency $| \omega_{ab} - \omega_{cd}|\geq \tau^{-1}_\mathrm{E}$, are neglected as they approximately average to zero.
\end{enumerate}

Second, the BR master equation does not, in principle, guarantee positivity of the density operator. That is, when propagating the system in time $\rho(t) = \Lambda_t[\rho_0]$, the populations of $\rho$ may become negative for some time $t>0$~\cite{Whitney2008}. For this reason, when propagating a density operator numerically, it is advisable to check its positivity. The following \texttt{python} script can be used to test positivity, hermitianity and normalisation condition of a density operator. The function \texttt{is\_state(rho)} returns 1 if a \texttt{rho} is a density operator, and a value $s<1$ if \texttt{rho} deviates from the conditions of positivity, hermitianity and normalisation, where $1-s$ is a measure of such deviation.
% ----------- code for density operator test --------------
\code{https://github.com/frnq/qme/blob/main/python/density_operator_test.py}{Is this operator still a state?}{code:density_operator_test}{python}{Scripts/python/density_operator_test.txt}

\subsection{Lindblad form of the Bloch-Redfield master equation}
\label{ss:bloch-redfield_lindblad}
Under certain conditions, it is possible to write the BR master equation in the Lindblad form of Eq.~\eqref{eq:lindblad_master_equation},
\begin{equation}
\label{eq:BR_lindblad_form}
    \dot{\rho}(t) = -\frac{i}{\hbar}[H_\mathrm{S},\rho(t)]+\sum_{\alpha\beta}\sum_{\omega} S_{\alpha\beta}(\omega) \bigg(A_\alpha^\phdagger(\omega)\rho(t) A_\beta^\dagger(\omega) -\frac{1}{2}\Big\{A_\alpha^\dagger(\omega)A_\beta^\phdagger(\omega),\rho(t)\Big\}\bigg),
\end{equation}
where $A_\alpha(\omega) = \sum_{\omega = \omega_b-\omega_a} A_{ab}^{(\alpha)}\ketbra{\omega_a}{\omega_b}$ are the coupling operators in the frequency domain, such that the sum over $\omega$ only needs to be carried out over the transition (Bohr) frequencies $\omega = \omega_b -\omega_a$, as in Eq.~\eqref{eq:BR_tensor}~\cite{Breuer2002}.

This form is useful, for example, to systematically compile the BR tensor from a list of system coupling operators $A_\alpha$ and noise-power spectra $S_{\alpha\alpha}$, or even to compose the full Liouville superoperator associated with the dynamics of Eq.~\eqref{eq:BR_lindblad_form}. 

\subsubsection{Example: Network with random energies and couplings}
\label{ss:bloch-redfield_random_sites}
Let us consider a system consisting of $N$ states $\ket{k}$ with energies $\varepsilon_k$, that interact via couplings $v_{jk}$, with associated Hamiltonian
\begin{equation}
    \label{eq:BR_random_network}
    H_\mathrm{S} = \sum_k \varepsilon_k \ketbra{k}{k} + \sum_{j<k} \bigg(v_{jk}\ketbra{j}{k}+h.c.\bigg).
\end{equation}
Let us assume that each state $\ket{k}$ couples with a local environment of uncorrelated bosonic modes characterised by some noise power spectrum $S_{k}(\omega)$. This type of system-environment model is typically used to model the transport of charge carriers (electrons, holes) or coupled electron-hole pairs (\textit{excitons}) in disordered organic semiconductors~\cite{Jang2018}. In the following \texttt{python} script we study the dynamics of an instance of such random quantum network using Bloch-Redfield theory, with the results shown in Fig.~\ref{fig:random_network}. The BR tensor is calculated using the general method introduced in script~\ref{code:BR_spin-boson}, while the propagator is calculated adaptively for different time scales. A robust and efficient method for the calculation of the Bloch-Redfield tensor is implemented in the \texttt{bloch\_redfield\_tensor} function of \texttt{QuTiP}'s module \texttt{bloch\_redfield}.
% ----------- code for random network --------------
\code{https://github.com/frnq/qme/blob/main/python/random_quantum_network.py}{Random quantum network \normalfont{\textsf{(requires script~\ref*{code:BR_spin-boson})}}}{code:random_quantum_network}{python}{Scripts/python/random_quantum_network.txt}

\subsection{Computational resources for Bloch-Redfield master equation}
\label{ss:bloch-redfield_resources}

Markovian master equations like Lindblad and Bloch-Redfield are generally numerically inexpensive when compared to methods involving memory kernels or environmental degrees of freedom~\cite{Breuer2002,Strathearn2018}. Nevertheless, as the size of the system increases, solving density operator master equations can become computationally demanding~\cite{Kondov2001}. Therefore, when implementing BR theory numerically it is important to keep track of the required computational resources.

\subsubsection{Memory requirements}
\label{sss:bloch-redfield_memory_requirements}
Let $d = \mathrm{dim}\mathcal{H}_\mathrm{S}$ be the dimension of the Hilbert space associated with system's Hamiltonian $H_\mathrm{S}$. For any density operator master equation, the amount of complex floating point (FP) numbers required to store the density operator scales with $d^2$, with the coherences (off-diagonal elements) taking up the majority of this memory requirement. Analogously, the memory requirements to store the Liouville superoperator associated with Eqs.~\eqref{eq:lindblad_master_equation} and~\eqref{eq:BR_standard_form} scale as $d^4$. When memory becomes an issue, it is possible to use \textit{stochastic wave function} methods to limit the memory scaling to that of the system dimension ($d$) for the state, and that of the Hamiltonian ($d^2$) for the propagation, as discussed in Sec.~\ref{ss:swfm}.

\subsubsection{Operations requirements}
\label{sss:bloch-redfield_operations_requirements}
There are three main computationally demanding tasks encountered when solving any density operator master equation numerically in Liouville space: 
\begin{itemize}
    \item Constructing the generator of the evolution $\mathcal{L}$, associated with $\dot{\bm{\rho}} = \mathcal{L} \bm{\rho}$;
    \item Computing the propagator $P_t = \exp[ \mathcal{L} t]$;
    \item Propagating the state $\bm{\rho}_t = P_t \bm{\rho}_0$.
\end{itemize}
As discussed in Sec.~\ref{ss:solving_dynamics}, there is an array of approaches to reduce the expense of these tasks, depending on the type of problem. \vspace{5pt}

\noindent
\textbf{\textsf{Propagation}} --- Starting from the bottom, propagating the state in Liouville space involves a matrix multiplication $P \bm{\rho}$ between a $d^2$-vector $\bm{\rho} = \mathrm{vec}(\rho)$ and a $d^2\times d^2$ operator $P$. Without any optimisation, the number of floating point operations required scales with $d^4$~\cite{Kondov2001}. \vspace{5pt}

\noindent
\textbf{\textsf{Matrix exponential}} --- The number of operations required to compute the propagator depends on the method used to calculate the exponential of the matrix associated with $\mathcal{L}$. For example, \href{https://scipy.org/}{\texttt{scipy}}'s implementation (\href{https://docs.scipy.org/doc/scipy/reference/generated/scipy.linalg.expm.html}{\texttt{scipy.linalg.expm}}) uses the Pad\'e method to approximate the matrix exponential (see Refs.~\cite{Moler2003,AlMohy2010} for details on the amount of operations required). This is generally a demanding task, for Lindblad and BR master equations alike: Some approaches to mitigate the computational costs associated with this task are discussed in Sec.~\ref{ss:solving_dynamics}.\vspace{5pt}

\noindent
\textbf{\textsf{Redfield tensor}} --- However, when it comes to constructing the generator of the evolution, calculating the Bloch-Redfield tensor $R$ becomes substantially more demanding than the bare Lindblad generator $\mathcal{L}$.
In essence, this is because each system coupling operator $A_\alpha$ may contribute to any of the $d^2$ transitions $\ketbra{\omega_a}{\omega_b}$ in the eigenbasis of $H_\mathrm{S}$. Therefore, when constructing a Redfield tensor from $m$ coupling operators $A_\alpha$ we may need to perform a number of operations that scales with $m^2 \times d^2$. In constrast, to construct a Lindblad superoperator $\mathcal{L}$ from $m$ \textit{jump} operators $L_k$ we only need a number of operations that scales with $m$. See Ref.~\cite{Kondov2001} for further information on the computational resources required for BR theory, and the efficiency of different numerical implementations.

\subsection{Pauli master equation}
\label{ss:bloch-redfield_pauli}
The computational cost of BR master equations reduces dramatically under some special circumstances. When the system's Hamiltonian $H_\mathrm{S}$ is non-degenerate, the equations of motion for the populations $p_a(t)$ of the eigenstates $\ket{\omega_a}$ are closed and decoupled from the equations of motion for the coherences~\cite{Breuer2002}. The result is a system of linear ordinary differential equations to the populations, known as the Pauli master equation (PME):
\begin{equation}
\label{eq:BR_pauli}
    \dot{p}_a(t) = \sum_b \big[W_{ab}p_b(t)-W_{ba}p_a(t)\big],
\end{equation}
where the matrix elements $W_{ab} = \sum_{\alpha\beta}A_{ba}^{(\alpha)}A_{ab}^{(\beta)}S_{\alpha\beta}(\omega_{ba})$ represent the transition rates between eigenstates $a$ and $b$. 

The Pauli equation~\eqref{eq:BR_pauli} can be written in the vector form $\dot{\bm{p}}(t) = W \bm{p}(t)$ and solved analytically or numerically using the matrix exponential $\bm{p}(t) = \exp[W t]\bm{p}(0)$. Since the population vector $\bm{p}$ is $d$-dimensional, the computational resources required to implement the PME scale with $d^2$. Pauli master equations find applications in scenarios where dephasing happens over a much shorter time scale than thermal relaxation. As an example, room-temperature exciton transport properties have been studied using this approach in Ref.~\cite{Davidson2020,Davidson2022}. The following script implements the PME associated with the problem set up in script~\ref{code:random_quantum_network}. The results are shown in Fig.~\ref{fig:random_network}.

% ----------- code for PME --------------
\code{https://github.com/frnq/qme/blob/main/python/pauli_master_equation.py}{Pauli master equation {\normalfont \textsf{(requires script~\ref*{code:random_quantum_network})}}}{code:pauli_master_equation}{python}{Scripts/python/pauli_master_equation.txt}

\begin{figure}
    \centering
    \includegraphics{Figures/random_network_pauli.pdf}
    \caption{Solution of Bloch-Redfield master equation and associated Pauli master equations for the random quantum network of scripts~\ref{code:random_quantum_network} and~\ref{code:pauli_master_equation}.}
    \label{fig:random_network}
\end{figure}