% don't remove the folling lines, and edit the defintion of \main if needed
%\documentclass[../report.tex]{subfiles}
\documentclass[../ReviewEPJC.tex]{subfiles}
\providecommand{\main}{..}
%\IfEq{\jobname}{\currfilebase}{\AtEndDocument{\bibliographystyle{report}\bibliography{references}}}{}
\IfEq{\jobname}{\currfilebase}{\AtEndDocument{\bibliographystyle{report}\bibliography{ReportBIBOnINSPIRE,ReportBIBNOTOnINSPIRE}}}{}
% until here

\begin{document}
%\linenumbers
\section{Outlook}\label{ConclSect}

In this Review we summarised the motivations, status and plans of the ongoing multi-disciplinary effort towards a muon collider.

These studies are relevant and timely for several reasons. The outcome of the first part of the LHC experimental programme suggests that an ambitious jump ahead in energy will be needed for a fruitful exploration of fundamental interaction physics. Also the extension to higher energies of established $ee$ and $pp$ collider concepts faces severe feasibility challenges in terms of size, cost and power consumption. We saw in Sections~\ref{ch2_phys_opp} and~\ref{sec:phys_studies} that a muon collider with 10~TeV energy or more in the centre of mass offers tremendous and varied exploration opportunities. The muon collider at 10~TeV centre of mass energy offers equivalent or superior physics reach to a $pp$ collider with the highest envisageable centre of mass energy, 100~TeV. The energy limit for a muon collider is not yet known, but it is expected to be above 10~TeV.

The second reason for the renewed interest in muon colliders stems from recent advances in technology and muon collider design. These include the outcome of the MAP studies, which demonstrated the feasibility of many critical components of the facility, as well as several proof-of-principle experiments and component tests like MICE and the MUCOOL RF programme. Previously-considered limits for cooling such as operation of RF cavities in very high magnetic fields and the feasibility of a 30~T solenoid for the final cooling are now demonstrated. Several other advances are detailed in Section~\ref{sec_fac}. Muon colliders are now included in the European Roadmap for Accelerator R\&{D}. No showstopper has been identified and an R\&{D} and demonstration plan has been defined to address the remaining challenges in the next few years. The first assessment of the experimental conditions of a muon collider also contributed to enhance the global confidence in the project. It demonstrated the possibility of running a comprehensive experimental programme coping with the BIB from muon decay and identified wide margins for progress as discussed in Section~\ref{sec:detectorandreconstruction}.

The technically limited timeline for the muon collider R\&{D} programme has been described in Section~\ref{sec_fac} and it is displayed in Figure~\ref{fig_muon:RDtimeline}. The next Update of the European Strategy for Particle Physics (ESPPU) in 2026/2027 is an important milestone. By then, the design of the facility and of the cooling demonstrator will be consolidated. If the necessary investments will be supported by a favourable ESPPU recommendation, the preparation of a Conceptual Design Report and the construction and operation of the demonstrator and hardware prototypes will then initiate. The muon collider is a long-term project, which offers immediate opportunities for R\&D.

%Work is needed today in order to bring the concept to maturity, triggering a positive ESPPU decision in 2026/2027.

%The muon collider is a long-term project but immediate progress is essential to develop the facility design so that a decision can be made on the future of the project  at the next update of the European strategy for particle physics. A positive outcome in Europe will support the decision making process in other regions such as America.

The muon collider technical feasibility is only one of the key aspects of the project that must be studied intensively in the next few years. The muon collider will be the first facility to collide leptons at such high energies, the first facility to collide second-generation leptons, and the first facility to collide particles that are unstable. These innovations entail novel opportunities and challenges for the exploitation of the muon collider facility, once built. Advances are thus necessary also in experimental and theoretical  physics and phenomenology in order to fully assess and consolidate the physics potential of the muon collider project. 

We saw in Section~\ref{sec:detectorandreconstruction} that the design of experiments at the muon collider will require an extensive investigation and development of detector technologies and of innovative reconstruction algorithms tailored to the suppression of the BIB from the decaying muons. 

We outlined in Sections~\ref{ch2_phys_opp} and~\ref{sec:phys_studies} that the current assessment of the potential of the muon collider for the exploration of fundamental interactions is incomplete. Preliminary sensitivity estimates have not been performed for many promising new physics search channels. It is likely that many novel physics channels have not been identified. For example, no comprehensive assessment is available of the muon collider perspectives to probe structured new physics scenarios such as for instance Supersymmetry, Composite Higgs, or extended Higgs sectors. Work is also needed towards a global analysis of the model-independent perspectives to probe new physics indirectly through the SM EFT.

Theoretical predictions at the muon collider will require major methodological advances, arguably comparable to those that were needed and achieved for the exploitation of the LHC. The principal challenges and opportunities stem from the copious emission of EW radiation that requires resummation for sufficiently accurate predictions, calls for the development of novel EW showering Monte Carlo codes and also challenges fixed-order calculations. While there are good perspectives for progress, building upon the vast experience with QED and QCD radiation, we remarked in Section~\ref{EWRadiation} a number of unique theoretical aspects of the challenges surrounding the treatment of the EW radiation. Novel theoretical ideas are thus needed, on top of the adaptation and implementation of existing methodologies. 

Increasingly complete and accurate theoretical predictions will have to be progressively integrated with detector design and simulation advances, towards a fully realistic assessment of the muon collider physics potential. Among the most urgent questions at the interface between theory and experiment one could mention for instance the detectability of very boosted SM particles emerging either from heavy resonance decays or from high energy scattering processes, the observability of unconventional new physics signatures and the feasibility of accurate per mil level cross-section measurements needed for Higgs coupling studies. Investigating these and other questions, related for example to muon-specific exploration opportunities, will provide theory targets to guide the design of the muon collider experiments and, conversely, drive the development and assess the adequacy of the theoretical predictions.

On top of ensuring a robust assessment of the muon collider project perspectives, the simultaneous advancement on accelerator, experimental and theoretical physics is needed also to exploit and develop inter-disciplinary synergies. The optimisation of the Machine Detector Interface described in Section~\ref{sec:environment} is only one of the tasks where an inter-disciplinary approach is mandatory. Others include the design of a possible forward detector for muons and of a very forward detector to exploit the collimated beam of energetic neutrinos from the decay of the colliding muons as described in Section~\ref{sec:fwdandlumi}. Inter-disciplinary work will be also needed in order to define the energy staging plan ensuring a good balance between physics case, technical risk and cost. The current baseline plan appears adequate so far, but it will need to be reconsidered as the design and the physics studies advance. Designing the cooling demonstrator such as to maximise the synergies with other projects, exploiting for example the opportunities for neutrino physics, offers additional inter-disciplinary opportunities.
%is evidently another inter-disciplinary enterprise.

Muon collider physics is still in its infancy. The studies presented in this Review represent the first exploration of the topic, but they enable us to identify the relevant questions and directions for rapid progress in the next few years. The muon collider programme offers appealing perspectives for ambitious innovative research that will advance particle collider physics as a whole.
%, on top of paving the way towards a muon collider.

\end{document}\

