% don't remove the folling lines, and edit the defintion of \main if needed
%\documentclass[../report.tex]{subfiles}
\documentclass[../ReviewEPJC.tex]{subfiles}
\providecommand{\main}{..}
\IfEq{\jobname}{\currfilebase}{\AtEndDocument{\bibliographystyle{report}\bibliography{ReportBIBOnINSPIRE,ReportBIBNOTOnINSPIRE}}}{}
% until here

%%%%%%%%%%%%%%%%%%%%%%%%%%%%%%%%%%%%%%%%%%%%%%%%%%%%%%%%%%%%%%%%%%%%

\begin{document}

\section{Physics studies}
\label{sec:phys_studies}

Tentative energies and luminosities of future muon colliders, as well as a possible staged scenario that foresees a first collider with 3~TeV energy in the centre of mass followed by a \mbox{$10^+$\,TeV}~MuC, have been described in Section~\ref{IntroSect}. Based on these benchmark parameters, Section~\ref{ch2_phys_opp} offers a broad overview of the physics exploration opportunities of the muon collider project, as they emerge from the investigations performed so far. 

A selection of these studies is described in the present section in more details, focusing on indirect and direct opportunities for new physics searches in the Higgs and electroweak sector (Section~\ref{sec:Higgs}) and for dark matter searches (Section~\ref{secdm}). Studies outlining muon-specific exploration opportunities, namely advantages of colliding muons rather than electrons or protons, will be reviewed in Section~\ref{muonspec}. 

We do not aim at a complete assessment of the muon collider physics potential, which is still under development, nor at an exhaustive illustration of the opportunities for progress in these three directions of investigation. We describe concrete studies as a possible starting point for future work towards increasingly complete and realistic sensitivity projections. On top of strengthening and consolidating the physics case, and advancing the theoretical and experimental methodologies that will be required for the future exploitation of the facility, an extensive investigation of muon collider physics is a desirable input at this stage of the project also in view of a possible reassessment of the target design parameters and of the staging plan.

\subsection{Electroweak and Higgs physics \label{sec:Higgs}}

Muon colliders can probe the physics of electroweak interactions, including the nature of the Higgs sector responsible for the breaking of the electroweak symmetry, by employing three distinct and complementary strategies. 

First, as emphasised in Section~\ref{VBF}, one can exploit the large luminosity for effective vector bosons to produce a large number of Higgs bosons and study its on-shell couplings with precision. The Higgs trilinear coupling can be measured as well thanks to the relatively large rate for Higgs pair production. The quadrilinear coupling could be also accessible.

Second, one can probe short-distance electroweak and Higgs interactions directly by performing cross section measurements at the large available $\mu^+\mu^-$ energy. These measurements can shed light on heavy new physics potentially responsible for modifications of the on-shell Higgs couplings, or/and extend the muon collider sensitivity to much higher new physics scales than those probed by on-shell measurements or by the direct production of new particles. We anticipated in Section~\ref{HEM} that such high-energy measurements offer unique opportunities to probe the composite nature of the Higgs particle. 

Finally, the muon collider can probe directly new physics extensions of the Higgs sector that foresee relatively light new particles. 

The present section is devoted to an assessment of the muon colliders' perspectives for progress in these three directions.

\subsubsection*{Higgs couplings}%\label{sec:Hc}}
At high-energy muon colliders, as the electroweak gauge boson content of the muon beam becomes sizeable, vector boson fusion (VBF) becomes the most important channel for production of SM particles, as illustrated in Figure~\ref{fig:EV}. In particular, the VBF Higgs production rate neatly overcomes the Higgsstrahlung ($ZH$) process that is instead dominant at low-energy $e^+ e^-$ Higgs factories. A total of half a million Higgs bosons will be produced at the 3~TeV MuC, and around ten million at 10~TeV, with the integrated luminosities of $1~{\rm{ab}}^{-1}$ and $10~{\rm{ab}}^{-1}$, respectively. These figures qualify muon colliders as ``Higgs factories'' and call for detailed Higgs couplings sensitivity projections.

An initial estimate of the Higgs measurements precision attainable via $W$ boson fusion (WBF, $\mu^+ \mu^- \to H \overline{\nu} \nu$) and $Z$ boson fusion (ZBF, $\mu^+ \mu^- \to H \mu^+ \mu^-$) was presented in \cite{Han:2020pif,AlAli:2021let}. These results are obtained with a fast detector simulation and realistic detector acceptance~\footnote{Namely, $|\eta|<2.5$, or $\theta\in [10^\circ,170^\circ]$, due to radiation-absorbing tungsten nozzles expectedly needed to suppress the beam induced backgrounds, see Section~\ref{sec:environment}.}, but they do not include backgrounds for most channels. In what follows we employ instead the more recent estimates of Ref.~\cite{Forslund:2022xjq}, that do incorporate physics backgrounds as well as detector effects through the muon collider DELPHES card~\cite{deFavereau:2013fsa,delphes_card_mucol}. 

\begin{table*}[t]
  \centering
    \caption{68$\%$ probability sensitivity to the Higgs couplings, assuming no BSM Higgs decay channels.
    \label{t:Kappa_sensitivity}}
    {\vspace{4pt}
      \begin{tabular}{c|c|c|cc|c} 
        %\cmrule
        %\ctoprule
        %\crowcolor
        \hline
        &
        HL-LHC &
        HL-LHC &
        HL-LHC &
        HL-LHC &
        HL-LHC  \\
         &
         &
        + 125 GeV MuC  &
        
        + 3 TeV MuC &

        + 10 TeV MuC &
        
        + 10 TeV MuC
        \\
        Coupling &
        &
        5 / 20 fb$^{-1}$&
        1/2 ab$^{-1}$ &
        10 ab$^{-1}$&
         + FCC-ee\\ %$e^+e^-$ $H$ fact\\
        %Coupling &
        %&
        %&
        %&
        %&
        %(240/365 GeV)\\
        
        %\cmrule
        \hline
%-----------------------------------
        $\kappa_W~[\%]$ &
          1.7   & %$0.986$ & % HLLHC
          
          1.3 / 0.9  & % HLLHC + mu coll 125 GeV 

          0.4 / 0.3   & % HLLHC + mu coll 3 TeV 
          
          0.1  & % HLLHC + mu coll 10 TeV 
          
          0.1    \\ [0.1cm] % HLLHC + mu coll 10 TeV + FCCee
        $\kappa_Z~[\%]$ &
          1.5 & %0.987$ & % HLLHC
          
          1.3 / 1.0 & % HLLHC + mu coll 125 GeV 

          0.9 / 0.7 & % HLLHC + mu coll 3 TeV 
          
          0.4 & % HLLHC + mu coll 10 TeV 
            
          0.1   \\ [0.1cm] % HLLHC + mu coll 10 TeV + FCCee
        $\kappa_g~[\%]$ &
          2.3 & %1.9& % HLLHC
          
          1.7 / 1.4 & % HLLHC + mu coll 125 GeV 

          1.2 / 1.0  & % HLLHC + mu coll 3 TeV 
          
          0.7  & % HLLHC + mu coll 10 TeV 
          
          0.6   \\ [0.1cm] % HLLHC + mu coll 10 TeV + FCCee
        $\kappa_\gamma~[\%]$ &
          1.9  & %1.5& % HLLHC
          
          1.6 / 1.5 & % HLLHC + mu coll 125 GeV 

          1.3 / 1.2  & % HLLHC + mu coll 3 TeV 
          
          0.8  & % HLLHC + mu coll 10 TeV 
          
          0.8   \\ [0.1cm] % HLLHC + mu coll 10 TeV + FCCee
        $\kappa_{Z\gamma}~[\%]$ &
          10  & %9.9& % HLLHC
          
          10 / 10 & % HLLHC + mu coll 125 GeV 

          9.3 / 8.6  & % HLLHC + mu coll 3 TeV 
          
          7.2 & % HLLHC + mu coll 10 TeV 
          
          7.1 \\ [0.1cm] % HLLHC + mu coll 10 TeV + FCCee
        $\kappa_c~[\%]$ &
          -& % HLLHC
          
          12 / 5.9 & % HLLHC + mu coll 125 GeV 

          6.2 / 4.4  & % HLLHC + mu coll 3 TeV 
          
          2.3  & % HLLHC + mu coll 10 TeV 
          
          1.1  \\ [0.1cm] % HLLHC + mu coll 10 TeV + FCCee

        $\kappa_b~[\%]$ &
          3.6  & %2.3& % HLLHC
          
          1.6 / 1.0 & % HLLHC + mu coll 125 GeV 

          0.8 / 0.7  & % HLLHC + mu coll 3 TeV 
          
          0.4  & % HLLHC + mu coll 10 TeV 
          
          0.4  \\ [0.1cm] % HLLHC + mu coll 10 TeV + FCCee
        $\kappa_\mu~[\%]$ &
          4.6  & %4.4& % HLLHC
          
          0.6 / 0.3 & % HLLHC + mu coll 125 GeV 

          4.2 / 4.0  & % HLLHC + mu coll 3 TeV 
          
          3.4  & % HLLHC + mu coll 10 TeV 
          
          3.2  \\ [0.1cm] % HLLHC + mu coll 10 TeV + FCCee
        $\kappa_\tau~[\%]$ &
          1.9  & %1.6& % HLLHC
          
          1.4 / 1.2 & % HLLHC + mu coll 125 GeV 

          1.2 / 1.0  & % HLLHC + mu coll 3 TeV 
          
          0.6  & % HLLHC + mu coll 10 TeV 
          
          0.4  \\ [0.1cm] % HLLHC + mu coll 10 TeV + FCCee
          %\cmrule
        $\kappa_t^\dagger~[\%]$ &
          3.3  & %2.6& % HLLHC
          
          3.2 / 3.1 & % HLLHC + mu coll 125 GeV 

          3.1 / 3.1  & % HLLHC + mu coll 3 TeV 
          
          3.1  & % HLLHC + mu coll 10 TeV 
          
          3.1 \\ [0.1cm] % HLLHC + mu coll 10 TeV + FCCee
          %
          %\cmrule
          \hline
          $\Gamma_H^\ddag~[\%]$ &
          5.3  & %3.4& % HLLHC
          
          2.7 / 1.7 & % HLLHC + mu coll 125 GeV 

          1.3 / 1.0  & % HLLHC + mu coll 3 TeV 
          
          0.5 & % HLLHC + mu coll 10 TeV 
          
          0.4 \\ [0.1cm]
          %\cmrule
          \hline
          \multicolumn{6}{l}{ {\scriptsize $^\dagger$ No input used for $\mu$ collider.}}\\
          \multicolumn{6}{l}{ {\scriptsize $^\ddag$ Prediction assuming only SM Higgs decay channels. Not a free parameter in the fits.}}\\
        \hline
      \end{tabular}
    }
\end{table*}


While realistic enough for a first quantitative assessment of the Higgs couplings precision, these results will have to be consolidated in the future by full detector simulation studies including beam induced backgrounds from the muon decays. A comparison between current full simulation results and DELPHES, performed in the $H\rightarrow b\bar{b}$ channel~\cite{Forslund:2022xjq}, displays a reduced precision for the $b$-jets energy determination. However the resolution degradation does not affect the $H\rightarrow b\bar{b}$ cross section measurement precision appreciably, leading to a good agreement between the fast and full simulation estimates in this channel. This is encouraging, taking also into account that detector and reconstruction design studies are at a very preliminary stage and present results definitely underestimate the attainable physics performances as described in Section~\ref{sec:detectorandreconstruction}. 

Further work is also needed for a robust assessment of the possibility, which we do assume in our estimates~\cite{Forslund:2022xjq}, to discriminate between the WBF and the ZBF Higgs production channels by tagging very forward muons well beyond $|\eta|\approx 2.5$. This would require a dedicated forward muon detector, as described in Section~\ref{sec:fwdandlumi}, that is still to be designed.

The projected sensitivities~\cite{Forslund:2022xjq} for the main Higgs decays ($b\bar{b}$, $WW^*$, etc) in single Higgs production are estimated at the few percent level at 3~TeV, whereas at 10~TeV with 10 ab$^{-1}$, sensitivities at the permille level are possible. Roughly these figures could be considered comparable or slightly superior to the HL-LHC measurements sensitivities in the 3~TeV case, and to those of future $e^+e^-$ Higgs factories~\cite{deBlas:2019rxi} for the 10~TeV MuC. Moreover the different production mechanisms make MuC results complementary to the other projects, as we will see in the Higgs couplings sensitivity projections presented below. 

It should also be noticed that some aspects of Higgs physics are challenging at muon colliders, and have not yet been investigated. For example the precision on the top Yukawa coupling ($y_t$) determination from the $t\bar{t}H$ measurement at 3 and 10~TeV is estimated to be $35\%$ and $53\%$~\cite{Forslund:2022xjq}, significantly below the LHC.  However muon colliders offer additional handles for $y_t$ determination, such as the measurement of $W^+W^-\rightarrow t\bar{t}$. Preliminary results in this channel are promising \cite{AlAli:2021let} but further study is needed.

\begin{figure*}[ht]
\centering
\includegraphics[width=\linewidth]{figures/Global_Kappa_HL_MC.pdf}
\caption{\label{fig:kappa}
Sensitivity to modified Higgs couplings in the $\kappa$ framework. We show the marginalized 68\% probability reach for each coupling modifier. For the 125 GeV MuC, light (dark) shades correspond to a luminosity of 5 (20) fb$^{-1}$. The same color code is used for the 3~TeV MuC with 1 or 2 ab$^{-1}$. %{\bf{bigger fonts in captions and MuC. The $k_C$ ratio pad is also somewhat misleading. Would it be possible to have a grey bar for the HL-LHC too (from the expected limit?)? If not, we should at least grey out the bottom pad and write something like "Too weak constraints from HL-LHC".}}
}
\end{figure*}

For a quantitative assessment of the muon collider potential to measure the properties of the Higgs boson, we perform here a series of fits to single-Higgs couplings in the so-called $\kappa$ framework~\cite{LHCHiggsCrossSectionWorkingGroup:2012nn,LHCHiggsCrossSectionWorkingGroup:2013rie}, where the interaction vertices predicted by the SM are modified by scaling factors $\kappa_i$. \footnote{For the effective 1-loop vertices to gluons, photons and $Z\gamma$ we use independent effective scaling parameters $\kappa_{g,\gamma,Z\gamma}$, to describe the possibility of extra particles running in the loops.}
In the $\kappa$ framework, the cross sections of the different production processes $i \to {H }$ at the different colliders, times the decay branching ratios
%
\begin{equation}
( \sigma \cdot \text{BR} ) ( i \to {H }\to f ) =  \frac{ \sigma_{i} \cdot \Gamma_{f}}{\Gamma_H},
\label{eq:kappa_zw}
\end{equation}
%
are parameterised as follows
%
\begin{eqnarray}
&& ( \sigma \cdot \text{BR} ) ( i \to {H }\to f )  =  \frac{ \sigma_{i}^{\text{SM}} \kappa_i^2 \cdot \Gamma_{f}^{\text{SM}}\kappa_f^2 }{\Gamma_H^{\text{SM}} \kappa_H^2} \\
&&  = \frac{\kappa_i^2\cdot\kappa_f^2}{\kappa_H^2} \left[ (\sigma \cdot \text{BR} ) ( i \to {H }\to f )\right]_{\text{SM}}\,, \nonumber
\end{eqnarray}
in terms of the SM predictions for cross sections and branching ratios. We are interested in studying the sensitivity to the couplings of the Higgs boson, not its putative decay to exotic final states. Therefore the Higgs width modifier $\kappa_{H}$ is determined by the other $\kappa$'s
\begin{equation}\label{kh}
\kappa_{H}^2 = \frac{\sum_{f} \kappa_f^2 \Gamma_{f}^\text{SM}}{
\Gamma_{H}^\text{SM}}=\sum_{f}\kappa_f^2 \text{BR}^{\text{SM}}(H\to f) \,,
\end{equation}
where the sum extends over the SM Higgs decay channels. We will comment later on the possibility of leaving the Higgs width as a free parameter in the fit. We further restrict our attention to the 10 coupling modifiers, listed in Table~\ref{t:Kappa_sensitivity}, that are most precisely determined in the different collider projects under examination.

The results of the fits, performed with the {\tt HEPfit} code~\cite{DeBlas:2019ehy}, are reported in Table~\ref{t:Kappa_sensitivity} and displayed in Figure~\ref{fig:kappa}. On top of the 3 and 10~TeV sensitivity projections previously described, the following input is considered for the other collider projects. For the HL-LHC, we employ the signal strengths measurement projections from Ref.~\cite{Cepeda:2019klc}, assuming the S2 scenario for the reduction of systematic uncertainties. We also consider single-Higgs measurements at the FCC-ee collider~\cite{FCC:2018byv,FCC:2018evy}, as a reference for the precision that is generically attainable at future low-energy $e^+e^-$ Higgs factories. These results assume that all intrinsic SM theory (and experimental) uncertainties are under control by the time any of these future colliders are built~\cite{Blondel:2019qlh,Freitas:2019bre}. These assumptions, and the inputs for the HL-LHC, are compatible with the ones employed in Ref.~\cite{deBlas:2019rxi} for a global comparative assessment of future collider sensitivities in preparation for the 2020 European Strategy Update process. Finally, the results employ sensitivity projections from Ref.~\cite{deBlas:2022aow} for the Higgs-pole muon collider operating at 125~GeV, to be discussed later in this section.

The second and the third columns of Table~\ref{t:Kappa_sensitivity} report the marginalised sensitivity projections for the 3 and 10~TeV MuC in combination with HL-LHC. A luminosity of 1 or of 2~ab$^{-1}$ is considered at 3~TeV, compatibly with the preliminary staging plan described in Section~\ref{IntroSect}. The 1~ab$^{-1}$ results represent the outcome of the first stage in the optimistic scenario where the 3~TeV MuC runs for a short time (of few years), leaving space soon to the 10~TeV collider that will rapidly supersede the 3~TeV measurements precision, as the table shows. The 2~ab$^{-1}$ sensitivities are attainable with a longer run (or possibly with the installation of two detectors) of the 3~TeV collider, in the event that the 10~TeV upgrade is delayed. In both scenarios, the 3~TeV MuC will advance the determination of several Higgs couplings relative to the HL-LHC. Furthermore it could play a crucial role in clarifying possible tensions with the SM of the HL-LHC  results by measuring the Higgs couplings with comparable or better statistical precision in a leptonic environment that is subject to different (and expectedly reduced) experimental and theoretical systematic uncertainties.

The 10~TeV stage of the muon collider will measure single-Higgs couplings at the sub-percent or permille level, enabling a jump ahead in the knowledge of Higgs physics that is comparable to the one of a dedicated $e^+e^-$ low-energy Higgs factory. Furthermore the muon collider measurements are complementary to those of $e^+e^-$ Higgs factories because different production modes ($ZH$ at $e^+e^-$ and WBF the MuC) dominate at the different colliders. This makes, for instance, the 10~TeV MuC more sensitive to $\kappa_W$, and the $e^+e^-$ Higgs factories more effective in the measurement of $\kappa_Z$. This complementarity is illustrated the fourth column of Table~\ref{t:Kappa_sensitivity}, where the 10~TeV MuC measurements are combined with those of an $e^+e^-$ Higgs factory. By comparing with the MuC-only results on the third column, we see that the muon collider dominates the combined sensitivity for several couplings. There would thus be space for improvements on single-Higgs physics at the muon collider even if constructed and operated after the completion of an $e^+e^-$ Higgs factory project.

\begin{table*}[t]
  \centering
    \caption{68$\%$ probability intervals for the Higgs trilinear coupling.
      \label{t:H3_sensitivity}
    }
  \vspace{4pt}
    {
      \begin{tabular}{c c c c c c} 
        %\ctoprule
        %\crowcolor
        \hline
         &
        HL-LHC
         &
        3 TeV MuC &

        10 TeV MuC &

        14 TeV MuC &

        30 TeV MuC 
        \\
         &
         &
        L$\approx$ 1 ab$^{-1}$ / 2 ab$^{-1}$ &
        L= 10 ab$^{-1}$ &
        L$\approx$ 20 ab$^{-1}$ &
        L= 90 ab$^{-1}$\\
       % &
        %\multicolumn{4}{c}{68\% prob. interval }\\
        \hline
%-----------------------------------
        $\delta \kappa_\lambda$ &
        
          [-0.5,0.5] & % HL-LHC
          [-0.27,0.35]~$\cup$~[0.85,0.94] / [-0.15,0.16] & % mu coll 3 TeV 
          
          [-0.035, 0.037] &% mu coll 10 TeV 
          
          [-0.024, 0.025] & % mu coll 14 TeV 
          
          [-0.011, 0.012] \\ [0.1cm] % mu coll 30 TeV 
          \hline
          comb. w HL-LHC &
          -- &
          [-0.2,0.22] / [-0.13,0.14] & % HLLHC + mu coll 3 TeV 
          [-0.035,0.036] & % HLLHC + mu coll 10 TeV 
          [-0.024,0.025] & % HLLHC + mu coll 14 TeV 
          [-0.011,0.012] \\ [0.05cm] % HLLHC + mu coll 30 TeV 
          %with HL-LHC &
          %&  
          %& 
          %&
          %\\  
        \hline
      \end{tabular}
    }
\end{table*}

\paragraph{The Higgs width and the 125~GeV muon collider} {\ } \\ 
\noindent
So far we studied new physics effects in the SM interaction vertices of the Higgs, excluding possible new vertices that mediate ``exotic'' decays of the Higgs particle either to light BSM states or to SM final states different from the decay channels (such as  $b\overline{b}$, $WW$, etc) foreseen by the SM. Specific exotic channels could be searched for individually. Or, the total Higgs branching ratio to exotic channels can be probed indirectly via its contribution to the Higgs total width, parametrised as~\cite{deBlas:2019rxi}
\begin{equation}
\Gamma_H = \Gamma_{H}^\text{SM} \frac{\kappa_H^2}{1-\text{BR}_{\rm exo}}\,,
\end{equation}
with $\kappa_H$ as in eq.~(\ref{kh}).


All the Higgs cross sections times branching ratios ($\sigma\cdot\text{BR}$) are sensitive to the total Higgs width. Therefore, precise measurements of these observables are powerful probes of new physics scenarios that foresee exotic Higgs decays. The precision on the determination of $\text{BR}_{\rm exo}$ can be estimated as the one of the most accurately measured coupling in Table~\ref{t:Kappa_sensitivity}, namely $\sim 0.1\,\%$ at the 10~TeV muon collider. However, it is technically possible to compensate the dependence of the $\sigma\cdot\text{BR}$ observables on  $\text{BR}_{\rm exo}$ by a correlated modification of the SM Higgs couplings. This defines a flat direction in the parameter space formed by the coupling modifiers $\kappa_i$ and the exotic $\text{BR}_{\rm exo}$, that requires additional measurements to be resolved. No (sufficently accurate) additional measurement seems possible at high-energy muon colliders and the flat direction can not be lifted. At low-energy $e^+e^-$ Higgs factories instead, an absolute measurement of the $ZH$ production cross section is possible by the missing mass method, resulting in a global sensitivity to $\text{BR}_{\rm exo}$ at the $1\,\%$ level even in an extended Higgs fit where $\text{BR}_{\rm exo}$ is added as a free parameter on top of the coupling modifiers. The possibility of performing this additional measurement is an additional element of complementarity between high-energy muon colliders and low-energy $e^+e^-$ Higgs factories.

Alternatively, or additionally, the flat direction could be resolved by a direct experimental determination of the Higgs boson width, performed at a muon collider operating close to the Higgs pole $\sqrt{s}=125$~GeV through the measurement of the Higgs resonance line shape~\cite{Han:2012rb,Greco:2016izi}. State-of-the art projections~\cite{deBlas:2022aow} (see also~\cite{Conway:2013lca}), that duly include initial state radiation (ISR) effects and physics backgrounds, foresee a sensitivity to $\Gamma_H$ ($\text{BR}_{\rm exo}$) at the $3\,\%$ ($2\,\%$) and the $2\,\%$ ($1.5\,\%$) level for, respectively, 5 and 20~ab$^{-1}$ integrated luminosity. These results ignore beam-induced backgrounds (that could be worse than at high energy), assume the feasibility of a collider with a beam energy spread as small as $R=0.003\%$, and an instantaneous luminosity well above the baseline scaling~in eq.~(\ref{lums}). These aspects should be investigated for a realistic assessment of the 125~GeV MuC physics potential and feasibility.


Apart from measuring $\Gamma_H$, the 125~GeV MuC can also study other aspects of Higgs physics, but with limited perspectives for progress in comparison with the HL-LHC.\footnote{The 125~GeV MuC would also enable an impressive determination of the Higgs mass with one part per million accuracy.} In fact, with a resonant $\mu^+ \mu^- \to H$ cross section of 70~pb, reduced to about 22~pb by the beam energy spread and ISR, a luminosity at the level of several fb$^{-1}$ would yield order $10^{5}$ Higgses, limiting a priori the statistical reach in terms of precision Higgs physics. This is illustrated by the 10-parameters couplings fit on the last column of Table~\ref{t:Kappa_sensitivity}. Some progress is possible, eminently in the coupling of the Higgs to muons, but globally the progress relative to the HL-LHC is mild, and inferior to the one attainable by the 3~TeV MuC. 



\paragraph{The trilinear Higgs coupling}
{\ } \\ 
\noindent
Unlike low-energy $e^+e^-$ Higgs factories, high-energy muon colliders enable the direct measurement of the Higgs boson self-interactions, starting from the triple Higgs coupling $\lambda_3$. The relevant process is the WBF production of Higgs boson pairs, $\mu^+ \mu^- \to HH \bar{\nu}\nu$, that attains a total yield of $3\cdot 10^4$ events at the 10~TeV MuC with 10~ab$^{-1}$ as shown in Figure~\ref{fig:EV}. 

The single Higgs couplings are very precisely determined as previously discussed. Therefore the measurement of the differential double Higgs production cross section can be directly translated into the exclusive determination of the trilinear Higgs coupling, expressed in terms of $\kappa_\lambda\equiv \lambda_3/\lambda_3^{\rm SM}$. We employ the likelihood from Ref.~\cite{Buttazzo:2020uzc}, based on the differential distribution in the Higgs pair invariant mass. This analysis includes physics backgrounds and realistic detector effects, and the results nicely agree with previous studies (eminently, with Ref.~\cite{Han:2020pif}). The resulting projections for the $68\,\%$ confidence level regions for the measurement of $\delta\kappa_\lambda=\kappa_\lambda-1$ are reported in Table~\ref{t:H3_sensitivity} and illustrated in Figure~\ref{tab:higgscouplingfit}.

\begin{table*}[t]
\caption{\label{tab:eft-silh} 
68\% probability reach on the Wilson coefficients in the Lagrangian~(\ref{eq:LSilh}) from the global fit.
In parenthesis we give the corresponding results from a fit assuming only one operator is generated by the UV physics.\\
}
\centering
{%\footnotesize

\begin{tabular}{c c}

\begin{tabular}{ c | c | c | c  }
\hline
        &       &\multicolumn{2}{c}{HL-LHC + MuC}  \\
      &HL-LHC   & 3 TeV   & 10 TeV    \\
      &   & (1 / 2 ab$^{-1}$)   & (10 ab$^{-1}$)    \\
%   &    &    &   \\   
\hline
$\frac{c_{\phi}}{\Lambda^2}[\mbox{TeV}^{-2}]$   & $0.51$   & $0.11 / 0.086$   & $0.037$  \\
   & $(0.28)^\dagger$   & $(0.1/0.074)$   & $(0.025)$  \\
%
$\frac{c_{T}}{\Lambda^2}[\mbox{TeV}^{-2}]$   & $0.0057$   & $0.0022 / 0.0021$   & $0.0019$  \\
   & $(0.0019)$   & $(0.0019 / 0.0019)$   & $(0.0018)$  \\
%
$\frac{c_{W}}{\Lambda^2}[\mbox{TeV}^{-2}]$   & $0.33$   & $0.0096 / 0.0068$   & $0.0011$  \\
   & $(0.021)$   & $(0.0033 / 0.0024)$   & $(0.00031)$  \\
%
$\frac{c_{B}}{\Lambda^2}[\mbox{TeV}^{-2}]$   & $0.33$   & $0.022 / 0.016$   & $0.0022$  \\
   & $(0.026)$   & $(0.0075 / 0.0054)$   & $(0.00065)$  \\
%
$\frac{c_{\phi W}}{\Lambda^2}[\mbox{TeV}^{-2}]$   & $0.31$   & $0.034 / 0.033$   & $0.024$  \\
   & $(0.033)$   & $(0.031 / 0.03)$   & $(0.017)$  \\
%
$\frac{c_{\phi B}}{\Lambda^2}[\mbox{TeV}^{-2}]$   & $0.32$   & $0.17 / 0.16$   & $0.1$  \\
   & $(0.19)$   & $(0.17 / 0.16)$   & $(0.097)$  \\
%
$\frac{c_{\gamma}}{\Lambda^2}[\mbox{TeV}^{-2}]$   & $0.0053$   & $0.0045 / 0.0042$   & $0.0026$  \\
   & $(0.0041)$   & $(0.0038 / 0.0036)$   & $(0.0022)$  \\
%
$\frac{c_{g}}{\Lambda^2}[\mbox{TeV}^{-2}]$   & $0.0012$   & $0.0011 / 0.00096$   & $0.00057$  \\
   & $(0.00052)$   & $(0.00037 / 0.0003)$   & $(0.00015)$  \\
%
\hline
\end{tabular}

&

\begin{tabular}{ c | c | c | c  }
\hline
        &       &\multicolumn{2}{c}{HL-LHC + MuC}  \\
      &HL-LHC   & 3 TeV   & 10 TeV    \\
      &   & (1 / 2 ab$^{-1}$)   & (10 ab$^{-1}$)    \\
%   &    &    &   \\   
\hline
%
$\frac{c_{y_{e}}}{\Lambda^2}[\mbox{TeV}^{-2}]$   & $0.25$   & $0.19 / 0.16$   & $0.086$  \\
   & $(0.2)$   & $(0.17 / 0.15)$   & $(0.081)$  \\
%
$\frac{c_{y_{u}}}{\Lambda^2}[\mbox{TeV}^{-2}]$   & $0.57$   & $0.47 / 0.42$   & $0.25$  \\
   & $(0.24)$   & $(0.16 / 0.13)$   & $(0.064)$  \\
%
$\frac{c_{y_{d}}}{\Lambda^2}[\mbox{TeV}^{-2}]$   & $0.46$   & $0.13 / 0.11$   & $0.059$  \\
   & $(0.25)$   & $(0.11 / 0.089)$   & $(0.052)$  \\
%
$\frac{c_{2B}}{\Lambda^2}[\mbox{TeV}^{-2}]$   & $0.087$   & $0.0036 / 0.0025$   & $0.00031$  \\
   & $(0.075)$   & $(0.0029 / 0.002)$   & $(0.00026)$  \\
%
$\frac{c_{2W}}{\Lambda^2}[\mbox{TeV}^{-2}]$   & $0.0087$   & $0.00097 / 0.00069$   & $0.000084$  \\
   & $(0.0075)$   & $(0.00078 / 0.00055)$   & $(0.00007)$  \\
%
$\frac{c_{3W}}{\Lambda^2}[\mbox{TeV}^{-2}]$   & $1.7$   & $1.7 / 1.7$   & $1.6$  \\
   & $(1.7)$   & $(1.7 / 1.7)$   & $(1.6)$  \\
%
$\frac{c_{6}}{\Lambda^2}[\mbox{TeV}^{-2}]$   & $8.4$   & $4.6 / 3.1$   & $0.64$  \\
   & $(7.8)$   & $(4.4 / 3.0)$   & $(0.6)$  \\
%
\hline 
\multicolumn{4}{c}{}\\[0.35cm]
\end{tabular}
%
\end{tabular}}\\
{ 
\footnotesize 
 \begin{flushleft}
 \justify{
$^\dagger$ As explained in~\cite{deBlas:2019rxi}, due to the treatment of systematic/theory uncertainties in the HL-LHC inputs, this number must be taken with caution, as it would correspond to an effect below the dominant theory uncertainties. A more conservative estimate accounting for 100$\%$ correlated theory errors would give $c_\phi/\Lambda^2\sim 0.42$ TeV$^{-2}$.}
\end{flushleft}
}

%
\end{table*}

Table~\ref{t:H3_sensitivity} shows that the 10~TeV MuC can measure the Higgs trilinear coupling with a precision of $4\,\%$, significantly better than the CLIC high-energy $e^+e^-$ future collider project, whose precision is limited to $12\,\%$~\cite{deBlas:2018mhx}. The comparison with the 100~TeV proton-proton collider (FCC-hh) is more uncertain because the FCC-hh sensitivity projections range from $3.5$ to $8\%$ depending on assumptions on the detector performances~\cite{Mangano:2020sao}. Muon colliders of even higher energy (14, or 30~TeV) could further improve the precision provided their integrated luminosity increases like the square of the centre of mass energy.

Our results for the 3~TeV stage are more structured. With an integrated luminosity of 1~ab$^{-1}$, the confidence region consists of two disjoint intervals, and it is significantly broader than the estimate (of around $18\,\%$ precision) one would naively obtain by cutting the likelihood at the Gaussian $1~\!\sigma$ level.  This is because the likelihood is highly non-Gaussian, due to a secondary local minimum at large $\delta\kappa_\lambda$. Recent projections~\cite{Cepeda:2019klc} suggest that the HL-LHC could offer a sufficiently accurate determination (at the $50\,\%$ level) of $\delta\kappa_\lambda$ to exclude the secondary minimum. Therefore the combination with HL-LHC projections produces a connected confidence region and a relative precise determination of $\delta\kappa_\lambda$ already with 1~ab$^{-1}$ luminosity. With 2~ab$^{-1}$ instead, the 3~TeV stage will not require combination with the HL-LHC for an accurate determination of $\delta\kappa_\lambda$ at the level of $16\,\%$.

Beyond double Higgs production, a multi-TeV muon collider could exploit triple-Higgs production to gain sensitivity to the quartic Higgs coupling, $\lambda_4$~\cite{Chiesa:2020awd}. The cubic and quartic Higgs interactions are related in the SM and in BSM scenarios where new physics is heavy and the couplings correction emerge from dimension-six effective operators. If this is the case, the measurement of $\lambda_4$ is irrelevant as it can not compete with the $\lambda_3$ determination. If this is not the case, for instance because new physics is light, $\lambda_4$ modifications are independent from those of $\lambda_3$, and possibly stronger. The quartic coupling is directly tested at leading order via, e.g. $\mu^+ \mu^- \to HHH\bar{\nu}\nu$, which has a cross section of 0.31 (4.18) ab at $\sqrt{s}=3$ (10) TeV~\cite{Chiesa:2020awd}. For realistic luminosities, this makes a 3 TeV option unable to probe the quartic coupling. At the 10~TeV MuC, $\lambda_4$ could be tested to a precision of tens of percent with integrated luminosities of several tens of ab, slightly above the current luminosity target.

\begin{figure*}[ht]
\centering
\includegraphics[width=0.95\linewidth]{figures/Global_SILH_HL_MC_4col.pdf}
\caption{\label{fig:SILH}
Global fit to the EFT operators in the Lagrangian (\ref{eq:LSilh}). We show the marginalised 68\% probability reach for each Wilson coefficient $c_i/\Lambda^2$ in eq.~(\ref{eq:LSilh}) from the global fit (solid bars). The reach of the vertical ``T'' lines indicate the results assuming only the corresponding operator is generated by the new physics.}
\end{figure*}

\subsubsection*{EFT probes of heavy new physics}%\label{sec:EFT}}

Measuring the properties of the Higgs boson is part of the broader endeavour to test the SM increasingly accurately and under unprecedented experimental conditions. Valid tests of the SM are those that can conceivably fail, revealing the presence of new physics effects. Theoretical BSM considerations thus provide a valuable guidance for the experimental exploration of the SM theory. This guidance becomes particularly strong and sharp under the hypothesis that all the new physics particles are heavy, such that their observable effects are encapsulated in Effective Field Theory (EFT) interaction operators of energy dimension larger than four. In this section we discuss the muon colliders sensitivity to putative EFT interactions beyond the SM, enabling a systematic and comprehensive exploration of high-scale new physics models.

Since hypothesising heavy new physic might seem a pessimistic attitude, for a collider project with great opportunities for direct discoveries, it is worth outlining the value of EFT studies like those of the present section if new physics is instead light. Even if not accurately described by the EFT, relatively light new physics could contribute to the same processes and observables, and be discovered by the same measurements that we are considering here to probe the EFT. Moreover even if the new light particles will be first discovered in other processes, probing their indirect effects will be an essential step for the characterisation of their properties and interactions with the SM particles. In other words, a program of characterisation for the newly discovered BSM physics would be similar to the program of SM characterisation based on the SM EFT.

For a first assessment of the muon colliders' potential to probe the SM EFT, we perform a global fit to the following dimension-6 EFT Lagrangian
%
\begin{eqnarray}
&{\cal L}_{\mathrm{SILH}}=
\frac{c_\phi }{\Lambda^2} \frac 12 \partial_\mu (\phi^\dagger \phi)\partial^\mu (\phi^\dagger \phi)+ 
\frac{c_T}{\Lambda^2} \frac 12(\phi^\dagger \lrD_\mu \phi)(\phi^\dagger \lrD^\mu \phi)\nonumber \\
%
&
- 
\frac{c_6}{\Lambda^2} \lambda (\phi^\dagger \phi)^3+\sum_f\left(\frac{c_{y_f} }{\Lambda^2} y^f_{ij} \phi^\dagger \phi  \bar{\psi}_{Li} \phi \psi_{Rj} + \hc \right)\nonumber \\
%
&
+\frac{c_W}{\Lambda^2} \frac{ig}{2}\phi^\dagger \lrDa_\mu \phi D_\nu W^{a~\!\mu\nu} 
+\frac{c_B}{\Lambda^2} \frac{ig^\prime}{2}\phi^\dagger \lrD_\mu \phi \partial_\nu B^{\mu\nu}
\nonumber \\
%
&+\frac{c_{\phi W}}{\Lambda^2} i g D_\mu\phi^\dagger \sigma_a D_\nu \phi W^{a~\!\mu\nu}+\frac{c_{\phi B} }{\Lambda^2} i g^\prime D_\mu\phi^\dagger \sigma_a D_\nu \phi B^{\mu\nu}
\nonumber \\
%
&+\frac{c_{\gamma} }{\Lambda^2} g^{\prime~\!2}  \phi^\dagger \phi B^{\mu\nu}B_{\mu\nu}+\frac{c_{g}}{\Lambda^2} g^{2}_s  \phi^\dagger \phi G^{A~\!\mu\nu}G^A_{\mu\nu}
\nonumber \\%-\frac{c_{2G}}{\Lambda^2}\frac{g_S^2}{2} (D^\mu G_{\mu\nu}^A)(D_\rho G^{A~\!\rho\nu})\\
%
&-\frac{c_{2W}}{\Lambda^2} \frac{g^2}{2} D^\mu W_{\mu\nu}^aD_\rho W^{a~\!\rho\nu}-\frac{c_{2B}}{\Lambda^2} \frac{g^{\prime~\!2}}{2}\partial^\mu B_{\mu\nu}\partial_\rho B^{\rho\nu}\nonumber \\
&
+\frac{c_{3W}}{ \Lambda^2}g^3\varepsilon_{abc}W_{\mu}^{a~\nu}W_\nu^{b~\rho}W_\rho^{c~\mu}. %+\frac{c_{3G}}{ \Lambda^2} g_S^3 f_{ABC}G_{\mu}^{A~\nu}G_\nu^{B~\rho}G_\rho^{C~\mu}.
\label{eq:LSilh}
\end{eqnarray}
%
While this is only a subset of the operators of the more general dimension-6 SM EFT, the operators above are of special relevance for several BSM scenarios. Explicit examples are Composite Higgs scenarios and $U(1)$ extensions of the SM, to be discussed later in this section. The selection of operator in eq.~(\ref{eq:LSilh}) follows the one done as part of the 2020 European Strategy Group studies \cite{EuropeanStrategyforParticlePhysicsPreparatoryGroup:2019qin,deBlas:2019rxi}, enabling a direct comparison with the sensitivity projections of other future collider projects.


\begin{figure*}[t]
\centering
\begin{tabular}{c c}
\hspace{-0.2cm}\includegraphics[width=0.49\linewidth]{figures/CHplotMuColl_Excl.pdf}
&
\includegraphics[width=0.49\linewidth]{figures/MZpplotMuColl_Excl.pdf}\\
%
\end{tabular}
%
\caption{\label{fig:CH_Zp}
(Left) The global reach for universal composite Higgs models at the HL-LHC and a high-energy muon collider. The figure compares the 2-$\sigma$ exclusion regions in the $(g_\star, m_\star)$ plane from the fit presented in Figure~\ref{fig:SILH}, using the power-counting in eq.~(\ref{eq:SILHpc}).
(Right) The same for a model featuring with an heavy replica of the $U(1)_Y$ gauge boson in the $(g_{Z^\prime}, m_{Z^\prime})$.
}
\end{figure*}
%


In the EFT fits to eq.~(\ref{eq:LSilh}) we include the following set of experimental inputs and projections:
%
\begin{itemize}
{\item The complete set of electroweak precision measurements from LEP/SLD~\cite{ALEPH:2005ab}, including the projected measurements of the $W$ mass at the HL-LHC~\cite{Azzi:2019yne}.  We also include the anomalous trilinear gauge coupling constraints from LEP2.}
%
{\item The HL-LHC projections for single Higgs measurements and for double Higgs production from~\cite{Cepeda:2019klc}. We assume the S2 scenario for the projected experimental and theoretical systematic uncertainties.}
%
{\item Also from the HL-LHC, the projections from two-to-two fermion processes, expressed in terms of the W and Y oblique parameters, from Ref.\cite{Farina:2016rws}, and the high energy diboson study from~\cite{Franceschini:2017xkh}.}
%
{\item The expected precision for single-Higgs observables at the 3 and 10 TeV muon colliders from the results of~\cite{Forslund:2022xjq}, previously described.}
%
{\item As in the HL-LHC case, we also include the projections from high-energy measurements in two-to-two fermion processes, expressed in terms of W and Y from~\cite{Chen:2022msz}, and in diboson processes such as $\mu^+\mu^-\to ZH, W^+W^-$, $\mu \nu \to WH,WZ$ from Ref.~\cite{Chen:2022msz,Buttazzo:2020uzc}.}
%
{\item The di-Higgs invariant mass distribution in $\mu^+ \mu^- \to \bar{\nu}\nu HH$ from Ref.~\cite{Buttazzo:2020uzc} (see also \cite{Han:2020pif}), as a probe of the $c_\phi$ and $c_6$ operators.
}
%
\end{itemize}
%
In all cases we assume the projected experimental measurements to be centred around the SM prediction. The assumptions in terms of theory uncertainties follow the same setup as in \cite{deBlas:2019rxi}.

Our analysis ignores the recent CDF determination of the $W$ mass~\cite{CDF:2022hxs}, which is in strong tension with the SM interpretation of the electroweak data and requires BSM physics. If confirmed, the CDF anomaly will become a major target for studies of new physics in the electroweak sector, and in particular for the SM EFT investigations described in the present section. The specific opportunities offered by the muon collider for exploring a possible BSM origin of the CDF anomaly have not been studied yet.

The results of these EFT fits are summarised in Table~\ref{tab:eft-silh} and Figure~\ref{fig:SILH}. Relative to the HL-LHC, a reach improvement of one order of magnitude is found at the 10~TeV~MuC for several operators, among which ${\cal O}_\phi$ and ${\cal O}_6$, as shown in the lower panel of the figure. The improvement on the ${\cal O}_\phi$ operator stems from the accurate measurements of the single-Higgs couplings (dominantly, $HZZ$ and $HWW$) and on the measurement of the invariant mass in VBF double-Higgs production, where ${\cal O}_\phi$ induces an energy-growing deformation. The improvement on the ${\cal O}_6$ operator simply follows from the accurate determination of $\delta\kappa_\lambda$. As previously discussed, these measurements exploit vector bosons initiated processes, and their accuracy reflects the effectiveness, emphasised in Section~\ref{VBF}, of the muon collider as a vector boson collider. 

A two orders of magnitude improvement is instead found for operators such as ${\cal O}_{W,B}$ and ${\cal O}_{2B, 2W}$. These operators induce growing with energy effects in diboson and difermion processes respectively, therefore they are very effectively probed by high-energy measurements as explained in Section~\ref{HEM}. 

As in the $\kappa$ analysis, we must also note that the muon collider potential on some operators such as ${\cal O}_{y_u}$ could be underestimated, due to the absence of detailed projections for the processes that could probe them most effectively. In the case of ${\cal O}_{y_u}$, a combination of different Higgs and top-quark measurements may improve the sensitivity~\cite{Franceschini:2021aqd}. 

Finally, we remark that all the sensitivity projections included in the fit assume unpolarised  muon beams. Polarised beams could bring extra information, allowing to test more directions in the SM EFT parameter space or resolving flat directions (see e.g.~\cite{Barklow:2017suo,DeBlas:2019qco}). For instance it was shown in~\cite{Chen:2022msz,Buttazzo:2020uzc} that the availability of polarised muon beams would further improve the reach on the ${\cal O}_{W,B}$ operators from diboson high-energy measurements, even for limited beam polarisation fraction. The  feasibility of polarised beams is not addressed by the current design studies, but it is not excluded.

The 3~TeV MuC is less performant than the 10~TeV one, but still enables a jump ahead of one order of magnitude or more, relative to HL-LHC, for several operators such as ${\cal O}_{W,B}$, ${\cal O}_{2B, 2W}$ and ${\cal O}_{\phi W,\phi B}$. The impact of these advances on the exploration of concrete BSM scenarios is detailed below and compared with the perspective for the progress attainable at all the other future collider projects that are currently under consideration. 

\begin{figure*}[ht]
\centering
\begin{tabular}{c c}
\hspace{-0.2cm}\includegraphics[width=0.49\linewidth]{figures/CHplotMuColl_Excl_ops_MC3TeV.pdf}
&
\includegraphics[width=0.49\linewidth]{figures/CHplotMuColl_Excl_ops_MC10TeV.pdf}\\
%
\end{tabular}
%\includegraphics[width=0.47\linewidth]{figures/CHplotMuColl_Excl.pdf}
\caption{\label{fig:SILH_gstar}
The breakdown of the global reach on composite Higgs, reported in Figure~\ref{fig:CH_Zp}, in the contribution of the individual processes. The 3 and the 10~TeV muon colliders are considered in the left and right panels, respectively.
}
\end{figure*}

\paragraph{BSM benchmark interpretations}{\ } \\ 
\noindent
The fit results can be used to perform sensitivity projections on specific new physics scenarios, enabling a concrete illustration of the muon colliders potential for indirect BSM searches. We consider a composite Higgs scenario and a simple $Z^\prime$ model, to be discussed in turn. 

Composite Higgs is described under the assumption that the new dynamics is parameterised in terms of a single coupling, $g_\star$, and mass, $m_\star$. Furthermore, as in \cite{deBlas:2019rxi}, we assume for definiteness that the new physics contributions to the operator Wilson coefficients in~(\ref{eq:LSilh}) follow the power-counting formula with unit numerical coefficients, namely we take
\begin{eqnarray}
&&\frac{c_{\phi,6,y_f}}{\Lambda^2}= \frac{g_\star^2}{ m_\star^2},\;\;\;\;\;
%
\frac{c_{W,B}}{\Lambda^2}= \frac{1}{m_\star^2},
\;\;\;\;\;
\frac{c_{2W,2B}}{\Lambda^2}= \frac{1}{g_\star^2}\frac{1}{m_\star^2},
\nonumber\\
&&
\frac{c_{T}}{\Lambda^2}= \frac{y_t^4}{16\pi^2}\frac{1}{m_\star^2},\;\;\;\;\;
%
\frac{c_{\gamma,g}}{\Lambda^2}= \frac{y_t^2}{16\pi^2}\frac{1}{m_\star^2},
\nonumber\\
&&
%
\frac{c_{\phi W,\phi B}}{\Lambda^2}= \frac{g_\star^2}{16\pi^2}\frac{1}{m_\star^2},
%
\frac{c_{3W}}{\Lambda^2}= \frac{1}{16\pi^2}\frac{1}{m_\star^2}.
%\frac{c_{3W,3G}}{\Lambda^2}= \frac{1}{16\pi^2}\frac{1}{m_\star^2}.
%
\label{eq:SILHpc}
\end{eqnarray}
%
The operators above define the so-called ``Universal'' manifestation of the Composite Higgs scenarios. Additional effects that depend on the degree of compositeness of the top quark, not considered here, have been studied in Ref.~\cite{Chen:2022msz}.

By projecting the EFT likelihood onto the $(g_\star,m_\star)$ plane we obtain the exclusion regions in the right panel in Figure~\ref{fig:CH_Zp} for the different muon collider options, combined and in comparison with the HL-LHC reach. The results agree with those of Figure~\ref{fig:HEM}.\footnote{The HL-LHC exclusion is in fact stronger than in Figure~\ref{fig:HEM} at large $g_\star$. This is because the conservative $c_\phi$ limit (see the footnote in Table~\ref{tab:eft-silh}) is  employed in Figure~\ref{fig:HEM}.}

We also show the EFT fit results interpretation in a simple BSM model featuring a single $Z^\prime$ massive vector boson. As in \cite{EuropeanStrategyforParticlePhysicsPreparatoryGroup:2019qin}, we consider a $Z^\prime$ coupled to the hypercharge current. In this case the dimension-6 effective Lagrangian only receives tree-level contributions to the operator with coefficient $c_{2B}/\Lambda^2=g_{Z^\prime}^2/(g^{\prime~\!4} M_{Z^\prime}^2)$. The corresponding indirect constraints in the $(g_{Z^\prime},M_{Z^\prime})$ plane are shown in the right panel of Figure~\ref{fig:CH_Zp}. 

While the bounds on the $Z^\prime$ model is obviously dominated by the high-energy measurements of difermion process, and the resulting constraints on the Y parameter (i.e., the ${\cal{O}}_{2B}$ operator coefficient), the situation is more complex for composite Higgs. The contributions from the different processes in setting the limits are shown separately in Figure~\ref{fig:SILH_gstar}, highlighting (see also~\cite{Chen:2022msz,Buttazzo:2020uzc}) the complementarity of different probes. The diboson constraints set the overall mass reach, independently of $g_\star$. The reach gets extended for low values of $g_\star$ by the difermion measurements. For high $g_\star$, $c_\phi$ bounds from Higgs coupling determinations and from the di-Higgs mass distribution measurement dominate the sensitivity.

The muon colliders sensitivity to composite Higgs, and to a $Z^\prime$ model that is representative of new physics affecting the electroweak interactions, was emphasised already in Section~\ref{HEM}. In particular we pointed out that the 10~TeV muon collider is more effective than any other future collider project that is currently under design or consideration. This can be seen also in Figure~\ref{fig:CH_Zp}, by comparing with the dashed line labelled as ``Others''. This line is formed by the envelop of the contours probed in the same plane by the FCC programme (including FCC-ee and FCC-hh), by all stages of CLIC and ILC, and all other collider projects studied in~\cite{EuropeanStrategyforParticlePhysicsPreparatoryGroup:2019qin} for the 2020 updated of the European Strategy for Particle Physics. The mass reach would further improve in proportion to the muon collider energy, provided the luminosity scales quadratically with the energy as in eq.~(\ref{lums}). 

Figure~\ref{fig:CH_Zp} also shows that the 3~TeV MuC sensitivity is similar to the one of the most effective alternative project (namely the FCC, including the FCC-hh), and vastly superior to the one of the HL-LHC. The figures conservatively assumes 1~ab$^{-1}$ integrated luminosity. EFT studies like those described in this section thus provide strong physics motivations for the first stage of the muon collider with a centre of mass energy of around 3~TeV.

\begin{figure*}[ht]
\begin{centering}
\includegraphics[width=0.67\linewidth]{5_Physics_Studies/figures/singlet_new}
\caption{95~\% 
C.L. reach (adapted from~\cite{Buttazzo:2018qqp}) on heavy singlet mixed with the 
SM Higgs doublet at the muon collider. The direct and indirect reach at other future projects and at the HL-LHC, documented in~\cite{EuropeanStrategyforParticlePhysicsPreparatoryGroup:2019qin}, is also shown for comparison.\label{fig:Singlet-reach}}
\par\end{centering}
\end{figure*}

\subsubsection*{Extended Higgs sectors}
%\label{sec:extendedHiggs}
The third exploration strategy by which muon colliders can advance knowledge of the electroweak and Higgs sector is to search directly for the resonant production of new particles. Few illustrative results are reported below, focusing in particular on BSM models that foresee an extension of the Higgs sector by additional scalar fields. The new particles in these scenarios do not carry QCD colour, therefore the mass reach of the LHC is intrinsically limited. The muon collider is thus generically expected to improve radically over the HL-LHC sensitivity projections already at the 3~TeV stage.

\paragraph{SM plus singlet}{\ } \\ 
\noindent
The simplest extension of the SM Higgs sector features one additional scalar field in the singlet of the SM gauge group. Its simplest interaction with the SM, if not forbidden by additional symmetries, is the ``Higgs portal'' (see also Section~\ref{dirr}) trilinear coupling $S\,H^\dagger H$ with the SM Higgs doublet $H$. By this coupling, the singlet mixes with the physical Higgs boson $h$. Following~\cite{Buttazzo:2018qqp}, we denote as $\sin\gamma$ the mixing parameter and as $m_S$ the physical mass of the singlet particle. We trade the fundamental Lagrangian parameter for $m_S$ and $\sin\gamma$, and we study the model in the $(m_S,\sin^2\gamma)$ plane as in Figure~\ref{fig:Singlet-reach}.

The mixing with the singlet scales down all the couplings of the Higgs particle by $\cos\gamma \simeq 1-1/2\,\sin^2\gamma$. It has the same effect as the ${\cal{O}}_\phi$ operator with $c_\phi / \Lambda^2 =\sin^2\gamma/v^2$. Therefore, the indirect sensitivity to $\sin^2\gamma$ can be read from Table~\ref{tab:eft-silh} and it is reported in Figure~\ref{fig:Singlet-reach} for the 3~TeV muon collider with 2~ab$^{-1}$ and for the 10~TeV MuC. The indirect sensitivity follows the general pattern previously described. The 3~TeV MuC improves Higgs coupling determination slightly and hence it slightly improves the HL-LHC sensitivity to $\sin^2\gamma$, by a factor 3 in this specific case. The 10~TeV MuC performances are comparable to the ones of an $e^+e^-$ Higgs factory, leading to the same reach on $\sin^2\gamma$.

Unlike low-energy Higgs factories, the muon collider also offers opportunities to produce the new scalar particle directly. The production occurs dominantly via the VBF $VV \rightarrow S$ process, that benefits from the large luminosity for effective vector bosons at the muon collider, as already discussed in Section~\ref{dirr}. The single production vertex $S\,V_L V_L$ involves longitudinally polarised vectors and it emerges directly from the $S\,H^\dagger H$ trilinear interaction via the Equivalence Theorem. The same interaction mediates the dominant decays of the singlet, to two massive vectors or to a pair of Higgs boson. The latter channel with Higgs decays to bottom quarks, namely $S \rightarrow hh \rightarrow 4b$, is the most sensitive one at high energy lepton colliders and it is employed~\cite{Buttazzo:2018qqp} in the direct reach estimates presented in Figure~\ref{fig:Singlet-reach}. 

\begin{figure*}
\centering{}\includegraphics[width=0.45\linewidth]{figures/MuC_EWPT_diHiggs_4.png}
%
\includegraphics[width=0.43\linewidth]{figures/MuC_EWPT_indirect_6.png}
\caption{\label{fig:EWpt-reach}Direct (left panel) and indirect (right panel) reach on the SM plus real scalar singlet scenario at the muon collider.  Dots indicate points with successful first-order EWPT, while red, green and blue dots represent signal-to-noise ratio for gravitational eave detection in the ranges $[50, +\infty)$, $[10, 50)$ and $[0, 10)$, respectively. Results adapted from~\cite{Liu:2021jyc}.
 }\end{figure*}

By combining the two blue lines (horizontal dashed and continuous), and comparing with the black lines, we can appreciate the potential progress of the 3~TeV MuC with respect to the HL-LHC.\footnote{The indirect HL-LHC sensitivity is most likely overestimated, as the aggressive bound on $c_\phi$ (see the footnote in Table~\ref{tab:eft-silh}) is employed in the figure.} We also notice a considerable region that can be only probed indirectly at the HL-LHC. In that region the presence of the singlet would produce tensions in the HL-LHC measurements of the Higgs couplings and the physics underlying such tension will be discovered directly at the 3~TeV MuC. 

The reach is greatly extended by the 10~TeV MuC. It covers almost all the region that could be probed by the combination (black dotted lines) of the FCC-ee and FCC-hh direct and indirect searches. Furthermore its reach extends to much higher mass in the region of small mixing angle. This improvement is of particular significance in light of the fact that a smaller mixing should be expected to emerge at higher mass, in concrete realisations of the singlet scenarios. This is illustrated in the figure by the grey dashed lines that correspond to two possible power-counting estimates for the scaling of the microscopic Lagrangian parameters. See~\cite{Buttazzo:2018qqp} for details.

The SM plus singlet model does not only provide a simple benchmark for future colliders comparison. It can mimic the signatures, and be reinterpreted, in strongly motivated BSM scenarios like Supersymmetry and Twin Higgs models~\cite{AlAli:2021let,Buttazzo:2018qqp}. ``Non-minimal'' (but plausible) Composite Higgs models could also feature a singlet or other scalar extensions of the Higgs sector~\cite{Gripaios:2009pe,Mrazek:2011iu}. SM plus a real singlet extension can also provide a strong first order ElectroWeak Phase Transition (EWPT), which is an essential ingredient for the electroweak baryogenesis mechanism potentially responsible for the matter-antimatter asymmetry in the Universe. Following~\cite{Liu:2021jyc} (see also~\cite{Ruhdorfer:2019utl}), we illustrate below the muon collider potential to probe this scenario.

In the left panel of Figure~\ref{fig:EWpt-reach}, the coloured solid curves show the muon collider 95\%~C.L. direct exclusion reach in the plane formed by the singlet mass and the product $\sin^2\gamma\times{\rm{BR}}(S\to hh)$.\footnote{The latter quantity, and not only $\sin^2\gamma$, is what controls the events yield in the di-Higgs final state. The branching ratio is set to $1/4$ in Figure~\ref{fig:Singlet-reach}, which is a good approximation when the singlet is heavy but not so in the mass range of Figure~\ref{fig:EWpt-reach}.} The points marked on the figure are obtained from a scan over the microscopic parameters of the specific model considered in Ref.~\cite{Liu:2021jyc}, and they correspond to configurations where the EWPT is of the first order and strong enough for electroweak baryogenesis. The 3~TeV MuC covers several of the relevant points, while the 10~TeV MuC enables an almost complete coverage. The points marked in red or in green (unlike those in blue) could perhaps also produce observable gravitation waves at LISA. Strong first order EWPT requires a modification of the Higgs potential. Therefore sizeable departures of the trilinear Higgs coupling with respect to the SM are expected in this scenario. This is shown on the right panel of Figure~\ref{fig:EWpt-reach}, in the plane formed by a universal modifier $\delta\kappa$ that affects all the single-Higgs couplings, and the trilinear coupling modifier $\delta\kappa_\lambda$. We see that the muon collider, already at the 3~TeV stage, has considerable chances to be sensitive to the predicted single- or triple-Higgs coupling modifications. It is in fact likely to observe correlated modifications in both couplings.

\begin{figure*}[ht]
\centering{}\includegraphics[width=0.45\linewidth]{figures/Scalar_Pair_3TeV.pdf}\hfill
\includegraphics[width=0.45\linewidth]{figures/Fermion_Associated_Radiation_Return_3TeV.pdf}
\caption{\label{fig:Charged-reach}
Cross sections versus the  non-SM Higgs mass for pair production (left panel), single production with a pair of fermions and radiative return production (right panel) at the 3~TeV muon collider. The value of $\tan\beta=1$, and the alignment limit $\cos(\alpha-\beta)=0$, is considered in the figure.}
\end{figure*}

\paragraph{Two Higgs Doublet Model}{\ } \\ 
\noindent
Models with two Higgs doublets (2HDM) are another important target for muon colliders.~\footnote{Other extensions of the Higgs sector, in particular those featuring a doubly-charged scalar, were studied in~\cite{Rodejohann:2010jh,Li:2023ksw}} While much work is still to be done for the detailed assessment of the muon collider potential, a rather complete characterisation of the relevant phenomenology was provided in Ref.~\cite{Han:2021udl}, whose findings are briefly summarised below. Like in the case of the singlet model, very significant progress on the 2HDM parameters space is possible already at the first 3~TeV stage of the muon collider. In what follows we stick to this energy for definiteness. At the higher energies muon colliders, which are also considered in~\cite{Han:2021udl}, the performances improve.

The scalar sector of the 2HDM consists of 5 physical particles: the SM-like Higgs $h$ with  $m_h=125$~GeV and the non-SM ones $H,A,H^\pm$. The tree-level couplings of the Higgs bosons are determined by the mixing angle between the neutral CP-even Higgs bosons, $\alpha$, and by a second parameter $\tan\beta=v_2/v_1$, with $v_{1,2}$ being the vacuum expectation value for two Higgs doublets. The dominant couplings of the Higgses with the SM gauge bosons typically involve two non-SM Higgses, for example, $ZHA$ or $W^\pm H^\mp H$.  The Yukawa couplings of the non-SM like Higgses with the SM fermions depends on how the two Higgs  doublets  are  coupled  to  the  leptons  and  quarks via Yukawa couplings. Four different patterns of Yukawa couplings are typically considered in the literature, giving rise to four different types of 2HDMs that we denote as Type-I, Type-II, Type-L and Type-F, following the notation of Ref.~\cite{Branco:2011iw}. 

The couplings of the SM Higgs bosons are generically modified in the 2HDM, potentially producing observable effects at the HL-LHC. In this event, the 3~TeV MuC could access directly the new heavy bosons, likely too heavy to be observed at the LHC, and establish the origin of the putative discrepancy. If instead the HL-LHC Higgs couplings measurements will not deviate from the SM predictions, the masses of the extra scalars could still be within the reach of the 3~TeV MuC, and even well below if the model parameters are in the so-called ``alignment limit'' where $\cos(\alpha-\beta)\approx 0$ and the Higgs coupling modifications are suppressed. In both cases it will be interesting to search directly for the new scalars. In what follows we thus discuss the 3~TeV MuC direct discovery potential, focusing on nearly ``aligned'' parameters configurations.

The heavy Higgs bosons can be produced in pairs via the direct $\mu^+\mu^-$ annihilation mediated by the exchange of a virtual photon or $Z$ boson, or via Vector Boson Fusion (VBF)
\begin{eqnarray}
\mu^+\mu^-\to H^+ H^-
,\qquad\mu^+\mu^-\to HA,
&&\nonumber \\
 V_1V_2\to H^+H^-, HA, H^\pm H/H^\pm A, HH/AA. &&
\label{eq:anni}
\end{eqnarray}
In the latter case, the process proceeds by the collinear emission of $V_{1,2}=W,Z$, or $\gamma$, out of the incoming muons. The cross section for the different processes at the 3~TeV MuC as a function of the new scalars mass is shown on the left panel of Figure~\ref{fig:Charged-reach}. The annihilation processes dominates for masses above around 500~GeV. The VBF cross section steeply increases, and it would dominate at lower masses that however could be probed also at the HL-LHC and thus are not worth discussing here. It is worth emphasising that the situation is different at the muon collider with 10~TeV center of mass energy or more. In that setup, VBF channels become more important and dominate at low mass~\cite{Han:2021udl}.

The figure shows that a very large number of events is expected with 1~ab$^{-1}$ integrated luminosity. Furthermore, considering the dominant decay channel of non-SM Higgs into third generation fermions, the SM backgrounds can be easily suppressed. Reach up to pair production threshold $m_\phi<1.5$~TeV is thus generically possible. A detailed comparison of the 3~TeV muon collider reach with the HL-LHC sensitivity is not yet available. For Type-II 2HDM~\cite{Craig:2016ygr}, the 3~TeV muon collider reach is superior in the intermediate region of $\tan\beta \in [2, 10]$. For larger $\tan\beta$, a reach above 1.5~TeV mass has already been attained at the LHC~\cite{ATLAS:2022yvl}.

In the parameter region with enhanced Higgs Yukawa couplings, single production of non-SM Higgs with a pair of fermions could play an important role and potentially extend the sensitivity above the pair production threshold. The production cross section for fermion associated production are shown in the right panel of Figure~\ref{fig:Charged-reach} for both the annihilation and VBF processes, with $\tan\beta=1$ and $\cos(\alpha-\beta)=0$. The dominant channel is $tbH^\pm$, followed by $t\bar{t}H/A$. Note that there is a strong dependence on $\tan\beta$ of the production cross section, depending on the types of 2HDM~\cite{Han:2021udl}. Finally, the radiative return process $\mu^+\mu^- \to \gamma H$ offers another production channel for the non-SM Higgs, which is relevant in regions with enhanced $H\mu^+\mu^-$ coupling. The cross section increases as the heavy Higgs mass approaches the collider c.m.~energy, closer to the $s$-channel resonant production. The production cross section is shown as the black curves in the right panel of Figure~\ref{fig:Charged-reach}. 

%%%%%%%%%%%%%%%%%%%%%
%%%%%%%%%%%%%%%%%%%%%
\subsection{Dark matter}\label{secdm}
%%%%%%%%%%%%%%%%%%%%% 
%%%%%%%%%%%%%%%%%%%%%
A global assessment of the MuC perspectives to search for Dark Matter (DM) is not yet available. The studies so far, reviewed below, investigated the MuC potential to probe scenarios where DM is a particle charged under the EW interactions and its observed abundance in the Universe emerges from the thermal freeze-out mechanism. This is a compelling possibility, and a one where the MuC will play a major role. However, the opportunities to probe other interesting scenarios for DM, either through muon collisions or with parasitical experiments, or as a byproduct of the muon collider demonstration program, should be also investigated.

If DM is a WIMP (i.e., a Weakly Interacting Massive Particle), it could be directly detected in ultra-low noise underground detectors by its interaction with the detector material. It can be also searched  in DM-rich astrophysical environments, where the DM pairs can annihilate and give rise to observable signals at cosmic ray observatories. These experimental investigation strategies have been and are being actively pursued, and are promising but suffer few potential roadblocks. Cosmic rays observation can be hampered by   large uncertainties on   astrophysical quantities and astrophysical processes that can mimic DM signals. Furthermore, these experiments exploit DM particles that are physically present in the Universe, whose local density is poorly known. This entails strong uncertainties on the expected signal. For instance, a recent analysis in Ref.~\cite{Rinchiuso:2020skh} quantifies the effect of density profile uncertainty for WIMP searches and finds order of magnitude effects on the (still very promising) cross section sensitivities of future generation experiments such as CTA~\cite{CTA:2020qlo}. Lab-based direct detection is less affected by profile uncertainties. But it suffers from being a very low momentum transfer process---even when DM is quite heavy---which makes background rejection very challenging. Future experiments like the Xenon's upgrades~\cite{Aalbers:2022dzr} will have sensitivity to a large variety of possible WIMP candidates, but also blind spots. See e.g.~\cite{Bottaro:2021snn,Bottaro:2022one} for an appraisal.

If DM is a WIMP, it can also be produced in the laboratory through EW interactions, provided a particle collider of sufficient energy and luminosity is available. The possibility of producing putative DM candidates and studying their signatures with precise particle detectors would, in the first place, firmly establish or conclusively exclude their existence. Furthermore, it will offer unique opportunities for the characterisation of the newly discovered DM particle. The vibrant forthcoming programme of non-collider DM searches, and the possible signals of WIMP DM that may emerge, thus provide additional motivations for collider studies. 

WIMP searches are also useful as sensitivity benchmarks to gauge the effectiveness of particle colliders to test DM, and to compare different collider projects. In fact, the WIMP scenario assumptions single out a relatively small set of compelling minimal benchmark models with no free parameters~(see~\cite{Bottaro:2021snn} and references therein). This is because the WIMP relic abundance is set by the (known) strength of the weak interactions coupling and the (unknown) mass of the WIMP. Therefore, for minimal models where the WIMP consists of a single ${\textrm{SU}}(2)_{L}$ $n$-plet it is possible to sharply predict the mass of the dark matter particle that produces the observed relic. Some examples are reported in Table~\ref{tab:thermal-masses}. As a general rule, the larger the $n$-plet the larger the mass of the WIMP. Smaller masses can be attained if the $n$-plet mixes with a state of lower multiplicity, e.g. a singlet. Therefore, testing the mass reach when only one ${\textrm{SU}}(2)_{L}$ $n$-plet is present as in the minimal models effectively demonstrates sensitivity to non-minimal models as well, as it demands to reach the highest mass for a given class of candidates.

\begin{table}[t]
\vspace{10pt}
\begin{centering}
\begin{tabular}{c|c||c|c|c|c}

%\hline 
$n$ & Y & Dirac & Majorana & Complex & Real\tabularnewline
\hline 
\hline 
2 & 1/2 & 1.08 & - & 0.58 & -\tabularnewline
\hline 
3 & 0 & -& 2.86%(1) 
& - & 2.53%(1)
\tabularnewline
3 & $\epsilon$ & 2.0 \& 2.4 & -%(1) 
& 1.6 \& 2.4 &-%(1)
\tabularnewline
3 & 1 & 2.84(14) & -%(1) 
& 2.1 & -%(1)
\tabularnewline
\hline 
4 & 1/2 & 4.79(9) & - & 4.98(5) & -\tabularnewline
\hline 
5 & 0 & - & 13.6(5) & - & 15.4(7)\tabularnewline
5 & $\epsilon$ & 9.1(5) & - & 11.3(7) & -\tabularnewline
5 & 1 & 9.9(7) &- & 11.5(7) &-\tabularnewline
\hline 
\end{tabular}
\par\end{centering}
\caption{\label{tab:thermal-masses}Thermal mass, in TeV, for pure \mbox{SU$(2)_L$} $n$-plet dark matter WIMP, from Ref.~\cite{Bottaro:2021snn,Bottaro:2022one}. Some of the candidates are endowed with a tiny hypercharge $\epsilon$. Effects of bound states and Sommerfeld enhancement of the annihilation cross section are included in the calculation of the thermal mass.
}
\end{table}

A crucial phenomenological parameter for the detection of WIMPs at colliders is the mass splitting between the lightest, neutral component of the $n$-plet, which constitutes the actual DM particle, and the other electrically charged and neutral components of the multiplet. When this mass splitting is greater than some threshold, typically around 10~GeV, it is relatively easy to observe the production of the heavier states in the $n$-plet. These particles decay to the actual DM particle, which is invisible, accompanied however by easily detectable SM particles. If instead the mass splitting is below the threshold, one needs to rely on ``Mono-X'', on ``Indirect'' or on ``Disappearing Tracks'' strategies for detection. We describe these strategies in turn in the rest of this section.

\subsubsection*{Mono-X}% \label{subsec:monoX}}

\begin{figure*}
\centering{}
\includegraphics[width=0.47\linewidth]{figures/lumin_3TeV.png}
\includegraphics[width=0.47\linewidth]{figures/BarsMonoX}\\
\includegraphics[width=0.47\linewidth]{figures/lumin_10TeV.pdf}
\includegraphics[width=0.47\linewidth]{figures/BarsMonoX10}\\
\caption{\label{fig:EWstates}Direct reach on electroweak states in mono-$X$ signals. Left: Luminosity needed to exclude  a Dirac fermion DM candidate for zero systematics~\cite{Han:2020uak} for $X=\gamma$ (solid), $X=\mu$ (dotted), $X=\mu\mu$ (dashed). Right: Mass reach on a fermionic DM candidate (assumed Majorana when $Y=0$, Dirac otherwise) at fixed 1~${\rm{ab}^{-1}}$ luminosity for the 3 TeV and 10~${\rm{ab}^{-1}}$ for 10 TeV muon collider in the channels $X=\gamma$ and $X=W$ for 0.1\% systematics~\cite{Bottaro:2021snn,Bottaro:2022one}. Black vertical lines denote the thermal mass for each
DM candidate.}
\end{figure*}

The Mono-X strategy is to observe DM production ``by contrast'', i.e. to observe a bunch of particles apparently recoiling against nothing.
At a muon collider the relevant reaction is
\begin{equation}
    \mu^{+}\mu^{-} \to \chi \chi +X \,,\label{eq:monoX}
\end{equation}
where $X$ denotes any SM particle or set of particles allowed by the interactions and $\chi$ is either the DM particle or a generic state belonging to the DM $n$-plet. In both cases, $\chi$ is not seen if the mass splitting is low and no dedicated strategy is adopted to detect disappearing track as discussed later in this section. 
%Section~\ref{secunc}.


\begin{figure*}
    \centering
    \includegraphics[width=0.45\textwidth]{figures/fermionic_ltplot_30000_horizontal_track_syst.pdf}\hfill
        \includegraphics[width=0.45\textwidth]{figures/scalar_ltplot_30000_horizontal_track_syst.pdf}
    \caption{Mass reach in the mono-$\gamma$, mono-$W$ and DT channels with luminosity scaling with energy as in~\eqref{lums} at muon colliders of different energy $\sqrt{s}$, from Ref.~\cite{Bottaro:2022one}. Vertical bars display the thermal mass of the candidates with its uncertainty. In the mono-$W$ and mono-$\gamma$ searches we show an error bar, which covers the range of possible exclusion as the systematic uncertainties are varies from 0 to 1\%. For single displaced tracks the   error bar covers a possible systematic uncertainty from 0\% to 10\%. The coloured bars are for an intermediate choice of systematics at 0.1\% (1\% for 1DT). 
    %All results consider only the masses range above $0.1\sqrt{s}$, where the Drell-Yan production mode is dominant. 
    Missing bars denoted by an asterisk * correspond to cases where no exclusion can be set in the mass range $M_\chi>0.1\sqrt{s}$. For such cases would be worth considering also the VBF production modes.}
    \label{fig:summary_all_colliders}
\end{figure*}

Searches for generic electroweak states have been studied for several types of observable particles $X$ accompanying the production of dark matter. The signal for $X=\gamma,W,Z,\mu^{\pm}$ and $\mu\mu$ have been studied in  \cite{Han:2020uak,Bottaro:2021snn}. Figure~\ref{fig:EWstates} summarises the reach illustrating on the left panels the luminosity needed to attain the 95\% CL exclusion of electroweak matter of a given mass in the production modes $X=\gamma,\mu,\mu\mu$. Among these, the mono-$\gamma$ search is the one placing the best bound for states heavier than about 500~GeV. The right panels summarise the reach with 1~${\rm{ab}^{-1}}$ and 10~${\rm{ab}^{-1}}$ for the 3 and the 10~TeV MuC (upper and lower panel). The mono-$W$ channel reach is reported on the right panel plots, together with mono-$\gamma$. This channel is effective for the same mass range in which mono-$\gamma$ leads the exclusion and in some cases exceeds mono-$\gamma$ results. All in all, the combination of mono-$\gamma$ and mono-$W$ dominates the mono-X strategy sensitivity and provide best mass reach for some DM candidates.

The 10~TeV MuC results show that at this energy it is possible to exclude fully a Dirac doublet DM at the thermal mass $1.1$~TeV in both the mono-$\gamma$ and the mono-$W$ channels. This specific WIMP candidate, known as ``higgsino'', is a widely-studied target for future colliders because of several reasons, including the fact that its detection is challenging for direct detection experiments based on scattering on Xenon nuclei~\cite{Bottaro:2022one}. Another popular candidate is the Majorana triplet with $2.9$~TeV mass, known as ``wino''. The 10~TeV MuC cannot access this candidate through Mono-X. Detection is instead possible through disappearing track searches as discussed later in this section.
%in Section~\ref{secunc}.

A grand summary encompassing higher energies,  heavier thermal relic DM candidates and including the Disappearing Tracks (DT) reach is displayed in Figure~\ref{fig:summary_all_colliders}. We see that at energies above 10~TeV the muon collider starts probing  WIMP DM $n$-plets with $n>2$. 

It is worth noticing that the detection of the Mono-X processes in eq.~(\ref{eq:monoX}) is affected by large SM backgrounds that originate, for instance, from the production of invisible neutrinos in place of $\chi$. The sensitivity is thus limited by the accuracy of the SM background predictions that need to be subtracted from the observed data. The error bars in Figure~\ref{fig:summary_all_colliders} report the mass reach for a relative accuracy ranging from $0\%$ (perfect background prediction) to $1\%$. We see that $1\%$ systematic uncertainties on the background prediction significantly degrades the sensitivity. Uncertainties at the $0.1\%$ level would be desirable and are assumed in the coloured bars. The possibility to attain such accurate predictions deserves further investigation.


\subsubsection*{%\label{subsec:2-body-processes}
Indirect reach through high-energy measurements}

Pure WIMP DM $n$-plets have a mass that scales roughly as $M_{\chi}\sim n^{5/2}$, as Table~\ref{tab:thermal-masses} shows, and can reach values above the kinematical threshold for resonant production of any collider we can imagine to build in the near and medium-term future. Indeed, masses around a fraction of 100~TeV  can be achieved, e.g. the thermal mass for a Majorana 7-plet is around 50~TeV. For large $n$ it becomes questionable if the weak interactions are still so weak, indeed scattering rates involving large $n$-plets end up being as large as 10\% of the  maximum allowed by unitarity for $n=9$ and a Landau pole for weak interactions emerges at energies less than two orders of magnitude from the mass of the $n=9$ dark matter candidate~\cite{Bottaro:2021snn,Bottaro:2022one}. Therefore, for $n\geq 9$ we do not possess a valid EFT description of the minimal scenario for DM in which the WIMP multiplet is the only new particle beyond the Standard Model. The SM plus a WIMP for $n\geq 9$ can be considered, but it should be interpreted as very rough sketch of some other non-minimal theory of dark matter that features more dynamics than what it is entailed by just adding one particle to the SM.

Keeping this caveat in mind, the overall picture is clear: WIMPs can be good dark matter candidates over a large range of masses from the TeV to the PeV. A muon collider programme where the energy is progressively raised in stages can probe the resonant production of heavier and heavier candidates, but a complete coverage of the heaviest WIMPs will have to exploit off-shell effects in precision measurements of SM processes, with a mass reach that is potentially above the direct production threshold. This indirect search strategy can be effectively pursued at the muon collider through the measurement of high energy cross sections~\cite{DiLuzio:2018jwd}. These measurements benefit from a boosted sensitivity to new physics above the collider reach as explained in Section~\ref{HEM}, and they fall in the same category of those employed in Section~\ref{sec:Higgs} for EFT searches.

In what follows we consider as  concrete examples the dark matter candidates with $Y=0,1/2,1$  listed in Table~\ref{tab:thermal-masses}. We refer to~\cite{Franceschini:2022sxc} for a study encompassing a larger set of candidates. The search strategy that we adopt leverages the observable effects that DM candidates can leave due to  their propagation as virtual states, which modify the rate and the distributions of SM processes such as 
\begin{eqnarray}
\mu^{+}\mu^{-} &\to& f \bar{f}\,, \label{proc:ffbar}\\
\mu^{+}\mu^{-} &\to& Z h \,, \label{proc:Zh} \\
\mu^{+}\mu^{-} &\to& W^{+}W^{-}\,, \label{proc:WW}
\end{eqnarray}
as well as $2\to3$ processes like
\begin{eqnarray}
    \mu^{+}\mu^{-}&\to& WWh\,,
    \label{proc:WWh}\\
\mu^{+}\mu^{-} &\to& f \bar{f^\prime} W\,. \label{proc:ffprimebar}
\end{eqnarray}
Measuring the total rate of eqs.(\ref{proc:ffbar}-\ref{proc:ffprimebar}) and using differential information on the angular distribution of the channels in which the charge of the final states, e.g. $f=e,\mu$,  can be tagged reliably, it is possible to probe the existence of new matter $n$-plets.

\begin{figure*}
\begin{centering}
\includegraphics[width=0.49\linewidth]{5_Physics_Studies/figures/1.5-14_TeV_indirect_WinoHiggsino_with_CC.pdf}
\includegraphics[width=0.49\linewidth]{5_Physics_Studies/figures/1.5-14_TeV_indirect_Dirac_with_CC.pdf}
\par\end{centering}
\caption{\label{fig:Indirect-reach-3plets} Minimal luminosity to exclude a thermal pure higgsino or wino dark matter (left panel) a 2.84~TeV Dirac triplet, 4.79~TeV Dirac 4-plet, a 13.6~TeV Majorana 5-plet (right panel) as function of the collider center of mass energy~\cite{Franceschini:2022sxc}. Lighter color lines are for polarized beams. The thickness of the wino, Dirac 3-plet, Dirac 4-plet, and Majoarana 5-plet bands covers the uncertainty on the thermal mass calculations. Diagonal lines mark the precision at which the total rate of the labeled channels are going to be measured. The shaded area indicates that at least one channel is going to be measured with 0.1\% uncertainty and systematic uncertainties need to be evaluated.}
\end{figure*}

It should be noted that the $2\to3$ processes cross sections, while formally of higher order in the EW loop expansion, are not suppressed relative to the $2\to 2$ cross sections, at the high energy MuC. This is a manifestation of the EW radiation enhancement that we described in Section~\ref{EWRadiation}. The enhancement emerges in the phase-space region where a $W$ boson is emitted with low energy and collinear to one of the initial muons or to one of the two other final state particles, which are instead energetic and central. The EW radiation enhancement offers novel opportunities to search for new physics. In the case at hand, it enables the high-rate production of new hard 2-body final states (namely $Wh$ and $f\bar{f^\prime}$ for eq.~(\ref{proc:WWh}) and (\ref{proc:ffprimebar}), respectively) to be exploited for WIMP searches. However, EW radiation effects also challenge theoretical predictions as they require not yet available systematic resummation techniques, as discussed in Section~\ref{EWRadiation}.  The estimates that follow do not include resummation. More work will thus be needed to turn them into fully quantitative sensitivity projections.

In Figure~\ref{fig:Indirect-reach-3plets} (left panel) we report the minimal luminosity needed to exclude a thermal pure wino DM (i.e., a Majorana triplet with $2.9$~TeV mass) and the higgsino (a Dirac doublet of $1.1$~TeV) as a function of the collider centre of mass energy. The luminosity curves feature a minimum around the direct production threshold $E_{{\rm{cm}}}= 2 M_{\chi}$, which provides the optimal energy for detection. For smaller $E_{{\rm{cm}}}$, more luminosity is needed as the effect of virtual $n$-plets decreases as $E_{{\rm{cm}}}^{2}/M_{\chi}^2$ below the production threshold. A larger luminosity is also needed moving above the threshold because the loop function (see~\cite{DiLuzio:2018jwd}) that describes the virtual DM exchange in the di-fermion final state happens to cross zero for some value of $E_{{\rm{cm}}}$ above $2\, M_{\chi}$. After crossing this second threshold, the required luminosity smoothly decreases with $E_{\rm{cm}}$ as the figure shows. A similar behaviour is observed for the other candidates considered in the right panel of Figure~\ref{fig:Indirect-reach-3plets}.

These studies are helped by the presence of left-handed fermions initial states, which source larger weak-boson mediated scattering. Therefore it is interesting to study the effect of beam polarization. In Figure~\ref{fig:Indirect-reach-3plets} the lighter colored lines give the necessary luminosity for an exclusion at a machine capable of 30\% left-handed polarization on the $\mu^{-}$ beam and -30\% for the $\mu^{+}$ beam. Even this modest polarization of the beams can reduce significantly the luminosity required for the exclusion.

When precision studies are involved it is important to keep in sight a possible bottleneck from systematic uncertainties. The origin of systematic uncertainties is difficult to assess at this stage, as there is not yet a fully developed experiment design. We identify the $0.1\%$ level as a possible reasonable level at which systematic uncertainties will need to be discussed. With this reference in mind we draw the shaded area of Figure~\ref{fig:Indirect-reach-3plets}, which indicates that the search for new electroweak matter is based on a high enough luminosity to have a statistical uncertainties of $0.1\%$ in the $q\bar{q}$ channel. As other channels are expected to be cleaner, and less statistically abundant, we reckon that in the shaded region of the plane a more careful evaluation of possible systematics needs to be performed before claiming a sensitivity. Along the baseline energy-luminosity line of eq.~(\ref{lums}), the required statistical precision is at the $1\%$ level, as the figure shows. The needs for accurate experimental measurements and theoretical predictions is thus reduced accordingly.

The left panel of Figure~\ref{fig:Indirect-reach-3plets} shows that the 10~TeV MuC with the baseline luminosity will probe the higgsino through indirect effects, on top of accessing it in Mono-X as we saw in the previous section. The wino is instead out of the indirect reach, but visible in mono-X. Interestingly, the 3~TeV MuC could access the higgsino with a luminosity slightly above $2~{\rm{ab}}^{-1}$, or by slightly lowering its centre of mass energy. The 3~TeV MuC seems instead unable to detect the higgsino with the Mono-X search strategy, according to the findings described in the previous section. 

The right panel of Figure~\ref{fig:Indirect-reach-3plets} displays the MuC sensitivity to higher-$n$ WIMP candidate multiplets. These are typically heavier than the higgsino and wino, but still they are easier to access indirectly because of the enhancement of the loop effects of large-$n$ electroweak multiplet. The 10~TeV MuC is thus sensitive to, for instance, a Dirac 4-plet with $4.8$~TeV thermal mass, very close to the kinematical threshold for resonant pair production. All fermionic WIMP candidates with $n=2$, 3 and 4 are also accessible with the baseline luminosity. Scalars in the same multiplets give dimmer signals and are hard to see unless higher luminosities are considered. A more promising way to observe the light scalars is the Mono-X search. Looking beyond 10~TeV, we see from  Figure~\ref{fig:Indirect-reach-3plets} that a collider in the 14~TeV ballpark can probe off-shell $n=5$ Majorana fermion dark matter with 14~TeV mass, way above the direct production threshold of 7~TeV. From Figure~\ref{fig:summary_all_colliders} we also see that a 14~TeV machine would be sensitive to  on-shell scalar $n=3$ $Y=1$ and to the scalar $n=4$ dark matter candidates. Looking at further high energies, results from Ref.~\cite{Franceschini:2022sxc} show that more candidates can be tested off-shell, e.g. the $n=7$ Majorana candidate, weighing about 50~TeV,  can be probed at a 30~TeV collider. Further dark matter candidate can be tested on-shell at larger $E_{{\rm{cm}}}$ as shown in Figure~\ref{fig:summary_all_colliders}. 

All in all the muon collider has a great potential to probe the idea of WIMP DM with several candidates that can be tested, some of which with multiple search strategies, strengthening the perspectives to establish a potential discovery. It should be noted however that both the strategies discussed so far would provide rather ``indirect'' signals. The Mono-X search is often considered as ``direct'' because the new particles are resonantly produced. However, neither the produced particles nor their decay products are detected. The signal would emerge as a small departure from the large SM background prediction in certain kinematical distributions, just as it would happen for an ``Indirect'' discovery based on loop effects. This is an additional motivation to investigate a truly direct detection strategy based on disappearing tracks, which is the subject of the following section. Another motivation is that disappearing tracks  searches display a better mass reach than Mono-X for several multiplets, as Figure~\ref{fig:summary_all_colliders} shows.

\subsubsection*{Unconventional signatures}%\label{secunc}

Disappearing tracks is one of the strategies to search for Long-Lived Particles (LLPs), which is among the priorities of the particle physics community~\cite{Curtin:2018mvb, Alimena:2019zri}. LLPs appear in a variety of new physics scenarios and yield a large range of signatures at colliders. Depending on the LLP quantum numbers and lifetime, these can span from LLP decay products appearing in the detector volume, even outside of the beam crossings, to metastable particles with anomalous ionisation disappearing after a short distance.

This wide range of ``unconventional'' experimental signatures is  intertwined with the development of detector technologies and their study can guide the design of the final detector layout. For example, the development of timing-sensitive detectors is crucial both to suppress the abundant beam-induced backgrounds and to detect the presence of heavy, slow-moving, particles that are travelling through the detector. A lively R\&D programme is ongoing to develop the reconstruction algorithms that will profit from these new technologies.

For heavy particles, whose production cross sections are dominated by the annihilation $s$-channel, there are two main features that make searches for unconventional signatures particularly competitive at a muon collider when compared to other future proposed machines like the FCC-hh. The produced particles tend to be more centrally distributed, impinging on the regions of the detector where reconstruction is comparatively easier, and furthermore they have a more sharply peaked Lorentz boost distribution, which can lead to effectively larger average observed lifetimes for the produced BSM states.

Searches for LLPs that decay within the volume of the tracking detectors (e.g. decay lengths between 1~mm and 500~mm) are particularly interesting as they directly probe the lifetime range motivated by compelling dark matter models. A rather detailed analysis including realistic simulation of the BIB from muon decays was performed in~\cite{Capdevilla:2021fmj} targeting higgsino and wino WIMPs, and it is summarised below. See~\cite{Han:2020uak} for simplified sensitivity estimates covering other candidates.

\paragraph{Search for disappearing tracks}{\ } \\ 
\noindent
The pure higgsino consists of a Dirac doublet with hypercharge $1/2$, with a thermal relic mass of 1.1~TeV. Due to loop radiative corrections, the charged state {\ensuremath{\tilde{\chi}^{\pm}}\xspace}\ splits from the neutral one {\ensuremath{\tilde{\chi}^{0}_{1}}\xspace}\ by 344~MeV, giving rise to a mean proper decay length of 6.6~mm for the charged state~\cite{Hisano:2006nn}. The {\ensuremath{\tilde{\chi}^{\pm}}\xspace}\ can thus travel a macroscopic distance before decaying into an invisible {\ensuremath{\tilde{\chi}^{0}_{1}}\xspace}\ and other low-energy Standard Model particles.

Searches at the LHC are actively targeting this scenario~\cite{ATLAS:2022rme,ATLAS:2017oal,ATLAS:2013ikw,CMS:2020atg,CMS:2014gxa}, but are not expected to cover the relic favoured mass~\cite{Saito:2019rtg, EuropeanStrategyforParticlePhysicsPreparatoryGroup:2019qin}. A muon collider operating at multi-TeV centre-of-mass energies could provide a perfect tool to look for these particles.

The production of {\ensuremath{\tilde{\chi}^{\pm}}\xspace}\ pairs  at a MuC proceeds via an s-channel photon or Z-boson, with other processes, such as vector boson fusion, being subdominant. The prospects to observe the disappearing track signal of {\ensuremath{\tilde{\chi}^{\pm}}\xspace}\ were investigated in detail in Ref.~\cite{Capdevilla:2021fmj} exploiting a detector simulation based on {\textsc{Geant}\xspace} 4~\cite{Agostinelli:2002hh} for the modeling of the response of the tracking detectors, which are crucial in the estimation of the backgrounds. The simulated events were overlaid with beam-induced background events simulated with MARS15~\cite{Mokhov:2017klc}.

The analysis strategy relies on requiring one ({SR\ensuremath{^{\gamma}_{1t}}\xspace}) or two ({SR\ensuremath{^{\gamma}_{2t}}\xspace}) disappearing tracks in each event in addition to a 25~GeV ISR photon. Additional requirements are imposed on the transverse momentum and angular direction of the reconstructed tracklet and on the distance between the two tracklets along the beam axis in the case of events with two candidates. The expected backgrounds are extracted from the full detector simulation and the results are presented assuming a 30\% (100\%) systematic uncertainty on the total background yields for the single (double) tracklet selections.
The corresponding discovery prospects and 95\% CL exclusion reach are shown in Figure~\ref{fig:higgsinoStubTracksReachMass} for each of the two selection strategies discussed above, considering pure-higgsino production cross sections and 10~TeV $\mu^{+}\mu^{-}$ collisions. The expected limits at 95\% CL at the 3~TeV MuC  are also overlaid for comparison.

Both event selections are expected to cover a wide range of higgsino masses and lifetimes, well in excess of current and expected collider limits. In the most favourable scenarios, the analysis of 10~ab$^{-1}$ of 10~TeV muon collisions is expected to allow the discovery {\ensuremath{\tilde{\chi}^{\pm}}\xspace} masses up to a value close to the kinematic limit of 5~TeV. The interval of lifetimes covered by the experimental search directly depends on the layout of the tracking detector, i.e. the radial position of the tracking layers, and the choices made in the reconstruction and identification of the tracklets, i.e. the minimum number of measured space-points. Considering the current detector design~\cite{CLICdp:2017vju,CLICdp:2018vnx,ILDConceptGroup:2020sfq,ILC:2007vrf}, the 10~TeV MuC is expected to allow to discover the higgsino thermal target, though only by a narrow margin. 

\begin{figure}[t]
\centering
\includegraphics[width=0.5\textwidth]{figures/hino_m_vs_lifetime}
\caption{Expected sensitivity~\cite{Capdevilla:2021fmj} to the higgsino, in the plane formed by the {\ensuremath{\tilde{\chi}^{\pm}}\xspace} mass and lifetime. The lifetime that corresponds to the thermal mass of 1.1~TeV, and to a mass-splitting of 344~MeV as in the pure-higgsino scenario, is reported as an horizontal dash-dotted line.}
\label{fig:higgsinoStubTracksReachMass}
\end{figure}

\begin{figure}[t]
\centering
\includegraphics[width=0.5\textwidth]{figures/wino_m_vs_lifetime}
\caption{Expected sensitivity~\cite{Capdevilla:2021fmj} to the wino, in the plane formed by the {\ensuremath{\tilde{\chi}^{\pm}}\xspace} mass and lifetime. The lifetime that corresponds to the thermal mass of 2.86~TeV, and to a mass-splitting of 166~MeV as in the pure-wino scenario, is reported as an horizontal dash-dotted line.}
\label{fig:winoStubTracksReachMass}
\end{figure}

An alternative tracking detector design, hard to realise in the presence of the BIB, with tracking layers significantly closer to the beam line would be needed to significantly boost the detection of such a signal. Other unconventional signatures, such as soft displaced tracks~\cite{Fukuda:2019kbp} detected in combination with an energetic ISR photon or kinked tracks should be investigated and have the potential to strongly enhance the sensitivity.  

Figure~\ref{fig:winoStubTracksReachMass} shows the expected sensitivity when considering a pure-wino scenario. The much longer predicted lifetime of the charged state significantly increases the likelihood of detecting at least one disappearing track, dramatically extending the reach.

In summary, the pure higgsino with thermal mass can be probed at the 5-$\sigma$ level by a 10~TeV MuC. Pure wino dark matter scenarios are well within the reach of a 10~TeV MuC and could be also probed at lower centre of mass energies.

\subsection{Muon-specific opportunities}
\label{muonspec}

We conclude our survey by reviewing a number a studies targeting new physics that preferentially couples to muons, entailing an inherent advantage of muon colliders over other facilities. As discussed in Section~\ref{muspec}, this scenario is, in the first place, a logical possibility that muon collisions will enable to probe for the first time systematically and effectively. Furthermore, it is motivated by the stronger coupling of second-generation particles to the Higgs, which typically results into a stronger coupling to new physics related with the breaking of the EW symmetry. New physics explaining the structure of leptonic flavour might also be probed more effectively with muons than with electrons.

%In Section~\ref{sec:g-2} we review 
We start by reviewing the work done in connection with the anomalous magnetic moment of the muon and its tension with the Standard Model (SM) prediction. Possible new physics explaining the anomaly definitely couples to muons and it preferentially couples to muons more strongly than to other particles like electrons because of existing constraints. The MuC thus is found to be a prime option for future  investigations of the possible new physics origin of the anomaly, if it will survive further scrutiny. Otherwise, the ones we present
%presented in Section~\ref{sec:g-2} 
will still define possible scenarios for new physics that the MuC can explore. Similar considerations hold for $B$-meson anomalies, which have been also studied extensively in connection with muon colliders~\cite{Huang:2021biu,Asadi:2021gah,Azatov:2022itm,Bause:2021prv,Qian:2021ihf,Allanach:2022blr,Bandyopadhyay:2021pld}. Since they do not incorporate the very recent LHCb results~\cite{LHCb:2022qnv,LHCb:2022zom}, these studies will not be reviewed here.

%Section~\ref{sec:LFV} discusses 
Next, we discuss the MuC opportunities to probe lepton flavour violation. High-energy muon collisions are found to be competitive and complementary with planned low-energy experiments, already at 3~TeV, entailing also opportunities to confirm and further investigate possible indirect evidences of new physics that might emerge from these very precise low-energy measurements. 

%Section~\ref{sec:muon-higgs} is devoted to 
We will finally outline the opportunities to explore the Higgs sector by exploiting the relatively large muon Yukawa coupling. In particular, we consider the possibility of detecting and studying heavy Higgs bosons through the radiative return process. 

\subsubsection*{The muon anomalous magnetic moment}
%\label{sec:g-2}}
%%%%%%%%%%%%%%%%%%%%%%%%%%%%%%%%%%%%%%%%%
%%%%%%%%%%%%%%%%%%%%%%%%%%%%%%%%%%%%%%%%%

The anomalous magnetic moment of the muon has provided, over the last ten years, an enduring hint for new physics. The experimental value of $a_\mu \!=\! (g_\mu \!-\! 2)/2$ from the E821 experiment at the Brookhaven National Lab~\cite{Muong-2:2006rrc} was recently confirmed by the E989 experiment at Fermilab~\cite{Muong-2:2021vma,Muong-2:2021ojo}, yielding the experimental average $a_\mu^{\scriptscriptstyle \rm EXP} \!=\! 116592061(41) \!\times\! 10^{-11}$. The comparison of this value with the SM prediction $a_\mu^{\scriptscriptstyle \rm SM} \!=\! 116591810(43) \times 10^{-11}$~\cite{Aoyama:2012wk,Gnendiger:2013pva,Keshavarzi:2018mgv,Davier:2019can,Kurz:2014wya,Melnikov:2003xd,Bijnens:2019ghy,Pauk:2014rta,Danilkin:2016hnh,Roig:2019reh,Colangelo:2014qya} 
shows a $4.2\,\sigma$ discrepancy
%
\begin{equation}
\Delta a_\mu = a_\mu^{\scriptscriptstyle \rm EXP}-a_\mu^{\scriptscriptstyle \rm SM} = 251 \, (59) \times 10^{-11}\,.
\label{eq:gmu}
\end{equation}
%
In the following, we refer to this as the $g$-2 anomaly. Recent lattice determinations of the hadronic vacuum polarization give a SM prediction that is more in agreement with the experimental result, but is in tension with the previous calculations based on dispersive methods~\cite{Borsanyi:2020mff}. Current and forthcoming plans to confirm the BSM origin of this anomaly include reducing the experimental uncertainty by a factor of four at E989, comparisons between phenomenological and Lattice determinations of the hadronic vacuum polarization contribution to $g$-2~\cite{Gerardin:2019vio,Chao:2021tvp,FermilabLattice:2017wgj,Budapest-Marseille-Wuppertal:2017okr,RBC:2018dos,Giusti:2019xct,Shintani:2019wai,FermilabLattice:2019ugu,Gerardin:2019rua,Giusti:2019hkz,Borsanyi:2020mff}, and new experiments aiming to probe the same physics~\cite{Sato:2017sdn,Abbiendi:2016xup}. 
If all of these efforts will confirm the presence of new physics, then the most urgent task at hand will be to probe this anomaly at higher energies, ultimately in order to discover and study the new BSM particles that give rise to the additional $\Delta a_\mu$ contributions. The $\mc$ is a uniquely well-suited machine for this endeavour, not least since it collides the actual particles displaying the anomaly, and hence the only particles guaranteed to couple to the new physics.

There are several ways in which a $\mc$ can provide a powerful high-energy test of the muon $g$-2 and discover the new physics responsible for the anomaly:
\begin{itemize}
\item If the physics responsible for $\Delta a_\mu$ is heavy enough, an Effective Field Theory (EFT) description holds up to the high $\mc$ energies. This was studied in \cite{Buttazzo:2020ibd}. In this case, scattering cross sections induced by the new physics effective operators grow at high energies (analogously to what discussed in Sections \ref{HEM}, \ref{sec:Higgs} and \ref{secdm}), so that a measurement with modest  precision at a sufficiently high energy will be sufficient to disentangle new physics effects from the SM background. These considerations are  independent from the specific underlying model.
\item In most motivated models of new physics, new particles responsible for $\Delta a_\mu$ are light enough to be directly produced in $\mu^+\mu^-$ collisions at attainable $\mc$ energies. Understanding such opportunities for direct production and discovery at a $\mc$ was studied in~\cite{Capdevilla:2020qel,Capdevilla:2021rwo}. It was found that a complete classification of perturbative BSM models that can give rise to the observed value of $\Delta a_\mu$, and of their experimental signatures, is possible. 
\item Additional effects in muon couplings to SM gauge and Higgs bosons, correlated with the muon $g$-2, can also be present at a level that can be probed by precision measurements at a $\mc$. Some of these effects can be predicted in a model-independent way, others arise in specific, motivated models.
\end{itemize}

These three strategies together make it possible to formulate a {\it no-lose theorem} for a high-energy $\mc$~\cite{Capdevilla:2020qel,Capdevilla:2021rwo}, if the experimental anomaly in the muon $g$-2 is really due to new physics.
%
The physics case of a high-energy determination of $\Delta a_\mu$, which is unique of a $\mc$, thus represents a striking example of the complementarity and interplay of the high-energy 
and high-intensity frontiers of particle physics, and it highlights the far reaching potential of a $\mc$ to probe new physics.


%%%%%%%%%%%%%%%%%%%%%%%%%%%%%%%%%%%%%%%%%%%%%%%%
%%%%%%%%%%%%%%%%%%%%%%%%%%%%%%%%%%%%%%%%%%%%%%%%
\paragraph{High-energy probes of the operators generating the \texorpdfstring{$g$}{g}-2}{\ }
\noindent
%%%%%%%%%%%%%%%%%%%%%%%%%%%%%%%%%%%%%%%%%%%%%%%%
%%%%%%%%%%%%%%%%%%%%%%%%%%%%%%%%%%%%%%%%%%%%%%%%

We start by reviewing the analysis of~\cite{Buttazzo:2020ibd}, which determined that precision measurements at high-energy $\mc$s can detect deviations in scattering rates that are generated by the same effective operators giving rise to the $g$-2 anomaly. Hence, they provide a powerful independent verification and detailed examination of the anomaly even if the responsible BSM degrees of freedom are too heavy to be produced on-shell at the collider. 
This would be a direct determination of the new physics contribution, not affected by the hadronic uncertainties that enter the SM prediction of $a_\mu$.


%%%%%%%%%%
\begin{figure*}
\centering%
\includegraphics[width=0.55\textwidth]{figures/feyn}
\caption{{\it Upper row:} Feynman diagrams contributing to the leptonic $g$-2 up to one-loop order in the Standard Model EFT.
{\it Lower row:} Feynman diagrams of the corresponding high-energy scattering processes.
Dimension-6 effective interaction vertices are denoted by a square.}
\label{fig:feyn}
\end{figure*}
%%%%%%%%%%

New interactions emerging at a scale $\Lambda$ larger than the EW scale can be described at energies $E \ll \Lambda$ 
by an effective Lagrangian containing non-renormalizable $SU(3)_c \otimes SU(2)_L \otimes U(1)_Y$ invariant operators.
The relevant effective Lagrangian contributing to $g$-2
%up to one-loop order,
reads~\cite{Buchmuller:1985jz}
\begin{eqnarray}
\mathcal{L} = 
&&\frac{C^\ell_{eB}}{\Lambda^2}
\left( \bar\ell_L \sigma^{\mu\nu}\ell_{R}\right) \! H B_{\mu\nu} + 
\frac{C^\ell_{eW}}{\Lambda^2}
\left( \bar\ell_L \sigma^{\mu\nu} \ell_{R} \right) \! \tau^I \! H W_{\mu\nu}^I  \nonumber \\ &&+ \frac{C^\ell_{T}}{\Lambda^2}( \overline{\ell}_L\sigma_{\mu\nu}\ell_{R}) (\overline{Q}_L\sigma^{\mu\nu} u_{R}) 
+ h.c..
\label{eq:L_SMEFT}
\end{eqnarray} 
%   
It includes not only the interactions that generate the dipole operator at tree level, but also four-fermion operators that generate the dipole at one loop. The Feynman diagrams relevant for $g$-2 are displayed in Figure~\ref{fig:feyn}, top row. After EW symmetry breaking, $H$ is replaced by its vacuum expectation value $v$, and one obtains the prediction 
\begin{eqnarray}
\Delta a_\ell  \simeq &&\frac{4m_\ell v}{e\Lambda^2} \, 
\bigg(
C^\ell_{e\gamma} - \frac{3\alpha}{2\pi} \frac{c^2_W \!-\! s^2_W}{s_W c_W} \,C^\ell_{eZ} \log\frac{\Lambda}{m_Z}
\bigg) \nonumber \\
&&- \sum_{q=c,t} \frac{4m_\ell m_q}{\pi^2} \frac{C_T^{\ell q}}{\Lambda^2}\,
\log\frac{\Lambda}{m_q},
\label{eq:Delta_a_ell}
\end{eqnarray}
%
where $s_W$ and $c_W$ are the sine and cosine of the weak mixing angle, $C_{e\gamma}=c_W C_{eB} - s_W C_{eW}$ and
$C_{eZ} = -s_W C_{eB} - c_W C_{eW}$.
Additional radiative contributions from the three operators $H^\dag H W_{\mu\nu}^IW^{I\mu\nu}$, $H^\dag H B_{\mu\nu}B^{\mu\nu}$ and $H^\dag \tau^I H W_{\mu\nu}^I B^{\mu\nu}$ can be neglected because they are suppressed by the small lepton Yukawa couplings.
For simplicity, $C_{eB}$, $C_{eW}$ and $C_{T}$ are assumed to be real. The one-loop renormalization effects to $C^\ell_{e\gamma}$ 
%
\begin{align}
\!\!\! C^\ell_{e\gamma}(m_\ell) \simeq C^\ell_{e\gamma}\!(\Lambda)\left(1 \!-\! \frac{3y^2_t}{16\pi^2} \log\frac{\Lambda}{m_t}
\!-\! \frac{4 \alpha}{\pi} \log\frac{m_t}{m_\ell}\right)
\label{eq:running_Cegamma}
\end{align}
%
can be straightforwardly included. Numerically~\cite{Buttazzo:2020ibd}
%
\begin{eqnarray}
\frac{\Delta a_\mu}{3 \!\times\! 10^{-9}} \approx  &&\left( \frac{250 \, {\rm TeV}}{\Lambda} \right)^{2} \times\nonumber\\
&&\left(C^\mu_{e\gamma} - 0.2 C^{\mu t}_T - 0.001 C^{\mu c}_T - 0.05 C^{\mu}_{eZ}\right). 
\nonumber
\end{eqnarray}
%


\noindent A few comments are in order: 
%
\begin{itemize}
\item The $\Delta a_\mu$ discrepancy can be solved for a new physics scale up to $\Lambda\approx 250~$TeV.
This requires a strongly coupled new physics sector where $C^\mu_{e\gamma}$ and/or $C^{\mu t}_T \sim g^2_{\rm\scriptscriptstyle NP}/16\pi^2 \sim 1$ 
and a chiral enhancement $v/m_\mu$ compared with the weak SM contribution.
Directly producing new particles at such high scales is far beyond the capabilities of any foreseen collider.
Nevertheless, this new physics can be tested at a $\mc$ through high-energy processes such as $\mu^+\!\mu^- \!\to h\gamma$ or $\mu^+\!\mu^- \!\to q\bar{q}$ (with $q=c,t$), that are affected by the very same operators that generate $\Delta a_\mu$.
%
\item If the underlying new physics sector is weakly coupled, $g_{\rm\scriptscriptstyle NP}\lesssim 1$, then
$C^\mu_{e\gamma}$ and $C^{\mu t}_T \lesssim 1/16\pi^2$, implying $\Lambda\lesssim 20~$TeV to solve the $g$-2 anomaly.
In this case, a $\mc$ could still be able to directly produce new physics particles~\cite{Capdevilla:2020qel}. 
Even so, the study of the processes $\mu^+\mu^-\to h\gamma$
and $\mu^+\mu^-\to q\bar{q}$ could be crucial to reconstruct the effective dipole vertex $\mu^+\mu^-\gamma$, as has been explicitly shown in~\cite{Paradisi:2022vqp}.
%
\item If the new physics sector is weakly coupled, and further $\Delta a_\mu$ scales with lepton masses as the SM weak contribution,
then $\Delta a_\mu \sim m^2_\mu/16\pi^2\Lambda^2$.
Here, the experimental value of $\Delta a_\mu$ can be accommodated only provided that $\Lambda \lesssim 1~$TeV.
For such a low new physics scale the EFT description breaks down at the typical multi-TeV $\mc$ energies, and new resonances cannot escape from direct production.
\end{itemize}

\paragraph*{Dipole operator.}\;The main contribution to $\Delta a_\mu$ comes from the dipole operator $O_{e\gamma}=\left(\bar\ell_L \sigma_{\mu\nu} e_R\right) H F^{\mu\nu}$. The same operator also induces a contribution to the process $\mu^+\mu^- \to h \gamma$ 
that grows with energy, and thus can become dominant over the SM cross section at a very high energy collider. 
Neglecting all masses, the
%
total $\mu^+\mu^- \!\to  h\gamma$ cross section is
%
\begin{align}
\sigma_{h\gamma} \!=\! \frac{s}{48\pi}\frac{|C^{\mu}_{e\gamma}|^2}{\Lambda^4}\!\!
\approx\!  0.7 {\rm ab} \left(\frac{\sqrt{s}}{30\, {\rm TeV}}\right)^{\!\!2} \!\! \left(\frac{\Delta a_\mu}{3 \times 10^{-9}}, \right)^{\!2}
\label{xsec_HA}
\end{align}
%
where in the last equation no contribution to $\Delta a_\mu$ other than the one from $C^{\mu}_{e\gamma}$ was assumed, and running effects for $C^{\mu}_{e\gamma}$, see eq.~(\ref{eq:running_Cegamma}), from a scale $\Lambda \approx 100$~TeV have been included. Notice that there is an identical contribution also to the process $\mu^+\mu^- \!\to\! Z\gamma$ since $H$ contains the longitudinal polarisations of the $Z$. Given the scaling with energy of the baseline luminosity~(\ref{lums}), one gets about 60 total $h\gamma$ events at $\sqrt{s}=30~$TeV. As it is discussed below, this is a signal that the $\mc$ is sensitive to.

The SM irreducible $\mu^+\mu^- \to h\gamma$ background is small, $\sigma_{h\gamma}^{\rm SM} \approx 2 \times 10^{-2}\,{\rm ab}\,\big({30\,{\rm TeV}}/{\sqrt{s}}\big)^{2}$,
with the dominant contribution arising at one-loop~\cite{Abbasabadi:1995rc} due to the muon Yukawa coupling suppression of the tree-level diagrams.
%
The main source of background comes from $Z\gamma$ events, where the $Z$ boson is incorrectly reconstructed as a Higgs. 
This cross section is large, due to the contribution from transverse polarisations.
%
There are two ways to isolate the $h\gamma$ signal from the background: by means of the different angular distributions of the two processes ---the SM $Z\gamma$ peaks in 
the forward region, while the signal is central---and by accurately distinguishing $h$ and $Z$ bosons from their decay products, e.g. by precisely reconstructing their invariant mass.
To estimate the reach on $\Delta a_\mu$ a cut-and-count experiment was considered in the $b\bar b$ final state, which has the highest signal yield. The significance of the signal is maximised in the central region $|\!\cos\theta| \lesssim 0.6$. At 30 TeV one gets
%
\begin{align}
\sigma_{h\gamma}^{\rm cut} &\approx 0.53\, {\rm ab} \,\bigg(\frac{\Delta a_\mu}{3 \times 10^{-9}} \bigg)^{\!2}, & ~~\sigma_{Z\gamma}^{\rm cut} &\approx 82\,{\rm ab}\,.
\end{align}
%
Requiring at least one jet to be tagged as a $b$, and assuming a $b$-tagging efficiency $\epsilon_b = 80\%$, one finds that a value $\Delta a_\mu = 3\!\times\! 10^{-9}$ can be tested at 95\% C.L.\ at a 30 TeV collider if the probability of reconstructing a $Z$ boson as a Higgs is less than 10\%. The resulting number of signal events is $N_S = 22$, and $N_S/N_B = 0.25$.
Figure~\ref{fig:hgamma} shows as a black line the 95\% C.L.\ reach from $\mu^+\mu^-\to h\gamma$ 
on the anomalous magnetic moment as a function of the collider energy. Note that since the number of signal events scales as the fourth power of the center-of-mass energy, only a collider with $\sqrt{s} \gtrsim 30$~TeV will have the sensitivity to test the $g$-2 anomaly in this channel.

%%%%%%%%%%
\begin{figure*}
\centering%
\includegraphics[width=0.6\textwidth]{5_Physics_Studies/figures/reach_gm2_tan.pdf}
\caption{Reach on the muon anomalous magnetic moment $\Delta a_\mu$ and muon EDM $d_\mu$, 
as a function of the $\mc$ collider center-of-mass energy $\sqrt{s}$, from the labeled processes. Figure taken from~\cite{Buttazzo:2020ibd}.}
\label{fig:hgamma}
\end{figure*}
%%%%%%%%%%

\paragraph*{Semi-leptonic interactions.}\;
If the anomalous magnetic moment arises at one loop from one of the other operators in \eqref{eq:Delta_a_ell}, their Wilson coefficients must be larger to reproduce the observed signal, and the new physics will be easier to test at a $\mc$.
We now derive the constraints on the semi-leptonic operators. The operator $O_T^{\mu t}$ that enters 
$\Delta a_\mu$ at one loop can be probed by $\mu^+\mu^-\to t\bar t$ (Figure~\ref{fig:feyn}). Its contribution 
to the cross section is
%
\begin{align}
\label{eq:mumu_tt}
\!\sigma_{t\bar{t}} =\! \frac{s}{6\pi} \frac{|C^{\mu t}_{T}|^2}{\Lambda^4} N_c 
\approx 
58 {\rm ab} \left(\frac{\sqrt{s}}{10 \, {\rm TeV}}\right)^{\!\!2} \!\!\! \left(\frac{\Delta a_\mu}{3 \times 10^{-9}} \right)^{\!2}
\end{align}
%
where the last equality assumes $\Lambda \approx 100~$TeV and uses $|\Delta a_\mu| \approx 3 \times 10^{-9} \left(100 \,{\rm TeV}/\Lambda\right)^2 |C^{\mu t}_T|$.
We estimate the reach on $\Delta a_\mu$ assuming an overall 50\% efficiency for reconstructing the top quarks, 
and requiring a statistically significant deviation from the SM $\mu\mu\to t\bar{t}$ background, with cross section
%
$\sigma_{t\bar{t}}^{\rm SM} \approx 1.7 \,{\rm fb}\,\big({10\,{\rm TeV}}/{\sqrt{s}}\big)^2$.
%

A similar analysis can be performed for semi-leptonic operator involving charm quarks. If the contribution from the charm loop dominates, we can probe $|\Delta a_\mu| \approx 3 \times 10^{-9}(10\,{\rm TeV}/\Lambda)^2 |C^{\mu c}_T|$ through the process $\mu\mu \to \bar{c} c$.
In this case, unitarity constraints on the new physics coupling $C_T^{\mu c}$ require a much lower new physics scale $\Lambda \lesssim 10$~TeV, 
so that our effective theory analysis will only hold for lower centre of mass energies.
Combining eq.~(\ref{eq:Delta_a_ell}) and (\ref{eq:mumu_tt}), with $c \leftrightarrow t$, we find
%
\begin{align}
\sigma_{c\bar{c}}
\, \approx \, 100 \,{\rm fb} \left(\frac{\sqrt{s}}{3 \, {\rm TeV}}\right)^{\!2} \!\! \left(\frac{\Delta a_\mu}{3 \times 10^{-9}} \right)^{\!2}.
\label{eq:mumu_cc_numerics}
\end{align}
%
The SM cross section for $\mu^+\mu^- \!\to c\bar{c}$ %is also given by eq.~\eqref{eq:DY} and
at $\sqrt{s}= 3~$TeV is $\sim 19$~fb. In Figure~\ref{fig:hgamma} we show the 95\% C.L.\ constraints on the top and charm contributions to $\Delta a_\mu$ as red and orange lines, respectively, as functions of the collider energy. Notice that the charm contribution can be probed already at $\sqrt{s} = 1$~TeV, while the top contribution can be probed at $\sqrt{s} = 10$~TeV. 

\paragraph*{Electric dipole moments.}\;
So far, CP conservation has been assumed. If however the coefficients $C_{e\gamma}$, $C_{eZ}$ or $C_T$ are complex, an electric dipole moment (EDM) $d_\mu$ is unavoidably generated for the muon. Since the cross sections in eq.~\eqref{xsec_HA} and \eqref{eq:mumu_tt} are proportional to the absolute values of the same coefficients, a $\mc$ offers a unique opportunity to test also $d_\mu$. The current experimental limit $d_\mu < 1.9 \times 10^{-19}\,e\,$cm was set by the BNL E821 experiment~\cite{Muong-2:2008ebm} 
and the new E989 experiment at Fermilab aims to decrease this by two orders of magnitude~\cite{Chislett:2016jau}. 
Similar sensitivities could be reached also by the J-PARC $g$-2 experiment~\cite{Gorringe:2015cma}.

From the model-independent relation~\cite{Giudice:2012ms}
%
\begin{align}
\frac{d_\mu}{\tan\phi_\mu} =  \frac{\Delta a_\mu}{2 m_\mu} \,e \,\simeq\, 3 \times 10^{-22} \left(\frac{\Delta a_\mu}{3\!\times\! 10^{-9}}\right) e\, {\rm cm}\,,
\label{eq:d_mu}
\end{align}
%
where $\phi_\mu$ is the argument of the dipole amplitude, the bounds on $\Delta a_\mu$ in Figure~\ref{fig:hgamma} can be translated into a nearly model-independent constraint on $d_\mu$ by assuming ${\tan\phi_\mu}\approx1$. We find that a 10~TeV $\mc$ can reach a sensitivity comparable to the ones expected at Fermilab~\cite{Chislett:2016jau} and J-PARC~\cite{Gorringe:2015cma}, while at a 30 TeV collider one gets the bound $d_\mu \lesssim 3\times 10^{-22}\, e$~cm.


\begin{figure*}[t]
\center
%\vspace*{-10mm}
\hspace{-0.75cm}
\includegraphics[width=0.5\textwidth]{figures/Singlets_Vector_Visible.pdf} ~
\includegraphics[width=0.5\textwidth]{figures/Singlets_Scalar_Visible.pdf}  \\ 
\hspace{-0.75cm}
\includegraphics[width=0.5\textwidth]{figures/Singlets_Vector_Invisible.pdf} ~
\includegraphics[width=0.5\textwidth]{figures/Singlets_Scalar_Invisible.pdf} 
\caption{
Singlet models for $g$-2 and their probes at different masses, assuming 100\% branching ratio to di-muons (top) and the minimum branching ratio to di-muon allowed by perturbative unitarity. \cite{Capdevilla:2021kcf}.
 \label{babar_singlet}}
\end{figure*}


%%%%%%%%%%%%%%%%%%%%%%%%%%%%%%%%%%%%%%%%%%%%%%%%
%%%%%%%%%%%%%%%%%%%%%%%%%%%%%%%%%%%%%%%%%%%%%%%%
\paragraph{Direct searches for BSM particles generating the \texorpdfstring{$g$}{g}-2}{\ } \\ 
\noindent
%%%%%%%%%%%%%%%%%%%%%%%%%%%%%%%%%%%%%%%%%%%%%%%%
%%%%%%%%%%%%%%%%%%%%%%%%%%%%%%%%%%%%%%%%%%%%%%%%

We now review the model-exhaustive analyses conducted in~\cite{Capdevilla:2020qel,Capdevilla:2021rwo} and \cite{Capdevilla:2021kcf}, examining all possible perturbative BSM solutions to the $g$-2 anomaly to understand the associated direct production signatures of new states at future $\mc$s, and we summarise the related no-lose theorem. This model-exhaustive analysis first finds the highest possible mass scale of new physics subject only to perturbative unitarity, and optionally the requirements of minimum flavour violation and/or naturalness. It is assumed that one-loop effects involving BSM states are responsible for the anomaly. Scenarios where new contributions only appear at higher loop order require a lower BSM mass scale to generate the required new contribution.
All possible one-loop BSM contributions to $\Delta a_\mu$ can be organised into two classes: {\emph{Singlet Scenarios}}, in which the BSM $g$-2 contribution only involves a muon and a new SM singlet boson that couples to the muon, and {\emph{electroweak (EW) Scenarios}}, in which new states with EW quantum numbers contribute to $g$-2. 

%%%%%%%%%%%%%%%%%%%%%%%%%%%%%%
\paragraph*{Singlet mediators.}
%%%%%%%%%%%%%%%%%%%%%%%%%%%%%%
\;\;\;Throughout this section, ``Singlet Models'' refers to the family of models where $\Delta a_\mu$ is generated by a muon-philic singlet, either scalar or vector, through the couplings
\begin{equation}
 g_S S (\mu_L \mu^c~ + \mu^{c  \dagger} \mu_L^\dagger),~~~~
  g_V  V_\nu (\mu^\dagger_L \bar \sigma^\nu \mu_L + \mu^{c  \dagger}\bar \sigma^\nu \mu^{c})~,
\label{lag-singlets}
\end{equation}
where $\mu_L$ and $\mu^c$ are the muon Weyl spinors. Realisations of these scenarios appear in multiple contexts. For example, vector singlets can be classified either into dark photon or $L_\mu-L_\tau$ like scenarios. The former are solutions to $g$-2 where couplings between the vector and first generation fermions are generated via loop-induced kinetic mixing. These scenarios are all excluded~\cite{Bauer:2018onh,Ballett:2019xoj} or soon to be~\cite{Mohlabeng:2019vrz}. The second, $L_\mu-L_\tau$ like scenarios, are vectors that do not couple to first generation fermions. These are highly constrained and a combination near-future experiments might probe the remaining parameter space relevant for the $g$-2 anomaly~\cite{Altmannshofer:2014pba,Krnjaic:2019rsv}. The muon colliders perspectives to probe these scenarios will be discussed later in this section.
%in Section~\ref{sec:LFV}.
Singlet scalar models can be UV-completed by extra scalars and/or fermions that, after being integrated out, generate the dimension-5 operator $(S/\Lambda)\, H^\dagger L\mu^c$. Once the Higgs gets a vev one reproduces the interaction in eq.~\eqref{lag-singlets}. These models are disfavoured for large singlet masses~\cite{Capdevilla:2021kcf}.

Figure~\ref{babar_singlet} shows the limits and projections on muon-philic vector (left) and scalar (right) singlets. In the upper panels, $100\%$ branching ratio to muon is assumed when kinematically allowed. The green/orange bands represent the parameter space for which the singlet scalars/vectors resolve $g$-2 within $2\sigma$. Existing experimental limits are shaded in gray, while projections are indicated with coloured lines. The  $M^3$ \cite{Kahn:2018cqs}, NA64$\mu$ \cite{Gninenko:2018ter}, and ATLAS fixed-target \cite{Galon:2019owl} experiments probe invisibly-decaying singlets; projections here assume a 100\% invisible branching fraction. The LHC limits and HL-LHC projections were obtained from $3\mu$/$4\mu$ muon searches. The purple muon collider projections are obtained from a combination of searches in the singlet plus photon final state, and from deviations in angular observables of Bhabha scattering~\cite{Capdevilla:2021rwo}. 

For scalar singlets whose width is determined entirely by the muon coupling (top right), Figure~\ref{babar_singlet} also shows the projections for a search for $S \to \gamma\gamma$ at a muon beam dump experiment~\cite{Chen:2017awl}~\footnote{See~\cite{Cesarotti:2022ttv} for the opportunities of a beam dump experiment that exploits the high energy MuC beams.} under the minimal assumption that the scalar-photon 
coupling arises solely from integrating out the muon. 

The bottom row plots of the figure include the same experiments, but assume that for $m_{S,V}  > 2m_\mu$, the singlets have the minimal branching ratio to di-muon that is consistent with the di-muon $g_{S,V}$ coupling strength and with the upper bound on the total singlet with-over-mass ratio from unitarity. The curves that are unaffected by this change of the muonic branching fraction correspond to searches that are insensitive to the singlet's decay modes. Notice however that the projections for $M^3$, NA64$\mu$, and ATLAS fixed-target experiments assume a $\simeq 100\%$ invisible branching fraction for $m_{S/V} > 2m_\mu$, which is model-dependent.

The upshot is that a 3 TeV $\mc$ can directly and indirectly probe the entire space of possible singlet explanations for the $g$-2 anomaly for masses above a few or 10 GeV. Lower masses are accessible by other experiments, with the possible exception of the 1--10~GeV window of mass, in the case of scalars, when the branching ratio to muons is small (bottom right panel of Figure~\ref{babar_singlet}). This region could however be covered by a lower energy muon collider, for instance by a 125~GeV MuC with 5 or with 20~${\textrm{fb}}^{-1}$ that is considered in the figure. The 125~GeV MuC is advantageous in this case, because of the following. The sensitivity is dominated by the mono-photon search and, in the signal, the energy of the photon is peaked at $E_{\rm{cm}}/2$ if the singlet is light. The background instead emerges from the production of a massive $Z$-boson decaying invisibly, therefore  the peak moves below $E_{\rm{cm}}/2$ by an amount that is controlled by the $Z$ mass. This enables an effective background reduction at the 125~GeV MuC. At 3 TeV instead the one due to the $Z$ mass is a relatively small correction to the energy. The peak displacement cannot thus be exploited for background rejection due to the finite photon energy resolution and to the smearing of the initial muons energy due to photon radiation in the initial state.


%%%%%%%%%%%%%%%%%%%%%%%%%%%%%%
\paragraph*{Electroweak mediators.}
%%%%%%%%%%%%%%%%%%%%%%%%%%%%%%
\;Scenarios with electroweak mediators can generate the necessary $g$-2 contribution even for new physics much above the TeV scale. In particular, the analysis of~\cite{Capdevilla:2020qel,Capdevilla:2021rwo}  carefully studied simplified models featuring new scalars and fermions that yield the largest possible BSM mass scale able to account for the anomaly.
%
By systematically scanning over the entire parameter space of all these models, subject to the constraint that they resolve the $g$-2 anomaly while maintaining perturbative unitarity (as well as other optional constraints), it is possible to derive an upper bound on the mass of the lightest charged BSM particle that has to exist in order to generate the observed $\Delta a_\mu$. 
%
The possibility of a high multiplicity of BSM states was also considered by allowing $N_{\rm BSM}$ copies of each BSM model to be present simultaneously. The results show that, in order to contribute ${2.8 \times 10^{-9}}$ to ${\Delta a_\mu}$ explaining the anomaly, EW scenarios must always have at least one new charged state lighter than
\begin{equation}
\label{e.MchargedmaxXresultpreview}
M^\mathrm{max, X}_\mathrm{BSM, charged} 
\approx
\left\{
\begin{array}{l}
(100~{\textrm{TeV}})  \ N_{\rm BSM}^{1/2} \\   \mathrm{for\;}  X = \mbox{(unitarity*)}
\\[5pt]
(20~{\textrm{TeV}})  \ N_{\rm BSM}^{1/2}  \\ \mathrm{for\;}  X = \mbox{(unitarity+MFV)} 
\\[5pt]
(20~{\textrm{TeV}})  \ N_{\rm BSM}^{1/6} \\
\mathrm{for\;}  X = \mbox{(unit.+naturalness*)}
\\[5pt]
(9~{\textrm{TeV}})  \ N_{\rm BSM}^{1/6} \\
\mathrm{for\;}  X = \mbox{(unit.+nat.+MFV)}
\end{array}
\right.
\end{equation}
This upper bound is evaluated under four different assumptions for the BSM model solving the $g$-2 anomaly: perturbative unitarity only; unitarity  and MFV (Minimal Flavour Violation); unitarity and naturalness (i.e., specifically, avoiding fine-tuning in the Higgs and in the muon mass); and unitarity combined with naturalness and MFV. 

The unitarity-only bound represents the very upper limit of what is possible within quantum field theory at the edge of perturbativity, but realising such high masses requires severe alignment, tuning, or another unknown mechanism to avoid stringent constraints from charged lepton flavour-violating decays~\cite{Calibbi:2017uvl,Aubert:2009ag}.
%
Therefore, every scenario without MFV has been marked with a star (*) above, to indicate additional tuning or unknown flavour mechanisms that have to also be present.

These values of new physics particle masses provide a rough estimate of the maximal needed collider energy. It has been shown in \cite{Paradisi:2022vqp} that, in concrete models with new scalars and fermions, a MuC with center-of-mass energy in the 10 TeV range can discover the new physics responsible for the muon $g$-2 by means of both direct searches for the new states, and high-energy scattering of SM particles such as $\mu^+\mu^-\to h\gamma$. The combination of direct and indirect searches in different final states can be a powerful handle to disentangle among the underlying models accommodating the anomaly.


The results summarised in the previous two paragraphs show that a MuC with energies from the test-bed-scale $\mathcal{O}$(100 GeV) to $\mathcal{O}$(10 TeV) and beyond has excellent prospects to discover the new particles necessary to explain the $g$-2 anomaly. %In the hypothesis that this anomaly is confirmed and due to new physics, it would thus provide a primary physics target for a high-energy muon collider.

%
\begin{figure*}[t]
\begin{center}
\includegraphics[width=0.46\linewidth]{figures/MuC_EFT}~\includegraphics[width=0.5\linewidth]{figures/xsection_tanb}
\end{center}
\vspace{-0.5cm}
\caption{{\it Left:} Cross sections for $hh$ (cyan) and $hhh$ (green) production as a function of $\sqrt{s}$ in models with VLF. {\it Right:} Cross sections for $hh$ (left axis) and $hhh$ (right axis) production as a function of $\tan\beta$ in models with VLF and 2HDM for $M_{L,E}\simeq m_{H,A,H^{\pm}}$. The dot-dashed and dashed lines correspond to the predictions corresponding to the central value of $\Delta a_{\mu}$ and $m_{H,A,H^{\pm}}=3\times M_{L,E}$ and $m_{H,A,H^{\pm}}=5\times M_{L,E}$, respectively. Both panels assume $\Delta a_{\mu}$ is within $1\sigma$ of the measured value (shaded ranges)~\cite{Dermisek:2021mhi}.}
\label{fig:xsection_plot}
\end{figure*} 
%

%%%%%%%%%%%%%%%%%%%%%%%%%%%%%%%%%%%%%%%%%%%%%%%%
%%%%%%%%%%%%%%%%%%%%%%%%%%%%%%%%%%%%%%%%%%%%%%%%
\paragraph{Multi-Higgs Signatures from Vector-like Fermions}{\ } \\ 
\noindent
%%%%%%%%%%%%%%%%%%%%%%%%%%%%%%%%%%%%%%%%%%%%%%%%
%%%%%%%%%%%%%%%%%%%%%%%%%%%%%%%%%%%%%%%%%%%%%%%%
Simple explanations for $g$-2 involve extensions of the SM with new Vector-Like Fermions (VLF) where the corrections to the muon magnetic moment are mediated by the SM Higgs and gauge bosons~\cite{Kannike:2011ng,Dermisek:2013gta}. These models generate effective interactions between the muon and multiple Higgs bosons leading to predictions for di- and tri-Higgs production at a $\mc$ that are directly correlated with the corrections to $\Delta a_{\mu}$. This section reviews the findings of~\cite{Dermisek:2021mhi,Dermisek:2022aec} on this subject. The authors consider extensions of the SM with VLF doublets, $L_{L,R}$, and singlets $E_{L,R}$ with masses $M_{L,E}$, respectively. It will be assumed that new $L_{L}$ and $E_{R}$ have the same quantum numbers as the SM leptons, but other possibilities will also be commented upon later.

The Yukawa interactions of interest are
%
\begin{eqnarray}
\mathcal{L}\supset && - y_{\mu}\bar{l}_L\mu_{R}H - \lambda_{E}\bar{l}_{L}E_{R}H  - \lambda_{L}\bar{L}_{L}\mu_{R}H  
\nonumber\\
&&- \lambda\bar{L}_{L}E_{R}H  - \bar{\lambda}H^{\dagger}\bar{E}_{L}L_{R}  + h.c.,
\label{eq:lagrangian}	
\end{eqnarray}
%
where $l_{L}=( \nu_{\mu},  \mu_{L} )^T$,  $ L_{L,R}= ( L_{L,R}^{0}, L_{L,R}^{-})^T$, and $H=(0,\;v + h/\sqrt{2})^{T}$ with $v=174$ GeV.
%
In the limit $v\ll M_{L,E}$, after integrating out the heavy leptons at tree level, eq.~(\ref{eq:lagrangian}) becomes
%
\begin{equation}
\mathcal{L}\supset - y_{\mu}\bar{l}_L\mu_{R}H - \frac{m_{\mu}^{LE}}{v^{3}}\bar{l}_L\mu_{R}H(H^\dagger H) + h.c.,
\label{eq:eff_lagrangian}
\end{equation}
where
\begin{equation}
m_\mu^{LE} \equiv \frac{\lambda_{L} \bar{\lambda} \lambda_{E}}{M_{L}M_{E}} v^3\,,
\label{eq:m^LE}
\end{equation}
is the contribution to the muon mass from mixing with new leptons.
%
Mixing of the muon with heavy leptons also leads to modifications of the muon couplings to $W$, $Z$, and $h$, and generates new couplings of the muon to new leptons. 

Assuming that $v\ll M_{L,E}$, the total one-loop correction to $g$-2 induced by these effects is well approximated by~\cite{Kannike:2011ng,Dermisek:2013gta}
%
\begin{equation}
\Delta a_{\mu}
= - \frac{1}{16\pi^{2}}  \frac{m_\mu m_\mu^{LE}}{v^2}.
\label{eq:dela}
\end{equation}
%
The explanation of the measured value of $\Delta a_{\mu}$ within $1\sigma$ requires that
%
\begin{equation}
m_\mu^{LE}/m_\mu = -1.07 \pm 0.25.
\label{eq:mmu_LE_range}
\end{equation}
%
For couplings of $\mathcal{O}(1)$, eq.~(\ref{eq:mmu_LE_range}) can be achieved for new lepton masses even as heavy as 7~TeV while simultaneously satisfying current relevant constraints~\cite{Dermisek:2021ajd}. For couplings close to the limit of perturbativity, $\sqrt{4\pi}$, this range extends to close to 50 TeV. This far exceeds the reach of the LHC and even projected expectations of possible future proton-proton colliders, such as the FCC-hh. However, there are related signals that could be fully probed at, for example, a 3 TeV $\mc$ through the effective interactions generated between the muon and multiple Higgs bosons. These interactions are all generated by eq.~(\ref{eq:eff_lagrangian})~\cite{Dermisek:2021mhi} and they lead to the following predictions
%
\begin{eqnarray}
\sigma_{\mu^+\mu^- \to hh} &=&  \frac{\left|\lambda^{hh}_{\mu \mu}\right|^2}{64 \pi}   = \frac{9}{64 \pi} \left(\frac{m_\mu^{LE}}{v^2}\right)^2 \label{eq:sigma_hh},\label{eq:EFT_xsections_1}\\
\sigma_{\mu^+\mu^- \to hhh}  &=& \frac{\left|\lambda^{hhh}_{\mu \mu}\right|^2}{6144 \pi^3} s = \frac{3}{4096 \pi^3} \left(\frac{m_\mu^{LE}}{v^3}\right)^2 s \label{eq:sigma_hhh}.
\label{eq:EFT_xsections_2}
\end{eqnarray}



\begin{table}[t]
\begin{center}
\begin{tabular}{ |c||c||c|c| } 
\hline
$SU(2)\times U(1)_{Y}$& c & $\sigma_{hh}(3\;\text{TeV})$ & $\sigma_{hhh}(3\;\text{TeV})$ \\
\hline
$\mathbf{2}_{-1/2}\oplus\mathbf{1}_{-1}$ & 1 & $244^{+141}_{-109}$~[ab]& $35.8^{+20.8}_{-15.9}$~[ab] \\
\hline
$\mathbf{2}_{-1/2}\oplus\mathbf{3}_{-1}$  & 5 &$10^{+6}_{-4}$~[ab]&$1.43^{+0.8}_{-0.6}$~[ab]\\
\hline
$\mathbf{2}_{-3/2}\oplus\mathbf{1}_{-1}$ & 3 & $27^{+16}_{-12}$~[ab] & $4.0^{+2.3}_{-1.8}~[ab]$\\
\hline
$\mathbf{2}_{-3/2}\oplus\mathbf{3}_{-1}$ & 3 &$27^{+16}_{-12}$~[ab] & $4.0^{+2.3}_{-1.8}$~[ab]\\
\hline
$\mathbf{2}_{-1/2}\oplus\;\mathbf{3}_{0}$ &1 & $244^{+141}_{-109}$~[ab]& $35.7^{+20.7}_{-15.9}$~[ab]\\
\hline
\end{tabular}
\caption{Quantum numbers of $L_{L,R}\oplus E_{L,R}$ under $SU(2)\times U(1)_{Y}$, corresponding $c$-factor for $\Delta a_{\mu}$, and predictions for di- and tri-Higgs cross sections running at $\sqrt{s}=3$ TeV, assuming $\Delta a_{\mu}\pm 1\sigma$.}
\label{table:models}
\end{center}
\end{table}



Thus, considering eq~(\ref{eq:dela}), one can see that the effective interactions of the muon with the Higgs are completely fixed by the muon mass and the predicted value of $\Delta a_{\mu}$.
The left panel of Figure~\ref{fig:xsection_plot} shows the total $\mu^{+}\mu^{-}\to hh$ and $\mu^{+}\mu^{-}\to hhh$ cross sections at a $\mc$ as a function of $\sqrt{s}$ calculated from the effective lagrangian and assuming that $\Delta a_{\mu}$ is achieved within $1\sigma$ (shaded ranges). Cross sections for a 3 TeV $\mc$ are highlighted with the red line. One can see that, for example, a $\mc$ running at $\sqrt{s}= 3$ TeV with 1 ab$^{-1}$ of integrated luminosity would see about 240 di-Higgs events and about 35 tri-Higgs events. It should be noted that already at $\sqrt{s}=1$ TeV this is roughly 4 (3) orders of magnitude larger than $\mu^{+}\mu^{-}\to hh$ and $\mu^{+}\mu^{-}\to hhh$ in the SM. Di- and tri-Higgs final states produced from vector boson fusion in the SM are characterised by a low total invariant mass. They are therefore easily distinguishable from those from direct $\mu^{+}\mu^{-}$ annihilation, which carry the entire energy of the collider. Backgrounds involving the $Z$-boson such as $\mu^{+}\mu^{-}\to Zh$ or $\mu^{+}\mu^{-}\to ZZ$, which may be comparable at the level of cross sections, should be instead suppressed by an invariant-mass cut on the $Z$-boson decay products.

Models with more exotic quantum numbers can also generate a similar correction to $\Delta a_{\mu}$ and, hence, similar predictions for di- and tri-Higgs cross sections. In total there are 5 different combinations of new lepton fields that can lead to mass-enhanced corrections to $\Delta a_{\mu}$ mediated by the SM Higgs. In each case, the correction as given in eq.~(\ref{eq:dela}) is simply multiplied by a corresponding $c$-factor. The resulting cross sections are then rescaled by a factor of $1/c^{2}$ compared to those in Figure~\ref{fig:xsection_plot}. Table~\ref{table:models} lists the $c$-factor multiplying eq.~(\ref{eq:dela}), and the corresponding predictions for di- and tri-Higgs cross sections for a $\mc$ running at $\sqrt{s}=3$ TeV, assuming $\Delta a_{\mu}\pm 1\sigma$. A $\mc$ can fully probe these scenarios even with moderate energies $\sqrt{s}\sim 1-3$ TeV.

%%%%%%%%%%%%%%%%%%%%%%%
%%%%%%%% start Dermisek

\begin{figure}[h]
%\includegraphics[width=0.65\linewidth]{contour_generick.pdf} 
\includegraphics[width=0.9\linewidth]{figures/contour_SMVLL_R1c_EPJC.png} 
%\includegraphics[width=0.65\linewidth]{contour_genericks2351025_final.pdf} 
\caption{Contours of constant $R_{h\to \mu^+\mu^-} = 1$ (solid), $1\pm 0.1$ (shaded), and 2.2 (dashed) in the $\Delta a_{\mu}$ -- $d_{\mu}$ plane in models with $c=1$, 3, and 5.}
\label{fig:SM_ellipse}
\end{figure}

Allowing for complex couplings, a parameter-free correlation emerges between  $\Delta a_\mu$, $d_\mu$, and $R_{h\to \mu^+\mu^-} \equiv BR(h\to \mu^+\mu^-) / BR(h\to \mu^+\mu^-)_{SM}$, with $R_{h\to \mu^+\mu^-} $ carving an ellipse in the plane of dipole moments~\cite{Dermisek:2022aec}:
\begin{flalign}
R_{h\to \mu^+\mu^-}=\left(\frac{\Delta a_{\mu}}{2\omega} - 1\right)^{2} + \left(\frac{m_{\mu}d_{\mu}}{e\omega}\right)^{2},
\label{eq:ellipse}
\end{flalign}
%
where $\omega = m_{\mu}^{2}/kv^{2}$. The $k$-factor relates the dimension 6 mass and dipole operators and for models with VLF it is given by
%
\begin{equation}
k=\frac{64\pi^{2}}{c},
\label{eq:tree_X_SM}
\end{equation}
%
where $c$-factors are listed in Table~\ref{table:models}. In Figure~\ref{fig:SM_ellipse} we show contours of constant $R_{h\to \mu^+\mu^-} = 1$, $1\pm 0.1$, and 2.2 in the plane of the muon dipole moments for models with $c=1$, 3, and 5. Different explanations of $\Delta a_\mu$ and the SM-like $R_{h\to \mu^+\mu^-}$ (or any other fixed value) require specific values of $d_{\mu}$.  Non-zero $d_{\mu}$ can only  increase the quoted rates for $\mu^+ \mu^- \to hh$ and  $\mu^+ \mu^- \to hhh$ and similar ellipses can be shown for the corresponding production cross sections. Models can be efficiently distinguished by these correlations. 

%%%%%
\begin{figure*}[t]
\centering
\includegraphics[width=0.75\textwidth]{figures/lfv_4fermi_1ab.png}
\caption{Summary of $\mc$ and low-energy constraints on flavor-violating 3-body lepton decays. The colored horizontal lines show the sensitivity to the $\tau 3\mu$ operator at various energies, all assuming $1\text{ ab}^{-1}$ of data. The dashed horizontal (vertical) lines show the current or expected sensitivity from $\tau \to 3\mu$ ($\mu \to 3e$) decays for comparison. The diagonal black lines show the expected relationship between different Wilson coefficients with various ansatz for the scaling of the flavor-violating operators (e.g., ``Anarchy'' assumes that all Wilson coefficients are $\mathcal{O}(1)$).
}\label{fig:lfv-operator-summary}
\end{figure*}
%%%%%

%%%%%%%%%%%%%%%%%%%%%%%%%%%%%%%%%%%%%%%%%%%%%%%%%
\paragraph*{Vector-like fermions and Two-Higgs-Doublet models.}
%%%%%%%%%%%%%%%%%%%%%%%%%%%%%%%%%%%%%%%%%%%%%%%%%
\;
It is straightforward to extend the discussion from the previous section to a 2HDM~\cite{Dermisek:2020cod,Dermisek:2021ajd}. For instance, in a type-II 2HDM where charged leptons couple exclusively to one Higgs doublet, $H_{d}$, (which can be achieved by assuming a $Z_{2}$ symmetry) the lagrangian in eq.~(\ref{eq:lagrangian}) from the previous section, is simply modified with the replacement $H\to H_{d}$. In this case both Higgs doublets develop a vev $\left\langle H_{d}^{0} \right\rangle=v_{d}$ and $\left\langle H_{u}^{0} \right\rangle=v_{u}$, where $\sqrt{v_{d}^{2}+v_{u}^{2}}=v=174$ GeV and $\tan\beta = v_{u}/v_{d}$. The effective interactions generated by integrating out heavy leptons is then
%
\begin{equation}
\mathcal{L}\supset y_{\mu}\bar{\mu}_{L}\mu_{R}H_{d} - \frac{m_{\mu}^{LE}}{v_{d}^{3}}\bar{\mu}_{L}\mu_{R}H_{d}(H_{d}^{\dagger} H_{d}).
\label{eq:eff_lagrangian_2HDM}
\end{equation}
%
Similar modifications to $Z$, $W$, and the SM-like Higgs couplings to the muon are also generated after EWSB.
Including the additional corrections to $\Delta a_{\mu}$ from heavy charged and neutral Higgs bosons leads to~\cite{Dermisek:2020cod,Dermisek:2021ajd}
%
\begin{equation}
\Delta a_{\mu}=-\frac{1+\tan^{2}\beta}{16\pi^{2}}\frac{m_{\mu}m_{\mu}^{LE}}{v^{2}},\;\;m_\mu^{LE} \equiv \frac{\lambda_{L} \bar{\lambda} \lambda_{E}}{M_{L}M_{E}} v_{d}^3,
\label{eq:gm2}
\end{equation}
%
where $M_{L,E}\simeq m_{H,A,H^{\pm}}$ is assumed for simplicity. The first term in eq.~(\ref{eq:gm2}) results from the same loops as in the SM, i.e. involving the $Z$, $W$, and SM-like Higgs, whereas the second term, enhanced in comparison by $\tan^{2}\beta$, results from the additional contributions from the heavy Higgses. The corresponding requirement to satisfy $\Delta a_{\mu}$ within $1\sigma$ then becomes
%
\begin{equation}
m_{\mu}^{LE}/m_{\mu}=(-1.07\pm0.25)/(1+\tan^{2}\beta).
\label{eq:gm2_range}
\end{equation}
Just as in the previous section, effective interactions between the muon and multiple Higgs bosons are generated via the single dimension-six operator in eq.~(\ref{eq:eff_lagrangian}). Thus, predictions for di- and tri-Higgs cross sections follow in the same way simply by replacing $m_{\mu}^{LE}$ with the corresponding definition in eq.~(\ref{eq:gm2}). Considering eq.~(\ref{eq:gm2_range}), it follows that $\sigma_{\mu^{+}\mu^{-}\to hh}$ and $\sigma_{\mu^{+}\mu^{-}\to hhh}$ cross sections in a type-II 2HDM decrease as $1/\tan^{4}\beta$. 

Figure~\ref{fig:xsection_plot} shows the $\tan\beta$ dependence of $\sigma_{\mu^{+}\mu^{-}\to hh}$ and $\sigma_{\mu^{+}\mu^{-}\to hhh}/s$ as obtained from the effective lagrangian when $\Delta a_{\mu}$ is achieved within $1\sigma$ (shaded range) and $M_{L,E}\simeq m_{H,A,H^{\pm}}$. The dot-dashed and dashed lines correspond to the predictions corresponding to the central value of $\Delta a_{\mu}$ and $m_{H,A,H^{\pm}}=3\times M_{L,E}$ and $m_{H,A,H^{\pm}}=5\times M_{L,E}$, respectively. Its expected that future measurements of $h\to \mu^{+}\mu^{-}$ will probe $\tan\beta$ up to $\sim 5$ and the inset zooms into this region~\cite{Dermisek:2021mhi}.

For a $\mc$ running at centre-of-mass energy of 3 TeV with, for example, 1 ab$^{-1}$ of luminosity, 3 di-Higgs events are expected in these scenarios for $\tan\beta \simeq 3$. For tri-Higgs the same sensitivity does not extend much above $\tan\beta \simeq 1$. When $m_{H,A,H^{\pm}}=5\times M_{L,E}$, the corresponding sensitivities to $\tan\beta$ increase to about $\tan\beta\simeq 5$ and 2.5 for di-Higgs and tri-Higgs signals, respectively.

These conclusions also extend to models with additional scalars where the SM Higgs is only one component of the scalar sector responsible for EWSB. Mixing within the Higgs sector (e.g. $\tan\beta$ in a 2HDM) introduces a free parameter to the predictions and correlations between the muon magnetic moment and effective Higgs couplings. Thus, the corresponding predictions for di- and tri-Higgs signals at a $\mc$ are not as sharp in these scenarios as compared to the SM. Though in a 2HDM the observables parametrically interpolate between the SM and models with scalars that do not participate in EWSB.

%%%%%% start Dermisek
%%%%%%%%%%%%%%%%%%%%%%

Allowing for complex couplings, the correlation between $\Delta a_\mu$, $d_\mu$, and $R_{h\to \mu^+\mu^-}$ discussed in the previous paragraph emerges in all models with chiral enhancement. In 2HDM the $k$ factor is determined by $\tan\beta$, and in models with scalars not participating in EWSB, for example those discussed in~\cite{Capdevilla:2021rwo}, the $k$ factor is directly linked to the coupling responsible for chiral enhancement~\cite{Dermisek:2022aec}. Just like for the 2HDM, the corresponding rates for $\mu^+ \mu^- \to hh$ and  $\mu^+ \mu^- \to hhh$ can be calculated in any model with chiral enhancement. Furthermore, in all these models, the correlation between $\Delta a_\mu$, $d_\mu$, and $R_{h\to \mu^+\mu^-}$ allows to set the upper bound on masses of new particles able to explain  $\Delta a_\mu$.

%%%%%%%%%%% end Dermisek
%%%%%%%%%%%%%%%%%%%%%%%%%

%%%%%%%%%%%%%%%%%%%%%%%%%%%%%%%%%%%%%%%%%%%%%%%%
%%%%%%%%%%%%%%%%%%%%%%%%%%%%%%%%%%%%%%%%%%%%%%%%
\subsubsection*{Lepton Flavour Violation}% \label{sec:LFV}}
%%%%%%%%%%%%%%%%%%%%%%%%%%%%%%%%%%%%%%%%%%%%%%%%
%%%%%%%%%%%%%%%%%%%%%%%%%%%%%%%%%%%%%%%%%%%%%%%%


The SM exhibits a distinctive pattern of fermion masses and mixing angles, for which we currently have no deep explanation.
Delicate symmetries also lead to a strong suppression of flavor-changing processes in the quark and lepton sectors, which may be reintroduced by new particles or interactions.
The non-observation of such processes thus leads to some of the most stringent constraints on BSM physics, while a positive signal could give us insight into the observed structure of the SM.
A number of precision experiments searching for lepton flavor violating (LFV) processes such as $\mu \to 3e$, $\tau \to 3\mu$ or $\mu$-to-$e$ conversion within atomic nuclei will explore these processes with orders of magnitude more precision in the coming decades~\cite{Baldini:2018uhj}.
As we will see, a high-energy $\mc$ has the unique capability to explore the same physics --- either via measuring effective interactions or by directly producing new states with flavor-violating interactions --- at the TeV scale.

%%%%%
\begin{figure*}[t!]
\centering
\includegraphics[width=7cm]{figures/3TeV_fixmB_plot.pdf}
~
\includegraphics[width=7cm]{figures/3TeV_mBscan_plot.pdf}
%
\caption{Constraints on lepton flavor violation in the MSSM in the $\Delta m^2 / \bar{m}^2$ vs. $\sin 2\theta_R$ plane (left) and the $\sin 2\theta_R$ vs. $M_1$ plane (right) from measurements of the slepton pair production process with flavor-violating final states (red band) at a 3 TeV $\mc$, assuming $1\,\textrm{ab}^{-1}$ of luminosity.
The width of the band represents the uncertainty on the reach from the measurement of the slepton and neutralino masses in flavor-conserving channels.
The purple and blue shaded lightly shaded regions indicate parameters preferred in Gauge-Mediated Supersymmetry Breaking scenarios and flavor-dependent mediator scenarios, respectively.
Both plots assume a mean slepton mass of $1\,\textrm{TeV}$. In the left plot we fix the neutralino mass $M_1 = 500\,\textrm{GeV}$, while in the right figure $\Delta m^2 / \bar{m}^2$ is fixed to 0.1.
The current (solid) and expected (dashed, dotted) limits from low-energy lepton flavor violation experiments are indicated by the blue, purple and green lines.}
\label{fig:mssm-3tev-summary}
\end{figure*}
%%%%%



%%%%%%%%%%%%%%%%%%%%%%%%%%%%%%%%%%%%%%%%%%%%%%%%%%%%%%%%%%%%
\paragraph{Effective LFV Contact Interactions}{\ } \\ 
\noindent
In this section, we study $\mc$ bounds on $\mu\mu \ell_i \ell_j$-type contact interactions, and demonstrate the complementarity with precision experiments looking for lepton-flavor violating decays, as first studied in~\cite{AlAli:2021let}. We will focus on $\tau 3\mu$ and $\mu 3e$ operators, since constraints on them can be compared directly with the sensitivity from $\tau \to 3\mu$ and $\mu \to 3 e$ decays. We parametrise the four-fermion operators relevant for the $\tau \to 3\mu$ decay via
\begin{eqnarray}
\mathcal{L} \supset &&
V_{LL}^{\tau 3\mu}\big( \bar{\mu} \gamma^{\mu} P_L \mu \big) \big(\bar{\tau} \gamma_{\mu} P_L \mu\big) \nonumber \\
&&+ V_{LR}^{\tau 3\mu} \big( \bar{\mu} \gamma^{\mu} P_L \mu\big) \big(\bar{\tau} \gamma_{\mu} P_R \mu\big) \nonumber \\
&&+ \big( L \leftrightarrow R \big) + \textrm{h.c.}\,,
\end{eqnarray}
with an equivalent set for the $\mu \to 3e$ decay. 
In what follows, we will assume all the $V_{ij}^{\tau 3\mu}$ coefficients are equal to $c^{\tau 3\mu} / \Lambda^2$, 
where $c^{\tau 3\mu}$ is a dimensionless coefficient and $\Lambda$ is to be interpreted as the scale of new physics, and similarly for $\mu 3 e$ coefficients.

At a $\mc$, the $\tau 3 \mu$ coefficients are probed via the $\mu^+\mu^- \to \mu \tau$ scattering process. 
Our analysis closely follows an analogous study at an $e^+ e^-$ collider in Ref.~\cite{Murakami:2014tna}. 
As discussed in~\cite{AlAli:2021let}, the SM backgrounds from $\tau^+\tau^-$ and $W^+W^-$ production can be substantially mitigated by a simple set of cuts, whereas the signal can be largely retained up to $\sim 10\%$ effects due to initial state radiation.
The resulting bounds, assuming fixed integrated luminosities of $1\,\textrm{ab}^{-1}$ at $0.125$, $3$, $10$ and $30\,\textrm{TeV}$ are shown in Figure~\ref{fig:lfv-operator-summary}, alongside current and future sensitivities of $\tau \to 3\mu$ and $\mu\to 3 e$ experiments.
A $3\,\textrm{TeV}$ machine would set a direct bound at the same level as the future Belle~II sensitivity. The sensitivity of higher energy MuCs is underestimated in Figure~\ref{fig:lfv-operator-summary} because the expected luminosity is higher, and vice versa for the 125~GeV MuC.

Given an ansatz regarding the flavour structure, the constraints on the $\tau 3\mu$ operators can be compared to the constraints on the analogous $\mu 3e$ operator in the $\mu \to 3 e$ decay. 
The diagonal lines in Figure~\ref{fig:lfv-operator-summary} show the expected relationship between the two Wilson coefficients for several different ansatz, including flavor anarchy (where all coefficients $\sim 1$), Minimal Leptonic Flavor Violation~\cite{Cirigliano:2005ck}, or scalings with different powers of the involved Yukawa couplings.
While muon decays set the strongest limits assuming anarchical coefficients, a $\mc$ could set competitive constraints for other ansatz: in the most extreme case, where the Wilson coefficients scale like the product of the Yukawas, a $3\,\textrm{TeV}$ machine would have sensitivity comparable to the final Mu3e sensitivity.

In addition to the $\tau 3\mu$ operators considered here, similar sensitivity should be attainable for the process $\mu^+ \mu^- \to \mu^{\pm} e^{\mp}$, as well as for the processes such as $\mu^+ \mu^- \to \tau^{\pm} e^{\mp}$ that violate lepton flavor by two units.
Overall, we see that a $\mc$ would be capable of directly probing flavor-violating interactions that are quite complementary to future precision constraints.




\paragraph{Direct Probes: LFV in the MSSM}{\ } \\ 
\noindent
An exciting possibility is that the flavor-changing processes that might be observed in low-energy experiments arise from loops of new particles near the TeV scale.
As a motivated example, consider the MSSM. The scalar superpartners of the SM leptons can have soft supersymmetry-breaking contributions to their mass- matrix that are off-diagonal in the SM lepton eigenbasis.
As a result, the slepton interactions with the leptons will be flavor-violating and lead to processes such as muon-to-electron conversion and rare muon decays at one loop. 
In well-motivated constructions, the mixing between the scalar partners of the electron and the muon states can be quite large, as the low-energy processes are protected by a ``Super-GIM'' mechanism~\cite{Arkani-Hamed:1996bxi}, allowing the new states to be near the TeV scale while consistent with current bounds.

A 3 TeV $\mc$ would dramatically extend the reach for electroweak-charged superpartners beyond a TeV, raising the possibility of directly producing the new states responsible for lepton flavor-violation. 
Moreover, the unique environment of a $\mc$ makes it possible to not only produce these new states, but measure their LFV interactions. 
This would provide detailed insight into both the mechanism of supersymmetry breaking and the origin of the flavor structure of the SM.
A detailed investigation of these prospects is carried out in~\cite{Homiller:2022iax};
here we briefly review results for 3 TeV. 

To understand the complementarity of low-energy LFV probes and the $\mc$ reach, we consider the scenario where only the right-handed selectron and smuon, along with one light neutralino (which we will assume to be a pure bino with mass $M_1$) are in the spectrum.
If the slepton masses $m_{\tilde{\ell}} > M_1$, the sleptons decay directly to a lepton and bino, and the LFV interactions can be measured directly via the pair-production process: $\mu^+ \mu^- \to \tilde{e}^+_{1,2} \tilde{e}^-_{1,2} \to \mu^{\pm} e^{\mp} \chi_1^0 \chi_1^0$, where the binos appear as missing momentum.
In this simplified scenario, both the low-energy LFV processes and the pair-production process at a $\mc$ depend only on the slepton masses and mixing angle, as well as $M_1$. 

\begin{figure}[t]
	\begin{center}
		%\hspace{-1cm}
\includegraphics[width=0.45\textwidth]{figures/contour.pdf}
	\end{center}
	\vspace{-0.3cm}
	\caption{The 2$\sigma$ sensitivity~\cite{Huang:2021nkl} of the 3~TeV MuC with $1~{\rm ab^{-1}}$ luminosity, given as orange regions. Other limits and projections are also shown for comparison. The region explaining the $(g-2)^{}_{\mu}$ anomaly is outlined by the yellow band.}
	\label{fig:contour}
\end{figure}

\begin{table*}[t]
\centering
\begin{tabular}{|c|c|c|c|}
  \hline
  % after \\: \hline or \cline{col1-col2} \cline{col3-col4} ...
  Coupling & $\kappa\equiv g/g_{\rm SM}$ & Type-II \& lepton-specific& Type-I \& flipped\\ \hline
  %$g_{H/A\mu^+\mu^-}$ & $\kappa_\mu$ & $-\sin\alpha-\cos\alpha \tan\beta$ & $-\sin\alpha-\cos\alpha \cot\beta$ \\ \hline
  $g_{H\mu^+\mu^-}$ & $\kappa_\mu$ &$\sin\alpha/\cos\beta$ & $\cos\alpha/\sin\beta$ \\ \hline
  $g_{A\mu^+\mu^-}$ & $\kappa_\mu$ &$\tan\beta$ & $-\cot\beta$ \\ \hline
  $g_{HZZ}$ & $\kappa_Z$ & $\cos(\beta-\alpha)$ & $\cos(\beta-\alpha)$ \\ \hline
  $g_{HAZ}$ & $1-\kappa^2_Z$ & $\sin(\beta-\alpha)$ & $\sin(\beta-\alpha)$ \\
  \hline
\end{tabular}
\caption[]{Coupling parameter values in different 2HDM setups.}
\label{tab:parameters}
\end{table*}


In Figure~\ref{fig:mssm-3tev-summary}, we show the $5\sigma$ reach for a $3\,\textrm{TeV}$ $\mc$, assuming an average slepton mass of $1\,\textrm{TeV}$.
The left panel shows the reach as a function of the mixing angle and mass-splitting, $\Delta m^2 = m_{\tilde{e},2}^2 - m_{\tilde{e},1}^2$, with $M_1 = 500\,\textrm{GeV}$. The right panel shows the constraints for fixed $\Delta m^2 / \bar{m}^2 = 0.1$ in the $M_1$ vs. $\sin 2\theta_R$ plane. Large mixing angles are motivated in models involving gauge-mediated supersymmetry breaking (GMSB), indicated by the purple region, while larger mass splittings are motivated in scenarios where the messengers carry flavor-dependent charges, such as $L_{\mu} - L_{\tau}$, indicated by the blue regions (see~\cite{Homiller:2022iax} for more details).
The complementary constraints from low-energy experiments searching for $\mu\to e\gamma$, $\mu\to 3e$ decays or $\mu$-to-$e$ transitions are shown in blue, purple and green, respectively.
We see that the $\mc$ reach extends to small mass splittings in the GMSB scenario, and covers a substantial part of the most well-motivated parameter space.


%%%%%%%%%%%%%%%%%%%%%%%%%%%%%%%%%%%%%%%%%%%%%%%%
%%%%%%%%%%%%%%%%%%%%%%%%%%%%%%%%%%%%%%%%%%%%%%%%
\paragraph{Gauge \texorpdfstring{$L_\mu-L_\tau$}{lmmt} Interactions}{\ } \\ 
\noindent
As a last example of new physics source of LFV, we consider the gauging of the $L_\mu-L_\tau$ charge. It is not straightforward to test this model at laboratories due to the preferred couplings to the second and third family leptons, unless we have a facility to directly collide muons. Here we summarise the findings of \cite{Huang:2021nkl} regarding searches of a gauged $L^{}_{\mu}{-}L^{}_{\tau}$ interaction at a MuC.
The discussion focuses on the 3~TeV MuC with $1~{\rm ab}^{-1}$ luminosity. 



The relevant interactions of the new boson $Z'$ read
\begin{eqnarray} \label{eq:L}
	\mathcal{L} \supset  g^{\prime} \left( \overline{\ell^{}_{\rm L}} Q^{\prime} \gamma^{\mu} \ell^{}_{\rm L} + \overline{E^{}_{\rm R}}Q^{\prime} \gamma^{\mu} E^{}_{\rm R} \right) Z^{\prime}_{\mu}\;,
\end{eqnarray}
where $g'$ stands for the coupling constant of gauged $L^{}_{\mu}{-}L^{}_{\tau}$ symmetry, $\ell \equiv (\nu, E)^{\rm T}$ is the lepton doublet with $\nu$ and $E$ being the neutrino and the charged lepton, respectively, and $Q^{\prime} = {\rm Diag}(0,1,-1)$ represents the charge matrix in the basis of $(e,\mu,\tau)$.
The $Z'$ will inevitably mix with the SM gauge bosons,  i.e., $\gamma$ and $Z$. It is found that the mixing with $\gamma$ is strongly suppressed by the $Z'$ mass, while the mixing with $Z$ can be relevant if their masses are of the similar order.
For simplicity, we assume a negligible mixing in the following, which actually represents a conservative estimate of the sensitivity.

In such a setup, the relevant processes for the analysis include the final-state signatures of dimuon plus photon, ditau photon as well as monophoton. Even though the process with  initial photon radiation is of higher order compared to the trivial two-body scatterings, its impact is comparable and in some circumstances even larger than the two-body ones, due to the radiative return of resonant $Z'$ production~\cite{Chakrabarty:2014pja,Karliner:2015tga}.



The two-body scattering is very clean, as the final back-to-back dimuon or ditau carries all the energy delivered by the initial colliding muons. The only background of our concern should be the intrinsic SM processes, such as $\mu^+  \mu^- \to \gamma/Z \to l^+  l^-$ as well as $t$-channel exchanges.

Here one also benefits from the interference between the $Z'$ and SM-mediated diagrams. For instance, consider $\mu^+ \mu^- \to \tau^+ \tau^-$. The interference contribution to the cross section 
\begin{equation}
\sigma \sim e^2 g'^2/(4\pi s) \quad {\rm for} \quad s \gg M^{2}_{Z'} \,,
\end{equation}
and 
\begin{equation}
\sigma \sim -e^2 g'^2/(4\pi M^2_{Z'}) \quad {\rm for} \quad s \ll M^{2}_{Z'}\,,
\end{equation}
dominates over the $Z'$-only cross section, which is proportional to  $\propto g'^4$, when $g'$ is small.
Comparing with the SM cross section
\begin{equation}
\sigma
\sim e^4 /(8\pi s) \sim 10^4~{\rm ab}~(3~{\rm TeV}/\sqrt{s})^2\,,
\end{equation}
one can readily estimate the excellent sensitivity to the gauge coupling  
\begin{eqnarray} %\label{eq:}
	g' < && 3.4 \times 10^{-2} \times \nonumber \\ && \left(\frac{\sqrt{s}}{3~{\rm TeV}} \right)^{\frac{1}{2}}  \left(\frac{1~{\rm ab^{-1}}}{\mathfrak{L}} \right)^{\frac{1}{4}}  {\rm max}\left(1, \frac{M^{}_{Z'}}{\sqrt{s}} \right)\,
\end{eqnarray}
for a collider with centre of mass energy $\sqrt{s}$ and integrated luminosity $\mathfrak{L}$.


\begin{figure*}[t]
\centering
%\includegraphics[bb=0.0in -0.1in 7in 6.0in,width=250pt]{figures/mrecoil.png}
\includegraphics[scale=0.45]{figures/mrecoil.png}
\caption{Recoil mass distribution~\cite{Chakrabarty:2014pja} for heavy Higgs mass of 0.5, 1, 1.5, 2, 2.5, 2.9 TeV with a total width 1 (red), 10 (blue), and 100 (green) GeV at a 3 TeV $\mc$. Background (black shaded region) includes all events with a photon of $p_T>10~$GeV.}
\label{fig:sigbkg}
\end{figure*}

For more detailed sensitivity projections, one has to make a few assumptions about the particle identification and detection prospects.
For the two-body scatterings, we assume an efficiency for dimuon identification of $100\%$ and that for ditau of $70\%$.
The  search of resonance for the radiative return process severely relies on the energy resolution of photon or equivalently dilepton.
The energy resolution for photons, which is detailed in Ref.~\cite{Chatrchyan:2008aa}, has been taken from the current CMS detector with ${\rm PbWO}^{}_{4}$ crystals, while for dimuon we take $\Delta m^{}_{\mu^+\mu^-} \simeq 5\times 10^{-5}~{\rm GeV^{-1}}\cdot s$~\cite{Freitas:2004hq}.
Moreover, a systematic uncertainty of $0.1\%$ level has been assumed.

The projected sensitivity is presented in Figure~\ref{fig:contour}. The limits using $\mu^+  \mu^- \to \ell^+ \ell^-$ (dashed and dotted curves for $\ell=\mu$ and $\tau$, respectively) are given as the darker orange region, while the radiative return process yields the lighter orange region.
Other limits and projections are also shown for comparison, such as $e^{+}   e^{-} \to \mu^{+}  \mu^{-}   Z',~Z' \to \mu^{+}   \mu^{-} $ from the BaBar experiment \cite{TheBABAR:2016rlg}, the LHC searches~\cite{Ma:2001md,Heeck:2011wj}, and 
the trident production in neutrino scattering experiments~\cite{Altmannshofer:2014pba}.


The Gauge \texorpdfstring{$L_\mu-L_\tau$}{lmmt} model can explain the muon $g$-2 anomaly in a particular region of the parameter space, outlined by the yellow band in the figure. In the parameter space of our concern, with $M^{}_{Z'} > 100~{\rm GeV}$, the anomaly-favoured region will be completely covered by the 3~TeV MuC.

\subsubsection*{Heavy Higgses through the Radiative Return Process}%\label{sec:muon-higgs}}

The relatively sizeable muon Yukawa coupling to the SM Higgs boson suggests that the muon coupling to new physics associated with the breaking of the EW symmetry could be directly observable at the MuC, or even be responsible for the discovery of new physics. This is illustrated below in an extended Higgs sector scenario featuring a second Higgs doublet.

%\paragraph{Heavy Higgses through the Radiative Return Process}{\ } \\ 
%\noindent
As discussed in Section~\ref{sec:Higgs}, the muon collider has much better perspectives than the HL-LHC to observe new heavy Higgs bosons already at the 3~TeV stage. It would also enable a very detailed characterisation of the newly discovered extended scalar sector by a number of precision measurements, including line shape studies of the new resonances produced in the $s$-channel~\cite{Eichten:2013ckl,Alexahin:2013ojp}, once their mass will be known with sufficient precision. If the mass is unknown, one can exploit the ``radiative return'' (RR) process. Namely
\begin{equation}
\mu^{+} \mu^{-} \rightarrow  \gamma H, \gamma A ,
\label{eq:rr}
\end{equation}
where we indicate with $H$ ($A$) the neutral CP-even (CP-odd) new heavy scalar states. The process proceeds through the emission of a photon from the initial state muons, which enables them to collide at the heavy Higgs boson mass even if the MuC centre of mass energy was slightly above. In what follows we illustrate the main points of this approach in the context of a 2-Higgs-doublet model (2HDM), summarising the findings of~\cite{Chakrabarty:2014pja}.

\begin{figure*}[t]
\centering
\includegraphics[scale=0.35]{figures/kappas}
\caption[5pt]{Comparison of sensitivities between different production mechanisms in the parameter plane $\kappa_\mu$\nobreakdash-$\kappa_Z$ for different masses of the heavy Higgs boson at the $3~{\textrm{TeV}}$ $\mc$. The shaded regions show a higher direct signal rate from the RR process than the $ZH$ associated production and $HA$ pair production channels. 
One can also see the allowed parameter regions (extracted from Ref.~\cite{Barger:2013ofa}). From~\cite{Chakrabarty:2014pja}.}
\label{fig:kappas}
\end{figure*}

In this model, the relevant heavy Higgs boson couplings can be parametrised as
\begin{eqnarray}
\label{eq:int}
\mathcal{L}_{int} = && -\kappa_\mu \frac {m_\mu} {v} H \bar \mu \mu + i \kappa_\mu \frac {m_\mu} {v} A \bar \mu \gamma_5 \mu 
\nonumber \\
    && 
    + \kappa_Z \frac {m_Z^2} {v} H Z^\mu Z_\mu \\ \nonumber
    && 
    +\frac {g \sqrt{ (1-\kappa_Z^2)}} {2\cos\theta_W} (H\partial^\mu A-A\partial^\mu H) Z_\mu\,,
\end{eqnarray}
where the two parameters $\kappa_\mu$ and $\kappa_Z$ characterise the coupling strength relative to the SM Higgs boson couplings. The coupling $\kappa_\mu$ controls the heavy Higgs resonant production and the radiative return cross sections while $\kappa_Z$ controls the cross sections for $ZH$ associated production and heavy Higgs pair $HA$ production.
We choose for simplicity equal coupling to muons of the CP-even $H$ and the CP-odd $A$. For the $HAZ$ coupling, the generic 2HDM relation is used where $\kappa_Z$ is proportional to $\cos(\beta-\alpha)$ and the $HAZ$ coupling is proportional to 
$\sin(\beta-\alpha)$. In the decoupling limit of the 2HDM at large $m_A$, $\kappa_{Z}\equiv \cos(\beta-\alpha)\sim m_{Z}^{2}/m_{A}^{2}$ is highly suppressed and $\kappa_\mu \approx \tan\beta\ (-\cot\beta)$ in Type-II and lepton-specific (Type-I and flipped) 2HDM. The value of the parameters in different 2HDM setups is shown in Table~\ref{tab:parameters}. 
%


 The signature of the RR process is quite striking as it results in  a monochromatic photon. For narrow scalars, the ``recoil mass'' is a sharp resonant peak at $m_{H/A}$, standing out of the continuous SM background. The reconstruction of the heavy Higgs boson from its decay product could provide an extra handle and a clean way to determine the heavy scalar branching ratios in a later stage after discovery. 

The characteristic of the RR signal is a photon energy
\begin{equation}
E_\gamma= \frac {\hat s - m_{H/A}^2} {2\sqrt{\hat s}},
\end{equation}
from which a recoil mass peaked at the heavy Higgs mass $m_{H/A}$ can be reconstructed. The energy of this photon is broadened by both detector effects, e.g.,  photon energy resolution, beam energy spread and physics effects, e.g., additional (soft) ISR/FSR, and the heavy Higgs width. 
The beam energy spread and additional soft ISR/FSR are expected to be at the GeV level~\cite{Delahaye:2013jla}. The recoil mass reconstruction uncertainty is dominated by the photon energy resolution, at least if the Higgs boson mass is significantly below the collider centre of mass energy. The heavy Higgs width, if sizeable, could effectively smear the energy of the photon. In the 2HDM it ranges between order 1 and 100~GeV for TeV-sized heavy Higgs masses. 

The inclusive cross section for the mono-photon background is substantial compared to the radiative return signal. The background can be estimated  from the M\"oller scattering with initial or final state photon emission, $\mu^+\mu^- \to \mu^+\mu^- \gamma$, and from the $W$ $t$-channel exchange with initial photon radiation, $\mu^+\mu^- \to \nu\nu \gamma$. 
The signal to background ratio is typically of order $10^{-3}$ for a 3 TeV $\mc$. Consequently, to discover through RR, we rely on exclusive processes and specific final states. 



For concrete illustration, a Type-II 2HDM has been adopted with the $b\bar b$ final state with $80\%$ decaying branching ratio. An $80\%$ $b$-tagging efficiency is assumed and at least one $b$-jet tagged required in the analysis. Madgraph5~\cite{Alwall:2011uj} has been used for parton level signal and background simulations and then Pythia~\cite{Sjostrand:2006za} for initial and final state photon radiation. Detector smearing and beam energy spread have been also implemented.

Figure~\ref{fig:sigbkg} shows the recoil mass distribution at a 3 TeV $\mc$. Both the signal and the background cross sections at fixed beam energy increase as the recoil mass increase from the photon emission. One can clearly distinguish the pronounced mass peaks. 


In order to assess the discovery perspectives of the RR process, in Figure~\ref{fig:kappas} we compare its reach with the one from the $ZH$ associated production and $HA$ pair production, in the $\kappa_\mu$\nobreakdash-$\kappa_Z$ plane. The RR production mode covers a large region of the plane, which expands when the heavy Higgs mass gets close to the MuC energy of 3~TeV. The region where RR dominates the discovery obviously shrinks when the heavy Higgs mass crosses the threshold for pair production. This is shown by the darker contour in the figure, for $1.4$~TeV mass. The figure also displays the LHC constraints from the measurements of the coupling of the SM Higgs. A more detailed discussions can be found in Ref.~\cite{Chakrabarty:2014pja}. 


\end{document}