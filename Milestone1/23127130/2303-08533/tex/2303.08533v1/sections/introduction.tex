\section{Introduction}\label{IntroSect}

Colliders are microscopes that explore the structure and the interactions of particles at the shortest possible length scale. Their goal is not to chase discoveries that are inevitable or perceived as such based on current knowledge. On the contrary, their mission is to explore the unknown in order to acquire radically novel knowledge.

The current experimental and theoretical situation of particle physics is particularly favourable to collider exploration. No inevitable discovery diverts our attention from pure exploration, and we can focus on the basic questions that best illustrate our ignorance. Why is electroweak symmetry broken and what sets the scale? Is it really broken by the Standard Model Higgs or by a more rich Higgs sector? Is the Higgs an elementary or a composite particle? Is the top quark, in light of its large Yukawa coupling, a portal towards the explanation of the observed pattern of flavor? Is the Higgs or the electroweak sector connected with dark matter? Is it connected with the origin of the asymmetry between baryons and anti-baryons in the Universe?

The next collider should deepen our understanding of the questions above, and offer broad and varied opportunities for exploration to enable radically unexpected discoveries. A comprehensive exploration must exploit the complementarity between energy and precision. Precise measurements allow us to study the dynamics of the particles we already know, looking for the indirect manifestation of yet unknown new physics. With a very high energy collider we can access the new physics particles directly. These two exploration strategies are normally associated with two distinct machines, either colliding electrons/positrons ($ee$) or protons ($pp$). 

With muons instead, both strategies can be effectively pursued at a single collider that combines the advantages of $ee$ and of $pp$ machines. Moreover, the simultaneous availability of energy and precision offers unique perspectives of indirect sensitivity to very heavy new physics, as well as unique perspectives for the characterisation of new heavy particles discovered at the muon collider itself. 

This is the picture that emerges from the investigations of the muon colliders physics potential performed so far, to be reviewed in this document in Sections~\ref{ch2_phys_opp} and~\ref{sec:phys_studies}. These studies identify a Muon Collider (MuC), with 10~TeV energy or more in the centre of mass and sufficient luminosity, as an ideal tool for a substantial ambitious jump ahead in the exploration of fundamental particles and interactions. Assessing its technological feasibility is thus a priority for the future of particle physics.

\paragraph{Muon collider concept}

Initial ideas for muon colliders were proposed long ago~\cite{Tikhonin:initial,Budker:initial,Neuffer:1979gq,Cline:1980sa,Skrinsky:1981ht,Neuffer:1983jr}. Subsequent studies culminated in the Muon Accelerator Program (MAP) in the US (see \cite{Ankenbrandt:1999cta,Palmer:2014nza,Boscolo:2018ytm,Neuffer:2017hnu} and~\cite{Delahaye:2019omf,Long:2020wfp} for an overview). The MAP concept for the muon collider facility is displayed in Figure~\ref{f:design}. The proton complex produces a short, high-intensity proton pulse that hits the target and produces pions. The decay channel guides the pions and collects the muons from their decay into a bunching and phase rotator system to form a muon beam. Several cooling stages then reduce the longitudinal and transverse emittance of the beam using a sequence of absorbers and radiofrequency (RF) cavities in a high magnetic field. A system of a linear accelerators (linac) and two recirculating linacs accelerate the beams to 60~GeV. They are followed by one or more rings to accelerate them to higher energy, for instance one to 300~GeV and one to 1.5~TeV, in the case of a 3~TeV centre of mass energy MuC. In the 10~TeV collider an additional ring from 1.5 to 5~TeV follows. These rings can be either fast-pulsed synchrotrons or Fixed-Field Alternating gradients (FFA) accelerators. Finally, the beams are injected at full energy into the collider ring. Here, they will circulate to produce luminosity until they are decayed. Alternatively they can be extracted once the muon beam current is strongly reduced by decay. There are wide margins for the optimisation of the exact energy stages of the acceleration system, taking also into account the possible exploitation of the intermediate-energy muon beams for muon colliders of lower centre of mass energy. 

\begin{figure}
    \begin{center}
    \includegraphics[width=0.4\textwidth,bb=0 0 17cm 12cm]{figures/facility/THPAB017f1.pdf}
    \end{center}
\caption{A conceptual scheme of the muon collider.}
\label{f:design}
\end{figure}

The concept developed by MAP provides the baseline for present and planned work on muon colliders, reviewed in Section~\ref{sec_fac}. Three main reasons sparked this renewed interest in muon colliders. First, the focus on high collision energy and luminosity where the muon collider is particularly promising and offers the perspective of revolutionising particle physics. Second, the advances in technology and muon colliders design. Third, the difficulty of envisaging a radical jump ahead in the high-energy exploration with $ee$ or $pp$ colliders.

In fact, the required increase of energy and luminosity in future high-energy frontier colliders poses severe challenges~\cite{Shiltsev:2019rfl}. Without breakthroughs in concept and in technologies, the cost and use of land as well as the power consumption are prone to increase to unsustainable levels.

The muon collider promises to overcome these limitations and allow to push the energy frontier strongly. Circular electron-positron colliders are limited in energy by the emission of synchrotron radiation that increases strongly with energy. Linear colliders overcome this limitation but require the beam to be accelerated to full energy in a single pass through the main linac and allow to use the beams only in a single collision~\cite{Stapnes:2019dcu}. The high mass of the muon mitigates synchrotron radiation emission, allowing them to be accelerated in many passes through a ring and to collide repeatedly in another ring. This results in cost effectiveness and compactness combined with a luminosity per beam power that roughly increases linearly with energy. Protons can be also accelerated in rings and made to collide with very high energy. However, protons are composite particles and therefore only a small fraction of their collision energy is available to probe short-distance physics through the collisions of their fundamental constituents. The effective energy reach of a muon collider thus corresponds to the one of a proton collider of much higher centre of mass energy. This concept is illustrated more quantitatively in Section~\ref{whym}.

Currently, the limit of the energy reach for muon colliders has not been identified. Ongoing studies focus on a 10~TeV design with an integrated luminosity goal of $10 ~ \rm ab^{-1}$. This goal is expected to provide a good balance between an excellent physics case and affordable cost, power consumption and risk. Once a robust design has been established at 10~TeV---including an estimate of the cost, power consumption and technical risk---other, higher energies will be explored.

The 2020 Update of the European Strategy for Particle Physics (ESPPU) recommended, for the first time in Europe, an R{\&}D programme on muon colliders design and technology. This led to the formation of the International Muon Collider Collaboration (IMCC)~\cite{IMCC} with the goal of initiating this programme and informing the next ESPPU process on the muon collider feasibility perspectives. This will enable the next ESPPU and other strategy processes to judge the scientific justification of a full Conceptual Design Report (CDR) and demonstration programme.

The European Roadmap for Accelerator R\&D~\cite{Adolphsen:2022ibf}, published in 2021, includes the muon collider. The report is based on consultations of the community at large, combined with the expertise of a dedicated Muon Beams Panel. It also benefited from significant input from the MAP design studies and US experts. The report assessed the challenges of the muon collider and did not identify any insurmountable obstacle. However, the muon collider technologies and concepts are less mature than those of electron-positron colliders. Circular and linear electron-positron colliders already have been constructed and operated but the muon collider would be the first of its kind. The limited muon lifetime gives rise to several specific challenges including the need of rapid production and acceleration of the beam. These challenges and the solutions under investigation are detailed in Section~\ref{sec_fac}.

The Roadmap describes the R\&D programme required to develop the maturity of the key technologies and the concepts in the coming few years. This will allow the assessment of realistic luminosity targets, detector backgrounds, power consumption and cost scale, as well as whether one can consider implementing a MuC at CERN or elsewhere. Mitigation strategies for the key technical risks and a demonstration programme for the CDR phase will also be addressed. The use of existing infrastructure, such as existing proton facilities and the LHC tunnel, will also be considered. This will allow the next strategy process to make an informed choice on the future strategy. Based on the conclusions of the next strategy processes in the different regions, a CDR phase could then develop the technologies and the baseline design to demonstrate that the community can execute a successful MuC project.

Important progress in the past gives confidence that this goal can be achieved and that the programme will be successful. In particular, the developments of superconducting magnet technology has progressed enormously and high-temperature superconductors have become a practical technology used in industry. Similarly, RF technology has progressed in general and experiments demonstrated the solution of the specific muon collider challenge---operating RF cavities in very high magnetic fields---that previously has been considered a showstopper. Component designs have been developed that can cool the initially diffuse beam and accelerate it to multi-TeV energy on a time scale compatible with the muon lifetime. However, a fully integrated design has yet to be developed and further development and demonstration of technology are required. 

The technological feasibility of the facility is one vital component of the muon collider programme, but the planning of the facility exploitation is equally important. This includes the assessment of the muon collider potential to address physics questions, as well as the design of novel detectors and reconstruction techniques to perform experiments with colliding muons.

\paragraph{The path to a new generation of experiments}

The main challenge to operating a detector at a muon collider is the fact that muons are unstable particles. As such, it is impossible to study the muon interactions without being exposed to decays of the muons forming the colliding beams. From the moment the collider is turned on and the muon bunches start to circulate in the accelerator complex, the products of the in-flight decays of the muon beams and the results of their interactions with beam line material, or the detectors themselves, will reach the experiments contributing to polluting the otherwise clean collision environment. The ensemble of all these particles is usually known as ``Beam Induced Backgrounds'', or BIB. The composition, flux, and energy spectra of the BIB entering a detector is closely intertwined with the design of the experimental apparatus, such as the beam optics that integrate the detectors in the accelerator complex or the presence of shielding elements, and the collision energy. However, two general features broadly characterise the BIB: it is composed of low-energy particles with a broad arrival time in the detector.

The design of an optimised muon collider detector is still in its infancy, but the work has initiated and it is reviewed in Section~\ref{sec:detectorandreconstruction}. It is already clear that the physics goals will require a fully hermetic detector able to resolve the trajectories of the outgoing particles and their energies. While the final design might look similar to those experiments taking data at the LHC, the technologies at the heart of the detector will have to be new. The large flux of BIB particles sets requirements on the need to withstand radiation over long periods of time, and the need to disentangle the products of the beam collisions from the particles entering the sensitive regions from uncommon directions calls for high-granularity measurements in space, time and energy. The development of these new detectors will profit from the consolidation of the successful solutions that were pioneered for example in the High Luminosity LHC upgrades, as well as brand new ideas. New solutions are being developed for use in the muon collider environment spanning from tracking detectors, calorimeters systems to dedicated muon systems. The whole effort is part of the push for the next generation of high-energy physics detectors, and new concepts targeted to the muon collider environment might end up revolutionising other future proposed collider facilities as well.

Together with a vibrant detector development program, new techniques and ideas needs to be developed in the interpretation of the energy depositions recorded by the instrumentation. The contributions from the BIB add an incoherent source of backgrounds that affect different detector systems in different ways and that are unprecedented at other collider facilities. The extreme multiplicity of energy depositions in the tracking detectors create a complex combinatorial problem that challenges the traditional algorithms for reconstructing the trajectories of the charged particles, as these were designed for collisions where sprays of particles propagate outwards from the centre of the detector. At the same time, the potentially groundbreaking reach into the high-energy frontier will lead to strongly collimated jets of particles that need to be resolved by the calorimeter systems, while being able to subtract with precision the background contributions. The challenging environment of the muon collider offers fertile ground for the development of new techniques, from traditional algorithms to applications of artificial intelligence and machine learning, to brand new computing technologies such as quantum computers.

\paragraph{Muon collider plans}

The ongoing reassessment of the muon collider design and the plans for R\&{D} allow us to envisage a possible path towards the realisation of the muon collider and a tentative technically-limited timeline, displayed in Figure~\ref{fig_muon:RDtimeline}.

The goal~\cite{Delahaye:2019omf,Long:2020wfp} is a muon collider with a centre of
mass energy of \qty{10}{\TeV} or more (a \mbox{$10^+$\,TeV}~MuC). Passing this energy threshold enables, among other things, a vast jump ahead in the search for new heavy particles relative to the LHC. The target integrated luminosity is obtained by considering the cross-section of a typical $2\to2$ scattering processes mediated by the electroweak interactions, $\sigma\sim 1~{\rm{fb}}\cdot(10~{\rm{TeV}})^2/E_{\rm{cm}}^2$. In order to measure such cross-sections with good (percent-level) precision and to exploit them as powerful probes of short distance physics, around ten thousand events are needed. The corresponding integrated luminosity is
\begin{equation}\label{lums}
\displaystyle
\mathfrak{L}_{\rm{int}}=10\,{\rm{ab}}^{-1}\left(\frac{E_{\rm{cm}}}{10\,{\rm{TeV}}}\right)^2\,.
\end{equation}
The luminosity requirement grows quadratically with the energy in order to compensate for the cross-section decrease. We will see in Section~\ref{sec_fac} that achieving this scaling is indeed possible at muon colliders.

Assuming a muon collider operation time of $10^7$ seconds per year, and one interaction point, eq.~(\ref{lums}) corresponds to an  instantaneous luminosity 
\begin{equation}\label{lumsin}
\displaystyle
\mathfrak{L}=\frac{5~{\rm{years}}}{\rm{time}}
\left(\frac{E_{\rm{cm}}}{10\,{\rm{TeV}}}\right)^22\cdot 10^{35}{\rm{cm}}^{-2}{\rm{s}}^{-1}\,.
\end{equation}
The current design target parameters (see Table~\ref{MC:t:param}) enable to collect the required integrated luminosity in a 5-year run, ensuring an appealingly compact temporal extension to the muon collider project even in its data taking phase. Furthermore this ambitious target leaves space to increase the integrated luminosity by running longer or by foreseeing a second interaction point. One could similarly compensate for a possible instantaneous luminosity reduction in the final design.

\paragraph{Muon collider stages}

In order to design the path towards a \mbox{$10^+$\,TeV} MuC, one could exploit the possibility of building it in stages. In fact, the design of many elements of the facility is simply independent of the collider energy. Once built and exploited for a lower $E_{\rm{cm}}$ MuC, they can thus be reused for a higher energy collider. This applies to the muon source and cooling, and to the accelerator complex as well because energy staging is anyway required for fast acceleration. Only the final collision ring of the lower $E_{\rm{cm}}$ collider could not be reused. However because of its limited size it is a minor addition to the total cost.

A staged approach has several advantages. First, it spreads the total cost over a longer time period and reduces the initial investment. This could enable a faster financing for the first energy stage and accelerate the whole project. Furthermore the reduced energy of the first stage allows, if needed, to make compromises on technologies that might not yet be fully developed, avoiding potential delays. In particular completing construction in 2045 as foreseen in Figure~\ref{fig_muon:RDtimeline} could be optimistic for a \mbox{$10^+$\,TeV} MuC, but realistic for a first lower-energy collider at few TeV. A centre of mass energy $E_{\rm{cm}}=3$~TeV is being tentatively considered for the first stage. This matches, with a much more compact design, the maximal $e^+e^-$ energy that could be achieved by the last stage of the CLIC linear collider~\cite{Stapnes:2019dcu}.

The 3~TeV stage of the muon collider offers amazing opportunities for progress with respect to the LHC and its high-luminosity successor (HL-LHC)~\cite{ZurbanoFernandez:2020cco}. These opportunities include a determination of the Higgs trilinear coupling, extended sensitivity to Higgs and top quark compositeness and to extended Higgs sectors. A selected summary of available studies is reported in Section~\ref{sec:phys_studies}. On the other hand, the physics potential of the \mbox{$10^+$\,TeV} collider is much superior to the one of the 3~TeV collider. The higher energy stage will radically advance the knowledge acquired with the first stage operation. Additionally, the energy upgrade would enable to investigate new physics discoveries or tensions with the SM that might emerge at the first stage.
%Generally speaking, the \mbox{$10^+$\,TeV} stage sensitivity or reach is just better than the one of the 3~TeV stage. Therefore the first stage will serve to collect more rapidly partial knowledge, waiting for the more complete findings of  the \mbox{$10^+$\,TeV} stage. However, the 3~TeV stage could also offer unique opportunities. For instance, it could be more effective than the \mbox{$10^+$\,TeV} stage to probe TeV or sub-TeV masses for resonances in the $\mu^+\mu^-$ channel.

The 3~TeV stage, following eq.~(\ref{lumsin}), would collect $0.9\simeq 1$~ab$^{-1}$ luminosity (with one detector) in five years of full luminosity, after an initial ramp-up phase of two to three years. In the most optimistic scenario the construction of the first stage will proceed rapidly. The first stage will terminate after seven years to leave space to the second stage with radically improved physics performances. If the second stage is instead delayed, the one at 3~TeV could run longer. The optimistic and pessimistic scenarios thus foresee 1 and  2~ab$^{-1}$ at 3~TeV, respectively.

\paragraph{Other muon colliders}

The tentative staging scenario detailed above should serve as the baseline for future investigations of alternative plans. In particular, one could consider the possibility of a first stage of much lower energy than 3~TeV, to be possibly built on an even shorter time scale. However, it is worth remarking in this context that the quadratic luminosity scaling with energy in eqs.~(\ref{lums}) and (\ref{lumsin}) is not only the aspirational target, but it is also the natural scaling of the luminosity at muon colliders. By following the scaling for low $E_{\rm{cm}}$ one immediately realises that the luminosity of a muon collider at order 100~GeV energy can not be competitive with the one of an $e^+e^-$ circular or linear collider. For instance this implies that while there is evidently a compelling physics case for a leptonic ``Higgs factory'' at around 250~GeV energy, a muon collider would be probably unable to collect the high luminosity needed for a successful program of Higgs coupling measurements, while this is possible for $e^+e^-$ colliders. In general, the luminosity scaling suggests that a physics-motivated first stage of the muon collider should either exploit some peculiarity of the muons that make $\mu^+\mu^-$ collisions more useful than $e^+e^-$ collisions, or target the TeV energy that is hard to reach with $e^+e^-$.

The possibility of operating a first muon collider at the Higgs pole $E_{\rm{cm}}=m_H=125$~GeV has been discussed extensively in the literature. The idea here is to exploit the large Yukawa coupling of the muon, much larger than the one of the electron, in order to produce the Higgs boson in the $s$-channel and study its lineshape. The physics potential of such Higgs-pole muon collider will be described in Section~\ref{sec:phys_studies}. The major results would be a rather precise and direct determination of the Higgs width and an astonishingly accurate measurement of the Higgs mass. However, the Higgs is a rather narrow particle, with a width over mass ratio $\Gamma_H/m_H$ as small as $3\cdot10^{-5}$. The muon beams would thus need a comparably small energy spread $\Delta E/E\hspace{-2pt}=\hspace{-2pt}3\cdot10^{-5}$ for the programme to succeed. Engineering such tiny energy spread might perhaps be possible. However it poses a challenge for the facility design that is peculiar to the Higgs-pole collider and of no relevance for higher energies, where a much higher spread is perfectly adequate for physics. For this reason, the Higgs-pole muon collider is not currently considered in the staging plan and further study is needed. 

Further work is also needed to assess the possible relevance of a muon collider at the $t{\overline{t}}$ production threshold $E_{\rm{cm}}\simeq 343$~GeV, aimed at measuring the top quark mass with precision. The top threshold could be reached also with $e^+e^-$ colliders. However the $e^+e^-$ Higgs factory at 250~GeV, to be possibly built before the muon collider, might not be easily and quickly upgradable to 343~GeV. Moreover, the naturally small (permille-level) beam energy spread and the reduction of initial state radiation effects give an advantage to muons over electrons for the threshold scan. Such ``first muon collider'' was proposed long ago~\cite{Berger:1997xu,Barger:1997yk}. Its modern relevance stems from the need of an improved top mass determination for establishing the possible instability of the SM Higgs potential~\cite{Franceschini:2022veh}. We will not discuss this option further and we refer the reader to the literature.

\paragraph{This Review}

A muon collider could be a sustainable innovative device for a big ambitious jump ahead in fundamental physics exploration. It is a long-term project, but with a tight schedule and a narrow temporal window of opportunity. The initiated work must continue in the next decade, fostered by a positive recommendation of the 2023 US Particle Physics Prioritization Panel (P5) and the next Update of the European Strategy for Particle Physics foreseen in 2026/2027. Progress must be made by then on the perspectives for a muon collider to be built and operated, for the outcome of its collisions to be recorded, interpreted and exploited to advance physics knowledge. This offers stimulating challenges for accelerator, experimental and theoretical physics. 

Muon colliders require innovative research in each of these three directions. The novelty of the theme and the lack of established solutions enable a high rate of progress, but it also requires that the three directions advance simultaneously because progress in one motivates and supports work in the others. Furthermore, exploiting synergies between accelerator, experimental and theoretical physics is of utmost importance at this initial stage of the muon collider project design.

In this spirit, the present Review summarises the state of the art and the recent progress in all these three areas, and outlines directions for future work. The aim is to provide a global perspective on muon colliders.

This Review is organised as follows. Section~\ref{ch2_phys_opp} summarises the key exploration opportunities offered by very high energy muon colliders and illustrates the potential for progress on selected physics questions. We also outline the challenges for the theoretical predictions needed to exploit this potential. These challenges constitute in fact a tremendous opportunity to advance knowledge of SM physics in a regime where the electroweak bosons are relatively light particles, entailing the emergence of the novel phenomenon of electroweak radiation. Section~\ref{sec_fac} describes the challenges and the opportunities of muon colliders for accelerator physics. It reviews the basic principles for the design of the muon production, cooling and fast acceleration systems. The required R\&{D}, and a tentative staging plan and timeline, are also outlined. Section~\ref{sec:detectorandreconstruction} describes the experimental conditions that are expected at the muon collider and the ongoing work on the design of the detector and of the event reconstruction software. We devote Section~\ref{sec:phys_studies} to selected muon collider sensitivity projection studies. The \mbox{$10^+$\,TeV} MuC is the main focus, but some opportunities of the 3~TeV stage are also described, as well as those of the Higgs-pole collider. A summary of the perspectives and opportunities for future work on muon colliders is reported in Section~\ref{ConclSect}.