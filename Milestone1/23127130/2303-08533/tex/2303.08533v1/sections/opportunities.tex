\section{Physics opportunities}\label{ch2_phys_opp}

\subsection{Why muons?}\label{whym}

Muons, like protons, can be made to collide with a centre of mass energy of $10$~TeV or more in a relatively compact ring, without fundamental limitations from synchrotron radiation. However, being point-like particles, unlike protons, their nominal centre of mass collision energy $E_{\rm{cm}}$ is entirely available to produce high-energy reactions that probe length scales as short as $1/E_{\rm{cm}}$. The relevant energy for proton colliders is instead the centre of mass energy of the collisions between the partons that constitute the protons. The partonic collision energy is distributed statistically, and approaches a significant fraction of the proton collider nominal energy with very low probability. A muon collider with a given nominal energy and luminosity is thus evidently way more effective than a proton collider with comparable energy and luminosity.

This concept is made quantitative in Figure~\ref{pvsm}. The figure displays the center of mass energy \parbox[c][0pt]{19pt}{${\sqrt{s\,}}_{\hspace{-2pt}p}$} that a proton collider must possess to be ``equivalent'' to a muon collider of a given energy \parbox[c][0pt]{54pt}{$E_{\rm{cm}}=\sqrt{s\,}_{\hspace{-2pt}\mu}$}. Equivalence is defined~\cite{Delahaye:2019omf,Costantini:2020stv,AlAli:2021let} in terms of the pair production cross-section for heavy particles, with mass close to the muon collider kinematical threshold of \parbox[c][0pt]{32pt}{$\sqrt{s\,}_{\hspace{-2pt}\mu}/2$}. The equivalent \parbox[c][0pt]{19pt}{${\sqrt{s\,}}_{\hspace{-2pt}p}$} is the proton collider centre of mass energy for which the cross-sections at the two colliders are equal. 

The estimate of the equivalent \parbox[c][0pt]{19pt}{${\sqrt{s\,}}_{\hspace{-2pt}p}$} depends on the relative strength $\beta$ of the heavy particle interaction with the partons and with the muons. If the heavy particle only possesses electroweak quantum numbers, $\beta=1$ is a reasonable estimate because the particles are produced by the same interaction at the two colliders. If instead it also carries QCD colour, the proton collider can exploit the QCD interaction to produce the particle, and a ratio of $\beta=10$ should be considered owing to the large QCD coupling and colour factors. The orange line on the left panel of Figure~\ref{pvsm}, obtained with $\beta=1$, is thus representative of purely electroweak particles. The blue line, with $\beta=10$, is instead a valid estimate for particles that also possess QCD interactions, as it can be verified in concrete examples.

\begin{figure*}[t]
\begin{minipage}{0.48\textwidth}
\begin{center}
\includegraphics[width=\textwidth]{figures/opportunities/muvsp.pdf}
\end{center}
\end{minipage}
\hfill
\begin{minipage}{0.47\textwidth}
\begin{center}
\includegraphics[width=\textwidth]{figures/opportunities/2to2.pdf}
\end{center}
\end{minipage}
\caption{Equivalent proton collider energy. The left plot~\cite{Delahaye:2019omf}, assumes that $qq$ and~$gg$~partonic initial states both contribute to the production. In the right panel~\cite{AlAli:2021let}, production from $qq$ and from $gg$ are considered separately.
\label{pvsm}}
\end{figure*}

The general lesson we learn from the left panel of Figure~\ref{pvsm} (orange line) is that at a proton collider with around $100$~TeV energy the cross-section for processes with an energy threshold of around $10$~TeV is quite smaller than the one of a muon collider (MuC) operating at \parbox[c][0pt]{54pt}{$E_{\rm{cm}}={\sqrt{s\,}}_{\hspace{-2pt}\mu}$} $\sim10$~TeV. The gap can be compensated only if the process dynamics is different and more favourable at the proton collider, like in the case of QCD production. The general lesson has been illustrated for new heavy particles production, where the threshold is provided by the particle mass. But it also holds for the production of light SM particles with energies as high as $E_{\rm{cm}}$, which are very sensitive indirect probes of new physics. This makes exploration by high energy measurements more effective at muon than at proton colliders, as we will see in Section~\ref{HEM}. Moreover the large luminosity for high energy muon collisions produces the copious emission of effective vector bosons. In turn, they are responsible at once for the tremendous direct sensitivity of muon colliders to ``Higgs portal'' type new physics and for their excellent perspectives to measure single and double Higgs couplings precisely as we will see in Section~\ref{dirr} and~\ref{VBF}, respectively.

On the other hand, no quantitative conclusion can be drawn from Figure~\ref{pvsm} on the comparison between the muon and proton colliders discovery reach for the heavy particles. That assessment will be performed in the following section based on available proton colliders projections.

\subsection{Direct reach}\label{dirr}

The left panel of Figure~\ref{dr} displays the number of expected events, at a $10$~TeV MuC with $10$~ab$^{-1}$ integrated luminosity, for the pair production due to electroweak interactions of Beyond the Standard Model (BSM) particles with variable mass M. The particles are named with a standard BSM terminology, however the results do not depend on the detailed BSM model (such as Supersymmetry or Composite Higgs) in which these particles emerge, but only on their Lorentz and gauge quantum numbers. The dominant production mechanism at high mass is the direct $\mu^+\mu^-$ annihilation, whose cross-section flattens out below the kinematical threshold at ${\rm{M}}=5$~TeV. The cross-section increase at low mass is due to the production from effective vector boson annihilation.

The figure shows that with the target luminosity of $10$~ab$^{-1}$ a $10$~TeV MuC can produce the BSM particles abundantly. If they decay to energetic and detectable SM final states, the new particles can be definitely discovered up to the kinematical threshold. Taking into account that the entire target integrated luminosity will be collected in $5$ years, a few month run could be sufficient for a discovery. Afterwards, the large production rate will allow us to observe the new particles decaying in multiple final states and to measure kinematical distributions. We will thus be in the position of characterising the properties of the newly discovered states precisely. Similar considerations hold for muon colliders with higher $E_{\rm{cm}}$, up to the fact that the kinematical mass threshold obviously grows to $E_{\rm{cm}}/2$. Notice however that the production cross-section decreases as $1/E_{\rm{cm}}^2$.\footnote{The scaling is violated by the vector boson annihilation channel, which however is relevant only at low mass.} 
Therefore, we obtain as many events as in the left panel of Figure~\ref{dr} only if the integrated luminosity grows quadratically with the energy as in eq.~(\ref{lums}). A luminosity that is lower than this by a factor of around $10$ would not affect the discovery reach, but it might reduce the potential for characterising the discoveries.

\begin{figure*}
\begin{minipage}{0.52\textwidth}
\begin{center}
\includegraphics[width=\textwidth]{figures/opportunities/Pair10TeV.pdf}
\end{center}
\end{minipage}
\hfill
\begin{minipage}{0.38\textwidth}
\begin{center}
\includegraphics[width=\textwidth]{figures/opportunities/reach.pdf}
\end{center}
\end{minipage}
\caption{Left panel: the number of expected events (from Ref.~\cite{Buttazzo:2020uzc}) at a $10$~TeV MuC, with  $10$~ab$^{-1}$ luminosity, for several BSM particles. Right panel: $95\%$~CL mass reach, from Ref.~\cite{EuropeanStrategyforParticlePhysicsPreparatoryGroup:2019qin}, at the HL-LHC (solid bars) and at the FCC-hh (shaded bars). The tentative discovery reach of a 10, 14 and 30~TeV MuC are reported as horizontal lines.
\label{dr}}
\end{figure*}

The direct reach of muon colliders vastly and generically exceeds the sensitivity of the High-Luminosity LHC (HL-LHC). This is illustrated by the solid bars on the right panel of Figure~\ref{dr}, where we report the projected HL-LHC mass reach~\cite{EuropeanStrategyforParticlePhysicsPreparatoryGroup:2019qin} on several BSM states. The $95\%$~CL exclusion is reported, instead of the discovery, as a quantification of the physics reach. Specifically, we consider Composite Higgs fermionic top-partners $T$ (e.g., the \parbox[c][0pt]{21pt}{$X_{5/3}$} and the \parbox[c][0pt]{19pt}{$T_{2/3}$}) and supersymmetric particles such as stops~\raisebox{2pt}{\parbox[c][0pt]{6pt}{${\widetilde{t}}$}}, charginos \raisebox{2pt}{\parbox[c][0pt]{14pt}{${\widetilde{\chi}}_1^\pm$}}, stau leptons~\raisebox{2pt}{\parbox[c][0pt]{6pt}{${\widetilde{\tau}}$}} and squarks~\raisebox{1pt}{\parbox[c][0pt]{6pt}{${\widetilde{q}}$}}. For each particle we report the highest possible mass reach, as obtained in the configuration for the BSM particle couplings and decay chains that maximises the hadron colliders sensitivity. The reach of a $100$~TeV proton-proton collider (FCC-hh) is shown as shaded bars on the same plot. The muon collider reach, displayed as horizontal lines for $E_{\rm{cm}}=10$, $14$ and~$30$~TeV, exceeds the one of the FCC-hh for several BSM candidates and in particular, as expected, for purely electroweak charged states. It should be noted that detailed muon collider sensitivity projections for the BSM candidates in Figure~\ref{dr} have not been performed yet. In general, a relatively limited literature exists on direct new physics searches at the MuC~\cite{Buttazzo:2018qqp,Ruhdorfer:2019utl,Liu:2021jyc,Asadi:2021gah,Huang:2021nkl,Liu:2021akf,Qian:2021ihf,Bandyopadhyay:2021pld,Capdevilla:2020qel,Capdevilla:2021rwo,Dermisek:2021ajd,Capdevilla:2021kcf,Homiller:2022iax,Bottaro:2022one,Bottaro:2021srh,Mekala:2023diu,Li:2023tbx,Kwok:2023dck}. More studies would be desirable also to offer targets to the design of the detector.

Several interesting BSM particles do not decay to easily detectable final states, and an assessment of their observability requires dedicated studies. A clear case is the one of minimal WIMP Dark Matter (DM) candidates. The charged state in the DM electroweak multiplet is copiously produced, but it decays to the invisible DM plus a soft undetectable pion, owing to the small mass-splitting. WIMP DM can be studied at muon colliders in several channels (such as mono-photon) without directly observing the charged state~\cite{Han:2020uak,Bottaro:2021snn}. Alternatively, one can instead exploit the disappearing tracks produced by the charged particle~\cite{Capdevilla:2021fmj}. The result is displayed on the left panel of Figure~\ref{dmfig} for the simplest candidates, known as Higgsino and Wino. A $10$~TeV muon collider reaches the ``thermal'' mass, marked with a dashed line, for which the observed relic abundance is obtained by thermal freeze out. Other minimal WIMP candidates become kinematically accessible at higher muon collider energies~\cite{Han:2020uak,Bottaro:2021snn}. Muon colliders could actually even probe some of these candidates when they are above the kinematical threshold, by studying their indirect effects on high-energy SM processes~\cite{DiLuzio:2018jwd,Franceschini:2022sxc}. A more extensive overview of the muon collider potential to probe WIMP DM is provided in Section~\ref{secdm}.

New physics particles are not necessarily coupled to the SM by gauge interaction. One setup that is relevant in several BSM scenarios (including models of baryogenesis, dark matter, and neutral naturalness; see Section~\ref{sec:extendedHiggs}) is the ``Higgs portal'' one, where the BSM particles interact most strongly with the Higgs field. By the Goldstone Boson Equivalence Theorem, Higgs field couplings are interactions with the longitudinal polarisations of the SM massive vector bosons $W$ and $Z$, which enable Vector Boson Fusion (VBF) production of the new particles. A muon collider is extraordinarily sensitive to VBF production, owing to the large luminosity for effective vector bosons. This is illustrated on the right panel of Figure~\ref{dmfig}, in the context of a benchmark model~\cite{Buttazzo:2018qqp,AlAli:2021let} (see also \cite{Ruhdorfer:2019utl,Liu:2021jyc}) where the only new particle is a real scalar singlet with Higgs portal coupling. The coupling strength is traded for the strength of the mixing with the Higgs particle, $\sin\gamma$, that the interaction induces. The scalar singlet is the simplest extension of the Higgs sector. Extensions with richer structure, such as involving a second Higgs doublet, are a priori easier to detect as one can exploit the electroweak production of the new charged Higgs bosons, as well as their VBF production. See Refs.~\cite{Han:2021udl,Chakrabarty:2014pja,Kalinowski:2020rmb,Rodejohann:2010jh,Li:2023ksw} for dedicated studies, and Section~\ref{sec:extendedHiggs} for a review.

\begin{figure*}
\begin{minipage}{0.5\textwidth}
\begin{center}
\includegraphics[width=1\textwidth]{figures/opportunities/BarChartPlot.pdf}
\vspace{-4pt}
\end{center}
\end{minipage}
\hfill
\begin{minipage}{0.51\textwidth}
\begin{center}
\includegraphics[width=\textwidth]{figures/physics_studies/singlet_new.pdf}
\vspace{-18pt}
\end{center}
\end{minipage}
\caption{Left panel: exclusion and discovery mass reach on Higgsino and Wino dark matter candidates at muon colliders from disappearing tracks, and at other facilities. The plot is adapted from Ref.~\cite{Capdevilla:2021fmj}. Right: exclusion contour~\cite{AlAli:2021let} for a scalar singlet of mass $m_\phi$ mixed with the Higgs boson with strength $\sin\gamma$. More details in Section~\ref{sec:extendedHiggs}.
\label{dmfig}}
\end{figure*}

In several cases the muon collider direct reach compares favourably to the one of the most ambitious future proton collider project. This is not a universal statement, in particular at a muon collider it is obviously difficult to access heavy particles that carry only QCD interactions. One might also expect a muon collider of $10$~TeV to be generically less effective than a $100$~TeV proton collider for the detection of particles that can be produced singly. For instance, for additional $Z'$ massive vector bosons, that can be probed at the FCC-hh well above the $10$~TeV mass scale. We will see in Section~\ref{HEM} that the situation is slightly more complex and that, in the case of $Z'$s, a $10$~TeV MuC sensitivity actually exceeds the one of the FCC-hh in most of the parameter space (see the right panel of Figure~\ref{fig:HEM}).

\subsection{A vector bosons collider}\label{VBF}

When two electroweak charged particles like muons collide at an energy much above the electroweak scale $m_{{\rm{\textsc{w}}}}\sim100~$GeV, they have a high probability to emit electroWeak (EW) radiation. There are multiple types of EW radiation effects that can be observed at a muon collider and play a major role in muon collider physics. Actually we will argue in Section~\ref{EWRadiation} that the experimental observation and the theoretical description of these phenomena emerges as a self-standing reason of interest in muon colliders. 

Here we focus on the practical implications~\cite{Delahaye:2019omf,Han:2020uid,Costantini:2020stv,Han:2020pif,Buttazzo:2020uzc,AlAli:2021let,Forslund:2022xjq} of the collinear emission of nearly on-shell massive vector bosons, which is the analog in the EW context of the Weizs\"{a}cker--Williams emission of photons. The vector bosons $V$ participate, as depicted in Figure~\ref{fig:EV}, to a scattering process with a hard scale \parbox[c][0pt]{15pt}{$\sqrt{\hat{s}}$} that is much lower than the muon collision energy \parbox[c][0pt]{20pt}{$E_{\rm{cm}}$}. The typical cross-section for $VV$ annihilation processes is thus enhanced by \parbox[c][0pt]{30pt}{$E_{\rm{cm}}^2/{\hat{s}}$}, relative to the typical cross-section for $\mu^+\mu^-$ annihilation, whose hard scale is instead $E_{\rm{cm}}$. The emission of the $V$ bosons from the muons is suppressed by the EW coupling, but the suppression is mitigated or compensated by logarithms of the separation between the EW scale and $E_{\rm{cm}}$ (see~\cite{Han:2020uid,Costantini:2020stv,AlAli:2021let} for a pedagogical overview). The net result is a very large cross-section for VBF processes that occur at $\sqrt{\hat{s}}\sim m_{\rm{\textsc{w}}}$, with a tail in {$\sqrt{\hat{s}}$} up to the TeV scale.

We already emphasised (see Figure~\ref{dr}) the importance of VBF for the direct production of new physics particles. The relevance of VBF for probing new physics indirectly simply stems for the huge rate of VBF SM processes, summarised on the right panel of Figure~\ref{fig:EV}. In particular we see that a $10$~TeV muon collider produces ten million Higgs bosons, which is around $10$ times more than future $e^+e^-$ Higgs factories. Since the Higgs bosons are produced in a relatively clean environment, without large physics backgrounds from QCD, a $10$~TeV muon collider (over-)qualifies as a Higgs factory~\cite{Forslund:2022xjq,AlAli:2021let,Han:2020pif,Bartosik:2019dzq,Bartosik:2020xwr}. Unlike $e^+e^-$ Higgs factories, a muon collider also produces Higgs pairs copiously, enabling accurate and direct measurements of the Higgs trilinear coupling~\cite{Costantini:2020stv,Han:2020pif,Buttazzo:2020uzc} and possibly also of the quadrilinear coupling~\cite{Chiesa:2020awd}. 

The opportunities for Higgs physics at a muon collider are summarised extensively in Section~\ref{sec:Hc}. In Figure~\ref{tab:higgscouplingfit} we report for illustration the results of a 10-parameter fit to the Higgs couplings in the $\kappa$-framework at a $10$~TeV MuC, and the sensitivity projections on the anomalous Higgs trilinear coupling $\delta\kappa_\lambda$. The table shows that a $10$~TeV MuC will improve significantly and broadly our knowledge of the properties of the Higgs boson. The combination with the measurements performed at an $e^+e^-$ Higgs factory, reported on the third column, does not affect the sensitivity to several couplings appreciably, showing the good precision that a muon collider alone can attain. However, it also shows complementarity with an $e^+e^-$ Higgs factory program. 

\begin{figure*}
\centering{
\begin{minipage}{0.35\textwidth}
\vspace{-10pt}
\includegraphics[width=\textwidth]{figures/opportunities/vvfig.pdf}
\end{minipage}
\hspace{30pt}
\begin{minipage}{0.5\textwidth}
\vspace{0pt}
\includegraphics[width=0.95\textwidth]{figures/opportunities/SMEvs.pdf}
\end{minipage}}
\caption{Left panel: schematic representation of vector boson fusion or scattering processes. The collinear $V$ bosons emitted from the muons participate to a process with hardness \mbox{$\sqrt{\hat{s}}\ll E_{\rm{cm}}$}. Right panel: number of expected events for selected SM processes at a muon collider with variable $E_{\rm{cm}}$ and luminosity scaling as in eq.~(\ref{lums}). 
\label{fig:EV}}
\end{figure*}

On the right panel of the figure we see that the performances of muon colliders in the measurement of $\delta\kappa_\lambda$ are similar or much superior to the one of the other future colliders where this measurement could be performed. In particular, CLIC measures $\delta\kappa_\lambda$ at the $10\%$ level~\cite{deBlas:2018mhx}, and the FCC-hh sensitivity ranges from $3.5$ to $8\%$ depending on detector assumptions~\cite{Mangano:2020sao}. A determination of $\delta\kappa_\lambda$ that is way more accurate than the HL-LHC projections is possible already at a low energy stage of a muon collider with $E_{\rm{cm}}=3$~TeV as discussed in Section~\ref{sec:Hc}.

The potential of a muon collider as a vector boson collider has not been explored fully. In particular a systematic investigation of vector boson scattering processes, such as $WW\hspace{-3pt}\to\hspace{-3pt} WW$, has not been performed. The key role played by the Higgs boson to eliminate the energy growth of the corresponding Feynman amplitudes could be directly verified at a muon collider by means of differential measurements that extend well above one TeV for the invariant mass  of the scattered vector bosons. Along similar lines, differential measurements of the $WW\hspace{-3pt}\to\hspace{-3pt} HH$ process has been studied in~\cite{Buttazzo:2020uzc,Han:2020pif} (see also~\cite{Costantini:2020stv}) as an effective probe of the composite nature of the Higgs boson, with a reach that is comparable or superior to the one of Higgs coupling measurements. A similar investigation was performed in~\cite{AlAli:2021let,Costantini:2020stv} (see also~\cite{Costantini:2020stv}) for $WW\hspace{-3pt}\to\hspace{-3pt} t{\overline{t}}$, aimed at probing Higgs-top interactions.

\subsection{High-energy measurements}\label{HEM}

Direct $\mu^+\mu^-$ annihilation, such as $HZ$ and $t{\overline{t}}$ production, displays a number of expected events of the order of several thousands, reported in Figure~\ref{fig:EV}. These are much less than the events where a Higgs or a $t{\overline{t}}$ pair are produced from VBF, but they are sharply different and easily distinguishable. The invariant mass of the particles produced by direct annihilation is indeed sharply peaked at the collider energy $E_{\rm{cm}}$, while the invariant mass rarely exceeds one tenth of $E_{\rm{cm}}$ in the VBF production mode. 

The good statistics and the limited or absent background thus enables few-percent level measurements of SM cross sections for hard scattering processes of energy $E_{\rm{cm}}=10$~TeV at the 10~TeV MuC. An incomplete list of the many possible measurements is provided in Ref.~\cite{Chen:2022msz}, including the resummed effects of EW radiation on the cross section predictions. It is worth emphasising that also charged final states such as $WH$ or $\ell\nu$ are copiously produced at a muon collider. The electric charge mismatch with the neutral $\mu^+\mu^-$ initial state is compensated by the emission of soft and collinear $W$ bosons, which occurs with high probability because of the large energy.

High energy scattering processes are as unique theoretically as they are experimentally~\cite{Delahaye:2019omf,Buttazzo:2020uzc,Chen:2022msz}. They give direct access to the interactions among SM particles with $10$~TeV energy, which in turn provide indirect sensitivity to new particles at the $100$~TeV scale of mass. In fact, the effects on high-energy cross sections of new physics at energy $\Lambda\gg E_{\rm{cm}}$ generically scale as $(E_{\rm{cm}}/\Lambda)^2$ relative to the SM. Percent-level measurements thus give access to $\Lambda\sim100$~TeV. This is an unprecedented reach for new physics theories endowed with a reasonable flavor structure. Notice in passing that high-energy measurements are also useful to investigate flavor non-universal phenomena, as we will see in Section~\ref{muonspec}.

\begin{figure*}
\begin{minipage}{0.55\textwidth}
\renewcommand{\arraystretch}{.89}
\setlength{\arrayrulewidth}{.2mm}
\setlength{\tabcolsep}{0.6 em}
\begin{center}
\begin{tabular}{c|c|c|c}
%\multicolumn{4}{c}{Fit Result [\%]} \\ \hline
\hline
& \multicolumn{1}{c|}{\makebox[35pt]{\small{HL-LHC}}} & \makebox[35pt]{\small{HL-LHC}} & \makebox[35pt]{\small{HL-LHC}} \\[0pt]
& \multicolumn{1}{c|}{\ } & \multicolumn{1}{l|}{\makebox[35pt]{+\small{$10\,\textrm{{TeV}}$}}} & \multicolumn{1}{l}{\makebox[35pt]{+\small{$10\,\textrm{{TeV}}$}}}  \\[-2pt]
\ & \ & \multicolumn{1}{c|}{\ } & \multicolumn{1}{l}{\hspace{0.5pt}+
{$e e$}} \\ 
\hline
$\kappa_W$ & 1.7 & 0.1 & 0.1 \\ \hline
$\kappa_Z$ & 1.5 & 0.4 & 0.1 \\ \hline
$\kappa_g$ & 2.3 & 0.7 & 0.6 \\ \hline
$\kappa_{\gamma}$ & 1.9 & 0.8 & 0.8 \\ \hline
$\kappa_{Z\gamma}$ & 10 & 7.2 & 7.1 \\ \hline 
$\kappa_c$ & - & 2.3 & 1.1 \\ \hline
$\kappa_b$ & 3.6 & 0.4 & 0.4\\ \hline
$\kappa_{\mu}$ & 4.6 & 3.4 & 3.2 \\ \hline
$\kappa_{\tau}$ & 1.9 & 0.6 & 0.4 \\ \hline\hline
%
$\kappa_t^*$ & 3.3 & 3.1 & 3.1 \\ \hline
\multicolumn{4}{l}{ {\scriptsize $^*$ No input used for the MuC}}\\
\end{tabular}
\end{center}
\end{minipage}
\hfill
\begin{minipage}{0.45\textwidth}
\vspace{0pt}
\includegraphics[width=.93\textwidth]{figures/opportunities/dk3.pdf}
\end{minipage}
\caption{Left panel: $1\sigma$ sensitivities (in \%) from a 10-parameter fit in the $\kappa$-framework at a $10$~TeV MuC with $10$~ab$^{-1}$, compared with HL-LHC. The effect of measurements from a $250$~GeV $e^+e^-$ Higgs factory is also reported. Right panel: sensitivity to $\delta\kappa_\lambda$ for different $E_{\rm{cm}}$. The luminosity is as in eq.~(\ref{lums}) for all energies, apart from $E_{\rm{cm}}\hspace{-2pt}=\hspace{-2pt}3$~TeV, where doubled luminosity (of 2~ab$^{-1}$) is assumed. More details in Section~\ref{sec:Hc}.
\label{tab:higgscouplingfit}}
\end{figure*}

This mechanism is not novel. Major progress in particle physics always came from raising the available collision energy, producing either direct or indirect discoveries. Among the most relevant discoveries that did not proceed through the resonant production of new particles, there is the one of the inner structure of nucleons. This discovery could be achieved~\cite{nobel} only when the transferred energy in electron scattering could reach a significant fraction of the proton compositeness scale $\Lambda_{\rm{\textsc{qcd}}}=1/r_{p}=300$~MeV. Proton-compositeness effects became sizeable enough to be detected at that energy, precisely because of the quadratic enhancement mechanism we described above.

Figure~\ref{fig:HEM} illustrates the tremendous reach on new physics of a $10$~TeV MuC with $10$~ab$^{-1}$ integrated luminosity. The left panel (green contour) is the sensitivity to a scenario that explains the microscopic origin of the Higgs particle and of the scale of EW symmetry breaking by the fact that the Higgs is a composite particle. In the same scenario the top quark is likely to be composite as well, which in turn explains its large mass and suggest a ``partial compositeness'' origin of the SM flavour structure. Top quark compositeness produces additional signatures that extend the muon collider sensitivity up to the red contour. The sensitivity is reported in the plane formed by the typical coupling $g_*$ and of the typical mass $m_*$ of the composite sector that delivers the Higgs. The scale $m_*$ physically corresponds to the inverse of the geometric size of the Higgs particle. The coupling $g_*$ is limited from around $1$ to $4\pi$, as in the figure. In the worst case scenario of intermediate $g_*$, a $10$~TeV MuC can thus probe the Higgs radius up to the inverse of $50$~TeV, or discover that the Higgs is as tiny as $(35$~TeV$)^{-1}$. The sensitivity improves in proportion to the centre of mass energy of the muon collider. 

The figure also reports, as blue dash-dotted lines denoted as ``Others'', the envelop of the $95\%$~CL sensitivity projections of all the future collider projects that have been considered for the~2020 update of the European Strategy for Particle Physics, summarised in Ref.~\cite{EuropeanStrategyforParticlePhysicsPreparatoryGroup:2019qin}. These lines include in particular the sensitivity of very accurate measurements at the EW scale performed at possible future $e^+e^-$ Higgs, electroweak and Top factories. These measurements are not competitive because new physics at $\Lambda\sim100$~TeV produces unobservable one part per million effects on $100$~GeV energy processes. High-energy measurements at a $100$~TeV proton collider are also included in the dash-dotted lines. They are not competitive either, because the effective parton luminosity at high energy is much lower than the one of a $10$~TeV MuC, as explained in Section~\ref{whym}. For example the cross-section for the production of an $e^+e^-$ pair with more than $9$~TeV invariant mass at the FCC-hh is only $40$~ab, while it is $900$~ab at a $10$~TeV muon collider. Even with a somewhat higher integrated luminosity, the FCC-hh just does not have enough statistics to compete with a $10$~TeV MuC.

The right panel of Figure~\ref{fig:HEM} considers a simpler new physics scenario, where the only BSM state is a heavy $Z'$ spin-one particle. The ``Others'' line also includes the sensitivity of the FCC-hh from direct $Z'$ production. The line exceeds the $10$~TeV MuC sensitivity contour (in green) only in a tiny region with $M_{Z'}$ around $20$~TeV and small $Z'$ coupling. This result substantiates our claim in Section~\ref{dirr} that a reach comparison based on the $2\to 1$ single production of the new states is simplistic. Single $2\to 1$ production couplings can produce indirect effect in $2\to 2$ scattering by the virtual exchange of the new particle, and the muon collider is extraordinarily sensitive to these effects. Which collider wins is model-dependent. In the simple benchmark $Z'$ scenario, and in the motivated framework of Higgs compositeness that future colliders are urged to explore, the muon collider is just a superior device.

\begin{figure*}
\begin{minipage}{0.47\textwidth}
\includegraphics[width=0.96\textwidth]{figures/opportunities/10TeVCTEC.pdf}
\end{minipage}
\hfill
\begin{minipage}{0.47\textwidth}
\vspace{0pt}
\includegraphics[width=0.96\textwidth]{figures/opportunities/Zprime10TeV.pdf}
\end{minipage}
\caption{Left panel: $95\%$ reach on the Composite Higgs scenario from high-energy measurements in di-boson and di-fermion final states~\cite{Chen:2022msz}. The green contour display the sensitivity from ``Universal'' effects related with the composite nature of the Higgs boson and not of the top quark. The red contour includes the effects of top compositeness. Right panel: sensitivity to a minimal $Z'$~\cite{Chen:2022msz}. Discovery contours at $5\sigma$ are also reported in both panels. 
\label{fig:HEM}}
\end{figure*}

We have seen that high energy measurements at a muon collider enable the indirect discovery of new physics at a scale in the ballpark of $100$~TeV. However the muon collider also offers amazing opportunities for direct discoveries at a mass of several TeV, and unique opportunities to characterise the properties of the discovered particles, as emphasised in Section~\ref{dirr}. High energy measurements will enable us take one step further in the discovery characterisation, by probing the interactions of the new particles well above their mass. For instance in the Composite Higgs scenario one could first discover Top Partner particles of few TeV mass, and next study their dynamics and their indirect effects on SM processes. This might be sufficient to pin down the detailed theoretical description of the newly discovered sector, which would thus be both discovered and theoretically characterised at the same collider. Higgs coupling determinations and other precise measurements that exploit the enormous luminosity for vector boson collisions, described in Section~\ref{VBF}, will also play a major role in this endeavour. 

We can dream of such glorious outcome of the project, where an entire new sector is discovered and characterised in details at the same machine, only because energy and precision are simultaneously available at a muon collider. 

\subsection{Electroweak radiation}\label{EWRadiation}

The novel experimental setup offered by lepton collisions at $10$~TeV energy or more outlines possibilities for theoretical exploration that are at once novel and speculative, yet robustly anchored to reality and to phenomenological applications. 

The muon collider will probe for the first time a new regime of EW interactions, where the scale $m_{{\rm{\textsc{w}}}}\hspace{-2pt}\sim\hspace{-2pt}100~$GeV of EW symmetry breaking plays the role of a small IR scale, relative to the much larger collision energy. This large scale separation triggers a number of novel phenomena that we collectively denote as ``EW radiation'' effects. Since they are prominent at muon collider energies, the comprehension of these phenomena is of utmost importance not only for developing a correct physical picture but also to achieve the needed accuracy of the theoretical predictions.

The EW radiation effects that the muon collider will observe, which will play a crucial role in the assessment of its sensitivity to new physics, can be broadly divided in two classes. 

The first class includes the emission of low-virtuality vector bosons from the initial muons. It  effectively makes the muon collider a high-luminosity vector boson collider, on top of a very high-energy lepton-lepton machine. The compelling associated physics studies described in Section~\ref{VBF} pose challenges for  fixed-order theoretical predictions and Monte Carlo event generation even at tree-level, owing to the sharp features of the Monte Carlo integrand induced by the large scale separation and the need to correctly handle QED and weak radiation at the same time, respecting EW gauge invariance. Strategies to address these challenges are available in  {\texttt{WHIZARD}}~\cite{Kilian:2007gr}, they have been recently implemented in {\texttt{MadGraph5\_aMC@NLO}}~\cite{Costantini:2020stv,Ruiz:2021tdt} and applied to several phenomenological studies in the muon collider context. Dominance of such initial-state collinear radiation will eventually require a systematic theoretical modelling in terms of EW Parton Distribution Function where multiple collinear radiation effects are resummed. First studies show that EW resummation effects can be significant at a 10~TeV MuC~\cite{Han:2020uid}.

The second class of effects are the virtual and real emissions of soft and soft-collinear EW radiation. They affect most strongly the measurements performed at the highest energy, described in Section~\ref{HEM}, and impact the corresponding cross-section predictions at order one~\cite{Chen:2022msz}. They also give rise to novel processes such as the copious production of charged hard final states out of the neutral $\mu^+\mu^-$ initial state, and to new opportunities to detect new short distance physics by studying, for one given hard final state, different patterns of radiation emission~\cite{Chen:2022msz}. The deep connection with the sensitivity to new physics makes the study of EW radiation an inherently multidisciplinary enterprise that overcomes the traditional separation between ``SM background'' and ``BSM signal'' studies. 

At very high energies EW radiation displays similarities with QCD and QED radiation, but also remarkable differences that pose profound theoretical challenges. 

First, being EW symmetry broken at low energy, different particles in the same EW multiplet---i.e., with different “EW color” like the $W$ and the $Z$---are distinguishable.
%First, since EW symmetry is broken at low energy, particles with different ``EW color'' are  distinguishable. 
In particular the beam particles (e.g., charged left-handed leptons) carry definite colour thus violating the KLN theorem assumptions. Therefore, no cancellation takes place between virtual and real radiation contributions, regardless of the final state observable inclusiveness~\cite{Ciafaloni:2000df,Ciafaloni:2000rp}. Furthermore, the EW colour of the final state particles can be measured, and it must be measured for a sufficiently accurate exploration of the SM and BSM dynamics. For instance, distinguishing the top from the bottom quark or the $W$ from the $Z$ boson (or photon) is necessary to probe accurately and comprehensively new short-distance physical laws that can affect the dynamics of the different particles differently. When dealing with QCD and QED radiation only, it is sufficient instead to consider ``safe'' observables where QCD/QED radiation effects can be systematically accounted for and organised in well-behaved (small) corrections. The relevant observables for EW physics at high energy are on the contrary dramatically affected by EW radiation effects. 

Second, in analogy with QCD and unlike QED, for EW radiation the IR scale is physical. However, at variance with QCD, the theory is weakly-coupled at the IR scale, and the EW ``partons'' do not ``hadronise''. EW~showering therefore always ends at virtualities of order 100~GeV, where  heavy EW states $(t,W,Z,H)$ coexist with light SM ones, and then decay.  Having a complete and consistent description of the evolution from high virtualities where EW symmetry is restored, to the weak scale where it is broken, to GeV scales, including also leading QED/QCD effects in all regimes is a new challenge~\cite{Han:2021kes}. 

Such a strong phenomenological motivation, and the peculiarities of the problem, stimulate work and offer a new perspective on resummation and showering techniques, or more in general trigger theoretical progress on IR physics. Fixed-order calculations will also play an important role. Indeed while the resummation of the leading logarithmic effects of radiation is mandatory at muon collider energies~\cite{Chen:2022msz,Bauer:2018xag}, subleading logarithms could perhaps be included at fixed order. Furthermore one should eventually develop a description where resummation is merged with fixed-order calculations in an exclusive way, providing the most accurate predictions in the corresponding regions of the phase space, as currently done for QCD computations. 

A significant literature on EW radiation exists, starting from the earliest works on double-logarithm resummations based on Asymptotic Dynamics~\cite{Ciafaloni:2000df,Ciafaloni:2000rp} or on the IR evolution equation~\cite{Fadin:1999bq,Melles:2000gw}. The factorisation of virtual massive vector boson emissions, leading to the notion of effective vector boson is also known since long~\cite{Kane:1984bb,Dawson:1984gx,Chanowitz:1985hj,Kunszt:1987tk}. More recent progress includes resummation at the next to leading log in the Soft-Collinear Effective Theory framework~\cite{Chiu:2007yn,Chiu:2007dg,Chiu:2009ft,Manohar:2014vxa,Manohar:2018kfx}, the operatorial definition of the distribution functions for EW partons~\cite{Bauer:2017isx,Fornal:2018znf,Bauer:2018xag} and the calculation of the corresponding evolution, as well as the calculation of the EW splitting functions~\cite{Chen:2016wkt} for EW showering and the proof of collinear EW emission factorisation~\cite{Borel:2012by,Wulzer:2013mza,Cuomo:2019siu}. Additionally, fixed-order virtual EW logarithms are known for generic process at the $1$-loop order~\cite{Denner:2000jv,Denner:2001gw} and are implemented in {\texttt{Sherpa}}~\cite{Bothmann:2020sxm} and  {\texttt{MadGraph5\_aMC@NLO}}~\cite{Pagani:2021vyk}. Exact EW corrections at NLO are available in an automatic form for arbitrary processes in the SM, for example in the {\texttt{MadGraph5\_aMC@NLO}}~\cite{Frederix:2018nkq} and in the {\texttt{Sherpa+Recola}}~\cite{Biedermann:2017yoi} packages or using {\texttt{WHIZARD+Recola}}~\cite{Bredt:2022dmm}. Implementations of EW showering are also available through a limited set of splittings in {\texttt{Pythia 8}}~\cite{Christiansen:2014kba,Christiansen:2015jpa} and in a complete way in {\texttt{Vincia}}~\cite{Brooks:2021kji}.

\begin{figure}[t]
\begin{center}
\includegraphics[width=0.5\textwidth]
{figures/physics_studies/reach_gm2_tan.pdf}
\end{center}
\caption{Summary, from Section~\ref{muonspec} of the muon collider perspectives to probe the muon $g$-2 anomaly. \label{fan}}
\end{figure}

While we are still far from an accurate systematic understanding of EW radiation, the present-day knowledge is sufficient to enable rapid progress in the next few years. The outcome will be an indispensable toolkit for muon collider predictions. Moreover, while we do expect that EW radiation phenomena can in principle be described by the Standard Model, they still qualify as  ``new phenomena'' until when we will be able to control the accuracy of the predictions and verify them experimentally. Such investigation is a self-standing reason of scientific interest in the muon collider project.

\subsection{Muon-specific opportunities\label{muspec}}

In the quest for generic exploration, engineering collisions between muons and anti-muons is in itself a unique opportunity. The concept can be made concrete by considering scenarios where the sensitivity to new physics stems from colliding muons, rather than electrons or other particles. An overview of such ``muon-specific'' opportunities is provided in Section~\ref{muonspec} based on the available literature~\cite{AlAli:2021let,Chakrabarty:2014pja,Capdevilla:2020qel,Buttazzo:2020ibd,Yin:2020afe,Capdevilla:2021rwo,Dermisek:2021ajd,Dermisek:2021mhi,Capdevilla:2021kcf,Huang:2021biu,Asadi:2021gah,Huang:2021nkl,Homiller:2022iax,Casarsa:2021rud,Han:2021lnp,Azatov:2022itm,Liu:2021akf,Cesarotti:2022ttv,Altmannshofer:2022xri,Bause:2021prv,Qian:2021ihf,Allanach:2022blr,Bandyopadhyay:2021pld,Dermisek:2022aec,Paradisi:2022vqp,Ruzi:2023atl,Qian:2022wxa}. A brief discussion is reported below.

It is worth emphasising in this context that lepton flavour universality is not a fundamental property of Nature. Therefore new physics could exist, coupled to muons, that we could not yet discover using electrons. In fact, it is not only conceivable, but even expected that new physics could couple more strongly to muons than to electrons. Even in the SM lepton flavour universality is violated maximally by the Yukawa interaction with the Higgs field, which is larger for muons than for electrons. New physics associated to the Higgs or to flavour could follow the same pattern, offering a competitive advantage to muon over electron collisions at similar energies. The comparison with proton colliders is less straightforward. By the same type of considerations one expects larger couplings with quarks, especially with the ones of the second and third generation. This expectation should be folded in with the much lower luminosity for heavier quarks at proton colliders than for muons at a muon collider. The perspectives of muon versus proton colliders are model-dependent and of course strongly dependent on the energy of the muon and of the proton collider.

Recently experimental anomalies in $g$-2 and in $B$-meson physics measurements triggered numerous studies of muon-philic new physics. These results provide interesting quantitative illustrations of the generic added value for exploration of a collider that employs second-generation particles. They show the muon collider potential to probe new physics that is presently untested because it couples mostly to muons. These models, and others with the same property, will still exist---though in a slightly different region of their parameter space---even if the anomalies will be explained by SM physics as the most recent LHCb results suggest for the $B$-meson anomalies~\cite{LHCb:2022qnv,LHCb:2022zom}.

Illustrative results are reported in Figure~\ref{fan}, displaying the minimal muon collider energy that is needed to probe different types of new physics potentially responsible for the $g$-2 anomaly. The solid lines correspond to limits on contact interaction operators due to unspecified new physics, that contribute at the same time to the muon $g$-2 and to high-energy scattering processes. Semi-leptonic muon-charm (muon-top) interactions that can account for the $g$-$2$ discrepancy can be probed by di-jets at a $3$~TeV ($10$~TeV) MuC, whereas a 30~TeV collider could even probe a tree-level contribution to the muon electromagnetic dipole operator directly through $\mu\mu\to h\gamma$. These sensitivity estimates are agnostic on the specific new physics model responsible for the anomaly. Explicit models typically predict light particles that can be directly discovered at the muon collider, and correlated deviations in additional observables. We will see in Section~\ref{muonspec} that a complete coverage of several models that accommodate the current discrepancy is possible already at a $3$~TeV MuC, and a collider of tens of TeV could provide a full-fledged no-lose theorem.