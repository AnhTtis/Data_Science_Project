\section{More Examples of CHEEM Learned Sequentially and Continually}
\label{sec:other-architectures}

Fig.~\ref{fig:vdd_arch_seq1_hee} shows the CHEEM learned  sequentially and continually  via the proposed HEE-based NAS on the VDD benchmark~\cite{vdd} with three different random seeds.


As comparisons, Fig. \ref{fig:vdd_arch_seq1_pe} shows the structure updates learned using the vanilla PE-based NAS. It can be seen that pure exploration does not reuse components from similar tasks. 
The pure exploration based method adds many unnecessary {\tt Adapt} and {\tt New} operations even though the tasks are similar (e.g., ImNet → C100), verifying the effectiveness of the proposed sampling method. While the pure exploration scheme adds many {\tt Skip} operations, thereby reducing the overall FLOPs, the average accuracy is low by a large margin, about 6\%. This shows that the pure exploration scheme cannot learn to choose operations in a task synergy aware way.

\newpage

\begin{figure}[h!]
    \centering
    \begin{subfigure}[t]{0.99\textwidth}
        \centering
        \includegraphics[width=\textwidth]{figures/vdd-order-1.pdf}
    \end{subfigure}
    ~
    \begin{subfigure}[t]{1.0\textwidth}
        \centering
        \includegraphics[width=\textwidth]{figures/structure-ee-100-4242.pdf}
    \end{subfigure}
    \vspace{.2em}
    ~
    \begin{subfigure}[t]{1.0\textwidth}
        \centering
        \includegraphics[width=\textwidth]{figures/structure-ee-100-42.pdf}
    \end{subfigure}
    \vspace{.2em}
    ~
    \begin{subfigure}[t]{1.0\textwidth}
        \centering
        \includegraphics[width=\textwidth]{figures/structure-ee-100-424242.pdf}
    \end{subfigure}
    \caption{Examples of the task-synergy memory (CHEEM) learned on the VDD benchmark~\cite{vdd} with the task sequence shown in the top \textbf{using our proposed HEE-based NAS} and three different random seeds. The overall performance is reported in Table~\ref{tab:vdd-results} in the main paper. \colorbox{skip}{S}, \colorbox{reuse}{R}, \colorbox{adapt}{A} and \colorbox{new}{N} represent {\tt Skip}, {\tt Reuse}, {\tt Adapt} and {\tt New} respectively. The first one (the 2nd row) is also shown in Fig.~\ref{fig:vdd_arch-sim} in the main paper.  The last two columns show the number of new task-specific parameters and added FLOPs respectively, in comparison with the first task, ImNet model. Overall, the learned task synergies make intutive sense and remain relatively stable across different random seeds. }
    \label{fig:vdd_arch_seq1_hee} 
\end{figure}

\FloatBarrier

\begin{figure}[h!]
    \centering
    \begin{subfigure}[t]{0.99\textwidth}
        \centering
        \includegraphics[width=\textwidth]{figures/vdd-order-1.pdf}
    \end{subfigure}
    ~
    \begin{subfigure}[t]{1.0\textwidth}
        \centering
        \includegraphics[width=\textwidth]{figures/structure-e-100-42.pdf}
    \end{subfigure}\vspace{.2em}
    ~
    \begin{subfigure}[t]{1.0\textwidth}
        \centering
        \includegraphics[width=\textwidth]{figures/structure-e-100-4242.pdf}
    \end{subfigure}\vspace{.2em}
    ~
    \begin{subfigure}[t]{1.0\textwidth}
        \centering
        \includegraphics[width=\textwidth]{figures/structure-e-100-424242.pdf}
    \end{subfigure}
    \caption{Examples of the task-synergy memory (CHEEM) learned on the VDD benchmark~\cite{vdd} with the task sequence shown in the top using the vanilla PE-based NAS and three different random seeds. \colorbox{skip}{S}, \colorbox{reuse}{R}, \colorbox{adapt}{A} and \colorbox{new}{N} represent {\tt Skip}, {\tt Reuse}, {\tt Adapt} and {\tt New} respectively. The overall performance is reported in Table~\ref{tab:uniform-hier} in the main paper and our proposed HEE-based NAS significantly improves the performance by an absolute 6\% average accuracy. In terms of the learned CHEEM, the PE-based NAS leads to much more {\tt New} operations, which shows it is less effective in terms of leveraging task synergies.}
    \label{fig:vdd_arch_seq1_pe} 
\end{figure}
\newpage