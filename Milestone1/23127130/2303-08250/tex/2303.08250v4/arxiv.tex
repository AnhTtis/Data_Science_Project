

\documentclass{article} %
\usepackage{arxiv,times}

\newcommand{\bbox}{\text{bbox}}
\newcommand{\alphapck}{\alpha_\bbox}
\newcommand{\kcycle}{\text{k-CyPCK}}
\newcommand{\cycle}{\text{-CyPCK}}

\newcommand{\I}{\mathbf{I}}
\newcommand{\Ia}{\I^\text{a}}
\newcommand{\Ib}{\I^\text{b}}
\newcommand{\Iatob}{\I^\text{a $\rightarrow$ b}}
\newcommand{\F}{\mathbf{F}}
\newcommand{\Fa}{\F^\text{a}}
\newcommand{\Fb}{\F^\text{b}}
\newcommand{\f}{\mathbf{f}}
\newcommand{\fa}{\f^\text{a}}
\newcommand{\fb}{\f^\text{b}}
\newcommand{\p}{\mathbf{p}}
\newcommand{\pa}{\p^\text{a}}
\newcommand{\pb}{\p^\text{b}}
\newcommand{\A}{\boldsymbol{\Phi}_\text{align}}
\newcommand{\G}{\mathbf{G}}
\newcommand{\C}{\mathbf{C}}
\newcommand{\Ca}{\C^\text{a}}
\newcommand{\Cb}{\C^\text{b}}
\newcommand{\cc}{\mathbf{c}}
\newcommand{\cca}{\cc^\text{a}}
\newcommand{\ccb}{\cc^\text{b}}
\newcommand{\Irec}{\I_\text{Recon}}
\newcommand{\M}{\mathbf{M}}
\newcommand{\Mrec}{\M_\text{Recon}}
\newcommand{\loss}{\mathcal{L}}
\newcommand{\T}{\mathcal{T}}
\newcommand{\W}{\mathcal{W}}
\newcommand{\Id}{\mathcal{I}}


\usepackage[utf8]{inputenc} %
\usepackage[T1]{fontenc}    %
\usepackage{hyperref}       %
\usepackage{url}            %
\usepackage{booktabs}       %
\usepackage{amsfonts}       %
\usepackage{nicefrac}       %
\usepackage{microtype}      %
\usepackage{xcolor}         %

\usepackage{booktabs}       %
\usepackage{xcolor}         %
\usepackage{wrapfig} %
\usepackage{amsmath} %
\usepackage{amssymb} %
\usepackage{enumitem} %
\usepackage{caption}
\usepackage{subcaption}
\usepackage{multirow}
\usepackage{multicol}
\usepackage{ulem} %
\usepackage{float}

\usepackage[
]{sidecap}   
\sidecaptionvpos{figure}{t} 

\setcounter{secnumdepth}{3}

\usepackage{graphicx}
\usepackage{pgffor}
\usepackage{subcaption} 
\usepackage{float}
\usepackage{url}            %
\usepackage{booktabs}       %
\usepackage{graphicx}
\usepackage{amsmath}
\usepackage{amssymb}
\usepackage{multicol}
\usepackage{multirow}
\usepackage{enumitem} %
\usepackage{array}
\usepackage{epigraph} 
\usepackage{placeins}


\usepackage{cuted}
\usepackage{capt-of}
\usepackage{pgfplots}
\usepackage{colortbl}

\usepackage{circledtext}
\usepackage{etoolbox}

\usepackage{tikz}
\usetikzlibrary{positioning,shapes}

\usepackage{caption}
\usepackage{subcaption}
\usepackage{colortbl}

\usepackage{placeins}

\definecolor{skip}{HTML}{F2ACCA}
\definecolor{reuse}{HTML}{9CCEA7}
\definecolor{adapt}{HTML}{FEB24C}
\definecolor{new}{HTML}{9EC9E2}


\title{Continual Learning via Learning a Continual Memory in Vision Transformer}
\makeatletter
\let\@fnsymbol\@arabic
\makeatother
\author{Chinmay Savadikar\thanks{North Carolina State University, $^2$Princeton University} \\
\texttt{csavadi@ncsu.edu}\\
\And
Michelle Dai$^2$ \\
\texttt{mdai@alumni.princeton.edu} \\
\And
Tianfu Wu$^1$ \\
\texttt{tianfu\_wu@ncsu.edu}
}


\newcommand{\fix}{\marginpar{FIX}}
\newcommand{\new}{\marginpar{NEW}}

\iclrfinalcopy %
\begin{document}


\maketitle

\begin{abstract}
This paper studies task-incremental continual learning (TCL) using Vision Transformers (ViTs).  Our goal is to improve the overall streaming-task performance without catastrophic forgetting by learning task synergies (e.g., a new task learns to automatically reuse/adapt modules from previous similar tasks, or to introduce new modules when needed, or to skip some modules when it appears to be an easier task). 
One grand challenge is how to tame ViTs at streaming diverse tasks in terms of balancing  their plasticity and stability in a task-aware way while overcoming the catastrophic forgetting. 
To address the challenge, we propose a simple yet effective approach that identifies a lightweight yet expressive ``sweet spot'' in the ViT block as the task-synergy memory in TCL.
We present a Hierarchical task-synergy Exploration-Exploitation (HEE) sampling based neural architecture search (NAS) method for effectively learning task synergies by structurally updating the identified memory component with respect to four basic operations ({\tt reuse, adapt, new and skip}) at streaming tasks.
The proposed method is thus dubbed as CHEEM (Continual Hierarchical-Exploration-Exploitation Memory). 
In experiments, we test the proposed CHEEM on the challenging Visual Domain Decathlon (VDD) benchmark and the 5-Dataset benchmark. It obtains consistently better performance than the prior art with sensible CHEEM learned continually. 
\end{abstract}

\setlength{\abovecaptionskip}{1pt}
\setlength{\belowcaptionskip}{-6pt}
\setlength{\belowdisplayskip}{1pt} \setlength{\belowdisplayshortskip}{1pt}
\setlength{\abovedisplayskip}{1pt} \setlength{\abovedisplayshortskip}{1pt} 

\section{Introduction}
\label{sec:introduction}
% \begin{itemize}
%     % Diffusion of FL
%     \item {\st{Diffusion of FL}}
%     % Security threats to FL
%     \item {\st{Security threats to FL with particular focus on model poisoning}}
%     % Limitations of existing countermeasures
%     \item {\st{Current countermeasures (e.g., KRUM) and their limitations}}
%     % Proposed method and its advantages
%     \item {\st{Intuitive description of the proposed method and its difference (i.e., advantages) w.r.t. state of the art}}
%     % Main contributions
%     \item {\st{Summary of the main contributions of this work}}
%     % Paper's structure and organization
%     \item {\st{Paper's structure and organization}}
% \end{itemize}

% Diffusion of FL
Recently, {\em federated learning} (FL) has emerged as the leading paradigm for training distributed, large-scale, and privacy-preserving machine learning (ML) systems~\cite{mcmahan2017googleai,mcmahan2017aistats}. 
The core idea of FL is to allow multiple edge clients to collaboratively train a shared, global model without disclosing their local private training data.
%Specifically, an FL system consists of a central server and many edge clients; 
A typical FL round involves the following steps: {\em(i)} the server randomly picks some clients and sends them the current, global model; {\em(ii)} each selected client locally trains its model with its own private data; then, it sends the resulting local model to the server;\footnote{Whenever we refer to global/local model, we mean global/local model {\em parameters}.} {\em(iii)} the server updates the global model by computing an \emph{aggregation function}, usually the average (FedAvg), on the local models received from clients.
% \begin{enumerate}
%     \item[{\em(i)}] the server sends the current, global model to the clients and appoints some of them for training;
%     \item[{\em(ii)}] each selected client locally trains its copy of the global model with its own private data; then, it sends the resulting local model back to the server;\footnote{Whenever we refer to global/local model, we mean global/local model {\em parameters}.}
%     \item[{\em(iii)}] the server updates the global model by computing an \emph{aggregation function} on the local models received from clients (by default, the average, also referred to as FedAvg~\cite{mcmahan2017aistats}).
% \end{enumerate}
This process goes on until the global model converges. %(e.g., after a certain number of rounds or other similar stopping criteria).
%\\
% The advantages of FL over the traditional, centralized learning paradigm are undoubtedly clear in terms of flexibility/scalability (clients can join/disconnect from the FL network dynamically), network communications (only model weights\footnote{We will use \textit{parameters} and \textit{weights} interchangeably.} are exchanged between clients and server), and privacy (each client's private training data is kept local at the client's end and not uploaded to the server).
\\
% Security threats to FL
%However, the growing adoption of FL also raises security concerns~\cite{costa2022covert}, particularly about its confidentiality, integrity, and availability.
Although its advantages over standard ML, FL also raises security concerns~\cite{costa2022covert}. %, particularly about its confidentiality, integrity, and availability~\cite{costa2022covert}.
% OLD, LONG VERSION
% Indeed, some work deals with privacy leakage that may expose the local data of some clients~\cite{melis2019sp}. 
% A large body of work, instead, investigates attacks that usually aim to detriment the predictive accuracy of the learned global model. For instance, \emph{data poisoning} attacks achieve this goal by letting an adversary pollute the training set of some corrupt FL clients with maliciously crafted examples~\cite{jagielski2018sp}.
% Similarly, in \emph{model poisoning} the attacker attempts to tweak the global model weights~\cite{bhagoji2019pmlr} by directly perturbing the local model's weights of some infected FL clients before these are sent to the central server for aggregation, usually via so-called Byzantine attacks. 
% It turns out that Byzantine model poisoning attacks severely impact standard FedAvg; therefore, more robust aggregation functions must be designed to make FL systems secure.
Here, we focus on \emph{untargeted model poisoning} attacks~\cite{bhagoji2019pmlr}, where an adversary attempts to tweak the global model weights %\footnote{We will use the terms \textit{parameters} and \textit{weights} interchangeably.} 
by directly perturbing the local model's parameters of some infected clients before these are sent to the central server for aggregation.
In doing so, the adversary aims to jeopardize the global model \textit{indiscriminately} at inference time.
Such model poisoning attacks severely impact standard FedAvg; therefore, more robust aggregation functions must be designed to secure FL systems.
\\
% In this paper, we focus on designing a novel robust aggregation scheme at the server's end to contrast the effect of Byzantine model poisoning attacks.
%
% Current countermeasures and their limitations
%Several countermeasures have been proposed in the literature to combat model poisoning attacks on FL systems.
% Some methods use simple statistics more robust than plain average to smooth the impact of malicious updates (e.g., Trimmed Mean and FedMedian~\cite{yin2018icml}). 
% Other defenses implement outlier detection techniques to discard malicious updates from the aggregation performed at the server's end. Those are either based on heuristics (e.g., Krum/Multi-Krum~\cite{blanchard2017nips} and Bulyan~\cite{mhamdi2018pmlr}) or data-driven approaches (e.g., K-means clustering~\cite{shen2016acm} or DnC via spectral analysis~\cite{shejwalkar2021ndss}). 
% Finally, some strategies rely on a centralized ``source of trust'' to spot potential malicious updates (e.g., FLTrust~\cite{cao2020fltrust}).
% Several countermeasures have been proposed in the literature to combat model poisoning attacks on FL systems, i.e., to discard possible malicious local updates from the aggregation performed at the server's end. 
% These techniques range from simple statistics more robust than plain average (e.g., Trimmed Mean and FedMedian~\cite{yin2018icml}) to outlier detection heuristics (e.g., Krum/Multi-Krum~\cite{blanchard2017nips} and Bulyan~\cite{mhamdi2018pmlr}) or data-driven approaches (e.g., spectral analysis via K-means clustering~\cite{shen2016acm} or spectral analysis), or methods based on ``source of trust'' (e.g., FLTrust~\cite{cao2020fltrust}).
% OLD, LONG VERSION
%Several countermeasures have been proposed in the literature to combat Byzantine model poisoning attacks on FL systems.
% Descriptive statistics
% For example, Trimmed Mean and FedMedian aggregate local model updates using more robust statistics than standard average~\cite{yin2018icml}.
%
% % Heuristics for outlier detection
% Many existing Byzantine-resilient strategies implement some outlier detection heuristics to discard the model updates sent by potentially malicious clients from the input of the aggregation function.
% One of the most popular heuristics is Krum~\cite{blanchard2017nips}.
% This strategy tries to mitigate the impact of Byzantine attacks by selecting as a global model the local model with the smallest sum of Euclidean distances to {\em all} the other local models.
% Although powerful, Krum requires the server to know (or, at least, estimate) the number of malicious FL clients upfront, which is generally impossible in a realistic attack scenario. %
% Moreover, Krum may become ineffective for complex, high-dimensional model parameter spaces due to the curse of dimensionality.
% Bulyan~\cite{mhamdi2018pmlr} tries to overcome this issue by combining Krum with a variant of Trimmed Mean.
% % Data-driven outlier detection
% Other strategies use data-driven outlier detection techniques -- e.g., via K-means clustering~\cite{shen2016acm} -- to spot potential malicious local model updates. 
% %For instance, Shen et al. propose to cluster local model updates with K-means and thus identify outliers.
%
% % Other techniques
% As far as the server is concerned, any local model received can be from a potential malicious client. 
% FLTrust~\cite{cao2020fltrust} assumes the server acts as a client, i.e., trains a local model on an additional {\em trustworthy} dataset at the server's end and compares it against all the local models from other clients. 
% This way, the server can rely on some ``source of trust'' when discarding potentially malicious clients.
%\\
% Limitations of existing Byzantine-resilient strategies
Unfortunately, existing defense mechanisms either rely on simple heuristics (e.g., Trimmed Mean and FedMedian by~\cite{yin2018icml}) or need strong and unrealistic assumptions to work effectively (e.g., foreknowledge or estimation of the number of malicious clients in the FL system, as for Krum/Multi-Krum~\cite{blanchard2017nips} and Bulyan~\cite{mhamdi2018pmlr}, which, however, cannot exceed a fixed threshold).
Furthermore, outlier detection methods using K-means clustering~\cite{shen2016acm} or spectral analysis like DnC~\cite{shejwalkar2021ndss} do not directly consider the temporal evolution of local model updates received.
Finally, strategies like FLTrust~\cite{cao2020fltrust} require the server to collect its own dataset and act as a proper client, thereby altering the standard FL protocol.
\\
% OLD, LONG VERSION
% Overall, existing Byzantine-resilient strategies are either simple heuristics (e.g., FedMedian) or, if they are more complex, they rely on strong and unrealistic assumptions to work effectively (e.g., knowing the number of malicious clients in the FL system in advance, as for Krum and alike).
% Furthermore, data-driven outlier detection methods do not consider the temporary evolution of local model updates received (e.g., K-means clustering). 
% Finally, strategies like FLTrust requires the server to collect its own dataset and act as a proper client, thereby altering the standard FL protocol.
%
% Description of the proposed method
This work introduces a novel pre-aggregation \textit{filter} robust to untargeted model poisoning attacks. Notably, this filter $(i)$ operates without requiring prior knowledge or constraints on the number of malicious clients and $(ii)$ inherently integrates temporal dependencies. 
The FL server can employ this filter as a preprocessing step before applying \textit{any} aggregation function, be it standard like FedAvg or robust like Krum or Bulyan.
Specifically, we formulate the problem of identifying corrupted updates as a multidimensional (i.e., matrix-valued) time series anomaly detection task. 
The key idea is that legitimate local updates, resulting from well-calibrated iterative procedures like stochastic gradient descent (SGD) with an appropriate learning rate, show \textit{higher predictability} compared to malicious updates. This hypothesis stems from the fact that the sequence of gradients (thus, model parameters) observed during legitimate training exhibit regular patterns, as validated in Section~\ref{subsec:intuition}. %until convergence. 
%This regularity may be more pronounced for smooth convex loss functions, but it can still be captured within an appropriate time window, even for more complex and convoluted loss surfaces. 
%We provide evidence of this claim in Appendix~B, where we show that the average mutual information (i.e., ``predictability''), calculated over pairs of legitimate model updates sent at different FL rounds, is significantly higher than the corresponding computation for a malicious client.
\\
Inspired by the matrix autoregressive (MAR) framework for multidimensional time series forecasting~\cite{chen2021je}, we propose the FLANDERS ({\em \textbf{F}ederated \textbf{L}earning meets \textbf{AN}omaly \textbf{DE}tection for a \textbf{R}obust and \textbf{S}ecure}) filter.
The main advantages of FLANDERS over existing strategies like FLDetector~\cite{zhao2020multivariate} are its resilience to large-scale attacks, where $50\%$ or more FL participants are hostile, and the capability of working under realistic non-iid scenarios.
We attribute such a capability to two key factors: $(i)$ FLANDERS works without knowing a priori the ratio of corrupted clients, and $(ii)$ it embodies temporal dependencies between intra- and inter-client updates, quickly recognizing local model drifts caused by evil players. Below, we summarize our main contributions:

\begin{itemize}
\item[{\em(i)}]
We provide empirical evidence that the sequence of models sent by legitimate clients is more predictable than those of malicious participants performing untargeted model poisoning attacks.
\\
\item[{\em(ii)}] 
We introduce FLANDERS, the first pre-aggregation filter for FL robust to untargeted model poisoning based on multidimensional time series anomaly detection.
\\
\item[{\em(iii)}] 
We integrate FLANDERS into Flower,\footnote{\scriptsize{\url{https://flower.dev/}}} a popular FL simulation framework for reproducibility.
\\
\item[{\em(iv)}] 
We show that FLANDERS improves the robustness of the existing aggregation methods under multiple settings: different datasets, client's data distribution (non-iid), models, and attack scenarios.
\\
\item[{\em(v)}] 
We publicly release all the implementation code of FLANDERS along with our experiments.\footnote{\scriptsize{\url{https://anonymous.4open.science/r/flanders_exp-7EEB}}}
\end{itemize}

% Paper's structure and organization
The remainder of the paper is structured as follows. %some related work and the current state-of-the-art solutions to security issues that FL entails. 
Section~\ref{sec:background} covers background and preliminaries. 
In Section~\ref{sec:related}, we discuss related work.
Section~\ref{sec:problem} and Section~\ref{sec:method} describe the problem formulation and the method proposed. % to tackle it. 
Section~\ref{sec:experiments} gathers experimental results. %, and Section~\ref{sec:limitations} discusses some limitations of this work.
Finally, we conclude in Section~\ref{sec:conclusion}.
 %discusses the limitations of this work and draws future research directions.
%reports conclusions and draws perspectives for future research directions.

%%%%%%% OLD %%%%%%%
%to overcome the resilience of Byzantine failures in distributed Stochastic Gradient Descent computations. 
% The strength of Krum is its time complexity, which is linear in the gradient dimension. 
% However, the robustness of the approach is guaranteed for gradient-based learning applications only when the majority of the clients are not compromised. 
% Besides, the aggregation mechanism of Krum, as well as that of similar methods, is robust from a coarse-grained perspective and does not provide solutions to errors and perturbations that may occur at inference time.
%A related approach to~\cite{blanchard2017nips} is the work of Su et al.~\cite{su2016dc}. Here, the authors propose an iterated approximate agreement to tackle a multi-layer scenario attacked by Byzantine agents. 
%However, the method works efficiently on the sole discrete context and it is inapplicable to continuous state environments.
%\gabri{Maybe, we should just talk about the main limitations of existing countermeasures without digging into their details (or, we can just mention Krum as this is the most popular one). I will move the description of all these methods to the Related Work section.}

\subsection{Approach}
This paper describes three concepts with the aim 
of a robust, energy-efficient robot control. 
While these concepts are rather straightforward and intuitive, they 
are not yet utilised in mainstream manipulator control.
It is not argued that all of these principles 
need to be used, but if the (sub)task allows it, using any 
of these principles can have a positive impact on the energy-efficiency.

\vspace{-4mm}
\subsubsection{Contextual prior knowledge:}
When humans perform a transportation task, they do  
not perform strict PTP motions such as traditional industrial 
robots. Instead, movements with a certain tolerance on the position 
are performed. 
This allows the natural dynamics of the system to be exploited, 
as will be explained in the following subsection. Typically, the 
tolerances come from 
knowledge about both the environment and the task context. For example, 
the spatial constraints, 
fragility of the payload, 
if a certain part of the task requires a higher precision, 
etc. 
It is clear that this knowledge precedes the task execution and 
determines how the human will perform the task.
The execution is generally done in multiple states, e.g., picking up 
the payload, moving and placing near the target position, making small 
adjustments when necessary.

This knowledge is used to split up the task in multiple 
subtasks and identify the different requirements. Robust 
controllers and monitors are then developed to perform and coordinate 
between these subtasks. 
Examples of such requirements are crane like operations such as:
 lifting the load to a certain 
height, transporting it without colliding, and lowering the load 
until contact is made.

The task also does not require high control precision throughout, 
but only for the initial grasping and final placement. 
In addition, this does not need to come only from the 
control.
Geometric constraints such as the environment or a previously placed
payload can be used to achieve this accuracy by sliding against them. 
This is further explained in section \ref{sec:discrete_control}.

By using this knowledge, lower-cost (and often also lower-weight) 
hardware can be used, so that a more robust, 
energy efficient execution can be developed. 
Thus, for a repetitive task, the cost of 
designing and implementing a task-specific controller is not 
necessarily higher than a generic, less energy-efficient controller.\\

\vspace{-8mm}
\subsubsection{Exploiting natural dynamics}
In this work, the natural dynamics of the system are used to inject 
as little energy as possible, resulting in energy-efficient 
motions.
However, precise control of the timing is lost when the system freely
follows its natural dynamics.

Due to the layout of the used cable robot (Fig.1), when the end effector is
in a fully constrained position, releasing the power of one (or more) 
of the motors, will result in a pendulum-like swing around the 
cables that are still powered, or braked. 
This swing is used in the control strategy to cover the horizontal 
distance while consuming a minimal amount of energy. 

\vspace{-4mm}
\subsubsection{Active use of brakes}
Based on the context, certain subtasks may occur where a joint 
does not need to move. Instead of producing a constant standstill
torque, it can also be opted to brake the joint.
Another case occurs when the demanded motion is in line with 
external forces such as gravity. In case of a continuous brake,
the brake force can be directly controlled to achieve a certain 
resulting force. 
With a discrete brake, a tolerance region can be determined between  
which the brake switches on-and-off to achieve a similar effect.
Section \ref{sec:continuous_control} utilises this concept to drop the
payload without driving the motor. The brakes can also be used to stop 
the natural dynamics, if necessary.
We present in section~\ref{ssec:faces} an application of PnP-HVAE on face images, using a pretrained state-of-the-art hierarchical VAE. 
Next, we study the application of our framework to natural images. To that end, we introduce  in section~\ref{ssec:patchVDVAE}  a patch hierachical VAE architecture, that is able to model natural images of different resolutions. In section~\ref{ssec:app_nat}, we provide deblurring, super-resolution and inpainting experiments to demonstrate the relevance of the proposed method.

Additional results are presented in Appendix~\ref{app:add}. All experiments can be reproduced using the code available at \url{https://github.com/jprost76/PnP-HVAE}.



\subsection{Face Image restoration (FFHQ)}\label{ssec:faces}
We first demonstrate the effectiveness of PnP-HVAE on highly structured data, by performing face image restoration.
Latent variable generative models can accurately model structured images such as face images \cite{karras2019style,vahdat2020nvae,child2021very,kingma2018glow}, and then be used to produce high quality restoration of such data. 
In our experiments, we use the VDVAE model of~\cite{child2021very}, pre-trained on the FFHQ dataset~\cite{karras2019style}, as our hierarchical VAE prior.
VDVAE has $L=66$ latent variable groups in its hierarchy and generates images at resolution $256\times256$.

We compare PnP-HVAE with the intermediate layer optimization algorithm (ILO)~\cite{daras2021intermediate} that is based on a different class of generative models than HVAE. ILO is a GAN inversion method which optimizes the image latent code along with the intermediate layer representation of a StyleGAN to generate an image consistent with a degraded observation.
We use the official implementation of ILO, along with a StyleGAN2 model~\cite{karras2020analyzing, stylegan2pytorch}, that was trained for 550k iterations on images of resolution $256\times256$ from FFHQ.  
As VDVAE and StyleGAN models are not trained on the same train-test split of FFHQ, we chose to evaluate the methods on a subset of 100 images from the CelebA dataset~\cite{liu2018large}. 
For super-resolution, the degradation model corresponds to the application of a gaussian low-pass filter followed by a $\times 4$ sub-sampling, and the addition of a gaussian white noise with $\sigma=3$.
For the deblurring, we considered motion blur and  gaussian kernels, both with a noise level $\sigma=8$. %

We provide quantitative comparisons in table~\ref{table:comp_ILO}, along with a visual comparison of the results in figure~\ref{fig:face_restoration}.
PnP-HVAE has the best  PSNR and SSIM results for all the considered restoration tasks, while ILO provides better results  for the perceptual distance.
By jointly optimizing the image and its latent variable, PnP-HVAE provides  results that are both realistic and consistent with the degraded observation.
On the other hand,  ILO  only optimizes on an extended latent space. This method generates  sharp and realistic images with better LPIPS scores,   
but the results lack  of consistency with respect to the observation, which explains the overall lower PSNR performance. 






\subsection{PatchVDVAE: a HVAE for natural images}\label{ssec:patchVDVAE}
Available generative models in the literature operate on images of  fixed resolutions and
are either restrained to datasets of limited diversity, or even to registered face images~\cite{kingma2018glow,child2021very, vahdat2020nvae, karras2019style}, or requiring additional class information~\cite{brock2018large, dhariwal2021diffusion, song2020score, luhman2022optimizing}.
Fitting an unconditional model on natural images appears to be a more difficult task, as their resolution can change, and their content is highly diverse.
The complexity of the problem can be reduced by learning a prior model on patches of reduced dimension. 
For image restoration problems, the patch model can be reused on images of higher dimensions~\cite{zoran2011learning,prost2021learning,altekruger2022patchnr}. When the model is a full CNN, the prior on the set of the  patches can  be computed efficiently by applying the network on the full image~\cite{prost2021learning}.

We thus introduce  patchVDVAE, a fully convolutional hierarchical VAE.
Contrary to existing HVAE models whose resolution is constrained by the constant tensor at the input of the top-down block, patchVDVAE can generate images of different resolutions by controlling the dimension of the input latent. 
This amounts to defining a prior on patches whose dimension corresponds to the receptive field of the VAE. A similar model is used for image denoising in~\cite{prakash2021interpretable}.

 
For PatchVDVAE architecture, we use the same bottom-up and top-down blocks as VDVAE~\cite{child2021very}, and replace the constant trainable input in the first top-down block by a latent variable, to make the model fully convolutional (details on the  architecture are given in Appendix~\ref{app:details}). 
The training dataset is composed of $128\times 128$ patches extracted from a combination of DIV2K~\cite{agustsson2017ntire} and Flickr2K~\cite{Lim_2017_CVPR_workshops} datasets.
We perform data augmentation by extracting  patches at $3$ resolutions: HR-images and $\times 2$ and $\times 4$ downscaled images. 
The model is trained for $7.10^5$ iterations with a batch size of $64$. Following the recommendation of~\cite{hazami2022efficient}, we use Adamax optimizer with an exponential moving average and gradient smoothing of the variance.
We set the decoder model to be a gaussian with diagonal covariance, as in~\cite{luhman2022optimizing}.
PatchVDVAE is fully convolutional and can generate images of dimension that are multiples of $64$ as illustrated by
figure~\ref{fig:vdvae}.

\newlength{\patchwidth}
\setlength{\patchwidth}{0.135\columnwidth}
\begin{figure}[!ht]
    \centering
    \begin{subfigure}[t]{.34\columnwidth}\hspace{0.1cm}
        \setlength{\tabcolsep}{0.02pt}
\renewcommand{\arraystretch}{0}
        \begin{tabular}{*{2}{p{1.03\patchwidth}}}
            \includegraphics[width=\patchwidth]{figures_arxiv/patchVDVAE/samples/generated/64x64/setup-5-image-0018.png} &
            \includegraphics[width=\patchwidth]{figures_arxiv/patchVDVAE/samples/generated/64x64/setup-5-image-0016.png} \\
            \includegraphics[width=\patchwidth]{figures_arxiv/patchVDVAE/samples/generated/64x64/setup-5-image-0008.png} &
            \includegraphics[width=\patchwidth]{figures_arxiv/patchVDVAE/samples/generated/64x64/setup-5-image-0019.png}   
        \end{tabular}
    \end{subfigure}\hspace{-0.15cm}
    \begin{subfigure}[t]{.64\columnwidth}
\begin{tabular}{cc}\vspace{-0.1cm}
\includegraphics[width=2\patchwidth]{figures_arxiv/patchVDVAE/samples/generated/256x256/setup-2-image-0009.png}&
        \includegraphics[width=2\patchwidth]{figures_arxiv/patchVDVAE/samples/generated/256x256/setup-2-image-0002.png}\end{tabular}

    \end{subfigure}
    \caption{\label{fig:vdvae} Left: $64\times64$ patches samples from our patchVDVAE model trained on patches from natural images.
    Right: PatchVDVAE is fully convolutional and it can generate images of higher resolution (here: $128\times128$).\vspace{-0.2cm}}
\end{figure}

\subsection{Natural images restoration}\label{ssec:app_nat}
We  evaluate PnP-HVAE on natural image restoration.
For each task, we report the average value of the PSNR, the SSIM, and the LPIPS metrics on $20$ images from the test set of the BSD dataset~\cite{MartinFTM01}.\\


\noindent
{\bf Image deblurring.}
In the experiments, we consider $2$ gaussian kernels and $2$ motion blur kernels from~\cite{levin2009understanding}, with $3$ different noise levels 
$\sigma \in \{2.55, 7.65, 12.75\}$.
As a baseline we consider  EPLL~\cite{zoran2011learning}, which learns a prior on image patches with a gaussian mixture model.
We also compare PnP-HVAE  with PnP-MMO and GS-PnP, $2$ competing convergent Plug-and-Play methods based on CNN denoisers.
PnP-MMO~\cite{pesquet2021learning} restricts the denoiser to be contraction in order to guarantee the convergence of the PnP forward-backard algorithm. GS-PnP~\cite{hurault2022gradient} considers a gradient step denoiser and reaches state-of-the-art performances of non converging methods~\cite{zhang2021plug}.
We set the temperature $\tau$  in our method as $0.95$, $0.8$ and $0.6$ for noise levels $2.55$, $7.65$ and $12.75$ respectively, and we let it run for a maximum of $50$ iterations. 
For the three compared methods we use the official implementations and pre-trained models provided by the respective authors. 
Details on the choice of hyperparameters for the concurrent methods are provided in the Appendix~\ref{app:details}
Figure~\ref{fig:deblurring_bsd} illustrates that our method provides correct deblurring results. 

According to table~\ref{tab:deb}, the performance of PnP-HVAE is between those of EPLL and GS-PnP and it outperforms PnP-MMO for large noise levels.\\

\begin{table}
\begin{center}\footnotesize
    \begin{tabular}{>{\centering}m{.3cm}*{5}{c}}
    $\sigma$ &Method & PSNR$\uparrow$ & SSIM$\uparrow$ & LPIPS$\downarrow$  \\ 
    \hline
    \multirow{4}{*}{\vcell{$2.55$}}
    & PnP-HVAE & $27.75$ & $0.79$ & $0.31$\\
    & GS-PNP \cite{hurault2022gradient} & $\mathbf{29.59}$ & $\mathbf{0.84}$ & $\mathbf{0.22}$\\
    & EPLL \cite{zoran2011learning} & $26.49$ & $0.71$ & $0.36$\\ 
    & PnP-MMO \cite{pesquet2021learning} & $\underbar{29.50}$ & $\underbar{0.83}$ & $\underbar{0.20}$ \\ \hline
    \multirow{4}{*}{\vcell{$7.65$}}
    & PnP-HVAE & $\underbar{26.36}$ & $\underbar{0.72}$ & $\underbar{0.40}$\\
    & GS-PNP \cite{hurault2022gradient} & $\mathbf{27.33}$ & $\mathbf{0.77}$ & $\mathbf{0.31}$\\
    & EPLL \cite{zoran2011learning} & $24.04$ & $0.66$ & $0.45$ \\ 
    & PnP-MMO \cite{pesquet2021learning} & $25.34$ & $0.69$ & $0.34$\\
    \hline
    \multirow{4}{*}{\vcell{$12.75$}}
    & PnP-HVAE & $\underbar{25.12}$ & $\mathbf{0.73}$ & $\underbar{0.47}$\\
    & GS-PNP \cite{hurault2022gradient} & $\mathbf{26.32}$ & $\mathbf{0.73}$ & $\mathbf{0.37}$\\
    & EPLL \cite{zoran2011learning} & $23.28$ & $0.61$ & $0.51$ \\ 
    & PnP-MMO \cite{pesquet2021learning} & $22.42$ & $0.53$& $0.54$ \\
    \hline
    &\vspace*{-.3cm}\\
            \multicolumn{2}{c}{Blur and motion kernels}& \multicolumn{3}{c}{
        \includegraphics*[scale=1]{figures_arxiv/kernels/4.png}\;\includegraphics*[scale=1]{figures_arxiv/kernels/7.png}\;\includegraphics*[scale=1]{figures_arxiv/kernels/9.png}\;\includegraphics*[scale=1]{figures_arxiv/kernels/11.png}} 
    \end{tabular}
        \caption{\label{tab:deb}Comparison  of PnP-HVAE  and other restoration methods on deblurring. Results are averaged on $4$ kernels.\vspace{-0.2cm}}% on image deblurring.}
    \end{center}
\end{table}

\begin{figure}
    
    \begin{subfigure}[h]{\linewidth}
        \centering
        \includegraphics*[width=\columnwidth]{figures_arxiv/deb_s255_k7.pdf}\vspace{-0.1cm}
        \caption{Gaussian blur, $\sigma=2.55$}
    \end{subfigure}
    \begin{subfigure}[h]{\linewidth}
        \centering
        \includegraphics*[width=\columnwidth]{figures_arxiv/deb_s765_k11.pdf}\vspace{-0.1cm}
        \caption{Motion blur, $\sigma=7.65$}
    \end{subfigure}\vspace*{-0.1cm}
    \caption{\label{fig:deblurring_bsd} Natural image deblurring\vspace{-0.1cm}}
\end{figure}

\noindent {\bf Effect of the temperature.}
PnP-HVAE gives control on the temperature of the prior over the latent space.
In figure~\ref{fig:temp_effect}, we illustrate that reducing the temperature increases the strength of the regularization prior. In this example the tuning $\tau=0.7$ produces the best performance.\\
\begin{figure}[!ht]
   
    \includegraphics[width=\columnwidth]{figures_arxiv/demo_temp.pdf}\vspace{-0.15cm}
    \caption{ \label{fig:temp_effect} Effect of the temperature in PnP-VAE on a deblurring problem, with $\sigma=7.65$.\vspace{-0.15cm}}
\end{figure}


\noindent
{\bf Image inpainting.}
Next we consider the task of noisy image inpainting. 
We compose a test-set of 10 images from the validation set of BSD~\cite{MartinFTM01} and we create masks
  by occluding diverse objects of small size in the images. 
A gaussian white noise with $\sigma=3$ is added to the images.
As a comparaison, we still consider GS-PnP and EPLL.
For PnP-HVAE, the temperature is set to $\tau=0.6$, and the algorithm is run for a maximum of $200$ iterations, unless the residual $||\x_{k+1}-\x_k||$ is on a plateau.
We provide on Table~\ref{tab:inpainting_bsd} the distortion metrics with the ground truth, as well as a visual
\begin{table}



\begin{center}
    \begin{tabular}{cccc}
        & PSNR$\uparrow$ & SSIM$\uparrow$ &LPIPS$\downarrow$ \\\hline
        PnP-HVAE  & $\mathbf{29.54}$ & $\mathbf{0.93}$ & $\mathbf{0.06}$\\
        GS-PNP & $28.52$ & $\mathbf{0.93}$ & $0.09$\\
        EPLL & $\underline{29.16}$ & $\mathbf{0.93}$ & $\mathbf{0.06}$\\
    \end{tabular}
    \caption{\label{tab:inpainting_bsd}Quantitative evaluation for inpainting on BSD.}
    \end{center}
\end{table}
comparison on figure~\ref{fig:inpainting_bsd}. 
With its hierarchical structure,  PnP-HVAE outperforms the compared methods. \vspace{0.05cm}



\begin{figure}[!h]
    \includegraphics[width=\columnwidth]{figures_arxiv/demo_inp_bsd2.pdf}\vspace{-0.1cm}
    \caption{\label{fig:inpainting_bsd}Natural image inpainting\vspace{-0.3cm}}
\end{figure}











\section{Related Work and Our Contributions} \vspace{-2mm}
\textit{Experience Replay Based approaches} aim to retain some exemplars from the previous tasks and replay them to the model along with the data from the current task~\citep{gradient-based-sample-selection,mir,large-scale-er,rainbow-memory,hindsight,agem,continual-prototype-evolution,remind,gem,gdumb,icarl,tiny-replay,hayes-icra,dark-experience-replay,large-scale-inc-learning,fast-and-slow,contrastive-continual-learning}. Instead of storing raw exemplars, \textit{Generative replay methods}~\citep{generative-replay,gan-memory} learn the generative process for the data of a task, and replay exemplars sampled from that process along with the data from the current task. For exemplar-free continual learning, \textit{Regularization Based approaches} explicitly control the plasticity of the model by preventing the parameters of the model from deviating too far from their stable values learned on the previous tasks when learning the current task~\citep{what-not-to-forget,selfless-sequential-learning,podnet,variational-continual-learning,kirkpatrick-overcoming,lwf,synaptic-intelligence,progress-and-compress}. Both these approaches aim to balance the stability and plasticity of a fixed-capacity model.

\textit{Dynamic Models} aim to use different parameters per task to avoid use of stored exemplars. Dynamically Expandable Network~\citep{dynamic-expandable-nets} adds neurons to a network based on learned sparsity constraints and heuristic loss thresholds. PathNet~\citep{pathnet} finds task-specific submodules from a dense network, and only trains submodules not used by other tasks. Progressive Neural Networks~\citep{pnn} learn a new network per task and adds lateral connections to the previous tasks' networks.~\citep{vdd} learns residual adapters which are added between the convolutional and batch normalization layers. \citep{network-of-experts} learns an expert network per task by transferring the expert network from the most related previous task. The L2G~\citep{learn-to-grow} uses Differentiable Architecture Search (DARTS~\citep{darts}) to determine if a layer can be reused, adapted, or renewed (3 fundamental skills: {\tt reuse}, {\tt adapt}, {\tt new}) for a task. Our approach is most closely related to Learn to Grow~\citep{learn-to-grow} which can also be interpreted as a Mixture of Experts framework. \citet{task-driven-priors} use task priors derived from a task similarity measure and use those to train a stochastic network and retain some layers of the most similar task, and retrain other layers. \citep{wang-task-difficulty-aware} use a task difficulty metric and threshold hyperparameters to either impose regularization constraints on the previous network, to use the same architecture as the previous tasks, or learn an entirely new architecture and parameters using NAS. Although similar to our method, they rely on (fixed) manually chosen threshold, whereas our method does not have any such heuristics. Dynamic models have also been explored for efficient transfer learning~\citep{nettailor,spottune,piggyback}.

Recently, there has been increasing interest in lifelong learning using Vision Transformers~\citep{learning-to-prompt,dualprompt,meta-attention,pool-of-adapters,dytox,towards-exemplar-free-continual-learning-vits,improving-vits,continual-obj-det-kd,memory-transformer,s-prompts}. \textit{Prompt Based approaches} learn external parameters appended to the data tokens that encode task-specific information useful for classification~\citep{learning-to-prompt,s-prompts,dytox}. Learning to Prompt (L2P)~\citep{learning-to-prompt} learns a pool of prompts and uses a key-value based retrieval to retrieve the correct set of prompts at test time. DualPrompt~\citep{dualprompt} learns generic and task-specific prompts and extends Learning to Prompt.~\citep{meta-attention} uses a ViT pretrained on ImageNet and learns binary masks to enable/disable parameters of the Feedforward Network (FFN), and the attention between image tokens for downstream tasks.

\textbf{Our Contributions} 
We make four main contributions to the field of lifelong learning with ViTs. (i) We propose and identify ArtiHippo in ViTs, i.e., the final projection layers of the multi-head self-attention blocks in a ViT.
We also present a new usage for the class-token in ViTs as the memory growing guidance. (ii) We present a hierarchical task-similarity-oriented exploration-exploitation-sampling-based NAS method for learning to grow ArtiHippo continually with respect to four basic growing operations: {\tt Skip}, {\tt Reuse}, {\tt Adapt}, and {\tt New} to overcome catstrophic forgetting. (iii) We are the first, to the best of our knowledge, to evaluate lifelong learning with ViTs on the large-scale, diverse and imbalanced VDD benchmark~\citep{vdd} with strong empirical performance obtained. We also materialize several state-of-the-art lifelong learning methods that were developed for ConvNets with ViTs on VDD for a comprehensive study.   
(iv) We show that our method is complementary to prompting-based approaches, and combining the two leads to  higher performance.
\section{Conclusion}
This paper presents a method of transforming Vision Transformers (ViTs) for resilient task-incremental continual learning (TCL) with catastrophic forgetting overcome. It identifies  the final projection layers of the multi-head self-attention blocks as the task-synergy memory in ViTs, which is then updated in a task-aware way using four operations, {\tt Skip}, {\tt Reuse}, {\tt Adapt} and {\tt New}. The learning of task-synergy memory is realized by a proposed hierarchical exploration-exploitation sampling based single-path one-short Neural Architecture Search algorithm, where the exploitation utilizes task similarities defined by the normalized cosine similarity between the mean class tokens of a new task and those of old tasks. The proposed method is dubbed as CHEEM (Continual Hierarchical-Exploration-Exploitation Memory). In experiments, the proposed method is tested on the challenging VDD and the 5-Datasets benchmarks.  
It obtains better performance than the prior art with sensible CHEEM learned continually. 
We also take great efforts in materializing several state-of-the-art baseline methods for ViTs and tested on the VDD, which are released in our code. 





\chapter*{Acknowledgement}
\addcontentsline{toc}{chapter}{Acknowledgement}
The authors thank Andrzej Kupsc, Sergey Barsuk, Olivier Callot and Wolfgang K{\"u}hn for their contribution on the CDR draft.
%The authors thank the international review committee XXX for their great effort in reading the CDR draft and providing valuable suggestions. 
The STCF working group thanks all 
the colleagues in the world-wide community for many profitable discussions
and expresses gratitude to the Hefei Comprehensive National Science Center for their strong support.  This work is supported by: international 
partnership program of the Chinese Academy of Sciences Grant No. 211134KYSB20200057.


\bibliography{iclr2025_conference}
\bibliographystyle{iclr2025_conference}

\newpage
\appendix
\section{Applications}
\label{sec:apps}
To demonstrate the wide range of usagages of our model, we implement a series of applications:
\begin{enumerate}
	\item Incremental surface \& color reconstruction
	\item 3D saliency detection
	\item Open vocabulary scene understanding
	\item Surface infrared field
	\item 3D style transfer
\end{enumerate}
Originating from our motivation in inspection and service robotics, we implement 1) Incremental surface \& color reconstruction for visualization of robot surroundings.
For robot exploration, we implement 2) 3D saliency detection to indicate the salient regions in maps.
For recovering object-level semantic information in environments, we implement 3) open vocabulary scene understanding to yield the regions containing the objects..
Furthermore, to demonstrate the flexibility, we implement 4) surface infrared fields and 5) 3D style transfer for artistic purposes. 

In~\cref{fig:latent_diff}, we classify those 3 applications into 3 categories: (a) directly obtaining the properties from sensor observation, such as application 1) and 4). (b) processing on sensor data and predict properties, such as application 2), 5). (c) extending (b) to operating beyond latent features, such as application 3).
%Thus, in the following, we discuss about those categories of applications.
% we mainly describe the application 1) (\cref{sec:incremental_reconstruction}) and 3) (\cref{sec:openvoc}).

Application 1) and 4) are in the first one category. Thus, we mainly describe 1) incremental surface \& color reconstruction (\cref{sec:incremental_reconstruction}), while for 4) we can easily exchange color with infrared.
%
For the second with 2) and 5) in~\cref{sec:fabircated_prop}, we mainly describe the usage of fabricated properties.
As the mapping part is redundant to previous category, it will not be detailed.
%
The third category is the application 3) that maps a LIM for high dimensional latent fields.
We demonstrate that this application provides a flexible inference in \cref{sec:openvoc}.


%Afterwards, we evaluate application 1) and 3) in~\cref{sec:exp} and extensively show demonstration for all application in~\cref{sec:exp:extensive_app}.

\section{Dataset Details}
\label{sec:dataset-details}
\begin{figure} [h]
\begin{minipage}{0.52\linewidth}
    \includegraphics[width=\textwidth]{figures/vdd-samples.pdf}
    \captionof{figure}{Example images from the VDD benchmark~\cite{vdd}. Each task has a significantly different domain than others, making VDD a challenging benchmark for lifelong learning.}
    \label{fig:vdd-examples}
\end{minipage}
\hfill
\begin{minipage}{0.45\linewidth}
    \centering
    \captionof{table}{The number of samples in training, validation and testing sets per task used in our experiments on the VDD benchmark~\cite{vdd}.}
    \resizebox{\linewidth}{!}{
    \begin{tabular}{c|c|c|c|c}
  \toprule
  Task & Train & Validation & Test & \#Categories \\
  \midrule
  ImageNet12 & 1108951 & 123216 & 49000 & 1000 \\
  CIFAR100 & 36000 & 4000 & 10000 & 10 \\
  SVHN & 42496 & 4721 & 26040 & 10\\
  UCF & 6827 & 758 & 1952 & 101 \\
  Omniglot & 16068 & 1785 & 6492 & 1623 \\
  GTSR & 28231 & 3136 & 7842 & 43 \\
  DPed & 21168 & 2352 & 5880 & 2 \\
  VGG-Flowers & 918 & 102 & 1020 & 102 \\
  Aircraft & 3001 & 333 & 3333 & 100 \\
  DTD & 1692 & 188 & 1880 & 47 \\
  \bottomrule
\end{tabular}}
    \label{tab:vdd-num-samples}
\end{minipage}
\end{figure}
\subsection{The VDD Benchmark}\label{sec:vdd}
It consists of 10 tasks: ImageNet-1k~\cite{imagenet}, CIFAR100~\cite{cifar}, SVHN~\cite{svhn}, UCF101 Dynamic Images (UCF)~\cite{ucf1,ucf2}, Omniglot~\cite{omniglot}, German Traffic Signs (GTSR)~\cite{gtsrb}, Daimler Pedestrian Classification (DPed)~\cite{daimlerpedcls},  VGG Flowers~\cite{vgg-flowers}, FGVC-Aircraft~\cite{aircraft},    and Describable Textures (DTD)~\cite{dtd}. All the images in the VDD benchmark have been scaled such that the shorter side is 72 pixels. Table~\ref{tab:vdd-num-samples} shows the number of samples in each task. Fig.~\ref{fig:vdd-examples} shows examples of images from each task of the VDD benchmark.
In our experiments, we use 10\% of {\tt the official training data} from each of the tasks for validation (e.g., used in the target network selection in Section 3.2.3 in main text), and report the accuracy on {\tt the official validation set} due to the unavailability of the ground-truth labels for the {\tt official test data}. In Table~\ref{tab:vdd-num-samples}, the \texttt{train, validation} and \texttt{test} splits are thus referred to 90\% of the official training data, 10\% of the official training data, and the entire official validation data respectively. When finetuning the learned architecture (i.e., the searched target network) for each task, we use {\tt the entire  official training data} to train and report results on {\tt the official validation set}.%




\begin{figure} [h]
\begin{minipage}{0.53\linewidth}
    \includegraphics[width=\textwidth]{figures/5-datasets-samples.pdf}
    \captionof{figure}{Example images from the 5-Datasets benchmark~\cite{adversarial-continual-learning}.}
    \label{fig:5-dataset-examples}
\end{minipage}
\hfill
\begin{minipage}{0.4\linewidth}
    \centering
    \captionof{table}{Number of samples in training, validation, and test sets per task in the 5-Datasets benchmark~\cite{adversarial-continual-learning}.
    }
    \resizebox{\linewidth}{!}{
    \begin{tabular}{c|c|c|c}
    \toprule
         Task & Train & Validation & Test  \\
         \midrule
         MNIST & 51000 & 9000 & 10000 \\
         not-MNIST & 12733 & 2247 & 3744 \\
         SVHN & 62269 & 10988 & 26032 \\
         CIFAR10 & 42500 & 7500 & 10000 \\
         Fashion MNIST & 51000 & 9000 & 10000 \\
    \bottomrule
    \end{tabular}}
    \label{tab:5-dataset-num-samples}
\end{minipage}
\end{figure}

\subsection{The 5-Datasets Benchmark}\label{sec:5datasets}
It consists of 5 tasks: CIFAR10~\cite{cifar}, MNIST~\cite{mnist}, Fashion-MNIST~\cite{fashion-mnist}, not-MNIST~\cite{notmnist}, and SVHN~\cite{svhn}, all having 10 categories. MNIST, Fashion-MNIST, and not-MNIST have a resolution of $28\times28$, and CIFAR10 and SVHN have a resolution of $32\times32$. We upsample the images to $72\times72$ to match the resolution of the ImageNet images on which the backbone ViT model is trained. Table~\ref{tab:5-dataset-num-samples} shows the data statistics. Fig.~\ref{fig:5-dataset-examples} shows examples of images from each task. To be consistent with the settings used on the VDD benchmark, we use 15\% of \texttt{the training data} for validation and report the results on \texttt{the official test data}, except for not-MNIST for which an official test split is not available. So, for the not-MNIST dataset, we use the small version of that dataset, with which we construct the test set by randomly sampling 20\% of the samples. From the remaining 80\%, we use 15\% for validation, and the rest as the training set.


\section{The Base Vision Transformer: ViT-B/8}
\label{sec:base-vit-model}
We use the base Vision Transformer (ViT) model, with a patch size of $8 \times 8$ (ViT-B/8) model from~\cite{vit}. The base ViT model contains 12 Transformer blocks. A Transformer block is defined by stacking a Multi-Head Self-Attention (MHSA) block and a Multi-Layer Perceptron (MLP) block with resudual connections for each block. ViT-B/8 uses 12 attention heads in each of the MHSA blocks, and a feature dimension of 768. The MLP block expands the dimension size to 3072 in the first layer and projects it back to 768 in the second layer. For all the experiments, we use an image size of $72\times72$ following the VDD setting. We base the implementation of the ViT on the {\tt timm} package ~\cite{timm}. 

\paragraph{Training the Base Model}
To train the ViT-B/8 model, we use the ImageNet data provided by the VDD benchmark (the \texttt{train} split in Table~\ref{tab:vdd-num-samples}). To save the training time, we initialize the weights from the ViT-B/8 trained on the full resolution ImageNet dataset (224$\times$224) and available in the \texttt{timm} package, and finetune it for 30 epochs on the downsized version of ImageNet (72$\times$72) in the VDD benchmark. We use a batch size of 2048 split across 4 Nvidia Quadro RTX 8000 GPUs. We follow the standard training/finetuning recipes for ViT models. The file {\tt cheem/artifacts/imagenet\_pretraining/args.yaml} in our code folder provides all the training hyperparameters used for training the the ViT-B/8 model on ImageNet.
During testing, we take a single center crop of 72$\times$72 from an image scaled with the shortest side to scaled to 72 pixels.
\section{Settings and Hyperparameters in the Proposed Lifelong Learning}\label{sec:training-details}
Starting with the ImageNet pretrained ViT-B/8, the proposed lifelong learning methods consists of three components in learning new tasks continually and sequentially: supernet training, evolutionary search for target network selection, and target network finetuning. The supernet training and target network finetuning use the \texttt{train} split, while the evolutionary search uses the \texttt{validation} split, both shown in Table~\ref{tab:vdd-num-samples} and Table~\ref{tab:5-dataset-num-samples}. We use the vanilla data augmentation in both supernet training and target network finetuning. We use a weight of 1 for the beta loss in all the experiments with $\beta$-DARTS.  

\begin{minipage}{0.48\linewidth}
    \centering
    \resizebox{\textwidth}{!}{
    \begin{tabular}{c|c|c|c}
         \toprule
         Task & Scale and Crop & Hor. Flip & Ver. Flip  \\
         \midrule
         CIFAR100 & Yes & p=0.5 & No \\
         Aircraft & Yes & p=0.5 & No \\
         DPed & Yes & p=0.5 & No \\
         DTD & Yes & p=0.5 & p=0.5 \\
         GTSR & Yes & p=0.5 & No \\
         OGlt & Yes & No & No \\
         SVHN & Yes & No & No \\
         UCF101 & Yes & p=0.5 & No \\
         Flwr. & Yes & p=0.5 & No \\
    \bottomrule
    \end{tabular}}
    \captionof{table}{Data augmentations for the 9 tasks in the VDD benchmark.}
    \label{tab:vdd-augmentations}
\end{minipage}
\hfill
\begin{minipage}{0.48\linewidth}
    \centering
    \resizebox{\textwidth}{!}{
    \begin{tabular}{c|c|c}
    \toprule
         Task & Scale and Crop & Hor. Flip  \\
         \midrule
         MNIST & Yes & No \\
         not-MNIST & Yes & No \\
         SVHN & Yes & No \\
         CIFAR100 & Yes & p=0.5 \\
         Fashion MNIST & Yes & No \\
    \bottomrule
    \end{tabular}}
    \captionof{table}{Data augmentations used for each task in the 5-Datasets benchmark.}
    \label{tab:5-dataset-augmentations}
\end{minipage}

\textbf{Data Augmentations.}
A full list of data augmentations used for the VDD benchmark is provided in Table \ref{tab:vdd-augmentations}, and the data augmentations used for the tasks in the 5-datasets benchmark is provided in Table \ref{tab:5-dataset-augmentations}. The augmentations are chosen so as not to affect the nature of the data. Scale and Crop transformation scales the image randomly between 90\% to 100\% of the original resolution and takes a random crop with an aspect ratio sampled from a uniform distribution over the original aspect ratio $\pm0.05$.
In evaluating the supernet and the finetuned model on the validation set and test set respectively, images are simply resized to $72\times72$ with bicubic interpolation.


\vspace{-1mm}
\subsection{Supernet Training}
\textit{VDD Benchmark}: For each task, we train the supernet for 150 epochs, unless otherwise stated. We use a label smoothing of 0.1. No other form of regularization is used since the {\tt skip} operation provides implicit regularization, which plays the role of Drop Path during training. We use a learning rate of 0.001 and the Adam optimier~\citep{adam} with a Cosine Decay Rule. For experiments with separate class tokens per task, we use a learning rate of 0.0005 for training the supernet, and 0.001 for training the task token. For each epoch, a minimum of 15 batches are drawn, with a batch size of 512. If the number of samples present in the task allows, we draw the maximum possible number of batches that covers the entire training data. For the Exploration-Exploitation sampling scheme, we use an exploration probability $\epsilon = 0.3$.

\textit{5-datasets Benchmark}: We use the same hyperparameters as those used in the VDD Benchmark, but train the supernet for 50 epochs.

\textit{L2G with DARTS and $\beta$-DARTS}: We train the supernet for Learn to Grow \citep{learn-to-grow} for 50 epochs for the VDD benchmark and 25 epochs for the 5-datasets benchmark.

\vspace{-1mm}
\subsection{Evolutionary Search}
\label{sec:evolutionary-search-params}
The evolutionary search is run for 20 epochs. We use a population size of 50. 25 candidates are generated by mutation, and 25 candidates are generated using crossover. The top 50 candidates are retained. The crossover is performed among the top 10 candidates, and the top 10 candidates are mutated with a probability of $0.1$. For the Exploration-Exploitation sampling scheme, we use an exploration probability $\epsilon = 0.5$ when generating the initial population.

\vspace{-1mm}
\subsection{Finetuning}
The target network for a task selected by the evolutionary search is finetuned for 30 epochs with a learning rate of 0.001, Adam optimizer, and a Cosine Learning Rate scheduler. Drop Path of 0.25 and label smoothing of 0.1 is used for regularization. We use a batch size of 512, and a minimum of 30 batches are drawn. When using a separate task token for each task, the task token is first finetuned with a learning rate of 0.001.

\subsection{Normalized Cosine Similarity}
\label{sec:normalized-cosine-similarity}
To verify the use of the Normalized Cosine Similarity as our similarity measure, we refer to Figure \ref{fig:similarities}. Figure \ref{subfig:cosine-sim} shows the Cosine Similarity between the mean class-tokens learned for tasks ImageNet, CIFAR100, SVHN, UCF101, and Omniglot (in order), and the mean class-tokens calculated for each expert using the data from the current task GTSR. Empirically, we observe that the Cosine Similarity between the mean class-tokens calculated using the data of the task associated with an expert and the mean class-token calculated with the current task in training is high. However, the difference between the similarity values for each expert are more important than the absolute values of the similarity. This difference can be increased by scaling the similarity such that it increases the magnitude difference between the similarities of different tasks, but maintains the relative similarity. This can be achieved by scaling the Cosine Similarities between -1 and 1 using the minimum and the maximum values from all the experts and all the blocks (Figure \ref{subfig:norm-cosine-sim}). Using the Normalized Cosine Similarity leads to better and more intuitive probability distributions for sampling candidate experts and the retention probabilities for the sampled experts. For example, comparing the probability values for sampling an expert at Block 6 calculated using Cosine Similarity (Figure \ref{subfig:expert-sampling}) vs. Normalized Cosine Similarity (Figure \ref{subfig:expert-sampling-normalized}), the probability of sampling the ImageNet expert increases, and those of sampling UCF101 and Omniglot decrease. Similarly, for Blocks 5 and 6, the retention probability calculated using the Normalized Cosine Similarity (Figure \ref{subfig:retention-probability-normalized}) reduces by a large factor than that calculated using the Cosine Similarity (Figure \ref{subfig:retention-probability}). This will encourage sampling the {\tt adapt} operation when these experts are sampled, thus adding plasticity to the network. The retention probability of the ImageNet experts at these blocks also reduces slightly, which will avoid imposing a strict prior.


\begin{figure}[H]
    \centering
    \begin{subfigure}[t]{0.4\textwidth}
        \centering
        \includegraphics[width=\textwidth]{figures/gtsrb_similarity/gtsrb_cosine_similarity.pdf}
        \caption{Cosine Similarity between the mean class-tokens from the previous tasks and mean class-token for each block.}
        \label{subfig:cosine-sim}
    \end{subfigure}
    ~
    \begin{subfigure}[t]{0.4\textwidth}
        \centering
        \includegraphics[width=\textwidth]{figures/gtsrb_similarity/gtsrb_normalized_cosine_similarity.pdf}
        \caption{Normalized Cosine Similarity at each block calculated by scaling the Cosine Similarities between $-1$ and $1$.}
        \label{subfig:norm-cosine-sim}
    \end{subfigure}\vspace{-1.5mm}
    ~
    \begin{subfigure}[t]{0.4\textwidth}
        \centering
        \includegraphics[width=\textwidth]{figures/gtsrb_similarity/gtsrb_softmax.pdf}
        \caption{Probabilities of sampling candidate Expert $e$ ($\psi_e^l$) at each block calculated using the Cosine Similarity}
        \label{subfig:expert-sampling}
    \end{subfigure}
    ~
    \begin{subfigure}[t]{0.4\textwidth}
        \centering
        \includegraphics[width=\textwidth]{figures/gtsrb_similarity/gtsrb_normalized_softmax.pdf}
        \caption{Probabilities of sampling candidate Expert $e$ ($\psi_e^l$) at each block calculated using the Normalized Cosine Similarity}
        \label{subfig:expert-sampling-normalized}
    \end{subfigure}\vspace{-1.5mm}
    ~
    \begin{subfigure}[t]{0.4\textwidth}
        \centering
        \includegraphics[width=\textwidth]{figures/gtsrb_similarity/gtsrb_sigmoid.pdf}
        \caption{Retention probabilities ($\rho_e^l$) for each expert in each block, calculated using the Cosine Similarity.}
        \label{subfig:retention-probability}
    \end{subfigure}
    ~
    \begin{subfigure}[t]{0.4\textwidth}
        \centering
        \includegraphics[width=\textwidth]{figures/gtsrb_similarity/gtsrb_normalized_sigmoid.pdf}
        \caption{Retention probabilities ($\rho_e^l$) for each expert, calculated using the Normalized Cosine Similarity.}
        \label{subfig:retention-probability-normalized}
    \end{subfigure}\vspace{1em}
    \caption{Comparison of probability values for sampling the Experts (Middle row) and the retention probabilities (Bottom row) using the Cosine Similarity and the Normalized Cosine Similarity. Using the Normalized Cosine Similarity gives better probability values for expert sampling probabilities $\psi_e$, as seen in Blocks 5 and 6, where the expert sampling probability for ImageNet increases, thus reducing the probability of sampling {\tt new} and {\tt skip}. This will encourage maximal reuse. The effect on the retention probability $\rho_e$ can be prominently seen on the Omniglot experts. The retention probability in Blocks 6 \ref{subfig:retention-probability-normalized} reduces, which will encourage the {\tt adapt} operation to be trained even if the Omniglot expert even if Omniglot experts were sampled.}
    \label{fig:similarities}
\end{figure}



\section{Learned architecture for different task order and pure exploration}
\label{sec:other-architectures}
Figure \ref{fig:vdd_arch_seq2} shows the architecture learned for a different task sequence. It can be seen that even with a different task sequence, the proposed method can learn to exploit task similarities. For example, an adapt operation is layer is learned at Block 4 for Omniglot, which is reused by GTSR. At Block 7, a New operation is learned for Omniglot, which is adapted for SVHN. Even though CIFAR100 is learnt as the last task, the search process can still learn to reuse many ImageNet experts. Figure \ref{fig:vdd_arch_seq1_exp} shows the architecture learned using a pure exploration strategy. It can be seen that pure exploration does not reuse components from similar tasks. For example, a large number of {\tt Adapt} and {\tt New} are added when learning CIFAR100. In contrast, the exploitation-exploitation strategy can learn to reuse the components from the ImageNet task (Figure \ref{fig:vdd_arch-sim} in the main text), and achieves better accuracy as well (Table \ref{tab:vdd-results} in the main text).

\begin{figure}[h]
    \centering
    \includegraphics[width=0.9\textwidth]{figures/vdd-iclr-42-ee-150-seq2.pdf}
    \caption{Architecture learned for task sequence ImNet, OGlt, UCF, Airc, Flwr, SVHN, DTD, GTSR, DPed, C100}
    \label{fig:vdd_arch_seq2} \vspace{-4mm}
\end{figure}\vspace{3mm}

\begin{figure}[h]
    \centering
    \includegraphics[width=0.9\textwidth]{figures/vdd-iclr-42-e-150.pdf}
    \caption{Architecture learned using pure exploration. Pure Exploration based method adds many unnecessary Adapt and New operations even though the tasks are similar (ImNet → C100), proving the effectiveness of the proposed sampling method}
    \label{fig:vdd_arch_seq1_exp} \vspace{-4mm}
\end{figure}

\begin{table}[h]
        \centering
        \resizebox{\textwidth}{!}{
            \begin{tabular}{c|l|l|l|l|l|clllllllllll} \toprule
    
          & \multicolumn{17}{c}{ImageNet $\rightarrow$ Omniglot under the lifelong learning setting} \\ \midrule
          & \multicolumn{2}{c|}{{\tt Adapter} in} &  \multirow{2}{1cm}{\#Param Added}     &    \multirow{2}{*}{Rel. $\uparrow$}          &   \multirow{2}{1cm}{Test Acc.}    &     \multicolumn{12}{c}{Learned Operation per Block} \\
          \cline{2-3} \cline{7-18}
          & NAS & Finetune & &  &    & 1 & 2 &   3 & 4 &   5 &   6 & 7 &    8 & 9 &   10 &   11 &   12 \\
          \cline{1-18}
            Shorcut & \multirow{2}{*}{w/o A \& S} & w/ A &  \multirow{2}{*}{2.96M} &         \multirow{2}{*}{3.47\%} & 82.18 & \multirow{2}{*}{\colorbox{adapt}{A}} & \multirow{2}{*}{\colorbox{adapt}{A}} &   \multirow{2}{*}{\colorbox{reuse}{R}} & \multirow{2}{*}{\colorbox{reuse}{R}} &   \multirow{2}{*}{\colorbox{adapt}{A}} &   \multirow{2}{*}{\colorbox{reuse}{R}} &   \multirow{2}{*}{\colorbox{adapt}{A}} &    \multirow{2}{*}{\colorbox{new}{N}} & \multirow{2}{*}{\colorbox{new}{N}} &    \multirow{2}{*}{\colorbox{new}{N}} &    \multirow{2}{*}{\colorbox{skip}{S}} &    \multirow{2}{*}{\colorbox{skip}{S}} \\
                 \cline{3-3}\cline{6-6}
                 in &           &        w/o S &      &             & 78.16 &                               &   &   &   &   &    &    &    &    \\
                 \cline{2-18}
            {\tt Adapter} &         w/ S \& A &         w/ S \& A &  4.14M &         4.89\% & 82.32 &                               \colorbox{adapt}{A} & \colorbox{adapt}{A} & \colorbox{adapt}{A} &   \colorbox{adapt}{A} &   \colorbox{adapt}{A} &   \colorbox{adapt}{A} &   \colorbox{new}{N} &    \colorbox{adapt}{A} & \colorbox{adapt}{A} &    \colorbox{new}{N} &    \colorbox{adapt}{A} &    \colorbox{new}{N} \\
            \bottomrule
        \end{tabular}
    }
\vspace{0.3em}
        \caption{Results of the ablation study on the {\tt Adapter} implementation (Section 3.2.1): with (w/) vs without (w/o) shortcut connection for the MLP {\tt Adapt} layer. We test lifelong learning from ImageNet to Omniglot in the VDD. The proposed combination of w/o shortcut in Supernet NAS training and target network selection and w/ shortcut in finetuning (retraining newly added layers) is the best in terms of the trade-off between performance and cost.}
        \label{tab:adapter_ablation} \vspace{-3mm}
\end{table}

\section{Ablation Studies}
\label{sec:ablations} \vspace{-2mm}

\subsection{The Structure of {\tt Adapter}}\vspace{-2mm} 
\label{sec:hybrid-adapter}
\paragraph{How to {\tt Adapt} in a sustainable way?} The proposed {\tt Adapt} operation will effectively increase the depth of the network in a plain way. In the worst case, if too many tasks use {\tt Adapt} on top of each other, we will end up stacking too many MLP layers together. This may lead to unstable training due to gradient vanishing and exploding. Shortcut connections~\citep{resnet} have been shown to alleviate the gradient vanishing and exploding  problems, making it possible to train deeper networks. Due to this residual architecture, the training can ignore an adapter if needed, and leads to a better performance. However, in the lifelong learning setup, where subsequent tasks might have different distributions, the search process might disproportionately encourage {\tt Adapt} operations because of this ability. To counter this, we propose a hybrid {\tt Adapter} which acts as a plain 2-layer MLP during Supernet training and target network selection, and a residual MLP during finetuning. With an ablation study (Table~1 in Supplementary), we show that much more compact models can be learned with negligible loss in accuracy.

We verify the effectiveness of the proposed hybrid adapter using a lifelong learning setup with 2 tasks: ImageNet and Omniglot. The Omniglot dataset presents two major challenges for a lifelong learning system. First, Omniglot is a few-shot dataset, for which we may expect a lifelong learning system can learn a model less complex than the one for ImageNet. Second,  Omniglot has a significantly different data distribution than ImageNet, for which we may expect a lifelong learning system will need to introduce new parameters, but hopefully in a sensible and explainable way. Table \ref{tab:adapter_ablation} shows the results. In terms of the learned neural architecture, a more compact model (row 3) is learned without the shortcut in the adapter during Supernet training and target network selection: the last two MHSA blocks are skipped and three blocks are reused. Skipping the last two MHSA blocks makes intuitive sense since Omniglot may not need those high-level self-attention (learned for ImageNet) due to the nature of the dataset. The three consecutive {\tt new} operations (in Blocks 8,9,10) also make sense in terms of learning new self-attention fusion layers (i.e., new memory) to account for the change of the nature of the task. Adding shortcut connection back in the finetuning shows significant performance improvement (from 78.16\% to 82.18\%), making it very close to the performance (82.32\%) obtained by the much more expensive and less intuitively meaningful alternative (the last row).

\begin{table*} [h]
    \centering
    \resizebox{\textwidth}{!}{
    \begin{tabular}{l|c|cccccccccc|c}
        \toprule
            Component &  ImNet &  C100 &    SVHN &   UCF &  OGlt &  GTSR &  DPed &  Flwr &  Airc. &   DTD &  Avg. Accuracy & Avg. Param. Inc./task (M) \\
        \midrule           
             Projection & $82.65$ & $\boldsymbol{90.86}$ & $\boldsymbol{96.06}$ & $\boldsymbol{75.63}$ & $84.06$ & $\boldsymbol{99.92}$ & $99.83$ & $\boldsymbol{89.28}$ & $\boldsymbol{51.94}$ & $\boldsymbol{55.78}$ & $\boldsymbol{82.60} \pm \boldsymbol{0.55}$ & $1.12 \pm 0.03$\\
            Value & $82.65$ & $85.59$ & $95.82$ & $72.25$ & $\boldsymbol{84.20}$ & $99.89$ &	$99.89$ & $84.05$ & $45.80$ &	$53.37$ & $80.35 \pm 1.10$ & $1.73 \pm 0.13$ \\ 
            Query & {$82.65$} & $90.00$ & $94.56$ & $70.08$ & $78.56$ & $99.83$ & $99.91$ & $85.00$ & $43.37$ & $55.53$ & $79.95 \pm 0.76$ & $2.65 \pm 0.16$ \\ 
            Key & {$82.65$} & $89.57$ & $94.41$ & $71.31$ & $81.12$ & $99.89$ & $\boldsymbol{99.92}$ & $87.03$ & $45.10$ & $\boldsymbol{56.01}$ & $80.00 \pm 0.70$ & $2.47 \pm 0.23$ \\ 
            FFN & {$82.65$} & $\boldsymbol{90.70}$ & $\boldsymbol{96.18}$ & $\boldsymbol{79.59}$ & $\boldsymbol{85.44}$ & $\boldsymbol{99.92}$ & $\boldsymbol{99.92}$ & $\boldsymbol{87.19}$ & $\boldsymbol{51.66}$ & $54.02$ & $\boldsymbol{82.73} \pm \boldsymbol{1.06}$ & $4.44 \pm 0.83$\\  
        \bottomrule
    \end{tabular}}
    \vspace{0.3em}
    \caption{Results of ablation study on the other components of the ViT used for realizing the ArtiHippo. Realizing ArtiHippo at the FFN shows slightly better performance than the Projection layer. However, the Projection layer is much more parameter efficient than the FFN. Using the Projection layer offers negligible drop in Average Accuracy without sacrificing parameter efficiency. The results have been averaged over 3 different seeds.}
    \label{tab:vdd-component-ablation} \vspace{-2mm}
\end{table*}

\subsection{Evaluating feasibility of other ViT components as ArtiHippo }
\label{sec:artihippo-components}
\vspace{-2mm}
Table \ref{tab:vdd-component-ablation} shows the accuracy with other components if the ViT used for learning the Mixture of Experts using the proposed NAS method. The Query component from the MHSA block and the Value do not perform as well as the Projection layer. The FFN performs only slightly better than the Projection layer, but as a much larger parameter cost This reinforces our identification of the ArtiHippo in Section \ref{sec:identify_artihippo} in the main text as a lightweight plastic component.


\begin{figure}[t]
    \centering
    \begin{subfigure}[t]{0.5\linewidth}
        \centering
        \includegraphics[width=0.8\linewidth]{figures/epochwise_plot.pdf}
        \caption{}
    \end{subfigure}
    \begin{subfigure}[t]{0.48\linewidth}
        \centering
        \includegraphics[width=\linewidth]{figures/taskwise_increase.pdf}
        \caption{}
    \end{subfigure}\vspace{2mm}
    \caption{(a) Results of the ablation study on the Exploration-Exploitation (EE) guided sampling in the Supernet NAS training using  the VDD benchmark~\citep{vdd}. The proposed EE sampling strategy is much more efficient than the pure exploration based strategy (i.e., the vanilla SPOS NAS~\citep{spos}). It uses the Supernet training efficiently even at 50 epochs and achieves better performance than pure Exploration, which is desirable for fast adaptation in dynamic environments using lifelong learning. The \% increase in parameters shows that EE strategy is effective in reusing experts from the previous tasks and limiting the increase in parameters. The results have been averaged over 3 runs with different seeds. (b) Percent increase in the number of parameters over tasks. All New refers to a new projection layer for every block as new task arrives (similar for All Adapt). This shows a linear increase in the number of parameters. The proposed exploration-exploitation method stays well below the ``All Adapt" curve as opposed to pure exploration which almost approaches ``All Adapt".}
    \label{fig:epochs-vs-acc}
\end{figure}


\subsection{The Exploration-Exploitation Sampling Method} 
\label{sec:exp-expl-sampling}
\vspace{-2mm}
Figure~\ref{fig:epochs-vs-acc}, left shows that the proposed exploration-exploitation strategy can consistently obtain higher accuracy than pure exploration even when the supernet is trained a small number of epochs. Even when the supernet is trained for a longer duration, the proposed exploration-exploitation strategy still outperforms pure exploration (Figure \ref{fig:epochs-vs-acc} top). Moreover, the exploration-exploitation strategy adds a lot less additional parameters than pure exploration (Figure \ref{fig:epochs-vs-acc} bottom). We thus verify that the proposed exploration-exploitation strategy is effective and efficient in utilizing the parameters learned by the previous tasks, thus making the ``selective addition" of parameters mentioned in Section \ref{sec:intro} possible. This also shows that proposed task-similarity metric is meaningful.

\subsection{Parameter growth over time}
\label{sec:param-growth-over-time}
Since dynamic model based methods add parameters as new tasks arrive, its necessary to study the rate at which the number of parameters grow. Since the proposed ArtiHippo adds new experts (parameters) dynamically, we cannot analytically determine the rate of growth. However, our experiments show that the proposed exploration-exploitation strategy achieves sub-linear growth in the number of parameters, which again shows that our method can effectively leverage the parameters learned in the previous tasks.

\section{Comparison with additional methods}
\label{sec:comparison-additional}
For completeness, we also compare with a baseline of L2 Parameter Regularization \citep{smith2023closer}, and Elastic Weight Consolidation \citep{kirkpatrick-overcoming} applied to the linear projection layer of the Multi Head Self-Attention block. We also compare with Experience Replay with a buffer update strategy used in \citet{icarl}. This comparison is not completely fair, since these methods are class-incremental in nature. However, this comparison serves as a baseline to observe meaningful trade-off. Table \ref{tab:additional-comp} shows that EWC, L2 Parameter Regularization and Experience Replay cannot completely overvome catastraophic forgetting, and hence lose accuracy over time.

\begin{table} %
    \centering
    \resizebox{\textwidth}{!}{
    \begin{tabular}{l|cccccccccc|l}
        \toprule
            Method-Backbone &  ImNet &  C100 &    SVHN &   UCF &  OGlt &  GTSR &  DPed &  Flwr &  Airc. &   DTD &  Avg. Accuracy \\
        \midrule
            {S-Prompts$^\dagger$} ($L$=12)~\citep{s-prompts} & $82.65$ & $89.32$ & $88.91$ & $64.52$ & $72.17$ & $99.29$ & $\boldsymbol{99.89}$ & $\boldsymbol{96.93}$ & $45.55$ & $60.76$ & $80.00 \pm 0.07$ \\
            {L2P$^\dagger$} ($L$=12)~\citep{learning-to-prompt} & $82.65$ & $89.32$ & $89.89$ & $65.63$ & $72.34$ & $99.55$ & $\boldsymbol{99.94}$ & $\boldsymbol{96.63}$ & $45.24$ & $59.57$ & $80.08 \pm 0.10$ \\
            \midrule
            $^*$L2G~\citep{learn-to-grow} (DARTS) & $82.65$ & $88.47$ & $85.20$ & $\boldsymbol{79.22}$ & $80.19$ & $99.28$ & $ 98.06$ & $76.14$ & $39.29$ & $46.01$ & $77.45 \pm 2.41$ \\ 
            $^*$L2G~\citep{learn-to-grow} ($\beta$-DARTS) & $82.65$ & $88.95$ & $94.73$ & $75.31$ & $79.76$ & $99.84$ & $99.76$ & $78.86$ & $34.50$ & $47.09$ & $78.14 \pm 0.54$ \\ 
            \midrule
            EWC \citep{kirkpatrick-overcoming} & $58.19$ & $87.69$ & $69.64$ & $57.27$ & $45.89$ & $95.01$ & $98.47$ & $90.20$ & $36.57$ & $\boldsymbol{61.97}$ & $70.09$ \\ 
            L2 Regularization \citep{smith2023closer} & $55.28$ & $87.10$ & $55.23$ & $58.86$ & $40.48$ & $95.07$ & $99.17$ & $90.20$ & $37.53$ & $\boldsymbol{62.55}$ & $68.15$ \\ 
            Experience Replay \citep{icarl} & $55.88$ & $78.70$ & $87.40$ & $58.20$ & $76.03$ & $97.92$ & $48.55$ & $84.41$ & $40.98$ & $54.68$ & $68.27$ \\ 
            \midrule
            Our ArtiHippo (Uniform, 150 epochs) & $82.65$ & $76.20$ & $95.60$ & $75.14$ & $80.72$ & $\boldsymbol{99.92}$ & $99.86$ & $76.41$ & $42.74$ & $41.74$ & $77.10 \pm 0.75$ \\ 
             Our ArtiHippo (Hierarchical, 50 epochs) & $82.65$ & $\boldsymbol{90.97}$ & $96.05$ & $75.20$ & $82.36$ & $\boldsymbol{99.91}$ & $99.58$ & $87.16$ & $42.10$ & $52.54$ & $\boldsymbol{80.85} \pm \boldsymbol{0.72}$ \\
             Our ArtiHippo (Hierarchical, 150 epochs) & $82.65$ & $\boldsymbol{90.86}$ & $\boldsymbol{96.06}$ & $75.63$ & $\boldsymbol{84.06}$ & $\boldsymbol{99.92}$ & $99.83$ & $89.28$ & $\boldsymbol{51.94}$ & $55.78$ & $\boldsymbol{82.60} \pm \boldsymbol{0.55}$ \\
            Our ArtiHippo (Hierarchical+$L$=1, 150 epochs) & $82.65$ & $\boldsymbol{90.50}$ & $\boldsymbol{96.19}$ & $\boldsymbol{79.70}$ & $\boldsymbol{85.71}$ & $\boldsymbol{99.91}$ & $99.83$ & $92.42$ & $\boldsymbol{52.23}$ & $58.99$ & $\boldsymbol{83.55} \pm \boldsymbol{0.09}$ \\
        \bottomrule
    \end{tabular}}
    \vspace{0.1em}
    \caption{Results on the VDD benchmark~\citep{vdd}. Our ArtiHippo shows clear improvements over the previous approaches. 
    All the results from our experiments are averaged over 3 different seeds. The 2 highest accuracies per task have been highlighted. All the methods use the same ViT-B/8 backbone containing 86.04M parameters and having 14.21G FLOPs unless otherwise stated. $^\dagger$ our modifications for the task-incremental setting. $^*$ our reproduction with the vanilla L2G method~\citep{learn-to-grow} for the ViT backbone. }
    \label{tab:additional-comp} \vspace{-3mm}
\end{table}




\section{Details of Modifying Baseline Methods on the VDD Benchmark}\label{sec:baseline}


\subsection{Modifying  S-Promts and L2P for Task-Incremental Setting on the VDD Benchmark}
\label{sec:task-inc-l2p-sprompt}
Both the S-Prompts~\cite{s-prompts} and the Learn-to-Prompt (L2P)~\cite{learning-to-prompt} can be modified for task-incremental setting (i.e., task ID is available in both training and inference) on the VDD benchmark without altering the core algorithm for learning the prompts. 

For the S-Prompts method, the modification is done by training $L$ prompts (randomly initialized) per task and then by retrieving the correct task-specific prompts with the task ID. Table \ref{tab:vdd-results-prompts} and Table \ref{tab:5-dataset-results-prompt} show the results of varying the number of prompts on the VDD benchmark and 5-dataset benchmark respectively, from which we can observe that varying the number of prompts beyond 10 does not affect the performance significantly, which is also observed in the original paper \cite{s-prompts}. 


\begin{table}[h]
    \centering
    \caption{Results of S-Prompts on the VDD benchmark~\cite{vdd} with various number of prompts. The results have been averaged over 3 different seeds.
    }
    \resizebox{\textwidth}{!}{
    \begin{tabular}{l|cccccccccc|l}
        \toprule
            Method &  ImNet &  C100 &    SVHN &   UCF &  OGlt &  GTSR &  DPed &  Flwr &  Airc. &   DTD &  Avg. Accuracy \\
        \midrule           
            {S-Prompts} (L=1/task) & $82.65$ & $87.06$ & $76.42$ & $54.82$ & $62.10$ & $96.74$ & $99.59$ & $95.52$ & $37.62$ & $57.78$ & $75.03 \pm 0.19$ \\
            {S-Prompts} (L=5/task) & $82.65$ & $88.91$ & $85.23$ & $62.23$ & $70.64$ & $99.08$ & $99.87$ & $97.35$ & $45.32$ & $60.74$ & $79.20 \pm 0.53$ \\
            {S-Prompts} (L=10/task) & $82.65$ & $89.62$ & $88.69$ & $65.20$ & $72.18$ & $99.37$ & $99.91$ & $97.06$ & $45.17$ & $60.94$ & $80.08 \pm 0.30$ \\
            {S-Prompts} (L=12/task)~\cite{s-prompts} & $82.65$ & $89.32$ & $88.91$ & $64.52$ & $72.17$ & $99.29$ & $99.89$ & $96.93$ & $45.55$ & $60.76$ & $80.00 \pm 0.07$ \\
            {S-Prompts} (L=15/task) & $82.65$ & $89.63$ & $89.36$ & $65.88$ & $72.54$ & $99.37$ & $99.94$ & $97.03$ & $45.07$ & $61.17$ & $80.26 \pm 0.09$ \\ 
        \bottomrule
    \end{tabular}}
    \vspace{0.1em}
    \label{tab:vdd-results-prompts} \vspace{-3mm}
\end{table}

\begin{table}[h]
    \centering
    \caption{Results of S-Prompts on the 5-Dataset benchmark~\cite{adversarial-continual-learning}. 
    The results have been averaged over 5 different task orders.}
    \resizebox{0.4\textwidth}{!}{
    \begin{tabular}{c|c|c}
        \toprule
        \textbf{Method} & \textbf{\#Prompts} & \textbf{Avg. Acc.} \\
        \toprule
        S-Prompts & 1 & $88.93 \pm 0.34$ \\
        S-Prompts & 5 & $91.14 \pm 0.78$ \\
        S-Prompts & 10 & $92.28 \pm 0.16$ \\
        S-Prompts & 12 & $92.42 \pm 0.11$ \\
        S-Prompts & 15 & $92.39 \pm 0.05$ \\
        \bottomrule
    \end{tabular}
    }
    \vspace{0.3em}
    \label{tab:5-dataset-results-prompt}
\end{table}

For the L2P method, we follow the official implementation\footnote{L2P official implementation: \href{https://github.com/google-research/l2p}{https://github.com/google-research/l2p}} which is tested on the 5-datasets benchmark~\cite{adversarial-continual-learning}. The vanilla L2P first trains a set of $N$ prompts of length $L_p$ (i.e. $N\cdot L_p$ tokens) per task. It then learns a set of $N$ keys such that the distance between the keys and the image encoding (using a fixed feature extractor) is maximized. 
In modifying the vanilla L2P for the task-incremental setting, we can directly retrieve the correct prompts using the Task ID instead of using a key-value matching. 
We initialize the  prompts for task $t$ from the trained prompts of task $t-1$ following the original implementation. We note that the prompt initialization is the only difference between the modified S-Prompts and the modified L2P.  Base on the above observations of performance changes w.r.t. the number of prompts in modifying S-Prompts, we use $L=12$ for L2P in our experiments.




\section{Modifying dynamic model based methods for Vision Transformers}
\label{sec:modification-for-vits}
Supermasks in Superposition (SupSup, \citep{supsup}), Efficient Feature Transformation (EFT, \citep{eft}), and Lightweight Learner (LL, \citep{ll}) have originally been developed for Convolutional Neural Networks. Here, we describe our modifications to theoriginal methods to make them compatible with Vision Transformers for a fair comparison with our ArtiHippo. Following ArtiHippo, which learns a base Vision Transformer model with ImageNet data from the VDD benchmark \citep{vdd}, we initialize the network for SupSup, EFT and LL with the same backbone. For SupSup, we learn masks for the weights of the final linear projection layer of the Multi-Head Self-Attention block using the straight through estimator \citep{straight-through-estimator}. We apply EFT on all the linear layers of the ViT by scaling all the activations by a learnable scaling vector using the Hadamard product following the original proposed formulation for fully-connected layers. Finally, for LL, which learns a task-specific bias vector which is added to all the feature maps of convolutional layers, we learn a similar bias vector and add it to the output of all the linear layers of the ViT. 



\end{document}
