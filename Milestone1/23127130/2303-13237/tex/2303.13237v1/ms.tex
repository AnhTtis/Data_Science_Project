\documentclass[12pt]{article}
\usepackage{amsmath}
\usepackage{graphicx}
\usepackage{enumerate}
\usepackage{natbib} 
\usepackage{url} 
\usepackage{pdflscape}
\usepackage{setspace}
\usepackage{multirow}
\usepackage{hyperref}
\usepackage{amsthm}
\usepackage{amsfonts}
\usepackage{bbm}
\RequirePackage[dvipsnames]{xcolor}
\usepackage{mathptmx}
\usepackage{breqn}


% DON'T change margins - should be 1 inch all around.
\addtolength{\oddsidemargin}{-.5in}%
\addtolength{\evensidemargin}{-1in}%
\addtolength{\textwidth}{1in}%
\addtolength{\textheight}{1.7in}%\\
\addtolength{\topmargin}{-1in}%

\numberwithin{equation}{section}

\newcommand{\NN}{\mathbb{N}}
\newcommand{\ZZ}{\mathbb{Z}}
\newcommand{\RR}{\mathbb{R}}
\newcommand{\PP}{\mathbb{P}}
\newcommand{\EE}{\mathbb{E}}

\DeclareMathOperator*{\argmax}{arg\,max}
\DeclareMathOperator*{\argmin}{arg\,min}

\newcommand{\RC}[2]{\mathrm{RC}_{#2}(#1)}
\def\CB#1{{\textcolor{Green}{[CB: #1]}}}
\def\CBnew#1{{\textcolor{Cyan}{ #1}}}

\newtheorem{theorem}{Theorem}[section]
\newtheorem{proposition}{Proposition}[section]
\newtheorem{property}{Property}[section]
\newtheorem{corollary}{Corollary}[theorem]
\newtheorem{lemma}[theorem]{Lemma}
\theoremstyle{remark}
\newtheorem*{remark}{Remark}
\theoremstyle{definition}
\newtheorem{definition}{Definition}[section]

\bibliographystyle{apalike}

\begin{document}

\def\spacingset#1{\renewcommand{\baselinestretch}%
{#1}\small\normalsize} \spacingset{1}

\title{Improving estimation for asymptotically independent bivariate extremes via global estimators for the angular dependence function}
\author{C. J. R. Murphy-Barltrop$^{1*}$, J. L. Wadsworth$^2$ and E. F. Eastoe$^2$\\
\small $^1$STOR-i Centre for Doctoral Training, Lancaster University LA1 4YR, United Kingdom \\
\small $^2$Department of Mathematics and Statistics, Lancaster University LA1 4YF, United Kingdom \\
\small $^*$Correspondence to: c.barltrop@lancaster.ac.uk}
\date{\today}
\maketitle

\bigskip
\begin{abstract}
    Modelling the extremal dependence of bivariate variables is important in a wide variety of practical applications, including environmental planning, catastrophe modelling and hydrology. The majority of these approaches are based on the framework of bivariate regular variation, and a wide range of literature is available for estimating the dependence structure in this setting. However, this framework is only applicable to variables exhibiting asymptotic dependence, even though asymptotic independence is often observed in practice. In this paper, we consider the so-called `angular dependence function'; this quantity summarises the extremal dependence structure for asymptotically independent variables. Until recently, only pointwise estimators of the angular dependence function have been available. We introduce a range of global estimators and compare them to another recently introduced technique for global estimation through a systematic simulation study, and a case study on river flow data from the north of England, UK. 
\end{abstract}

\noindent%
{\it Keywords:}  Bivariate Extremes, Dependence Modelling, Asymptotic Independence, Angular Dependence Function
\vfill

\newpage
\spacingset{1.8}

\section{Introduction} \label{Sec1}
Bivariate extreme value theory is a branch of statistics that deals with the modelling of dependence between the extremes of two variables. This type of analysis is useful in a variety of fields, including finance \citep{Castro-Camilo2018}, engineering \citep{Ross2020}, and environmental science \citep{Brunner2016}, where understanding and predicting the behaviour of rare, high-impact events is important. 

In certain applications, interest lies in understanding the risk of observing simultaneous extreme events at multiple locations; for example, in the context of flood risk modelling, widespread flooding can result in damaging consequences to properties, businesses, infrastructure, communications and the economy \citep{Lamb2010,Keef2013a}. To support resilience planning, it it imperative to identify locations at high risk of joint extremes. 

Classical theory for bivariate extremes is based on the framework of regular variation. Given a random vector $(X,Y)$ with standard exponential margins, we say that $(X,Y)$ is bivariate regularly varying if, for any measurable $B \subset [0,1]$,
\begin{equation} \label{eqn:MRV} 
    \lim_{r \to \infty} \Pr (V \in B, R > sr \mid R>r) = H(B) s^{-1}, \; s \geq 1,
\end{equation}
with $R:= e^X+e^Y$, $V := e^X/R$  and $H(\partial B) = 0$, where $\partial B$ is the boundary of $B$ \citep{Resnick1987}. Note that bivariate regular variation is most naturally expressed on standard Pareto margins, and the mapping $(X,Y) \mapsto (e^X,e^Y)$ performs this transformation. We refer to $R$ and $V$ as radial and angular components, respectively. Equation \eqref{eqn:MRV} implies that for the largest radial values, the radial and angular components are independent. Furthermore, the quantity $H$, which is known as the spectral measure, must satisfy the moment constraint $\int_0^1v\mathrm{d}H(v) = 1/2$. 

The spectral measure summarises the extremal dependence of $(X,Y)$, and a wide range of approaches exist for its estimation \citep[e.g.,][]{Einmahl2009,Carvalho2014,Eastoe2014}. Equivalently, one can consider Pickands' dependence function \citep[PDF;][]{Pickands1981}, which has a direct relationship to $H$ via
\begin{equation*}
    A(t) = \int_0^1\max\{vt, (1-v)(1-t) \}2\mathrm{d}H(v), \hspace{1em} t \in [0,1].
\end{equation*}
This function again captures the extremal dependence of $(X,Y)$, and many approaches also exist for its estimation \citep[e.g.,][]{Guillotte2016,Marcon2016,Vettori2018}. Moreover, estimation approaches for the spectral measure and PDF encompass a wide range of statistical methodologies, with parametric, semi-parametric, and non-parametric modelling techniques proposed in both Bayesian and frequentist settings. 
% illustrating the wide variety of literature available under the framework described in equation \eqref{eqn:MRV}. 

However, methods based on bivariate regular variation are limited in the forms of extremal dependence they can capture. This dependence can be classified through the coefficient $\chi$ \citep{Joe1997}, defined as
\begin{equation*}
    \chi := \lim_{u \to \infty}\Pr(Y > u \mid X > u) \in [0,1],
\end{equation*}
where this limit exists. If $\chi >0$, then $X$ and $Y$ are asymptotically dependent, and the most extreme values of either variable can occur simultaneously. If $\chi = 0$, $X$ and $Y$ are asymptotically independent, and the most extreme values of either variable occur separately. 

Under asymptotic independence, the spectral measure $H$ places all mass on the points $\{0\}$ and $\{1\}$; equivalently, $A(t) = 1$ for all $t \in [0,1]$. Consequently, for this form of dependence, the framework given in equation \eqref{eqn:MRV} is degenerate and is unable to accurately extrapolate into the joint tail \citep{Ledford1996,Ledford1997}. Practically, an incorrect assumption of asymptotic dependence between two variables is likely to result in an overly conservative estimate of joint risk.

To overcome this limitation, several models have been proposed that can capture both classes of extremal dependence. The first was given by \citet{Ledford1996}, in which they assume that as $u \to \infty$, the joint tail can be represented as 
\begin{equation} \label{eqn:led_tawn}
    \Pr (X > u, Y > u) = \Pr(\min(X,Y)>u) = L(e^u)e^{-u/\eta},
\end{equation}
where $L$ is a slowly varying function at infinity, i.e., $\lim_{u \to \infty}L(cu)/L(u) = 1$ for $c>0$, and $\eta \in (0,1]$. The quantity $\eta$ is termed the coefficient of tail dependence, with $\eta=1$ and $\lim_{u \to \infty}L(u) > 0$ corresponding to asymptotic dependence and either $\eta < 1$ or $\eta = 1$ and $\lim_{u\to \infty}L(u) = 0$ corresponding to asymptotic independence. Many extensions to this approach exist \citep[e.g.,][]{Ledford1997,Resnick2002,Ramos2009}; however, all such approaches are only applicable in regions where both variables are large, limiting their use in many practical settings. Since many extremal bivariate risk measures, such as environmental contours \citep{Haselsteiner2021} and return curves \citep{Murphy‐Barltrop2023}, are defined both in regions where both variables are extreme and in regions where only one variable is extreme, methods based on equation \eqref{eqn:led_tawn} are inadequate for their estimation.   

Several copula-based models have been proposed that can capture both classes of extremal dependence, such as those given in \citet{Coles2002}, \citet{Wadsworth2017} and \citet{Huser2019}. Unlike equation \eqref{eqn:led_tawn}, these can be used to evaluate joint tail behaviour in all regions where at least one variable is extreme. However, these techniques typically require strong assumptions about the parametric form of the bivariate distribution, thereby offering reduced flexibility.

\citet{Heffernan2004} proposed a modelling approach, known as the conditional extremes model, which also overcomes the limitations of the framework described in equation \eqref{eqn:led_tawn}. This approach assumes the existence of normalising functions $a:\RR_+ \to \RR$ and $b:\RR_+ \to \RR_+$ such that
\begin{equation} \label{eqn:heff_tawn}
    \lim_{u \to \infty}\Pr\left[(Y-a(X))/b(X) \leq z, \; X-u > x \mid X>u\right] = D(z)e^{-x}, \; x>0, 
\end{equation}
for a non-degenerate distribution function $D$. Note that the choice of conditioning on $X>u$ is arbitrary, and an equivalent formulation exists for normalised $X$ given $Y>u$. This framework can capture both asymptotic dependence and asymptotic independence, with the former arising when $a(x) = x$ and $b(x) = 1$, and can also be used to describe extremal behaviour in regions where only one variable is large. 

Finally, \citet{Wadsworth2013} proposed a general extension of equation \eqref{eqn:led_tawn}. As $u \to \infty$, they assume that for each ray $w \in [0,1]$, 
\begin{equation} \label{eqn:wads_tawn}
    \Pr(\min\{X/w,Y/(1-w)\}>u) = L(e^u;w)e^{-\lambda(w)u}, \;\; \lambda(w) \geq \max(w,1-w),
\end{equation}
where $L(\cdot \; ;w)$ is slowly varying for each $w \in [0,1]$ and $\lambda$ is the angular dependence function (ADF), which generalises the coefficient $\eta = 1/\{2\lambda(0.5)\}$. This extension captures both extremal dependence regimes, with asymptotic dependence implying the lower bound, i.e., $\lambda(w) = \max(w,1-w)$ for all $w \in [0,1]$. Evaluation of the ADF for rays $w$ close to $0$ and $1$ corresponds to regions where one variable is larger than the other.

The ADF can be viewed as the counterpart of the PDF for asymptotically independent variables, and shares many of its theoretical properties \citep{Wadsworth2013}. It can be used to differentiate between different forms of asymptotic independence, with both positive and negative associations captured, alongside complete independence, which implies $\lambda(w) = 1$ for all $w \in [0,1]$. Figure \ref{fig:Sec1_lambdas} illustrates the ADFs for three copulas. We observe a variety in shapes, corresponding to differing degrees of positive extremal dependence in the underlying copulas. The weakest dependence is observed for the inverted logistic copula, while the ADF for the asymptotically dependent logistic copula is equal to the lower bound. 

\begin{figure}[!h]
    \centering
    \includegraphics[width=\textwidth]{Section1/example_lambdas.pdf}
    \caption{The true ADFs (given in red) for three example copulas. Left: bivariate Gaussian copula with coefficient $\rho = 0.5$. Centre: inverted logistic copula with dependence parameter $r = 0.8$. Right: logistic copula with dependence parameter $r = 0.8$. The lower bound for the ADF is denoted by the black dotted line. }
    \label{fig:Sec1_lambdas}
\end{figure}

Despite these modelling advances, the majority of approaches for quantifying the risk of bivariate extreme events still require bivariate regular variation. Many of the approaches that do allow for asymptotic independence use the conditional extremes model of equation \eqref{eqn:heff_tawn} despite some well known limitations of this approach \citep{Liu2014}. 

One particular application of the model described in equation \eqref{eqn:wads_tawn} is the estimation of so-called bivariate return curves, $\RC{p}{} := \left\{ (x,y) \in \RR^2: \Pr(X>x,Y>y) = p \right\}$, which requires knowledge of extremal dependence in regions where either variable is large; see Section \ref{Subsec5.4}. \citet{Murphy‐Barltrop2023} obtain estimates of return curves, finding that estimates derived using equation \eqref{eqn:wads_tawn} were preferable to those from the conditional extremes model. \citet{Mhalla2019} and \citet{Murphy-Barltrop2022} also provide non-stationary extensions and inference methods for the ADF.

In this paper, we propose a global methodology for ADF estimation in order to improve extrapolation into the joint upper tail for bivariate random vectors exhibiting asymptotic independence. Until recently, the ADF has been estimated only in a pointwise manner using the Hill estimator \citep{Hill1975} on the tail of $\min\{X/w,Y/(1-w)\}$, resulting in unrealistic rough functional estimates and, as we demonstrate in Section \ref{Sec4}, high degrees of variability. Further, \citet{Murphy‐Barltrop2023} showed that pointwise ADF estimates result in non-smooth return curve estimates, which are again unrealistic.

The first smooth ADF estimator was proposed in \citet{Simpson2022} based on a theoretical link between a limit set derived from the shape of appropriately scaled sample clouds and the ADF \citep{Nolde2022}. The authors introduce global estimation techniques for the limit set, from which smooth ADF estimates follow; see Section \ref{Sec2} for further details. 

We introduce several novel smooth ADF estimators, and compare their performance with the pointwise Hill estimator, as well as the estimator given in \citet{Simpson2022}. In Section \ref{Sec2}, we review the literature on ADF estimation. In Section \ref{Sec3}, we introduce a range of novel estimators, and select tuning parameters for each proposed estimation technique. In Section \ref{Sec4}, we compare each of the available estimators through a systematic simulation study, finding certain estimators to be favourable over others. A subset of estimators are then applied to river flow data sets in Section \ref{Sec5} and used to obtain estimates of return curves for different combinations of river gauges. We conclude in Section \ref{Sec6} with a discussion. 

\section{Existing techniques for ADF estimation} \label{Sec2}

% \subsection{Existing techniques for ADF estimation} \label{Subsec2.1}
In this section, we introduce existing estimators for the ADF, with $(X,Y)$ denoting a random vector with standard exponential margins throughout. To begin, for any ray $w \in [0,1]$, define the min-projection at $w$ as $T_w:= \min\{X/w,Y/(1-w)\}$. Equation \eqref{eqn:wads_tawn} implies that for any $w \in [0,1]$ and $t > 0$, 
\begin{equation} \label{eqn:cond_wads_tawn}
    \Pr( T_w > u + t \mid T_w > u) = \frac{L(e^{u+t};w)}{L(e^u;w)} e^{-\lambda(w)t} \to e^{-\lambda(w)t} = t_{*}^{-\lambda(w)}, 
\end{equation}
as $u \to \infty$, with $t_{*} := e^t$. Since the expression in equation \eqref{eqn:cond_wads_tawn} has a univariate regularly varying tail with positive index, \citet{Wadsworth2013} propose using the Hill estimator \citep{Hill1975} to obtain a pointwise estimator of the ADF; we denote this `base' estimator $\hat{\lambda}_{H}$. A major drawback of this technique is that the estimator is pointwise, leading to rough and often unrealistic estimates of the ADF. Furthermore, this estimator need not satisfy the theoretical constraints on the ADF identified in \citet{Wadsworth2013}, such as the endpoint conditions $\lambda(0) =\lambda(1) = 1$. Moreover, no information is shared across different rays, increasing the variability in the resulting estimates. 

\citet{Simpson2022} recently proposed a novel estimator for the ADF using a theoretical link with the limiting shape of scaled sample clouds. Let $C_n := \{ (X_i,Y_i)/\log n; \; i = 1, \hdots, n\}$ denote $n$ scaled, independent copies of $(X,Y)$. \citet{Nolde2022} explain how, as $n \to \infty$, the asymptotic shape of $C_n$ provides information on the underlying extremal dependence structure. In many situations, $C_n$ converges onto the compact limit set $\bar{G} = \{(x,y): g(x,y) \leq 1\} \subseteq [0,1]^2$, where $g$ is the gauge function of $\bar{G}$. A sufficient condition for this convergence to occur is that the joint density, $f$, of $(X,Y)$ exists, and that 
\begin{equation} \label{eqn:gauge}
    -\log f(tx,ty) \sim tg(x,y), \; \; x,y \geq 0, \; t \to \infty. 
\end{equation}
Following \citet{Nolde2014}, we also define the unit-level set $G = \{(x,y): g(x,y) = 1\} \subset [0,1]^2.$ Given fixed margins, the shapes of $\bar{G}$ and $G$ are completely determined by the extremal dependence structure of $(X,Y)$. Furthermore, \citet{Nolde2022} show that the boundary set $G$ is also directly linked to the modelling frameworks described in equations \eqref{eqn:led_tawn}, \eqref{eqn:heff_tawn} and \eqref{eqn:wads_tawn}, as well as the approach of \citet{Simpson2020}. In particular, letting $R_w := (w/\max(w,1-w),\infty] \times ((1-w)/\max(w,1-w),\infty]$ for all $w \in [0,1]$, we have that 
\begin{equation} \label{eqn:gauge_lambda}
    \lambda(w) = \max(w,1-w) \times r_w^{-1},
\end{equation}
where 
\begin{equation*}
    r_w = \min \{ r \in [0,1] : rR_w \bigcap G = \emptyset \}.
\end{equation*}
The boundary sets $G$ for each of the copulas in Figure \ref{fig:Sec1_lambdas} are given in Figure \ref{fig:Sec2_gauges}, alongside the coordinates $(r_ww/\max(w,1-w),r_w(1-w)/\max(w,1-w))$ for all $w \in [0,1]$; these coordinates represent the relationship between $G$ and the ADF via equation \eqref{eqn:gauge_lambda}. One can again observe the variety in shapes. For the asymptotically dependent logistic copula, we have that $(1,1) \in G$; this is true for all asymptotically dependent bivariate random vectors. 

\begin{figure}[!h]
    \centering
    \includegraphics[width=\textwidth]{Section2/example_gauges.pdf}
    \caption{The boundary set $G$ (given in red) for three example copulas, with coordinate limits denoted by the black dotted lines and the blue lines representing the coordinates $(r_ww/\max(w,1-w),r_w(1-w)/\max(w,1-w))$ for all $w \in [0,1]$. Left: bivariate Gaussian copula with coefficient $\rho = 0.5$. Centre: inverted logistic copula with dependence parameter $r = 0.8$. Right: logistic copula with dependence parameter $r = 0.8$.}
    \label{fig:Sec2_gauges}
\end{figure}

In practice, both the limit set, $\bar{G}$, and its boundary, $G$, are unknown. \citet{Simpson2022} propose an estimator for $G$, which is then used to derive an estimator $\hat{\lambda}_{ST}$ for the ADF via equation \eqref{eqn:gauge_lambda}. The resulting estimator $\hat{\lambda}_{ST}$ was shown to outperform $\hat{\lambda}_{H}$ in a wide range of scenarios \citep{Simpson2022}. 

Estimation of $G$ uses an alternative radial-angular decomposition of $(X,Y)$, with $R^*:= X+Y$ and $V^*:=X/(X+Y)$. \citet{Simpson2022} assume the tail of $R^* \mid V^*=v^*$, $v^* \in [0,1]$, follows a generalised Pareto distribution (GPD) \citep{Davison1990} and then use generalised additive models to capture trends over angles in both the threshold and GPD scale parameter \citep{Youngman2019}. Next, high quantile estimates from the conditional distributions $R^* \mid V^*=v^*$, $v^* \in [0,1]$ are computed using the fitted GPDs. They are then transformed back to the original scale using $X=R^*V^*$ and $Y = R^*(1-V^*)$ and finally scaled onto the set $[0,1]^2$ to give an estimate of $G$; see \citet{Simpson2022} for further details. 

\citet{Wadsworth2022} also provide methodology for estimation of $G$, though their focus is on estimation of tail probabilities more generally, including in dimensions greater than two. Furthermore, their approach requires prior selection of a parametric form for $g$. We therefore restrict our attention to the work of \citet{Simpson2022} as their main focus is non-parametric estimation for $G$ in two dimensions.

When applying the estimators $\hat{\lambda}_{H}$ and $\hat{\lambda}_{ST}$ in Section \ref{Sec4} and \ref{Sec5}, we use the tuning parameters suggested in the original approaches. In the case of $\hat{\lambda}_H$, we set $u$ to be the empirical 90\% quantile of $T_w$. The default tuning parameters for $\hat{\lambda}_{ST}$ can be found in \citet{Simpson2022}, and example estimates of the set $G$ obtained using the suggested parameters are given in the Supplementary Material. 

\section{Novel estimators for the ADF} \label{Sec3}
% \subsection{Novel estimators for the ADF} \label{Subsec2.2}
Motivated by the goal of global estimation, we propose a range of novel estimators for the ADF. Since the ADF and PDF bear many theoretical similarities, we begin by reviewing estimation of the PDF. A smooth functional estimate for the ADF is desirable, so we restrict attention to approaches for the PDF which achieve this: spline-based techniques \citep{Hall2000,Cormier2014} and techniques that utilise the family of Bernstein-B\'ezier polynomials \citep{Guillotte2016,Marcon2016,Marcon2017a}. In this paper, we focus on to the latter category, since spline-based techniques typically result in more complex formulations and a larger number of tuning parameters. Moreover, approaches based on Bernstein-B\'ezier polynomials have been shown to improve estimator performance across a wide range of copula examples \citep{Vettori2018}. For functions on the interval $[0,1]$, the family of Bernstein-B\'ezier polynomials of degree $k \in \NN$ is given by 
\begin{equation*}
    \mathcal{B}_k = \left\{ \sum_{i=0}^k \beta_i \binom{k}{i} w^i(1-w)^{k-i} \; \Bigg\vert \; \pmb{\beta} \in [0,1]^{k+1}, \; w \in [0,1] \right\}.
\end{equation*}
Many approaches assume that the PDF $A \in \mathcal{B}_k$ and propose techniques for estimating the coefficient vector $\pmb{\beta}$, resulting in an estimator $\hat{\pmb{\beta}}$. This automatically ensures $A(t) \leq 1$ for all $t \in [0,1]$, thereby satisfying the theoretical upper bound of the PDF. 

We make a similar assumption about the ADF, and use this to propose novel estimators. However, unlike the PDF, the ADF is unbounded from above, meaning functions in $\mathcal{B}_k$ cannot represent all forms of extremal dependence captured by equation \eqref{eqn:wads_tawn}. Moreover, the endpoint conditions $\lambda(0) = \lambda(1) = 1$ are not necessarily satisfied by functions in $\mathcal{B}_k$. We therefore propose an alternative family of polynomials: given $k \in \NN$, let 
\begin{align} %
    \begin{split} \label{eqn:new_BP}
        \mathcal{B}^*_k = \Bigg\{ (1-w)^k + \sum_{i=1}^{k-1} \beta_i \binom{k}{i} &w^i(1-w)^{k-i} + w^k =: f(w) \; \Bigg\vert \; w \in [0,1], \\
        & \pmb{\beta} \in [0,\infty)^{k-1} \; \; \text{such that} \; \; f(w) \geq \max(w,1-w)\Bigg\}.   
    \end{split}
\end{align}
Functions in this family are unbounded from above, and $f(0) = f(1) = 1$ for all $f \in \mathcal{B}^*_k$. Furthermore, the parameter vector $\pmb{\beta}$ is constrained such that each $f \in \mathcal{B}^*_k$ satisfies the lower bound of the ADF. 

For the remainder of this section, let $\lambda(\cdot ; \; \pmb{\beta}) \in \mathcal{B}^*_k$ represent a form of the ADF from $\mathcal{B}^*_k$. Interest now lies in estimating the coefficient vector $\pmb{\beta}$, which requires choice of the degree $k \in \NN$. This is a trade-off between flexibility and computational complexity; polynomials with small values of $k$ may not be flexible enough to capture all extremal dependence structures, resulting in bias, while high values of $k$ will increase computational burden and parameter variance. 

\subsection{Composite likelihood approach} \label{Subsec3.1}
One consequence of equation \eqref{eqn:cond_wads_tawn} is that, for all $w \in [0,1]$, the conditional variable $T^*_w := (T_w - u_w \mid T_w > u_w) \sim$ Exp($\lambda(w)$), approximately, for large $u_w$. The density of this variable is $f_{T^*_w}(t^*_w) \approx \lambda(w)e^{-\lambda(w)t^*_w}, t^*_w>0$, resulting in a likelihood function for min-projection exceedances of $u_w$. Let $(\mathbf{x},\mathbf{y}) := \{ (x_i,y_i): \;i = 1, \hdots, n\}$ denote $n$ independent observations from the joint distribution of $(X,Y)$. For each $w \in \mathcal{W}$, where $\mathcal{W}$ denotes some finite subset spanning the interval $[0,1]$, let $\mathbf{t}_w := \{ \min(x_i/w,y_i/(1-w)): \; i = 1, \hdots, n\}$ and take $u_w$ to be the empirical $100q\%$ quantile of $\mathbf{t}_w$, with $q$ close to $1$ and fixed across $w$. Letting $\mathbf{t}^*_w := \{ t_w - u_w \mid  t_w \in \mathbf{t}_w, t_w > u_w \}$, we have a set of realisations from the conditional variable $T^*_w$. 

One approach to obtain an estimate of $\lambda(w)$ while considering all $w \in \mathcal{W}$ simultaneously is to use a composite likelihood, in which multiple likelihood components are treated as independent whether or not they are independent. Provided each component is a valid density function, the resulting likelihood function provides unbiased parameter estimates; see \citet{Varin2011} for further details. For this model, the likelihood function is given by
\begin{equation*}
    \mathcal{L}_C(\pmb{\beta}) = \prod_{w \in \mathcal{W}} \prod_{t^*_w \in \mathbf{t}^*_w} \lambda(w ; \; \pmb{\beta}) e^{-\lambda(w ; \; \pmb{\beta}) t^*_w} = \left[\prod_{w \in \mathcal{W}}\lambda(w;\; \pmb{\beta})^{\lvert \mathbf{t}^*_w \rvert}\right] \times e^{-\sum_{w \in \mathcal{W}}\sum_{t^*_w \in \mathbf{t}^*_w}\lambda(w;\; \pmb{\beta})t_w},
\end{equation*}
where $\lvert \mathbf{t}^*_w \rvert$ denotes the cardinality of the set $\mathbf{t}^*_w$. This composite likelihood function has equal weights across all $w \in \mathcal{W}$ (the `components'). An estimator of the ADF, $\hat{\lambda}_{CL}$, is given by $\lambda(\cdot; \; \hat{\pmb{\beta}}_{CL})$, where $\hat{\pmb{\beta}}_{CL}$ is the maximum likelihood estimator of $\pmb{\beta}$. 

To apply this method in practice, one must first select a set $\mathcal{W}$ and a value for the probability $q$; see Section \ref{Subsec3.4} for further details. The former is akin to selecting the degree of smoothing, while the latter is analogous to selecting a threshold for the GPD Defined in Section \ref{Sec2} in the univariate setting. 

\subsection{Probability ratio approach}\label{Subsec3.2}
With $\mathcal{W}$ and $\mathbf{t}_w$ defined as in Section \ref{Subsec3.1}, consider two probabilities $q<p<1$, both close to one. Given any $w \in \mathcal{W}$, let $u_w$ and $v_w$ denote the $100q\%$ and $100p\%$ empirical quantiles of $\mathbf{t}_w$, respectively. Assuming the distribution function of $T_w$ is strictly monotonic, equation \eqref{eqn:cond_wads_tawn} implies that 
\begin{equation} \label{eqn:quant_ratio}
    \frac{1-p}{1-q} = \Pr( T_w > v_w \mid T_w > u_w) \approx e^{-\lambda(w)(v_w - u_w)} \Rightarrow \left\vert \frac{1-p}{1-q} - e^{-\lambda(w)(v_w - u_w)} \right\vert \approx 0,   
\end{equation}
Similarly to \citet{Murphy-Barltrop2022}, we exploit equation \eqref{eqn:quant_ratio} to obtain an estimator for the ADF. Firstly, we observe that this representation holds for all $w \in \mathcal{W}$, hence 
\begin{equation*}
    \sum_{w \in \mathcal{W}}\left\vert \frac{1-p}{1-q} - e^{-\lambda(w)(v_w - u_w)} \right\vert \approx 0. 
\end{equation*}
To ensure equation  \eqref{eqn:quant_ratio} holds requires careful selection of $q$ and $p$. This selection also represents a bias-variance trade off: probabilities too small (big) will induce bias (high variability). Moreover, owing to the different rates of convergence to the limiting ADF \citep{Wadsworth2013}, a single pair $(q,p)$ is unlikely to be appropriate across all data structures. We instead consider a range of probability pairs simultaneously. Specifically, letting $\{(q_{j},p_{j}) \mid q_{j} < p_{j} < 1, 1 \leq j \leq m\}$, $m \in \NN$, be pairs of probabilities near one, consider the expression
\begin{equation*}
    S(\pmb{\beta}) := \sum_{w \in \mathcal{W}}\sum_{j=1}^m \left\vert   \left[\frac{1-p_j}{1-q_j} \right] - e^{-\lambda(w;\; \pmb{\beta})(v_{w,j} - u_{w,j})} \right\vert, 
\end{equation*}
in which $u_{w,j}$ and $v_{w,j}$ denote $100q_j\%$ and $100p_j\%$ empirical quantiles of $\mathbf{t}_w$, respectively, for each $j = 1, \hdots, m$. Since it is desirable for $S(\pmb{\beta})=0$, we set $\hat{\pmb{\beta}}_{PR} = \argmin_{\pmb{\beta} \in [0,\infty)^{k-1}} S(\pmb{\beta})$ and denote by $\hat{\lambda}_{PR}$ the estimator $\lambda(\cdot;\; \hat{\pmb{\beta}}_{PR})$. Similarly to $\hat{\lambda}_{CL}$, one must select the sets $\mathcal{W}$ and $\{(q_{j},p_{j}) \mid q_{j} < p_{j} < 1, 1 \leq j \leq m\}$, $m \in \NN$ prior to applying this estimator; see Section \ref{Subsec3.4}. 

\subsection{Incorporating knowledge of conditional extremes parameters}\label{Subsec3.3}
As noted in Section \ref{Sec2}, the set $G$ links different representations of bivariate extremes. Recall the conditional extremes modelling framework described in equation \eqref{eqn:heff_tawn}. Let $a_{y\mid x}$ and $a_{x\mid y}$ be the normalising functions for conditioning on the events $X>t$ and $Y>t$ respectively, and let $\alpha_{y\mid x} := \lim_{t \to \infty}a_{y\mid x}(t)/t$ and $\alpha_{x\mid y} := \lim_{t \to \infty}a_{x\mid y}(t)/t$, with $\alpha_{y\mid x},\alpha_{x\mid y} \in [0,1]$. From \citet{Nolde2022},  we have that $g(1,\alpha_{y \mid x}) = 1$ and $g(\alpha_{x \mid y},1) = 1$, with $g$ defined as in equation \eqref{eqn:gauge}. Assuming that the values of $\alpha_{y \mid x}$ and $\alpha_{x \mid y}$ are known, it follows from equation \eqref{eqn:gauge_lambda} that since the set $R_{1/(1+\alpha)}$ will always intersect the point $(1,\alpha_{y \mid x}) \in G$ for any $\alpha \in [0,\alpha_{y \mid x}]$,
\begin{equation} \label{eqn:gauge_lambda_heff_tawn}
    \lambda\left(\frac{1}{1+\alpha}\right) = \max\left(\frac{1}{1+\alpha},\frac{\alpha}{1+\alpha}\right) \times r_{1/(1+\alpha)}^{-1} = \frac{1}{1+\alpha} \times 1^{-1} = \frac{1}{1+\alpha}, 
\end{equation}
for all $\alpha \in [0,\alpha_{y \mid x}]$. Similarly, $\lambda(\alpha/(1+\alpha)) = 1/(1+\alpha)$ for all $\alpha \in [0,\alpha_{x \mid y}]$. Consequently, for all $w \in [0,\alpha^*_{x \mid y}] \bigcup [\alpha^*_{y \mid x},1]$, with $\alpha^*_{x \mid y}:=\alpha_{x \mid y}/(1+\alpha_{x \mid y})$ and $\alpha^*_{y \mid x}:=1/(1+\alpha_{y \mid x})$, $\lambda(w) = \max(w,1-w)$.  

This result is illustrated in Figure \ref{fig:gauge_lambda_result} for a Gaussian copula with $\rho=0.5$. Here, $\alpha_{x \mid y}=\alpha_{y \mid x} = 0.25$, implying $\lambda(w,1-w) = \max(w,1-w)$ for all $w \in [0,0.2] \bigcup [0.8,1]$; these rays correspond to the blue lines in the figure. One can observe that for any region $R_w$ defined along either of the blue lines (such as the shaded regions illustrated for $w = 0.1$ and $w = 0.9$), we have that $r_w = 1$, since these regions will intersect $G$ at either the coordinates $(0.25,1)$ or $(1,0.25)$.   

\begin{figure}[!h]
    \centering
    \includegraphics[width=.6\textwidth]{Section3/gauge_lambda_illustration.pdf}
    \caption{Pictoral illustration of the result described in equation \eqref{eqn:gauge_lambda_heff_tawn}. The boundary set $G$, given in red, is from the bivariate Gaussian copula with $\rho=0.5$, with the points $(1,\alpha_{y \mid x})$ and $(\alpha_{x \mid y},1)$ given in green. The blue lines represent the rays $w \in [0,\alpha^*_{x \mid y}] \bigcup [\alpha^*_{y \mid x},1]$, while the yellow and pink shaded regions represent the set $R_w$ for $w = 0.1$ and $w=0.9$, respectively. }
    \label{fig:gauge_lambda_result}
\end{figure}

In practice, $\alpha^*_{x \mid y}$ and $\alpha^*_{y \mid x}$ are unknown; however, estimates $\hat{\alpha}_{y \mid x}$ and $\hat{\alpha}_{x \mid y}$ are commonly obtained using a likelihood function based on a misspecified model for the distribution $D$ in equation \eqref{eqn:heff_tawn} \citep[e.g. ][]{Jonathan2014}. The resulting estimates, denoted $\hat{\alpha}^*_{x \mid y}$, $\hat{\alpha}^*_{y \mid x}$, can be used to approximate the ADF for $w \in [0,\hat{\alpha}^*_{x \mid y}) \bigcup (\hat{\alpha}^*_{y \mid x},1]$. What now remains is to combine this with an estimator for $\lambda(w)$ on $ [\hat{\alpha}^*_{x \mid y},\hat{\alpha}^*_{y \mid x}]$. 

A crude way to obtain an estimator via this framework would be to set $\lambda(w) = \max(w,1-w)$ for $w \in [0,\hat{\alpha}^*_{x \mid y}) \bigcup (\hat{\alpha}^*_{y \mid x},1]$ and $\lambda(w) = \hat{\lambda}_{H}(w)$, $\hat{\lambda}_{CL}(w)$ or $\hat{\lambda}_{PR}(w)$ for $w \in [\hat{\alpha}^*_{x \mid y},\hat{\alpha}^*_{y \mid x}]$. However, this results in discontinuities at $\hat{\alpha}^*_{x \mid y}$ and $\hat{\alpha}^*_{y \mid x}$. Instead, for the smooth estimators, we rescale $\mathcal{B}_k^*$ such that the resulting ADF estimate is continuous for all $w \in [0,1]$. Consider the set of polynomials given by
\begin{align} %
    \begin{split} \label{eqn:heff_tawn_BP}
         \mathcal{B}^{'}_k = \Bigg\{ &\hat{\alpha}^*_{x \mid y}\left(1- \frac{v-\hat{\alpha}^*_{x \mid y}}{\hat{\alpha}^*_{y \mid x}-\hat{\alpha}^*_{x \mid y}} \right)^k + \sum_{i=1}^{k-1} \beta_i \binom{k}{i} \left(\frac{v-\hat{\alpha}^*_{x \mid y}}{\hat{\alpha}^*_{y \mid x}-\hat{\alpha}^*_{x \mid y}}\right)^i\left(1-\frac{v-\hat{\alpha}^*_{x \mid y}}{\hat{\alpha}^*_{y \mid x}-\hat{\alpha}^*_{x \mid y}}\right)^{k-i} + \\
        & \hat{\alpha}^*_{y \mid x}\left(\frac{v-\hat{\alpha}^*_{x \mid y}}{\hat{\alpha}^*_{y \mid x}-\hat{\alpha}^*_{x \mid y}}\right)^k =:f(v) \Bigg\vert v \in [\hat{\alpha}^*_{x \mid y},\hat{\alpha}^*_{y \mid x}], \; \pmb{\beta} \in [0,\infty)^{k-1} \; \; \text{such that} \; \; f(v) \geq \max(v,1-v)\Bigg\}.   
    \end{split}
\end{align}
For all $f \in \mathcal{B}^{'}_k$, we have that $f(\hat{\alpha}^*_{x \mid y}) = \hat{\alpha}^*_{x \mid y}$ and $f(\hat{\alpha}^*_{y \mid x}) = \hat{\alpha}^*_{y \mid x}$, and each $f$ is only defined on the interval $[\hat{\alpha}^*_{x \mid y},\hat{\alpha}^*_{y \mid x}]$. Letting $\lambda^{'}(\cdot \; ; \pmb{\beta}) \in \mathcal{B}^{'}_k$ represent a form of the ADF for $w \in [\hat{\alpha}^*_{x \mid y},\hat{\alpha}^*_{y \mid x}]$, the techniques introduced in Sections \ref{Subsec3.1} and \ref{Subsec3.2} can be used to obtain estimates of the coefficient vectors, which we denote $\hat{\pmb{\beta}}_{CL2}$ and $\hat{\pmb{\beta}}_{PR2}$, respectively. The resulting estimators for $\lambda$ are given by 
\begin{equation*}
    \hat{\lambda}_{CL2}(w) = \begin{cases}
        \lambda^{'}(w;\hat{\pmb{\beta}}_{CL2}) \; & \text{for} \; w \in [\hat{\alpha}^*_{x \mid y},\hat{\alpha}^*_{y \mid x}], \\
        \max(w,1-w) \; & \text{for} \; w \in [0,\hat{\alpha}^*_{x \mid y}) \bigcup (\hat{\alpha}^*_{y \mid x},1],
    \end{cases} 
\end{equation*}
with $\hat{\lambda}_{PR2}$ defined analogously. We lastly define the discontinuous estimator $\hat{\lambda}_{H2}$ as
\begin{equation*}
    \hat{\lambda}_{H2} := \begin{cases}
        \hat{\lambda}_{H}(w) \; & \text{for} \; w \in [\hat{\alpha}^*_{x \mid y},\hat{\alpha}^*_{y \mid x}], \\
        \max(w,1-w) \; & \text{for} \; w \in [0,\hat{\alpha}^*_{x \mid y}) \bigcup (\hat{\alpha}^*_{y \mid x},1].
    \end{cases} 
\end{equation*}
This is obtained by combining the pointwise Hill estimator with the information provided by the estimates $\hat{\alpha}^*_{x \mid y},\hat{\alpha}^*_{y \mid x}$. Illustrations of all the estimators discussed in this section, as well as in Section \ref{Sec2}, can be found in the Supplementary Material. 

\subsection{Tuning parameter selection} \label{Subsec3.4}

Prior to using any of the ADF estimators introduced in this section, we are required to select at least one tuning parameter. For the probability values required by the estimators introduced in Sections \ref{Subsec3.1} and \ref{Subsec3.2}, we set $q=0.90$, $\{q_{j} \}_{j=1}^m := \{0.87 + (j-1)\times 0.002\}_{j=1}^m$ and $p_{j} = q_{j} + 0.05$ for $j = 1, \ldots, m$, with $m=31$. These values were chosen to evaluate whether the resulting estimators improve upon the base estimator $\hat{\lambda}_H$ using (approximately) the same amount of tail information in all cases. We tested a range of probabilities for both estimators and found that the ADF estimates were not massively sensitive to these across different extremal dependence structures. For example, for $\hat{\lambda}_{CL}$, a lower $q$ resulted in mild improvements for asymptotically independent copulas, while simultaneously worsening the quality of ADF estimates for asymptotically dependent examples, while a higher $q$ led to higher variance. 

We also set $\mathcal{W} := \{0, 0.001, 0.002, \hdots,0.999, 1\}$, i.e., a finite set of equally spaced rays spanning the interval $[0,1]$. This set was sufficient to ensure a high degree of smoothness in the resulting ADF estimates without too high a computational burden.  

For each of the novel estimators (except $\hat{\lambda}_{H2}$), we must also specify the degree $k \in \NN$ for the polynomial families described by equations \eqref{eqn:new_BP} and \eqref{eqn:heff_tawn_BP}. In the case of the PDF, studies have found that higher values of $k$ are preferable for very strong positive dependence, while the opposite is true for weak dependence \citep{Marcon2017a,Vettori2018}. We prefer to select a single value of $k$ that performs well across a range of dependence structures, while minimising the computational burden; this avoids the need to select this parameter when obtaining ADF estimates in practice. 

To achieve this objective, we estimated the root mean integrated squared error (RMISE), as defined in Section \ref{Subsec4.1}, of the estimators $\hat{\lambda}_{CL}$ and $\hat{\lambda}_{PR}$ with $k = 2,3,\hdots,11$ using $200$ samples from two Gaussian copula examples, corresponding to strong ($\rho=0.9$) and weak ($\rho=0.1$) positive dependence. Assessment of how the RMISE estimates vary over $k$ for both estimators suggests that $k=7$ is sufficient to accurately capture both dependence structures. The full results can be found in the Supplementary Material.  

For each of the `combined' estimators in Section \ref{Subsec3.3}, we take the same tuning parameters as for the `non-combined' counterpart, since the combined estimators have near identical formulations only defined on a subset of $[0,1]$. For example, the empirical 90\% threshold of the min-projection is used for both $\hat{\lambda}_{H}$ and $\hat{\lambda}_{H2}$. Finally, when estimating the conditional extremes parameters, empirical $90\%$ conditioning thresholds are used.

\section{Simulation Study} \label{Sec4}

\subsection{Overview}\label{Subsec4.1}
In this section, we use simulation to compare the estimators proposed in Section \ref{Sec3} to the existing techniques described in Section \ref{Sec2}. For the comparison, we introduce nine copula examples, representing a wide variety of extremal dependence structures, and encompassing both extremal dependence regimes. 

The first three examples are from the bivariate Gaussian distribution, for which the dependence is characterised by the correlation coefficient $\rho \in [-1.1]$. We consider $\rho \in \{-0.6,0.1,0.6\}$, resulting in data structures exhibiting medium negative, weak positive, and medium positive dependence, respectively. Note that in the case of $\rho = -0.6$, the choice of exponential margins will hide the dependence structure \citep{Keef2013,Nolde2022}. 

For the next two examples, we consider the bivariate extreme value copula with logistic and asymmetric logistic families \citep{Gumbel1960,Tawn1988}. In both cases, the dependence is characterised by the parameter $r \in (0,1]$; we set $r = 0.8$, corresponding to weak positive extremal dependence. For the asymmetric logistic family, we also require two asymmetry parameters $(k_1,k_2) \in [0,1]^2$, which we fix to be $(k_1,k_2) = (0.3,0.7)$, resulting in a mixture structure. 

We next consider the inverted bivariate extreme value copula \citep{Ledford1997} for the logistic and asymmetric logistic families, with the dependence again characterised by the parameters $r$ and $(r,k_1,k_2)$, respectively. We set $r = 0.4$, corresponding to moderate positive dependence, and again fix $(k_1,k_2) = (0.3,0.7)$. Note that for this copula, the model described in equation \eqref{eqn:wads_tawn} is exact: see \citet{Wadsworth2013}.

Lastly, we consider the bivariate student t copula, for which dependence is characterised by the correlation coefficient $\rho \in [-1,1]$ and the degrees of freedom $\nu > 0$. We consider $\rho = 0.8$, $\nu = 2$ and $\rho = 0.2$, $\nu = 5$, corresponding to strong and weak positive dependence. 

Illustrations of the true ADFs for each copula are given in Figure \ref{fig:true_ADFs_copulas}, showing a range of extremal dependence structures. For examples where the ADF equals the lower bound, the copula exhibits asymptotic dependence. While the fifth copula exhibits asymmetric dependence, the limiting ADF is symmetric; the same is not true for its inverted counterpart.   

\begin{figure}[!h]
    \centering
    \includegraphics[width=\textwidth]{Section4/sim_study_examples.pdf}
    \caption{True ADFs (in red) for each copula introduced in Section \ref{Subsec4.1}, along with the theoretical lower bound (black dotted line).}
    \label{fig:true_ADFs_copulas}
\end{figure}

To evaluate estimator performance, we use the RMISE 
\begin{equation*}
    \text{RMISE}\left(\hat{\lambda}_{-}\right) = \left( \mathbb{E} \left[\int_{0}^1\left\{\hat{\lambda}_{-}(w) - \lambda(w)\right\}^2 \mathrm{d}w \right] \right)^{1/2},
\end{equation*}
where $\hat{\lambda}_{-}$ denotes an arbitrary estimator. Simple rearrangement shows that this metric is equal to the square root of the sum of integrated squared bias (ISB) and integrated variance (IV) \citep{Gentle2002}, i.e.,  
\begin{equation*}
    \text{RMISE}\left(\hat{\lambda}_{-}\right) = \left[ \underbrace{\int_{0}^1\left\{ \mathbb{E}\left[\hat{\lambda}_{-}(w)\right] - \lambda(w)\right\}^2 \mathrm{d}w}_{\text{ISB}} + \underbrace{\int_{0}^1 \mathbb{E} \left( \left\{ \mathbb{E} \left[\hat{\lambda}_{-}(w)\right] - \hat{\lambda}_{-}(w)\right\}^2 \right) \mathrm{d}w}_{\text{IV}} \right]^{1/2}.
\end{equation*}
Therefore, the RMISE summarises the quality of an estimator in terms of both bias and variance, and can be used as a means to compare different estimators. 

\subsection{Results} \label{Subsec4.4}
For the copulas described in Section \ref{Subsec4.1}, data from each copula example was simulated on standard exponential margins with a sample size of $n = $10,000, and the integrated squared error (ISE) of each estimator was approximated for $1,000$ samples using the trapezoidal rule; see the Supplementary Material for further information. The square root of the mean of these estimates was then computed, resulting in a Monte-Carlo estimate of the RMISE. For each of the estimators $\hat{\lambda}_{CL}, \hat{\lambda}_{PR}, \hat{\lambda}_{CL2}$ and $\hat{\lambda}_{PR2}$, the optimisation functions were constrained to ensure the resulting ADF estimates satisfy the theoretical lower bound. The RMISE estimates for each estimator and copula combination are shown in Table \ref{table:RMISE_values}. Tables for ISB and IV can be found in the Supplementary Material. For each estimator, the bias varies significantly across the different copulas. However, in the majority of cases, the bias/variance are similar across most of the estimators. 


\renewcommand{\arraystretch}{1.5}
\begin{table}[!h]

\caption{RMISE values (multiplied by 100) for each estimator and copula combination. Smallest RMISE values in each row are highlighted in bold, with values reported to 3 significant figures.
\label{table:RMISE_values}}
\centering
\begin{tabular}[t]{c|c|c|c|c|c|c|c}
\hline
Copula & $\hat{\lambda}_{H}$ & $\hat{\lambda}_{CL}$ & $\hat{\lambda}_{PR}$ & $\hat{\lambda}_{H2}$ & $\hat{\lambda}_{CL2}$ & $\hat{\lambda}_{PR2}$ & $\hat{\lambda}_{ST}$\\
\hline
Copula 1 & \textbf{61.1} & 61.3 & 66.2 & 61.4 & 61.9 & 66.7 & 63.7\\
\hline
Copula 2 & 3.55 & 3.33 & 3.64 & 3.51 & 3.33 & 3.63 & \textbf{2.95}\\
\hline
Copula 3 & 3.78 & 3.48 & 3.84 & 3.27 & 3.22 & 3.57 & \textbf{1.09}\\
\hline
Copula 4 & 4.9 & 4.79 & 6.92 & 4.28 & 4.25 & 6.17 & \textbf{2.77}\\
\hline
Copula 5 & 14.1 & 14.1 & 17.1 & 14.1 & 14.1 & 17 & \textbf{12.1}\\
\hline
Copula 6 & 2.51 & 1.97 & 2.15 & 2 & \textbf{1.74} & 1.9 & 2.12\\
\hline
Copula 7 & 2.93 & \textbf{2.64} & 2.88 & 2.87 & 2.66 & 2.89 & 3.96\\
\hline
Copula 8 & 2.49 & 2.72 & 2.95 & 0.66 & \textbf{0.6} & 0.789 & 1.87\\
\hline
Copula 9 & 12.1 & 12 & 14.9 & 12 & 12 & 14.9 & \textbf{11.1}\\
\hline
\end{tabular}
\end{table}

While no estimator consistently outperforms the others, $\hat{\lambda}_{CL2}$ and $\hat{\lambda}_{ST}$ tend to have lower RMISE, ISB and IV values, on average. This is especially the case when comparing to the base estimator $\hat{\lambda}_H$. Furthermore, the `combined' estimators outperform their non-combined counterparts in many cases, suggesting that incorporating parameter estimates from the conditional extremes model can reduce bias and variance. The Gaussian copula with $\rho = -0.6$ has significantly higher RMISE values, indicating that none of the estimators capture negative dependence well, though this is in part due to the choice of exponential margins. 

Overall, these results indicate that no one estimator is preferable across all extremal dependence structures. However, we suggest using the estimators $\hat{\lambda}_{CL2}$ and $\hat{\lambda}_{ST}$ since, on average, these appeared to result in less bias and variance. The form of extremal dependence appears to affect the performance of both of these estimators; since this is often difficult to quantify a priori, we suggest using both estimators and evaluating relative performance via diagnostics, as we do in Section \ref{Sec5}. 

\section{Case Study} \label{Sec5}
\subsection{Overview} \label{Subsec5.1} 
Understanding the probability of observing extreme river flow events (i.e., floods) at multiple sites simultaneously is important in a variety of sectors, including insurance \citep{Keef2013a,Quinn2019,Rohrbeck2021a} and environmental management \citep{Lamb2010,Gouldby2017}. Valid risk assessments therefore require accurate evaluation of the extremal dependence between different sites. 

In this section, we estimate the ADF of river flow data sets obtained from gauges in the north of England, UK, which can be subsequently used to construct bivariate return curves. Daily average flow values ($m^3/s$) at six river gauge locations on different rivers were considered. The gauge sites are illustrated in Figure \ref{fig:riv_spatial_plot}. For each location, data is available between May 1993 and September 2020; however, we only consider dates where a measurement is available for every location. To avoid seasonality, we consider the interval October-March only; from our analysis, it appears that the highest daily flow values are observed in this period. This results in $n=4,659$ data points for each site. Plots of the daily flow time series can be found in the Supplementary Material; these plots suggest that an assumption of  stationarity is reasonable for the extremes of each data set. 

\begin{figure}[!h]
    \centering
    \includegraphics[width=.7\textwidth]{Section5/river_plot.pdf}
    \caption{Locations of river gauges in the north of England, UK. Individual rivers illustrated in blue alongside the corresponding gauge locations.}
    \label{fig:riv_spatial_plot}
\end{figure} 

We fix the site on the river Lune to be our reference site and consider the extremal dependence between this and all other gauges. We first estimate the extremal dependence measure $\chi$ and the coefficient $\eta$ using the upper $10\%$ of the corresponding joint tails. Both $\chi$ and $\eta$ are limiting values; however, in practice, we are unable to evaluate such limits without a closed form for the joint distribution. We therefore calculate these values empirically. Taking $\chi$, for example, an estimate is given by $\hat{\chi}_{q} = \hat{\Pr}(X>\hat{x}_{q},Y>\hat{y}_{q})/\hat{\Pr}(X>\hat{x}_{q})$, where $\hat{\Pr}(\cdot)$ denotes an empirical probability estimate and $\hat{x}_{q}$ and $\hat{y}_{q}$ denote empirical $100q$\% quantiles estimates for the variables $X$ and $Y$, respectively, and $q$ is some value close to $1$. Specifically, we take $q=0.9$. In practice, we are unlikely to observe $\chi = \hat{\chi}_{q}$, even at the most extreme quantile levels, i.e., as $q \to 1$. This can be problematic when trying to quantify the form of extremal dependence, since $\hat{\chi}_q>0$ may arise for asymptotically independent data sets (for example). Therefore, the estimated coefficients should act only as a rough guide for this quantification.

The dependence measure estimates and $95\%$ confidence intervals are shown in Figure \ref{fig:dep_coeff} as a function of distance from the reference site. Here and throughout, all confidence intervals are obtained via block bootstrapping with block size $b=40$; this value appears appropriate to account for the varying degrees of temporal dependence observed across the six gauge sites. These estimates suggest that asymptotic independence may exist for at least three of the site pairings; therefore, modelling techniques based on bivariate regular variation would likely fail to capture the observed extremal dependencies in this scenario. 


\begin{figure}[!h]
    \centering
    \includegraphics[width=\textwidth]{Section5/dep_coeff.pdf}
    \caption{Estimated dependence coefficients as a function of distance from the Lune gauge, with $95\%$ pointwise confidence intervals given by the shaded regions. Left: Estimates of $\chi$ (blue). Right: Estimates of $\eta$ (grey).}
    \label{fig:dep_coeff}
\end{figure}

\subsection{ADF Estimation} \label{Subsec5.2} 

We transform each marginal data set to exponential margins using the semi-parametric approach of \citet{Coles1991}, whereby a GPD is fitted to the upper tail and the body is modelled empirically. The GPD thresholds are selected using the technique proposed in \citet{Varty2021}. Diagnostic plots found in the Supplementary Material indicate good model fits. Since our results from Section \ref{Sec4} suggest that the estimators $\hat{\lambda}_{CL2}$ and $\hat{\lambda}_{ST}$ perform best overall, we used these, alongside the base estimator $\hat{\lambda}_{H}$, to estimate the ADF for each combination of the reference gauge and the other five gauges. The resulting ADF estimates can be found in Figure \ref{fig:adf_ests}.

\begin{figure}[!h]
    \centering
    \includegraphics[width=\textwidth]{Section5/adf_ests_sub.pdf}
    \caption{ADF estimates for each pair of gauge sites. The purple, pink and green lines represent the estimates from $\hat{\lambda}_{H}$, $\hat{\lambda}_{CL2}$ and $\hat{\lambda}_{ST}$, respectively, with the theoretical lower bound denoted by the black dotted lines.}
    \label{fig:adf_ests}
\end{figure}

For the most part, there is decent agreement across the estimators. One can observe contrasting shapes across the different pairs of gauges, illustrating the variety in the observed extremal dependence structures. We note that the estimates obtained using $\hat{\lambda}_H$ demonstrate some of the drawbacks mentioned in Section \ref{Sec2}, such as non-smoothness and not respecting theoretical results for the ADF. Moreover, these results illustrate that on the whole, the estimator $\hat{\lambda}_{CL2}$ is very much a smoothed version of $\hat{\lambda}_{H}$, owing to the form of likelihood function used.

\subsection{Assessing goodness of fit for ADF estimates} \label{Subsec5.3} 
Recall that, from equation \eqref{eqn:cond_wads_tawn}, we have $T^*_w \sim \text{Exp}(\lambda(w))$ as $u_w \to \infty$ for all $w \in [0,1]$. We exploit this result to assess the goodness of fit for ADF estimates. 

Let $\hat{\lambda}(w), w \in [0,1]$, denote an estimated ADF obtained using the sample $\{ (x_i,y_i): \;i = 1, \hdots, n\}$. Given $w \in [0,1]$, let $u_w$ denote some high empirical quantile from the sample $\mathbf{t}_w$, and consider the sample $\mathbf{t}^*_w$, with $\mathbf{t}_w$ and $\mathbf{t}^*_w$ defined as in Section \ref{Subsec3.2}. If $\mathbf{t}^*_w$ is indeed a sample from an $\text{Exp}(\hat{\lambda}(w))$ distribution, we would expect good agreement between the empirical and model quantiles. Letting $n_w = \lvert\mathbf{t}^*_w\rvert$ and $\mathbf{t}^*_{w(j)}$ denote the $j$-th order statistic of $\mathbf{t}^*_{w}$, we consider the set of pairs
\begin{equation*} 
    \left\{ \left( \frac{ -\log(1-j/(n_w + 1))}{\hat{\lambda}(w)} , \mathbf{t}^*_{w(j)} \right) : j = 1, \hdots, n_w \right\},
\end{equation*}
for different rays $w \in [0,1]$, corresponding to a range of joint survival regions. With $u_w$ fixed to be the 90\% empirical quantile of $\mathbf{t}_w$, quantile-quantile (QQ) plots for five individual rays, $w \in \{0.1,0.3,0.5,0.7,0.9\}$, are given in Figure \ref{fig:adf_diag} for the first pair of gauges and the $\hat{\lambda}_{CL2}$ estimator. Uncertainty intervals are obtained via block bootstrapping. On the whole, the estimated exponential quantiles appear in good agreement with the observed quantiles, indicating the underlying ADF estimate accurately captures the tail behaviour for each min-projection variable. Analogous plots for $\hat{\lambda}_H$ and $\hat{\lambda}_{ST}$ are given in the Supplementary Material. Similar plots were obtained when the other pairs of gauges were considered. 

\begin{figure}[!h]
    \centering
    \includegraphics[width=\textwidth]{Section5/adf_diag_plots.pdf}
    \caption{Individual ADF QQ plots for $w \in \{0.1,0.3,0.5,0.7,0.9\}$ for first pair of gauges, using the ADF estimate obtained via $\hat{\lambda}_{CL2}$. Estimates given in black, with 95\% pointwise confidence intervals represented by the grey shaded regions. Red lines correspond to the $y=x$ line.}
    \label{fig:adf_diag}
\end{figure}

\subsection{Estimating return curves} \label{Subsec5.4} 
To quantify the risk of joint flooding events across sites, we follow \citet{Murphy‐Barltrop2023} and use the ADF to estimate a bivariate risk measure known as a return curve, as defined in Section \ref{Sec1}. This measure is the direct bivariate extension of a return level, which is commonly used to quantify risk in the univariate setting \citep{Coles2001}. Taking probability values close to zero gives a summary of the joint extremal dependence, thus allowing for comparison across different data structures. In the context of extreme river flows, return curves can be used to evaluate at which sites joint extremes (floods) are more/less likely to occur. For illustration, we fix $p$ to correspond to a $5$ year return period, i.e., $p = 1/(5n_y)$, where $n_y$ is the average number of points observed in a given year \citep{Brunner2016}. Excluding missing observations, we have $28$ years of data, hence the resulting curve should be within the range of data whilst simultaneously representing the joint tail. The resulting return curve estimates for each ADF estimator and pair of gauge sites can be found in Figure \ref{fig:original_return_curves}. 

\begin{figure}[!h]
    \centering
    \includegraphics[width=\textwidth]{Section5/return_curves_original_sub.pdf}
    \caption{Estimated 5-year return curves (on original margins) for each pair of gauges. The purple, pink and green lines represent the curve estimates from $\hat{\lambda}_{H}$, $\hat{\lambda}_{CL2}$ and $\hat{\lambda}_{ST}$, respectively.}
    \label{fig:original_return_curves}
\end{figure}

There is generally good agreement among the estimated curves. The almost-square shapes of the estimates for the first two pairs of gauges indicate higher likelihoods of observing simultaneous flood events at the corresponding gauge sites; this is as expected given the close spatial proximity of these sites. In all cases, the curves derived via $\hat{\lambda}_H$ are quite rough and unrealistic, and are subsequently ignored. To assess the goodness of fit of the remaining return curve estimates, we consider the first and fifth examples and apply the diagnostic introduced in \citet{Murphy‐Barltrop2023}. Our results suggest good quality model fits for both of the estimates obtained using $\hat{\lambda}_{CL2}$ and $\hat{\lambda}_{ST}$. Furthermore, we also obtain $95\%$ return curve confidence intervals for these examples. The resulting plots illustrating the diagnostics and confidence intervals, along with a brief explanation of the diagnostic tool, are given in the Supplementary Material.

\section{Discussion} \label{Sec6}
We have introduced a range of novel global estimators for the ADF, as detailed in Section \ref{Sec3}. We compared these estimators to existing techniques through a systematic simulation study and found our novel estimators to be competitive in many cases. In particular, the estimators derived via the composite likelihood approach of Section \ref{Subsec3.1}, alongside the estimator of \citet{Simpson2022}, appear to have lower bias and variance, on average, compared to alternative estimation techniques. We also applied ADF estimation techniques to a range of river flow data sets, and obtained estimates of return curves for each data set. The results suggest that our estimation procedures are able to accurately capture the range of extremal dependence structures exhibited in the data, allowing for a more robust risk analysis of joint flooding events. 

From Section \ref{Sec4}, one can observe that the `combined' estimators proposed in Section \ref{Subsec3.3} outperform their `uncombined' counterparts in the majority of instances. This indicates that incorporating the knowledge obtained from the conditional extremes parameters leads to improvements in ADF estimates. Furthermore, in most cases, ADF estimates obtained via approximations of the set $G$ appeared to have lower bias compared to alternative estimation techniques. More generally, these results suggest that inferential techniques that exploit the results of \citet{Nolde2022} are superior to techniques which do not. Estimation of $G$, and its impact on estimation of other extremal dependence properties, represents an important line of research.

As noted in Section \ref{Sec1}, few applications of the modelling framework described in equation \eqref{eqn:wads_tawn} exist, even though this model offers advantages over the widely used approach of \citet{Heffernan2004} when evaluating joint tail probabilities. Inference via the ADF ensures consistency in extremal dependence properties, and one can obtain accurate estimates of certain risk measures, such as return curves. 

For each of the existing and novel estimators introduced in Sections \ref{Sec2} and \ref{Sec3}, we were required to select quantile levels, which is equivalent to selecting thresholds of the min-projection. With the exception of $\hat{\lambda}_{ST}$, similar quantile levels were considered for each estimator so as to provide some degree of comparability. However, due to variation in estimation procedures, we acknowledge that the selected quantile levels are not readily comparable since the quantity of joint tail data used for estimation varies across different estimators. Moreover, as noted in Section \ref{Subsec3.4}, trying to select `optimal' quantile levels appears a fruitless exercise since the performance of each estimator does not appear to alter much across different quantile levels.


Finally, we acknowledge the lack of theoretical treatment for the proposed ADF estimators which is an important consideration for understanding properties of the methodology. However, theoretical results of this form typically require in-depth analyses and strict assumptions, which themselves may be hard to verify, whilst in practice one can only ever look at diagnostics obtained from the data. We have therefore opted for a more practical treatment of the proposed estimators.

\section*{Supplementary Material}
\begin{itemize}
    \item \textbf{Supplementary Material for ``Improving estimation for asymptotically independent bivariate extremes via global estimators for the angular dependence functions"}: File containing additional figures and tables. (.pdf file)
    \item \textbf{Code and data}: Zip file containing an R script and the case study data sets. The script can be used to reproduce case study results from Section \ref{Sec5}. (.zip file)
\end{itemize}



\section*{Declarations}
\subsection*{Funding}
This paper is based on work completed while Callum Murphy-Barltrop was part of the EPSRC funded STOR-i centre for doctoral training (EP/L015692/1). 
\subsection*{Competing interests}
The authors have no relevant financial or non-financial interests to disclose.
\subsection*{Data availability}
The data sets analysed in Section \ref{Sec5} are freely available online from the National River Flow Archive \citep{National2022}. The gauge ID numbers for the six considered stations are as follows: 
River Lune - 72004, River Wenning - 72009, River Kent - 73012, River Irwell - 69002, River Aire - 27028, River Derwent - 23007

\bibliography{library,nrfa}

\end{document}
