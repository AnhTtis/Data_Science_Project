\section{Methodology} \label{sec:Methodology}

\begin{figure}[tb]
    \centering
    % \includegraphics[width=\linewidth]{Figure/SMT_circuit.pdf}
    \includegraphics[width=\linewidth]{Figure/SMT-Circuit.pdf}
    \caption{Quantum circuit for our $\mathcal{BV}$ SMT solver}
    \label{fig:SMT_circuit}
\end{figure}

In this section, we introduce our quantum SMT solver for the $\mathcal{BV}$ theory based on Grover's algorithm. Fig.~\ref{fig:SMT_circuit} shows its block diagram, echoing the overall structure in Fig~\ref{fig:Oracle_High-level}~{(a)}. As mentioned in Section~\ref{sec:Introduction}, the oracle is the key component for marking the solutions so that the difusser knows which elements to increase/maximize their probability to be measured. Since the diffuser is standard and independent from the search problem, we will not discuss it further. Instead, in the following, we put emphasise on how to construct the oracle.



\subsection{SAT Circuit} \label{subsec:SAT-Circuit-Degisn}

Fernandes et al.~\cite{FS19} showed an example of constructing the oracle circuit for a $3$-SAT problem and solved the problem based on Grover’s algorithm. However, they did not provide a systematic way to construct the oracle circuit. In this section, we propose a method to constructively generate the oracle circuit for an arbitrary $3$-SAT formula and prove its correctness. Consider the following grammar for $3$-SAT formulas in conjunctive normal form (CNF):
\[
\begin{matrix}
    F & \simeq & C \mid F_1 \wedge F_2 \\
    C & \simeq & l_1 \vee l_2 \vee l_3 \\
    l & \simeq & v \mid \neg v \mid 0
\end{matrix}
\]

Fig.~\ref{fig:SAT_clause_circuit} shows how to construct a quantum circuit for a clause $C: l_1 \vee l_2 \vee l_3$. Notice that the $Q$ gate depends on each literal $l_i$. If $l_i$ is a negative literal $\neg v_i$, $Q$ would be the $\mathtt{I}$ gate, i.e., the identity gate; otherwise, $Q$ would be the $\mathtt{X}$ gate, also called the $\mathtt{NOT}$ gate. The qubit $q_a(C)$ is an ancilla bit\footnote{The notation $q_a(C)$ is not a function application. It just represents that $q_a$ is the ancilla bit of clause $C$. Similarly, $q_o(C)$ represents the output bit of clause $C$, and $q_o(F)$ represents the output bit of formula $F$.} for internal computation, and $q_o(C)$ is the output bit for the truth value of $C$. Theorem~\ref{thm:SAT-Clause} proves that the quantum circuit construction for a clause $C$ is correct.

\begin{theorem} \label{thm:SAT-Clause}
{\bf [Clause Correctness]} (1) $q_o'(C) = 1$ $\iff$ clause $C$ is true, (2) $q_a'(C) = 0$, and (3) $v_i' = v_i$ for all $i \in \{1,2,3\}$.
\end{theorem}

\begin{proof}
Given a clause $C: l_1 \vee l_2 \vee l_3$, since if $l_i$ is a negative literal $\neg v_i$, $Q$ would be the $\mathtt{I}$ gate; otherwise, $Q$ would be the $\mathtt{X}$ gate, we have $Q(v_i) = \neg l_i$, as the red notations in Fig.~\ref{fig:SAT_clause_circuit}. Let $a$ be the value of $q_a(C)$ after apply the first $\mathtt{CCNOT}$ gate on it.

To prove condition~${(1)}$: $q_o' (C) = 0 \Leftrightarrow (\neg l_3 = 1) \wedge (a = 1)$ $\iff$ $(\neg l_3 = 1) \wedge (\neg l_1 = 1) \wedge (\neg l_2 = 1)$. If we apply negation on both side, we have
$\neg (q_o' (C) = 0) \Leftrightarrow \neg ((\neg l_1 = 1) \wedge (\neg l_2 = 1) \wedge (\neg l_3 = 1))$. Thus,
$q_o' (C) = 1 \Leftrightarrow \neg (\neg l_1 = 1) \vee \neg (\neg l_2 = 1) \vee \neg (\neg l_3 = 1)$. That is, 
$q_o' (C) = 1 \Leftrightarrow (l_1 = 1) \vee (l_2 = 1) \vee (l_3 = 1)$.

To prove condition~${(2)}$: $q_a' (C) = (\neg l_1 \wedge \neg l_2) \oplus a$. Since $a$ is obtained by applying $\mathtt{CCNOT}$ gate on $q_a(C)$, with $\neg l_1$ and $\neg l_2$ as the control bits, we have
$q_a' (C) = (\neg l_1 \wedge \neg l_2) \oplus ((\neg l_1 \wedge \neg l_2) \oplus 0)$
$= (\neg l_1 \wedge \neg l_2) \oplus (\neg l_1 \wedge \neg l_2) = 0$.

To prove condition~${(3)}$: $v_i' = Q(\neg l_i) = Q(Q(v_i)) = v_i$, because $QQ = I$, as $Q$ is either $\mathtt{X}$ or $\mathtt{I}$ gate. \qed
\end{proof}

\begin{figure}[tb]
\centering
\begin{minipage}{0.56 \linewidth}
\centering
\includegraphics[width=0.9\linewidth]{Figure/SAT_clause_circuit.pdf}
\end{minipage}
~
\begin{minipage}{0.4\linewidth}
\small
\begin{displaymath}
    Q: \left\{
    \begin{array}{ll}
      \mathtt{X} \mbox{ , if} & l_i \mbox{ is } v_i \mbox{ or } 0 \\
      \mathtt{I} \mbox{ , if} & l_i \mbox{ is } \neg v_i
    \end{array}
    \right.
\end{displaymath}
\end{minipage}
\caption{Quantum circuit for a clause $C: l_1 \vee l_2 \vee l_3$}
\label{fig:SAT_clause_circuit}
\end{figure}


Fig.~\ref{fig:SAT_circuit} shows how to construct a quantum circuit for a formula $F: F_1 \wedge F_2$. It is required to construct the quantum circuits for $F_1$ and $F_2$ first, which are then conjuncted by a $\mathtt{CCNOT}$ gate. The qubit $q_o(F)$ is the output bit for the truth value of formula $F$. Theorem~\ref{thm:SAT-Formula} proves that the quantum circuit construction for a formula $F$ is correct.

\begin{theorem} \label{thm:SAT-Formula}
{\bf [Formula Correctness]} (1) $q_o'(F) = 1$ $\iff$ formula $F$ is true, (2) $q_a'(F) = 0$, and (3) $v_i' = v_i$ for all $i \in \{1,2, \ldots, n\}$.

\end{theorem}

\begin{proof}
We prove this theorem by structural induction on the grammar to generate an arbitrary formula $F$.
The basic case, when $F$ is a clause $C$, has been proved in Theorem~\ref{thm:SAT-Clause}.

Induction assumption : Let $F_1$ and $F_2$ be two formulas satisfying the three conditions: (a) $q_o'(F_1) = 1$ $\iff$ formula $F_1$ is true, and $q_o'(F_2) = 1$ $\iff$ formula $F_2$ is true. (b) $q_a'(F_1) = 0$ and $q_a'(F_2) = 0$. (c) $v_i' = v_i$ for $i \in \{1,2,\ldots, n\}$.

Consider the induction step, the circuit for $F: F_1 \wedge F_2$ is constructed based on $F_1$ and $F_2$, as shown in Figure \ref{fig:SAT_circuit}. 

To prove condition~{(1)}: Because of the $\mathtt{CCNOT}$ gate, $q_o' (F) = 1 \Leftrightarrow ((q_o' (F_1) = 1) \wedge (q_o' (F_2) = 1)) \oplus 0$ $ \Leftrightarrow (q_o' (F_1) = 1) \wedge (q_o' (F_2) = 1)$. Based on induction assumption~{(b)}, we can conclude that $q_o' (F) = 1$ $\iff$ $F_1$ is true and $F_2$ is true $\iff$ $F$ is true.

To prove condition~{(2)}: $q_a' (F) = q_a' (F_2) = 0$, according to the induction assumption~{(b)}.

To prove condition~{(3)}: Since $v_i'$ of $F$ is equal to $v_i'$ of $F_2$, based on induction assumption~{(c)}, $v_i' = v_i$ for all $i \in \{1,2,\ldots, n\}$. \qed
\end{proof}

\begin{figure}[tb]
    \centering
    \includegraphics[width=0.65\linewidth]{Figure/SAT_circuit.pdf}
    \caption{Quantum circuit construction for a formula $F$}
    \label{fig:SAT_circuit}
\end{figure}

\subsection{Theory Circuit} \label{subsec:Theory-Circuit}
Now, we illustrate how to construct the theory circuit for atoms consisting of arithmetic and comparison operations. Consider the syntax grammar for an arbitrary atom, as shown in Fig.~\ref{fig:BV-Syntax}. An expression could be a variable $\Tilde{a}$, a constant $\mathcal{C}$, or the result of an arithmetic operation between two expressions $E_1$ and $E_2$. Constructing the circuit for a variable $\Tilde{a}$ or a constant $\mathcal{C}$ is trivial, which is omitted here. The rest is the case of $\Tilde{a} \odot \Tilde{b}$. Here, we do not list all the arithmetic operations, as there are many. They can either be constructed based on primitive quantum gates or were developed in related works (c.f.~Section~\ref{sec:RelatedWorks}). Instead, we abstract the circuit for arithmetic operations as the general one, shown in Fig.~\ref{fig:arithmetic-comparator}~{(a)}, for easy illustration. Depending on different arithmetic operations, one can obtain its corresponding final circuit constructively in a bottom-up manner.

% For the word concatenation and sub-word selection operation, it can be easily implemented by connecting the corresponding qubits to the arithmetic circuit or comparator circuit. 
% For the modulo sum, multiplication and shift operation, the circuits proposed by \cite{CS08}, \cite{KT14}, \cite{KT11} can be adopted to provide those operations respectively. 
% The bitwise operations such as AND, OR, XOR and NOT can be implemented by using the $NOT$ gate, $CNOT$ gate and $CCNOT$ gate. 
% Take OR operation for example, both of the two operands first apply NOT operation and then become the inputs of control qubits of $CCNOT$ gate. By setting the input of the target qubit to 1, it will output the OR operation result at the output of the target qubit of the $CCNOT$ gate.
% While obtaining the result of each $\mathcal{BV}$ term, they will be sent to the comparator circuit. 

% There is a topic that needs to be concerned when applying the arithmetic circuit is that the original operand may no longer exist after the circuit’s computation. 
% If the variable is necessary for the succeeding procedure, there are two ways to solve this problem. 
% The first approach is to use the reversible nature of quantum computing. 
% If the calculation result can be stored in other qubits, a reversed arithmetic circuit can restore the operand qubits to their original state. 
% The other way is to duplicate the input or the output of the arithmetic circuit before or after its calculation. 
% The $CNOT$ gate can be used to implement the duplication by placing the variable at the control qubit and setting the input of the target qubit to 0. 
% Which approach is more appropriate depending on the adopted arithmetic circuit. 
% Take the adder \cite{CS08} we use for example. 
% Since the carry qubits are necessary during the calculation procedure but become useless after the computation, we can reverse the control and target qubit of the last layer’s $CNOT$ gates to store the summation result in those carry qubits, and then add $CNOT$ gates between two operands to restore the variable (which is variable $b$ in this circuit) back to its original condition. The circuit of the modified 2 bit adder is shown in Figure \ref{fig:modified_adder}.

% \begin{figure}[tb]
%     \centering
%     \includegraphics[width=0.7\linewidth]{Figure/modified_adder.pdf}
%     \caption{The circuit of modified 2-bit adder.}
%     \label{fig:modified_adder}
% \end{figure}

For each atom of the form $E_1 \rhd E_2$, we adopt the comparator~\cite{OR07} to compare $E_1$ and $E_2$. The overall structure of the comparator is shown in Fig.~\ref{fig:arithmetic-comparator}~{(b)}. Of course, the output $E_1 \odot E_2$ of the arithmetic circuit would be the input of the comparator circuit. The two output bits $O_1$ and $O_2$ of the comparator indicate the relation between $E_1$ and $E_2$, as follows. Notice that the case of $(1,1)$ does not exist in the comparator design~\cite{OR07}.

\begin{displaymath}
(O_1, O_2) = \left\{ 
\begin{matrix}
(0,1) & \mbox{if} & E_1 < E_2 \\
(1,0) & \mbox{if} & E_1 > E_2 \\
(0,0) & \mbox{if} & E_1 = E_2
\end{matrix}
\right.
\end{displaymath}

Based on the two output bits $O_1$ and $O_2$, one can construct the corresponding atom for each of the six different cases, as shown in Fig.~\ref{fig:atom_circuit}~{(a)}--{(f)}. Theorem~\ref{thm:Atom-correctness} shows the correctness of our circuit construction for atoms.

\begin{theorem} \label{thm:Atom-correctness}
{\bf [Atom correctness]} For each atom $E_1 \rhd E_2$ and its corresponding $atom$ bit, we have $atom = 1 \iff (E_1 \rhd E_2) = 1$.
\end{theorem}

\begin{proof}
The proof is straightforward by examining the $atom$ output for each case based on the truth table of primitive quantum gates. We omit it here due to the page limit. \qed
\end{proof}

% \begin{enumerate}
%     \item {$(\mathbf{a} > \mathbf{b})$}: $|atom\rangle = |O_1 \rangle$
%     \item {$(\mathbf{a} < \mathbf{b})$}: $|atom\rangle = |O_2 \rangle$
%     \item {$(\mathbf{a} = \mathbf{b})$}: $|atom\rangle = |\neg O_1 \wedge \neg O_2 \rangle$ \\
%     $|O_1 O_2 \rangle \otimes |atom\rangle$
%     $= (NOT^{\otimes 2} \otimes I) CCNOT (NOT^{\otimes 2} \otimes I) (|O_1 O_2 \rangle \otimes |0\rangle)$ \\
%     $= (NOT^{\otimes 2} \otimes I) CCNOT (|\neg O_1 \neg O_2 \rangle \otimes |0\rangle)$ \\
%     $= (NOT^{\otimes 2} \otimes I) (|\neg O_1 \neg O_2 \rangle \otimes |\neg O_1 \wedge \neg O_2\rangle) = (|O_1 O_2 \rangle \otimes |\neg O_1 \wedge \neg O_2\rangle)$
%     \item {$(\mathbf{a} \neq \mathbf{b})$}: $|atom\rangle = |O_1 \vee O_2 \rangle$ \\
%     $|O_1 O_2 \rangle \otimes |atom\rangle$
%     $= (NOT^{\otimes 2} \otimes I) CCNOT (NOT^{\otimes 2} \otimes I) (|O_1 O_2 \rangle \otimes |1\rangle)$ \\
%     $= (NOT^{\otimes 2} \otimes I) CCNOT (|\neg O_1 \neg O_2 \rangle \otimes |1\rangle)$ \\
%     $= (NOT^{\otimes 2} \otimes I) (|\neg O_1 \neg O_2 \rangle \otimes |(\neg O_1 \wedge \neg O_2) \oplus 1\rangle) = (|O_1 O_2 \rangle \otimes |O_1 \vee O_2\rangle)$
%     \item {$(\mathbf{a} \leq \mathbf{b})$}: $|atom\rangle = |(\neg O_1 \wedge \neg O_2) \vee O_2 \rangle = |\neg O_1 \rangle$, since $|O_1 O_2 \rangle = |11 \rangle$ does not exists.
%     \item {$(\mathbf{a} \geq \mathbf{b})$}: $|atom\rangle = |(\neg O_1 \wedge \neg O_2) \vee O_1 \rangle = |\neg O_2 \rangle$, since $|O_1 O_2 \rangle = |11 \rangle$ does not exists.
% \end{enumerate}

\begin{figure}[tb]
\begin{minipage}{0.47\linewidth}
\centering
\includegraphics[width=0.6\linewidth]{Figure/Arithmetic_Circuit.pdf} 
\end{minipage}
~
\begin{minipage}{0.47\linewidth}
\centering
\includegraphics[width=0.55\linewidth]{Figure/Comparator-circuit.pdf} 
\end{minipage}
\\[2mm]
\begin{minipage}{0.47\linewidth}
\centering
\footnotesize
(a) Arithmetic Circuit
\end{minipage}
~
\begin{minipage}{0.47\linewidth}
\centering
\footnotesize
(b) Comparator Circuit
\end{minipage}
\caption{Theory Circuit}
\label{fig:arithmetic-comparator}
\end{figure}

\begin{figure}[tb]
\centering
\includegraphics[width=0.96\linewidth]{Figure/Atom-circuit.pdf}
\caption{Quantum circuit construction for atoms}
\label{fig:atom_circuit}
\end{figure}

% \begin{comment}
%     \begin{enumerate}
%         \item {$(\mathbf{a} > \mathbf{b})$}: $|atom\rangle = |O_1 \rangle$
%         \item {$(\mathbf{a} < \mathbf{b})$}: $|atom\rangle = |O_2 \rangle$
%         \item {$(\mathbf{a} = \mathbf{b})$}: $|atom\rangle = |\neg O_1 \wedge \neg O_2 \rangle$ \\
%         $|O_1 O_2 \rangle \otimes |atom\rangle$
%         $= (NOT^{\otimes 2} \otimes I) CCNOT (NOT^{\otimes 2} \otimes I) (|O_1 O_2 \rangle \otimes |0\rangle)$ \\
%         $= (NOT^{\otimes 2} \otimes I) CCNOT (|\neg O_1 \neg O_2 \rangle \otimes |0\rangle)$ \\
%         $= (NOT^{\otimes 2} \otimes I) (|\neg O_1 \neg O_2 \rangle \otimes |\neg O_1 \wedge \neg O_2\rangle) = (|O_1 O_2 \rangle \otimes |\neg O_1 \wedge \neg O_2\rangle)$
%         \item {$(\mathbf{a} \neq \mathbf{b})$}: $|atom\rangle = |O_1 \vee O_2 \rangle$ \\
%         $|O_1 O_2 \rangle \otimes |atom\rangle$
%         $= (NOT^{\otimes 2} \otimes I) CCNOT (NOT^{\otimes 2} \otimes I) (|O_1 O_2 \rangle \otimes |1\rangle)$ \\
%         $= (NOT^{\otimes 2} \otimes I) CCNOT (|\neg O_1 \neg O_2 \rangle \otimes |1\rangle)$ \\
%         $= (NOT^{\otimes 2} \otimes I) (|\neg O_1 \neg O_2 \rangle \otimes |(\neg O_1 \wedge \neg O_2) \oplus 1\rangle) = (|O_1 O_2 \rangle \otimes |O_1 \vee O_2\rangle)$
%         \item {$(\mathbf{a} \leq \mathbf{b})$}: $|atom\rangle = |(\neg O_1 \wedge \neg O_2) \vee O_2 \rangle$ \\
%         $|O_1 O_2 \rangle \otimes |atom\rangle$ \\
%         $= (NOT^{\otimes 2} \otimes I) CCNOT (NOT^{\otimes 2} \otimes I) (I \otimes CNOT) (|O_1 O_2 \rangle \otimes |0\rangle)$ \\
%         $= (NOT^{\otimes 2} \otimes I) CCNOT (NOT^{\otimes 2} \otimes I) (|O_1 O_2 \rangle \otimes |O_2 \oplus 0\rangle)$ \\
%         $= (NOT^{\otimes 2} \otimes I) CCNOT (|\neg O_1 \neg O_2 \rangle \otimes |O_2\rangle)$ \\
%         $= (NOT^{\otimes 2} \otimes I) (|\neg O_1 \neg O_2 \rangle \otimes |(\neg O_1 \wedge \neg O_2) \oplus O_2\rangle)$ \\
%         $= |O_1 O_2 \rangle \otimes |(\neg O_1 \wedge \neg O_2) \vee O_2 \rangle$
%         \item {$(\mathbf{a} \geq \mathbf{b})$}: $|atom\rangle = |(\neg O_1 \wedge \neg O_2) \vee O_1 \rangle$ \\
%         $|O_1 O_2 \rangle \otimes |atom\rangle$ \\
%         $= (NOT^{\otimes 2} \otimes I) CCNOT (NOT^{\otimes 2} \otimes I) (CNOT') (|O_1 O_2 \rangle \otimes |0\rangle)$, where $CNOT'$ is the $CNOT$ gate with its control and target qubit connect to $|O_1\rangle$ and $|0\rangle$ reaspectively \\
%         $= (NOT^{\otimes 2} \otimes I) CCNOT (NOT^{\otimes 2} \otimes I) (|O_1 O_2 \rangle \otimes |O_1 \oplus 0\rangle)$ \\
%         $= (NOT^{\otimes 2} \otimes I) CCNOT (|\neg O_1 \neg O_2 \rangle \otimes |O_1\rangle)$ \\
%         $= (NOT^{\otimes 2} \otimes I) (|\neg O_1 \neg O_2 \rangle \otimes |(\neg O_1 \wedge \neg O_2) \oplus O_1\rangle)$ \\
%         $= |O_1 O_2 \rangle \otimes |(\neg O_1 \wedge \neg O_2) \vee O_1 \rangle$
%     \end{enumerate}
% \end{comment}

\subsection{Consistency Extractor} \label{subsec:Consistency_Extractor}

The task of {\em consistency extractor} is to extract those assignments that are consistent in both Boolean and the bit-vector domains. That is, given an $atom_i$ with its Boolean abstract variable $v_{B_i}$, we want to make sure that $v_{B_i}' = 1$ iff $atom_i$ and $v_{B_i}$ are logical equivalent.

% For example, if one atom is $(a>b)$, the assignments that the circuit will extract are (a) $\{(a>b)$ and $(T2B(a>b) = True)\}$ and (b) $\{(a<b \vee a=b \vee a \neq b)$ and $(T2B(a>b) = False)\}$. 

As shown in Fig.~\ref{fig:SMT_extract_circuit}~{(a)}, consistency extractor is composed of a $\mathtt{CNOT}$ gate and a $\mathtt{X}$ gate. The input $atom_i$ serves as the control bit of the $\mathtt{CNOT}$ gate, while the other input $v_{B_i}$ serves as the target bit, followed by a $\mathtt{X}$ gate to flip its result. With this design, the output qubit $v_{B_i}'$ would be $1$ iff $atom_i$ and $v_{B_i}$ have the same truth value. Theorem~\ref{thm:Consistency_Extractor_Correctness} proves the correctness of our quantum circuit for consistency extractor.

% The control qubit of the $CNOT$ gate will be connected to the output of the comparator circuit and the target qubit will be connected to the corresponding qubit of the boolean abstract variable. 
% Since the $CNOT$ gate operates XOR on its output of the target qubit, it can identify whether the two inputs of the gate are consistent or not. 
% Take atom $(a>b)$ for example, the control qubit of the $CNOT$ gate will connect to the corresponding $|atom\rangle$ that identifies the “$>$” result (which is $O_1$ for the comparator circuit we use), and the target qubit will connect to the corresponding qubit that represents $T2B(a>b)$. 
% The output of target qubit will be 0 if the assignment is $\mathcal{BV}$ consistent (which is (1) $\{(a>b)$ and $(T2B(a>b) = True)\}$ and (2) $\{(a<b \vee a=b \vee a \neq b)$ and $(T2B(a>b) = False)\}$). 
% In order to control the succeeding procedure, the $NOT$ gates will be applied to the output of the $CNOT$ gates to let the outputs of the consistent extract circuit become 1 while the result is $\mathcal{BV}$ consistent.

\begin{theorem} \label{thm:Consistency_Extractor_Correctness}
{\bf [Correctness of Consistency Extractor]} \\ 
($v_{Bi}' = 1) \iff (v_{B_i} \equiv \; atom_i)$.
\end{theorem}

\begin{proof}
Let $a$ be the result of $v_{Bi}$ after applying the $\mathtt{CNOT}$ gate, as marked in red in Fig.~\ref{fig:SMT_extract_circuit}.
\begin{displaymath}
\begin{array}{lcl}
v'_{Bi} = 1 & \Leftrightarrow & a = 0 \;\; \Leftrightarrow \;\; (v_{Bi} \oplus atom_i) = 0 \\
            & \Leftrightarrow & (v_{Bi} = 0 \wedge atom_i = 0) \vee (v_{Bi} = 1 \wedge atom_i = 1) \\
            & \Leftrightarrow & v_{B_i} \; \equiv \; atom_i \qed
\end{array}
\end{displaymath}
\end{proof}


\begin{figure}[tb]
\begin{minipage}{0.4\linewidth}
\centering
\includegraphics[width=0.95\linewidth]{Figure/Consistency_Extractor.pdf}
\end{minipage}
~
\begin{minipage}{0.56\linewidth}
\centering
\includegraphics[width=0.98\linewidth]{Figure/Solution_Inverter.pdf}
\end{minipage} \\[2mm]
\begin{minipage}{0.4\linewidth}
\centering
\footnotesize
(a) Consistency Extractor
\end{minipage}
~
\begin{minipage}{0.56\linewidth}
\centering
\footnotesize
(b) Solution Inverter
\end{minipage}
\caption{Consistency Extractor and Solution Inverter}
\label{fig:SMT_extract_circuit}
\end{figure}

\subsection{Solution Inverter} \label{subsec:solution_inverter}

{\em Solution inverter} is the key component of the oracle. It marks the solutions to the SMT formula by inversing them, i.e., giving them a ``$-1$'' phase such that the diffuser knows which elements to increase/maximize their probability being measured.

The circuit for solution inverter, as shown in Fig.~\ref{fig:SMT_extract_circuit}~{(b)}, is composed of two $(m+1)$-$\mathtt{CNOT}$ gates and a $\mathtt{Z}$ gate, where $m$ is the number of Boolean abstract variables. Theorem~\ref{thm:Solution_Inverter_Correctness} proves the correctness of solution inverter.

% The $CNOT$ gate is responsible for generating the evaluation result of the $\mathcal{BV}$ SMT problem. The satisfiable solution is the assignments that all atoms (i.e. boolean abstract variables) are $\mathcal{BV}$ consistent and the SAT result is also satisfiable. 
% The $CNOT$ gate will set the SMT output qubit (i.e. $q_{SMT}$) to $|1\rangle$ for those assignments that satisfy the condition. 
% Next, the $control-Z$ gate will take place to inverse the assignments which their SMT output qubit is $|1\rangle$. 
% Based on the characteristic of Grover’s algorithm, the algorithm is suitable for searching the elements which are the minor category of the database (i.e. the number of target element $\leq$ the number of non-target elements). 
% In order to let the circuit works properly under any assignment situation, the additional qubit $q_{addition}$ is used to double the number of the database element. 
% Since the additional qubit applies the $H$ gate initially, it has already doubled the element number of the database, and then it will be used as the control qubits of the $control-Z$ gate of the inversion circuit. 
% As the result, only the quantum state that both $q_{SMT}$ and $q_{addition}$ are $|1\rangle$ will be inverse. 
% The purpose of this operation is to let half part of the database element (i.e. the states with $q_{addition}$ are 0) will not be inversed under any situation, so the number of the inverted (target) elements will always be the minor part compared to the whole database and the Grover search procedure can works properly.


% The following is the structural proof of $m-CNOT$ gate construction and our inversion circuit design.

% Proof. A $m-CNOT$ gate can be implemented by the quantum circuit shown in Figure \ref{fig:mCNOT_circuit}, where the output $o'_{m} = (i_1 \wedge i_2 \wedge ... \wedge i_{m})$ while $o_{m} = 0$. \\
% Proof by induction. \\
% For basic case, the $2-CNOT$ gate can be implemented by one $CCNOT$ gate. \\
% $o'_2 = (i_1 \wedge i_2) \oplus o_2$ and $o'_2 = (i_1 \wedge i_2)$ while $o_2 = 0$ \\
% Induction assumption : \\
% The output of the $k-CNOT$ gate $o'_k$ equals to $(i_1 \wedge i_2 \wedge ... \wedge i_k)$ while $o_k = 0$ \\
% Induction case : \\
% The output of the $(k+1)-CNOT$ gate $o'_{k+1} = (o'_k \wedge i_{k+1}) \oplus o_{k+1} = (i_1 \wedge i_2 \wedge ... \wedge i_k \wedge i_{k+1}) \oplus o_{k+1}$ and $o'_{k+1} = (i_1 \wedge i_2 \wedge ... \wedge i_k \wedge i_{k+1})$ while $o_{k+1} = 0$.

% \begin{figure}[tb]
%     \centering
%     \includegraphics[width=0.55\linewidth]{Figure/mCNOT_circuit.pdf}
%     \caption{Quantum circuit of $m-CNOT$ gate.}
%     \label{fig:mCNOT_circuit}
% \end{figure}

\begin{theorem} \label{thm:Solution_Inverter_Correctness}
{\bf [Correctness of Solution Inverter]} \\
(1) $q'_{\mathtt{SMT}} = 1$ iff $q'_o(F_B) = 1$ and $v_{B_i} \equiv atom_i$, $\forall i \in \{1,2,\ldots, m\}$. \\
(2) All the solutions are added a ``$-1$'' phase.
\end{theorem}

\begin{proof}
Let $q'_{\mathtt{SMT}}$ be the value of the $q_{\mathtt{SMT}}$ bit after applying the first $(m+1)$-$\mathtt{CNOT}$ gate, as marked in red in Fig.~\ref{fig:SMT_extract_circuit}~{(b)}.
Because $q_{\mathtt{SMT}}$ (initialized as $0$) is the target bit of the $(m+1)$-$\mathtt{CNOT}$ gate, provided that $v_{B_i}'$ is one of the $m$ controlled bit, we know that
$q'_{SMT} = 1 \Leftrightarrow (v'_{B_1} \wedge v'_{B_2} \wedge ... \wedge v'_{B_m} \wedge q_o'(F_B)) = 1$. By Theorem~\ref{thm:Consistency_Extractor_Correctness}, we can conclude that condition~{(1)} holds.

To prove condition~{(2)}, let us consider the state of the quantum circuit. Let $| v'_{B_1}, \ldots, v'_{B_m}, \ldots, q_o'(F_B), q'_{\mathtt{SMT}} \rangle$ be the quantum state after applying the first $(m+1)$-$\mathtt{CNOT}$ gate. As the value of $q'_{\mathtt{SMT}}$ could be $0$ or $1$, the system space $S$ can be split into two disjoint sets $S_0$ and $S_1$. That is, $S = S_0 \cup S_1$, where $S_0 = \{| v'_{B_1}, \ldots, v'_{B_m}, \ldots, q_o'(F_B), 0 \rangle \}$ and $S_1 = \{| v'_{B_1}, \ldots, v'_{B_m}, \ldots, q_o'(F_B), 1 \rangle \}$. By condition~{(1)} we just proved, $S_1$ is exactly the set of all solutions to the SMT problem. After that, a $\mathtt{Z}$ gate is applied on the $q'_{\mathtt{SMT}}$ bit. Since $\mathtt{Z}(|0\rangle) = |0\rangle$ and $\mathtt{Z}(|1\rangle) = -1 |1\rangle$, the $Z$ gate will give a ``$-1$'' phase for each element in $S_1$, i.e., all solutions are added a ``$-1$'' phase. \qed
\end{proof}

% \begin{figure}[tb]
%     \centering
%     \includegraphics[width=0.6\linewidth]{Figure/Solution_Inverter.pdf}
%     \caption{Quantum circuit of inversion.}
%     \label{fig:SMT_inverse_circuit}
% \end{figure}


% \begin{comment}
%     Proof. The extract operation $U_{extract}$ will set the boolean abstract qubits to $|1,1,...,1 \rangle$ while all of them are consistent with the status of their corresponding atoms (i.e. $T2B(atom) = atom\ status$), and the operation can be implemented by the existing quantum gates.
    
%     First, the operation $U'_{CNOT}$ behaves as $CNOT$ gate with its control and target qubit reversed, i.e. $U'_{CNOT}(|ab\rangle) = |a \oplus b \rangle \otimes |b\rangle$.
    
%     Next, the operation $U'_{bCNOT} = \prod_{i=0}^{n-1} U'_{bCNOT\_i}$ applies bitwise $U'_{CNOT}$ operation to two quantum bit string, such that : $U'_{bCNOT}(|\mathbf{a} \mathbf{b} \rangle) = |\mathbf{a} \oplus \mathbf{b} \rangle \otimes |\mathbf{b} \rangle$, where $\mathbf{a}$ and $\mathbf{b}$ are two $n$ bits quantum bit string, and $U'_{bCNOT\_i}$ is the $U'_{CNOT}$ operation with the $i$th qubit of $\mathbf{a}$ and $\mathbf{b}$ (i.e. $\mathbf{a}$[i] and $\mathbf{b}$[i]) as its input.
    
%     According to our circuit design and the deduction of previous section, the quantum state after SAT, arithmetic and comparator circuit can be represented as 
%     \begin{center}
%         $|\psi \rangle = \sum_{B,atom=0}^{B,atom=2^n-1} P_{B,atom} (|\mathbf{B} \rangle \otimes |\mathbf{atom} \rangle \otimes |SAT_B \rangle \otimes |0\rangle \otimes |+\rangle)$
%     \end{center}
%     , where $|P_{B,atom}|^2$ is the probability of the state, $\mathbf{B}$ and $\mathbf{atom}$ are the integer representation of $n$-bit boolean assignment and atom status $|atom\rangle$ respectively, and $SAT_B$ is the SAT evaluation result of assignment $\mathbf{B}$.
    
%     The consistent extract operation $U_{extract}$ can be represented as 
%     \begin{center}
%         $U_{extract} = (NOT^{\otimes n} \otimes I^{^{\otimes n+3}}) (U'_{bCNOT} \otimes I^{\otimes 3})$
%     \end{center}
    
%     \begin{center}
%         $U_{extract} (|\psi \rangle)$ 
%         $= (NOT^{\otimes n} \otimes I^{^{\otimes n+3}}) (U'_{bCNOT} \otimes I^{\otimes 3}) \sum_{B,atom=0}^{B,atom=2^n-1} P_{B,atom} (|\mathbf{B} \rangle \otimes |\mathbf{atom} \rangle \otimes |SAT_B \rangle \otimes |0\rangle \otimes |+\rangle)$
%         $= (NOT^{\otimes n} \otimes I^{^{\otimes n+3}}) \sum_{B,atom=0}^{B,atom=2^n-1} P_{B,atom} (|\mathbf{B} \oplus \mathbf{atom} \rangle \otimes |\mathbf{atom} \rangle \otimes |SAT_B \rangle \otimes |0\rangle \otimes |+\rangle)$
%         $= (NOT^{\otimes n} \otimes I^{^{\otimes n+3}}) \sum_{B,atom=0}^{B,atom=2^n-1} P_{B,atom} (|\mathbf{k} \rangle \otimes |\mathbf{atom} \rangle \otimes |SAT_B \rangle \otimes |0\rangle \otimes |+\rangle)$
%         , where
%         $\left \{ \begin{array}{rcl}
%         |\mathbf{k} \rangle = |0,0,...,0 \rangle & \mbox{for}
%         & \mathbf{B} = \mathbf{atom} \\ 
%         |\mathbf{k} \rangle \neq |0,0,...,0 \rangle & \mbox{for} & \mathbf{B} \neq \mathbf{atom}
%         \end{array} \right $\\
%         $= P_{B,atom} (|\mathbf{k'} \rangle \otimes |\mathbf{atom} \rangle \otimes |SAT_B \rangle \otimes |0\rangle \otimes |+\rangle)$
%         , where
%         $\left \{ \begin{array}{rcl}
%         |\mathbf{k'} \rangle = |1,...,1 \rangle & \mbox{for}
%         & \mathbf{B} = \mathbf{atom} \\ 
%         |\mathbf{k'} \rangle = |\neg \mathbf{k} \rangle & \mbox{for} & \mathbf{B} \neq \mathbf{atom}
%         \end{array} \right $
%     \end{center}
%     \\
%     Proof. Based on the result of the previous Proof, the inverse operation $U_{inverse} = (I^{\otimes 2n+1} \otimes CZ)(U_{SMT} \otimes I)$ will inverse the phases of those states that satisfy the following 3 conditions, and the operation can be implemented by existing quantum gates. \\
%     (1) $\mathcal{BV}$ consistent for all atoms (i.e. $|\mathbf{k'} \rangle = |1,1,...,1 \rangle$), \\
%     (2) the boolean assignment is satisfiable (i.e. $|SAT_B \rangle = |1 \rangle$), \\
%     (3) the additional qubit $|q\_addition \rangle = |1 \rangle$.
    
%     The operation $U_{SMT}$ is constructed by one $(n+1)-CNOT$ gate with its control qubits connect to $|k'\rangle$ and $|SAT_B\rangle$, in order to present the $\mathcal{BV}$ SMT result for each assignments according to condition (1) and (2).
%     \begin{center}
%         $(U_{SMT} \otimes I) (\sum_{B,atom=0}^{B,atom=2^n-1} P_{B,atom} (|\mathbf{k'} \rangle \otimes |\mathbf{atom} \rangle \otimes |SAT_B \rangle \otimes |0\rangle \otimes |+\rangle))$
%         $= \sum_{B,atom=0}^{B,atom=2^n-1} P_{B,atom} (|\mathbf{k'} \rangle \otimes |\mathbf{atom} \rangle \otimes |SAT_B \rangle \otimes |\wedge_n \mathbf{k'} \wedge SAT_B \rangle \otimes |+\rangle)$
%     \end{center}
    
%     Next, the operation $CZ$ (i.e. $control-Z$ gate) will apply phase inverse to the states that its SMT result is satisfiable (i.e. $|\wedge_n \mathbf{k'} \wedge SAT_B \rangle = |1\rangle$) and also satisfy condition (3). 
%     \begin{center}
%         $(I^{\otimes 2n+1} \otimes CZ) (\sum_{B,atom=0}^{B,atom=2^n-1} P_{B,atom} (|\mathbf{k'} \rangle \otimes |\mathbf{atom} \rangle \otimes |SAT_B \rangle \otimes |\wedge_n \mathbf{k'} \wedge SAT_B \rangle \otimes |+\rangle))$
%         $= \sum_{B,atom=0}^{B,atom=2^n-1} P_{B,atom} ((|\mathbf{k'} \rangle \otimes |\mathbf{atom} \rangle \otimes |SAT_B \rangle \otimes |\wedge_n \mathbf{k'} \wedge SAT_B \rangle \otimes |0\rangle)+(-1)^{(\wedge_n \mathbf{k'} \wedge SAT_B)} (|\mathbf{k'} \rangle \otimes |\mathbf{atom} \rangle \otimes |SAT_B \rangle \otimes |\wedge_n \mathbf{k'} \wedge SAT_B \rangle \otimes |1\rangle))$
%     \end{center}
% \end{comment}

\subsection{Reverse circuit}
After inversing the solutions, the last part of the oracle function is the {\em reverse circuit}, which is constructed by the reversed circuits corresponding to the four aforementioned components. 
The purpose of the reverse circuit is to restore the qubits back to their original states, following the reversible nature of quantum computing.
In addition, Grover's algorithm may take several iterations to amplify the probability of solutions. Ancilla bits need to be restored to be used for the following iterations.

% The detailed design and implementation can be found in [XXX]. \lsw{Create an anonymous website to host the files.}

% Another benefit of the reverse circuit is that it can also restore the ancilla qubits which are used in the previous circuit, so the diffusion circuit can reuse those ancilla qubits to construct the $m-CNOT$ gate instead of adding new ones to reduce the required number of qubits of the circuit.

% \subsection{Measurement}
% Finally, the satisfiable solutions of the SMT problem can be obtained by measurement. 
% As we descript previously, the circuit will always output the correct result directly for most solution situations since the search database has already been doubled. 
% But there are two special cases, which are unsatisfiable SMT problems and the tautology SMT problems (i.e. all assignments are satisfiable solution for the SMT formula), we can not conclude the answer directly from the measurement result. 
% In those two situations, none of the assignments has been marked as solutions for the unsatisfiable problems and exactly half of the assignments in the database have been marked for the tautology problems. 
% Both conditions will let the “inversion about the average” operation not amplify the probability of any of the assignments, the measured probability of every assignment is the same. 
% So if the measurement result presents a uniform probability distribution for all assignments, the result is one of the two situations, and they can be distinguished easily by applying an SMT formula evaluation for one arbitrary assignment. If the evaluation result is satisfiable, then all assignments are satisfiable for the SMT problem and vice versa.
