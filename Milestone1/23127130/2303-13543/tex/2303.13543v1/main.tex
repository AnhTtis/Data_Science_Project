
\documentclass[journal]{IEEEtran}

\usepackage{cite}
\usepackage{bm}
% \usepackage{algorithm}
% \usepackage{algorithmic}
\usepackage[linesnumbered,ruled]{algorithm2e}
\usepackage{graphicx}
\usepackage{subfigure}
\usepackage{lineno,hyperref}
\usepackage{amsmath}
\usepackage{pifont}
\usepackage{epstopdf}
\usepackage{multirow}
\usepackage{amsthm,amsmath,amsfonts,amssymb,bm}
\usepackage{array}
\usepackage{booktabs}
\usepackage{makecell}
\usepackage[all]{xy}
\def\diag{\mbox{diag}}
\modulolinenumbers[5]




% *** GRAPHICS RELATED PACKAGES ***
%
\ifCLASSINFOpdf
  % \usepackage[pdftex]{graphicx}
  % declare the path(s) where your graphic files are
  % \graphicspath{{../pdf/}{../jpeg/}}
  % and their extensions so you won't have to specify these with
  % every instance of \includegraphics
  % \DeclareGraphicsExtensions{.pdf,.jpeg,.png}
\else
  % or other class option (dvipsone, dvipdf, if not using dvips). graphicx
  % will default to the driver specified in the system graphics.cfg if no
  % driver is specified.
  % \usepackage[dvips]{graphicx}
  % declare the path(s) where your graphic files are
  % \graphicspath{{../eps/}}
  % and their extensions so you won't have to specify these with
  % every instance of \includegraphics
  % \DeclareGraphicsExtensions{.eps}
\fi

\hyphenation{op-tical net-works semi-conduc-tor}


\begin{document}

\title{Labeled Subgraph Entropy Kernel}

\author{Chengyu~Sun, Xing~Ai, Zhihong~Zhang$^{\ast}$, and~Edwin~R~Hancock,~\IEEEmembership{Fellow,~IEEE}
% <-this % stops a space
    % \IEEEcompsocitemizethanks{
    %     \IEEEcompsocthanksitem Chengyu Sun are with School of Informatics, Xiamen University, Xiamen, Fujian, China.
    %     \protect\\ E-mail: 24320191152507@stu.xmu.edu.cn, 30920201153942@stu.xmu.edu.cn
    %     \IEEEcompsocthanksitem Corresponding author: Zhihong Zhang is with School of Informatics, Xiamen University, Xiamen, Fujian, China.
    %     \protect\\ E-mail: zhihong@xmu.edu.cn
    %     \IEEEcompsocthanksitem Edwin R. Hancock is with University of York, York, UK.  
    %     \protect\\ E-mail: edwin.hancock@york.ac.uk}
% note need leading \protect in front of \\ to get a newline within \thanks as
% \\ is fragile and will error, could use \hfil\break instead.
% <-this % stops an unwanted space
}

% note the % following the last \IEEEmembership and also \thanks - 
% these prevent an unwanted space from occurring between the last author name
% and the end of the author line. i.e., if you had this:
% 
% \author{....lastname \thanks{...} \thanks{...} }
%                     ^------------^------------^----Do not want these spaces!
%
% a space would be appended to the last name and could cause every name on that
% line to be shifted left slightly. This is one of those "LaTeX things". For
% instance, "\textbf{A} \textbf{B}" will typeset as "A B" not "AB". To get
% "AB" then you have to do: "\textbf{A}\textbf{B}"
% \thanks is no different in this regard, so shield the last } of each \thanks
% that ends a line with a % and do not let a space in before the next \thanks.
% Spaces after \IEEEmembership other than the last one are OK (and needed) as
% you are supposed to have spaces between the names. For what it is worth,
% this is a minor point as most people would not even notice if the said evil
% space somehow managed to creep in.



% The paper headers
\markboth{Journal of \LaTeX\ Class Files,~Vol.~14, No.~8, August~2015}%
{Shell \MakeLowercase{\textit{et al.}}: Bare Demo of IEEEtran.cls for IEEE Journals}
% The only time the second header will appear is for the odd numbered pages
% after the title page when using the twoside option.
% 
% *** Note that you probably will NOT want to include the author's ***
% *** name in the headers of peer review papers.                   ***
% You can use \ifCLASSOPTIONpeerreview for conditional compilation here if
% you desire.




% If you want to put a publisher's ID mark on the page you can do it like
% this:
%\IEEEpubid{0000--0000/00\$00.00~\copyright~2015 IEEE}
% Remember, if you use this you must call \IEEEpubidadjcol in the second
% column for its text to clear the IEEEpubid mark.



% use for special paper notices
%\IEEEspecialpapernotice{(Invited Paper)}




% make the title area
\maketitle

% As a general rule, do not put math, special symbols or citations
% in the abstract or keywords.
\begin{abstract}
In recent years, kernel methods are widespread in tasks of similarity measuring. Specifically, graph kernels are widely used in fields of bioinformatics, chemistry and financial data analysis. However, existing methods, especially entropy based graph kernels are subject to large computational complexity and the negligence of node-level information. In this paper, we propose a novel labeled subgraph entropy graph kernel, which performs well in structural similarity assessment. We design a dynamic programming subgraph enumeration algorithm, which effectively reduces the time complexity. Specially, we propose labeled subgraph, which enriches substructure topology with semantic information. Analogizing the cluster expansion process of gas cluster in statistical mechanics, we re-derive the partition function and calculate the global graph entropy to characterize the network. In order to test our method, we apply several real-world datasets and assess the effects in different tasks. To capture more experiment details, we quantitatively and qualitatively analyze the contribution of different topology structures. Experimental results successfully demonstrate the effectiveness of our method which outperforms several state-of-the-art methods.
\end{abstract}
% Note that keywords are not normally used for peerreview papers.
\begin{IEEEkeywords}
Graph kernel, Subgraph, Graph entropy
\end{IEEEkeywords}


\IEEEpeerreviewmaketitle
\section{Introduction}

The increasing complexity of source code poses a key challenge to the reliability of large-scale software systems. Software bugs in these systems can lead to safety issues~\cite{bug_safety} for users around the world as well as cause non-negligible financial losses~\cite{bug_loss}. As such, developers have to spend a large amount of time and effort on bug fixing. Consequently, \aprfull (\apr), designed to automatically generate patches to fix software bugs, has attracted wide attention from both academia and industry~\cite{long2016prophet, legoues2012genprog, long2015spr, lou2020can, tufano2018empstudy}. 


To achieve \apr, one popular approach is known as Generate-and-Validate (G\&V)~\cite{qi2015gv, ghanbari2019prapr, lou2020can, le2016hdrepair, legoues2012genprog, wen2018capgen, hua2018sketchfix, martinez2016astor, koyuncu2020fixminder, liu2019tbar, liu2019avatar}, which is typically based on the following pipeline: First, fault localization techniques~\cite{wong2016fl, abreu2007ochiai, zhang2013injecting, papadakis2015metallaxis, li2019deepfl, li2017transforming} are applied to determine the suspicious locations in programs where bugs are likely to exist. Then, the buggy locations are used by the \apr tools to generate a list of patches that replace buggy lines with correct lines. Afterward, each patch is validated against the original test suite to identify any \emph{plausible patches} (i.e., passing all tests in the test suite). Finally, to determine the \emph{correct patches}, developers examine the list of plausible patches to see if any of them can correctly fix the bug. 

Traditional \apr tools can mainly be categorized into heuristic-based~\cite{legoues2012genprog, le2016hdrepair, wen2018capgen}, constraint-based~\cite{mechtaev2016angelix, le2017s3, demacro2014nopol, long2015spr} and \template~\cite{ghanbari2019prapr, hua2018sketchfix, martinez2016astor, liu2019tbar, liu2019avatar}. Among these traditional tools, \template \apr tools~\cite{ghanbari2019prapr, liu2019tbar, benton2020effectiveness} have been able to achieve state-of-the-art results. \Template \apr tools typically leverage pre-defined templates (e.g., adding a nullness check) for bug fixing. However, since these fix templates are typically handcrafted, the number and types of bugs they are able to fix can be limited. 



To address the limitations of traditional \apr, researchers have proposed various \learning \apr tools~\cite{li2020dlfix, chen2018sequencer, jiang2021cure, lutellier2020coconut, zhu2021recoder, ye2022rewardrepair} based on the \nmtfull (\nmt) architecture~\cite{sutskever2014mt} where the input is the buggy code snippets and the goal is to translate the buggy code snippets into a fixed version. To accomplish this, \learning \apr tools require supervised training datasets with pairs of both buggy and fixed code snippets in order to learn how to perform this translation step. These training data are usually obtained by mining historical bug fixes using heuristics/keywords~\cite{dallmeier2007benchmark}, which can be imprecise for identifying bug-fixing commits; even the actual bug-fixing commits can include irrelevant code changes, leading to further pollution in the dataset~\cite{xia2022alpharepair}.
% 
Moreover, it can be hard for such \apr tools to generalize and fix bug types unseen during training. 



To better leverage recent advances in \plmfull{s} (\plm{s}), researchers~\cite{xia2022alpharepair, xia2023repairstudy, kolak2022patch, prenner2021codexws} have directly applied \plm{s} to generate patches without bug-fixing datasets. These \llm-based \apr tools work by either directly generating a complete code function~\cite{prenner2021codexws, xia2023repairstudy} or predict/infill the correct code snippet given its surrounding context~\cite{xia2022alpharepair, xia2023repairstudy}. By directly using \llm{s} that are pre-trained on billions of open-source code snippets, \llm-based \apr tools can achieve state-of-the-art performance on many repair datasets~\cite{xia2022alpharepair}. 


% 
%
%

Traditional \apr tools have long used the insight of the \emph{plastic surgery hypothesis}~\cite{barr2014plastic} where it states that the code ingredients to fix a bug already exist within the same project. Traditional \apr tools have manually designed pattern-~\cite{ghanbari2019prapr, saha2017elixir} or heuristic-based~\cite{jiang2018simfix, legoues2012genprog} approaches to finding and using such relevant code ingredients to generate fixes for bugs. However, the plastic surgery hypothesis has been largely ignored in \llm-based \apr. In fact, \llm provides a unique opportunity to fully automate the plastic surgery hypothesis idea via fine-tuning (learning project-specific information via model updates from the buggy project) and prompting (directly providing relevant code ingredients to the model), and make it directly applicable to different languages (since the \llm{s} are typically multi-lingual).%
Moreover, despite the intensive manual efforts involved, traditional \apr tools still cannot fully leverage project-specific information due to large search space for leveraging/composing existing code ingredients. In contrast, the project-specific information can effectively leveraged by \llm{s} due to their power in code understanding/vectorization, e.g., even partial/imprecise information may still guide \llm{s} in correct patch generation!
 To this end, we ask the question: \emph{How useful is the plastic surgery hypothesis in the era of \plm{s}}?








\mypara{Our Work.} To answer the question, we present \ourtech{\xspace} -- a \llm-based approach that automatically utilizes the plastic surgery hypothesis by systematically combining multiple fine-tuning and prompting strategies for \apr. \ourtech fine-tunes \plm{s} using two novel domain-specific training strategies: \textbf{\epfinetune} -- we fine-tune using the original buggy project by aggressively masking out a high percentage of tokens, which allows \plm to learn project-specific code tokens and programming styles; and \textbf{\rofinetune} -- which only masks out a single continuous code sequence per training sample, allowing the model to get used to the final \csapr task of predicting a single continuous code sequence. Furthermore, we directly leverage the ability for \plm{s} to understand natural language instructions and introduce a novel prompting strategy, \textbf{\idprompting}, which uses information retrieval and static analysis to obtain a list of relevant identifiers for the buggy lines. While such relevant identifiers are critical for fixing some difficult bugs, they may not be seen by the \llm during inference due to limited context window size. Through the use of prompting, we directly tell the model to use these extracted identifiers (relevant code ingredients) to generate the correct code. Finally, to perform repair, we combine all four model variants (including the base model, both fine-tuned models and the base model with prompting) for the final repair.





While our insight of leveraging the plastic surgery hypothesis for \llm-based \apr is generalizable across different types of \plm{s}, to implement \ourtech, we choose a recent \plm{\xspace}, \ctfive~\cite{wang2021codet5}, which is pre-trained on millions of open-source code snippets. \ctfive is an encoder-decoder model trained using \mspfull (\msp) objective where a percentage of tokens are masked out and each continuous masked token sequence is referred to as a masked span. Also, although we only extract relevant identifiers from the current buggy project (since this paper focuses on the plastic surgery hypothesis), our work can be easily extended to obtain other code information (such as relevant statements or functions) from other sources, such as  the massive pre-training corpora~\cite{husain2020codesearchnet} or historical bug-fixing datasets~\cite{jiang2019infer}, which can provide more coding knowledge for \llm{s}. Besides, although we mainly focus on using traditional string comparison algorithms for information retrieval in this paper, these techniques can be easily replaced by other frequency-based retrieval~\cite{robertson2009probabilistic} and neural search (or embedding-based search)~\cite{reimers2019sentence}.
  In summary, this paper makes the following contributions:


%


\begin{itemize}[noitemsep, leftmargin=*, topsep=0pt]
    \item \textbf{Dimension.} This paper is the first to revisit the important plastic surgery hypothesis in the era of \llm{s}. It opens up a new dimension for \llm-based \apr to incorporate previously neglected information from the buggy project itself to boost \apr performance. Furthermore, it demonstrates the promising future of retrieval-based prompting for modern \llm-based \apr.
    \item \textbf{Implementation.} We implement \ourtech based on the recent \ctfive model. We augment the model using two novel fine-tuning strategies: \epfinetune and \rofinetune, along with a novel prompting strategy based on information retrieval and static analysis: \idprompting. We combine the patches generated by all four models together and perform patch ranking to speed up \apr.% 
    \item \textbf{Evaluation Study.} We conduct an extensive evaluation against state-of-the-art \apr tools. On the widely studied \dfj 1.2 and 2.0 datasets~\cite{just2014dfj}, \ourtech is able to achieve the new state-of-the-art results of 89 and 44 correct bug fixes (15 and 8 more than best baseline) respectively.  Furthermore, we perform a broad ablation study to justify our design. \ourtech demonstrates for the first time that the plastic surgery hypothesis can substantially boost \llm-based \apr and advance state-of-the-art \apr, while being fully automated and general. Moreover, even partial/imprecise code ingredients may still effectively guide \llm{s} for \apr!
\end{itemize}


\section{Related work}
\noindent \textbf{Video foundation models.}
With sufficient computational power and an abundant source of data, there have been attempts to build a single large-scale foundation model that can be adapted to diverse downstream tasks.
Along with the success of foundations models in the natural language processing domain~\cite{brown2020language,chen2021evaluating,devlin2019bert} and in computer vision~\cite{bertasius2021space,jia2021scaling,radford2021learning}, video data has become another data type of interest, as it has grown in scale due to numerous internet video-sharing platforms.
Accordingly, several methods to train a video foundation model have been proposed.
Due to the innate multi-modality of video data, \textit{i.e.}, a combination of visual $\cdot$ vocal $\cdot$ textual context, most works have centered around the variations of the cross-modal attention mechanism \cite{akbari2021vatt,bertasius2021space,gabeur2020multi,luo2020univl,neimark2021video,tan2021look,wei2020multi,yang2021taco}.
In addition, as most video data lack proper labels or descriptions, contrastive learning methods were studied to learn meaningful feature representations or enhance video-text alignment in a self-supervised manner \cite{akbari2021vatt,kuang2021video,luo2020univl,yang2021taco}.

More specifically, MERLOT \cite{zellers2021merlot} proposed a multi-modal representation learning method for visual commonsense reasoning, which also performed well in twelve video reasoning tasks.
VATT \cite{akbari2021vatt} introduced a multi-modal learning method via contrastive learning. 
The pre-trained model performed well in a variety of vision tasks from image classification to video action recognition and zero-shot video retrieval.
Another representative work, UniVL \cite{luo2020univl} proposed a straightforward pre-training method with auxiliary loss functions. 
After fine-tuning on a specific task, the pre-trained model performed outstandingly in a wide range of tasks of text-to-video retrieval, action segmentation, action step localization, video sentiment analysis, and video captioning.
Other foundation models for multiple video tasks include \cite{li2020hero,sun2019learning,sun2019videobert,zhu2020actbert,fu2021violet,wang2022all}. 

\noindent \textbf{Auxiliary learning.}
In order to enhance the performance of one or a multitude of primary tasks, auxiliary learning methods can be incorporated.
\cite{ruder2017overview} introduced Multi-task learning (MTL) to the deep neural networks by training a single model with multiple task losses to assist learning on the main task.
Such a method is generally adapted to pre-train the foundation models in the self-supervised manner~\cite{li2020hero,sun2019learning,sun2019videobert,zhu2020actbert,fu2021violet,wang2022all}.
However, these various pretext task losses used in the pre-training phase are ignored in the fine-tuning phase, and only the primary task loss is minimized.

Recently, meta-learning methods have been introduced for auxiliary learning.
\cite{liu2019self,navon2020auxiliary,shu2019meta} proposed a meta-learning method in which the model learns auxiliary tasks to generalize well to unseen data. 
In these settings, a separate subset of data is held out as the primary task, while the others are used as auxiliary tasks that aid the primary task's performance.
Similar methods were adopted for computer vision tasks such as semantic segmentation \cite{xu2021leveraging}.
Other domain applications include navigation tasks with reinforcement learning \cite{ye2021auxiliary}, or self-supervised learning methods on graph data \cite{hwang2020self}.
\section{Labeled Subgraph Entropy Kernel}
As mentioned above, our work is consists of three parts: subgraph counting algorithm, subgraph based entropy calculating and kernel definition. In this section, we provide a detailed description of our statistical process. We propose a network statistic element called Labeled Graphlet and design the corresponding searching algorithm. Next, we demonstrate the re-derivation process of subgraph-based cluster expansion and calculate the global entropy of the network. Finally, we define the subgaph entropy kernel and prove its corresponding required properties. In order to make our description clearer, we demonstrate the whole process of calculating subgraph entropy embedding in Algorithm \ref{alg:entropy}.


\iffalse
\begin{table}
\centering
\begin{tabular}{lll}
\hline
Symbol  & Definition  \\
\hline
$G$&Graph set\\
$GL$&Graph labels\\
$G_i$&The $i$th graph in $G$\\
$L$ & Node label set \\
$l$ & $|L|$,the number of node label types\\
$t$ & Number of designated graphlet topology types\\
$N$ & $|G|$, the size of graph-set $G$\\
$n$&Number of nodes in $G_i$ \\

\hline
\end{tabular}
\caption{Important notation used in this paper and their descriptions.}
\label{tab:plain}
\end{table}
\fi

\begin{algorithm}[htb]
\label{alg:entropy}
\caption{Subgraph Entropy Algorithm}
\KwIn{Adjacency matrix $adj$ and node label set $L$ of individual graph $G$, specific subgraph type $v$.}
\KwOut{Thermodynamics entropy value $S$ of $G$ which measured by $v$-th subgraph}
\textbf{Initialize super parameters}: inverse temperature $\beta$, scale parameter $\sigma$, the subgraph node number $l_v$ and edge number $d_v$ of $v-$th subgraph;\\
\textbf{Graphlet counting}: calculate the occurrence frequency $n=\{n_1,n_2...n_L\}$ of $v-$th subgraph starting from each node label within $G$;\\
\textbf{Edge configuration integral}: compute edge configuration integral $\zeta_v$ with formula(3);\\
\textbf{Partition function}: compute Subgraph configuration integral $q_v$ and the partition function $z_v$ of $v-$th labeled graphlet with formula(1) and formula(2);\\
\textbf{Subgraph entropy}: Calculate the subgrpah entropy $S_v$ of $v-$th  graphlet;\\
\textbf{Return $S_v$}
\end{algorithm}


\subsection{Labeled Subgraph Counting}
Graphlet is known as a representation of local structure, but existing graphlet-based statistical methods consider only topology information. Not only that, existing enumeration algorithms are limited by graphlet scale because of expensive computation costs. In attempting to solve the problems, we proposed a novel subgraph form called {\it Labeled Graphlet}. Labeled Graphlets are only marked by their starting node to simplify the complexity of the subgraph class set. As \ref{fig:graphlet_vs_motif} shows, Labeled Graphlet would tell the difference between two pairs of substructures in wireframes. Apparently, Labeled Graphlets share the same label-set $L$ with network nodes. Compared to existing graphlet statistics, our method provides additional details with little extra cost.

Generally speaking, subgraphs can be classified into three groups according to their structural characteristics: trees, paths, and circuits. Considering the variation of size, finding all subgraphs is an NP-hard problem. To simplify this procedure, we adopt twelve types of topologies as \ref{fig:graphlet_types} shows. Considering the derivation or mutually exclusive relationship between graphlets, our counting process follows the “grow” principle in a breadth-first manner. For small-scale subgraphs of size $k=3$ or $k=4$, we employ the Parallel Parameterized Graphlet Decomposition (PGD) algorithm. PGD breaks down the global searching problem into several local tasks and then merges the results for all edges. Merging and searching over low-dimensional spaces of edge neighborhoods is clearly more efficient than searching over the global high-dimensional space \cite{ahmed2015icdm}.

\begin{figure*}
\centering
\subfigure[Difference between motif and graphlet]{
\label{fig:graphlet_vs_motif}
\includegraphics[width=0.4\textwidth]{fig/graphlet_motif.pdf}
}
\quad
\subfigure[All twelve types of the structures mentioned in this paper.]{
\label{fig:graphlet_types}
\includegraphics[width=0.4\textwidth]{fig/graphlet_type.pdf}
}
\caption{\ref{fig:graphlet_vs_motif} shows the different between motif and labeled graphlet, as the former couldn't distinguish subtrees with different root. \ref{fig:graphlet_types} demonstrates all the graphlet topology in this paper.}
\end{figure*}

For the subgraphs of size $k\geq 5$, we design a dynamic programming algorithm to improve computing efficiency. The pseudocode of this process is shown in Algorithm \ref{alg:counting}. Starting from saving the first order neighbor sets of each node $a\in V$, we select tree-shaped subgraphs which meet the condition of degree. We have noticed that high-order neighbor sets can be synthesized by existing neighbor sets. For example, a 2-hop path consists of two 1-hop paths, and an 8-hop path consists of two 4-hop paths. Thus, we replace large-scale traversal with exponential steps superposition, which remarkably enhances searching efficiency with little extra space. 
The computational complexity of our algorithm is determined by line 3 of Algorithm \ref{alg:counting}, which is no more than $|V|^3$.

% \begin{breakablealgorithm}
\begin{algorithm}[h]
\label{alg:counting}
\caption{Path and circuit counting}
\KwIn{graph $G(V,E)$, maximum depth $D$, degree threshold $t$}
\KwOut{subgraph sets $Tree$, $Circuit$ and $Path$}
\textbf{Initialize}: $Circuit=Path=\emptyset$, $d$-hop adjacency matrix set $\{A^d\in \mathbb{R}^{\vert V\vert\times \vert V\vert} \vert d=2^1,2^2...2^D\}$, path set: $\{P_b(a)=\emptyset \vert\forall a,b\in V\}$\\
\For{$d\in [0,D]$}{
\For{$a,b,c \in V$}{
$N(a)=A_a^1$\\
\If{$|N(a)|\geq t$}{$Tree={N(a)+a}\cup Tree$}
\For{$i\in [2^{d}+1,2^{d+1}]$}{
\If{$A_{ab}^{(2^{d})}==1 \land A_{bc}^{(i-2^{d})}==1$}{
\If{$c \notin P_{b}(a)$}{
\If{$P_{c}(a)==\emptyset$}{$A_{ac}^{(i)}=A_{ca}^{(i)}=1$\\
$P_{c}(a)=P_{a}(c)=P_{c}(b)\cup P_{b}(a)$\\
$Path=P_{v}(a)\cup Path$\\
}
\Else{$Circuit=\{P_b(a)\cup P_c(b)\cup P_a(c)\} \cup Circuit$}
}
}   
}
}
}
\textbf{Return} $Tree, Circuit, Path$\\
\end{algorithm}
% \end{breakablealgorithm}


% \begin{breakablealgorithm}[H]
% \label{alg:counting}
% \caption{Path and circuit counting}
% \begin{algorithmic}[1]
% \STATE {//Initialization part}
%             ............
% \STATE {//Iterative part}
% \STATE  {$count\Leftarrow count+1$} 
% \UNTIL{The given termination criterion is met.}

% \end{algorithmic}
% \end{breakablealgorithm}



\subsection{Subgraph Cluster Expansion}
As mentioned above, the classical cluster expansion can be used to describe the motif structure. In order to solve this problem, we present the subgraph expansion algorithm, which omits the integral of a single node integral. In this section, we make an analogy between network subgraphs and interacting particles in the thermodynamic gas model. Thus, we can express configuration integral from the perspective of network topology, especially the network subgraph. According to this thought, Zhang et al.\cite{zhang2020graph} map the network motifs to the classical cluster expansion, then calculate the motif configuration integral and the single node integral, respectively. For graph $G$ with node set $L_v$, they re-write partition function $Z$ for the network as a sum over the individual motif contributions $z_v$,
\begin{equation}
\label{f1}
\begin{split}
    Z&=\sum \limits_v\prod \limits_{n_v}^{|L_v|}\frac{1}{n_v!}\{rq_v\}^{n_v}=\sum \limits_{n_v}z_v\\
    &=\sum \limits_{n_v} \frac{1}{n_v!}(rq_v)^{n_v}\frac{1}{N-l_vn_v}(rq_0)^{N-l_vn_v}\\
\end{split}
\end{equation}
where $r$, $n_v$, $l_v$ and $q_v$ is the radial variable, occurred frequency, number of nodes and the configuration integral (the product overall edges connecting nodes) of the $v^{th}$ motif. According to the definition presented above, motifs describe global information of networks by constructing graph representations, while graphlets attach more attention to local graph attributes. One main difference between these two statistic elements is whether the counting algorithms reuse nodes. The former one enumerates independent motifs and abandons individual nodes; The latter one count preset structures from each node while overlapping searching is inevitable. Thus, there are probably no singleton nodes left under the graphlets enumeration process. In other words, the item $N-l_vn_v$ in formula \ref{f1} can be 0 or even negative, and it becomes meaningless in the circumstances. Specifically, the configuration integral of $v^{th}$ graphlet (or the broader concept "subgraph") $q_v$ can be calculated with only connected nodes:

\begin{equation}
    q_v=\frac{1}{l_v!r}\zeta_v=\frac{1}{l_v!r}\epsilon^{d_v}
\end{equation}
$\zeta_v$ is the configuration integral obtained through the product over all edges connecting nodes, and $d_v$ is the edges number of the $v^{th}$ motif. According to the Mayer function, the configuration integral $\epsilon$ for one edge is given by
\begin{equation}
\begin{split}
    \epsilon&=\int^\infty_0(e^{-\beta v(r)}-1)dr\\
    &=\exp\left[\beta\sum\limits^{r_{max}}_{r=r_{min}}e^{-4\epsilon\left[(\frac{\sigma}{r})^{12}-(\frac{\sigma}{r})^{6}\right]}\right]+R
\end{split}
\end{equation}

where $\beta=\frac{1}{T}$ is the inverse temperature, $\sigma$ is scale parameter and $R=-\frac{r_{max}-r_{min}}{\Delta r}$. It is important to note that we confine the interval of integration to $[r_{min},r_{max}]$. The graph entropy of $v^{th}$ pattern consists of two parts: configuration integral $z_v$ and average energy $\langle U_v\rangle$.

\begin{equation}
    \begin{split}
        S_v&=\ln z_v+\beta\langle U_v\rangle\\
        &=n_v\left\{d_v\left[ \log\epsilon-\beta\frac{\epsilon-R}{\epsilon}\right]-l_v\log l_v -\log n_v \right\}\\
        %&=n_v\{ d_v\log\epsilon-l_v\log l_v-\log n_v\}+\beta\frac{n_vd_vpe^\beta}{pe^\beta+R}\\
    \end{split}
\end{equation}
where $\epsilon=pe^\beta+R$.
%the expansion formula of $v^{th}$ subgraph is

\subsection{Labeled Subgraph Entropy Kernel}
In this section, we introduce the concept of Labeled Subgraph Entropy Kernel. We replacing the occurrence frequency of graphlet with corresponding graph entropy. We provide a detailed framework description of our method and prove necessary kernel properties.
Before we define our novel kernel, we here summarize key concepts and notations. Graph $G=(V,E)$, where $V$ is a set of ordered vertices and $E\subseteq (V\times V)$ is a set of edges. If there is a mapping $V \Rightarrow L$ that assigns labels from a set $L$ to vertices, we call $G$ a labeled graph. $G$ is called undirected if $(vi, vj)\in E$ iff $(vj, vi)\in E$ otherwise, it is referred to as directed. Although many of our techniques are applicable to both directed and undirected graphs, for ease of exposition, we will exclusively deal with undirected graphs in this paper.

\textbf{Definition 1}: As for graph $G=(V,E,L)$, the corresponding labeled subgraph entropy representation around $v$ types of graphlet is
\begin{equation}
    S(G)=(S_1^1,S_2^1,...S_v^L)^T
\end{equation}

where $S_v^L$ denotes the thermodynamics entropy value measured by $v^{th}$ graphlet starting from node with label $L$. As algorithm \ref{alg:entropy} shows, $S_v^L=0$ iff the number of corresponding labeled graphlet is 0.

\textbf{Definition 2}: Let $G_1=(V_1,E_1,L)$ and $G_2=(V_2,E_2,L)$ be a pair of sample graphs. The subgraph entropy kernel $k_{SE}^{(v)}$ adopting $v$ types of graphlet is

\begin{equation}
\label{f7}
    k_{SE}^{(v)}(G_1,G_2)=k(S(G_1),S(G_2))
\end{equation}

where $k$ is the kernel function such as the Linear Product, Sigmoid Function, Polynomial Kernel Function or Radial Basis Function.

\textbf{Theorem 1}: The Labeled Subgraph Entropy Kernel $k_{SE}^{(v)}$ is symmetrical and positive definite.

\textbf{Proof}: We firstly embed two graphs into Euclidean space by formula \ref{f7}, and then calculate the kernel value using the above functions. Obviously, this process computes the inner products of two vectors in Hilbert space. Thus our Labeled Subgraph Entropy Kernel is symmetrical and positive definite.

\section{Experimental Results}
\label{sec:experiments}
\subsection{Training Details}
\cite{Kalantari2017DeepHD} provides the first dataset specifically designed for multi-exposure HDR fusion under large motion. It consists of 74 training sets, which we use to supervise the training of our model. We crop the input images to patches of size \(256 \times 256\) at a step size of 64. This totally generates 20128 training samples. To augment training samples, we randomly rotate and flip the training images. The training adopts Adam optimizer. The learning rate is initialized to \(10^{-4}\) and is reduced to \(10^{-5}\) after 20 epochs. It is observed that 40 epochs are sufficient for the training to converge.    

\subsection{Numerical Evaluation}
We numerically measure the performance of our method on the 15 test sets of \cite{Kalantari2017DeepHD}, by Peak Signal-to-Noise Ratio (PSNR) and Structure Similarity, computed in both tonemapping domain (-\(\mu\)) and HDR linear domain (-L). Visual difference metric HDR-VDP-2 is also adopted, where the parameters are set as same as in previous works \cite{wu2018end} and \cite{niu2021hdrgan}. 

Table \ref{table_metrics} compares our model with state-of-the-art models. For \cite{yan2020nonlocal} and \cite{xiong2021hierarchical}, we use the results reported in the publications. Note that \cite{sen2012robust} and \cite{hu2013hdr} are not machine learning based methods. Moreover,  \cite{Kalantari2017DeepHD} and \cite{wu2018end} apply optical flow and homography transformation to preprocess the input images respectively, and hence entail extra computation. 

Table \ref{table_metrics} shows that our method outperforms competing method in terms of PSNR-L, SSIM-$\mu$, SSIM-L and HDR-VDP-2. It ranks the second best in PSNR-$\mu$, being slightly (0.1dB) inferior to \cite{xiong2021hierarchical}. Note that \cite{xiong2021hierarchical} utilizes a pretrained model to detect ghosting regions for training, whereas our method does not require any pretrained model. The high PSNR and SSIM scores varify that our model has strong HDR reconstruction ability and can accurately restore the radiance and structure of the scene in both tonemapping domain and HDR linear domain. Furthermore, its high performance in term of HDR-VDP-2\cite{mantiuk2011hdr} performance indicates that our method can generate HDR image visually close to the target image.

\begin{table*}[ht]
\centering
\begin{tabular}{l|c|c|c|c|c}
\hline
& PSNR-$\mu$ & PSNR-L & SSIM-$\mu$ & SSIM-L & HDR-VDP-2 \\
\hline
\bfseries Sen & 40.97 & 38.36 & 0.9830 & 0.9746 & 60.60\\
\hline
\bfseries Hu  & 35.65 & 30.80 & 0.9725 & 0.9491 & 58.34\\
\hline
\bfseries Kalantari & 42.69 & 41.22 & 0.9888 & 0.9845 & 65.05\\
\hline
\bfseries DeepHDR& 41.99 & 41.22 & 0.9878 & 0.9859 & \underline{65.91}\\
\hline
\bfseries AHDR & 43.62 & 41.03 & 0.9900  &\underline{0.9883} & 63.85 \\
\hline 
\bfseries NHDRRNet& 42.414 & - & 0.9887 & - & 61.21 \\
\hline 
\bfseries HDR-GAN &43.92 & \underline{41.57} &\underline{0.9905} &0.9865 & 65.45\\
\hline 
\bfseries HFNet & \textbf{44.28} & 41.47 & - & - & - \\
\hline 
\bfseries Ours & \underline{44.18} & \textbf{42.19}&\textbf{0.9912} & \textbf{0.9883}& \textbf{67.07} \\
\hline
\end{tabular}
\caption{Numerical performance of the proposed model, evaluated on the dataset by Kalantari-Ramamoorthi. The best and second best results for each metric are marked in \textbf{bold} and \underline{underlined}, respectively}
\label{table_metrics}
\end{table*}

\subsection{Visual Performance Evaluation}

\begin{figure*}[!htb]
\centering
\includegraphics[width=\textwidth]{experiments/kalantari_test.png}
\caption{Visual comparison on the test set of Kalantari-Ramamoorthi dataset. Zoom-in views of reconstruction by each method are presented on the saturated regions that contain moving objects. Our network built with gated Swin Transformer yields noticeably better visual results than other methods.}
\label{fig_kalantari_test}
\end{figure*}
Fig. \ref{fig_kalantari_test} present the visual performance of our method and comparable methods on two examples from \cite{Kalantari2017DeepHD}. We present the zoom-in views of two challenging cases, where large saturated regions contain substantial non-rigid motion in the reference image. The two patch-based methods do not reconstruct the missing details in the saturated regions, as they heavily rely on the details provided by the reference image for registration. Image reconstructed by the optical flow based method \cite{Kalantari2017DeepHD} suffers motion blur artifacts. This is because the convolutions of DeepHDR and HDR-GAN have limited receptive fields, and hence are hampered to repair missing content in misaligned regions by aligned regions. The gating mechanism of AHDR is only applied to low-level features, so the high-level outliers may deteriorate the HDR fusion. In contrast to comparable methods, our model remarkably overcomes the ghosting artifacts.

\begin{figure}[ht]
\centering
\includegraphics[width=\columnwidth]{experiments/sen_test.pdf}
\caption{Visual performance comparison on example images from the dataset by Sen et al. Zoom in views on challenging areas are presented. Although the ground truth is unavailable, it can be clearly observed that our method visually performs better than comparable methods.}
\label{sen_test}
\end{figure}

\begin{figure}[ht]
\centering
\includegraphics[width=\columnwidth]{experiments/tursun_test.pdf}
\caption{Visual performance comparison on example images from the dataset by Tursun et al. Compared to state of the art methods, our method suffers less ghosting artifact.}
\label{tursun_test}
\end{figure}

Fig.\ref{sen_test} and Fig.\ref{tursun_test} present visual performance of our method on two examples from benchmark datasets \cite{sen2012robust} and \cite{tursun2016objective}. As these test datasets   do not provide ground truth image. we mark the visual difference on the results generated by different methods. It can be seen that our method suffers less artifacts than other methods in various scenes with various motion patterns, achieving better visual results. Our method creates high-quality HDR more robustly and generalizes well. 

\subsection{Ablation Study}

\begin{table}[h]
\centering
\resizebox{\columnwidth}{!}{
\begin{tabular}{l|c|c|c|c|c}
\hline
                         & PSNR-$\mu$ & PSNR-l & SSIM-$\mu$ & SSIM-l & HDR-VDP-2 \\ \hline
restormer(w/o ssim loss) & 44.00  & 41.5   & 0.9906 & 0.9873 & 64.72  \\ \hline
Ours(w/o ssim loss)      & 44.07  & 41.83  & 0.9909 & 0.9879 &  64.78  \\ \hline
Ours                     & 44.18  & 42.19  & 0.9912 & 0.9883 & 67.07      \\ \hline
\end{tabular}
}
\caption{Experimental results of ablation study. We compare using Gated Swin Transformer v.s. Gated Transformer, and the combined loss function v.s. the traditional $l_{1}$ norm loss function.}
\label{table_ablation_block_loss}
\end{table}

We verify various components of our method, including Swin Transformer, loss function, and gating mechanism by ablation study.

\subsubsection{Ablation Study on Block Design}
Our model has similar architecture to Restormer, which uses modified Transformer, whereas we use modified Swin Transformer as the building unit. For comparison, we replace the residual modules in each block in our model with multiple transformer layers as in Restormer, with same number of transformer layers. Table \ref{table_ablation_block_loss} presents the results, which show that using Swin Transformer achieves superior performance in all measures. The reason is that the attention module of Restormer is computed channel-wise, but forgoes the cross-exposure spatial dependency to repair the non-aligned area. 

\subsubsection{Ablation Study on Loss Function}
We trained our model under different loss function configurations, as shown in \ref{table_ablation_block_loss}. The results validate that the SSIM loss benefits detail reconstruction.

\subsubsection{Ablation Study on Gating Mechanism}
\begin{table}[h]
\resizebox{\columnwidth}{!}{
\begin{tabular}{l|c|c|c|c|c}
\hline
           & PSNR-$\mu$ & PSNR-l & SSIM-$\mu$ & SSIM-l & HDR-VDP-2 \\ \hline
w/o gating & 43.14  & 41.03  & 0.9904 & 0.9868 &     64.88      \\ \hline
one gating & 43.44  & 41.42  & 0.9909 & 0.9882 &    67.13   \\ \hline
Ours       & 43.61  & 41.74  & 0.9909 & 0.9881 & 66.96     \\ \hline
\end{tabular}
}
\caption{Ablation experimental results to verify the effectiveness of the gating mechanism}
\label{table_ablation_gating}
\end{table}

The gating mechanism is an important component in our model. Ablation study is conducted in the gating mechanism as follows.

\textbf{w/o gating}: The gating mechanism is not used in the feed forward network of all transformer layers in the model, that it, our GST unit degenerate to the vanilla Swin Transformer.

\textbf{one gating}: The gating mechanism is only used in the first Swin Transformer layers subsequent to the embedding layer, but not used for other layers. 

 Table \ref{table_ablation_gating} shows the results of the ablation experiments, where the model is trained for 20 epochs. By removing the gating mechanism, the network relies on self-attention for image alignment, resulting in the lowest performance. On top of it, adding gates to low level layers notably improves the HDR reconstruction. Furthermore, by integrating the gating mechanism with all Swin Transformer layers, the model effectively inpaints information in non-aligned regions and obtains the highest HDR reconstruction results, thus validates the effectiveness of the gating mechanism in our model.

\section{Conclusion}\label{sec:conclusion}
In this work, we focus on addressing the fundamental challenge of OOD detection tasks, which is how to fully understand the semantic discrepancy between the ID/OOD samples. We reveal that the key to success in the realistic SCOOD task is to allocate as many ID samples in the unlabeled set correctly as possible. To this end, we propose a novel uncertainty-aware optimal transport scheme that introduces class-specific energy scores as guidance for effective label assignment. Experimental results show that our method achieves better performance than previous state-of-the-art methods on SCOOD benchmarks.

\textbf{Limitations.} In addition to temperature scaling, other techniques such as feature clipping applied in ReAct~\cite{sun2021react} also enhance the performance of energy score, so how to obtain an OOD score that best fits the SCOOD task can be further explored. Moreover, a setting highly related to SCOOD has been proposed in \cite{katz2022training} and formulated as a constrained optimization problem. We will also theoretically analyze these practical OOD settings in our feature work.

% \section*{Acknowledgments}
\textbf{Acknowledgments.} 
This work is supported by National Key R\&D Program of China under Grant 2020AAA0105701, National Natural Science Foundation of China (NSFC) under Grants 61872327, Major Special Science and Technology Project of Anhui, National Natural Science Foundation of China (62033012) and Ant Group through Ant Research Intern Program.

The authors would like to acknowledge funding through the SNSF Sinergia grant called "Robust Deep Density Models for High-Energy Particle Physics and Solar Flare Analysis (RODEM)" with funding number CRSII$5\_193716$, the SNSF project grant 200020\_212127 called "At the two upgrade frontiers: machine learning and the ITk Pixel detector", and the Alexander von Humboldt foundation Feodor Lynen fellowship programme.
\bibliographystyle{ieeetr}
\bibliography{ref.bib}


\end{document}