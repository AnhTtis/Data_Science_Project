% Felix Finster, Eduardo Guendelman and Claudio Paganini
\documentclass[reqno,11pt]{amsart}
%\usepackage[notref,notcite]{showkeys}
\newcommand\hmmax{0}
\newcommand\bmmax{0}
\usepackage{hyperref}
\usepackage{graphicx}
\usepackage{slashed}
\usepackage{amscd}
\usepackage{tensor}
\usepackage{amsmath,amssymb,mathabx,bm}
\usepackage{esint}
\usepackage{dsfont}
\usepackage{enumitem}
\usepackage{accents}
%\usepackage{mathabx}
%\usepackage{amsmath, amscd}
%\usepackage{ngerman}
%\usepackage[latin1]{inputenc}
%\usepackage{refcheck}
%\usepackage{auto-pst-pdf} % langsame Version, alle Bilder werden als PDF erstellt
\usepackage[off]{auto-pst-pdf} % schnelle Version, Bilder müssen dazu schon vorliegen
%\usepackage[usenames,dvipsnames]{pstricks}
%\usepackage{epsfig}
\usepackage{pst-grad} % For gradients
\usepackage{pst-plot} % For axes
\usepackage[mathscr]{eucal}
\textheight 22cm
\textwidth 14.38cm

%\input{../../Sony/sony.tex} % Sollte für Sony-Reader unkommentiert sein

\oddsidemargin=0.9cm
\evensidemargin=0.9cm
\topmargin=-0.5cm
\numberwithin{equation}{section}
\allowdisplaybreaks[1]

%\newcommand{SetFigFont}[3]{}

\title[ Comparing Modified Measures and Causal Fermion Systems]{Modified Measures as an Effective Theory for Causal Fermion Systems}
%Options:
% Marrying Causal Fermion Systems with Non-Riemannian Measure Theories.
% 

\author[F.\ Finster]{Felix Finster}
\address{Fakult\"at f\"ur Mathematik \\ Universit\"at Regensburg \\ D-93040 Regensburg \\ Germany}
\email{finster@ur.de}

\author[E. Guendelman]{Eduardo Guendelman}
\address{Ben Gurion Universiy}
\email{}

\author[C.F.\ Paganini]{Claudio F. Paganini  }
\address{Fakult\"at f\"ur Mathematik \\ Universit\"at Regensburg \\ D-93040 Regensburg \\ Germany}
\address{Max Planck Institute for Gravitational Physics (Albert Einstein Institute), Am M\"uhlenberg 1, D-14476 Potsdam, Germany}
\email{claudio.paganini@ur.com}



\newtheorem{Def}{Definition}[section]
\newtheorem{Thm}[Def]{Theorem}
\newtheorem{Conjecture}[Def]{Conjecture}
\newtheorem{Prp}[Def]{Proposition}
\newtheorem{Lemma}[Def]{Lemma}
\newtheorem{Remark}[Def]{Remark}
\newtheorem{Corollary}[Def]{Corollary}
\newtheorem{Example}[Def]{Example}
\newtheorem{Assumption}[Def]{Assumption}
\newtheorem{axiom}{Axiom}

\renewcommand{\L}{\mathcal{L}}
\renewcommand{\H}{\mathscr{H}}
\newcommand{\la}{\langle}
\newcommand{\ra}{\rangle}
\newcommand{\Lin}{\text{\rm{L}}}
\newcommand{\N}{\mathbb{N}}
\DeclareMathOperator{\supp}{supp}
\newcommand{\C}{\mathbb{C}}
\newcommand{\Sact}{{\mathcal{S}}}
\newcommand{\Z}{\mathbb{Z}}
\newcommand{\R}{{\mathord{\mathbb R}}}
\newcommand{\tr}{\mbox{tr}}
\newcommand{\F}{\mathscr{F}}
\newcommand{\Reals}{{\mathbb R}}         %the real numbers
\newcommand{\eps}{\varepsilon}
\newcommand{\NatNum}{\mathbb{N}}
\newcommand{\Mcal}{\mathcal{M}}
\newcommand{\HH}{\mathcal H}
\newcommand{\Pdd}{\mbox{$\partial$ \hspace{-1.2 em} $/$}}
\newcommand*{\myfont}{\fontfamily{ppl}\selectfont}		%ppl
\DeclareFontFamily{OT1}{rsfso}{}
\DeclareFontShape{OT1}{rsfso}{m}{n}{ <-7> rsfso5 <7-10> rsfso7 <10-> rsfso10}{}
\DeclareMathOperator{\Sl}{\prec\!}
\DeclareMathOperator{\Sr}{\!\succ}

\newcommand\mpar[1]{\marginpar {\flushleft\sffamily\footnotesize #1}}
\setlength{\marginparwidth}{3.0cm}



\begin{document}

\maketitle

\begin{abstract} 
We compare the structures of the theory of causal fermion systems (CFS), an approach to unify Quantum Theory with General Relativity, with those of modified measure theories (MMT), which are a set of modified gravity theories. 
Spacetimes with MMT can be obtained as the continuum limit of a CFS.
This suggests that MMT could serve as effective descriptions of modifications to General Relativity implied by CFS. The goal is to lay the foundation for future research on exploring which MMTs are consistent with the causal action principle of CFS. 

% More over we show that for generic causal fermion systems, deviation from the Riemannian measure are to be expected in the continuum limit.
\end{abstract}

\tableofcontents

\section{Introduction }
This paper is part of a series of papers comparing the structures and ideas of different approaches to fundamental physics that was started with \cite{ethcfs} and will be continued in \cite{cfstdqg}. In order to be self-consistent, each of these papers contains an introduction to the theories under consideration including an overview of their accomplishments. They proceed with a detailed comparison between these two theories. Ultimately, the goal is to motivate the community to establish an extensive collection of such articles as a sort of "Rosetta's stone" for approaches to fundamental physics. The hope is that such a set of dictionaries of ideas helps the exchange across approaches and thereby catalyzes progress in the foundations of physics. This addresses a distinctly different goal from overview articles such as \cite{loll2022quantum, Mielczarek_2018, deBoer:2824341} or \cite{addazi2022quantum} (which originated the present work) that try to cover the development across many approaches simultaneously. By focusing on two sets of ideas at a time, a greater level of depth can be covered and more specific issues addressed. 

Two of the main challenges in fundamental theoretical physics are finding a theory unifying General Relativity and the Standard Model of particle physics and explaining the ``dark'' components of the universe, namely dark matter and dark energy. Assuming dark energy to be given by the cosmological constant, its value in the present universe can be determined from the observed accelerated expansion of the universe  \cite{acceleratinguniverse, acceleratinguniverse2, acceleratinguniverse3, acceleratinguniverse4} and is found to be positive. Explaining the value of the cosmological constant from first principle is an open problem. Deriving it as the vacuum energy of the quantum fields leads to a number that is off by about 120 orders of magnitude. This is sometimes referred to as the worst prediction in physics and goes by the name of the \textit{cosmological constant problem}. 

Addressing these challenges, in the present paper we will compare modified measure theories (MMT), an approach to modified theories of gravity, and causal fermion systems (CFS), a novel approach to fundamental physics. 
 
 MMT are a set of modified theories of gravity developed by Guendelman and collaborators \cite{NGVE,scaleinvtmt, Fieltheory, manymeasures, Benisty:2019jqz, measuresModifiedGravityandCosmology}. 
 Originally, these theories were developed  with the goal to modify the variational principle for gravity in such a way that the equations of motion become invariant under the addition of a constant to the  Lagrangian.  This puts gravity conceptually on the same footing as the fields of the Standard Model which satisfy this symmetry naturally. If one considers the cosmological constant to be an effective description for the energy of the vacuum state in QFT, then this symmetry corresponds to the old wisdom from classical mechanics that you can never measure absolute energy levels but only relative ones.
 
 Thereby, MMT are one possible solution to the cosmological constant problem. In the simplest versions of MMT, the cosmological constant  $\Lambda$ becomes an integration constant to be determined by observations. 
 In the second order formalism, the measure density introduces an additional degree of freedom consisting of a scalar field. Variations of this scheme with two or more measures have been used successfully to model inflation, for a review see \cite{measuresModifiedGravityandCosmology}.  In this case, the integration constants do not only allow for a shift in the cosmological constant, but also determine the shape of the scalar field potentials. In particular, they determine the level of the flat sectors that could serve to describe the inflationary period as well as the sector responsible for the late slow inflation phase of the universe.
 
 
CFS was developed by Finster and collaborators \cite{cfs, website} as a new approach to unify General Relativity with the Standard Model of particle physics which has been worked out in details in perturbations around Minkowski space and has proven viable in general spacetimes. In particular, it succeeds in deriving the three generations of fermions in the Standard Model and in resolving the hierarchy problem. The latter is resolved by the fact that the Standard Model interactions/ gauge fields are obtained at leading order in the expansion with respect to the regularization length $\varepsilon$ while the coupling of matter to gravity is only comes in at the next-to-next-to-leading-order in an expansion in powers of this regularization length \footnote{In causal fermion systems one has to introduce a regularization length $\varepsilon$ for the various structures to be mathematically well-defined. One then studies the limit where $\varepsilon\searrow0$. In the context of physics the regularization length can be thought of as the Planck length $l_p$. }.

In recent years also the Quantum Field Theory limit for the Standard Model fields has been worked out \cite{qft, fockentangle, finster2018complex, fockfermionic}. This completes the derivation of our well-established theories in suitable limits (except for the Higgs field which has not yet been worked out).

The focus of current research is in further solidifying the mathematical foundations of the theory and in obtaining novel predictions for open problems. In this direction it was recently shown \cite{baryogenesis} that the CFS theory allows for a novel mechanism for baryogenesis. Phenomenological consequences of this mechanism are subject of ongoing research, see for example \cite{cosmo}. The present comparison paper was motivated by the fact, that the mechanism of baryogenesis in \cite{baryogenesis} in fact requires to go from the metric measures over to a modified measure. 

For a detailed introduction to CFS tailored for physicists see~\cite{dice2014,dice2018,review}, 
whereas~\cite{cfsrev,nrstg} might be a good starting point for mathematicians. For a more gravitational perspective see \cite{grossmann}.
We also refer the interested reader to the textbooks~\cite{cfs, intro} and the website \cite{website}. The website is built in a modular way for the interested reader to pick those aspects of the theory they is most interested. 

Some connections between CFS and MMT have already been touched in \cite{baryogenesis}. 
The present comparison should lay the foundation to exploring these connections in more detail and in particular to launch a systematic investigation as to which of the phenomenological results in MMT can be recovered in the setting of CFS. On the one hand, this would be a major step in connecting CFS to observations/experiments. On the other hand, this would allow for a constraint on MMTs as an effective theory of an underlying fundamental theory. 
 
In an upcoming paper \cite{cfstdqg} some of the authors will compare the CFS theory with thermodynamic approaches to gravity \cite{padmanabhan2011entropy,padmanabhan2014general,padmanabhan2015one,padmanabhan2017cosmic,padmanabhan2017atoms}. The underlying motivation behind this approach is the same as for MMT to resolve the cosmological constant problem. In the above-mentioned paper, we will argue that CFS in a certain sense incorporates both sets of ideas to resolve the cosmological constant problem. The reason to mention the thermodynamic approach here is that it has recently been claimed that the correct theory of gravity emerging from these thermodynamic considerations is unimodular gravity \cite{alonso2022thermodynamics}. It is worth noting that a covariant formulation of unimodular gravity is a subset of two-measure-theories (TMT), as we will elaborate below. This connection will be further explored in \cite{gravvariance}.


 
\subsection{Organization}
The paper is organized as follows. In Section \ref{sec:background} we give a brief introduction to the theory of CFS and MMTs. The core of the paper is Section \ref{sec:comparison}, where we compare various aspects of the two theories, in particular their fundamental structures (Section \ref{sec:foundations}) and the equations of motion (Section \ref{sec:eom}). In Section \ref{sec:metric} we discuss which new structures are introduced to replace the metric and how the different tasks the metric fulfills in the standard second order formulation of GR are reassigned to those structures at the fundamental level. Finally, in Section \ref{sec:conclusion} we give a brief conclusion and outlook on future research.

\section{Background}\label{sec:background}
In this section we will give a basic introduction to CFS and MMT. The main purpose of this section is to make the paper self contained. We will keep these introductions relatively short and in turn point the interested reader to more in depth reviews in the literature.

% The CFS theory is an entirely new mathematical framework developed by Finster (see \cite{website} for an introduction and \cite{cfs} for a detailed treatment of the theory) with the aim to unify the standard model and general relativity. NRM theories on the other hand are a collection of modified gravity theories developed by Guendelman and collaborators (see \cite{measuresModifiedGravityandCosmology} for an introduction and an overview) mainly with the intention to resolve the cosmological constant problem. This is the fact that the equation of motion for the Einstein-Hilbert action for gravity is not invariant upon adding a constant to the Lagrangian and the observed cosmological constant for the late time evolution of the universe is many orders of magnitude off from the predicted value due to vacuum fluctuations based on Quantum Field theory. We will present the state of the fields independently for both fields and then discuss the potential contact points when we present the details of the project. 
 \subsection{Causal Fermion Systems} \label{sec:cfs}
As we do not assume the reader to be familiar with CFS
we give a brief overview of the basic concepts and explain how they relate to more familiar structures in physics.
The central postulate of the CFS theory is the \textit{causal action principle}, from which the field equations of GR and the SM can be derived in the continuum limit (see ~\cite{cfs}). The theory thus provides a unification of the weak, the strong and the electromagnetic forces with gravity. Moreover, the approach has led to concise notions of ``quantum spacetime'' and ``quantum geometry''
(see~\cite{rrev, lqg}) and the Quantum Field Theory limit has recently been established rigorously (see~\cite{qft, fockbosonic, fockfermionic,fockentangle}).

For the present introduction we begin with the abstract definition. A CFS consists of three objects: a Hilbert space $\H$, a suitably chosen subset $\F$ of the bounded linear operators on the Hilbert space, which we denote by ~$\Lin(\H)$,  and a measure $\rho$ defined on the Borel $\sigma$-algebra with respect to the norm-topology on $\Lin(\H)$.


\begin{Def} \label{def:cfs}
Let~${(\H, \la .|. \ra_\H)}$ be a Hilbert space. 
Given parameter {$n \in \N$} (``{spin dimension}''), we let $\F$ be the set
\begin{align*}
{\F} := \Big\{ \;x &\in \Lin(\H) \text{ with the properties:} \\
&\blacktriangleright\; \text{$x$ is {self-adjoint} and has {finite rank}} \\
&\blacktriangleright\; \text{$x$ has at most $n$ positive and at most $n$ negative eigenvalues} \;\Big\} .
\end{align*}
Moreover, let~$\rho$ be a Borel measure on~$\F$.
We refer to~$(\H, \F, \rho)$ as a {\bf{causal fermion system}} (CFS).
\end{Def} 

If we restrict to the subset $\F^{\text{reg}}$ of \textit{regular points} in $\F$, that is operators with exactly $n$ positive and $n$ negative eigenvalues, then one can show \cite{gaugefix, banach} that $\F^{\text{reg}}$ is a Banach manifold. Hence, the theory of causal fermion systems is based on the mathematical structure of an \textit{operator manifold}, thereby fusing the underlying structure of General Relativity and Quantum Theory. The measure then encodes spacetime and all structures therein. In order to understand how this comes about, we need to be able to connect these fundamental descriptions to established notions of spacetime and matter in General Relativity. In the following, we will only give a schematic description of how this works.  To that end we introduce the local correlation map

\begin{align} \label{eq:localcorrelation}
    F\,[g_{\mu\nu}, A_\mu,\dots]: \mathcal{M}  \qquad&\rightarrow \qquad \;\; \qquad\F \\
    x \qquad&\rightarrow \qquad F\,[g_{\mu\nu}, A_\mu,\dots](x).
\end{align}
This map allows us to identify points in a classical spacetime $\mathcal{M}$ with operators in $\F$. The map $F$ depends on the metric $g_{\mu\nu}$ as well as the matter fields defined on the spacetime under consideration. Hence all these structures are encoded in the map itself. We will give a precise construction of this map in Minkowski space in Section \ref{sec:continuum}.
 We can then use the local correlation map to define the measure $\mu$ of a subset $\Omega \subset \F$ as the push forward of the measure $\mu$
\begin{equation}\label{eq:pushforward}
  \rho (\Omega):= \mu(F^{-1}(\Omega))= \int_{F^{-1}(\Omega)} \Phi\, d^4x.
\end{equation}
Here $\Phi$ is the density function of the measure on the manifold. If one considers the metric measure on the manifold then $\Phi=\sqrt{|g|}$. However the construction of the push forward measure also works for more general measures on the manifold $\mathcal{M}$. In particular it allows for the measures studied in the context of MMT that we will introduce in Section \ref{sec:nrmt}. \textit{This insight is the key point we would like to highlight in this paper. }

For further considerations it is convenient to introduce the notation
\begin{align*}
S_{F(x)} &:= F(x)(\H) \subset \H &&\hspace*{-2cm} \text{subspace of dimension $\leq 2n$} \\
\pi_{F(x)} &: \H \rightarrow S_x{F(x)}\subset \H &&\hspace*{-2cm} \text{orthogonal projection on~$S_{F(x)}$} \:.
\end{align*}
With that notation at hand, the local correlation map allows us to give an explicit realization of the Hilbert space in $\mathcal{M}$ in terms of physical wave functions
\begin{equation}
    \psi^u(x) = \pi_{F(x)} u.
\end{equation}
Here $u$ is a vector in the Hilbert space $\H$. As a result of this construction, for every point $x$, $\psi^u(x)$ is a $2n$ dimensional vector that is an element in the vector space $S_{F(x)}$. One can think of $S_{F(x)}$ as a fibre space attached to a point $x$ on the operator manifold. This allows us to formulate a causal fermion system entirely in the language of spacetimes and structures therein. Taking these structures together, the local correlation map allows us to describe a classical configuration of spacetime and matter as a causal fermion system. These classical structures can be thought of as an (approximate)\footnote{For the map to be well defined we need to introduce a regularization on the scale of the Planck length. The classical structures are only recovered in the limit where the regularization length goes to zero.  } effective description. 

With the local correlation map at hand, we can introduce the notion of "spacetime" in the setting of CFS as the support\footnote{The
{\em{support}} of a measure is defined as the complement of the largest open set of measure zero, i.e.
\[ \supp \rho := \F \setminus \bigcup \big\{ \text{$\Omega \subset \F$ \,\big|\,
$\Omega$ is open and $\rho(\Omega)=0$} \big\} \:. \]} of the measure, 
\[  M := \supp \rho \:. \]
Hence, in the contrast to GR where the support of the measure is the entire manifold on which the variation principle is defined, in the context of CFS the measure is the fundamental variable and its support is usually only a proper subset of $\F$. In particular, the dimension of the spacetime in CFS can be much lower than the dimension of $\F$ (or the manifold $\F^{\text{reg}}$ for that matter) which can in principal be infinite. 
Furthermore, despite $\F^{\text{reg}}$ being a manifold, $M$ need not be so but can in fact also have a discrete structure. 
This explains how a scalar quantity, like the density function of the measure, alone can describe such a plethora of physical systems: By changing the configuration of the metric and matter fields in $F[g_{\mu\nu}, A_\mu,\dots]$ we map into a different subset of $\F$. This also gives us a different realizations of the Hilbert space in $\mathcal{M}$ in terms of physical wave functions\footnote{ A change in the local correlation map identifies a different operator with a particular point in the classical spacetime. This changes the value of the physical wave functions at that point. Therefore, if we vary the measure by changing the local correlation map, we vary the family of all physical wave functions in spacetime. 
In fact, as we will show below, in examples of causal fermion systems describing classical spacetimes we reverse the order of the construction.
Hence, we change the realization of the Hilbert space in terms of physical wave functions to vary the measure by changing the local correlation map. Therefore, in the continuum limit, a variation of the measure can be thought of equivalently as a variation of the family of all physical wave functions. }

To summarize the notation, the points in~$M$ are called {\em{spacetime points}}, and~$M$ is referred to as {\em{spacetime}}. Given this identification, in the remainder of this paper we will use $x$ both as a label for the spacetime point as well as for the associated operator and omit the local correlation map except if it is not clear from context what is meant.

We now proceed to show that, starting from the abstract definition of a CFS, one can obtain the usual spacetime structures such as causal relations. To achieve this, given a CFS, one introduces
additional mathematical objects which are {\em{inherent}} in the sense that they
only use information already encoded in the CFS.
Then one shows that these inherent structures correspond to familiar
structures in a Lorentzian spacetime, at least in suitable limiting cases.
This procedure is carried out in detail in~\cite{cfs}. Here we only
mention those few structures which will be essential for what follows. 

In order to define a causal structure in spacetime,
we study the spectral properties of the operator product $xy$.
Note that this operator product is an operator
of rank at most~$2n$, but in general it is not symmetric (note that $(xy)^* = yx \neq xy$ unless the
operators commute) and therefore not an element of~$\F$.
We denote the non-trivial eigenvalues of~$xy$, counting algebraic multiplicities,
by~$\lambda^{xy}_1, \ldots, \lambda^{xy}_{2n} \in \C$
(more precisely,
denoting the rank of~$xy$ by~$k \leq 2n$, we choose~$\lambda^{xy}_1, \ldots, \lambda^{xy}_{k}$ as all
the non-zero eigenvalues and set~$\lambda^{xy}_{k+1}, \ldots, \lambda^{xy}_{2n}=0$).
These eigenvalues give rise to the following notion of a causal structure.
\begin{Def} \label{def:causalstructure}
The points~$x,y \in \F$ are said to be
\[ \left\{ \begin{array}{cll}
\text{{\bf{spacelike}} separated} &\!&  {\mbox{if $|\lambda^{xy}_j|=|\lambda^{xy}_k|$
for all~$j,k=1,\ldots, 2n$}} \\[0.4em]
\text{{\bf{timelike}} separated} &&{\mbox{if $\lambda^{xy}_1, \ldots, \lambda^{xy}_{2n}$ are all real}} \\[0.2em]
&& \text{and $|\lambda^{xy}_j| \neq |\lambda^{xy}_k|$ for some~$j,k$} \\[0.2em]
\text{{\bf{lightlike}} separated} && \text{otherwise}\:.
\end{array} \right. \]
\end{Def} \noindent
More specifically, the points $x$ and~$y$ are lightlike 
separated if
not all the eigenvalues have the same absolute value and if 
not all of them are real.
We point out that this notion of causality does not rely on an underlying metric.
But it reduces to the causal structure of Minkowski space or general curved spacetimes in certain limiting
cases (for Minkowski space see~\cite[Section~1.2]{cfs} or Section~\ref{sec:continuum};
for curved spacetime see~\cite{lqg}).

A causal fermion system also distinguishes a {\em{direction of time}}. 
Indeed, we can introduce the functional
\[ {\mathscr{C}} :  M \times M \rightarrow \R\:,\qquad
{\mathscr{C}}(x, y) := i \,\tr \big( y\,x \,\pi_y\, \pi_x - x\,y\,\pi_x \,\pi_y \big) \:, \]
which leads to the following notion of time direction (for details see~\cite[Section~1.1.2]{cfs}).
\begin{Def}\label{def:order}
For timelike separated points~$x,y \in M$,
\[ \left\{ \begin{array}{cll}
\text{$y$ lies in the {\bf{future}} of~$x$} &\!&  {\mbox{if ${\mathscr{C}}(x,y)>0$}} \\[0.4em]
\text{$y$ lies in the {\bf{past}} of~$x$} && {\mbox{if ${\mathscr{C}}(x,y)<0$}}
\end{array} \right. \]
\end{Def} \noindent
We point out that this notion of ``lies in the future of'' is not necessarily transitive.
This corresponds to our physical conception that the transitivity of the causal relations
could be violated both on the cosmological scale (there might be closed timelike curves)
and on the microscopic scale (there seems no compelling reason why the causal
relations should be transitive down to the Planck scale).
We remark that a causal fermion system also gives rise to other notions of causality
which are transitive; we refer the interested reader to~\cite[Section~4]{linhyp}. 

Next we want to define a connection. While defined point-wise in the spacetime context, its function, as its name implies, is to connect between the fibres of neighbouring points in a spacetime, i.e. to encode how objects in the fibre evolve as we move within the manifold. A connection in the context of causal fermion systems is therefore a unitary map from $S_x$ to $S_y$. This can be constructed from the kernel of the fermionic projector $P(x,y)=\pi_y x$. For details of this construction see \cite{lqg}. The so called {\em{spin connection}} is given by
\begin{equation} \label{eq:cfsconnection}
    D_{x,y} \::\: S_y \rightarrow S_x \qquad \text{unitary}\:.
\end{equation}
It also gives rise to a corresponding metric connection, being a mapping between corresponding Clifford structures. The associated curvature is introduced as the holonomy of the connection.

We now come to the core of the theory of causal fermion systems: the {\em{causal action principle}}.
In order to single out the physically admissible
causal fermion systems, one must formulate constraints in the form of physical equations. To this end, we require that
the measure~$\rho$ be a minimizer of the causal action~${\mathcal{S}}$ defined by
\begin{align}
\text{\em{Lagrangian:}} && \L(x,y) &= \frac{1}{4n} \sum_{i,j=1}^{2n} \big(
|\lambda^{xy}_i| - |\lambda^{xy}_j| \big)^2  \label{Lagrange} \\
\text{\em{causal action:}} && \Sact(\rho) &= \iint_{\F \times \F} \L(x,y)\: d\rho(x)\, d\rho(y) \label{Sdef}
\end{align}
under suitable constraints (for the detailed form of the constraints see~\cite[Section~1.1.1]{cfs}).
A minimizing measure satisfies corresponding Euler-Lagrange equations (see~\cite[Section~1.4.1]{cfs}).
These equations describe the dynamics of the causal fermion system.
In a suitable limiting case referred to as the {\em{continuum limit}} these give rise to the equations of motion for the fields in the Standard Model and GR.

To illustrate the concepts introduced in this section in a specific example we will now discuss Minkowski space as a CFS. 

\subsubsection{Causal Fermion Systems in Minkowski Space}\label{sec:continuum}
In this section we explain in the example of the Minkowski vacuum how the abstract ideas introduced above are implemented in practice. 
For notational simplicity, we work in a fixed reference frame and identify Minkowski
space with~$\R^{1,3}$, endowed with the standard Minkowski inner product with signature
convention $(+,-,-,-)$. We consider the smooth, spatially compact solutions of the
Dirac equation in Minkowski space
\[ %\begin{equation}\label{eq:dirac}
    (i \gamma^k \partial_k - m) \,\psi = 0 \:, \]
endowed with the usual scalar product 
\[ ( \psi\, | \,\phi ) := \int_{t=\text{const}} (\overline{\psi} \gamma^0 \phi)(t, \vec{x})\, d\vec{x} \:, \]
where $\overline{\psi} = \psi^\dagger\gamma^0$ is the adjoint spinor. We then choose the Hilbert space~$\H$ to be the completion of the subspace of all negative-energy solutions.
We denote the restriction of the scalar product by~$\la .|. \ra := (.|.)|_{\H \times \H}$.

We now want to build the local correlation map assigning an operator~$F(x)$ for any point~$x$ of Minkowski space. This is obtained by applying Riesz`s representation theorem to the bi-linear map
\begin{equation}
    b_x(\psi, \phi) = -\overline{\psi(x)} \, \phi(x)  \qquad \forall \psi, \phi \in \H.
\end{equation}
We thus obtain the map to be
\begin{equation}\label{eq:correlation}
    \la \psi\, |\, {F(x)}\, \phi \ra = -\overline{\psi(x)} \, \phi(x)  \qquad \forall \psi, \phi \in \H.
\end{equation}
This operator encodes information on the local densities and correlations of all the wave functions in~$\H$
at the spacetime point~$x$. Therefore, it is referred to as the {\em{local correlation operator}}.
By construction, this operator is self-adjoint, has rank at most four and has
at most two positive and at most two negative eigenvalues.
Therefore, following Definition~\ref{def:cfs}, the local correlation operator~$F(x)$ is an element of~$\F$
if we choose the spin dimension~$n=2$.
Hence we have successfully constructed the local correlation map~$F[\eta_{\mu\nu}]$
from Minkowski vacuum to $\F$. According to the construction \eqref{eq:pushforward} this gives us the measure 
\begin{equation} \label{pushforward}
\rho = F_* \mu \qquad \text{or equivalently} \qquad
    \rho(\Omega) := %\int_{{F^{-1}(\Omega)}} d^4 x =
    \mu \big( {F^{-1}(\Omega)} \big)
\end{equation}
where $\mu$ is the standard Lebesgue measure on Minkowski. We thus have constructed a CFS~$(\F, \H, \rho)$ of spin dimension two.

The above explanation was oversimplified because the wave functions in~$\H$
are defined only up to sets of measure zero, and therefore the right side of~\eqref{eq:correlation}
is generally ill-defined point-wise. As a consequence we need to introduce a regularization by setting
\begin{equation} \label{psireg}
\psi_\varepsilon={\mathfrak{R}}_\varepsilon(\psi) \:,
\end{equation}
where the {\em{regularization operator}} ${\mathfrak{R}}_\varepsilon : \H \rightarrow C^0(\R^{1,3}, \C^4) \cap \H$
is a linear operator which maps to continuous Dirac solutions\footnote{Here, for simplicity of presentation, we have chosen a regularization which is compatible with the Dirac equation. This need not be the case for general regularizations in general spacetimes.} and fulfills the relation
\[ \psi=\lim_{\varepsilon \searrow 0}{\mathfrak{R}}_\varepsilon(\psi) \]
(a simple example for a regularization is the convolution by a suitable mollifier).
Working with the regularized wave functions, the right side of~\eqref{eq:correlation}
is a bounded sesquilinear form. Therefore, we can introduce the
{\em{regularized local correlation operator}}~$F^\varepsilon(x)$ by
\[ %\label{eq:regcorrelation}
    \la \psi_\varepsilon \,|\, {F^\varepsilon(x)} \,\phi_\varepsilon \ra := -\overline{\psi_\varepsilon(x)} \, \phi_\varepsilon(x)  \qquad \forall \, \psi, \phi \in \H \]
and applying the above construction to the regularized local correlation map~$F^\varepsilon$ gives a CFS~$(\H, \F, \rho\varepsilon)$.

It is shown in~\cite[Section~1.2]{cfs}, that all the structures of Minkowski space can be recovered
from this CFS in the limit $\varepsilon \searrow 0$.
For example, in this limit the causal structure of Definitions ~\ref{def:causalstructure} and~\ref{def:order},
reproduces the causal structure of Minkowski space and the connection \eqref{eq:cfsconnection} agrees with the spin connection on Minkowski space. 

It turns out that to describe Minkowski vacuum we have to choose $\H$ as the space of all negative-energy solutions of the Dirac equation. This in a sense realizes Dirac's original concept of the {\em{Dirac sea}}, that in vacuum, all the states of negative energy should be occupied. In the theory of causal fermion systems, the Dirac sea arises as the realization of the Hilbert space $\H$ in terms of physical wave functions. Therefore CFS does not share the problems of the original concept (like the infinite negative energy density of the sea) because the Dirac sea, describing the underlying Hilbert space, drops out of the Euler-Lagrange equations derived from the causal action principle.

The regularization operator can be visualized as ``smoothing'' the wave functions on a microscopic scale.
The length scale $\varepsilon$ can be thought of as the Planck length. In the theory of causal fermion systems, the regularization is not merely a technical tool in order to make divergent expressions finite, but it realizes the concept that on microscopic length scales, the structure of spacetime itself is modified. Thus we always consider the regularized objects as the physical objects.

In order to describe systems involving particles and/or anti-particles, following Dirac's hole theory one extends $\H$ by Dirac solutions of positive energy and/or removes vectors of negative energy. Bosonic fields (like electromagnetic or gravitational fields), on the other hand, correspond to collective ``excitations'' of the Dirac sea and the particle wave functions described by modifying the Dirac equation by a potential~${{\mathcal{B}}}$,
\[ (i \Pdd + {{\mathcal{B}}} - m ) \, \psi = 0 \:. \]
If we build the local correlation map $F \,[\eta_{\mu\nu}, B_\mu]$ from these solutions, varying the field $B_\mu$ induces a variation of the measure constructed by \eqref{eq:pushforward}. This measure can again be analysed in the limit $\varepsilon \searrow 0$ to obtain the equations of motion for the vector potential $B_\mu$. To do so we can use the fact that Minkowski vacuum with $B_\mu=0$ is a critical point of the causal action. We can therefor study the linearization of the Euler Lagrange equations around this minimizer. This procedure is carried out systematically and in computational detail in the {\em{continuum limit}}. As already mentioned above, this analysis makes it possible to derive classical field equations like the Maxwell and Einstein equations from the causal action principle (for details see~\cite{cfs}).




\subsection{ Modified Measure Theories} \label{sec:nrmt}
MMT were originally motivated by the cosmological constant problem. On the mathematical level this boils down to an inconsistency between the matter sector and the gravitational sector. As it is well known, in non-gravitational physics, like in particle mechanics for example,  the origin from which we measure energy is not important. In mathematical terms that means that the equations of motion are invariant under addition of a constant to the matter Lagrangian $\mathcal{L}_m$
\begin{equation}
    \mathcal{L}_m \longrightarrow  \mathcal{L}_m+C.
\end{equation}
The same does not hold true for the standard second order variational formulation of GR based on the Einstein Hilbert action
\begin{equation}
    S= \int_M R(g) \, \sqrt{-g} \, d^4x.
\end{equation}
If we add a constant to the Lagrangian 
\begin{equation}
    \mathcal{L}_G \longrightarrow  \mathcal{L}_G+C
\end{equation}
the equations of motion get an extra contribution of the form $C g_{\mu\nu}$ from the variation with respect to the metric $g$ of the additional $C \int_M \sqrt{-g}d^4x$ term in the action. In the semi classical setup this makes the gravitational equations dependent on the vacuum expectation value of all the quantum fields. To make these calculations compatible with observations requires a lot of fine tuning of the parameters in the matter Lagrangian.

MMT are one way to resolve this conceptual inconsistency between the matter and the gravitational sector. Instead of integrating the Lagrangian against the measure $\sqrt{-g}d^4x$ we pick the measure to be an independent quantity. The total action is then given by 
\begin{equation}
    S= \int_M L \, \Phi(A) \, d^4x \qquad \text{ with } \qquad L=\frac{-1}{\kappa} \, R(\Gamma, g) +L_m,
    \label{ActionwithPhi}
\end{equation}
where $\kappa$ is the gravitational coupling constant, the scalar curvature is given in terms of the connection and the metric and the measure by 
\begin{equation}
    \Phi(A)= \frac{1}{6}\varepsilon^{\alpha\beta\mu\nu}\partial_\alpha A_{\beta\mu\nu},
\end{equation}
where $A_{\beta\mu\nu}$ is the tensor gauge potential of a non-singular exact $4$-form $\omega=dA$. Thus the modified volume element density $\Phi(A)$ is given by the scalar density of the dual field-strength associated with that potential. As a result, $\Phi(A) d^4x$ is invariant under general coordinate transformations.  This approach to modify gravity was first introduced in \cite{NGVE} and later expanded upon in \cite{scaleinvtmt} to include applications concerning spontaneous symmetry breaking of scale invariance. Field theoretic aspects were discussed in \cite{Fieltheory}. After developing some examples using only one measure, multi-measure theories were developed, for example see \cite{manymeasures} where it was demonstrated that those can accommodate cosmological solutions with an early inflationary phase and a late time exponential growth phase with a small cosmological constant. Modified Measures have also been used in the formulation of string and brane theories with dynamical tensions \cite{stringsandmeasures}. For a recent review on the applications of  Modified Measures to  Modified Gravity Theories and Cosmology see \cite{measuresModifiedGravityandCosmology}. 

In a recent study \cite{measuresandfullhistoryofuniverse} the application of TMT to all phases of the universe, stating from a non singular emergent phase, followed by inflation and then by a Dark matter and Dark Energy era has been investigated. In such multi measure theories DM and DE owe their existence to the multiple measures. Further it has been shown that the well known theory of unimodular gravity when formulated in a generally covariant form \cite{UnimodularGRI} 
\begin{equation}
    \mathcal{S}= \int d^4x \,\sqrt{-g} \, (R + 2 \Lambda + \mathcal{L}_m) - \int d^4 x \, \Phi(B) \,2 \Lambda, 
\end{equation}
appears as a special case of TMT. 
Here $\Lambda$ is a priori a dynamical scalar field, and $\Phi(B)$ as above. Variation w.r.t. $B$ implies $\Lambda= const$ whereas variation w.r.t. $\Lambda$ yields $ \Phi(B) =\sqrt{-g}$. 




\subsubsection{Single Measure Theory}\label{sec:singlemeasure}

For the purpose of this paper we will only discuss the theory with a single modified measure in detail because CFS only features a single fundamental measure and it is hence unclear how multiple measures would be accommodated. 
Among the papers using modified measure  using only one measure, we can emphasize \cite{NGVE}, 
 which was in fact the first paper on modified measures, and then further generalizations which allowed to include gauge fields, like in \cite{GravityCosmologyandParticlePhysicswithouttheCosmologicalConstantProblem}, \cite{Hehl} and \cite{spacefillingbranesandhigherdimensions},  also string theories with a modified measures have been formulated with only  one measure see \cite{stringsandmeasures}  and other follow up papers.

In two dimensions, where Einstein gravity becomes trivial because the Einstein tensor vanishes identically, a single modified measure provides an acceptable theory of gravity. That is, we can consider  
\begin{equation}
    S= \int_M  R(\Gamma, g) \,\Phi\, d^2x 
\end{equation}
where
\begin{equation}
    \Phi(A)= \frac{1}{6}\,\varepsilon^{\alpha\beta}\partial_\alpha A_{\beta},
\end{equation}
and where the variation with respect to the gauge field $ A_{\beta}$ leads to the condition that the curvature $R = M $, where $M$ is a constant. This is in fact the Jackiw Teitelboinm model \cite{JT}.

Returning to $3+1$ dimensions let us recall the difference between the first order, or Palatini, formalism and the second order formalism. In the first order formalism the connection $\gamma$ is considered as an independent degree of freedom, while in the second order formalism it is by definition given by the Levi-Civita connection. Now for the case of pure gravity with the usual measure
 
\begin{equation}
    S= \int_M  R(\Gamma, g) \, \sqrt{-g}\,d^4x 
\end{equation}
the two formalism turn out to give the same equations of motion. However this is not necessarily true anymore when additional degrees of freedom come into play, be that a non-trivial matter Lagrangian or modified measures. 
Following \cite{}, if we vary the action \eqref{ActionwithPhi} with respect to the field $A_{\beta \mu\nu}$ we get
\begin{equation}
    \frac{-1}{\kappa} R(\Gamma, g) +L_m= M
\end{equation}
where $M$ is a constant of integration. For the first order formalism that is the end of the story. $M$ corresponds to the cosmological constant and has to be determined by observation. In the second order formalism however we get more. Here it is convenient to introduce the scalar field $\chi=\frac{\Phi(A)}{\sqrt{-g}}$. 
Following \cite{NGVE} we get the following system of equations
\begin{align}
    R_{\mu\nu}-\frac{1}{2}R \,g_{\mu\nu}= \frac{\kappa}{2}\left(T_{\mu\nu}+ M g_{\mu\nu}\right) + \frac{1}{\chi}\left( \chi_{,\mu;\nu} -\right.&\left.g_{\mu\nu} \, \Box\,\chi  \right) \\
   \Box \, \chi- \frac{\kappa}{D-1}\left[ \left(M + \frac{1}{2} \, T\right)+ \frac{(D-2)}{2}\,L_m           \right]&\,\chi =0
\end{align}
Where $T$ is the trace of the stress energy tensor and the integration constant $M=-\Lambda$ takes the role of the cosmological constant. We see that this system of equations can only give rise to solutions compatible with the equations of Einstein's General Relativity if $\left(M + \frac{1}{2} T\right)+ \frac{(D-2)}{2}L_m    =0$. The significance of this condition will be explored in an upcoming paper \cite{gravvariance}. The condition is in particular satisfied by a Minkowski vacuum spacetime, however it fails in de-Sitter spacetimes. Therefore de-Sitter spacetimes are not a solution to modified measure theories with a single measure in the second order formulation. 

An interesting  consequence of the above results in the second order formalism is get a non-conservation of the conventionally defined matter stress energy, that does not include a contribution from the field $\chi$ ,  in the original frame, 
\begin{equation}\label{eq:nonconservation}
    T_{\mu\nu}^{\phantom{\mu\nu};\mu}= -2\, \frac{\partial \mathcal{L}_m}{\partial g^{\mu\nu}}\,g^{\mu\alpha}\nabla_\alpha \ln{\chi}
\end{equation}
The non conservation is dependent on the gradient of scalar field $\chi$ . That means in a sense we can transfer "energy" from the measure, hence the gravitational sector of the theory, to the matter sector of the theory. This turns out to be an important property of the modified measure theories as an effective description of causal fermion systems in the context of baryogenesis \cite{baryogenesis}. 

To examine the physical implications of the theory it is typically more convenient to work in the so called Einstein frame. This is possible whenever the scalar curvature couples to the measure $\Phi(A)$ and there are no higher curvature terms.  When working with an action of the type defined by eq. \ref{ActionwithPhi} , to get to the Einstein frame we apply a local conformal transformation

\begin{equation}
    g_{\mu\nu}\longrightarrow \bar{g}_{\mu\nu} =\chi \, g_{\mu\nu}
    \label{EinsteinFrame}
\end{equation}
which restores the gravitational equations to the  Einstein form and expresses all additional equations as well in term of the metric $\bar{g}_{\mu\nu}$.
In the reformulation of the theory in Einstein frame $\chi$
corresponds to an additional scalar field. As mentioned above, in the first order formulation, where the affine connection is an independent degree of freedom that is determined from the equations of motion,  $\chi$  does not introduce new degrees of freedom, instead $\chi$ and therefore the measure is solved in terms of other fields through a constraint equation. 

Meanwhile in the second order formulation, in the Einstein Frame, we can rewrite  the field as $\chi= \exp{u}$ and then it is found that $u$ has a standard kinetic term in the action and represents a new degree of freedom and the constraint equation that is obtained in the first order formalism becomes a dynamical equation instead. Because the Einstein Tensor of the re-scaled metric is divergence free, a total conserved energy momentum can be defined, but it contains a contribution from  $u$ in addition to the matter.

To some extent this reformulation implies running coupling constants in the Einstein frame, since when performing the conformal transformation \eqref{EinsteinFrame}, a rearrangement of the interactions take place where non conformally invariant terms  in the Lagrangian  get a $\chi$ dependent weight. For a general discussion see \cite{FOUNDATION}. In the case of the standard model all interactions are dimensionless, except for the Higgs field's potential terms. 

When working in the Einstein frame, the scalar field $\chi$ is determined via the constraint as a function of the matter fields and constants of integration.
This then rearranges the interactions when reintroducing  $\chi$ into the equations of motion. This effect plays a key role in the dynamics of TMT models. The rearrangement manifests for example in the masses of fermions and their interactions with scalar fields as well as in the unusual structure of the fermionic contributions to the energy-momentum tensor.
All these quantities appear to be $\chi$  dependent which leads to the appearance of unsuspected interactions.

The result in the Einstein frame is an effective Lagrangian containing effective scalar field potentials  that are non trivial functions of the integration constants and the original potentials that coupled to the measures  $\Phi$ and  $\sqrt{-g}$ . 
In the context of cosmology, this can lead to a scalar field potentials with two flat regions, one flat used for inflation and the other for the slowly accelerated phase of the universe observed now, see for example \cite{GUENDELMANKATZ} and many other papers, including more recent ones, like \cite{measuresandfullhistoryofuniverse} that can even produce three flat regions\footnote{One for a non singular emergent universe and inflation, another one for early dark energy and the last one for a late dark energy phase.}. 


 
\section{Comparison} \label{sec:comparison} 
In this section, which is the core of our paper, we will highlight and compare the mathematical structures and the conceptional ideas behind CFS and MMT.
\subsection{Foundations of the Approaches}\label{sec:foundations}
We begin with a discussion of the basic building blocks of the theories which is straight forward.

\subsubsection{MMT}
MMT are formulated on classical manifolds with the Lagrangian for both matter and gravitation unchanged. It makes the measure a priori independent of the metric. In the second order formalism it thereby introduces  one  additional scalar degree of freedom  not present in the standard formulation of GR and the Standard Model. Meanwhile in the first order formalism the measure, through the equations of motion,  can be resolved in terms of other fields. In this case the measure does not introduce new degrees of freedom, but just adds a constant of integration. Furthermore it reorganizes the interactions of the matter model, as we explained in Section \ref{sec:singlemeasure}. 


\subsubsection{CFS}
CFS is formulated on an operator manifold $\F^{\text{reg}}$ over a Hilbert space and the measure is the only ``dynamical'' degree of freedom in the theory. However, the fact that the support of the measure is typically restricted to a proper, lower dimensional subset of the operator manifold together with the intrinsic causal structures allows for the measure to encode a plethora of physical systems. In contrast to GR for example the operator manifold on which the variational principle is formulated does not fix the topology of the spacetime described by a minimizing measure. 


\subsubsection{Discussion}
Given the different starting points for the theories,  a comparison is only possible close to the continuum limit where the regularized local correlation map $F_\varepsilon \,[g_{\mu\nu}, A_\mu, \dots]$ allows us to think of a CFS in terms of classical spacetime structures. It is immediately clear from the construction of the measure \eqref{eq:pushforward} that, in this setup, the measure on the manifold does not have to coincide with the measure derived from the metric $g_{\mu\nu}$ which entered the local correlation map $F_\varepsilon[g_{\mu\nu}, A_\mu, \dots]$. Beyond the obvious compatibility it was shown in \cite{baryogenesis} that at least in the context of baryogenesis, CFS requires MMT to describe the appropriate continuum limit. 

\subsection{Derivation of the Field Equations}\label{sec:eom}
Ultimately, the predictive power of any theory lies in its field equations. Here we discuss how the field equations are obtained. 
\subsubsection{MMT}
The variation principle in MMT is identical to standard GR except for the fact that we have an additional degree of freedom in the measure and to obtain the full set of field equations accordingly we also have to vary the action with respect to this degree of freedom. While for GR the predictions of the first order formulation and second order formulation coincide, this is not true anymore for MMT. The field $\chi$ only becomes a dynamical scalar degree of freedom in the second order formulation. 
By considering additional measures for different parts of the action (e.g. one measure for the gravitational sector and one for the matter sector) additional constants of integration  can be added to the theory which can for example be used to model an inflationary universe with a late time low lambda phase see \cite{GUENDELMANKATZ} for example.

\subsubsection{CFS}
A priori, in CFS there is only a single object, the measure $\rho$, to be varied in the action principle. However, close to the continuum limit, where we can approximate the CFS by standard fields in a spacetime, we can vary the measure in a more controlled fashion according to the effective degrees of freedom in the continuum description. This is achieved by varying the fields that enter the local correlation map $F_\varepsilon \,[g_{\mu\nu}, A_\mu, \dots]$. The standard results for matter and gravitational field equations are obtained in the continuum limit where $\varepsilon\searrow0$. However, given the fact that matter only couples to gravity as a third order effect of an expansion in the regularization length $\varepsilon$ we know that in reality $\varepsilon\neq 0$. Accordingly, one expects deviations from the standard results in the effective description. This was demonstrated as a proof of principle in the case of the CFS mechanism for baryogenesis \cite{baryogenesis} and requires modified measures for a minimizing spacetime and matter configuration. This in fact turns out to be crucial for the consistency of the approximate effective description, as the departure from metric measures implies a non-conservation of the matter stress-energy tensor; a property necessary when describing baryogenesis as a transition from a vacuum configuration. In a sense, the mechanism transfers ``energy'' from the gravitational sector, to the matter sector. 



\subsubsection{Discussion}
MMT is the minimal possible modification of gravity, and its field equations are accordingly compatible with standard predictions in the local universe. However, there is no fundamental argument that allows to distinguish which formulation, first order, second order or multiple measures, is fundamental. Accordingly, when building phenomenological models, there are many free parameters to fit observations. CFS, on the other hand, comes with a single fixed variation principle. However, in this case working out the full phenomenology is a difficult task. Ultimately it provides the chance to constrain the freedom in MMT models. In particular, it might help to distinguish between the first order formalism and the second order formalism, while multi-measure theories at first glance are incompatible with CFS. Working out the precise dynamics of the measure field $\chi$ for a CFS in the continuum limit will be subject of future work. 



\subsection{Splitting the Tasks of the Metric}\label{sec:metric}
In the standard second order formulation of GR the metric encodes all gravitational degrees of freedom in the Einstein Hilbert action. In particular the metric encodes 
\begin{enumerate}
    \item the map from the tangent space to the co-tangent space,
    \item the causal structure,
    \item the connection, 
    \item and the measure. 
\end{enumerate}
Furthermore, the action is composed from the matter fields and the metric, whereby the gravitational part of the action is exclusively composed from the metric.
It is plausible that in a fundamental theory not all of these properties will be encoded by the same mathematical object and that additional degrees of freedom are introduced by such a cut. 

First a comment on the standard formulations of gravity for comparison. In the first order formulation the connection is promoted to an independent field that appears in the action in combination with the metric. It turns out that for pure gravity this is equivalent to the second order formulation. However, this equivalence is not true in a more general setting. For example when matter is added to the model or, when curvature square terms appear in the action. In these cases it becomes more than just a formalism, although the name is still used. In fact, by making the connection independent of the metric we obtain new gravitational theories. 


\subsubsection{MMT}
It is well known, that geometry is give by the combination of causal structure plus volume, i.e. measure (see e.g.\cite[p.9]{surya2019causal} for a discussion of this fact in the context of Causal Sets Theory). With that in mind it is quite a natural cut to consider the measure as an independent degree of freedom as is the case in MMT. In the second order formalism the metric still encodes (1)-(3) and the measure becomes an independent dynamic degree of freedom encoded in the field $\chi$. The gravitational action is now composed of the metric and the measure density. In the first order formulation however the metric now only encodes (1) and (2) while the connection and the measure are independent variables. In this case the field $\chi$ can be resolved in terms of other fields. The gravitational action is now composed of the metric, the connection and the measure. As discussed in section \ref{sec:singlemeasure} in contrast to GR this now leads to different equations of motion from the second order formulation.

In many cases one can apply a conformal transformation to change to the Einstein frame, where we recover the usual variation principle with an additional scalar field. It is crucial to notice, that the causal structure is not affected by such a conformal transformation.



\subsubsection{CFS}
In CFS the causal structure and the connection is already encoded in the operator manifold $\F^{\text{reg}}$. The causal structure through the Eigenvalues of the operator product $xy$ between any two points in the manifold \ref{def:causalstructure} and the connection through the polar decomposition of the fermionic projector $\pi_y x$ mapping from $S_x$ to $S_y$. Given the fact that $S_x$ is a subspace of a Hilbert space, it is itself a Hilbert space with the induced scalar product.  
In the limiting case the space $S_x$ in some sense corresponds to the fibre at $x$ and we obtain the bundle as the emergent limiting structure. 

The measure $\rho$ plays in CFS the role of the fundamental degree of freedom encoding the geometry and the matter content of the limiting spacetime. Again it is crucial to note, that the support of the measure varies for different minimizers thereby encoding physical systems with a different causal structure and ultimately even a different topology of the limiting spacetimes. 

The map between the tangent and the co-tangent space is obtained in CFS as follows. First of all, for the tangent and co-tangent space to be well-defined, we need to assume that the spacetime $M:= \supp \rho$ is a four-dimensional smooth manifold. Under this and a few technical assumptions (more precisely, that the tangent cone measure be non-degenerate; see~\cite[Definition~6.7]{topology}), it is shown in~\cite[Section~6]{topology} with measure-theoretic methods that at every spacetime point~$x \in M$ there is a distinguished Clifford subspace~${\mathscr{C}}\ell_x$ together with a canonical mapping~$\gamma_x$ from the tangent space~$T_xM$ to the Clifford subspace,
\[ \gamma_x \::\: T_xM \rightarrow {\mathscr{C}}\ell_x \:. \]
This mapping corresponds to the usual Clifford multiplication. The anti-commutation relations~$\{ \gamma_x(u), \gamma_x(v) =: 2 g(u,v)$ define a Lorentzian metric, which gives rise to the usual identification of the tangent and co-tangent space.

The Lagrangian is a bi-distribution $\mathcal{L}(x,y)$ built from the eigenvalues of the operator product of the two spacetime points $x,y \in \F$ and integrated against the measure over both variables $d\rho(x)\,d\rho(y)$ to obtain the causal action. Only the measure is varied to find a minimizer of the action. 


\subsubsection{Discussion}
MMT in either the first or the second order formalism makes one of the cuts that CFS makes by separating the measure from the metric. In the first order formalism it makes an additional cut by promoting the connection to an independent variable. It is arguable whether this cut is conceptually compatible with CFS as in CFS the causal structure and the connection are both encoded in similar mathematical objects, namely the eigenvalues of the operator product and the polar decomposition of the fermionic projector. However, those are not really variables in the CFS setting as the theory is much more radical, promoting the measure to be the only fundamental degree of freedom while the connection and the causal structure are already encoded in the operator manifold $\F$.

The present comparison shows that in absence of a full theory of quantum gravity it is worthwhile to study the effect of separating different aspects of the functions the metric fulfills in the second order formulation of GR into independent variables/degrees of freedom. This gives a set of conceptually well motivated modifications of gravity which stand a chance to be emergent as an appropriate limit of an underlying theory of quantum gravity. 

\section{Outlook and Conclusion}\label{sec:conclusion}
In the present paper we have discussed the relationship between MMT and CFS. While it is clear that a connection exists, a substantial amount of work will be needed to constrain those MMT that are compatible with CFS. In particular it will be important to work out whether CFS gives rise to the dynamics of the first or the second order formulation of GR or whether one obtains something more general. If deriving such constraints in a rigorous manner is successful this will allow to study these MMT models as effective phenomenological models for CFS in cosmological/astrophysical settings without having to take the full CFS formalism into consideration.

On the conceptual level there are open questions concerning the scalar field $\chi$ that appears in MMT. It is not apriori clear how to interpret this field exactly. It is tightly connected to the volume of the spacetime and can play the role of the inflaton field when using the second order formalism \cite{NONRIEMANNIANVOLINFLATION}. However, it only encodes deviations from the metric volume. Hence, if one were to quantize the modified measure \cite{dzhunushaliev2022quantization} and with it the scalar field $\chi$, then quanta of this field can not directly be interpreted as a quantum of volume. It is interesting to note that in \cite{dzhunushaliev2022quantization} modified measures are related to a change in the fundamental length scale. This is very similar to the relation discovered in \cite{baryogenesis} where the modification of the measure is related to a change in the regularization length.  These questions are subject of ongoing research and we hope to shed some light on them in our upcoming papers \cite{cfstdqg, gravvariance}. 

We believe that the present paper is an indication that it might be worth to study the various possible splittings of the tasks the metric fulfills in the second order formulation of GR more systematically. In particular with respect to  possible phenomenology on cosmological scales. Having a literature with a wide variety of MMT models available will help significantly to speed up the investigation of phenomenological implications of CFS once the connection can be made rigorous. Similar research efforts regarding other, well motivated, modifications of GR might have similar synergies with various approaches to Quantum Gravity. 



\section*{Acknowledgement}
We would like to thank the COST Action 18108 QGMM for facilitating the exchange between different approaches to Quantum Gravity. The idea for this paper was born after two of us met during the writing of the survey article \cite{addazi2022quantum}. 

\bibliographystyle{amsplain}
%\bibliography{../../felix}
\bibliography{claudio}

\end{document}

