Our results showed that participant ratings of media trust and perceived accuracy before and after exposure to provenance information were different to a statistically significant degree. Additionally, those differences varied with respect to the provenance state and media characteristics. We will now discuss higher-level observations from our findings and pose implications for the design of future provenance-bearing UIs on social media.

\subsection{Permeable boundary between provenance and content credibility}
Ideally, having access to provenance information should increase trust, perceived accuracy, and correction in authentic media (i.e. non-composited content that agrees with its claim) \cite{gundecha2013tool}. For inauthentic media, trust and perceived accuracy should decrease (as long as the claim states what is supposed to be represented), while correction should remain positive \cite{gundecha2013tool, emily2022usable, sherman2021designing}. We see from Table \ref{t:edit-topic} that non-composited content had positive $\Delta t$ and $\Delta \alpha$ in some cases, while the only significant $a_{imp}$ was negative. While the effect of provenance here does not match with the ideal scenario, it is important to note that a couple pieces of non-composited media did not agree with their claims, which lends some justification for the negative changes.

%false content \cite{jahanbakhsh2021nudges}, as well as improving the accuracy of user interpretations \cite{bode2018see, seo2019trust}. These techniques can also have the reverse effect, however, making viewers more ignorant of factual information \cite{jahanbakhsh2021nudges, dias2020emphasizing, garrett2013corrections}.

When we introduce provenance state in our analysis, we \remove{can add the additional dimension to our expectations for the ideal scenario. Content that does not display state may see a positive or negative change in trust and perceived accuracy ratings depending on the edits and claims, but correction should increase given the additional insights into the media's backstory. We }expect that participants will be more skeptical of media with provenance incompleteness and invalidity, drawing from previous work on misinformation warnings \cite{ozturk2015combating, seo2019trust}. For incomplete states, there is no additional information given to evaluate perceived accuracy, so we do not expect a change there; however, invalid states (which occur when the provenance chain is tampered with \cite{c2pa-ux}) may justify lowering perceived accuracy due to suspected malicious intent. %For incomplete and invalid states, we might expect trust to decrease due to the , but agreement should remain at similar levels for in the incomplete state while it may drop for the invalid state.
However, we see from Table \ref{t:agreement-state} that correction and change in perceived accuracy are negative for both incomplete and invalid states. That is, provenance had the opposite effect for correction as originally intended---instead of correcting perceptions, it caused participants to stray further from the truth rating. We see a similar pattern with previous work in credibility indicators, where users exposed to the indicators mistakenly saw true content as false \cite{ dias2020emphasizing, garrett2013corrections}. This is particularly noticeable in the invalid state, where an authentic news image garnered significantly less trust and perceived accuracy than its normal state counterparts. In general, incomplete and invalid states both appear to decrease trust and perceived accuracy in content regardless of content claim agreement. 

One likely explanation for this is that participants confuse \textit{content} credibility with \textit{provenance} credibility. Indeed, we see evidence of this confusion in our qualitative analysis. For example, P97, P311, and P340 all provided feedback on the icon colour while implicitly or explicitly referring to the contents of the media rather than the provenance information displayed in the UI. Colour was used to indicate the credibility of the provenance chain but participants appeared to treat it as a glanceable indicator of content credibility. This is not surprising given that the current landscape of credibility indicators mostly involve action-oriented warnings about the content itself \cite{sherman2021designing, twitter-manip-label, jahanbakhsh2021nudges} and the hasty consumption of information on social media \cite{flintham2018falling}. However, content and provenance credibilities are related but orthogonal concepts. \add{For example, a deceptively composited photo (with low content credibility) may bear a normal provenance state (with high provenance credibility) because its provenance information was updated properly at every edit stage. A viewer who is not yet aware of the composition may incorrectly assume that the photo's content is more credible than another that is unedited but has an invalid provenance state.} %The boundary between content and provenance credibility is currently blurry, with assertions of provenance validity often mixing with those of content.

As our example hints, the lack of distinction between content and provenance credibility can lead users to over-trust inauthentic media with a good provenance record or under-trust authentic media with imperfect provenance. Invalid provenance can indicate deceptive intent, but incomplete provenance can arise in benign situations. Malicious actors may intentionally introduce incompleteness to lower perceived credibility or flood a social media feed with red invalid provenance indicators. It is therefore essential to supplement, or even preface, the deployment of provenance-enabled systems with sufficient user education and on differences between content and provenance credibilities.


% \subsection{Heavy graphical edits may not necessarily be important ones}
\subsection{Not all edits are created equal}
\label{s:edits-unequal}
We see from Table \ref{t:edit-topic} that trust and perceived accuracy both decreased while correction was positive in composited content. While those movements are favourable in the context of this study as both pieces composited media were deceptive (i.e. did not agree with their claims), they may not be in all composited media. Compositions can take on a wide variety of forms. Placing a watermark over or a logo in the corner of an image or video is a common act of composition, and while the media has been edited, it can still authentically depict original content. We cannot extrapolate the movements in trust, perceived accuracy, and correction to media in our study to innocent compositions, but we are aware of this possibility given participants' skepticism over content manipulations in our free response question. Universal decrease in trust and perceived accuracy in composited media may present significant challenges in provenance adoption. Content creators who add a logo or watermark to their media may hesitate publishing to provenance-enabled platforms when users will unjustifiably view their content as less reliable and trustworthy. More consideration of how to differentiate routine compositions from malicious ones is needed.

This need is not solely constrained to compositions, but applies to edits more generally. Edits can be graphically intensive but not significantly change the meaning of media. For example, the Pulitzer-winning photo taken by John Filo during the Kent State Massacre in 1970 was anonymously edited for aesthetic reasons before appearing in several high-profile magazines \cite{petapixel}. On the other hand, a small edit gives opportunity to craft an entirely different story about a piece of media. For example, After the French national football team won the World Cup in 2018, a photo circulated on Facebook that appeared to show Black players in the second row at a photo op at the French presidential palace \cite{observers}, leading to accusations of racial discrimination. The photo was later revealed to have been cropped and the original photo showed Black players in the front row. These two examples illustrate how the graphical complexity of an edit may not correspond to its importance in changing the meaning of media, and edits of similar complexities may have different levels of importance. This separation of importance is currently not represented in our provenance UIs. \add{Relatedly, some manipulations may make deceptive intentions clear to users if they appeared in the provenance chain, while others may not and can even further confuse the user. We can envision the future development of a taxonomy of manipulations based on their efficacy in revealing deceptions via provenance.}


\subsection{Design implications}
\label{s:design-implications}
In this study, we placed provenance indicators in the top right corner of media in a post, in line with the C2PA UX specifications \cite{c2pa-ux}. However, there are many alternatives for placement and design of provenance indicators, such as appearing below or in the corner of the media instead of on top of it. This may require changing the indicator design and making it text-based and native to the platform UI. Text-based indicators that conform to platform design guidelines may be perceived as more trustworthy than graphical icons, as demonstrated in previous work on end-to-end encrypted messaging. \cite{stransky2021encryption}. The use of colour is also a factor to consider in indicator design. Currently, many participants have requested bolder use of colour to create more obvious signals of problematic content. However, because content and provenance credibilities are still not well-distinguished concepts, we think using colour in the way participants suggested can actually be misleading. Early deployments of provenance indicators can consider removing colour altogether to sidestep this confusion until the users establish a clearer mental model of provenance and its relationship to content credibility. 

This study's provenance details panels were relatively simple, with provenance chains of no longer than 3 manifests. However, they may be much more complex in practice. As suggested in Section \ref{s:edits-unequal}, highlighting the importance of an edit with respect to visual semantics rather than graphical complexity can help ``cut through the noise.'' The current design of the provenance does not establish visual hierarchy among edits details (see Fig. \ref{fig:stateful-l2}). Designing a hierarchical system for visually displaying edits according to their semantic relevance may be essential for managing complexity in real-world provenance detail panels. 

On a higher level, we observe design challenges stemming from a foundational tension that exists between provenance and current social media norms. Participants' desire for quick and easy ways to tell whether a piece of media is authentic (as they explained in Section \ref{s:qual}), along with current treatment of misinformation labels as warnings, reveal that users are increasingly accustomed to \textit{being told outright} what they should and should not believe. This engages what Kahneman \cite{kahneman2011thinking} calls ``System 1'' thinking---effortless, intuitive, and sometimes illogical. Provenance, however, takes a different approach: provide users with resources they need to make their own informed decisions on whether to believe what they see. In other words, users still need to \textit{make their own judgements} rather than having one prescribed to them. Kahneman calls this more effortful and rational style of thinking ``System 2.'' Design affordances that allow users to better switch between ``System 1'' and ``System 2'' modes of thought while evaluating media can allow them to effectively leverage provenance information to make credibility decisions.

%A careful balancing act now presents itself when considering design choices for provenance indicators. On one hand, using existing visual cues and mental models for provenance indicator iconography can provide users with familiar interactions for a gentler introduction to a new concept. It is also aligned with some participant suggestions in Section \ref{s:qual}. On the other hand, crafting a distinct visual language for provenance may help to accelerate the aforementioned mentality shift and and improve comprehension in the long run. 
When designing provenance indicators, it is important to balance using familiar visual cues (in line with some participant suggestions in Section \ref{s:qual}), and using a distinct visual language that may improve conceptual understanding. Case studies on user interfaces used when introducing new (now well-known) web conventions, such as the AdChoices program \cite{adchoices}, as well as past work in functional metaphors for end-to-end encryption \cite{demjaha2018metaphors}, may have valuable lessons going forward.


