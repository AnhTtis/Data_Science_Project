This study was primarily quantitative, so we could not capture nuanced interactions with specific elements of the provenance UI or account for different interpretations of survey questions, which may have led to inconsistent definitions of \textsc{trust} and \textsc{perceived accuracy}. A complementary qualitative study with participants completing tasks on a provenance-enabled prototype and being interviewed about their experiences would provide more insights. \add{Our deceptive media were also modified through composition, so we did not observe participants use other means of editing to reason about credibility. We anticipate gßenerative AI to be an increasingly popular method of media creation and manipulation on social media in the coming years due to the public availability of text-to-image models, such as DALL-E \cite{ramesh2021dalle} and Stable Diffusion \cite{rombach2022diffusion}. We are thus excited about the potential of provenance standards to provide much-needed transparency and disclosures to viewers of AI-generated content.} 

Our study used a social media prototype resembling Twitter. Variation in participants' past experiences with and general perceptions of the platform have affected our results in ways we did not account for. Future work can expand this study to other platforms as well as other provenance standards, such as Arweave \cite{arweave} and Four Corners \cite{fourcorners}. Even within the same standard, different variations in provenance indicator placement and iconography can be further studied to better address the tension between automatic and intentional methods of information consumption outlined in Section \ref{s:design-implications}. Additionally, as mentioned in Section \ref{s:design-implications}, we had relatively simple provenance details panels in this study, which may not accurately reflect real-world UIs once the system is deployed. Future work, perhaps of interest to the computer vision community, can study more complex provenance details panels and develop techniques to highlight edits with more visual semantic relevance to users.

We measured participants' media credibility perceptions through two quantitative variables---trustworthiness and perceived accuracy of claim---but there are many other approaches that could also be used, as outlined in Section \ref{s:rw-measures}. Our content was also limited to 9 pieces of media per feed to keep the study completion time reasonable and participant fatigue low. This constrained the combination of levels within \textsc{media topic}, \textsc{edit status}, \textsc{claim agreement}, and \textsc{provenance state} we could have for each piece of media. While we managed to include at least one piece of media in each of those level combinations, we did not have any composited news media, nor did we have any composited invalid media. Our media also had concrete ``ground truth'' claims from Snopes, but other content may not have an easily identifiable claim, or have a claim that can be easily agreed upon. Future work can explore impacts of provenance on media with uncertain or nonexistent claims. 

 %Finally, this study assumes unfamiliarity with provenance UIs and simulates users' first exposure to them, which may not reflect perceptions as users become accommodated to interacting with the UIs over time. We did not investigate longitudinal effects of provenance, nor any byproducts of provenance with a temporal component. For example, one of the most popular sites of collective sensemaking on social media is the comments section. Widely accessible provenance information may generate new discussions about media in the comments of the media's post that may influence credibility perceptions more powerfully than the provenance information itself. Discussions about provenance may also alter incentives for content creators to publish on provenance-enabled platforms: some may see it as a tool for mitigating misinformation and be more comfortable posting into a reliable content community, while others may fear being on the receiving end of false accusations and harassment. We see this as a rich avenue of future work.
Finally, his study assumed unfamiliarity with provenance UIs and simulated users' first exposure to them, but did not investigate longitudinal effects or byproducts with a temporal component, such as discussions in comments sections or incentives for content creators. We see these as rich avenues for future work.

