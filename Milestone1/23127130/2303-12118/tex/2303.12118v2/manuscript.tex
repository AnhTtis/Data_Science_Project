%%
%% This is file `sample-manuscript.tex',
%% generated with the docstrip utility.
%%
%% The original source files were:
%%
%% samples.dtx  (with options: `manuscript')
%% 
%% IMPORTANT NOTICE:
%% 
%% For the copyright see the source file.
%% 
%% Any modified versions of this file must be renamed
%% with new filenames distinct from sample-manuscript.tex.
%% 
%% For distribution of the original source see the terms
%% for copying and modification in the file samples.dtx.
%% 
%% This generated file may be distributed as long as the
%% original source files, as listed above, are part of the
%% same distribution. (The sources need not necessarily be
%% in the same archive or directory.)
%%
%% Commands for TeXCount
%TC:macro \cite [option:text,text]
%TC:macro \citep [option:text,text]
%TC:macro \citet [option:text,text]
%TC:envir table 0 1
%TC:envir table* 0 1
%TC:envir tabular [ignore] word
%TC:envir displaymath 0 word
%TC:envir math 0 word
%TC:envir comment 0 0
%%
%%
%% The first command in your LaTeX source must be the \documentclass command.
\documentclass[acmsmall]{acmart}
% \documentclass[anonymous, acmsmall, review]{acmart}

%%
%% \BibTeX command to typeset BibTeX logo in the docs
\AtBeginDocument{%
  \providecommand\BibTeX{{%
    \normalfont B\kern-0.5em{\scshape i\kern-0.25em b}\kern-0.8em\TeX}}}

%% Rights management information.  This information is sent to you
%% when you complete the rights form.  These commands have SAMPLE
%% values in them; it is your responsibility as an author to replace
%% the commands and values with those provided to you when you
%% complete the rights form.
\setcopyright{acmcopyright}
\copyrightyear{2023}
\acmYear{2023}
\acmDOI{XXXXXXX.XXXXXXX}

%% These commands are for a PROCEEDINGS abstract or paper.
\acmConference[CSCW '23]{Make sure to enter the correct
  conference title from your rights confirmation email}{October 14-18, 2023}{Minneapolis, MN}
\acmPrice{15.00}
\acmISBN{978-1-4503-XXXX-X/18/06}

\usepackage{rotating}
\usepackage{graphicx}  
\usepackage{url}
\usepackage{booktabs}  
\usepackage{caption}
\usepackage{subcaption}
\usepackage{tabularx}
\usepackage{tablefootnote}
\usepackage{multirow}
\usepackage[normalem]{ulem}


\graphicspath{{figures/}{pictures/}{images/}{./}}
%%
%% Submission ID.
%% Use this when submitting an article to a sponsored event. You'll
%% receive a unique submission ID from the organizers
%% of the event, and this ID should be used as the parameter to this command.
%%\acmSubmissionID{123-A56-BU3}

%%
%% For managing citations, it is recommended to use bibliography
%% files in BibTeX format.
%%
%% You can then either use BibTeX with the ACM-Reference-Format style,
%% or BibLaTeX with the acmnumeric or acmauthoryear sytles, that include
%% support for advanced citation of software artefact from the
%% biblatex-software package, also separately available on CTAN.
%%
%% Look at the sample-*-biblatex.tex files for templates showcasing
%% the biblatex styles.
%%

%%
%% The majority of ACM publications use numbered citations and
%% references.  The command \citestyle{authoryear} switches to the
%% "author year" style.
%%
%% If you are preparing content for an event
%% sponsored by ACM SIGGRAPH, you must use the "author year" style of
%% citations and references.
%% Uncommenting
%% the next command will enable that style.
%%\citestyle{acmauthoryear}

% \newcommand{\addd}[1]{\textcolor{cyan}{\noindent{#1}}}
% \newcommand{\removee}[1]{\textcolor{red}{\noindent \sout{#1}}}
\newcommand{\addd}[1]{\textcolor{black}{\noindent{#1}}}
\newcommand{\removee}[1]{\textcolor{black}{\noindent \sout{#1}}}
\newcommand{\add}[1]{\textcolor{black}{\noindent{#1}}}
\newcommand{\remove}[1]{\textcolor{black}{\unskip}}

%%
%% end of the preamble, start of the body of the document source.
\begin{document}

%%
%% The "title" command has an optional parameter,
%% allowing the author to define a "short title" to be used in page headers.
\title[Impact of Provenance-Enabled Media]{Examining the Impact of Provenance-Enabled Media on Trust and Accuracy Perceptions}
% Brainstorming titles:
% The Impact of Media Provenance Indicators within Social Media Feeds on User Perceptions of Trust...and?

%%
%% The "author" command and its associated commands are used to define
%% the authors and their affiliations.
%% Of note is the shared affiliation of the first two authors, and the
%% "authornote" and "authornotemark" commands
%% used to denote shared contribution to the research.


\author{K. J. Kevin Feng}
\affiliation{%
  \institution{University of Washington}
  \city{Seattle}
  \country{USA}}
\email{kjfeng@uw.edu}

\author{Nick Ritchie}
\affiliation{%
  \institution{BBC UX\&D}
  \city{London}
  \country{United Kingdom}}
\email{nick.ritchie@bbc.co.uk}

\author{Pia Blumenthal}
\affiliation{%
  \institution{Adobe, Inc.}
  \city{New York}
  \country{USA}}
\email{piab@adobe.com}

\author{Andy Parsons}
\affiliation{%
  \institution{Adobe, Inc.}
  \city{New York}
  \country{USA}}
\email{andyp@adobe.com}

\author{Amy X. Zhang}
\affiliation{%
  \institution{University of Washington}
  \city{Seattle}
  \country{USA}}
\email{axz@cs.uw.edu}

%%
%% By default, the full list of authors will be used in the page
%% headers. Often, this list is too long, and will overlap
%% other information printed in the page headers. This command allows
%% the author to define a more concise list
%% of authors' names for this purpose.
\renewcommand{\shortauthors}{Feng et al.}

%%
%% The abstract is a short summary of the work to be presented in the
%% article.
\begin{abstract}

In recent years, industry leaders and researchers have proposed to use technical provenance standards to address visual misinformation spread through digitally altered media. By adding immutable and secure provenance information such as authorship and edit date to media metadata, social media users could potentially better assess the validity of the media they encounter. However, it is unclear how end users would respond to provenance information, or how to best design provenance indicators to be understandable to laypeople. We conducted an online experiment with 595 participants from the US and UK to investigate how provenance information altered users' accuracy perceptions and trust in visual content shared on social media. We found that provenance information often lowered trust and caused users to doubt deceptive media, particularly when it revealed that the media was composited. We additionally tested conditions where the provenance information itself was shown to be incomplete or invalid, and found that these states have a significant impact on participants' accuracy perceptions and trust in media, leading them, in some cases, to disbelieve honest media. Our findings show that provenance, although enlightening, is still not a concept well-understood by users, who confuse \textit{media} credibility with the orthogonal (albeit related) concept of \textit{provenance} credibility. We discuss how design choices may contribute to provenance (mis)understanding, and conclude with implications for usable provenance systems, including clearer interfaces and user education.


% Current reactive techniques to combat visual misinformation, such as crowdsourcing human fact-checkers to flag problematic media and algorithmically comparing similar media for edits, are easily outpaced by the scale and speed of digital media distribution as well as increasing sophistication in media editing tools. Provenance standards illuminate a promising path forward. By allowing provenance information such as authorship and edit date to be immutably and securely chained onto media metadata, these standards can empower users to make a more informed decision about media validity without having to manually research the media's backstory themselves. However, understanding how users respond to provenance information in social media settings is not a well-studied area, despite its importance to the successful adoption of usable provenance systems. To address this gap, we conducted an online study with 595 participants from the US and UK to investigate how user perceptions of media validity---as defined by trust and truth judgements---changed upon the introduction of provenance information in a social media feed. We found that provenance mostly lowered trust and caused users to doubt the truth of media, even with authentic content. Although provenance was helpful in correcting initially misled truth judgements in deceptive media, it led participants farther away from the truth in some honest media. Additionally, signals of incomplete and invalid provenance triggered a more unfavourable outlook on media validity. Our findings show that provenance, although enlightening in some cases, is still not a concept well-understood by users, and that provenance authenticity can be easily confused with the orthogonal (albeit related) concept of media authenticity. We posit granular and high-level implications on the design of usable provenance systems going forward, with an emphasis on conceptual clarity in provenance-enabled UIs and user education.
\end{abstract}

%%
%% The code below is generated by the tool at http://dl.acm.org/ccs.cfm.
%% Please copy and paste the code instead of the example below.
%%
\begin{CCSXML}
<ccs2012>
   <concept>
       <concept_id>10003120.10003121.10011748</concept_id>
       <concept_desc>Human-centered computing~Empirical studies in HCI</concept_desc>
       <concept_significance>500</concept_significance>
       </concept>
   <concept>
       <concept_id>10003120.10003121.10003122.10003334</concept_id>
       <concept_desc>Human-centered computing~User studies</concept_desc>
       <concept_significance>500</concept_significance>
       </concept>
 </ccs2012>
\end{CCSXML}

\ccsdesc[500]{Human-centered computing~Empirical studies in HCI}
\ccsdesc[500]{Human-centered computing~User studies}

%%
%% Keywords. The author(s) should pick words that accurately describe
%% the work being presented. Separate the keywords with commas.
\keywords{media provenance, social media, credibility perceptions}


%%
%% This command processes the author and affiliation and title
%% information and builds the first part of the formatted document.
\maketitle

\section{Introduction}
\section{Introduction}

The increasing complexity of source code poses a key challenge to the reliability of large-scale software systems. Software bugs in these systems can lead to safety issues~\cite{bug_safety} for users around the world as well as cause non-negligible financial losses~\cite{bug_loss}. As such, developers have to spend a large amount of time and effort on bug fixing. Consequently, \aprfull (\apr), designed to automatically generate patches to fix software bugs, has attracted wide attention from both academia and industry~\cite{long2016prophet, legoues2012genprog, long2015spr, lou2020can, tufano2018empstudy}. 


To achieve \apr, one popular approach is known as Generate-and-Validate (G\&V)~\cite{qi2015gv, ghanbari2019prapr, lou2020can, le2016hdrepair, legoues2012genprog, wen2018capgen, hua2018sketchfix, martinez2016astor, koyuncu2020fixminder, liu2019tbar, liu2019avatar}, which is typically based on the following pipeline: First, fault localization techniques~\cite{wong2016fl, abreu2007ochiai, zhang2013injecting, papadakis2015metallaxis, li2019deepfl, li2017transforming} are applied to determine the suspicious locations in programs where bugs are likely to exist. Then, the buggy locations are used by the \apr tools to generate a list of patches that replace buggy lines with correct lines. Afterward, each patch is validated against the original test suite to identify any \emph{plausible patches} (i.e., passing all tests in the test suite). Finally, to determine the \emph{correct patches}, developers examine the list of plausible patches to see if any of them can correctly fix the bug. 

Traditional \apr tools can mainly be categorized into heuristic-based~\cite{legoues2012genprog, le2016hdrepair, wen2018capgen}, constraint-based~\cite{mechtaev2016angelix, le2017s3, demacro2014nopol, long2015spr} and \template~\cite{ghanbari2019prapr, hua2018sketchfix, martinez2016astor, liu2019tbar, liu2019avatar}. Among these traditional tools, \template \apr tools~\cite{ghanbari2019prapr, liu2019tbar, benton2020effectiveness} have been able to achieve state-of-the-art results. \Template \apr tools typically leverage pre-defined templates (e.g., adding a nullness check) for bug fixing. However, since these fix templates are typically handcrafted, the number and types of bugs they are able to fix can be limited. 



To address the limitations of traditional \apr, researchers have proposed various \learning \apr tools~\cite{li2020dlfix, chen2018sequencer, jiang2021cure, lutellier2020coconut, zhu2021recoder, ye2022rewardrepair} based on the \nmtfull (\nmt) architecture~\cite{sutskever2014mt} where the input is the buggy code snippets and the goal is to translate the buggy code snippets into a fixed version. To accomplish this, \learning \apr tools require supervised training datasets with pairs of both buggy and fixed code snippets in order to learn how to perform this translation step. These training data are usually obtained by mining historical bug fixes using heuristics/keywords~\cite{dallmeier2007benchmark}, which can be imprecise for identifying bug-fixing commits; even the actual bug-fixing commits can include irrelevant code changes, leading to further pollution in the dataset~\cite{xia2022alpharepair}.
% 
Moreover, it can be hard for such \apr tools to generalize and fix bug types unseen during training. 



To better leverage recent advances in \plmfull{s} (\plm{s}), researchers~\cite{xia2022alpharepair, xia2023repairstudy, kolak2022patch, prenner2021codexws} have directly applied \plm{s} to generate patches without bug-fixing datasets. These \llm-based \apr tools work by either directly generating a complete code function~\cite{prenner2021codexws, xia2023repairstudy} or predict/infill the correct code snippet given its surrounding context~\cite{xia2022alpharepair, xia2023repairstudy}. By directly using \llm{s} that are pre-trained on billions of open-source code snippets, \llm-based \apr tools can achieve state-of-the-art performance on many repair datasets~\cite{xia2022alpharepair}. 


% 
%
%

Traditional \apr tools have long used the insight of the \emph{plastic surgery hypothesis}~\cite{barr2014plastic} where it states that the code ingredients to fix a bug already exist within the same project. Traditional \apr tools have manually designed pattern-~\cite{ghanbari2019prapr, saha2017elixir} or heuristic-based~\cite{jiang2018simfix, legoues2012genprog} approaches to finding and using such relevant code ingredients to generate fixes for bugs. However, the plastic surgery hypothesis has been largely ignored in \llm-based \apr. In fact, \llm provides a unique opportunity to fully automate the plastic surgery hypothesis idea via fine-tuning (learning project-specific information via model updates from the buggy project) and prompting (directly providing relevant code ingredients to the model), and make it directly applicable to different languages (since the \llm{s} are typically multi-lingual).%
Moreover, despite the intensive manual efforts involved, traditional \apr tools still cannot fully leverage project-specific information due to large search space for leveraging/composing existing code ingredients. In contrast, the project-specific information can effectively leveraged by \llm{s} due to their power in code understanding/vectorization, e.g., even partial/imprecise information may still guide \llm{s} in correct patch generation!
 To this end, we ask the question: \emph{How useful is the plastic surgery hypothesis in the era of \plm{s}}?








\mypara{Our Work.} To answer the question, we present \ourtech{\xspace} -- a \llm-based approach that automatically utilizes the plastic surgery hypothesis by systematically combining multiple fine-tuning and prompting strategies for \apr. \ourtech fine-tunes \plm{s} using two novel domain-specific training strategies: \textbf{\epfinetune} -- we fine-tune using the original buggy project by aggressively masking out a high percentage of tokens, which allows \plm to learn project-specific code tokens and programming styles; and \textbf{\rofinetune} -- which only masks out a single continuous code sequence per training sample, allowing the model to get used to the final \csapr task of predicting a single continuous code sequence. Furthermore, we directly leverage the ability for \plm{s} to understand natural language instructions and introduce a novel prompting strategy, \textbf{\idprompting}, which uses information retrieval and static analysis to obtain a list of relevant identifiers for the buggy lines. While such relevant identifiers are critical for fixing some difficult bugs, they may not be seen by the \llm during inference due to limited context window size. Through the use of prompting, we directly tell the model to use these extracted identifiers (relevant code ingredients) to generate the correct code. Finally, to perform repair, we combine all four model variants (including the base model, both fine-tuned models and the base model with prompting) for the final repair.





While our insight of leveraging the plastic surgery hypothesis for \llm-based \apr is generalizable across different types of \plm{s}, to implement \ourtech, we choose a recent \plm{\xspace}, \ctfive~\cite{wang2021codet5}, which is pre-trained on millions of open-source code snippets. \ctfive is an encoder-decoder model trained using \mspfull (\msp) objective where a percentage of tokens are masked out and each continuous masked token sequence is referred to as a masked span. Also, although we only extract relevant identifiers from the current buggy project (since this paper focuses on the plastic surgery hypothesis), our work can be easily extended to obtain other code information (such as relevant statements or functions) from other sources, such as  the massive pre-training corpora~\cite{husain2020codesearchnet} or historical bug-fixing datasets~\cite{jiang2019infer}, which can provide more coding knowledge for \llm{s}. Besides, although we mainly focus on using traditional string comparison algorithms for information retrieval in this paper, these techniques can be easily replaced by other frequency-based retrieval~\cite{robertson2009probabilistic} and neural search (or embedding-based search)~\cite{reimers2019sentence}.
  In summary, this paper makes the following contributions:


%


\begin{itemize}[noitemsep, leftmargin=*, topsep=0pt]
    \item \textbf{Dimension.} This paper is the first to revisit the important plastic surgery hypothesis in the era of \llm{s}. It opens up a new dimension for \llm-based \apr to incorporate previously neglected information from the buggy project itself to boost \apr performance. Furthermore, it demonstrates the promising future of retrieval-based prompting for modern \llm-based \apr.
    \item \textbf{Implementation.} We implement \ourtech based on the recent \ctfive model. We augment the model using two novel fine-tuning strategies: \epfinetune and \rofinetune, along with a novel prompting strategy based on information retrieval and static analysis: \idprompting. We combine the patches generated by all four models together and perform patch ranking to speed up \apr.% 
    \item \textbf{Evaluation Study.} We conduct an extensive evaluation against state-of-the-art \apr tools. On the widely studied \dfj 1.2 and 2.0 datasets~\cite{just2014dfj}, \ourtech is able to achieve the new state-of-the-art results of 89 and 44 correct bug fixes (15 and 8 more than best baseline) respectively.  Furthermore, we perform a broad ablation study to justify our design. \ourtech demonstrates for the first time that the plastic surgery hypothesis can substantially boost \llm-based \apr and advance state-of-the-art \apr, while being fully automated and general. Moreover, even partial/imprecise code ingredients may still effectively guide \llm{s} for \apr!
\end{itemize}



\section{Related Work}

\section{Related Work}\label{sec:related-work}



Over the last few years, several benchmarks for stream processing frameworks have been proposed and stream processing benchmarking studies have been conducted. The differentiation between benchmarks and experimental studies applying them is sometimes blurry. Many publications that present benchmarks perform also an experimental study with them. On the other hand, many experimental studies utilize existing benchmarks, but modify them.
Nevertheless, we structure this section into two parts: First, we give an overview of stream processing benchmarks to justify our benchmark selection for this study. Second, we discuss related stream processing benchmarking studies.

\subsection{Related Work on Stream Processing Benchmarks}

Besides the Theodolite benchmarks for event-driven microservices used in this study, several other benchmarks for stream processing frameworks have been proposed.
\cref{tab:related-benchmarks} summarizes characteristics of the discussed benchmarks. 


\begin{table*}
	\begin{threeparttable}[b]
		\caption{Overview of the characteristics and implementations of stream processing benchmarks.}
		\label{tab:related-benchmarks}
		\footnotesize
		\newcommand{\cmark}{\ding{51}}%
		\newcommand{\xmark}{\ding{55}}%
		\newcommand{\qmark}{\makebox[0pt][l]{\textbf{\textit{?}}}\phantom{\cmark}}%
		
		\newcommand{\txnote}[1]{\makebox[0pt][l]{\tnote{#1}}}
		
		\newcommand\undefcolumntype[1]{\expandafter\let\csname NC@find@#1\endcsname\relax}
		\newcommand\forcenewcolumntype[1]{\undefcolumntype{#1}\newcolumntype{#1}}
		
		\newcommand*\rot{\rotatebox{90}}
		\newcolumntype{L}{>{\raggedright\arraybackslash}X}
		\newcolumntype{R}{>{\raggedleft\arraybackslash}X}
		\newcolumntype{C}{>{\centering\arraybackslash}X}
		\newcolumntype{o}{p{0pt}}
		\renewcommand{\arraystretch}{1.2}
		\newcommand{\fnoptional}{a}
		\newcommand{\fnbeam}{b}
		\newcommand{\fnriottasksamples}{d}
		\newcommand{\fnbeamnexmark}{c}
		\begin{tabularx}{\textwidth}{ll o C o C o CCC o CCCCCCC o C o CCC}
			\toprule
			&&& && && \multicolumn{3}{c}{Messaging} && \multicolumn{7}{c}{Stream processing framework} && && \multicolumn{3}{c}{Cloud-native} \\
			\cmidrule{8-10} \cmidrule{12-18} \cmidrule{22-24}
			Benchmark & Published && \rot{Task samples} && \rot{Open source} && \rot{Kafka} & \rot{Others} & \rot{None} && \rot{Flink} & \rot{Spark} & \rot{Storm} & \rot{Samza} & \rot{Kafka Streams} & \rot{Hazelcast Jet} & \rot{Others} && \rot{Database} && \rot{Containers} & \rot{Kubernetes} & \rot{Others} \\
			\midrule
			Theodolite \cite{BDR2021} & \citeyear{BDR2021}
			& %
			& 4
			& %
			& \cmark %
			& %
			& \cmark %
			& %
			& %
			& %
			& \cmark %
			& %
			& %
			& \cmark\txnote{\fnbeam} %
			& \cmark %
			& \cmark %
			& \cmark\txnote{\fnbeam} %
			& %
			& \phantom{\cmark}\txnote{\fnoptional} %
			& %
			& \cmark %
			& \cmark %
			& %
			\\
			Beam Nexmark \cite{BeamNexmark2022} & \citeyear{BeamNexmark2022}\txnote{\fnbeamnexmark}
			& %
			& 13
			& %
			& \cmark %
			& %
			& \cmark %
			& \cmark %
			& %
			& %
			& \cmark\txnote{\fnbeam} %
			& \cmark\txnote{\fnbeam} %
			& %
			& \qmark\txnote{\fnbeam} %
			& %
			& \qmark\txnote{\fnbeam} %
			& \cmark\txnote{\fnbeam} %
			& %
			& %
			& %
			& %
			& %
			& %
			\\
			ESPBench \cite{Hesse2021} & \citeyear{Hesse2021}
			& %
			& 5
			& %
			& \cmark %
			& %
			& \cmark %
			& %
			& %
			& %
			& \cmark\txnote{\fnbeam} %
			& \cmark\txnote{\fnbeam} %
			& %
			& \qmark\txnote{\fnbeam} %
			& %
			& \cmark\txnote{\fnbeam} %
			& \qmark\txnote{\fnbeam} %
			& %
			& \cmark %
			& %
			& %
			& %
			& %
			\\
			OSPBench \cite{vanDongen2020} & \citeyear{vanDongen2020}
			& %
			& 5
			& %
			& \cmark %
			& %
			& \cmark %
			& %
			& %
			& %
			& \cmark %
			& \cmark %
			& %
			& %
			& \cmark %
			& %
			& %
			& %
			& %
			& %
			& \cmark %
			& %
			& \cmark %
			\\
			DSPBench \cite{Bordin2020} & \citeyear{Bordin2020}
			& %
			& 5
			& %
			& \cmark %
			& %
			& \cmark %
			& %
			& %
			& %
			& %
			& \cmark %
			& \cmark %
			& %
			& %
			& %
			& %
			& %
			& \cmark %
			& %
			& %
			& %
			& %
			\\
			\citet{Shahverdi2019} & \citeyear{Shahverdi2019}
			& %
			& 1
			& %
			& \cmark %
			& %
			& \cmark %
			& %
			& %
			& %
			& \cmark %
			& \cmark %
			& \cmark %
			& %
			& \cmark %
			& \cmark %
			& %
			& %
			& \cmark %
			& %
			& %
			& %
			& %
			\\
			\citet{Karimov2018} & \citeyear{Karimov2018}
			& %
			& 2
			& %
			& %
			& %
			& %
			& %
			& \cmark %
			& %
			& \cmark %
			& \cmark %
			& \cmark %
			& %
			& %
			& %
			& %
			& %
			& %
			& %
			& %
			& %
			& %
			\\
			RIoTBench \cite{Shukla2017} & \citeyear{Shukla2017}
			& %
			& 4\txnote{\fnriottasksamples} %
			& %
			& \cmark %
			& %
			& %
			& \cmark %
			& %
			& %
			& %
			& %
			& \cmark %
			& %
			& %
			& %
			& %
			& %
			& \cmark %
			& %
			& %
			& %
			& %
			\\
			YSB \cite{Chintapalli2016} & \citeyear{Chintapalli2016}
			& %
			& 1
			& %
			& \cmark %
			& %
			& \cmark %
			& %
			& %
			& %
			& \cmark %
			& \cmark %
			& \cmark %
			& %
			& %
			& %
			& %
			& %
			& \cmark %
			& %
			& %
			& %
			& %
			\\
			SparkBench \cite{Li2015} & \citeyear{Li2015}
			& %
			& 10
			& %
			& \cmark %
			& %
			& %
			& %
			& \cmark %
			& %
			& %
			& \cmark %
			& %
			& %
			& %
			& %
			& %
			& %
			& %
			& %
			& %
			& %
			& %
			\\
			StreamBench \cite{Lu2014} & \citeyear{Lu2014}
			& %
			& 7
			& %
			& %
			& %
			& \cmark %
			& %
			& %
			& %
			& %
			& \cmark %
			& \cmark %
			& %
			& %
			& %
			& %
			& %
			& %
			& %
			& %
			& %
			& %
			\\
			Linear Road \cite{Arasu2004} & \citeyear{Arasu2004}
			& %
			& 5
			& %
			& %
			& %
			& %
			& %
			& \cmark %
			& %
			& %
			& %
			& %
			& %
			& %
			& %
			& \cmark %
			& %
			& %
			& %
			& %
			& %
			& %
			\\
			\bottomrule
		\end{tabularx}
		\begin{tablenotes}\footnotesize
			\item[\fnoptional] optional
			\item[\fnbeam] using Apache Beam
			\item[\fnbeamnexmark] the Beam Nexmark benchmarks are based on the Nexmark paper \cite{Tucker2010} published in \citeyear{Tucker2010}
			\item[\fnriottasksamples] RIoTBench's 4 application benchmarks are composed of 27 microbenchmarks
		\end{tablenotes}
	\end{threeparttable}
\end{table*}



StreamBench~\cite{Lu2014} is one of the earliest benchmarks for modern stream processing frameworks. While originally only implemented for Spark and Storm, it has later been used to benchmark Apache Apex, Beam, Flink, and Samza as well \cite{Hesse2019, Qian2016}.
As its name suggests, SparkBench~\cite{Li2015} is a benchmark tailored to Apache Spark.
The Yahoo Streaming Benchmark (YSB) \cite{Chintapalli2016} is frequently used and adapted in research \cite{Lopez2016, Yang2017, Karakaya2017, Nasiri2019, Zeuch2019, Chu2020, vanDongen2020}.
Worth highlighting is the work of \citet{Shahverdi2019}, who extend YSB with implementations for the frameworks Kafka Streams and Hazelcast Jet. As discussed in \cref{sec:frameworks}, these frameworks are particularly suited for building event-driven microservices.
RIoTBench \cite{Shukla2017} provides four application benchmarks for Storm composed of 27~small task samples. \citet{Nasiri2019} adopt RIoTBench for Flink and Spark.
\citet{Karimov2018} present a benchmark with two task samples, derived from a real industrial context, yet without providing open-source implementations.

More recently, DSPBench \cite{Bordin2020}, OSPBench~\cite{vanDongen2020}, and ESPBench \cite{Hesse2021} have been proposed.
DSPBench contains 15~benchmarks, which resample typical stream processing applications, derived from reviewing the literature.
OSPBench provides benchmarks for analyzing traffic sensor data. Besides evaluations of latency, throughput, and resource usage, \citeauthor{vanDongen2020} used OSPBench to also evaluate scalability~\cite{vanDongen2021b} and fault recovery~\cite{vanDongen2021a}.
In contrast to most other benchmarks, OSPBench provides implementations for the rather new framework Kafka Streams, which is also evaluated in this study.
The Enterprise Stream Processing Benchmark (ESPBench) builds upon the Senska benchmark \cite{Hesse2018}.
It is special in the sense that it integrates a relational database management system.
In contrast to most other benchmarks, ESPBench's task samples are implemented with Apache Beam. While \citet{Hesse2021} only perform evaluations with Spark, Flink, and Hazelcast Jet, we expect that also other Beam runners can be used to run the benchmark.

The Nexmark benchmark \cite{Tucker2010} has originally been proposed as the \textit{Niagara Extension to the XMark benchmark} addressed to first-generation stream processing systems (see the survey of \citet{Fragkoulis2023} for a discussion of first and second-generation stream processing systems).
The Apache Beam community adapted and extended Nexmark with implementations for Beam to benchmark the performance of different runners~\cite{BeamNexmark2022}.
Documentation and benchmark results are provided for the direct runner as well as for the Flink, the Spark, and the Google Cloud Dataflow runners.
However, running the benchmark with other runners should be possible as well.
Recently, there seems to be an effort to implement the Nexmark task samples with other frameworks in an open-source project.\footnote{\url{https://github.com/nexmark/nexmark}}
However, currently this project only provides implementations for Apache Flink.
Moreover, \citet{Gencer2021} implemented the Nexmark benchmark for their performance evaluation of Hazelcast Jet.

Worth mentioning is also the Linear Road benchmark presented by \citet{Arasu2004}. Although published years before all modern stream processing frameworks considered in this work have been released, it is still used in research \cite{Zhang2017,Zeuch2019,Sax2020} and compared to newer benchmarks \cite{Bordin2020,Hesse2021}.
\citet{Pagliari2020} and \citet{Garcia2022a, Garcia2022b} present approaches to generate benchmarks.






From \cref{tab:related-benchmarks}, we can see that a lot of open-source benchmarks have been proposed. Apart from the Theodolite benchmarks, none of these benchmarks is particularly addressed to scalability.
Often originating in data management research, many benchmarks are defined as ``queries'' over data streams~\cite{Tucker2010,Karimov2018,Hesse2021}.
Most benchmarks include a messaging system as a middleware component between workload generation and stream processing framework. In the vast majority of cases, this is Apache Kafka.
\citet{Karimov2018} exclude such a system to not let it become the benchmark's bottleneck. Our Theodolite benchmarks purposely include Kafka to represent more realistic event-driven microservice deployments~\cite{BDR2021}.
Flink, Spark, and Storm are by far the most supported frameworks. Only a few benchmarks exist for Samza, Kafka Streams, and Hazelcast Jet, which are frameworks particularly suited for implementing event-driven microservice. Our Theodolite benchmarks are the only ones providing implementations for all of them.
While some benchmarks include an interaction with a database in their setup, others do not.
With the Theodolite benchmarks, a database can optionally be used as we did in a previous study~\cite{IC2E2022FaaSStreaming}.
Besides our Theodolite benchmarks, there is only one other benchmark (OSPBench) that is provided as container images to be used in a cloud-native setting. No other benchmark provides Kubernetes manifests.





\subsection{Related Work on Stream Processing Benchmarking}


\begin{table*}
	\begin{threeparttable}[b]
		\caption{Overview of employed benchmarks, frameworks, and experimental setup of stream processing benchmarking studies.}
		\label{tab:related-experiments}
		\footnotesize
		\newcommand{\cmark}{\ding{51}}%
		\newcommand{\xmark}{\ding{55}}%
		\newcommand{\qmark}{\makebox[0pt][l]{\textbf{\textit{?}}}\phantom{\cmark}}%
		
		\newcommand{\txnote}[1]{\makebox[0pt][l]{\tnote{#1}}}
		
		\newcommand\undefcolumntype[1]{\expandafter\let\csname NC@find@#1\endcsname\relax}
		\newcommand\forcenewcolumntype[1]{\undefcolumntype{#1}\newcolumntype{#1}}
		
		\newcommand*\rot{\rotatebox{90}}
		\newcolumntype{L}{>{\raggedright\arraybackslash}X}
		\newcolumntype{R}{>{\raggedleft\arraybackslash}X}
		\newcolumntype{C}{>{\centering\arraybackslash}X}
		\newcolumntype{o}{p{0pt}}
		\renewcommand{\arraystretch}{1.2}
		\newcommand{\fnvandenpoel}{a}
		\newcommand{\fnbeam}{b}
		\begin{tabularx}{\textwidth}{ll o CCCCCCCCCCCCC o CCCCCCC o CCCCCC}
			\toprule
			&&& \multicolumn{13}{c}{Benchmark} && \multicolumn{7}{c}{Framework} && \multicolumn{6}{c}{Execution} \\
			\cmidrule{4-16} \cmidrule{18-24} \cmidrule{26-31}
			Publication & Year &&
			\rot{Theodolite \cite{BDR2021}} &
			\rot{Beam Nexmark \cite{BeamNexmark2022}} &
			\rot{ESPBench \cite{Hesse2021}} &
			\rot{OSPBench \cite{vanDongen2020}} &
			\rot{DSPBench \cite{Bordin2020}} &
			\rot{\citet{Shahverdi2019}} &
			\rot{\citet{Karimov2018}} &
			\rot{RIoTBench \cite{Shukla2017}} &
			\rot{YSB \cite{Chintapalli2016}} &
			\rot{SparkBench \cite{Li2015}} &
			\rot{StreamBench \cite{Lu2014}} &
			\rot{Linear Road \cite{Arasu2004}} &
			\rot{Others}
			&&
			\rot{Flink} &
			\rot{Spark} &
			\rot{Storm} &
			\rot{Samza} &
			\rot{Kafka Streams} &
			\rot{Hazelcast Jet} &
			\rot{Others}
			&&
			\rot{Cloud environment} &
			\rot{Distributed} &
			\rot{Different resource amounts} &
			\rot{\dots in isolated experiments} &
			\rot{Different load intensities} &
			\rot{\dots in isolated experiments}
			\\
			\midrule
			This work &
				& %
				& \cmark %
				& %
				& %
				& %
				& %
				& %
				& %
				& %
				& %
				& %
				& %
				& %
				& %
				& %
				& \cmark %
				& %
				& %
				& \cmark\txnote{\fnbeam} %
				& \cmark %
				& \cmark %
				& %
				& %
				& \cmark %
				& \cmark %
				& \cmark %
				& \cmark %
				& \cmark %
				& \cmark %
			\\
			\citet{IC2E2022FaaSStreaming} & \citeyear{IC2E2022FaaSStreaming}
				& %
				& \cmark %
				& %
				& %
				& %
				& %
				& %
				& %
				& %
				& %
				& %
				& %
				& %
				& %
				& %
				& \cmark\txnote{\fnbeam} %
				& %
				& %
				& \cmark\txnote{\fnbeam} %
				& %
				& %
				& \cmark\txnote{\fnbeam} %
				& %
				& \cmark %
				& \cmark %
				& \cmark %
				& \cmark %
				& \cmark %
				& \cmark %
			\\
			\citet{Hesse2021} & \citeyear{Hesse2021}
				& %
				& %
				& %
				& \cmark %
				& %
				& %
				& %
				& %
				& %
				& %
				& %
				& %
				& %
				& %
				& %
				& \cmark\txnote{\fnbeam} %
				& \cmark\txnote{\fnbeam} %
				& %
				& %
				& %
				& \cmark\txnote{\fnbeam} %
				& %
				& %
				& %
				& \cmark %
				& \cmark %
				& \cmark %
				& %
				& %
			\\
			van Dongen\tnote{\fnvandenpoel} \cite{vanDongen2021b} & \citeyear{vanDongen2021b}
				& %
				& %
				& %
				& %
				& \cmark %
				& %
				& %
				& %
				& %
				& %
				& %
				& %
				& %
				& %
				& %
				& \cmark %
				& \cmark %
				& %
				& %
				& \cmark %
				& %
				& %
				& %
				& \cmark %
				& \cmark %
				& \cmark %
				& %
				& \cmark %
				& \cmark %
			\\
			van Dongen\tnote{\fnvandenpoel} \cite{vanDongen2021a} & \citeyear{vanDongen2021a}
				& %
				& %
				& %
				& %
				& \cmark %
				& %
				& %
				& %
				& %
				& %
				& %
				& %
				& %
				& %
				& %
				& \cmark %
				& \cmark %
				& %
				& %
				& \cmark %
				& %
				& %
				& %
				& \cmark %
				& \cmark %
				& %
				& %
				& \cmark %
				& %
			\\
			\citet{Bordin2020} & \citeyear{Bordin2020}
				& %
				& %
				& %
				& %
				& %
				& \cmark %
				& %
				& %
				& %
				& %
				& %
				& %
				& %
				& %
				& %
				& %
				& \cmark %
				& \cmark %
				& %
				& %
				& %
				& %
				& %
				& \cmark %
				& \cmark %
				& %
				& %
				& \cmark %
				& \cmark %
			\\
			\citet{Chu2020} & \citeyear{Chu2020}
				& %
				& %
				& %
				& %
				& %
				& %
				& %
				& %
				& %
				& \cmark %
				& %
				& %
				& %
				& \cmark %
				& %
				& \cmark %
				& %
				& \cmark %
				& %
				& %
				& %
				& \cmark %
				& %
				& %
				& \cmark %
				& \cmark %
				& %
				& %
				& %
			\\
			\citet{Vikash2020} & \citeyear{Vikash2020}
				& %
				& %
				& %
				& %
				& %
				& %
				& %
				& %
				& %
				& %
				& %
				& %
				& %
				& \cmark %
				& %
				& \cmark %
				& \cmark %
				& \cmark %
				& %
				& %
				& %
				& \cmark %
				& %
				& %
				& \cmark %
				& %
				& %
				& \cmark %
				& \cmark %
			\\
			van Dongen\tnote{\fnvandenpoel} \cite{vanDongen2020} & \citeyear{vanDongen2020}
				& %
				& %
				& %
				& %
				& \cmark %
				& %
				& %
				& %
				& %
				& %
				& %
				& %
				& %
				& %
				& %
				& \cmark %
				& \cmark %
				& %
				& %
				& \cmark %
				& %
				& %
				& %
				& \cmark %
				& \cmark %
				& \cmark %
				& %
				& %
				& %
			\\
			\citet{Nasiri2019} & \citeyear{Nasiri2019}
				& %
				& %
				& %
				& %
				& %
				& %
				& %
				& %
				& \cmark %
				& \cmark %
				& %
				& %
				& %
				& %
				& %
				& \cmark %
				& \cmark %
				& \cmark %
				& %
				& %
				& %
				& %
				& %
				& %
				& \cmark %
				& \cmark %
				& \cmark %
				& \cmark %
				& \cmark %
			\\
			\citet{Shahverdi2019} & \citeyear{Shahverdi2019}
				& %
				& %
				& %
				& %
				& %
				& %
				& \cmark %
				& %
				& %
				& %
				& %
				& %
				& %
				& %
				& %
				& \cmark %
				& \cmark %
				& \cmark %
				& %
				& \cmark %
				& \cmark %
				& %
				& %
				& \cmark %
				& \cmark %
				& \cmark %
				& \cmark %
				& %
				& %
			\\
			\citet{Zeuch2019} & \citeyear{Zeuch2019}
				& %
				& %
				& %
				& %
				& %
				& %
				& %
				& %
				& %
				& \cmark %
				& %
				& %
				& \cmark %
				& \cmark %
				& %
				& \cmark %
				& \cmark %
				& \cmark %
				& %
				& %
				& %
				& \cmark %
				& %
				& %
				& \cmark %
				& %
				& %
				& \cmark %
				& \cmark %
			\\
			\citet{Karimov2018} & \citeyear{Karimov2018}
				& %
				& %
				& %
				& %
				& %
				& %
				& %
				& \cmark %
				& %
				& %
				& %
				& %
				& %
				& %
				& %
				& \cmark %
				& \cmark %
				& \cmark %
				& %
				& %
				& %
				& %
				& %
				& %
				& \cmark %
				& \cmark %
				& %
				& \cmark %
				& \cmark %
			\\
			\citet{Truong2018} & \citeyear{Truong2018}
				& %
				& %
				& %
				& %
				& %
				& %
				& %
				& %
				& %
				& %
				& %
				& %
				& %
				& \cmark %
				& %
				& %
				& %
				& %
				& %
				& %
				& %
				& \cmark %
				& %
				& \cmark %
				& \cmark %
				& %
				& %
				& \cmark %
				& \cmark %
			\\
			\citet{Karakaya2017} & \citeyear{Karakaya2017}
				& %
				& %
				& %
				& %
				& %
				& %
				& %
				& %
				& %
				& \cmark %
				& %
				& %
				& %
				& %
				& %
				& \cmark %
				& \cmark %
				& \cmark %
				& %
				& %
				& %
				& %
				& %
				& %
				& \cmark %
				& %
				& %
				& \cmark %
				& \cmark %
			\\
			\citet{Shukla2017} & \citeyear{Shukla2017}
				& %
				& %
				& %
				& %
				& %
				& %
				& %
				& %
				& \cmark %
				& %
				& %
				& %
				& %
				& %
				& %
				& %
				& %
				& \cmark %
				& %
				& %
				& %
				& %
				& %
				& \cmark %
				& \cmark %
				& \cmark %
				& %
				& %
				& %
			\\
			\citet{Yang2017} & \citeyear{Yang2017}
				& %
				& %
				& %
				& %
				& %
				& %
				& %
				& %
				& %
				& \cmark %
				& %
				& %
				& %
				& \cmark %
				& %
				& \cmark %
				& \cmark %
				& \cmark %
				& %
				& %
				& %
				& %
				& %
				& \cmark %
				& \cmark %
				& %
				& %
				& %
				& %
			\\
			\citet{Chintapalli2016} & \citeyear{Chintapalli2016}
				& %
				& %
				& %
				& %
				& %
				& %
				& %
				& %
				& %
				& \cmark %
				& %
				& %
				& %
				& %
				& %
				& \cmark %
				& \cmark %
				& \cmark %
				& %
				& %
				& %
				& %
				& %
				& %
				& \cmark %
				& \cmark %
				& \cmark %
				& %
				& %
			\\
			\citet{Lopez2016} & \citeyear{Lopez2016}
				& %
				& %
				& %
				& %
				& %
				& %
				& %
				& %
				& %
				& %
				& %
				& %
				& %
				& \cmark %
				& %
				& \cmark %
				& \cmark %
				& \cmark %
				& %
				& %
				& %
				& %
				& %
				& %
				& \cmark %
				& %
				& %
				& \cmark %
				& \cmark %
			\\
			\citet{Qian2016} & \citeyear{Qian2016}
				& %
				& %
				& %
				& %
				& %
				& %
				& %
				& %
				& %
				& %
				& %
				& \cmark %
				& %
				& %
				& %
				& %
				& \cmark %
				& \cmark %
				& \cmark %
				& %
				& %
				& %
				& %
				& %
				& \cmark %
				& \cmark %
				& \cmark %
				& %
				& %
			\\
			\citet{Lu2014} & \citeyear{Lu2014}
				& %
				& %
				& %
				& %
				& %
				& %
				& %
				& %
				& %
				& %
				& %
				& \cmark %
				& %
				& %
				& %
				& %
				& \cmark %
				& \cmark %
				& %
				& %
				& %
				& %
				& %
				& %
				& \cmark %
				& \cmark %
				& \cmark %
				& %
				& %
			\\
			\bottomrule
		\end{tabularx}
		\begin{tablenotes}\footnotesize
			\item[\fnvandenpoel] and van den Poel
			\item[\fnbeam] using Apache Beam
		\end{tablenotes}
	\end{threeparttable}
\end{table*}

\cref{tab:related-experiments} provides an overview of experimental performance evaluation and benchmarking studies. It indicates the applied benchmark, the evaluated stream processing, and information regarding the experiment setup and method. The latter includes whether the respective study was performed in a cloud environment, in a distributed fashion with multiple instances of the framework deployed. Moreover, it shows whether the benchmarks have been executed with different resource amounts and different load intensities and whether different resource amounts and load intensities are evaluated in isolated experiments. In previous work, we argued that scalability should be evaluated with isolated experiments for different combinations of load and resources~\cite{LTB2021,EMSE2022}.

We can observe that there is no established stream processing benchmark. Only YSB is used in several studies. However, YSB can be considered a micro-benchmark~\cite{Bermbach2017} and, hence, is less suited to benchmark entire microservices.
Except for the preliminary evaluation of our Theodolite benchmarks~\cite{BDR2021}, there is no benchmarking study addressed to stream processing frameworks employed within microservice architectures.

Flink, Spark, and Strom are by far the most frequently benchmarked frameworks. Kafka Stream, Hazelcast Jet, and Samza, which are particularly suited for implementing event-driven microservices, are only benchmarked in a few studies and there is no study benchmarking all of them.

9 out of 20 studies report on experiments in public or private clouds.
Except for this and our previous study~\cite{IC2E2022FaaSStreaming}, there are no evaluations in Kubernetes.
Likewise, there are no further studies evaluating scalability with a systematic approach as we do in this study. \citet{Vikash2020}, \citet{Nasiri2019}, \citet{Karakaya2017}, and \citet{vanDongen2021b} explicitly evaluate scalability, however, without testing different load intensities against different resource amounts in isolated experiments. \citet{Nasiri2019} conduct independent evaluations of scaling load and computing resources and, thus, address another aspect than our study.
Our previous study~\cite{IC2E2022FaaSStreaming} applies our Theodolite method as well, but benchmarks scalability with respect to costs and is addressed to comparing stream processing deployments against Function-as-a-Service offerings.




\section{Methods}
\section{PoseRAC Model}
\label{sec4}

\begin{figure*}[t]
\centering
\includegraphics[width=1.0\textwidth]{figure5.pdf}
\caption{Overview of our proposed PoseRAC. For a input video, the repetitive count can be obtained through Pose Estimation, Transformer Encoder, Pose Mapping and Action-trigger, where only the Encoder and the Pose Mapping need to be trained. We use Triplet Margin Loss to train the Encoder while Binary Cross Entropy Loss to train both the Encoder and the Pose Mapping. In addition to achieving the state-of-the-art performance so far, the biggest highlight of our PoseRAC is that it is lightweight enough to be easily trained on a CPU.}
\label{fig5}
\end{figure*}

Given a video $V={\{x_i\}}^{T}_{1}\in \mathbb{R}^{C\times H\times W\times T}$ with $T$ RGB frames, repetitive action counting model aims to predict a certain value $Y$, which is the number of repetitive actions. In this section, we will introduce our PoseRAC in detail.

\subsection{Model Overview}

As shown in Figure \ref{fig5}, PoseRAC consists of four parts. 

\begin{itemize}

\item The first is a state-of-the-art and lightweight Pose Estimation Network~($\S\ref{first}$), which is used to estimate the poses represented by lots of human pose key points from each frame of the original video sequence. 

\item The second is a simple Transformer Encoder~($\S\ref{second}$) to embed the key points of poses into high-level feature space, where the same class have similar distances, while the distances of different classes are far apart.

\item The third is a Pose Mapping Module~($\S\ref{third}$), where the unique mapping relationship between the salient poses and the action classes can be learned. Each pose can be mapped to the action class with the highest probability after the previous encoding.

\item The fourth part is a lightweight Action-trigger Module~($\S\ref{fourth}$). When we get the salient action classification results of all frames of the entire video sequence, we can use this module to calculate the repetition count in a short time.

\end{itemize}

\subsection{Pose Estimation Network}
\label{first}
Our model first converts the video sequence into a sequence of human pose key points, which can be defined as: 
\begin{equation}
\begin{split}
&V={\{x_i\}}^{T}_{1}\in \mathbb{R}^{C\times H\times W\times T}\\
&V\xrightarrow{\mathrm{Pose Estimation}} P={\{p_i\}}^{T}_{1}\in \mathbb{R}^{D\times K\times T}
\end{split}
\label{eq1}
\end{equation}
where each $x_i$ represents a single RGB frame, and each $p_i$ represents the key points of each frame. To express the key points of each frame, we use $D\times K$ sequence, which includes two parts, one ($K$) is the number of key points to fully represent the current pose, the other ($D$) is the dimension of each key point, generally three, which are the two coordinates of the planes and the depth estimation.

Here we use state-of-the-art pose estimation models such as Vitpose\cite{xu2022vitpose} and BlazePose\cite{bazarevsky2020blazepose}. The pose estimation algorithms themselves are not designed by us, but we introduce pose information into the action counting task, which is a novel design not explored by previous work.

Moreover, our pose-level poses estimation processes the primitive information of video, which is similar to the feature extraction network in all video-level algorithms such as I3D\cite{carreira2017quo}, VideoSwinTransformer\cite{liu2022video}, and TSN\cite{wang2016temporal}. But the difference is that the result of video-level incorporates all information, while pose-level only produces core information, which greatly improves the performance. Additionally, using pose information can contribute to the lightweight of model. For instance, for a 1024-frame video, video-level feature extraction with an output dimension of 512 would produce a data volume of $1024\times 512=524288$, while using pose information with 33 key points produces a data volume of only $1024\times 33 \times 3=101376$.

\subsection{Encoding Poses with Transformer}
\label{second}
Here we specify our data representation for the Transformer Encoder, which requires input batch size, sequence length, and embedding dimensions. In our pose-level approach, each frame is a batch, the number of key points in each frame is the sequence length, and the feature dimension of each key point is the embedding dimension.

First we get the pose of each frame ${p_i}\in \mathbb{R}^{D\times K}$ through the Pose Estimation Network, where $i\in {1, 2, \dots, T}$ is the frame index, $K$ is the number of key points, and $D$ is the dimension of each key point. We further define $p_i = {\{k_j\}}^{K}_{1}$ to represent each key point, where $k_j\in \mathbb{R}^D$, and we embed it to obtain richer information. Our embedding projection $\mathrm{\bf{E}}$ is a simple MLP network with ReLU as the activation function. These calculations can be defined as:
\begin{equation}
\begin{split}
\mathrm{\bf{Z}}^0 = [\mathrm{\bf{E}}(k_1), \mathrm{\bf{E}}(k_2), \dots, \mathrm{\bf{E}}(k_K)]^T
\end{split}
\end{equation}
where $\mathrm{\bf{E}}(k_j)\in \mathbb{R}^{D^{\prime}}$ is the embedding feature. Then the next Transformer takes $\mathrm{\bf{Z}}^0$ as input and encodes it with self-attention. Given $\mathrm{\bf{Z}}^0\in \mathbb{R}^{K\times D^{\prime}}$ with $K$ key point features, each of which is $D^{\prime}$-dimensional, $\mathrm{\bf{Z}}^0$ is projected using $\mathrm{\bf{W}}_Q\in \mathbb{R}^{D^{\prime}\times D_q}$, $\mathrm{\bf{W}}_K\in \mathbb{R}^{D^{\prime}\times D_k}$, $\mathrm{\bf{W}}_V\in \mathbb{R}^{D^{\prime}\times D_v}$, where $D_k=D_q$, to extract feature representations query($\mathrm{\bf{Q}}$), key($\mathrm{\bf{K}}$) and value($\mathrm{\bf{V}}$), which can be defined as:
\begin{equation}
\begin{split}
&\mathrm{\bf{Q}}=\mathrm{\bf{Z}}^0\times \mathrm{\bf{W}}_Q\\
&\mathrm{\bf{K}}=\mathrm{\bf{Z}}^0\times \mathrm{\bf{W}}_K\\
&\mathrm{\bf{V}}=\mathrm{\bf{Z}}^0\times \mathrm{\bf{W}}_V
\end{split}
\end{equation}
and the output of self-attention can be computed as:
\begin{equation}
\begin{split}
\mathrm{\bf{Attn}}=\mathrm{Softmax}(\frac{\mathrm{\bf{Q}}\mathrm{\bf{K}}^T}{\sqrt{D_q}})\mathrm{\bf{V}}
\end{split}
\end{equation}
where $\mathrm{\bf{Attn}}\in \mathbb{R}^{K\times D^{\prime}}$. Also, we use common multi-head self-attention (MHSA) to make several self-attention operations calculate in parallel.

Now we introduce the overall architecture of Transformer Encoder, which has $L$ layers with each layer consisting of MHSA and MLP blocks. Also, LayerNorm and Residual Connection are applied before and after every MHSA or MLP block, respectively. Because the number of key points of each frame is  a bit less, so our encoder does not include the downsampling module that other models may have. The overall process can be defined as:
\begin{equation}
\begin{split}
&\mathrm{\bf{\hat{Z}}}^l = \mathrm{MHSA}(\mathrm{LN}(\mathrm{\bf{Z}}^{l-1})) + \mathrm{\bf{Z}}^{l-1}\\
&\mathrm{\bf{Z}}^l = \mathrm{MLP}(\mathrm{LN}(\mathrm{\bf{\hat{Z}}}^l)) + \mathrm{\bf{\hat{Z}}}^l
\end{split}
\end{equation}
where $\mathrm{\bf{Z}}^{l-1}$, $\mathrm{\bf{\hat{Z}}}^l$, $\mathrm{\bf{Z}}^l\in \mathbb{R}^{K\times D^{\prime}}$.


\subsection{Pose Mapping}
\label{third}
Taking the Encoder output $\mathrm{\bf{Z}}^L\in \mathbb{R}^{K\times D^{\prime}}$ as input, Pose Mapping module outputs probability scores $\mathrm{\bf{S}}\in \mathbb{R}^{C}$ of the current frame over all action classes. We perform binary classification after Sigmoid activation for each class, with the two salient poses of each class represented by the same bit data. To realize such a module, we use a very lightweight MLP network, which avoids the complexity. First, the two dimensions $K$ and $D^{\prime}$ of $\mathrm{\bf{Z}}^L$ are flattened into $\mathbb{R}^{KD^{\prime}}$, and then it passes through an MLP module, where the output channels is set to $C$, which can be defined as:
\begin{equation}
\begin{split}
\mathrm{\bf{S}} = \sigma(\mathrm{MLP}(\mathrm{Flatten}(\mathrm{\bf{Z}}^L)))
\end{split}
\end{equation}
where $\sigma$ represents the Sigmoid activation function.

With such Pose Mapping, we can obtain the scores of single frame. It should be noted that we extract the poses of all frames, and use the convenience of matrix operations to obtain scores in parallel, which is actually consistent with the idea of mini batch. So at last, we combine the scores of all frames to get the video score matrix $\mathrm{\bf{\hat{S}}}\in \mathbb{R}^{C\times T}$, where $T$ represents the number of frames in the current video. 


\subsection{Action-trigger Module}
\label{fourth}
We use the lightweight Action-trigger Module to obtain the final output $Y$, the repetitive action count, which has a time complexity of $\mathcal{O}(n)$. First, we get the scores $S_c\in \mathbb{R}^T$ of a given action class from $\mathrm{\bf{\hat{S}}}$. Then, we scan all frames and use the action-trigger mechanism to count when the two salient poses of the action class occur sequentially. We set upper and lower bounds to distinguish the scores of the two salient poses, which cluster non-salient poses in the middle and easily classify the salient poses to the two ends.

\subsection{Losses and Metric Learning}

The modules need to be trained are Embedding, Transformer Encoder and Pose Mapping, and because we perform binary classification for each class, so we use the Binary Cross Entropy Loss, which can be defined as follows:
\begin{gather}
\mathcal{L}_{bce} = -\frac{1}{N}\sum\limits_{i=1}^{N}(\frac{1}{C}\sum\limits_{j=1}^{C}loss(i,j))  \\
 loss(i,j)=y_{ij}\log p_{ij} + (1-y_{ij})\log(1-p_{ij})
\end{gather}
where $N$ represents the batch size (in our method, each frame is a batch), $C$ represents the number of classes, $y$ and $p$ are the labels and our predictions, respectively.

Moreover, we use Metric Learning to improve our Encoder and introduce the Pose Triplet Loss. Given a pose, Encoder produces higher-level features $\mathrm{\bf{Z}}^L$, which should be more representative. As shown in Figure \ref{fig5}, we achieve this with Triplet Margin Loss function, which selects anchors, same class positive samples, and different classes negative samples in a batch. It can be expressed as:
\begin{equation}
\begin{split}
\mathcal{L}_{tri} = \mathrm{max}(\mathrm{CS}(a,p)-\mathrm{CS}(a,n)+\mathrm{margin},0)
\end{split}
\end{equation}
where $a$, $p$, $d$ are anchors, positive and negative samples, and $\mathrm{CS}$ represents the Cosine Similarity to measure the distance between features. We pay more attention to hard samples, where the distances between anchors and negative samples are even smaller than those of positive samples. After Metric Learning, the poses of each action can be distinguishable, which cluster in the high-level space.

At last, our overall training combines these two losses:
\begin{equation}
\begin{split}
\mathcal{L} = \mathcal{L}_{bce} + \alpha\mathcal{L}_{tri}
\end{split}
\end{equation}
where $\alpha$ is the weight factor to control the two losses in the same numeric scale.
\subsection{Implementation Details}

\noindent{\bf Training.} We use the \emph{RepCount-pose} and \emph{UCFRep-pose} dataset we created to train our model. Only the frames with salient poses are inputted into the network instead of the entire video to speed up the fitting.

\noindent{\bf Inference.} During inference, the entire video sequence is inputted into the model. The poses of all frames pass through the Encoder and Pose Mapping, and then enter the Action-trigger Module to output the repetitive count.

\section{Results}
\section{Results}
\label{results}

\begin{figure*}[ht]
    \centering
    \includegraphics[scale=0.15,trim={0 2.5cm 0 5cm},clip]{images/aoi-single_burst}
    \caption{The time average peak Age of Information with burst and \gls{soa} loss values against the dynamic reliability logic for different network topologies.}
    \label{fig:aoi_burst}\vspace{-0.4cm}
\end{figure*}


This paper focuses on both transport layer and application layer metrics to determine the feasibility of dynamic reliability. For this, we have selected the session packet volume, as transmitted, retransmitted, lost and backlogged packets as \glspl{kpi} for the transport layer; while focusing on the \gls{aoi} for the application layer. The \gls{aoi} was chosen as a crucial indicator for the freshness of packets in real-time applications. More specifically, this work adopts the time average peak \gls{aoi} equation \cite{aoi_equation} depicted in Eq. \ref{aoi}, where $\Delta(r_{i+1})$ is the $i$th update at the time it was received at the server, for a session time period of $\tau$.

\begin{equation}
    \label{aoi}
    \gls{aoi}_\tau = \frac{1}{n-1}\sum_{i=1}^{n-1} \Delta(r_{i+1})
\end{equation}

We include a comparison between the vanilla QUIC implementation which does not enjoy the dynamic reliability extension, with a number of dynamic reliability policies. The tests were run a number of times for statistical significance, with the mean value of vanilla implementation used as a baseline for comparison. The topology utilised both random loss and bursty loss to explore the bounds of dynamic reliability. The \gls{soa} loss in the figures correspond to the loss values presented in Table. \ref{tab:path_char}, for ease of comparison between bursty and random loss scenarios.

\subsection{Transport-Layer KPIs}

To analyse the performance gain at the transport layer due to dynamic reliability, the volume of transmitted and backlogged packets is examined. The figures are in the form of boxplots, which take the vanilla implementation as a benchmark, depicted as the red dashed line.

As seen in Fig. \ref{fig:sent_burst}, the loss plays a crucial role in the performance of the reliability policies. The policies under random loss did incredibly well for the networks with a larger capacity, namely \gls{mmwave} and Sub-6~GHz, whereas for burst loss, the lower network capacities had a larger packet reduction. With the increase in burst loss, the behaviour of the set split reliable policies became unpredictable, if a reliable assignment happened to coincide with a burst loss, the number of transmitted packets increases, and vice versa. On the other hand, in smarter policies, such as Loss-Aware, the performance lightly matched the vanilla baseline, as the reliable assignment dominated the session to compensate for a higher burst loss. Not only that but, the burst loss also impacted the variance of the transmitted packets for the policies.

Unsurprisingly, the unreliable focused policy, 80-20 split, outperformed other policies for all topologies in random and bursty loss scenarios, with an approximate reduction of 80\%. That being said, the majority of the policies reduced the transmitted packets on the link by approximately 70\% for random loss, while the reduction started at $\approx 15\%$ and decreased as the loss increased for the burst loss scenario.

The retransmitted and lost packets, not shown due to space limitations, followed the same trend as the transmitted packets for the random loss scenarios. However, for the burst loss scenarios, the larger capacity networks had a lower reduction in the retransmitted and lost packets. This can be seen as a favorable outcome since the lower capacity networks are scarce on resources. It is important to note that the Loss-Aware policy mimicked the vanilla approach as the burst loss increased, signifying the overwhelming appointment of reliable packets in adapting to the harsh burst loss conditions.
 
Alternatively, Fig. \ref{fig:backlog_burst} clearly shows a stark comparison between the policies and loss scenario in the reduction of the backlogged packets. The Loss-Aware policy for random loss scenario reduced the backlogged packets by up to 50\%, beating all other policies by approximately 30\%. Furthermore, it is clear that the unreliability focused policies resulted in the lowest backlog for the session. In comparison, we notice that the burst loss and the backlogged frequency have a positive correlation, where the maximum reduction of the backlogged packets for the policies is at most 20\%. Much like the transmitted packets, the probability of a burst loss occurrence plays a vital role in the number of retransmissions sent and by extension the number of backlogged packets. Thus, we can conclude that the stress placed on the buffer is a result of the reliable packets which is tightly coupled with the congestion on the session. Whereas, unreliable focused policies did not encounter such a phenomenon regardless if it was experiencing a burst loss.


\subsection{Application-Layer KPIs}

The feasibility of dynamic reliability for real-time applications can be determined by the \gls{aoi}, with comparison across different topologies and policies. If we take a strict approach and consider anything below $10$~ms is real-time \cite{real-time}, then all the reliability policies passed that requirement, which is attractive for real-time applications, as shown in Fig. \ref{fig:aoi_burst}. Utilising the median as an estimate of the runs, the policies in the WLAN and Sub-6~GHz topology with random loss floated around $4-5$~ms with negligible difference, while the \gls{aoi} for \gls{mmwave} was $\approx 2-3$~ms. It is clear that the \gls{aoi} and the network capacity have a negative correlation, as the network capacity decreases, the \gls{aoi} increases. The same correlation is extended to the bursty loss scenarios, where \gls{mmwave} dominated the other topologies. That being said, it is crucial to note that the \gls{aoi} for the reliability policies is often slightly better than or equal to the \gls{aoi} of the vanilla implementation, proving that dynamic reliability reduces the congestion of the session at no cost to the \gls{aoi}.


\section{Discussion}
We provide some comments on the growth conditions which constituted the majority of our analysis in sections \ref{sec:Hmixing} and \ref{sec:Hsigma}. In the simplest cases of Lemma \ref{lemma:unstableGrowth}, growth was established in an analogous fashion to the old one-step expansion condition (\ref{eq:oldOneStepExpansion}), finding the relevant Jacobians $M_j$ and checking that their expansion factors $K(M_j)$ satisfy
\begin{equation}
    \label{eq:discussionOneStep}
    \sum_j \frac{1}{K(M_j)} <1.
\end{equation}
For the more complicated cases, the inductive method used to establish growth near the accumulation points in Lemma \ref{lemma:unstableGrowth} and the weakened one-step expansion condition (\ref{eq:oneStep}) both address the same fundamental issue: the splitting of unstable curves by singularities into an unbounded number of small components. They circumvent this obstacle in rather different ways, however. While (\ref{eq:oneStep}) generalises (\ref{eq:discussionOneStep}) to ensure an growth of unstable curves `on average' (see \cite{chernov_statistical_2009} for a precise statement), our inductive method is a more direct adaptation of (\ref{eq:discussionOneStep}), using it to generate contradictory geometric conditions which a hypothetical non-growing unstable curve must satisfy. It may be possible to prove Theorem \ref{sec:Hmixing} using (\ref{eq:oneStep}) as the basis for growth. Since we required (\ref{eq:oneStep}) anyway for proving Theorem \ref{thm:HsigmaExp}, this could potentially condense our analysis, but only to a minor extent. A convenience of the method used in section \ref{sec:Hmixing} is that, by way of the `simple intersection' property, it naturally gives geometric information on the images of manifolds, useful for proving the property \textbf{(M)} of Theorem \ref{thm:katok-strelcyn}.

We expect that essentially analogous analysis can be applied to establish mixing properties in a wide class of piecewise linear non-uniformly hyperbolic maps, including those (like the OTM) which sit on the boundary of ergodicity and beyond. While we have relied on the precise partition structure of $H_\sigma$, its fundamental feature (self-similar sequences of elements $A^k$, sharing boundaries with its neighbours $A^{k-1},A^{k+1}$ and accumulating onto some point $p$) is quite typical to return map systems. See, for example, those of various stadium billiards \cite{chernov_chaotic_2006,chernov_improved_2008,chernov_statistical_2009} and LTMs \cite{springham_polynomial_2014}. Indeed, the same method can be used to prove the Bernoulli property for non-monotonic LTMs \cite{myers_hill_mixing_2022}, where monotonicity of the manifold images cannot be assumed and the classical argument \cite{sturman_mathematical_2006} fails. The OTM is the pointwise limit of these maps as the boundary shrinks to null measure. It further has utility in proving growth conditions for maps which are uniformly hyperbolic but possess regions $A_j$ where the hyperbolicity is very weak, signified by $K(M_j) \approx 1$, so that (\ref{eq:discussionOneStep}) fails. Typically this leads to suboptimal bounds on mixing windows, see e.g. \cite{wojtkowski_model_1981,przytycki_ergodicity_1983,myers_hill_family_2022}. The map $H_{(\eta,\eta)}$ for $\eta \approx 1/2$ is another example, possessing weak hyperbolicity over $A_2, A_3$. Letting $\varepsilon = |\eta-1/2|>0$, there is an upper bound $N = N(\varepsilon)$ on escape times from the intersections $A_2\cap \sigma, A_3 \cap \sigma$. The growth lemma then follows by applying the inductive step roughly $N$ times and can be established for arbitrarily small $\varepsilon$, opening the door to establishing optimal mixing windows.

The above gives two examples of piecewise linear perturbations to $H$ where mixing with respect to Lebesgue is preserved and our methods can be applied. Nonlinear perturbations to the shear profiles complicate the analysis in several ways. Firstly as the map's Jacobians takes on a broader range of values, cone invariance becomes an increasingly harder condition to establish. Cones must be widened, giving looser bounds on expansion factors, which may already be weak due to new regions of weaker stretching. This, together with the change from polygonal to curvilinear return time partition elements and nonlinear local manifolds, adds some complexity to showing growth conditions. This does not rule out certain (small) nonlinear perturbations however. There is some leeway in the inequalities which govern cone invariance and growth of local manifolds, the latter of which is not too dissimilar from the piecewise linear setting (see Lemmas \ref{lemma:piecewiseApprox}, \ref{lemma:componentLength}). Certain small perturbations would not alter the \emph{topological} structure of the return time partition, i.e. which elements share boundaries, the key information needed for setting up the induction. Finally while the partition elements would no longer be polygonal, only coarse geometric information is required for verifying each inductive step. Following the above, a potential perturbation could be to replace the linear portions of each shear by a cubic, perturbing the tent profile
\[  f(t) = \begin{cases} 2t & 0 \leq t \leq 1/2, \\ 2(1-t) & 1/2 \leq t \leq 1 ,\end{cases} \]
of the OTM shears to
\[  f_a(t) = \begin{cases} \frac{1}{8} t \left(16 - a + 6at - 8at^{2} \right) & 0 \leq t \leq 1/2, \\ \frac{1}{8}\left(1-t\right)\left( 16 - a + 6a\left(1-t\right) - 8a\left(1-t\right)^{2}\right)  & 1/2 \leq t \leq 1, \end{cases}   \]
for $a>0$. For small enough $a$ the gradient range $f'(t)$ is restricted to small neighbourhoods of $\{ 2, -2\}$ and the escape time partition retains a similar structure. We illustrate this in Figure \ref{fig:perturbations}, showing escapes from the square $S_3$ under the map $G \circ F$, equivalent to escapes from the perturbed $A_3$ under the $G \circ F$, but with a cleaner geometry for comparison. When $a$ is too large the analogy to the OTM breaks down. At $a=16$ the map is twice differentiable everywhere and features a new source of slowed mixing, the Jacobian is the identity at the corner points $x,y \in \{  0, 1/2 \}$ giving locally parabolic behaviour (visible in the escape time partition). 

\begin{figure}
    \centering
    \includegraphics[width=0.24 \linewidth]{0.png}
    \includegraphics[width=0.24 \linewidth]{4.png}
    \includegraphics[width=0.24 \linewidth]{8.png}
    \includegraphics[width=0.24 \linewidth]{16.png}
    \caption{Partition of escape times from $S_3$ under the mapping $F \circ G$ for $a= 0,4,8,16$. }
    \label{fig:perturbations}
\end{figure}

\section{Limitations and Future Work}
This study was primarily quantitative, so we could not capture nuanced interactions with specific elements of the provenance UI or account for different interpretations of survey questions, which may have led to inconsistent definitions of \textsc{trust} and \textsc{perceived accuracy}. A complementary qualitative study with participants completing tasks on a provenance-enabled prototype and being interviewed about their experiences would provide more insights. \add{Our deceptive media were also modified through composition, so we did not observe participants use other means of editing to reason about credibility. We anticipate generative AI to be an increasingly popular method of media creation and manipulation on social media in the coming years due to the public availability of text-to-image models, such as DALL-E \cite{ramesh2021dalle} and Stable Diffusion \cite{rombach2022diffusion}. We are thus excited about the potential of provenance standards to supplement budding frameworks in this space \cite{pai-framework} and provide much-needed transparency and disclosures to viewers of AI-generated content.} 

\addd{Our study only included one representation of provenance---the C2PA standard. Thus, it also included C2PA-specific metaphors and design patterns (e.g., cryptographically signed chains), which may not be available in other provenance standards, such as Arweave \cite{arweave} and Four Corners \cite{fourcorners}. Our findings were therefore unable to speak for all provenance standards. Future work may include a broader range of provenance standards to make a more general claim about how users may operationalize the \textit{conceptual knowledge} derived from general provenance exposure---rather than UI-specific signals---to evaluate media.} Additionally, as mentioned in Section \ref{s:design-implications}, we had relatively simple provenance details panels in this study, which may not accurately reflect real-world UIs once the system is deployed. Future work, perhaps of interest to the computer vision community, can study more complex provenance details panels and develop techniques to highlight edits with more visual semantic relevance to users. Our study also used a social media prototype resembling Twitter. Variation in participants' past experiences with and general perceptions of the platform have affected our results in ways we did not account for, and we hope future work may expand similar studies to other platforms.

We measured participants' media credibility perceptions through two quantitative variables---trustworthiness and perceived accuracy of claim---but there are many other approaches that could also be used, as outlined in Section \ref{s:rw-measures}. Our content was also limited to 9 pieces of media per feed to keep the study completion time reasonable and participant fatigue low. This constrained the combination of levels within \textsc{media topic}, \textsc{edit status}, \textsc{claim agreement}, and \textsc{provenance state} we could have for each piece of media. While we managed to include at least one piece of media in each of those level combinations, we did not have any composited news media, nor did we have any composited invalid media. Our media also had concrete ``ground truth'' claims from Snopes, but other content may not have an easily identifiable claim, or have a claim that can be easily agreed upon. Future work can explore impacts of provenance on media with uncertain or nonexistent claims. 

 %Finally, this study assumes unfamiliarity with provenance UIs and simulates users' first exposure to them, which may not reflect perceptions as users become accommodated to interacting with the UIs over time. We did not investigate longitudinal effects of provenance, nor any byproducts of provenance with a temporal component. For example, one of the most popular sites of collective sensemaking on social media is the comments section. Widely accessible provenance information may generate new discussions about media in the comments of the media's post that may influence credibility perceptions more powerfully than the provenance information itself. Discussions about provenance may also alter incentives for content creators to publish on provenance-enabled platforms: some may see it as a tool for mitigating misinformation and be more comfortable posting into a reliable content community, while others may fear being on the receiving end of false accusations and harassment. We see this as a rich avenue of future work.
Finally, this study assumed unfamiliarity with provenance UIs and simulated users' first exposure to them, but did not investigate longitudinal effects or byproducts with a temporal component, such as discussions in comments sections or incentives for content creators. We see these as rich avenues for future work.



\section{Conclusion}
\section{Conclusion}\label{sec:conclusion}
In this work, we focus on addressing the fundamental challenge of OOD detection tasks, which is how to fully understand the semantic discrepancy between the ID/OOD samples. We reveal that the key to success in the realistic SCOOD task is to allocate as many ID samples in the unlabeled set correctly as possible. To this end, we propose a novel uncertainty-aware optimal transport scheme that introduces class-specific energy scores as guidance for effective label assignment. Experimental results show that our method achieves better performance than previous state-of-the-art methods on SCOOD benchmarks.

\textbf{Limitations.} In addition to temperature scaling, other techniques such as feature clipping applied in ReAct~\cite{sun2021react} also enhance the performance of energy score, so how to obtain an OOD score that best fits the SCOOD task can be further explored. Moreover, a setting highly related to SCOOD has been proposed in \cite{katz2022training} and formulated as a constrained optimization problem. We will also theoretically analyze these practical OOD settings in our feature work.

% \section*{Acknowledgments}
\textbf{Acknowledgments.} 
This work is supported by National Key R\&D Program of China under Grant 2020AAA0105701, National Natural Science Foundation of China (NSFC) under Grants 61872327, Major Special Science and Technology Project of Anhui, National Natural Science Foundation of China (62033012) and Ant Group through Ant Research Intern Program.



%%
%% The acknowledgments section is defined using the "acks" environment
%% (and NOT an unnumbered section). This ensures the proper
%% identification of the section in the article metadata, and the
%% consistent spelling of the heading.

\begin{acks}
We warmly thank all our participants and reviewers. We also extend a special thanks to Dave Kozma for engineering assistance on the prototype, as well as members of the C2PA Technical Working Group, C2PA UX Taskforce, and the Social Futures Lab for discussion and feedback. This work was supported by a research grant from the C2PA.
\end{acks}

%%
%% The next two lines define the bibliography style to be used, and
%% the bibliography file.
\bibliographystyle{ACM-Reference-Format}
\bibliography{refs}

%%
%% If your work has an appendix, this is the place to put it.
\newpage

\appendix
\section{Appendix for Proofs}

\paragraph{Proof of Theorem \ref{thm:main}.}

\begin{proof}
\label{proof:main}
Our proof has two steps. In Step 1, we will show that SimCLR is equivalent to minimizing the cross entropy loss defined in Eqn.~(\ref{eqn:cross-entropy}). 
In Step 2, we will show  that minimizing the cross-entropy loss 
is equivalent to spectral clustering on $\bfpi$. 
Combining the two steps together, we have proved our theorem. 

\textbf{Step 1: } SimCLR is equivalent to minimizing the cross entropy loss.

The cross-entropy loss takes expectation over 
$\bfW_\bfX\sim \mathbb{P}(\cdot ; \bfpi)$, 
which means $\bfW_\bfX$ has exactly one non-zero entry in each row $i$. By Lemma~\ref{lem:multinomial}, we know every row $i$ of $\bfW_\bfX$ is independent of other rows. Moreover, 
$\bfW_{\bfX,i}\sim \mathcal{M}(1, \bfpi_i/\sum_j \bfpi_{i,j})=\mathcal{M}(1, \bfpi_i)$, because $\bfpi_i$ itself is a probability distribution.
Similarly, we know $\bfW_\bfZ$ also has the row-independent property by sampling over $\mathbb{P}(\cdot;\bfK_\bfZ)$.
Therefore, by Lemma~\ref{lem:cross_split}, we know Eqn.~(\ref{eqn:cross-entropy}) is equivalent to:
\[
 -\sum_{i=1}^n \mathbb{E}_{\bfW_{\bfX,i}}[\log \mathbb{P}(\bfW_{\bfZ,i}=\bfW_{\bfX,i};\bfK_\bfZ)],
\]

This expression takes expectation over $\bfW_{\bfX,i}$ for the given row $i$. Notice that 
$\bfW_{\bfX,i}$ has exactly one non-zero entry, which equals $1$ (same for $\bfW_{\bfZ,i}$). 
As a result
we expand the above expression to be:
\begin{equation}
 -\sum_{i=1}^n \sum_{j\neq i} \Pr(\bfW_{\bfX,i,j}=1)\log \Pr(\bfW_{\bfZ,i,j}=1).
\label{eqn:detailed-expansion}    
\end{equation}


By Lemma~\ref{lem:multinomial}, $\Pr(\bfW_{\bfZ,i,j}=1)=\bfK_{\bfZ,i,j}/\|\bfK_{\bfZ,i}\|_1$ for $j\neq i$. Recall that $\bfK_\bfZ=(k(\bfZ_i-\bfZ_j))_{(i,j)\in[n]^2}$, which means 
$\bfK_{\bfZ,i,j}/\|\bfK_{\bfZ,i}\|_1=\frac{\exp(-\|\bfZ_i-\bfZ_j\|^2/{2\tau})}{\sum_{k\neq i}
\exp(-\|\bfZ_i-\bfZ_k\|^2/{2\tau})
}$ for $j\neq i$, when $k$ is the Gaussian kernel with variance $\tau$. 

Notice that $\bfZ_i=f(\bfX_i)$, so we know
\begin{equation}
-\log \Pr(\bfW_{\bfZ,i,j}=1)=
-\log \frac{\exp(-\|f(\bfX_i)-f(\bfX_j)\|^2/{2\tau})}{\sum_{k\neq i}
\exp(-\|f(\bfX_i)-f(\bfX_k)\|^2/{2\tau}),
}
\label{eqn:infonce-equivalence}    
\end{equation}


The right hand side is exactly the InfoNCE loss defined in Eqn.~(\ref{eqn:infonce}).
Inserting Eqn.~(\ref{eqn:infonce-equivalence}) into Eqn.~(\ref{eqn:detailed-expansion}), we get the SimCLR algorithm, which first samples augmentation pairs $(i,j)$ with $\Pr(\bfW_{\bfX,i,j}=1)$ for each row $i$, and then optimize the InfoNCE loss. 

\textbf{Step 2: } minimizing the cross entropy loss 
is equivalent to spectral clustering on $\bfpi$.


By Lemma~\ref{lem:convert_to_spectral}, we may further convert the loss to 
\begin{equation}
\label{eqn:main-theorem-repul-attr}
\min_{\bfZ}
-\sum_{(i,j)\in [n]^2} \mathbf{P}_{i,j}
\log k (\bfZ_i-\bfZ_j)+\log \mathbf{R}(\bfZ).
\end{equation}
Since $k$ is the Gaussian kernel, this reduces to \[
\min_\bfZ \mathrm{tr}(\bfZ^\top \mathbf{L}(\bfpi) \bfZ)
+\log \mathbf{R}(\bfZ),
\]

where we use the fact that $\mathbb{E}_{\bfW_\bfX\sim \mathbb{P}(\cdot; \bfpi)}[\mathbf{L}(\bfW_\bfX)]
=\mathbf{L}(\bfpi)
$, because the Laplacian operator is linear and $
\mathbb{E}_{\bfW_\bfX\sim \mathbb{P}(\cdot; \bfpi)}(\bfW_\bfX)=\bfpi
$.
\end{proof}

\paragraph{Proof of Theorem \ref{thm:clip}.}
\begin{proof}
Since $\bfW_\bfX\sim \mathbb{P}(\cdot;\bfpi_{\mathbf{A}, \mathbf{B}})$, we know 
$\bfW_\bfX$ has exactly one non-zero entry in each row, denoting the pair that got sampled. 
A notable difference compared to the previous proof is we now have $n_\mathcal{A}+n_\mathcal{B}$ objects in our graph. CLIP deals with this by taking a mini-batch of size $2N$, 
such that $n_\mathcal{A}=n_\mathcal{B}=N$, and adding the $2N$ InfoNCE losses together. We label the objects in $\mathcal{A}$ as $[n_\mathcal{A}]$, and the objects in $\mathcal{B}$ as $\{n_\mathcal{A}+1, \cdots, n_\mathcal{A}+n_\mathcal{B}\}$. 

Notice that $\bfpi_{\mathbf{A}, \mathbf{B}}$ is a bipartite graph, so the edges of objects in $\mathcal{A}$ will only connect to object in $\mathcal{B}$ and vice versa. We can define the similarity matrix in $\cZ$ as $\bfK_\bfZ$, 
where $\bfK_\bfZ(i, j+n_\mathcal{A})=\bfK_\bfZ(j+n_\mathcal{A},i)= k(\bfZ_i-\bfZ_j)$ for $i\in [n_\mathcal{A}], j\in [n_\mathcal{B}]$, and otherwise we set $\bfK_\bfZ(i,j)=0$. 
The rest is same as the previous proof. 
\end{proof}

\paragraph{Proof of Theorem \ref{thm:exponential}.}

\begin{proof}
\label{proof:exponential}
Since the objective function consists of a linear term combined with an entropy regularization, which is a strongly concave function, the maximization problem is a convex optimization problem. Owing to the implicit constraints provided by the entropy function, the problem is equivalent to having only the equality constraint. We then introduce the Lagrangian multiplier $\lambda$ and obtain the following relaxed problem:

$$
\widetilde{E}(\boldsymbol{\alpha})=\psi_{1}-\sum_{i=1}^n \alpha_{i} \psi_{i}+\tau \sum_{i=1}^n \alpha_{i}\log \alpha_{i}+\lambda\left(\boldsymbol{\alpha}^{\top} \mathbf{1}_n-1\right).
$$

As the relaxed problem is unconstrained, taking the derivative with respect to $\alpha_{i}$ yields

$$
\frac{\partial \widetilde{E}(\boldsymbol{\alpha})}{\partial \alpha_{i}}=-\psi_{i}+\tau\left(\log \alpha_{i}+\alpha_{i} \frac{1}{\alpha_{i}}\right)+\lambda=0.
$$

Solving the above equation implies that $\alpha_{i}$ takes the form
$
\alpha_{i}=\exp \left(\frac{1}{\tau} \psi_{i}\right) \exp \left(\frac{-\lambda}{\tau}-1\right).
$ Since $\alpha_{i}$ lies on the probability simplex, the optimal $\alpha_{i}$ is explicitly given by
$
\alpha^{*}_{i}=\frac{\exp \left(\frac{1}{\tau} \psi_{i}\right)}{\sum_{i^{\prime}=1}^n \exp \left(\frac{1}{\tau} \psi_{i^{\prime}}\right)} .
$ Substituting the optimal point into the objective function, we obtain
$$
\begin{aligned}
E\left(\boldsymbol{\alpha}^*\right)  &=\psi_1-\sum_{i=1}^n \frac{\exp \left(\frac{1}{\tau} \psi_{i}\right)}{\sum_{i^{\prime}=1}^n \exp \left(\frac{1}{\tau} \psi_{i^{\prime}}\right)} \psi_{i}+\tau \sum_{i=1}^n \frac{\exp \left(\frac{1}{\tau} \psi_{i}\right)}{\sum_{i^{\prime}=1}^n \exp \left(\frac{1}{\tau} \psi_{i^{\prime}}\right)}\log \frac{\exp \left(\frac{1}{\tau} \psi_{i}\right)}{\sum_{i^{\prime}=1}^n \exp \left(\frac{1}{\tau} \psi_{i^{\prime}}\right)} \\
& =\psi_1 - \tau \log \left(\sum_{i=1}^n \exp \left(\frac{1}{\tau} \psi_{i}\right)\right).
\end{aligned}
$$
Thus, the Lagrangian dual function is given by
\begin{equation*}
-E\left(\boldsymbol{\alpha}^*\right)= -\tau \log \frac{\exp \left(\frac{1}{\tau} \psi_{1}\right)}{\sum_{i=1}^n \exp \left(\frac{1}{\tau} \psi_{i}\right)}.\qedhere
\end{equation*}
\end{proof}



\section{More on Experiments} \label{section: experiment_details}

\paragraph{CIFAR-10 and CIFAR-100} CIFAR-10 ~\citep{krizhevsky2009learning} and CIFAR-100 ~\citep{krizhevsky2009learning} are well-known classic image classification datasets. Both CIFAR-10 and CIFAR-100 contain a total of 60k $32 \times 32$ labeled images of different classes, with 50k for training and 10k for testing. CIFAR-10 is similar to CIFAR-100, except there are 10 different classes in CIFAR-10 and 100 classes in CIFAR-100.

\paragraph{TinyImageNet} TinyImageNet ~\citep{le2015tiny} is a subset of ImageNet ~\citep{deng2009imagenet}. There are 200 different object classes in TinyImageNet, with 500 training images, 50 validation images, and 50 test images for each class. All the images in TinyImageNet are colored and labeled with a size of $64 \times 64$.

\textbf{Pseudo-code.} Algorithm \ref{alg:Training Procedure} presents the pseudo-code for our empirical training procedure.

\begin{algorithm}[!htbp]
\caption{Training Procedure}
\label{alg:Training Procedure}
\begin{algorithmic}[1]
\REQUIRE trainable encoder network $f$, batch size $N$, augmentation strategy \textit{aug}, loss function $L$ with hyperparameters \textit{args}
\FOR {sampled minibatch ${x_i}_{i=1}^N$}
\FORALL{$i \in { 1, ..., N }$}
\STATE draw two augmentations $t_i = \textit{aug}\left(x_i\right) $, $t_i' = \textit{aug}\left(x_i\right) $
\STATE $z_i = f\left(t_i\right)$, $z_i' = f\left(t_i'\right)$
\ENDFOR
\STATE compute loss $\mathcal{L} = L(N, z, z', \textit{args})$
\STATE update encoder network $f$ to minimize $\mathcal{L}$
\ENDFOR
\STATE \textbf{Return} encoder network $f$
\end{algorithmic}
\end{algorithm}

We also provide the pseudo-code for our core loss function used in the training procedure in Algorithm \ref{alg:Core loss}. The pseudo-code is almost identical to SimCLR's loss function, with the exception of an extra parameter $\gamma$.

\begin{algorithm}[!htbp]
\caption{Core loss function $\mathcal{C}$}
\label{alg:Core loss}
\begin{algorithmic}[1]
\REQUIRE batch size $N$, two encoded minibatches $z_1, z_2$, $\gamma$, temperature $\tau$
\STATE $z = \textit{concat}\left(z_1, z_2\right)$
\FOR {$i \in {1, ..., 2N }, j \in {1, ..., 2N}$ }
\STATE $s_{i,j} = \Vert z_i - z_j \Vert_2^{\gamma}$
\ENDFOR
\STATE \textbf{define} $l(i, j)$ \textbf{as} $l(i, j) = - \log \frac{exp\left(s_{i,j}/\tau \right)}{\sum_{k=1}^{2N} \mathbf{1}{[k \ne i]} exp\left(s{i, j} / \tau \right)} $
\STATE \textbf{Return} $\frac{1}{2N} \sum_{k=1}^N\left[l(i, i+N) + l(i+N, i)\right]$
\end{algorithmic}
\end{algorithm}

Utilizing the core loss function $\mathcal{C}$, we can define all kernel loss functions used in our experiments in Table \ref{table: loss definition}. For all $z_i \in z$ with even dimensions $n$, we define $z_{L_i} = z_i\left[0:n/2\right]$ and $z_{R_i} = z_i\left[n/2:n\right]$.

\begin{table}[ht]
\centering
\begin{tabular}{{@{}l|l@{}}}
Kernel  &  Loss function \\ \midrule
Laplacian & $\mathcal{C}\left(N, z, z', \gamma=1, \tau\right)$\\ \midrule
Sum       & $\lambda * \mathcal{C}\left(N, z, z', \gamma=1, \tau_1\right) + (1-\lambda) * \mathcal{C}\left(N, z, z', \gamma=2, \tau_2\right)$  \\ \midrule
Concatenation Sum&$\lambda * \mathcal{C}\left(N, z_L, z'_L, \gamma=1, \tau_1\right) + (1-\lambda) * \mathcal{C}\left(N, z_R, z'_R, \gamma=2, \tau_2\right)$\\ \midrule
$\gamma = 0.5$ & $\mathcal{C}\left(N, z, z', \gamma=0.5, \tau\right)$          \\ 

\end{tabular}

\caption{Definition of kernel loss functions in our experiments}
\label {table: loss definition}
\end{table}

\textbf{Baselines.} We reproduce the SimCLR algorithm using PyTorch Lightning~\citep{PytorchLightning}.

\textbf{Encoder details.}
The encoder $f$ consists of a backbone network and a projection network. We employ ResNet50~\citep{ResNet} as the backbone and a 2-layer MLP (connected by a batch normalization~\citep{ioffe2015batch} layer and a ReLU \cite{nair2010rectified} layer) with hidden dimensions 2048 and output dimensions 128 (or 256 in the concatenation kernel case).

\textbf{Encoder hyperparameter tuning.}
For each encoder training case, we randomly sample 500 hyperparameter groups (sample details are shown in Table \ref{table: Hyperparameter sample}) and train these samples simultaneously using Ray Tune ~\citep{RayTune}, with the ASHA scheduler~\citep{li2018massively}. Ultimately, the hyperparameter group that maximizes the online validation accuracy (integrated in PyTorch Lightning) within 5000 validation steps is chosen for the given encoder training case.

\begin{table}[ht]
\centering

\begin{tabular}{@{}l|l|l@{}}
\midrule
Hyperparameter  & Sample Range & Sample Strategy \\ \midrule
start learning rate & $\left[10^{-2}, 10\right]$ & log uniform \\ \midrule
$\lambda$       & $\left[0, 1\right]$ & uniform \\ \midrule
$\tau$, $\tau_1$, $\tau_2$ & $\left[0, 1\right]$ & log uniform \\ \midrule
\end{tabular}

\caption{Hyperparameters sample strategy}
\label {table: Hyperparameter sample}
\end{table}

\textbf{Encoder training.} 
We train each encoder using the LARS optimizer~\citep{LARSOptimizer}, LambdaLR Scheduler in PyTorch, momentum 0.9, weight decay $10^{-6}$, batch size 256, and the aforementioned hyperparameters for 400 epochs on a single A-100 GPU.

\textbf{Image transformation.} The image transformation strategy, including augmentation, is identical to the default transformation strategy provided by PyTorch Lightning.

\textbf{Linear evaluation.}
The linear head is trained using the SGD optimizer with a cosine learning rate scheduler, batch size 64, and weight decay $10^{-6}$ for 100 epochs. The learning rate starts at $0.3$ and ends at $0$.

\textbf{Moco Experiments.} We also tested our method based on MoCo~\citep{he2019moco}. The results are summarized in Table \ref{tab:results-moco}. Here we choose ResNet18~\citep{ResNet} as the backbone and set a temperature of $0.1$ as default. For our simple sum kernel, we set $\lambda=0.8$. The results show that our method outperforms the original MoCo method.

\begin{table}[thb]
\centering
\caption{MoCo Experiment Results on CIFAR-10 and CIFAR-100.}
\label{tab:results-moco}
\resizebox{\textwidth}{!}{%
\begin{tabular}{@{}c|ccc|ccc@{}}
\toprule
\multirow{3}{*}{Method} & \multicolumn{3}{c|}{CIFAR-10} & \multicolumn{3}{c}{CIFAR-100} \\ \cmidrule(lr){2-4} \cmidrule(lr){5-7} 
                        & 200 epochs & 400 epochs    & 1000 epochs   & 200 epochs & 400 epochs & 1000 epochs         \\ \midrule
MoCo (repro.)         & $76.41 \pm 0.12$    & $80.01 \pm 0.15$          & $84.45 \pm 0.08$    & $\mathbf{47.02 \pm 0.11}$ & $52.50 \pm 0.07$ & $57.62 \pm 0.15$            \\
\midrule
Laplacian Kernel        & ${78.09 \pm 0.10}$    & $\mathbf{83.85 \pm 0.09}$          & $\mathbf{88.34 \pm 0.16}$    & $46.12 \pm 0.22$   & $53.44 \pm 0.17$ & $59.10 \pm 0.14$        \\
Simple Sum Kernel & $\mathbf{78.12 \pm 0.15}$   & $83.23 \pm 0.18$ & $87.50 \pm 0.20$ & $46.65 \pm 0.06$ & $\mathbf{53.62 \pm 0.19}$ & $\mathbf{59.83 \pm 0.12}$\\
\bottomrule
\end{tabular}
}
\end{table}



\section{More Experiments on Synthetic Data}


Consider a scenario with $n$ clusters, each containing $k$ vertices. Let the probability of vertices $u$ and $v$ from the same cluster belonging to $\bfpi$ be $p$. Conversely, for vertices $u$ and $v$ from different clusters, let the probability of belonging to $\pi$ be $q$. We generate the graph $\bfpi$ randomly, based on $p$ and $q$. We experiment with values of $k=100$ and $n=6$ for ease of visualization, embedding all points in a two-dimensional space. Each vertex's initial position originates from a normal distribution. In each iteration, we sample a subgraph of $\bfpi$ uniformly, ensuring each vertex has an out-degree of $1$. We then optimize the corresponding vectors using InfoNCE loss with an SGD optimizer and iterate until convergence. Our experimental setup consists of an SGD learning rate of $1$, an InfoNCE loss temperature of $0.5$, and a batch size of $50$. We evaluate two scenarios with different $p$ and $q$ values: $p=1$, $q=0$, and $p=0.75$, $q=0.2$. The results of these experiments are visualized in Figure \ref{fig:vis-spectral-cluster}. The obtained embeddings exhibit the hallmark pattern of spectral clustering of graph $\bfpi$.

\begin{figure}[!tb]
\centering
\subfigure{
\includegraphics[width=1\textwidth]{Figures/cluster_pi.png}
\label{fig:vis-cluster}
}
\subfigure{
\includegraphics[width=1\textwidth]{Figures/noised_cluster_pi.png}
\label{fig:vis-noised-cluster}
}
\caption{Visualizations of the optimization process using InfoNCE Loss on the vectors corresponding to $\bfpi$. Points of identical color belong to the same cluster within $\bfpi$. To showcase the internal structure of $\bfpi$, we randomly select 10 vertices from each cluster to display the edge distribution of $\bfpi$.}
\label{fig:vis-spectral-cluster}
\end{figure}




\end{document}
\endinput
%%
%% End of file `manuscript.tex'.
