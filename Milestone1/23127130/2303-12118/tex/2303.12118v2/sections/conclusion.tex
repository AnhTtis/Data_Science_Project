Today, the barrier for creating and publishing misleading visual content on social media is low. Media editing tools are increasingly capable of manipulating media in sophisticated and convincing ways, posing a nearly impossible task for human fact-checkers or even automated algorithms to tease apart truth and deception. Provenance protocols present a promising solution: every time a piece of media is created or edited, a record is automatically stamped into the media's metadata. The provenance information can then be surfaced by front-end clients for end users to inspect and use to make an informed judgement about the media's credibility. 

Empowering users with provenance is appealing in theory, but our study revealed that it can also be challenging in practice. We found that access to provenance information can indeed help users correct perceptions on deceptive and manipulated media, but may decrease trust and perceived accuracy with media claims even with perfectly authentic content. Moreover, an additional layer of complexity is introduced in provenance-enabled systems where credibility of \textit{provenance information} co-exists with credibility of the \textit{content itself}. Our study showed that users are not yet prepared to distinguish the two concepts, which can lead to misinterpretations of credibility. Despite this, the average participant self-reported an adequate level of understanding of the provenance UIs' functionality.

Our study provides concrete design suggestions for UIs of provenance-enabled social media platforms, as well as problems for researchers beyond the immediate human-computer interaction and social computing community. Technical work on provenance protocol infrastructure and empirical studies on provenance usability can and should be synergetic. Interdisciplinary work is crucial for addressing key usability challenges to pave the way for successful deployment of usable provenance on the web. 