%%%%%%%%% ABSTRACT
\begin{abstract}
% \wei{For co-authors: Careful, abstrac and introduction (current version is from some time ago) needs to be re-written to be coherent with the method section. Method section is already pretty complete. }
    Large scale \vlfull{} (\vl{}) models have shown fantastic results in terms of aligning between visual and text modalities representations enabling great progress in \zsfull{} recognition, image generation \& editing, and many other exciting tasks. However, these models still have some fundamental weaknesses requiring additional training before applying them to \zsfull{} action recognition tasks. For example, \vl{} models tend to over-represent objects while paying much less attention to verbs, and require additional tuning on video data for best \zsfull{} action recognition performance.
    While previous work relied on large (expensive to obtain) supervised data, in this work we propose an unsupervised approach. We adapt a \vl{} model for zero-shot and few-shot action recognition using a collection of unlabeled videos and a set of language sources, such as unpaired action dictionaries, Large Language Models, and \vl{} models for matching and captioning. 
    % We attain improvement of up to XXX\% in \zs{} transfer to novel action recognition tasks, in some cases even improving upon supervised baselines by up to YYY\%.
    Although finetuned on unlabeled video data, our resulting models demonstrate high transferability to numerous unseen zero-shot downstream tasks, improving the base \vl{} model performance by up to 14\%, and even comparing favorably to fully-supervised baselines in both zero-shot and few-shot video recognition transfer.
    The code is provided in supplementary and will be released upon acceptance.
    
   % The ABSTRACT is to be in fully-justified italicized text, at the top
   % of the left-hand column, below the author and affiliation
   % information. Use the word ``Abstract'' as the title, in 12-point
   % Times, boldface type, centered relative to the column, initially
   % capitalized. The abstract is to be in 10-point, single-spaced type.
   % Leave two blank lines after the Abstract, then begin the main text.
   % Look at previous ICCV abstracts to get a feel for style and length.
\end{abstract}