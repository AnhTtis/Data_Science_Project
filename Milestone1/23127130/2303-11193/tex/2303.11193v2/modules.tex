
\usetikzlibrary{shapes, shapes.misc, backgrounds}


\tikzset{%
	on layer/.code={
		\pgfonlayer{#1}\begingroup
		\aftergroup\endpgfonlayer
		\aftergroup\endgroup
}}

\newlength{\ModuleDiagramSymbolSize}
\setlength{\ModuleDiagramSymbolSize}{.2ex}
\tikzset{
	really densely dotted/.style={line cap=round, dash pattern=on 0pt off 2\pgflinewidth},
	module diagram symbol size/.style={
		generator/.style={shape=circle, fill, outer sep=0, inner sep=#1},
		relation/.style={shape=diamond, fill, outer sep=0, inner sep=#1},
		syzygy/.style={shape=cross out, draw, outer sep=0, inner sep=#1}
	},
	every module diagram/.style={
		module diagram symbol size=.2ex,
		line cap=round,
	},
	module diagram/.style={
		every module diagram,
		every picture
	}
}

\newcommand{\Chains}[3][]{
	\node[#1, generator, at=(#2), label={[#1]below:#3}] (G) {};
	\begin{scope}[#1]
		\fill[opacity=0.2] (#2) |- (inf) |- cycle;
		\draw[] (inf -| G) -- (G) -- (G -| inf);
	\end{scope}

}

\NewDocumentCommand{\CochainsAbsolute}{
	s
	O{} % style
	m % node  the simplex grade
	O{} % style for the label
	m % label
	O{below left} % positioning of the label
}{
	\begin{scope}[#2]
		\IfBooleanTF{#1}{
				\coordinate[at=(#3), label={[#2,#4]#6:#5}] (S);
				\coordinate[at=(#3 |- inf)] (R1);
				\coordinate[at=(#3 -| inf)] (R2);
				\coordinate[at=(inf)] (G);
		}{
			\node[syzygy, at=(#3), label={[#2,#4]#6:#5}] (S) {};
			\node[relation, at=(#3 |- inf)] (R1) {};
			\node[relation, at=(#3 -| inf)] (R2) {};
			\node[generator, at=(inf), opacity=.5] (G) {};
		}
		\fill[opacity=0.2] (S.center) |- (inf) |- cycle;
		\IfBooleanTF{#1}{
			\draw (inf -| #3) -- (S) -- (#3 -| inf);
		}{
			\draw
				(pinf -| #3) -- (S) -- (#3 -| pinf)
				(qinf -| #3) -- (R1) -- (G) -- (R2) -- (#3 -| qinf);
			\draw[every picture, really densely dotted]
				(pinf -| #3) -- (qinf -| #3)
				(pinf |- #3) -- (qinf |- #3);
		}
%	\end{scope}
}

\NewDocumentCommand{\CochainsRelative}{%
	s
	O{} % style
	m % name of the node of the simplex
	O{} % style for the labels
	m % label at the simplex grade
	O{above right} % positioning of the label of the simplex grade
	O{} % label at the first generator
	O{} % label at the second generator
}{
	\begin{scope}[#2]
		\IfBooleanTF{#1}{
			\coordinate[at=(#3 |- inf), label={[#4]above:#7}] (G1);
			\coordinate[at=(#3 -| inf), label={[#4]right:#8}] (G2);
			\coordinate[at=(#3), label={[#4]#6:#5}] (R);
		}{
			\node[generator, at=(#3 |- inf), label={[#4]above:#7}] (G1) {};
			\node[generator, at=(#3 -| inf), label={[#4]right:#8}] (G2) {};
			\node[relation, at=(#3), label={[#4]#6:#5}] (R) {};
		}
		\fill[opacity=0.2] (#3) |- (P |- inf) |- (P -| inf) |- (#3);
			\IfBooleanTF{#1}{
				\draw (#3 |- inf) -- (R) -- (#3 -| inf);
			}{
				\draw
					(#3 |- qinf) -- (G1) -- (P |- inf)
					(#3 -| qinf) -- (G2) -- (P -| inf)
					(#3 |- pinf) -- (R) -- (#3 -| pinf);
				\draw[really densely dotted]
					(#3 |- pinf) -- (#3 |- qinf)
					(#3 -| pinf) -- (#3 -| qinf);
			}
	\end{scope}
}

\newcommand{\infinityCoordinate}{4,4}
\newcommand{\coordinates}{
	\coordinate (inf) at (\infinityCoordinate);
	\coordinate (pinf) at ($(inf)-(0.75,0.75)$);
	\coordinate (qinf) at ($(inf)-(0.25,0.25)$);
	\coordinate (O) at (0,0);
	\coordinate (P) at (-0.5,-0.5);
}

\ProvideDocumentCommand{\DrawGrid}{O{0} D(){0.5} m O{0} D(){0.5} m}{
	\begin{scope}[on background layer]
		\foreach \i in {#1, #2, ..., #3}{
			\foreach \j in {#4, #5, ..., #6}{
				\fill[gray] (\i, \j) circle (0.3pt);
			}
		}
	\end{scope}
}

\newcommand{\AxesHomology}{
	\coordinates
	\begin{scope}[]
		\draw[->] (-0.75,0) -- (O -| inf);
		\draw[->] (0,-0.75) -- (O |- inf);
	\end{scope}
}
\NewDocumentCommand{\AxesCohomology}{
	s % without star, draw the 'finitized' version.
}{
	\coordinates
	\begin{scope}[]
		\IfBooleanTF{#1}{
			\draw[<-, shorten <=-8] (0,0) -- (O -| inf);
			\draw[<-, shorten <=-8] (0,0) -- (O |- inf);
		}{
			\draw[<-, shorten <=-8] (0,0) -- (O -| pinf);
			\draw[<-, shorten <=-8] (0,0) -- (O |- pinf);
			\draw[really densely dotted]  (O -| pinf) -- (O -| qinf);
			\draw[really densely dotted] (O |- pinf) -- (O |- qinf);
			\draw          (O -| qinf) -- (O -| inf);
			\draw          (O |- qinf) -- (O |- inf);
		}
	\end{scope}
}


\newcommand{\FreeModule}[3][]{
	\begin{scope}[#1]
		\node[#3, at=(#2)] {};
		\coordinate[at=(#2)] (n) {};
		\fill[opacity=.25] (#2) rectangle (inf);
		\draw (n) -- (n |- inf) (n) -- (n -| inf);
	\end{scope}
}

\newcommand{\InjectiveModule}[2][]{
	\path (#2) coordinate (n);
	\fill[#1, opacity=.25] (n) rectangle (P);
	\draw[#1] (n) -- (n |- P) (n) -- (n -| P);
}
