\section{Introduction}

The advent of large language models (LLMs)---e.g., GPT-3~\cite{brown2020language}, GPT-NeoX~\cite{gpt-neox-20b}, Jurassic-1~\cite{J1WhitePaper}, LaMDA~\cite{lamda}---has transformed the writing process.
Instead of manually drafting passages of text, writers can now hand over this effort to these models and almost instantly generate passages from an initial sentence or phrase.
Beyond their generative capabilities, LLMs demonstrate significant few-shot and zero-shot performance~\cite{brown2020language} meaning that they are able to perform previously unseen tasks with only an instruction and/or a couple of examples---i.e., a prompt.
By leveraging this ability of LLMs, writers can potentially automate or augment specific tasks in their workflows by using adequate prompts and, thus, further facilitate the writing process.
For instance, based only on prompt examples provided for GPT-3~\cite{examplesopenai}, writers can use LLMs to correct grammar, create an outline, produce analogies, or even change the point-of-view of a scene.

To seize the opportunity presented by LLMs, an assortment of products and interfaces have been created that leverage these models to provide writers with specific tools that automate steps in their writing workflows. 
For example, tools such as WordTune~\cite{wordtune} and NotionAI~\cite{notionai} provide editing buttons that the user can click after selecting text to automatically rewrite it, change its tone, summarize it, elaborate on it, etc.
Additionally, a variety of LLM-powered copywriting tools~\cite{jasper, copyai} have also been created that provide writers with a variety of template forms that they can fill to generate specific types of writing (e.g., video description or script, blog introduction, article headline).
Similarly in academia, various interfaces have been designed to leverage LLMs to support specific tasks: generate various forms of figurative language~\cite{chakrabarty2022help}, summarize a writer's writing~\cite{dang2022beyond}, brainstorm and combine ideas~\cite{di2022idea}, or propagate writing edits across a story~\cite{lee2022interactive}.

\begin{figure}[hb]
    \centering
    \includegraphics[width=1.0\columnwidth]{figures/textblocks.png}
    \caption{Illustrations for the \textbf{concatenate}, \textbf{split}, and \textbf{input} interactions supported in text blocks.}
    \label{fig:textblock}
\end{figure}

While the proliferation of these LLM-driven tools means that various writing tasks can now be supported, the individual needs and challenges of writers might not be fulfilled by these tools.
Due to their type of writing, their fluency with a language, or other factors such as their style and workflow, a writer may have specific needs and challenges during their writing process.
However, while existing interfaces provide a general set of tools, they provide limited or no support for the writer to create their own tools to support their unique tasks.
Further, an interface may not provide a comprehensive set of tools that supports all of the writer's tasks and, thus, the writer may need to constantly switch between multiple interfaces to support their workflows.
As a result, the writer needs to scatter and adapt their writing workflow across a variety of interfaces.

In this work, we envision a canvas-based interface that enables writers to create their own personalized LLM-driven tools and configure them into one cohesive writing environment.
Inspired by object-oriented interaction~\cite{xia2016object, ciolfi2016beyond, xia2018dataink, xia2017collection, han2022passages, beaudouin2000instrumental} and block-based programming~\cite{resnick2009scratch},  we present the design for \textbf{LMCanvas}, an interface that enables users to interact with text and model blocks to flexibly create and arrange LLM-powered tools.
Through the interface, users can create text blocks to encapsulate both their writing and LLM prompts, keep drafts as separate blocks, and organize them in the canvas. 
By connecting text blocks to model blocks (i.e., blocks that represent a set of model configurations), users can create LM pipelines, tools, that generate outputs as text blocks based on the input text and model block.
After creating a set of tools, the user can flexibly arrange them in the canvas to create a writing environment customized to their needs and preferences. 
In this workshop paper, we discuss our design of the envisioned LMCanvas and future work to develop this concept. 
