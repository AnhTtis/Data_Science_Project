\section{Future Work}

In this workshop paper, we outlined the foundational objects and interactions that we aim to support in our envisioned LMCanvas.
Our goal with this interface is to enable writers to more effectively leverage LLMs by personalizing their use to fit their unique workflows, needs, and challenges.
At the current stage of the project, we have developed an initial prototype that supports the three objects presented in this paper and their basic interactions.
With this prototype, we are planning to conduct formative studies to understand the various tools that writers can create with LMCanvas, the benefits and drawbacks of the interfaces, and additional blocks and interactions that writers might need.
Based on the findings, we plan to improve and expand on the concept.
Beyond the directions for improvement to be distilled from the formative study, there are additional future directions that we plan to pursue with LMCanvas.

First, an additional benefit of representing writing as text blocks is that this can enable the interface to maintain a separate history for each text block.
With this modularized history, writers can check on and revert changes for only specific parts of their writing, and they can also reflect back on their generation attempts by seeing what inputs and parameter configurations were previously used.
We are planning to implement this modularized history for text blocks and to enable users to interact with it---e.g., dragging the text input that generated a text block out from the history and into the canvas.

Second, in future versions of LMCanvas, we aim to support various types of output containers for pipeline blocks. 
Currently, the prototype only supports containers that keep generated text blocks as a list.
However, when dealing with a large quantity of generated outputs, writers may need alternative methods to look at and explore generated outputs.
For example, generations could be encoded in a scatterplot~\cite{matejka2018dreamlens} to enable the writer to visualize the output space. 

Finally, identifying effective prompts (i.e., prompt engineering) is a major hurdle in leveraging LLMs. 
While tools have been designed to facilitate this in well-defined tasks where there is a ``ground-truth''~\cite{strobelt2022promptide}, there is limited work that investigated how to support prompt engineering in open-ended and more creative tasks.
Through our initial versions of LMCanvas, we aim to investigate mechanisms to facilitate prompt engineering in open-ended writing and to incorporate these into the interface.
For example, the interface could allow writers to drag-and-drop text blocks into pipeline blocks as positive or negative examples, and leverage these in the back-end to produce outputs more aligned to the writers' preferences.