\section{LMCanvas: Design Concept}

In our envisioned interface, \textbf{LMCanvas}, writers can create four types of objects in an infinite canvas: \textit{text blocks}, \textit{model blocks}, and \textit{pipeline blocks}.
All of these blocks can be flexibly moved, copy-pasted, deleted, and connected to each other.
Below, we detail the specific interactions that we aim to support for each type of block.

\subsubsection{Text Blocks}

In the canvas, the writer can create text blocks, which are objects that the writer can type text into and edit.
When writing with LLMs, text can represent different types of content: actual writing, prompts, examples for prompts, generated outputs, etc.
By compartmentalizing text into modular blocks, our interface allows the writer to flexibly organize and structure these different forms of text in their writing environment.
For example, the writer can use a text block as their main text editor, maintain text blocks on the side containing alternative versions for certain paragraphs, and keep a text block with a prompt template to reuse when creating LLM-powered tools.

\begin{figure}[b]
    \centering
    \includegraphics[width=0.8\columnwidth]{figures/modelblocks.png}
    \caption{Model blocks represent a set of parameter configurations that the writer can configure, copy, and reuse. By connecting model blocks to text blocks, writers can create pipeline blocks that allow them to generate outputs based on the nested text and parameters.}
    \label{fig:modelblock}
\end{figure}

To support flexible use of text blocks, we design the following interactions specific to these blocks. 
\textbf{Resizing} allows the writer to change the format of text blocks for different types of usage (e.g., a larger block for a text editor) or to decrease their size to decrease clutter in the screen.
When the writer decides that they do not have to keep two text blocks separate anymore (e.g., decided on the final versions for the first two verses of their poem), they can \textbf{concatenate} these blocks by drag-and-dropping one text block into the other.
Alternatively, if the writer needs to modularize or separate certain parts of a text (e.g., to only draft one part of a paragraph), they can \textbf{split} off text by selecting it and dragging it outwards---creating a new text block.
To allow writers to create reusable LLM-powered tools, the interface allows the user to create text blocks to which  they can \textbf{input} other text blocks.
Specifically, the user types the ``\verb|[[input]]|'' command in a text block to create an ``input prong''. Then they can attach other blocks into this prong to replace the ``\verb|[[input]]|'' command with the content of the attached text block.
Finally, as writers may want writing support tools to act on selected text (e.g., generate metaphor for selected phrase or edit selected text to be shorter), the interface also allows users to create \textbf{select} blocks by typing the ``\verb|[[select]]|'' command in a text block.
The content of these blocks are replaced by any text that the user selects in the canvas.

\begin{figure*}[!ht]
    \centering
    \includegraphics[width=1.0\textwidth]{figures/pipelineblocks.png}
    \caption{The output container of pipeline blocks can be connected to text blocks to add generations as continuations, or to the input prongs of text blocks to chain pipelines.}
    \label{fig:pipelineblocks}
\end{figure*}

\subsubsection{Model Blocks}

LLMs possess various parameters that control the generation process. 
For example, the temperature parameter determines the probability of the model  generating more out-of-distribution or improbable text. 
Prior work has demonstrated that, when writing with LLMs, different configurations of these parameters can satisfy different user needs~\cite{lee2022coauthor}.
Thus, to support writers to set, test, and reuse parameter configurations, LMCanvas allows users to create multiple model blocks with different combinations of parameters (\autoref{fig:modelblock}).
These blocks represents an instance of parameter configurations that the writer can re\textbf{configure} by clicking on a parameter and using the displayed widgets to change its value. 

\subsubsection{Pipeline Blocks}

To generate text with the LLM, the user can \textbf{connect} a text block to a model block to create a pipeline block (\autoref{fig:modelblock}).
When a writer clicks on ``generate'' in a pipeline block, the interface uses the nested text block as input and the model block as the parameter configurations to generate an output, which is presented as a text block.
To test multiple inputs and parameter configurations, the writer can also \textbf{expand} a pipeline block by adding additional text and model blocks.
In this case, when the pipeline block generates, it produces a generation for each pairing of text and model blocks inside the pipeline block.

By default, each time a pipeline block generates, it adds the generation as a text block in the output container---the box containing ``1'' that prongs out of the pipeline block as seen in \autoref{fig:modelblock}.
However, through drag-and-drop, the writer can connect this output container to (1) a text block to add the generations from the pipeline as \textbf{continuations} to that text block (left in \autoref{fig:pipelineblocks}), or (2) an input prong to \textbf{chain} multiple pipeline blocks together and create more complex tools~\cite{wu2021aichains} (right in \autoref{fig:pipelineblocks}).
Additionally, if the writer connects the output of a pipeline to a \textbf{select} block, the interface replaces any text selected across all text blocks in the canvas with the generation produced by the pipeline.
These various forms of connecting pipeline blocks can enable the writer to create a variety of tools from the same basic blocks.