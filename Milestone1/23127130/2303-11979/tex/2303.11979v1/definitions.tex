%DEFINICIONS DE FORMAT DE TEXT
\newcommand{\xMat}[1]{\mathbf{#1}}  
\newcommand{\vect}[1]{\mathbf{#1}}                                    % vectors
\newcommand{\boxcomment}[1]{\noindent\fbox{\parbox{\textwidth}{#1}}\medskip\\}
\newcommand{\xAbs}[1]{| #1 |}
\newcommand{\mbf}{\mathbf}

\newcommand{\parcial}[2]{\frac{\partial#1}{\partial#2}}
\newcommand{\secparcial}[3]{\frac{\partial^2#1}{\partial#2\partial#3}}
\newcommand{\secparcialab}[1]{\secparcial{#1}{\alpha}{\beta}}
\newcommand{\derivada}[2]{\frac{\text{d}#1}{\text{d}#2}}
\newcommand{\eval}[2]{|^{#1}_{#2}}
%
% comanda eloi ALGORITMES
%\renewcommand{\algorithmicensure}{\textbf{Return :}}
%
% COMANDES TEORIA NOVA (REFERENCIANT A NOMS DE COSES):
% objective function
\newcommand{\xeta}[1]{{\eta_{#1}}}
\newcommand{\xetaParam}[1]{{\tilde{\eta}_{#1}}}
\newcommand{\xetaParamBis}[2]{{\tilde{\eta}_{#1}^{#2}}}
\newcommand{\xetaRestr}[1]{{\hat{\eta}_{#1}}}
\newcommand{\xdeviationLinear}{\xdistortionLinear}
\newcommand{\xdistortionLinear}{\eta}
\newcommand{\xdistortionLinearMod}{\xdistortionLinear_*}
\newcommand{\xDistortionLinear}{\xdistortionLinear}
\newcommand{\xDistortionDelta}{\xDistortionLinear_\delta}
\newcommand{\xDev}[1]{{\eta_{#1}}}
\newcommand{\xDevPlane}{\xDev{}}
\newcommand{\xDevPlaneRestr}{\hat{\xDev{}}}
\newcommand{\xDevSrf}{\xDev{\null_\xSurface}}
\newcommand{\xDevParam}{\xDev{\null_\xParamPlane}}
\newcommand{\xDevParamBis}[1]{\xDev{\null_{\xParamPlane_{#1}}}}
\newcommand{\xDevParamRestr}{\hat{\xDev{}}_{\null_\xParamPlane}}
\newcommand{\xDevParamRestrBis}[1]{\hat{\xDev{}}_{\null_{\xParamPlane_{#1}}}}
\newcommand{\xQuality}[1]{q_{#1}}
\newcommand{\xQualityPlane}{q}
\newcommand{\xQualitySrf}{\xQuality{\null_\xSurface}}
\newcommand{\xQualityParam}{\xQuality{\null_\xParamPlane}}%[1]{\tilde{\xQuality{}}_{#1}}
\newcommand{\xQualityRestr}{{\hat{\xQuality{}}_{\null_\xParamPlane}}}
\newcommand{\xedge}[1]{\vect{e}_{#1}}
\newcommand{\xedgeN}[1]{\tilde{\xedge{}}_{#1}}
\newcommand{\eprev}{\xedge{2}^{\text{prev}}}
% nodes of a triangle 
\newcommand{\xPointSrf}[1]{\vect{x}_{#1}}
\newcommand{\xPointAuxPlane}[1]{\vect{y}_{#1}}
% \newcommand{\xnodeCoord}[1]{x_{#1}}
% \newcommand{\xnodeBasic}{\vect{x}}
% \newcommand{\xnode}[1]{\xnodeBasic_{#1}}
% \newcommand{\xCoordx}[1]{(x_{#1},y_{#1})}
% % \newcommand{\ynode}[1]{\vect{y}_{#1}}
% \newcommand{\xCoordy}[1]{(x_{#1},y_{#1},z_{#1})}
\newcommand{\xnodeCoord}[1]{x^{#1}}
\newcommand{\xnodeBasic}{\vect{x}}
\newcommand{\xnode}[1]{\xnodeBasic_{#1}}
\newcommand{\xCoordx}[1]{(x_{#1},y_{#1})}
\newcommand{\ynode}[1]{\vect{y}_{#1}}
\newcommand{\xCoordy}[1]{(x_{#1},y_{#1},z_{#1})}
% nodes of a triangle tilde
\newcommand{\xnodeTilde}[1]{\tilde{\vect{x}}_{#1}}
% nodes of a triangle hat
\newcommand{\xnodeHat}[1]{\vect{x}_{#1}}
% triangle space (3 coord, 2 coord)
\newcommand{\xTriangleSpace}[1]{\altmathcal{T}_{#1}}
\newcommand{\xTriangleSpaceSrf}{\altmathcal{T}_{\xSurface}}
\newcommand{\xTriangleSpacePlane}{\altmathcal{T}}
\newcommand{\xTriangleSpaceParam}{\altmathcal{T}_{\xParamPlane}}
% 3D triangle
\newcommand{\xTriangleIIID}{\xTriangle_{\null_{\xSurface}}}
\newcommand{\xTriangleSrf}{\xTriangle_{\null_{\xSurface}}}
% 3D triangle with z=0
\newcommand{\xTriangleIID}{\xTriangle}
\newcommand{\xTrianglePlane}{\xTriangle}
% 2D triangle
\newcommand{\xTriangle}{E}
\newcommand{\xElement}{E}
\newcommand{\xElementInd}[1]{E_{#1}}
\newcommand{\xTriangleParamPlane}{\xTriangle_{\null_{\xParamPlane}}}
% function from TriangleSpace3 to TriangleSpace3'
\newcommand{\xT}{\vect{T}}
\newcommand{\xText}{\tilde{\xT}}
% parametrization of the surface
\newcommand{\xparametrization}{\boldsymbol{\varphi}}
\newcommand{\xparametrizationTri}{\tilde{\xparametrization}}
\newcommand{\xParamPlane}{\altmathcal{U}}
\newcommand{\xSurface}{\Sigma}
% projection 3 coord to 2 coord
\newcommand{\xprojection}{\vect{\pi}}
% jacobian matrices
\newcommand{\xJacobianRP}{\mat{A}} %jacobian matrix ReferenceTri to PhysicalTri
\newcommand{\xJacobianRI}{\mat{W}} %jacobian matrix ReferenceTri to IdealTri
\newcommand{\xJacobianIP}{\mat{S}} %jacobian matrix IdealTri to PhysicalTri
% R de real
\newcommand{\xreal}[1]{\mathbb{R}^{#1}}
% funció continua
\newcommand{\xcont}[1]{\altmathcal{C}^{#1}}
%punts a l'espai paramètric
\newcommand{\xParamPoint}[1]{\vect{u}_{#1}}
\newcommand{\xParamCoord}[1]{(u_{#1},v_{#1})}

\newcommand{\xscaleFun}{\tilde{\mathbf{\lambda}}}
\newcommand{\xscaleFunRestr}{\mathbf{\lambda}}
\newcommand{\xscaleParam}{\Lambda}

\newcommand{\mesh}{\altmathcal{M}}
\newcommand{\xMesh}{\altmathcal{M}}
\newcommand{\xMeshIdeal}{\xMesh_I}
\newcommand{\xMeshPhysical}{\xMesh_P}
\newcommand{\xMeshRef}{\xMesh_M}
\newcommand{\submesh}[1]{\xMesh_{#1}}

% HIGH ORDER ELEMENTS
\newcommand{\physTri}[1]{\xTriangle_{#1}}
\newcommand{\xphysTri}[1]{\xTriangle_{#1}}
\newcommand{\xrefTri}[1]{\xTriangle_{R_{#1}}}
\newcommand{\xidealTri}[1]{\xTriangle_{I_{#1}}}
\newcommand{\physElem}[1]{\xElement_{#1}^P}
\newcommand{\xphysElem}[1]{\physElem{#1}}
\newcommand{\xPhysElem}[1]{\physElem{#1}}
\newcommand{\xElementPhys}[1]{\physElem{#1}}
\newcommand{\xrefElem}[1]{\xElement^{M}_{#1}}
\newcommand{\xRefElem}[1]{\xrefElem{#1}}
\newcommand{\xidealElem}[1]{\xElement^{I}_{#1}}
\newcommand{\xIdealElem}[1]{\xidealElem{#1}}
\newcommand{\xElementRef}[1]{\xRefElem{#1}}
\newcommand{\xElementIdeal}[1]{\xIdealElem{#1}}
\newcommand{\xIdealCoord}[1]{\mathbf{y}_{#1}}
\newcommand{\xReferenceCoord}[1]{\xNodeRef{#1}}
\newcommand{\xPhysicalCoord}[1]{\xnode{#1}}

\newcommand{\xNodeRef}[1]{\boldsymbol{\xi}_{#1}}
\newcommand{\xNodeRefCoord}[1]{\xi^{#1}}
\newcommand{\xrepresentationHO}{\boldsymbol{\phi}}
\newcommand{\xrepresentationRP}{\xrepresentationHO_P}
\newcommand{\xrepresentationRI}{\boldsymbol{\phi}_I}
%\newcommand{\xrepresentationIP}{\boldsymbol{\phi}_{\xElement{}}}
\newcommand{\xrepresentationIP}{\boldsymbol{\phi}}
\newcommand{\xrepresentationIPind}[1]{\boldsymbol{\phi}_{\xElementInd{#1}}}
\newcommand{\xDomMaphE}{\boldsymbol{\phi}_{h_{|_\xElement}}}
\newcommand{\xDomMaphEInd}[1]{\boldsymbol{\phi}_{h_{|_{\xPhysElem{#1}}}}}
%\newcommand{\HOind}[1]{\boldsymbol{\phi}_{#1}}
%\newcommand{\xmapRPind}[1]{\xrepresentationHO_{P_#1}}
%\newcommand{\xmapRIind}[1]{\xrepresentationHO_{I_#1}}
%\newcommand{\xmapIPind}[1]{\xrepresentationIPind{#1}}
\newcommand{\xrepresentationIPelem}{\boldsymbol{\phi}_{h_{NO}}}
\newcommand{\expDeviation}{{2}}
\newcommand{\expNormQual}{{2}}
%\newcommand{\xdeviationHO}{\hat{\eta}_{\expDeviation}}
%\newcommand{\deviationHO}{\xdeviationHO}
%\newcommand{\xdeviationHOexp}[1]{\hat{\eta}_{#1}}
%\newcommand{\deviationHOexp}[1]{\xdeviationHOexp{#1}}
\newcommand{\xdeviationHO}{\eta_{{}}}%\hat{\eta}_{\expDeviation}}
\newcommand{\xDistortionHO}{\xdeviationHO}
\newcommand{\xDistortionHOind}[1]{\eta_{{#1}}}

\newcommand{\xDistortionHOReg}{\xDistortionLinear_{\delta,\xElement{}}}
\newcommand{\xDistOperator}[1]{\mathrm{M}#1}
\newcommand{\xDistOperatorRegularized}[1]{\mathrm{M}_{\delta}#1}
\newcommand{\xDistortionHOPunctMesh}{\xDistOperator{\xDomMaph}}
\newcommand{\xDistortionHOPunctElem}{\xDistOperator{\xrepresentationIP}}
\newcommand{\xDistortionHOPunctElemInd}[1]{\xDistOperator{\xrepresentationIPind{#1}}}
\newcommand{\xDistortionHOPunctMeshReg}{\xDistOperatorRegularized{\xDomMaph}}
\newcommand{\xDistortionHOPunctElemReg}{\xDistOperatorRegularized{\xrepresentationIP}}
\newcommand{\xDistortionHOPunctElemIndReg}[1]{\xDistOperatorRegularized{\xrepresentationIPind{#1}}}
\newcommand{\xDistortionHOMesh}{\xDistortionLinear_{{\xMesh{}}}}
\newcommand{\xDistortionHOMeshReg}{\xDistortionLinear_{\delta,{\xMesh{}}}}

%\newcommand{\xDistortionHOind}[1]{\eta_{\xElementInd{#1}}}
%\newcommand{\xDistOperator}[1]{\mathrm{M}#1}
\newcommand{\xDistortionHOPunct}{\xDistOperator{\xrepresentationIP}}
\newcommand{\xDistortionHOPunctInd}[1]{\xDistOperator{\xrepresentationIPind{#1}}}
\newcommand{\xDistortionHOPunctSurf}[1]{\mathrm{M}^{#1}{\xrepresentationIP}}
\newcommand{\xDistortionHOPunctSurfElem}[2]{\mathrm{M}^{#1}{\xrepresentationIPind{#2}}}
\newcommand{\xDistortionHOsurf}{\eta^{\xparametrization}_{{}}}%\hat{\eta}_{\expDeviation}}
\newcommand{\xDistortionHOsurfInd}[1]{\eta^{\xparametrization}_{{#1}}}%\hat{\eta}_{\expDeviation}}
% \newcommand{\xdeviationHOexp}[1]{\hat{\eta}_{#1}}
% \newcommand{\deviationHOexp}[1]{\xdeviationHOexp{#1}}
\newcommand{\xqualityHO}{q_{\xElement{}}}
\newcommand{\qualityHO}{\xqualityHO}
\newcommand{\setOfHOnodes}[2]{{#1}_{1},\ldots,{#1}_{\numberNodesHO{#2}}}
\newcommand{\setOfHOnodesComplete}[1]{{#1}_1,\ldots,{#1}_{\numberNodesHO{\xOrderHO}}}
\newcommand{\completeRepresentationHO}[3]{\xrepresentationHO(#1;\setOfHOnodes{#2}{#3})}
\newcommand{\completeRepresentationRP}[3]{\xrepresentationRP(#1;\setOfHOnodes{#2}{#3})}
\newcommand{\xOrderHO}{p}
% \newcommand{\completeRepresentationHO}[#2]{\representationHO(#1,#2)}%;\xnode{1},\ldots,\xnode{\numberNodesHO{#2}})}
\newcommand{\numberNodesHO}[1]{{{n}_{#1}}}
\newcommand{\shapeFunction}[1]{N_{#1}}
\newcommand{\xetaScaled}[1]{{\overline{\eta}_{#1}}}

\newcommand{\xDifferential}[1]{\xMat{D}#1}

\newcommand{\xObjFun}[1]{f_{#1}}

\newcommand{\xArea}[1]{|#1|}
\newcommand{\xDet}{\text{det }}

% % % % 
\newcommand{\xIdealManifold}{\altmathcal{M}}
\newcommand{\xPhysicalManifold}{\altmathcal{N}}
\newcommand{\xTangent}[2]{T_{#1}#2}
\newcommand{\xetaDiff}[1]{\mu_{#1}}
\newcommand{\xdeviationHODiff}{\hat{\mu}_{\expDeviation}}



\newcommand{\xAffine}{\mathbf{\psi}}
\newcommand{\xA}{\mathbf{A}}
\newcommand{\xb}{\mathbf{b}}


\newcommand{\xSign}{\gamma}


%%%%%%%%%%%%%%%% NEW HO SURFACE SMOOTHING
\newcommand{\numberNodesHOComplete}{\numberNodesHO{\xOrderHO}}
\newcommand{\xProductSpace}[2]{{#1}^{\times#2}}
\newcommand{\xProductSpaceBracket}[2]{\left(#1\right)^{\times#2}}

\newcommand{\xDistortionHOSRF}{\xDistortionHOSRFpunct^\xElement{}}
\newcommand{\xDistortionHOSRFpunct}{{\eta}_{\null_{\xParamPlane}}}
% {{\eta}_{\xrepresentationIP_{\xparametrization}}}
\newcommand{\xqualityHOSRF}{{q}_{\null_{\xParamPlane}}^\xElement{}}
% {{q}_{\xrepresentationIP_{\xparametrization}}}
\newcommand{\xDistortionHOSRFsrf}{\xDistortionHOSRFsrfPunct^\xElement{}}
\newcommand{\xDistortionHOSRFsrfPunct}{{\eta}_{\null_{\xSurface}}}
\newcommand{\xqualityHOSRFsrf}{{q}_{\null_{\xSurface}}^\xElement{}}
\newcommand{\xDistortionHOSRFBis}[1]{{\eta}_{\null_{\xParamPlane_{#1}}}}


\newcommand{\xTangentSpace}[2]{\text{T}_{#1}#2}
\newcommand{\xTmatrix}{\mat{M}}

\newcommand{\xNodeIdeal}[1]{\xIdealCoord{#1}}
\newcommand{\xNodeIdealCoord}[2]{{y}^{#1}_{#2}}


%\newcommand{\xModSigma}{\sigma_*}%varpi}
\newcommand{\xModSigma}[1]{\sigma_{#1}}%varpi}
\newcommand{\xModSigmaD}{\sigma_{\delta}}%varpi}
\newcommand{\xParameterDet}{\varepsilon}
\newcommand{\xTol}{\tau}
\newcommand{\xDeteen}{\mathbb{D}}
\newcommand{\xConstantTerm}{\kappa}

\newcommand{\xNumElements}{n_{E}}
\newcommand{\xNumNodes}{n_N}
\newcommand{\xFreeNodes}{n_F}
\newcommand{\xFreeIndices}{\altmathcal{I}_F}
\newcommand{\xNumNodesElement}{n_{N,e}} %{n_{N_e^k}}
\newcommand{\xNumNodesLoop}{n_L}
\newcommand{\xNumElementsNeigh}[1]{n_{E}^{#1}}
\newcommand{\xNumElementsSurface}{n_{E_s}}
\newcommand{\xNumNodesSurface}{n_{N_s}}

\newcommand{\xGlobalObjFun}{f}
\newcommand{\xElemObjFun}[1]{f_{#1}}
\newcommand{\xDistSubMeshParam}{\hat{f}}%K_{\xParamPlane}}
\newcommand{\xDistSubMeshSrf}{K_{\xSurface}}
\newcommand{\xbnabla}{\bm{\nabla}}
\newcommand{\xHessian}{\xbnabla^2}
\newcommand{\xStep}[2]{\alpha_{#1}^{#2}}
\newcommand{\xDirection}[2]{\vect{p}_{#1}^{#2}}
\newcommand{\xUpdatedVec}[2]{\vect{w}_{#1}^{#2}}


\newcommand{\xSizeField}{h}
\newcommand{\xShapeMeasure}{\xdeviationLinear_{sh}}
\newcommand{\xSizeMeasure}{\xdeviationLinear_{si}}
\newcommand{\xSizeShapeMeasure}{\xdeviationLinear_{ss}}
\newcommand{\xMin}{\mu}
\newcommand{\xModmin}{\mu_*}



\newcommand{\xScalarProduct}[3]{\langle{#1},{#2}\rangle_{#3}}
\newcommand{\xNorm}[2]{\|{#1}\|_{#2}}
\newcommand{\xScalarProductMesh}[2]{\langle{#1},{#2}\rangle_{\xMesh{}}}
\newcommand{\xScalarProductMeshI}[2]{\langle{#1},{#2}\rangle_{\xMeshIdeal}}
\newcommand{\xNormMesh}[1]{\|{#1}\|_{\xMesh{}}}
\newcommand{\xNormMeshIdeal}[1]{\|{#1}\|_{\xMeshIdeal{}}}
\newcommand{\xExtraConstant}{}%{\frac{1}{2}}



\newcommand{\xElemNod}[1]{\hat{#1}}
\newcommand{\xElemNodTotal}{\hat{n}_\xOrderHO}
\newcommand{\numberNodesHOLocal}[1]{{\xElemNod{n}_{#1}}}

\newcommand{\xFunctionSpace}{\altmathcal{U}}
\newcommand{\xFunctionSpaceVect}{\boldsymbol{\xFunctionSpace}}
\newcommand{\xPolynomialSpace}{\altmathcal{P}}

\newcommand{\xDomain}[1]{\Omega_{#1}}
\newcommand{\xDomainIdeal}{\xDomain{I}}
\newcommand{\xDomainPhysical}{\xDomain{P}}

\newcommand{\xDim}{d}

\newcommand{\xelind}{e}
\newcommand{\xnodind}{e}

\newcommand{\xDomMap}{\phi}
\newcommand{\xDomMaph}{\phi_{h}}
\newcommand{\xDirichlet}{g}
\newcommand{\xDirichleth}{\xDirichlet_{h}}


\newcommand{\argmin}{\textrm{argmin}}

\newcommand{\xFunction}{\phi}
\newcommand{\xMappingLinearElements}{\xrepresentationIP}


\newcommand{\aposteriori}{\textit{a posteriori }}
\newcommand{\xScaledJacobian}{\mu}

\newcommand{\xPseudoNormal}{\vect{n}_{\xnode{}}}
\newcommand{\xDiamDisc}{D}
\newcommand{\tetgen}{{TetGen}}
\newcommand{\alya}{{Alya}}
\newcommand{\alyaCode}{{the Alya solver}}
\newcommand{\zR}{\mathbb{R}}
\newcommand{\zdim}{d}
\newcommand{\zmetric}{\textbf{M}}
\newcommand{\zvep}{\textbf{U}}
\newcommand{\zvepT}{\textbf{U}^\mathrm{T}}
\newcommand{\zvap}{\textbf{D}}
\newcommand{\zeuclideanaxis}{\ze^\triangle}
\newcommand{\zidealaxis}{\ze^I}
\newcommand{\zmasteraxis}{\ze^M}
\newcommand{\zphysicalaxis}{\ze^P}
\newcommand{\zisotropicphysicalaxis}{\ze^{P_\triangle}}
\newcommand{\zfield}{\textbf{F}}
\newcommand{\ztr}{\mathrm{tr}}
\newcommand{\zmaster}{E^M}
\newcommand{\zbmaster}{E^{\hat{M}}}
\newcommand{\zmastercircle}{E^M}
\newcommand{\zequilater}{E^{\triangle}}
\newcommand{\zequilatercircle}{E^{\triangle}}
\newcommand{\zideal}{E^I}
\newcommand{\zidealcircle}{E^I}
\newcommand{\zphysical}{E^P}
\newcommand{\zbphysical}{E^{\hat{P}}}
\newcommand{\zphysicalcircle}{E^P}
\newcommand{\zequilaterphysicalmap}{\zphi_E}
\newcommand{\zidealphysicalmap}{\zphi_E}
\newcommand{\zequilatermap}{\zphi_{\triangle}}
\newcommand{\zidealmap}{\zphi_I}
\newcommand{\zphysicalmap}{\zphi_P}
\newcommand{\zmastermap}{\zphi_M}
\newcommand{\zbphysicalmap}{\zphi_{\hat{P}}}
\newcommand{\zisotropicphysical}{E^{P_\triangle}}
\newcommand{\zisotropicphysicalmap}{\zphi_{P^{\triangle}}}
\newcommand{\zJacobianisotropicphysical}{\textbf{D}\zphi_{P^{\triangle}}}
\newcommand{\zfieldmap}{\zfield}%{\zphi_\textbf{F}}
\newcommand{\zequilaterisotropicphysicalmap}{\zphi_{U}}
\newcommand{\zJacobianfield}{\textbf{D}\zfieldmap}
%\newcommand{\zJacobianequilaterphysical}{\zJacobianphysical \cdot \zJacobianequilater^{-1}}
\newcommand{\zJacobianequilaterphysical}{\textbf{D}\zequilaterphysicalmap}
\newcommand{\zJacobianequilaterisotropicphysical}{\textbf{D}\zequilaterisotropicphysicalmap}
\newcommand{\zJacobianequilaterisotropicphysicalcircle}{\textbf{D}\zequilaterisotropicphysicalcirclemap}
\newcommand{\zJacobianidealphysical}{\textbf{D}\zidealphysicalmap}
\newcommand{\zJacobianequilater}{\textbf{D}\zequilatermap}
\newcommand{\zJacobianideal}{\textbf{D}\zidealmap}
\newcommand{\zJacobianphysical}{\textbf{D}\zphysicalmap}
\newcommand{\zmetricspace}{\left(\zR^\zdim,\zmetric\right)}
\newcommand{\zeuclideanspace}{\left(\zR^\zdim,\mathds{1}\right)}
\newcommand{\zconjugate}{\left(\zJacobianphysical\cdot\zJacobianequilater^{-1}\right)^\mathrm{T}\cdot\zmetric\cdot\zJacobianphysical\cdot\zJacobianequilater^{-1}}
%\newcommand{\zconjugatexi}{\left(\zJacobianphysical(\zxi)\cdot\left(\zJacobianequilater(\zxi)\right)^{-1}\right)^\mathrm{T}\cdot\zmetric(\zphysicalmap(\zxi))\cdot\zJacobianphysical(\zxi)\cdot\left(\zJacobianequilater(\zxi)\right)^{-1}}
\newcommand{\zconjugatexi}{\zJacobianequilaterphysical(\zxi)^\mathrm{T}\cdot\zmetric(\zphysicalmap(\zxi))\cdot\zJacobianequilaterphysical(\zxi)}
\newcommand{\zrotation}{\textbf{R}(\theta)}
\newcommand{\zmetricrotation}{\textbf{R}_\zmetric(\theta)}
\newcommand{\ze}{\textbf{e}}
\newcommand{\zf}{\textbf{f}}
\newcommand{\zv}{\textbf{v}}
\newcommand{\zu}{\textbf{u}}
\newcommand{\zparameterization}{\boldsymbol{\varphi}}
\newcommand{\zG}{\textbf{G}}
\newcommand{\zx}{\textbf{x}}
\newcommand{\zxi}{\boldsymbol{\xi}}
\newcommand{\zbxi}{\hat{\boldsymbol{\xi}}}
\newcommand{\zy}{\textbf{y}}
\newcommand{\zp}{\textbf{p}}
\newcommand{\zbp}{\hat{\textbf{p}}}
\newcommand{\zA}{\textbf{A}}
\newcommand{\zid}{\mathrm{Id}}
\newcommand{\zphi}{\boldsymbol{\phi}}
\newcommand{\zpsi}{\boldsymbol{\psi}}
%\newcommand{\zbpsi}{\hat{\boldsymbol{\psi}}}
\newcommand{\zbpsi}{\hat{N}}
\newcommand{\zallbpsi}{\hat{\textbf{N}}}
\newcommand{\zT}{\textbf{T}}
\newcommand{\zmesh}{\altmathcal{M}}
\newcommand{\zbmesh}{\hat{\zmesh}}
\newcommand{\zidealmesh}{\zmesh_I}
\newcommand{\zequilateralmesh}{\zmesh_\triangle}
\newcommand{\zmeshspace}{\left(\zmesh,\mathds{1}\right)}
\newcommand{\zequilateralmeshspace}{\left(\zequilateralmesh,\mathds{1}\right)}
\newcommand{\zmeshmetricspace}{\left(\zmesh,\zmetric\right)}
\newcommand{\zequilateralmeshmetricspace}{\left(\zequilateralmesh,\zmetric\right)}
\newcommand{\zmetricdistortion}{\eta_{\zmetric}\left(\zJacobianequilaterphysical\right)}
\newcommand{\zmetricdistortiony}{\eta_{\zmetric(\zequilaterphysicalmap(\zy))}\left(\zJacobianequilaterphysical(\zy)\right)}
\newcommand{\zdistortiony}{\eta\left(\zJacobianequilaterphysical(\zy)\right)}
\newcommand{\zmetricdistortionideal}{\eta_{\zmetric\circ\zequilaterphysicalmap}\left(\zJacobianequilaterphysical\right)}
\newcommand{\zeuclideandistortionoperator}{\altmathcal{N}\zequilaterisotropicphysicalmap}
\newcommand{\zeuclideandistortionoperatorreg}{\altmathcal{N}_0\zequilaterisotropicphysicalmap}
\newcommand{\zdistortionoperator}{\altmathcal{N}\zequilaterphysicalmap}
\newcommand{\zmetricdistortionoperator}{\altmathcal{N}_{\zmetric}\zidealphysicalmap}
\newcommand{\zmetricdistortionoperatormesh}{\altmathcal{N}_{\zmetric}\phi_{h}}
\newcommand{\zelementspace}{\left(E,\mathds{1}\right)}
\newcommand{\zelementmetricspace}{\left(\zphysical,\zmetric\right)}
\newcommand{\zder}{\frac{d}{dt}}
\newcommand{\zders}{\frac{d}{ds}}
\newcommand{\zderx}{\frac{\partial}{\partial x}}
\newcommand{\zderxi}{\frac{\partial}{\partial x_i}}
\newcommand{\zderyi}{\frac{\partial}{\partial y_i}}
\newcommand{\zAconjugate}{\zA^\mathrm{T}\cdot\zmetric\cdot\zA}
\newcommand{\zJacobianderxequilaterphysical}{\textbf{D}\frac{\partial}{\partial (\zx_0,\zx_1,...,\zx_n)}\zequilaterphysicalmap}
\newcommand{\zJacobianderxiequilaterphysical}{\textbf{D}\zderxiequilaterphysicalmap}
\newcommand{\zderxiequilaterphysicalmap}{\frac{\partial\zequilaterphysicalmap}{\partial x_i}}
\newcommand{\zmetricdistortioneval}{\eta_{\zmetric(\zequilaterphysicalmap(\zy))}\left(\zJacobianequilaterphysical(\zy)\right)}
\newcommand{\zanisotropicratiophysical}{\rho_{\zphysical}}
\newcommand{\zanisotropicratioideal}{\rho_{\zideal}}
\newcommand{\zanisotropicratiophysicaloperator}{\mathfrak{r}_{\zphysical}}
\newcommand{\zanisotropicratioidealoperator}{\mathfrak{r}_{\zideal}}
\newcommand{\zmetricball}{B_{r,\zmetric}}
\newcommand{\zmetricsphere}{S_{\zmetric}}
\newcommand{\zball}{B_{r,\mathds{1}}}
\newcommand{\zsphere}{S_{\mathds{1}}}
\newcommand{\zmetricaxis}{\zG}
\newcommand{\zaxisball}{B_{r,\zmetricaxis}}
\newcommand{\zidealball}{B_{r,\zmetricaxis(\ze^{I})}}
\newcommand{\zidealballvar}{B_{r,\zmetricaxis(\ze^{I}(\zx))}}
\newcommand{\zmetricballvar}{B_{r,\zmetric(\zx)}}
\newcommand{\zmetricspherevar}{S_{\zmetric(\zx)}}
\newcommand{\zvol}{\mathrm{vol}}
\newcommand{\ztangentmetricspace}{T_{\zy}\zmetricspace}
\newcommand{\ztangenteuclideanspace}{T_{\zy}\zeuclideanspace}
\newcommand{\zsign}{\mathrm{sign}}

%\newcommand{\zDline}{\textbf{D}_{\text{line}}}
\newcommand{\zDline}{\textbf{D}}
\newcommand{\zDcross}{\textbf{D}_{\text{cross}}}
%\newcommand{\zmapline}{\psi_{\text{line}}}
\newcommand{\zmapline}{\psi}
\newcommand{\zmapcross}{\psi_{\text{cross}}}
\DeclareMathAlphabet{\altmathcal}{OMS}{cmsy}{m}{n}
\newcommand{\loc}{\psi}

%geometry

\newcommand{\zmodel}{\Lambda}
\newcommand{\znurbs}{\Gamma}
% \newcommand{\zchull}{\mathcal{C}\mathcal{H}}
% \newcommand{\zinout}{\mathcal{I}\mathcal{O}}
\newcommand{\zchull}{\text{CH}}
\newcommand{\zinout}{\text{IO}}
\newcommand{\ztrimmednurbs}{\overline{\Gamma}}
\newcommand{\zmatrix}{\mathbb{N}}
% \newcommand{\zmat}{\zmatrix\left(\zx\right)}
\newcommand{\zmat}{\zmatrix}
% \newcommand{\zadj}{\zmatrix^*\left(\zx\right)}
\newcommand{\zadj}{\adj{\zmatrix\left(\zx\right)}}
\newcommand{\zadjj}{\adj{\zmatrix}}
\newcommand{\adj}[1]{\text{adj}\left(#1\right)}
\newcommand{\gfun}{\gamma}
\newcommand{\gfunn}{\gfun_{\znurbs}}
% \newcommand{\gfun}{\gamma\left(\zx\right)}
\newcommand{\gnfun}{\hat{\gamma}}
\newcommand{\gnfunn}{\gnfunn_{\znurbs}}
\newcommand{\tr}{\text{tr}}

%\algnewcommand{\algorithmiccommentt}[1]{*/ #1 */}