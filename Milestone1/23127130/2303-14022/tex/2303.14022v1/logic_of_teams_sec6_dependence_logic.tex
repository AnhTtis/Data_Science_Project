%!TEX root = logic_of_teams.tex


\section{Axiomatising valuational team semantics}\label{sec:axiomatising}

We have seen that LT is compact, and that the class of finite Boolean
algebras is adequate. We also know that every finite Boolean algebra is
isomorphic to a Boolean algebra of the form $(\P S , \emptyset ,\cdot^C, \cup,\cap)$
for some finite set $S$. Thus, the class of Boolean algebras with
domain $\P S $ for finite $S$ is adequate for LT. Knowing this
we can show how standard valuational team semantics relate to the logic
LT. This construction follows in essence the construction presented
in \cite{LorimerOlsson2022} with some refinement and generalisation. The
construction and proof will be developed through the following steps.

\begin{enumerate}
	\item We start by defining the propositional team logic $\text{PT}^+$ by
	its denotations in valuational team semantics. This is an expressive logic
	with many interesting logics as syntactic fragments.
	\item Then we show that we formulate this semantics in term of the algebra $\P\P
	2^{\mathbb{N}}$ with a specific homomorphism $H_V$ in the style of the
	semantics we have given for LT. We can then prove that
	LT is in a sense a conservative extension of a logic weaker than
	$\text{PT}^+$.
	\item Next we observe that the homomorphism $H_V$ belongs to a special
	class of homomorphism that are nicely definable and axiomatisable in
	LT using defined connectives from Section \ref{sec:definable}.
	\item In the final step we show that every homomorphism of this class on a
	certain adequate collection of Boolean algebras can be mapped to a
	set of teams of $\P2^\mathbb{N}$ such that $H_V$ expresses the memberhood
	of the original homomorphism by the mapping.
	\item We can then conclude that $\text{PT}^+$ is axiomatised as a fragment
	of LT by the axioms of the class of homomorphisms.
\end{enumerate}

In this section we will treat several logics defined for different sets of
connectives, and thus with different sets of formulas. For a logic L we
denote by $\Fm_{\text{L}}$ the formulas of the logic and $\text{L}:\Delta \vDash \phi$ for the
entailment of the logic. In particular, the formulas of LT will be referred to
as $\Fm_{\text{LT}}$, and it's entailment will be denoted by $\text{LT}: \Delta
\vDash \phi$ in contrast to other sections of this paper.

\subsection{Valuational team semantics}\label{sec:valuational}

We are interested in formulating a logic in a sense generalising some
propositional team logics given in team semantics on valuations. We do so by
identifying the set of connectives included and define them by their
denotation on the teams of valuations $\P2^\mathbb{N}$. In order to cover a
majority of the more interesting logics we focus our attention on
\textit{strong propositional team logic}, $\text{PT}^+$, one of the
strongest logics presented in \cite{Yang2017}.


\begin{table}
	\caption{Correspondence between notations in the current paper and in the
	literature.}
	\label{translation}
\begin{tabular}{cccc}
	\toprule
	\cite{Yang2017} & \cite{Lueck2020} & \cite{Yang2022} & This paper \\
	\midrule
	$p_i$  		& $p_i$ 	& $p_i$ 	& $P_i$\\
	$\neg$ 		& $\neg$	& $\neg$	& $\slnot$ \\
	$\bot$ 		& $\bot$ 	& $\bot$	& $\ibot$ \\
	$\otimes  $ & $\vee $	& $\vee$	& $ \ilor $ \\
	$\wedge$ 	& $\wedge$	& $\wedge$  & $\land $ \\
	$ \vee $ 	& $\ovee$	& $\ilor$   & $ \lor $ \\
	$\text{NE}$ & ~ 		& ~			& $\NB$ \\
	\bottomrule
\end{tabular}
\end{table}

\begin{defin}
The set of formulas $\Fm_{\text{PT}^+}$ of \emph{Strong propositional team logic},
$\text{PT}^+$,  is generated by the following grammar $$ \phi::= P_i
\mathrel|  \slnot P_i \mathrel|\ibot \mathrel| \NB\mathrel|
\phi \ilor \phi\mathrel|   \phi \land \phi \mathrel|  \phi \lor \phi $$ and
it's valuational team semantics can be described by defining the denotations
for formulas $\llbracket \phi \rrbracket \subseteq \P 2^\mathbb{N} $
recursively for cases of the main connective as follows:

\begin{align*}
	\llbracket  P_i \rrbracket &= \set{ X | \text {for all } s \in X, s(i)= 1}\\
	\llbracket \slnot P_i \rrbracket &= \set{ X | \text {for all } s \in X, s(i)= 0}\\
	\llbracket \ibot \rrbracket &= \set{ \emptyset}\\
	\llbracket \NB \rrbracket &= \set {X| X\neq \emptyset} \\
	\llbracket \phi \ilor \psi \rrbracket &= \set{ X\cup Y | X\in \llbracket \phi \rrbracket, Y\in \llbracket \psi \rrbracket} \\
	\llbracket \phi \land \psi \rrbracket &= \llbracket \phi \rrbracket \cap \llbracket \psi \rrbracket\\
	\llbracket \phi \lor \psi \rrbracket &= \llbracket \phi \rrbracket \cup \llbracket \psi \rrbracket\\
\end{align*}
The logical entailment of $\text{PT}^+$ is then defined as follows:
$$\text{PT}^+: \Delta \vDash \phi \quad \text{iff} \quad
\bigcap_{\delta \in \Delta}  \llbracket \delta \rrbracket \subseteq \llbracket
\phi \rrbracket.$$
Elements $s\in 2^\mathbb{N}$ are viewed as \textit{valuations} for the set of propositional variables, and sets of valuations $X\in \P 2^\mathbb{N}$ are reffered to as \textit{teams (of valuations)}. 
\end{defin}
There is no standard notation for the connectives in the literature, and we have
chosen notation that corresponds best to the notation for LT. Table
\ref{translation} indicates the correspondence between our notation and
notation elsewhere. 

As defined it is clear that for the set of formulas, including defined
connectives, we have that $\Fm_{PT^+}\subset
\Fm_{\text{LT}}$ even though the logics are described with different semantics
in mind. By reinterpreting the semantics of $\text{PT}^+$ we will be able to
find axioms to define it in LT. In the literature there are multiple weaker
propositional team logics described and studied, in particular propositional
logics of dependence \cite{Yang2016}. Many important logics however are given
as, or is expressively equivalent to, logics that can be given as fragments of
$\text{PT}^+$. This means that our results regarding the axiomatisation of
$\text{PT}^+$ in LT will be directly applicable for these logics too. Table
\ref{list_of_logics} describes what syntactical restrictions constitutes what
propositional logic in the literature. The logic $\text{PD}^\vee$ is
described in \cite{Yang2016}, and the others are described in \cite{Yang2017}.

\begin{table}
	\begin{tabular}{ll}
	\toprule
		Propositional team logics. 								& Connectives \\
		\midrule 
		Classical propositional logic (CPL) 					&$\slnot P_i,\ibot,\ilor,\land$\\
		Strong classical propositional logic ($\text{CPL}^+$) 	&$\slnot P_i,\ibot,\ilor,\land,\NB$\\
		Propositional union closed logic ($\text{PU}$)		&$\slnot P_i,\ibot,\ilor,\land,\circledast$ \\
		Strong propositional union closed logic ($\text{PU}^+$)&$\slnot P_i,\ibot,\ilor,\land,\circledast,\NB$ \\
		Propositional dependence logic w. int. disj. ($\text{PD}^\vee$)& $ \slnot P_i,\ibot,\ilor, \land,\lor$ \\
		Propositional team logic ($\text{PT}$)				& $\slnot P_i,\ibot,\ilor,\land,\lor,\circledast$ \\
		Strong propositional team logic ($\text{PT}^+$)		& $\slnot P_i,\ibot,\ilor,\land,\lor,\NB$\\
		\bottomrule
	\end{tabular} 
\caption{Names and included connectives of logics described in
\cite{Yang2016} and \cite{Yang2017} as fragments of $\text{PT}^+$. The
connective $\circledast$ can be defined in $\text{PT}^+$ as $\phi \circledast
\psi := (\phi \land \NB) \protect\ilor (\psi \land \NB)$. By $\slnot P_i$ we mean that
$\slnot$ is only allowed to be applied to propositional variables.}
\label{list_of_logics}
\end{table}

\subsection{Interpreted as a model of LT}
Observe that the denotations of valuational team semantics are elements of the
set $\P\P 2^{\mathbb{N}}$, which can be interpreted as a model of LT by
interpreting $\P 2^{\mathbb{N}}$ as a Boolean algebra using the standard set
operations. We will refer to this as \textit{the valuation model}. With this
reading we can see that the interpretation of atomic formulas imposes a
specific homomorphism that maps every atomic formula to the set of teams for
which every member evaluates it to true.

Define the \textit{valuation homomorphism} $H_V : \Fm_{\text{LT}} \to \P\P
2^\mathbb{N}$ as the unique homomorphism such that
\[ 
H_V(P_i) = \set { X\in \P 2^{\mathbb{N}} | \text{ for all } s\in X, s(i)= 1 }
= \P \set{ s\in 2^{\mathbb{N}} | s(i) = 1}.
\] 
Thus, 
\[
H_V :\Delta \vDash \phi \quad \text{iff} \quad \bigcap_{\delta \in
\Delta}H_V(\delta) \subseteq H_V(\phi).
\]
We can restate this result as in following theorem.

\begin{thm}\label{directThm}
	For all formulas $\Delta\cup \{\phi\} \subseteq \Fm_{\text{PT}^+}$:
	\[
		\text{PT}^+: \Delta \vDash \phi \quad \text{ iff } \quad  H_V
	: \Delta \vDash \phi.
	\]
	Hence, if  $ \text{LT} : \Delta \vDash\phi$, then $\text{PT}^+: \Delta \vDash \phi$.
\end{thm} 


\subsection{Axiomatising a specific class of homomorphisms}\label{Ax-class-hom}

We observe that the valuation homomorphism $H_V$ has the following special
property. 
\[ 
H_V (P_i)= \P X \text{ for some } X \in \P 2^{\mathbb{N}}
\]
Algebraically speaking, every propositional variable is mapped to a non-empty
principal ideal of the Boolean algebra on $2^\mathbb{N}$. By corollaries
\ref{flat_eqs} and \ref{flatness_formula} we can directly conclude that 
\[
H_V : {}\vDash \slnot \slnot P_i \to  P_i \quad \text{for all } i \in
\mathbb{N}.
\]
We will see that this is the crucial categorisation of the homomorphisms that
relate to valuational team logics. We therefore identify the class defined by
these formulas, and the corresponding axiomatisation as discussed in Section
\ref{sec:def-class}.

\begin{defin}
	Let $\mathcal{H}_{\text{FV}}$ denote the class of homomorphisms defined by
	$\set{\slnot \slnot P_i\to P_i | i\in \mathbb{N}}$. We say that a
	homomorphism $H$ \textit{has flat variables} iff
	$H\in\mathcal{H}_{\text{FV}}$. Furthermore, let \emph{the flat variable
	axioms} be the set
	\[
	\text{FVA}=\set{\Box (\slnot \slnot P_i \to P_i) |i\in	\mathbb{N}}.
	\]
\end{defin}
 
 It follows directly that for all $\Delta,\set{\phi}\subseteq \Fm_{\text{LT}}$
 \[
 	\mathcal{H}_{\text{FV}} :	\Delta \vDash \phi \quad \text{iff} \quad \text{LT}:\text{FVA},
	\Delta\vDash \phi.
\]
It is evident that $H_V\in \mathcal{H}_{\text{FV}}$. It is also clear that a
homomorphism with respect to a complete  Boolean algebra has flat variables if
and only if every variable is mapped to a non-empty principal ideal.

\subsection{Axiomatisation of $\text{PT}^+$ as a fragment}

In this section we prove that the axioms FVA axiomatises $\text{PT}^+$ in
the sense of the following theorem.
\begin{thm}\label{thm:mainAx}
	For all $\Delta \cup \{ \phi \} \subseteq \Fm_{\text{PT}^+}$ 
	\[ 
	\text{PT}^+: \Delta \vDash \phi \quad \text{iff} \quad \text{LT} : \text{FVA}, \Delta \vDash \phi.
	\]
\end{thm}

To prove this we first need the following lemma, regarding the
representability of homomorphisms in $\mathcal{H}_{\text{FV}}$ by the valuation
homomorphism $H_V$

\begin{lemma}\label{f_rep_in_Val}
For every Boolean algebra of the form $(\P S, \emptyset, \cdot^C, \cup
,\cap)$ and every homomorphism $H:\Fm_{\text{LT}} \to \P \P S$ with flat
variables there is a mapping $f:S \to 2^{\mathbb{N}}$  such that for all
formulas $\phi \in \Fm_{\text{PT}^+}$ and all $X\in \P S$:
\[ 
X\in H(\phi) \quad \text{iff} \quad f^*(X) \in H_V(\phi),
\] 
where $f^*:\P S \to \P 2^{\mathbb{N}}$ is defined by $f^*(X) = \set{f(x) | x
\in X}$, and $H_V:\Fm_{\text{LT}} \to \P\P 2^\mathbb{N}$ denotes the valuation
homomorphism.
\end{lemma}
\begin{proof}
	Assume $H:\Fm_{\text{LT}} \to \P \P S$ has flat variables. Let $f: S\to
	2^{\mathbb{N}}$ be defined by
	\[
		f(s)(i)= \begin{cases} 1 & \text{ if } \{s\} \in H(P_i) \\ 
								0 & \text{ if } \{s\} \notin  H(P_i).
					\end{cases}
	\]

	\begin{quote} 
	\textbf{Claim:} For all $X,Y\in \P S$, $f^*(X\cup Y) = f^*(X)\cup f^*(Y)$.
	Furthermore, if $f^*(X) = U\cup V$, then there exist $Y,Z\in \P S$ such
	that $f^*(Y)= U$, $f^*(Z)= V$ and $Y\cup Z= X$.
	\end{quote}

	The first part is self-evident by the definition of $f^*$ from $f$. For
	the second part, let $Y= \set{s\in X| f(s)\in U}$ and $Z= \set{s\in X
	| f(s) \in V}$.
	
	We can now prove that the function $f$ is what we looked for in the lemma
	by induction over formulas $\phi\in	\Fm_{\text{PT}^+}$.
	
	We have four types of base cases:  $\ibot, \NB, P_i, \slnot P_i $:
	\begin{itemize}
		\item By definition $X\in H(\ibot)$ iff $X= \emptyset$, and
		since $f^*(\emptyset) = \emptyset $ this case is evident.
		\item The case for $\NB$ is proven by contraposition of the previous
		case.
		\item By assumption $H(P_i)$ is flat in the complete atomic Boolean
		algebra $\P S$. Hence, for all $X \in H(P_i)$, by flatness, we have
		for all $s\in X$ that $\{s\} \in H(P_i)$. By construction then
		$f^*(\{s\}) \in H_V (P_i)$ for all $s\in X$, and thus since $H_V$ has
		flat variables and and the valuation algebra is atomic $f^*(X)\in
		H_V(P_i)$. The opposite direction is proven with a similar chain of
		arguments.
		\item The proof when $\phi = \slnot P_i$ is by first using Theorems
		\ref{slnot_form_to_subset} and \ref{basic_flatness} to assert that
		$H(\slnot P_i)$ is flat regardless of the homomorphism, the proof is
		similar to the previous case.
	\end{itemize}

	For the induction step we have three cases for the main connectives:
	$\land ,\lor,\ilor$
	\begin{itemize}
		\item Let $\phi = \psi \land \chi$. $X\in H(\phi)$ by definition if
		and only if $X\in H(\psi)$ and $X\in H(\chi)$. By induction
		hypothesis, we can conclude that is the case iff $f^*(X)\in
		H_V(\psi)$ and $f^*(X)\in H_V(\chi)$, which is equivalent to stating
		that $f^*(X) \in H_V(\phi)$.
		\item for $\phi = \psi \lor \chi$ similar to previous case.
		\item Let  $\phi = \psi \ilor \chi$. For one direction, assume $X \in
		H(\phi)$. Then there exists $Y\in H(\psi)$ and $Z\in H(\chi)$ such
		that  $X= Y\cup Z$. By induction hypothesis $f^*(Y)\in H_V(\psi)$ and
		$f^*(Z)\in H_V(\chi)$. By the first part of the Claim about $f^*$ we
		see that $f^*(X)= f^*(Y)\cup f^*(Z)$ and thus $f^*(X)\in H_V(\phi)$.
		
		For the other direction, assume $f^*(X) \in H_V(\phi)$.  Then there
		exist $U\in H_V(\psi), V\in H_V(\chi)$ such that $f^*(X)=U\cup V$.
		Then by the second part of the Claim about $f^*$, there exists $Z, Y$
		such that $f^*(Z)=U, f^*(Y)= V$ and $Z\cup Y= X$. By induction
		hypothesis therefore $Z\in H(\psi)$ and $Y\in H(\chi)$.  We conclude
		that $X\in H(\phi)$.
	\end{itemize}
	This concludes the proof of the lemma.
\end{proof}

We are now ready to prove Theorem \ref{thm:mainAx}.

\begin{proof}[Proof of Theorem \ref{thm:mainAx}] 
	The theorem is equivalent to the statement that for all $\Delta \cup \{
	\phi \} \subseteq \Fm_{\text{PT}^+}$ 
	\[
	\text{PT}^+:\Delta \vDash \phi \quad \text{iff} \quad
	\mathcal{H}_{\text{FV}} :  \Delta \vDash \phi.
	\]
	One direction follows directly from Theorem \ref{directThm} and the fact
	that $H_V\in\mathcal{H}_{\text{FV}} $. 

	The other direction is proved by contraposition. Assume
	$\mathcal{H}_{\text{FV}} :  \Delta \nvDash \phi$ and thus  $\text{LT} :
	\text{FVA}, \Delta \nvDash \phi$. Then by the adequacy ensured by Theorem
	\ref{thm:finite_bas} there exists a Boolean algebra with domain $\P S$, an
	homomorphism $H: \Fm_{\text{LT}} \to  \P\P S$ such that $ H:
	\text{FVA},\Delta \nvDash \phi.$ In other words, $H\in
	\mathcal{H}_{\text{FV}}$ and there exists a set $X\in \P S$ such that
	\[
	X\in \bigcap_{\delta\in \Delta} H(\delta), \text{ but } X\notin H(\phi).
	\]  
	We can then find a map $f$ as described in Lemma \ref{f_rep_in_Val} and
	conclude that 
	\[
	f^*(X) \in \bigcap_{\delta\in\Delta} H_V(\delta), \text{ but }
	f^*(X)\notin H_V(\phi).
	\]
	Thus, $H_V :\Delta \nvDash \phi$ and by Theorem \ref{directThm}: 
	\[
	\text{PT}^+:\Delta \nvDash \phi.
	\] 
	This finalises the proof of Theorem	\ref{thm:mainAx}.
\end{proof}

Observe that to evaluate $\text{PT}^+$ it is sufficient to consider the
valuational algebra $\P 2^\mathbb{N}$ i.e. for all  $\Delta, \{\phi\}
\subseteq\Fm_{\text{PT}^+}$ we have 
\[
\text{PT}^+ : \Delta \vDash \phi \quad\text{iff}\quad \P 2^\mathbb{N}:
\text{FVA},\Delta \vDash \phi.
\]
In this sense $\P 2^\mathbb{N}$ can be seen as canonical for $\text{PT}^+$. It
is however \textit{not} canonical for the logic in LT axiomatized by FVA,
since the canonicity only holds when the formulas are restricted to the
language $\Fm_{\text{PT}^+}$.

