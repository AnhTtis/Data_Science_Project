\documentclass[]{asl}
\expandafter\def\csname ver@amsmath.sty\endcsname{2019/01/01}
\expandafter\def\csname aligned@a\endcsname{}

\usepackage{stmaryrd}
\usepackage{braket}
\usepackage{mathtools}
\usepackage{enumerate}
\usepackage{proof}
\usepackage{booktabs}
\usepackage{rotating}
%\usepackage{showframe}

%\usepackage{mathabx}
%\usepackage{graphicx}
%\usepackage{cmll}
%\usepackage{fontspec}
%\usepackage[english]{babel}
%\usepackage{unicode-math}
%\usepackage{trimclip}
%\usepackage{rotating}
%\usepackage{framed}

%\usepackage[autostyle]{csquotes}
%\usepackage[backend=biber,style=alphabetic,url=true,doi=true,eprint=true]{biblatex}
%\addbibresource{refs.bib}

%Theorem environments
\newtheorem{thm}{Theorem}[section]
\newtheorem{prop}[thm]{Proposition}
\newtheorem{lemma}[thm]{Lemma}
\newtheorem{cor}[thm]{Corollary}
\theoremstyle{definition}
\newtheorem{defin}[thm]{Definition}

%\newtheorem{conj}[thm]{Conjecture}
%\newtheorem{que}{Question}
%\newtheorem{example}{Example}

%A few new commands
\renewcommand\P{\mathop{\mathcal P}}
\newcommand{\den}[2][a]{\llbracket #2 \rrbracket_{#1}}
\renewcommand\phi\varphi % Redefine \phi as \varphi to disginguish it better from \emptyset
\newcommand\dep{\mathtt{D}}
\newcommand\ruleskip{\qquad\qquad}

%Better marginpar command
%\let\oldmarginpar\marginpar
%\renewcommand\marginpar[1]{\-\oldmarginpar[\raggedleft\footnotesize #1]%
%{\raggedright\footnotesize #1}}

%Internal Boolean operators.
\makeatletter
%\let\oldlor\lor
%\let\oldland\land
%\renewcommand{\land}{\mathrel{\vphantom{\oldland}\mathpalette\land@\relax}}
%\renewcommand{\lor}{\mathrel{\vphantom{\oldlor}\mathpalette\lor@\relax}}
\newcommand{\iland}{\mathrel{\vphantom{\land}\mathpalette\iland@\relax}}
\newcommand{\ilor}{\mathrel{\vphantom{\lor}\mathpalette\ilor@\relax}}
\newcommand{\ilnot}{\mathord{\vphantom{\lnot}\mathpalette\ilnot@\relax}}
\newcommand{\ibot}{\mathord{\vphantom{\bot}\mathpalette\ibot@\relax}}
\newcommand{\itop}{\mathord{\vphantom{\top}\mathpalette\itop@\relax}}
%\newcommand{\land@}[2]{\ooalign{\rotatebox[origin=c]{90}{$\m@th#1>$}}}
%\newcommand{\lor@}[2]{\ooalign{\rotatebox[origin=c]{-90}{$\m@th#1>$}}}
\newcommand{\iland@}[2]{\ooalign{\raisebox{.15ex}{\rotatebox[origin=c]{-90}{$\m@th#1\leqslant$}}}}
\newcommand{\ilor@}[2]{\ooalign{\raisebox{.15ex}{\rotatebox[origin=c]{-90}{$\m@th#1\geqslant$}}}}
\newcommand{\ibot@}[2]{\ooalign{$\m@th#1\bot$\cr\kern.2em$\m@th#1\bot$}}
\newcommand{\itop@}[2]{\ooalign{$\m@th#1\top$\cr\kern.2em$\m@th#1\top$}}
%\newcommand{\ilnot@}[2]{\ooalign{$\m@th#1\neg$\cr\kern.2em$\m@th#1\neg$}}
\newcommand{\ilnot@}[2]{\ooalign{\raisebox{.2ex}{$\m@th#1\neg$}\cr\raisebox{-.2ex}{$\m@th#1\neg$}}}
\makeatother

% Auxillary operators

\newcommand\Fm{\ensuremath{\mathtt{Fm}}}
\newcommand\Lb{\ensuremath{\mathtt{Lb}}}

\newcommand\of{\mathbin:}
\newcommand\ekv\leftrightarrow
\newcommand{\up}{\mathop{\uparrow}}
\newcommand{\dwn}{\mathop{\downarrow}}
\newcommand{\NB}{\mathord{\text{NB}}}
\newcommand{\NT}{\mathord{\text{NT}}}
\newcommand{\slnot}{\mathord{{\sim}}}

\renewcommand{\oe}{\ensuremath{\lor\mathtt{E}}}
\newcommand{\oi}{\ensuremath{\lor\mathtt{I}}}
\renewcommand{\ae}{\ensuremath{\land\mathtt{E}}}
\newcommand{\ai}{\ensuremath{\land\mathtt{I}}}
\newcommand{\ioe}{\ensuremath{\mathord{\ilor}\mathtt{E}}}
\newcommand{\ioi}{\ensuremath{\mathord{\ilor}\mathtt{I}}}
\newcommand{\iae}{\ensuremath{\mathord{\iland}\mathtt{E}}}
\newcommand{\iai}{\ensuremath{\mathord{\iland}\mathtt{I}}}
\renewcommand{\ne}{\ensuremath{\lnot\mathtt{E}}}
\renewcommand{\ni}{\ensuremath{\lnot\mathtt{I}}}
\newcommand{\ine}{\ensuremath{\ilnot\mathtt{E}}}
\newcommand{\ini}{\ensuremath{\ilnot\mathtt{I}}}
\newcommand{\be}{\ensuremath{\bot\mathtt{E}}}
%\newcommand{\bi}{\ensuremath{\bot\mathtt{I}}}
\newcommand{\ibe}{\ensuremath{\ibot\mathtt{E}}}
\newcommand{\ibi}{\ensuremath{\ibot\mathtt{I}}}
\newcommand\taut{\ensuremath{\mathtt{taut}}}
\newcommand\sub{\ensuremath{\mathtt{sub}}}
\newcommand{\orr}{\ensuremath{\lor\mathtt{R}}}
\newcommand{\orl}{\ensuremath{\lor\mathtt{L}}}
\newcommand{\ar}{\ensuremath{\land\mathtt{R}}}
\newcommand{\al}{\ensuremath{\land\mathtt{L}}}
\newcommand{\nl}{\ensuremath{\lnot\mathtt{R}}}
\newcommand{\nr}{\ensuremath{\lnot\mathtt{L}}}
\newcommand{\ior}{\ensuremath{\ilor\mathtt{R}}}
\newcommand{\iol}{\ensuremath{\ilor\mathtt{L}}}
\newcommand{\iar}{\ensuremath{\iland\mathtt{R}}}
\newcommand{\ial}{\ensuremath{\iland\mathtt{L}}}
\newcommand{\inl}{\ensuremath{\ilnot\mathtt{R}}}
\newcommand{\inr}{\ensuremath{\ilnot\mathtt{L}}}
\newcommand{\raa}{\ensuremath{\mathtt{RAA}}}

\newcommand{\ti}{\ensuremath{\top\mathtt{I}}}


\title{The propositional logic of teams} 
\author{Engström, Fredrik \and Lorimer Olsson, Orvar}
\date{\today}

\begin{document}

\begin{abstract}
Since the introduction by Hodges, and refinement by Väänänen, team semantic
constructions have been used to generate expressively enriched logics still
conserving nice properties, such as compactness or decidability. In contrast,
these logics fail to be substitutional, limiting any algebraic treatment, and
rendering schematic uniform proof systems impossible. This shortcoming can be
attributed to \emph{the flatness principle}, commonly adhered to when
generating team semantics.

Investigating the formation of team semantics from algebraic semantics, and
disregarding the flatness principle, we present \emph{the logic of teams}, LT,
a substitutional logic for which important propositional team logics are
axiomatisable as fragments. Starting from classical propositional logic and
Boolean algebras, we give semantics for LT by considering the algebras that
are powersets of Boolean algebras $B$, i.e.,of the form $\P B$, equipped with
\emph{internal} (point-wise) and \emph{external} (set-theoretic) connectives.
Furthermore, we present a well-motivated complete and sound labelled natural
deduction system for LT.
\end{abstract} 

\maketitle

%!TEX root = logic_of_teams.tex
\section{Introduction}

Team semantics was invented by Hodges \cite{Hodges1997} to give the
Independence Friendly logic (IF-logic) of Hintikka and Sandu
\cite{Hintikka1989} a compositional semantics. Team semantics was later used
by Väänänen to define Dependence logic \cite{Vaeaenaenen2007}, a formalism
extending first-order logic in which functional dependencies between variables
are explicitly expressed by atomic formulas. The intended meaning of these
atomic formulas $\dep(\bar x,y)$ is that the value of the variable $y$ is
functionally determined by the values of the finitely many variables $\bar x$.
Even though Dependence logic uses only first-order quantifiers it can express
any existential second-order property or statement. The reason is that there
are second-order quantifiers hidden in the truth conditions, most
interestingly in the truth conditions for $\lor$ and $\exists$.

Since then many logics based on team semantics have been introduced and
investigated, such as Independence logic, Propositional dependence logics and
Modal dependence logics.

\subsection{Lifting Tarskian semantics to team semantics}

In the classical Tarskian semantics of first-order logic the denotation of a
formula, given a structure, is defined to be the set of all assignments that
satisfies the formula. Similarly, the denotation of a propositional formula is
the set of valuations satisfying the formula, and the denotation of a modal
logic formula, given a Kripke model, is the set of all worlds satisfying the
formula. Thus, the denotation of a formula in this classical setting is an
element of $\P X$, the powerset of the set $X$ of all assignments, all
valuations or all worlds, respectively.

Team semantics of first-order, propositional and modal logic lifts the
denotations of formulas to be sets of subsets of $X$, i.e., elements of $\P\P
X$, instead of elements of $\P X$. Thus, instead of asking if a single
assignment, valuation or world satisfies a formula, team semantics asks if a
\emph{set} of assignments, valuations or worlds satisfies a formula. Such sets
are called \emph{teams}. In this sense, the team semantical denotation is the
image of the classical denotation under a specific function, a lift, from $\P
X$ to $\P \P X$.

For most team semanticists the \emph{flatness principle} has guided the
definition of the semantics. This principle can be formulated as follows:
\begin{quote}
    A team satisfies a formula iff every individual assignment, valuation, or
    world in the team satisfies the formula in the usual classical sense.
\end{quote}
Note that this only applies to formulas in the original language, without any
dependence atoms or other added connectives; for these formulas it make sense
to say that an assignment/valuation/world satisfies a formula. 
%\marginpar{ XXX
%maybe a more informative sentence also incorporating how establishing flatness
%has been a common step in formulatin team semantics conservatively extending a
%language /OLO} 
Flatness directly gives conservativity in the sense that the
singleton team of the empty assignment satisfies a sentence in a model iff the
sentence is true in the classical sense in the model.

In terms of denotations of formulas the flatness principle naturally
translates to the equation
\begin{equation}\label{eq:powerset}
    \den[h]{\phi} =\P \den[c]{\phi}
\end{equation} 
where $\den[h]{\phi}$ is the team semantical denotation of $\phi$ ($h$ for
Hodges) and $\den[c]{\phi}$ is the ordinary Tarskian denotation of $\phi$ ($c$
for classical). By subscribing to the flatness principle most team
semanticists also subscribe to this powerset lift of classical semantics to
team semantics.

\subsection{Substitutionality and logics}

Dependence logic and its variants have some nice properties such as
compactness, Löwenheim-Skolem properties \cite{Vaeaenaenen2007} and that 
first-order consequences of theories can be axiomatized \cite{Kontinen2013} to
name a few. But they are not substitutional; for example $$\forall x\, (P(x)
\lor P(x))\vDash \forall x\, P(x) \text{, but } \forall x\, (\dep(x) \lor
\dep(x)) \nvDash \forall x\, \dep(x),$$
where $\dep(x)$ is the dependence atom with only one free variable, intuitively
stating that there is only one value of $x$ in the team, i.e., that $x$ is
constant.

In general, a logic is substitutional if $\varphi \vDash \psi$ implies
$\phi[\sigma/p] \vDash \psi[\sigma/p]$, i.e., an entailment does not break by
substituting a formula $\sigma$ for an atom $p$. Substitutionality was already
used by Bolzano to define the concept of validity. In \cite{Bolzano1837}
Bolzano defines \emph{universally valid propositions} as propositions with all
their variants true. In his terminology a variant is nothing but a
substitutional instance. Bolzano, and many after him, thus took
substitutionality not only as an important property for a logic, but as the
basic principle for the concept of logical validity.

Any attempt to algebraise a logic that is not substitutional
will not work in the strict sense since the elements of the algebraic
structure need to be partitioned, or divided, into different types. Indeed,
the examples of algebraisations of variants of propositional dependence logic
that appear in the literature
\cite{Quadrellaro2021,Quadrellaro2020,Bezhanishvili2021} all introduce
additional predicates on the resulting algebras in order to partition the
elements into different types.

The powerset lift of the Tarskian semantics to team semantics, as in
\eqref{eq:powerset}, could be blamed for the lack of substitutionality in
Dependence logic. In fact, any specific lift will limit the possible
denotations of atoms and the resulting logics will not, in general, be
substitutional.

In this paper we will therefore generalize away from using a specific lift.
Instead we will, in a certain sense, quantify over all possible lifts of the
atoms: Given a function $\mathcal L : \P X \to \P\P X$ that lifts
the denotations of atoms from a Tarskian setting to a team semantical setting,
we extend this compositionally to give team semantical denotations
$\den[\mathcal L]{\phi}$ to all formulas $\phi$ in our logic. We define
the entailment relation by $\phi \vDash \psi$ if $\den[\mathcal L]{\phi}
\subseteq \den[\mathcal L]{\psi}$ for all functions $\mathcal L : \P X \to
\P\P X$.

The function $\phi \mapsto \den[\mathcal L]{\phi}$ is nothing but a
homomorphism from the absolutely free term algebra of formulas to a specific
algebraic structure of sets of teams. Describing a logic by quantifying over 
all homomorphisms from a term language into an algebraic structure is exactly 
the starting point for constructing algebraic semantics.

\subsection{Lifting algebraic semantics}

The semantics of classical propositional logic can be defined in terms of
Boolean algebras. One may even say that classical propositional logic
\emph{is} the logic of Boolean algebras. The set of propositional formulas is
the term algebra generated by atoms using the Boolean operators $\bot, \lnot,
\lor$, and $\land$; and the semantics of propositional logic can be stated
using homomorphisms from this term algebra to a Boolean algebra. A formula is
a tautology if its image under any such homomorphism is the top element of the
Boolean algebra. This is the starting point when we define the Logic of Teams,
or LT for short: Lifting the algebraic semantics for classical propositional
logic to the setting of teams. A team in this setting is nothing but a set of
elements of the Boolean algebra.

The connectives, and the corresponding operators on the sets of teams, we are
interested in are the ordinary Boolean connectives $\bot,\lnot, \lor,\land$
that correspond to the empty set, the complement, the union and the
intersection. We will call these connectives and the corresponding operators
\emph{external} Boolean connectives and operators. We will also add the
\emph{internal} connectives and operators that are defined by pointwise
application of  the operators of the Boolean algebra, for example $$A \ilor B
= \set{a \lor b | a \in A, b \in B},$$ where $\ilor$ is the internal
disjunction. We will use blackbord boldface versions of the Boolean
connectives to denote these internal connectives: $\ibot,\ilnot,\ilor,\iland$.
When the underlying Boolean algebra is $\P 2^{\mathbb N}$ with the
set-theoretic Boolean operators, the connective $\ilor$ is exactly the
``splitjunction'' used in Dependence logic, and $\iland$ is, in Dependence
logic, equivalent to conjunction.

\subsection{Related constructions} 

An algebraic structure generated by the powerset of a Boolean algebra with
internal connectives is not new to mathematics and logic. Brink
\cite{Brink1984,Brink1986,Brink1993} contributes to these investigations and
calls them \emph{power algebras}, whereas Goldblatt calls them \emph{complex
algebras} \cite{Goldblatt1989} referring to a subset of an algebraic group as
a complex. It is also worth noting that these algebras are special cases of
\emph{Boolean algebras with operators} as described by J\'onsson and Tarski
\cite{Jonsson1993}. These play a notable role in the algebraic treatment of
Modal logic, see \cite{Venema2007}.

In other related work, Priest has utilised the powerset lift in order to
investigate resulting families of \textit{plurivalent logics}
\cite{Priest2017}, an effort elaborated on by Humberstone in
\cite{Humberstone2014} coining the term \textit{power matrices} for the
resulting constructions. In these papers, semantics of multivalued logics is
lifted into evaluations on subsets of possible truth values, and the logical
connectives are interpreted in terms of pointwise operations. They do,
however, not include any connectives relating to the set-theoretic Boolean
operations on the powerset algebra.

In particular, in \cite{Goranko99} Goranko and Vakarelov use power sets of
Boolean algebras, referred to as \textit{hyperboolean algebras}, in order to
define what they call the \textit{hyperboolean modal logic}, HBML. This
construction treats a Boolean algebra, expressed as a partial order, as a
Kripke frame and utilise the Boolean structure to define modal operators.
% the following unary and trinary accessibility relations given for all $b,c,d
% \in B$
% \begin{itemize}
% 	\item $R_\bot(b)$ iff $b=0$
% 	\item $R_\to(b,c,d)$ iff $b= c\to d$ 
% \end{itemize} 
% These relations result in a Kripke-style semantics in terms of satisfacion
% with the constant and binary Modal operators $\langle 0\rangle$ and $\langle
% \to \rangle$ for which the defining clauses in the definition of satisfaction
% $M,b \Vdash \phi$ for all Models $(B,V)$ and all points $b\in B$:
% \begin{itemize}
% 	\item $M,b \Vdash \langle 0 \rangle$ iff $b=0$ 
% 	\item $M,b \Vdash \phi \langle \to \rangle \psi $ iff there is $c,d \in B$
% 	s.t.{} $b= c\to d$, $M,c \Vdash \phi $ and $M,d \Vdash \psi.$
% \end{itemize}
% \fi 
The algebraic counterpart of the logic HBML is thus the powerset algebra of
the original Boolean algebra, with the modal operators defined as the internal
pointwise operations on the underlying algebra.

% \iffalse  $\langle 0 \rangle$ and $\langle \to \rangle$ identified with the
% pointwise application of the operations in the underlying algebra.

% Goranko and Vakarelov also identifies some connectives definable in HBML, in
% particular a unary difference modality $\langle \neq \rangle $ and a unary
% only operator $\mathcal{O}$ with quite hairy definitions in HBML, but fairly
% simple semantics interpretations
% \begin{itemize}
% 	\item $M,b \Vdash \langle \neq \rangle \phi$ iff there is $c\in B$ s.t.
% 	$c\neq b$ and $M,c \Vdash \phi $.
% 	\item $M,b \Vdash \mathcal{O} \phi $ iff for all $c\in B$: $M, c\vDash
% 	\phi$ only when $c=b$.
% \end{itemize}

% They present a sound and complete Hilbert-style system for validity utilising
% the difference modality to hinge on previous work, and the only operator to
% axiomatise the underlying Boolean structure.

The logic of teams that we define in this paper has different motivation and
origin than that of HBML, but building on the same class of models. It is easy
to see that the connectives of LT and HBML are  interdefinable such that the
two logics share validities and can in this sense be viewed as having the same
theorems. However, where Goranko and Vakarelov only defines HBML in terms of
validity with a proof system fundamentally structured around an elaborately
defined \textit{difference modality} and an \textit{only operator}, we define
LT for a full entailment notion and present a labelled natural deduction
system for which the rules directly correspond to the basic connectives of the
logic. Even so, the correspondence between the two logics is strong enough for
some important properties of HBML presented in \cite{Goranko99} to also apply
to LT, see Section \ref{sec:canonical}.

\subsection{Structure of this paper} 

In the next section we formally introduce the syntax and semantics of the
Logic of Teams, LT. The semantics is defined in an algebraic manner in terms
of homomorphisms into algebraic structures based on Boolean algebras. 
% This is
% done in the most natural way and mirrors the interpretations of team semantics
% in respect to its denotation. 
In the same section we also introduce a labelled
natural deduction system for LT. The formulas are decorated by labels and the
labels are themselves classical propositional formulas. By including rules in
the deduction system identifying classically equivalent formulas as equivalent
labels we establish the role of the labels as references to elements in a
Boolean algebra, and this paves the way for the completeness proof via
Lindenbaum--Tarski algebras presented in the next section.

Section 3 is devoted to prove the completeness of the natural deduction system
and the consequences that can be observed by a more careful investigation of
the proof. This section also includes results regarding non-canonicity and
adequacy of sets of Boolean algebras.

In Section 4 we introduce some important definable connectives in LT. In
particular we define the \textit{strict negation} that, apart from being an
interesting type of negation, will be important in Section 6. We also define a
universal $\Box$-modality, which plays an important role in Section 5, where
definable classes of homomorphisms and axiomatisations of their corresponding
logics are discussed.

In section 6 we finally utilise the strict negation to define classes of
homomorphisms, and give axioms expressing the strong propositional team logic,
PT$^+$ in \cite{Yang2017}, as a part of LT. In this way we establish the
connection between LT and the propositional team logics found in the
literature with semantics based on teams of valuations, here referred to as
\textit{valuational team semantics}. These results establish LT as a well
motivated, substitutional, and expressively rich propositional team logic that
is highly relevant for a better understanding of team semantics for
propositional logics. 

In the final section of the paper we reflect generally on the construction
that we have presented, and discuss some further topics of investigation that
are implicated by our work.
%!TEX root = logic_of_teams.tex

\section{The Logic of Teams}

Let us now define the Logic of Teams, LT, that is our main object of study in
this paper.

Let $B$ be a Boolean algebra, allowing for $B$ to be the trivial, one element,
Boolean algebra. $\P B$, the set of all subsets of $B$, is the domain of a
Boolean algebra with respect to the usual set-theoretic operators $(\emptyset,
\cdot^C, \cup, \cap)$ with some additional operators describable; the
internal, or point-wise operators $\ibot$, $\ilnot$, $\ilor$, and $\iland$.
These operators are defined as follows:
\begin{align*}
    \ibot &= \set{\bot}, \\
    \ilnot X &= \set{\lnot a | a \in X},\\
    X \ilor Y &= \set{a \lor b | a \in X, b \in Y}, \text{ and}\\
    X \iland Y &=\set{a \land b | a \in X, b \in Y}.
\end{align*}
Where $X,Y \subseteq B$ and $a \lor b \in B$ is the result of the operator
$\lor$ in the Boolean algebra $B$ applied to $a$ and $b$.

Formulas of LT are elements in the term algebra (absolutely free algebra) generated
by the propositional variables $P_0$, $P_1$, \dots\ using the external Boolean
connectives $\bot$, $\lnot$, $\lor$, $\land$ and the internal Boolean
connectives $\ibot$, $\ilnot$, $\ilor$, $\iland$. We will denote this term
algebra by $\Fm$:

$$\phi ::= \bot \mathrel| \ibot \mathrel| P_i \mathrel| \lnot \phi
\mathrel| \ilnot \phi \mathrel| \phi \lor \phi \mathrel| \phi \land \phi
\mathrel| \phi \ilor \phi \mathrel| \phi \iland \phi $$

Entailment is defined as the subset relation for the images of the
formulas under arbitrary homomorphisms.

\begin{defin}
    Let $\Delta$ be a set of formulas, then $\Delta \vDash \psi$ iff for all
    Boolean algebras $B$ and all homomorphisms $H:\Fm \to \P B$:
    $$\bigcap_{\phi \in\Delta} H(\phi) \subseteq H(\psi).$$
\end{defin}

Note that this definition of entailment does not correspond to the standard
entailment in algebraic logic in which $\phi$ entails $\psi$ if $H$
is a homomorphism with $H(\phi)=B$ then $H(\psi)=B$. The fact that $\ibot
\nvDash \ilnot \ibot$ shows this. More on this in Section \ref{sec:definable}.

Substituition instances of formulas are nothing but images under
homomorphisms: $\sigma : \Fm \to \Fm$ and thus substitutionality follows directly
from the definition of the entailment relation.

\begin{thm}[Substitutionality]
    If $\Delta \vDash \psi$ then $\sigma(\Delta) \vDash \sigma(\psi)$ for
    every algebra homomorphism $\sigma:\Fm \to \Fm$.
\end{thm}
\begin{proof}
    Follows directly from the definitions, and the fact that if $H:\Fm \to \P B$
    is a homomorphism then so is $H\circ \sigma$.
\end{proof}

%We will denote this logic by LT as a short-hand for the Logic of Teams. 

\subsection{Natural deduction} \label{deduction-system}

To define a deductive system for LT we will introduce \emph{labels} and
\emph{labelled formulas}. The labels have their intended reading as elements
in a Boolean algebra and the formulas as subsets of the same Boolean algebra.

Formally, labels are classical propositional formulas: 
$$a::= \bot \mathrel|  p_i \mathrel| \lnot a \mathrel| a \lor a \mathrel| a \land a$$

The propositional variables $p_i$ are called \emph{atomic labels}. The term
algebra of labels is denoted by $\Lb$.

A \emph{labelled formula} is of the form $a\of \phi$ where $a$ is a label and
$\phi$ a formula. The intended meaning of which is that $a$ is an element of
$\phi$, i.e., that under any homomorphisms $h$ and $H$, $h(a) \in H(\phi)$.
Thus, we define entailment in the natural way:

\begin{defin}
    Let $\Gamma$ be a set of labelled formulas, then $ \Gamma \vDash b\of \psi$
    iff for all Boolean algebras $B$ and all homomorphisms $h: \Lb \to B$,
    $H:\Fm \to \P B$ if $h(a) \in H(\phi)$ for all $a\of \phi \in
    \Gamma$ then $h(b) \in H(\psi)$.
\end{defin}

We say that homomorphisms $H$ and $h$ make $a\of \phi$ true if $h(a) \in H(\phi)$.

\begin{prop}
    $\Delta \vDash \psi$ iff $p\of \Delta \vDash p\of \psi$, where $p$ is an atomic
    label and $p\of \Delta = \set{p\of \phi | \phi \in \Delta}$.
\end{prop}
\begin{proof}
    Follows directly from the definitions.
\end{proof}

We will now present a sound and complete proof system for the relation $\Gamma
\vDash b\of \psi$, where $\Gamma$ is a set of labelled formulas. For this we will
use the usual shorthands $a \leftrightarrow b$ for $(a\lor
\lnot b)\land (\lnot a \lor b)$ and $\itop$ for $\ilnot \ibot$. Also $a=b$ is a shorthand for the labelled formula $a \leftrightarrow b\of  \itop$. Note that 
if $H$ and $h$ are homomorphisms that make $a=b$ true then $h(a)=h(b)$.

\subsubsection{Rules for external Boolean connectives}

The following rules are the standard rules for propositional logic, with
labels added to the formulas.

$$\infer[\ai]{a\of \phi\land \psi}{a\of \phi & a\of  \psi} \ruleskip
    \infer[\ae]{a\of \phi}{a\of \phi \land \psi}\ruleskip
    \infer[\ae]{a\of \psi}{a\of \phi \land \psi}$$

$$\infer[\oi]{a\of \phi\lor \psi}{a\of \phi} \ruleskip
    \infer[\oi]{a\of \phi\lor \psi}{a\of \psi} \ruleskip
    \infer[\oe]{b\of \sigma}{a\of \phi \lor \psi & \infer*{b\of \sigma}{[a\of \phi]} & \infer*{b\of \sigma}{[a\of \psi]}}
$$

$$\infer[\ni]{a\of \lnot \phi}{\infer*{b\of \bot}{[a\of \phi]}} \ruleskip
    \infer[\ne]{a\of  \bot}{a\of \phi & a\of \lnot\phi} $$

%$$\infer[\bi]{\bot\of \bot}{\bot\of \itop} \ruleskip 

$$\infer[\raa]{a\of \phi}{\infer*{b\of \bot}{[a\of \lnot\phi]}} \ruleskip \infer[\be]{b\of \phi}{a\of \bot} $$

%Note that $\bi$ is only sound for \emph{non-trivial} Boolean algebra's.

\subsubsection{Rules for internal Boolean connectives}

Next we add rules for the internal operations. The rule $\iae$ is modelled on
the fact that $x \in H(\phi \iland \psi)$ iff there are $y,z$ such that $x
\in H(\phi)$, $y \in H(\psi)$ and $x=y \land z$ holds in the Boolean algebra $B$. Similar for
$\ioe$. 
$$\infer[\iai]{a \land b\of  \phi \iland \psi}{a\of \phi & b\of  \psi}
    \ruleskip
    \infer[\iae]{b\of \sigma}{a\of \phi \iland \psi & \infer*{b\of \sigma}{\deduce{[a=p\land
q]}{\deduce{[q\of \psi]}{[p\of \phi]}}}}$$ 

$$\infer[\ioi]{a \lor b\of  \phi \ilor \psi}{a\of \phi & b\of  \psi}
    \ruleskip
    \infer[\ioe]{b\of \sigma}{a\of \phi \ilor \psi & \infer*{b\of \sigma}{\deduce{[a=p\lor q]}{\deduce{[q\of \psi]}{[p\of \phi]}}}}$$ 

$$\infer[\ini]{\lnot a\of \ilnot \phi}{a\of \phi} \ruleskip
    \infer[\ine]{\lnot a\of \phi}{a\of \ilnot \phi}$$

Note that vacuous discharges are allowed in the elimination rules for $\ilor$
and $\iland$.

%$$\infer[\ibe]{a=\bot}{a\of \ibot} \ruleskip \infer[\ibi]{a\of \ibot}{a=\bot}$$

\subsubsection{Rules for labels}

We add two rules for dealing with labels.

$$\infer[\taut]{b\of  \itop}{a_1 \of  \itop & \cdots & a_k\of  \itop} \ruleskip
    \infer[\sub]{a\of \phi}{a =b & b\of \phi}$$

The rule $\taut$ is only applicable if $a_1, \ldots, a_k \vdash b$ in
classical propositional logic, i.e., if $a_1 \land \ldots
\land a_k \rightarrow b$ is a tautology. Note that we allow for the special
case when $k=0$ in this rule.

In $\iae$ and $\ioe$ the $p$ and $q$ are distinct propositional variables not
occurring in any uncancelled assumptions, nor in $a$ or $b$. This
distinctiveness criterion is needed as otherwise the derivation
$$\infer[\iae_1^\ast]{a\of P\land Q} { a\of P
\iland Q
    & 
    \infer[\sub]{a\of P\land Q}
        {
        \infer[\taut]{a= b}{[a = b \land b ]^1} 
        &
        \infer[\ai]{b\of P\land Q}{[b\of P]^1 & [b\of Q]^1} 
        }
    }  
$$
would be valid, but $a\of P \iland Q \nvDash a\of P\land Q $.

For an example of a non-trivial derivation of an entailment, see Figure
\ref{fig:derivation}.

It is easy to see that the derivability relation satisfies the following lemma: 

\begin{lemma}\label{lem:basicfacts} 
$\Gamma \vdash a\of \phi$ iff $\Gamma,a\of
\lnot \phi \vdash a\of \bot$.
%    \item $\lnot a\of  \ilnot \phi \dashv\vdash a\of \phi$ 
%    \item $\lnot a\of  \phi \dashv\vdash a\of  \ilnot \phi$
%    \item 
%\end{enumerate}
\end{lemma}

\begin{thm}[Soundness]
    If $\Gamma \vdash a\of \phi$ then $\Gamma \vDash a\of \phi$.
\end{thm}
\begin{proof}
    Follows from a straight-forward induction on proof trees. For the case of
    $\iae$, $\ioe$, $\taut$ and $\sub$ note that $h(a)\in H(\itop)$ implies
    that $h(a)$ is $\top$, the top element of $B$. Thus, if $h(a
    \leftrightarrow b) \in H(\itop)$ then $h(a)=h(b)$.
\end{proof}

%!TEX root = logic_of_teams.tex


\section{Completeness and adequacy}

Next we will prove the completeness of the proof system, but first we need some 
definitions and a few easy facts.

Observe that if $\Gamma \vdash \bot\of \bot$  then $\Gamma \vdash a\of \phi$
for all labels $a$ and all formulas $\phi$. We say that $\Gamma$ is
\emph{consistent} if $\Gamma \nvdash \bot\of \bot$ or equivalently that there
are $a$ and $\phi$ such that $\Gamma \nvdash a\of \phi$.

\begin{prop}\label{prop:max}
    If $\Gamma$ is consistent then so is either $\Gamma,a\of \phi$ or
    $\Gamma, a\of \lnot \phi$.
\end{prop}
\begin{proof}
    Directly by using the $\ni$ and $\ne$ rules.
\end{proof}

In the next proposition remember that if $a$ and $b$ are labels then $a=b$ denotes the labelled formula $a \leftrightarrow b : \itop$.

\begin{prop}\label{prop:resp}
    If $\Gamma,a\of \phi\ilor \psi$ is consistent and $p$ and $q$ are
    propositional variables not occurring in $\Gamma$, nor in $a$, then
    $\Gamma, a\of \phi\ilor \psi,p\of \phi,q\of \psi,a = p \lor q$ is
    consistent. And similarly for $\iland$.
\end{prop}
\begin{proof}
    Directly by using the $\ioe$ rule and the $\iae$ rule.
\end{proof}

\begin{defin}
    $\Gamma$ is \emph{$\ilor$-saturated} if whenever $a\of \phi\ilor
    \psi \in \Gamma$ then there are labels $b$ and $c$ such that $\set{b\of
    \phi,c\of \psi,a = b \lor c} \subseteq
    \Gamma$. The dual notion of \emph{$\iland$-saturation} is defined
    similarly. We say that $\Gamma$ is \emph{saturated} if it is both
    $\ilor$-saturated and $\iland$-saturated.
\end{defin}

% \begin{prop}
%     If $\Gamma$ is consistent then the set $T=\set{a | a\of  \itop \in \Gamma}$ of labels is
%     consistent in propositional logic as a set of propositional formulas.
% \end{prop}
% \begin{proof}
%         If $T$ is inconsistent then by  $\taut$ we get that $\Gamma \vdash
%         \bot \of  \ibot$ and so $\Gamma \vdash \bot\of \bot$ by $\bi$ and is thus
%         inconsistent.
% \end{proof}

\begin{lemma}
Let $\Gamma$ be a set of labelled formulas and $\Gamma'$ be the result of
renaming the atomic labels $p_i$ by $p_{2i}$. Then 
\begin{enumerate}
    \item $\Gamma$ is consistent iff $\Gamma'$ is, and
    \item $\Gamma$ is satisfiable iff $\Gamma'$ is.
\end{enumerate}
\end{lemma}
\begin{proof}
\begin{enumerate}
    \item 
    If $\Gamma \vdash \bot$ then the same derivation with all atomic labels
    renamed shows that $\Gamma' \vdash \bot$. On the other hand if $\Gamma'
    \vdash \bot$, there is a finite subset $\Gamma^f$ of $\Gamma'$ such that
    $\Gamma^f \vdash \bot$. By renaming the atoms $p_{2i}$ by $p_i$ and
    $p_{2i+1}$ by $p_{i+k}$ where $k$ is large enough, i.e., larger than all
    indices of atoms occurring in $\Gamma^f$, we see
    that $\Gamma \vdash \bot$.
    \item Given a Boolean algebra $B$ and homomorphisms $H$ and $h$
    such that $h(a) \in H(\varphi)$ for all $a:\varphi \in \Gamma$ we can
    define $h'$ by setting $h'(p_{2i}) = h(p_i)$ and $h'(p_{2i+1}) = \bot \in B$. Then
    $B,H,h'$ satisfies $\Gamma'$. Also, if $B,H,h$ satisfies $\Gamma'$ then
    clearly for $h'$ defined by $h'(p_{i})=h(p_{2i})$ we have that  $B,H,h'$ satisfies
    $\Gamma$.
\end{enumerate}
\end{proof}

\begin{thm}[Completeness]
    If $\Gamma \vDash a\of \phi$ then $\Gamma \vdash a\of \phi$.
\end{thm}
\begin{proof}
    Assume that $\Gamma \nvdash a\of \phi$. By Lemma \ref{lem:basicfacts} the
    set $\Gamma_0 =\Gamma,a\of \lnot \phi$ is consistent. We will extend it to
    a maximal consistent saturated set $\Gamma^\ast$. By applying the previous
    lemma we may assume that there are an infinite set of atomic labels that
    are not mentioned in $\Gamma_0$.

    We construct $\Gamma^\ast$ as the union of $\Gamma_n$ where each
    $\Gamma_n$ is a finite extension of $\Gamma_0$. First, we enumerate all
    labelled formulas and when constructing $\Gamma_{n+1}$ we pick the $n$:th
    labelled formula $b \of \psi$ and add either $b\of\psi$ or
    $b\of\lnot\psi$. By Proposition \ref{prop:max} one of these is consistent
    with $\Gamma_n$.

    By Proposition \ref{prop:resp} we can assure that if we add a labelled
    formula $b\of \sigma \ilor \theta$ then we also add $p\of \sigma$, $q\of
    \theta$ and $b= p \lor q$ for some new propositional variables $p$ and
    $q$, and keeping $\Gamma_{n+1}$ consistent. 
    This construction assures that $\Gamma^\ast$ is maximal consistent and
    saturated.

    Now, let $T=\set{a | a\of  \itop \in \Gamma^\ast}$. Let $B$ be the
    Lindenbaum--Tarski Boolean algebra of labels over $T$, i.e., its elements
    are equivalence classes of labels under the relation of $T$-provable
    equivalence: 
    \[
        B = \Lb / \mathord\sim_T = \set{[a]_T | a \in \Lb},
    \]
    where $a  \sim_T b $ iff $T \vdash a \leftrightarrow b$ (in classical
    propositional logic) and $[a]_T=\set{b \in \Lb | a \sim_T b}$. Observe
    that $B$ is the trivial one-element Boolean algebra iff $T$ is the
    inconsistent theory, i.e., iff $\bot \in T$.

    Define the homomorphisms $h: \Lb \to B$  and $H: \Fm \to \P B$ by
    \begin{gather*}
       h(a)=[a]_T \in B \text{ for atoms $a$, and}\\ H(\varphi) = \set{h(a) |
        a\of \varphi \in\Gamma^\ast} \text{ for atoms $\varphi$}.
    \end{gather*}

    {\sc Claim.} $h(a) \in H(\varphi)$ iff $a\of \varphi \in \Gamma^\ast$.

    The claim is proved by induction on formulas. The base case follows
    immediately from the definition of $H$ and the observation that 
    $$H(\ibot) =
    \{[\bot]_T\}. $$  This is seen by taking $h(b) \in H(\ibot)$ and observing
    that $b\of \ibot
    \in \Gamma^\ast$ and so $\lnot b \of \itop
    \in \Gamma^\ast$ and, thus, $\lnot b \in T$ and $b \in [\bot]_T$.
    \begin{itemize}
        \item 
    If $\varphi$ is $\psi \lor \sigma$, $\psi \land \sigma$ or $\lnot \psi$
    the induction step is straight-forward, as for example $a \of  \psi \land
    \sigma \in \Gamma^\ast$ iff $a\of \psi \in \Gamma^\ast$ and $a\of \sigma
    \in \Gamma^\ast$ and, by the induction hypothesis, this is equivalent to
    $h(a) \in H(\psi)$ and $h(a) \in H(\sigma)$, i.e., $h(a) \in H(\psi) \cap
    H(\sigma) = H(\psi
    \land \sigma)$.

\item 
    For the case when $\varphi$ is $\ilnot \psi$ note that $a \of  \ilnot \psi
    \in \Gamma^\ast$ iff $\lnot a \of \psi \in \Gamma^\ast$ iff $h(\lnot a)
    \in H(\psi)$ iff $\lnot h(a) \in H(\psi)$ iff $h(a) \in H(\ilnot \psi)$.

\item 
    When $\varphi$ is $\psi \ilor \sigma$ note that $a \of  \psi \ilor \sigma \in
    \Gamma^\ast$ iff there are labels $b$ and $c$ such that $b \of \psi \in
    \Gamma^\ast$, $c\of \sigma \in \Gamma^\ast$ and $a = b \lor
    c \in \Gamma^\ast$. By the induction hypothesis this is
    equivalent to $h(b) \in H(\psi)$, $h(c) \in H(\sigma)$ and $h(a) = h(b\lor
    c) = h(b) \lor h(c)$. Thus this is equivalent to $h(a) \in H(\psi \ilor
    \sigma)$.
\item 
    The case when $\varphi$ is $\psi \iland \sigma$ is treated similarly, ending the proof of the claim.
\end{itemize}
    It now follows immediately that for these choices
    of $B$, $H$, and $h$ we have $h(a) \in H(\varphi)$ for all $a\of\varphi \in
    \Gamma_0$ and thus $\Gamma \nvDash a\of\phi$.
\end{proof}

\begin{cor}
    The logic LT is compact.
\end{cor}
\begin{proof}
    Follows directly from soundness and completeness.
\end{proof}

\subsection{Canonical algebras}\label{sec:canonical}

The semantics of classical propositional logic can be defined both in terms of
the single two-element Boolean algebra or in terms of all Boolean algebras,
i.e., the two-element Boolean algebra is \emph{canonical} for propositional
logic. In LT the situation is a bit different as we will see.

We write 
\[
  X: \Delta \vDash \phi \quad\text{and}\quad B : \Delta \vDash \phi
\]
for the restricted version of the entailment relation where the quantification
is done over a set $X$ of Boolean algebras or a single Boolean algebra $B$. 
For example, $B: \Delta \vDash \phi$ iff for all homomorphisms $H:\Fm \to \P B$,
\[
    \bigcap_{\psi \in \Delta} H(\psi) \subseteq H(\phi).
\]

\begin{defin}
    A set $X$ of Boolean algebras is \emph{adequate} (for LT) if
    for every $\Delta$ and $\phi$ we have
    \[
    \Delta \vDash \phi \text{ iff } X:\Delta \vDash \phi.
    \] 
   % A single Boolean algebra $B$ is said to be \emph{canonical} if $\set{B}$
   % is adequate.
\end{defin}

We omit an empty theory and write $B : {} \vDash \phi$ instead of $B:
\emptyset \vDash \phi$. Let us make some observations:
\[
B: {} \vDash \ibot \text{ iff } B=1,
\] 
where $1$ is the trivial one-element Boolean algebra.  If $\phi$ has no
propositional variables then 
\[
B:{} \nvDash \varphi \text{ iff } B: {} \vDash \lnot\ibot \lor (\ibot \iland \lnot \phi).
\]
Thus, there is a $\psi$ such that
\[
 B: {} \vDash \psi \text{ iff } B \neq 1.
\]
Therefore, if $X$ is an adequate set of Boolean algebras then $1
\in X$.\footnote{Note that, if we add $\psi$ as an axiom we get a (sound
and) complete derivation system for the semantics restricted to non-trivial Boolean
algebras} We have a similar result for $B=2$: 
\[
B: {} \vDash \ibot \lor \itop  \text{ iff } B = 1,2,
\] 
where $\itop$ is $\ilnot\ibot$. Thus, if $X$ is adequate then $2 \in X$. And
so, if $X$ is adequate then $\{1,2\} \subsetneq X$. But, we can say more. Due
to the similarities of LT and HBML we can follow the procedure in
\cite{Goranko99} to observe that no set of finite Boolean algebras is
adequate. Interpreting the models of LT as Kripke frames, the operation
$\dwn$, defined in LT in section \ref{def.clos}, directly corresponds to the
`diamond' operation of the partial order $\leq$  of the underlying Boolean
algebra. For the equally definable dual box operation
$\Box\scriptscriptstyle{\leq}$, it then follows by an established result for
any extension of the modal logic S4, that \textit{Grzegorczyk's formula}
$$\textbf{Grz:
}\Box_{\scriptscriptstyle{\leq}}(\Box_{\scriptscriptstyle{\leq}}(P\to
\Box_{\scriptscriptstyle{\leq}} P)\to P)\to P $$ is valid on a partially
ordered Kripke frame if and only if it is finite.\footnote{It is fairly
straight-forward to see that \textbf{Grz} is falsifiable in a partially
ordered Kripke frame $F$ if and only if it is possible to choose a valuation
$V$ such that there is a $\leq$-sequence of points ${p_1\leq p_2 \leq \dots }$
such that satisfaction of the member-hood $p_i \in V(P)$ is infinitely
alternating. See \cite{Bull1984} for details.} Since LT can express this
formula, we can conclude that:
\begin{thm}
	The set of finite Boolean algebras is not adequate (for LT).
\end{thm} 
\iffalse 
 \marginpar{Swap this theorem to reference to Goranko}
\begin{thm}
    No finite set of finite Boolean algebras is adequate.
\end{thm}
\begin{proof}
    We prove this by investigating the formulas $\phi_n$ defined by recursion
    on $n$:
    \begin{align*}
        \phi_0 &= P \\
        \phi_{n+1}&= P \ilor \phi_{n}
    \end{align*}
    Now, if $H$ is a homomorphism then $H(\phi_n) \subseteq H(\phi_{n+1})$
    since the elements of $H(\phi_n)$ are all of the form $a_0 \lor a_1 \lor
    \ldots \lor a_n$, where $a_i \in H(P)$, and thus $a_0\lor a_0 \lor a_1
    \lor \ldots \lor a_n \in H(\phi_{n+1})$.

    Assume now that $H$ is a homomorphism into $\P B$ where $B$ is finite with
    $n$ elements. Then, since the sequence $H(\phi_0) \subseteq  H(\phi_1)
    \subseteq  \ldots$ is increasing and the number of elements of $\P B$ is
    $2^n$ we have $H(\phi_{2^n}) = H(\phi_{2^n+1})$.

    Therefore, if $X$ is a finite set of finite Boolean algebras then $X :
    \phi_{n+1}
    \vDash \phi_n$ for some large enough $n$. 

    However in the full logic LT, $\phi_{n+1} \nvDash \phi_n$  since if $B= \P\mathbb N$
    and $H(P) = \set{\{i\} | i \in \mathbb N}$ then $H(\phi_n)$ is the set of
    all (non-empty) sets with at most $n$ elements and thus $H(\phi_{n+1})
    \nsubseteq H(\phi_n)$ for all $n \in \mathbb N$.
\end{proof}
\fi

On the other hand, the completeness theorem gives us a set of Boolean algebras
that is adequate, the set of Lindenbaum--Tarski algebras. These are all
countable Boolean algebras.

\begin{thm}
    The set of finite and countable Boolean algebras is adequate.
\end{thm}
\begin{proof}
    The proof of the completeness theorem constructs a Boolean algebra $B$
    that is a Lindenbaum--Tarski algebra over a countably infinite set of
    propositional variables. Thus, if $\Gamma \nvdash a\of \phi$ there is
    $H:\Fm \to \P B$, where $B$ is a finite or countable Boolean algebra, such
    that $\bigcap_{\psi \in \Gamma} H(\psi) \nsubseteq H(\phi)$.
\end{proof}

% In fact, by analysing the construction in the proof of the completeness
% theorem and using the compactness of LT we may restrict ourselves to finite
% Boolean algebras:

% \begin{thm}\label{thm:finite_bas}
%     The set of finite Boolean algebras is adequate.
% \end{thm}
% Before proving this theorem we need the following lemma.

% \begin{lemma}
%     If $\Gamma$ is a finite consistent set of labelled formulas and $A$ the
%     set of atomic labels that occur in labels in $\Gamma$; then
%     there is a finite consistent set of labelled formulas $\widehat \Gamma
%     \supseteq \Gamma$ such that:
%     \begin{itemize}
%         \item If $a\of \lnot (\phi \ilor \psi) \in \widehat\Gamma$ and $b$ and
%         $c$ are labels with all propositional variables in $A$, then there are $a',b'$
%         and $c'$ such that $\vdash a \leftrightarrow a'$, $\vdash b \leftrightarrow b'$, $\vdash c
%         \leftrightarrow c'$ and either $b' \of  \lnot \phi \in \widehat\Gamma$, $c' \of 
%         \lnot \psi \in \widehat\Gamma$ or $(a' \leftrightarrow b' \lor c') \of 
%         \lnot \itop \in \widehat\Gamma$.
%         \item If $a\of \lnot (\phi \iland \psi) \in \widehat\Gamma$ and $b$ and
%         $c$ are labels with all propositional variables in $A$, then there are $a', b'$
%         and $c'$ such that $\vdash a \leftrightarrow a'$, $\vdash b \leftrightarrow b'$, $\vdash c
%         \leftrightarrow c'$ and either $b' \of  \lnot \phi \in \widehat\Gamma$, $c' \of 
%         \lnot \psi \in \widehat\Gamma$ or $ (a' \leftrightarrow b' \land c') \of 
%         \lnot \itop \in \widehat\Gamma$.
%     \end{itemize}
% \end{lemma}
% \begin{proof}
%     If $a\of \lnot (\phi \ilor \psi) \in \Gamma$ and $b$ and $c$ are labels then
%     at least one of the three theories $\Gamma, b\of \lnot \phi$; $\Gamma, c\of \lnot
%     \psi$; or $\Gamma,a \leftrightarrow b \lor c\of \lnot \itop$ is consistent.
%     If not, then $\Gamma$ would prove $b\of \phi$, $c\of \psi$, and $a
%     \leftrightarrow b \lor c\of \itop$ and by applications of $\ioi$, $\sub$, and
%     $\ne$, would prove $a\of \bot$.

%     A similar statement for $a\of \lnot (\phi \iland \psi)$ holds as well.

%     There are only a finite number of equivalence classes (under provability
%     in classical propositional logic) of labels with atoms in $A$. Thus, we
%     can take a finite maximal list of such non-equivalent labels: $a_0, a_1,
%     \ldots, a_m$. Also, let $\sigma_0,\sigma_1,$ \ldots, $\sigma_n$ be a list
%     of all subformulas and negated subformulas of formulas in $\Gamma$.

%     We construct $\widehat{\Gamma}$ by a finite number of steps, step $k$
%     resulting in a finite extension $\Gamma_k$ of $\Gamma$. In each such step
%     $k$ we focus on one quadruple: $\sigma_i,a_j,a_r,a_s$, and make sure that that
%     if $\sigma_i$ is of the form $\lnot (\phi \ilor \psi)$ (or $\lnot (\phi
%     \iland \psi)$) and $a \of \sigma_i$ is in $\Gamma_{k-1}$, for some $\vdash a
%     \leftrightarrow a_j$, then at least one of $a_r\of \lnot
%     \phi$, $a_s\of \lnot
%     \psi$ or $a_j \leftrightarrow a_r \lor a_s\of \lnot \itop$ (or $a_j
%     \leftrightarrow a_r \land a_s\of \lnot \itop$) is in $\Gamma_k$. 

%     Since there are only a finite number of such quadruples this process will
%     end and we let $\widehat \Gamma$ be the last such $\Gamma_k$. This ensures
%     that for every labelled formula  $a:\neg(\phi\ilor\psi)$ (or $a:\lnot
%     (\phi \iland \psi)$), and labels $b,c$ with all propositional variables in
%     $A$, we can find $a_j,a_r,a_s$ taking the roles of $a',b',c'$ in the
%     statement of the theorem.
% \end{proof}

% \begin{proof}[Proof of Theorem \ref{thm:finite_bas}]
%     We will prove that if $\phi$ is a consistent formula, i.e., $p\of \phi
%     \nvdash \bot\of \bot$, then there is a finite Boolean algebra $B$ and a
%     homomorphism $H$ into $\P B$ such that $H(\phi) \neq \emptyset$. Together
%     with the compactness theorem for LT this proves that the set of finite
%     Boolean algebras is adequate for LT.

%     Let $\Gamma_0= \set{p\of \phi}$ and define $\Gamma_{n+1}$ by picking a new
%     (not picked before) labelled complex formula $a\of \psi$ in $\Gamma_n$ and
%     let $\Gamma_{n+1}$ be $\widehat\Gamma_{n}'$ where $\Gamma_{n}'$ is
%     constructed from $\Gamma_n$ as follows:
%     \begin{enumerate}
%         \item If $\psi$ is $\sigma \land \chi$ add both $a\of  \sigma$ and
%         $a\of \chi$.
%         \item If $\psi$ is $\sigma \lor \chi$ add $a\of  \sigma$ or $a\of
%         \chi$ depending on which is consistent with $\Gamma_n$.
%         \item If $\psi$ is $\sigma \ilor \chi$ take new atomic labels $p$ and
%         $q$ and add $p\of  \sigma$, $q\of \chi$, and $a = p\lor q$.
%         \item If $\psi$ is $\sigma \iland \chi$ take new atomic labels $p$ and
%         $q$ and add $p\of  \sigma$, $q\of \chi$, and $a = p\land q$.
%         \item If $\psi$ is $\ilnot \sigma$ add $\lnot a\of  \sigma$.
%         \item If $\psi$ is $\lnot\lnot \sigma$ add $a\of  \sigma$.
%         \item If $\psi$ is $\lnot (\sigma \land \chi)$ add $a\of  \lnot
%         \sigma$ or $a\of \lnot \chi$ depending on which is consistent with
%         $\Gamma_n$.
%         \item If $\psi$ is $\lnot (\sigma \lor \chi)$ add both $a\of
%         \lnot\sigma$ and $a\of \lnot\chi$.
%         \item If $\psi$ is $\lnot \ilnot \sigma$ add $\lnot a \of \lnot
%         \sigma$.
%         \item If $\psi$ is $\lnot (\sigma \ilor \chi)$ or $\lnot (\sigma
%         \iland \chi)$ add nothing.
%     \end{enumerate}
    
%     Let $\Gamma^\ast$ be the union of the $\Gamma_n$'s. Let $A$ the set of
%     atomic labels mentioned in $\Gamma^\ast$. As the only case where we add
%     atomic labels is when dealing with $\ilor$ and $\iland$, it should be
%     clear that if there are $k$ occurrences of $\iland$ and $\ilor$ in $\phi$
%     then we add at most $2k$ atomic labels, and so $|A| \leq 2k+1$.

%     The rest of the proof follows the proof of the completeness theorem
%     closely with the important difference that the Lindenbaum-Tarski algebra
%     $B$ are the equivalence classes of labels \emph{with atomic labels from
%     $A$} under the relation of $T$-provable equivalence. Thus $B$ is a finite
%     Boolean algebra.

%     As before we define $h(a)=[a]_T \in B$ and $H(P_i) = \set{h(a) | a\of P_i
%     \in \Gamma^\ast}$ and by induction prove that $h(a) \in H(\varphi)$ iff
%     $a\of \varphi \in \Gamma^\ast$.

%     Therefore $h(p) \in H(\phi)$ and so $H(\phi)\neq \emptyset$ and thus 
%     $\phi \nvDash \bot$.
% \end{proof}
 
% As a direct corollary we get that Boolean algebras of the form $\P X$ are
% enough to define the semantics of LT:

% \begin{cor}
%     The set of Boolean algebras of the form $\P X$ is adequate.
% \end{cor}

% Theorem \ref{thm:finite_bas} tells us that if $\nvDash \phi$ then there is a
% finite Boolean algebra $B$ and a homomorphism $H :  \Fm \to \P B$ such that
% $H(\phi) \neq B$. Also, for each finite $B$ and each $\phi$ there are only
% finitely many possible homomorphic images of $\phi$. Thus, the set of
% non-theorems of LT is recursively enumerable. Combining this with the fact
% that the set of theorems of LT is recursively enumerable and that $\phi \vDash
% \psi$ iff $\vDash \lnot \phi \lor \psi$ we can conclude the following:

% \begin{cor}
%     Entailment in LT is decidable.
% \end{cor}

%!TEX root = logic_of_teams.tex

\section{Definability}\label{sec:definable}

 We have defined an algebra based on external and internal Boolean
 connectives. Using these connectives, we can define a set of other constants
 and connectives; some of the usual suspects, and some that will be useful for
 the connection to valuational team semantics. It will be clear from the
 definitions we give that there is plenty of similar constructions to define
 similar connectives. In this paper we focus on the connectives that will be
 important for expressing traditional propositional team semantics, and leave
 further exploration for now. We name connectives after their interpretation
 in the intended semantics.

 \subsection{Constants}

We define the following constants with corresponding semantic interpretation
for every Boolean algebra $B$ and every homomorphism $H:\Fm \to \P B$:
\begin{itemize}
 	\item $\top~~= \lnot\bot$. Then $H(\top) =\P(B)$ and is called
 	\emph{the external top}.
 	\item $\NB =\lnot \ibot$.  Then $H(\text{NB})= \{\bot\}^C=
 	\set{x | x\neq \bot}$ and is called the  \emph{not-bottom constant}.
 \end{itemize}
Remember that we previously defined $\itop =\ilnot \ibot$, \emph{the internal
top}. 

In addition to the constants $\bot$ and $\ibot$ these are the constants it
will be useful for us to name. There are more constants definable in LT. Note
however that the interpretations of these constants may coincide for some
algebras $\P B$. This observation underlies the failure of canonicity for
single algebras explored in section \ref{sec:canonical}.


\subsection{Closures}
We define two useful closure operators with corresponding semantic
interpretation for every Boolean algebra $B$ and every homomorphism $H:\Fm \to
\P B$:
\begin{itemize}
		\item $\dwn \phi = \phi \iland \top$.  Then $H(\dwn \phi) = \set{ b \in B|
		\text {there exists } a \in H(\phi) , \text{ s.t. } b\leq a}$, called the
		\textit{downwards closure} operator.
		\item $\up \phi = \phi \ilor \top$.  Then  $H(\up \phi) = \set{ b \in B | \text
		{there exists } a \in H(\phi), \text{ s.t. } b\geq a}$, called the
		\textit{upwards closure} operator.
\end{itemize} 
For elements of a Boolean algebra we use the standard understanding of $a \leq
b$, as $a\land b = a$.

Note that the closure operators are projections in the sense that $H(\dwn \dwn
\phi)= H(\dwn \phi)$. Furthermore, $\top$ and $\bot$ are their only common
fix-points: $H(\dwn\bot)= H(\bot) = H(\up \bot)$. We also have the following
properties:
\begin{gather*}
H(\dwn \phi) = \emptyset \text{ iff } H(\phi) = \emptyset, \text{ and
otherwise } \bot \in H(\dwn \phi), \\ 
H(\up \phi) = \emptyset \text{ iff } H(\phi) = \emptyset, \text{ and otherwise }
 \top \in  H(\up \phi),\\
H(\dwn \phi) = \P B \text{ iff } \top \in H(\phi), \text{ and} \\
H(\up \phi) = \P B \text{ iff  } \bot \in H(\phi).
\end{gather*}

We can use these properties to define universal modal operators.
%%%(XXXX maybe see these closures as "forward and backwards " modalities?)

\subsection{Modal operators}
When considering a homomorphism $H : \mathtt{Fm} \to \P{B}$, it is interesting
to have an `existence', or `possibility' operator $\Diamond$ such that for all
formulas $\phi$, algebras $\P B$, and all homomorphisms $H$,

$$ H(\Diamond \phi)  = \begin{cases} \P B & \text{if there exists } x\in H(\phi)\\ \emptyset & \text{otherwise} \end{cases} $$

Similarly, we consider a `universal' or `always' operator $\Box$,
such that

$$H(\Box\phi)  =\begin{cases} \P B & \text{if for all }  x,\quad  x\in H(\phi)\\ \emptyset & \text{
otherwise} \end{cases} $$

with the symbols $\Box$ and $\Diamond$ chosen for the clear role as universal
modal operators for the Boolean algebra. Inspecting the described properties
of the closure operators $\dwn$ and $\up$ it is clear that these connectives
can be defined in the following manner.

\begin{defin}
For any algebra $\P B$ we define the universal and existential operators
$\Box$ and $\Diamond$ as follows:
\begin{align*}
	\Diamond \phi &= \up \dwn \phi  = (  \phi\iland \top) \ilor\top\\
	\Box A &= \lnot \Diamond \lnot A
\end{align*}
\end{defin}

Interestingly enough, we have in some sense found a decomposition of the
universal modalities on a set of worlds that happen to be a Boolean algebra,
so that we have the following.

\begin{thm}
	The fragment $(\bot, \lnot, \lor ,\land, \Box, \Diamond)$ of LT captures exactly the\\
	modal logic S5.
\end{thm}

Note that the interpretation of the connectives mentioned in the above theorem
do not depend on the underlying/internal structure of the algebra, and the
underlying Boolean algebra functions merely as `a set of worlds' with $\Box$
and $\Diamond$ interpreted for the universal relation.

The $\Box$ operation however plays a more important role in this paper
in that it facilitates internalisation into formulas of classifications of
important global properties of homomorphism.


\subsection{Definable classes of homomorphisms}\label{sec:def-class}

The semantic definition of entailment in LT is given as a universal
satisfaction of a property evaluated independently for all homomorphisms for
all Boolean algebras. Just like how, in terms of Kripke models, semantics for
intuitionistic logic is given by a restriction on valuations to those
satisfying a persistence criteria, we are interrested in logics
that can be defined by a restriction on the class of homomorphisms that are
considered. We also describe a notion of \textit{definability} in LT of such classes, and by
using the $\Box$ operator achieve axiomatisations in LT of the logics of
definable classes of homomorphisms.

\begin{defin}
Let $H$ be a homomorphism $\Fm \to \P B$. We
define the \emph{local entailment} of $H$ denoted $H: \Delta \vDash \phi $ in
the expected way:
$$H: \Delta \vDash \phi \quad \text{ iff } \quad
\bigcap_{\delta\in \Delta} H(\delta) \subseteq H(\phi)$$
Given a class of homomorphisms $\mathcal{H}$ (not necessarily all to the same
algebra), we can define the logic of the  class-entailment
$$ \mathcal{H} : \Delta \vDash \phi \text{ iff for all }  H\in
\mathcal{H}. \quad H:\Delta \vDash \phi  $$
\end{defin}
We will be able to find sets of formulas identifying classes of homomorphisms, in the sense
that they are all valid exactly for the homomorphisms of that class.

\begin{defin}
A class of homomorphism $\mathcal{H}$  is \textit{definable} in LT if there
exists a set of formulas $\Pi$ such that for all homomorphism $H: \Fm \to
\P B$,
$$H\in \mathcal{H} \quad \text{iff } \quad H:~\vDash \pi \text{
for all } \pi \in \Pi. $$
\end{defin}

Note that to axiomatize the logic of a class of homomorphisms $\mathcal{H}_{\text{a}}$,
it is not enough to take a defining set of formulas $\Pi$ as axioms. The
reason is that definability imposes a global condition on homomorphism, whereas the
entailment is a local condition. As a simple example, consider the class
$\mathcal{H}_{\text{tr}}$ of homomorphisms onto the powerset of the trivial Boolean algebra $\P B=
\set{1}$. It is clear that this class is defined by the formula $\ibot$ $$H\in
\mathcal{H}_{\text{tr}}\quad \text{iff}\quad H:~\vDash \ibot$$ On the
other hand we have that $$\mathcal{H}_{\text{tr}} : ~ \vDash \itop \quad
\text{but} \quad \ibot \nvDash \itop$$ However, by using the universal
modality $\Box$, we can internalise global conditions and ensure that $$x\in
H(\Box \eta) \quad \text{ iff} \quad H(\eta)= \P B.$$ That is, for
all $x \in B$, $$ x\in H(\Box \eta)\quad \text{iff} \quad H :~
\vDash \eta.$$ We can then conclude the following theorem:

\begin{thm} 
If a class of homomorphism $\mathcal{H}$ is defined by a set of formulas
$\Pi$, let $\Box\Pi= \set{\Box \pi | \pi\in \Pi}$. Then we have
$$ \Box \Pi, \Delta \vDash \phi \quad \text{ iff } \quad
\mathcal{H} : \Delta \vDash \phi,$$ 
so that $\Box \Pi$ serves as an axiomatization in LT of the logic defined
by the restriction to homomorphisms of the class $\mathcal{H}$.
\end{thm}

This usage of $\Box$ also serves the purpose to relate the semantics of LT to
standardly considered algebraic semantics, in which $H(\top)$ is the special
designated value corresponding to truth \cite{Font16} :
\begin{thm} 
$\Box \phi \vDash \psi$ iff for all homomorphisms $H$, $H(\phi) = H(\top)$
implies $H(\psi)=H(\top)$.
\end{thm}

\subsection{Implication}
Not surprisingly we define \textit{external implication} and
\emph{equivalence} in the standard Boolean way:
\begin{gather*}
	\phi \to \psi = \lnot \phi \lor \psi \\ 
	\phi \leftrightarrow \psi = (\phi \rightarrow \psi) \land (\psi \rightarrow \phi)
\end{gather*}
	With standard Boolean considerations it is then
easy to see that the deduction theorem holds in LT for this external
implication:

\begin{thm}[Deduction theorem for LT]
	For all $\Delta \cup \set{\phi,\psi}\subseteq\Fm$
	$$\Delta, \phi  \vDash \psi \quad \text{iff} \quad \Delta \vDash \phi \to \psi.$$
\end{thm}

With several types of negations and disjunctions, there are several other type
of implication that are natural to consider, but they will not be of focus in
this paper. In Section \ref{Ax-class-hom}, leaning on the deduction theorem
for $\to$ we will use this connective  to define and, together with $\Box$, to
axiomatise the logic for an important class of homomorphism.

\subsection{Strict negation}
When we in Section \ref{sec:axiomatising} are going to relate LT to valuational team semantics, we will
have to consider another type of negation. We will denote this negation by
$\slnot$. For the following discussion it will be useful to define $\slnot$
not just as an operation on formulas of LT, but as an operation on subsets of
Boolean algebras:

\begin{defin}
  For subset $A$ and an element $b$ of a Boolean algebra $B$ we say that $b$
	is \emph{separate} from $A$ if for all $a \in A$, $b\land a = \bot$. We
	define $\slnot A$ as the set of all elements separate from $A$, i.e.,
	$$\slnot A = \set{ b \in B | \text{for all } a\in A, b\land a = \bot }.$$
\end{defin}

This operation is definable as an operation in LT, and
we call it \textit{strict negation} 

\begin{defin}
	In LT we define the unary operation $\slnot$ by the following
	$$ \slnot \phi= \lnot\up( \dwn \phi\land \NB )$$ 
	we call this operation the \textit{strict negation}
\end{defin}

\begin{thm}\label{slnot_form_to_subset}
	For every Boolean algebra $B$, every homomorphism $H:\Fm \to \P B$, and
	every formula $\phi\in\Fm$ we have $$ H(\slnot \phi) = \slnot H(\phi)$$
\end{thm}
\begin{proof}
	We first convince ourselves that for any algebra, homomorphism and formula
	as prescribed we have
	$$H(\dwn \phi \land NB)= \set{ a \in  B |  a \neq \bot \text{ and } a\leq b
	\text{ for some } b\in H(\phi)}.$$
	Therefore $H(\up(\dwn \phi \land NB))$ is the set of elements of $B$ that
	have non-trivial intersection with some element in $H(\phi)$, i.e
	$$H(\up (\dwn \phi \land NB)= \set{ a\in B | a\land b \neq \bot \text{ for
	some } b\in H(\phi)}.$$
	This is exactly the complement of $\slnot H(\phi)$, and thus
	\[ 
	H(\slnot\phi)=  H (\lnot\up( \dwn \phi\land \NB ))= \slnot H(\phi).
	\]
\end{proof}

The purpose of defining strict negation in this paper is in order to identify
the relation between LT and traditional propositional team semantics defined
by valuational semantics. The connective is however interesting in its own
right, and it can be viewed as a type of intuitionistic negation in the
following sense.
 
\begin{thm}\label{prop_strict_neg}
	The following statements hold for LT.
	\begin{enumerate}
		\item $\vDash P \to \slnot \slnot P$ 
		\item $\nvDash \slnot \slnot P \to P $ 
		\item $\vDash \slnot  \slnot \slnot  P \leftrightarrow \slnot P$ 
		\item $\vDash (\slnot P \iland \slnot \slnot P )\leftrightarrow \ibot$
		\item $\vDash \slnot P \ilor \slnot \slnot P $
	\end{enumerate}  
\end{thm}

As a simple proof of statement (2), consider for any Boolean algebra a
homomorphism $H$ such that $H(P)= \emptyset$. Then $H(\slnot \slnot P) =
\set{\bot}$. Clearly then $H(\slnot \slnot P)\nsubseteq H( P)$ and the
statement is asserted. The other statements can be proven using the natural
deduction system given in Section \ref{deduction-system}, but since the
statements include many defined connectives the proof trees become fairly
large. As example, statement (1) is written directly without defined
connectives as $$(1) \vDash \neg P \lor \lnot((([\lnot (((P \iland \lnot
\bot)\land \lnot \ibot)\ilor \lnot \bot)]\iland \lnot \bot) \land \lnot \ibot
)\iland \lnot \bot)$$ Figure \ref{fig:derivation} shows a proof tree for the
core of such proof by proving $a:P \vdash a:\slnot\slnot P$, still using
defined connectives as notational abbreviations.

\begin{sidewaysfigure}
\vspace{.6\textwidth}
\hspace{10pt}
$
\infer[\ni_1]{a\of \slnot \slnot P}{
	\infer[\ioe_2]{a\of \bot}{
		[a\of \up(\dwn \slnot P \land \NB)]^1
		&
		\infer[\iae_3]{a\of \bot}{
			[p \of \dwn\slnot P \land \NB]^2
			& \hspace{-50pt}
			\infer[\be]{a\of \bot}{
				\infer[\ne]{s\of \bot}{
					[s\of \slnot P]^3
					&\hspace{-180pt}
					\infer[\taut /\sub]{s\of \up (\dwn P \land \NB)}{
						\infer[\ioi]{(s\land r)\lor(s\land q)\lor (s\land \lnot (r\lor q))\of \up(\dwn P \land \NB)}{
							\infer[\ai]{(s\land r )\lor (s\land q) \of \dwn P \land \NB}{
								\infer[\taut / \sub]{(s\land r)\lor(s\land q)\of\dwn P}{
									\infer[\iai]{((s\land r)\lor q)\land s \of \dwn P}{
										\infer[\sub]{(s\land r)\lor q \of P}{
											\infer[\taut ]{a= (s\land r) \lor q}{
												[a=p\lor q]^2
												&
												[p= s\land r]^3}
											&
											a\of P}
										&
										\infer[\ni]{s \of \top}{[s\of \bot]}}}
								&
								\infer[\ni]{(s\land r)\lor(s\land q)\of \NB}{
									\infer[\ne]{(s\land r)\of \bot}{
										\infer[\sub]{s\land r \of \NB}{
											[p=s\land r]^3
											&
											\infer[\ae]{p\of \NB}{
												[p\of \dwn \slnot P \land \NB]^2}}	
										& \hspace{-10pt}
										\infer[\taut/\sub]{(s\land r) \of \ibot}{
											\infer[\ine]{\lnot\lnot(s\land r) \of \ibot}{
												\infer[\taut]{\lnot (s\land r) \of \itop}{
													\infer[\ini]{\lnot((s\land r) \lor (s\land q)) \of \itop }{
														[(s\land r)\lor (s\land q) \of \ibot]}}}}}}}						
							&\hspace{-60pt}
							\infer[\ni]{s \land \lnot (r\lor q) \of \top}{
								[s\land \lnot (r\lor q)\of \bot]}}}}}}}}
$
\caption{Derivation showing that $a:P \vdash a:\slnot\slnot P$. Here
$\taut/\sub$ is a shorthand for a combination of $\taut$ and $\sub$.}
\label{fig:derivation}
\end{sidewaysfigure}

At the end of next section, when we have further investigated semantic
properties of $\slnot$, we will instead give semantic proofs for the
statements  of Theorem \ref{prop_strict_neg}.

%!TEX root = logic_of_teams.tex

\section{Flatness in LT}

We will now spend some effort understanding $\slnot$ as an operation on
subsets of a Boolean algebra. Recall that a Boolean \textit{ideal} is a
subset of a Boolean algebra that is downwards closed and closed under
disjunction.

\begin{thm}\label{negid}
	For all Boolean algebras $B$ and all subsets $A\subseteq B$ we have that $\slnot
	A$ is a non-empty Boolean ideal.
\end{thm}
\begin{proof}
	First clearly $\bot\in \slnot A$, so it is not empty. Then, for all $a \in
	A$, if $b\land a = \bot$ then for all $c\leq b$, $c\land a = \bot$, and hence
	$\slnot A$ is downwards closed. Similarly, if $a\land  b= \bot$ and $a\land
	c = \bot$, then $a\land (b\lor c)= \bot$ by distributivity, and hence
	$\slnot A$ is closed under disjunction.
\end{proof}

This is a necessary property for the sets $\slnot A$ but not fully
categorising in the sense that there exists Boolean algebras $B$  with ideals
that cannot be identified as $\slnot A$ for any set $A\in \P B$. However, we
can strengthen the categorisation in the following way by identifying the
specific type of ideals.

\begin{defin}
		For a Boolean algebra $B$ a subset  $A\subseteq B$ is said to be
	\begin{itemize}
		\item  \emph{closed under existing infinite disjunctions} if for all
		$ C \subseteq A$ such that $\bigvee C$ exists in $B$ then $\bigvee C \in A$, and
		\item \emph{flat} if it is both downwards closed and closed under existing
		infinite disjunction.
	\end{itemize}
\end{defin}

\begin{thm}\label{basic_flatness}
	For any Boolean algebra $B$ and subset $A\subseteq B$:
	\begin{enumerate}
		\item $\slnot A$ is flat, and
		\item $A$ is flat iff $\slnot \slnot A = A$.
	\end{enumerate} 
\end{thm}
\begin{proof}
	(1) Downwards closure follows directly from Theorem \ref{negid} that asserts
	$\slnot A$ is an ideal of $B$. Note also that $\bot$ is
	the result of the empty disjunction so that $\bot$ is an element of every
	flat set. 
	
	Let $C\subseteq \slnot A$ be a set of elements
	such that $\bigvee C \in B$. We need to show that $\bigvee C \in
	\slnot A$. For arbitrary $a\in A$ we have, that $a\land \bigvee C =
	\bigvee_{c\in C} c \land a $ and the right side of this equation
	exists.\footnote{See for example Lemma 1.33 on p. 22 of \cite{Monk1989}.} Then by
	assumption $\bigvee_{c \in C} c \land a = \bigvee_{c\in C} \bot = \bot$ and thus
	$\bigvee C \in \slnot A$.
	
	(2) One direction follows directly from (1). For the other direction,  it is easy
	to see that in general $A\subseteq \slnot \slnot A$, since for all $a\in A$
	we have for all $c \in \slnot A$ that $a\land c= \bot$ by definition of
	$\slnot A$. What is left to show is that in the case when $A$ is flat we
	also have  $\slnot \slnot A\subseteq A$. 
	
	Thus, assume $A$ is flat and $b\in \slnot \slnot A$. We need to show that
	$b\in A$. Consider the set $\set{b\land a | a\in A}$. By the downwards
	closure of $A$, it is clear that $\set{b\land a | a\in A}\subseteq A$, so that
	by closure under existing infinite disjunctions, if it exists,
	$\bigvee_{a\in A} b\land a\in A$.
	
	We thus finish the proof by showing that $b= \bigvee_{a\in A} b\land a$,
	i.e., $b$ is the least upper bound of the set $\set{b\land a | a\in A}$.
	Since $b\geq b\land a $ for all $a\in A$, clearly $b$ is an upper bound for
	$\set{b\land a}_{a\in A}$. Assume for the sake of contradiction, that it is
	not the least upper bound. Then there exists $c\in B$ such that $c$ is an
	upper bound for $\set{b\land a | a\in A}$, and $c\lneq b$. We then
	investigate the element $d = b\land \lnot c \neq \bot$. For every $a\in A$
	we have
	$$a\land d = a \land  (b \land \lnot c) = (a\land b) \land \lnot c=
	\bot,$$ 
	where the last equality is due to $(a\land b)\leq c$ since $c$ is an upper
	bound for $\set{b\land a | a\in A}$. Consequently, $d\in \slnot A$, but we
	also have $d\leq b$ by construction, and thus $b\notin \slnot\slnot A$ which
	is a contradiction. Thus $b$ is necessarily the least upper bound of
	$\set{b\land a | a\in A}$, and hence  $b= \bigvee_{a\in A} b\land a\in A$
	and thus $b\in A$.
\end{proof}

As stated in the proof $A\subseteq \slnot \slnot A$ holds for any subset of a
Boolean algebra, and it is just the reverse inclusion that essentially
categorizes flatness. Using this result we can extend this language to
formulas of LT.

\begin{defin}
	A formula $\phi\in\Fm$, is called \textit{flat} if $\slnot \slnot \phi \vDash 
	\phi$, that is, for every Boolean algebra $B$, and every homomorphism $H:\Fm 
	\to \P B$, $H(\phi)$ is flat as a subset of	$B$.
\end{defin}

\subsection{Flatness in complete and atomic algebras}

By the adequacy result of theorem \ref{thm:finite_bas}, we are for the
purpose of LT able to restrict our attention to finite Boolean algebras. Since
finite Boolean algebras are both complete and atomic, it is relevant to
investigate flatness in the particular case when the Boolean algebra is
complete and atomic. These specific observations will play a role in Section
\ref{sec:axiomatising} when relating LT to propositional dependence logics.
Recall the following definitions of complete and atomic Boolean algebras:

\begin{defin}
	A Boolean algebra is said to be \textit{complete} if it is closed under
	infinite disjunctions and conjunctions, and it is said to be
	\textit{atomic} if every element is a least upper bound of some set of
	atoms.
\end{defin}

It is clear that a Boolean algebra $B$ is atomic iff 
\[b=\bigvee \set{ a\in B | a \text{ is an atom and } a\leq b}\] 
for all $b \in B$. Furthermore, we recall the following theorems:

\begin{thm}[\cite{Halmos2009}]
	Every finite Boolean algebra is complete and atomic.
\end{thm}

\begin{thm}[\cite{Halmos2009}]
	Every complete atomic Boolean algebra is equivalent to an algebra of the
	form $(\P S, \emptyset, \cdot^C, \cup, \cap)$ for some set $S$.
\end{thm}

We can relatively directly observe the following simpler categorisation of
flatness in these classes of Boolean algebras.

\begin{thm}\label{comp_slnot_ideal}
	If $B$ is a complete Boolean algebra, then for all $A\in \P B$, $A$ is
	flat iff $A$ is a non-empty principal ideal, i.e., if there
	exists $a\in A$ such that $$ A= \set{ b \in B| b\leq a}.$$
\end{thm} 
\begin{proof}
	Assume $A$ is flat. Take $a= \bigvee A$, then by flatness $a\in A$ and by
	downwards closure $A$ is the principal ideal generated by $a$. For the other
	direction, assume $A$ is a non-empty principal ideal. Then $A$ is downwards
	closed. Furthermore, there exists $a\in A$ such that $A=\set{ b \in B| b\leq
		a}$. Consequently, for all $C \subseteq A$ it is evident that $\bigvee C
	\leq a$, and thus again by downwards closure, $\bigvee C \in A$. Therefore
	$A$ is closed under existing infinite disjuctions, and thus is flat.
\end{proof} 

For atomic Boolean algebras flatness is equivalent to membership being
reducable to atomic membership in the following sense:

\begin{thm}
	Let $B$ be an atomic Boolean algebra, then a subset $A \subseteq B$ is flat if
	and only if
	\[
	A = \set{b \in B | a \in A \text{ for all atoms }  a \leq b}.
	\]
\end{thm}
\begin{proof}
	Assume $A$ is flat, then it is non-empty. Assume $b\in A$. Then for all
	$a\leq b$, $a\in A$, by downwards closure of A, and thus 
	\[ 
		A\subseteq \set{b \in B | a \in A \text{ for all atoms }  a \leq b}.
	\] 
	For the opposite inclusion, assume $a\in A$ for all atoms $a\leq b$. Then
	since $B$ is atomic, $b=\bigvee \set{a \in B | a \text{ atomic and } a\leq
	b }$ so that, by the closure under existing infinite disjunctions, $b\in
	A$.
	
	For the opposite direction, assume $b \in A \text{ iff } a \in A \text{
	for all atoms }  a \leq b$. We need to show that $A$ is flat. Downwards
	closure is clear, since if $b\in A$ then for any $c\leq b$, $\set{a \in B
	|  a \text{ atomic and } a\leq c} \subseteq \set{a \in B |  a \text{
	atomic and } a\leq b} $. For the closure under existing disjunctions,
	assume, for $C \subseteq A$, that $\bigvee C=b\in B$. Clearly, for all
	atoms, $a\in B$, $a\leq b$ iff there exists $c \in C$  such that $a\leq
	c$. Consequently, for all $a\leq b$, $a\in A$, so that by assumption $b\in
	A$. Thus, $A$ is flat.
\end{proof} 

We have previously established that for any Boolean algebra $B$, a subset
$A\subseteq B$ is flat iff $\slnot \slnot A = A$. We can thus
summarise these results for the nicely behaved complete and atomic Boolean
algebras:

\begin{cor}\label{flat_eqs}
	For a complete atomic Boolean algebra $B$, and a subset $A\subseteq B$ the
	following statements are equivalent:
	\begin{itemize}
		\item $A$ is a non-empty principal ideal of $B$. 
		\item $A$ is flat.
		\item $A =  \set{b \in B | a \in A \text{ for all atoms }  a \leq b}$.
		\item $ \slnot \slnot A\subseteq A $. 
	\end{itemize}
\end{cor}

Since the set of finite Boolean algebras is adequate for LT, Theorem
\ref{thm:finite_bas}, and all finite Boolean algebras are complete and atomic
this corollary will be useful in the next section to identify flatness in the
following way:

\begin{cor}\label{flatness_formula}
	If $B$ is a complete atomic Boolean algebra, then for all homomorphism $H:
	\Fm \to \P B$ and all formulas $\phi$ 
	\[
	H(\phi) \text{ is flat iff }  H : ~ \vDash \slnot \slnot \phi \to \phi.
	\]
\end{cor} 

To finish of this section we can use the results to give a semantic proof of
the properties of strict negation stated in Theorem \ref{prop_strict_neg}.

\begin{proof}[Proof of Theorem \ref{prop_strict_neg}]
	Statement (1) follows directly from the observation in the proof of Theorem
    \ref{basic_flatness} part (2) that for any subset of a Boolean algebra $A$ 
    we have that $A\subseteq \slnot\slnot A$.
	For statement (4) observe that for any formula $\phi$ and any homomorphism $H$, 
	if $a\in H(\phi)$,and $b\in H(\slnot\phi)$, then by definition 
	$a\land b= \bot $. Thus, as long as $H(\phi)$ and 
	$H(\slnot\phi)$ are non-empty, $H(\phi\iland \slnot \phi)= \set{\bot}$.
	This is assured by Theorem \ref{negid} when $\phi = \slnot P$.
	To semantically prove (3) and (5) we use Theorem \ref{thm:finite_bas} asserting that the class
	of complete atomic Boolean algebras are adequate for LT, so we may restrict our 
	attention to this class of algebras.  We can then use the following claim.
	
		\noindent\textbf{Claim:}  If $B$ is  complete Boolean algebra, and 
		$A\subseteq B$ is a non-empty principal ideal with top element 
		$a= \bigvee \!\!A$, then $\neg a = \bigvee\!\! \slnot\! A$ and thus is 
		the top element of the principal ideal $\slnot A$. 

		\begin{proof}[Proof of claim:] Assume $a$ is the top element of 
			the principal ideal $A$. Then for all $b\in A$ we have that 
			$b\land a = b$ and thus  $\neg a \land b = \neg a \land a \land b = \bot$,
			and thus $\neg a \in \slnot A$. Furthermore, if $b\in \slnot A$, 
			then $ a\land b = \bot$. 
			Therefore, $\neg a \lor (a \land b)= \neg a \lor b = \neg a$, 
			i.e. $b\leq \neg a$ and we have established that 
			$\neg a $ is the top element of the principal ideal $\slnot A$. 	
		\end{proof}
	To establish statement (3) we see for every complete 
	(atomic) Boolean algebra, by Theorem \ref{comp_slnot_ideal}, 
	that $H(\slnot P)$ is a non-empty principal ideal for every homomorphism 
	$H$. Let $a$ be the top element of $H(\slnot P)$. Then by the claim $a$ 
	is also the top element of 
	$\slnot \slnot H(\slnot P)= H(\slnot \slnot \slnot P)$, and 
	thus $H(\slnot \slnot \slnot P)= H(\slnot P)$. 
	This is enough to assert the validity of formula (3).
	
	For formula (5), we first acknowledge by Theorem \ref{comp_slnot_ideal}
	and the claim that for any complete (atomic) Boolean algebra $B$ and any 
	homomorphism $H$ there is an element $a$ such that $a$ and $\neg a$ are the 
	top elements of the principal ideals $H(\slnot P)$ and $H(\slnot\slnot P)$ 
	respectively. Then, for every $b\in B$ we have
	since $a \lor \neg a= \top$ that
	$$b= b \land (a\lor \neg a)= (b\land a ) \lor (b\land \neg a)$$
	being the top elements of respective principal ideals we observe that 
	$$b\land a \in  H(\slnot P)\quad \text{and} \quad b \land \neg a \in H(\slnot \slnot P)$$ 
	and conclude that  $ H(\slnot P \ilor \slnot \slnot P)= B$. 
	This holds for every homomorphism $H$ for a complete Boolean algebra B, and 
	hence (5) holds for the logic LT.   
\end{proof}
%!TEX root = logic_of_teams.tex


\section{Axiomatising valuational team semantics}\label{sec:axiomatising}

In this section we show how standard team semantics based on valuations relate
to the logic LT. This construction follows in essence the construction
presented in \cite{LorimerOlsson2022} with some refinements and
generalisations. The construction and proof will be developed through the
following steps, enumerated by the corresponding subsection.

\begin{enumerate}
	\item[6.1.] We start by defining the propositional team logic
	$\text{PT}^+$ in terms of its standard semantics, here referred to as
	\textit{valuational team semantics}, by defining its denotations in terms
	of collections of teams of valuations. This is an expressive logic with
	many interesting logics as syntactic fragments.
	\item[6.2.] Then we show how to formulate this semantics in terms of the
	algebra $\P\P 2^{\mathbb{N}}$ with a specific homomorphism $H_V$ in the
	style of the semantics we have given for LT. We can then prove that LT is
	in a sense a conservative extension of a logic weaker than $\text{PT}^+$.
	\item[6.3.] Next we observe that the homomorphism $H_V$ belongs to a special
	class of homomorphisms $\mathcal{H}_{\text{PV}}$ that are nicely definable
	and axiomatisable in LT using defined connectives from Section
	\ref{sec:definable}.
	\item[6.4.] We can then show that the axiomatisation of this class
	axiomatises $\mathrm{PT}^+$ as a fragment of LT. This is done by
	essentially showing that for the restricted language of $\mathrm{PT}^+$
	the homomorphism $H_V$ is canonical for the logic of the whole class
	$\mathcal{H}_\mathrm{PV}$. The proof includes two main steps expressed in
	two separate lemmas:
	\begin{itemize} 
		\item Lemma \ref{embedding_lemma} states that for embeddings of
		Boolean algebras all homomorphism in $\mathcal{H}_{\text{PV}}$ to the
		domain of the embedding, can be paired with homomorphism in
		$\mathcal{H}_{\text{PV}}$ to the co-domain such that memberhood in
		interpretations of formulas of $\text{PT}^+$ are preserved by the
		embedding. This together with Stone's theorem establishes that the
		complete atomic Boolean algebras are adequate for the logic of the
		class $\mathcal{H}_{\text{PV}}$.
		\item Lemma \ref{f_rep_in_Val} then establishes that every
		homomorphism in $\mathcal{H}_\mathrm{PV}$ with a complete atomic
		Boolean algebra as co-domain can be represented by the homomorphism
		$H_V : \Fm \to \P2^\mathbb{N}$ in such a way that $H_V$ expresses the
		memberhood of the original homomorphism by the mapping for
		interpretations of formulas in $\text{PT}^+$.
	\end{itemize}
	With these lemmas together we conclude that $\text{PT}^+$ is axiomatised
	as a fragment of LT by the axioms of the class of homomorphisms.
\end{enumerate} 

In this section we will treat several logics defined for different sets of
connectives, and thus with different sets of formulas. For a logic L we denote
by $\Fm_{\text{L}}$ the formulas of the logic and $\text{L}:\Delta \vDash
\phi$ for the entailment of the logic. In particular, the formulas of LT will
be referred to as $\Fm_{\text{LT}}$, and its entailment will be denoted by
$\text{LT}: \Delta \vDash \phi$ in contrast to other sections of this paper.

\subsection{Valuational team semantics}\label{sec:valuational}

We are interested in formulating a logic in a sense generalising some
propositional team logics given in team semantics where the notion of
\textit{team} refers to sets of valuations being the basic semantic object. We
call this \textit{valuational team semantics}. We represent the valuational
team semantics by recursively defining the denotation of formulas on the set
of teams of valuations $\P2^\mathbb{N}$. In order to cover a majority of the
more interesting logics we focus our attention on \textit{strong propositional
team logic}, $\text{PT}^+$, one of the strongest logics presented in
\cite{Yang2017}.

\begin{defin}
The set of formulas $\Fm_{\text{PT}^+}$ of \emph{Strong propositional team logic},
$\text{PT}^+$,  is generated by the following grammar $$ \phi::= P_i
\mathrel|  \slnot P_i \mathrel|\ibot \mathrel| \NB\mathrel|
\phi \ilor \phi\mathrel|   \phi \land \phi \mathrel|  \phi \lor \phi $$ and
its valuational team semantics can be described by defining the denotations
for formulas $\llbracket \phi \rrbracket \subseteq \P 2^\mathbb{N} $
recursively for cases of the main connective as follows:
\begin{align*}
	\llbracket  P_i \rrbracket &= \set{ X | \text {for all } s \in X, s(i)= 1}\\
	\llbracket \slnot P_i \rrbracket &= \set{ X | \text {for all } s \in X, s(i)= 0}\\
	\llbracket \ibot \rrbracket &= \set{ \emptyset}\\
	\llbracket \NB \rrbracket &= \set {X| X\neq \emptyset} \\
	\llbracket \phi \ilor \psi \rrbracket &= \set{ X\cup Y | X\in \llbracket
	\phi \rrbracket, Y\in \llbracket \psi \rrbracket} \\
	\llbracket \phi \land \psi \rrbracket &= \llbracket \phi \rrbracket \cap
	\llbracket \psi \rrbracket\\
	\llbracket \phi \lor \psi \rrbracket &= \llbracket \phi \rrbracket \cup
	\llbracket \psi \rrbracket\\
\end{align*}
The logical entailment of $\text{PT}^+$ is then defined as follows:
$$\text{PT}^+: \Delta \vDash \phi \quad \text{iff} \quad
\bigcap_{\delta \in \Delta}  \llbracket \delta \rrbracket \subseteq \llbracket
\phi \rrbracket.$$
Elements $s\in 2^\mathbb{N}$ are viewed as \textit{valuations} for the set of
propositional variables, and sets of valuations $X\in \P 2^\mathbb{N}$ are
referred to as \textit{teams (of valuations)}.
\end{defin}

There is no standard notation for the connectives in the literature, and we have
chosen notation that corresponds best to the notation for LT. Table
\ref{translation} indicates the correspondence between our notation and
notation elsewhere. 

\begin{table}
	\label{translation}
\begin{tabular}{cccc}
	\toprule
	\cite{Yang2017} & \cite{Lueck2020} & \cite{Yang2022} & This paper \\
	\midrule
	$p_i$  		& $p_i$ 	& $p_i$ 	& $P_i$\\
	$\neg$ 		& $\neg$	& $\neg$	& $\slnot$ \\
	$\bot$ 		& $\bot$ 	& $\bot$	& $\ibot$ \\
	$\otimes  $ & $\vee $	& $\vee$	& $ \ilor $ \\
	$\wedge$ 	& $\wedge$	& $\wedge$  & $\land $ \\
	$ \vee $ 	& $\ovee$	& $\ilor$   & $ \lor $ \\
	$\text{NE}$ & ~ 		& ~			& $\NB$ \\
	\bottomrule
\end{tabular}
\caption{Correspondence between notations in the current paper and other
relevant papers on propositional team logics.}
\end{table}

As defined, it is clear that for the set of formulas, including defined
connectives, we have that $\Fm_{\text{PT}^+}\subset
\Fm_{\text{LT}}$ even though the logics are described using different
semantics. In this section we will find axioms in the language of LT that
axiomatises $\text{PT}^+$ in LT. In the literature there are multiple weaker
propositional team logics described and studied, in particular propositional
logics of dependence \cite{Yang2016}. Many important logics, however, are
given as, or is expressively equivalent to, logics that can be given as
fragments of $\text{PT}^+$. This means that our result regarding the
axiomatisation of $\text{PT}^+$ in LT will be directly applicable to these
logics too. Table \ref{list_of_logics} gives an overview of these weaker
logics that can be found in the literature. The logic $\text{PD}^\vee$ is
described in \cite{Yang2016}, and the others are described in \cite{Yang2017}.

\begin{table}
	\begin{tabular}{ll}
	\toprule
		Propositional team logics. 								& Connectives \\
		\midrule 
		Classical propositional logic (CPL) 					&$\slnot P_i,\ibot,\ilor,\land$\\
		Strong classical propositional logic ($\text{CPL}^+$) 	&$\slnot P_i,\ibot,\ilor,\land,\NB$\\
		Propositional union closed logic ($\text{PU}$)		&$\slnot P_i,\ibot,\ilor,\land,\circledast$ \\
		Strong propositional union closed logic ($\text{PU}^+$)&$\slnot P_i,\ibot,\ilor,\land,\circledast,\NB$ \\
		Propositional dependence logic w. int. disj. ($\text{PD}^\vee$)& $ \slnot P_i,\ibot,\ilor, \land,\lor$ \\
		Propositional team logic ($\text{PT}$)				& $\slnot P_i,\ibot,\ilor,\land,\lor,\circledast$ \\
		Strong propositional team logic ($\text{PT}^+$)		& $\slnot P_i,\ibot,\ilor,\land,\lor,\NB$\\
		\bottomrule
	\end{tabular} 
\caption{Names and included connectives of logics described in
\cite{Yang2016} and \cite{Yang2017} as fragments of $\text{PT}^+$. The
connective $\circledast$ can be defined in $\text{PT}^+$ as $\phi \circledast
\psi := (\phi \land \NB) \protect\ilor (\psi \land \NB)$. By $\slnot P_i$ we mean that
$\slnot$ is only allowed to be applied to propositional variables.}
\label{list_of_logics}
\end{table}

\subsection{$\text{PT}^+$ as a model of LT} 

Observe that the denotations of valuational team semantics are elements of the
set $\P\P 2^{\mathbb{N}}$, which can be interpreted as a model of LT by
interpreting $\P 2^{\mathbb{N}}$ as a Boolean algebra using the standard set
operations. We will refer to this as \textit{the valuation model}. With this
reading we can see that the interpretation of atomic formulas imposes a
specific homomorphism that maps every atomic formula to the set of teams for
which every member evaluates it to true.

Define the \textit{valuation homomorphism} $H_V : \Fm_{\text{LT}} \to \P\P
2^\mathbb{N}$ as the unique homomorphism such that
\[ 
H_V(P_i) = \set { X\in \P 2^{\mathbb{N}} | \text{ for all } s\in X, s(i)= 1 }
= \P \set{ s\in 2^{\mathbb{N}} | s(i) = 1}.
\] 
Thus, 
\[
H_V :\Delta \vDash \phi \quad \text{iff} \quad \bigcap_{\delta \in
\Delta}H_V(\delta) \subseteq H_V(\phi).
\]
Let us formulate this in the following theorem.

\begin{thm}\label{directThm}
	For all formulas $\Delta\cup \{\phi\} \subseteq \Fm_{\mathrm{PT}^+}$:
	\[
		\mathrm{PT}^+: \Delta \vDash \phi \quad \text{ iff } \quad  H_V
	: \Delta \vDash \phi.
	\]
	Hence, if  $ \text{LT} : \Delta \vDash\phi$, then $\mathrm{PT}^+: \Delta \vDash \phi$.
\end{thm} 


\subsection{Axiomatising a specific class of homomorphisms}\label{Ax-class-hom}

We observe that the valuation homomorphism $H_V$ has the following special
property. 
\[ 
H_V (P_i)= \P X \text{ for some } X \in \P 2^{\mathbb{N}}
\]
Algebraically speaking, every propositional variable is mapped to a non-empty
principal ideal of the Boolean algebra on $\P(2^\mathbb{N})$, that is, a
subset $A$ of the Boolean algebra $B$ such that there is a maximal element
$a\in A$ generating $A$, i.e., such that $$A= \set{ b\in B | b\leq a}.$$ In
other words, $A$ is downwards closed and $\bigvee A\in A$.

We can in fact express the property of being a principal ideal in LT in the
following formula akin to the excluded middle:

\begin{thm}\label{excluded_middle} 
	For all Boolean algebras $B$, all homomorphism $ H: \Fm \to \P B$, and all
	formulas $\phi \in \Fm$ we have $$ H: ~ \vDash \phi \ilor \slnot \phi
	\quad \text{ if and only if } H(\phi) \text{ is a principal ideal} $$
\end{thm}

As part of proving this statement we will use a small result that will be
useful to refer to multiple times in upcoming proofs. We therefore state it
as a separate lemma.
\begin{lemma}\label{negated_principal}
	For a Boolean algebra $B$, if $A\subseteq \P B$ is a principal ideal with maximal element $a\in
	B$, then $\slnot A$ is a principal ideal with maximal element $\lnot a$.
\end{lemma} 
\begin{proof}[Proof of lemma]
First note that for all $A\subseteq B$ we have that $\slnot A$ is downwards
closed, since if $c\leq c'\in \slnot A$ then for all $b\in A$, $ b\land c \leq
b\land c' = \bot$. What is left to show is that, if $a= \bigvee A$, then
$\neg a = \bigvee \slnot A$. Clearly, for all $a'\in A$ we have
$a'\land a = a'$ and thus $\lnot a \land a' = \bot$, and thus $\lnot a \in \slnot
A$. Furthermore, if $b\in \slnot A$, then $a\land b = \bot$ and
therefore $\lnot a = \lnot a \lor (a \land b)= \lnot a \lor b$. i.e $b\leq
\lnot a$. thus $\lnot a $ is an upper bound of $ \slnot A$ included in
the set, and thus the set is a principal ideal, and $\lnot a = \bigvee
\slnot A$.
\end{proof}
\begin{proof}[Proof of Theorem \ref{excluded_middle}]

For one direction, assume $H(\phi \ilor \slnot \phi)= B$, we then need to
prove that  $H(\phi)$ is a principal ideal. First we establish that it is
downwards closed.  In search of a contradiction, assume that for
some $a\in H(\phi)$ we can find $b\in B$ such that $b\leq a$ and $b\notin
H(\phi)$. We want to show that then $b\notin H(\phi \ilor \slnot
\phi)$. If $b\in H(\phi\ilor \slnot \phi)$, then there exists $c\in H(\phi)$
and $d \in \slnot H(\phi)$ such that $c\lor d= b $. Since $b\leq a \in
H(\phi)$ we have that $$ b\land d \leq a\land d = \bot$$ since $d\in
\slnot H(\phi)$. But then, since clearly $c\leq b$, we have $$b= b\land b
= (c\lor d ) \land b = ( c\land b )\lor (b\land d)= c\in H(\phi)$$ This is a
contradiction. We can therefore conclude, under the main assumption, that
$H(\phi)$ is downwards closed. Next we show that $\bigvee H(\phi)\in H(\phi)$.
By assumption we have that  $$\top \in H(\phi \ilor \slnot
\phi)$$ Then there exists $c \in  H(\phi)$ and $d \in H(\slnot \phi )$ such
that $c\lor d=\top$. Furthermore, for all $a \in H(\phi)$ we have that $$a =
\top \land a = (c\lor d) \land a = c \land a, $$ since $d \land a  =
\bot$ for all $a \in H( \phi) $. Consequently $a \leq c $ for all $a
\in H( \phi)$ and thus $ c$ is an upper bound for $H(\phi)$ included in
the set, i.e. $c  = \bigvee H(\phi)$. With downwards closure established, this
also means that $H(\phi)$ is a principal ideal.
	
	
For the other direction, assume $H(\phi)$ is a principal ideal. Then there
exists $a \in B$ such that $a$ is the top element of $H(\phi)$. Then by Lemma
\ref{negated_principal} we see that  $\lnot a$ is the top element of the
principal ideal $H(\slnot \phi)$. Therefore, since $a \lor \neg a =
\top$ we have for every $b\in B$ that $$b= b \land (a\lor \neg a)= (b\land a )
\lor (b\land \neg a).$$ Being the top elements of the respective principal
ideals we observe that $$b\land a \in  H(\phi)\quad \text{and} \quad b
\land \neg a \in H(\slnot \phi)$$ and conclude that  $ H(\phi  \ilor
\slnot \phi)= B$, in other words $H\vDash \phi \ilor \slnot \phi$.
\end{proof}

From this theorem we can directly conclude for the valuation homomorphism
$H_V$ that
\[
H_V : {}\vDash P_i \ilor \slnot P_i \quad \text{for all } i \in
\mathbb{N}.
\]
We will see that this is the crucial categorisation of the homomorphisms that
relate to valuational team logics. We therefore identify the class defined by
these formulas, and the corresponding axiomatisation as discussed in Section
\ref{sec:def-class}.

\begin{defin}
	Let $\mathcal{H}_{\text{PV}}$ denote the class of homomorphisms defined by
	$\set{ P_i\ilor \slnot P_i | i\in \mathbb{N}}$. We say that a
	homomorphism $H$ \textit{has principal  variables} iff
	$H\in\mathcal{H}_{\text{PV}}$. Furthermore, let \emph{the principal variable
	axioms} be the set
	\[
	\text{PVA}=\set{\Box ( P_i \ilor P_i) |i\in	\mathbb{N}}.
	\]
\end{defin}
 
 It follows directly from Theorem \ref{Def_to_ax} that, for all $\Delta,\set{\phi}\subseteq \Fm_{\text{LT}}$,
 \[
 	\mathcal{H}_{\text{PV}} :	\Delta \vDash \phi \quad \text{if and only if} \quad \text{LT}:\text{PVA},
	\Delta\vDash \phi.
\]
It is evident that $H_V\in \mathcal{H}_{\text{PV}}$. 


\subsection{Axiomatisation of $\text{PT}^+$ as a fragment}

In this section we prove that the axioms PVA axiomatise $\text{PT}^+$ in the
sense of the following theorem.
\begin{thm}\label{thm:mainAx}
	For all $\Delta \cup \{ \phi \} \subseteq \Fm_{\mathrm{PT}^+}$ 
	\[ 
	\mathrm{PT}^+: \Delta \vDash \phi \quad \text{iff} \quad \mathrm{LT} :
	\mathrm{PVA}, \Delta \vDash \phi.
	\]
\end{thm}
One direction is already proved by Theorem \ref{Def_to_ax} and
\ref{directThm}. To prove the other direction we first recall the fundamental
result by Stone in the theory of Boolean algebras.

\begin{thm}[\cite{Halmos2009}]\label{thm:completeBA}
	Every Boolean algebra can be embedded into a complete atomic Boolean
	algebra of the form $(\P S, \emptyset, \cdot^C, \cup, \cap)$ for some set
	$S$.
\end{thm}

Using this theorem we can establish Theorem \ref{thm:mainAx} in a two step
process. First showing that any homomorphism $H\in \mathcal{H}_{PV}$ can be
faithfully represented by a homomorphism in a Boolean algebra of subsets, and
then that any such homomorphism can be represented by the specific valuational
homomorphism $H_V: \Fm  \to \P \P 2^{\mathbb{N}}$. These steps are established
below in Lemma \ref{embedding_lemma} and Lemma \ref{f_rep_in_Val}
respectively. Both proofs are similar in structure, where the statement is
proven by induction over the complexity of formulas in $\mathrm{PT}^+$ for
which particular care is needed to handle the internal connective $\ilor$. 
In the first case this invokes the formulation of an additional sub-lemma, and in
the second case it suffices with a simpler result expressed as a claim.

%\marginpar{additional sentence or to involved allready?}
\begin{lemma}\label{embedding_lemma}
	Let $B,B'$ be Boolean algebras, and $e:B\hookrightarrow B'$ an embedding
	of Boolean algebras. Then for each homomorphism $H:
	\Fm_{\text{LT}} \to \P B $ with principal variables there  is a
	homomorphism $H': \Fm_{\text{LT}} \to \P B'$  with principal variables
	such that for all formulas $\phi \in \Fm_{\text{PT}^+}$ $$ b\in H(\phi)
	\quad \text{ if and only if } \quad e(b) \in H'(\phi).$$
\end{lemma} 
\begin{proof}
	Assume $H: \Fm \to \P B$ has principal variables. Thus, for all $i$ there
	exists $b_i\in B$ such that $$H(P_i)= \set{ a\in B | a\leq b_i  }.$$ We
	then define $H': \Fm \to \P B'$ as the homomorphism that maps each
	propositional variable $P_i$ to the principal ideal of $e(b_i)$, i.e.,
	$$H'(P_i)= \set{ a'\in B'| a'\leq e(b_i)}.$$
	
	Clearly, $H'$ has principal variables, so what is left to show is that the
	equivalence in the theorem holds for all $b\in B$ and all formulas $\phi$
	of PT$^+$. In order to do so, we need an additional lemma best described
	using the notation of Boolean intervals.\footnote{The introduction of
	Boolean intervals in this context is inspired by their usage in
	\cite{Hella2023} in which minimal covers of Boolean intervals are used to
	define complexity measures for expressions in first-order team semantics.
	Even if our usage is noticeably different, the proof construction we
	present for forming included covering intervals is strongly related to the
	constructive methods of generating minimal interval covers for collections
	of teams satisfying a formula presented in that paper.}
	\begin{defin} 
		In a Boolean algebra $B$, if $a,b\in B$ and $a\leq b$, then let
		$[a,b]\subseteq B$ denote the set of elements between $a$ and $b$,
		that is $$[a,b]= \set{c\in B | a\leq c \text{ and } c\leq b  }. $$ We
		call this the closed interval of $a$ and $b$, and if $a=b$ we may
		write $[a]$ instead of $[a,a]$.
	\end{defin}  

\begin{lemma}[Interval lemma]For all Boolean algebras $B,B'$, Boolean
embedding
$e:B\hookrightarrow B'$, homomorphisms $H:Fm\to \P B$ and the
generated homomorphism $H': \Fm \to \P B'$  as described above, we have that for
all $\phi\in \Fm_{\text{PT}^+}$ and all $b'\in B'$, if $b'\in H'(\phi)$,
then there is a $b\in B$ such that $b'\leq e(b)$ and $[b',e(b)]\subseteq
H'(\phi)$.
\end{lemma} 
\begin{proof}
	We prove this statement for all $b\in B$ by induction over the complexity
	of formulas. Let $b_i$ denote the generating element of the principal
	ideal $H(P_i)$.
	\begin{itemize}
		\item Assume $\phi = \ibot$. Then $b'\in H'(\ibot)$ only if $b'=  \bot
		= e(\bot)$, and thus $[b', e(\bot)]= [\bot] \subseteq H'(\ibot)$.
		\item Assume $\phi = \NB $. Then if $b'\in H'(\NB)$ clearly
		$[b',e(\top)]\subseteq H'(\NB)$.
		\item Assume $\phi = P_i$. If $b'\in H'(P_i)$, then $b'\leq e(b_i)$
		and, since $H'(P_i)$ is a principal ideal, clearly
		$[b',e(b_i)]\subseteq  H'(P_i)$.
		\item Assume $\phi = \slnot P_i $. By Lemma \ref{negated_principal}
		and a similar argument as for the previous case we can conclude that
		if $b'\in H'(\slnot P_i)$ then $[b', e(\lnot b_i)] \subseteq H'(\slnot
		P_i)$.
		\item Assume $\phi = \psi \land \chi$. If $b'\in H'(\phi)$, then
		$b'\in H'(\psi)$ and $b'\in H'(\chi)$. by induction, we may find
		$b_\psi,b_\chi\in B$ such that $[b',e(b_\psi)]\subseteq H'(\psi)$ and
		$[b',e(b_\chi)]\subseteq H'(\chi)$. Since $e$  is a
		homomorphism $$[b',e(b_\psi \land b_\chi)]\subseteq H'(\phi).$$
		\item $\phi = \psi \lor \chi$. If $b' \in H'(\phi)$, then $b'\in
		H'(\psi)$ or $b'\in H'(\chi)$, and without loss of generality we may
		assume the former. By induction hypothesis there is some $b_\psi \in
		B$ such that $[b',e(b_\psi)]\subseteq H'(\psi)$, and thus
		$[b',e(b_\psi)]\subseteq H'(\phi)$.
		\item $\phi = \psi \ilor \chi$. If $b'\in H'(\phi)$, then there is $
		c'\lor d'= b'$ such that $c'\in H'(\psi)$ and $d'\in H'(\chi)$. By
		induction hypothesis, we can then find $b_\psi, b_\chi \in B$ such
		that $[c',e(b_\psi)]\subseteq H'(\psi)$ and $[d',e(b_\chi)]\subseteq
		H'(\chi)$. Then since $b'= c'\lor d'$ and $e$ is a homomorphism, it is
		easy to see that $$H'(\phi)\supseteq \set{ p\lor q | p\in
		[c',e(b_\psi)], q \in [d', e(b_\chi)]} = [b', e(b_\psi \lor
		b_\chi)].$$
	\end{itemize}
	This concludes the proof of the interval lemma.
\end{proof}

Now we are ready to finish the proof of Lemma \ref{embedding_lemma}. Again
this is achieved for all $b\in B$ by induction over the complexity of
formulas.
	\begin{itemize}
	\item Assume $\phi = \ibot$. First note that $b\in H(\ibot)$ if and only
	if $ b= \bot\in B$ and  $b'\in H'(\ibot)$ if and only if $b'= \bot \in
	B'$. Then note that $e(\bot)= \bot\in B'$ by $e$ being a homomorphism, and
	by $e$ being injective $e(b)= \bot$ if and only if $b= \bot\in B$. These
	observations suffice to prove the statement.
	\item Assume $\phi = \NB$. Since regardless of homomorphism and algebra
	$H(\NB)$ is the complement set of $H(\ibot)$, a similar argument proves
	the statement.
	\item Assume $\phi = P_i$. For all $b\in B$ we have that  $b\in H(\P_i)$
	if and only if $b\leq b_i$. Then by $e$ being an embedding this holds if
	and only if  $e(b)\leq e(b_i)$, which is equivalent to $e(b)\in H'(P_i)$.
	\item Assume $\phi = \slnot P_i$. By Lemma \ref{negated_principal}
	$H(\slnot P_i)$ is the principal ideal generated by $\lnot b_i$. Similarly,
	$H'(\slnot P_i) $  is the principal ideal generated by $\lnot e(b_i)$.
	Thus, a similar argument as for $\phi = P_i$ suffices.
	\item Assume $\phi = \psi \land \chi$. Then $b\in H(\phi)$ if and only if
	$b\in H(\psi)$ and $b\in H(\chi)$. By induction we can conclude that this
	holds if and only if  $e(b)\in H'(\psi)$ and $e(b)\in H'(\chi)$ which
	holds if and only if  $e(b)\in H'(\phi)$.
	\item Assume $\phi = \psi \lor \chi$. The result follows in a similarly
	standard way as the previous case.
	\item Assume $\phi = \psi \ilor \chi$. If $b\in H(\phi)$, then there
	exists $c\lor d = b$ such that $c\in H(\psi)$ and $d\in H(\chi)$. By
	induction hypothesis we directly see that $e(b)= e(c)\lor e(d) \in
	H'(\phi)$. For the other direction, assume $e(b)\in H'(\phi)$. Then there
	exists $c'\lor d' = e(b)$ such that $c'\in H'(\psi) $ and $d'\in
	H'(\chi)$. Now by the interval lemma above we can find $b_\psi, b_\chi \in
	B$ such that $[c',e(b_\psi)]\subseteq H'(\psi)$ and
	$[d',e(b_\chi)]\subseteq H'(\chi)$. Now, let $c= b\land b_\psi$ then
	$$e(c)= e(b)\land e(b_\psi)= (c'\lor d')\land e(b_\psi)= c'\lor  (d' \land
	e(b_\psi))$$ so that clearly $e(c)\in [c', e(b_\psi)]\subseteq H'(\psi)$,
	and by induction hypothesis $c\in H(\psi)$. Similarly define $d= b\land
	b_\chi$ and conclude that $d\in H(\chi)$. What is left to show is that $b=
	c\lor d$. We see that
	\begin{multline*}
	c\lor d= (b\land b_\psi) \lor (b \land b_\chi)= (b
	\lor (b \land b_\chi)) \land (b_\psi\lor (b \land b_\chi))= \\b\land
	((b_\psi \lor b ) \land (b_\psi\lor b_\chi) )= b, 
	\end{multline*}
	where the last  equality comes from the observation that $e(b)= c'\lor d'
	\leq e(b_\psi)
	\lor e( b_\chi) = e(b_\psi\lor b_\chi)$, and by $e$ being injective
	letting us conclude that $b\leq b_\psi\lor b_\chi$.
\end{itemize} This concludes the proof of Lemma \ref{embedding_lemma}.
\end{proof}

\begin{lemma}\label{f_rep_in_Val}
For every Boolean algebra of the form $(\P S, \emptyset, \cdot^C, \cup ,\cap)$
and every homomorphism $H:\Fm_{\text{LT}} \to \P \P S$ with principal
variables there is a mapping $f_H:S \to 2^{\mathbb{N}}$  such that for all
formulas $\phi \in \Fm_{\text{PT}^+}$ and all $X\in \P S$:
\[ 
X\in H(\phi) \quad \text{iff} \quad f_H^*(X) \in H_V(\phi),
\] 
where $f_H^*:\P S \to \P 2^{\mathbb{N}}$ is defined by $f_H^*(X) = \set{f_H(x) | x
\in X}$, and $H_V:\Fm_{\text{LT}} \to \P\P 2^\mathbb{N}$ denotes the valuation
homomorphism.
\end{lemma}
\begin{proof}
	Assume $H:\Fm_{\text{LT}} \to \P \P S$ has principal variables. Let $f: S\to
	2^{\mathbb{N}}$ be defined by
	\[
		f(s)(i)= \begin{cases} 1 & \text{ if } \{s\} \in H(P_i) \\ 
								0 & \text{ if } \{s\} \notin  H(P_i).
					\end{cases}
	\]

	\begin{quote} 
	\textbf{Claim:} For all $X,Y\in \P S$, $f^*(X\cup Y) = f^*(X)\cup f^*(Y)$.
	Furthermore, if $f^*(X) = U\cup V$, then there exist $Y,Z\in \P S$ such
	that $f^*(Y)= U$, $f^*(Z)= V$ and $Y\cup Z= X$.
	\end{quote}

	The first part of the claim is self-evident by the definition of $f^*$
	from $f$. For the second part, let $Y= \set{s\in X| f(s)\in U}$ and $Z=
	\set{s\in X | f(s) \in V}$.
	
	We can now prove that the function $f$ is what we looked for in the lemma
	by induction over formulas $\phi\in	\Fm_{\text{PT}^+}$.
	
	We have four types of base cases:  $\ibot, \NB, P_i, \slnot P_i $:
	\begin{itemize}
		\item Assume $\phi = \ibot$. By definition $X\in H(\ibot)$ iff $X= \emptyset$, and
		since $f^*(X) = \emptyset $ if and only if $X= \emptyset$ this case is evident.
		\item Assume $\phi = \NB$. This case is proved by contraposition of the previous
		case.
		\item Assume $\phi = P_i$. By assumption $H(P_i)$ is a principal ideal in the Boolean
		algebra $\P S$. Hence, for all $X \in H(P_i)$, by downwards closure we have
		for all $s\in X$ that $\{s\} \in H(P_i)$. By construction then
		$f^*(\{s\}) \in H_V (P_i)$ for all $s\in X$, and thus since $H_V$ has
		principal variables we find that $f^*(X)\in
		H_V(P_i)$. The opposite direction is proved with a similar chain of
		arguments.
		\item Assume $\phi = \slnot P_i$. Then by Lemma
		\ref{negated_principal} we assert that for all $H\in \mathcal{H}_{PV}$
		we have that $H(\slnot P_i)$ is a principal ideal. The proof is then
		similar to the previous case.
	\end{itemize}

	For the induction step we have three cases for the main connectives:
	$\land ,\lor,\ilor$
	\begin{itemize}
		\item Assume $\phi = \psi \land \chi$. $X\in H(\phi)$ by definition if
		and only if $X\in H(\psi)$ and $X\in H(\chi)$. By induction
		hypothesis, we can conclude that is the case if and only if $f^*(X)\in
		H_V(\psi)$ and $f^*(X)\in H_V(\chi)$, which is equivalent to stating
		that $f^*(X) \in H_V(\phi)$.
		\item Assume $\phi = \psi \lor \chi$. The proof is similarly standard
		as the previous case.
		\item Assume  $\phi = \psi \ilor \chi$. For one direction, assume $X \in
		H(\phi)$. Then there exist $Y\in H(\psi)$ and $Z\in H(\chi)$ such
		that  $X= Y\cup Z$. By induction hypothesis $f^*(Y)\in H_V(\psi)$ and
		$f^*(Z)\in H_V(\chi)$. By the first part of the Claim about $f^*$ we
		see that $f^*(X)= f^*(Y)\cup f^*(Z)$ and thus $f^*(X)\in H_V(\phi)$.
		
		For the other direction, assume $f^*(X) \in H_V(\phi)$.  Then there
		exist $U\in H_V(\psi), V\in H_V(\chi)$ such that $f^*(X)=U\cup V$.
		Then by the second part of the Claim about $f^*$, there exists $Z, Y$
		such that $f^*(Z)=U, f^*(Y)= V$ and $Z\cup Y= X$. By the induction
		hypothesis  $Z\in H(\psi)$ and $Y\in H(\chi)$.  We conclude
		that $X\in H(\phi)$.
	\end{itemize}
	This concludes the proof of the lemma.
\end{proof}

We are now ready to prove Theorem \ref{thm:mainAx}.

\begin{proof}[Proof of Theorem \ref{thm:mainAx}] 
	The theorem is equivalent to the statement that for all $\Delta \cup \{
	\phi \} \subseteq \Fm_{\text{PT}^+}$ 
	\[
	\text{PT}^+:\Delta \vDash \phi \quad \text{iff} \quad
	\mathcal{H}_{\text{PV}} :  \Delta \vDash \phi.
	\]
	One direction follows directly from Theorem \ref{directThm} and the fact
	that $H_V\in\mathcal{H}_{\text{PV}} $. 

	The other direction is proved by contraposition. Assume
	$\mathcal{H}_{\text{PV}} :  \Delta \nvDash \phi$ and thus  $\text{LT} :
	\text{PVA}, \Delta \nvDash \phi$.  Then there is some Boolean algebra $B$ and 
	homomorphism $H: \Fm_{\text{LT}} \to  \P B$ such that $ H:
	\text{PVA},\Delta \nvDash \phi.$ In other words, $H\in
	\mathcal{H}_{\text{PV}}$ and there exists an element $ b\in B $ such that
	\[
	b \in \bigcap_{\delta\in \Delta} H(\delta), \text{ but } b \notin H(\phi).
	\]  
	By Theorem \ref{thm:completeBA}, we can then find an embedding $e :B \hookrightarrow \P{S}$ of $B$ into a complete atomic Boolean algebra of the form $(\P S, \emptyset, \cdot^C, \cup
	,\cap)$. Then by lemma \ref{embedding_lemma} and \ref{f_rep_in_Val}  we can find a homomorphism $H':\Fm_{\text{LT}}\to \P\P S$, and a mapping $f_{H'}: S\to 2^\mathbb{N}$  such that 
	\[
	f_{H'}^*(e(b)) \in \bigcap_{\delta\in\Delta} H_V(\delta), \text{ but }
	f_{H'}^*(e(b))\notin H_V(\phi).
	\]
	Thus, $H_V :\Delta \nvDash \phi$ and by Theorem \ref{directThm}: 
	\[
	\text{PT}^+:\Delta \nvDash \phi.
	\] 
	This finalises the proof of Theorem	\ref{thm:mainAx}.
\end{proof}

Observe that to evaluate $\text{PT}^+$ it is sufficient to consider the
valuational algebra $\P 2^\mathbb{N}$ i.e. for all  $\Delta, \{\phi\}
\subseteq\Fm_{\text{PT}^+}$ we have 
\[
\text{PT}^+ : \Delta \vDash \phi \quad\text{iff}\quad \P 2^\mathbb{N}:
\text{PVA},\Delta \vDash \phi.
\]
In this sense $\P 2^\mathbb{N}$ can be seen as canonical for $\text{PT}^+$. It
is however \textit{not} canonical for the logic in LT axiomatized by PVA,
since the canonicity only holds when the formulas are restricted to the
language $\Fm_{\text{PT}^+}$.


%!TEX root = logic_of_teams.tex
\section{Conclusion}

In this paper, we have introduced a new substitutional logic of teams, LT,
with a natural semantics inspired by algebraic semantics together with a sound
and complete labelled natural deduction system. We have also shown that the
logic has a finite model property in that the finite Boolean algebras are
adequate for the semantics of LT, proving that the entailment of LT is
decidable. Furthermore this gives the possibility to axiomatise different
propositional team logics, such as propositional dependence logic, as a
fragment of LT. From the initial choice of a more algebraic perspective, we
see that the constructions of the semantics, the natural deduction system and
the relative axiomatisation presented, seem fairly straight-forward and
sensible without significant arbitrary choices. This indicates that an
analysis of the structure of these constructions should be informative for an
over-arching view of team logics from an algebraic perspective. By focusing on
different parts of this construction we highlight some research topics that
are exposed by the work in this paper.

The logic LT is fully substitutional, so the full toolbox of algebraisation is
open to us. Furthermore, in the language of abstract algebraic logic
\cite{Font16}, LT is a finitary implicative logic (with respect to the
exterior implication). We can then conclude by application of general results
that:
\begin{thm}
	LT is algebraisable and its equivalent algebraic semantics
	\\$\mathcal{A}\textit{lg}^*\text{LT}$ (the largest class of algebras LT is
	algebraisable with respect to) is a quasi-variety that can be presented by
	the equations and quasi-equations that result by applying the
	transformation $\phi \mapsto \phi \approx \top $ to the axioms and rules
	of any Hilbert-style presentation of LT.
\end{thm}
\begin{proof}
	This conclusion can be drawn from the general results presented in
	\cite{Font16}. The theorem follows from   Definitions 2.3 and 2.5;
	Propositions 2.7 and 3.15; and the direct comment after Defininion 3.19 in
	the aforementioned work.
\end{proof}

We have at this point not investigated Hilbert-style proof systems further. By
compactness together with the deduction theory for external implication, we
can however conclude that it is possible to present a fully axiomatic Hilbert
system, with modus-ponens and substitution as its only rules, and hence that
$\mathcal{A}\textit{lg}^*\text{LT}$ is a variety. Apart from the direct
observation that the algebras $\P B $ where $B$ is a Boolean algebra are
included in $\mathcal{A}\textit{lg}^*\text{LT}$ we know little about this
class at the moment and it should be further investigated. One way of doing so
is by identifying an informative choice of Hilbert system for the logic with
corresponding equational theory of the $\mathcal{A}\textit{lg}^*\text{LT}$.
Another approach is to find more general ways than the powerset-operation to
generate algebras of $\mathcal{A}\textit{lg}^*\text{LT}$ from Boolean
algebras. If $\mathcal{A}\textit{lg}^*\text{LT}$ can be exhausted by algebras
uniformly generated by Boolean algebras it would illuminate and identify the
relation between these classes of algebras and provide a clear semantic
connection between LT and classical propositional logic.

To relate LT to other propositional team logics we have identified a set of
flat variable axioms (FVA). This set of axioms can be viewed as the obvious
set of axioms that captures the denotational semantics of the other logics in
the semantics of LT. It is therefore expected that similar algebraic
constructions and axiomatisations are possible for other types of team
semantics such as modal team semantics \cite{Vaeaenaenen2008}. Furthermore, this
axiomatisation provides a way to construct proofs of the entailment statements
of these propositional team logics. It does however not directly constitute a
natural deduction system for the axiomatised logics per se, since the terms of
these proofs will in general not be confined to the syntactical fragment of
the logics. Our natural deduction system may however motivate, and be seen as
a guide in the construction of deduction systems for these propositional team
logics, and indicates a suitability of labelled systems.

In the labelled natural deduction for LT, the rules $\sub$ and $\taut$
establishe a notion of equivalence of labels determined by classical
propositional logic. From this perspective these rules can be viewed as
structural rules of the deduction system. The rules for the formulas of LT
consists of introduction rules together with the elimination rules that are
the direct inverses of the introduction rules (up to equivalence of
labels).\footnote{The rule $\ine$ is not directly the inverse of $\ini$ but
can be seen to be equivalent to a direct inverse rule. } In this sense, the
rules of LT harmonise, and it is possible to use more advanced proof
theoretic methods to investigate LT. For example, it seems to be easy to turn
the system into a sequent system that could be analysed with respect to cut
rules and cut elimination. This analysis may lead up to a proof theoretic
explanation of the internal connectives and so also of the connectives in
other propositional team logics through their axiomatisation.


By thinking of the $\taut$ rule as provability in an \emph{inner logic} we can
generalise the construction the labelled natural deduction system into a proof
theoretic teamification, or combination of two logics, an inner and an outer
logic. Of such constructions LT represents the special case for which both are
classical propositional logic. This opens the door for a purely
proof-theoretic approach to team logics, and more general relatives, and we
see this as an interesting future research topic. In particular this indicates
the possibility for similar constructions in a first order setting that could
give new general insight into first order team logics, which is an active
field of study with many applications.

%\input{logic_of_teams_sec99_appendix}

\section*{Funding}

The first author was supported by grant 2022-01685 of the Swedish Research Council, Vetenskapsrådet.

%\printbibliography
\bibliographystyle{asl}
\bibliography{refs}

\end{document}
