%!TEX root = logic_of_teams.tex

\section{Definability}\label{sec:definable}

 We have defined an algebra based on external and internal Boolean
 connectives. Using these connectives, we can define a set of other constants
 and connectives; some of the usual suspects, and some that will be useful for
 the connection to valuational team semantics. It will be clear from the
 definitions we give that there is plenty of similar constructions to define
 similar connectives. In this paper we focus on the connectives that will be
 important for expressing traditional propositional team semantics, and leave
 further exploration for now. We name connectives after their interpretation
 in the intended semantics.

 \subsection{Constants}

We define the following constants with corresponding semantic interpretation
for every Boolean algebra $B$ and every homomorphism $H:\Fm \to \P B$:
\begin{itemize}
 	\item $\top~~= \lnot\bot$. Then $H(\top) =\P(B)$ and is called
 	\emph{the external top}.
 	\item $\NB =\lnot \ibot$.  Then $H(\text{NB})= \{\bot\}^C=
 	\set{x | x\neq \bot}$ and is called the  \emph{not-bottom constant}.
 \end{itemize}
Remember that we previously defined $\itop =\ilnot \ibot$, \emph{the internal
top}. 

In addition to the constants $\bot$ and $\ibot$ these are the constants it
will be useful for us to name. There are more constants definable in LT. Note
however that the interpretations of these constants may coincide for some
algebras $\P B$. This observation underlies the failure of canonicity for
single algebras explored in section \ref{sec:canonical}.


\subsection{Closures}
We define two useful closure operators with corresponding semantic
interpretation for every Boolean algebra $B$ and every homomorphism $H:\Fm \to
\P B$:
\begin{itemize}
		\item $\dwn \phi = \phi \iland \top$.  Then $H(\dwn \phi) = \set{ b \in B|
		\text {there exists } a \in H(\phi) , \text{ s.t. } b\leq a}$, called the
		\textit{downwards closure} operator.
		\item $\up \phi = \phi \ilor \top$.  Then  $H(\up \phi) = \set{ b \in B | \text
		{there exists } a \in H(\phi), \text{ s.t. } b\geq a}$, called the
		\textit{upwards closure} operator.
\end{itemize} 
For elements of a Boolean algebra we use the standard understanding of $a \leq
b$, as $a\land b = a$.

Note that the closure operators are projections in the sense that $H(\dwn \dwn
\phi)= H(\dwn \phi)$. Furthermore, $\top$ and $\bot$ are their only common
fix-points: $H(\dwn\bot)= H(\bot) = H(\up \bot)$. We also have the following
properties:
\begin{gather*}
H(\dwn \phi) = \emptyset \text{ iff } H(\phi) = \emptyset, \text{ and
otherwise } \bot \in H(\dwn \phi), \\ 
H(\up \phi) = \emptyset \text{ iff } H(\phi) = \emptyset, \text{ and otherwise }
 \top \in  H(\up \phi),\\
H(\dwn \phi) = \P B \text{ iff } \top \in H(\phi), \text{ and} \\
H(\up \phi) = \P B \text{ iff  } \bot \in H(\phi).
\end{gather*}

We can use these properties to define universal modal operators.
%%%(XXXX maybe see these closures as "forward and backwards " modalities?)

\subsection{Modal operators}
When considering a homomorphism $H : \mathtt{Fm} \to \P{B}$, it is interesting
to have an `existence', or `possibility' operator $\Diamond$ such that for all
formulas $\phi$, algebras $\P B$, and all homomorphisms $H$,

$$ H(\Diamond \phi)  = \begin{cases} \P B & \text{if there exists } x\in H(\phi)\\ \emptyset & \text{otherwise} \end{cases} $$

Similarly, we consider a `universal' or `always' operator $\Box$,
such that

$$H(\Box\phi)  =\begin{cases} \P B & \text{if for all }  x,\quad  x\in H(\phi)\\ \emptyset & \text{
otherwise} \end{cases} $$

with the symbols $\Box$ and $\Diamond$ chosen for the clear role as universal
modal operators for the Boolean algebra. Inspecting the described properties
of the closure operators $\dwn$ and $\up$ it is clear that these connectives
can be defined in the following manner.

\begin{defin}
For any algebra $\P B$ we define the universal and existential operators
$\Box$ and $\Diamond$ as follows:
\begin{align*}
	\Diamond \phi &= \up \dwn \phi  = (  \phi\iland \top) \ilor\top\\
	\Box A &= \lnot \Diamond \lnot A
\end{align*}
\end{defin}

Interestingly enough, we have in some sense found a decomposition of the
universal modalities on a set of worlds that happen to be a Boolean algebra,
so that we have the following.

\begin{thm}
	The fragment $(\bot, \lnot, \lor ,\land, \Box, \Diamond)$ of LT captures exactly the\\
	modal logic S5.
\end{thm}

Note that the interpretation of the connectives mentioned in the above theorem
do not depend on the underlying/internal structure of the algebra, and the
underlying Boolean algebra functions merely as `a set of worlds' with $\Box$
and $\Diamond$ interpreted for the universal relation.

The $\Box$ operation however plays a more important role in this paper
in that it facilitates internalisation into formulas of classifications of
important global properties of homomorphism.


\subsection{Definable classes of homomorphisms}\label{sec:def-class}

The semantic definition of entailment in LT is given as a universal
satisfaction of a property evaluated independently for all homomorphisms for
all Boolean algebras. Just like how, in terms of Kripke models, semantics for
intuitionistic logic is given by a restriction on valuations to those
satisfying a persistence criteria, we are interrested in logics
that can be defined by a restriction on the class of homomorphisms that are
considered. We also describe a notion of \textit{definability} in LT of such classes, and by
using the $\Box$ operator achieve axiomatisations in LT of the logics of
definable classes of homomorphisms.

\begin{defin}
Let $H$ be a homomorphism $\Fm \to \P B$. We
define the \emph{local entailment} of $H$ denoted $H: \Delta \vDash \phi $ in
the expected way:
$$H: \Delta \vDash \phi \quad \text{ iff } \quad
\bigcap_{\delta\in \Delta} H(\delta) \subseteq H(\phi)$$
Given a class of homomorphisms $\mathcal{H}$ (not necessarily all to the same
algebra), we can define the logic of the  class-entailment
$$ \mathcal{H} : \Delta \vDash \phi \text{ iff for all }  H\in
\mathcal{H}. \quad H:\Delta \vDash \phi  $$
\end{defin}
We will be able to find sets of formulas identifying classes of homomorphisms, in the sense
that they are all valid exactly for the homomorphisms of that class.

\begin{defin}
A class of homomorphism $\mathcal{H}$  is \textit{definable} in LT if there
exists a set of formulas $\Pi$ such that for all homomorphism $H: \Fm \to
\P B$,
$$H\in \mathcal{H} \quad \text{iff } \quad H:~\vDash \pi \text{
for all } \pi \in \Pi. $$
\end{defin}

Note that to axiomatize the logic of a class of homomorphisms $\mathcal{H}_{\text{a}}$,
it is not enough to take a defining set of formulas $\Pi$ as axioms. The
reason is that definability imposes a global condition on homomorphism, whereas the
entailment is a local condition. As a simple example, consider the class
$\mathcal{H}_{\text{tr}}$ of homomorphisms onto the powerset of the trivial Boolean algebra $\P B=
\set{1}$. It is clear that this class is defined by the formula $\ibot$ $$H\in
\mathcal{H}_{\text{tr}}\quad \text{iff}\quad H:~\vDash \ibot$$ On the
other hand we have that $$\mathcal{H}_{\text{tr}} : ~ \vDash \itop \quad
\text{but} \quad \ibot \nvDash \itop$$ However, by using the universal
modality $\Box$, we can internalise global conditions and ensure that $$x\in
H(\Box \eta) \quad \text{ iff} \quad H(\eta)= \P B.$$ That is, for
all $x \in B$, $$ x\in H(\Box \eta)\quad \text{iff} \quad H :~
\vDash \eta.$$ We can then conclude the following theorem:

\begin{thm} 
If a class of homomorphism $\mathcal{H}$ is defined by a set of formulas
$\Pi$, let $\Box\Pi= \set{\Box \pi | \pi\in \Pi}$. Then we have
$$ \Box \Pi, \Delta \vDash \phi \quad \text{ iff } \quad
\mathcal{H} : \Delta \vDash \phi,$$ 
so that $\Box \Pi$ serves as an axiomatization in LT of the logic defined
by the restriction to homomorphisms of the class $\mathcal{H}$.
\end{thm}

This usage of $\Box$ also serves the purpose to relate the semantics of LT to
standardly considered algebraic semantics, in which $H(\top)$ is the special
designated value corresponding to truth \cite{Font16} :
\begin{thm} 
$\Box \phi \vDash \psi$ iff for all homomorphisms $H$, $H(\phi) = H(\top)$
implies $H(\psi)=H(\top)$.
\end{thm}

\subsection{Implication}
Not surprisingly we define \textit{external implication} and
\emph{equivalence} in the standard Boolean way:
\begin{gather*}
	\phi \to \psi = \lnot \phi \lor \psi \\ 
	\phi \leftrightarrow \psi = (\phi \rightarrow \psi) \land (\psi \rightarrow \phi)
\end{gather*}
	With standard Boolean considerations it is then
easy to see that the deduction theorem holds in LT for this external
implication:

\begin{thm}[Deduction theorem for LT]
	For all $\Delta \cup \set{\phi,\psi}\subseteq\Fm$
	$$\Delta, \phi  \vDash \psi \quad \text{iff} \quad \Delta \vDash \phi \to \psi.$$
\end{thm}

With several types of negations and disjunctions, there are several other type
of implication that are natural to consider, but they will not be of focus in
this paper. In Section \ref{Ax-class-hom}, leaning on the deduction theorem
for $\to$ we will use this connective  to define and, together with $\Box$, to
axiomatise the logic for an important class of homomorphism.

\subsection{Strict negation}
When we in Section \ref{sec:axiomatising} are going to relate LT to valuational team semantics, we will
have to consider another type of negation. We will denote this negation by
$\slnot$. For the following discussion it will be useful to define $\slnot$
not just as an operation on formulas of LT, but as an operation on subsets of
Boolean algebras:

\begin{defin}
  For subset $A$ and an element $b$ of a Boolean algebra $B$ we say that $b$
	is \emph{separate} from $A$ if for all $a \in A$, $b\land a = \bot$. We
	define $\slnot A$ as the set of all elements separate from $A$, i.e.,
	$$\slnot A = \set{ b \in B | \text{for all } a\in A, b\land a = \bot }.$$
\end{defin}

This operation is definable as an operation in LT, and
we call it \textit{strict negation} 

\begin{defin}
	In LT we define the unary operation $\slnot$ by the following
	$$ \slnot \phi= \lnot\up( \dwn \phi\land \NB )$$ 
	we call this operation the \textit{strict negation}
\end{defin}

\begin{thm}\label{slnot_form_to_subset}
	For every Boolean algebra $B$, every homomorphism $H:\Fm \to \P B$, and
	every formula $\phi\in\Fm$ we have $$ H(\slnot \phi) = \slnot H(\phi)$$
\end{thm}
\begin{proof}
	We first convince ourselves that for any algebra, homomorphism and formula
	as prescribed we have
	$$H(\dwn \phi \land NB)= \set{ a \in  B |  a \neq \bot \text{ and } a\leq b
	\text{ for some } b\in H(\phi)}.$$
	Therefore $H(\up(\dwn \phi \land NB))$ is the set of elements of $B$ that
	have non-trivial intersection with some element in $H(\phi)$, i.e
	$$H(\up (\dwn \phi \land NB)= \set{ a\in B | a\land b \neq \bot \text{ for
	some } b\in H(\phi)}.$$
	This is exactly the complement of $\slnot H(\phi)$, and thus
	\[ 
	H(\slnot\phi)=  H (\lnot\up( \dwn \phi\land \NB ))= \slnot H(\phi).
	\]
\end{proof}

The purpose of defining strict negation in this paper is in order to identify
the relation between LT and traditional propositional team semantics defined
by valuational semantics. The connective is however interesting in its own
right, and it can be viewed as a type of intuitionistic negation in the
following sense.
 
\begin{thm}\label{prop_strict_neg}
	The following statements hold for LT.
	\begin{enumerate}
		\item $\vDash P \to \slnot \slnot P$ 
		\item $\nvDash \slnot \slnot P \to P $ 
		\item $\vDash \slnot  \slnot \slnot  P \leftrightarrow \slnot P$ 
		\item $\vDash (\slnot P \iland \slnot \slnot P )\leftrightarrow \ibot$
		\item $\vDash \slnot P \ilor \slnot \slnot P $
	\end{enumerate}  
\end{thm}

As a simple proof of statement (2), consider for any Boolean algebra a
homomorphism $H$ such that $H(P)= \emptyset$. Then $H(\slnot \slnot P) =
\set{\bot}$. Clearly then $H(\slnot \slnot P)\nsubseteq H( P)$ and the
statement is asserted. The other statements can be proven using the natural
deduction system given in Section \ref{deduction-system}, but since the
statements include many defined connectives the proof trees become fairly
large. As example, statement (1) is written directly without defined
connectives as $$(1) \vDash \neg P \lor \lnot((([\lnot (((P \iland \lnot
\bot)\land \lnot \ibot)\ilor \lnot \bot)]\iland \lnot \bot) \land \lnot \ibot
)\iland \lnot \bot)$$ Figure \ref{fig:derivation} shows a proof tree for the
core of such proof by proving $a:P \vdash a:\slnot\slnot P$, still using
defined connectives as notational abbreviations.

\begin{sidewaysfigure}
\vspace{.6\textwidth}
\hspace{10pt}
$
\infer[\ni_1]{a\of \slnot \slnot P}{
	\infer[\ioe_2]{a\of \bot}{
		[a\of \up(\dwn \slnot P \land \NB)]^1
		&
		\infer[\iae_3]{a\of \bot}{
			[p \of \dwn\slnot P \land \NB]^2
			& \hspace{-50pt}
			\infer[\be]{a\of \bot}{
				\infer[\ne]{s\of \bot}{
					[s\of \slnot P]^3
					&\hspace{-180pt}
					\infer[\taut /\sub]{s\of \up (\dwn P \land \NB)}{
						\infer[\ioi]{(s\land r)\lor(s\land q)\lor (s\land \lnot (r\lor q))\of \up(\dwn P \land \NB)}{
							\infer[\ai]{(s\land r )\lor (s\land q) \of \dwn P \land \NB}{
								\infer[\taut / \sub]{(s\land r)\lor(s\land q)\of\dwn P}{
									\infer[\iai]{((s\land r)\lor q)\land s \of \dwn P}{
										\infer[\sub]{(s\land r)\lor q \of P}{
											\infer[\taut ]{a= (s\land r) \lor q}{
												[a=p\lor q]^2
												&
												[p= s\land r]^3}
											&
											a\of P}
										&
										\infer[\ni]{s \of \top}{[s\of \bot]}}}
								&
								\infer[\ni]{(s\land r)\lor(s\land q)\of \NB}{
									\infer[\ne]{(s\land r)\of \bot}{
										\infer[\sub]{s\land r \of \NB}{
											[p=s\land r]^3
											&
											\infer[\ae]{p\of \NB}{
												[p\of \dwn \slnot P \land \NB]^2}}	
										& \hspace{-10pt}
										\infer[\taut/\sub]{(s\land r) \of \ibot}{
											\infer[\ine]{\lnot\lnot(s\land r) \of \ibot}{
												\infer[\taut]{\lnot (s\land r) \of \itop}{
													\infer[\ini]{\lnot((s\land r) \lor (s\land q)) \of \itop }{
														[(s\land r)\lor (s\land q) \of \ibot]}}}}}}}						
							&\hspace{-60pt}
							\infer[\ni]{s \land \lnot (r\lor q) \of \top}{
								[s\land \lnot (r\lor q)\of \bot]}}}}}}}}
$
\caption{Derivation showing that $a:P \vdash a:\slnot\slnot P$. Here
$\taut/\sub$ is a shorthand for a combination of $\taut$ and $\sub$.}
\label{fig:derivation}
\end{sidewaysfigure}

At the end of next section, when we have further investigated semantic
properties of $\slnot$, we will instead give semantic proofs for the
statements  of Theorem \ref{prop_strict_neg}.
