%!TEX root = logic_of_teams.tex

\section{Flatness in LT}

We will now spend some effort understanding $\slnot$ as an operation on
subsets of a Boolean algebra. Recall that a Boolean \textit{ideal} is a
subset of a Boolean algebra that is downwards closed and closed under
disjunction.

\begin{thm}\label{negid}
	For all Boolean algebras $B$ and all subsets $A\subseteq B$ we have that $\slnot
	A$ is a non-empty Boolean ideal.
\end{thm}
\begin{proof}
	First clearly $\bot\in \slnot A$, so it is not empty. Then, for all $a \in
	A$, if $b\land a = \bot$ then for all $c\leq b$, $c\land a = \bot$, and hence
	$\slnot A$ is downwards closed. Similarly, if $a\land  b= \bot$ and $a\land
	c = \bot$, then $a\land (b\lor c)= \bot$ by distributivity, and hence
	$\slnot A$ is closed under disjunction.
\end{proof}

This is a necessary property for the sets $\slnot A$ but not fully
categorising in the sense that there exists Boolean algebras $B$  with ideals
that cannot be identified as $\slnot A$ for any set $A\in \P B$. However, we
can strengthen the categorisation in the following way by identifying the
specific type of ideals.

\begin{defin}
		For a Boolean algebra $B$ a subset  $A\subseteq B$ is said to be
	\begin{itemize}
		\item  \emph{closed under existing infinite disjunctions} if for all
		$ C \subseteq A$ such that $\bigvee C$ exists in $B$ then $\bigvee C \in A$, and
		\item \emph{flat} if it is both downwards closed and closed under existing
		infinite disjunction.
	\end{itemize}
\end{defin}

\begin{thm}\label{basic_flatness}
	For any Boolean algebra $B$ and subset $A\subseteq B$:
	\begin{enumerate}
		\item $\slnot A$ is flat, and
		\item $A$ is flat iff $\slnot \slnot A = A$.
	\end{enumerate} 
\end{thm}
\begin{proof}
	(1) Downwards closure follows directly from Theorem \ref{negid} that asserts
	$\slnot A$ is an ideal of $B$. Note also that $\bot$ is
	the result of the empty disjunction so that $\bot$ is an element of every
	flat set. 
	
	Let $C\subseteq \slnot A$ be a set of elements
	such that $\bigvee C \in B$. We need to show that $\bigvee C \in
	\slnot A$. For arbitrary $a\in A$ we have, that $a\land \bigvee C =
	\bigvee_{c\in C} c \land a $ and the right side of this equation
	exists.\footnote{See for example Lemma 1.33 on p. 22 of \cite{Monk1989}.} Then by
	assumption $\bigvee_{c \in C} c \land a = \bigvee_{c\in C} \bot = \bot$ and thus
	$\bigvee C \in \slnot A$.
	
	(2) One direction follows directly from (1). For the other direction,  it is easy
	to see that in general $A\subseteq \slnot \slnot A$, since for all $a\in A$
	we have for all $c \in \slnot A$ that $a\land c= \bot$ by definition of
	$\slnot A$. What is left to show is that in the case when $A$ is flat we
	also have  $\slnot \slnot A\subseteq A$. 
	
	Thus, assume $A$ is flat and $b\in \slnot \slnot A$. We need to show that
	$b\in A$. Consider the set $\set{b\land a | a\in A}$. By the downwards
	closure of $A$, it is clear that $\set{b\land a | a\in A}\subseteq A$, so that
	by closure under existing infinite disjunctions, if it exists,
	$\bigvee_{a\in A} b\land a\in A$.
	
	We thus finish the proof by showing that $b= \bigvee_{a\in A} b\land a$,
	i.e., $b$ is the least upper bound of the set $\set{b\land a | a\in A}$.
	Since $b\geq b\land a $ for all $a\in A$, clearly $b$ is an upper bound for
	$\set{b\land a}_{a\in A}$. Assume for the sake of contradiction, that it is
	not the least upper bound. Then there exists $c\in B$ such that $c$ is an
	upper bound for $\set{b\land a | a\in A}$, and $c\lneq b$. We then
	investigate the element $d = b\land \lnot c \neq \bot$. For every $a\in A$
	we have
	$$a\land d = a \land  (b \land \lnot c) = (a\land b) \land \lnot c=
	\bot,$$ 
	where the last equality is due to $(a\land b)\leq c$ since $c$ is an upper
	bound for $\set{b\land a | a\in A}$. Consequently, $d\in \slnot A$, but we
	also have $d\leq b$ by construction, and thus $b\notin \slnot\slnot A$ which
	is a contradiction. Thus $b$ is necessarily the least upper bound of
	$\set{b\land a | a\in A}$, and hence  $b= \bigvee_{a\in A} b\land a\in A$
	and thus $b\in A$.
\end{proof}

As stated in the proof $A\subseteq \slnot \slnot A$ holds for any subset of a
Boolean algebra, and it is just the reverse inclusion that essentially
categorizes flatness. Using this result we can extend this language to
formulas of LT.

\begin{defin}
	A formula $\phi\in\Fm$, is called \textit{flat} if $\slnot \slnot \phi \vDash 
	\phi$, that is, for every Boolean algebra $B$, and every homomorphism $H:\Fm 
	\to \P B$, $H(\phi)$ is flat as a subset of	$B$.
\end{defin}

\subsection{Flatness in complete and atomic algebras}

By the adequacy result of theorem \ref{thm:finite_bas}, we are for the
purpose of LT able to restrict our attention to finite Boolean algebras. Since
finite Boolean algebras are both complete and atomic, it is relevant to
investigate flatness in the particular case when the Boolean algebra is
complete and atomic. These specific observations will play a role in Section
\ref{sec:axiomatising} when relating LT to propositional dependence logics.
Recall the following definitions of complete and atomic Boolean algebras:

\begin{defin}
	A Boolean algebra is said to be \textit{complete} if it is closed under
	infinite disjunctions and conjunctions, and it is said to be
	\textit{atomic} if every element is a least upper bound of some set of
	atoms.
\end{defin}

It is clear that a Boolean algebra $B$ is atomic iff 
\[b=\bigvee \set{ a\in B | a \text{ is an atom and } a\leq b}\] 
for all $b \in B$. Furthermore, we recall the following theorems:

\begin{thm}[\cite{Halmos2009}]
	Every finite Boolean algebra is complete and atomic.
\end{thm}

\begin{thm}[\cite{Halmos2009}]
	Every complete atomic Boolean algebra is equivalent to an algebra of the
	form $(\P S, \emptyset, \cdot^C, \cup, \cap)$ for some set $S$.
\end{thm}

We can relatively directly observe the following simpler categorisation of
flatness in these classes of Boolean algebras.

\begin{thm}\label{comp_slnot_ideal}
	If $B$ is a complete Boolean algebra, then for all $A\in \P B$, $A$ is
	flat iff $A$ is a non-empty principal ideal, i.e., if there
	exists $a\in A$ such that $$ A= \set{ b \in B| b\leq a}.$$
\end{thm} 
\begin{proof}
	Assume $A$ is flat. Take $a= \bigvee A$, then by flatness $a\in A$ and by
	downwards closure $A$ is the principal ideal generated by $a$. For the other
	direction, assume $A$ is a non-empty principal ideal. Then $A$ is downwards
	closed. Furthermore, there exists $a\in A$ such that $A=\set{ b \in B| b\leq
		a}$. Consequently, for all $C \subseteq A$ it is evident that $\bigvee C
	\leq a$, and thus again by downwards closure, $\bigvee C \in A$. Therefore
	$A$ is closed under existing infinite disjuctions, and thus is flat.
\end{proof} 

For atomic Boolean algebras flatness is equivalent to membership being
reducable to atomic membership in the following sense:

\begin{thm}
	Let $B$ be an atomic Boolean algebra, then a subset $A \subseteq B$ is flat if
	and only if
	\[
	A = \set{b \in B | a \in A \text{ for all atoms }  a \leq b}.
	\]
\end{thm}
\begin{proof}
	Assume $A$ is flat, then it is non-empty. Assume $b\in A$. Then for all
	$a\leq b$, $a\in A$, by downwards closure of A, and thus 
	\[ 
		A\subseteq \set{b \in B | a \in A \text{ for all atoms }  a \leq b}.
	\] 
	For the opposite inclusion, assume $a\in A$ for all atoms $a\leq b$. Then
	since $B$ is atomic, $b=\bigvee \set{a \in B | a \text{ atomic and } a\leq
	b }$ so that, by the closure under existing infinite disjunctions, $b\in
	A$.
	
	For the opposite direction, assume $b \in A \text{ iff } a \in A \text{
	for all atoms }  a \leq b$. We need to show that $A$ is flat. Downwards
	closure is clear, since if $b\in A$ then for any $c\leq b$, $\set{a \in B
	|  a \text{ atomic and } a\leq c} \subseteq \set{a \in B |  a \text{
	atomic and } a\leq b} $. For the closure under existing disjunctions,
	assume, for $C \subseteq A$, that $\bigvee C=b\in B$. Clearly, for all
	atoms, $a\in B$, $a\leq b$ iff there exists $c \in C$  such that $a\leq
	c$. Consequently, for all $a\leq b$, $a\in A$, so that by assumption $b\in
	A$. Thus, $A$ is flat.
\end{proof} 

We have previously established that for any Boolean algebra $B$, a subset
$A\subseteq B$ is flat iff $\slnot \slnot A = A$. We can thus
summarise these results for the nicely behaved complete and atomic Boolean
algebras:

\begin{cor}\label{flat_eqs}
	For a complete atomic Boolean algebra $B$, and a subset $A\subseteq B$ the
	following statements are equivalent:
	\begin{itemize}
		\item $A$ is a non-empty principal ideal of $B$. 
		\item $A$ is flat.
		\item $A =  \set{b \in B | a \in A \text{ for all atoms }  a \leq b}$.
		\item $ \slnot \slnot A\subseteq A $. 
	\end{itemize}
\end{cor}

Since the set of finite Boolean algebras is adequate for LT, Theorem
\ref{thm:finite_bas}, and all finite Boolean algebras are complete and atomic
this corollary will be useful in the next section to identify flatness in the
following way:

\begin{cor}\label{flatness_formula}
	If $B$ is a complete atomic Boolean algebra, then for all homomorphism $H:
	\Fm \to \P B$ and all formulas $\phi$ 
	\[
	H(\phi) \text{ is flat iff }  H : ~ \vDash \slnot \slnot \phi \to \phi.
	\]
\end{cor} 

To finish of this section we can use the results to give a semantic proof of
the properties of strict negation stated in Theorem \ref{prop_strict_neg}.

\begin{proof}[Proof of Theorem \ref{prop_strict_neg}]
	Statement (1) follows directly from the observation in the proof of Theorem
    \ref{basic_flatness} part (2) that for any subset of a Boolean algebra $A$ 
    we have that $A\subseteq \slnot\slnot A$.
	For statement (4) observe that for any formula $\phi$ and any homomorphism $H$, 
	if $a\in H(\phi)$,and $b\in H(\slnot\phi)$, then by definition 
	$a\land b= \bot $. Thus, as long as $H(\phi)$ and 
	$H(\slnot\phi)$ are non-empty, $H(\phi\iland \slnot \phi)= \set{\bot}$.
	This is assured by Theorem \ref{negid} when $\phi = \slnot P$.
	To semantically prove (3) and (5) we use Theorem \ref{thm:finite_bas} asserting that the class
	of complete atomic Boolean algebras are adequate for LT, so we may restrict our 
	attention to this class of algebras.  We can then use the following claim.
	
		\noindent\textbf{Claim:}  If $B$ is  complete Boolean algebra, and 
		$A\subseteq B$ is a non-empty principal ideal with top element 
		$a= \bigvee \!\!A$, then $\neg a = \bigvee\!\! \slnot\! A$ and thus is 
		the top element of the principal ideal $\slnot A$. 

		\begin{proof}[Proof of claim:] Assume $a$ is the top element of 
			the principal ideal $A$. Then for all $b\in A$ we have that 
			$b\land a = b$ and thus  $\neg a \land b = \neg a \land a \land b = \bot$,
			and thus $\neg a \in \slnot A$. Furthermore, if $b\in \slnot A$, 
			then $ a\land b = \bot$. 
			Therefore, $\neg a \lor (a \land b)= \neg a \lor b = \neg a$, 
			i.e. $b\leq \neg a$ and we have established that 
			$\neg a $ is the top element of the principal ideal $\slnot A$. 	
		\end{proof}
	To establish statement (3) we see for every complete 
	(atomic) Boolean algebra, by Theorem \ref{comp_slnot_ideal}, 
	that $H(\slnot P)$ is a non-empty principal ideal for every homomorphism 
	$H$. Let $a$ be the top element of $H(\slnot P)$. Then by the claim $a$ 
	is also the top element of 
	$\slnot \slnot H(\slnot P)= H(\slnot \slnot \slnot P)$, and 
	thus $H(\slnot \slnot \slnot P)= H(\slnot P)$. 
	This is enough to assert the validity of formula (3).
	
	For formula (5), we first acknowledge by Theorem \ref{comp_slnot_ideal}
	and the claim that for any complete (atomic) Boolean algebra $B$ and any 
	homomorphism $H$ there is an element $a$ such that $a$ and $\neg a$ are the 
	top elements of the principal ideals $H(\slnot P)$ and $H(\slnot\slnot P)$ 
	respectively. Then, for every $b\in B$ we have
	since $a \lor \neg a= \top$ that
	$$b= b \land (a\lor \neg a)= (b\land a ) \lor (b\land \neg a)$$
	being the top elements of respective principal ideals we observe that 
	$$b\land a \in  H(\slnot P)\quad \text{and} \quad b \land \neg a \in H(\slnot \slnot P)$$ 
	and conclude that  $ H(\slnot P \ilor \slnot \slnot P)= B$. 
	This holds for every homomorphism $H$ for a complete Boolean algebra B, and 
	hence (5) holds for the logic LT.   
\end{proof}