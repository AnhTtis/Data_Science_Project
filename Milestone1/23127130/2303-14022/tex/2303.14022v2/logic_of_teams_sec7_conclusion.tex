%!TEX root = logic_of_teams.tex
\section{Conclusion}

In this paper, we have introduced a new substitutional logic of teams, LT,
with a natural semantics inspired by algebraic semantics together with a sound
and complete labelled natural deduction system. We have also shown that the
logic has a finite model property in that the finite Boolean algebras are
adequate for the semantics of LT, proving that the entailment of LT is
decidable. Furthermore this gives the possibility to axiomatise different
propositional team logics, such as propositional dependence logic, as a
fragment of LT. From the initial choice of a more algebraic perspective, we
see that the constructions of the semantics, the natural deduction system and
the relative axiomatisation presented, seem fairly straight-forward and
sensible without significant arbitrary choices. This indicates that an
analysis of the structure of these constructions should be informative for an
over-arching view of team logics from an algebraic perspective. By focusing on
different parts of this construction we highlight some research topics that
are exposed by the work in this paper.

The logic LT is fully substitutional, so the full toolbox of algebraisation is
open to us. Furthermore, in the language of abstract algebraic logic
\cite{Font16}, LT is a finitary implicative logic (with respect to the
exterior implication). We can then conclude by application of general results
that:
\begin{thm}
	LT is algebraisable and its equivalent algebraic semantics
	\\$\mathcal{A}\textit{lg}^*\text{LT}$ (the largest class of algebras LT is
	algebraisable with respect to) is a quasi-variety that can be presented by
	the equations and quasi-equations that result by applying the
	transformation $\phi \mapsto \phi \approx \top $ to the axioms and rules
	of any Hilbert-style presentation of LT.
\end{thm}
\begin{proof}
	This conclusion can be drawn from the general results presented in
	\cite{Font16}. The theorem follows from   Definitions 2.3 and 2.5;
	Propositions 2.7 and 3.15; and the direct comment after Defininion 3.19 in
	the aforementioned work.
\end{proof}

We have at this point not investigated Hilbert-style proof systems further. By
compactness together with the deduction theory for external implication, we
can however conclude that it is possible to present a fully axiomatic Hilbert
system, with modus-ponens and substitution as its only rules, and hence that
$\mathcal{A}\textit{lg}^*\text{LT}$ is a variety. Apart from the direct
observation that the algebras $\P B $ where $B$ is a Boolean algebra are
included in $\mathcal{A}\textit{lg}^*\text{LT}$ we know little about this
class at the moment and it should be further investigated. One way of doing so
is by identifying an informative choice of Hilbert system for the logic with
corresponding equational theory of the $\mathcal{A}\textit{lg}^*\text{LT}$.
Another approach is to find more general ways than the powerset-operation to
generate algebras of $\mathcal{A}\textit{lg}^*\text{LT}$ from Boolean
algebras. If $\mathcal{A}\textit{lg}^*\text{LT}$ can be exhausted by algebras
uniformly generated by Boolean algebras it would illuminate and identify the
relation between these classes of algebras and provide a clear semantic
connection between LT and classical propositional logic.

To relate LT to other propositional team logics we have identified a set of
flat variable axioms (FVA). This set of axioms can be viewed as the obvious
set of axioms that captures the denotational semantics of the other logics in
the semantics of LT. It is therefore expected that similar algebraic
constructions and axiomatisations are possible for other types of team
semantics such as modal team semantics \cite{Vaeaenaenen2008}. Furthermore, this
axiomatisation provides a way to construct proofs of the entailment statements
of these propositional team logics. It does however not directly constitute a
natural deduction system for the axiomatised logics per se, since the terms of
these proofs will in general not be confined to the syntactical fragment of
the logics. Our natural deduction system may however motivate, and be seen as
a guide in the construction of deduction systems for these propositional team
logics, and indicates a suitability of labelled systems.

In the labelled natural deduction for LT, the rules $\sub$ and $\taut$
establishe a notion of equivalence of labels determined by classical
propositional logic. From this perspective these rules can be viewed as
structural rules of the deduction system. The rules for the formulas of LT
consists of introduction rules together with the elimination rules that are
the direct inverses of the introduction rules (up to equivalence of
labels).\footnote{The rule $\ine$ is not directly the inverse of $\ini$ but
can be seen to be equivalent to a direct inverse rule. } In this sense, the
rules of LT harmonise, and it is possible to use more advanced proof
theoretic methods to investigate LT. For example, it seems to be easy to turn
the system into a sequent system that could be analysed with respect to cut
rules and cut elimination. This analysis may lead up to a proof theoretic
explanation of the internal connectives and so also of the connectives in
other propositional team logics through their axiomatisation.


By thinking of the $\taut$ rule as provability in an \emph{inner logic} we can
generalise the construction the labelled natural deduction system into a proof
theoretic teamification, or combination of two logics, an inner and an outer
logic. Of such constructions LT represents the special case for which both are
classical propositional logic. This opens the door for a purely
proof-theoretic approach to team logics, and more general relatives, and we
see this as an interesting future research topic. In particular this indicates
the possibility for similar constructions in a first order setting that could
give new general insight into first order team logics, which is an active
field of study with many applications.
