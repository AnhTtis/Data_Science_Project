%!TEX root = logic_of_teams.tex
\section{Introduction}

Team semantics was invented by Hodges \cite{Hodges1997} to give the
Independence Friendly logic (IF-logic) of Hintikka and Sandu
\cite{Hintikka1989} a denotational compositional semantics. Team semantics was
later used by Väänänen to define Dependence logic \cite{Vaeaenaenen2007}, a
formalism extending first-order logic in which functional dependence between
variables are explicitly expressed by atomic formulas. The intended meaning of
these atomic formulas $\dep(\bar x,y)$ is that the value of the variable $y$
is functionally determined by the values of the finitely many variables $\bar
x$. Even though Dependence logic uses only first-order quantifiers it can
express any existential second-order property or statement. The reason is that
there are second-order quantifiers hidden in the truth conditions of these
logics, most interestingly in the truth conditions for $\lor$ and $\exists$.

Since then many logics based on team semantics have been constructed and
investigated, such as Independence logic, Propositional dependence logics and
Modal dependence logics.

\subsection{Lifting Tarskian semantics to team semantics}

In the classical Tarskian semantics of first-order logic the denotation of a
formula, given a structure $A$, is defined to be the set of all assignments
that satisfies the formula in $A$. Similarly, the denotation of a
propositional formula is the set of valuations satisfying the formula, and the
denotation of a modal logic formula, given a Kripke model, is the set of all
worlds satisfying the formula. Thus, the denotation of a formula in this
classical setting is an element of $\P X$, the powerset of the set $X$ of all
assignments, all valuations or all worlds respectively.

Team semantics of first-order, propositional and modal logic lifts the
denotations of formulas to be sets of subsets of $X$, i.e., elements of $\P\P
X$, instead of elements of $\P X$. Thus, instead of asking if a single
assignment, valuation or world satisfies a formula, team semantics asks if a
\emph{set} of assignments, valuations or worlds satisfies a formula. Such sets
are called \emph{teams}. In this sense the team semantical denotation is the
image of the classical denotation under a specific function, a lift, from $\P
X$ to $\P \P X$.

For most team semanticists the \emph{flatness principle} has guided the
definition of the semantics. This principle can be formulated as follows:
\begin{quote}
    A team satisfies a formula iff every individual assignment, valuation, or
    world in the team satisfies the formula in the usual classical sense.
\end{quote}
Note that this principle only applies to formulas in the original language,
without any dependence atoms or other added connectives. For these formulas it
makes sense to say that an assignment/valuation/world satisfies a formula. In
terms of denotations of formulas the flatness principle naturally translates
to the equation 
\begin{equation}\label{eq:powerset}
    \den[h]{\phi} =\P \den[c]{\phi}
\end{equation} 
where $\den[h]{\phi}$ is the team semantical denotation of $\phi$ ($h$ for
Hodges) and $\den[c]{\phi}$ is the ordinary Tarskian denotation of $\phi$ ($c$
for classical).

By subscribing to the flatness principle most team semanticists also subscribe
to this powerset lift of classical semantics to team semantics.

\subsection{Substitutionality and logics}

Dependence logic and its variants have some nice properties such as
compactness and Löwenheim-Skolem properies \cite{Vaeaenaenen2007} and that 
first-order consequences of theories can be axiomatized \cite{Kontinen2013} to
name a few. But they are not substitutional; for example $$\forall x\, (P(x)
\lor P(x))\vDash \forall x\, P(x) \text{, but } \forall x\, (\dep(x) \lor
\dep(x)) \nvDash \forall x\, \dep(x),$$
where $\dep(x)$ is the dependence atom with only one free variable intuitively
stating that there's only one value of $x$ in the team, i.e., that $x$ is
constant.

In general, a logic is substitutional if $\varphi \vDash \psi$ implies
$\phi[\sigma/p] \vDash \psi[\sigma/p]$, i.e., an entailment isn't destroyed by
substituting a formula $\sigma$ for an atom $p$. 
%Most of the commonly studied
%logics, such as propositional logic, first-order logic and modal logics are
%substitutional and
Substitutionality was already used by Bolzano to define the
concept of validity. In \cite{Bolzano1837} Bolzano defines \emph{universally
valid propositions} as propositions with all their variants true. In Bolzano's
terminology a variant is nothing but a substitutional instance. Bolzano, and
many after him, thus took substitutionality not only as an important property
for a logic, but as the basic principle for the concept of a logical validity.

Substitutionality is a prerequisite of the logic to be able to define a
uniform proof system since if the logic is not substitutional a proof system
need to distinguish between formulas of different kinds. This is indeed the
case for all known proof systems for Dependence logic and its siblings. Also,
any attempt to algebraise a logic that is not substitutional will not result
in algebras in the original sense, since the elements of the algebraic
structure need to be partitioned, or divided, into different types. Indeed,
the examples of algebrisations of variants of propositional dependence logic
that appear in the literature
\cite{Quadrellaro2021,Quadrellaro2020,Bezhanishvili2021} all introduce
additional predicates on the resulting algebras in order to partition the
elements into different types.

The powerset lift of the Tarskian semantics to team semantics, as in
\eqref{eq:powerset}, could be blamed for the lack of substitutionality in
Dependence logic. In fact, any specific lift will limit the possible
denotations of atoms and the resulting logics will not, in general, be
substitutional.

In this paper we will therefore generalize away from using a specific lift.
Instead we will, in a certain sense, quantify over all possible lifts of the
atoms. Thus, given a function $\mathcal L : \P X \to \P\P X$ that lifts
the denotations of atoms from a Tarskian setting to a team semantical setting
we may extend this compositionally to give team semantical denotations
$\den[\mathcal L]{\phi}$ to all formulas $\phi$ in our logic. We define
the entailment relation by $\phi \vDash \psi$ if $\den[\mathcal L]{\phi}
\subseteq \den[\mathcal L]{\psi}$ for all functions $\mathcal L : \P X \to
\P\P X$.

The function $\phi \mapsto \den[\mathcal L]{\phi}$ is nothing but a
homomorphism from the absolutely free term algebra of formulas to a specific
algebraic structure of sets of teams. Describing a logic by quantifying over 
all homomorphism from a term language into an algebraic structure is exactly 
the starting point of constructing algebraic
semantics.

\subsection{Lifting algebraic semantics}

The semantics of classical propositional logic can be defined in terms of
Boolean algebras. One may even say that classical propositional logic
\emph{is} the logic of Boolean algebras. When thinking about the set of
propositional formulas as the term algebra generated by the propositional
variables using the Boolean operators: $\bot, \lnot, \lor$, and  $\land$ the
semantics of propositional logic can be stated using homomorphisms from the
term algebra to a Boolean algebra: A formula is a tautology if its image under
any such homomorphism is the top element of the Boolean algebra. This is the
starting point when we define the Logic of Teams, or LT for short, by lifting
the algebraic semantics for classical propositional logic to the setting of
teams. A team here is nothing but a set of elements of the Boolean algebra. 

The connectives, and the corresponding operators on the sets of teams, we are
interested in are the ordinary Boolean connectives $\bot,\lnot, \lor,\land$
that correspond to the empty set, the complement, the union and the
intersection. We will call these connectives and the corresponding operators
\emph{external} Boolean connectives and operators. We will also add the
\emph{internal} connectives and operators that use the operators of the
Boolean algebra pointwise. We will denote these internal connectives and
operators with $\ibot,\ilnot,\ilor,\iland$. The connective $\ilor$ is the
splitjunction used in Dependence logic and $\iland$ is the dual of $\ilor$,
which in Dependence logic is equivalent to conjunction.

\subsection{Structure of this paper}

In the next section we will formally introduce the syntax and semantics of the
Logic of Teams, LT. The semantics is defined in an algebraic manner in terms
of homomorphisms into algebraic structures based on Boolean algebras. This
is done in the most natural way and mirrors the interpretations of team
semantics in respect to it's denotation. In that section we also introduce a
labelled natural deduction system for LT. The formulas are decorated by labels
and the labels are themselves classical propositional formulas. By including
rules in the deduction system identifying classically equivalent formulas as
equivalent labels we establish the role of the labels as references to
elements in a Boolean algebra, and this paves the way for the completeness
proof via Lindenbaum-Tarski algebras presented in the next section.

Section 3 is devoted to the completeness of the natural deduction system and
the consequences that can be observed by more careful investigation of the
proof. An important result is that the set of finite Boolean algebras is
sufficient to describe the semantics of LT. This in turn has as a consequence
that the logic is decidable.

In Section 4 we introduce some important definable connectives together with
a notion of definability of classes of homomorphisms and axiomatisation of their
corresponding logic.

An important defined connective is what we call $\textit{strict negation}$,
and section 5 elaborates on the semantic properties of this operation
establishing the notion of flatness for subsets of Boolean algebras, in
particular we consider the special case of compact and atomic Boolean
algebras, bearing in mind that the adequacy of finite Boolean algebras, which
satisfy both qualifiers.

In section 6 we finally utilise the strict negation to define classes of
homomorphisms, and give axioms expressing the Strong propositional team logic,
PT$^+$, and therefore also many other propositional team logics as fragments.
In this way we establish the connection of LT with other propositional team
logics in the literature, and in some way establish LT as a well
motivated, substitutional, and expressively rich propositional team logic.

In the final section of the paper we reflect generally on the construction
that we have presented, and discuss some further topics of investigation that
are implicated by our work.