%!TEX root = logic_of_teams.tex


\section{Completeness and adequacy}

Observe that if $\Gamma \vdash \bot\of \bot$  then $\Gamma \vdash a\of \phi$
for all labels $a$ and all formulas $\phi$. We say that $\Gamma$ is
\emph{consistent} if $\Gamma \nvdash \bot\of \bot$ or equivalently that there
are $a$ and $\phi$ such that $\Gamma \nvdash a\of \phi$.

\begin{prop}\label{prop:max}
    If $\Gamma$ is consistent then so is either $\Gamma,a\of \phi$ or
    $\Gamma, a\of \lnot \phi$.
\end{prop}
\begin{proof}
    Directly by using the $\ni$ and $\ne$ rules.
\end{proof}

\begin{prop}\label{prop:resp}
    If $\Gamma,a\of \phi\ilor \psi$ is consistent and $p$ and $q$ are
    propositional variables not occurring in $\Gamma$, nor in $a$, then
    $\Gamma, a\of \phi\ilor \psi,p\of \phi,q\of \psi,a\leftrightarrow p \lor q
    \of  \itop$ is consistent. And similar for $\iland$.
\end{prop}
\begin{proof}
    Directly by using the $\ioe$ rule and the $\iae$ rule.
\end{proof}

\begin{defin}
    $\Gamma$ is \emph{$\ilor$-saturated} if whenever $a\of \phi\ilor
    \psi \in \Gamma$ then there are labels $b$ and $c$ such that $\set{b\of
    \phi,c\of \psi,a = b \lor c} \subseteq
    \Gamma$. The dual notion of \emph{$\iland$-saturation} is defined
    similarly. We say that $\Gamma$ is \emph{saturated} if it is both
    $\ilor$-saturated and $\iland$-saturated.
\end{defin}

% \begin{prop}
%     If $\Gamma$ is consistent then the set $T=\set{a | a\of  \itop \in \Gamma}$ of labels is
%     consistent in propositional logic as a set of propositional formulas.
% \end{prop}
% \begin{proof}
%         If $T$ is inconsistent then by  $\taut$ we get that $\Gamma \vdash
%         \bot \of  \ibot$ and so $\Gamma \vdash \bot\of \bot$ by $\bi$ and is thus
%         inconsistent.
% \end{proof}


\begin{thm}
    If $\Gamma \vDash a\of \phi$ then $\Gamma \vdash a\of \phi$.
\end{thm}
\begin{proof}
    Assume that $\Gamma \nvdash a\of \phi$. By Lemma \ref{lem:basicfacts} the
    set $\Gamma_0 =\Gamma,a\of \lnot \phi$ is consistent. We will extend it to
    a maximal consistent saturated set $\Gamma^\ast$. To make sure we have
    enough fresh propositional variables we first rename the propositional
    variables in $\Gamma_0$ by replacing $p_i$ with $p_{2i}$ and thus freeing
    up an infinite number of propositional variables.

    We construct $\Gamma^\ast$ as the union of $\Gamma_n$ where each $\Gamma_n$
    is a finite extension of $\Gamma_0$. 

    We enumerate all labelled formulas and when constructing $\Gamma_{n+1}$ we
    pick the $n$:th labelled formula $b \of \psi$ and add either $b\of\phi$ or
    $b\of\lnot\phi$. By Proposition \ref{prop:max} one of these is consistent
    with $\Gamma_n$.

    By proposition \ref{prop:resp} we can assure that if we added a labelled
    formula $b\of \sigma \ilor \theta$ then we also add $p\of \sigma$, $q\of
    \theta$ and $b= p \lor q$ for some new propositional variables $p$ and
    $q$, and keeping $\Gamma_{n+1}$ consistent. 

    This construction assures that $\Gamma^\ast$ is maximal consistent and
    saturated.

    Let $T=\set{a | a\of  \itop \in \Gamma^\ast}$. Let $B$ be the
    Lindenbaum-Tarski Boolean algebra of labels over $T$, i.e., its elements
    are equivalence classes of labels under the relation of $T$-provable
    equivalence: $$B = \Lb / \sim_T = \set{[a]_T | a \in \Lb},$$ where $a
    \sim_T b $ iff $T \vdash a \leftrightarrow b$ (in classical propositional
    logic) and $[a]_T=\set{b \in \Lb | a \sim_T b}$. Observe that $B$ is the
    trivial one-element Boolean algebra iff $T$ is the inconsistent theory,
    i.e., if $\bot \in T$.

    Define the homomorphism $h: \Lb \to B$ by $h(p_i)=[p_i]_T \in B$ and the
    homomorphism $H: \Fm \to \P B$ by $H(P_i) = \set{h(a) | a\of P_i \in
    \Gamma^\ast}$. By induction on formulas we prove that $h(a) \in
    H(\varphi)$ iff $a\of \varphi \in \Gamma^\ast$. The base case follows
    immediately from the definition of $H$.

    If $\varphi$ is $\psi \lor \sigma$, $\psi \land \sigma$ or $\lnot \psi$
    the induction step is straight-forward as for example $a \of  \psi \land
    \sigma \in \Gamma^\ast$ iff $a\of \psi \in \Gamma^\ast$ and $a\of \sigma
    \in \Gamma^\ast$ and, by the induction hypothesis, this is equivalent to
    $h(a) \in H(\psi)$ and $h(a) \in H(\sigma)$, i.e., $h(a) \in H(\psi) \cap
    H(\sigma) = H(\psi
    \land \sigma)$.

    For the case when $\varphi$ is $\ilnot \psi$ note that $a \of  \ilnot \psi
    \in \Gamma^\ast$ iff $\lnot a \of \psi \in \Gamma^\ast$ iff $h(\lnot a)
    \in H(\psi)$ iff $\lnot h(a) \in H(\psi)$ iff $h(a) \in H(\ilnot \psi)$.

    When $\varphi$ is $\psi \ilor \sigma$ note that $a \of  \psi \ilor \sigma \in
    \Gamma^\ast$ iff there are labels $b$ and $c$ such that $b \of \psi \in
    \Gamma^\ast$, $c\of \sigma \in \Gamma^\ast$ and $a \leftrightarrow b \lor
    c \of  \itop \in \Gamma^\ast$. By the induction hypothesis this is
    equivalent to $h(b) \in H(\psi)$, $h(c) \in H(\sigma)$ and $h(a) = h(b\lor
    c) = h(b) \lor h(c)$. Thus this is equivalent to $h(a) \in H(\psi \ilor
    \sigma)$.

    The case when $\varphi$ is $\psi \iland \sigma$ is treated similarly. 

    Therefore $h$ and $H$ witnesses that $\Gamma \nvDash a\of \phi$.
\end{proof}

\begin{cor}
    The logic LT is compact.
\end{cor}
\begin{proof}
    Follows directly from soundness and completeness.
\end{proof}


\subsection{Canonical algebras}\label{sec:canonical}

The semantics of classical propositional logic can both be defined in terms of
the single two elements Boolean algebra or in terms of all Boolean algebras,
i.e., the two elements Boolean algebra is \emph{canonical} for propositional logic.
In LT the situation is a bit different as we will see.

We will write 
\[
  X: \Delta \vDash \phi \quad\text{and}\quad B : \Delta \vDash \phi
\]
for the restricted version of the entailment relation where the quantification
is done over a set $X$ of Boolean algebras or a single Boolean algebra $B$. 
For example, $B: \Delta \vDash \phi$ iff for all homomorphisms $H:\Fm \to \P B$,
\[
    \bigcap_{\psi \in \Delta} H(\psi) \subseteq H(\phi).
\]

\begin{defin}
    A set $X$ of Boolean algebras is \emph{adequate} (for LT) if
    for every $\Delta$ and $\phi$ we have
    \[
    \Delta \vDash \phi \text{ iff } X:\Delta \vDash \phi.
    \] 
    A single Boolean algebra $B$ is said to be \emph{canonical} if $\set{B}$
    is adequate.
\end{defin}
We will omit an empty theory and write $B : {} \vDash \phi$ instead of $B:
\emptyset \vDash \phi$.

Let us make some observations:
\[
B: {} \vDash \ibot \text{ iff } B=1,
\] 
where $1$ is the trivial one element Boolean algebra.  If $\phi$ has no
propositional variables then 
\[
B:{} \nvDash \varphi \text{ iff } B: {} \vDash \lnot\ibot \lor (\ibot \iland \lnot \phi).
\]
Thus, there is a $\psi$ such that
\[
 B: {} \vDash \psi \text{ iff } B \neq 1.
\]
Therefore, if $X$ is an adequate set of Boolean algebras then $1
\in X$.\footnote{Note that, if we add $\psi$ as an axiom then we have a (sound
and) complete derivation system for the semantics in which all Boolean
algebras are non-trivial.} We have a similar result for $B=2$: 
\[
B: {} \vDash \ibot \lor \itop  \text{ iff } B = 1,2,
\] 
where $\itop$ is $\ilnot\ibot$. Thus, if $X$ is adequate then $2 \in X$.

\begin{thm}
    No single Boolean algebra is canonical. In fact, if $X$ is adequate then
    $\{1,2\} \subsetneq X$.
\end{thm}

We can say more:

\begin{thm}
    No finite set of finite Boolean algebras is adequate.
\end{thm}
\begin{proof}
    We prove this by investigating the formulas $\phi_n$ defined by recursion
    on $n$:
    \begin{align*}
        \phi_0 &= P \\
        \phi_{n+1}&= P \ilor \phi_{n}
    \end{align*}
    Now, if $H$ is a homomorphism then $H(\phi_n) \subseteq H(\phi_{n+1})$
    since the elements of $H(\phi_n)$ are all of the form $a_0 \lor a_1 \lor
    \ldots \lor a_n$, where $a_i \in H(P)$, and thus $a_0\lor a_0 \lor a_1
    \lor \ldots \lor a_n \in H(\phi_{n+1})$.

    Assume now that $H$ is a homomorphism into $\P B$ where $B$ is finite with
    $n$ elements. Then, since the sequence $H(\phi_0) \subseteq  H(\phi_1)
    \subseteq  \ldots$ is increasing and the number of elements of $\P B$ is
    $2^n$ we have $H(\phi_{2^n}) = H(\phi_{2^n+1})$.

    Therefore, if $X$ is a finite set of finite Boolean algebras then $X :
    \phi_{n+1}
    \vDash \phi_n$ for some large enough $n$. 

    However in the full logic LT, $\phi_{n+1} \nvDash \phi_n$  since if $B= \P\mathbb N$
    and $H(P) = \set{\{i\} | i \in \mathbb N}$ then $H(\phi_n)$ is the set of
    all (non-empty) sets with at most $n$ elements and thus $H(\phi_{n+1})
    \nsubseteq H(\phi_n)$ for all $n \in \mathbb N$.
\end{proof}

On the other hand, the completeness theorem gives us a set of Boolean algebras
that is adequate, the set of Lindenbaum-Tarski algebras. These are either
finite Boolean algebras (including the one element trivial Boolean algebra) or
the unique countable atomless Boolean algebra.\footnote{There is precisely one
(up to isomorphism) countable atomless Boolean algebra \cite{Monk1989}.}

\begin{thm}
    The set of finite Boolean algebras together with the countable atomless
    Boolean algebra is adequate.
\end{thm}
\begin{proof}
    The proof of the completeness theorem constructs a Boolean algebra $B$
    that is a Lindenbaum-Tarski algebra over a countably infinite set of
    propositional variables. It's easy to see that such Boolean algebras are
    countable, and if it's infinite then also atomless. Thus, if $\Gamma
    \nvdash a\of \phi$ there is $H:\Fm \to \P B$, where $B$ is the countable
    atomless Boolean algebra or a finite Boolean algebra, such that
    $\bigcap_{\psi \in \Gamma} H(\psi) \nsubseteq H(\phi)$.
\end{proof}

In fact, by analysing the construction in the proof of the completeness
theorem and using the compactness of LT we may restrict ourselves to finite
Boolean algebras:

\begin{thm}\label{thm:finite_bas}
    The set of finite Boolean algebras is adequate.
\end{thm}
Before proving this theorem we need the following lemma.

\begin{lemma}
    If $\Gamma$ is a finite consistent set of labelled formulas and $A$ the
    set of atomic labels that occur in labels in $\Gamma$; then
    there is a finite consistent set of labelled formulas $\widehat \Gamma
    \supseteq \Gamma$ such that:
    \begin{itemize}
        \item If $a\of \lnot (\phi \ilor \psi) \in \widehat\Gamma$ and $b$ and
        $c$ are labels with all propositional variables in $A$, then there are $a',b'$
        and $c'$ such that $\vdash a \leftrightarrow a'$, $\vdash b \leftrightarrow b'$, $\vdash c
        \leftrightarrow c'$ and either $b' \of  \lnot \phi \in \widehat\Gamma$, $c' \of 
        \lnot \psi \in \widehat\Gamma$ or $(a' \leftrightarrow b' \lor c') \of 
        \lnot \itop \in \widehat\Gamma$.
        \item If $a\of \lnot (\phi \iland \psi) \in \widehat\Gamma$ and $b$ and
        $c$ are labels with all propositional variables in $A$, then there are $a', b'$
        and $c'$ such that $\vdash a \leftrightarrow a'$, $\vdash b \leftrightarrow b'$, $\vdash c
        \leftrightarrow c'$ and either $b' \of  \lnot \phi \in \widehat\Gamma$, $c' \of 
        \lnot \psi \in \widehat\Gamma$ or $ (a' \leftrightarrow b' \land c') \of 
        \lnot \itop \in \widehat\Gamma$.
    \end{itemize}
\end{lemma}
\begin{proof}
    If $a\of \lnot (\phi \ilor \psi) \in \Gamma$ and $b$ and $c$ are labels then
    at least one of the three theories $\Gamma, b\of \lnot \phi$; $\Gamma, c\of \lnot
    \psi$; or $\Gamma,a \leftrightarrow b \lor c\of \lnot \itop$ is consistent.
    If not, then $\Gamma$ would prove $b\of \phi$, $c\of \psi$, and $a
    \leftrightarrow b \lor c\of \itop$ and by applications of $\ioi$, $\sub$, and
    $\ne$, would prove $a\of \bot$.

    A similar statement for $a\of \lnot (\phi \iland \psi)$ holds as well.

    There are only a finite number of equivalence classes (under provability
    in classical propositional logic) of labels with atoms in $A$. Thus, we
    can take a finite maximal list of such non-equivalent labels: $a_0, a_1,
    \ldots, a_m$. Also, let $\sigma_0,\sigma_1,$ \ldots, $\sigma_n$ be a list
    of all subformulas and negated subformulas of formulas in $\Gamma$.

    We construct $\widehat{\Gamma}$ by a finite number of steps, step $k$
    resulting in a finite extension $\Gamma_k$ of $\Gamma$. In each such step
    $k$ we focus on one quadruple: $\sigma_i,a_j,a_r,a_s$, and make sure that that
    if $\sigma_i$ is of the form $\lnot (\phi \ilor \psi)$ (or $\lnot (\phi
    \iland \psi)$) and $a \of \sigma_i$ is in $\Gamma_{k-1}$, for some $\vdash a
    \leftrightarrow a_j$, then at least one of $a_r\of \lnot
    \phi$, $a_s\of \lnot
    \psi$ or $a_j \leftrightarrow a_r \lor a_s\of \lnot \itop$ (or $a_j
    \leftrightarrow a_r \land a_s\of \lnot \itop$) is in $\Gamma_k$. 

    Since there are only a finite number of such quadruples this process will
    end and we let $\widehat \Gamma$ be the last such $\Gamma_k$. This ensures
    that for every labelled formula  $a:\neg(\phi\ilor\psi)$ (or $a:\lnot
    (\phi \iland \psi)$), and labels $b,c$ with all propositional variables in
    $A$, we can find $a_j,a_r,a_s$ taking the roles of $a',b',c'$ in the
    statement of the theorem.
\end{proof}

\begin{proof}[Proof of Theorem \ref{thm:finite_bas}]
    We will prove that if $\phi$ is a consistent formula, i.e., $p\of \phi
    \nvdash \bot\of \bot$, then there is a finite Boolean algebra $B$ and a
    homomorphism $H$ into $\P B$ such that $H(\phi) \neq \emptyset$. Together
    with the compactness theorem for LT this proves that the set of finite
    Boolean algebras is adequate for LT.

    Let $\Gamma_0= \set{p\of \phi}$ and define $\Gamma_{n+1}$ by picking a new
    (not picked before) labelled complex formula $a\of \psi$ in $\Gamma_n$ and
    let $\Gamma_{n+1}$ be $\widehat\Gamma_{n}'$ where $\Gamma_{n}'$ is
    constructed from $\Gamma_n$ as follows:
    \begin{enumerate}
        \item If $\psi$ is $\sigma \land \chi$ add both $a\of  \sigma$ and
        $a\of \chi$.
        \item If $\psi$ is $\sigma \lor \chi$ add $a\of  \sigma$ or $a\of
        \chi$ depending on which is consistent with $\Gamma_n$.
        \item If $\psi$ is $\sigma \ilor \chi$ take new atomic labels $p$ and
        $q$ and add $p\of  \sigma$, $q\of \chi$, and $a = p\lor q$.
        \item If $\psi$ is $\sigma \iland \chi$ take new atomic labels $p$ and
        $q$ and add $p\of  \sigma$, $q\of \chi$, and $a = p\land q$.
        \item If $\psi$ is $\ilnot \sigma$ add $\lnot a\of  \sigma$.
        \item If $\psi$ is $\lnot\lnot \sigma$ add $a\of  \sigma$.
        \item If $\psi$ is $\lnot (\sigma \land \chi)$ add $a\of  \lnot
        \sigma$ or $a\of \lnot \chi$ depending on which is consistent with
        $\Gamma_n$.
        \item If $\psi$ is $\lnot (\sigma \lor \chi)$ add both $a\of
        \lnot\sigma$ and $a\of \lnot\chi$.
        \item If $\psi$ is $\lnot \ilnot \sigma$ add $\lnot a \of \lnot
        \sigma$.
        \item If $\psi$ is $\lnot (\sigma \ilor \chi)$ or $\lnot (\sigma
        \iland \chi)$ add nothing.
    \end{enumerate}
    
    Let $\Gamma^\ast$ be the union of the $\Gamma_n$'s. Let $A$ the set of
    atomic labels mentioned in $\Gamma^\ast$. As the only case where we add
    atomic labels is when dealing with $\ilor$ and $\iland$, it should be
    clear that if there are $k$ occurrences of $\iland$ and $\ilor$ in $\phi$
    then we add at most $2k$ atomic labels, and so $|A| \leq 2k+1$.

    The rest of the proof follows the proof of the completeness theorem
    closely with the important difference that the Lindenbaum-Tarski algebra
    $B$ are the equivalence classes of labels \emph{with atomic labels from
    $A$} under the relation of $T$-provable equivalence. Thus $B$ is a finite
    Boolean algebra.

    As before we define $h(a)=[a]_T \in B$ and $H(P_i) = \set{h(a) | a\of P_i
    \in \Gamma^\ast}$ and by induction prove that $h(a) \in H(\varphi)$ iff
    $a\of \varphi \in \Gamma^\ast$.

    Therefore $h(p) \in H(\phi)$ and so $H(\phi)\neq \emptyset$ and thus 
    $\phi \nvDash \bot$.
\end{proof}
 
As a direct corollary we get that Boolean algebras of the form $\P X$ are
enough to define the semantics of LT:

\begin{cor}
    The set of Boolean algebras of the form $\P X$ is adequate.
\end{cor}

Theorem \ref{thm:finite_bas} tells us that if $\nvDash \phi$ then there is a
finite Boolean algebra $B$ and a homomorphism $H :  \Fm \to \P B$ such that
$H(\phi) \neq B$. Also, for each finite $B$ and each $\phi$ there are only
finitely many possible homomorphic images of $\phi$. Thus, the set of
non-theorems of LT is recursively enumerable. Combining this with the fact
that the set of theorems of LT is recursively enumerable and that $\phi \vDash
\psi$ iff $\vDash \lnot \phi \lor \psi$ we can conclude the following:

\begin{cor}
    Entailment in LT is decidable.
\end{cor}
