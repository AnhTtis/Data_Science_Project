%!TEX root = logic_of_teams.tex

\section{The Logic of Teams}

Let us now define the Logic of Teams, LT, that is our main object of study in
this paper.

Let $B$ be a Boolean algebra, allowing for $B$ to be the trivial, one element,
Boolean algebra. $\P B$, the set of all subsets of $B$, is the domain of a
Boolean algebra with respect to the usual set-theoretic operators $(\emptyset,
\cdot^C, \cup, \cap)$ with some additional operators describable; the
internal, or point-wise operators $\ibot$, $\ilnot$, $\ilor$, and $\iland$.
These operators are defined as follows:
\begin{align*}
    \ibot &= \set{\bot}, \\
    \ilnot X &= \set{\lnot a | a \in X},\\
    X \ilor Y &= \set{a \lor b | a \in X, b \in Y}, \text{ and}\\
    X \iland Y &=\set{a \land b | a \in X, b \in Y}.
\end{align*}
Where $X,Y \subseteq B$ and $a \lor b \in B$ is the result of the operator
$\lor$ in the Boolean algebra $B$ applied to $a$ and $b$.

Formulas of LT are elements in the term algebra (absolutely free algebra) generated
by the propositional variables $P_0$, $P_1$, \dots\ using the external Boolean
connectives $\bot$, $\lnot$, $\lor$, $\land$ and the internal Boolean
connectives $\ibot$, $\ilnot$, $\ilor$, $\iland$. We will denote this term
algebra by $\Fm$:

$$\phi ::= \bot \mathrel| \ibot \mathrel| P_i \mathrel| \lnot \phi
\mathrel| \ilnot \phi \mathrel| \phi \lor \phi \mathrel| \phi \land \phi
\mathrel| \phi \ilor \phi \mathrel| \phi \iland \phi $$

Entailment is defined as the subset relation for the images of the
formulas under arbitrary homomorphisms.

\begin{defin}
    Let $\Delta$ be a set of formulas, then $\Delta \vDash \psi$ iff for all
    Boolean algebras $B$ and all homomorphisms $H:\Fm \to \P B$:
    $$\bigcap_{\phi \in\Delta} H(\phi) \subseteq H(\psi).$$
\end{defin}

Note that this definition of entailment does not correspond to the standard
entailment in algebraic logic in which $\phi$ entails $\psi$ if $H$
is a homomorphism with $H(\phi)=B$ then $H(\psi)=B$. The fact that $\ibot
\nvDash \ilnot \ibot$ shows this. More on this in Section \ref{sec:definable}.

Substituition instances of formulas are nothing but images under
homomorphisms: $\sigma : \Fm \to \Fm$ and thus substitutionality follows directly
from the definition of the entailment relation.

\begin{thm}[Substitutionality]
    If $\Delta \vDash \psi$ then $\sigma(\Delta) \vDash \sigma(\psi)$ for
    every algebra homomorphism $\sigma:\Fm \to \Fm$.
\end{thm}
\begin{proof}
    Follows directly from the definitions, and the fact that if $H:\Fm \to \P B$
    is a homomorphism then so is $H\circ \sigma$.
\end{proof}

%We will denote this logic by LT as a short-hand for the Logic of Teams. 

\subsection{Natural deduction} \label{deduction-system}

To define a deductive system for LT we will introduce \emph{labels} and
\emph{labelled formulas}. The labels have their intended reading as elements
in a Boolean algebra and the formulas as subsets of the same Boolean algebra.

Formally, labels are classical propositional formulas: 
$$a::= \bot \mathrel|  p_i \mathrel| \lnot a \mathrel| a \lor a \mathrel| a \land a$$

The propositional variables $p_i$ are called \emph{atomic labels}. The term
algebra of labels is denoted by $\Lb$.

A \emph{labelled formula} is of the form $a\of \phi$ where $a$ is a label and
$\phi$ a formula. The intended meaning of which is that $a$ is an element of
$\phi$, i.e., that under any homomorphisms $h$ and $H$, $h(a) \in H(\phi)$.
Thus, we define entailment in the natural way:

\begin{defin}
    Let $\Gamma$ be a set of labelled formulas, then $ \Gamma \vDash b\of \psi$
    iff for all Boolean algebras $B$ and all homomorphisms $h: \Lb \to B$,
    $H:\Fm \to \P B$ if $h(a) \in H(\phi)$ for all $a\of \phi \in
    \Gamma$ then $h(b) \in H(\psi)$.
\end{defin}

We say that homomorphisms $H$ and $h$ make $a\of \phi$ true if $h(a) \in H(\phi)$.

\begin{prop}
    $\Delta \vDash \psi$ iff $p\of \Delta \vDash p\of \psi$, where $p$ is an atomic
    label and $p\of \Delta = \set{p\of \phi | \phi \in \Delta}$.
\end{prop}
\begin{proof}
    Follows directly from the definitions.
\end{proof}

We will now present a sound and complete proof system for the relation $\Gamma
\vDash b\of \psi$, where $\Gamma$ is a set of labelled formulas. For this we will
use the usual shorthands $a \leftrightarrow b$ for $(a\lor
\lnot b)\land (\lnot a \lor b)$ and $\itop$ for $\ilnot \ibot$. Also $a=b$ is a shorthand for the labelled formula $a \leftrightarrow b\of  \itop$. Note that 
if $H$ and $h$ are homomorphisms that make $a=b$ true then $h(a)=h(b)$.

\subsubsection{Rules for external Boolean connectives}

The following rules are the standard rules for propositional logic, with
labels added to the formulas.

$$\infer[\ai]{a\of \phi\land \psi}{a\of \phi & a\of  \psi} \ruleskip
    \infer[\ae]{a\of \phi}{a\of \phi \land \psi}\ruleskip
    \infer[\ae]{a\of \psi}{a\of \phi \land \psi}$$

$$\infer[\oi]{a\of \phi\lor \psi}{a\of \phi} \ruleskip
    \infer[\oi]{a\of \phi\lor \psi}{a\of \psi} \ruleskip
    \infer[\oe]{b\of \sigma}{a\of \phi \lor \psi & \infer*{b\of \sigma}{[a\of \phi]} & \infer*{b\of \sigma}{[a\of \psi]}}
$$

$$\infer[\ni]{a\of \lnot \phi}{\infer*{b\of \bot}{[a\of \phi]}} \ruleskip
    \infer[\ne]{a\of  \bot}{a\of \phi & a\of \lnot\phi} $$

%$$\infer[\bi]{\bot\of \bot}{\bot\of \itop} \ruleskip 

$$\infer[\raa]{a\of \phi}{\infer*{b\of \bot}{[a\of \lnot\phi]}} \ruleskip \infer[\be]{b\of \phi}{a\of \bot} $$

%Note that $\bi$ is only sound for \emph{non-trivial} Boolean algebra's.

\subsubsection{Rules for internal Boolean connectives}

Next we add rules for the internal operations. The rule $\iae$ is modelled on
the fact that $x \in H(\phi \iland \psi)$ iff there are $y,z$ such that $x
\in H(\phi)$, $y \in H(\psi)$ and $x=y \land z$ holds in the Boolean algebra $B$. Similar for
$\ioe$. 
$$\infer[\iai]{a \land b\of  \phi \iland \psi}{a\of \phi & b\of  \psi}
    \ruleskip
    \infer[\iae]{b\of \sigma}{a\of \phi \iland \psi & \infer*{b\of \sigma}{\deduce{[a=p\land
q]}{\deduce{[q\of \psi]}{[p\of \phi]}}}}$$ 

$$\infer[\ioi]{a \lor b\of  \phi \ilor \psi}{a\of \phi & b\of  \psi}
    \ruleskip
    \infer[\ioe]{b\of \sigma}{a\of \phi \ilor \psi & \infer*{b\of \sigma}{\deduce{[a=p\lor q]}{\deduce{[q\of \psi]}{[p\of \phi]}}}}$$ 

$$\infer[\ini]{\lnot a\of \ilnot \phi}{a\of \phi} \ruleskip
    \infer[\ine]{\lnot a\of \phi}{a\of \ilnot \phi}$$

Note that vacuous discharges are allowed in the elimination rules for $\ilor$
and $\iland$.

%$$\infer[\ibe]{a=\bot}{a\of \ibot} \ruleskip \infer[\ibi]{a\of \ibot}{a=\bot}$$

\subsubsection{Rules for labels}

We add two rules for dealing with labels.

$$\infer[\taut]{b\of  \itop}{a_1 \of  \itop & \cdots & a_k\of  \itop} \ruleskip
    \infer[\sub]{a\of \phi}{a =b & b\of \phi}$$

The rule $\taut$ is only applicable if $a_1, \ldots, a_k \vdash b$ in
classical propositional logic, i.e., if $a_1 \land \ldots
\land a_k \rightarrow b$ is a tautology. Note that we allow for the special
case when $k=0$ in this rule.

In $\iae$ and $\ioe$ the $p$ and $q$ are distinct propositional variables not
occurring in any uncancelled assumptions, nor in $a$ or $b$. This
distinctiveness criterion is needed as otherwise the derivation
$$\infer[\iae_1^\ast]{a\of P\land Q} { a\of P
\iland Q
    & 
    \infer[\sub]{a\of P\land Q}
        {
        \infer[\taut]{a= b}{[a = b \land b ]^1} 
        &
        \infer[\ai]{b\of P\land Q}{[b\of P]^1 & [b\of Q]^1} 
        }
    }  
$$
would be valid, but $a\of P \iland Q \nvDash a\of P\land Q $.

For an example of a non-trivial derivation of an entailment, see Figure
\ref{fig:derivation}.

It is easy to see that the derivability relation satisfies the following lemma: 

\begin{lemma}\label{lem:basicfacts} 
$\Gamma \vdash a\of \phi$ iff $\Gamma,a\of
\lnot \phi \vdash a\of \bot$.
%    \item $\lnot a\of  \ilnot \phi \dashv\vdash a\of \phi$ 
%    \item $\lnot a\of  \phi \dashv\vdash a\of  \ilnot \phi$
%    \item 
%\end{enumerate}
\end{lemma}

\begin{thm}[Soundness]
    If $\Gamma \vdash a\of \phi$ then $\Gamma \vDash a\of \phi$.
\end{thm}
\begin{proof}
    Follows from a straight-forward induction on proof trees. For the case of
    $\iae$, $\ioe$, $\taut$ and $\sub$ note that $h(a)\in H(\itop)$ implies
    that $h(a)$ is $\top$, the top element of $B$. Thus, if $h(a
    \leftrightarrow b) \in H(\itop)$ then $h(a)=h(b)$.
\end{proof}
