%!TEX root = logic_of_teams.tex
\section{Conclusion}\label{section:Conclusion}

In this paper, we have introduced a new substitutional logic of teams, LT,
with a natural semantics inspired by algebraic semantics together with a sound
and complete labelled natural deduction system. We have also given an
axiomatisation of different propositional team logics, such as propositional
dependence logic, in LT. From the initial choice of a more algebraic
perspective, we see that the constructions of the semantics, the natural
deduction system and the relative axiomatisation presented, seem fairly
straight-forward and sensible without significant arbitrary choices. This
indicates that an analysis of the structure of these constructions should be
informative for an over-arching view of team logics from an algebraic
perspective. By focusing on different parts of this construction we highlight
some research topics that are exposed by the work in this paper.

The logic LT is fully substitutional, so the full toolbox of algebraisation is
open to us. In the language of abstract algebraic logic \cite{Font16}, LT is a
semilattice-based logic; and by applying some general theory we can see that LT
is, in fact, not algebraisable in the strict meaning. However, its assertional
companion, $\text{LT}^\top$, defined by the designated element $\top$, is an
implicative logic and thus algebraisable. We suggest that both these two
logics should be analysed from the view of abstract algebraic logic in future
work.

% a finitary implicative logic (with respect to the
% exterior implication). This is a very well behaved class of logics, and many
% general results apply.  In particular, we can conclude that many different
% notions of classes of algebras corresponding to LT coincide, and this class of
% algebras is a variety. We have at this moment not investigated any further
% details regarding this variety.%\iffalse 

% We can then conclude by application of general results
% that:
% \begin{thm}
% 	LT is algebraisable and its equivalent algebraic semantics
% 	$\mathcal{A}\textit{lg}^*(\text{LT})$ (the largest class of algebras LT is
% 	algebraisable with respect to) is a variety. Furthermore,
% 	$$ \mathcal{A}\textit{lg}^*(\text{LT}) = \mathcal{V}(\P \mathcal B),$$
% 	where $\P \mathcal B$ is the class of algebras of the form $\P B$ and $\mathcal V(\P\mathcal B)$ is the smallest variety including $\P \mathcal B$.
% \end{thm}
% \begin{proof}
% 	That LT is algebraisable and that $\mathcal{A}\textit{lg}^*(\text{LT})$ is
% 	a quasi-variety follows from the general results presented in
% 	\cite{Font16}. The theorem follows from Definitions 2.3 and 2.5;
% 	Propositions 2.7 and 3.15; and the direct comment after Defininion 3.19 in
% 	the aforementioned work. By compactness of LT together with the deduction
% 	theory for external implication, we can also conclude that it is possible
% 	to present a fully axiomatic Hilbert system for LT, with modus ponens and
% 	substitution as its only rules,\marginpar{Is it clear that we can find a Hilbert system?} and hence that
% 	$\mathcal{A}\textit{lg}^*\text{LT}$ is a variety.

% 	That $ \mathcal{A}\textit{lg}^*(\text{LT}) = \mathcal{V}(\set{\P B |
% 	\text{$B$ is a Boolean algebra}})$ follows from some general results
% 	regarding semilattice-based logics, see for example Theorem 7.46 in
% 	\cite{Font16}.
% \end{proof}

% It is important to note that the theorem implies that $\psi \vDash \varphi$
% iff for all $A \in \mathcal V(\P \mathcal B)$ and all homomorphisms $H: \Fm
% \to A$, if $H(\psi)=\top$ then $H(\varphi)=\top$. However, as mentioned in
% Section \ref{sec:logic} the non-entailment $\ibot \nvDash \ilnot \ibot$ of LT
% shows that is not enough to quantify over all algebras $A$ of the form $\P B$.

% \marginpar{Say something about to what extent the Hilbert-style system for HBML given in
% \cite{Goranko99}, or any general results regarding varieties of powerset
% algebras in \cite{Goldblatt1989} apply to
% $\mathcal{A}\textit{lg}^*\text{LT}$.}

% One way of doing so would be by identifying an informative choice of Hilbert
% system for LT with corresponding equational theory. Another approach is to
% find more general ways than the powerset operation to generate algebras of the
% class from Boolean algebras. If the class of algebras can be exhausted by
% algebras uniformly generated by Boolean algebras it would illuminate and
% identify the relation between these classes of algebras and provide a clear
% semantic connection between LT and classical propositional logic.

To relate LT to other propositional team logics we have identified a set of
principle variable axioms (PVA). This set of axioms can be viewed as a
somewhat natural set of axioms that captures the denotational semantics of the
valuational team logic PT$^+$ in the semantics of LT, bearing in mind how
classical propositional logic is captured in PT$^+$. It is therefore expected
that similar algebraic constructions and axiomatisations are possible for
other types of team semantics such as modal team semantics
\cite{Vaeaenaenen2008}. Furthermore, this axiomatisation provides a way to
construct proofs of the entailment statements of these propositional team
logics. It does however not directly constitute a natural deduction system for
the axiomatised logics per se, since the terms of these proofs will in general
not be confined to the syntactical fragment of the logics. Our natural
deduction system may however motivate, and be seen as a guide in the
construction of deduction systems for these propositional team logics, and
indicates the suitability of labelled systems.

In the labelled natural deduction for LT, the rules $\sub$ and $\taut$
establish a notion of equivalence of labels determined by classical
propositional logic. From this perspective these rules can be viewed as
structural rules of the deduction system. The rules for the formulas of LT
consist of introduction rules together with the elimination rules that are
the direct inverses of the introduction rules (up to equivalence of
labels).\footnote{The rule $\ine$ is not directly the inverse of $\ini$ but
can be seen to be equivalent to a direct inverse rule. } In this sense, the
rules of LT harmonise, and it is possible to use more advanced proof-theoretic
methods to investigate LT. For example, it seems to be easy to turn the system
into a sequent system that could be analysed with respect to cut rules and cut
elimination. This analysis may lead up to a proof-theoretic explanation of the
internal connectives and so also of the connectives in other propositional
team logics through their axiomatisations.

By thinking of the $\taut$ rule as provability in an \emph{inner logic} we can
generalise the construction of the labelled natural deduction system into a
proof-theoretic teamification, or combination of two logics, an inner and an
outer logic. Of such constructions LT represents the special case for which
both are classical propositional logic. This opens the door for a purely
proof-theoretic approach to team logics, and more general relatives, and we
see this as an interesting future research topic. In particular this indicates
the possibility for similar constructions in a first-order setting that could
give new general insight into first-order team logics, which is an active
field of study with many applications.
