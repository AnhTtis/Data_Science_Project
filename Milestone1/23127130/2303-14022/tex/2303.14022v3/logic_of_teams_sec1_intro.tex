%!TEX root = logic_of_teams.tex
\section{Introduction}

Team semantics was invented by Hodges \cite{Hodges1997} to give the
Independence Friendly logic (IF-logic) of Hintikka and Sandu
\cite{Hintikka1989} a compositional semantics. Team semantics was later used
by Väänänen to define Dependence logic \cite{Vaeaenaenen2007}, a formalism
extending first-order logic in which functional dependencies between variables
are explicitly expressed by atomic formulas. The intended meaning of these
atomic formulas $\dep(\bar x,y)$ is that the value of the variable $y$ is
functionally determined by the values of the finitely many variables $\bar x$.
Even though Dependence logic uses only first-order quantifiers it can express
any existential second-order property or statement. The reason is that there
are second-order quantifiers hidden in the truth conditions, most
interestingly in the truth conditions for $\lor$ and $\exists$.

Since then many logics based on team semantics have been introduced and
investigated, such as Independence logic, Propositional dependence logics and
Modal dependence logics.

\subsection{Lifting Tarskian semantics to team semantics}

In the classical Tarskian semantics of first-order logic the denotation of a
formula, given a structure, is defined to be the set of all assignments that
satisfies the formula. Similarly, the denotation of a propositional formula is
the set of valuations satisfying the formula, and the denotation of a modal
logic formula, given a Kripke model, is the set of all worlds satisfying the
formula. Thus, the denotation of a formula in this classical setting is an
element of $\P X$, the powerset of the set $X$ of all assignments, all
valuations or all worlds, respectively.

Team semantics of first-order, propositional and modal logic lifts the
denotations of formulas to be sets of subsets of $X$, i.e., elements of $\P\P
X$, instead of elements of $\P X$. Thus, instead of asking if a single
assignment, valuation or world satisfies a formula, team semantics asks if a
\emph{set} of assignments, valuations or worlds satisfies a formula. Such sets
are called \emph{teams}. In this sense, the team semantical denotation is the
image of the classical denotation under a specific function, a lift, from $\P
X$ to $\P \P X$.

For most team semanticists the \emph{flatness principle} has guided the
definition of the semantics. This principle can be formulated as follows:
\begin{quote}
    A team satisfies a formula iff every individual assignment, valuation, or
    world in the team satisfies the formula in the usual classical sense.
\end{quote}
Note that this only applies to formulas in the original language, without any
dependence atoms or other added connectives; for these formulas it make sense
to say that an assignment/valuation/world satisfies a formula. 
%\marginpar{ XXX
%maybe a more informative sentence also incorporating how establishing flatness
%has been a common step in formulatin team semantics conservatively extending a
%language /OLO} 
Flatness directly gives conservativity in the sense that the
singleton team of the empty assignment satisfies a sentence in a model iff the
sentence is true in the classical sense in the model.

In terms of denotations of formulas the flatness principle naturally
translates to the equation
\begin{equation}\label{eq:powerset}
    \den[h]{\phi} =\P \den[c]{\phi}
\end{equation} 
where $\den[h]{\phi}$ is the team semantical denotation of $\phi$ ($h$ for
Hodges) and $\den[c]{\phi}$ is the ordinary Tarskian denotation of $\phi$ ($c$
for classical). By subscribing to the flatness principle most team
semanticists also subscribe to this powerset lift of classical semantics to
team semantics.

\subsection{Substitutionality and logics}

Dependence logic and its variants have some nice properties such as
compactness, Löwenheim-Skolem properties \cite{Vaeaenaenen2007} and that 
first-order consequences of theories can be axiomatized \cite{Kontinen2013} to
name a few. But they are not substitutional; for example $$\forall x\, (P(x)
\lor P(x))\vDash \forall x\, P(x) \text{, but } \forall x\, (\dep(x) \lor
\dep(x)) \nvDash \forall x\, \dep(x),$$
where $\dep(x)$ is the dependence atom with only one free variable, intuitively
stating that there is only one value of $x$ in the team, i.e., that $x$ is
constant.

In general, a logic is substitutional if $\varphi \vDash \psi$ implies
$\phi[\sigma/p] \vDash \psi[\sigma/p]$, i.e., an entailment does not break by
substituting a formula $\sigma$ for an atom $p$. Substitutionality was already
used by Bolzano to define the concept of validity. In \cite{Bolzano1837}
Bolzano defines \emph{universally valid propositions} as propositions with all
their variants true. In his terminology a variant is nothing but a
substitutional instance. Bolzano, and many after him, thus took
substitutionality not only as an important property for a logic, but as the
basic principle for the concept of logical validity.

Any attempt to algebraise a logic that is not substitutional
will not work in the strict sense since the elements of the algebraic
structure need to be partitioned, or divided, into different types. Indeed,
the examples of algebraisations of variants of propositional dependence logic
that appear in the literature
\cite{Quadrellaro2021,Quadrellaro2020,Bezhanishvili2021} all introduce
additional predicates on the resulting algebras in order to partition the
elements into different types.

The powerset lift of the Tarskian semantics to team semantics, as in
\eqref{eq:powerset}, could be blamed for the lack of substitutionality in
Dependence logic. In fact, any specific lift will limit the possible
denotations of atoms and the resulting logics will not, in general, be
substitutional.

In this paper we will therefore generalize away from using a specific lift.
Instead we will, in a certain sense, quantify over all possible lifts of the
atoms: Given a function $\mathcal L : \P X \to \P\P X$ that lifts
the denotations of atoms from a Tarskian setting to a team semantical setting,
we extend this compositionally to give team semantical denotations
$\den[\mathcal L]{\phi}$ to all formulas $\phi$ in our logic. We define
the entailment relation by $\phi \vDash \psi$ if $\den[\mathcal L]{\phi}
\subseteq \den[\mathcal L]{\psi}$ for all functions $\mathcal L : \P X \to
\P\P X$.

The function $\phi \mapsto \den[\mathcal L]{\phi}$ is nothing but a
homomorphism from the absolutely free term algebra of formulas to a specific
algebraic structure of sets of teams. Describing a logic by quantifying over 
all homomorphisms from a term language into an algebraic structure is exactly 
the starting point for constructing algebraic semantics.

\subsection{Lifting algebraic semantics}

The semantics of classical propositional logic can be defined in terms of
Boolean algebras. One may even say that classical propositional logic
\emph{is} the logic of Boolean algebras. The set of propositional formulas is
the term algebra generated by atoms using the Boolean operators $\bot, \lnot,
\lor$, and $\land$; and the semantics of propositional logic can be stated
using homomorphisms from this term algebra to a Boolean algebra. A formula is
a tautology if its image under any such homomorphism is the top element of the
Boolean algebra. This is the starting point when we define the Logic of Teams,
or LT for short: Lifting the algebraic semantics for classical propositional
logic to the setting of teams. A team in this setting is nothing but a set of
elements of the Boolean algebra.

The connectives, and the corresponding operators on the sets of teams, we are
interested in are the ordinary Boolean connectives $\bot,\lnot, \lor,\land$
that correspond to the empty set, the complement, the union and the
intersection. We will call these connectives and the corresponding operators
\emph{external} Boolean connectives and operators. We will also add the
\emph{internal} connectives and operators that are defined by pointwise
application of  the operators of the Boolean algebra, for example $$A \ilor B
= \set{a \lor b | a \in A, b \in B},$$ where $\ilor$ is the internal
disjunction. We will use blackbord boldface versions of the Boolean
connectives to denote these internal connectives: $\ibot,\ilnot,\ilor,\iland$.
When the underlying Boolean algebra is $\P 2^{\mathbb N}$ with the
set-theoretic Boolean operators, the connective $\ilor$ is exactly the
``splitjunction'' used in Dependence logic, and $\iland$ is, in Dependence
logic, equivalent to conjunction.

\subsection{Related constructions} 

An algebraic structure generated by the powerset of a Boolean algebra with
internal connectives is not new to mathematics and logic. Brink
\cite{Brink1984,Brink1986,Brink1993} contributes to these investigations and
calls them \emph{power algebras}, whereas Goldblatt calls them \emph{complex
algebras} \cite{Goldblatt1989} referring to a subset of an algebraic group as
a complex. It is also worth noting that these algebras are special cases of
\emph{Boolean algebras with operators} as described by J\'onsson and Tarski
\cite{Jonsson1993}. These play a notable role in the algebraic treatment of
Modal logic, see \cite{Venema2007}.

In other related work, Priest has utilised the powerset lift in order to
investigate resulting families of \textit{plurivalent logics}
\cite{Priest2017}, an effort elaborated on by Humberstone in
\cite{Humberstone2014} coining the term \textit{power matrices} for the
resulting constructions. In these papers, semantics of multivalued logics is
lifted into evaluations on subsets of possible truth values, and the logical
connectives are interpreted in terms of pointwise operations. They do,
however, not include any connectives relating to the set-theoretic Boolean
operations on the powerset algebra.

In particular, in \cite{Goranko99} Goranko and Vakarelov use power sets of
Boolean algebras, referred to as \textit{hyperboolean algebras}, in order to
define what they call the \textit{hyperboolean modal logic}, HBML. This
construction treats a Boolean algebra, expressed as a partial order, as a
Kripke frame and utilise the Boolean structure to define modal operators.
% the following unary and trinary accessibility relations given for all $b,c,d
% \in B$
% \begin{itemize}
% 	\item $R_\bot(b)$ iff $b=0$
% 	\item $R_\to(b,c,d)$ iff $b= c\to d$ 
% \end{itemize} 
% These relations result in a Kripke-style semantics in terms of satisfacion
% with the constant and binary Modal operators $\langle 0\rangle$ and $\langle
% \to \rangle$ for which the defining clauses in the definition of satisfaction
% $M,b \Vdash \phi$ for all Models $(B,V)$ and all points $b\in B$:
% \begin{itemize}
% 	\item $M,b \Vdash \langle 0 \rangle$ iff $b=0$ 
% 	\item $M,b \Vdash \phi \langle \to \rangle \psi $ iff there is $c,d \in B$
% 	s.t.{} $b= c\to d$, $M,c \Vdash \phi $ and $M,d \Vdash \psi.$
% \end{itemize}
% \fi 
The algebraic counterpart of the logic HBML is thus the powerset algebra of
the original Boolean algebra, with the modal operators defined as the internal
pointwise operations on the underlying algebra.

% \iffalse  $\langle 0 \rangle$ and $\langle \to \rangle$ identified with the
% pointwise application of the operations in the underlying algebra.

% Goranko and Vakarelov also identifies some connectives definable in HBML, in
% particular a unary difference modality $\langle \neq \rangle $ and a unary
% only operator $\mathcal{O}$ with quite hairy definitions in HBML, but fairly
% simple semantics interpretations
% \begin{itemize}
% 	\item $M,b \Vdash \langle \neq \rangle \phi$ iff there is $c\in B$ s.t.
% 	$c\neq b$ and $M,c \Vdash \phi $.
% 	\item $M,b \Vdash \mathcal{O} \phi $ iff for all $c\in B$: $M, c\vDash
% 	\phi$ only when $c=b$.
% \end{itemize}

% They present a sound and complete Hilbert-style system for validity utilising
% the difference modality to hinge on previous work, and the only operator to
% axiomatise the underlying Boolean structure.

The logic of teams that we define in this paper has different motivation and
origin than that of HBML, but building on the same class of models. It is easy
to see that the connectives of LT and HBML are  interdefinable such that the
two logics share validities and can in this sense be viewed as having the same
theorems. However, where Goranko and Vakarelov only defines HBML in terms of
validity with a proof system fundamentally structured around an elaborately
defined \textit{difference modality} and an \textit{only operator}, we define
LT for a full entailment notion and present a labelled natural deduction
system for which the rules directly correspond to the basic connectives of the
logic. Even so, the correspondence between the two logics is strong enough for
some important properties of HBML presented in \cite{Goranko99} to also apply
to LT, see Section \ref{sec:canonical}.

\subsection{Structure of this paper} 

In the next section we formally introduce the syntax and semantics of the
Logic of Teams, LT. The semantics is defined in an algebraic manner in terms
of homomorphisms into algebraic structures based on Boolean algebras. 
% This is
% done in the most natural way and mirrors the interpretations of team semantics
% in respect to its denotation. 
In the same section we also introduce a labelled
natural deduction system for LT. The formulas are decorated by labels and the
labels are themselves classical propositional formulas. By including rules in
the deduction system identifying classically equivalent formulas as equivalent
labels we establish the role of the labels as references to elements in a
Boolean algebra, and this paves the way for the completeness proof via
Lindenbaum--Tarski algebras presented in the next section.

Section 3 is devoted to prove the completeness of the natural deduction system
and the consequences that can be observed by a more careful investigation of
the proof. This section also includes results regarding non-canonicity and
adequacy of sets of Boolean algebras.

In Section 4 we introduce some important definable connectives in LT. In
particular we define the \textit{strict negation} that, apart from being an
interesting type of negation, will be important in Section 6. We also define a
universal $\Box$-modality, which plays an important role in Section 5, where
definable classes of homomorphisms and axiomatisations of their corresponding
logics are discussed.

In section 6 we finally utilise the strict negation to define classes of
homomorphisms, and give axioms expressing the strong propositional team logic,
PT$^+$ in \cite{Yang2017}, as a part of LT. In this way we establish the
connection between LT and the propositional team logics found in the
literature with semantics based on teams of valuations, here referred to as
\textit{valuational team semantics}. These results establish LT as a well
motivated, substitutional, and expressively rich propositional team logic that
is highly relevant for a better understanding of team semantics for
propositional logics. 

In the final section of the paper we reflect generally on the construction
that we have presented, and discuss some further topics of investigation that
are implicated by our work.