%!TEX root = logic_of_teams.tex


\section{Axiomatising valuational team semantics}\label{sec:axiomatising}

In this section we show how standard team semantics based on valuations relate
to the logic LT. This construction follows in essence the construction
presented in \cite{LorimerOlsson2022} with some refinements and
generalisations. The construction and proof will be developed through the
following steps, enumerated by the corresponding subsection.

\begin{enumerate}
	\item[6.1.] We start by defining the propositional team logic
	$\text{PT}^+$ in terms of its standard semantics, here referred to as
	\textit{valuational team semantics}, by defining its denotations in terms
	of collections of teams of valuations. This is an expressive logic with
	many interesting logics as syntactic fragments.
	\item[6.2.] Then we show how to formulate this semantics in terms of the
	algebra $\P\P 2^{\mathbb{N}}$ with a specific homomorphism $H_V$ in the
	style of the semantics we have given for LT. We can then prove that LT is
	in a sense a conservative extension of a logic weaker than $\text{PT}^+$.
	\item[6.3.] Next we observe that the homomorphism $H_V$ belongs to a special
	class of homomorphisms $\mathcal{H}_{\text{PV}}$ that are nicely definable
	and axiomatisable in LT using defined connectives from Section
	\ref{sec:definable}.
	\item[6.4.] We can then show that the axiomatisation of this class
	axiomatises $\mathrm{PT}^+$ as a fragment of LT. This is done by
	essentially showing that for the restricted language of $\mathrm{PT}^+$
	the homomorphism $H_V$ is canonical for the logic of the whole class
	$\mathcal{H}_\mathrm{PV}$. The proof includes two main steps expressed in
	two separate lemmas:
	\begin{itemize} 
		\item Lemma \ref{embedding_lemma} states that for embeddings of
		Boolean algebras all homomorphism in $\mathcal{H}_{\text{PV}}$ to the
		domain of the embedding, can be paired with homomorphism in
		$\mathcal{H}_{\text{PV}}$ to the co-domain such that memberhood in
		interpretations of formulas of $\text{PT}^+$ are preserved by the
		embedding. This together with Stone's theorem establishes that the
		complete atomic Boolean algebras are adequate for the logic of the
		class $\mathcal{H}_{\text{PV}}$.
		\item Lemma \ref{f_rep_in_Val} then establishes that every
		homomorphism in $\mathcal{H}_\mathrm{PV}$ with a complete atomic
		Boolean algebra as co-domain can be represented by the homomorphism
		$H_V : \Fm \to \P2^\mathbb{N}$ in such a way that $H_V$ expresses the
		memberhood of the original homomorphism by the mapping for
		interpretations of formulas in $\text{PT}^+$.
	\end{itemize}
	With these lemmas together we conclude that $\text{PT}^+$ is axiomatised
	as a fragment of LT by the axioms of the class of homomorphisms.
\end{enumerate} 

In this section we will treat several logics defined for different sets of
connectives, and thus with different sets of formulas. For a logic L we denote
by $\Fm_{\text{L}}$ the formulas of the logic and $\text{L}:\Delta \vDash
\phi$ for the entailment of the logic. In particular, the formulas of LT will
be referred to as $\Fm_{\text{LT}}$, and its entailment will be denoted by
$\text{LT}: \Delta \vDash \phi$ in contrast to other sections of this paper.

\subsection{Valuational team semantics}\label{sec:valuational}

We are interested in formulating a logic in a sense generalising some
propositional team logics given in team semantics where the notion of
\textit{team} refers to sets of valuations being the basic semantic object. We
call this \textit{valuational team semantics}. We represent the valuational
team semantics by recursively defining the denotation of formulas on the set
of teams of valuations $\P2^\mathbb{N}$. In order to cover a majority of the
more interesting logics we focus our attention on \textit{strong propositional
team logic}, $\text{PT}^+$, one of the strongest logics presented in
\cite{Yang2017}.

\begin{defin}
The set of formulas $\Fm_{\text{PT}^+}$ of \emph{Strong propositional team logic},
$\text{PT}^+$,  is generated by the following grammar $$ \phi::= P_i
\mathrel|  \slnot P_i \mathrel|\ibot \mathrel| \NB\mathrel|
\phi \ilor \phi\mathrel|   \phi \land \phi \mathrel|  \phi \lor \phi $$ and
its valuational team semantics can be described by defining the denotations
for formulas $\llbracket \phi \rrbracket \subseteq \P 2^\mathbb{N} $
recursively for cases of the main connective as follows:
\begin{align*}
	\llbracket  P_i \rrbracket &= \set{ X | \text {for all } s \in X, s(i)= 1}\\
	\llbracket \slnot P_i \rrbracket &= \set{ X | \text {for all } s \in X, s(i)= 0}\\
	\llbracket \ibot \rrbracket &= \set{ \emptyset}\\
	\llbracket \NB \rrbracket &= \set {X| X\neq \emptyset} \\
	\llbracket \phi \ilor \psi \rrbracket &= \set{ X\cup Y | X\in \llbracket
	\phi \rrbracket, Y\in \llbracket \psi \rrbracket} \\
	\llbracket \phi \land \psi \rrbracket &= \llbracket \phi \rrbracket \cap
	\llbracket \psi \rrbracket\\
	\llbracket \phi \lor \psi \rrbracket &= \llbracket \phi \rrbracket \cup
	\llbracket \psi \rrbracket\\
\end{align*}
The logical entailment of $\text{PT}^+$ is then defined as follows:
$$\text{PT}^+: \Delta \vDash \phi \quad \text{iff} \quad
\bigcap_{\delta \in \Delta}  \llbracket \delta \rrbracket \subseteq \llbracket
\phi \rrbracket.$$
Elements $s\in 2^\mathbb{N}$ are viewed as \textit{valuations} for the set of
propositional variables, and sets of valuations $X\in \P 2^\mathbb{N}$ are
referred to as \textit{teams (of valuations)}.
\end{defin}

There is no standard notation for the connectives in the literature, and we have
chosen notation that corresponds best to the notation for LT. Table
\ref{translation} indicates the correspondence between our notation and
notation elsewhere. 

\begin{table}
	\label{translation}
\begin{tabular}{cccc}
	\toprule
	\cite{Yang2017} & \cite{Lueck2020} & \cite{Yang2022} & This paper \\
	\midrule
	$p_i$  		& $p_i$ 	& $p_i$ 	& $P_i$\\
	$\neg$ 		& $\neg$	& $\neg$	& $\slnot$ \\
	$\bot$ 		& $\bot$ 	& $\bot$	& $\ibot$ \\
	$\otimes  $ & $\vee $	& $\vee$	& $ \ilor $ \\
	$\wedge$ 	& $\wedge$	& $\wedge$  & $\land $ \\
	$ \vee $ 	& $\ovee$	& $\ilor$   & $ \lor $ \\
	$\text{NE}$ & ~ 		& ~			& $\NB$ \\
	\bottomrule
\end{tabular}
\caption{Correspondence between notations in the current paper and other
relevant papers on propositional team logics.}
\end{table}

As defined, it is clear that for the set of formulas, including defined
connectives, we have that $\Fm_{\text{PT}^+}\subset
\Fm_{\text{LT}}$ even though the logics are described using different
semantics. In this section we will find axioms in the language of LT that
axiomatises $\text{PT}^+$ in LT. In the literature there are multiple weaker
propositional team logics described and studied, in particular propositional
logics of dependence \cite{Yang2016}. Many important logics, however, are
given as, or is expressively equivalent to, logics that can be given as
fragments of $\text{PT}^+$. This means that our result regarding the
axiomatisation of $\text{PT}^+$ in LT will be directly applicable to these
logics too. Table \ref{list_of_logics} gives an overview of these weaker
logics that can be found in the literature. The logic $\text{PD}^\vee$ is
described in \cite{Yang2016}, and the others are described in \cite{Yang2017}.

\begin{table}
	\begin{tabular}{ll}
	\toprule
		Propositional team logics. 								& Connectives \\
		\midrule 
		Classical propositional logic (CPL) 					&$\slnot P_i,\ibot,\ilor,\land$\\
		Strong classical propositional logic ($\text{CPL}^+$) 	&$\slnot P_i,\ibot,\ilor,\land,\NB$\\
		Propositional union closed logic ($\text{PU}$)		&$\slnot P_i,\ibot,\ilor,\land,\circledast$ \\
		Strong propositional union closed logic ($\text{PU}^+$)&$\slnot P_i,\ibot,\ilor,\land,\circledast,\NB$ \\
		Propositional dependence logic w. int. disj. ($\text{PD}^\vee$)& $ \slnot P_i,\ibot,\ilor, \land,\lor$ \\
		Propositional team logic ($\text{PT}$)				& $\slnot P_i,\ibot,\ilor,\land,\lor,\circledast$ \\
		Strong propositional team logic ($\text{PT}^+$)		& $\slnot P_i,\ibot,\ilor,\land,\lor,\NB$\\
		\bottomrule
	\end{tabular} 
\caption{Names and included connectives of logics described in
\cite{Yang2016} and \cite{Yang2017} as fragments of $\text{PT}^+$. The
connective $\circledast$ can be defined in $\text{PT}^+$ as $\phi \circledast
\psi := (\phi \land \NB) \protect\ilor (\psi \land \NB)$. By $\slnot P_i$ we mean that
$\slnot$ is only allowed to be applied to propositional variables.}
\label{list_of_logics}
\end{table}

\subsection{$\text{PT}^+$ as a model of LT} 

Observe that the denotations of valuational team semantics are elements of the
set $\P\P 2^{\mathbb{N}}$, which can be interpreted as a model of LT by
interpreting $\P 2^{\mathbb{N}}$ as a Boolean algebra using the standard set
operations. We will refer to this as \textit{the valuation model}. With this
reading we can see that the interpretation of atomic formulas imposes a
specific homomorphism that maps every atomic formula to the set of teams for
which every member evaluates it to true.

Define the \textit{valuation homomorphism} $H_V : \Fm_{\text{LT}} \to \P\P
2^\mathbb{N}$ as the unique homomorphism such that
\[ 
H_V(P_i) = \set { X\in \P 2^{\mathbb{N}} | \text{ for all } s\in X, s(i)= 1 }
= \P \set{ s\in 2^{\mathbb{N}} | s(i) = 1}.
\] 
Thus, 
\[
H_V :\Delta \vDash \phi \quad \text{iff} \quad \bigcap_{\delta \in
\Delta}H_V(\delta) \subseteq H_V(\phi).
\]
Let us formulate this in the following theorem.

\begin{thm}\label{directThm}
	For all formulas $\Delta\cup \{\phi\} \subseteq \Fm_{\mathrm{PT}^+}$:
	\[
		\mathrm{PT}^+: \Delta \vDash \phi \quad \text{ iff } \quad  H_V
	: \Delta \vDash \phi.
	\]
	Hence, if  $ \text{LT} : \Delta \vDash\phi$, then $\mathrm{PT}^+: \Delta \vDash \phi$.
\end{thm} 


\subsection{Axiomatising a specific class of homomorphisms}\label{Ax-class-hom}

We observe that the valuation homomorphism $H_V$ has the following special
property. 
\[ 
H_V (P_i)= \P X \text{ for some } X \in \P 2^{\mathbb{N}}
\]
Algebraically speaking, every propositional variable is mapped to a non-empty
principal ideal of the Boolean algebra on $\P(2^\mathbb{N})$, that is, a
subset $A$ of the Boolean algebra $B$ such that there is a maximal element
$a\in A$ generating $A$, i.e., such that $$A= \set{ b\in B | b\leq a}.$$ In
other words, $A$ is downwards closed and $\bigvee A\in A$.

We can in fact express the property of being a principal ideal in LT in the
following formula akin to the excluded middle:

\begin{thm}\label{excluded_middle} 
	For all Boolean algebras $B$, all homomorphism $ H: \Fm \to \P B$, and all
	formulas $\phi \in \Fm$ we have $$ H: ~ \vDash \phi \ilor \slnot \phi
	\quad \text{ if and only if } H(\phi) \text{ is a principal ideal} $$
\end{thm}

As part of proving this statement we will use a small result that will be
useful to refer to multiple times in upcoming proofs. We therefore state it
as a separate lemma.
\begin{lemma}\label{negated_principal}
	For a Boolean algebra $B$, if $A\subseteq \P B$ is a principal ideal with maximal element $a\in
	B$, then $\slnot A$ is a principal ideal with maximal element $\lnot a$.
\end{lemma} 
\begin{proof}[Proof of lemma]
First note that for all $A\subseteq B$ we have that $\slnot A$ is downwards
closed, since if $c\leq c'\in \slnot A$ then for all $b\in A$, $ b\land c \leq
b\land c' = \bot$. What is left to show is that, if $a= \bigvee A$, then
$\neg a = \bigvee \slnot A$. Clearly, for all $a'\in A$ we have
$a'\land a = a'$ and thus $\lnot a \land a' = \bot$, and thus $\lnot a \in \slnot
A$. Furthermore, if $b\in \slnot A$, then $a\land b = \bot$ and
therefore $\lnot a = \lnot a \lor (a \land b)= \lnot a \lor b$. i.e $b\leq
\lnot a$. thus $\lnot a $ is an upper bound of $ \slnot A$ included in
the set, and thus the set is a principal ideal, and $\lnot a = \bigvee
\slnot A$.
\end{proof}
\begin{proof}[Proof of Theorem \ref{excluded_middle}]

For one direction, assume $H(\phi \ilor \slnot \phi)= B$, we then need to
prove that  $H(\phi)$ is a principal ideal. First we establish that it is
downwards closed.  In search of a contradiction, assume that for
some $a\in H(\phi)$ we can find $b\in B$ such that $b\leq a$ and $b\notin
H(\phi)$. We want to show that then $b\notin H(\phi \ilor \slnot
\phi)$. If $b\in H(\phi\ilor \slnot \phi)$, then there exists $c\in H(\phi)$
and $d \in \slnot H(\phi)$ such that $c\lor d= b $. Since $b\leq a \in
H(\phi)$ we have that $$ b\land d \leq a\land d = \bot$$ since $d\in
\slnot H(\phi)$. But then, since clearly $c\leq b$, we have $$b= b\land b
= (c\lor d ) \land b = ( c\land b )\lor (b\land d)= c\in H(\phi)$$ This is a
contradiction. We can therefore conclude, under the main assumption, that
$H(\phi)$ is downwards closed. Next we show that $\bigvee H(\phi)\in H(\phi)$.
By assumption we have that  $$\top \in H(\phi \ilor \slnot
\phi)$$ Then there exists $c \in  H(\phi)$ and $d \in H(\slnot \phi )$ such
that $c\lor d=\top$. Furthermore, for all $a \in H(\phi)$ we have that $$a =
\top \land a = (c\lor d) \land a = c \land a, $$ since $d \land a  =
\bot$ for all $a \in H( \phi) $. Consequently $a \leq c $ for all $a
\in H( \phi)$ and thus $ c$ is an upper bound for $H(\phi)$ included in
the set, i.e. $c  = \bigvee H(\phi)$. With downwards closure established, this
also means that $H(\phi)$ is a principal ideal.
	
	
For the other direction, assume $H(\phi)$ is a principal ideal. Then there
exists $a \in B$ such that $a$ is the top element of $H(\phi)$. Then by Lemma
\ref{negated_principal} we see that  $\lnot a$ is the top element of the
principal ideal $H(\slnot \phi)$. Therefore, since $a \lor \neg a =
\top$ we have for every $b\in B$ that $$b= b \land (a\lor \neg a)= (b\land a )
\lor (b\land \neg a).$$ Being the top elements of the respective principal
ideals we observe that $$b\land a \in  H(\phi)\quad \text{and} \quad b
\land \neg a \in H(\slnot \phi)$$ and conclude that  $ H(\phi  \ilor
\slnot \phi)= B$, in other words $H\vDash \phi \ilor \slnot \phi$.
\end{proof}

From this theorem we can directly conclude for the valuation homomorphism
$H_V$ that
\[
H_V : {}\vDash P_i \ilor \slnot P_i \quad \text{for all } i \in
\mathbb{N}.
\]
We will see that this is the crucial categorisation of the homomorphisms that
relate to valuational team logics. We therefore identify the class defined by
these formulas, and the corresponding axiomatisation as discussed in Section
\ref{sec:def-class}.

\begin{defin}
	Let $\mathcal{H}_{\text{PV}}$ denote the class of homomorphisms defined by
	$\set{ P_i\ilor \slnot P_i | i\in \mathbb{N}}$. We say that a
	homomorphism $H$ \textit{has principal  variables} iff
	$H\in\mathcal{H}_{\text{PV}}$. Furthermore, let \emph{the principal variable
	axioms} be the set
	\[
	\text{PVA}=\set{\Box ( P_i \ilor P_i) |i\in	\mathbb{N}}.
	\]
\end{defin}
 
 It follows directly from Theorem \ref{Def_to_ax} that, for all $\Delta,\set{\phi}\subseteq \Fm_{\text{LT}}$,
 \[
 	\mathcal{H}_{\text{PV}} :	\Delta \vDash \phi \quad \text{if and only if} \quad \text{LT}:\text{PVA},
	\Delta\vDash \phi.
\]
It is evident that $H_V\in \mathcal{H}_{\text{PV}}$. 


\subsection{Axiomatisation of $\text{PT}^+$ as a fragment}

In this section we prove that the axioms PVA axiomatise $\text{PT}^+$ in the
sense of the following theorem.
\begin{thm}\label{thm:mainAx}
	For all $\Delta \cup \{ \phi \} \subseteq \Fm_{\mathrm{PT}^+}$ 
	\[ 
	\mathrm{PT}^+: \Delta \vDash \phi \quad \text{iff} \quad \mathrm{LT} :
	\mathrm{PVA}, \Delta \vDash \phi.
	\]
\end{thm}
One direction is already proved by Theorem \ref{Def_to_ax} and
\ref{directThm}. To prove the other direction we first recall the fundamental
result by Stone in the theory of Boolean algebras.

\begin{thm}[\cite{Halmos2009}]\label{thm:completeBA}
	Every Boolean algebra can be embedded into a complete atomic Boolean
	algebra of the form $(\P S, \emptyset, \cdot^C, \cup, \cap)$ for some set
	$S$.
\end{thm}

Using this theorem we can establish Theorem \ref{thm:mainAx} in a two step
process. First showing that any homomorphism $H\in \mathcal{H}_{PV}$ can be
faithfully represented by a homomorphism in a Boolean algebra of subsets, and
then that any such homomorphism can be represented by the specific valuational
homomorphism $H_V: \Fm  \to \P \P 2^{\mathbb{N}}$. These steps are established
below in Lemma \ref{embedding_lemma} and Lemma \ref{f_rep_in_Val}
respectively. Both proofs are similar in structure, where the statement is
proven by induction over the complexity of formulas in $\mathrm{PT}^+$ for
which particular care is needed to handle the internal connective $\ilor$. 
In the first case this invokes the formulation of an additional sub-lemma, and in
the second case it suffices with a simpler result expressed as a claim.

%\marginpar{additional sentence or to involved allready?}
\begin{lemma}\label{embedding_lemma}
	Let $B,B'$ be Boolean algebras, and $e:B\hookrightarrow B'$ an embedding
	of Boolean algebras. Then for each homomorphism $H:
	\Fm_{\text{LT}} \to \P B $ with principal variables there  is a
	homomorphism $H': \Fm_{\text{LT}} \to \P B'$  with principal variables
	such that for all formulas $\phi \in \Fm_{\text{PT}^+}$ $$ b\in H(\phi)
	\quad \text{ if and only if } \quad e(b) \in H'(\phi).$$
\end{lemma} 
\begin{proof}
	Assume $H: \Fm \to \P B$ has principal variables. Thus, for all $i$ there
	exists $b_i\in B$ such that $$H(P_i)= \set{ a\in B | a\leq b_i  }.$$ We
	then define $H': \Fm \to \P B'$ as the homomorphism that maps each
	propositional variable $P_i$ to the principal ideal of $e(b_i)$, i.e.,
	$$H'(P_i)= \set{ a'\in B'| a'\leq e(b_i)}.$$
	
	Clearly, $H'$ has principal variables, so what is left to show is that the
	equivalence in the theorem holds for all $b\in B$ and all formulas $\phi$
	of PT$^+$. In order to do so, we need an additional lemma best described
	using the notation of Boolean intervals.\footnote{The introduction of
	Boolean intervals in this context is inspired by their usage in
	\cite{Hella2023} in which minimal covers of Boolean intervals are used to
	define complexity measures for expressions in first-order team semantics.
	Even if our usage is noticeably different, the proof construction we
	present for forming included covering intervals is strongly related to the
	constructive methods of generating minimal interval covers for collections
	of teams satisfying a formula presented in that paper.}
	\begin{defin} 
		In a Boolean algebra $B$, if $a,b\in B$ and $a\leq b$, then let
		$[a,b]\subseteq B$ denote the set of elements between $a$ and $b$,
		that is $$[a,b]= \set{c\in B | a\leq c \text{ and } c\leq b  }. $$ We
		call this the closed interval of $a$ and $b$, and if $a=b$ we may
		write $[a]$ instead of $[a,a]$.
	\end{defin}  

\begin{lemma}[Interval lemma]For all Boolean algebras $B,B'$, Boolean
embedding
$e:B\hookrightarrow B'$, homomorphisms $H:Fm\to \P B$ and the
generated homomorphism $H': \Fm \to \P B'$  as described above, we have that for
all $\phi\in \Fm_{\text{PT}^+}$ and all $b'\in B'$, if $b'\in H'(\phi)$,
then there is a $b\in B$ such that $b'\leq e(b)$ and $[b',e(b)]\subseteq
H'(\phi)$.
\end{lemma} 
\begin{proof}
	We prove this statement for all $b\in B$ by induction over the complexity
	of formulas. Let $b_i$ denote the generating element of the principal
	ideal $H(P_i)$.
	\begin{itemize}
		\item Assume $\phi = \ibot$. Then $b'\in H'(\ibot)$ only if $b'=  \bot
		= e(\bot)$, and thus $[b', e(\bot)]= [\bot] \subseteq H'(\ibot)$.
		\item Assume $\phi = \NB $. Then if $b'\in H'(\NB)$ clearly
		$[b',e(\top)]\subseteq H'(\NB)$.
		\item Assume $\phi = P_i$. If $b'\in H'(P_i)$, then $b'\leq e(b_i)$
		and, since $H'(P_i)$ is a principal ideal, clearly
		$[b',e(b_i)]\subseteq  H'(P_i)$.
		\item Assume $\phi = \slnot P_i $. By Lemma \ref{negated_principal}
		and a similar argument as for the previous case we can conclude that
		if $b'\in H'(\slnot P_i)$ then $[b', e(\lnot b_i)] \subseteq H'(\slnot
		P_i)$.
		\item Assume $\phi = \psi \land \chi$. If $b'\in H'(\phi)$, then
		$b'\in H'(\psi)$ and $b'\in H'(\chi)$. by induction, we may find
		$b_\psi,b_\chi\in B$ such that $[b',e(b_\psi)]\subseteq H'(\psi)$ and
		$[b',e(b_\chi)]\subseteq H'(\chi)$. Since $e$  is a
		homomorphism $$[b',e(b_\psi \land b_\chi)]\subseteq H'(\phi).$$
		\item $\phi = \psi \lor \chi$. If $b' \in H'(\phi)$, then $b'\in
		H'(\psi)$ or $b'\in H'(\chi)$, and without loss of generality we may
		assume the former. By induction hypothesis there is some $b_\psi \in
		B$ such that $[b',e(b_\psi)]\subseteq H'(\psi)$, and thus
		$[b',e(b_\psi)]\subseteq H'(\phi)$.
		\item $\phi = \psi \ilor \chi$. If $b'\in H'(\phi)$, then there is $
		c'\lor d'= b'$ such that $c'\in H'(\psi)$ and $d'\in H'(\chi)$. By
		induction hypothesis, we can then find $b_\psi, b_\chi \in B$ such
		that $[c',e(b_\psi)]\subseteq H'(\psi)$ and $[d',e(b_\chi)]\subseteq
		H'(\chi)$. Then since $b'= c'\lor d'$ and $e$ is a homomorphism, it is
		easy to see that $$H'(\phi)\supseteq \set{ p\lor q | p\in
		[c',e(b_\psi)], q \in [d', e(b_\chi)]} = [b', e(b_\psi \lor
		b_\chi)].$$
	\end{itemize}
	This concludes the proof of the interval lemma.
\end{proof}

Now we are ready to finish the proof of Lemma \ref{embedding_lemma}. Again
this is achieved for all $b\in B$ by induction over the complexity of
formulas.
	\begin{itemize}
	\item Assume $\phi = \ibot$. First note that $b\in H(\ibot)$ if and only
	if $ b= \bot\in B$ and  $b'\in H'(\ibot)$ if and only if $b'= \bot \in
	B'$. Then note that $e(\bot)= \bot\in B'$ by $e$ being a homomorphism, and
	by $e$ being injective $e(b)= \bot$ if and only if $b= \bot\in B$. These
	observations suffice to prove the statement.
	\item Assume $\phi = \NB$. Since regardless of homomorphism and algebra
	$H(\NB)$ is the complement set of $H(\ibot)$, a similar argument proves
	the statement.
	\item Assume $\phi = P_i$. For all $b\in B$ we have that  $b\in H(\P_i)$
	if and only if $b\leq b_i$. Then by $e$ being an embedding this holds if
	and only if  $e(b)\leq e(b_i)$, which is equivalent to $e(b)\in H'(P_i)$.
	\item Assume $\phi = \slnot P_i$. By Lemma \ref{negated_principal}
	$H(\slnot P_i)$ is the principal ideal generated by $\lnot b_i$. Similarly,
	$H'(\slnot P_i) $  is the principal ideal generated by $\lnot e(b_i)$.
	Thus, a similar argument as for $\phi = P_i$ suffices.
	\item Assume $\phi = \psi \land \chi$. Then $b\in H(\phi)$ if and only if
	$b\in H(\psi)$ and $b\in H(\chi)$. By induction we can conclude that this
	holds if and only if  $e(b)\in H'(\psi)$ and $e(b)\in H'(\chi)$ which
	holds if and only if  $e(b)\in H'(\phi)$.
	\item Assume $\phi = \psi \lor \chi$. The result follows in a similarly
	standard way as the previous case.
	\item Assume $\phi = \psi \ilor \chi$. If $b\in H(\phi)$, then there
	exists $c\lor d = b$ such that $c\in H(\psi)$ and $d\in H(\chi)$. By
	induction hypothesis we directly see that $e(b)= e(c)\lor e(d) \in
	H'(\phi)$. For the other direction, assume $e(b)\in H'(\phi)$. Then there
	exists $c'\lor d' = e(b)$ such that $c'\in H'(\psi) $ and $d'\in
	H'(\chi)$. Now by the interval lemma above we can find $b_\psi, b_\chi \in
	B$ such that $[c',e(b_\psi)]\subseteq H'(\psi)$ and
	$[d',e(b_\chi)]\subseteq H'(\chi)$. Now, let $c= b\land b_\psi$ then
	$$e(c)= e(b)\land e(b_\psi)= (c'\lor d')\land e(b_\psi)= c'\lor  (d' \land
	e(b_\psi))$$ so that clearly $e(c)\in [c', e(b_\psi)]\subseteq H'(\psi)$,
	and by induction hypothesis $c\in H(\psi)$. Similarly define $d= b\land
	b_\chi$ and conclude that $d\in H(\chi)$. What is left to show is that $b=
	c\lor d$. We see that
	\begin{multline*}
	c\lor d= (b\land b_\psi) \lor (b \land b_\chi)= (b
	\lor (b \land b_\chi)) \land (b_\psi\lor (b \land b_\chi))= \\b\land
	((b_\psi \lor b ) \land (b_\psi\lor b_\chi) )= b, 
	\end{multline*}
	where the last  equality comes from the observation that $e(b)= c'\lor d'
	\leq e(b_\psi)
	\lor e( b_\chi) = e(b_\psi\lor b_\chi)$, and by $e$ being injective
	letting us conclude that $b\leq b_\psi\lor b_\chi$.
\end{itemize} This concludes the proof of Lemma \ref{embedding_lemma}.
\end{proof}

\begin{lemma}\label{f_rep_in_Val}
For every Boolean algebra of the form $(\P S, \emptyset, \cdot^C, \cup ,\cap)$
and every homomorphism $H:\Fm_{\text{LT}} \to \P \P S$ with principal
variables there is a mapping $f_H:S \to 2^{\mathbb{N}}$  such that for all
formulas $\phi \in \Fm_{\text{PT}^+}$ and all $X\in \P S$:
\[ 
X\in H(\phi) \quad \text{iff} \quad f_H^*(X) \in H_V(\phi),
\] 
where $f_H^*:\P S \to \P 2^{\mathbb{N}}$ is defined by $f_H^*(X) = \set{f_H(x) | x
\in X}$, and $H_V:\Fm_{\text{LT}} \to \P\P 2^\mathbb{N}$ denotes the valuation
homomorphism.
\end{lemma}
\begin{proof}
	Assume $H:\Fm_{\text{LT}} \to \P \P S$ has principal variables. Let $f: S\to
	2^{\mathbb{N}}$ be defined by
	\[
		f(s)(i)= \begin{cases} 1 & \text{ if } \{s\} \in H(P_i) \\ 
								0 & \text{ if } \{s\} \notin  H(P_i).
					\end{cases}
	\]

	\begin{quote} 
	\textbf{Claim:} For all $X,Y\in \P S$, $f^*(X\cup Y) = f^*(X)\cup f^*(Y)$.
	Furthermore, if $f^*(X) = U\cup V$, then there exist $Y,Z\in \P S$ such
	that $f^*(Y)= U$, $f^*(Z)= V$ and $Y\cup Z= X$.
	\end{quote}

	The first part of the claim is self-evident by the definition of $f^*$
	from $f$. For the second part, let $Y= \set{s\in X| f(s)\in U}$ and $Z=
	\set{s\in X | f(s) \in V}$.
	
	We can now prove that the function $f$ is what we looked for in the lemma
	by induction over formulas $\phi\in	\Fm_{\text{PT}^+}$.
	
	We have four types of base cases:  $\ibot, \NB, P_i, \slnot P_i $:
	\begin{itemize}
		\item Assume $\phi = \ibot$. By definition $X\in H(\ibot)$ iff $X= \emptyset$, and
		since $f^*(X) = \emptyset $ if and only if $X= \emptyset$ this case is evident.
		\item Assume $\phi = \NB$. This case is proved by contraposition of the previous
		case.
		\item Assume $\phi = P_i$. By assumption $H(P_i)$ is a principal ideal in the Boolean
		algebra $\P S$. Hence, for all $X \in H(P_i)$, by downwards closure we have
		for all $s\in X$ that $\{s\} \in H(P_i)$. By construction then
		$f^*(\{s\}) \in H_V (P_i)$ for all $s\in X$, and thus since $H_V$ has
		principal variables we find that $f^*(X)\in
		H_V(P_i)$. The opposite direction is proved with a similar chain of
		arguments.
		\item Assume $\phi = \slnot P_i$. Then by Lemma
		\ref{negated_principal} we assert that for all $H\in \mathcal{H}_{PV}$
		we have that $H(\slnot P_i)$ is a principal ideal. The proof is then
		similar to the previous case.
	\end{itemize}

	For the induction step we have three cases for the main connectives:
	$\land ,\lor,\ilor$
	\begin{itemize}
		\item Assume $\phi = \psi \land \chi$. $X\in H(\phi)$ by definition if
		and only if $X\in H(\psi)$ and $X\in H(\chi)$. By induction
		hypothesis, we can conclude that is the case if and only if $f^*(X)\in
		H_V(\psi)$ and $f^*(X)\in H_V(\chi)$, which is equivalent to stating
		that $f^*(X) \in H_V(\phi)$.
		\item Assume $\phi = \psi \lor \chi$. The proof is similarly standard
		as the previous case.
		\item Assume  $\phi = \psi \ilor \chi$. For one direction, assume $X \in
		H(\phi)$. Then there exist $Y\in H(\psi)$ and $Z\in H(\chi)$ such
		that  $X= Y\cup Z$. By induction hypothesis $f^*(Y)\in H_V(\psi)$ and
		$f^*(Z)\in H_V(\chi)$. By the first part of the Claim about $f^*$ we
		see that $f^*(X)= f^*(Y)\cup f^*(Z)$ and thus $f^*(X)\in H_V(\phi)$.
		
		For the other direction, assume $f^*(X) \in H_V(\phi)$.  Then there
		exist $U\in H_V(\psi), V\in H_V(\chi)$ such that $f^*(X)=U\cup V$.
		Then by the second part of the Claim about $f^*$, there exists $Z, Y$
		such that $f^*(Z)=U, f^*(Y)= V$ and $Z\cup Y= X$. By the induction
		hypothesis  $Z\in H(\psi)$ and $Y\in H(\chi)$.  We conclude
		that $X\in H(\phi)$.
	\end{itemize}
	This concludes the proof of the lemma.
\end{proof}

We are now ready to prove Theorem \ref{thm:mainAx}.

\begin{proof}[Proof of Theorem \ref{thm:mainAx}] 
	The theorem is equivalent to the statement that for all $\Delta \cup \{
	\phi \} \subseteq \Fm_{\text{PT}^+}$ 
	\[
	\text{PT}^+:\Delta \vDash \phi \quad \text{iff} \quad
	\mathcal{H}_{\text{PV}} :  \Delta \vDash \phi.
	\]
	One direction follows directly from Theorem \ref{directThm} and the fact
	that $H_V\in\mathcal{H}_{\text{PV}} $. 

	The other direction is proved by contraposition. Assume
	$\mathcal{H}_{\text{PV}} :  \Delta \nvDash \phi$ and thus  $\text{LT} :
	\text{PVA}, \Delta \nvDash \phi$.  Then there is some Boolean algebra $B$ and 
	homomorphism $H: \Fm_{\text{LT}} \to  \P B$ such that $ H:
	\text{PVA},\Delta \nvDash \phi.$ In other words, $H\in
	\mathcal{H}_{\text{PV}}$ and there exists an element $ b\in B $ such that
	\[
	b \in \bigcap_{\delta\in \Delta} H(\delta), \text{ but } b \notin H(\phi).
	\]  
	By Theorem \ref{thm:completeBA}, we can then find an embedding $e :B \hookrightarrow \P{S}$ of $B$ into a complete atomic Boolean algebra of the form $(\P S, \emptyset, \cdot^C, \cup
	,\cap)$. Then by lemma \ref{embedding_lemma} and \ref{f_rep_in_Val}  we can find a homomorphism $H':\Fm_{\text{LT}}\to \P\P S$, and a mapping $f_{H'}: S\to 2^\mathbb{N}$  such that 
	\[
	f_{H'}^*(e(b)) \in \bigcap_{\delta\in\Delta} H_V(\delta), \text{ but }
	f_{H'}^*(e(b))\notin H_V(\phi).
	\]
	Thus, $H_V :\Delta \nvDash \phi$ and by Theorem \ref{directThm}: 
	\[
	\text{PT}^+:\Delta \nvDash \phi.
	\] 
	This finalises the proof of Theorem	\ref{thm:mainAx}.
\end{proof}

Observe that to evaluate $\text{PT}^+$ it is sufficient to consider the
valuational algebra $\P 2^\mathbb{N}$ i.e. for all  $\Delta, \{\phi\}
\subseteq\Fm_{\text{PT}^+}$ we have 
\[
\text{PT}^+ : \Delta \vDash \phi \quad\text{iff}\quad \P 2^\mathbb{N}:
\text{PVA},\Delta \vDash \phi.
\]
In this sense $\P 2^\mathbb{N}$ can be seen as canonical for $\text{PT}^+$. It
is however \textit{not} canonical for the logic in LT axiomatized by PVA,
since the canonicity only holds when the formulas are restricted to the
language $\Fm_{\text{PT}^+}$.

