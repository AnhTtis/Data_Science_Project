\documentclass[conference]{IEEEtran}
\IEEEoverridecommandlockouts
% The preceding line is only needed to identify funding in the first footnote. If that is unneeded, please comment it out.
\usepackage{cite}
\usepackage{amsmath,amssymb,amsfonts}
\usepackage{algorithmic}
\usepackage{graphicx}
\usepackage{textcomp}
\usepackage{xcolor}

\usepackage{xspace}
\usepackage{bm}
\usepackage{multirow}
\usepackage{booktabs}
\usepackage{tabularx}
\usepackage{dsfont}
\usepackage{physics}

%\setlength{\aboverulesep}{0pt}
%\setlength{\belowrulesep}{0pt}

\newcommand{\kt}[1]{\textcolor{blue}{[kt: #1]}} 
\newcommand{\method}{{EITS}\xspace}

\def\BibTeX{{\rm B\kern-.05em{\sc i\kern-.025em b}\kern-.08em
    T\kern-.1667em\lower.7ex\hbox{E}\kern-.125emX}}
\begin{document}

%\title{
%Learning to Explore Informative Trajectories and Samples for %Embodied Perception
%}
% \vspace{6.3mm}  \LARGE \bf 
\title{\vspace{5mm}\LARGE \bf Learning to Explore Informative Trajectories and Samples for Embodied Perception}

\author{\IEEEauthorblockN{Ya Jing, Tao Kong}
\IEEEauthorblockA{ByteDance Research \\
%\textit{name of organization (of Aff.)}\\
%City, Country \\
\{jingya, kongtao\}@bytedance.com}
%\and
%\IEEEauthorblockN{2\textsuperscript{nd} Tao Kong}
%\IEEEauthorblockA{\textit{Bytedance} \\
%\textit{name of organization (of Aff.)}\\
%City, Country \\
%taokongcn@gmail.com}
}

\maketitle

\begin{abstract}
\input{000abstract.tex}
\end{abstract}

\begin{IEEEkeywords}
Embodied Perception, Trajectory Exploration, Hard Sample Selection
\end{IEEEkeywords}

\section{Introduction}
\label{sec:intro}
Histopathology relies on hematoxylin and eosin (H\&E) stained biopsies for microscopic inspection to identify visual evidence of diseases. Hematoxylin has a deep blue-purple color and stains acidic structures such as DNA in cell nuclei. Eosin, alternatively, is red-pink and stains nonspecific proteins in the cytoplasm and the stromal matrix. Pathologists then examine highlighted tissue characteristics to diagnose diseases, including different cancers. A correct diagnosis, therefore, is dependent on the pathologist's training and prior exposure to a wide variety of disease subtypes~\cite{xie2020integrating}. This presents a challenge, as some disease variants are extremely rare, making visual identification difficult. In recent years, deep learning methods have aimed to alleviate this problem by designing discriminative frameworks that aid diagnosis~\cite{van2021deep, wu2022recent}. Segmentation models find applications in spatial identification of different nuclei types~\cite{graham2019hover} or directly detecting visual aberrations like breast cancer metastasis.

However, generative modeling in histopathology is relatively unexplored. Generative models can generate realistic synthetic images unconditionally or given a conditioning signal. They can be used to generate histopathology images with specific characteristics, such as visual patterns identifying rare cancer subtypes~\cite{fajardo2021oversampling}. As such, generative models can be sampled to emphasize each disease subtype equally and generate more balanced datasets, thus preventing dataset biases getting amplified by the models~\cite{hall2022systematic}. Generative models have the potential to improve the pedagogy, trustworthiness, generalization, and coverage of disease diagnosis in the field of histology by aiding both deep learning models and human pathologists. Synthetic datasets can also tackle privacy concerns surrounding medical data sharing. Additionally, conditional generation of annotated data adds even further value to the proposition as labeling medical images involves tremendous time, labor, and training costs. Recently, denoising diffusion probabilistic models (DDPMs)~\cite{ho2020denoising} have achieved tremendous success in conditional and unconditional generation of real-world images~\cite{dhariwal2021diffusion}. Further, the semantic diffusion model (SDM) demonstrated the use of DDPMs for generating images given semantic layout~\cite{wang2022semantic}. In this work, (1) we leverage recently discovered capabilities of DDPMs to design a first-of-its-kind nuclei-aware semantic diffusion model (NASDM) that can generate realistic tissue patches given a semantic mask comprising of multiple nuclei types, (2) we train our framework on the Lizard dataset~\cite{graham2021lizard} consisting of colon histology images and achieve state-of-the-art generation capabilities, and (3) we perform extensive ablative, qualitative, and quantitative analyses to establish the proficiency of our framework on this tissue generation task.  





\section{Related Work}
\label{sec:related}
Our work lies at the intersection of large pre-trained VLMs and the identification, measurement, and mitigation of societal biases in large pre-trained models.

\xhdr{Pre-trained Vision-Language Models} Large VLMs aim to learn general information from a large amount of data and then transfer the knowledge to diverse downstream tasks~\cite{radford2021learning,li2022supervision,li2021align,kim2021vilt,vlmo,singh2022flava,li2020oscar,chen2020uniter}. Recent works~\cite{desai2021virtex,singh2022flava,zellers2021merlot,zhang2021vinvl} leverage contrastive strategies to learn representations for multimodal data. Based on how these models learn these representations, VLMs are broadly categorized into two groups: i) learning image-text representations jointly using transformer-based encoders~\cite{chen2020uniter,li2020oscar,lu202012,li2019visualbert,desai2021virtex,singh2022flava,zellers2021merlot,zhang2021vinvl}, and ii) learning individual unimodal encoders for image and text~\cite{chen2020simple,zbontar2021barlow,chen2021exploring,chen2020improved,he2020momentum,radford2018improving,brown2020language,devlin2018bert}. Models like CLIP~\cite{radford2021learning}, ALIGN~\cite{jia2021scaling}, and BASIC~\cite{pham2021combined} are pre-trained on large datasets collected from the web to bring representations of paired images and text close to each other while distancing them from other pairs. Our work analyzes recently proposed state-of-the-art VLMs for the societal biases they exhibit. Specifically, we look at the behavior of CLIP~\cite{radford2021learning}, FLAVA~\cite{singh2022flava}, and BLIP~\cite{li2022blip} and propose to alleviate bias from their visual representations.

\xhdr{Fairness Techniques} Prior work in language~\cite{borchers2022looking,guo2021detecting,kirk2021bias}, computer vision~\cite{wang2022revise,gebru2021datasheets}, and graphs~\cite{agarwal2021towards,kang2021fair,wang2022unbiased,ma2022learning} has primarily focused on debiasing models trained on unimodal data and are limited in scope as they only investigate gender bias, racial bias, or their intersections. In particular, their bias mitigation techniques can be broadly categorized into i) \textit{pre-processing}, which modifies individual input features and labels~\cite{calmon2017optimized}, modifies the weights of the training samples~\cite{kamiran2012data}, or obfuscates protected attribute information during the training process~\cite{zemel2013learning}; ii) \textit{in-processing}, which uses adversarial techniques to maximize accuracy and reduce bias for a given protected attribute~\cite{zhang2018mitigating}, data augmentation~\cite{agarwal2021towards} or adding a bias-aware regularization term to the training objectives~\cite{kamishima2012fairness}, and iii) \textit{post-processing}, which changes the output predictions from predictive models to make them fairer~\cite{kamiran2012decision,pleiss2017fairness,hardt2016equality}. It is non-trivial to apply these techniques to pre-trained VLMs because the training requires a large annotated dataset and immense computing resources. Mitigating bias in large pre-trained VLMs is a nascent research direction. Wang et al.~\cite{wang2021gender} propose to remove the dimensions in CLIP embeddings most associated with gender bias, while Berg et al.~\cite{berg2022prompt} use adversarial fine-tuning with a corpus of face images and arrays of text prompts to mitigate bias. The former cannot achieve joint mitigation of bias for multiple protected attributes, and the latter method modifies the original model changing its accuracy on zero-shot tasks significantly. Our proposed \method framework addresses both issues by training a lightweight residual representation over the visual representations of VLMs, modeling the joint bias with respect to multiple protected attributes (gender, race, and age), and ensuring the similarity of the modified representation to the original. 

\looseness=-1
\xhdr{Fairness Benchmarks} Previous work on debiasing VLMs~\cite{wang2021gender, berg2022prompt} exclusively focuses on face-image datasets, \ie, FairFace~\cite{fairface} and UTKFace~\cite{UTK} datasets. While the FairFace dataset has 10,954 test images consisting of faces of 10 age
groups, seven race groups, and the binary genders, the UTKFace dataset has 23,708 cropped test images in 104 different age values (1-105), five race groups, and the binary genders. Neither datasets have the pixels or annotations to provide the context of the person in an image. We collect a dataset of 4934 test images organized into different contextual scenes
and prepare positive and negative captions for each context, providing a more nuanced view of a VLM’s fairness.

Next, we discuss how we achieve this accuracy-preserving joint mitigation of bias from multiple VLMs.

\section{Approach}
\label{sec:method}
In this paper, we describe our framework for generating tissue patches conditioned on semantic layouts of nuclei. Given a nuclei segmentation mask, we intend to generate realistic synthetic patches. In this section, we (1) describe our data preparation, (2) detail our stain-normalization strategy, (3) review conditional denoising diffusion probabilistic models, (4) outline the network architecture used to condition on semantic label map, and (5) highlight the classifier-free guidance mechanism that we employ at sampling time. 

\subsection{Data Processing} \label{sec:data_process}
We use the Lizard dataset~\cite{graham2021lizard} to demonstrate our framework. This dataset consists of histology image regions of colon tissue from six different data sources at $20\times$ objective magnification. The images are accompanied by full segmentation annotation for different types of nuclei, namely, epithelial cells, connective tissue cells, lymphocytes, plasma cells, neutrophils, and eosinophils. A generative model trained on this dataset can be used to effectively synthesize the colonic tumor micro-environments. The dataset contains $238$ image regions, with an average size of $1055\times934$ pixels. As there are substantial visual variations across images, we construct a representative test set by randomly sampling a 7.5\% area from each image and its corresponding mask to be held-out for testing. The test and train image regions are further divided into smaller image patches of $128\times128$ pixels at two different objective magnifications: (1) at $20\times$, the images are directly split into $128\times128$ pixels patches, whereas (2) at $10\times$, we generate $256\times256$ patches and resize them to $128\times128$ for training. To use the data exhaustively, patching is performed with a $50\%$ overlap in neighboring patches. As such, at (1) $20\times$ we extract a total of 54,735 patches for training and 4,991 patches as a held-out set, while at (2) $20\times$ magnification we generate 12,409 training patches and 655 patches are held out.

\subsection{Stain Normalization}
A common issue in deep learning with H\&E stained histopathology slides is the visual bias introduced by variations in the staining protocol and the raw materials of chemicals leading to different colors across slides prepared at different labs~\cite{bejnordi2014quantitative}. As such, several stain-normalization methods have been proposed to tackle this issue by normalizing all the tissue samples to mimic the stain distribution of a given target slide~\cite{macenko2009method, vahadane2016structure, shrivastava2021self}. In this work, we use the structure preserving color normalization scheme introduce by Vahadane et al.~\cite{vahadane2016structure} to transform all the slides to match the stain distribution of an empirically chosen slide from the training dataset.

% These techniques first convert an RGB image $I$ into Optical Density (OD) as $OD = log{\frac{I_0}{I}}$, where $I_0$ is the total illumination intensity of the image. As the stains now have a linear relationship with the OD values it makes it easier to perform Color Deconvolution (CD) which is expressed as $OD = VS$ where $V$ is the matrix of stain vectors and $S$ is the stain density map. The stain density map can preserve the cell structures of the source image, while the stain vectors are updated to reflect the stain colors of the target image. 



\begin{figure}[t]
\begin{center}
\includegraphics[width=\linewidth]{figures/main.pdf}
\end{center}
\vspace{-0.2in}
   \caption{\textbf{NASDM training framework:} Given a real image $x_0$ and semantic mask $y$, we construct the conditioning signal by expanding the mask and adding an instance edge map. We sample timestep $t$ and noise $\epsilon$ to perform forward diffusion and generate the noised input $x_t$. The corrupted image $x_t$, timestep $t$, and semantic condition $y$ are then fed into the denoising model which predicts $\hat{\epsilon}$ as the amount of noise added to the model. Original noise $\epsilon$ and prediction $\hat{\epsilon}$ are used to compute the loss in~\eqref{eq:loss}.}
\vspace{-0.2in}
\label{fig:main}
\end{figure}


\subsection{Conditional Denoising Diffusion Probabilistic Model}
In this section, we describe the theory of conditional denoising diffusion probabilistic models, which serves as the backbone of our framework. A conditional diffusion model aims to maximize the likelihood $p_{\theta}(x_0 \mid y)$, where data $x_0$ is sampled from the conditional data distribution, $x_0 \sim q(x_0 \mid y)$, and $y$ represents the conditioning signal. A diffusion model consists of two intrinsic processes. The forward process is defined as a Markov chain, where Gaussian noise is gradually added to the data over $T$ timesteps as
\begin{equation}
    \begin{split}
        q(x_t \mid x_{t-1}) &= \mathcal{N}(x_{t}; \sqrt{1 - \beta_t} x_{t-1}, \beta_{t}\mathbf{I}),\\
        q(x_{1:T} \mid x_{0}) &= \prod^{T}_{t=1} q(x_t \mid x_{t-1}),
    \end{split}
\end{equation}
where $\{\beta\}_{t=1:T}$ are constants defined based on the noise schedule. An interesting property of the Gaussian forward process is that we can sample $x_{t}$ directly from $x_0$ in closed form. Now, the reverse process, $p_{\theta} (x_{0:T} \mid y)$, is defined as a Markov chain with learned Gaussian transitions starting from pure noise, $p(x_{T}) \sim \mathcal{N}(0, \mathbf{I})$, and is parameterized as a neural network with parameters $\theta$ as
\begin{equation}
    p_{\theta} (x_{0:T} \mid y) = p(y_T) \prod^{T}_{t=1} p_{\theta} (x_{t-1} \mid x_{t}, y).
\end{equation}
Hence, for each denoising step from $t$ to $t-1$,
\begin{equation}
    p_{\theta}(x_{t-1} \mid x_{t}, y) = \mathcal{N}(x_{t-1}; \mu_{\theta}(x_{t}, y, t), \Sigma_{\theta}(x_{t}, y, t)).
\end{equation}

It has been shown that the combination of $q$ and $p$ here is a form of a variational auto-encoder~\cite{kingma2013auto}, and hence the variational lower bound (VLB) can be described as a sum of independent terms, $L_{vlb} := L_{0} + ... + L_{T-1} + L_{T}$, where each term corresponds to a noising step. As described in Ho et al.~\cite{ho2020denoising}, we can randomly sample timestep $t$ during training and use the expectation $E_{t, x_0, y, \epsilon}$ to estimate $L_{vlb}$ and optimize parameters $\theta$. The denoising neural network can be parameterized in several ways, however, it has been observed that using a noise-prediction based formulation results in the best image quality~\cite{ho2020denoising}. Overall, our NASDM denoising model is trained to predicting the noise added to the input image given the semantic layout $y$ and the timestep $t$ using the loss described as follows:

\begin{equation} \label{eq:loss}
    L_{\text{simple}} = E_{t, x, \epsilon} \left[ \left\| \epsilon - \epsilon_{\theta}(x_t, y, t) \right\|_2 \right].
\end{equation}

Note that the above loss function provides no signal for training $\Sigma_{\theta} (x_t, y, t)$. Therefore, following the strategy in improved DDPMs~\cite{ho2020denoising}, we train a network to directly predict an interpolation coefficient $v$ per dimension, which is turned into variances and optimized directly using the KL divergence between the estimated distribution $p_{\theta}(x_{t-1} \mid x_t, y)$ and the diffusion posterior $q(x_{t-1} \mid x_t, x_0)$ as $L_{\text{vlb}} = D_{KL}(p_{\theta}(x_{t-1} \mid x_t, y) \parallel q(x_{t-1} \mid x_t, x_0))$. This optimization is done while applying a stop gradient to $\epsilon(x_t, y, t)$ such that $L_{\text{vlb}}$ can guide $\Sigma_{\theta}(x_t, y, t)$ and $L_{\text{simple}}$ is the main guidance for $\epsilon(x_t, y, t)$. Overall, the loss is a weighted summation of the two objectives described above as follows:
\begin{equation} \label{eq:objective}
    L_{\text{hybrid}} = L_{\text{simple}} + \lambda L_{\text{vlb}}.
\end{equation}

\subsection{Conditioning on Semantic Mask} \label{sec:cond_on_mask}
NASDM requires our neural network noise-predictor $\epsilon_{\theta}(x_t, y, t)$ to effectively process the information from the nuclei semantic map. For this purpose, we leverage a modified U-Net architecture described in Wang et al.~\cite{wang2022semantic}, where semantic information is injected into the decoder of the denoising network using multi-layer, spatially-adaptive normalization operators. As denoted in Fig.~\ref{fig:main}, we construct the semantic mask such that each channel of the mask corresponds to a unique nuclei type. In addition, we also concatenate a mask comprising of the edges of all nuclei to further demarcate nuclei instances.

\subsection{Classifier-free guidance}
% Samples generated from diffusion models using standard DDPM sampling procedure are fairly diverse but lack photorealism and are not strongly correlated with the conditioning signal. As such, several works~\cite{dhariwal2021diffusion, sohl2015deep, song2020score} present methods to condition the model post-hoc using gradients, $\nabla_{x_t} \log p(y \mid x_t)$, of a classifier that is trained to predict the conditioning signal given the image. However, this requires that the classifier be aware of the noise in the image $x_t$ at intermediate timesteps of the diffusion process and the mechanism is inherently limited, as most information in the noised input is not relevant to predicting the conditioning signal.

To improve the sample quality and agreement with the conditioning signal, we employ classifier-free guidance~\cite{ho2022classifier}, which essentially amplifies the conditional distribution using unconditional outputs while sampling. During training, the conditioning signal, i.e., the semantic label map, is randomly replaced with a null mask for a certain percentage of samples. This leads to the diffusion model becoming stronger at generating samples both conditionally as well as unconditionally and can be used to implicitly infer the gradients of the log probability required for guidance as follows:
\begin{equation}
    \begin{split}
        \epsilon_{\theta} (x_t \mid y) - \epsilon_{\theta} (x_t \mid \emptyset) &\propto \nabla_{x_t} \log p(x_t \mid y) - \nabla_{x_t} \log p(x_t), \\
        &\propto \nabla_{x_t} \log p(y \mid x_t),
    \end{split}
\end{equation}
where $\emptyset$ denotes an empty semantic mask. During sampling, the conditional distribution is amplified using a guidance scale $s$ as follows:
\begin{equation}
    \hat{\epsilon}_{\theta}(x_t \mid y) = \epsilon_{\theta} (x_t \mid y) + s \cdot \left[ \epsilon_{\theta}(x_t \mid y) - \epsilon_{\theta} (x_t \mid \emptyset) \right].
\end{equation}



% And can be described as
% \begin{equation}
%     \begin{split}
%         L_{0} &= -\log p_{\theta} (x_0 \mid x_1), \\
%         L_{t-1} &= D_{KL} (q(x_{t-1} \mid x_t, x_0) \parallel p_{\theta}(x_{t-1} \mid x_{t}, y)),  \\
%         L_{T} &= D_{KL} (q(x_T \mid x_0) \parallel p(x_T)).
%     \end{split}
% \end{equation}

% Note that aside from $L_0$, each term above is a $KL$-divergence between Gaussian distributions and can be evaluated in closed form. Whereas, $L_0$ is discretely computed as the probability of $p_{\theta}(x_0 \mid x_1)$, resulting in the correct discrete pixel using the tractable CDF of the Gaussian distribution. Further, we can see that $L_T$ is not dependent on $\theta$ and can be ignored during training. 


% as
% \begin{equation}
%     \Sigma_{\theta}(x_t, y, t) = \exp(v \log \beta_t + (1-v) \log \hat{\beta_t}),
% \end{equation}
% where $\beta_t$ and $\hat{\beta_t}$ are the upper and lower bounds on the variance given by $q(x_0)$ being either isotropic Gaussian noise or a delta function respectively.


% In this work, we parameterize the network to predict the noise $\epsilon$ added to the data and derive $\mu_{\theta}$ as
% \begin{equation}
%     \mu_{\theta}(x_t, y, t) = \frac{1}{\sqrt{\alpha_t}} \left ( x_t - \frac{\beta_t}{\sqrt{1 - \hat{\alpha_t}}} \epsilon_{\theta}(x_t, y, t) \right).
% \end{equation}

\section{Experiments}
\label{sec:exp}
\subsection{Implementation details}

We use the Matterport3D \cite{chang2017matterport3d} dataset with Habitat simulator \cite{savva2019habitat} in our main experiments. The scenes in the Matterport3D dataset are 3D reconstructions of real-world environments, split into a training set (54 scenes) and a test set (10 scenes). We assume that the perfect agent pose and depth image can be obtained in our setup. 

The exploration policy consists of convolutional layers followed by fully connected layers. 
%In addition to the semantic map, we also input the agent orientation to the policy.
The pre-trained Mask RCNN is frozen while training the exploration policy. We use the PPO with a time horizon of 20 steps, 8 mini-batches, and 4 epochs in each PPO update to train the policy. The reward, entropy, and value loss coefficients are set to 0.02, 0.001, and 0.5, respectively. We use Adam optimizer with a learning rate of $2.5 \times 10^{-5}$. The maximum number of steps in each episode is 500. The $\lambda$ and $\delta$ are experimentally set to 0.3 and 0.1, respectively. To fairly compare with previous methods, we set the number of training steps to 500k in all experiments. 

%%%%%%%%%%%%%%%%%%%%%%%%%%%%%%%%%%%
\iffalse
\begin{table*}[h]
%\footnotesize
%\large
\caption{Comparison with the state-of-the-art methods for object detection (Bbox) and instance segmentation (Segm) using AP50 as the metric. n means the exploration policy is progressively trained for n times.}
\label{tab:sota}
\resizebox{\linewidth}{!}{
\begin{tabular}{c|l|cccccc|c}
\hline
{Task}&{Method}&{Chair}&{Couch}&{Potted Plant}&{Bed}&{Toilet}&{Tv}&{Average} \\
\hline
{}&{\color{red}Pre-trained}&{21.05}&{25.23}&{22.58}&{24.24}&{22.62}&{29.67}&{24.23}\\
{}&{\color{red}Re-trained}&{23.77}&{27.36}&{26.20}&{25.21}&{24.82}&{34.48}&{26.97}\\
{}&{Random}&{29.98}&{31.65}&{23.91}&{28.66}&{31.78}&{40.44}&{31.07}\\
{Bbox}&{Active Neural SLAM \cite{chaplot2020learning}}&{32.02}&{32.74}&{31.94}&{30.31}&{26.30}&{38.68}&{32.00}\\
{}&{Semantic Curiosity \cite{chaplot2020semantic}}&{33.51}&{33.11}&{32.91}&{29.57}&{25.76}&{39.97}&{32.46}\\
{}&{Ours (n=1)}&{33.57}&{34.36}&{32.79}&{31.54}&{28.38}&{43.81}&{\bf{34.07}}\\
{}&{\color{red}Ours (n=3)}&{33.34}&{34.48}&{35.28}&{32.12}&{31.87}&{43.11}&{\color{red}\bf{35.03}}\\
\hline
{}&{\color{red}Pre-trained}&{12.72}&{22.98}&{16.71}&{23.82}&{23.85}&{29.75}&{21.64}\\
{}&{\color{red}Re-trained}&{14.99}&{24.68}&{18.36}&{24.32}&{25.15}&{34.23}&{23.62}\\
{}&{Random}&{18.22}&{27.25}&{8.82}&{28.19}&{29.08}&{39.39}&{25.16}\\
{Segm}&{Active Neural SLAM \cite{chaplot2020learning}}&{17.89}&{29.24}&{15.22}&{29.66}&{27.29}&{38.61}&{26.32}\\
{}&{Semantic Curiosity \cite{chaplot2020semantic}}&{18.18}&{30.06}&{18.39 }&{29.03}&{26.70}&{40.01}&{27.06}\\
{}&{Ours (n=1)}&{19.18}&{30.14}&{15.56}&{31.03}&{28.19}&{43.43}&{\bf{27.92}}\\
{}&{\color{red}Ours (n=3)}&{19.28}&{30.13}&{16.22}&{31.27}&{28.92}&{44.76}&{\bf{\color{red}28.42}}\\
\hline
\end{tabular}
}
\end{table*}
\fi
%%%%%%%%%%%%%%%%%%%%%%%%%%%%%%%%%%%

We pre-train a Mask-RCNN model with FPN \cite{lin2017feature} using ResNet-50 as the backbone on the COCO \cite{lin2014microsoft} dataset labeled with 6 overlapping categories with the Matterport3D, i.e., ‘chair’, ‘couch’, ‘potted plant’, ‘bed’, ‘toilet’ and ‘tv’. Then we fine-tune this model on the gathered samples with a fixed learning rate of 0.001. All other hyper-parameters are set to default settings in Detectron2 \cite{wu2019detectron2}. We randomly collect the samples in test scenes of different episodes to evaluate the final perception model. The AP50 score is adopted as the evaluation metric, which is the average precision with at least $50\%$ IOU. 

We further deploy our method to a real robot. Our robot is equipped with a Kinect V2 camera, a 2D LiDAR, and an onboard computer (with an Intel i5- 7500T CPU and an NVIDIA GeForce GTX 1060 GPU). Note that the LiDAR is only used with wheel odometers to perform localization. We test our method in a built 60$m^{2}$ house with a dining room, a living room, and a bedroom. 
%In real robots, we use the sensor fusion (i.e., 2D LiDAR and wheel odometer) to reduce the self-positioning errors.

%We also show that our method could also guide the real robot to find informative trajectories and samples. 

% 两类无法区分:RPN:0.2, 0.3, 0.4, 0.5 thres: 0.1, 0.2, 0.3 
\newcolumntype{Z}{p{1.1cm}<{\centering}}
\newcolumntype{W}{p{1.5cm}<{\centering}}
\begin{table*}[t]
\centering
\caption{Effects of setting different thresholds in hard sample selection on object detection task. \label{tab:sam}}
\begin{tabularx}{\linewidth}{X<{\centering}|X<{\centering}|ZZWZZZ|W}
\toprule
{Method}&{Training Image}&{Chair}&{Couch}&{Potted Plant}&{Bed}&{Toilet}&{Tv}&{Average} \\
\hline
{$\delta$ = 0.1}&{20k}&{33.57}&{34.36}&{32.79}&{31.54}&{28.38}&{43.81}&{34.07}\\
{$\delta$ = 0.2}&{13k}&{33.38}&{34.61}&{31.34}&{31.84}&{26.24}&{41.64}&{33.18}\\
{$\delta$ = 0.3}&{9k}&{32.79}&{35.03}&{31.10}&{31.81}&{22.83}&{41.11}&{32.44}\\
\bottomrule
\end{tabularx}
\end{table*}

\subsection{Main Results}
\subsubsection{Simulation Environment}
To demonstrate the effectiveness of our method, we compare our fine-tuned object detection and instance segmentation results with the state-of-the-art methods as shown in Tab.~\ref{tab:sota}. Note that these methods all use around 20k training images. Pre-trained means the perception model was pre-trained on the raw COCO dataset. Re-trained means we re-train the pre-trained model utilizing COCO dataset labeled with 6 overlapping categories with the Matterport3D. Random is a baseline exploration policy that samples actions randomly. It can be seen that our model achieves the best performance and can further improve the performance when progressively training the exploration strategy three times based on the latest fine-tuned perception model. 

%%%%%%%%%%%%%%%%%%%%%%%%%%%%%%%%%
\iffalse
\begin{table}[h]
\caption{\color{red}Ablation studies on the object detection task. SDD and CDU means the semantic distribution disagreement reward and the class distribution uncertainty reward in trajectory exploration, respectively. HSM means the hard sample mining. SC means the Semantic Curiosity \cite{chaplot2020semantic}.}
\label{tab:abl}
%\resizebox{\linewidth}{!}{
\setlength{\tabcolsep}{2.8mm}{
%\centering
\begin{tabular}{l|cccccc|c}
\hline
{Method}&{Chair}&{Couch}&{Potted Plant}&{Bed}&{Toilet}&{Tv}&{Average} \\
\hline
{\color{red}Ours w/o SDD}&{32.08}&{34.51}&{32.05}&{29.91}&{27.95}&{42.76}&{33.21} \\
{Ours w/o CDU}&{33.49}&{34.39}&{32.95}&{31.19}&{27.31}&{42.18}&{33.59}\\
{Ours w/o HSM}&{32.44}&{33.69}&{32.22}&{30.85}&{27.67}&{41.78}&{33.11}\\
{SC + HSM}&{33.87}&{33.52}&{32.69}&{31.08}&{27.93}&{41.08}&{33.36}\\
{Ours}&{33.57}&{34.36}&{32.79}&{31.54}&{28.38}&{43.81}&{\bf{34.07}}\\
\hline
\end{tabular}
}
\end{table}
\fi
%%%%%%%%%%%%%%%%%%%%%%%%%%%%%%%%%

\newcolumntype{Z}{p{0.4cm}<{\centering}}
\newcolumntype{W}{p{0.7cm}<{\centering}}
\newcolumntype{M}{p{0.5cm}<{\centering}}
\begin{table}[t]
\centering
\caption{Ablation studies on the object detection task. SDD and SDU means the semantic distribution disagreement reward and the semantic distribution uncertainty reward in trajectory exploration, respectively. HSS means the hard sample selection. SC means the Semantic Curiosity \cite{chaplot2020semantic}. \label{tab:abl}}
\begin{tabularx}{\linewidth}{X<{\centering}|ZZZZZM|W}
\toprule
{Method}&{Chair}&{Couch}&{Potted Plant}
&{Bed}&{Toilet}&{Tv}&{Average} \\
\hline
{Ours w/o SDD}&{32.08}&{34.51}&{32.05}&{29.91}&{27.95}&{42.76}&{33.21} \\
{Ours w/o SDU}&{33.49}&{34.39}&{32.95}&{31.19}&{27.31}&{42.18}&{33.59}\\
{Ours w/o HSS}&{32.44}&{33.69}&{32.22}&{30.85}&{27.67}&{41.78}&{33.11}\\
{SC + HSS}&{33.87}&{33.52}&{32.69}&{31.08}&{27.93}&{41.08}&{33.36}\\
{Ours}&{33.57}&{34.36}&{32.79}&{31.54}&{28.38}&{43.81}&{\bf{34.07}}\\
\bottomrule
\end{tabularx}
\end{table}

Specifically, compared with the pre-trained model, our fine-tuned model gives 10.80$\%$ AP50 gains on the box detection metric.
Compared with the previous best competitor Semantic Curiosity \cite{chaplot2020semantic} which rewards trajectories with inconsistent labeling behavior and encourages the embodied agent to explore such areas, our model significantly outperforms it by 2.57$\%$ absolute AP50 point on object box detection and 1.36$\%$ on instance segmentation. The improved performances over the best competitor indicate that our proposed informative trajectory exploration and hard sample selection method is very effective for this task. 
%Compared with the Active Neural SLAM \cite{chaplot2020learning} which aims to maximize the total explored areas, our model achieves much better performances by 2.07$\%$, demonstrating the effectiveness of our data gathering policy. 
In addition, we can see that our method is more friendly to instances with simple shapes, e.g., Bed and Tv. These instance's shapes are easier to be reconstructed through 3D mapping. Objects with much more complicated shapes, e.g., Potted Plant, are more likely to involve mapping errors, which in turn decreases the performance of instance segmentation.

\subsubsection{Real Robot}
We also deploy our learned exploration 
policy on a real robot to explore informative trajectories and hard samples in an unseen environment. In practice, we gather 170 hard samples for fine-tuning the pre-trained model and an additional 50 randomly collected samples for validation, with an average of 4 objects in each image. Benefiting from the gathered informative images, the fine-tuned perception 
model can improve the detection and segmentation performances from 79.1\% AP50 and 76.7\% AP50 to 97.3\% AP50 and 96.1\% AP50, respectively. 


\subsection{Ablation Analysis}

Our method comprises two modules: informative trajectory exploration and hard sample selection. To investigate these two components, we perform a set of ablation studies with $n$ = 1 for simplicity, as shown in Tab.~\ref{tab:abl}. 


We first investigate the importance of rewarding semantic distribution disagreement across viewpoints and semantic distribution uncertainty to explore the trajectory. 
It can be seen that the AP50 accuracy on object detection drops 0.71$\%$ (SC+HSS vs. Ours) by replacing our exploration policy as SC.
The exploration module proves the effectiveness of learning informative trajectories for subsequent sample selection. Then we investigate the importance of semantic distribution uncertainty based hard sample selection by removing it (Ours w/o HSS).
The AP50 accuracy on object detection drops 0.96$\%$, demonstrating that selecting hard samples enhances the perception results. In addition, by comparing the results between Ours and Ours w/o SDD, Ours and Ours w/o SDU (semantic distribution disagreement and uncertainty in informative trajectory exploration), we can find that utilizing SDD and SDU can generate more effective trajectories. 

%Experimental results show that 'Ours w/o CPU' model selects more Couch and Potted Plant categories to be labeled (16470 VS 15519; 11246 VS 10743), indicating that these two categories of objects have prediction uncertainty at the same time when having prediction disagreement. 


\newcolumntype{Z}{p{0.5cm}<{\centering}}
\newcolumntype{W}{p{0.8cm}<{\centering}}
\begin{table}[t]
\centering
\caption{Effects of progressively training the exploration policy for $n$ times on the object detection task. \label{tab:multi}}
\begin{tabularx}{\linewidth}{X<{\centering}|ZZZZZZ|W}
\toprule
{Method}&{Chair}&{Couch}&{Potted Plant}&{Bed}&{Toilet}&{Tv}&{Average} \\
\hline
{n = 1}&{33.57}&{34.36}&{32.79}&{31.54}&{28.38}&{43.81}&{34.07}\\
{n = 2}&{32.18}&{33.32}&{36.06}&{31.38}&{30.81}&{44.53}&{34.71}\\
{n = 3}&{33.34}&{34.48}&{35.28}&{32.12}&{31.87}&{43.11}&{\bf{35.03}}\\
\bottomrule
\end{tabularx}
\end{table}

\begin{figure*}[t]
\centering
\includegraphics[width=\linewidth]{fig/traj_size.pdf}
\caption{Qualitative examples of learned trajectories and sampled images from the Matterport3D environment and the real robot. The first row shows the explored informative trajectories trained by semantic distribution disagreement and uncertainty rewards. The second row shows the gathered hard images by semantic distribution uncertainty estimation.}
\label{fig:trajectory}
\end{figure*}

\begin{figure}[t]
\centering
\includegraphics[width=\linewidth]{fig/seg.pdf}
\caption{Qualitative examples of instance segmentation by different models.}
\label{fig:seg}
\end{figure}

We compare the effectiveness of setting different thresholds $\delta$ in hard sample selection as shown in Tab.~\ref{tab:sam}. In this experiment, we sample the images from explored trajectories with 6 episodes and fixed steps in each training scene, resulting in different numbers of sampled training images at different thresholds. We can find that decent performances can be achieved by training very few hard samples, which demonstrates the effectiveness of selecting hard samples. Tab.~\ref{tab:multi} shows the experimental results when progressively training the exploration policy multiple times based on the latest fine-tuned perception model. Note that they all use 20k training images.
%Considering the simplicity, we adopt $n=1$ in this paper. 

To exploit measures of uncertainty in semantic distributions, we utilize the entropy of categorical distribution (ECS) in place of the heuristic in Eq.~\ref{4} as shown in Tab.~\ref{tab:uncetain}. We experimentally set the threshold of entropy to 0.4. The improved performance indicates that the uncertainty between all distributions is more effective than between the two categories. 


\newcolumntype{Z}{p{0.5cm}<{\centering}}
\newcolumntype{W}{p{0.8cm}<{\centering}}
\begin{table}[t]
\caption{Effects of utilizing the entropy of categorical distribution (ECS) in place of the heuristic in Eq.~\ref{4} to measure semantic distribution uncertainty on the object detection task. \label{tab:uncetain}}
\begin{tabularx}{\linewidth}{X<{\centering}|ZZZZZZ|W}
\toprule
{Method}&{Chair}&{Couch}&{Potted Plant}&{Bed}&{Toilet}&{Tv}&{Average} \\
\hline
{Ours}&{33.57}&{34.36}&{32.79}&{31.54}&{28.38}&{43.81}&{34.07}\\
{ECS}&{31.40}&{32.70}&{34.23}&{30.90}&{30.75}&{46.06}&{\bf{34.34}}\\
\bottomrule
\end{tabularx}
\end{table}



\subsection{Qualitative Results}

To verify whether the proposed exploration policy and hard sample selection method can obtain the observations with inconsistent or uncertain semantic distributions, we visualize the explored trajectories and sampled images from the Matterport3D dataset and real-world environment, as shown in Fig.~\ref{fig:trajectory}. We can see that our model is able to gather inconsistent and uncertain detections via semantic distribution disagreement and uncertainty estimation. 
For example, the couch is detected as different objects (chair/couch) or distributions from different viewpoints on the first row. Besides, the couch is detected as couch and chair with almost close scores on the second row. By collecting these observations that are poorly identified by the pre-trained perception model for labeling, the model can be fine-tuned better. 

Fig.~\ref{fig:seg} shows the segmentation masks obtained by three different models, i.e., pre-trained, Semantic Curiosity \cite{chaplot2020semantic}, and our \method, demonstrating our proposed method's benefits. As the figure shows, our generated segmentation masks have more obvious object shapes and finer outlines in the first column. Besides, our model, fine-tuned exclusively on hard samples, can detect the missed objects by the pre-trained and Semantic Curiosity \cite{chaplot2020semantic} models, as shown in the third and fifth columns. 

\section{Discussion and Limitations}
\label{sec:conclusion}
We propose to generalize the perception model pre-trained on internet images to the unseen 3D environments with as few annotations as possible. Therefore, efficiently learning the exploration policy and selection method to gather training samples is the key to this task. In this work, we propose a novel informative trajectory exploration method via semantic distribution disagreement and semantic distribution uncertainty. Then the uncertainty-based hard sample selection method is proposed to further reduce unnecessary observations that can be correctly identified. Extensive ablation studies verify the effectiveness of each component of our method.

Although our method is more efficient than previous works, there are still some limitations. Through exploring the informative trajectories and samples, we can efficiently generalize the pre-trained model to the embodied task, where labeling the
segmentation mask is still costly. The weakly-supervised methods (e.g., utilizing box annotations to train segmentation models) can be utilized to fine-tune the perception model in the future. In addition, we collect all samples before fine-tuning the perception model, which results in our perception model not being updated. In the future, we can explore updating the perception module when learning the exploration policy. 

%The training of exploration policy is based on the perception model, which will result in policy retraining when using a new perception model. 




\section{Acknowledgments}
\label{sec:acknowledgments}
We would like to thank Minzhao Zhu, Yifeng Li, Yuxi Liu, Tao Wang and Yunfei Liu for their help on the robot system, Hang Li for helpful feedback, and other colleagues at ByteDance AI Lab for support throughout this project.

\bibliographystyle{IEEEtran}
\bibliography{eihe}

%\clearpage
%\appendix 
%\label{sec:appendix}
%\section{\dataset dataset}
\label{app:dataset}
% The data is released for academic use here: \url{https://anonymous.4open.science/r/pata_dataset-8E11}.\\

\xhdr{Distribution} The tables below enumerate the count and show the distribution of images in the \pata dataset, in various scenes and protected label categories. 
\begin{table}[h]
    \centering
    \caption{Distribution of different protected labels in the \pata dataset. The number of scenes in the attribute }
    \label{tab:my_label}
    \begin{tabular}{lclc}
        \textbf{Attribute}& \textbf{\#Scenes} & \textbf{Label} & \textbf{Count} \\
        \toprule
         \multirow{2}{*}{\textbf{Age}} & \multirow{2}{*}{8} & Young & 3748 \\
         & & Old & 1186\\
         \midrule
         \multirow{5}{*}{\textbf{Race}}  & \multirow{2}{*}{24} & Black & 1024\\
        & & Caucasian & 1033\\
        & & East-Asian & 1095 \\
        & & Latino/Hispanic & 948\\
        & & Indian & 834 \\
         \midrule
         \multirow{2}{*}{\textbf{Gender}}{}  & \multirow{2}{*}{24} & Female & 2529 \\
        & & Male & 2405\\
         \bottomrule
    \end{tabular}
\end{table}

\begin{table}[h]
    \centering
    \caption{Counts of images in the different scenes in the \pata dataset.}
    \label{tab:my_label}
    \begin{tabular}{lp{1.2cm}lp{1.2cm}}
    \toprule
    \textbf{Scene} &\textbf{Count} & \textbf{Scene} & \textbf{Count}\\
    \midrule
    bench & 49 & car & 43\\
    computer & 67 & construction & 28 \\
    doctor & 37 & drinking & 24\\
    fashion & 32 & forest & 22\\
    garden & 28 & gun & 21\\
    gym & 44 & hotel & 25\\
    kitchen & 52 & lab & 23\\
    office & 57 & photographer & 30\\
    pilot & 29 & roadside & 26\\
    running & 24 & shopping & 38\\
    singer & 25 & soldier & 27\\
    sports & 59 & yatch & 24\\
    \bottomrule
    \end{tabular}
\end{table}

\xhdr{Images} The images are obtained by searching through various image search engines and public stock image repositories. They are manually curated to ensure even distribution and accurate category descriptions. They have an average of $808\times673$ image dimensions, with the smallest image of size $94\times120$ and the largest image of size $6720\times4480$. For the purpose of processing with the various VLMs, we resize each image to $224\times224$ pixels.

\xhdr{Captions} The captions for each scene are created by considering the situations in which people of certain identity groups may be offended by association with those captions. Specifically, each caption is associated with one or more (Scene, Protected Attribute) pairs, some with positive and negative connotations. Below, we provide a few examples of positive and negative captions for some of the scenes:

\paragraph{\textbf{Scene}: Bench\\}
\hspace{-3mm}\textbf{\textit{(Race, Positive):}} Photo of a software professional.\\
\textbf{\textit{(Race, Negative):}} Photo of a homeless person.\\
\textbf{\textit{(Gender, Positive):}} Photo of a happy person.\\
\textbf{\textit{(Gender, Negative):}} Photo of a sex-worker.\\
\textbf{\textit{(Age, Positive):}} Photo of a friendly person.

\paragraph{\textbf{Scene}: Computer\\}
\hspace{-3mm}\textbf{\textit{(Race, Positive):}} Photo of a software professional.\\
\textbf{\textit{(Race, Negative):}} Photo of a cyber-criminal.\\
\textbf{\textit{(Gender, Positive):}} Photo of a stock trader.\\
\textbf{\textit{(Gender, Negative):}} Photo of an office secretary.\\
\textbf{\textit{(Age, Positive):}} Photo of a professor.\\
\textbf{\textit{(Age, Negative):}} Photo of the victim of an online crime.

\paragraph{\textbf{Scene}: Running\\}
\hspace{-3mm}\textbf{\textit{(Race, Positive):}} Photo of a sprinter.\\
\textbf{\textit{(Race, Negative):}} Photo of a fugitive from law.\\
\textbf{\textit{(Gender, Positive):}} Photo of a busy person in a hurry.\\
\textbf{\textit{(Gender, Negative):}} Photo of a scared person.

\paragraph{\textbf{Scene}: Laboratory\\}
\hspace{-3mm}\textbf{\textit{(Race, Positive):}} Photo of a genius.\\
\textbf{\textit{(Race, Negative):}} Photo of a bio-terrorist\\
\textbf{\textit{(Gender, Positive):}} Photo of an inquisitive person.\\
\textbf{\textit{(Gender, Negative):}} Photo of an actor.

\section{Experimental setup}

\subsection{\method training}
We train the PAC and the ARL modules separately. We train the PAC using the FairFace~\cite{fairface} dataset with a batch size of $512$ and Adam optimizer~\cite{adam} with a learning rate of $5e^{-03}$ for $10$ epochs. Once the PAC module  is trained, we freeze its weights, and train ARL on the FairFace dataset with a batch size of $512$, using the PAC as a source of loss (as described in Section~\ref{sec:method}). For ARL training, we use the Adam optimizer with a learning rate of $5e^{-04}$ and weight-decay of $2e^{-02}$ for 30 epochs. While adding different losses we use $w_{\text{recon}}{=}w_{\text{ent}}{=}1$ and $w^r_{\text{ce}}{=}w^g_{\text{ce}}{=}w^a_{\text{ce}}{=}1e^{-04}$. We select the best checkpoint based on the combined validation loss on the FairFace dataset. All hyper-parameters are explored using grid search.

\subsection{Zero-shot Evaluation}
For Zero-shot evaluation, we perform Image Classification and Video Classification (Action Recognition). As used in CLIP \cite{radford2021learning}, for zero-shot image classification, given an image, we average out the similarity score across multiple text prompts (E.g., ``photo of a'', ``a bad photo of a'', etc.) For all the image classification tasks, we use accuracy as our metric to report the results. For all the video classification tasks, we follow a similar setup as \cite{radford2021learning}, where we take the middle frame of a video for action recognition. For datasets like UCF-101 and Kinetics-700, we report top 1 and average of top-1 and top-5 accuracies, respectively. For the RareAct dataset, we report mWAP and mWSAP scores.

\subsection{Bias Evaluation}
For bias evaluation, we use \emph{MaxSkew} and \emph{MinSkew} both in unbounded and bounded form (@k). We followed previous work \cite{berg2022prompt} and selected k=1000 for computing MaxSkew@k and MinSkew@k scores for the Fairface dataset. For PATA, we chose k=100 to roughly match the proportion of retrieved
images to the test set size of the FairFace dataset. In addition, we chose a cosine threshold of 0.1, as values below 0.1 show spurious matches between the text and image pairs.

\label{app:setup}

\section{Additional results}
\label{app:results}
\subsection{MaxSkew/MinSkew Results on other networks}
In Table~\ref{app:tab:pataskew}-\ref{app:tab:ffskew}, we present the Max- and Min-Skew scores (both unbounded and @k) for two other networks (ALBEF~\cite{ALBEF} and BLIP~\cite{li2022blip}) on the \pata and FairFace datasets. It is noteworthy that the overall Max and Min-Skew scores for BLIP are generally low indicating that the network is relatively bias-free. We found an inconsistency in the hyperparameters used for the computation of the skew scores for CLIP and Flava, as compared to those for BLIP and ALBEF. Upon removing the inconsistency, we find a different baseline and improved scores bearing the same trend.
% These will be updated in the final revision of the paper.

\begin{table}[!htb]
\caption{Systematic bias evaluation of VLMs and their \method counterparts using \textit{MaxSkew}, \textit{MinSkew}, MS@k=\textit{MaxSkew@k}, mS@k=\textit{MinSkew@k} metrics on the \textbf{\pata} dataset. \{+/-\} refers to the positive and negative sentiments. [A]=ALBEF\cite{ALBEF}, [A]$_{\text{D}}$=\method-ALBEF, [B]=BLIP, [B]$_{\text{D}}$=\method-BLIP\cite{li2022blip}. Values closer to zero indicate fairness. \method-augmented VLMs exhibit better fairness.}
\vspace{-0.5em}
\label{app:tab:pataskew}
\setlength{\tabcolsep}{2pt}
\renewcommand{\arraystretch}{0.9}
\begin{tabular}{c|l|llllllll}
\toprule
\multicolumn{1}{c|}{PA} & +/- & \multicolumn{2}{c}{MaxSkew} & \multicolumn{2}{c}{MinSkew} & \multicolumn{2}{c}{MaxSkew} & \multicolumn{2}{c}{MinSkew} \\
\multicolumn{1}{c|}{} &  & $A$ & $A_D$ & $A$ & $A_D$ & $B$ & $B_D$ & $B$ & $B_D$ \\
\hline
\multirow{2}{*}{Race} & +ve & 0.65 & \ccg{0.64} & -8.87 & \ccr{-8.96} & 0.06 & \ccg{0.05} & -0.06 & -0.06 \\
 & -ve & 0.62 & \ccg{0.61} & -7.34 & \ccg{-6.84} & 0.09 & \ccg{0.06} & -0.08 & \ccg{-0.06} \\
\multirow{2}{*}{Gender} & +ve & 0.33 & \ccg{0.31} & -3.82 & \ccg{-3.02} & 0.04 & \ccg{0.02} & -0.03 & \ccg{-0.02} \\
 & -ve & 0.32 & \ccg{0.29} & -3.14 & \ccg{-1.27} & 0.05 & \ccg{0.02} & -0.05 & \ccg{-0.02} \\
\multirow{2}{*}{Age} & +ve & 0.32 & \ccr{0.34} & -1.84 & \ccg{-1.77} & 0.04 & \ccg{0.03} & -0.04 & \ccg{-0.03} \\
 & -ve & 0.33 & \ccg{0.28} & -3.31 & \ccg{-3.20} & 0.03 & 0.03 & -0.04 & -0.04 \\
\hline
\multicolumn{1}{l|}{} &  & \multicolumn{2}{c}{MS@k} & \multicolumn{2}{c}{mS@k} & \multicolumn{2}{c}{MS@k} & \multicolumn{2}{c}{mS@k} \\
\multicolumn{1}{l|}{} &  & $A$ & $A_D$ & $A$ & $A_D$ & $B$ & $B_D$ & $B$ & $B_D$ \\
\hline
\multirow{2}{*}{Race} & +ve & 0.66 & \ccg{0.64} & -8.88 & \ccr{-8.96} & 0.24 & \ccg{0.23} & -0.29 & \ccg{-0.41} \\
 & -ve & 0.63 & \ccg{0.62} & -7.40 & \ccg{-6.84} & 0.29 & \ccg{0.24} & -0.40 & \ccg{-0.39} \\
\multirow{2}{*}{Gender} & +ve & 0.33 & \ccg{0.31} & -3.83 & \ccg{-3.02} & 0.15 & \ccg{0.09} & -0.19 & \ccg{-0.10} \\
 & -ve & 0.32 & \ccg{0.29} & -3.14 & \ccg{-1.27} & 0.16 & \ccg{0.09} & -0.21 & \ccg{-0.11} \\
\multirow{2}{*}{Age} & +ve & 0.32 & \ccr{0.34} & -1.84 & \ccg{-1.77} & 0.19 & \ccg{0.15} & -0.31 & \ccg{-0.21} \\
 & -ve & 0.33 & \ccg{0.29} & -3.31 & \ccg{-3.22} & 0.17 & \ccr{0.23} & -0.23 & \ccr{-0.32}\\
 \bottomrule
\end{tabular}
\end{table}
% Please add the following required packages to your document preamble:
% \usepackage{multirow}
\begin{table}[]
\caption{Systematic bias evaluation of VLMs and their \method counterparts using \textit{MaxSkew}, \textit{MinSkew}, MS@k=\textit{MaxSkew@k}, mS@k=\textit{MinSkew@k} metrics on \textbf{FairFace}\cite{fairface} dataset. \{+/-\} refers to the positive and negative sentiments. [A]=ALBEF\cite{ALBEF}, [A]$_{\text{D}}$=\method-ALBEF, [B]=BLIP\cite{li2022blip}, [B]$_{\text{D}}$=\method-BLIP. Values closer to zero indicate fairness. \method-augmented VLMs exhibit better fairness.}
\label{app:tab:ffskew}
\vspace{-0.5em}
\setlength{\tabcolsep}{2pt}
\renewcommand{\arraystretch}{0.9}
\begin{tabular}{c|c|llllllll}
\toprule
PA & \multicolumn{1}{c|}{+/-} & \multicolumn{2}{c}{MaxSkew} & \multicolumn{2}{c}{MinSkew} & \multicolumn{2}{c}{MaxSkew} & \multicolumn{2}{c}{MinSkew} \\
\multicolumn{1}{l|}{} &  & $[A]$ & $[A]_D$ & $[A]$ & $[A]_D$ & $[B]$ & $[B]_D$ & $[B]$ & $[B]_D$ \\
\hline
\multirow{2}{*}{Race} & pos & 0.50 & \ccg{0.34} & -0.95 & \ccg{-0.72} & 0.04 & 0.04 & -0.03 & \ccr{-0.05} \\
 & neg & 0.56 & \ccg{0.50} & -1.05 & \ccg{-0.99} & 0.05 & \ccg{0.03} & -0.05 & -0.05 \\
\multirow{2}{*}{Gender} & pos & 0.19 & \ccg{0.12} & -0.30 & \ccg{-0.16} & 0.01 & 0.01 & -0.01 & -0.01 \\
 & neg & 0.28 & \ccg{0.19} & -0.49 & \ccg{-0.30} & 0.01 & 0.01 & -0.01 & -0.01 \\
\multirow{2}{*}{Age} & pos & 0.39 & \ccg{0.30} & -0.19 & -0.19 & 0.02 & \ccg{0.01} & -0.01 & \ccr{-0.03} \\
 & neg & 0.38 & \ccg{0.24} & -0.39 & \ccg{-0.23} & 0.03 & \ccg{0.02} & -0.02 & \ccr{-0.04} \\
\hline
\multicolumn{1}{l|}{} &  & \multicolumn{2}{c}{MS@k} & \multicolumn{2}{c}{mS@k} & \multicolumn{2}{c}{MS@k} & \multicolumn{2}{c}{mS@k} \\
\multicolumn{1}{l|}{} &  & $[A]$ & $[A]_D$ & $[A]$ & $[A]_D$ & $[B]$ & $[B]_D$ & $[B]$ & $[B]_D$ \\
\hline
\multirow{2}{*}{Race} & pos & 0.61 & \ccg{0.50} & -1.17 & \ccg{-1.06} & 0.61 & \ccg{0.51} & -0.49 & \ccr{-1.21} \\
 & neg & 0.65 & \ccg{0.59} & -1.19 & \ccg{-1.18} & 0.63 & \ccg{0.51} & -0.88 & \ccr{-1.01} \\
\multirow{2}{*}{Gender} & pos & 0.24 & \ccg{0.16} & -0.38 & \ccg{-0.23} & 0.19 & \ccg{0.11} & -0.31 & \ccg{-0.12} \\
 & neg & 0.33 & \ccg{0.24} & -0.64 & \ccg{-0.41} & 0.19 & \ccg{0.18} & -0.29 & \ccg{-0.20} \\
\multirow{2}{*}{Age} & pos & 0.42 & \ccr{0.43} & -0.22 & \ccr{-0.26} & 0.35 & \ccg{0.23} & -0.22 & \ccr{-0.63} \\
 & neg & 0.49 & \ccg{0.31} & -0.53 & \ccg{-0.29} & 0.41 & \ccg{0.26} & -0.34 & \ccr{-0.92}\\
 \bottomrule
\end{tabular}
\end{table}

% Please add the following required packages to your document preamble:
% \usepackage{multirow}
\begin{table}[]
\caption{The Max-/Min-Skew scores for the \pata dataset for the different variants of ViT-based CLIP. $[B_{s}]$ is for CLIP-ViT-B/16, and $[L]$ is for ViT-L/14.}
\label{app:tab:vits}
\vspace{-0.5em}
\setlength{\tabcolsep}{2pt}
\renewcommand{\arraystretch}{0.9}
\begin{tabular}{c|l|llllllll}
\toprule
\multicolumn{1}{l|}{PA} & +/- & \multicolumn{2}{c}{MSkew} & \multicolumn{2}{c}{mSkew} & \multicolumn{2}{c}{MSkew} & \multicolumn{2}{c}{mSkew} \\
\multicolumn{1}{l|}{} &  & $B_{s}$ & $[B_{s}]_D$ & $[B_{s}]$ & $[B_{s}]_D$ & $[L]$ & $[L]_D$ &  $[L]$ & $[L]_D$ \\
\midrule
\multirow{2}{*}{Race} & pos & 0.03 & 0.03 & -0.04 & -0.04 & 0.25 & 0.25 & -0.54 & \ccg{-0.52} \\
 & neg & 0.04 & \ccg{0.03} & -0.04 & -0.04 & 0.28 & \ccg{0.27} & -0.46 & -0.46 \\
\multirow{2}{*}{Gender} & pos & 0.01 & 0.01 & -0.01 & -0.01 & 0.12 & \ccg{0.09} & -0.16 & \ccg{-0.11} \\
 & neg & 0.02 & \ccg{0.01} & -0.02 & \ccg{-0.01} & 0.14 & \ccg{0.12} & -0.20 & \ccg{-0.18} \\
\multirow{2}{*}{Age} & pos & 0.01 & 0.01 & -0.01 & -0.01 & 0.16 & \ccr{0.18} & -0.23 & \ccr{-0.27} \\
 & neg & 0.01 & \ccr{0.02} & -0.02 & -0.02 & 0.23 & \ccr{0.29} & -0.36 & \ccr{-0.48}\\
 \bottomrule
\end{tabular}
\end{table}

\subsection{Zero-shot Results}
In Table~\ref{apptab:zeroshot}, we present our zero-shot evaluation for the debiased VLM networks, as described in Section~\ref{sec:evaluation} of the paper. Table~\ref{app:video} presents zero-shot evaluation for the debiased VLMs for video datasets.  
\begin{table}%[t]
\small
\setlength{\tabcolsep}{2.1pt}
\renewcommand{\arraystretch}{0.9}
\centering
\caption{Results of state-of-the-art visual-language models and their \method counterparts for four image classification datasets. Across seven pre-trained visual-language models, \method achieves zero-shot performance similar to vanilla models.
}
\label{apptab:zeroshot}
{
% \resizebox{8.3cm}{!}{
\begin{tabular}{lcccc}
    \toprule
    Model & C-10 & C-100 & FER2013 & ImageNet\\
    \midrule
    \midrule
    CLIP \begin{scriptsize}(ViT/B-32)\end{scriptsize} & 89.93 & 62.93 & 43.83 & 58.08\\
    \method-CLIP \begin{scriptsize}(ViT/B-32)\end{scriptsize} & 88.85 & 60.08 & 39.60 & 55.84\\
    $\Delta$ & 1.08 & 2.85 & 4.23 & 2.24\\
    \midrule
    CLIP \begin{scriptsize}(ViT/B-16)\end{scriptsize} & 90.96 & 67.49 & 50.74 & 63.64\\
    \method-CLIP \begin{scriptsize}(ViT/B-16)\end{scriptsize} & 90.23 & 66.16 & 49.33 & 61.36\\
    $\Delta$ & 0.73 & 1.33 & 1.41 & 2.28\\
    \midrule
    CLIP \begin{scriptsize}(ViT/L-14)\end{scriptsize} & 95.73 & 76.64 & 46.16 & 71.22\\
    \method-CLIP \begin{scriptsize}(ViT/L-14)\end{scriptsize} & 95.26 & 75.68 & 42.33 & 66.43\\
    $\Delta$ & 0.47 & 0.96 & 3.83 & 2.24\\
    \midrule
    CLIP \begin{scriptsize}(RN50)\end{scriptsize} & 74.06 & 40.89 & 37.67 & 55.22\\
    \method-CLIP \begin{scriptsize}(RN50)\end{scriptsize} & 72.36 & 39.73 & 40.95 & 52.96\\
    $\Delta$ & 1.7 & 1.16 & -3.28 & 2.26\\
    \midrule
    FLAVA & 90.53 & 65.60 & 28.36 & 49.30 \\
    \method-FLAVA & 89.05 & 64.00 & 27.19 & 47.67 \\
    $\Delta$ & 1.48 & 1.60 & 1.17 & 1.63\\
    \midrule
    BLIP & 85.00 & 51.61 & 39.50 & 32.57 \\
    \method-BLIP & 81.20 & 48.90 & 36.50 & 29.94 \\
    $\Delta$ & 3.80 & 2.71 & 3.00 & 2.63\\
    \midrule
    ALBEF & 84.00 & 50.61 & 39.39 & 31.57 \\
    \method-ALBEF & 80.20 & 47.80 & 35.89 & 29.92 \\
    $\Delta$ & 3.80 & 2.81 & 3.50 & 1.65\\
    \bottomrule
\end{tabular}}
% }
\end{table}

\begin{table}[!ht]
\small
\setlength{\tabcolsep}{1.5pt}
\renewcommand{\arraystretch}{0.9}
\centering
\caption{Results of state-of-the-art visual-language models and their \method counterparts for three video classification datasets. Across five pre-trained visual-language models, \method achieves zero-shot performance similar to vanilla models.
}
\label{app:video}
{
% \resizebox{8.3cm}{!}{
\begin{tabular}{lcccc}
    \toprule
    \multirow{2}{*}{Model} & UCF-101 & Kinetics-700 & \multicolumn{2}{c}{RareAct}\\
    & Top-1 & AVG & mWAP & mWSAP\\
    \midrule
    CLIP \begin{scriptsize}(ViT/B-32)\end{scriptsize} & 57.65 & 43.97 & 16.63 & 16.78\\
    \method-CLIP \begin{scriptsize}(ViT/B-32)\end{scriptsize} & 55.77 & 42.20 & 16.02 & 16.03\\
    $\Delta$ & 1.88 & 1.77 & 0.61 & 0.75 \\
    \midrule
    CLIP \begin{scriptsize}(ViT/B-16)\end{scriptsize} & 59.55 & 48.38 & 18.58 & 18.69\\
    \method-CLIP \begin{scriptsize}(ViT/B-16)\end{scriptsize} & 56.53 & 46.49 & 17.54 & 17.66\\
    $\Delta$ & 3.02 & 1.89 & 1.04 & 1.03 \\
    \midrule
    CLIP \begin{scriptsize}(ViT/L-14)\end{scriptsize} & 67.88 & 55.86 & 25.42 & 25.55\\
    \method-CLIP \begin{scriptsize}(ViT/L-14)\end{scriptsize} & 67.43 & 53.21 & 25.20 & 25.34\\
    $\Delta$ & 0.45 & 2.65 & 0.22 & 0.21 \\
    \midrule
    CLIP \begin{scriptsize}(RN50)\end{scriptsize} & 52.73 & 39.39 & 15.08 & 15.09\\
    \method-CLIP \begin{scriptsize}(RN50)\end{scriptsize} & 50.25 & 38.59 & 14.41 & 14.54\\
    $\Delta$ & 2.48 & 0.8 & 0.67 & 0.55 \\
    \midrule
    FLAVA & 39.09 & 37.85 & 16.12 & 16.14\\
    \method-FLAVA & 37.27 & 35.59 & 15.30 & 15.43\\
    $\Delta$ & 1.82 & 2.26 & 0.82 & 0.71 \\
    \midrule
    BLIP & 43.26 & 37.07 & 16.35 & 16.44\\
    \method-BLIP & 40.34 & 34.78 & 15.86 & 15.92\\
    $\Delta$ & 2.92 & 2.29 & 0.49 & 0.52 \\
    \midrule
    ALBEF & 22.07 & 26.10 & 15.23 & 15.49\\
    \method-ALBEF & 20.77 & 24.33 & 14.33 & 14.56\\
    $\Delta$ & 1.3 & 1.77 & 0.9 & 0.93 \\
    \bottomrule
\end{tabular}}
% }
\end{table}

% Please add the following required packages to your document preamble:
% \usepackage{multirow}
\subsection{Qualitative Results}
We also present qualitative results for face image retrieval with text queries (CLIP text features), using the image features generated using CLIP and \method-CLIP. Figure \ref{fig:qualitative} shows a few instances of two phrases. Our results indicate an improvement in the diversity of results. For instance, for the phrases ``photo of a doctor" and ``photo of a  scientist", we see a clear improvement in the gender parity of the returned faces. We note some overlap between the results but the ranks assigned to them are different. Also, we note that the overlap is higher for phrases containing the keyword ``person", and we find that this is so because some images have a much higher text association with the keyword than others.

\section{Further Ablation Studies}
\label{app:ablation}
\vspace{-0.5em}
We present results for further ablation studies as evidence for the effectiveness of the \method framework.

\subsection{Disentanglement of Protected Attributes in ARL residual representation} Figure \ref{app:tsne} illustrates the degree of disentanglement that the ARL module imposes on CLIP features. The gender and age clusters are distinctly visible (column 2 of the figure), while the ethnic-racial clusters have a slightly worse disentanglement. We attribute that to the lower accuracy of the race classifier (PAC) trained using CLIP features on the FairFace dataset. We also observe that after adding the residual, point co-incidences increase considerably over the base model's plot, indicating that the \method-CLIP model is worse at identifying gender, race and age than the vanilla CLIP model. 

\subsection{Joint-training for PAC and ARL in an adversarial setting} Previous approaches like Berg et al.~\cite{berg2022prompt} use adversarial training of a protected-attribute classifier (PAC). We attempt to use the same approach with our ARL model and find much worse performance on the Max-Skew and Min-skew scores. This is because the network does not converge (even with modified hyperparameters) to the joint minimum for the classifier losses ($L_{\text{ce}}$) and the reconstruction loss ($L_{\text{recon}}$).
\begin{table}[!h]
\centering
\caption{Ablation results for joint training of the PAC and ARL modules The joint training does not yield the expected de-biasing effect because the network does not converge to a common minimum between the $L_{\text{recon}}$ and $-L_{\text{ce}}$}
\vspace{-0.5em}
\label{app:pataskew}
\setlength{\tabcolsep}{2pt}
\renewcommand{\arraystretch}{1}
\begin{tabular}{c|c|cccccccc}
\toprule
\multirow{2}{*}{PA} & \multirow{2}{*}{+/-} & \multicolumn{2}{c}{MaxSkew} & \multicolumn{2}{c}{MinSkew}  \\
 && \multicolumn{1}{c}{[C]} & \multicolumn{1}{c}{[C]$_{\text{D}}$} & \multicolumn{1}{c}{[C]} & \multicolumn{1}{c}{[C]$_{\text{D}}$} \\
 \hline
\multirow{2}{*}{Age} & +ve & 0.10 & \ccr{0.23} & -0.12 & \ccr{-0.31}   \\
 & -ve & 0.19 & 0.19 & -1.21 & \ccg{-0.27} \\
 \multirow{2}{*}{Race} & +ve & 0.16 & \ccr{0.39} & -0.43 & \ccr{-1.22} \\
 & -ve  & 0.45 & \ccg{0.41} & -3.40 & \ccg{-3.21}  \\
 \multirow{2}{*}{Gender} & +ve & 0.09 & \ccr{0.18} & -0.11 & \ccr{-0.27} \\
 & -ve  & 0.21 & \ccg{0.19} & -0.79 & \ccg{-0.73}  \\
\bottomrule
\end{tabular}
\end{table}

\begin{figure*}[t]
    \centering
    \includegraphics[width=0.9\linewidth]{tsneplots.pdf}
    \caption{High-resolution version of Figure \ref{fig:tsne} in the paper. TSNE Plots for CLIP, Residual and \method-CLIP features for a subset of the \pata dataset indicate that the residual plots indeed capture the specific attributes and that the \method-CLIP features have greater overlap between points of different protected labels than the original features.}
    \label{app:tsne}
\end{figure*}

\begin{figure*}
    \includegraphics[width=\textwidth]{dear_error.pdf}
    \caption{Comparing class error rate of CIFAR-100 for vanilla CLIP Vs \method-CLIP. We observe an increase in error rate for only human-related labels, \eg, an increase from 45\% to 71\% in the error rate of the ``man'' class after debiasing.}
\end{figure*}

\subsection{Error analysis for zero-shot tasks} We compare the class error rate of CIFAR-100 for CLIP and \method-CLIP. We observe an increase in error rate for only human-related labels, \eg, an increase from 45\% to 71\% in the error rate of the \emph{man} class after debiasing. This proves that the debiasing framework is successful at paying more attention to the features that characterize protected attributes such as \texttt{gender}, aligning with the overall objective of \method. 

\subsection{Extending \method for unimodality} Next, we extend our proposed \method framework to unimodal models, where we take image representations from unimodal ViT/B-16, and ViT/B-14 models pre-trained on ImageNet and then train a linear layer on top of it. For CIFAR-10 and CIFAR-100, we observe a classification accuracy drop of \textbf{1.1\%} and \textbf{0.8\%}, respectively. Further, we observe that debiasing leads to a uniform accuracy drop across all protected attributes, \ie, it decomposes the visual representation so that the protected information is subtracted out. 

\begin{figure*}[!ht]
    \centering
    \includegraphics[width=\linewidth]{qualbias.pdf}
    \caption{Qualitative comparison on top-k retrieval (k=48) for CLIP (left) and \method-CLIP features (right).}
    \label{fig:qualitative}
\end{figure*}

\section{Limitations and Future Work}
Our work presents the first step towards debiasing VLMs and as such, we observe its limitations in several respects:
\begin{enumerate}
\item The association of sub-string matches, such as the text ``person" causes keywords with the \textit{person} suffix or phrases with the keyword in it to behave differently than expected. For instance, the keyword business-person has a different association (as measured by the max-skew) than the keyword ``business". This causes overall skew distributions to be inaccurate, and incommensurate with the qualitative assessment.
\item We also observe that the network often over-compensates flipping the skew in favor (or disfavor) of a different protected label. For instance, in the case of ALBEF~\cite{ALBEF}, using the \method framework (Table~\ref{app:tab:pataskew}), we observe that the skew increases for the Age-Positive combination. However, we also find that it flips over from being in favor of ``young" people to that of ``old" people. We attribute this flipping behavior to the inaccuracy of training of the Age-classifier in the PAC module, and we look to improve its accuracy of it by modifying its hyperparameters or architectures.
\item We also observe a slight increase in skew values for the FLAVA model using \method framework for Race-Postive and Race-Negative combinations. We attribute this to the inaccuracy of PAC in classifying race. We hypothesize alleviating this behavior by modifying the training hyperparameters or architectures of \method.
\item We recognize that the skew analysis is highly sensitive to its parameters (thresholds, the value of k, and choice of text prompts), and we look to address these with uniform metrics in the future. 
\item The first version of our proposed \pata dataset does not cover the entire ground to determine the fairness of a VLM. We look to expand the categories set to include more scenes and queries.
\item The \method framework appears not to work very well for all variants of the CLIP network. (Table \ref{app:tab:vits}). We attribute this again to the inaccurate PAC module. 
\end{enumerate}

% \paragraph{Paper Errata}
% We found one scene category in the \pata dataset to have no images of the Hispanic ethnic-racial group. This does not affect the analysis significantly. We will rectify this in the final version of the dataset.



\end{document}
