\section{Experimental Evaluation} \label{sec:experiments}
We implement \name on top of the Rust compiler, version \texttt{nightly-2023-01-26}. We use
\crust with version 0.16.0. For the SAT solver, we rely on a Rust-binding of \texttt{z3}\cite{z3}
with version 0.11.2.
We run all our experiments on a MacBook Pro with an Apple M1 chip, with 8 cores (4 performance
and 4 efficiency). The computer has 16GB RAM and runs macOS Monterey 12.5.1.


{\bf Benchmark selection.}
To evaluate the utility of \name, we collected a benchmark suite of 20 programs (Table~\ref{tab:benchmarks}).
These include benchmarks from Laertes~\cite{DBLP:journals/pacmpl/EmreSDH21}'s accompanying 
artifact~\cite{emre_mehmet_2021_5442253} (marked by * in Table~\ref{tab:benchmarks})\footnote{We excluded \texttt{json-c}, \texttt{optipng}, \texttt{tinycc}
where \name crashes because
of the uses of unions and variadic arguments as discussed in Section~\ref{sec:discussion}. Additional programs (\texttt{qsort}, \texttt{grabc}, \texttt{xzoom}, \texttt{snudown}, \texttt{tmux},
\texttt{libxml2}) are mentioned in the paper~\cite{DBLP:journals/pacmpl/EmreSDH21} but are either missing 
or incomplete in the artifact~\cite{emre_mehmet_2021_5442253}.}, and additionally 8 real-world projects (\texttt{binn}, \texttt{brotli}, \texttt{buffer},
\texttt{heman}, \texttt{json.h}, \texttt{libtree}, \texttt{lodepng}, \texttt{rgba}) together with 4 commonly used data structure libraries (\texttt{avl}, \texttt{bst}, \texttt{ht}, \texttt{quadtree}).


\vspace{-1cm}
\begin{table}[!ht]
\caption{Benchmark information}
    \centering
    \begin{tabular}{lrrrr|lrrrr}
        Benchmark & Files         & Structs          & Functions            & LOC    & Benchmark         & Files         & Structs          & Functions            & LOC   \\
        avl       & 1             & 2                & 11                   & 229    & libcsv*           & 1             & 6                & 23                   & 976   \\
        binn      & 1             & 5                & 165                  & 4426   & libtree           & 1             & 18               & 32                   & 2610  \\
        brotli    & 30            & 237              & 867                  & 537723 & libzahl*          & 49            & 65               & 108                  & 4655  \\
        bst       & 1             & 1                & 6                    & 154    & lil*              & 2             & 9                & 136                  & 5670  \\
        buffer    & 2             & 3                & 42                   & 1207   & lodepng           & 1             & 19               & 236                  & 14153 \\
        bzip2*    & 9             & 39               & 126                  & 14829  & quadtree          & 5             & 14               & 31                   & 1216  \\
        genann*   & 6             & 10               & 27                   & 2410   & rgba              & 2             & 3                & 19                   & 1855  \\
        heman     & 24            & 52               & 302                  & 13762  & robotfindskitten* & 1             & 8                & 18                   & 1508  \\
        ht        & 1             & 3                & 10                   & 264    & tulipindicators*  & 111           & 18               & 229                  & 22363 \\
        json.h    & 1             & 13               & 53                   & 3860   & urlparser*        & 1             & 1                & 21                   & 1379 
        \end{tabular}
    \label{tab:benchmarks}
\end{table}
\vspace{-1cm}


\subsection{Research questions}
We aim at answering the following research questions.
\begin{itemize}
    \item[] RQ1. How many raw pointers/pointer uses can \name translate to safe pointers/pointer uses?
    \item[] RQ2. How does \name's result compare with the state-of-the-art~\cite{DBLP:journals/pacmpl/EmreSDH21}?
    \item[] RQ3. What is the runtime performance of \name?
\end{itemize}


{\bf RQ 1: Unsafe pointer reduction.}
In order to judge \name's efficacy, we measure the reduction rate of raw pointer declarations and uses.
This is a direct indicative of the improvement in safety, as safe pointers are always checked by the Rust compiler (even inside \lstinline{unsafe} regions).  
As previously mentioned, we focus on mutable non-array pointers. %
The results are presented in Table~\ref{tab:unsafe-reduce},
where \#ptrs counts the number of raw pointer declarations in a given benchmark, \#uses counts the
number of times raw pointers are being used, and the Laertes and Crown headers denote the reduction rates of the
number of raw pointers and raw pointer uses achieved by the two tools, respectively.
For instance, for benchmark \texttt{avl}, the rate of 100\% means that 
all raw pointer declarations and their uses are translated into safe ones.
Note that the ``-'' symbols on the row corresponding to \lstinline{robotfindskitten}
are due to the fact that the benchmark contains 0 raw pointer uses.

The median reduction rates achieved by \name for raw pointers and raw pointer uses are
37.3\% and 62.1\%, respectively. 
\name achieves a 100\% reduction rate for many non-trivial data
structures (\lstinline{avl}, \lstinline{bst}, \lstinline{buffer}, \lstinline{ht}), as well as for \lstinline{rgba}. 
For \texttt{brotli}, a lossless data compression algorithm developed by Google, which is our largest benchmark,  
\name achieves reduction rates of 21.4\% and 20.9\%, respectively.  
The relatively low reduction rates for \lstinline{brotli} and a few other  
benchmarks (\texttt{tulipindicators}, \texttt{lodepng}, \texttt{bzip2}, \texttt{genann},
\texttt{libzahl}) is due to their use of non-standard memory management strategies (discussed in detail in
Section \ref{sec:discussion}). %

Notably, all the translated benchmarks compile under the aforementioned Rust compiler version.
As a check of semantics preservation, for the benchmarks that provide test suites
(\lstinline{libtree}, \lstinline{rgba}, \lstinline{quadtree}, \lstinline{urlparser}, \lstinline{genann}, \lstinline{buffer}),
our translated benchmarks pass all the provided tests.

\vspace{-.9cm}
\begin{table}[]
    \centering
\caption{Reduction of (mutable, non-array) raw pointer declarations and uses}
\begin{tabular*}{\textwidth}{c @{\extracolsep{\fill}} lrrrrrr}
    Benchmark         & \#ptrs & Laertes         & Crown            & \#uses & Laertes & Crown            \\
    avl               & 8      & 0.0\%           & \textbf{100.0\%} & 41     & 0.0\%   & \textbf{100.0\%} \\
    binn              & 103    & 46.6\%          & \textbf{65.0\%}  & 247    & 62.3\%  & \textbf{71.3\%}  \\
    brotli            & 846    & 0.0\%           & \textbf{21.4\%}  & 3686   & 0.0\%   & \textbf{20.9\%}  \\
    bst               & 5      & 0.0\%           & \textbf{100.0\%} & 22     & 0.0\%   & \textbf{100.0\%} \\
    buffer            & 38     & 0.0\%           & \textbf{100.0\%} & 56     & 0.0\%   & \textbf{100.0\%} \\
    bzip2*            & 126    & 14.3\%          & \textbf{26.2\%}  & 2946   & 2.2\%   & \textbf{3.7\%}   \\
    genann*           & 28     & 0.0\%           & \textbf{7.1\%}   & 160    & 0.0\%   & \textbf{15.0\%}  \\
    heman             & 360    & 30.3\%          & \textbf{35.0\%}  & 926    & 50.2\%  & \textbf{60.2\%}  \\
    ht                & 6      & 33.3\%          & \textbf{100.0\%} & 28     & 42.9\%  & \textbf{100.0\%} \\
    json.h            & 128    & 2.3\%           & \textbf{23.4\%}  & 647    & 1.2\%   & \textbf{62.1\%}  \\
    libcsv*           & 20     & 65.0\%          & \textbf{70.0\%}  & 141    & 97.9\%  & 97.9\%           \\
    libtree           & 48     & 29.2\%          & \textbf{39.6\%}  & 227    & 33.0\%  & \textbf{62.1\%}  \\
    libzahl*          & 87     & 2.2\%           & \textbf{16.1\%}  & 279    & 4.1\%   & \textbf{16.8\%}  \\
    lil*              & 202    & 9.2\%           & \textbf{18.8\%}  & 1018   & 51.4\%  & \textbf{69.4\%}  \\
    lodepng           & 227    & \textbf{46.3\%} & 44.9\%           & 1232   & 40.4\%  & 37.7\%           \\
    quadtree          & 33     & 0.0\%           & \textbf{42.4\%}  & 117    & 0.0\%   & \textbf{48.7\%}  \\
    rgba              & 6      & 83.3\% & 83.3\%  & 12     & 100.0\% & 100.0\%          \\
    robotfindskitten* & 1      & 0.0\%           & 0.0\%            & 0      & -       & -                \\
    tulipindicators*  & 134    & 0.0\%           & \textbf{0.7\%}   & 625    & 0.0\%   & 0.0\%            \\
    urlparser*        & 9      & 0.0\%           & \textbf{11.1\%}  & 40     & 0.0\%   & \textbf{45.0\%} 
\end{tabular*}
\label{tab:unsafe-reduce}
\end{table}
\vspace{-.7cm}


{\bf RQ 2: Comparing with state-of-the-art.}
The comparison of \name with Laertes~\cite{DBLP:journals/pacmpl/EmreSDH21}
is also shown in Table \ref{tab:unsafe-reduce}, with bold font highlighting
better results. The data on Laertes is either directly extracted from 
the artifact~\cite{emre_mehmet_2021_5442253} or has been confirmed by 
the authors through private correspondence. We can see that \name outperforms 
the state-of-the-art (often by a significant degree) in most cases, with \texttt{lodepng} being the only exception, where we suspect that the reason also
lies with non-standard memory management strategies mentioned before. Laertes
is less affected by this as it does not rely on ownership analysis. 


{\bf RQ 3: Runtime performance.}
Although our analysis relies on solving a constraint satisfaction problem that is proven to
be NP-complete, in practice the runtime performance of \name is consistently  high.
The execution time of the analysis and the rewrite for the whole benchmark suite is within 60 seconds. 




