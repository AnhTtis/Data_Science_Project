\section{Architecture} \label{sec:architecture}
In this section, we give a brief overview of \name's architecture.
\name takes as input Rust programs automatically translated by \crust.
These programs are very similar to the original C ones,
where the C syntax is replaced by Rust syntax.
\name applies several static analyses on the MIR of Rust
to infer properties of pointers:
\begin{itemize}

\item{\bf Ownership analysis}: computes ownership information about the pointers in the code, i.e. for each pointer it infers whether it is owning/non-owning at particular program locations.

\item{\bf Mutability analysis}: infers which pointers are used to modify the object they point to (inspired by \cite{10.1145/2384616.2384680,10.1145/1297027.1297051}).

\item{\bf Fatness analysis}: distinguishes array pointers from non-array pointers (inspired by \cite{10.1145/3527322}).
\end{itemize}  

The results of these analyses are summarised as type qualifiers~\cite{10.1145/1186632.1186635}.
A type qualifier is an atomic property (i.e., ownership, mutability, and fatness) that `qualifies' the standard pointer type.
These qualifiers are then utilised for pointer retyping. For example, an owning,
non-array pointer is retyped to \boxptr. %
After pointers have been retyped, \name rewrites their usages accordingly.











