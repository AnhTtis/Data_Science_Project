\documentclass[a4paper,12pt]{article}
\pdfoutput=1 % if your are submitting a pdflatex (i.e. if you have
             % images in pdf, png or jpg format)

\usepackage{jheppub} % for details on the use of the package, please
                     % see the JHEP-author-manual

\usepackage[T1]{fontenc} % if needed

\usepackage{graphicx}
\usepackage{amsmath}
\usepackage{amsfonts}
\usepackage{amssymb}
\usepackage{url}
\usepackage{hyperref}
\usepackage{subfigure}
\usepackage{array}
\newcolumntype{M}[1]{>{\centering\arraybackslash}m{#1}}


\title{\boldmath Modular $A_4$ symmetry in 3+1 active-sterile neutrino masses and mixings}


%% %simple case: 2 authors, same institution
%% \author{A. Uthor}
%% \author{and A. Nother Author}
%% \affiliation{Institution,\\Address, Country}

% more complex case: 4 authors, 3 institutions, 2 footnotes
\author[a]{Mayengbam Kishan Singh}
\author[a,b]{S. Robertson Singh}
\author[a,b,1]{N. Nimai Singh\note{Corresponding author}}
%\author[a,2]{and Fourth}

% The "\note" macro will give a warning: "Ignoring empty anchor..."
% you can safely ignore it.

\affiliation[a]{Department of Physics, Manipur University,\\Imphal-795003, India}
\affiliation[b]{Research Institute of Science and Technology\\Imphal-795003, India}
%\affiliation[c]{A School for Advanced Studies,\\some-location, Country}

% e-mail addresses: one for each author, in the same order as the authors
\emailAdd{kishan@manipuruniv.ac.in}
\emailAdd{robsoram@gmail.com}
\emailAdd{nimai03@yahoo.com}
%\emailAdd{fourth@one.univ}


%We observe the $Im[\tau]$ to satisfy neutrino oscillation data and the latest cosmological Planck upper bound on the sum of neutrino masses $\sum m_i <0.12$eV in different ranges $i.e.$ $Im[\tau] \sim [1.047,1.101]$ for NH and $Im[\tau] \sim [0.793,0.928]$ for IH. 

\abstract{Motivated by the significance of modular symmetry in describing neutrino masses and flavour structure, we apply the modular $A_4$ symmetry in 3+1 scheme of active-sterile neutrino mixings. Neutrino oscillation observables in 3$\sigma$ range are successfully reproduced through the vev of the modulus $\tau$ in normal hierarchy(NH) as well as inverted hierarchy(IH). Our model predicts $\sin^2\theta_{23}>0.523$ in both NH and IH. We also study other phenomenologies regarding effective neutrino masses $m_{\beta}$ in tritium beta decay and $m_{\beta\beta}$ in neutrinoless double beta decay. Mixings between active and sterile neutrinos are analysed in detail. Dirac CP-violating phase $\delta_{CP}$ as well as Majonara phases $\alpha$ and $\beta$ are also predicted from the model. The best-fit values of the neutrino mixing angles and ratios of two mass squared differences are determined using minimum $\chi^2$ analysis. Our model predicts best-fit values of the neutrino oscillation observables as $\sin^2\theta_{23}=0.572$, $ \sin^2\theta_{12}=0.313$, $\sin^2\theta_{13}=0.022$ and $r  = \sqrt{\Delta m_{21}^2/\Delta m_{31}^2}= 0.172$ for NH, whereas $\sin^2\theta_{23}=0.582,$ $\sin^2\theta_{12}=0.301,$ $ \sin^2\theta_{13}=0.022$ and $r = 0.171$ for IH. Finally, we also estimate the allowed range of the heavy Majorana neutrino mass scale $\lambda$ from experimental bounds of $m_{\beta\beta}$, $m_{\beta}$ and  $\vert U_{14}\vert^2$.}



\begin{document} 
\maketitle
\flushbottom

\section{Introduction}
Origin of neutrino masses and flavour structure is one of the most important problems of Standard Model(SM) of particle physics. Since the discovery of neutrino oscillation in various experiments such as SNO, SuperKamiokande, etc., various extensions of SM have been studied. Models with non-Abelian discrete symmetries such as $A_4$ \cite{altarelli2010discrete,king2007a4,babu2003underlying}, $S_4$ \cite{ma2006neutrino,altarelli2009revisiting,vien2022lepton,bazzocchi2013neutrino}, $S_3$ \cite{mohapatra2006s3,morisi2006flavor,Vien:2021uxw,grimus2005s3,koide2007s}, etc. find their distinct places in describing some of the experimental results as well as new predictions regarding unresolved problems of SM. Among these, the absolute mass scale of neutrino, Dirac CP violating phase, Baryogenesis, Dark matter, etc. are some of the main questions in high energy physics. 

Inspired by the observations of LSND \cite{aguilar2001evidence} and MiniBooNE \cite{aguilar2018significant,2021microboone}, many authors proposed the existence of a fourth state of neutrino called sterile neutrino. Sterile neutrinos are incorporated with the three neutrino theory in literatures in a 3+1 scheme  \cite{Zhang2011,giunti2020katrin,guintianrev, vien2022b,miranda2019revisiting,vien2021}, 3+1+1 scheme  \cite{kuflik2012neutrino,huang2013mev,fan2012light,nelson2011effects}, 3+2 scheme  \cite{donini2012minimal,archidiacono2012testing,babu2016light,karagiorgi2007leptonic} etc. One of the simplest extensions is the 3+1 scheme where a singlet sterile neutrino is added to the three active neutrinos, and sterile neutrino gets mass through minimal extended seesaw mechanism(MES) \cite{Zhang2011}. Many authors have used discrete symmetries to describe the neutrino masses and their flavour structures. The main disadvantages of such approaches are the presence of many hypothetical scalar fields as well as additional symmetry groups. Recently, an interesting approach  \cite{Feruglio} has been proposed in which modular symmetry is used along with the discrete symmetries as its subgroups.  An important feature of modular symmetry framework is that the Yukawa couplings can be transformed non-trivially as modular forms under the modular group. These modular forms are written as a function of a complex parameter $\tau$, called the modulus. A minimal or no scalar flavons are required to break the symmetry as the lepton masses are generated from the symmetry breaking by the vev of the modulus $\tau$. Modular forms of level 3 denoted by $\Gamma(3)$ is isomorphic to $A_4$. Detailed analysis on modular $A_4$ groups and its application in neutrino model building are studied in Refs.   \cite{Feruglio,kobayashi2018neutrino,kobayashi2018modular}. There are other works based on numerous modular groups such as modular $A_4$  \cite{zhang2020modular,Beherascoto,mishra2022type,kobayashi2020type,okada2020radiative,
nomura2019modular,abbas2021fermion,nomurascoto,behera2022implications}, modular $A_5$  \cite{ding2019neutrino,novichkov2019modular,PhysRevD.103.095013,PhysRevD.103.076005,Behera2021eut}, modular $S_4$ \cite{wang2020minimal,penedo2019lepton,wang2021dirac,ding2019modular,
KobayasiS4,LiuS4}, etc. to study neutrino phenomenologies, dark matter, leptogenesis, etc. For instance, in Ref. \cite{abbas2021fermion},  modular $A_4$ is used  to study lepton and quark masses and mixings based on inverse seesaw mechanism, Ref. \cite{behera2022implications} studies lepton mixings and leptogenesis in linear seesaw while scotogenic dark matter scenario is studied in Ref. \cite{nomurascoto,Beherascoto}. There are other works in the literature which study 3+1 active-sterile mixing using the conventional $A_4$ discrete symmetry group \cite{das2019active,mksingh,vien2022b,das3}. However,  modular $A_4$ symmetry has not been applied in 3+1 active-sterile neutrino mixing in minimal extended seesaw (MES) mechanism. The main objectives and features of the present work are 
\begin{enumerate}
\item[i.] To develop a new model of neutrino masses and flavour mixings based on modular $A_4$ symmetry for 3+1 scheme where we extend the SM by an $A_4$ triplet right-handed neutrino $N_i$ which is a singlet under $SU(2)$, and one SM singlet sterile neutrino $S$.
\item[ii.] To establish a detailed analysis of active-sterile mixing for sterile neutrinos in the mass range $m_s\sim(0.8,10)$ eV. 
\item[iii.] To search the specific ranges of the modulus $\tau$ and Dirac Yukawa coefficient $g$ for both NH and IH constrained by neutrino oscillation data as well as the cosmological Planck upper limit $\sum m_{i}<0.12$ eV.
\item[iv.] To reproduce active neutrino mixing angles, mass-squared differences as well as Dirac and Majorana phases $\delta_{CP},\alpha$ and $\beta$ for both NH and IH in the 3$\sigma$ range. We also  include the non-unitarity effect produced in the $(3\times 3)$ lepton mixing matrix originated from the mixing with sterile neutrino. 
\item[v.] To predict the neutrino effective mass parameters $m_{\beta}$ and $m_{\beta\beta}$ from tritium beta decay and neutrinoless double beta decay $(0\nu\beta\beta)$.
\item[vi.] To put an upper and lower bound on the heavy right-handed Majorana mass scale $\lambda$ using the experimental bounds of $\vert U_{i4}\vert$, $m_{\beta\beta}$ and $m_{\beta}$ from various experiments such as KamLAND-Zen + GERDA\cite{agostini2018improved}, KATRIN\cite{aker2020first,katrin2022direct}, etc.
\item[vii.] Finally, to determine the best-fit values of the model parameters and neutrino observables using the minimum $\chi^2$ analysis.
\end{enumerate}
Significant results are obtained for the active-sterile mixing parameters $\vert U_{i4}\vert^2 $, (where $i=1,2,3$) within the experimental bounds. As we shall discuss later, exciting results are obtained for the modulus $\tau$ in a very narrow range each for normal hierarchy(NH) and inverted hierarchy(IH). Predictions for other phenomenological studies reveal new results which could be probed by future experiments. The present paper is organised as follows. We present a detailed description of the model in section \ref{section2}, followed by the numerical analysis of the model in section \ref{section3}. Results of the analysis are presented in section \ref{section4}. We conclude with a brief summary and discussion in section \ref{section6}.

\section{Description of the model}\label{section2}
In this model, we use modular $A_4$ symmetry for studying lepton masses and mixings. We consider three right-handed neutrinos $N_{i}$ as triplet under $A_4$ and a sterile neutrino field $S$ as singlet. The SM lepton doublet $L$ also transforms as triplet of $A_4$ while the Higgs represented as $H_u$ and $H_d$ having hypercharges $+1/2 $ and $-1/2$ respectively are taken as singlets of $A_4$. They develop  VEVs due to symmetry breaking along $\langle H_u\rangle = \left(0, v_u/\sqrt{2}\right)^T$ and $\langle H_d\rangle = \left( v_d/\sqrt{2}, 0\right)$ inducing SM fermion mass terms. The right-handed charged leptons $e_r, \mu_r$ and $\tau_r$ are assigned $1,1^{\prime\prime},1^{\prime}$ of $A_4$ respectively. As a result, there are three independent coupling constants $\alpha^{\prime},\beta^{\prime}$ and $\gamma^{\prime}$ in the Lagrangian of the charged lepton sector. We consider one scalar field $\zeta$ as an $A_4$ singlet in order to generate sterile neutrino mass matrix $M_s$. The role of triplet scalar field $\phi$ with zero modular weight is to simplify our analysis by making the charged lepton mass matrix diagonal. It does not have any significance in the neutrino sector. Finally, the Yukawa coupling is transformed as a modular function $Y =(y_1,y_2,y_3)^T$, which is triplet under $A_4.$  The complete particle contents of the model with their corresponding group charges and modular weights $k_i$ are given in Table \ref{Table1}. The components of modular form $Y$ of weight 2, which transforms as triplet under $A_4$ can be expressed in terms of Dedekind eta-function $\eta(\tau)$ and its derivative $\eta^{\prime}(\tau)$ as \cite{Feruglio} 
\begin{table*}
\centering
\begin{tabular}{|M{1.5cm}|M{1.3cm}M{1.4cm}|M{1.5cm}|M{0.5cm}M{0.5cm}|M{0.5cm}M{0.5cm}|M{0.6cm}|}
\hline
\vspace*{0.2cm}
$\frac{Fields}{Charges}$ & $L$ & $e_r, \mu_r,\tau_r$ & $H_{u,d}$ & $N_i$ & $S$ & $\phi$& $\zeta$ &$Y$  \\
\hline
$SU(2)_L$& 2 & 1 & 2 & 1 & 1 &1 &1&1   \\
$U(1)_Y$ & $-1/2$ & $-1$& $\pm$ 1/2 & 0& 0& 0&0&0 \\
$A_4$& 3 &1,1$^{\prime\prime}$,1$^{\prime}$ &1 & 3 & 1 &3&1 &3  \\
$k_i$&$k_L$ &$k_e$  &$k_H$ &$k_N$ & $k_S$ & $k_{\phi}$ &$k_{\zeta}$& $k_Y$  \\
\hline
\end{tabular}
\caption{\centering Particle contents of the model and their group charges.}
\label{Table1}
\end{table*}

\begin{align}
y_1(\tau)= & \frac{i}{2\pi}\left[\frac{\eta^{\prime}(\frac{\tau}{3})}{\eta (\frac{\tau}{3})}+ \frac{\eta^{\prime}(\frac{\tau+1}{3})}{\eta (\frac{\tau+1}{3})} + \frac{\eta^{\prime}(\frac{\tau+2}{3})}{\eta (\frac{\tau+2}{3})} - \frac{27\eta^{\prime}(3\tau)}{\eta(3\tau)}\right],\nonumber
\end{align}
\begin{align}
y_2(\tau)= &\frac{-i}{\pi}\left[\frac{\eta^{\prime}(\frac{\tau}{3})}{\eta (\frac{\tau}{3})}+\omega^2 \frac{\eta^{\prime}(\frac{\tau+1}{3})}{\eta (\frac{\tau+1}{3})} +\omega \frac{\eta^{\prime}(\frac{\tau+2}{3})}{\eta (\frac{\tau+2}{3})} \right],\nonumber\\
y_3(\tau)= &\frac{-i}{\pi}\left[\frac{\eta^{\prime}(\frac{\tau}{3})}{\eta (\frac{\tau}{3})}+\omega \frac{\eta^{\prime}(\frac{\tau+1}{3})}{\eta (\frac{\tau+1}{3})} +\omega^2 \frac{\eta^{\prime}(\frac{\tau+2}{3})}{\eta (\frac{\tau+2}{3})} \right].\label{etafn}
\end{align}

where $\omega = e^{2\pi i/3}$ and $\eta(\tau)$ is defined as 
\begin{equation}
\eta(\tau)=q^{1/24}\prod_{n=1}^{\infty}(1-q^n),  \ \ \ \ q \equiv e^{i 2\pi \tau}.
\end{equation}
The overall coefficients in eq. (\ref{etafn}) is one possible choice; it cannot be uniquely determined. The $q-$expansions of the Yukawa components are expressed as 
\begin{align}
y_1(\tau)& = 1+12q+36q^2+12q^3+... \nonumber\\
y_2(\tau)& = -16q^{1/3}(1+7q+8q^2+...)\\
y_3(\tau)& = -18q^{2/3}(1+2q+5q^2+...).\nonumber
\end{align}
In modular symmetry framework, an interaction is invariant under the symmetry when sum of the modular weights for each associated fields and coupling is zero and it is also invariant under the discrete symmetry (such as $A_4$ symmetry). 


The Lagrangian of charged lepton sector invariant under $A_4$ symmetry is given below.
\begin{align}
-\mathcal{L}_{c} =\ \alpha^{\prime} e_r^c( H_d L \phi)_1 + \beta^{\prime} \mu_r^c (H_d L \phi)_{1^{\prime}} + \gamma^{\prime}\tau_r^c(H_d L \phi)_{1^{\prime\prime}} + h.c.
\label{chargelepton}
\end{align}
For the neutrino sector, the Lagrangian is 
\begin{align}\label{md}
-\mathcal{L}_{D}=&\ g_i (N_i^c)^TH_uLY + h.c \\ \nonumber
=&\ g_1 (N_i^c)^TH_u(LY)_{3S} + g_2 (N_i^c)^TH_u(LY)_{3A} + h.c \\
-\mathcal{L}_{R} =&\ \lambda Y (N_{i}^c)^T N_i + h.c \\
-\mathcal{L}_s= &\ \delta \zeta Y S^cN_i +h.c
\label{MR}
\end{align}
where, $\alpha^{\prime}, \beta^{\prime}$,$\gamma^{\prime},g_1,g_2,\lambda$ and $\delta$ are constant coefficients. The subscripts $`3S$' and $`3A$' respectively denote the symmetric and antisymmetric product of two triplet fields in $A_4$ symmetry. The above interaction terms for the charged lepton and neutrino sector are invariant under modular $A_4$ symmetry if the weights $k_i$ satisfy the following conditions,
\begin{align}
k_e + k_H+k_L+k_{\phi} =0,\ \ 
k_H+k_L+k_N+k_Y=0, \nonumber \\  
k_N+k_N+k_Y=0, \ \ \ \ \
k_{\zeta}+k_S+k_N+k_Y=0.
\end{align}
For modular form of weight $-k_Y=2,$ we have considered the following assignments of modular weights,
\begin{equation}
k_N=k_L=1, \ k_{\phi}=k_H=0,k_S=0,\ k_{\zeta}=1, \ \mbox{and} \ k_e=-1.
\end{equation}
Choosing the vev of $\phi$ along $v_{\phi}(1,0,0)$, eq.  (\ref{chargelepton}) gives a diagonal charged lepton mass matrix as
\begin{equation}
M_L= diag(\alpha^{\prime},\beta^{\prime},\gamma^{\prime})\ v_{\phi}v_d.
\label{ml}
\end{equation}
The charged lepton masses can be reproduced by adjusting the values the coefficients $\alpha^{\prime}, \beta^{\prime}$ and $\gamma^{\prime}$. 

Other higher dimensional terms in the neutrino sector, such as $\frac{g^{\prime}}{\Lambda}H_u\zeta S^c(LY)_1$, $\frac{\lambda_1}{\Lambda^2}YY\zeta\zeta(N_i^c)^TN_i$, $\frac{\delta_1}{\Lambda^2}YYNS\zeta\zeta\zeta$, $\frac{\delta_2}{\Lambda^3}YY\zeta\zeta\zeta\zeta SS$, etc. are also allowed by the symmetry of the model. However, their contributions are negligibly small and they have been ignored in our analysis. From eq. (\ref{md}) - eq. (\ref{MR}), after electroweak symmetry breaking, the Dirac, Majorana and sterile neutrino mass matrices are given as
\begin{align*}
M_D = &\ v_u \left(
\begin{array}{ccc}
 2 g_1 y_1 & y_3 (g_2-g_1) & y_2 (-g_1-g_2) \\
 y_3 (-g_1-g_2) & 2 g_1 y_2 & y_1 (g_2-g_1) \\
 y_2 (g_2-g_1) & y_1(-g_1-g_2) & 2 g_1 y_3 \\
\end{array}
\right); \\
M_R=&\ \lambda\left(
\begin{array}{ccc}
 2y_1 & -y_3 & -y_2 \\
 -y_3 & 2 y_2 & -y_1 \\
 -y_2 & -y_1 & 2 y_3 \\
\end{array}
\right);\\
M_s =&\ \delta v_{\zeta}\left(
\begin{array}{ccc}
y_1 & y_3 & y_2 \\
\end{array}
\right).
\end{align*}
where $v_{\zeta}=\langle \zeta\rangle$ is the vev of the scalar $\zeta.$ 
 
In the 3+1 MES mechanism, the $4\times 4$ active-sterile neutrino mass matrix is given by \cite{Zhang2011} 
\begin{equation}
M_{\nu}^{4\times 4} = -\left(\begin{matrix}
M_DM_R^{-1}M_D^T & M_DM_R^{-1}M_S^T \\
 
M_S(M_R^{-1})^TM_D^T & M_SM_R^{-1}M_S^T
\end{matrix} \right).
\label{4by4}
\end{equation}
Here, $det( M_{\nu}^{4\times 4}) =0$. Thus, at least one of the neutrino mass eigenvalues is zero. Applying the seesaw condition, $M_{\nu}$ is further diagonalised and the $(3\times 3)$ active neutrino mass matrix $m_{\nu}$ and the sterile neutrino mass $m_s$ are expressed as
\begin{align}
m_{\nu} \simeq &\  M_DM_R^{-1}M_S^T\left(M_S M_R^{-1}M_S^T\right)^{-1}M_S\left(M_R^{-1}\right)^T M_D^T -M_DM_R^{-1}M_D^T\ ;
\label{m} \\ 
m_s \simeq & - M_SM_R^{-1}M_S^T.
\label{ms}
\end{align}
It is important to note that the mass matrix $M_s$ is a $(1\times 3)$ matrix while $M_R$, $M_D$ are all $(3\times 3)$ matrices. The $(4\times 4)$ active-sterile mass matrix is diagonalised by a unitary $(4\times 4)$ mixing matrix given by  \cite{v441982}
\begin{equation}
U \simeq \left(\begin{matrix}
(1-\frac{1}{2}RR^{\dagger})\mathcal{U} & R \\ 
-R^{\dagger}\mathcal{U} & 1-\frac{1}{2}R^{\dagger}R
\end{matrix} \right),
\label{V44}
\end{equation}
where $\mathcal{U}$ represents the $3\times 3$ active neutrino mixing matrix and $R$ represents the strength of active-sterile mixing given by
\begin{align}
R=&M_DM_R^{-1}M_S^T(M_SM_R^{-1}M_S^T)^{-1}.
\end{align}
Deviation of $\mathcal{U}$ from unitarity due to the presence of sterile neutrino is determined by $1- \frac{1}{2}RR^{\dagger}$. The $4\times 4$ neutrino mixing matrix can be parameterized by six mixing angles $(\theta_{12},\theta_{13}, \theta_{23},\theta_{14},\theta_{24}, \theta_{34})$, three Dirac phases $(\delta_{13},\delta_{14},\delta_{24})$ and three Majorana phases $(\alpha,\beta,\gamma)$ \cite{gariazzo2016light} as,
\begin{equation}
U = R_{34}\tilde{R}_{24}\tilde{R}_{14}R_{23}\tilde{R}_{13}R_{12}.P
\label{V44p}
\end{equation}
where $R_{ij}$ are rotation matrices and P is a diagonal phase matrix.
 %\left(\begin{matrix}
%c_{12}c_{13}c_{14} & c_{13}c_{14}s_{12}e^{i\frac{\alpha}{2}} & c_{14}s_{13} e^{i\frac{\beta}{2}} & s_{14}e^{-i\frac{\gamma}{2}} \\ 
%U_{\mu 1} & U_{\mu 2} & U_{\mu 3} & c_{14}s_{24}e^{-i\left(\frac{\gamma}{2}-\delta_{14}+\delta_{24}\right)} \\ 
%U_{\tau 1} & U_{\tau 2} & U_{\tau 3} & c_{14}c_{24}s_{34}e^{-i\left(\frac{\gamma}{2}-\delta_{14}\right)} \\ 
%U_{s1} & U_{s2} & U_{s3} & c_{14}c_{24}c_{34}e^{-i\left(\frac{\gamma}{2}-\delta_{14}\right)}
%\end{matrix} \right).
%\label{V44p}
%\end{equation}

The six neutrino mixing angles can be determined from the mixing elements of $U$ using the following relations \cite{DEV2019401},
\begin{align} \label{anglessolve}
\sin^2\theta_{14}\ &=\ \vert U_{e4}\vert ^2,\ \  
\sin^2\theta_{24}\ =\ \frac{\vert U_{\mu 4}\vert ^2}{1-\vert U_{e4}\vert ^2},\ \ 
\sin^2\theta_{34}\ =\ \frac{\vert U_{\tau 4}\vert ^2}{1-\vert U_{e4}\vert ^2-\vert U_{\mu 4}\vert ^2},\nonumber \\
\sin^2\theta_{12}\ &=\ \frac{\vert U_{e2}\vert ^2}{1-\vert U_{e4}\vert ^2-\vert U_{e3}\vert ^2}, \ \ 
\sin^2\theta_{13}\ =\ \frac{\vert U_{e3}\vert ^2}{1-\vert U_{e4}\vert ^2},
\end{align}
\begin{align}
\sin^2\theta_{23}\ =&\ \frac{\vert U_{e3}\vert ^2(1-\vert U_{e4}\vert ^2)-\vert U_{e4}\vert ^2\vert U_{\mu 4}\vert ^2}{1-\vert U_{e4}\vert ^2-\vert U_{\mu 4}\vert ^2}+ \frac{\vert U_{e1}U_{\mu 1}+U_{e2}U_{\mu 2}\vert ^2(1-\vert U_{e4}\vert ^2)}{(1-\vert U_{e4}\vert ^2-\vert U_{e3}\vert ^2)(1-\vert U_{e4}\vert ^2-\vert U_{\mu 4}\vert ^2)}.\nonumber 
\end{align} 
where $U_{ij}$ are the elements of mixing matrix in eq. (\ref{V44}). We also try do determine the Jarlskog invariant for the active sterile mixing. According to the parameterisation in eq. (\ref{V44p}), the Jarlskog invariant $J_{3+1} = Im[U_{e1}U_{\mu 2}U^*_{e2}U^*_{\mu 1}]$   takes the form  \cite{KUMAR2020115082}
\begin{equation}
J_{3+1} = J_3^{cp} c_{14}^2c_{24}^2 + s_{24}s_{14}c_{24}c_{23}c^2_{14}c^3_{13}c_{12}s_{12}\sin(\delta_{14}-\delta_{24}),
\label{deltasolve}
\end{equation}
where $J_3^{cp} = s_{23}c_{23}s_{12}c_{12}s_{13}c_{13}^2 \sin \delta_{13}$ is the Jarlskog invariant for the three neutrino framework and $s_{ij} = \sin\theta_{ij},c_{ij}=\cos\theta_{ij}$ are the mixing angles. The two physical Majorana phases $\alpha$ and $\beta$ are determined from $U$ using the invariants $I_1$ and $I_2$ defined as follows 
\begin{align}\label{majoranaphase1}
I_1 = Im[U_{e1}^*U_{e2}]\ =\ c_{12} c_{13}^2 c_{14}^2 s_{12} \sin\left(\frac{\alpha}{2}\right), \\ 
I_2 = Im[U_{e1}^*U_{e3}]\ =\ c_{12}c_{13}c_{14}^2s_{13}\sin\left(\frac{\beta}{2}-\delta_{13}\right).
\label{majoranaphase2}
\end{align}

Other important parameters in neutrino physics are the effective neutrino mass $m_{\beta\beta}$ and effective electron neutrino mass $m_{\beta}$. A combined analysis from KAMLand-Zen \cite{Kamland} and GERDA provided an upper bound on $m_{\beta\beta}$ in the range $m_{\beta\beta} < (0.071-0.161$) eV \cite{agostini2018improved,goswami}. Recent result from the KATRIN-2020 \cite{aker2020first} experiment  constrains the effective electron neutrino mass $m_{\beta}$ to be less than 1.1 eV. However, this upper bound has been updated to $m_{\beta}<0.8$ eV$c^{-2}$ in the latest results of KATRIN-2022  \cite{katrin2022direct}. These parameters are determined from the neutrinoless double beta decay and tritium beta decay respectively using the relations \cite{Hagstot}
\begin{equation}
m_{\beta\beta}=\vert\sum_{j=1}^4\vert U_{ej}\vert ^2m_j\vert,
\label{mbbeq}
\end{equation}

\begin{equation}
m_{\beta} = \left(\sum_{i=1}^4\vert U_{ei}\vert ^2 m_{i}^2\right)^{1/2}.
\label{mbeq}
\end{equation}

\begin{table}
\begin{center}
\renewcommand{\arraystretch}{1}
\begin{tabular}{c|c|c}
 \hline 
Parameter &	Normal Hierarchy (best-fit$\pm 1\sigma$) &	Inverted Hierarchy (best-fit$\pm 1\sigma$)  \\
\hline
$\vert\Delta m^2_{21}\vert: [10^{-5} eV^2]$ & 6.82 – 8.03 $(7.41^{+0.21}_{-0.20})$  &	 6.82 – 8.03 $(7.41^{+0.21}_{-0.20})$ \\
$\vert\Delta m^2_{31}\vert: [10^{-3} eV^2]$	& 2.428 – 2.597 $(2.511^{+0.028}_{-0.027})$  & 2.408 – 2.581 $(2.498^{+0.032}_{-0.025})$ \\

$\sin^2\theta_{12} $	& 0.270 – 0.341 $(0.303^{+0.012}_{-0.011})$ &  0.270 – 0.341 $(0.303^{+0.012}_{-0.011})$  \\
$\sin^2\theta_{23}$ &0.406 – 0.620	$(0.572^{+0.018}_{-0.023})$  &0.412 – 0.623 $(0.578^{+0.016}_{-0.021})$ 		 \\
$\sin^2\theta_{13}/10^{-2}$ & 2.029 – 2.391	$(2.203^{+0.056}_{-0.059})$ & 2.047 – 2.396 $(2.219^{+0.060}_{-0.057})$  \\
$\delta_{\rm CP}/^o$ &	108 - 404 $(197^{+42}_{-0.25})$	& 192 - 360 $(286^{+27}_{-32})$	 \\
$r=\sqrt{\frac{\Delta m_{21}^2}{\vert\Delta m_{3l}^2\vert}} $ & 0.1675 - 0.1759 (0.1718)  & 0.1683 - 0.1765 (0.1722)\\
$\vert U_{14}\vert^2 $  & 0.012 - 0.047 & 0.012 - 0.047 \\
$\vert U_{24}\vert^2 $  & 0.005 - 0.03 &  0.005 - 0.03 \\
$\vert U_{34}\vert^2 $  & 0 - 0.16 & 0 - 0.16 \\
\hline
\end{tabular}  
\end{center}
\caption{Updated global-fit data for three neutrino oscillation, $Nufit$ 2022 \cite{nufit}. For 3+1 mixing, data taken from  \cite{barrylight,vien2022b,Gariazzo_2016}.}
\label{data}
\end{table} 

\section{Numerical analysis}\label{section3}
In this section, we describe the steps of the detailed numerical analysis to determine the allowed regions of the free parameters in the model, which satisfy the current neutrino oscillation data. The active neutrino mass matrix in eq. (\ref{m}) depends on the parameters $g_2/g_1, \lambda $ and $\tau.$ On fixing $\tau$, the modular symmetry is broken and the neutrino mass eigenvalue, mixing angles and Dirac as well as Majorana phases are completely determined. The absolute scale of active neutrino masses can be fixed by adjusting the overall factor $v_u^2g_1/\lambda$ in $m_{\nu}.$ We define the complex parameters $\tau$ and $g_2/g_1$ as 

\begin{equation}
\tau=Re[\tau] + i\ Im[\tau],\ \  \frac{g_2}{g_1}= ge^{i \phi_g} 
\end{equation} 

The fundamental domain of $\tau$ is given in Ref. \cite{Feruglio}. We randomly scan the free parameters in the following ranges
\begin{align}
&Re[\tau]=[-1.5,1.5],\ \ \ \  \ \ Im[\tau] = [0.6,1.5],\ \ \ \ \ \ \ g = [1,2],\nonumber \\ 
&\lambda =[10^{14},10^{16}]\ \mbox{GeV},\ \ \ \ \ \ \  \ \phi_g = [-\pi,\pi].
\end{align}
The active neutrino mass matrix $m_{\nu}$ is numerically diagonalised using the relation $m_{\nu}= \mathcal{U} diag(m_1,m_2,m_3)\mathcal{U}^{\dagger}$, where $\mathcal{U}$ is a unitary $3\times 3$ matrix and $m_1,m_2,m_3$ are the active neutrino mass eigenvalues. We constrain the values of the model parameters using the 3$\sigma$ bounds of the three mixing angles $\sin^2\theta_{12},\sin^2\theta_{13},\sin^2\theta_{23}$ and the ratio of neutrino mass squared differences $r= \sqrt{\Delta m_{21}^2/\Delta m_{31}^2}= m_2/m_3$ for NH $(m_1 \approx 0<< m_2 < m_3<m_4)$ and $ r=\sqrt{\Delta m_{21}^2 /\vert\Delta m_{32}^2\vert}= \sqrt{1-m_1^2/m_2^2}$ for IH $(m_3\approx0<< m_2<m_1<m_4)$ given in Table \ref{data}. For the sterile sector, the coefficient $k \equiv\delta v_{\zeta}$ is solved by constraining the sterile neutrino mass $m_s$ in the range $[0.8, 10]$eV.
\begin{figure}[htp]
\subfigure[]{
\includegraphics[width=.45\textwidth]{imagesNH/gvsReImtauNH.pdf}}
\quad
\subfigure[]{
\includegraphics[width=.45\textwidth]{imagesIH/gvsreimtau.pdf}}
\quad
\caption{\footnotesize{Variation plots showing the allowed ranges of Re[$\tau$] and Im[$\tau$] with $\vert g\vert$ for NH in (a) and IH in (b).  }}
\label{gplot}
\end{figure}

\begin{figure}
\subfigure[]{
\includegraphics[width=.45\textwidth]{imagesNH/y1y2y3vsImtauNH.pdf}}
\quad
\subfigure[]{
\includegraphics[width=.45\textwidth]{imagesNH/y1y2vsy3.pdf}}
\quad
\caption{\footnotesize{(a).Dependence of Yukawa couplings    $y_1,y_2,y_3$ with Im[$\tau$] for NH. (b).Variation plot showing the dependence of Yukawa couplings among themselves for NH.}}
\label{yplotnh}
\end{figure}

The allowed ranges of Re$[\tau]$, Im$[\tau]$ and $g$ for both NH and IH are shown in Fig. \ref{gplot}. The ranges of Yukawa couplings $\vert y_1\vert,\vert y_2\vert$ and $\vert y_3\vert$, determined from these values are shown as variation plots in Fig. \ref{yplotnh} and Fig. \ref{yplotih} for NH and IH respectively. It is observed that the couplings lie in the ranges $0.984\leq \vert y_1(\tau)\vert\leq 0.989$, $ 0.593\leq \vert y_2(\tau)\vert\leq 0.664$, $ 0.178\leq \vert y_3(\tau)\vert\leq 0.223$ for NH. The values of Im$(\tau)$ vary in the range $(1.047 - 1.101)$ while the values of Re$(\tau)$ are concentrated at specific regions around $\pm 0.5$ and $\pm 1.5$ in the fundamental domain of $\tau$ as evident from Fig. {\ref{gplot}(a). Whereas, in the case of IH, the observed values of couplings are found as $0.948 \leq \vert y_1(\tau)\vert\leq 1.016 $, $0.847 \leq \vert y_2(\tau)\vert\leq 1.142$ and  $0.367 \leq \vert y_3(\tau)\vert\leq 0.648$. It is important to note that $Im[\tau]$ in IH is found to in a very narrow range $(0.793, 0.928)$, which was not allowed by NH.

\begin{figure}[htp]
\subfigure[]{
\includegraphics[width=.45\textwidth]{imagesIH/y1y2y3vsimtau.pdf}}
\quad
\subfigure[]{
\includegraphics[width=.45\textwidth]{imagesIH/y3vsy1y2.pdf}}
\quad
\caption{\footnotesize{(a).Dependence of Yukawa couplings    $y_1,y_2,y_3$ with Im[$\tau$] for IH. (b).Variation plot showing the dependence of Yukawa couplings among themselves for IH.}}
\label{yplotih}
\end{figure}

For the charged lepton sector, by comparing eq. (\ref{ml}) with the experimental values for masses of the charged leptons given in Ref. \cite{pdg2022}, $m_e = 0.51099$ MeV, $m_{\mu} = 105.65837$ MeV and $m_{\tau}= 1776.86$ MeV, the parameters $\alpha^{\prime}, \beta^{\prime}, \gamma^{\prime}$ are determined to be 
\begin{equation}
\frac{\alpha^{\prime}}{\gamma^{\prime}} = 0.287\times 10^{-2},\ \  \ \ \ \frac{\beta^{\prime}}{\gamma^{\prime}} = 0.059.
\end{equation}

\begin{figure}
\subfigure[]{
\includegraphics[width=.45\textwidth]{imagesNH/anglesNH.pdf}}
\quad
\subfigure[]{
\includegraphics[width=.45\textwidth]{imagesNH/sumvsm2m3NH.pdf}}
\quad
\caption{\footnotesize{(a). Plot between mixing angle $\sin^2 \theta_{13}$ with $\sin^2 \theta_{23}$ and $\sin^2 \theta_{12}$ for NH (b). Variation of active neutrino masses $m_2$ and $m_3$ ($m_1=0$) with $\sum m_i$ for NH.   }}
\label{anglesnh}
\end{figure}
\begin{figure}
\subfigure[]{
\includegraphics[width=.45\textwidth]{imagesNH/si4vskNH.pdf}}
\quad
\subfigure[]{
\includegraphics[width=.45\textwidth]{imagesIH/si4vsk.pdf}}
\quad
\caption{\footnotesize{ Variation between active-sterile mixing angles $\sin^2 \theta_{i4}$ where $i=1,2,3$ with sterile neutrino mass coefficient $k\equiv \delta v_{\zeta}$ for NH in (a) and for IH in (b).}}
\label{si4nh}
\end{figure}

\section{Results of the analysis}\label{section4}
In this section we shall show the results of the numerical analysis discussed in the previous section. We have divided the analysis of the results into two subsections : Normal hierarchy (NH) and Inverted hierarchy (IH).
\subsection{ Normal hierarchy(NH)}

\subsubsection{Mixing angles and active neutrino masses}
For NH, the predicted values of neutrino mixing angles $\sin^2\theta_{13}$ are plotted with $\sin^2\theta_{12}$ and $\sin^2\theta_{23}$ in Fig. \ref{anglesnh}(a). From these plots, we observe that the values of $\sin^2\theta_{23}$ are concentrated at regions greater than $0.55$ implying the higher octant of $\theta_{23}$. This result is consistent with the latest improved measurement of neutrino oscillation parameters by the NOvA experiment \cite{acero2022improved}. Plots between individual neutrino masses $m_2, m_3$  ($m_1=0$) with sum of active neutrino masses $\sum m_{i}$  are shown in Fig. \ref{anglesnh}(b). Choosing the factor $v_u^2g_1/\lambda =0.01$ eV, we observe that the sum of active neutrino masses $\sum m_{i}$ satisfies the cosmological Planck upper bound  $\sum m_{i}<0.12$ eV \cite{collaboration2020n}. It is observed in a very narrow range of $\sum m_{i} \sim (0.015 - 0.044 )$ eV. The individual neutrino masses are found in the ranges $ m_2 \sim (0.0022 - 0.0066)$ eV, $ m_3 \sim (0.0130 - 0.0379)$ eV. For the mixings between active and sterile neutrino, we have plotted the active-sterile mixing angles $\sin^2\theta_{i4}$ as a function of parameter $k$ in Fig. \ref{si4nh}(a). The non-unitarity effect due to the presence of active-sterile mixing is determined by $\frac{1}{2}RR^{\dagger}$. For our analysis it is observed to be in the range $\mathcal{O}(10^{-4} -10^{-1}).$

\subsubsection{Jarlskog invariant and phases}
Proceeding further, we have calculated the model prediction of the Jarlskog invariant in the 3+1 scenario. Fig. \ref{J} (a) shows the variation of mixing angles with $J_{3+1}$ for NH. Dirac CP-violating phase $\delta_{13}$ is solved from eq. (\ref{deltasolve}). In this relation, we have randomly varied the unknown phases for active-sterile mixing $\delta_{14}$ and $\delta_{24}$ in the range $[-\pi,\pi]$. Majorana phases $\alpha$ and $\beta$ are also evaluated using eq. (\ref{majoranaphase1}) - (\ref{majoranaphase2}).  The predictions of $\alpha$ and $\beta$ from our model are shown in Fig. \ref{phasenh}(a)(b) as variation plots with Dirac CP-violating phase $\delta_{13}$. From Fig. \ref{deltaphig}(a) which shows the correlation of $\delta_{13}$ with the phase of Yukawa coefficient $\phi_g$ in NH, it is observed that $\phi_g$ is concentrated around $\pm 3 \pi$. Variation between the invariants $I_1$ and $I_2$ defined in eq. (\ref{majoranaphase1}) are plotted in Fig. \ref{J}(b).

\begin{figure}
\subfigure[]{
\includegraphics[width=.45\textwidth]{imagesNH/J3+1NH.pdf}}
\quad
\subfigure[]{
\includegraphics[width=.45\textwidth]{imagesNH/I1vsI2NH.pdf}}
\quad
\caption{\footnotesize{(a). Plot between Jarlskog invariant $J_{3+1}$ in the 3+1 sector with active neutrino mixing angles $\sin^2 \theta_{12}$ and $\sin^2 \theta_{23}$. (b). Variation between the invariants $I_1$ and $I_2$ for NH.}}
\label{J}
\end{figure}

\subsubsection{Variation of $m_{\beta\beta}$ and $m_{\beta}$ with $\sum m_i$ for NH }
Phenomenological studies on beta decay and neutrinoless double beta decay experiments are also carried out. The effective mass parameter $m_{\beta\beta}$ and effective electron neutrino mass $m_{\beta}$ are determined using eq. (\ref{mbbeq})-(\ref{mbeq}). Fig. \ref{mbnh}(a) shows the variation of $m_{\beta\beta}$ as a function of $\sum m_{i}$ for NH. The blue shaded region is disallowed by Planck-2018 data \cite{collaboration2020n} ($\sum m_i <0.12$eV) and the region above horizontal yellow strip ( due to NME uncertainty \cite{engel2017status,kotila2012phase}) is excluded by combined results from KamLAND-Zen and GERDA \cite{agostini2018improved} experiments. The red data points are predictions of $m_{\beta\beta}$ in the 3-$\nu$ mixing i.e. the effect of active-sterile mixing $\vert U_{14}\vert$ and $m_4$ is not considered. This region is observed beyond the future sensitivities of experiments such as nEXO  \cite{kharusi2018nexo}, Legend-1K \cite{abgrall1894legend,agostini2017discovery}, etc. As a result, it will be very hard to detect such signals. If we include the effects of sterile neutrino mixing, our model predicts an improved region, shown as black points, where some of the data points are observed within the future sensitivity limits. Fig. \ref{mbnh}(b) shows the variation of $m_{\beta}$ with $\sum m_i$. The upper green panel is the region excluded by the KATRIN-2020 experiment \cite{aker2020first} ($m_{\beta}<1.1$eV). This upper bound is updated at $m_{\beta}<0.8$ eV$c^{-2}$ in the latest results of KATRIN- 2022 \cite{katrin2022direct}. In this plot also, in the case of 3-$\nu$ mixings scheme, predictions lie well beyond reach of future sensitivity of Project 8 \cite{project8} experiment ($m_{\beta}\sim 0.04$eV).  However, the region gets enhanced within this sensitivity if sterile neutrino mixing in 3+1 scheme is taken into account. This means that the existence of sterile neutrinos for NH in eV scale will be highly disfavoured if Project 8 does not observe any signal in the future run. In our analysis, it is observed that the effective mass parameters lie in the range $0.03$ eV$ \leq m_{\beta}\leq 1.22$ eV and $4.82$ meV $\leq m_{\beta\beta}\leq 181.43$ meV for NH.


\begin{figure}
\subfigure[]{
\includegraphics[width=.45\textwidth]{imagesNH/deltaalphaNH.pdf}}
\quad
\subfigure[]{
\includegraphics[width=.45\textwidth]{imagesNH/deltabetaNH.pdf}}
\quad
\caption{\footnotesize{Phase plots for NH. (a) Variation of CP-violating Dirac phase $\delta_{13}$ with  Majorana phases $\alpha$ and (b) shows the plot with $\beta$.}}
\label{phasenh}
\end{figure}

\begin{figure}
\subfigure[]{
\includegraphics[width=.45\textwidth]{imagesNH/sumvsmbbNH.pdf}}
\quad
\subfigure[]{
\includegraphics[width=.45\textwidth]{imagesNH/sumvsmbNH.pdf}}
\quad
\caption{\footnotesize{ Dependence of effective masses $m_{\beta}$ and  $m_{\beta\beta}$ with  $\sum m_i$ for NH.}}
\label{mbnh}
\end{figure}


\begin{figure}
\subfigure[]{
\includegraphics[width=.45\textwidth]{imagesNH/deltaphigNH.pdf}}
\quad
\subfigure[]{
\includegraphics[width=.45\textwidth]{imagesIH/deltaphig.pdf}}
\quad
\caption{\footnotesize{Variation of CP-violating Dirac phase $\delta_{13}$ with phase of Yukawa coefficient $\phi_g$ for NH in (a) and IH in (b).}}
\label{deltaphig}
\end{figure}

\subsection{Inverted hierarchy(IH)}
\subsubsection{Mixing angles and neutrino masses}
Active neutrino mixing angles $\sin^2\theta_{12},\sin^2\theta_{23}$ with $\sin^2\theta_{13}$ for IH are shown in Fig. \ref{anglesih}(a). The minimum value of $\sin^2\theta_{23}$ is observed at 0.522. This result implies that higher octant of $\theta_{23}$ is also favoured in IH similar to NH. Taking $v_u^2g_1/\lambda =  0.01$ eV, we have evaluated the absolute scales of the active neutrino masses $m_1,m_2$ while the lightest $m_3=0$ in MES mechanism. Plot between active neutrino masses $m_1,m_2$ with $\sum m_i$ is shown in Fig. \ref{anglesih}(b). From this analysis, as constrained by $\sum m_i<0.12$ eV, we can put an upper bound on the individual neutrino masses. The upper and lower bounds of the  active neutrino masses are obtained in the ranges $0.017\ \mbox{eV}\leq m_1\leq 0.059\ \mbox{eV},\ 0.018\ \mbox{eV} \leq m_2\leq 0.060\ \mbox{eV} $ while $0.035\ \mbox{eV}\leq \sum m_i\leq 0.119\ \mbox{eV}$. The active-sterile mixing angles  $\sin^2\theta_{i4}$, ($i=1,2,3$) for IH are shown in Fig. \ref{si4nh}(b).

\begin{figure}
\subfigure[]{
\includegraphics[width=.45\textwidth]{imagesIH/angleplot.pdf}}
\quad
\subfigure[]{
\includegraphics[width=.45\textwidth]{imagesIH/massplot.pdf}}
\quad
\caption{\footnotesize{ (a). Plot between mixing angle $\sin^2 \theta_{13}$ with $\sin^2 \theta_{23}$ and $\sin^2 \theta_{12}$ for IH (b). Variation of active neutrino masses $m_2$ and  $\sum m_i$ ($m_3=0$) with $m_2$ for IH.}}
\label{anglesih}
\end{figure}

\subsubsection{Jarlskog invariant and phases}
The plot of active neutrino mixing angles with Jarlskog invariant $J_{3+1}$ for IH is shown in Fig. \ref{Jih}(a). The other invariants $I_1$ and $I_2$ are plotted in Fig. \ref{Jih}(b).  The values of $J_{3+1}$ are observed in the range between $\pm 0.06$ which is out of the bounds for $J_{max}^{CP} = 0.0332 \pm 0.0008$ at 1$\sigma$  for both orderings \cite{nufit}. The Dirac CP-violating phase $\delta_{13}$ and Majorana phases $\alpha, \beta$ are determined from eq. (\ref{majoranaphase1}). In this calculation, we have excluded the regions of $J_{3+1}$ outside $\pm 0.0332.$ The variation of the phases are shown in Fig. \ref{phasesih}. The Dirac phase is  predicted in the ranges $\delta_{13} \sim (180.04^o-249.56^o)$ and $\delta_{13} \sim (285.13^o-359.67^o)$.

\begin{figure}
\subfigure[]{
\includegraphics[width=.45\textwidth]{imagesIH/jvss23.pdf}}
\quad
\subfigure[]{
\includegraphics[width=.45\textwidth]{imagesIH/i1i2plot.pdf}}
\quad
\caption{\footnotesize{ (a). Plot between Jarlskog invariant $J_{3+1}$ with active neutrino mixing angles $\sin^2 \theta_{12}$ and $\sin^2 \theta_{23}$ for IH. (b). Variation of invariants $I_1$ and $I_2$ for IH.}}
\label{Jih}
\end{figure}
\begin{figure}
\subfigure[]{
\includegraphics[width=.45\textwidth]{imagesIH/alphadelta.pdf}}
\quad
\subfigure[]{
\includegraphics[width=.45\textwidth]{imagesIH/betadelta.pdf}}
\quad
\caption{\footnotesize{ Plot between the Dirac and Majorana phases for IH. }}
\label{phasesih}
\end{figure}

\subsubsection{Variation of $m_{\beta\beta}$ and $m_{\beta}$ with $\sum m_i$ for IH }
Similarly, for the case of IH, we have plotted $m_{\beta}$ and $m_{\beta\beta}$ as a function of $\sum m_i$ as shown in Fig. \ref{mbih}. In 3-$\nu$ mixing, most of the regions of $m_{\beta}$ allowed by $\sum m_i$ of Planck-2018 data are available beyond the sensitivity of Project 8 experiment ($m_{\beta} \sim 0.04$ eV) represented by horizontal dotted line and future sensitivity of KATRIN experiment ($m_{\beta}\sim 0.02$ eV) represented by the horizontal solid green line. However, a very small region which is allowed by the cosmological bound is available within the project 8 sensitivity. If it detects no signal in future, IH will be disfavoured for 3$\nu$ mixings scheme in our model. For the case of 3+1 mixing, all the regions of $m_{\beta}$ are obtained inside sensitivity of Project 8. As a result, data from future run of Project 8 experiment may be able to decide the fate of sterile neutrino and 3+1 mixings in IH. Therefore, based on our analysis, we can infer that the future results of $m_{\beta}$ experiments may completely prove or disprove the existence of an eV-scale sterile neutrino in both orderings of neutrino mass. The effective mass parameter is observed in the range  $0.043$ eV$ \leq m_{\beta}\leq 1.76$ eV.

For $0\nu\beta\beta,$ the plot between $m_{\beta\beta}$ and $\sum m_i$ is shown in Fig. \ref{mbih}(a). The effective neutrino mass parameter is in the range  $0.54$ meV $\leq m_{\beta\beta}\leq 345.53$ meV. Many data points of $m_{\beta\beta}$ allowed by cosmological constraints, are observed within the reach of nEXO and Legend-1K. Detection or non-detection such signals in future may be used to distinguish between NH and IH. However, it may not be conclusive since there are still some regions outside these sensitivities. The major difference between NH and IH is the enhancement of $m_{\beta}$ and $m_{\beta\beta}$ in IH for both 3-$\nu$ mixings and 3+1 mixings. In the IH case of 3+1 scheme, most of the predictions of $m_{\beta\beta}$ can be probed by these experiments. 

\begin{figure}
\subfigure[]{
\includegraphics[width=.45\textwidth]{imagesIH/mbbvsmi.pdf}}
\quad
\subfigure[]{
\includegraphics[width=.45\textwidth]{imagesIH/mbvsmi.pdf}}
\quad
\caption{\footnotesize{ Dependence of effective masses $m_{\beta\beta}$ and  $m_{\beta}$ with  $\sum m_i$ for IH. }}
\label{mbih}
\end{figure}

\begin{figure}
\subfigure[]{
\includegraphics[width=.45\textwidth]{imagesNH/lambdavsmbNH.pdf}}
\quad
\subfigure[]{
\includegraphics[width=.45\textwidth]{imagesNH/lambdavsmbbNH.pdf}}
\quad
\caption{\footnotesize{Variation of $m_{\beta}$ and $m_{\beta\beta}$ with $\lambda$ for NH. In these plots, the black points represent the parameters in 3+1 mixings while the red points represent 3$\nu$ mixings.}}
\label{mblambda}
\end{figure}

\subsection{Constrains on $\lambda$}
The active-sterile mixing elements are related to $\lambda$ through $k$ as $\vert U_{i4}\vert \sim v_u g/\sqrt{m_s \lambda}$. Consequently therefore, the magnitudes of effective mass parameters $m_{\beta}$ and $m_{\beta\beta}$ depend on $\lambda$ as evident from eq. (\ref{mbbeq}) and eq. (\ref{mbbeq}). In our analysis, we also deduce an estimate on the range of $\lambda$ through the calculated results from the model and the experimental bounds of $\vert U_{i4}\vert^2, m_{\beta\beta}$ and $m_{\beta}$. The variation of these parameters are shown in Fig. \ref{mblambda} for NH and Fig. \ref{mblambdaih} for IH. The dependence of $\vert U_{14}\vert^2$ on $\lambda$ is shown in Fig. \ref{ui4lambda} for both orderings. Based on the regions excluded by experimental bounds these parameters, $\lambda<10^{14}$ GeV is excluded in the 3+1 scheme. Whereas, the upper bound on $\lambda$ is provided by $\vert U_{14}\vert^2$ in Fig. \ref{ui4lambda}(a) for NH. Here, $\lambda>1.7\times 10^{15}$ GeV fails to satisfy the experimental bounds of $\vert U_{14}\vert^2$ indicated by the horizontal black lines. This allowed range of $\lambda$ is expanded in case of IH, as shown in Fig. \ref{ui4lambda}(b). In this plot, we can observe some values of $\lambda$ between $2.1\times^{14}$ GeV and $6.2\times 10^{15}$ GeV, allowed by bounds of $\vert U_{14}\vert^2$. As we can observe, there is no dependence of $\lambda$ on $m_{\beta\beta}$ and $m_{\beta}$ in the 3$\nu$ scheme represented by the red data points in these plots.

\begin{figure}
\subfigure[]{
\includegraphics[width=.45\textwidth]{imagesIH/mbvslambda.pdf}}
\quad
\subfigure[]{
\includegraphics[width=.45\textwidth]{imagesIH/mbbvslambda.pdf}}
\quad
\caption{\footnotesize{Variation of $m_{\beta}$ and $m_{\beta\beta}$ with $\lambda$ for IH. In these plots, the black points represent the parameters in 3+1 mixings while the red points represent 3$\nu$ mixings.}}
\label{mblambdaih}
\end{figure}

\begin{figure}
\subfigure[]{
\includegraphics[width=.45\textwidth]{imagesNH/u14vslambdaNH.pdf}}
\quad
\subfigure[]{
\includegraphics[width=.45\textwidth]{imagesIH/u14vslambda.pdf}}
\quad
\caption{\footnotesize{  Plot between active-sterile mixing  element $\vert U_{14}\vert^2$ with $\lambda$ for NH in (a) and IH in (b).}}
\label{ui4lambda}
\end{figure}



\subsection{$\chi^2$ analysis}
Finally, the best-fit values of the neutrino observables and the corresponding best-fit values of model parameters $\tau$ and $g$ are evaluated using the $\chi^2$ analysis. We use the $\chi^2$ function defined as 
\begin{equation}
\chi^2(x_i) = \sum_{j}\left(\frac{y_j(x_i)-y_j^{bf}}{\sigma_j}\right)^2
\label{chitest}
\end{equation}
where $x_i$ are the free parameters in the model and $j$ is summed over the observables $\{\sin^2\theta_{12},\sin^2\theta_{13},\sin^2\theta_{23},r\}$. Here, $y_j(x_i)$ denotes the model predictions for the observables and $y_j^{bf}$ are their best-fit values obtained from the global analysis. $\sigma_j$ denotes the corresponding uncertainties obtained by symmetrizing $1\sigma$ range of the neutrino observables given in Table \ref{data}. By minimizing the $\chi^2$ function, we can calculate the best-fit values of our model parameters and predict the values of neutrino observables. 

For NH at $\chi^2_{min}=0.74$, the best-fit of the parameter $\tau$ is obtained at Re$[\tau]= 0.432$, Im$[\tau]= 1.076$ and $g= 1.534$. The corresponding values of Yukawa couplings are found to be $\vert y_1\vert=0.987,\vert y_2\vert=0.624,\vert y_3\vert=0.197$. The best-fit values of neutrino observables are obtained at $\sin^2\theta_{23}=0.581$, $ \sin^2\theta_{12}=0.309$, $\sin^2\theta_{13}=0.022$ and $r = 0.172$. The best-fit Dirac and Majorana phases are observed at $\delta_{13}=358.41^o,\ \alpha=245.98^o$ and $ \beta=245.77^o$. Similarly, for IH at $\chi^2_{min}=1.28$, the values of the model parameters are found at Re$[\tau]= -0.77$, Im$[\tau]= 0.81$ and $g= 1.14$, while the Yukawa couplings are found to be $\vert y_1\vert=1.01,\vert y_2\vert=1.11$ and $\vert y_3\vert=0.61$. The corresponding best-fit values of the neutrino observables are $\sin^2\theta_{23}=0.582,$ $\sin^2\theta_{12}=0.301,$ $ \sin^2\theta_{13}=0.022$ and $r = 0.171$. Lastly, Dirac and Majorana phases are observed at $\delta_{13}=346.78^o,\ \alpha=14.53^o$ and $\beta=321.62^o$.

\section{Summary and Discussion}\label{section6}
We have successfully constructed a new neutrino mass model based on modular $A_4$ by extending the SM with an $A_4$ triplet right-handed neutrino $N_i$ and a singlet sterile neutrino $S$ in the 3+1 scheme. MES mechanism is used to generate the active and eV-scale sterile neutrino masses in both NH and IH. Our main motivation is to avoid the hypothetical scalar flavons using modular symmetry and at the same time, to reproduce all the neutrino observables through the vev of a single parameter $\tau,$ for a particular range of Yukawa coefficient factor $g.$ However, to simplify our analysis, we have used a triplet scalar $\phi$ whose vev makes the charged lepton mass matrix diagonal.  Two free model parameters $\tau$ and $g$ are scanned randomly in a particular  domain and active neutrino mass matrix is numerically diagonalised.  We have conducted the numerical analysis using the 3$\sigma$ bounds of neutrino observables in such a way that all the neutrino observables evaluated from the model, simultaneously satisfy these bounds.  Our analysis on neutrino masses is also consistent with the cosmological upper bound on the sum of neutrino masses $\sum m_i <0.12$ eV. Effects of the eV-scale sterile neutrino on neutrino mixing angles, effective mass parameters $m_{\beta}, m_{\beta\beta}$ and unitarity of the active neutrino mixing matrix, are studied. Neutrino mixing angles are evaluated from the $(4\times 4)$ active-sterile mixing matrix $U$ by including the non-unitarity effects of sterile neutrino. Jarlskog invariant in the 3+1 sector is also determined. CP-violating Dirac phase is successfully predicted for NH in the ranges $\delta_{13} \sim (180.12^o - 230.70^o)$  and $\delta_{13} \sim (305.15^o - 358.83^o)$.

From the analysis of $m_{\beta\beta}$ and $m_{\beta}$, it is observed that the inclusion of mixing with sterile neutrino enhances the results within the future sensitivities of various experiments such as KamLAND-Zen, GERDA, Project 8, KATRIN, Legend-1K, nEXO, etc. If these experiments fail to detect any signal, the existence of eV scale sterile neutrino will be disfavoured. Numerical results of $m_{\beta}$ and $ m_{\beta\beta}$ in NH are observed to be comparatively larger than IH. This fact may be employed to discriminate between NH and IH in future experiments. As constrained by $\vert U_{14}\vert^2$, $m_{\beta}$ and $m_{\beta\beta}$, the bounds on Majorana mass scale $\lambda$ is obtained in the range $1.0\times 10^{14}$ GeV $\leq \lambda \leq 1.7\times 10^{15}$ GeV for NH case. In case of IH, $\lambda$ is constrained in the range $2.1\times 10^{14}$ GeV $\leq \lambda \leq 6.2\times 10^{15}$ GeV.

Finally, we have used the minimum $\chi^2$ analysis to predict the best-fit values of the model parameters as well as the neutrino oscillation data. Our model predicts higher octant of $\theta_{23}$ in both NH and IH. This result is in agreement  with the latest results from improved NOVA experiment  \cite{acero2022improved}. The MES mechanism has an advantage in generating keV-MeV scale sterile neutrino along with the active neutrino mass in the eV scale. Study on the possibility of keV-MeV sterile neutrino as a dark matter candidate, will be addressed in future communication. To conclude, we emphasize that modular $A_4$ symmetry is very successful in reproducing neutrino phenomenology and other issues which are beyond Standard Model, without the need for extra flavons as in conventional discrete symmetry models.


\section*{Acknowledgements}
One of the authors(MKS) would like to thank DST-INSPIRE, govt. of India for providing the fellowship for the research under INSPIRE fellowship (ID IF180349).

\bibliographystyle{JHEP}
\bibliography{modulara4}
\end{document}
