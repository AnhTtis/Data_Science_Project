%%%%%%%%%%%%%%%%%%%%%%%%%%%%%%%%%%%%%%%%%%%%%%%%%%%%%%%%%%%%%%%%%%%%%%%%
%    INSTITUTE OF PHYSICS PUBLISHING                                   %
%                                                                      %
%   `Preparing an article for publication in an Institute of Physics   %
%    Publishing journal using LaTeX'                                   %
%                                                                      %
%    LaTeX source code `ioplau2e.tex' used to generate `author         %
%    guidelines', the documentation explaining and demonstrating use   %
%    of the Institute of Physics Publishing LaTeX preprint files       %
%    `iopart.cls, iopart12.clo and iopart10.clo'.                      %
%                                                                      %
%    `ioplau2e.tex' itself uses LaTeX with `iopart.cls'                %
%                                                                      %
%%%%%%%%%%%%%%%%%%%%%%%%%%%%%%%%%%
%
%
% First we have a character check
%
% ! exclamation mark    " double quote  
% # hash                ` opening quote (grave)
% & ampersand           ' closing quote (acute)
% $ dollar              % percent       
% ( open parenthesis    ) close paren.  
% - hyphen              = equals sign
% | vertical bar        ~ tilde         
% @ at sign             _ underscore
% { open curly brace    } close curly   
% [ open square         ] close square bracket
% + plus sign           ; semi-colon    
% * asterisk            : colon
% < open angle bracket  > close angle   
% , comma               . full stop
% ? question mark       / forward slash 
% \ backslash           ^ circumflex
%
% ABCDEFGHIJKLMNOPQRSTUVWXYZ 
% abcdefghijklmnopqrstuvwxyz 
% 1234567890
%
%%%%%%%%%%%%%%%%%%%%%%%%%%%%%%%%%%%%%%%%%%%%%%%%%%%%%%%%%%%%%%%%%%%
%
\documentclass[12pt]{iopart}
\newcommand{\gguide}{{\it Preparing graphics for IOP Publishing journals}}
%Uncomment next line if AMS fonts required
%\usepackage{iopams} 
\usepackage{graphicx}
%\usepackage{amsmath}
\usepackage{amsfonts}
%\eqnobysec
\usepackage[labelformat=simple]{subcaption}
%\usepackage{url}
\usepackage{hyperref}
\renewcommand\thesubfigure{(\alph{subfigure})}
%\usepackage{subfigure}
\usepackage{array}
\newcolumntype{M}[1]{>{\centering\arraybackslash}m{#1}}

\begin{document}

\title{Modular $A_4$ symmetry in 3+1 active-sterile neutrino masses and mixings}

\author{Mayengbam Kishan Singh$^1$, S. Robertson Singh$^{1,2}$, N. Nimai Singh$^{1,2}$}

\address{$^1$ Department of Physics, Manipur University, Imphal, India - 795003}
\address{$^2$ Research Institute of Science and Technology, Imphal, India - 795003}
\ead{kishan@manipuruniv.ac.in}
\vspace{10pt}
%\begin{indented}
%\item[]February 2024
%\end{indented}

\begin{abstract}
Motivated by the significance of modular symmetry in generating neutrino masses and flavor mixings, we apply the modular $A_4$ symmetry in a 3+1 scheme of active-sterile neutrino mixings. Neutrino oscillation observables in the 3$\sigma$ range are successfully reproduced through the vacuum expectation value of the modulus $\tau$ in the Inverted Hierarchy(IH) only, whereas Normal Hierarchy (NH) is ruled out. We also study phenomenologies related to the effective neutrino masses $m_{\beta}$ in tritium beta decay and $m_{\beta\beta}$ in neutrinoless double beta decay. Mixings between active neutrinos and eV scale sterile neutrino are analyzed in detail. The model also predicts the Dirac CP-violating phase $\delta_{CP}$ and Majonara phases $\alpha$ and $\beta$. The best-fit values of the neutrino mixing angles and ratios of two mass-squared differences are determined using minimum $\chi^2$ analysis. The best-fit values of the neutrino oscillation observables are predicted as $\sin^2\theta_{23}=0.566,$ $\sin^2\theta_{12}=0.307,$ $ \sin^2\theta_{13}=0.023$ and $r = 0.172$. The Dirac and Majorana phases are observed at $\delta_{CP}=350.19^o,\ \alpha=355.75^o$ and $\beta=333.10^o$. We also observe that the predictions of effective neutrino mass parameters in the 3+1 scheme are significantly different from the three neutrino paradigm.
\end{abstract}

%
% Uncomment for keywords
%\vspace{2pc}
%\noindent{\it Keywords}: XXXXXX, YYYYYYYY, ZZZZZZZZZ
%
% Uncomment for Submitted to journal title message
%\submitto{\JPA}
%
% Uncomment if a separate title page is required
%\maketitle
% 
% For two-column output uncomment the next line and choose [10pt] rather than [12pt] in the \documentclass declaration
%\ioptwocol
%

\section{Introduction}
The origin of neutrino masses and flavor structure is one of the most critical problems of particle physics's Standard Model (SM). Since the discovery of neutrino oscillation in various experiments such as SNO, SuperKamiokande, etc., various extensions of SM have been studied. Models with non-Abelian discrete symmetries such as $A_4$  \cite{altarelli2010discrete,king2007a4,babu2003underlying}, $S_4$  \cite{ma2006neutrino,altarelli2009revisiting,vien2022lepton,bazzocchi2013neutrino}, $S_3$ \cite{mohapatra2006s3,morisi2006flavor,Vien:2021uxw,grimus2005s3,koide2007s}, etc. find their distinct places in describing some of the experimental results as well as new predictions regarding unresolved problems of SM. Among these, the absolute mass scale of neutrino, Dirac CP-violating phase, Baryogenesis, Dark matter, etc., are some of the main questions in high-energy physics. 

Inspired by the observations of LSND  \cite{aguilar2001evidence} and MiniBooNE  \cite{aguilar2018significant,2021microboone}, many authors proposed the existence of a fourth state of neutrino called the sterile neutrino. Sterile neutrinos are incorporated along with the three neutrino theory in literatures in a 3+1 scheme  \cite{Zhang2011,giunti2020katrin,guintianrev, vien2022b,miranda2019revisiting,vien2021}, 3+1+1 scheme  \cite{kuflik2012neutrino,huang2013mev,fan2012light,nelson2011effects}, 3+2 scheme  \cite{donini2012minimal,archidiacono2012testing,babu2016light,karagiorgi2007leptonic} etc. One of the simplest extensions is the 3+1 scheme, where a singlet sterile neutrino is added to the three active neutrinos, and the sterile neutrino gets mass through a minimally extended seesaw mechanism (MES) \cite{Zhang2011}.
Many authors have used discrete symmetries to describe the neutrino masses and their flavor structures. The main disadvantages of such approaches are the inclusion of many hypothetical scalar fields and additional symmetry groups. Recently, an interesting approach has been proposed in which modular symmetry is used in which the non-Abelian discrete symmetries are its subgroups \cite{feruglio2019neutrino}. An important feature of the modular symmetry framework is that the Yukawa couplings can be transformed non-trivially as modular forms under the modular group. These modular forms are written as a function of a complex parameter $\tau$, called the modulus. Minimal or no scalar flavons are required to break the symmetry as the lepton masses are generated from the symmetry breaking by the eve of the modulus $\tau$. Modular forms of level 3 denoted by $\Gamma(3)$ are isomorphic to $A_4$. Detailed analysis on modular $A_4$ groups and their application in neutrino model building is studied in  \cite{feruglio2019neutrino,kobayashi2018neutrino,kobayashi2018modular}. There are other works based on numerous modular groups such as modular $A_4$  \cite{zhang2020modular,Beherascoto,mishra2022type,kobayashi2020type,okada2020radiative,
nomura2019modular,abbas2021fermion,nomurascoto,behera2022implications}, modular $A_5$  \cite{ding2019neutrino,novichkov2019modular,PhysRevD.103.095013,PhysRevD.103.076005,Behera2021eut}, modular $S_4$ \cite{wang2020minimal,penedo2019lepton,wang2021dirac,ding2019modular,
KobayasiS4,LiuS4}, etc., to study neutrino phenomenologies, dark matter, leptogenesis, etc. For instance, in  \cite{abbas2021fermion},  modular $A_4$ is used to study lepton and quark masses and mixings based on inverse seesaw mechanism,  \cite{behera2022implications} studies lepton mixings and leptogenesis in linear seesaw. In contrast, the scotogenic dark matter scenario is studied in  \cite{nomurascoto,Beherascoto}. Other works in the literature study 3+1 active-sterile mixing using the conventional $A_4$ discrete symmetry group  \cite{das2019active,mksingh,vien2022b,Das3}. However,  modular $A_4$ symmetry has not been applied in 3+1 active-sterile neutrino mixing in MES mechanism. The main objectives and features of the present work are: (i) to develop a new model of neutrino masses and flavor mixings based on modular $A_4$ symmetry for 3+1 scheme based on MES mechanism, (ii) to establish a detailed analysis of active-sterile mixing for sterile neutrinos in the mass in eV scale, (iii) to reproduce low energy active neutrino mixing angles, mass-squared differences as well as Dirac and Majorana phases $\delta_{CP},\ \alpha$ and $\beta$ in the 3$\sigma$ ranges constrained by the cosmological Planck upper limit $\sum m_{i}<0.12$ eV, (iv) to predict the effective neutrino mass parameters $m_{\beta}$ and $m_{\beta\beta}$ from tritium beta decay and neutrinoless double beta decay $(0\nu\beta\beta)$ and (v) finally, to determine the best-fit values of the model parameters and neutrino observables using the minimum $\chi^2$ analysis.

As we shall discuss later, exciting results are obtained for the modulus $\tau$ in a narrow range for IH only, while the NH is ruled out at 3$\sigma$. The model's predictability is improved by excluding hypothetical scalar triplet flavons compared to conventional model building. Significant results are obtained for the active-sterile mixing parameters $\vert U_{i4}\vert^2 $, (where $i=1,2,3$). Predictions in the phenomenological studies of effective neutrino masses $m_{\beta}$ and $m_{\beta\beta}$ reveal new results that future experiments could probe. For instance, we obtain distinctive results between the values of $m_{\beta}$ and $m_{\beta\beta}$ in three neutrinos (3$\nu$) mixings and 3+1 neutrino mixings. Future experiments could probe these results, and we may use them to validate or discard the existence of eV-scale sterile neutrinos. With hints of the non-existence of sterile neutrinos from the initial results of MicroBooNE, we shall study a phenomenological approach to the constraints of active-sterile neutrino mixings. The present paper is organized as follows. We present a detailed description of the model in section \ref{c5s2}, followed by the numerical analysis of the model in section \ref{section3}. Results of the analysis are presented in section \ref{section4}. We conclude with a summary and discussion in section \ref{section6}.


\section{The structure of the model}\label{c5s2}
This model uses modular $A_4$ symmetry to study lepton masses and mixings in the 3+1 framework. We extend the SM particle contents with three right-handed neutrinos $N$ as triplet under modular $ A_4$ and a singlet sterile neutrino field $S$. The SM lepton doublet $L$ transforms as a triplet of $A_4$ while the Higgs represented as $H_u$ and $H_d$ having hypercharges $+1/2 $ and $-1/2$  are taken as singlets of $A_4$. They develop VEVs after the spontaneous symmetry breaking (SSB) with alignments along $\langle H_u\rangle = \left(0, v_u/\sqrt{2}\right)^T$ and $\langle H_d\rangle = \left( v_d/\sqrt{2}, 0\right)$ inducing fermion mass terms. The right-handed charged leptons $e_r, \mu_r$ and $\tau_r$ are assigned $1,\ 1^{\prime\prime},\ 1^{\prime}$ of modular $A_4$ respectively. As a result, there are three independent coupling constants $\alpha^{\prime},\ \beta^{\prime}$, and $\gamma^{\prime}$ in the Lagrangian of the charged lepton sector. We consider one scalar field $\zeta$, which is an $A_4$ singlet so that its vacuum expectation value (VEV) $v_{\zeta}$ fixes the scale of sterile neutrino mass matrix $M_S$. %We have the freedom of choosing a particular scale of $v_{\zeta}$ because 
The MES mechanism has the advantage of naturally generating sterile neutrinos in eV or KeV scale based on the scale of $M_S$\footnote{ $M_S\sim \mathcal{O}(10^2)$GeV will give sterile neutrinos in eV scale. In contrast, $M_S\sim \mathcal{O}(10)$TeV will generate KeV scale sterile neutrino mass}. In modular $A_4$ symmetry, the Yukawa coupling transforms as an $A_4$ triplet modular function $Y =(y_1,y_2,y_3)^T$ having weight $k_Y$.  The complete particle contents of the model with their corresponding group charges and modular weights $k_i$ are given in Table \ref{Table1}. The components of modular form $Y$ of weight 2, which transforms as triplet under $A_4$ can be expressed in terms of the Dedekind eta-function $\eta(\tau)$ and its derivative $\eta^{\prime}(\tau)$ as \cite{feruglio2019neutrino}

\begin{eqnarray}
y_1(\tau)= & \frac{i}{2\pi}\left[\frac{\eta^{\prime}(\frac{\tau}{3})}{\eta (\frac{\tau}{3})}+ \frac{\eta^{\prime}(\frac{\tau+1}{3})}{\eta (\frac{\tau+1}{3})} + \frac{\eta^{\prime}(\frac{\tau+2}{3})}{\eta (\frac{\tau+2}{3})} - \frac{27\eta^{\prime}(3\tau)}{\eta(3\tau)}\right],\nonumber \\
y_2(\tau)= &\frac{-i}{\pi}\left[\frac{\eta^{\prime}(\frac{\tau}{3})}{\eta (\frac{\tau}{3})}+\omega^2 \frac{\eta^{\prime}(\frac{\tau+1}{3})}{\eta (\frac{\tau+1}{3})} +\omega \frac{\eta^{\prime}(\frac{\tau+2}{3})}{\eta (\frac{\tau+2}{3})} \right],\nonumber\\
y_3(\tau)= &\frac{-i}{\pi}\left[\frac{\eta^{\prime}(\frac{\tau}{3})}{\eta (\frac{\tau}{3})}+\omega \frac{\eta^{\prime}(\frac{\tau+1}{3})}{\eta (\frac{\tau+1}{3})} +\omega^2 \frac{\eta^{\prime}(\frac{\tau+2}{3})}{\eta (\frac{\tau+2}{3})} \right].\label{C5etafn}
\end{eqnarray}

where $\omega = e^{2\pi i/3}$ and $\eta(\tau)$ is defined as 
\begin{equation}
\eta(\tau)=q^{1/24}\prod_{n=1}^{\infty}(1-q^n),  \ \ \ \ q \equiv e^{i 2\pi \tau}.
\end{equation}
The overall coefficients in (\ref{C5etafn}) is one possible choice; it cannot be uniquely determined. The $q-$expansions of the Yukawa components are expressed as 
\begin{eqnarray}
y_1(\tau)& = 1+12q+36q^2+12q^3+... \nonumber\\
y_2(\tau)& = -16q^{1/3}(1+7q+8q^2+...)\\
y_3(\tau)& = -18q^{2/3}(1+2q+5q^2+...).\nonumber
\end{eqnarray}
In a modular symmetry framework, an interaction is invariant under the symmetry when the sum of the modular weights for each associated field and coupling is zero, and it is also invariant under the discrete symmetry (such as $A_4$ symmetry). 

\begin{table*}
\centering
\begin{tabular}{|m{1.5cm}|m{1.3cm}m{1.4cm}|m{1.5cm}|m{0.5cm}m{0.5cm}|m{0.5cm}|}
\hline
$\frac{Fields}{Charges}$ & $L$ & $e_r^c, \mu_r^c,\tau_r^c$ & $H_{u,d}$ & $N^c$ & $S^c$ & $\zeta$   \\
\hline
$SU(2)_L$& 2 & 1 & 2 & 1 & 1  &1   \\
$U(1)_Y$ & $-1/2$ & $1$& $\pm$ 1/2 & 0& 0&0 \\
$A_4$& 3 &1,1$^{\prime\prime}$,1$^{\prime}$ &1 & 3 & 1 &1   \\
$k_i$&$k_L$ &$k_e$  &$k_H$ &$k_N$ & $k_S$ &$k_{\zeta}$ \\
\hline
\end{tabular}
\caption{\centering Particle contents of the model and their group charges.}
\label{Table1}
\end{table*}

The Lagrangian of charged lepton sector invariant under $A_4$ symmetry is given below.
\begin{eqnarray}
-\mathcal{L}_{c} =\ \alpha^{\prime} e_r^c( H_d L Y)_1 + \beta^{\prime} \mu_r^c (H_d L Y)_{1^{\prime}} + \gamma^{\prime}\tau_r^c(H_d L Y)_{1^{\prime\prime}} + h.c.
\label{C5chargelepton}
\end{eqnarray}
For the neutrino sector, the Lagrangian is 
\begin{eqnarray}\label{C5md}
-\mathcal{L}_{D} &=\ g_i (N^c)^TH_uLY + h.c \\ \nonumber
  & =\ g_1 (N^c)^TH_u(LY)_{3S} + g_2 (N^c)^TH_u(LY)_{3A} + h.c \\
-\mathcal{L}_{R} &= \ \lambda Y (N^c)^T N + h.c \\
-\mathcal{L}_s &= \ \zeta Y S^cN +h.c
\label{C5MR}
\end{eqnarray}
where, $\alpha^{\prime}, \beta^{\prime}$,$\gamma^{\prime},\ g_1,\ g_2,$ and $\lambda$ are constant coefficients. The subscripts $`3S$' and $`3A$' denote the symmetric and antisymmetric product of two triplet fields in $A_4$ symmetry. The above interaction terms for the charged lepton and neutrino sector are invariant under modular $A_4$ symmetry if the weights $k_i$ satisfy the following conditions,
\begin{eqnarray}
k_e + k_H+k_L =0,\ \ 
k_H+k_L+k_N+k_Y=0, \nonumber \\  
k_N+k_N+k_Y=0, \ \ \ \ \
k_{\zeta}+k_S+k_N+k_Y=0.
\end{eqnarray}
For modular form of weight $k_Y=2,$ we have used the following assignments of modular weights,
\begin{equation}
k_N=k_L=-1, \ k_H=0,k_S=1,\ k_{\zeta}=-2, \ \mbox{and} \ k_e=-1.
\end{equation}
Other higher dimensional terms such as $\frac{g^{\prime}}{\Lambda}H_u\zeta S^c(LY)_1$, $\frac{\lambda_1}{\Lambda^2}YYY(N_i^c)^TN_i\zeta\zeta$, $\frac{\delta_1}{\Lambda^2}YYNS\zeta\zeta\zeta$, $\frac{\delta_2}{\Lambda^3}YYSS\zeta\zeta\zeta\zeta$, etc. in the neutrino sector are also allowed by the symmetry of the model. However, their contributions are negligibly small and have been ignored in our analysis. 

After the electroweak symmetry breaking, using (\ref{C5md}) - (\ref{C5MR}), the Dirac, Majorana, and sterile neutrino mass matrices are given as

\begin{eqnarray}
M_D = &\ v_u \left(
\begin{array}{ccc}
 2 g_1 y_1 & y_3 (g_2-g_1) & y_2 (-g_1-g_2) \\
 y_3 (-g_1-g_2) & 2 g_1 y_2 & y_1 (g_2-g_1) \\
 y_2 (g_2-g_1) & y_1(-g_1-g_2) & 2 g_1 y_3 \\
\end{array}
\right); \\
M_R=&\ \lambda\left(
\begin{array}{ccc}
 2y_1 & -y_3 & -y_2 \\
 -y_3 & 2 y_2 & -y_1 \\
 -y_2 & -y_1 & 2 y_3 \\
\end{array}
\right);\\
M_s =&\ v_{\zeta}\left(
\begin{array}{ccc}
y_1 & y_3 & y_2 \\
\end{array}
\right).
\label{matrix}
\end{eqnarray}
Whereas, from (\ref{C5chargelepton}), the charged lepton mass matrix becomes 
\begin{eqnarray}
m_L = v_d \left(\begin{array}{ccc}
\alpha^{\prime} & 0 & 0 \\ 
0 & \beta^{\prime} & 0 \\ 
0 & 0 & \gamma^{\prime}
\end{array} \right)\times 
\left(\begin{array}{ccc}
y_1 & y_3 & y_2 \\ 
y_2 & y_1 & y_3 \\ 
y_3 & y_2 & y_1
\end{array} \right).
\end{eqnarray}
 It is convenient to diagonalise the $m_L$ using a hermitian matrix $M_{L}=m^{\dagger}_L m_L$ as
\begin{eqnarray}
M^{diag}_L = \mathcal{U}^{\dagger}_L M_L \mathcal{U}_L.
\label{mldiag}
\end{eqnarray}

In the MES mechanism, we obtain a leading order of the active neutrino mass matrix $m_{\nu}$ as well as the sterile mass $m_s$ given as follows.
\begin{equation}
m_{\nu} \simeq M_DM_R^{-1}M_S^T\left(M_S M_R^{-1}M_S^T\right)^{-1}M_S\left(M_R^{-1}\right)^T M_D^T-M_DM_R^{-1}M_D^T;
\label{C5mv}
\end{equation}
\begin{equation}
m_s \simeq - M_SM_R^{-1}M_S^T.
\label{C5ms}
\end{equation}

The full $(4\times 4)$ active-sterile neutrino mass matrix is diagonalised by a unitary $(4\times 4)$ mixing matrix given by  \cite{v441982}
\begin{equation}
V \simeq \left(\begin{array}{ccc}
(1-\frac{1}{2}RR^{\dagger})U & R \\ 
-R^{\dagger}U & 1-\frac{1}{2}R^{\dagger}R
\end{array} \right),
\label{C5V44}
\end{equation}
where $U$ represents the $3\times 3$ active neutrino mixing matrix and $R$ represents the strength of active-sterile mixing given by
\begin{eqnarray}
R=&M_DM_R^{-1}M_S^T(M_SM_R^{-1}M_S^T)^{-1}.
\label{C5R}
\end{eqnarray}
Using (\ref{C5ms}) and (\ref{C5R}), the sterile neutrino mass and actie-sterile mixing strength are given by 
 \begin{eqnarray}
 m_s =\ \frac{v_\zeta^2\ y_1 (y_1^3 - 12 y_1 y_2 y_3 - 8 (y_2^3 + y_3^3))}{2\lambda (y_1^3 + y_2^3 - 3 y_1 y_2 y_3 + y_3^3)} \\
\fl R=\ f \left(\begin{array}{c}
 y_1^3 + (1 + 3 g) y_2^3 - 3 y_1 y_2 y_3 + (1 - 3 g) y_3^3 \\ 
 y_3 (y_1^3 + y_2^3 - 3 y_1 y_2 y_3 + y_3^3) + 
     g (-3 y_1^2 y_2^2 - 2 y_1^3 y_3 + y_2^3 y_3 + 3 y_1 y_2 y_3^2 + 
        y_3^4) \\ 
 (y_1^3 (y_2 + 2 g y_2) - 
     3 (1 + g) y_1 y_2^2 y_3 + 
     3 g y_1^2 y_3^2 - (-1 + g) y_2 (y_2^3 + y_3^3))\end{array} \right)
 \end{eqnarray}
where \begin{equation}
f = \frac{2 g_1 v}{
  v_{\zeta}\left[ y_1^3 - 12 y_1 y_2 y_3 - 
     8 (y_2^3 + y_3^3)\right]}.
\end{equation}   

Deviation of $U$ from unitarity due to the presence of sterile neutrino is determined by $1- \frac{1}{2}RR^{\dagger}$. The $4\times 4$ neutrino mixing matrix $V$ can be parametrized by six mixing angles $(\theta_{12},\ \theta_{13},\ \theta_{23},\ \theta_{14},\theta_{24},\ \theta_{34})$, three Dirac phases $(\delta_{CP},\ \delta_{14},\ \delta_{24})$ and three Majorana phases $(\alpha,\ \beta,\ \gamma)$ \cite{gariazzo2016light}. However, in the active neutrino sector, we note that the lightest neutrino mass in the MES mechanism is vanishing, i.e., $m_1=0(m_3=0)$ for NH(IH). Thus, we are left with only one Majorana CP-violating phase $\beta$ while $\alpha$ will be vanishing (Here, $\gamma$ can be re-phased out)\cite{nath2016understanding}. This result will be evident from the numerical analysis as well.
 
In the above parametrization, the unitary matrix V can be written as,
\begin{equation}
V = O_{34}\tilde{O}_{24}\tilde{O}_{14}O_{23}\tilde{O}_{13}O_{12}.P
\label{V44p}
\end{equation}
where $O_{ij}$ are rotation matrices and P is a diagonal phase matrix given by 
\begin{equation}
P = diag\{e^{i\alpha},e^{i (\beta + \delta_{13})},e^{i(\gamma +\delta_{14})},1\}.
\end{equation}

The six neutrino mixing angles are determined from the mixing elements of $V$ using the following relations 

\begin{eqnarray} \label{C1anglessolve}
\sin^2\theta_{14}\ =\ \vert V_{e4}\vert ^2,\ \  
\sin^2\theta_{24}\ =\ \frac{\vert V_{\mu 4}\vert ^2}{1-\vert V_{e4}\vert ^2},\ \ 
\sin^2\theta_{34}\ =\ \frac{\vert V_{\tau 4}\vert ^2}{1-\vert V_{e4}\vert ^2-\vert V_{\mu 4}\vert ^2},\nonumber \\
\sin^2\theta_{12}\ =\ \frac{\vert V_{e2}\vert ^2}{1-\vert V_{e4}\vert ^2-\vert V_{e3}\vert ^2}, \ \ 
\sin^2\theta_{13}\ =\ \frac{\vert V_{e3}\vert ^2}{1-\vert V_{e4}\vert ^2},
\end{eqnarray}
\begin{eqnarray}
\fl \sin^2\theta_{23}\ =&\ \frac{\vert V_{e3}\vert ^2(1-\vert V_{e4}\vert ^2)-\vert V_{e4}\vert ^2\vert V_{\mu 4}\vert ^2}{1-\vert V_{e4}\vert ^2-\vert V_{\mu 4}\vert ^2}+ \frac{\vert V_{e1}V_{\mu 1}+V_{e2}V_{\mu 2}\vert ^2(1-\vert V_{e4}\vert ^2)}{(1-\vert V_{e4}\vert ^2-\vert V_{e3}\vert ^2)(1-\vert V_{e4}\vert ^2-\vert V_{\mu 4}\vert ^2)}.\nonumber 
\end{eqnarray}
where $V_{ij}$ are the elements of mixing matrix in (\ref{V44p}).

The Dirac CP-violating phase $\delta_{CP}$, is related to the $4\times 4$ mixing matrix $V$ through the Jarlskog invariant $J.$ According to the parametrisation given in (\ref{V44p}), the Jarlskog invariant defined as $J = Im[V_{e1}V_{\mu 2}V^*_{e2}V^*_{\mu 1}]$  takes the form  \cite{KUMAR2020115082}
\begin{equation}
J = J_3^{cp} c_{14}^2c_{24}^2 + s_{24}s_{14}c_{24}c_{23}c^2_{14}c^3_{13}c_{12}s_{12}\sin(\delta_{14}-\delta_{24}),
\label{C5deltasolve}
\end{equation}
where $J_3^{cp} = s_{23}c_{23}s_{12}c_{12}s_{13}c_{13}^2 \sin \delta_{CP}$ is the Jarlskog invariant in the three neutrino framework and $s_{ij} = \sin\theta_{ij},c_{ij}=\cos\theta_{ij}$ are the neutrino mixing angles. Similarly, the two physical Majorana phases $\alpha$ and $\beta$ are determined from $V$ using the invariants $I_1$ and $I_2$ defined as follows 
\begin{eqnarray}\label{C5majoranaphase1}
I_1 = Im[V_{e1}^*V_{e2}]\ =\ c_{12} c_{13}^2 c_{14}^2 s_{12} \sin\left(\frac{\alpha}{2}\right), \\ 
I_2 = Im[V_{e1}^*V_{e3}]\ =\ c_{12}c_{13}c_{14}^2s_{13}\sin\left(\frac{\beta}{2}-\delta_{CP}\right).
\label{C5majoranaphase2}
\end{eqnarray}

One of the most important parameters in neutrino physics are the effective neutrino mass $m_{\beta\beta}$ and effective electron neutrino mass $m_{\beta}$. A combined analysis from KAMLand-Zen \cite{Kamland} and GERDA provided an upper bound on $m_{\beta\beta}$ in the range $m_{\beta\beta} < (0.071-0.161$) eV \cite{agostini2018improved,goswami}. Recent results from the KATRIN-2020 \cite{aker2020first} experiment constrains the effective electron neutrino mass $m_{\beta}$ to be less than 1.1 eV. However, this upper bound has been updated to $m_{\beta}<0.8$ eV in the latest results of KATRIN-2022  \cite{katrin2022direct}. These parameters are determined from the neutrinoless double beta decay and tritium beta decay respectively, using the relations \cite{Hagstot}

\begin{eqnarray}
m_{\beta\beta}=\vert\sum_{j=1}^4 V_{ej}^2 m_j\vert
\label{mbbeq}
\end{eqnarray}
\begin{equation}
m_{\beta} = \left(\sum_{i=1}^4\vert V_{ei}^2\vert m_{i}^2\right)^{1/2}.
\label{mbeq}
\end{equation}
The presence of eV scale sterile neutrino mixings with active neutrinos have significant affects on the predictions of $m_{\beta\beta}$ and $m_{\beta}$ as pointed out in Ref.\cite{giunti2015predictions}.

\section{Numerical analysis}\label{section3}
In this section, we describe the steps of the detailed numerical analysis of the model. The charged lepton mass matrix $m_L$ depends on the free parameters $\alpha^{\prime}, \beta^{\prime}$, $\gamma^{\prime} $ and $\tau.$ We define the complex parameter $\tau$ as 
\begin{equation}
\tau=Re[\tau] + i\ Im[\tau].
\end{equation} 
The fundamental domain of $\tau$ is given in \cite{feruglio2019neutrino}. Without the loss of generality,  $\alpha^{\prime},\ \beta^{\prime}$ and $\gamma^{\prime} $ can be taken to be real and positive. Given the complex parameter $\tau,$ we can solve these unknown coefficients using the following three identities 
\begin{eqnarray}
Tr[M_L] = m_e^2 + m_{\mu}^2 + m_{\tau}^2, \\
Det[M_L] = m_e^2 \times m_{\mu}^2 \times m_{\tau}^2, \\
Tr[M_L]^2/2 - Tr[M_L^2]/2 = m_em_{\mu} + m_{\mu}m_{\tau} + m_{\tau}m_e.
\end{eqnarray}
where $m_e , m_{\mu}$ and $ m_{\tau}$ are the charged lepton masses.


We randomly scan the real and imaginary parts of $\tau$ in the positive half of the fundamental domain, 
\begin{equation}
 Re[\tau] = [0,0.5],\ Im[\tau]= [0.6,2].
\end{equation}
Using $v_d = 246$ GeV and $m_e = 0.51099$ MeV, $m_{\mu} = 105.65837$ MeV and $m_{\tau}= 1776.86$ MeV, taken from PDG \cite{pdg2022}, the values of $\alpha^{\prime},\ \beta^{\prime}$ and $\gamma^{\prime} $ are solved using the above identities. Once the charged lepton mass matrix is completely determined, it can be numerically diagonalized using (\ref{mldiag}).

The active neutrino mass matrix given in (\ref{C5mv}) for the neutrino sector depends only on two complex parameters $g\equiv g_2/g_1$ and $\tau$,  up to an overall factor $v_u^2g_1^2/\lambda$. The modular symmetry is broken on fixing $\tau$, and the neutrino mass eigenvalues, three mixing angles, and the Dirac and Majorana phases are completely determined. The model becomes very predictive since the number of free parameters is less than the number of neutrino observables. The absolute scale of active neutrino masses is fixed by adjusting the overall factor $v_u^2g_1^2/\lambda$. With the values of $\tau$ obtained from the charged lepton sector, we give the following values as input parameters, 
\begin{equation}
 \vert g\vert = [0,10], \ \ \lambda = 10^{14}\ \mbox{GeV}, \ \ v_{\zeta} = [10,100]\ \mbox{GeV}\ \  \mbox{and}\ \ \phi_g = [-\pi,\pi].
\label{inputs}
\end{equation} 

Similar to the charged lepton sector, the active neutrino mass matrix $m_{\nu}$ is numerically diagonalized using the relation 
\begin{eqnarray}
 M_{\nu}^{diag} = \mathcal{U}^{\dagger}_{\nu} M_{\nu} \mathcal{U}_{\nu}
\end{eqnarray}
where $M_{\nu} = m_{\nu}^{\dagger}m_{\nu}$ is a hermitian matrix and $M_{\nu}^{diag} = diag(m_1^2,m_2^2,m_3^2)$. Here, the active neutrino mass eigenvalues are $m_1,\ m_2$, and $m_3$. Thus, the $(3\times 3)$ active neutrino PMNS mixing matrix will be given by 
\begin{eqnarray}
U = \mathcal{U}^{\dagger}_{L}\mathcal{U}_{\nu}.
\end{eqnarray}

\begin{table*}
\begin{center}
\renewcommand{\arraystretch}{1}
\begin{tabular}{c|c|c}
 \hline 
Parameter &	Normal Hierarchy (best-fit$\pm 1\sigma$) &	Inverted Hierarchy (best-fit$\pm 1\sigma$)  \\
\hline
$\vert\Delta m^2_{21}\vert: [10^{-5} eV^2]$ & 6.82 – 8.03 $(7.41^{+0.21}_{-0.20})$  &	 6.82 – 8.03 $(7.41^{+0.21}_{-0.20})$ \\
$\vert\Delta m^2_{31}\vert: [10^{-3} eV^2]$	& 2.428 – 2.597 $(2.511^{+0.028}_{-0.027})$  & 2.408 – 2.581 $(2.498^{+0.032}_{-0.025})$ \\

$\sin^2\theta_{12} $	& 0.270 – 0.341 $(0.303^{+0.012}_{-0.011})$ &  0.270 – 0.341 $(0.303^{+0.012}_{-0.011})$  \\
$\sin^2\theta_{23}$ &0.406 – 0.620	$(0.572^{+0.018}_{-0.023})$  &0.412 – 0.623 $(0.578^{+0.016}_{-0.021})$ 		 \\
$\sin^2\theta_{13}/10^{-2}$ & 2.029 – 2.391	$(2.203^{+0.056}_{-0.059})$ & 2.047 – 2.396 $(2.219^{+0.060}_{-0.057})$  \\
$\delta_{\rm CP}/^o$ &	108 - 404 $(197^{+42}_{-0.25})$	& 192 - 360 $(286^{+27}_{-32})$	 \\
$r=\sqrt{\frac{\Delta m_{21}^2}{\vert\Delta m_{3l}^2\vert}} $ & 0.1675 - 0.1759 (0.1718)  & 0.1683 - 0.1765 (0.1722)\\
$\vert U_{14}\vert^2 $  & 0.012 - 0.047 & 0.012 - 0.047 \\
$\vert U_{24}\vert^2 $  & 0.005 - 0.03 &  0.005 - 0.03 \\
$\vert U_{34}\vert^2 $  & 0 - 0.16 & 0 - 0.16 \\
\hline
\end{tabular}  
\end{center}
\caption{Updated global-fit data for three neutrino oscillation, NuFIT 5.2, 2022 \cite{nufit}. For 3+1 mixing, data is taken from  \cite{barrylight,vien2022b,Gariazzo_2016}.}
\label{data}
\end{table*} 



We filter the values of the model parameters using the 3$\sigma$ bounds of the three mixing angles $\sin^2\theta_{12},\sin^2\theta_{13},\sin^2\theta_{23}$ and the ratio of neutrino mass squared differences $r= \sqrt{\Delta m_{21}^2/\Delta m_{31}^2}= m_2/m_3$ for NH $(m_1 \approx 0\ll m_2 < m_3\ll m_4)$ and $ r=\sqrt{\Delta m_{21}^2 /\vert\Delta m_{32}^2\vert}= \sqrt{1-m_1^2/m_2^2}$ for IH $(m_3\approx0\ll m_2<m_1 \ll m_4)$ given in table \ref{data}. The sum of the active neutrino masses $\sum m_i = m_1 + m_2 + m_3 $ is also constrained by the Cosmological Planck upper limit $\sum m_i <0.12$ eV.

The allowed solutions for the charged lepton Yukawa coefficients $\alpha^{\prime},\ \beta^{\prime}$ and $\gamma^{\prime}$ are shown as correlation plots among themselves in figure \ref{chargelepton}. The model predicts an inverted hierarchy of neutrino masses  while the normal hierarchy is ruled out at 3$\sigma$. 
\begin{figure}
\centering
\begin{subfigure}[b]{0.43\textwidth}
    \includegraphics[width=\textwidth]{chargelepton}
   \subcaption{}
    \label{chargelepton}
  \end{subfigure}
  \hspace{0.2cm}
  \begin{subfigure}[b]{0.43\textwidth}
    \includegraphics[width=\textwidth]{tauplot}
    \subcaption{}
    \label{tauplot}
  \end{subfigure}
  %
%    \captionsetup{justification=raggedright, singlelinecheck=false,  width=0.95\linewidth}
  \caption{\footnotesize (a) Variations of charged lepton Yukawa coefficients among themselves. (b)  Plot of allowed regions of $\tau$ constrained by neutrino oscillation data. }
  \label{C5tauchargeplot}
\end{figure}
\begin{figure}
\centering
\begin{subfigure}[b]{0.43\textwidth}
    \includegraphics[width=\textwidth]{Imtauvsyukawa}
    \subcaption{}
    \label{imtauvsyukawa}
  \end{subfigure}
  \hspace{0.2cm}
  \begin{subfigure}[b]{0.43\textwidth}
    \includegraphics[width=\textwidth]{gvsyukawa}
    \subcaption{}
    \label{gvsyukawa}
  \end{subfigure}
  %
%    \captionsetup{justification=raggedright, singlelinecheck=false,  width=0.95\linewidth}
  \caption{\footnotesize Figure shows the  variation of Yukawa components $\vert y_1\vert,\vert y_2\vert$ and $\vert y_3\vert$ with Im$[\tau]$ and  $\vert g\vert$.  }
  \label{C5yukawaplot}
\end{figure}


The allowed ranges of Re$[\tau]$, Im$[\tau]$ and $g$ is shown in figure \ref{tauplot}. The values of Im$(\tau)$ vary in the range $[ 0.67 - 1.59 ]$ while the values of Re$(\tau)$ are continuously varied in a very small region between 0.001 to $0.26$ in the fundamental domain of $\tau$. The ranges of Yukawa couplings $\vert y_1\vert,\ \vert y_2\vert$ and $\vert y_3\vert$, determined from these values, are shown in figure  \ref{C5yukawaplot} as variation plots with Im[$\tau$] and $\vert g\vert$. The values of $\vert y_2\vert$ and $\vert y_3\vert$ decrease with increase in Im[$\tau$] while $\vert y_1\vert$ almost remains constant throughout. It is observed that the couplings lie in the ranges $0.993 \leq \vert y_1(\tau)\vert\leq 1.071 $, $ 0.213 \leq \vert y_2(\tau)\vert\leq 1.489 $ , and  $ 0.023 \leq \vert y_3(\tau)\vert\leq 1.066 $.  

\section{Results of the analysis}\label{section4}
In this section, we shall show the results of the numerical analysis of the model. It is exciting to see in figure \ref{C5gtauvssum}, which shows the dependence of Im$[\tau]$ and $\vert g\vert$ with the sum of active neutrino masses $\sum m_i$. It shows that $\vert g\vert$ and Im[$\tau$] are heavily constrained by the Planck upper bound $\sum m_i < 0.12$ eV, indicated by the vertical dotted line. The upper limit on $\sum m_i$  is satisfied if we fix the overall scale factor $v_u^2g_1^2/\lambda = 0.00123$ eV in our analysis. Figure \ref{mass} shows the variation of active neutrino mass ratios $m_2/\sum m_i$ and $m_1/\sum m_i$ with $\sum m_i$. 
\begin{figure}
\centering
\begin{subfigure}[b]{0.4\textwidth}
    \includegraphics[width=\textwidth]{gvssum}
    \subcaption{}
    \label{gvssum}
  \end{subfigure}
  \hspace{0.2cm}
  \begin{subfigure}[b]{0.4\textwidth}
    \includegraphics[width=\textwidth]{sumvsimtau}
    \subcaption{}
    \label{sumvsimtau}
  \end{subfigure}
  %
%    \captionsetup{justification=raggedright, singlelinecheck=false,  width=0.95\linewidth}
  \caption{\footnotesize  The dependence of $\vert g\vert$  and Im$[\tau]$ on the sum of the active neutrino masses $\sum m_i$.  }
  \label{C5gtauvssum}
\end{figure}

\begin{figure}
\centering
  \begin{subfigure}[b]{0.4\textwidth}
    \includegraphics[width=\textwidth]{mass}
    \subcaption{}
    \label{mass}
  \end{subfigure}
  \hspace{0.2cm}
  \begin{subfigure}[b]{0.4\textwidth}
    \includegraphics[width=\textwidth]{angles}
    \subcaption{}
    \label{angles}
  \end{subfigure}
  %
%    \captionsetup{justification=raggedright, singlelinecheck=false,  width=0.95\linewidth}
  \caption{\footnotesize (a) Predicted values of active neutrino mixing angles $\sin^2\theta_{12},\sin^2\theta_{23}$ and $\sin^2\theta_{13}$. (b) Predicted values of active neutrino masses $m_1,m_2$ ($m_3=0$).  }
  \label{C5anglemass}
\end{figure}


In figure \ref{angles}, we show the predicted values of the active neutrino mixing angles $\sin^2\theta_{12}$, $\sin^2\theta_{23}$ and $\sin^2\theta_{13}$. The variation of these angles with the sum of the active neutrino masses $\sum m_i$ are shown in figure \ref{C5anglevssum}. It suggests that the mixing angles $\sin^2\theta_{12}$ and $\sin^2\theta_{13}$ run over their experimental ranges while the values of $\sin^2\theta_{23}$ are concentrated more in regions above 0.5 for higher values of $\sum m_i$. However, it needs to be more conclusive to identify the octant degeneracy of $\theta_ {23}$. 
\begin{figure}
\centering
\begin{subfigure}[b]{0.4\textwidth}
    \includegraphics[width=\textwidth]{anglevssum}
    \subcaption{}
  %  \label{an}
  \end{subfigure}
  \hspace{0.2cm}
  \begin{subfigure}[b]{0.4\textwidth}
    \includegraphics[width=\textwidth]{s13vssum}
    \subcaption{}
 %   \label{gvsyukawa}
  \end{subfigure}
  %
%    \captionsetup{justification=raggedright, singlelinecheck=false,  width=0.95\linewidth}
  \caption{\footnotesize Dependence of active neutrino mixing angles $\sin^2\theta_{12},\ \sin^2\theta_{23}$ and $\sin^2\theta_{13}$ on $\sum m_i$.  }
  \label{C5anglevssum}
\end{figure}


The variation of Jarlskog invariant $J$ with $\sum m_i$ is shown in figure \ref{jplot}. The other two invariants $I_1$ and $I_2$ related to the Majorana phases are also shown in figure \ref{i1vsi2}. The model predictions of Dirac and Majorana CP-violating phases are solved from these invariants, as shown in figure \ref{C5phase}. In the MES mechanism, the lightest neutrino mass is always zero ($m_1 = 0$ for NH and $m_3=0$ for IH). It implies that one of the two Majorana phases is vanishing ($\alpha \simeq 0^o$ or $360^o$). From figure \ref{delphi2}, we can observe that the Majorana phase $\beta$ varies almost linearly with the Dirac CP-violating phase $\delta_{CP}.$ Figure \ref{phi1phi2} shows the correlation between Majorana phases $\alpha$ and $\beta$. We observe a vanishing $\alpha$ near $0^o$ or $360^o$ while $\beta$ varies continuously from $0^o $ to $ 360^o$.


In the sterile neutrino sector, the input parameters in (\ref{inputs}) evaluate the sterile neutrinos in the eV mass scale. The active-sterile mixing elements $\vert U_{14}\vert^2,\ \vert U_{24}\vert^2$ and $\vert U_{34}\vert^2$ are shown as variation plot vs the sterile neutrino mass $m_4$, in figure \ref{C5ui4}. These plots suggest that our model can reproduce active-sterile mixings in the experimental bounds given in Table \ref{data}. %The active-sterile mixing angles $\sin^2\theta_{14}$ (red points), $\sin^2\theta_{24}$ (blue points) and $\sin^2\theta_{34}$(black points) are shown as a function of sterile neutrino mass $m_4$ in Fig.(). 

\begin{figure}
\centering
\begin{subfigure}[b]{0.43\textwidth}
    \includegraphics[width=\textwidth]{Jplot}
    \subcaption{}
    \label{jplot}
  \end{subfigure}
  \hspace{0.2cm}
  \begin{subfigure}[b]{0.43\textwidth}
    \includegraphics[width=\textwidth]{i1i2}
    \subcaption{}
   \label{i1vsi2}
  \end{subfigure}
  %
%    \captionsetup{justification=raggedright, singlelinecheck=false,  width=0.95\linewidth}
  \caption{\footnotesize Figures show the model predictions of Jarlskog invariant $J$, Majorana phase invariants $I_1$ and $I_2$. }
  \label{C5jplot}
\end{figure}
\begin{figure}
\centering
\begin{subfigure}[b]{0.43\textwidth}
    \includegraphics[width=\textwidth]{delphi2}
    \subcaption{}
    \label{delphi2}
  \end{subfigure}
  \hspace{0.2cm}
  \begin{subfigure}[b]{0.43\textwidth}
    \includegraphics[width=\textwidth]{phi1phi2}
    \subcaption{}
   \label{phi1phi2}
  \end{subfigure}
  %
%    \captionsetup{justification=raggedright, singlelinecheck=false,  width=0.95\linewidth}
  \caption{\footnotesize Figures show the model predictions of CP-violating Dirac phase $\delta_{CP}$ and Majorana phases $\phi_1$, $\phi_2$. }
  \label{C5phase}
\end{figure} 

The model predictions of effective mass parameters $m_{\beta\beta}$ and $m_{\beta}$ are shown in figure \ref{C5effmass}. The black data points represent the predictions for the 3+1 mixings, while the red data points are for the active neutrino sector. In the MES mechanism for IH, the contribution from $m_3$ vanishes as the lightest neutrino mass, $m_3=0$. There is a significant difference in the predictions for three neutrino and 3+1 neutrino mixings. It is observed that the presence of active-sterile mixings considerably increases the values of $m_{\beta}$ and $m_{\beta\beta}$. The dotted, horizontal red line in figure \ref{mbbplot} indicates the future sensitivity of the nEXO experiment. Based on the present model, most of the data points in 3+1 mixings are obtained inside nEXO and Legend-1K sensitivities, which have the potential to rule out or verify active-sterile mixings. However, predictions in three neutrino mixings still exceed this sensitivity. In the case of $m_{\beta}$, model predictions in 3+1 mixings are comparatively larger than active neutrino mixings. The effective mass parameters are observed in the ranges $1.45\ \mbox{meV}\leq m_{\beta}\leq 58.94\ \mbox{meV}$ and $1.09\ \mbox{meV} \leq m_{\beta\beta} \leq 58.08 $ meV for active neutrino mixing. In the case of 3+1 mixings, these parameters are observed in the ranges $11.13\ \mbox{meV}\leq m_{\beta}\leq 7487.80\ \mbox{meV}$ and $3.71\ \mbox{meV}\leq m_{\beta\beta}\leq 449.01\ \mbox{meV}$.


\subsection{$\chi^2$ analysis}\label{C}
Finally, the best-fit values of the neutrino observables and the corresponding best-fit values of model parameters $\tau$ and $g$ are evaluated using the $\chi^2$ analysis. We use the $\chi^2$ function defined as 
\begin{equation}
\chi^2(x_i) = \sum_{j}\left(\frac{y_j(x_i)-y_j^{bf}}{\sigma_j}\right)^2,
\label{chitest}
\end{equation}
where $x_i$ are the free parameters in the model and $j$ is summed over the observables $\{\sin^2\theta_{12},\ \sin^2\theta_{13},\ \sin^2\theta_{23},r\}$. Here, $y_j(x_i)$ denotes the model predictions for the observables, and $y_j^{bf}$ are their best-fit values obtained from the global analysis. $\sigma_j$ denotes the corresponding uncertainties obtained by symmetrizing $1\sigma$ range of the neutrino observables given in Table \ref{data}. By minimizing the $\chi^2$ function, we can calculate the best-fit values of our model parameters and predict the values of neutrino observables.
\begin{figure}
\centering
\begin{subfigure}[b]{0.43\textwidth}
    \includegraphics[width=\textwidth]{u14plot}
    \subcaption{}
    \label{u14plot}
  \end{subfigure}
  \hspace{0.2cm}
  \begin{subfigure}[b]{0.43\textwidth}
    \includegraphics[width=\textwidth]{u34u24}
    \subcaption{}
   \label{u34u24}
  \end{subfigure}
  %
%    \captionsetup{justification=raggedright, singlelinecheck=false,  width=0.95\linewidth}
  \caption{\footnotesize Figures show the variation of active-sterile mixing elements $\vert U_{i4}\vert^2$ ($i=1,2,3$) with the mass of sterile neutrino $m_4$. The shaded regions indicate the experimental bounds of $\vert U_{14}\vert^2$ in (a) and $\vert U_{24}\vert^2$ in (b). The magenta horizontal line in (b) represents the upper limit on $\vert U_{34}\vert^2$, given in Table \ref{data}. }
  \label{C5ui4}
\end{figure} 

\begin{figure}
\centering
\begin{subfigure}[b]{0.45\textwidth}
    \includegraphics[width=\textwidth]{mbbplot}
    \subcaption{}
    \label{mbbplot}
  \end{subfigure}
  \hspace{0.2cm}
  \begin{subfigure}[b]{0.45\textwidth}
    \includegraphics[width=\textwidth]{mbeta}
    \subcaption{}
   \label{mbplot}
  \end{subfigure}
  %
%    \captionsetup{justification=raggedright, singlelinecheck=false,  width=0.95\linewidth}
  \caption{\footnotesize Dependence of effective mass parameters $m_{\beta\beta}$ and $m_{\beta}$ with $\sum m_i$. }
  \label{C5effmass}
\end{figure}


At $\chi^2_{min}=3.76$, the values of the free parameters of the model are found at Re$[\tau]= 0.00735$, Im$[\tau]= 0.81780$ and $g= 3.66295$, while the Yukawa couplings are found to be $\vert y_1\vert=1.07157,\ \vert y_2\vert=1.12685$ and $\vert y_3\vert=0.59249$. The corresponding best-fit values of the neutrino observables are $\sin^2\theta_{23}=0.566,$ $\sin^2\theta_{12}=0.307,$ $ \sin^2\theta_{13}=0.023$ and $r = 0.172$. Lastly, Dirac and Majorana phases are observed at $\delta_{CP}=350.19^o,\ \alpha=355.75^o$ and $\beta=333.10^o$. 

We also evaluate the best-fit value of the $3\times 3$ active neutrino mixing matrix predicted from the model as 

\begin{equation}
\fl U_{bf} = \left(
\begin{array}{ccc}
 0.82303\, +0.00013 i & 0.54717\, +0.00109 i & 0.15231\, -0.00464 i \\
 0.36325\, +0.02537 i & -0.52165+0.169030 i & -0.11278-0.74430 i \\
 -0.41495+0.13355 i & 0.63201\, -0.02156 i & -0.04768-0.63858 i \\
\end{array}
\right)
\end{equation}

Again, the best-fit active-sterile mixing strength is given by 
\begin{eqnarray}
R_{bf} = \left(
\begin{array}{c}
 0.03445\, -0.00023 i \\
 0.00093\, -0.00149 i \\
 -0.00537-0.00987 i \\
\end{array}
\right)
\end{eqnarray}

We have summarized the best values of the model parameters and neutrino observables in Table \ref{C5bestfit}. The best-fit results suggest that the model prefers higher octant of $\theta_{23}$. The absolute masses of active neutrinos are predicted as $m_1 = 33.28$ meV, $m_2 = 33.79$ meV and $m_3 =0$ for IH while the sterile neutrino mass is predicted at $m_4 = 5.31$ eV. The best-fit values of the active-sterile mixing elements are obtained as $\vert U_{14}\vert = 0.079,\ \vert U_{24}\vert = 0.049 $ and $\vert U_{34}\vert = 0.009.$ 

\section{Summary and Discussion}\label{section6}
We have successfully constructed a new neutrino mass model based on modular $A_4$ by extending the SM with an $A_4$ triplet right-handed neutrino $N$ and a singlet sterile neutrino $S$ in the 3+1 scheme. The model only generates the active and eV-scale sterile neutrino masses in IH, while NH is not allowed at 3$\sigma$. Our primary motivation is to avoid the hypothetical scalar flavons using modular symmetry and, simultaneously, to reproduce all the neutrino observables through the vev of a single parameter $\tau,$ for a particular range of Yukawa coefficient factor $g.$ Two free model parameters, $\tau$ and $g$, are scanned randomly in a particular domain, and the active neutrino mass matrix is numerically diagonalized.

 We have conducted the numerical analysis using the 3$\sigma$ bounds of neutrino observables so that all the neutrino observables evaluated from the model simultaneously satisfy these bounds. Our analysis of neutrino masses is also consistent with the cosmological upper bound on the sum of neutrino masses $\sum m_i <0.12$ eV.

\begin{table*}
\centering
\begin{tabular}{|c|c|c|}
\hline 
Observables & Best-fit & 3$\sigma$ \\ 
\hline 
$\sin^2\theta_{23} $ & 0.566 & - \\ 

$\sin^2\theta_{13}$ & 0.023 & - \\ 

$\sin^2\theta_{12}$ & 0.307 & - \\ 

$r$ & 0.172 & - \\ 
$\delta_{CP}/^{o}$ & 350.19 & [180.36 , 359.73] \\
$\alpha /^{o}$ & 355.75 & [0 , 360] \\
$\beta /^{o}$ & 333.10 & [0 , 360] \\
$m_1$ & 33.28 meV & [1.464 , 59.464] meV\\
$m_2$ & 33.79 meV & [1.487 , 60.349] meV\\
$m_4$ & 5.31 eV & [0.061 , 210.717] eV \\
$\sum$ $m_i=m_1+m_2$ & 69.91 meV & [2.951, 119.814] meV\\
\hline 
Model parameters & Best-fit & 3$\sigma$ \\ 
\hline 
Re$[\tau]$ & 0.007 & [0.001 , 0.29 ] \\ 
Im$[\tau]$ & 0.817 & [0.68 , 1.59] \\ 
$\alpha^{\prime}$ & 6.13$\times 10^{-3}$ & [152.16 , 997.74] $\times 10^{-5}$  \\  
$\beta^{\prime}$ & 0.36 $\times 10^{-3}$ & [1.11 , 99.85]$\times 10^{-4}$ \\ 
$\gamma^{\prime}$ & 3.18 $\times 10^{-6}$& [31.19 , 785.28] $\times 10^{-5}$ \\
$\vert g\vert$ & 3.66 & [0.17 , 5.92]\\
$\phi_g$ & -0.41$\pi$ & [-0.94 $\pi$ , 0.96 $\pi$] \\
\hline 
\end{tabular} 
\caption{Best-fit values and 3$\sigma$ ranges of the model parameters and the corresponding predictions of neutrino observables for $\chi^2_{min} = 3.76 $. Neutrino observables $\sin^2\theta_{12},\sin^2\theta_{13},\sin^2\theta_{23}$ and $r$ are constrained by 3$\sigma$ values from experimental data.}
\label{C5bestfit}
\end{table*} 
 

The effects of an eV-scale sterile neutrino on the active neutrino mixing angles, effective mass parameters $m_{\beta},\ m_{\beta\beta}$, and unitarity of the active neutrino mixing matrix are studied. Neutrino mixing angles are evaluated from the $(4\times 4)$ active-sterile mixing matrix $V$ by including the non-unitarity effects of sterile neutrino. %Active-sterile mixing angles $\sin^2\theta_{14}$,  $\sin^2\theta_{24}$ and  $\sin^2\theta_{34}$ are observed to be  $0.000144\leq \sin^2\theta_{14}\leq 0.00236$,  $0.000133\leq \sin^2\theta_{24}\leq 0.00394$ and  $0.00023\leq \sin^2\theta_{34}\leq 0.00796$ for NH. In case of IH, these mixing angles are observed in the range $0.001032\leq \sin^2\theta_{14}\leq 0.002937$,  $0.00172\leq \sin^2\theta_{24}\leq 0.00359$ and  $0.00248\leq \sin^2\theta_{34}\leq 0.00606$.
Jarlskog invariant in the 3+1 sector is also determined. CP-violating Dirac phase is successfully predicted in the ranges $\delta_{CP} \sim (180.36^o - 359.73^o)$  after re-phasing the angles in the third quadrant. 
From the analysis of $m_{\beta\beta}$ and $m_{\beta}$, it is observed that the inclusion of mixing with sterile neutrino enhances the results within the future sensitivities of various experiments such as KamLAND-Zen, GERDA, Project 8, KATRIN, Legend-1K, nEXO, etc. If these experiments fail to detect any signal, the existence of eV scale sterile neutrino will be disfavoured.

Finally, we have used the minimum $\chi^2$ analysis to predict the best-fit values of the model parameters and the neutrino oscillation data. The best-fit analysis of the present model predicts a higher octant of $\theta_{23}$. This result agrees with the latest results from the improved NOVA experiment  \cite{acero2022improved}. The MES mechanism has an advantage in generating KeV-MeV scale sterile neutrino along with the active neutrino mass in the eV scale. The role of the singlet scalar $\zeta$ is to fix the scale of sterile neutrino mass and keep the term invariant under modular $A_4$. A study on the possibility of KeV-MeV sterile neutrino as a dark matter candidate will be addressed in future. We have successfully removed any ad-hoc triplet scalars in this work and the number of free parameters is significantly reduced. In conclusion, the modular $A_4$ symmetry successfully reproduces neutrino phenomenology and other issues beyond the Standard Model without needing extra flavons as in conventional discrete symmetry models.

\section*{Acknowledgements}
We would like to thank V.V. Vien, Dept. of Physics, Tay Nguyen University, Vietnam, for the useful suggestions and comments. One of the authors(MKS) would like to thank DST-INSPIRE, govt. of India for providing the fellowship for the research under INSPIRE fellowship (ID IF180349).

\section*{References}
\bibliographystyle{unsrt}
\bibliography{references}



\end{document}

