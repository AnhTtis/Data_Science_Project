\documentclass[aps,a4paper, preprint, superscriptaddress,preprintnumbers,amsmath,amssymb]{revtex4-2}
\usepackage{graphicx}
\usepackage{amsmath}
\usepackage{amsfonts}
\usepackage{amssymb}
\usepackage{url}
\usepackage{hyperref}
\usepackage{subfigure}
\usepackage{array}

%
\begin{abstract}
Motivated by the significance of modular symmetry in describing neutrino masses and flavor structure, we apply the $A_4$ modular symmetry in 3+1 scheme of active-sterile neutrino mixings. Neutrino oscillation observables in 3$\sigma$ range are successfully reproduced through the vev of the modulus $\tau$ in normal hierarchy(NH) as well as inverted hierarchy(IH). We have also studied other phenomenologies regarding effective  masses $m_{\beta}$ and $m_{\beta\beta}$ in neutrinoless double beta decay. Mixings between active and sterile neutrinos are analysed in detail and the mixing elements $\vert V_{i4}\vert $ are found to satisfy the experimental bounds. The best-fit values of the neutrino mixing angles and mass squared differences ratio are determined using minimum $\chi^2$ analysis. This model predicts best-fit values of the neutrino oscillation observables as $\sin^2\theta_{23}=0.572$, $ \sin^2\theta_{12}=0.313$, $\sin^2\theta_{13}=0.022$ and $r = 0.172$ for NH whereas $\sin^2\theta_{23}=0.602,$ $\sin^2\theta_{12}=0.288,$ $ \sin^2\theta_{13}=0.022$ and $r = 0.172$ for IH. Our analysis is also consistent with the latest Planck cosmological upper bound on the sum of neutrino masses $\sum m_i <0.12$ eV.
%%%%%%%%%%%%%%%%%%%%%%%%%%%%%%%%%%%%%%%%%%%%%%%%%%%%

\end{abstract}

\begin{document}

\title{Modular $A_4$ symmetry in 3+1 active-sterile neutrino masses and mixings}






\author{Mayengbam Kishan Singh}
\email{kishan@manipuruniv.ac.in}
 \affiliation{Department of Physics, Manipur University, Imphal-795003, India}                                    
  \author{S. Robertson Singh}
\email{robsoram@gmail.com}
  \author{N. Nimai Singh}
\email{nimai03@yahoo.com}  
 \affiliation{Research Institute of Science and Technology, Imphal - 795003, India}
  

%%%%%%%%%%%%%%%%%%%%%%%%%%%%%%%%%%%%%%%%%%%%%%%%%%%


\maketitle

\titlepage
\thispagestyle{empty}

%=============================================
\section{Introduction}
Origin of neutrino masses and flavour structure is one of the most important problems of Standard Model(SM) of particle physics. Since the discovery of neutrino oscillation in various experiments such as SNO, SuperKamiokande, etc., various extensions of SM have been studied. Models with non-Abelian discrete symmetries such as $A_4$\cite{altarelli2010discrete,king2007a4,babu2003underlying}, $S_4$\cite{ma2006neutrino,altarelli2009revisiting,bazzocchi2013neutrino}, $S_3$\cite{mohapatra2006s3,morisi2006flavor,grimus2005s3,koide2007s}, etc. found their distinct places in describing some of the experimental results as well as new predictions regarding unresolved problems of SM. Among these, the absolute mass scale of neutrino, Dirac CP violating phase, Baryogenesis, Dark matter, etc. are some of the main questions in high energy physics. 

Motivated by the observations of LSND \cite{aguilar2001evidence} and MiniBooNE \cite{aguilar2018significant,2021microboone}, many authors proposed the existence of a fourth state of neutrino called sterile neutrino. Sterile neutrinos are incorporated with the three neutrino theory in various literatures in a 3+1 scheme, 3+1+1, 3+2 etc. One of the simplest extension is the 3+1 scheme where a singlet sterile neutrino is added to the three active neutrinos and sterile neutrino gets mass through minimal extended seesaw mechanism(MES)\cite{Zhang2011}. Many authors have used discrete symmetries to describe the neutrino masses and its flavor structures. The main drawbacks of such approaches are the presence many hypothetical scalar fields as well as additional symmetry groups. Recently, an interesting method has been proposed in which modular symmetry is used along with the discrete symmetries as its subgroup.  An important feature of modular symmetry framework is that the Yukawa couplings can transform non-trivially as modular forms under the modular group. These modular forms are written as a function of a single parameter $\tau$ which is called the modulus. A minimal or no scalar flavons are required to break the symmetry as the lepton masses are generated from the symmetry breaking by the vev of the modulus $\tau$. Modular forms of level 3 denoted by $\Gamma(3)$ is isomorphic to $A_4$. Detailed analysis on $A_4$ modular groups and its application in neutrino model building are studied in Refs.  \cite{feruglio2019neutrino,kobayashi2018neutrino,kobayashi2018modular}. There are other works based on numerous modular groups $A_4$ \cite{zhang2020modular,Beherascoto,mishra2022type,kobayashi2020type,okada2020radiative,
nomura2019modular,abbas2021fermion,nomurascoto,behera2022implications,behera2022inverse}, $A_5$ \cite{ding2019neutrino,novichkov2019modular,PhysRevD.103.095013,PhysRevD.103.076005,wang2021double,Behera2021eut}, $S_4$\cite{wang2020minimal,penedo2019lepton,wang2021dirac,ding2019modular,
KobayasiS4,LiuS4}, etc. to study neutrino phenomenologies, dark matter, leptogenesis, etc. For instance, in Ref.\cite{abbas2021fermion},  modular $A_4$ is used  to study lepton and quark masses and mixing based on inverse seesaw mechanism, Ref.\cite{behera2022implications} studies lepton mixing and leptogenesis in linear seesaw while scotogenic dark matter scenario is studied in Ref.\cite{nomurascoto,Beherascoto}.


The special feature of this work is that we have performed a detailed study of the effects of an eV-scale sterile neutrino on neutrino phenomenology using modular $A_4$ symmetry. We have successfully reproduced neutrino oscillation data within the 3$\sigma$ bounds in normal hierarchy(NH) as well as Inverted hierarchy(IH). The particle contents and model predictions are different from other works. Dirac CP-violating phase and the two Majorana phases are also determined from the active-sterile mixing matrix. Finally, the best-fit values of the model parameters and neutrino observables are determined using the $\chi^2$ analysis. Significant results are obtained for the active-sterile mixing parameters $\vert V_{i4}\vert $, (where $i=1,2,3$) within the experimental bounds. There are literatures which study 3+1 active-sterile mixing using the general $A_4$ discrete symmetry group \cite{das2019active,mksingh,vien2021,vien2022b,das3}. However, to the best of our knowledge, modular $A_4$ group has not been used in MES mechanism  for 3+1 scheme. The structure of this manuscript is organised as follows. We present a detailed description of the model in section \ref{section2}, followed by the numerical analysis of the model in section \ref{section3}. Results of the analysis is presented in section \ref{section4}. We conclude with a brief summary and discussion in section \ref{section5}. 



\begin{table*}
\begin{center}
\renewcommand{\arraystretch}{1}
\begin{tabular}{c|c|c}
 \hline 
Parameter &	Normal Hierarchy (best-fit$\pm 1\sigma$) &	Inverted Hierarchy (best-fit$\pm 1\sigma$)  \\
\hline
$\vert\Delta m^2_{21}\vert: [10^{-5} eV^2]$ & 6.82 – 8.03 $(7.41^{+0.21}_{-0.20})$  &	 6.82 – 8.03 $(7.41^{+0.21}_{-0.20})$ \\
$\vert\Delta m^2_{31}\vert: [10^{-3} eV^2]$	& 2.428 – 2.597 $(2.511^{+0.028}_{-0.027})$  & 2.408 – 2.581 $(2.498^{+0.032}_{-0.025})$ \\

$\sin^2\theta_{12} $	& 0.270 – 0.341 $(0.303^{+0.012}_{-0.011})$ &  0.270 – 0.341 $(0.303^{+0.012}_{-0.011})$  \\
$\sin^2\theta_{23}$ &0.406 – 0.620	$(0.572^{+0.018}_{-0.023})$  &0.412 – 0.623 $(0.578^{+0.016}_{-0.021})$ 		 \\
$\sin^2\theta_{13}/10^{-2}$ & 2.029 – 2.391	$(2.203^{+0.056}_{-0.059})$ & 2.047 – 2.396 $(2.219^{+0.060}_{-0.057})$  \\
$\delta_{\rm CP}/^o$ &	108 - 404 $(197^{+42}_{-0.25})$	& 192 - 360 $(286^{+27}_{-32})$	 \\
$r=\sqrt{\frac{\Delta m_{21}^2}{\vert\Delta m_{32}^2\vert}} $ & 0.1675 - 0.1759 (0.1718)  & 0.1683 - 0.1765 (0.1722)\\
$\vert U_{14}\vert^2 $  & 0.012 - 0.047 & 0.012 - 0.047 \\
$\vert U_{24}\vert^2 $  & 0.005 - 0.03 &  0.005 - 0.03 \\
$\vert U_{34}\vert^2 $  & 0 - 0.16 & 0 - 0.16 \\
\hline
\end{tabular}  
\label{data}
\end{center}
\caption{Updated global-fit data for three neutrino oscillation, $Nufit$ 2022 \cite{nufit}. For 3+1 mixing, data taken from \cite{barrylight,vien2022b,Gariazzo_2016}.}
\end{table*} 


\section{Description of the model}\label{section2}
In this model, we use $A_4$ modular symmetry for studying lepton masses and mixings. We consider three right-handed neutrinos as triplet under $A_4$ and a sterile neutrino field $S$ as singlet. The SM lepton doublet $L$ is also transformed as triplet of $A_4$ while the Higgs $H_u$ and $H_d$ having hypercharge $+1/2 $ and $-1/2$ respectively are represented at singlets of $A_4$. The right-handed charged leptons $e_r, \mu_r$ and $\tau_r$ are assigned $1,1^{\prime\prime},1^{\prime}$ respectively. As a result, there are three independent coupling constants $\alpha^{\prime},\beta^{\prime}$ and $\gamma^{\prime}$ for the superpotential of the charged lepton sector. We have considered one scalar field $\eta$ as an $A_4$ singlet in order to generate sterile neutrino mass matrix $M_s$. Finally, the Yukawa coupling is transformed as a modular function $Y_i(\tau)$ which is triplet under $A_4.$ The role of triplet scalar field $\phi$ with zero modular weight is to simplify our analysis by making the charged lepton mass matrix diagonal. It does not have any significance in the neutrino sector.  The complete particle contents with their corresponding group charges and modular weights are given in Table \ref{Table1}. The modular forms $Y_i(\tau)$ of weight 2 which transform as triplet under $A_4$ can be expressed in terms of Dedekind eta-function $\eta(\tau)$ as \cite{feruglio2019neutrino} 
\begin{align}
y_1= & \frac{i}{2\pi}\left[\frac{\eta^{\prime}(\frac{\tau}{3})}{\eta (\frac{\tau}{3})}+ \frac{\eta^{\prime}(\frac{\tau+1}{3})}{\eta (\frac{\tau+1}{3})} + \frac{\eta^{\prime}(\frac{\tau+2}{3})}{\eta (\frac{\tau+2}{3})} - \frac{27\eta^{\prime}(3\tau)}{\eta(3\tau)}\right] \nonumber \\
y_2= &\frac{-i}{\pi}\left[\frac{\eta^{\prime}(\frac{\tau}{3})}{\eta (\frac{\tau}{3})}+\omega^2 \frac{\eta^{\prime}(\frac{\tau+1}{3})}{\eta (\frac{\tau+1}{3})} +\omega \frac{\eta^{\prime}(\frac{\tau+2}{3})}{\eta (\frac{\tau+2}{3})} \right]  \label{deta} \\
y_3= &\frac{-i}{\pi}\left[\frac{\eta^{\prime}(\frac{\tau}{3})}{\eta (\frac{\tau}{3})}+\omega \frac{\eta^{\prime}(\frac{\tau+1}{3})}{\eta (\frac{\tau+1}{3})} +\omega^2 \frac{\eta^{\prime}(\frac{\tau+2}{3})}{\eta (\frac{\tau+2}{3})} \right] \nonumber
\end{align}
where $\omega = e^{2\pi i/3}$ and $\eta(\tau)$ is defined as 
\begin{equation}
\eta(\tau)=q^{1/24}\prod_{n=1}^{\infty}(1-q^n),  \ \ \ \ q \equiv e^{i 2\pi \tau}.
\end{equation}
The overall coefficient in eq.(\ref{deta}) is one possible choice and it cannot be determined.


\begin{table*}
\begin{center}
\begin{tabular}{|m{2cm}|m{0.3cm}m{1cm}|m{1.5cm}|m{0.5cm}m{0.5cm}|m{0.5cm}m{0.5cm}|m{0.6cm}|}
\hline
$\frac{Fields}{Charges}$ & $L$ & $e_i$ & $H_{u,d}$ & $N_i$ & $S$ & $\phi$& $\zeta$ &$Y$  \\
\hline
$SU(2)_L$& 2 &1  & 2&1 & 1 &1 &1&1   \\
$A_4$& 3 &1,1$^{\prime\prime}$,1$^{\prime}$ &1 & 3 & 1 &3&1 &3  \\
$k_i$&$k_L$ &$k_e$  &$k_H$ &$k_N$ & $k_S$ & $k_{\phi}$ &$k_{\zeta}$& $k_Y$  \\
\hline
\end{tabular}
\end{center}
\caption{\centering Particle contents of the model and their group charges.}
\label{Table1}
\end{table*}

The modular invariant Lagrangian of the charged lepton is given below.
\begin{align}
\mathcal{L}_{c} =\ \alpha^{\prime} e_r (H_d \bar{L}\phi)_1 + \beta^{\prime} \mu_r (H_d\bar{L}\phi)_{1^{\prime\prime}} + \gamma^{\prime} \tau_r( H_d\bar{L}\phi)_{1^{\prime}}
\label{ml}
\end{align}
The charged lepton masses can be reproduced by adjusting the values the coefficients $\alpha^{\prime}, \beta^{\prime}$ and $\gamma^{\prime}$. Choosing the vev of $\phi$ along $(1,0,0)$, eq.(\ref{ml}) gives a diagonal charged lepton mass matrix 
\begin{equation}
M_L= diag(\alpha^{\prime},\beta^{\prime},\gamma^{\prime})\ v_{\phi}v_d
\end{equation}
For the neutrino sector, the invariant Lagrangian is 
\begin{align}
\mathcal{L}_{D}=&\ g H_u\bar{L}YN_i\\ \nonumber
=&\ g_1H_u  (\bar{L}Y)_{3S}N_i + g_2 H_u( \bar{L}Y)_{3A}N_i \\
\mathcal{L}_{R} =&\ \lambda Y N_{i}^c N_i \\
\mathcal{L}_s= &\ \delta \zeta Y S^cN_i
\end{align}
The weights $k_i$ should satisfy the following conditions,
\begin{align}
k_e + k_H+k_L+k_{\phi} =0\\
k_H+k_L+k_N= k_Y\\
k_N+k_N=k_Y\\
k_{\zeta}+k_S+k_N=k_Y
\end{align}
For $k_Y = 2$,  we get 
\begin{equation}
k_N=k_L=1, \ k_{\phi}=k_H=0,k_S=2,\ k_{\zeta}=-1, \ \mbox{and} \ k_e=-1
\end{equation}
The neutrino mass matrices are
\begin{align*}
M_D = & v_u \left(
\begin{array}{ccc}
 2 g_1 y_1 & y_3 (g_2-g_1) & y_2 (-g_1-g_2) \\
 y_3 (-g_1-g_2) & 2 g_1 y_2 & y_1 (g_2-g_1) \\
 y_2 (g_2-g_1) & y_1(-g_1-g_2) & 2 g_1 y_3 \\
\end{array}
\right); \\
M_R=& \lambda\left(
\begin{array}{ccc}
 2y_1 & -y_3 & -y_2 \\
 -y_3 & 2 y_2 & -y_1 \\
 -y_2 & -y_1 & 2 y_3 \\
\end{array}
\right);\\
M_s =& \delta v_{\zeta}\left(
\begin{array}{ccc}
y_1 & y_2 & y_3 \\
\end{array}
\right).
\end{align*}

In the 3+1 MES mechanism, the active neutrino mass and the sterile neutrino mass are calculated by the following relations \cite{Zhang2011}
\begin{align}
m_{\nu} \simeq &\  M_DM_R^{-1}M_S^T\left(M_S M_R^{-1}M_S^T\right)^{-1}M_S\left(M_R^{-1}\right)^T M_D^T -M_DM_R^{-1}M_D^T\ ;
\label{m} \\ 
m_s \simeq & - M_SM_R^{-1}M_S^T.
\label{ms}
\end{align}

In the 3+1 neutrino framework, the $(4\times 4)$ active-sterile mass matrix is diagonalised by a unitary $(4\times 4)$ mixing matrix given by \cite{v441982}
\begin{equation}
V \simeq \left(\begin{matrix}
(1-\frac{1}{2}RR^{\dagger})U & R \\ 
-R^{\dagger}U & 1-\frac{1}{2}R^{\dagger}R
\end{matrix} \right),
\label{V44}
\end{equation}
where $R$ represents the strength of active-sterile mixing given by
\begin{align}
R=&M_DM_R^{-1}M_S^T(M_SM_R^{-1}M_S^T)^{-1}
\end{align}
Deviation of the active neutrino mixing matrix $U$ from unitarity due to the presence of sterile neutrino is determined by $\frac{1}{2}RR^{\dagger}$. The $4\times 4$ neutrino mixing matrix can be parameterized by six mixing angles $(\theta_{12},\theta_{13}, \theta_{23},\theta_{14},\theta_{24}, \theta_{34})$, three Dirac phases $(\delta_{13},\delta_{14},\delta_{24})$ and three Majorana phases $(\alpha,\beta,\gamma)$ \cite{gariazzo2016light} as,
\begin{equation}
V^{4\times 4} = \left(\begin{matrix}
c_{12}c_{13}c_{14} & c_{13}c_{14}s_{12}e^{i\frac{\alpha}{2}} & c_{14}s_{13} e^{i\frac{\beta}{2}} & s_{14}e^{-i\frac{\gamma}{2}} \\ 
U_{\mu 1} & U_{\mu 2} & U_{\mu 3} & c_{14}s_{24}e^{-i\left(\frac{\gamma}{2}-\delta_{14}+\delta_{24}\right)} \\ 
U_{\tau 1} & U_{\tau 2} & U_{\tau 3} & c_{14}c_{24}s_{34}e^{-i\left(\frac{\gamma}{2}-\delta_{14}\right)} \\ 
U_{s1} & U_{s2} & U_{s3} & c_{14}c_{24}c_{34}e^{-i\left(\frac{\gamma}{2}-\delta_{14}\right)}
\end{matrix} \right).
\label{V44p}
\end{equation}
The six neutrino mixing angles can be determined from the mixing elements of $V$ using the following relations \cite{DEV2019401},
\begin{align} \label{anglessolve}
\sin^2\theta_{14}\ &=\ \vert V_{e4}\vert ^2,\ \  
\sin^2\theta_{24}\ =\ \frac{\vert V_{\mu 4}\vert ^2}{1-\vert V_{e4}\vert ^2},\ \ 
\sin^2\theta_{34}\ =\ \frac{\vert V_{\tau 4}\vert ^2}{1-\vert V_{e4}\vert ^2-\vert V_{\mu 4}\vert ^2}\nonumber \\
\sin^2\theta_{12}\ &=\ \frac{\vert V_{e2}\vert ^2}{1-\vert V_{e4}\vert ^2-\vert V_{e3}\vert ^2}, \ \ 
\sin^2\theta_{13}\ =\ \frac{\vert V_{e3}\vert ^2}{1-\vert V_{e4}\vert ^2},
\end{align}
\begin{align}
\sin^2\theta_{23}\ =&\ \frac{\vert V_{e3}\vert ^2(1-\vert V_{e4}\vert ^2)-\vert V_{e4}\vert ^2\vert V_{\mu 4}\vert ^2}{1-\vert V_{e4}\vert ^2-\vert V_{\mu 4}\vert ^2}+ \frac{\vert V_{e1}V_{\mu 1}+V_{e2}V_{\mu 2}\vert ^2(1-\vert V_{e4}\vert ^2)}{(1-\vert V_{e4}\vert ^2-\vert V_{e3}\vert ^2)(1-\vert V_{e4}\vert ^2-\vert V_{\mu 4}\vert ^2)}.\nonumber 
\end{align} 
where $V_{ij}$ are the elements of mixing matrix in eq.(\ref{V44}). We also try do determine the Jarlskog invariant $J$ for the active sterile mixing. According to the parameterisation in eq.(\ref{V44p}), the Jarlskog invariant $J_{3+1} = Im[V_{e1}V_{\mu 2}V^*_{e2}V^*_{\mu 1}]$   takes the form \cite{KUMAR2020115082}
\begin{equation}
J_{3+1} = J_3^{cp} c_{14}^2c_{24}^2 + s_{24}s_{14}c_{24}c_{23}c^2_{14}c^3_{13}c_{12}s_{12}\sin(\delta_{14}-\delta_{24}),
\end{equation}
where $J_3^{cp} = s_{23}c_{23}s_{12}c_{12}s_{13}c_{13}^2 \sin \delta_{13}$ is the Jarlskog invariant for the three neutrino framework and $s_{ij} = \sin\theta_{ij},c_{ij}=\cos\theta_{ij}$ are the mixing angles. The two physical Majorana phases $\alpha$ and $\beta$ are determined from $V^{4\times 4}$ using the invariants $I_1$ and $I_2$ defined as follows 
\begin{align}\label{majoranaphase1}
I_1 = Im[V_{e1}^*V_{e2}]\ =\ c_{12} c_{13}^2 c_{14}^2 s_{12} \sin\left(\frac{\alpha}{2}\right), \\ 
I_2 = Im[V_{e1}^*V_{e3}]\ =\ c_{12}c_{13}c_{14}^2s_{13}\sin\left(\frac{\beta}{2}-\delta_{13}\right).
\label{majoranaphase2}
\end{align}


Another important parameter in neutrino physics is the effective neutrino mass $m_{\beta\beta}$ and effective electron mass $m_{\beta}$. A combined analysis from KAMLand-Zen\cite{Kamland} and GERDA provided an upper bound on $m_{\beta\beta}$ in the range $m_{\beta\beta} < (0.071-0.161$) eV \cite{agostini2018improved,goswami}. Recent result from latest KATRIN \cite{Katrin2020} experiment  constrains the effective electron neutrino mass $m_{\beta}$ to be less than 1.1 eV. These parameters are determined from the neutrinoless double beta decay and beta decay respectively using the relations
\begin{equation}
m_{\beta\beta}=\vert\sum_{j=1}^4\vert V_{ej}\vert ^2m_j\vert,
\label{mbbeq}
\end{equation}

\begin{equation}
m_{\beta} = \left(\sum_{i=1}^4\vert V_{ei}\vert ^2 m_{i}^2\right)^{1/2}.
\label{mbeq}
\end{equation}

\begin{figure}
\subfigure[]{
\includegraphics[width=.45\textwidth]{y2y1vsy3.pdf}}
\quad
\subfigure[]{
\includegraphics[width=.45\textwidth]{y1vsretau.pdf}}
\quad
\subfigure[]{
\includegraphics[width=.45\textwidth]{y1y2vsretau.pdf}}
\quad
\subfigure[]{
\includegraphics[width=.45\textwidth]{y123vsimtau.pdf}}
\quad
\caption{\footnotesize{Variation plot showing the dependence of Yukawa ($y_1, y_2,y_3$) among themselves in (a). Relation of $y_1$ with Re[$\tau$]  is shown in (b) while (c) shows the variation of $y_2,y_3$ with Re[$\tau$]. The dependence of Im[$\tau$] with ($y_1, y_2,y_3$) is shown in (d).} for NH.}
\label{yplotnh}
\end{figure}

\section{Numerical analysis}\label{section3}
In this section, we carry out the detailed numerical analysis to determine the allowed regions of the free parameters in the model, which satisfy the current neutrino oscillation data. The lepton mass matrices in eq.(\ref{ml}) and eq.(\ref{m}) depend on the effective parameters $\alpha^{\prime}, \beta^{\prime}, \gamma^{\prime}, g_1, g_2 $ and $\tau.$ Parameters $\alpha^{\prime}, \beta^{\prime}, \gamma^{\prime}$ in the charged lepton sector can be taken as real and they are determined by the ratios of the charged lepton masses $m_e/m_{\tau}$ and $m_{\mu}/m_{\tau}$. On fixing $\tau$, the modular symmetry is broken and the neutrino mass eigenvalue, mixing angles and Dirac as well as Majorana phases are completely determined. We define the ratio of neutrino mass squared differences $r= \sqrt{\Delta m_{21}^2/\Delta m_{31}^2}= m_2/m_3$ for NH $(m_1 \approx 0<< m_2 < m_3<m_4)$ and $ r=\sqrt{\Delta m_{21}^2 /\vert\Delta m_{32}^2\vert}= \sqrt{1-\frac{m_1^2}{m_2^2}}$ for IH $(m_3\approx0<< m_1<m_2<m_4)$. The absolute scale of active neutrino masses can be fixed by adjusting the overall factor $v_u^2g_1/\lambda.$ The neutrino mixing angles and $r$ depend on only two free parameters $g_2/g_1$ and the modulus $\tau.$  Since these parameters are complex we consider
\begin{equation}
\tau=Re[\tau] + i\ Im[\tau],\ \  \frac{g_2}{g_1}= ge^{i \phi_g} 
\end{equation} 
The fundamental domain of $\tau$ is given in Ref.\cite{feruglio2019neutrino}. We randomly scan the parameters in the range Re$[\tau]=[-1.5,1.5]$ for both ordering. Further, we take $\ g = [0.1,3] $ and Im$[\tau]=[0.6,3]$ for NH while $g = [1.5,2] $, Im$[\tau]=[0.6,1]$ for IH . We also consider $\phi_g = [-\pi,\pi]$. The active neutrino mass matrix is numerically diagonalised using the relation $m_{\nu}= U m_{\nu}^{d}U^{\dagger}$, where $m_{\nu}^d = diag(m_1,m_2,m_3)$, and $m_1,m_2,m_3$ are the neutrino mass eigenvalues.  The mixing angles can be calculated using the general formula given in eq.(\ref{anglessolve}). We filter the allowed values of the model parameters using the 3$\sigma$ bounds of the three mixing angles and ratio $r$ given in Table \ref{data}. For the sterile sector, the coefficient $k \equiv\delta v_{\zeta}$ is solved by constraining the sterile neutrino mass $m_s$ in the range $[0.8, 10]$eV. Finally, the best-fit values of the neutrino observables and the corresponding best-fit values of model parameters $\tau$ and $g$ are evaluated using the $\chi^2$ analysis. We use the $\chi^2$ function given by 
\begin{equation}
\chi^2(x_i) = \sum_{j}\left(\frac{y_j(x_i)-y_j^{bf}}{\sigma_j}\right)^2
\label{chitest}
\end{equation}
 where $x_i$ are the free parameters in the model and $j$ is summed over the observables $\{\sin^2\theta_{12},\sin^2\theta_{13},\sin^2\theta_{23},r\}$. Here, $y_j(x_i)$ denotes the model predictions for the observables and $y_j^{bf}$ are their best-fit values obtained from the global analysis. $\sigma_j$ denotes the corresponding uncertainties obtained by symmetrizing $1\sigma$ range of the neutrino observables given in Table \ref{data}. By minimizing the overall $\chi^2$ function, we can calculate the best-fit values of our model parameters and predict the values of neutrino observables.
 

\begin{figure}
\subfigure[]{
\includegraphics[width=.45\textwidth]{yvsimtauih.pdf}}
\quad
\subfigure[]{
\includegraphics[width=.45\textwidth]{yvsretauih.pdf}}
\quad
\caption{\footnotesize{(a) and (b) show the dependence of Yukawa ($y_1, y_2,y_3$) with Im[$\tau$] and Re[$\tau$] respectively for IH.}}
\label{yplotih}
\end{figure}

\section{Results}\label{section4}
The range of Yukawa couplings which satisfy all the neutrino observables are shown in Fig.(\ref{yplotnh})-(\ref{yplotih}) as variation plot with Re[$\tau$] and Im[$\tau$] respectively for NH and IH respectively. It is observed that the couplings lie in the ranges $0.984\leq \vert y_1(\tau)\vert\leq 0.989$, $ 0.593\leq \vert y_2(\tau)\vert\leq 0.662$, $ 0.178\leq \vert y_3(\tau)\vert\leq 0.222$ for NH. The allowed values of Im$(\tau)$ continuously vary in the range $1.048 - 1.102$ while the values of Re$(\tau)$ are concentrated at specific regions around $\pm 0.5$ and $\pm 1.5$ in the fundamental domain of $\tau$ as evident from Fig.({\ref{yplotnh})(b) and (c). Predicted values of neutrino mixing angles $\sin^2\theta_{13}$ are plotted with $\sin^2\theta_{12}$ and $\sin^2\theta_{23}$ for both orderings in Fig.(\ref{angles}). From these plots, the values of $\sin^2\theta_{23}$ are concentrated at regions greater than $0.55$ implying the higher octant of $\theta_{23}$.  Plots between individual neutrino masses $m_2, m_3$ for NH ($m_1=0$) and $m_1, m_2$ for IH ($m_3=0$) with sum of active neutrino masses $\sum m_{i}$  are shown respectively in Fig.(\ref{mass}). Choosing the factor $v_u^2g_1/\lambda \sim \mathcal{O}(10^{-2})$, we observe that the sum of active neutrino masses $\sum m_{i}$ satisfies the Planck upper bound $\sum m_{i} < 0.12$eV. For NH, sum of active neutrino masses is observed in a very narrow range $\sum m_{i} \sim (0.0145 - 0.045 )$eV while it is $\sum m_{i} \sim (0.083 - 0.121 )$eV for IH. The individual neutrino masses are found in the range $ m_2 \sim (0.00214 - 0.00658)$eV, $ m_3 \sim (0.0124 - 0.0382)$eV  for NH and $ m_1 \sim (0.041 - 0.06)$eV, $ m_2 \sim (0.042 - 0.061)$eV for IH. For the mixings between active and sterile neutrino, we have plotted the active-sterile mixing elements $\vert V_{14}\vert, \vert V_{24}\vert $ and $\vert V_{34}\vert$ in Fig.(\ref{vi4}) as a function of sterile neutrino mass $m_s$. It is evident from these plots that the active-sterile mixing elements satisfy the observed bounds given in Table.\ref{Table1}.  Variation of  active-sterile mixing angles $\sin^2\theta_{i4}$ ($i=1,2,3$) with parameter $k$ is shown in Fig.(\ref{si4}) for both ordering. The non-unitarity effect due to the presence of active-sterile mixing is determined by $\frac{1}{2}RR^{\dagger}$. For our analysis it is observed to be less than $\mathcal{O}(10^{-3}).$ It is also important to note from the plots shown above that the number of data points in case of NH is much larger compared to IH. Thus, we can infer that our model favours NH.

\begin{figure}
\subfigure[]{
\includegraphics[width=.45\textwidth]{anglesnew.pdf}}
\quad
\subfigure[]{
\includegraphics[width=.45\textwidth]{anglesih.pdf}}
\quad
\caption{\footnotesize{  Plot between mixing angle $\sin^2 \theta_{13}$ with $\sin^2 \theta_{23}$ and $\sin^2 \theta_{12}$ for NH in (a) and IH in (b).}}
\label{angles}
\end{figure}
\begin{figure}
\subfigure[]{
\includegraphics[width=.45\textwidth]{m1m2vssum.pdf}}
\quad
\subfigure[]{
\includegraphics[width=.45\textwidth]{m1m2vssumih.pdf}}
\quad
\caption{\footnotesize{ Variation between neutrino masses $m_2$ and $m_3$ ($m_1 =0$) with sum of active neutrino masses $\sum m_i $ for NH is shown in (a). For IH, (b) shows plot between $m_1$ and $m_2$ ($m_3=0$) with $\sum m_i $. The vertical dotted line is the cosmological Planck upper bound $\sum m_i<0.12$eV.}}
\label{mass}
\end{figure}

Proceeding further, we have calculated the model prediction of the Jarlskog invariant in the 3+1 scenario. Fig.(\ref{J})(a) shows the variation of mixing angles with $J_{3+1}$ for NH.  Prediction of Dirac CP-violating phase $\delta_{13}$ is shown in Fig.(\ref{phase})(a). Majorana phases $\alpha$ and $\beta$ are also evaluated using eq.(\ref{majoranaphase1}) - (\ref{majoranaphase2}) and the predictions from our model are shown in Fig.(\ref{phase})(b). For the case of IH, the variation of active neutrino mixing angles with Jarlskog invariant $J_{3+1}$ is shown in Fig.(\ref{J})(b). The value of $J_{3+1}$ is observed in the vicinity of $\pm 0.06$ which is out of the bounds for $J_{max}^{CP} = 0.0332 \pm 0.0008$ at 1$\sigma$  for both orderings \cite{nufit}.


Other phenomenological studies on neutrinoless double beta decay experiments are also carried out. The effective mass parameter $m_{\beta\beta}$ and effective electron mass $m_{\beta}$ are determined from the using eq.(\ref{mbbeq})-(\ref{mbeq}). Unknown phases for the active-sterile sector $\delta_{14}$ and $\delta_{24}$ are randomly chosen in the range $[-\pi,\pi]$. Fig.(\ref{mbnh}) shows the variation of $m_{\beta\beta}$ and $m_{\beta}$ as a function of $\sum m_{i}$ for NH. It is observed that these effective mass parameters lie in the range $0.03$ eV$ \leq m_{\beta}\leq 0.18$ eV and $0.001$ eV $\leq m_{\beta\beta}\leq 0.006$ eV. In case of IH shown in Fig.(\ref{mbih}), the effective parameters are in range  $0.179$ eV$ \leq m_{\beta}\leq 0.413$ eV and $0.005$ eV $\leq m_{\beta\beta}\leq 0.017$ eV. Values of $m_{\beta\beta}$ and $m_{\beta}$ are observed to be comparatively larger in case of IH.

\begin{figure}
\subfigure[]{
\includegraphics[width=.45\textwidth]{s14s24.pdf}}
\quad
\subfigure[]{
\includegraphics[width=.45\textwidth]{s14s34.pdf}}
\quad
\caption{\footnotesize{ Plot between active-sterile mixing angle $\sin^2 \theta_{14}$ with $\sin^2 \theta_{24}$ and $\sin^2 \theta_{34}$ for NH.}}
\label{anglemass}
\end{figure}

\begin{figure}
\subfigure[]{
\includegraphics[width=.45\textwidth]{si4vsk.pdf}}
\quad
\subfigure[]{
\includegraphics[width=.45\textwidth]{si4vskih.pdf}}
\quad
\caption{\footnotesize{ Variation between active-sterile mixing angle $\sin^2 \theta_{i4}$ where $i=1,2,3$ with sterile neutrino mass coefficient $k=\delta v_{\zeta}$ for NH in (a) and IH in (b).}}
\label{si4}
\end{figure}


\begin{figure}
\subfigure[]{
\includegraphics[width=.45\textwidth]{vi4vsms.pdf}}
\quad
\subfigure[]{
\includegraphics[width=.45\textwidth]{msvsvi4ih.pdf}}
\quad
\caption{\footnotesize{  Dependence of active-sterile mixing elements $V_{i4}$ where $i=1,2,3$ on sterile neutrino mass $m_s$ for NH in (a) and IH in (b).}}
\label{vi4}
\end{figure}

\begin{figure}
\subfigure[]{
\includegraphics[width=.45\textwidth]{Jvss23s12.pdf}}
\quad
\subfigure[]{
\includegraphics[width=.45\textwidth]{anglesvsjih.pdf}}
\quad
\caption{\footnotesize{ (a). Plot between Jarlskog invariant $J_{3+1}$ in the active-sterile sector with active neutrino mixing angles $\sin^2 \theta_{12}$ and $\sin^2 \theta_{23}$ for NH. (b). Variation of $J_{3+1}$ with $\sin^2 \theta_{13}, \sin^2 \theta_{12}$ and $\sin^2 \theta_{23}$ for IH.}}
\label{J}
\end{figure}

\begin{figure}
\subfigure[]{
\includegraphics[width=.45\textwidth]{deltas23.pdf}}
\quad
\subfigure[]{
\includegraphics[width=.45\textwidth]{majorana.pdf}}
\quad
\caption{\footnotesize{(a). Variation of CP-violating Dirac phase $\delta_{13}$ with  $\sin^2 \theta_{23}$ for NH. (b). Plot between the two Majorana phases $\alpha$ and $\beta$ for NH.}}
\label{phase}
\end{figure}
\begin{figure}
\subfigure[]{
\includegraphics[width=.45\textwidth]{mivsmb.pdf}}
\quad
\subfigure[]{
\includegraphics[width=.45\textwidth]{mivsmbb.pdf}}
\quad
\caption{\footnotesize{ Dependence of (a). Effective electron mass $m_{\beta}$ with $\sum m_i$ and (b). Effective neutrino mass $m_{\beta\beta}$ from $\nu 0\beta\beta$ with  $\sum m_i$ for NH. }}
\label{mbnh}
\end{figure}

\begin{figure}
\subfigure[]{
\includegraphics[width=.45\textwidth]{mbvssumih.pdf}}
\quad
\subfigure[]{
\includegraphics[width=.45\textwidth]{mbbvssumih.pdf}}
\quad
\caption{\footnotesize{ Dependence of (a). Effective electron mass $m_{\beta}$ with $\sum m_i$ and (b). Effective neutrino mass $m_{\beta\beta}$ from $\nu 0\beta\beta$ with  $\sum m_i$ for IH.}}
\label{mbih}
\end{figure}


Finally, the best-fit values of the model parameters and predictions of neutrino observables are evaluated using $\chi^2$ analysis as shown in eq.(\ref{chitest}). For NH at $\chi^2_{min}=1.20$, the best-fit of the parameter $\tau$ is obtained at Re$[\tau]= -0.45$, Im$[\tau]= 1.07$ and $g= 1.49$. The corresponding values of Yukawa are found to be $\vert y_1\vert=0.987,\vert y_2\vert=0.644,\vert y_3\vert=0.197$. The best-fit values of neutrino observables are obtained at $\sin^2\theta_{23}=0.572$, $ \sin^2\theta_{12}=0.313$, $\sin^2\theta_{13}=0.022$ and $r = 0.172$. Similarly, for IH at $\chi^2_{min}=3.54$, the values of the model parameters are found as Re$[\tau]= 0.35$, Im$[\tau]= 0.87$ and $g= 1.89$, while the Yukawa are found to be $\vert y_1\vert=0.969,\vert y_2\vert=0.951,\vert y_3\vert=0.466$. The corresponding best-fit values of the neutrino observables are $\sin^2\theta_{23}=0.602,$ $\sin^2\theta_{12}=0.288,$ $ \sin^2\theta_{13}=0.022$ and $r = 0.172$.

For the charged lepton sector, by comparing eq.(\ref{ml}) with the experimental values for masses of the charged leptons given in Ref.\cite{pdg2022}, $m_e = 0.51099$ MeV, $m_{\mu} = 105.65837$ MeV and $m_{\tau}= 1776.86$ MeV, the parameters $\alpha^{\prime}, \beta^{\prime}, \gamma^{\prime}$ are determined to be 
\begin{equation}
\frac{\alpha^{\prime}}{\gamma^{\prime}} = 0.287\times 10^{-2},\ \  \ \ \ \frac{\beta^{\prime}}{\gamma^{\prime}} = 0.059.
\end{equation}




\section{Summary and Discussion}\label{section5}
We have successfully constructed a new neutrino model based on modular $A_4$ by extending the SM with an $A_4$ triplet right-handed neutrino $N$ and a singlet sterile neutrino in a 3+1 scheme. MES mechanism is imposed to generate the active and eV-scale sterile neutrino masses in both NH and IH. The main motivation is to avoid hypothetical scalar flavons using modular symmetry and at the same time, reproduce all the neutrino observables through the vev of a single parameter $\tau.$ However, to simplify our analysis, we have used a triplet scalar $\phi$ whose vev makes the charged lepton mass matrix diagonal.  Two free model parameters $\tau$ and $g$ are scanned randomly in a particular  domain and active neutrino mass matrix is numerically diagonalised.  We have conducted the numerical analysis using the 3$\sigma$ bounds of neutrino observables in such a way that all the neutrino observables evaluated from the model simultaneously satisfy these bounds.  Our analysis of neutrino mass is also consistent with the cosmological upper bound on the sum of neutrino masses $\sum m_i <0.12$ eV. Effects of the eV-scale sterile neutrino on neutrino mixing angles, effective mass parameters $m_{\beta}, m_{\beta\beta}$ and unitarity of the active neutrino mixing matrix are studied in detail.  Neutrino mixing angles are evaluated from the $(4\times 4)$ active-sterile mixing matrix $V$ without avoiding the non-unitarity effects of sterile neutrino. Active-sterile mixing angles $\sin^2\theta_{14}$,  $\sin^2\theta_{24}$ and  $\sin^2\theta_{34}$ are observed to be  $0.000144\leq \sin^2\theta_{14}\leq 0.00236$,  $0.000133\leq \sin^2\theta_{24}\leq 0.00394$ and  $0.00023\leq \sin^2\theta_{34}\leq 0.00796$ for NH and $0.001032\leq \sin^2\theta_{14}\leq 0.002937$,  $0.00172\leq \sin^2\theta_{24}\leq 0.00359$. In case of IH, these mixing angles are observed in the range $0.00248\leq \sin^2\theta_{34}\leq 0.00606$. Jarlskog invariant in the 3+1 sector is also determined. CP-violating Dirac phase is successfully predicted in the ranges $\delta_{13} \sim (180.12^o - 230.70^o)$  and $\delta_{13} \sim (305.15^o - 358.83^o)$. Finally, we have used the minimum $\chi^2$ analysis to predict the best-fit values of the model parameters as well as the neutrino oscillation data. From our analysis, we can conclude that NH is more favoured in our model compared to IH. The MES mechanism has an advantage in generating keV-MeV scale sterile neutrino also along with the active neutrino mass in the eV scale. There are other works which study the possibility of keV-MeV sterile neutrino to be a dark matter candidate which we shall leave for future study. To conclude, we emphasize that modular $A_4$ is very successful in reproducing neutrino phenomenology and other beyond Standard Model problems without the need for extra flavons as in conventional discrete symmetry models.


\section*{Acknowledgements}
One of the authors(MKS) would like to thank DST-INSPIRE, govt. of India for providing the fellowship for the research under INSPIRE fellowship (ID IF180349).

\bibliographystyle{unsrt}
\bibliography{a4modular}

\end{document}