% CVPR 2023 Paper Template
% based on the CVPR template provided by Ming-Ming Cheng (https://github.com/MCG-NKU/CVPR_Template)
% modified and extended by Stefan Roth (stefan.roth@NOSPAMtu-darmstadt.de)

\documentclass[10pt,twocolumn,letterpaper]{article}

%%%%%%%%% PAPER TYPE  - PLEASE UPDATE FOR FINAL VERSION
%\usepackage[review]{cvpr}      % To produce the REVIEW version
\usepackage{cvpr}              % To produce the CAMERA-READY version
%\usepackage[pagenumbers]{cvpr} % To force page numbers, e.g. for an arXiv version

% Include other packages here, before hyperref.
\usepackage{graphicx}
\usepackage{amsmath}
\usepackage{amssymb}
\usepackage{booktabs}
\usepackage{multirow}
\usepackage{float}
\usepackage{tabularx}
\makeatletter
\@namedef{ver@everyshi.sty}{}
\makeatother
\usepackage{tikz}
\usepackage{comment}
\usepackage{color}
\usepackage{amsfonts}
\usepackage{caption}
\usepackage{mathtools}
\usepackage{tablefootnote}
\usepackage{dsfont}
\usepackage{ragged2e}
\usepackage{array}

\newcommand{\slo}[1]{\textcolor{red}{#1}}

% It is strongly recommended to use hyperref, especially for the review version.
% hyperref with option pagebackref eases the reviewers' job.
% Please disable hyperref *only* if you encounter grave issues, e.g. with the
% file validation for the camera-ready version.
%
% If you comment hyperref and then uncomment it, you should delete
% ReviewTempalte.aux before re-running LaTeX.
% (Or just hit 'q' on the first LaTeX run, let it finish, and you
%  should be clear).
\usepackage[pagebackref,breaklinks,colorlinks]{hyperref}

\newcommand{\OURS}{RoITr}


% Support for easy cross-referencing
\usepackage[capitalize]{cleveref}
\crefname{section}{Sec.}{Secs.}
\Crefname{section}{Section}{Sections}
\Crefname{table}{Table}{Tables}
\crefname{table}{Tab.}{Tabs.}


%%%%%%%%% PAPER ID  - PLEASE UPDATE
\def\cvprPaperID{1048} % *** Enter the CVPR Paper ID here
\def\confName{CVPR}
\def\confYear{2023}

\newcommand{\JI}[1]{\textbf{\textcolor{purple}{JI: #1}}}
\begin{document}

%%%%%%%%% TITLE - PLEASE UPDATE
\title{Rotation-Invariant Transformer for Point Cloud Matching}

\author{Hao Yu$^1$ \; Zheng Qin$^2$ \;  Ji Hou$^3$ \; Mahdi Saleh$^1$ \; Dongsheng Li$^2$ \; Benjamin Busam$^1$ \; Slobodan Ilic$^{1, 4}$ \\ \vspace{-.1cm} \\
$^{1}$TUM \quad $^{2}$NUDT \quad $^{3}$Meta Reality Labs \quad$^{4}$Siemens AG, Munich
}

%Institution1 address\\
%{\tt\small firstauthor@i1.org}
% For a paper whose authors are all at the same institution,
% omit the following lines up until the closing ``}''.
% Additional authors and addresses can be added with ``\and'',
% just like the second author.
% To save space, use either the email address or home page, not both
%First line of institution2 address\\
%{\tt\small secondauthor@i2.org}
\maketitle
\IEEEtitleabstractindextext{%
\begin{abstract}
Vision transformer (ViT) expands the success of transformer models from sequential data to images. The model decomposes an image into many smaller patches and arranges them into a sequence. Multi-head self-attentions are then applied to the sequence to learn the attention between patches. 
Despite many successful interpretations of transformers on sequential data, little effort has been devoted to the interpretation of ViTs, and many questions remain unanswered. For example, among the numerous attention heads, which one is more important? 
How strong are individual patches attending to their spatial neighbors in different heads? What attention patterns have individual heads learned? 
In this work, we answer these questions through a visual analytics approach. Specifically, we first identify \textbf{\textit{what}} heads are more important in ViTs by introducing multiple pruning-based metrics. 
Then, we profile the spatial distribution of attention strengths between patches inside individual heads, as well as the trend of attention strengths across attention layers.
Third, using an autoencoder-based learning solution, we summarize all possible attention patterns that individual heads could learn. Examining the attention strengths and patterns of the important heads, we answer \textbf{\textit{why}} they are important. 
Through concrete case studies with experienced deep learning experts on multiple ViTs, we validate the effectiveness of our solution that deepens the understanding of ViTs from \textit{head importance}, \textit{head attention strength}, and \textit{head attention pattern}.
\end{abstract}

% Note that keywords are not normally used for peerreview papers.
\begin{IEEEkeywords}
Vision transformer, multi-head self-attention, deep learning, explainable artificial intelligence, visual analytics.
\end{IEEEkeywords}}
\section{Introduction}
\label{sec:introduction}

Suppose that we want to \emph{fit} and \emph{validate} a model on the basis of a single dataset.  Two example scenarios are as follows:
\begin{list}{}{}
\item{\emph{Scenario 1.}} We  want to use the data both to generate and to test a hypothesis. 
\item{\emph{Scenario 2.}} We want to use the  data both to fit a complicated model, and to obtain an accurate estimate of the expected prediction error. 
\end{list}
In either case, it is clear that a naive approach that fits and validates a model on the same data is deeply problematic. In Scenario 1, testing a hypothesis on the same data used to generate it will lead to hypothesis tests that do not control the Type 1 error, and to confidence intervals that do not attain the nominal coverage \citep{fithian2014optimal}.   And in Scenario 2, estimating the expected prediction error on the same data used to fit the model will lead to massive downward bias  \citep[see][for recent reviews]{tian2020prediction,oliveira2021unbiased}.

In the case of Scenario 1, recent interest  has focused on \emph{selective inference}, a framework that enables a data analyst to generate and test a hypothesis on the same data \citep[see, e.g.,][]{taylor2015statistical}. The main idea is as follows: to test a hypothesis generated from the data, we should condition on the event that we selected this particular hypothesis. Despite promising applications of this framework to a number of problems, such as inference after regression \citep{lee2016exact}, changepoint detection \citep{jewell2022testing,hyun2021post}, clustering \citep{gao2020selective,chen2022selective,yun2023selective}, and outlier detection \citep{chen2020valid}, it suffers from some drawbacks: 
\begin{enumerate}
\item To perform selective inference, the  procedure used to generate the null hypothesis must be fully-specified in advance.  For instance, if a researcher wishes to cluster the data and then test for a difference in means between the clusters, as in \cite{gao2020selective} and \cite{chen2022selective}, then they must fully specify the clustering procedure (e.g., hierarchical clustering with squared Euclidean distance and complete linkage, cut to obtain $K$ clusters) in advance. 
\item Finite-sample selective inference typically requires an assumption of multivariate Gaussianity, though in some cases this can be relaxed to obtain asymptotic results \citep{taylor2018post,tian2017asymptotics,tibshirani2018uniform,tian2018selective}.
\end{enumerate}
Thus, it is clear that selective inference does not provide a flexible, ``one-size-fits-all" approach to Scenario 1. 

In the case of Scenario 2, proposals to de-bias the ``in-sample" estimate of expected prediction error  tend to be specialized to simple models, and thus do not provide an all-purpose tool that is broadly applicable to complex contemporary settings \citep{oliveira2021unbiased}.

\emph{Sample splitting} \citep{cox1975note} is an intuitive  approach that is broadly applicable to a variety of settings, including Scenarios 1 and 2; see the left-hand panel of Figure~\ref{fig:samplesplit_vs_datathin}. We split a dataset containing $n$ observations into two sets, containing $n_1$ and $n_2$ observations, respectively (where $n_1+n_2=n$). Then we can generate a hypothesis based on the first set and test it on the second set (Scenario 1), or we can fit a model to the first set and estimate its error on the second set (Scenario 2). Sample splitting also forms the basis for cross-validation, an important tool for a practicing data scientist \citep{hastie2009elements}. 

While sample splitting often can 
 adequately address both Scenarios 1 and 2, it also suffers from some drawbacks: 
\begin{enumerate} 
    \item If the data contain outliers, then each outlier is assigned to a single subsample. %Again, this may not be desirable.
    \item If the observations are not independent (for instance, if they correspond to a time series) then the subsamples that result from sample splitting are not independent, and so sample splitting does not provide a solution to either Scenario 1 or Scenario 2.
    \item If one is interested in drawing conclusions at a per-observation level, then sample splitting is unsuitable.  For example, if sample splitting is applied to a dataset consisting of the 50 states of the United States, then one can only conduct inference or perform validation on those states not used in fitting.
    \item If the model of interest is fit using  unsupervised learning, then  sample splitting may  not provide an adequate solution in either Scenario 1 or 2.  The issue relates to \#3 above. This is discussed in \cite{gao2020selective,chen2022selective}, and \cite{neufeld2022inference} in the context of Scenario 1. 
\end{enumerate}

In recent work, \cite{neufeld2023data} proposed an approach for \emph{convolution-closed data thinning} that addresses these drawbacks. They consider splitting, or \emph{thinning}, a random variable $X$ drawn from a convolution-closed family into $K$ independent random variables $\Xt{1},\ldots,\Xt{K}$ such that
$X=\sum_{k=1}^K\Xt{k}$, 
and $\Xt{1},\ldots,\Xt{K}$ come from the same family of distributions as $X$ (see the right-hand panel of Figure~\ref{fig:samplesplit_vs_datathin}). 
For instance, they show that $X \sim N(\mu, \sigma^2)$ can be thinned into two independent $N(\epsilon \mu, \epsilon \sigma^2)$ and $N((1-\epsilon) \mu, (1-\epsilon) \sigma^2)$ random variables that sum to $X$. 
Finally, and most critically, if  $X$ is drawn from a Gaussian, Poisson, negative binomial, binomial, multinomial, or gamma distribution, then they can thin it  \emph{even when parameters of its distribution are unknown}. Because the thinned random variables are independent, this
provides a new approach to tackle Scenarios 1 and 2:  
one thins the dataset into two independent datasets.  One then fits a model to one dataset, and  validates it on the other. 




On the surface, it is quite remarkable that one can break up a random variable $X$ into two or more {\em independent} random variables that sum to $X$ without knowing some (or sometimes any) of the parameters.  In this paper, we seek to explain the underlying principles that make this possible.  In doing so, we show that convolution-closed data thinning can be generalized so as to make it more flexible and much more widely applicable. The convolution-closed data thinning property $X=\sum_{k=1}^K\Xt{k}$ is desirable because it ensures that no information has been lost in the thinning process. However, clearly this would be equally true if we were to replace the summation by any other deterministic function.  Likewise, the fact that $\Xt{1},\ldots,\Xt{K}$ are from the same family as $X$, while convenient, is nonessential. 

Our generalization of convolution-closed data thinning is thus a procedure for splitting $X$ into $K$ random variables such that  the following two properties hold: 
$$
\text{  (i) } X=T(\Xt{1},\ldots,\Xt{K}); \text{ and  (ii) }\Xt{1},\ldots,\Xt{K} \text{ are mutually independent}.
$$
This generalization is broad enough to simultaneously encompass both convolution-closed data thinning and sample splitting. Furthermore, it greatly increases the scope of distributions that can be thinned. In the $K=2$ case, this generalized goal has been stated before \citep[see][``P1'' property]{leiner2022data} as we will describe later.  However, we are the first to develop a widely applicable strategy for achieving this goal.   Not only are we able to thin exponential families that were not previously possible (such as the  beta family), but we can even thin outside of the exponential family.  For example, generalized thinning enables us to thin $X \sim \text{Unif}(0, \theta)$ into
 $\Xt{k} \overset{\text{iid}}{\sim} \theta \cdot\text{Beta}\left(\frac{1}{K},1\right)$, for $k=1,\dots,K$, in such a way that $X=\max\{\Xt{1},\ldots,\Xt{K}\}$.




The primary contributions of our paper are as follows:
\begin{enumerate}
\item We propose \emph{generalized data thinning}, a general strategy for thinning a single random variable $X$ into two or more independent random variables, $\xo,\ldots,\Xt{K}$, without knowledge of the parameter value(s).  
%Unlike in \cite{neufeld2023data}, we recover the original random variable $X$ using the function $T(\xo,\ldots,\Xt{k})$, where $T(\cdot)$ need not be addition, and the distributions of the $\Xt{k}$'s need not be the same as $X$.  
Importantly, we show that {\em sufficiency} is the key property underlying the choice of the function $T(\cdot)$.
\item We show that it is possible to apply generalized data thinning to distributions far outside the scope of consideration of \cite{neufeld2023data}: these include the beta, uniform, and shifted exponential distributions, among others.  A summary of distributions covered by this work is provided in Table~\ref{table:maintable}.  In light of results by \cite{darmois, koopman}, and \cite{pitman_1936} (see the end of Section~\ref{sec:method}), we believe our examples are representative of the full range of cases to which this sufficiency-based approach can be applied. 
\item We show that sample splitting --- which, on its surface, bears little resemblance to convolution-closed data thinning --- is in fact based on the same principle: both are special cases of generalized data thinning with different choices of the function $T(\cdot)$.  In other words, our proposal is a direct \emph{generalization} of sample splitting. 
\end{enumerate}



We are not the first to propose generalizations of sample splitting.  Inspired by \cite{tian2018selective}'s use of randomized responses, \cite{rasines2021splitting} define what they call the $(U,V)$-decomposition, which injects independent noise $W$ to create two independent random variables $U=u(X,W)$ and $V=v(X,W)$ that together are jointly sufficient for the unknown parameters.  However, they do not describe how to perform a $(U,V)$-decomposition other than in the special case of a Gaussian random vector with known covariance.  Our generalized thinning framework achieves the goal set out in their paper, providing a concrete recipe for finding such decompositions in a broad set of examples.  Another paper with a similar goal is \cite{leiner2022data}.  They define ``data fission'', which seeks to find random variables $f(X)$ and $g(X)$ for which the distributions of $f(X)$ and $g(X) \mid f(X)$ are known and for which $X=h(f(X),g(X))$. When these two random variables are independent (which they describe as the ``P1'' property), their proposal aligns with generalized thinning.  However, like \cite{rasines2021splitting}, they do not provide a general strategy for performing P1-fission, and the only two examples they provide are the Gaussian vector with known covariance and the Poisson.

The rest of our paper is organized as follows. In Section~\ref{sec:method}, we define generalized data thinning, present our main theorem, and provide a simple recipe for thinning that is followed throughout the paper. In Section~\ref{sec:natural-exp-fam}, we consider the case of thinning natural exponential families; this section also revisits the convolution-closed data thinning proposal of \cite{neufeld2023data}, and clarifies the class of distributions that can be thinned using that approach. In Section~\ref{sec:general-exp}, we show that we can apply data thinning to  general exponential families. We consider distributions outside of the exponential family in Section~\ref{sec:outside-exp-fam}. Section~\ref{sec:counterexamples} contains examples of distributions that \emph{cannot} be thinned using the approaches in this paper; these examples provide insight into the fact that sufficiency is the key property needed for (generalized) data thinning to ``work".  We verify our results numerically in Section~\ref{sec:experiments}.  Finally, we close with a discussion in Section~\ref{sec:discussion}; derivations and additional technical details are deferred to the appendix.  


\begin{figure}
  \hspace{39mm}  Sample splitting    \hspace{10mm} Generalized data thinning 
  \vspace{-3mm}
\begin{center} 
\centering
\includegraphics[scale=0.18,trim={3cm 8cm 39cm 8cm},clip]{figures/schematics-002.png}
\includegraphics[scale=0.18,trim={3cm 8cm 40cm 8cm},clip]{figures/schematics-v3-003.png} 
\caption{\emph{Left:} Sample splitting assigns each observation into either a training set or a test set. 
\emph{Right:} Generalized data thinning, the proposal of this paper, splits each observation into two parts that are independent and that can be used to recover the original observation $T(\xo, \Xt{2})=X$. In some cases, they are drawn from the same distributional family as $X$.  
\label{fig:samplesplit_vs_datathin}}
\end{center}
\end{figure}

\setlength{\tabcolsep}{2.5pt}
\begin{table*}[t]
\small
\centering
\caption{Experimental results of proposed method.}
\begin{tabular}{lcccccccccccccccclccccc}
\hline
                              &  &               &              &  & \multicolumn{7}{c}{Total   Power Consumption {[}mW{]}}                &  & \multicolumn{2}{c}{}          &  &                                                                           &  & \multicolumn{1}{l}{}                                                            \\ \cline{6-12}
                              &  & \multicolumn{2}{c}{Accuracy} &  &
			      \multicolumn{3}{c}{Standard HW} &  &
			      \multicolumn{3}{c}{Optimized HW} &  &
			      \multicolumn{2}{c}{\#Selected} &  &
			      \multirow{2}{*}{\begin{tabular}[c]{@{}c@{}}Max
				Delay\\ Red.\end{tabular}} &  &
				\multirow{2}{*}{\begin{tabular}[c]{@{}c@{}}Voltage
				  Scaling\\ Factor\end{tabular}} &
				  \multirow{2}{*}{\begin{tabular}[c]{@{}c@{}}
				    \\ V\_SHW\end{tabular}} &
				    \multirow{2}{*}{\begin{tabular}[c]{@{}c@{}}\\
				    V\_OHW\end{tabular}}\\ \cline{3-4} \cline{6-8} \cline{10-12} \cline{14-15}
Network-Dataset               &  & Orig.         & Prop.        &  & Orig.   & Prop.     & Red.      &  & Orig.   & Prop.     & Red.      &  & Wei.            & Act.          &  &                                                                           &  &                                                                                 \\ \hline
LeNet-5-CIFAR-10              &  & 80.6\%        & 78.5\%       &  & 375.5   & 149.6   & 60.2\%  &  & 360.7    & 78.3    & 78.3\%  &  & 35            & 210           &  & 40 ps                                                                    &  & 0.71/0.8  & 10.1\%  & 5.4\%                                                                       \\
ResNet-20-CIFAR-10            &  & 91.9\%        & 89.6\%       &  & 718.9   & 361.0   & 49.8\%  &  & 663.9    & 288.3   & 56.6\%  &  & 35            & 210           &  & 40 ps                                                                    &  & 0.71/0.8    & 13.2\%  & 11.5\%                                                                     \\
ResNet-50-CIFAR-100           &  & 79.9\%        & 78.5\%       &  & 708.7   & 293.8   & 58.5\%  &  & 701.8    & 157.1   & 77.6\%  &  & 41            & 223           &  & 30 ps                                                                    &  & 0.73/0.8  & 8.1\%  & 4.2\%                                                                      \\
EfficientNet-B0-Lite-ImageNet &  & 73.8\%        & 69.7\%       &  & 21.2    & 19.3    & 9.0\%   &  & 2.4      & 1.9     & 20.8\%  &  & 50            & 236           &  & 20 ps                                                                    &  & 0.75/0.8  & 6.4\%  & 6.5\%                                                                      \\ \hline
\end{tabular}
\label{tab:results}
\end{table*}






\section{Related Work}
\label{sec:related}

\noindent\textbf{Models with Extrinsic Rotation Invariance.} The mainstream of deep learning-based point cloud matching approaches is intrinsically rotation-sensitive. Pioneers \cite{zeng20173dmatch,deng2018ppfnet} learn to describe local patches from a rotation-variant input. FCGF~\cite{choy2019fully} leverages fully-convolutional networks to accelerate the geometry description. D3Feat~\cite{bai2020d3feat} jointly detects and describes sparse keypoints for matching. Predator~\cite{huang2021predator} incorporates the global context to enhance the local descriptors and predicts the overlap regions for keypoint sampling. CoFiNet~\cite{yu2021cofinet} extracts coarse-to-fine correspondences to alleviate the repeatability issue of keypoints. GeoTrans~\cite{qin2022geometric} considers the geometric information in fusing the intra-frame context globally. However, the awareness of spatial positions is missing in the cross-frame aggregation. Lepard~\cite{li2022lepard} extends the non-rigid shape matching~\cite{trappolini2021shape,saleh2022bending,tang2022neural} to point clouds~\cite{qin2023deep} and proposes a re-positioning module to alleviate the pose variations. REGTR~\cite{yew2022regtr} directly regresses the corresponding coordinates and registers point clouds in an end-to-end fashion. Nonetheless, all of these methods suffer from instability with additional rotations. 


\noindent\textbf{Methods with Intrinsic Rotation Invariance.} A branch of handcrafted descriptors~\cite{chua1997point, tombari2010unique, guo2013rotational} aligns the input to a canonical representation according to an estimated local reference frame~(LRF), while the others~\cite{rusu2008aligning,rusu2009fast,drost2010model} mine the rotation-invariant components and encode them as the representation of the local geometry. Inspired by that, some deep learning-based methods~\cite{deng2018ppf,gojcic2019perfect,barroso2020hdd,saleh2020graphite,ao2021spinnet,yu2022riga,saleh2022bending} are designed to be intrinsically rotation-invariant to make the neural models focus on the pose-agnostic pure geometry. As a pioneer, PPF-FoldNet~\cite{deng2018ppf} consumes PPF-based patches and learns the descriptors using a FoldingNet~\cite{yang2018foldingnet}-based architecture without supervision. LRF-based works~\cite{gojcic2019perfect,saleh2020graphite,ao2021spinnet,saleh2022bending} achieve rotation invariance by aligning their input to the defined canonical representation. YOHO~\cite{wang2022you} adopts a group of rotations to learn a rotation-equivariant feature group and further obtain the invariance via group pooling. A common problem of the rotation-invariant methods is the less distinctive features. Although RIGA~\cite{yu2022riga} incorporates the global context into local descriptors to enhance the feature distinctiveness, its ineffective local geometry encoding and global position description learned by PointNet~\cite{qi2017pointnet} still constraint the representation ability of its descriptors. 


 \begin{figure*}
  \includegraphics[width=\textwidth]{figures/pipeline.pdf}
  \vspace{-0.5cm}
  \caption{\textbf{An Overview of \OURS{}.} From left to right: \textit{\textbf{(0).}} \OURS{} takes as input a pair of triplets $\mathcal{P} = (\mathbf{P}, \mathbf{N}, \mathbf{X})$ and $\mathcal{Q} = (\mathbf{Q}, \mathbf{M}, \mathbf{Y})$, each with three dimensions referring to the point cloud, the estimated normals, and the initial features. \textit{\textbf{(1).}}[$\S{}$.~\ref{sec:local_geometry}] A stack of encoder blocks hierarchically downsamples the points to coarser superpoints and encodes the local geometry, yielding superpoint triplets $\mathcal{P}^\prime$ and $\mathcal{Q}^\prime$. Each encoder block consists of an Attentional Abstraction Layer~(AAL) for downsampling and abstraction, followed by $e\times$ PPF Attention Layers~(PALs) for local geometry encoding and context aggregation. Both of them are based on our proposed PPF Attention Mechanism~(PAM), which enables the pose-agnostic encoding of pure geometry.~(See Fig.~\ref{fig:differences} and Fig.~\ref{fig:local_attention}). \textit{\textbf{(2).}}[$\S{}$.~\ref{sec:global_context}] Global information is fused to enhance the superpoint features of $\mathcal{P}^\prime$ and $\mathcal{Q}^\prime$. The geometric cues are globally aggregated as a rotation-invariant position representation, which introduces spatial awareness in the consecutive cross-frame context aggregation. After a stack of $g\times$ global transformers, the globally-enhanced triplets $\widetilde{\mathcal{P}}^\prime$ and  $\widetilde{\mathcal{Q}}^\prime$ are produced. \textit{\textbf{(3).}}[$\S{}$.~\ref{sec:local_geometry}] Superpoint triplets $\mathcal{P}^\prime$ and $\mathcal{Q}^\prime$ are decoded to point triplets $\hat{\mathcal{P}}$ and $\hat{\mathcal{Q}}$ by a stack of decoder blocks. Each block consists of a Transition Up Layer~(TUL) for upsampling and context aggregation, followed by $d\times$ PALs.  \textit{\textbf{(4).}}[$\S{}$.~\ref{sec:matching}] By adopting the coarse-to-fine matching~\cite{yu2021cofinet}, $\widetilde{\mathcal{P}}^\prime$ and $\widetilde{\mathcal{Q}}^\prime$ are matched to generate superpoint correspondences, which are consecutively refined to point correspondences between $\hat{\mathcal{P}}$ and $\hat{\mathcal{Q}}$. \textit{\textbf{(5).}} $\hat{\mathcal{C}}$ is established between $\hat{\mathcal{P}}$ and $\hat{\mathcal{Q}}$.}
  \label{fig:pipeline}
  \vspace{-0.5cm}
\end{figure*}


\vspace{-2mm}
\section{Proposed Framework} \label{method}
\vspace{-2mm}

\begin{figure}[!tbp]
\centering

\includegraphics[width=12cm]{figure/teaser.pdf}
\caption{Illustration of the whole workflow. (a) shows how hypergraph information bottleneck utilised to optimize the representation $Z$ to capture the minimal sufficient information within the input data $D=(G,I)$ to predict the MCI conversion label $Y$. (b) is the overall workflow integrating HGIB into the hypergraph neural network. HGNNP is a kind of hypergraph convolutional layer.}
\label{teaser}

\end{figure}

This section presents a detailed description of the crucial components of our proposed framework. Firstly, we elucidate the process of constructing the hypergraph from a given set of multi-modal data, and we delve into the specifics of the hypergraph convolution definition. Secondly, we explicate the fundamental principle of information bottleneck and integrate it into hypergraph neural networks.
The overview of the proposed method is illustrated in Fig.~\ref{teaser}.

% Introduction why using hypergraph for mult-modality learning + HGIB



% \Angie{to be updated}



\smallskip
\subsection{Hypergraph Modelling and Hypergraph Convolution}

To overcome the
challenges associated with multi-modality data ($I_1, I_2, ..., I_m$), we leverage a hypergraph
structure $G$ to represent the multi-modal features ($X_1, X_2, ..., X_m$) extracted from backbones. Subsequently, we employ a hypergraph neural network to predict MCI conversion.

% The main purpose of the proposed framework is to optimally balance the expressiveness and robustness of the learned representation of hypergraph structure data for accurate MCI conversion prediction.
% 
%Hypergraph is general framework to incorporate with multi-modality data which is common strategy for AD diagnosis.
% 
% To encourage the minimal and sufficient representation learning, we introduce  information bottleneck by applying this principle to hypergraph neural networks.
% 
%We will first elaborate hypergraph modelling and hypergraph convolution operation and then introduce the hypergraph information bottleneck.

\medskip
\noindent
\textbf{Hypergraph representation learning.}
We consider an undirected attributed hypergraph $G = (V, E, \textbf{H})$ with a vertex set $V$, a hyperedge set $E$, and an adjacency matrix $\textbf{H} \in \mathbb{R}^{|V| \times |E|}$ for hyperedge weight. 
% 
Each vertex in our hypergraph structure corresponds to a patient, while each hyperedge represents the relationship between a subset of vertices. Unlike in a graph structure, where an edge connects only two vertices, a hyperedge in a hypergraph connects multiple vertices, enabling the representation of higher-order relationships. This feature facilitates the grouping of subsets of vertices with common features or properties, enhancing the ability of the hypergraph to model complex relationships within the data.
% 
In our hypergraph structure, each vertex corresponds to a patient and each hyperedge represents the relationship between a subset of the vertices. Unlike in a graph structure, a hyperedge in a hypergraph connects multiple vertices instead of just two, allowing for the representation of higher-order relationships. This can be seen as the hyperedges grouping together subsets of vertices that have common features or properties.
% 
Specifically, the hyperedge weight between vertex $v$ and hyperedge $e$ can be defined as 
$h_{v,e}= 
\left\{ 
    \begin{array}{lc}
        1& \text{if} \  v \in e \\
        0& \text{otherwise}
    \end{array}
\right.
$.
Moreover, we denote the vertex attributes as $X$, which can be seen as a feature embedding. The input data can be represented as $D=(G, X)$. In the multi-modal setting, we assume $m$ modalities as input and denote them as $D=(G, (X_1, X_2, ..., X_m))$.
%the input data is
%can be overall 
%denoted as $D=(G, (X_1, X_2, ..., X_M))$ for M modalities
% and the hypergraph $G$.
%can represent high-order correlations of the data.

%How to generate the hypergraph 



% Given the input data $D$ with a hypergraph and corresponding embedding, 

% A crucial step in hypergraph learning is how to contruct the hypergraph structure. To do this, 
% %To generate such hypergraph structure, 
% we first obtain the feature embeddings $X$, from the given multi-modality data, using a pre-trained network backbone, as shown in Fig.~\ref{teaser}(b).
% We then use a neighbour strategy, in feature space, to generate the hyperedge groups following the same protocol as in \cite{gao2022hgnn}. Specifically, given a vertex as the centroid, its  k-nearest neighbours in the feature space can be connected by a hyperedge:
% \begin{equation}
%    E_k=\{N_{\text{KNN}_k}(v)|v \in V\}. 
% \end{equation}
% These hyperedge groups are further concatenated together to form a hypergraph for each modality data.
% To effectively utilize the multi-modality knowledge, we concat k different hypergraphs together to generate the final hypergraph. Specifically, the incidence matrices $H_k$ are concatenated directly, as $H=H_1\|H_2\|...\|H_k$.
% Then, we can feed the data D into Hypergraph Convolution Layer for further computation.



A crucial step in hypergraph learning is the construction of the hypergraph structure. To achieve this, we first obtain the feature embeddings $X=\{X_1, X_2, ...,\\ X_m\}$ from the multi-modality data using a pre-trained network backbone, as illustrated in Fig.~\ref{teaser}(b). We then employ a neighbor strategy in feature space to generate the hyperedge groups following the same protocol as described in \cite{gao2022hgnn}. Specifically, for each vertex, its $k$-nearest neighbors in the feature space are connected by a hyperedge, resulting in the set of hyperedges $E_k=\{N_{\text{KNN}_k}(v)|v \in V\}$.
% \begin{equation}
% E_k={N_{\text{KNN}_k}(v)|v \in V}.
% \end{equation}
% 
These hyperedge groups are concatenated together to form a hypergraph for each modality data. To effectively utilize the multi-modality knowledge, we concatenate $k$ different hypergraphs to generate the final hypergraph by $H=H_1\|H_2\|...\|H_k$. Then, we feed the resulting data $D$ into a Hypergraph Convolution Layer for further computation.

% To update the vertex information, we aggregate its neighbor vertex messages along the hyperpath:

\medskip
\noindent
\textbf{Hypergraph Convolution.}
We use spatial hypergraph convolution layers \cite{gao2022hgnn} for message aggregation. Messages can be passed either from vertex to hyperedge or from hyperedge to vertex using hyperpaths $P$, which is defined as $P(v_1,v_k) = (v_1,e_1,v_2,...,e_{k-1}, v_k)$.
% We utilise the Inter-Neighbor Relation $N$ of hypergraph $G$ as $N=\{ ( v,e  ) | w_{v,e}=1, v \in V, \ and\  e \in E \}$.
% 
% The vertex inter-neighbor set of hyperedge $e$ is  defined as $N_v(e)=\{v|vNe \}$ and the hyperedge inter-neighbor set of vertex is defined as $N_e(v)=\{e|vNe\}$.
% Therefore, to update the vertex information, we need to aggregate the messages from its hyperedge inter-neighbors $N_e(v)$. And the hyperedge inter-neighbor message is updated according to their vertex inter-neighbors $N_v(e)$. Such two-step message aggregation realises a closed message passing loop among vertices.
% Then the spatial hypergraph convolution layer reads:
% \begin{equation}
% h_e=w_e \cdot \sum_{v \in N_v(e)} \frac{x_v}{|N_v(e)|}, \ \ \ \ 
% y_v=\sigma \Bigg(\sum_{e \in N_e(v)} \frac{h_e}{|N_e(v)|} \cdot \Theta \Bigg),
% \end{equation}
% where $x_v$, $h_e$, and $y_v$ are the input, hidden, and output feature vectors. $w_e$ is a weight associated to hyperedge $e$, and $\Theta$ is a trainable parameter of current hypergraph convolution layer. $\sigma$ is a non-linear activation function,\textit{ e.g.}, ReLU($\cdot$). 
% 
We define the inter-neighbor relation $N$ of hypergraph $G$ as $N=\{ (v,e) | w_{v,e}=1, v \in V, \text{ and } e \in E \}$. The vertex inter-neighbor set of hyperedge $e$ is defined as $N_v(e)=\{v|vNe\}$, and the hyperedge inter-neighbor set of vertex $v$ is defined as $N_e(v)=\{e|vNe\}$. To update the vertex information, we aggregate the messages from its hyperedge inter-neighbors $N_e(v)$, and to update the hyperedge information, we use the vertex inter-neighbors $N_v(e)$.
% 
Thus, the spatial hypergraph convolution layer is defined as
\begin{equation}
f_e= \sum_{v \in N_v(e)} h_{v,e}  \cdot \frac{x_v}{|N_v(e)|}, \ \ \ \
{f}'_v=\sigma \Bigg(\sum_{e \in N_e(v)} \frac{f_e}{|N_e(v)|} \cdot \Theta \Bigg),
\end{equation}
where $x_v$, $f_e$, and ${f}'_v$ are the input, hidden, and output feature vectors. $\Theta$ is a trainable parameter of the current hypergraph convolution layer. $\sigma$ is a non-linear activation function, such as ReLU. The two-step message aggregation realizes a closed message passing loop among vertices, which enables the model to capture higher-order relationships between the vertices in the hypergraph.


\smallskip
\subsection{Hypergraph Information Bottleneck (HGIB)}
To balance the expressiveness and robustness of the model, we aim to optimize the vertex representation to capture the minimal sufficient information required for downstream tasks via the information bottleneck approach \cite{tishby2000information}. The Hypergraph Information Bottleneck (HGIB) approach, as shown in Fig. \ref{teaser}(a), is derived from the Graph Information Bottleneck \cite{wu2020graph}, which requires the node representation $Z_v$ to minimize the information from hypergraph-structured data $D$ while maximizing the information to prediction $Y$.
% 
% At the optimisation level, a major challenge for HGIB numerical realisation is
% is that the independent and identically distributed (IID) assumption of vertices are not feasible for many real-world scenarios.
% 
% Therefore, we rely on a local-dependence assumption for hypergraph-structural data: given the data related to the limited number of neighbours of vertex $v$, the data in the rest of the hypergraph is independent of $v$.
 % 
 However, a major challenge in realizing HGIB numerically is the assumption of independent and identically distributed (IID) vertices, which is not feasible for many real-world scenarios. Therefore, we rely on a local dependence assumption for hypergraph-structured data, whereby given data related to a limited number of neighbors of vertex $v$, the data in the rest of the hypergraph is independent of $v$.
 % 
 % 
 % The optimal representation follows the Markovian dependence. The representation of each vertex is updated by incorporating its neighbours with respect to the hypergraph representation $X$.
 We assume a Markovian dependence to obtain the optimal representation, whereby the representation of each vertex is updated by incorporating its neighbors with respect to the hypergraph representation $X$.
 % 
The information bottleneck seeks to optimise
 %is reduced to the following optimization:
\begin{equation}
\underset{\mathbb{P}(Z^l|D) \in \Omega }{min} \mathcal{L}_{\text{HGIB}}(X,Y;Z^l):= [-I(Y;Z^l)+\beta I(X;Z^l)],
\end{equation}
where $\Omega$ characterises the space of conditional distribution of $Z^l$ given data $D$, and $\beta$ is a balancing weight. $l$ represents the $l$-th hypergraph convolution layer. 

We now define the mutual information $I(X;Z^{l})$,  between the initial vertex embedding and updated vertex embedding, following~\cite{nguyen2010estimating}. This yield to the Cross-Entropy loss that reads:
\vspace{-2mm}
\begin{equation}
    I(Y;Z^l) \rightarrow -\sum_{v \in V} \text{CE}(Z_v^lW_{out};Y_v),
\end{equation}
% \vspace{-2mm}
where $W_{out}$ is the weight of projector to predict the MCI conversion labels.
%
%
% \subsubsection{HGIB Estimation}\\
% \\
% \textbf{Lemma 1} (Nguyen, Wainright & Jordan's bound) For any two random variables $X_1$, $X_2$, and any brivariate function $g:g(X_1, X_2) \in \mathbb{R}$, we have
% \begin{equation*}
% I(X_1, X_2) \geqslant \mathbb{E} [g(X_1, X_2)]-\mathbb{E}_{\mathbb{P}(X_1)\mathbb{P}(X_2)} [\text{exp}\left (g\left (X_1, X_2\right )-1\right )].
% \end{equation*}
% We use this lemma for $I(Y;Z^l)$ and set $g(Y,Z^l)=1+\text{log}\frac{\prod_{v \in V} Cat(Z^lW_{out})}{\mathbb{P}(Y)}$. Then $I(Y;Z^l)$ reduces to the Cross-Entropy loss by ignoring constants, \textit{i.e.,}
% $$
% I(Y;Z^l) \rightarrow -\sum_{v \in V} \text{CE}(Z_v^lW_{out};Y_v).
% $$
%
% \if 0
% \textbf{Proposition 1. }
% % 
% % \begin{equation*}
% % \underset{\mathbb{P}(X|D) \in \Omega }{min} \text{GIB} _\beta(D,Y;X):= [-I(Y;X)+\beta I(D;X)],
% % \end{equation*}
% % 
% $I(X;Z^{l})  \leqslant I(X;Z^{l-1})$.
%
% \textbf{Proof.} Since the conditional distribution of $Z_v^l$ depends only on 
% {$X,Z^{l-1},Z^l$} forms a Markov chain in this order $X \rightarrow Z^{l-1} \rightarrow Z^l$.
% According to data-processing inequality \cite{beaudry2011intuitive}, we have $I(X;Z^{l})  \leqslant I(X;Z^{l-1})$.
% \fi
%
% $I(X;Z^{l})$ measures the mutual information between the initial vertex embedding and updated vertex embedding. It has an upper bound and can be derived to a tractable objective to optimize as follow.
%
%\textbf{Proposition 1. } For any distribution $\mathbb{Q}(Z^l)$ for $Z^l$, we have 
%$I(X;Z^{l})  \leqslant  \text{KL}(\mathbb{P}(Z^l|X)\|\mathbb{Q}(Z^l))$.
%
%To specify the upper bound for $I(X;X^{l})$, we assume $\mathbb{Q}(Z^l)$ is a non-informative prior and the elements in $X^{l}$ are IID Bernoulli distributions: $Z^l = \bigcup_{i,j} \{ z_{i,j} \in  \{ 0,1  \} |z_{i,j}\overset{IID}{\sim } \text{Bernoulli(0.5)} \}$. We assume the elements in $\mathbb{Q}(Z^l)$ have a probability of 0.5. Thus the estimation of $I(X;X^{l})$ is written as
%$$I(X;X^{l})=\frac{1}{nm} \sum_{i=1}^{n}\sum_{j=1}^{m} \text{KL}\left (\text{Bernoulli}\left (z_{ij}^l\right)\| \text{Bernoulli}\left (0.5\right )\right ).$$
We then assume that the elements in $X^{l}$ are IID Bernoulli distributions: $Z^l = \bigcup_{i,j} \{ z_{i,j} \in  \{ 0,1  \} |z_{i,j}\overset{IID}{\sim } \text{Bernoulli(0.5)} \}$. So the mutual information can be defined as $I(X;Z^{l})=\frac{1}{nm} \sum_{i=1}^{n}\sum_{j=1}^{m} \text{KL}\left (\text{Bernoulli}\left (z_{ij}^l\right)\| \text{Bernoulli}\left (0.5\right )\right )$. Here, $\text{KL}$ denotes the Kullback-Leibler divergence between two Bernoulli distributions. 

\vspace{-2mm}
\subsection{Optimisation Scheme}
\vspace{-2mm}

Our main task is MCI prediction conversion. This problem is taken from the perspective of a three class prediction task (NC, MCI, and AD). We have as basis a cross-entropy loss for classification. In the medical domain, it is usual to encounter with the class imbalance problem. To address this issue, we incorporate a focal loss. We then define our overall optimisation scheme as
% 
%In this paper, we focus on the three class prediction task. So the cross entropy loss is basically applied for classification.
%Considering the class imbalance problem for the task setting, we further incorporate focal loss.
%Therefore, the final optimization loss function is the combination of CE loss, focal loss, and HGIB loss:
\begin{equation}
    \mathcal{L}_{total} = \frac{1}{|V|} \sum_{v \in V}\Bigl\{\text{CE}(P_v;Y_v) + \mu \left [-\alpha (1-P_v)^\gamma \text{log}(P_v) \right ]\Bigl\}  + \xi \frac{1}{L}\sum_{l=1}^{L} \mathcal{L}_{\text{HGIB}},
\end{equation}
where $\mu$ and $\xi$ are balancing parameters. $\alpha$ and $\gamma$ are two hyper-parameters for the focal loss~\cite{lin2017focal} in our experiments, we set their values to 2 and 0.5 respectively.
%nd been set as 2 and 0.5, respectively.
\section{Experiments}
\label{sec:experiments}

\begin{table*}[ht]
\center
\caption{Comparison to the deterministic state-of-the-arts on H3.6M \cite{h36m_pami} and 3DPW \cite{vonMarcard2018} datasets. $^{\dagger}$ means using temporal cues. The methods are not strictly comparable because they may have different backbones and training datasets. We provide the numbers only to show proof-of-concept results.}
\label{tab:state_of_the_art_det}
\setlength{\tabcolsep}{6pt}
\resizebox{\linewidth}{!}{
\begin{tabular}{l l l | c c c | c c c}
    \hline 
    \multirow{2}{*}{Method} & \multirow{2}{*}{Venue} & Intermediate & \multicolumn{3}{c}{H3.6M \cite{h36m_pami}} & \multicolumn{3}{|c}{3DPW \cite{vonMarcard2018}} \\

    \cline{4-6} \cline{7-9}
    & & Representation & MPVE$\downarrow$ & MPJPE$\downarrow$ & PA-MPJPE$\downarrow$ & MPVE$\downarrow$ & MPJPE$\downarrow$ & PA-MPJPE$\downarrow$ \\
    \hline 
    % $^{\dagger}$ Arnab \etal \cite{arnab2019exploiting} & CVPR'19  & 2D skeleton & - & 77.8 & 54.3 & - & - & 72.2 \\
    $^{\dagger}$ HMMR \cite{kanazawa2019learning} & CVPR'19 & - & - & - & 56.9 & 139.3 & 116.5 & 72.6 \\
    $^{\dagger}$ DSD-SATN \cite{sun2019human} & ICCV'19 & 3D skeleton & - & 59.1 & 42.4 & - & - & 69.5 \\
    $^{\dagger}$ VIBE \cite{kocabas2020vibe} & CVPR'20 & - & - & 65.9 & 41.5 & 99.1 & 82.9 & 51.9 \\
    $^{\dagger}$ TCMR \cite{choi2021beyond} & CVPR'21 & - & - & 62.3 & 41.1 & 102.9 & 86.5 & 52.7 \\
    $^{\dagger}$ MAED \cite{wan2021encoder} & ICCV'21 & 3D skeleton & - & 56.3 & 38.7 & 92.6 & 79.1 & 45.7 \\
    \hline
    SMPLify \cite{bogo2016keep} & ECCV'16 & 2D skeleton & - & - & 82.3 & - & - & - \\
    % \cline{2-8}
    HMR \cite{kanazawa2018end} & CVPR'18 & - & 96.1 & 88.0 & 56.8 & 152.7 & 130.0 & 81.3 \\
    GraphCMR \cite{kolotouros2019convolutional} & CVPR'19 & 3D vertices & - & - & 50.1 & - & - & 70.2 \\
    SPIN \cite{kolotouros2019learning} & ICCV'19 & - & - & - & 41.1 & 116.4 & 96.9 & 59.2 \\
    DenseRac \cite{xu2019denserac} & ICCV'19 & IUV image & - & 76.8 & 48.0 & - & - & - \\
    DecoMR \cite{zeng20203d} & CVPR'20 & IUV image & - & 60.6 & 39.3 & - & - & - \\
    ExPose \cite{choutas2020monocular} & ECCV'20 & - & - & - & - & - & 93.4 & 60.7 \\
    Pose2Mesh \cite{choi2020pose2mesh} & ECCV'20 & 3D skeleton & 85.3 & 64.9 & 46.3 & 106.3 & 88.9 & 58.3 \\
    I2L-MeshNet \cite{moon2020i2l} & ECCV'20 & 3D vertices & 65.1 & 55.7 & 41.1 & 110.1 & 93.2 & 57.7 \\
    PC-HMR \cite{luan2021pc} & AAAI'21 & 3D skeleton & - & - & - & 108.6 & 87.8 & 66.9  \\
    HybrIK \cite{li2021hybrik} & CVPR'21 & 3D skeleton & 65.7 & 54.4 & 34.5 & 86.5 & 74.1 & 45.0  \\
    METRO \cite{lin2021end} & CVPR'21 & 3D vertices & - & 54.0 & 36.7 & 88.2 & 77.1 & 47.9 \\
    ROMP \cite{sun2021monocular} & ICCV'21 & - & - & - & - & 108.3 & 91.3 & 54.9 \\
    Mesh Graphormer\cite{Lin_2021_ICCV} & ICCV'21 & 3D vertices & - & 51.2 & 34.5 & 87.7 & 74.7 & 45.6 \\
    PARE \cite{Kocabas_2021_ICCV} & ICCV'21 & Segmentation & - & - & - & 88.6 & 74.5 & 46.5 \\
    THUNDR \cite{zanfir2021thundr} & ICCV'21 & 3D markers & - & 55.0 & 39.8 & 88.0 & 74.8 & 51.5 \\
    PyMaf \cite{zhang2021pymaf} & ICCV'21 & IUV image & - & 57.7 & 40.5 & 110.1 & 92.8 & 58.9 \\
    OCHMR \cite{Khirodkar_2022_CVPR} & CVPR'22 & 2D heatmap & - & - & - & 107.1 & 89.7 & 58.3 \\
    3DCrowdNet \cite{Choi_2022_CVPR} & CVPR'22 & 3D skeleton & - & - & - & 98.3 & 81.7 & 51.5 \\
    CLIFF \cite{li2022cliff} & ECCV'22 & - & - & \textbf{47.1} & 32.7 & 81.2 & 69.0 & 43.0 \\
    FastMETRO \cite{cho2022FastMETRO} & ECCV'22 & 3D vertices & - & 52.2 & 33.7 & 84.1 & 73.5 & 44.6 \\
    VisDB \cite{yao2022learning} & ECCV'22 & 3D vertices & - & 51.0 & 34.5 & 85.5 & 73.5 & 44.9 \\
    \rowcolor{mygray}
    \rowcolor{mygray}
    \textbf{\vmname\ (Ours)} \cite{ma20233d} & CVPR'23 & Virtual marker & \textbf{58.0} & {47.3} & \textbf{32.0} & \textbf{77.9} & \textbf{67.5} & \textbf{41.3} \\
    \hline 
\end{tabular}}
\end{table*}


\begin{table*}[t]
\center
\caption{Comparison to the probabilistic state-of-the-arts on H3.6M \cite{h36m_pami} and 3DPW \cite{vonMarcard2018} datasets. The methods are not strictly comparable because they may have different backbones and training datasets. We provide the numbers only to show proof-of-concept results.}
\label{tab:state_of_the_art_pro}
\setlength{\tabcolsep}{8pt}
\resizebox{\linewidth}{!}{
\begin{tabular}{l l | c c c | c c c}
    \hline 
    \multirow{2}{*}{Method} & Hypothesis & \multicolumn{3}{c}{H3.6M \cite{h36m_pami}} & \multicolumn{3}{|c}{3DPW \cite{vonMarcard2018}} \\

    \cline{3-5} \cline{6-8}
    & Number $\hyponum$ & MPVE$\downarrow$ & MPJPE$\downarrow$ & PA-MPJPE$\downarrow$ & MPVE$\downarrow$ & MPJPE$\downarrow$ & PA-MPJPE$\downarrow$ \\
    \hline 
    
    \multirow{2}{*}{Biggs \etal \cite{biggs2020multibodies} NeurIPS'20} & 10 & - & 59.2 & 42.2 & - & 79.4 & 56.6\\
     & 25 & - & 58.2 & 42.2 & - & 75.8 & 55.6\\
    \hline 
    Sengupta \etal \cite{sengupta2021hierarchical} ICCV'21 & 25   & - & - & - & - & 75.1 & 47.0\\
    \hline 
    \multirow{4}{*}{ProHMR \cite{Kolotouros_2021_ICCV} ICCV'21} & 10 & - & - & - & - & 88.9 & 55.0\\
     & 25 & - & - & - & - & 85.1 & 52.1\\
     & 100 & - & - & - & - & 80.1 & 48.1\\
     & 200 & - & - & - & - & 77.9 & 46.5\\
    \hline 
    \multirow{4}{*}{HuManiFlow \cite{Sengupta_2023_CVPR} CVPR'23} & 10 & - & - & - & - & 75.6 & 47.9\\
     & 25 & - & - & - & - & 71.9 & 44.5\\
     & 100 & - & - & - & - & 65.1 & 39.9\\
     & 200 & - & - & - & - & 64.5 & 38.8\\
    \hline 
    HMDiff \cite{Foo_2023_ICCV} ICCV'23 & 25 & - & 49.3 & 32.4 & 82.4 & 72.7 & 44.5\\
    \hline 
    \rowcolor{mygray}
     & 10  & 55.8 & 45.4 & 31.3 & 77.3 & 66.6 & 42.5   \\
    \rowcolor{mygray}
     & 25 & 52.9 & 42.9 & 29.8 & 72.9 & 63.2 & 40.0   \\
    \rowcolor{mygray}
     & 100 & 49.2 &  39.9 & 27.6 & 67.4 & 58.2 & 36.4   \\
    \rowcolor{mygray}
    \multirow{-4}{*}{\textbf{\vmproname\ (Ours)}} & 200 & \textbf{47.7} & \textbf{38.6} & \textbf{26.6} & \textbf{64.7} & \textbf{56.2} & \textbf{35.0}  \\

    \hline 
\end{tabular}}
\end{table*}


\subsection{Datasets and metrics}
\label{subsec:dataset}
\noindent\textbf{H3.6M \cite{h36m_pami}.} We use (S1, S5, S6, S7, S8) for training and (S9, S11) for testing. 
For the deterministic estimation task, as in \cite{kanazawa2018end, choi2020pose2mesh, lin2021end, Lin_2021_ICCV}, we report Mean Per Joint Position Error (MPJPE) and PA-MPJPE for poses that are derived from the estimated meshes. We also report Mean Per Vertex Error (MPVE) for the whole mesh. For the probabilistic estimation task, we report the minimum errors over $\hyponum$ hypotheses to measure the estimated distribution accuracy following a standard practice in recent works \cite{Sengupta_2023_CVPR, Foo_2023_ICCV}. \\


\noindent\textbf{3DPW \cite{vonMarcard2018}} is collected in natural scenes. 
Following previous works \cite{lin2021end, Lin_2021_ICCV, Kocabas_2021_ICCV, zanfir2021thundr, Sengupta_2023_CVPR, Foo_2023_ICCV}, we use the train set of 3DPW to learn the model and evaluate on the test set. The same evaluation metrics as H3.6M are used. We further evaluate the performance of our probabilistic framework \vmproname\ on two subsets of 3DPW, \ie 3DPW-OC \cite{vonMarcard2018, Zhang_2020_CVPR} and 3DPW-PC \cite{vonMarcard2018, sun2021monocular} which are composed by object- and person-specific occlusion, respectively. To ensure a fair comparison, as suggested by \cite{Li_2023_ICCV}, we do not use the 3DPW training set in these particular evaluations. \\

\noindent\textbf{SURREAL \cite{varol2017learning}} is a large-scale synthetic dataset with GT SMPL annotations and has diverse samples in terms of body shapes, backgrounds, \etc We use its training set to train a model and evaluate the test split following \cite{choi2020pose2mesh}. The same evaluation metrics as H3.6M \cite{h36m_pami} are reported.


\subsection{Implementation Details}
\label{subsec:implementation}
We learn $64$ virtual markers on the H3.6M \cite{h36m_pami} training set. We use the same set of markers for all datasets instead of learning a separate set for each one. Following \cite{kanazawa2018end, choi2020pose2mesh, moon2020i2l, zanfir2021thundr, kolotouros2019convolutional, kocabas2020vibe, Lin_2021_ICCV, lin2021end}, we conduct mix-training by using MPI-INF-3DHP \cite{mehta2017monocular}, UP-3D \cite{lassner2017unite}, and COCO \cite{lin2014microsoft} training set for experiments on the H3.6M and 3DPW datasets. 

We adapt a 3D pose estimator \cite{sun2018integral} with HRNet-W48 \cite{sun2019deep} as the image feature backbone for estimating the 3D virtual markers. We set the number of voxels in each dimension to be $64$, \ie $D = H = W = 64$ for 3D heatmaps. Following \cite{kanazawa2018end, kolotouros2019convolutional, moon2020i2l}, we crop every single human region from the input image and resize it to $256 \times 256$. The dimensions of the 2D feature map are $64$, \ie $H_b = W_b = 64$. The 2D feature map has $C_b = 48$ feature channels. The processed $\featimg$ feature channels are set to be $C=64$. The denoiser network has $B=3$ blocks. 
To train \vmname, we use Adam \cite{kingma2015adam} optimizer to train the whole framework for $40$ epochs with a batch size of $32$. The learning rates for the 3D VM estimation branch and the updating branch are set to $5 \times 10^{-4}$ and $1 \times 10^{-3}$, respectively, which are decreased by half after the $30^{th}$ epoch. To train \vmproname, we use Adam \cite{kingma2015adam} optimizer to train the whole framework for $50$ epochs with a batch size of $80$. The learning rates for the 3D VM estimation branch, the updating branch, and the denoiser are set to $4 \times 10^{-5}$, $2 \times 10^{-4}$, and $2 \times 10^{-3}$, respectively. They are decayed by $0.5$, $0.5$, and $0.1$ after $30^{th}$ and $40^{th}$ epochs, respectively. For inference, we employ DDIM \cite{song2021denoising} strategy and generate each hypothesis in $T=10$ steps with $\eta = 0$.
Please check the supplementary for more details.


\begin{table}[t]
\center
 \caption{Comparison to the state-of-the-arts on SURREAL \cite{varol2017learning} dataset. $^{*}$ means training on the test split with 2D supervisions. ``Skel. + Seg.'' means using skeleton and segmentation together. $\hyponum$ is the number of hypotheses.}
\label{tab:state_of_the_art_surreal}
\setlength{\tabcolsep}{4pt}
\resizebox{\linewidth}{!}{
\begin{tabular}{l l l | c c c}
    \hline
    \multirow{2}{*}{Method} & Intermediate & \multirow{2}{*}{$\hyponum$} & \multirow{2}{*}{MPVE$\downarrow$} & \multirow{2}{*}{MPJPE$\downarrow$} & \multirow{2}{*}{PA-MPJPE$\downarrow$} \\
    & Representation &  & &  &  \\
    \hline
    HMR \cite{kanazawa2018end} & -  & - & 85.1 & 73.6 & 55.4 \\
    BodyNet \cite{varol2018bodynet} & Skel. + Seg. & - & 65.8 & - & - \\
    GraphCMR \cite{kolotouros2019convolutional} & 3D vertices & - & 103.2 & 87.4 & 63.2  \\
    SPIN \cite{kolotouros2019learning} & - & - & 82.3 & 66.7 & 43.7 \\
    DecoMR \cite{zeng20203d} & IUV image & - & 68.9 & 52.0 & 43.0 \\
    Pose2Mesh \cite{choi2020pose2mesh} & 3D skeleton & - & 68.8 & 56.6 & 39.6 \\
    PC-HMR \cite{luan2021pc} & 3D skeleton & - & 59.8 & 51.7 & 37.9  \\
    $^{*}$ DynaBOA \cite{guan2022out} & - & - & 70.7 & 55.2 & 34.0 \\
    \rowcolor{mygray}
    \textbf{\vmname\ (Ours)}          & Virtual marker & - & 44.7 & 36.9 & 28.9 \\
    \hline
    \rowcolor{mygray}
     &  & 10 & 44.3 & 36.6 & 28.0 \\
    \rowcolor{mygray}
    \textbf{\vmproname} &  & 25 & 42.0 & 34.8 & 26.2 \\
    \rowcolor{mygray}
    \textbf{(Ours)} &  & 100 & 38.1 & 31.0 & 23.5 \\
    \rowcolor{mygray}
    & \multirow{-4}{*}{Virtual marker}  & 200 & \textbf{36.5} & \textbf{29.8} & \textbf{22.4} \\
    \hline 
\end{tabular}}

\end{table}


\begin{table*}[t]
\center
\caption{Comparison to the state-of-the-art methods on the challenging occlusion-specific 3DPW-OC \cite{vonMarcard2018, Zhang_2020_CVPR} and 3DPW-PC \cite{vonMarcard2018, sun2021monocular} datasets. The top and bottom blocks are deterministic and probabilistic methods, respectively.}
\label{tab:sota_occlusion}
\setlength{\tabcolsep}{8pt}
\resizebox{\linewidth}{!}{
\begin{tabular}{l l l |c c c| c c c}
    \hline
    \multirow{2}{*}{Method} & \multirow{2}{*}{Venue} & Hypothesis & \multicolumn{3}{c|}{3DPW-OC \cite{vonMarcard2018, Zhang_2020_CVPR}} & \multicolumn{3}{c}{3DPW-PC \cite{vonMarcard2018, sun2021monocular}} \\ 
    \cline{4-9}
    & & Number $\hyponum$  & MPVE$\downarrow$ & MPJPE$\downarrow$ & PA-MPJPE$\downarrow$  
    & MPVE$\downarrow$ & MPJPE$\downarrow$ & PA-MPJPE$\downarrow$ \\ 
    \hline
    SPIN \cite{kolotouros2019learning} & ICCV'19 & -     & 121.4  & 95.5  & 60.7 & 159.8  & 122.1  & 77.4  \\
    I2L-MeshNet \cite{moon2020i2l} &  ECCV'20 & -         & 129.5  & 92.0  & 61.4 & 160.2  & 117.3  & 80.0  \\
    PyMAF \cite{zhang2021pymaf} & ICCV'21 & -            & 113.7  & 89.6  & 59.1 & 154.6  & 117.5  & 74.5  \\
    ROMP \cite{sun2021monocular} & ICCV'21 & -           & -      & 91.0  & 62.0 & 152.8  & 117.9  & 79.7  \\
    PARE \cite{Kocabas_2021_ICCV} & ICCV'21 & -          & 101.5  & 83.5  & 57.0 & 122.4  & 96.8   & 64.5  \\
    OCHMR \cite{Khirodkar_2022_CVPR} & CVPR'22 & -       & 145.9  & 112.2 & 75.2 & 149.6  & 117.5  & 77.1     \\
    3DCrowdNet \cite{Choi_2022_CVPR} & CVPR'22 & -       & 101.5  & 83.5  & 57.1 & 114.8  & 90.9   & 64.4  \\
    JOTR \cite{Li_2023_ICCV} & ICCV'23 & -               & 92.6   & 75.7  & 52.2 & 109.7  & 86.5   & 58.3  \\ 
    \rowcolor{mygray}
    \textbf{\vmname\ (Ours)} \cite{ma20233d} & CVPR'23 & -         & 92.8   & 78.4  & 49.0 & 110.8  & 93.9   & 66.0  \\ 
    \hline
    HMDiff \cite{Foo_2023_ICCV} & ICCV'23 & 25           & -      & -     & -    & 143.1  & 114.2  & 73.5  \\
    \rowcolor{mygray}
     & & 10 & 90.4   & 77.3  & 48.6 & 109.4  & 89.8   & 60.2  \\
    \rowcolor{mygray}
     & & 25 & 85.4   & 73.3  & 46.0 &  102.9  &  84.7   &  56.6  \\ 
    \rowcolor{mygray}
     & & 100 & 77.7   & 67.1  &  42.5 & 94.0   & 78.0   & 52.0  \\
    \rowcolor{mygray}
    \multirow{-4}{*}{\textbf{\vmproname\ (Ours)}} & \multirow{-4}{*}{-} & 200 & \textbf{75.2}  & \textbf{64.9} & \textbf{40.9} & \textbf{90.3}  & \textbf{74.6}   & \textbf{49.7}  \\
    \hline
\end{tabular}}
\end{table*}


\subsection{Comparison to the State-of-the-arts}
\subsubsection{Deterministic Track}
\label{subsec:sota_det}

\noindent\textbf{Results on H3.6M \cite{h36m_pami}.}
Table \ref{tab:state_of_the_art_det} compares our approach to the deterministic state-of-the-art (SOTA) methods on the H3.6M dataset. Our method achieves superior performance. In particular, it outperforms the methods that use skeletons (Pose2Mesh \cite{choi2020pose2mesh}, DSD-SATN \cite{sun2019human}), body markers (THUNDR) \cite{zanfir2021thundr}, or IUV image \cite{zeng20203d, zhang2021pymaf} as proxy representations, demonstrating the effectiveness of the virtual marker representation. \\

\noindent\textbf{Results on 3DPW \cite{vonMarcard2018}.}
We compare our method to the deterministic SOTA methods on the 3DPW dataset in Table \ref{tab:state_of_the_art_det}. Our approach achieves SOTA results among all the methods, validating the advantages of the virtual marker representation over the skeleton representation used in Pose2Mesh \cite{choi2020pose2mesh}, DSD-SATN \cite{sun2019human}, and other representations like IUV image used in PyMAF \cite{zhang2021pymaf}. In particular, our approach outperforms I2L-MeshNet \cite{moon2020i2l}, METRO \cite{lin2021end}, and Mesh Graphormer \cite{Lin_2021_ICCV} by a notable margin, suggesting that virtual markers are more suitable and effective representations than using all vertices as most of them are not discriminative enough to be accurately detected. \\



\noindent\textbf{Results on SURREAL \cite{varol2017learning}.}
This dataset has more diverse samples in terms of body shapes. The results are shown in Table \ref{tab:state_of_the_art_surreal}. Our approach \vmname, outperforms the SOTA methods by a notable margin, especially in terms of MPVE. Figure \ref{fig:teaser} (top) shows some challenging cases without cherry-picking. The skeleton representation loses the body shape information so the method \cite{choi2020pose2mesh} can only recover mean shapes. In contrast, our approach generates much more accurate mesh estimation results. 




\subsubsection{Probabilistic Track}
\label{subsec:sota_pro}

\noindent\textbf{Results on H3.6M \cite{h36m_pami} and 3DPW \cite{vonMarcard2018}.}
Table \ref{tab:state_of_the_art_pro} presents a comparison between \vmproname\ and other SOTA probabilistic methods on the H3.6M \cite{h36m_pami} and 3DPW \cite{vonMarcard2018} datasets. Our method significantly surpasses the existing SOTA approaches in terms of accuracy when estimating data distributions. This serves as a validation of the effectiveness of our probabilistic framework, which benefits from the generation capability of diffusion models. Furthermore, as the number of hypotheses ($\hyponum$) increases, our method consistently improves, highlighting the diverse and accurate nature of our approach when modeling data distributions. Notably, compared to HMDiff \cite{Foo_2023_ICCV}, which is also a diffusion model-based method, our incorporation of 2D image cues and the virtual marker representations contributes to the observed notable improvement. \\

\noindent\textbf{Results on 3DPW-OC \cite{vonMarcard2018, Zhang_2020_CVPR} and 3DPW-PC \cite{vonMarcard2018, sun2021monocular}.}
Table \ref{tab:sota_occlusion} shows the comparison results of our method against existing methods on the 3DPW-OC and 3DPW-PC datasets. Both 3DPW-OC \cite{vonMarcard2018, Zhang_2020_CVPR} and 3DPW-PC \cite{vonMarcard2018, sun2021monocular} are subsets of the 3DPW \cite{vonMarcard2018} dataset, containing scenarios with occlusions by objects and people, respectively. It can be seen that our deterministic estimation results are comparable to the SOTA performance. When adopting the proposed probabilistic modeling approach, the accuracy of our method in estimating the data distribution is significantly improved, thanks to the enhanced accuracy in 3D virtual marker estimation. Figure \ref{fig:3dpw_oc} presents a comparative visualization between \vmname\ and \vmproname. It can be seen that the deterministic approach, \vmname, may encounter difficulties under occlusions, \eg the wrongly estimated right arm and the incorrect arm length. Conversely, the probabilistic method \vmproname, demonstrates a capability to accurately model realistic human poses, offering feasible estimations for the occluded parts. \\

\begin{figure}[ht]
    \centering
    \vspace{-2em}
    \includegraphics[width=0.95\linewidth]{imgs/experiments/3dpw_oc.pdf}
    \caption{Qualitative comparison of \vmname\ and \vmproname\ on 3DPW-OC \cite{vonMarcard2018,sun2021monocular} testset. For each mesh, we show a projecting view and a side view (with shadow). \vmname\ wrongly estimates the right arm when heavy occlusion happens while \vmproname\ provides two reasonable estimates. }
    \label{fig:3dpw_oc}
\end{figure}




\noindent\textbf{Results on SURREAL \cite{varol2017learning}.}
Table \ref{tab:state_of_the_art_surreal} displays the results of our probabilistic approach on the SURREAL \cite{varol2017learning} dataset. It can be seen that when $S=10$, our probabilistic framework already surpasses existing approaches. This indicates that our method can accurately model the more diverse data distribution. As the number of hypotheses ($S$) increases, the performance of our model continues to improve.


\subsection{Ablation Study on \vmname}
\label{subsec:ablation}
\noindent\textbf{Virtual marker representation.}
\label{subsubsec:ablation_effect}
We compare our method to two baselines in Table \ref{tab:ba_effect}. First, in baseline (a), we replace the virtual markers of our method with the skeleton representation. The rest are kept the same as ours (c). Our method achieves a much lower MPVE than the baseline (a), demonstrating that the virtual markers help to estimate body shapes more accurately than the skeletons. In baseline (b), we randomly sample $64$ from the $6890$ mesh vertices as virtual markers. We repeat the experiment five times and report the average number. We can see that the result is worse than ours, which is because the randomly selected vertices may not be expressive enough to reconstruct the other vertices or can not be accurately detected from images as they lack distinguishable visual patterns. The results validate the effectiveness of our learning strategy. 


\begin{table}[t]
\center
\caption{Ablation study of the virtual marker representation for our approach \vmname\ on H3.6M \cite{h36m_pami} and SURREAL \cite{varol2017learning} datasets. 
``Skeleton'' means the sparse landmark joint representation is used. 
``Rand virtual marker'' means the virtual markers are randomly selected from all the vertices without learning. (c) is our method, where the learned virtual markers are used. MPVE error is reported.}
\label{tab:ba_effect}
\setlength{\tabcolsep}{10pt}
\resizebox{\linewidth}{!}{
\begin{tabular}{c | l | c | c }
    \hline 
    \multirow{2}{*}{No.} & Intermediate & \multicolumn{2}{c}{MPVE$\downarrow$} \\

    \cline{3-4}
    & Representation & H3.6M & SURREAL \\
    \hline
    (a) & Skeleton & 64.4 & 53.6 \\ 
    (b) & Rand virtual marker & 63.0 & 50.1 \\
    \rowcolor{mygray}
    (c) & Virtual marker & \textbf{58.0} & \textbf{44.7}\\
    \hline 
\end{tabular}}
\end{table}

\begin{figure}[t]
	\centering
	\includegraphics[width=\linewidth]{imgs/experiments/joint_diff.pdf}
	\caption{Mesh estimation results of different methods on H3.6M \cite{h36m_pami} test set. Our method \vmname\ with virtual marker representation gets better shape estimation results than Pose2Mesh \cite{choi2020pose2mesh} which uses skeleton representation. Note the waistline of the body and the thickness of the arm. }
	\label{fig:joint_diff}
\end{figure}



Figure \ref{fig:teaser} (top) shows some qualitative results on the SURREAL test set. The meshes estimated by the baseline which uses skeleton representation, \ie Pose2Mesh \cite{choi2020pose2mesh}, have inaccurate body shapes. This is reasonable because the skeleton is oversimplified and has very limited capability to recover shapes. Instead, it implicitly learns a mean shape for the whole training dataset. In contrast, the mesh estimated by using virtual markers has much better quality due to its strong representation power and therefore can handle different body shapes elegantly. Figure \ref{fig:joint_diff} also shows some qualitative results on the H3.6M test set. For clarity, we also draw the intermediate representation (blue balls) in it. \\


\noindent\textbf{Number of virtual markers.} 
\label{subsubsec:ablation_K}
We evaluate how the number of virtual markers affects estimation quality on H3.6M \cite{h36m_pami} dataset. Figure \ref{fig:body_arche} visualizes the learned 64 markers. Figure \ref{fig:different_K_vis} visualizes the learned virtual markers when $K=16,32,96 $, which are all located on the body surface and close to the extreme points of the mesh. This is expected as mentioned in Section \ref{subsec:body_arche}.  Table \ref{tab:different_K} (GT) shows the mesh reconstruction results when we have GT 3D positions of the virtual markers in objective (Eq. \ref{eq:aa}). When we increase the number of virtual markers, both mesh reconstruction error, \ie MPVE, and the regressed landmark joint error, \ie MPJPE, steadily decrease. This is expected because using more virtual markers improves the representation power. However, using more virtual markers cannot guarantee smaller estimation errors when we need to estimate the virtual marker positions from images as in our method. This is because the additional virtual markers may have large estimation errors which affect the mesh estimation result. The results are shown in Table \ref{tab:different_K} (Est). Increasing the number of virtual markers $K$ steadily reduces the MPVE errors when $K$ is smaller than $96$. However, if we keep increasing $K$, the error begins to increase. This is mainly because some of the newly introduced virtual markers are difficult to detect from images and therefore bring errors to mesh estimation. \\



\begin{figure}[t]
    \noindent\begin{minipage}{\linewidth}
        \captionof{table}{Ablation study of the different number of virtual markers ($K$) on H3.6M \cite{h36m_pami} dataset. (GT) Mesh reconstruction results when GT 3D positions of the virtual markers are used in objective (\ref{eq:aa}). (Est) Mesh estimation results obtained by our method \vmname\ when we use different numbers of virtual markers ($K$).}
        \label{tab:different_K}
        \setlength{\tabcolsep}{13pt}
        \resizebox{\linewidth}{!}{
        \begin{tabular}{c | c  c | c  c }
            \hline 
            \multirow{2}{*}{$K$} & \multicolumn{2}{c|}{GT}  & \multicolumn{2}{c}{Est} \\
        
            \cline{2-3} \cline{4-5}
            & MPVE$\downarrow$ & MPJPE$\downarrow$ & MPVE$\downarrow$ & MPJPE$\downarrow$\\
            \hline
            16 & 46.8 & 39.8 & 58.7 & 47.8 \\ 
            32 & 20.1 & 14.2 & 58.2 & 48.3 \\
            64 & 11.0 & 7.5 & \textbf{58.0} & \textbf{47.3} \\
            96 & \textbf{9.9} & \textbf{5.6} & 59.6 & 48.2\\
            \hline 
        \end{tabular}}
        \vspace{0.8em}
    
        \includegraphics[width=\linewidth]{imgs/experiments/different_K_vis.pdf}
        \captionof{figure}{Visualization of the learned virtual markers when $K = 16, 32, 96$, from left to right, respectively.}
        \label{fig:different_K_vis}
    \end{minipage}
\end{figure}



\noindent\textbf{Updating coefficient matrix.}
\label{subsubsec:blending_effect}
We compare our method to a baseline which uses the fixed coefficient matrix $\widetilde{\mathbf{A}}^{sym}$. 
We show the quality comparison in Figure \ref{fig:blending_nr_diff}. We can see that the estimated mesh by (a) a fixed coefficient matrix has mostly correct pose and shape but there are some artifacts on the mesh while using the (b) updated coefficient matrix can get better mesh estimation results. 
As shown in Table \ref{tab:quan_bm_nr_effect}, using a fixed coefficient matrix gets larger MPVE and MPJPE errors than using the updated one. This is caused by the estimation errors of virtual markers when occlusion happens, which is inevitable since the virtual markers on the back will be self-occluded by the front body. As a result, inaccurate marker positions would bring large errors to the final mesh estimates if we directly use the fixed matrix.


\begin{figure}[t]
    \noindent\begin{minipage}{\linewidth}
        \captionof{table}{Ablation study of the coefficient matrix for \vmname\ on H3.6M \cite{h36m_pami} dataset. ``fixed'' means using the fixed coefficient matrix $\widetilde{\mathbf{A}}^{sym}$ to reconstruct the mesh. }
        \label{tab:quan_bm_nr_effect}
        \setlength{\tabcolsep}{5pt}
        \renewcommand\arraystretch{1.35}
        \resizebox{\linewidth}{!}{
        \begin{tabular}{l | c c |c c}
            \hline 
            Method & Fixed $\widetilde{\mathbf{A}}^{sym}$ & Updated $\Aest$ & MPVE$\downarrow$ & MPJPE$\downarrow$ \\
            \hline
            (a) \vmname\ (fixed) & \cmark &  & 64.7 & 51.6 \\
            \rowcolor{mygray}
            (b) \vmname  & & \cmark  & \textbf{58.0} & \textbf{47.3} \\
            \hline 
        \end{tabular}}
        \vspace{0.8em}
    
        \includegraphics[width=\linewidth]{imgs/experiments/blending_diff.pdf}
        \captionof{figure}{Mesh estimation comparison results when using (a) fixed coefficient matrix $\widetilde{\mathbf{A}}^{sym}$, and (b) updated $\Aest$ in our method \vmname. Please zoom in to better see the details.}
        \label{fig:blending_nr_diff}
    \end{minipage}
\end{figure}


\begin{table}[t]
\center
\caption{Ablation study of the architecture design for the probabilistic method \vmproname\ on 3DPW \cite{vonMarcard2018} dataset. (a-b) ablates the use of 2D feature $\feat$ and 2D VMs $\vmestuv$ in denoiser network, respectively. Ablation (c) removes the 2D VM $\vmestuv$ sampling in 2D feature $\feat$ and directly transforms the global 2D feature $\feat$ to $\featimg$ for the denoiser. In ablation (d), the denoiser directly estimates the noise $\noiseest$ following the original formula \cite{ho2020denoising}, while keeping the rest design the same. (e) is the proposed \vmproname\ framework.}
\label{tab:vmpro_ablate}
\setlength{\tabcolsep}{3pt}
\renewcommand\arraystretch{1.35}
\resizebox{\linewidth}{!}{
\begin{tabular}{l | c c c | c c}
    \hline 
    \multirow{2}{*}{Method} & 2D feature  & 2D VM & Regress & \multirow{2}{*}{MPVE$\downarrow$} & \multirow{2}{*}{MPJPE$\downarrow$} \\
      &  $\feat$ &  $\vmestuv$ &  $\hmapestuv$ &  &  \\
    \hline
    (a) \textit{w/o} $\feat$ & & \cmark & \cmark & 80.1 & 68.2 \\
    (b) \textit{w/o} $\vmestuv$ & \cmark & & \cmark & 80.6 & 68.8 \\
    (c) \textit{w/o} sampling in $\feat$ & \cmark & \cmark & \cmark & 80.4 & 69.2 \\
    (d) $\rightarrow \noiseest$ & \cmark & \cmark & \cmark & 80.6 & 68.9 \\
    \rowcolor{mygray}
    (e) \vmproname  & \cmark & \cmark & \cmark & \textbf{76.9} & \textbf{66.1} \\
    \hline 
\end{tabular}}
\end{table}

\begin{figure*}[t]
	\centering
	\includegraphics[width=\linewidth]{imgs/experiments/design.pdf}
	\caption{Mesh estimation comparison results of different ablated methods of \vmproname. (a) We ablate the usage of 2D image features $\feat$. (b) We ablate the usage of 2D estimated VM $\vmestuv$. (c) We remove the 2D VM $\vmestuv$ sampling in 2D feature $\feat$ and directly transform the global 2D feature $\feat$ to $\featimg$ for the denoiser. (d) We change the estimation target of the denoiser to the noise $\noiseest$ instead of the signal itself. (e) is our proposed \vmproname\ method. We show the GT mesh and its side view on the far right. For baselines (a-e), we show each hypothesis that is denoised from a zero noise for a fair comparison. Each hypothesis presents a projecting view and a side view.}
	\label{fig:vmpro_ablate}
\end{figure*}

\subsection{Ablation Study on \vmproname}
\noindent\textbf{Architecture design of the denoiser.}
Table \ref{tab:vmpro_ablate} compares \vmproname\ to four baselines. We report the metric when predicting $S=10$ hypotheses. In baseline (a) and (b), we remove the 2D feature $\feat$ and 2D estimated VMs $\vmestuv$ input for the denoiser, respectively. It can be seen that the errors increase a lot compared to the full model, which validates the effectiveness of the two guidance. In baseline (c), we remove the local sampling by 2D VMs $\vmestuv$ on 2D feature $\feat$, \ie the purple arrow in Figure \ref{fig:diff_model}. Instead, we directly transform the global 2D image feature $\feat$ to $\featimg$ when feeding into the denoiser and the performance degrades notably. In ablation study (d), we alter the denoiser to estimate the noise $\noiseest$ rather than the clean 3D VMs $\vmest(0)$, adjusting the inference process to align with the original denoising formulation by Ho \etal \cite{ho2020denoising}. This modification results in a decrease in performance. A similar observation was made in previous research by Tevet \etal \cite{tevet2023human}, where it is suggested that noise estimation might require substantially larger batch sizes to be effective. Unfortunately, we lack the resources to confirm this hypothesis in our settings. 

Figure \ref{fig:vmpro_ablate} presents the qualitative comparison of different designs. As observed in the image, the left arm of the subject is nearly fully extended. However, due to depth ambiguity, the estimations from baselines (a-d) align with the 2D perspective, yet inaccuracies arise in the 3D estimation of the left forearm when viewed from the side view. Our method (e), which integrates 2D local features, more accurately captures precise 3D poses, achieving estimations that most closely match the GT. This highlights the significance of our design, which not only utilizes 2D local features $\feat$ but also leverages 2D VMs $\vmestuv$ to locally sample these features effectively.\\




\begin{figure}[t]
    \noindent\begin{minipage}{\linewidth}
        \captionof{table}{Results of using different denoising steps $T$ for \vmproname\ on H3.6M \cite{h36m_pami} and 3DPW \cite{vonMarcard2018} datasets.}
        \label{tab:denoising_step}
        \resizebox{\linewidth}{!}{
        \centering
        \setlength{\tabcolsep}{8pt}
        \begin{tabular}{c | c  c | c  c  | c}
            \hline 
            \multirow{2}{*}{Steps $T$} & \multicolumn{2}{c|}{H3.6M}  & \multicolumn{2}{c|}{3DPW} & \multirow{2}{*}{FPS$\uparrow$} \\
        
            \cline{2-3} \cline{4-5}
            & MPVE$\downarrow$ & MPJPE$\downarrow$ & MPVE$\downarrow$ & MPJPE$\downarrow$ &\\
            \hline
            1 & 386.5 & 370.4  & 372.2 & 358.9 & 339.6 \\
            2 & 77.0 & 66.2 & 104.7 & 92.6 & 326.6 \\
            4 & 58.8 & 48.3 & 80.0 & 69.2 & 301.0 \\
            5 & 57.1 & 46.9 & 78.3 & 67.6 & 292.6 \\
            10 & 55.8 & 45.4 & 77.3 & 66.6  & 247.4 \\
            15 & 55.5 & 45.0 & 76.4 & 65.5  & 208.0 \\
            20 & 55.8 & 45.3 & 76.8 & 65.9 & 189.0 \\
            100 & 55.6 & 45.2 & 76.9 & 65.9 & 66.4 \\
            \hline 
        \end{tabular}}
        \vspace{0.8em}
    
        \includegraphics[width=\linewidth]{imgs/experiments/stepsT.pdf}
        \captionof{figure}{Mesh estimation results of using different denoising steps $T$ for \vmproname\ on a sample data from H3.6M \cite{h36m_pami} dataset. Each hypothesis is denoised from a zero initial noise for a fair comparison.}
        \label{fig:denoising_step}
    \end{minipage}
\end{figure}

\noindent\textbf{Denoising steps.}
Table \ref{tab:denoising_step} shows the metrics on the H3.6M \cite{h36m_pami} and 3DPW \cite{vonMarcard2018} test sets when using different numbers of denoising steps $T$. We report the metrics of predicting $S=10$ hypotheses. We also report the speed of the whole model inferring a hypothesis on an NVIDIA A100-PCIE-80GB GPU. It can be seen that using only $T=1$ denoising step results in significant errors, and the estimated mesh is not human-shaped, as shown in Figure \ref{fig:denoising_step}. As the number of denoising steps $T$ increases, the accuracy of the estimated mesh gradually improves, and the error decreases. When $T>=10$, the accuracy of the mesh estimation stabilizes, and further steps yield only marginal improvements, as validated by the marginal reduction in error and the indistinguishable differences in visualization. Therefore, based on the acceleration of DDIM \cite{song2021denoising}, we choose $T=10$ steps for a good balance between estimation performance and inference speed. \\


\begin{table}[t]
\center
\caption{Robustness to occlusion for \vmname\ and \vmproname\ on 3DPW \cite{vonMarcard2018} test set when different body parts are occluded. We report MPVE error.
}
\label{tab:ablate_occ}
\setlength{\tabcolsep}{15pt}
\resizebox{\linewidth}{!}{
\begin{tabular}{c | c | c  c }
    \hline 
    \multirow{2}{*}{Occ. Parts} & \multirow{2}{*}{\vmname}  & \multicolumn{2}{c}{\vmproname} \\

    \cline{3-4}
    & & $\hyponum=10$ & $\hyponum=100$ \\
    \hline
    \rowcolor{mygray}
    None & 77.9 & 77.3 & 67.4 \\
    2 Arms & 79.2 & 78.7 & 68.8 \\
    2 Legs & 78.3 & 77.6  & 67.8 \\
    Body & 78.6 & 78.1 & 68.2 \\
    Random & 78.7 & 77.9 & 68.4 \\
    \hline 
\end{tabular}}
\end{table}
\vspace{-2em}





\begin{figure*}[ht]
	\centering
	\includegraphics[width=\linewidth]{imgs/experiments/quality_result.pdf}
	\caption{ \textbf{Top:} Meshes estimated by our deterministic approach \vmname\ on images from 3DPW test set. The rightmost case in the dashed box shows a typical failure. \textbf{Bottom:} Meshes estimated by \vmname\ on Internet images with challenging cases (extreme shapes or in a long dress).}
	\label{fig:quality_result_w_failure}
\end{figure*}




\begin{figure}[t]
	\centering
	\includegraphics[width=\linewidth]{imgs/experiments/locality.pdf}
	\caption{(a) For each virtual marker (represented by a star), we highlight the top 30 most affected vertices (represented by a colored dot) based on average coefficient matrix $\Aest$. (b) For each vertex (dot), we highlight the top 3 virtual markers (star) that contribute the most. The dependency has a strong locality which improves the robustness when some virtual markers cannot be accurately detected. }
	\label{fig:locality}
\end{figure}



\begin{figure*}[t]
	\centering
	\includegraphics[width=0.95\linewidth]{imgs/experiments/3dpw_det_pro.pdf}
	\caption{Qualitative comparison of \vmname\ and \vmproname\ on 3DPW \cite{vonMarcard2018} testset. \vmname\ may fail when facing ambiguity, while \vmproname\ could generate multiple reasonable solutions. For each mesh estimate, we present both a projected view and a side view in each column. We show three hypotheses generated by \vmproname.}
	\label{fig:3dpw_det_pro}
\end{figure*}


\begin{figure*}[ht!]
	\centering
	\includegraphics[width=0.95\linewidth]{imgs/experiments/quality_vmpro.pdf}
	\caption{Meshes estimated by \vmproname\ on Internet images. For each case in a row, we present three hypotheses, showing both a projected view and a side view in each column. The rightmost column shows the overlapped side views to unveil the subtle differences.}
	\label{fig:quality_result_w_failure_pro}
\end{figure*}

\begin{figure}[ht!]
	\centering
	\includegraphics[width=\linewidth]{imgs/experiments/vmpro_failure_latin.pdf}
	\caption{A typical failure case of \vmproname. The heavy occlusion and ambiguity cause the model to incorrectly estimate the right arm and right leg of the male dancer.}
	\label{fig:vmpro_failure}
\end{figure}
\vspace{-0.8em}

\noindent\textbf{Robustness to occlusion.}
\label{subsec:occlusion}
We report results of \vmname\ and \vmproname\ when different image regions of corresponding virtual markers are occluded by a synthetic random mask in Table \ref{tab:ablate_occ}. The errors of \vmname\ are slightly larger than the original image (None). We further analyze how inaccurate virtual markers would affect the mesh estimation, \ie when part of human body is occluded or truncated. According to the finally learned coefficient matrix $\Aest$ of our model \vmname, we highlight the relationship weights among virtual markers and all vertices in Figure \ref{fig:locality}. We can see that our model actually learns \emph{local and sparse} dependency between each vertex and the virtual markers, \eg for each vertex, the virtual markers that contribute the most are in a near range as shown in Figure \ref{fig:locality} (b). Therefore, in inference, if a virtual marker has inaccurate position estimation due to occlusion or truncation, the dependent vertices may have inaccurate estimates, while the rest will be barely affected. 

This locality enhances the robustness of the \vmproname\ method against occlusion as well. When sampling $S=10$ hypotheses, \vmproname\ exhibits a minor increase in error under various circumstances of VM occlusion. 



\subsection{Qualitative Results}
\label{subsec:quality}
\noindent\textbf{\vmname.}
Figure \ref{fig:quality_result_w_failure} (top) presents some meshes estimated by our approach \vmname\ on natural images from the 3DPW test set. The rightmost case shows a typical failure where our method has a wrong pose estimate of the left arm due to heavy occlusion. We can see that the failure is constrained to the local region and the rest of the body still gets accurate estimates. As discussed in Sec. \ref{subsec:occlusion}, we suppose that this is due to the advantages brought by the locality of the virtual marker representation, which can confine erroneous mesh estimates to a local range and thereby enhance robustness.
Figure \ref{fig:body_arche} (right) shows more examples where occlusion or truncation occurs, and our method can still get accurate or reasonable estimates robustly. Note that when truncation occurs, our method still guesses the positions of the truncated virtual markers. 

Figure \ref{fig:quality_result_w_failure} (bottom) shows our estimated meshes on challenging cases, which indicates the strong generalization ability of our model on diverse postures and actions in natural scenes. Please refer to the supplementary for more quality results. Note that since the datasets do not provide supervision of head orientation, face expression, hands, or feet, the estimates of these parts are just in canonical poses inevitably. \\




\noindent\textbf{\vmproname.} 
Figure \ref{fig:3dpw_det_pro} compares the visualization of \vmname\ and \vmproname\ on the 3DPW \cite{vonMarcard2018} test set. We draw three hypotheses generated by \vmproname, depicted in blue. Additionally, we include the results from the deterministic model \vmname\ (green). It is noticeable that, due to self-occlusion or object occlusion, \vmname\ might yield results that seem accurate in 2D projections yet are erroneous in 3D space, \eg the incorrect estimation of the right leg in the second case. This issue stems from the inherent ambiguity associated with occlusions. By leveraging the advantages of the data distribution learned by the denoising model, \vmproname\ can generate more reasonable solutions, such as the naturally standing legs. 

In Figure \ref{fig:quality_result_w_failure_pro}, we show more results of \vmproname\ on unseen Internet images. It can be observed that \vmproname\ possesses strong generalization capabilities, obtaining reasonable 3D estimations when faced with uncertain ambiguities, such as the various plausible poses of the runner's right arm in the second case (as can be seen from the side view). Furthermore, even under complex actions and backgrounds, such as the person climbing in the third image, \vmproname\ also demonstrates robust estimation results.

Figure \ref{fig:vmpro_failure} shows a typical failure case of \vmproname\ on Internet image, where our method fails when heavy occlusion and ambiguity happen. Our method incorrectly estimates the right arm and right leg of the male dancer, which may be addressed using more powerful estimators or more diverse training datasets in the future. 
\vspace{-0.2cm}
\section{Conclusion}
\label{sec:conclusion}
We introduced \OURS{} - an intrinsically rotation-invariant model for point cloud matching. We proposed PAM~(PPF Attention Mechanism) that embeds PPF-based local coordinates to encode rotation-invariant geometry. This design lies at the core of AAL~(Attention Abstraction Layer), PAL~(PPF Attention Layer), and TUL~(Transition Up Layer) which are consecutively stacked to compose PPFTrans~(PPF Transformer) for representative and pose-agnostic geometry description. We further enhanced features by introducing a novel global transformer architecture, which ensures the rotation-invariant cross-frame spatial awareness.
%The global context is then aggregated for feature enhancement via the global transformer structure with the rotation-invariant cross-frame spatial awareness. 
Extensive experiments are conducted on both rigid and non-rigid benchmarks to demonstrate the superiority of our approach, especially the remarkable robustness against arbitrary rotations. %However, as \OURS{} does not explicitly handle the occlusion, it may fail in cases with extremely limited overlap. 
Limitations are discussed in the Appendix.

\noindent\textbf{Acknowledgment.} This paper is supported by the National Natural Science Foundation of China under Grant No. 62025208. We appreciate the help from Lennart Bastian, Mert Karaoglu, Ning Liu, and Zhiying Leng.

\renewcommand\thesection{\Alph{section}}
\setcounter{section}{0}
\section{Appendix for Proofs}

\paragraph{Proof of Theorem \ref{thm:main}.}

\begin{proof}
\label{proof:main}
Our proof has two steps. In Step 1, we will show that SimCLR is equivalent to minimizing the cross entropy loss defined in Eqn.~(\ref{eqn:cross-entropy}). 
In Step 2, we will show  that minimizing the cross-entropy loss 
is equivalent to spectral clustering on $\bfpi$. 
Combining the two steps together, we have proved our theorem. 

\textbf{Step 1: } SimCLR is equivalent to minimizing the cross entropy loss.

The cross-entropy loss takes expectation over 
$\bfW_\bfX\sim \mathbb{P}(\cdot ; \bfpi)$, 
which means $\bfW_\bfX$ has exactly one non-zero entry in each row $i$. By Lemma~\ref{lem:multinomial}, we know every row $i$ of $\bfW_\bfX$ is independent of other rows. Moreover, 
$\bfW_{\bfX,i}\sim \mathcal{M}(1, \bfpi_i/\sum_j \bfpi_{i,j})=\mathcal{M}(1, \bfpi_i)$, because $\bfpi_i$ itself is a probability distribution.
Similarly, we know $\bfW_\bfZ$ also has the row-independent property by sampling over $\mathbb{P}(\cdot;\bfK_\bfZ)$.
Therefore, by Lemma~\ref{lem:cross_split}, we know Eqn.~(\ref{eqn:cross-entropy}) is equivalent to:
\[
 -\sum_{i=1}^n \mathbb{E}_{\bfW_{\bfX,i}}[\log \mathbb{P}(\bfW_{\bfZ,i}=\bfW_{\bfX,i};\bfK_\bfZ)],
\]

This expression takes expectation over $\bfW_{\bfX,i}$ for the given row $i$. Notice that 
$\bfW_{\bfX,i}$ has exactly one non-zero entry, which equals $1$ (same for $\bfW_{\bfZ,i}$). 
As a result
we expand the above expression to be:
\begin{equation}
 -\sum_{i=1}^n \sum_{j\neq i} \Pr(\bfW_{\bfX,i,j}=1)\log \Pr(\bfW_{\bfZ,i,j}=1).
\label{eqn:detailed-expansion}    
\end{equation}


By Lemma~\ref{lem:multinomial}, $\Pr(\bfW_{\bfZ,i,j}=1)=\bfK_{\bfZ,i,j}/\|\bfK_{\bfZ,i}\|_1$ for $j\neq i$. Recall that $\bfK_\bfZ=(k(\bfZ_i-\bfZ_j))_{(i,j)\in[n]^2}$, which means 
$\bfK_{\bfZ,i,j}/\|\bfK_{\bfZ,i}\|_1=\frac{\exp(-\|\bfZ_i-\bfZ_j\|^2/{2\tau})}{\sum_{k\neq i}
\exp(-\|\bfZ_i-\bfZ_k\|^2/{2\tau})
}$ for $j\neq i$, when $k$ is the Gaussian kernel with variance $\tau$. 

Notice that $\bfZ_i=f(\bfX_i)$, so we know
\begin{equation}
-\log \Pr(\bfW_{\bfZ,i,j}=1)=
-\log \frac{\exp(-\|f(\bfX_i)-f(\bfX_j)\|^2/{2\tau})}{\sum_{k\neq i}
\exp(-\|f(\bfX_i)-f(\bfX_k)\|^2/{2\tau}),
}
\label{eqn:infonce-equivalence}    
\end{equation}


The right hand side is exactly the InfoNCE loss defined in Eqn.~(\ref{eqn:infonce}).
Inserting Eqn.~(\ref{eqn:infonce-equivalence}) into Eqn.~(\ref{eqn:detailed-expansion}), we get the SimCLR algorithm, which first samples augmentation pairs $(i,j)$ with $\Pr(\bfW_{\bfX,i,j}=1)$ for each row $i$, and then optimize the InfoNCE loss. 

\textbf{Step 2: } minimizing the cross entropy loss 
is equivalent to spectral clustering on $\bfpi$.


By Lemma~\ref{lem:convert_to_spectral}, we may further convert the loss to 
\begin{equation}
\label{eqn:main-theorem-repul-attr}
\min_{\bfZ}
-\sum_{(i,j)\in [n]^2} \mathbf{P}_{i,j}
\log k (\bfZ_i-\bfZ_j)+\log \mathbf{R}(\bfZ).
\end{equation}
Since $k$ is the Gaussian kernel, this reduces to \[
\min_\bfZ \mathrm{tr}(\bfZ^\top \mathbf{L}(\bfpi) \bfZ)
+\log \mathbf{R}(\bfZ),
\]

where we use the fact that $\mathbb{E}_{\bfW_\bfX\sim \mathbb{P}(\cdot; \bfpi)}[\mathbf{L}(\bfW_\bfX)]
=\mathbf{L}(\bfpi)
$, because the Laplacian operator is linear and $
\mathbb{E}_{\bfW_\bfX\sim \mathbb{P}(\cdot; \bfpi)}(\bfW_\bfX)=\bfpi
$.
\end{proof}

\paragraph{Proof of Theorem \ref{thm:clip}.}
\begin{proof}
Since $\bfW_\bfX\sim \mathbb{P}(\cdot;\bfpi_{\mathbf{A}, \mathbf{B}})$, we know 
$\bfW_\bfX$ has exactly one non-zero entry in each row, denoting the pair that got sampled. 
A notable difference compared to the previous proof is we now have $n_\mathcal{A}+n_\mathcal{B}$ objects in our graph. CLIP deals with this by taking a mini-batch of size $2N$, 
such that $n_\mathcal{A}=n_\mathcal{B}=N$, and adding the $2N$ InfoNCE losses together. We label the objects in $\mathcal{A}$ as $[n_\mathcal{A}]$, and the objects in $\mathcal{B}$ as $\{n_\mathcal{A}+1, \cdots, n_\mathcal{A}+n_\mathcal{B}\}$. 

Notice that $\bfpi_{\mathbf{A}, \mathbf{B}}$ is a bipartite graph, so the edges of objects in $\mathcal{A}$ will only connect to object in $\mathcal{B}$ and vice versa. We can define the similarity matrix in $\cZ$ as $\bfK_\bfZ$, 
where $\bfK_\bfZ(i, j+n_\mathcal{A})=\bfK_\bfZ(j+n_\mathcal{A},i)= k(\bfZ_i-\bfZ_j)$ for $i\in [n_\mathcal{A}], j\in [n_\mathcal{B}]$, and otherwise we set $\bfK_\bfZ(i,j)=0$. 
The rest is same as the previous proof. 
\end{proof}

\paragraph{Proof of Theorem \ref{thm:exponential}.}

\begin{proof}
\label{proof:exponential}
Since the objective function consists of a linear term combined with an entropy regularization, which is a strongly concave function, the maximization problem is a convex optimization problem. Owing to the implicit constraints provided by the entropy function, the problem is equivalent to having only the equality constraint. We then introduce the Lagrangian multiplier $\lambda$ and obtain the following relaxed problem:

$$
\widetilde{E}(\boldsymbol{\alpha})=\psi_{1}-\sum_{i=1}^n \alpha_{i} \psi_{i}+\tau \sum_{i=1}^n \alpha_{i}\log \alpha_{i}+\lambda\left(\boldsymbol{\alpha}^{\top} \mathbf{1}_n-1\right).
$$

As the relaxed problem is unconstrained, taking the derivative with respect to $\alpha_{i}$ yields

$$
\frac{\partial \widetilde{E}(\boldsymbol{\alpha})}{\partial \alpha_{i}}=-\psi_{i}+\tau\left(\log \alpha_{i}+\alpha_{i} \frac{1}{\alpha_{i}}\right)+\lambda=0.
$$

Solving the above equation implies that $\alpha_{i}$ takes the form
$
\alpha_{i}=\exp \left(\frac{1}{\tau} \psi_{i}\right) \exp \left(\frac{-\lambda}{\tau}-1\right).
$ Since $\alpha_{i}$ lies on the probability simplex, the optimal $\alpha_{i}$ is explicitly given by
$
\alpha^{*}_{i}=\frac{\exp \left(\frac{1}{\tau} \psi_{i}\right)}{\sum_{i^{\prime}=1}^n \exp \left(\frac{1}{\tau} \psi_{i^{\prime}}\right)} .
$ Substituting the optimal point into the objective function, we obtain
$$
\begin{aligned}
E\left(\boldsymbol{\alpha}^*\right)  &=\psi_1-\sum_{i=1}^n \frac{\exp \left(\frac{1}{\tau} \psi_{i}\right)}{\sum_{i^{\prime}=1}^n \exp \left(\frac{1}{\tau} \psi_{i^{\prime}}\right)} \psi_{i}+\tau \sum_{i=1}^n \frac{\exp \left(\frac{1}{\tau} \psi_{i}\right)}{\sum_{i^{\prime}=1}^n \exp \left(\frac{1}{\tau} \psi_{i^{\prime}}\right)}\log \frac{\exp \left(\frac{1}{\tau} \psi_{i}\right)}{\sum_{i^{\prime}=1}^n \exp \left(\frac{1}{\tau} \psi_{i^{\prime}}\right)} \\
& =\psi_1 - \tau \log \left(\sum_{i=1}^n \exp \left(\frac{1}{\tau} \psi_{i}\right)\right).
\end{aligned}
$$
Thus, the Lagrangian dual function is given by
\begin{equation*}
-E\left(\boldsymbol{\alpha}^*\right)= -\tau \log \frac{\exp \left(\frac{1}{\tau} \psi_{1}\right)}{\sum_{i=1}^n \exp \left(\frac{1}{\tau} \psi_{i}\right)}.\qedhere
\end{equation*}
\end{proof}



\section{More on Experiments} \label{section: experiment_details}

\paragraph{CIFAR-10 and CIFAR-100} CIFAR-10 ~\citep{krizhevsky2009learning} and CIFAR-100 ~\citep{krizhevsky2009learning} are well-known classic image classification datasets. Both CIFAR-10 and CIFAR-100 contain a total of 60k $32 \times 32$ labeled images of different classes, with 50k for training and 10k for testing. CIFAR-10 is similar to CIFAR-100, except there are 10 different classes in CIFAR-10 and 100 classes in CIFAR-100.

\paragraph{TinyImageNet} TinyImageNet ~\citep{le2015tiny} is a subset of ImageNet ~\citep{deng2009imagenet}. There are 200 different object classes in TinyImageNet, with 500 training images, 50 validation images, and 50 test images for each class. All the images in TinyImageNet are colored and labeled with a size of $64 \times 64$.

\textbf{Pseudo-code.} Algorithm \ref{alg:Training Procedure} presents the pseudo-code for our empirical training procedure.

\begin{algorithm}[!htbp]
\caption{Training Procedure}
\label{alg:Training Procedure}
\begin{algorithmic}[1]
\REQUIRE trainable encoder network $f$, batch size $N$, augmentation strategy \textit{aug}, loss function $L$ with hyperparameters \textit{args}
\FOR {sampled minibatch ${x_i}_{i=1}^N$}
\FORALL{$i \in { 1, ..., N }$}
\STATE draw two augmentations $t_i = \textit{aug}\left(x_i\right) $, $t_i' = \textit{aug}\left(x_i\right) $
\STATE $z_i = f\left(t_i\right)$, $z_i' = f\left(t_i'\right)$
\ENDFOR
\STATE compute loss $\mathcal{L} = L(N, z, z', \textit{args})$
\STATE update encoder network $f$ to minimize $\mathcal{L}$
\ENDFOR
\STATE \textbf{Return} encoder network $f$
\end{algorithmic}
\end{algorithm}

We also provide the pseudo-code for our core loss function used in the training procedure in Algorithm \ref{alg:Core loss}. The pseudo-code is almost identical to SimCLR's loss function, with the exception of an extra parameter $\gamma$.

\begin{algorithm}[!htbp]
\caption{Core loss function $\mathcal{C}$}
\label{alg:Core loss}
\begin{algorithmic}[1]
\REQUIRE batch size $N$, two encoded minibatches $z_1, z_2$, $\gamma$, temperature $\tau$
\STATE $z = \textit{concat}\left(z_1, z_2\right)$
\FOR {$i \in {1, ..., 2N }, j \in {1, ..., 2N}$ }
\STATE $s_{i,j} = \Vert z_i - z_j \Vert_2^{\gamma}$
\ENDFOR
\STATE \textbf{define} $l(i, j)$ \textbf{as} $l(i, j) = - \log \frac{exp\left(s_{i,j}/\tau \right)}{\sum_{k=1}^{2N} \mathbf{1}{[k \ne i]} exp\left(s{i, j} / \tau \right)} $
\STATE \textbf{Return} $\frac{1}{2N} \sum_{k=1}^N\left[l(i, i+N) + l(i+N, i)\right]$
\end{algorithmic}
\end{algorithm}

Utilizing the core loss function $\mathcal{C}$, we can define all kernel loss functions used in our experiments in Table \ref{table: loss definition}. For all $z_i \in z$ with even dimensions $n$, we define $z_{L_i} = z_i\left[0:n/2\right]$ and $z_{R_i} = z_i\left[n/2:n\right]$.

\begin{table}[ht]
\centering
\begin{tabular}{{@{}l|l@{}}}
Kernel  &  Loss function \\ \midrule
Laplacian & $\mathcal{C}\left(N, z, z', \gamma=1, \tau\right)$\\ \midrule
Sum       & $\lambda * \mathcal{C}\left(N, z, z', \gamma=1, \tau_1\right) + (1-\lambda) * \mathcal{C}\left(N, z, z', \gamma=2, \tau_2\right)$  \\ \midrule
Concatenation Sum&$\lambda * \mathcal{C}\left(N, z_L, z'_L, \gamma=1, \tau_1\right) + (1-\lambda) * \mathcal{C}\left(N, z_R, z'_R, \gamma=2, \tau_2\right)$\\ \midrule
$\gamma = 0.5$ & $\mathcal{C}\left(N, z, z', \gamma=0.5, \tau\right)$          \\ 

\end{tabular}

\caption{Definition of kernel loss functions in our experiments}
\label {table: loss definition}
\end{table}

\textbf{Baselines.} We reproduce the SimCLR algorithm using PyTorch Lightning~\citep{PytorchLightning}.

\textbf{Encoder details.}
The encoder $f$ consists of a backbone network and a projection network. We employ ResNet50~\citep{ResNet} as the backbone and a 2-layer MLP (connected by a batch normalization~\citep{ioffe2015batch} layer and a ReLU \cite{nair2010rectified} layer) with hidden dimensions 2048 and output dimensions 128 (or 256 in the concatenation kernel case).

\textbf{Encoder hyperparameter tuning.}
For each encoder training case, we randomly sample 500 hyperparameter groups (sample details are shown in Table \ref{table: Hyperparameter sample}) and train these samples simultaneously using Ray Tune ~\citep{RayTune}, with the ASHA scheduler~\citep{li2018massively}. Ultimately, the hyperparameter group that maximizes the online validation accuracy (integrated in PyTorch Lightning) within 5000 validation steps is chosen for the given encoder training case.

\begin{table}[ht]
\centering

\begin{tabular}{@{}l|l|l@{}}
\midrule
Hyperparameter  & Sample Range & Sample Strategy \\ \midrule
start learning rate & $\left[10^{-2}, 10\right]$ & log uniform \\ \midrule
$\lambda$       & $\left[0, 1\right]$ & uniform \\ \midrule
$\tau$, $\tau_1$, $\tau_2$ & $\left[0, 1\right]$ & log uniform \\ \midrule
\end{tabular}

\caption{Hyperparameters sample strategy}
\label {table: Hyperparameter sample}
\end{table}

\textbf{Encoder training.} 
We train each encoder using the LARS optimizer~\citep{LARSOptimizer}, LambdaLR Scheduler in PyTorch, momentum 0.9, weight decay $10^{-6}$, batch size 256, and the aforementioned hyperparameters for 400 epochs on a single A-100 GPU.

\textbf{Image transformation.} The image transformation strategy, including augmentation, is identical to the default transformation strategy provided by PyTorch Lightning.

\textbf{Linear evaluation.}
The linear head is trained using the SGD optimizer with a cosine learning rate scheduler, batch size 64, and weight decay $10^{-6}$ for 100 epochs. The learning rate starts at $0.3$ and ends at $0$.

\textbf{Moco Experiments.} We also tested our method based on MoCo~\citep{he2019moco}. The results are summarized in Table \ref{tab:results-moco}. Here we choose ResNet18~\citep{ResNet} as the backbone and set a temperature of $0.1$ as default. For our simple sum kernel, we set $\lambda=0.8$. The results show that our method outperforms the original MoCo method.

\begin{table}[thb]
\centering
\caption{MoCo Experiment Results on CIFAR-10 and CIFAR-100.}
\label{tab:results-moco}
\resizebox{\textwidth}{!}{%
\begin{tabular}{@{}c|ccc|ccc@{}}
\toprule
\multirow{3}{*}{Method} & \multicolumn{3}{c|}{CIFAR-10} & \multicolumn{3}{c}{CIFAR-100} \\ \cmidrule(lr){2-4} \cmidrule(lr){5-7} 
                        & 200 epochs & 400 epochs    & 1000 epochs   & 200 epochs & 400 epochs & 1000 epochs         \\ \midrule
MoCo (repro.)         & $76.41 \pm 0.12$    & $80.01 \pm 0.15$          & $84.45 \pm 0.08$    & $\mathbf{47.02 \pm 0.11}$ & $52.50 \pm 0.07$ & $57.62 \pm 0.15$            \\
\midrule
Laplacian Kernel        & ${78.09 \pm 0.10}$    & $\mathbf{83.85 \pm 0.09}$          & $\mathbf{88.34 \pm 0.16}$    & $46.12 \pm 0.22$   & $53.44 \pm 0.17$ & $59.10 \pm 0.14$        \\
Simple Sum Kernel & $\mathbf{78.12 \pm 0.15}$   & $83.23 \pm 0.18$ & $87.50 \pm 0.20$ & $46.65 \pm 0.06$ & $\mathbf{53.62 \pm 0.19}$ & $\mathbf{59.83 \pm 0.12}$\\
\bottomrule
\end{tabular}
}
\end{table}



\section{More Experiments on Synthetic Data}


Consider a scenario with $n$ clusters, each containing $k$ vertices. Let the probability of vertices $u$ and $v$ from the same cluster belonging to $\bfpi$ be $p$. Conversely, for vertices $u$ and $v$ from different clusters, let the probability of belonging to $\pi$ be $q$. We generate the graph $\bfpi$ randomly, based on $p$ and $q$. We experiment with values of $k=100$ and $n=6$ for ease of visualization, embedding all points in a two-dimensional space. Each vertex's initial position originates from a normal distribution. In each iteration, we sample a subgraph of $\bfpi$ uniformly, ensuring each vertex has an out-degree of $1$. We then optimize the corresponding vectors using InfoNCE loss with an SGD optimizer and iterate until convergence. Our experimental setup consists of an SGD learning rate of $1$, an InfoNCE loss temperature of $0.5$, and a batch size of $50$. We evaluate two scenarios with different $p$ and $q$ values: $p=1$, $q=0$, and $p=0.75$, $q=0.2$. The results of these experiments are visualized in Figure \ref{fig:vis-spectral-cluster}. The obtained embeddings exhibit the hallmark pattern of spectral clustering of graph $\bfpi$.

\begin{figure}[!tb]
\centering
\subfigure{
\includegraphics[width=1\textwidth]{Figures/cluster_pi.png}
\label{fig:vis-cluster}
}
\subfigure{
\includegraphics[width=1\textwidth]{Figures/noised_cluster_pi.png}
\label{fig:vis-noised-cluster}
}
\caption{Visualizations of the optimization process using InfoNCE Loss on the vectors corresponding to $\bfpi$. Points of identical color belong to the same cluster within $\bfpi$. To showcase the internal structure of $\bfpi$, we randomly select 10 vertices from each cluster to display the edge distribution of $\bfpi$.}
\label{fig:vis-spectral-cluster}
\end{figure}




%%%%%%%%% REFERENCES
{\small
\bibliographystyle{ieee_fullname}
\bibliography{egbib}
}

\end{document}
