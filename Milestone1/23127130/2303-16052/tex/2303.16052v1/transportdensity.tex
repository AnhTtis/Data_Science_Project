\section{Horizontal Transport Density on $\mathbb{H}^n$}\label{sectrandens}
In this section we introduce the notion of horizontal transport density, extending to the Heisenberg group setting the presentation provided in \cite{Santambrogio2}. an horizontal transport density  is a measure representing the density of transport along horizontal curves and it is computed using geodesics of the space. Then we study conditions under which the transport density is Lebesgue absolutely continuous, with density in $L^p$.

Let $\Omega\subset\mathbb{H}^{n}$ be compact and geodesically convex, and $\mu,\nu\in\mathcal{P}(\Omega)$. One can associate to any optimal transport plan $\gamma\in\Pi_1(\mu,\nu)$ a positive and finite Radon measure $a_\gamma\in\mathcal{M}_+(\Omega)$,
\begin{equation}\label{transport density}
	\int_{\Omega} \phi(x)da_\gamma(x):=\int_{\Omega\times\Omega}\bigg{(}\int_0^1\phi(S_t(x,y))|\dot{S}_t(x,y)|_Hdt\bigg{)}d\gamma(x,y),\quad \forall\phi\in C(\Omega),
\end{equation}
where $[0,1]\ni t\mapsto S_t(x,y)$ is the minimizing geodesic between $x$ and $y$, see \eqref{geod01}, $\dot{S}_t(x,y):=\frac{d}{dt}S_t(x,y)$ and $|\dot S_t(x,y)|_H=d_{CC}(x,y)$ is its constant speed and
\begin{displaymath}
	C(\Omega):=\{\phi:\Omega\rightarrow\mathbb{R},\, \phi\, \textnormal{continuous}\}.
\end{displaymath}
From Proposition \ref{pi1.1} it follows that, if $\gamma\in\Pi_1(\mu,\nu)$, then for $\gamma$-a.e. $(x,y)$ there exists a unique minimizing geodesic between $x$ and $y$, and hence $a_\gamma$ is well-defined.
According to the terminology used in literature, we will call this measure \textit{horizontal transport density} and it represents the amount of transport taking place in each region of $\Omega$.
If we look at the action of $a_{\gamma}$ on sets, we have that for every Borel set $A$,
\begin{equation}\label{densitylength}
	a_\gamma(A)=\int_{\Omega\times\Omega}\mathcal{H}^1(A\cap S(x,y))d\gamma(x,y).
\end{equation}

\subsection{Horizontal transport densities absolutely continuous w.r.t. $\mathcal{L}^{2n+1}$}
The first goal is to prove the existence of at least one horizontal transport density, absolutely continuous w.r.t. the Haar measure of the group. Let $\mu_t$ be the displacement interpolation between $\mu$ and $\nu$ associated with a generic $\gamma\in\Pi_1(\mu,\nu)$, 
\begin{equation*}\label{interpolation}
	\mu_t:=((S_t)_\sharp \gamma)_{t\in [0,1]},
\end{equation*}
the horizontal transport density $a_\gamma$, associated with $\gamma$, may be written as 
\begin{equation*}\label{density2}
	a_\gamma=\int_0^1(S_t)_\sharp (d_{CC}\gamma)dt
\end{equation*}
where $d_{CC}\gamma$ is a positive measure on $\Omega\times\Omega$.
Since $\Omega$ is bounded, there exist $C>0$ such that $d_{CC}(x,y)\leq C$ for any $x,y$ in $\Omega$ and hence 
\begin{equation}
	\label{density3}
	a_\gamma\leq C\int_0^1\mu_t dt.
\end{equation}
To prove that $a_\gamma$ is absolutely continuous w.r.t. $\mathcal{L}^{2n+1}$, it is sufficient to prove that $\mu_t$ is absolutely continuous w.r.t. $\mathcal{L}^{2n+1}$ for almost every $t\in[0,1]$. In this way we get that, whenever $\mathcal{L}^{2n+1}(A)=0$, then 
\begin{equation}\label{density4}
a_\gamma(A)\leq C\int_0^1\mu_t(A)dt=0.
\end{equation}
\\

 We introduce now  the following lemma which guarantees that minimizing geodesics used by optimal plans cannot intersect at intermediate points. This result will be useful in the proof of Theorem \ref{absolutecont}.
\begin{lemma} \label{pi1.3}
	Let $\gamma\in \Pi_1(\mu,\nu)$. Then $\gamma$ is concentrated on a set $\Gamma$ such that $\forall(x,y),(x',y')\in \Gamma$ with $(x,y)\not=(x',y')$, if the minimizing geodesics between these two pairs of points intersect at a point $z\in\Omega$, then all points $x$, $x'$, $y$, $y'$ and $z$ lie on the same minimizing geodesic.
	Moreover if $\gamma\in\Pi_2(\mu,\nu)$, then either $x\leq x'\leq z\leq y\leq y'$ or $x'\leq x\leq z\leq y'\leq y$.
\end{lemma}
\begin{proof}
	We first recall that \eqref{c-CM} reads as
	\begin{equation}\label{aaa}
		d_{CC}(x,y)+d_{CC}(x',y')\leq d_{CC}(x,y')+d_{CC}(x',y),
	\end{equation}
	$\forall (x,y),(x',y')\in\Gamma$.
	Let $\sigma:[0,d_{CC}(x,y)]\rightarrow\Omega$ a minimizing geodesic between $x$ and $y$, $\tilde\sigma:[0,d_{CC}(x',y')]\rightarrow\Omega$ a minimizing geodesic between $x'$ and $y'$, $z\in\sigma(0,d_{CC}(x,y))\cap\tilde\sigma(0,d_{CC}(x',y'))$, so $z=\sigma(d_{CC}(x,z))=\tilde\sigma(d_{CC}(x',y')-d_{CC}(z,y'))$. We denote by $\beta$ the curve between $x$ and $y'$ defined in the following way:
	\begin{displaymath}
		\beta(t):=\begin{cases}
			\sigma(t),\quad \textnormal{if}\,\ t\in[0,d_{CC}(x,z)],\\
			\tilde\sigma(t),\quad \textnormal{if}\,\ t\in[d_{CC}(x,z),d_{CC}(x',y')].
		\end{cases}
	\end{displaymath}
	We will prove that $\beta$ is minimizing geodesic between $x$ and $y'$. Indeed, otherwise we would have 
	\begin{equation}\label{aab}
		d_{CC}(x,y')<l_H(\beta)=l(\sigma_{|[0,d_{CC}(x,z)]})+l_H(\tilde\sigma_{|[d_{CC}(x,z),d_{CC}(x',y')]})=d_{CC}(x,z)+d_{CC}(z,y').
	\end{equation}
	Since $z$ lies on both minimizing geodesic between $x$ and $y$ and minimizing geodesic between $x'$ and $y'$, it follows that 
	\begin{equation}\label{aac}
		\begin{cases}
			d_{CC}(x,y)=d_{CC}(x,z)+d_{CC}(z,y);\\
			d_{CC}(x',y')=d_{CC}(x',z)+d_{CC}(z,y').
		\end{cases}	
	\end{equation}
	By replacing (\ref{aac}) in (\ref{aab}), we obtain:
	\begin{equation}\label{aad}
		d_{CC}(x,y')+d_{CC}(z,y)+d_{CC}(x',z)<d_{CC}(x,y)+d_{CC}(x',y').
	\end{equation}
	By the triangle inequality follows that:
	\begin{displaymath}
		d_{CC}(x',y)\leq d_{CC}(x',z)+d_{CC}(z,y),
	\end{displaymath}
	and then, by replacing this last inequality in (\ref{aad}), we obtain
	\begin{displaymath}
		d_{CC}(x,y')+d_{CC}(x',y)<d_{CC}(x,y)+d_{CC}(x',y'),
	\end{displaymath}
	and this contraddicts (\ref{aaa}). It follows that $\sigma$ and $\beta$ are minimizing geodesics that coincide on the non trivial interval $[0,d_{CC}(x,z)]$. Since $\mathbb{H}^n$ is non-branching, this implies that $\sigma$ and $\beta$ are sub-arcs of the same minimizing geodesic, namely $\sigma$ if $d_{CC}(z,y')\leq d_{CC}(z,y)$ and $\beta$ otherwise, on which all points $x,z,y',y$ lie.
	
	The thesis follows from Proposition \ref{monotone}.
\end{proof}
If $T:\Omega\rightarrow\Omega$, we set
\begin{equation*}
	T_t:=S_t\circ(\text{Id}\otimes T):\Omega\rightarrow\Omega,
\end{equation*}
where $T_t(x)$ is the point at distance $td_{CC}(x, T(x))$ from $x$ on the selected minimizing geodesic $S(x,T(x))$ between $x$ and $T(x)$. We observe that, in this case, $\mu_t=(S_t)_\sharp\gamma=T_{t\,\sharp}\mu$, where $\gamma:=(\text{Id}\otimes T)_\sharp\mu\in\Pi(\mu,\nu)$.

The previous lemma makes us able to prove the following result.

\begin{prop}\label{absolutecontinterp}
	Suppose that $\mu\ll\mathcal{L}^{2n+1}$ then, there exists $\overline{\gamma}\in\Pi_1(\mu,\nu)$ such that
	\begin{equation}\label{absolutecontinter1}
		\overline\mu_t:=(S_t)_\sharp\overline\gamma\ll\mathcal{L}^{2n+1},\quad \forall t\in[0,1).
	\end{equation}
\end{prop}

\begin{proof}	
	First we suppose that $\nu$ is finitely atomic, with atoms $(y^i)_{i=1}^M$. Let $\gamma\in\Pi_2(\mu,\nu)\subset\Pi_1(\mu,\nu)$, as in Theorem \ref{mainthmbis}, which is monotone in the sense of \eqref{orderrelation} and induced by a transport map $T$. 
	
	We denote by $\Omega_i:=T^{-1}(\{y^i\})\cap \pi_1(\Gamma)$, where $\Gamma$ is the set on which $\gamma$ is concentrated: obviously these sets are mutually disjoint. Moreover $\hat{\Omega}:=\bigcup_{i=1}^N\Omega_i\subseteq\Omega$ and $\mu(\hat{\Omega})=1$.
	
	Now we denote by $\Omega_i(t):=T_t(\Omega_i)$: if we fix $t\in[0,1[$, then $\Omega_i(t)\cap\Omega_j(t)=\emptyset$ for every $i,j=1,\ldots,N$. Indeed, if $\exists\ z\in\Omega_i(t)\cap\Omega_j(t)$ then $\exists\ x^i\in \Omega_i$ and $x^j\in\Omega_j$ such that $(x^i,y^i), (x^j,y^j)\in \Gamma, (x^i,y^i)\not=(x^j,y^j)$ and the minimizing geodesics between these two pairs of points intersect at $z$. Since $\gamma\in\Pi_2(\mu,\nu)$, by Theorem \ref{pi1.3} we can suppose that $x^i,y^i,x^j,y^j,z$ belong to the same unit-speed minimizing geodesic and $x^i\leq x^j\leq z\leq y^i\leq y^j$. In particular this means, on the one hand, that $td_{CC}(x^i,y^i)=d_{CC}(x_i,z)\geq d_{CC}(x^j,z)=td_{CC}(x^j,y^j)$, hence $d_{CC}(x^i,y^i)\geq d_{CC}(x^j,y^j)$. On the other hand $(1-t)d_{CC}(x^i,y^i)=d_{CC}(z,y_i)\leq d_{CC}(z,y^j)=(1-t)d_{CC}(x^j,y^j)$, hence $d_{CC}(x^i,y^i)\leq d_{CC}(x^j,y^j)$. It follows that $d_{CC}(x^i,y^i)= d_{CC}(x^j,y^j)$ and hence $d_{CC}(x^i,z)=d_{CC}(x^j,z)$ and $d_{CC}(z,y^i)=d_{CC}(z,y^j)$, which in turn implies that $x^j=x^i$ and $y^i=y^j$ and gives a contradiction.
	
	Remember also that $\mu$ is absolutely continuous and hence there exists a correspondence $\varepsilon\mapsto\delta=\delta(\varepsilon)$ such that 
	\begin{equation*}
		\mathcal{L}^{2n+1}(A)<\delta(\varepsilon)\Rightarrow\mu(A)<\varepsilon.
	\end{equation*}
	
	Let $A$ a Borel set, $t\in[0,1)$,
	\begin{equation*}
		\mu_t(A)=\sum_{i=1}^{M}\mu_t(A\cap\Omega_i(t))=\sum_{i=1}^M\mu(T_t^{-1}(A\cap\Omega_i(t)))=\mu\left(\bigcup_{i=1}^M(T_t^{-1}(A\cap\Omega_i(t)))\right),
	\end{equation*}
	since the sets $T_t^{-1}(A\cap\Omega_i(t))\subseteq\Omega_i$ are disjoint.
	We observe that for any $x\in\Omega_i$, $T_t(x)=S_t(x,y^i)$, hence by \eqref{MCP} follows that
	\begin{equation*}
		\mathcal{L}^{2n+1}(E)\leq\frac{1}{(1-t)^{2n+3}}\mathcal{L}^{2n+1}(T_t(E)),
	\end{equation*}
	for any $E\subset\Omega_i$. This in turn implies that
	\begin{equation*}
		\mathcal{L}^{2n+1}(T_t^{-1}(A\cap\Omega_i(t)))\leq\frac{1}{(1-t)^{2n+3}}\mathcal{L}^{2n+1}(A\cap \Omega_i(t)),
	\end{equation*}
	and so 
	\begin{equation*}
		\mathcal{L}^{2n+1}\left(\bigcup_{i=1}^M(T_t^{-1}(A\cap\Omega_i(t)))\right)\leq\frac{1}{(1-t)^{2n+3}}\mathcal{L}^{2n+1}(A).
	\end{equation*}
	Hence, it is sufficient to suppose that $\mathcal{L}^{2n+1}(A)<(1-t)^{2n+3}\delta(\varepsilon)$ to get $\mu_t(A)<\varepsilon$.	This proves that $\mu_t\ll\mathcal{L}^{2n+1}$.
	
	Now, if $\nu$ is not finitely atomic, we can take a sequence $(\nu_k)_{k\in\mathbb{N}}$ of atomic measures weakly converging to $\nu$. The corresponding optimal plans $\gamma_k$ weakly converge to an optimal plan $\overline\gamma$ and $\mu_t^k$ weakly converge to the corresponding $\overline\mu_t:=(S_t)_{\sharp}\overline{\gamma}$, see \cite[Theorem 1.50]{Santambrogiolibro}. Take a set $A$ such that $\mathcal{L}^{2n+1}(A)<(1-t)^{2n+3}\delta(\varepsilon)$. Since the Lebesgue measure is regular, $A$ is included in an open set $B$ such that $\mathcal{L}^{2n+1}(B)<(1-t)^{2n+3}\delta(\varepsilon)$. Hence $\mu_t^k(B)<\varepsilon$. Passing to the limit and using Portmanteau's Theorem, see \cite[Theorem 2.1]{Billingsley}, we get 
	$$\overline\mu_t(A)\leq\overline\mu_t(B)\leq\liminf_k \mu_t^k(B)\leq\varepsilon.$$
	This proves that $\overline\mu_t\ll\mathcal{L}^{2n+1}$.
\end{proof}

Now we are able to find at least an optimal transport plan $\overline\gamma\in\Pi_1(\mu,\nu)$ such that the interpolation measures $\mu_t$ constructed from $\overline\gamma$ are absolutely continuous for $t<1$.
\\

\begin{teo}\label{absolutecont}
	Suppose that $\mu\ll\mathcal{L}^{2n+1}$ then $\exists\overline\gamma\in\Pi_1(\mu,\nu)$ such that $a_{\overline\gamma}\ll\mathcal{L}^{2n+1}$.
\end{teo}
\begin{proof}
	Let $\overline\gamma\in\Pi_1(\mu,\nu)$ satisfying \eqref{absolutecontinter1}. Then, the thesis follows immediately from \eqref{density4} applied to $a_{\overline\gamma}$.
\end{proof}

Obviously the previous argument depends only on one of the two marginals and it is completely symmetric: if $\nu\ll\mathcal{L}^{2n+1}$, again one can get the existence of an optimal transport plan $\overline\gamma\in\Pi_1(\mu,\nu)$ such that the associated horizontal transport density $a_{\overline\gamma}$ is absolute continuous w.r.t. the $(2n+1)$-dimensional Lebesgue measure.
 

%if $\mu\ll\mathcal{L}^{2n+1}$, resp. $\mu_s\ll\mathcal{L}^{2n+1}$ for $s\in(0,1)$, we will denote by $\rho$ its density, $\rho_s$ resp. $\rho_s$. We will write $\mu\in L^p$, resp. $\mu_s\in L^p$, if $\mu\ll\mathcal{L}^{2n+1}$ with $\rho\in L^p(\Omega)$ and $\|\mu\|_p=\|\rho\|_{L^p(\Omega)}$ , resp. $\mu_s\ll\mathcal{L}^{2n+1}$, with $\rho_s\in L^p(\Omega)$ and $\|\mu_s\|_p=\|\rho_s\|_{L^p(\Omega)}$. 

\subsection{p-summable horizontal transport densities}
The next step is to prove, under some suitable assumptions, the existence of at least one optimal plan $\gamma\in\Pi_1(\mu,\nu)$, whose associated horizontal transport density belongs to $L^p$, for some $p\in[1,\infty]$. From now on, given $\overline{\mu}\in\mathcal{M}_+(\Omega)$ we will write that $\overline{\mu}\in L^p$ if $\overline{\mu}\ll\mathcal{L}^{2n+1}$, with density $\overline{\rho}\in L^p(\Omega)$. We will denote by $\|\overline{\mu}\|_p:=\|\overline{\rho}\|_{L^p(\Omega)}$. 

Let $\gamma\in\Pi_1(\mu,\nu)$ be an optimal transport plan as in Theorem \ref{absolutecont}, using Minkowski inequality in \eqref{density3} it holds
\begin{equation}\label{Minkowski}
	\|a_\gamma\|_p\leq C\int_{0}^1\|\mu_t\|_pdt.
\end{equation}

In order to prove $p$- summability of $a_\gamma$, for some $\gamma\in\Pi_1(\mu,\nu)$, it is sufficient to prove that almost every measure $\mu_t$ is in $L^p$ and to estimate their $L^p$ norms, choosing a posteriori $p$ such that the integral above converges.

In the following theorem we will estimate for all $t\in(0,1)$ the $L^p$ norm of the interpolation measures $\mu_t$, associated with transport plans $\gamma\in\Pi_1(\mu,\nu)$ that satisfy the thesis of Theorem \ref{absolutecont}.

\begin{prop}\label{summabilityinterp}
	If $\mu\in L^p$, then $\exists\overline\gamma\in\Pi_1(\mu,\nu)$ such that $\overline\mu_t:=(S_t)_{\sharp}\overline\gamma$ satisfies
	\begin{equation}\label{bb3}
		\|\overline\mu_t\|_p\leq (1-t)^{-(2n+3)/p'}\|\mu\|_p,\quad \forall t\in(0,1),
	\end{equation}
 where $p':=\frac{p}{p-1}$ is the conjugate exponent of $p$.
\end{prop}
\begin{proof}
	Let us denote by $\rho$ the density of $\mu$ w.r.t. $\mathcal{L}^{2n+1}$. Consider first the discrete case: let us assume that the target measure $\nu$ is finitely atomic and let us denote by $(y^i)_{i=1,\ldots,M}$ its atoms, as in the previous proof. Let us consider an optimal transport plan $\gamma\in\Pi_2(\mu,\nu)$ as in Theorem \ref{mainthmbis}. As before, since $\gamma$ is induced by a map $T$, $\Omega$ can be decomposed in subsets $\Omega_i$ mutually disjoint, $i\in\{1,\ldots,M\}$, such that for $\gamma$-a.e. $(x,y)\in\Omega_i\times\Omega$, we have $y=y_i$.
	Let us consider the corresponding interpolation measures $\mu_t$. As in the proof of Theorem \ref{absolutecontinterp} we get that $\mu_t\ll\mathcal{L}^{2n+1}$ for every $t\in[0,1)$; moreover, for all $\phi\in C(\Omega)$, by definition of push-forward, we get that 
	\begin{align*}
		\int_\Omega\phi(x)d\mu_t(x)=&\sum_{i=1}^M\int_{\Omega_i\times\Omega}\phi(S_t(x,y_i))d\gamma(x,y_i)=\\
		=&\sum_{i=1}^M\int_{\Omega_i}\phi(T_t(x))d\mu(x).
	\end{align*}
	Let us fix $i\in\{1,\ldots,M\}$ and let us denote by $\rho_t$ the density of $\mu_t$ w.r.t. $\mathcal{L}^{2n+1}$ and by $\rho_t^i:={\rho_t}_{\lfloor \Omega_i}$. Let us take the change of variable $z=S_t(x,y_i)={T_t}_{\lfloor\Omega_i}(x)$. We know, from Lemma \ref{pi1.3} and disjointness of the sets $\Omega_i(t)$, that this map is injective. Then, for all $\phi\in C(\Omega_i)$, we get 
	\begin{align*}
		\int_{\Omega_i}\phi(x)d\mu^i_t(x)&=\int_{\Omega_i}\phi(T_t(x))\rho(x)dx=\\&=
		\int_{\Omega_i(t)}\phi(z)\rho(T_t^{-1}(z))|\det\mathcal{J}_x(S_t(x,y^i))|^{-1}dz.
	\end{align*}
	Hence, we have that 
	\begin{equation*}
		\rho_t^i(z)=\rho(T_t^{-1}(z))|\det\mathcal{J}_x(S_t(x,y^i))|^{-1},\quad \text{for a.e. } z\in\Omega_i(t).
	\end{equation*}
	Consequently, we get 
	\begin{align*}
		\|\rho_t^i\|^p_{L^p(\Omega_i(t))}&=\int_{\Omega_i(t)}\rho(T_t^{-1}(z))^p|\det\mathcal{J}_x(S_t(x,y^i))|^{-p}dz=\\&=\int_{\Omega_i}\rho(x)^p|\det\mathcal{J}_x(S_t(x,y^i))|^{1-p}dx.
	\end{align*}
	Hence from \eqref{det} it follows that
	\begin{align*}
		\|\rho_t^i\|^p_{L^p(\Omega_i(t))}\leq (1-t)^{(1-p)(2n+3)}\|\rho\|^p_{L^p(\Omega_i)},\quad \forall i\in\{1,\ldots,M\}.
	\end{align*}
	Then, we have
	\begin{equation*}
		\|\mu_t\|_p\leq (1-t)^{-(2n+3)/p'}\|\mu\|_p,\quad \forall t\in(0,1).
	\end{equation*}
	
	As in the proof of Proposition \ref{absolutecontinterp},  if $\nu$ is not finitely atomic, we can take a sequence $(\nu_k)_{k\in\mathbb{N}}$ of atomic measures weakly converging to $\nu$. The corresponding optimal plans $\gamma_k$ weakly converge to an optimal plan $\overline\gamma$ and $\mu_t^k$ weakly converge to the corresponding $\overline\mu_t:=(S_t)_{\sharp}\overline{\gamma}$. Hence, we get that
	\begin{equation*}
		\|\overline\mu_t\|_p\leq\liminf_{k \rightarrow 0}\|\mu_t^k\|_p\leq (1-t)^{-(2n+3)/p'}\|\mu\|_p.
	\end{equation*}
\end{proof}

Now we are able to prove the following theorem:
\begin{prop}\label{summability1}
	If $\mu\in L^p$ for some $p\in[1,\infty]$, then
	\begin{itemize}
		\item If $p<\frac{2n+3}{2n+2}$ then $\exists\overline\gamma\in\Pi_1(\mu,\nu)$ such that $a_{\overline\gamma}\in L^p(\Omega)$.
		\item If $p\geq\frac{2n+3}{2n+2}$ then $\exists\overline\gamma\in\Pi_1(\mu,\nu)$ such that $a_{\overline\gamma}\in L^s(\Omega)$ for $s<\frac{2n+3}{2n+2}$.
	\end{itemize} 
\end{prop}
\begin{proof}
	Let $\overline\gamma\in\Pi_1(\mu,\nu)$ satisfying \eqref{bb3}. Then, it follows from \eqref{Minkowski} applied to $a_{\overline\gamma}$ that
	\begin{equation*}
		\|a_{\overline{\gamma}}\|_p\leq C\int_{0}^1 \|\overline\mu_t\|_pdt\leq C\|\mu\|_p\int_0^1(1-t)^{-(2n+3)/p'}dt.
	\end{equation*}
	The last integral is finite whenever $p'>2n+3$, i.e. $p<\frac{2n+3}{2n+2}$.
	
	If $p\geq\frac{2n+3}{2n+2}$ the thesis follows from the fact that any density in $L^p$ also belongs to any $L^s$ space for $s<p$.
\end{proof}

If also $\nu\in L^p$, by symmetry and using the same strategy as before, it is obvious that one can also show the same $L^p$ estimates on $\mu_t$ but from the other side: i.e., with the same notations as in proof of Proposition \ref{summabilityinterp}, by approximating $\mu$ with a sequence of atomic measures we get
\begin{equation}\label{bb1}
	\|\overline\mu_t\|_p\leq t^{-(2n+3)/p'}\|\nu\|_p,\quad \forall t\in(0,1).
\end{equation}
Combining \eqref{bb3} e \eqref{bb1}, we infer that, $\forall t\in(0,1)$, it holds
\begin{align*}
	\|\overline\mu_t\|_p&\leq\min\left\{(1-t)^{-(2n+3)/p'}\|\mu\|_p,t^{-(2n+3)/p'}\|\nu\|_p\right\}\leq 2^{(2n+3)/p'}\max\{\|\mu\|_p,\|\nu\|_p\}.
\end{align*}
However the previous $L^p$ estimates on $\mu_t$ for $t\leq\frac{1}{2}$ and $t\geq\frac{1}{2}$ have been obtained by discrete approximations on $\mu$ and $\nu$, respectively. If the two approximations converge to two different transport plans between $\mu$ and $\nu$, then we cannot glue together the two estimates on $\overline\mu_t$ and deduce anything about the summability of $a_{\overline{\gamma}}$. In the Euclidean setting, Santambrogio in \cite{Santambrogio2} used the uniqueness of the monotone optimal transport plan, see for instance \cite[Theorem 3.18]{Santambrogiolibro}. As far as we know, this result is still true in the Riemannian setting, see \cite{Feldman}, but it has not been proven in the Heisenber group yet. Using a different technique, as in \cite{Dweik}, we will prove the existence of at least one $p$-summable horizontal transport density, for any $p\in[1,\infty]$, provided that the two measures $\mu$ and $\nu$ stay in $L^p$ as well. 

\begin{teo}\label{summability2}
	If $\mu,\nu\in L^p$, then $\exists\overline\gamma\in\Pi_1(\mu,\nu)$ such that $a_{\overline\gamma}\in L^p$, for any $p\in[1,\infty]$.
\end{teo}
\begin{proof} 
We consider the Kantorovich problem between $\mu$ and $\nu$ associated with the transport cost $d_{CC}+\varepsilon d_{CC}^2$, with $\varepsilon>0$, which has a unique solution $\gamma_\varepsilon$ (see Theorem \ref{mainthmtris}). If we denote by $\mu_t^\varepsilon:=({S_t})_{\#}\gamma_\varepsilon$, for $t\in[0,1)$ then one can get the same $L^p$ estimates \eqref{bb3} and \eqref{bb1} and then
	\begin{equation*}
		\|\mu_t^\varepsilon\|_p\leq 2^{(2n+3)/p'}\max\{\|\mu\|_p,\|\nu\|_p\},\quad \forall t\in[0,1).
	\end{equation*}
	If $\varepsilon\rightarrow 0$ then $\gamma_\varepsilon\rightharpoonup\overline{\gamma}$, for some optimal transport $\overline{\gamma}\in\Pi_2(\mu,\nu)$. Moreover, $\mu_t^\varepsilon\rightharpoonup\overline{\mu}_t$, where $\overline{\mu}_t:=({S_t})_{\#}\overline{\gamma}$ so we can pass to the limit, uniformly w.r.t. $\varepsilon$
	\begin{equation}\label{estimate1}
		\|\overline{\mu}_t\|_p\leq\liminf_{\varepsilon\rightarrow0}\|\mu_t^\varepsilon\|_p\leq 2^{(2n+3)/p'}\max\{\|\mu\|_p,\|\nu\|_p\},\quad \forall t\in[0,1).	
	\end{equation}
	The thesis will follow by applying \eqref{estimate1} and \eqref{Minkowski} to the horizontal transport density $a_{\overline{\gamma}}$ associated with $\overline{\gamma}$.
\end{proof}

Using the same technique we can improve Theorem \ref{summability1}:

\begin{teo} 
	If $\mu\in L^{p_1}$ and $\nu\in L^{p_2}$, with $p_1>p_2\geq1$ hence: 
	\begin{enumerate}
		\item If $p_1<\frac{2n+3}{2n+2}$, then $\exists\overline\gamma\in\Pi_1(\mu,\nu)$ such that $a_{\overline\gamma}\in L^{p_1}$;
		\item If $p_1\geq\frac{2n+3}{2n+2}$, then $\exists\overline{\gamma}\in\Pi_1(\mu,\nu)$ such that $a_{\overline\gamma}\in L^s$ for all the exponents $s$ satisfying
		\begin{equation*}
			s<s_0=s_0(p_1,p_2,2n+3):=\frac{p_2(2n+3)(p_1-1)}{(2n+3)(p_1-1)-(p_1-p_2)}.
		\end{equation*}
	\end{enumerate} 
	\begin{proof}
		We take $\overline{\gamma}$ and $\overline{\mu}_t$ as in the previous proof. Using exactly the same argument as in Theorem \ref{summability2} we get
		\begin{equation*}
			\|\overline{\mu}_t\|_{p_1}\leq (1-t)^{-(2n+3)/p_1^*}\|\mu\|_{p_1}.
		\end{equation*}
		The first part of the statement (the case $p_1<\frac{2n+3}{2n+2})$ follows exactly as in Theorem \ref{summability1}. In order to prove the second one, we observe that we can get also
		\begin{equation*}
			\|\overline{\mu}_t\|_{p_2}\leq t^{-(2n+3)/p_2^*}\|\nu\|_{p_2}.
		\end{equation*}
		We then apply standard H\"older inequality to derive the usual interpolation estimate for any exponent $p_2<s<p_1$:
		\begin{equation*}
			\|f\|_s\leq\|f\|^\alpha_{p_1}\|f\|^{1-\alpha}_{p_2} \quad \textnormal{with }\alpha=\frac{p_1(s-p_2)}{s(p_1-p_2)},\ \textnormal{and }1-\alpha=\frac{p_2(p_1-s)}{s(p_1-p_2)}.
		\end{equation*}
		This implies 
		\begin{equation*}
			\|\overline{\mu}_t\|_s\leq\|\overline{\mu}_t\|_{p_1}^1\|\overline{\mu}_t\|_{p_2}^0\leq(1-t)^{-(2n+3)/p_1^*}\|\mu\|_{p_1}\leq C\|\mu\|_{p_1}, \quad \text{for } t<\frac{1}{2};
		\end{equation*}
		and
		\begin{align*}
			\|\overline{\mu}_t\|_s&\leq\|\overline{\mu}_t\|_{p_1}^\alpha\|\overline{\mu}_t\|_{p_2}^{1-\alpha}\leq \big{(}(1-t)^{-(2n+3)/p_1^*}\|\mu\|_{p_1}\big{)}^{\alpha}\big{(} t^{-(2n+3)/{p_2}^*}\|\nu\|_{p_2}\big{)}^{1-\alpha}\leq\\ &\leq C' (1-t)^{-\alpha(2n+3)/p_1^*}\|\mu\|_{p_1}^{\alpha}\|\nu\|_{p_2}^{1-\alpha},\quad \text{for }\frac{1}{2}<t<1.
		\end{align*}
		Then, take $s<s_0$, so that $\alpha(2n+3)/p_1^*<1$ is ensured and hence the $L^s$ norm is integrable, it follows, again by \eqref{Minkowski}, a bound on $\|a_{\overline{\gamma}}\|_s$. 
	\end{proof}
\end{teo}
\begin{Remark}
	We observe that, if $p_1>p_2\geq1$ and $p_1\geq(2n+3)/(2n+2)$ then it holds
	\begin{equation}\label{betterexp}
		\frac{p_2(2n+3)(p_1-1)}{(2n+3)(p_1-1)-(p_1-p_2)}\geq \frac{2n+3}{2n+2}.
	\end{equation}
It is a direct computation. Let us first  note that the denominator can be written as $(2n+2)(p_1-1)+(p_2-1)$ so that \eqref{betterexp}
is equivalent to $$\frac{p_2(2n+2)(p_1-1)}{(2n+2)(p_1-1)+(p_2-1)}\geq 1
$$
This is equivalent to 
$$	p_2(2n+2)(p_1-1)-(2n+2)(p_1-1)-(p_2-1)\geq 0$$
and to 
$$	 (p_2-1)((2n+2)(p_1-1)-1) \geq 0.$$
By hypothesis we know that $p_2-1\geq0$ and $(2n+2)(p_1-1)\geq(2n+2)\left(\frac{2n+3}{2n+2}-1\right)=1$, hence the conclusion follows. 

As in the Euclidean case (see \cite{Santambrogio2}), we do not know whether this exponent $s_0=s_0(p_1,p_2,2n+3)$ is sharp or not and whether there exists $\gamma\in\Pi_1(\mu,\nu)$ such that $a_\gamma$ belongs or not to $L^{s_0}$. 
\end{Remark}
