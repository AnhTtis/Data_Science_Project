For every $n\geq1$, the $n$-th \textit{Heisenberg group} $\mathbb{H}^n$ is the connected, simply connected and nilpotent Lie group, whose Lie algebra $\mathfrak{h}^n$ is stratified of step 2; i.e. it is the direct sum of two linear subspaces
$$ \mathfrak{h}^n= \mathfrak{h}^n_1 \oplus \mathfrak{h}^n_2,$$

where $\mathfrak{h}_1= \mathrm{span} \{X_1, \dots, X_n, X_{n+1}, \dots, X_{2n}\}$, $\mathfrak{h}_2=\mathrm{span} \{X_{2n+1} \}$ and the only non-trivial bracket-relation between the vector fields $X_1, \dots, X_n, X_{n+1}, \dots, X_{2n}, X_{2n+1}$ is 
\begin{equation*}
	[X_j, X_{n+j}]=X_{2n+1},\quad \forall j=1,\ldots n.
\end{equation*}

Vector fields of $\mathfrak{h}^n_1$ are called \textit{horizontal vector fields}. The horizontal layer $\mathfrak{h}^n_1$ of the algebra is isomorphic to a linear subspace $V$ of $T_e\mathbb{H}^n$.
The disjoint union of $ \{ (d\ell_{q})_e(V),q) \}_{q \in \mathbb{H}^n}$, where $d\ell_q$ is the differential of the left translation in $\mathbb{H}^n$
\begin{equation*}
	\label{translation}
	\ell_q: \mathbb{H}^n \to \mathbb{H}^n,\quad \ell_q(g):= q\cdot g, \ q \in\mathbb{H}^n,
\end{equation*}
is a sub-bundle of the tangent bundle; we call it the \textit{horizontal bundle} and we denote it by $H \mathbb{H}^n$. Since we are dealing with left-invariant vector fields, the fibre of $H\mathbb{H}^n$ at $q \in \mathbb{H}^n$ is generated by the vector fields $\{X_1, \dots, X_n, X_{n+1}, \dots, X_{2n}\}$ evaluated at $q$:
$$H_q \mathbb{H}^n=\mathrm{span} \{ X_1(q), \dots, X_n(q), X_{n+1}(q), \dots, X_{2n}(q) \}.$$

We fix an inner product $\langle \cdot, \cdot \rangle_H$ on the horizontal layer $\mathfrak{h}_1^n$ such that $\{X_1, \dots, X_n, X_{n+1}, \dots, X_{2n}\}$ is an orthonormal basis and we will denote by $|\cdot|_H$ the norm associated with this scalar product. Since $\mathfrak{h}_1^n$ can be identified for any $q \in \mathbb{H}^n$ with $H_q \mathbb{H}^n$, then $\langle \cdot, \cdot \rangle_{H,q}$ is the corresponding inner product on $H_q \mathbb{H}^n$. For notational simplicity we will denote it as $\langle \cdot, \cdot \rangle_H,\forall q\in\mathbb{H}^n$.\\

Given $X\in\mathfrak{h}^n$ and $q_0\in\mathbb{H}^n$, we denote by $\exp(tX)(q_0):=\sigma(t)$, where $\sigma$ is the curve that solves 
\begin{equation*}
	\begin{cases}
		\dot{\sigma}(t)=X(\sigma(t)),\\
		\sigma(0)=q_0.
	\end{cases}
\end{equation*}
This map is well-defined for all  whole $\mathbb{R}$, see \cite{Bonfiglioli} and \cite[Appendix]{Stein3}. Moreover it holds that $$\exp(tX)(q_0)=q_0\exp(tX)(e),\quad \forall q_0\in\mathbb{H}^n, \forall X\in\mathfrak{h}.$$ If $\mathfrak{h}^n\ni X=\sum_{j=1}^{2n+1}x_jX_j$, then the exponential map $\mathrm{exp}: \mathfrak{h}^n\to \mathbb{H}^n$
\begin{equation*}
	\exp(X):=\exp(X)(e).
\end{equation*}
induces a global diffeomorphism between $\mathbb{R}^{2n+1}$ and $\mathbb{H}^n$. Hence, every $q\in\mathbb{H}^n$ can be written in an unique way as
\begin{equation*}
	\label{coordinates}
	q= \mathrm{exp}(x_1 X_1+\dots+x_n X_n+x_{n+1}X_{n+1}+\dots x_{2n}X_{2n}+x_{2n+1}X_{2n+1}),
\end{equation*} 
where $x_i\in\mathbb{R},\forall i=1,\ldots,2n+1$. From now on we will work in this system of coordinates, known as \textit{canonical coordinates of $1^{st}$ kind}:  we will identify any point $q \in \mathbb{H}^n$ with $(x_1,\ldots,x_{2n+1})\in \mathbb{R}^{2n+1}$ and we will write $x=(x_1,\ldots,x_{2n+1})\in\mathbb{H}^n\simeq\mathbb{R}^{2n+1}$. 

In this system of coordinates the group law will be given, through the Baker-Campbell-Hausdorff formula which defines a group structure on $\mathfrak{h}^n$, by  
\begin{equation*}\label{grouplaw}
	x \cdot y := \bigg{(}x_1+y_1,\ldots,x_{2n}+y_{2n}, x_{2n+1}+y_{2n+1}+\frac{1}{2}\sum_{j=1}^{2n} (x_jy_{n+j}-x_{n+j}y_j)\bigg{)}.
\end{equation*}
See Section 2 in \cite{Bonfiglioli}, in particular Theorem 2.2.18, for more details.

Always in coordinates, the unit element $e\in\mathbb{H}^n$ is $0_{\mathbb{R}^{2n+1}}$, the center of the group is 
\begin{equation*}
	L := \{(0,\ldots,0,x_{2n+1}) \in \mathbb{H}^n ;\; x_{2n+1} \in \mathbb{R}\},
\end{equation*}
and, according to the two steps stratification of $\mathfrak{h}^n$, the algebra, and consequently $\mathbb{H}^n$, is endowed with a family of intrinsic non-isotropic dilations that reads as 
\begin{equation*}
	\delta_\lambda((x_1,\ldots,x_{2n+1})):= (\lambda x_1,\ldots,\lambda x_{2n},\lambda^2x_{2n+1}),\quad \forall x\in\mathbb{H}^n, \forall \lambda>0.
\end{equation*} 


The homogeneous dimension of $\mathbb{H}^n$ is 
\begin{equation}\label{homogdim}
	N:=\sum_{i=1}^2i\dim(\mathfrak{h}^n_i)=2n+2.
\end{equation}
The vector fields $X_1,\ldots,X_{2n+1}$, left invariant w.r.t. \eqref{grouplaw}, in coordinates read as
\begin{equation*}
	\begin{cases}
		X_j := \partial_{x_j} -\frac{x_{n+j}}{2} \partial_{x_{2n+1}} \,, j=1,\dots,n,\\
		X_{n+j} := \partial_{x_{n+j}}+\frac{x_j}{2} \partial_{x_{2n+1}},\ \ j=1,\dots,n,\\
		X_{2n+1}=\partial_{x_{2n+1}}.	
	\end{cases}
\end{equation*}
The Lebesgue measure $\mathcal{L}^{2n+1}$ on $\mathbb{R}^{2n+1}$ is the Haar measure of the group, that we shall also denote sometimes by $dx$, on $\mathbb{H}^n\simeq\mathbb{R}^{2n+1}$.

Let $\Omega\subseteq\mathbb{H}^n$ be an open set and $f:\Omega\to\mathbb{R}$ be a measurable function, we denote by $\nabla_Hf=(X_1f,\ldots,X_{2n}f)$ its horizontal derivatives in the sense of distributions. For every $1\leq p\leq\infty$, the space
\begin{equation}\label{horsob}
    HW^{1,p}(\Omega):=\left\{f:\Omega\to\mathbb{R} \text{ measurable : }f\in L^{p}(\Omega),\nabla_Hf\in L^{p}(\Omega)\right\},
\end{equation}
is a Banach space equipped with the norm
\begin{equation*}
    \|f\|_{HW^{1,p}(\Omega)}:=\|f\|_{L^p(\Omega)}+\|\nabla_H f\|_{L^p(\Omega)}.
\end{equation*}
Moreover, for $1\leq p<\infty$, we denote by
\begin{equation*}
    HW_0^{1,p}(\Omega):=\overline{C_0^{\infty}(\Omega)}^{HW^{1,p}(\Omega)}
\end{equation*}
and by
\begin{equation*}
    HW^{-1,p'}(\Omega):=( HW_0^{1,p}(\Omega))'.
\end{equation*}



%Let , $\mathcal{L}^{2n+1}$ is $N$-homogeneous with respect to the dilations, 
%\begin{equation*}
%	\mathcal{L}^{2n+1}(\delta_\lambda(A)) = \lambda^N \mathcal{L}^{2n+1}(A),
%\end{equation*}
%for all Borel set $A$ and all $\lambda>0$, where  is the homogeneous dimension of $\mathbb{H}^n$. If $x\in\mathbb{H}^n$ and $r>0$, we denote by
%\begin{equation*}
%	B_{H}(x,r):=\left\{y\in\mathbb{H}^n: d_{CC}(x,y)\leq r\right\},
%\end{equation*}
%then $B_H(x,r) = x\cdot \delta_\lambda (B_H(0,1))$. Hence, it follows that 
%\begin{equation*} \label{e:measball}
%	\mathcal{L}^{2n+1}(B_H(x,r)) =r^N \underbrace{\mathcal{L}^{2n+1}(B_{H}(0,1))}_{=:c_n} \, ,
%\end{equation*}
%for all $x\in \mathbb{H}^n$ and all $r>0$.


\subsubsection{\textbf{Carnot-Carath\'eodory distance}}
We can equip $\mathbb{H}^n$ with the so called \textit{Carnot-Carath\'eodory distance}, that will make it a polish space.

We say that an absolutely continuous curve $\sigma\in AC([a,b],\mathbb{R}^{2n+1})$, is \textit{horizontal} if its tangent vector $\dot{\sigma}(t)$ belongs to $H_{\sigma(t)}\mathbb{H}^n$ at almost every $t\in[a,b]$ where it exists.
We will denote by
\begin{equation*}
	H([a,b],\mathbb{R}^{2n+1}):=\left\{\sigma\in AC\left([a,b],\mathbb{R}^{2n+1}\right):\sigma\text{ is horizontal}\right\}.
\end{equation*}
Given $\sigma\in H\left([a,b],\mathbb{R}^{2n+1}\right)$, its \textit{horizontal length} is defined as
\begin{equation}\label{horlength}
	l_{H}(\sigma):=\int_a^b|\dot{\sigma}(t)|_H dt,
\end{equation}
% or equivalently
% \begin{equation*}
% 	l_H(\sigma)=\sup\left\{\sum_{i=1}^Md_{CC}(\sigma(t_{i-1}),\sigma(t_i)): 0=t_0<t_1<\ldots<t_M=1\right\}.
% \end{equation*}


The Rashevsky-Chow's theorem guarantees the existence of an horizontal curve connecting any two points of $\mathbb{H}^n$, see \cite{Chow}. Hence, given $x,y\in\mathbb{H}^n$ one can define the \textit{Carnot-Caratheodory distance} (shortly \textit{CC-distance}) between them as follows:
\begin{equation}\label{CCdistance}
	d_{CC}(x,y):= \inf \  \left\{  l_{H}( \sigma) \ | \  \sigma\in H\left([a,b],\mathbb{R}^{2n+1}\right) , \ \sigma(a)=x, \ \sigma(b)=y \right\}.
\end{equation}

Always by the Rashevsky-Chow's theorem the topology induced by this distance on $\mathbb{H}^n$ is globally euclidean. 


%One can prove that the CC-distance is left invariant and 1-homogeneous with respect to the dilations, i.e.
%\begin{equation*}
%	d_{CC}(x \cdot y, x\cdot z) = d_{CC}(y,z) \quad \text{and} \quad d_{CC}(\delta_\lambda(y), \delta_\lambda(z)) = \lambda\,d_{CC}(y,z)
%\end{equation*}
%for all $x$, $y$, $z\in\mathbb{H}^n$ and all $\lambda>0$. 

\subsubsection{\textbf{Minimizing geodesic}}
Following the terminology introduced in \cite{FigalliRifford}, we call \textit{minimizing horizontal curve} any curve $\sigma\in H([a,b],\mathbb{R}^{2n+1})$ such that
\begin{equation*}
	l_{H}(\sigma)=d_{CC}(\sigma(a),\sigma(b)).
\end{equation*}
Up to a reparametrization one can always assume that a minimizing horizontal curve $\sigma:[a,b]\rightarrow\mathbb{R}^{2n+1}$ is parametrized proportionally to arc-length , i.e. $d_{CC} (\sigma(t),\sigma(t'))=|t-t'|v$, where $v=\frac{d_{CC}(\sigma(a),\sigma(b))}{b-a}$ is the constant speed of $\sigma$. We will call \textit{minimizing geodesic} any minimizing horizontal curve parametrized proportionally to the arc-length. By the Hopf-Rinow Theorem, see for instance \cite{FigalliRifford}, it follows that $(\mathbb{H}^n, d_{CC})$ is a geodesic space.

Usually, in this kind of spaces, minimizing geodesics between two points may not be  unique: in the case of the Heisenberg group minimizing geodesics can be computed explicitly and it is possible to detect a set in which these are unique.

We denote by
\begin{equation} \label{KAPPA}
	K:= \{(x,y)\in \mathbb{H}^n\times \mathbb{H}^n;\,\, x^{-1}\cdot y \not \in L\},
\end{equation}
then it holds the following characterization for minimizing geodesics parametrized on $[0,1]$.

\begin{teo} [Characterization of minimizing geodesics] \label{geod} 
	A non trivial $\sigma\in H\left([0,1],\mathbb{R}^{2n+1}\right)$ starting from a generic point $x\in\mathbb{H}^n$ is a minimizing geodesic if and only if
	\begin{equation*}\label{geodform}
		\sigma(t)=\exp\left(t \sum_{j=1}^n\left( \left( \chi_j-\theta x_{n+j}\right) X_j +\left(\chi_{n+j}+\theta x_j\right)X_{n+j}\right) \right)(x),
	\end{equation*}
	for some $\chi \in \mathbb{R}^{2n}\setminus\{0\}$ and $\theta\in [-2\pi,2\pi]$. Moreover one has $|\dot\sigma|_H=|\chi|_E=d_{CC}(x,\sigma(1))$, where $|\cdot|_E$ is the Euclidean norm in $\mathbb{R}^{2n}$.
	
	Moreover:
	\begin{itemize}
		\item For all $(x,y)\in K$, there is a unique minimizing geodesic between $x$ and $y$, as in \eqref{geodform}, for some $\chi \in \mathbb{R}^{2n}\setminus\{0\}$ and some $\theta\in\, (-2\pi,2\pi)$. Moreover one has $|\chi|_E =d_{CC}(x,y)$.
		\item If $(x,y)\not\in K$, $x^{-1}\cdot y = (0,\ldots,0,z_{2n+1})$ for some $z_{2n+1}\in \mathbb{R}\setminus\{0\}$, there are infinitely many minimizing geodesics between $x$ and $y$. These curves are all curves of the form \eqref{geodform} with $\theta=2\pi$, resp. $\theta=-2\pi$, if $z_{2n+1}>0$, resp. if $z_{2n+1}<0$, for all $\chi \in \mathbb{R}^{2n}$ such that $|\chi|_E = \sqrt{4\pi|z_{2n+1}|} $.
	\end{itemize}
\end{teo}
From \cite[Theorem 7.29]{Villani} follows that $(\mathbb{H}^n, d_{CC})$ is a non-branching metric space, any two minimizing geodesics which coincide on a non-trivial interval coincide on the whole intersection of their intervals of definition..

%Let $t_0\in \mathbb{R}$, $x\in\mathbb{H}$ we denote by
%\begin{equation*}
%	\exp^H_{t_0}(\cdot,\cdot)(x):\mathbb{R}^{2n}\times[-2\pi,2\pi]\setminus\{0\}\longrightarrow\mathbb{H}^n
%\end{equation*}
%where
%\begin{equation}\label{SRexp}
%	\exp^H_{t_0}(\chi,\theta)(x):=\exp\left(t_0 \sum_{j=1}^n\left( \left( \chi_j-\frac{\theta}{\tau} x_{n+j}\right) X_j +\left(\chi_{n+j}+\frac{\theta}{\tau} x_j\right)Y_j\right) \right)(x).
%\end{equation}

\begin{deff} 
	We denote by \textnormal{Geo}$(\mathbb{H}^n)$ the space of minimizing geodesics parametrized on $[0,1]$.
\end{deff}

We fix a measurable selection of minimal curves, i.e. a Borel map
\begin{equation*}
	S:\underset{(x,y)\quad\mapsto\quad S(x,y)}{\mathbb{H}^n\times\mathbb{H}^n\rightarrow \textnormal{Geo}(\mathbb{H}^n)}
\end{equation*}
where $S(x,y)$ is a minimizing geodesic joining $x$ and $y$. The existence of such a map follows from general theorems about measurable selections, see e.g. \cite[Section 6.9]{Bogachev}. Moreover $S_{\lfloor K}$ is continuous and for all $(x,y)\in K$. 

If we denote by $e_t$ the evaluation map, i.e. $e_t(\sigma) := \sigma(t)$ for all $\sigma \in C\left([0,1],\mathbb{R}^{2n+1}\right)$,  for $t\in [0,1]$, the map
\begin{equation}\label{geod01}
	S_t:=e_t\circ S:\mathbb{H}^n\times\mathbb{H}^n\rightarrow\mathbb{H}^n,
\end{equation}
associates to any two points $x,y\in\mathbb{H}^n$ the point $S_t(x,y)$ of $\mathbb{H}^n$ at distance $t \, d_{CC}(x,y)$ from $x$ on the selected minimizing geodesic $S(x,y)$ between $x$ and $y$. If we fix $\overline{y}\in\mathbb{H}^n$ and $t\in(0,1)$, then the function $S_t(\cdot,\overline{y})$ is $C^{\infty}$ on $\mathbb{H}^n\setminus(\overline{y}\cdot L)$ and it holds that
\begin{equation}\label{det}
	\det \mathcal{J}_x(S_t(x,\overline{y}))\geq(1-t)^{2n+3}
\end{equation}
for all $x\in\mathbb{H}^n\setminus(\overline{y}\cdot L)$. Moreover
\begin{equation}\label{MCP}
	\mathcal{L}^{2n+1}(E) \leq \frac{1}{(1-t)^{2n+3}} \,\mathcal{L}^{2n+1}(S_t(E,y))
\end{equation}
for any $y\in\mathbb{H}^n$ and $E\subset \mathbb{H}^n$, which roughly means that $(\mathbb{H}^n,d_{CC},\mathcal{L}^{2n+1})$ satisfies a so-called Measure Contraction Property.  For proofs of these results see \cite[Section~2]{Juillet}. Instead of the exponent $2n+3$ one would expect the topological dimension $2n+1$, as in the Euclidean setting, or at most the homogeneous dimension $N=2n+2$. However Juillet, in the \cite[Remark 2.3]{Juillet}, shows that this exponent is sharp.

%how this exponent arises for the unit ball $B_H(0,1)$ in $\mathbb{H}^1$.
