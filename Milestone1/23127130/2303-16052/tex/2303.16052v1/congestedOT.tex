

\section{Weighted distance induced by a $L^p$ density}

In the previous section we introduced the notion of transport density in $\mathbb{H}^n$ and we provided assumptions to ensure that it is in $L^p$. A measure with continuous density naturally induces a weighted length of curves which leads to the definition of control distance. Scope of this section is to extend the definition of this kind of metric to $L^p$ densities. The statement of the results are  apparently similar to the analogous ones in the Euclidean setting (contained in \cite{Santambrogio1}), but the proofs are totally different, due to the geometric properties of the space. 


\subsection{Distance induced by a $p$-summable weight}
Through this section, we suppose that $\Omega\subset\mathbb{H}^{n}$ is an open, bounded set and $\mu,\nu\in\mathcal{P}(\overline{\Omega})$. We denote by 
\begin{equation}\label{acca}
	H:=\big{\{}\sigma\in AC([0,1],\overline{\Omega}):\ \sigma\ \textnormal{is\ horizontal}\big{\}}	 
\end{equation}
the set of horizontal curves on $\overline{\Omega}$ parametrized on $[0,1]$,
viewed as subset of $C([0,1],\overline{\Omega})$ equipped with the topology of uniform convergence.

Let us first recall the definition of weighted length in the regular setting. Let $\phi \in C(\overline{\Omega},\mathbb{R}_+)$ and $\sigma\in C([0,1],\overline{\Omega})$, we call 
\begin{multline*}\label{LPHI}
	L_{\phi}(\sigma):=\sup\bigg{\{}\sum_{i=1}^n \left(\inf_{[t_i,t_{i+1}]}(\phi\circ\sigma)\right)d_{CC}(\sigma(t_i), \sigma(t_{i+1})):\\ 
	([t_i, t_{i+1}])_i \mbox{ is a partition of }[0,1]\bigg{\}}. 
\end{multline*}
As it is well known, see for example \cite[Lemma 2.7]{Santambrogio1}, the function  $\sigma\mapsto L_{\phi}(\sigma)$ is l.s.c., hence Borel, on $C([0,1],\overline{\Omega})$ w.r.t. the uniform convergence. If $\phi\in C(\overline{\Omega},\mathbb{R})$, let us write $\phi=\phi_+-\phi_-$, where $\phi_+$ and $\phi_-$ are the positive and negative part of $\phi$ respectively, hence $L_\phi:=L_{\phi_+}-L_{\phi_-}$ is Borel.
In particular if $\sigma\in H$ and $\phi\in C(\overline{\Omega})$, then $L_{\phi}(\sigma)$ is the  length of the curve $\sigma$ with respect to the weight $\phi$, 
and can be expressed as
\begin{equation}\label{length}
	L_{\phi}(\sigma)=\int_0^1 \phi(\sigma(t))|\dot{\sigma}(t)|_H dt.
\end{equation} 
Moreover it follows that 
\begin{equation}\label{boundlphi}
	0\leq |L_{\phi}(\sigma)|\leq\|\phi\|_{\infty}l_H(\sigma).
\end{equation} 



%%%%%%%%%%%%%%


%%%%%%%%%%%%%


If $\phi\in C(\overline{\Omega},\mathbb{R}_+)$, $\forall x,y\in\overline{\Omega}$ we denote by
\begin{equation}\label{ccontinuous}
	c_\phi(x,y):=\inf\{L_{\phi}(\sigma)\,:\,\sigma\in H^{x,y}\},
\end{equation}
where
    \begin{equation}\label{horcrvxy}
        H^{x, y}:=\{\sigma\in H:\ \sigma(0)=x,\ \sigma(1)=y\}.
    \end{equation}

Let us explicitly recall that the cost function $c_\phi(x,y)$
is a distance, if $\phi$ is continuous and  strictly positive, and it is only a pseudo distance, if $\phi$ is non-negative. 
In order to extend this pseudo distance to summable functions  $\phi$, we start with an estimate of the regularity of $c_\phi(x,y)$ in terms of the $L^{p'}$ norm of $\phi$, 
where
$p':=\frac{p}{p-1}$ is the conjugate exponent of some $p\in(1,+\infty)$. 
 The proof is inspired by \cite[Proposition 3.2]{Santambrogio1} but requires many non trivial changes, due to the geometric structure of $\mathbb{H}^n$. 
\begin{prop}\label{cxicomp}
	If $p'>N$, then there exists $C>0$ such that for every $\phi\in C(\overline{\Omega},\mathbb{R}_+)$ and every $(x,y), (x',y')\in\Omega\times\Omega$, one has:
	\begin{equation*}\label{holderest}
		\vert c_{\phi}(x,y)-c_{\phi}(x',y')\vert \leq C\Vert \phi\Vert_{L^{p'}(\Omega)} \left(d_{CC}(x,x')^{\alpha}+ d_{CC}(y,y')^{\alpha} \right),
	\end{equation*}
	where $\alpha:=1-\frac{N}{p'}$.
	Moreover, if $(\phi_n)_{n\in\mathbb{N}}\subset  C(\overline{\Omega},\mathbb{R}_+)$ is bounded in $L^{p'}$, then $(c_{\phi_n})_{n\in\mathbb{N}}$ admits a sub-sequence that converges in $C(\overline{\Omega}\times\overline{\Omega},\mathbb{R}_+)$. 
\end{prop}
\begin{proof}
   
	Let $\phi\in C(\overline{\Omega},\mathbb{R}_+)$ and $x,y\in\Omega$. For $k>0$ let $\sigma_k\in H^{x,y}$ be such that
	\begin{equation*}
		\int_0^1 \phi(\sigma_k(t))\vert \dot \sigma_k(t)\vert_H dt\leq c_{\phi}(x,y)+\frac{1}{k}.
	\end{equation*}
 In order to study the regularity of $c_\phi$ with respect to the second variable $y$, we choose a point $z_\varepsilon$ 
 which can be connected to $y$ by an horizontal segment. Indeed, we fix  a constant coefficient unitary horizontal vector field i.e a vector field
$ \sum_{j=1}^na_jX_j+b_jX_{n+j}$
such that $(a_1,\ldots,a_n,b_1,\ldots,b_n)\in\mathbb{R}^{2n}$ and $\vert(a_1,\ldots,a_n,b_1,\ldots,b_n)\vert_E=1$, and we choose for all $\varepsilon>0$ the points 
	\begin{equation*}
		z_\varepsilon:=\exp\bigg{(}{\varepsilon\bigg{(}\sum_{j=1}^na_jX_j+b_jX_{n+j}}\bigg{)}\bigg{)}(y),
	\end{equation*}
such that $z_\varepsilon\in \Omega$. Now we modify the curve $\sigma_k$ into a curve  $\sigma_{k,t_0}\in H^{x,z_\varepsilon}$: we choose $t_0\in (0,1)$ and define
	\begin{equation*}
		\sigma_{k,t_0}(t):=
		\begin{cases}
			\sigma_k\big{(}{t\over t_0}\big{)} &\mbox{ if }t\in[0,t_0]\\  
			\tilde{\sigma}_{\varepsilon,y}\big{(}\frac{t-t_0}{1-t_0}\big{)} &\mbox{ if }t\in]t_0,1],
		\end{cases}
	\end{equation*}
	where
	\begin{equation*}
		\tilde{\sigma}_{\varepsilon, y}(t)=\exp\bigg{(}t\varepsilon\bigg{(}\sum_{j=1}^na_jX_j+b_jX_{n+j}\bigg{)}\bigg{)}(y),\quad t\in[0,1].
	\end{equation*}
	We then have, for all $k>0$
	\begin{align*}
		c_\phi(x,z_\varepsilon)&\leq \int_0^1\phi(\sigma_{k,t_0}(t))\vert\dot\sigma_{k,t_0}(t)\vert_H dt=\\
		&=\int_0^1\phi(\sigma_k(t))\vert \dot\sigma_k(t)\vert_H dt+\int_0^1\phi(\tilde{\sigma}_{\varepsilon, y}(t))\vert \dot{\tilde{\sigma}}_{\varepsilon, y}(t)\vert_H dt\leq\\
		&\leq c_\phi(x,y)+\frac{1}{k}+\varepsilon\int_0^1 \phi(\tilde{\sigma}_{\varepsilon, y}(t))dt.
	\end{align*}
	Now, if $k\rightarrow+\infty$, we get
	\begin{equation*}
		\frac{1}{\varepsilon}\bigg{[}c_{\phi}\bigg{(}x,\exp\bigg{(}{\varepsilon\bigg{(}\sum_{j=1}^na_jX_j+b_jX_{n+j}}\bigg{)}\bigg{)}(y)\bigg{)}-c_{\phi}(x,y)\bigg{]}\leq \int_0^1 \phi(\tilde{\sigma}_{\varepsilon, y}(t))dt,
	\end{equation*}
	and, by similar argument:
	\begin{equation*}
		\frac{1}{\varepsilon}\bigg{[}c_{\phi}(x,y)-c_{\phi}\bigg{(}x,\exp\bigg{(}{\varepsilon\bigg{(}\sum_{j=1}^na_jX_j+b_jX_{n+j}}\bigg{)}\bigg{)}(y)\bigg{)}\bigg{]}\leq \int_0^1 \phi(\tilde{\sigma}_{\varepsilon, y}(1-t))dt.
	\end{equation*}

Integrating with respect to $y$, raising to the power $p'$ and using the fact that the function $y\mapsto\tilde{\sigma}_{\varepsilon, y}(t) $ has Jacobian determinant $1$, 
this implies that $c_{\phi}(x,\cdot)\in HW^{1,p'}(\Omega)$, see \eqref{horsob}, and
	\begin{equation}\label{EST}
		\|\nabla_Hc_{\phi}(x,\cdot)\|_{p'}\leq\|\phi\|_{p'},\quad \forall x\in\Omega.
	\end{equation}
	By symmetry we also get that
	\begin{equation}\label{ESTx}
		\|\nabla_Hc_{\phi}(\cdot,y)\|_{p'}\leq\|\phi\|_{p'},\quad \forall y\in\Omega.	
	\end{equation}
	Since $p'>N$ then if follows by \eqref{EST}, \eqref{ESTx} and Morrey's Theorem (see \cite{Capogna}, Chapter 5), that there exists $C>0$ such that
	\begin{align*}
		\vert c_{\phi}(x,y)-c_{\phi}(x,y')\vert \leq C\Vert \phi\Vert_{L^{p'}}  d_{CC}(y,y')^{\alpha},\quad  \forall x, y, y'\in\Omega,\\
		\vert c_{\phi}(x,y)-c_{\phi}(x',y)\vert \leq C\Vert \phi\Vert_{L^{p'}}  d_{CC}(x,x')^{\alpha},\quad   \forall x, x', y\in\Omega.	
	\end{align*}
	This proves \eqref{holderest}. The second claim in the proposition then follows from \eqref{holderest}, the identity $c_{\phi_n}(x,x)=0$ and Ascoli-Arzelà's theorem. 
\end{proof}

From now on, we further assume that $p<\frac{N}{N-1}$. The next goal is to give an equivalent definition for $c_\phi$, that extends the notion for functions just in $L^{p'}$.

\begin{prop}\label{ccoincid}
	If $\phi\in C(\Omega,\mathbb{R}_+)$, then
\begin{equation}\label{barcphi}
	c_\phi(x,y)=\sup\left\{c(x,y)\,:c\in\mathcal{C}(\phi)\right\},
\end{equation}
where 
\begin{equation}\label{Setcphi}
\mathcal{C}(\phi)=\left\{c=\lim_{n\rightarrow+\infty} c_{\phi_n}\,\mbox{ in }C(\overline{\Omega}\times\overline{\Omega})\,:\,(\phi_n)_{n\in\mathbb{N}}\subset C(\overline{\Omega}),\,\phi_n\geq 0,\,
\phi_n\to\phi\,\mbox{ in }L^{p'}\right\}.
\end{equation}
\end{prop}

We first state two technical remarks that will be useful in the proof.

 
\begin{Remark}
Note that if we have a constant coefficient unitary 
horizontal vector 
 $W_1:=a_1X_1+\ldots +a_nX_n+a_{n+1}X_{n+1}+\ldots+a_{2n}X_{2n}\in\mathfrak{h}_1^1$, 
 it is  possible to perform a change of variable which sends the 
 vector $W_1$ to the first element of the canonical orthonormal basis. 
 Indeed, if we  denote by $W_2,\ldots,W_{2n}$ a basis of orthogonal complement $W_1^\perp$ in $\mathfrak{h}_1^1$ with respect to $\left\langle\cdot,\cdot\right\rangle_H$, and by $x$ a point, we can consider the change of variable
\begin{equation}\label{changeofcoordinates}
	\Psi:\mathbb{R}^{2n+1}\to\mathbb{H}^{n},
\quad \Psi(e_1,\ldots,e_{2n+1})=\exp(e_1W_1)\exp\left(\sum_{i=2}^{2n} e_{i}W_{i}+e_{2n+1}X_{2n+1}\right)(x).
\end{equation}
 In this system of coordinates the vector field $W_1$ reads as $d \Psi(W_1)=\partial_{e_1},$ and the point $x$ will be the origin in the new  coordinate system.
\end{Remark}

\begin{Remark} 
First we note that any function $c\in \mathcal{C}(\phi)$ satisfies the triangular inequality. Given a continuous function $\phi$, and three points $x^0,x^1,x^2$, it follows that 
	\begin{align*}
		c_{\phi}(x_0,x_2)&\leq \inf\{L_{\phi}(\sigma_1) + L_{\phi}(\sigma_2):\sigma_1\in H^{x^0,x^1}, \sigma_2\in H^{x^1,x^2}\}\\
		&\leq\sum_{i=1}^2\inf\{L_{\phi}(\sigma):\sigma\in H^{x^{i-1},x^i}\}=\sum_{i=1}^2c_\phi(x^{i-1},x^i).
	\end{align*}
	Hence, passing to the limit in the definition of $c$ we obtain
	\begin{equation}\label{ineq}
		c(x,y)=\lim_{n\to+\infty}c_{\phi_n}(x,y)\leq\lim_{n\to+\infty}\left( \sum_{i=1}^2c_{\phi_n}(x^{i-1},x^i)\right)=\sum_{i=1}^2c(x^{i-1},x^i).
	\end{equation}
\end{Remark}

\begin{proof}[{\it Proof of Proposition \ref{ccoincid}}]
To simplify notations we call
$$\overline{c}_{\phi}= \sup\left\{c(x,y)\,:c\in\mathcal{C}(\phi)\right\},$$
so that we have to prove that $ \overline{c}_{\phi} = c_\phi.$
	First we consider the constant sequence $\phi_n:=\phi,\quad\forall n\in\mathbb{N}$. Then $c_\phi\in\mathcal{C}(\phi)$ and we get that $\overline{c}_{\phi}\geq c_\phi$.
	
	Let us prove the converse inequality. 
	Let $\bar{x},\bar{y}\in\Omega$, $k>0$ and $\sigma\in H^{\bar{x},\bar{y}}$ such that $L_{\phi}(\sigma)<c_\phi(\bar{x},\bar{y})+1/k$. 
 %The fact that $\sigma$ is horizontal means that there exist measurable functions $h_j:[0,1]\to\mathbb{R}$ such that
%	\begin{equation*}
%		\dot\sigma(t)=\sum_{j=1}^{2n}h_j(t)X_j(\sigma(t)).
%	\end{equation*} 
 %a functions $h_j$'s are simple and hence
%	\begin{align*}
%		\dot\sigma(t)=\sum_{j=1}^{2n}\left(\sum_{i=1}^{M}\mathbbm{1}_{[t_{i-1},t_i]}(t)a_{i,j}\right)X_j(\sigma(t)).
%	\end{align*}
	Let us fix a sequence $\phi_n\rightarrow\phi$ in $L^{p'}$ such that $c_{\phi_n}$ converges uniformly to some $c$, we want to prove that $c\leq c_{\phi}$.
  From density of simple functions and continuity of $\phi$ we can assume that there exists 
 a finite decomposition $\{t_0, t_1, \cdots t_M\}$ of the interval $[0,1]$ such that  $\dot\sigma$ is constant and horizontal on the interval $[t_{i-1},t_i]$; in particular
	\begin{equation*}
		L_{\phi_n}(\sigma)=\sum_{i=1}^{M}\int_{t_{i-1}}^{t_i}\phi_n(\sigma(t))|\dot\sigma(t)|_Hdt.
	\end{equation*} 
Let us consider a single interval $[t_{i-1}, t_i]$: up to  a change of coordinates, we can also assume that $|\dot\sigma|_H=1$ on this interval. For this reason, in the change of coordinates  $\Psi_i:\mathbb{R}^{2n+1}\rightarrow\mathbb{H}^{n}$, introduced in \eqref{changeofcoordinates}, we can choose 
$\Phi_i(\sigma(t_{i-1}))=(t_{i-1},0)$ so that  
$\Phi_i(\sigma(t_{i}))=(t_{i},0)$ ,
 %where $W^i_1:=\sum_{j=1}^{2n}a_{i,j}X_j(\sigma(t))$ and we denote by $x^0=\bar{x},x^N=\bar{y}, x^{i}=\sigma(t_i),x^{i-1}=\sigma(t_{i-1})$. Let us fix $i$: we will denote by $W^i_2,\ldots,W_{2n}^i$ a basis of orthogonal complement $(W_1^i)^\perp$ in $\mathfrak{h}_1^1$ with respect to $\left\langle\cdot,\cdot\right\rangle_H$.  Using the change of coordinates. 
and $$\Phi_i \circ \sigma : [t_{i-1}, t_{i}] \to\mathbb{R}^{2n+1}, \quad (\Phi_i \circ \sigma)(t) = (t, 0).$$

We now consider, for every $\delta>0$ and for every $i$, cylindrical neighborhoods $C_{i, \delta} = \{(t, \hat e): t \in [t_{i-1}, t_i], |(0,\hat e)|_H \leq \delta\},$ of the curve $\Phi_i \circ \sigma$, with basis
$S_{i-1} =\{(t_{i-1}, \hat e): |(0,\hat e)|_H \leq \delta\} .$
For every $\hat e\in \mathbb{R}^{2n},$ with  $|(0,\hat e)|_H\leq \delta$, we call  $\sigma_e (t) =\Psi_i(t, \hat e)$. By definition 
$$
c_{\phi_n}\Big(\Psi_i(t_{i-1}, \hat e) , \Psi_i(t_i, \hat e)\Big)\leq 
L_{\phi_n}(\sigma_e \circ \theta_i),
$$
where $\theta_i$ is a  change of coordinate which sends $[0,1]$ to $[t_{i-1}, t_{i}]$. 
Note that 
\begin{equation}\label{questa}
L_{\phi_n}(\sigma_e \circ \theta_i) =  L_{\phi_n\circ\Psi_i }( \Phi_i \circ \sigma_e \circ \theta_i) = \int_{t_{i-1}} ^{t_i}  ( \phi_n\circ\Psi_i)(t, \hat e) dt.
\end{equation}
Hence, integrating on $S_{i-1}$  we get
\begin{equation}\label{cpne}
	\int_{S_{i-1}}c_{\phi_n}\Big(\Psi_i(t_{i-1}, \hat e), \Psi_i(t_i, \hat e)\Big) d\mathcal{H}^{2n}(\hat e)\leq \int_{S_{i-1}} \int_{t_{i-1}}^{t_i} ( \phi_n\circ\Psi_i)(t, \hat e) dt d\mathcal{H}^{2n}(\hat e).
\end{equation}
For $ n \to \infty$ using the uniform convergence of $c_{\phi_n}$ to $c$ and the $L^{p'}$ convergence of $\phi_n$ to $\phi$ we get
that

	\begin{equation*}
		\int_{S_{i-1}} c\Big(\Psi_i(t_{i-1}, \hat e), \Psi_i(t_i, \hat e)\Big))d\mathcal{H}^{2n}(\hat e)\leq\int_{C_i}(\phi\circ\Psi_i) (t, \hat e) d\mathcal{L}^{2n+1} (t, \hat e).
  	\end{equation*}

	Now we divide by the measure of $S_{i-1}$  and pass to the limit as $\delta\rightarrow0^+$.
	Using the fact that $c$ is continuous
	\begin{equation*}
		\lim_{\delta\to0^+}\frac{1}{d\mathcal{H}^{2n}(S_{i-1})}\int_{S_{i-1}} c\Big(\Psi_i(t_{i-1}, \hat e), \Psi_i(t_i, \hat e)\Big))d\mathcal{H}^{2n}(\hat e)=c\Big(\Psi_i(t_{i-1}, 0), \Psi_i(t_i, 0)\Big) = c(x^{i-1},x^i),
	\end{equation*}
where  $x^i = \sigma(t_i)$, 
Analogously  the integral over $C_i =[t_{i-1}, t_i] \times S_{i-1}$ divided by the measure of $S_{i-1} $ converges to the integral on $[t_{i-1}, t_i] $, which is the integral along the curve $\Phi_i\circ\sigma(t)$
	\begin{equation*}
		\lim_{\delta\to 0^+}\frac{1}{d\mathcal{H}^{2n}(S_{i-1})}\int_{C_i}(\phi\circ\Psi_i)(t, \hat e) d\mathcal{L}^{2n+1}(t, \hat e)= \int_{t_{i-1}}^{t_i}
  (\phi\circ\Psi_i)(t, 0) dt = 
\int_{t_{i-1}}^{t_i}  \phi(\sigma(t))|\dot\sigma(t)|_Hdt.
	\end{equation*}
	Then, using \eqref{cpne}, we get that 
	\begin{equation*}
 c(x^{i-1},x^{i}) \leq \int_{t_{i-1}}^{t_i}\phi(\sigma(t))|\dot\sigma(t)|_Hdt,\quad \forall i=1,\ldots,M,
	\end{equation*}
	and then, from \eqref{ineq},
	\begin{equation*}
 c(\bar{x},\bar{y})\leq
 \sum_{i=1}^{M}c(x^{i-1},x^{i}) 
		\leq\sum_{i=1}^{M}\int_{t_{i-1}}^{t_i}\phi(\sigma(t))|\dot\sigma(t)|_Hdt=L_{\phi}(\sigma).
	\end{equation*}
	This gives 
	\begin{equation*}
		c(\bar{x},\bar{y})\leq c_{\phi}(\bar{x},\bar{y})+\frac{1}{k}
	\end{equation*}
	for the choice of $\sigma$ and, since $k$ is arbitrary, it follows that $c(\bar{x},\bar{y})\leq c_{\phi}(\bar{x},\bar{y})$.
\end{proof}


Since  definition \eqref{barcphi},  makes sense also for $L^{p'}$ functions, and extends \eqref{ccontinuous}, we will use it as definition of $c_\phi$
for any non-negative function $\phi\in L^{p'}(\Omega)$: 

\begin{deff}
    If $\phi\in L^{p'}(\Omega)$, then we define
\begin{equation}\label{barcphi}
	c_\phi(x,y)=\sup\left\{c(x,y)\,:c\in\mathcal{C}(\phi)\right\},
\end{equation}
where $\mathcal{C}(\phi)$ has been defined in 
\ref{Setcphi}. 
\end{deff}
 
Let us finally note that the function $c_\phi$ is a pseudo distance. Indeed the properties of the pseudo distance can be obtained passing  to the limit in the inequalities satisfied by $c_{\phi_n}$. In particular 
 the triangular inequality follows from \eqref{ineq}.



\section{Congested Optimal Transport in $\mathbb{H}^n$}
Starting from a transport plan $\gamma$ we defined a transport density
$a_\gamma$, and an associated distance $c_{\phi}$, with $\phi = a_\gamma$. Hence it is possible to study Monge-Kantorovich problem associated to this distance
\begin{equation}\inf_{\gamma\in\Pi(\mu,\nu)}\int_{\overline{\Omega}\times\overline{\Omega}}c_{a_\gamma}(x,y)d\gamma(x,y).
	\end{equation}
This means that a transport plan induces a metric, and a metric induces an optimal transport plan. However, this optimal transport plan will not coincide with the starting one, and will change again the  metric of the space. In order to be able to reach an equilibrium, in which the transport plan remains stable under this minimization, we need to introduce the more abstract notion of \textit{traffic plan}, which depends on  all possible paths, and use  the  metric associated to its \textit{traffic intensity} in the Monge-Kantorovich problem. This will lead to the notion of \textit{Wardrop equilibrium}.

Hence, scope of this section is to adapt to the Heisenberg Group setting the notion of congested optimal transport, and its equilibria, proposed in \cite{Santambrogio1} by Carlier et al. We suppose that $\Omega\subset\mathbb{H}^{n}$ is an open, bounded and geodesically convex set, that models the cortical (or geographical) area on which the dynamic takes place and $\mu,\nu\in\mathcal{P}(\overline{\Omega})$ represent the initial and  final  cortical activity (or distributions of  vehicles and their destinations), respectively.





\subsection{Horizontal traffic plans and traffic intensity}
We now introduce a probability measure $Q$ on the set of continuous curves $C([0,1],\overline{\Omega})$, concentrated on the set $H$ of horizontal paths, defined in \eqref{acca}. According to the notation introduced in \cite{Morel} for the Euclidean setting, we will call \textit{horizontal traffic plan} a probability measure $Q\in\mathcal{P}([0,1],\overline{\Omega})$ such that $Q(H)=1$ and
\begin{equation}\label{avlength}
	\int_{C([0,1],\overline{\Omega})}l_H(\sigma)dQ(\sigma)<+\infty.
\end{equation}
Such measures are introduced to take into account the structure of the space in which we will work. We say that an horizontal traffic plan $Q$ is \textit{admissible} between the measures $\mu$ and $\nu$ if $(e_0)_{\#}Q=\mu$ and $(e_1)_{\#}Q=\nu$, where $e_0$ and $e_1$ are the evaluation maps at times $t=0$ and $t=1$. We denote by 
$$\mathcal{Q}_{\mathbb{H}}(\mu,\nu):=\{\text{admissible horizontal traffic plans between $\mu$ and $\nu$}\};$$ 
this set is not empty, indeed if $\gamma\in\Pi(\mu,\nu)$, we may define \begin{equation}\label{traffpl}
    Q_\gamma:=\int_{\overline{\Omega}\times\overline{\Omega}}\delta_{S(x,y)}d\gamma(x,y)\in\mathcal{Q}_{\mathbb{H}}(\mu,\nu).
\end{equation}


 One can associate to any $Q\in\mathcal{Q}_{\mathbb{H}}(\mu,\nu)$ a positive and finite Borel measure $i_{Q}\in\mathcal{M}_+(\overline{\Omega})$ that we will call \textit{horizontal traffic intensity}, representing the transiting mass associated with such horizontal traffic plan and defined by duality as
\begin{align}\label{defiQ}
    \int_{\overline{\Omega}} \phi(x) di_{Q}(x):=\int_{C([0,1], \overline{\Omega})} L_{\phi}(\sigma) d Q(\sigma),\quad \forall \phi \in C(\overline{\Omega}).
\end{align}
Note that the right hand side is well defined. Indeed, from \eqref{boundlphi} and \eqref{avlength} it follows that $L_{\phi}\in L^1(C([0,1],\overline{\Omega}),Q)$, for every $Q\in\mathcal{Q}_{\mathbb{H}}(\mu,\nu)$.

If we look at the action of $i_{Q}$ on a set $A$, then $i_{Q}(A)$ 
represents the total cumulated traffic in $A$ induced by $Q$.
The notion of horizontal traffic intensity is a path-dependent version of horizontal transport density introduced in Section \ref{sectrandens}. Namely, let $\gamma\in\Pi_1(\mu,\nu)$ and let $Q_\gamma$ be as in \eqref{traffpl}, then
\begin{equation}\label{transporttraffic}
i_{Q_\gamma}=a_\gamma.
\end{equation}


Let us denote by
\begin{equation}
	\mathcal{Q}_{\mathbb{H}}^p(\mu,\nu):=\left\{Q\in\mathcal{Q}_{\mathbb{H}}(\mu,\nu):i_{Q}\in L^p(\Omega)\right\}.
\end{equation}
From now on we further assume that $\mathcal{Q}_{\mathbb{H}}^p(\mu,\nu)\not=\emptyset$. In general this could be not true, but  if $\mu,\nu\in L^p$, Theorem \ref{summability2} and \eqref{transporttraffic} imply the existence of a traffic plan whose associated traffic intensity is a $L^p$ function. 


Using the same argument as in \cite[Section 3.2]{Santambrogio1} one can 
can extend for $\phi\geq 0, \phi\in L^{p'}(\Omega)$ the notion of weighted length  $L_{\phi}$,  and prove the following result.

\begin{teo}\label{prolLxi} Let $Q\in \mathcal{Q}^p_{\mathbb{H}}(\mu,\nu)$, $\phi\in L^{p'},\phi\geq0$, $p<\frac{N}{N-1}$ and $(\phi_n)_{n\in\mathbb{N}}\subset C(\overline{\Omega})$, $\phi_n\geq0$ $\forall n\in\mathbb{N}$, $\phi_n\rightarrow\phi$ in $L^{p'}$, then:
	\begin{enumerate}
		\item[(i)] $(L_{\phi_n})_{n\in\mathbb{N}}$ converges strongly in $L^1(C([0,1],\overline{\Omega}),Q)$ to some limit, independent of the  approximating sequence $(\phi_n)_{n\in\mathbb{N}}$. This limit will be denoted by $L_\phi$. 
		\item[(ii)] The following equality holds:
		\begin{equation}\label{eglc}
			\int_{\Omega} \phi(x) i_{Q}(x)\ dx=\int_{C([0,1],\overline{\Omega})} L_{\phi}(\sigma)\ dQ(\sigma).
		\end{equation}
		\item[(iii)] The following inequality holds for $Q$-a.e. $\sigma\in H$:
		\begin{equation}\label{ineglc}
			L_{\phi}(\sigma)\geq c_{\phi}(\sigma(0),\sigma(1)),
		\end{equation}
	\end{enumerate}
where $c_{\phi}$ is defined in  \eqref{barcphi}.
\end{teo}

\subsection{Wardrop equlibria in $\mathbb{H}^n$}

The congestion effects are captured by a metric associated with $Q$. We consider a \textit{congestion function} $g:\Omega\times\mathbb{R}_+\rightarrow\mathbb{R}_+$, which is continuous and such that
\begin{enumerate}
	\item $g(x,\cdot):\mathbb{R}_+\rightarrow\mathbb{R}_+$ is strictly increasing $\forall x\in\overline{\Omega}$;
	\item $\lim_{i\to\infty}g(x,i)=+\infty, \forall x\in\overline{\Omega}$;
	\item $g(x,0)=c, \forall x\in\overline{\Omega}$, for some $c\in\mathbb{R}, c>0$.
\end{enumerate}

The quantity $g(x,i)$ can be seen as the cost to be paid for passing through $x$ with an amount of traffic $i$. This partially justifies the assumptions on the cost function: the fact that $g(x,\cdot)$ is an increasing function of $i$ means that the more traffic there is, the highest is the cost to reach the destination 
; the assumption $\lim_{i\to\infty}g(x,i)=+\infty$ models the fact that, if there is too much traffic, it is impossible to reach the destination, and $g(x,0)>0$ because the cost is positive even if there is no traffic (in the case of vehicles for instance fuel cost, wear costs ecc.).


Given $Q\in\mathcal{Q}_{\mathbb{H}}(\mu,\nu)$, we denote by
\begin{equation}\label{congfun}
	\phi_{Q}(x):=\begin{cases}
		g(x,i_{Q}(x)),\quad \text{if }i_{Q}\ll\mathcal{L}^{2n+1},\\
		+\infty,\quad \text{otherwise},
	\end{cases}
\end{equation}
where, with abuse of notation, $i_Q(x)$ is the density of the measure $i_Q$ with respect to the Lebesgue measure.
The existence of at least one $Q$ such that $i_{Q}\ll\mathcal{L}^{2n+1}$ depends on $\mu$ and $\nu$. For instance if either $\mu\ll\mathcal{L}^{2n+1}$ or $\nu\ll\mathcal{L}^{2n+1}$, the existence of such a $Q$ follow from  \eqref{transporttraffic} and Theorem \ref{absolutecont}. Let $Q\in\mathcal{Q}(\mu,\nu)$ such that $i_Q\ll\mathcal{L}^{2n+1}$, hence the quantity   
\begin{equation}\label{vehicles}
	\int_{\overline{\Omega} }\phi_{Q}(x)i_{Q}(x)dx
\end{equation}
is well defined and 
represents the total cost paid for commuting between $\mu$ and $\nu$, corresponding to the traffic assignment $Q$. We can also express the transport problem in terms of transport plans. Let $H^{x,y}$ the set defined in \eqref{horcrvxy}
    then any transport from $x$ to $y$ that is performed along a path $\sigma\in H^{x, y}$, pays a cost 
    \begin{equation*}
		L_{\phi_{Q}}(\sigma)=\int_0^1 g\big{(}\sigma(t),i_{Q}(\sigma(t))\big{)}\vert\dot{\sigma}(t)\vert_Hdt.
	\end{equation*}
	An optimal transportation problem with traffic congestion, analogous to the classical Monge-Kantorovich problem, can be stated as 
	\begin{equation}\label{condition2}\inf_{\gamma\in\Pi(\mu,\nu)}\int_{\overline{\Omega}\times\overline{\Omega}}c_{\phi_{Q}}(x,y)d\gamma(x,y),
	\end{equation}
    
    where the distance function has been replaced by the  cost 
    $c_{\phi_{Q}}$ defined in the previous section. Clearly the infimum depends on the choice of $Q$. Moreover, the fact that  $ Q \in \mathcal{Q}_{\mathbb{H}}(\mu,\nu)$, implies that $\gamma_{Q} := (e_0, e_1)_{\#}Q \in\Pi(\mu,\nu)$, but it will be in general different from the minimizer. However a change in all the transport plans can induce to change  the traffic plan. In a urban scenario this change can be imposed by an external authority,  while it takes place naturally in living systems, as in the visual cortex, which we choose as a motivating example for this study. Here the traffic plan is represented by the cortical connectivity and its strength represents its traffic intensity. Within a fixed architecture, the signal tends to choose the best transport plan, minimizing \eqref{condition2}. However, due to cortical plasticity, a learning mechanism is able to change the structure of the connectivity network, in order to optimize the propagation, leading to a changement of the traffic plan. Hence 
    we are interested in horizontal traffic plans $Q$ such that the minimum in this problem is exactly the transport plan $\gamma_{Q}$. 
This is the horizontal version of the notion of Wardrop equilibrium provided in \cite{Santambrogio1}:

\begin{deff}\label{Wardrop}
	A Wardrop equilibrium is an horizontal traffic plan $Q\in\mathcal{Q}_{\mathbb{H}}(\mu,\nu)$ such that
	\begin{enumerate}
		\item $Q\big{(}\left\{\sigma\in H: L_{\phi_Q}(\sigma)=c_{\phi_{Q}}(\sigma(0),\sigma(1))\right\}\big{)}=1$;
		\item $\gamma_{Q}:=(e_0,e_1)_{\#}Q\in\Pi(\mu,\nu)$ solves the Monge-Kantorovich problem
		\begin{equation*}
			\inf_{\gamma\in\Pi(\mu,\nu)}\int_{\overline{\Omega}\times\overline{\Omega}}c_{\phi_{Q}}(x,y)d\gamma(x,y).
		\end{equation*}
	\end{enumerate}
\end{deff}

    
    
%	Obviously the metric above is well-posed only if $i_{Q}\ll\mathcal{L}^{2n+1}$ and $\phi$ is continuous (or at least lower semicontinuous). The goal of the next section is to prove that the metric can be rigorously defined also when $g\left( x,i_{Q}(\cdot)\right)$ is an $L^{p'}$ function for every $x\in\overline{\Omega}$, where $p'>N$ and $N$ is the homogeneous dimension of $\mathbb{H}^n$ defined in \eqref{homogdim}. The reason will be clarified in the last section.

%\begin{Remark}
%	The condition $p<\frac{N}{N-1}$ in Theorem \ref{prolLxi} is not very restrictive. Following \cite[Remark 3.7]{Santambrogio1} and \cite[Proposition 4.4]{Brasco2}, if we consider two discrete measure $\mu$ and $\nu$, with $\mu\not=\nu$, then $\mathcal{Q}_{\mathbb{H}}^p(\mu,\nu)\not=\emptyset$ for $p\in\left(1,\frac{N}{N-1}\right)$ but $\mathcal{Q}_{\mathbb{H}}^{\frac{N}{N-1}}(\mu,\nu)=\emptyset$. Indeed, if we assume that there exists $Q\in\mathcal{Q}_{\mathbb{H}}(\mu,\nu)$ such that $i_{Q}\in L^{\frac{N}{N-1}}(\Omega)$, we can define the vector measure $\textbf{\textsc{w}}_Q$
%	\begin{equation*}\label{vectormeasure}
%		\int_{\overline{\Omega}}X(x)\cdot d\textbf{\textsc{w}}_Q=\int_{C([0,1],\overline{\Omega})}\left(\int_0^1\left\langle X(\sigma(t)),\dot{\sigma}(t)\right\rangle_H dt\right)dQ(\sigma),\quad \forall X\in C(\overline{\Omega},H\mathbb{H}^{n}).
%	\end{equation*}
%	It follows that $|\textbf{\textsc{w}}_Q|\leq i_{Q}$, and hence $\|\textbf{\textsc{w}}_Q\|_{\frac{N}{N-1}}\leq\|i_{Q}\|_{\frac{N}{N-1}}<+\infty$.
%	Moreover it follows that
%	\begin{equation*}\label{flowdivergence}
%		\nabla_H\cdot\textbf{\textsc{w}}_Q=\mu-\nu.
%	\end{equation*} 
%	Since $\mu-\nu\notin HW^{-1,\frac{N}{N-1}}(\Omega)$, we get a contradiction. We have just proved that $\mathcal{Q}_{\mathbb{H}}^{\frac{N}{N-1}}(\mu,\nu)=\emptyset$ as soon as $\mu-\nu\notin HW^{-1,\frac{N}{N-1}}(\Omega)$. We can conclude that if $p\geq\frac{N}{N-1}$ the congestion effects are so strong that the total congested cost in \eqref{lepbme1} is always $+\infty$ as soon as $\mu-\nu\notin HW^{-1,\frac{N}{N-1}}(\Omega)$.
%\end{Remark}

We will see in the next section that the transport plan 
which realizes the equilibrium can be found as a solution of a suitable convex optimization problem. 


\subsection{Existence of Wardrop equilibria as minima of a convex optimization problem}
The following  convex optimization problem has been proposed in \cite{Santambrogio1}, in order to get the existence of equilibria introduced posed in the previous section. We will always assume that
\begin{equation}
\mathcal{Q}_{\mathbb{H}}^p(\mu,\nu):=\left\{Q\in\mathcal{Q}_{\mathbb{H}}(\mu,\nu):i_{Q}\in L^p\right\}\not=\emptyset,
\end{equation}
with $p<\frac{N}{N-1}$, and we call \textit{horizontal congested optimal transport problem}:
\begin{equation}\label{lepbme1}
	\inf_{Q\in \mathcal{Q}_{\mathbb{H}}^p(\mu,\nu)}\int_{\Omega}G(x,i_{Q}(x))dx,
\end{equation}

%\begin{equation}\label{lepbme0}
%	\inf_{Q\in \mathcal{Q}_{\mathbb{H}}(\mu,\nu)} \mathcal{G}(i_{Q}),
%\end{equation}
%where
%\begin{equation}\label{totalcost}
%	\mathcal{G}(i)=\begin{cases}
%		\int_{\Omega}G(x,i(x))dx,\quad  &\mbox{ if $i\ll\mathcal{L}^{2n+1}$},\\
%		+\infty,\quad&\mbox{ otherwise,}
%	\end{cases}	
%\end{equation}
where $G(x,i):=\int_0^ig(x,z)dz$, for some cost function $g$. 
% As we can see in \eqref{congfun}, congestion effects in this model are quite strong: only very diffused horizontal traffic intensity (i.e. absolutely continuous w.r.t $\mathcal{L}^{2n+1}$) are allowed; as soon as there is a low-dimensional concentration of the traffic, there will be an infinite total cost and vehicles get stuck in traffic.
The quantity $\int_{\Omega}G(i_{Q}(x))dx$ can be seen as the total cost of congestion for someone who has the right to impose \textit{who goes where}, in order to minimize the total cost, which in general differs from the total cost paid by a single user, given by \eqref{vehicles}. For this reason we will refer to %\eqref{lepbme1} as the \textit{horizontal congested optimal transport problem} and to
the functional in \eqref{lepbme1} as the \textit{total cost functional}.

We further assume that $\exists a,b\in\mathbb{R}_+$ such that
\begin{equation*}\label{growthg}
	ai^{p-1}\leq g(x,i)\leq b(i^{p-1}+1), \forall i\in\mathbb{R}_+,\forall x\in\Omega,
\end{equation*}
which implies that, for every $x\in\Omega$, the function $g(x,i_{Q}(\cdot))\in L^{p'}(\Omega)$ for every $Q\in\mathcal{Q}^p_{\mathbb{H}}(\mu,\nu)$.


%{\color{blue} - NOTA -  CONCORDO CON BRASCO. NON SI CAPISCE COSA SERVONO QUESTE PROPRIETA', IMMAGINO SERVANO NELLA PROVA, MA NO SI CAPISCE COME}

%Note that Ascoli-Arzelà theorem guarantees that the set 
%\begin{equation*}
%	H_K:=\left\{\sigma\in H:|\dot\sigma|_H\leq K\right\}
%\end{equation*}
%are compact (w.r.t. the uniform convergence) for every $K>0$. Indeed these sets are equicontinuous because 
%\begin{equation*}
%	d_{CC}(\sigma(t_1),\sigma(t_2))\leq\int_{t_1}^{t_2}|\dot\sigma(t)|_Hdt\leq K|t_1-t_1|,\quad \forall t_1,t_2\in[0.1].
%\end{equation*}

%{\color{blue} - NOTA -  END COSA NON COMPRESA}

Under the previous hypothesis, following the same strategy as \cite{Santambrogio1} and using the compactness argument above, one can prove that:

\begin{teo}
	The minimization problem \eqref{lepbme1} admits a solution.
\end{teo}

%\subsection{Optimality Conditions}\label{foc}

We consider the variational inequality characterizing solutions of the convex problem \eqref{lepbme1}. Precisely transport plan  $\overline{Q}\in \mathcal{Q}_{\mathbb{H}}^p(\mu,\nu)$,  solves \eqref{lepbme1} if and only if
\begin{equation}\label{varineq}
	\int_{\Omega} \phi_{\overline{Q}}(x) i_{\overline{Q}}(x)dx=\inf \left\{ \int_{\Omega} \phi_{\overline{Q}}(x) i_{Q}(x)dx \; :\; Q\in\mathcal{Q}_{\mathbb{H}}^p(\mu,\nu)\right\},
\end{equation}
where $\phi_{\overline{Q}}(x)=g(x,i_{\overline{Q}}(x))$.
Following the same strategy as in \cite[Proposition 3.9]{Santambrogio1}, one can also prove that if $\overline{Q}\in \mathcal{Q}_{\mathbb{H}}^p(\mu,\nu)$ solves \eqref{lepbme1}, then
\begin{equation}
	\int_{\Omega} \phi_{\overline{Q}}(x) i_{\overline{Q}}(x)dx=\inf_{\gamma\in\Pi(\mu,\nu)}\int_{\overline{\Omega}\times\overline{\Omega}}c_{\phi_{\overline{Q}}}(x,y)d\gamma(x,y).
\end{equation}

Now we will see that solutions of \eqref{lepbme1} are equilibrium configurations for the Wardrop problem,  introduced in Definition \ref{Wardrop}. 
\begin{Remark}
    If $\overline{Q}$ solves \eqref{lepbme1}, and $\gamma_{\bar Q}:=(e_0,e_1)_{\#}\overline{Q}$,  it follows that
\begin{align*}
	\int_{\overline{\Omega}\times \overline{\Omega}} c_{\phi_{\overline{Q}}}(x,y)d\gamma_{\bar{Q}}(x,y)=\int_{C([0,1],\overline{\Omega})}c_{\phi_{\overline{Q}}}(\sigma(0),\sigma(1))d\overline{Q}(\sigma)\leq\\ \underset{\eqref{ineglc}}{\leq}\int_{C([0,1],\overline{\Omega})} L_{\phi_{\overline{Q}}}(\sigma) d\overline{Q}(\sigma)= \int_{\Omega} \phi_{\overline{Q}}(x) i_{\overline{Q}}(x)dx=\\= \inf_{\gamma\in \Pi(\mu,\nu)} \int_{\overline{\Omega}\times \overline{\Omega}} c_{\phi_{\overline{Q}}}(x,y)d\gamma(x,y).
\end{align*}
Hence, $\gamma_{\bar Q}$ solves the  Monge-Kantorovich problem, associated with the cost function $c_{\phi_{\overline{Q}}}$
\begin{equation}
	\inf_{\gamma\in \Pi(\mu,\nu)} \int_{\overline{\Omega}\times \overline{\Omega}} c_{\phi_{\overline{Q}}}(x,y)d\gamma(x,y).
\end{equation}
\end{Remark}
We also observe that
\begin{align*}
	\int_{C([0,1],\overline{\Omega})} L_{\phi_{\overline{Q}}}(\sigma) d\overline{Q}(\sigma)= \int_{\overline{\Omega}\times\overline{\Omega}} c_{\phi_{\overline{Q}}}(x,y)d\gamma_{\bar Q}(x,y)\\
	=\int_{C([0,1],\overline{\Omega})}c_{\phi_{\overline{Q}}}(\sigma(0),\sigma(1))d\overline{Q}(\sigma)
\end{align*}
and, since $L_{\phi_{\overline{Q}}}(\sigma)\geq c_{\phi_{\overline{Q}}}(\sigma(0),\sigma(1))$, we get
\[L_{\phi_{\overline{Q}}}(\sigma)=c_{\phi_{\overline{Q}}}(\sigma(0),\sigma(1)) \quad\mbox{ for }  \overline{Q} \mbox{-a.e. } \sigma.\]

We remark the fact that the hypothesis $p<\frac{N}{N-1}$ guarantees the conjugate exponent $p'>N$: from \eqref{growthg} it follows that $\phi_{\bar Q}\in L^{p'}(\Omega)$ and then the metric $c_{\phi_{\bar Q}}$ is well defined.

%\subsection{Existence of Wardrop Equilibria }


Following the same argument as in \cite{Santambrogio1}, one can prove the existence of such equilibria:

\begin{teo}\label{cwe}
	Let us assume that $1<p<\frac{N}{N-1}$ then, if $\mathcal{Q}^p_{\mathbb{H}}(\mu,\nu)\not=\emptyset$, there exists an equilibrium. Moreover $Q\in\mathcal{Q}_{\mathbb{H}}(\mu,\nu)$ is an equilibrium if and only if $Q$ solves the minimization problem \eqref{lepbme1}.
\end{teo}

\begin{Remark}
    If we furthermore assume that $G(x,\cdot)$ is strictly convex for every $x\in\Omega$, then given $Q_{1}$ and $Q_{2}$ which solves \eqref{lepbme1} it follows that $i_{Q_{1}}=i_{Q_{2}}$. In other words, equilibria are not necessarily unique but they all induce the same intensity or, equivalently, the same metric.
\end{Remark}
 
As we said before, from a modelistic viewpoint minimizing \eqref{lepbme1}, with the constraint $(e_0)_{\#}Q=\mu$ and $(e_1)_{\#}Q=\nu$, describes a situation we only know which is the the initial and  final  cortical activity (or the  distributions of vehicles and their destination), and we are interested in  minimizing the total cost: in mathematical terms, this corresponds to say that the set of admissible couplings $(e_0, e_1)_{\#}Q$ coincides with the whole set of transport plans $\Pi(\mu,\nu)$. This is often referred as SO (System Optimum) type of movement. 

More constraints could be possibly imposed to the problem. Indeed one could minimize the total cost under the constraint of $(e_0,e_1)_{\#}Q\in\Pi\subset\Pi(\mu,\nu)$. As a particular case, we have $\Pi=\{\bar{\gamma}\}$, that is the coupling is given and, roughly speaking, we a priori know the probability $\bar\gamma(x,y)$ for a vehicle in $x$ (or the signal starting from $x$) to reach a destination $y$. This is often called User Equilibrium (UE) type of movement. 
In this case we look for a traffic plan in the set 
\begin{eqnarray*}
	\mathcal{Q}_{\mathbb{H}}^p(\overline{\gamma}):=\bigg{\{}Q\in \mathcal{P}(C([0,1],\overline{\Omega})) \mbox{ : } Q(H)=1,\ \int_{C([0,1],\overline{\Omega})}l_H(\sigma)dQ(\sigma)<+\infty,\\ (e_0,e_1)_\sharp Q=\overline{\gamma} \text{ and } i_{Q}\in L^p(\Omega)\bigg{\}} 
\end{eqnarray*}
and equilibria are traffic plans belonging to this set and satisfying the first condition of Definition \ref{Wardrop}. If $\mathcal{Q}_{\mathbb{H}}^p(\overline{\gamma}) \not=\emptyset$ the previous arguments can be adapted to this new situation and we can get the existence of equilibria.
Interestingly enough, while in the urban scenario UE is more realistic, since each vehicle has a preferred destination, it has been proved in \cite{Wolfson} (in the  Euclidean setting) that a SO type of movement is more natural for the description of signal propagation in the brain.

