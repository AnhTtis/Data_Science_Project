
This section contains some well-known results about optimal transportation theory in the Heisenberg Group. Precisely we collect some results about the Monge-Kantorovich problem associated with Carnot-Carath\'eodory distance, 
following the presentation in \cite{DePascale2}.

Given a Polish metric space $(M,d)$, we denote by $\mathcal{M}_+(M)$ the set of positive and finite Radon measures on $M$;  we denote by $\mathcal{P}(M)$, resp. $\mathcal{P}_c(M)$, the subset of probability measures on $M$, resp. the subset of probability measures on $M$ with compact support. 

If $(M_1,d_1)$ and $(M_2,d_2)$ are two Polish metric spaces, $f:M_1\to M_2$ is a Borel map and $\mu\in\mathcal{M}_+(M_1)$, we denote by $f_{\#}\mu\in\mathcal{M}_+(M_2)$ the measure defined as $f_{\#}\mu(A):=\mu(f^{-1}(A))$, $\forall A$ Borel set in $M_2$.
Let $\mu,\nu\in\mathcal{P}_c(\mathbb{H}^n)$, we denote by
\begin{equation*}
	\Pi(\mu,\nu)=\big{\{}\gamma\in\mathcal{P}(\mathbb{H}^n\times\mathbb{H}^n): (\pi_1)_{\#}\gamma=\mu, (\pi_2)_{\#}\gamma=\nu\big{\}}
\end{equation*}
the set of transport plans between $\mu$ and $\nu$, where $\pi_1$ and $\pi_2$ are the projection on the first and second factor, respectively. The set $\Pi(\mu,\nu)$ is compact w.r.t. the weak convergence of measures.

Given a lower semicontinuous cost function $k:\mathbb{H}^n\times\mathbb{H}^n \rightarrow [0,+\infty]$, the Monge-Kantorovich problem between $\mu$ and $\nu$, associated with the cost $k$,
\begin{equation} \label{e:MK} 
	\inf_{\gamma \in \Pi(\mu,\nu)} \int_{\mathbb{H}^n\times\mathbb{H}^n} k(x,y)\,d\gamma(x,y),
\end{equation}
admits solutions. We call 
these solutions \textit{optimal transport plans} for the cost function $k$ and we denote by
\begin{equation*}
	\Pi_k(\mu,\nu):=\{\gamma\in\Pi(\mu,\nu):\gamma \text{ solves } \eqref{e:MK}\}. 
\end{equation*}
This set is closed in $\Pi(\mu,\nu)$, w.r.t. the weak convergence of measures. Moreover, if $\gamma\in\Pi_k(\mu,\nu)$ and $\int_{\mathbb{H}^n\times\mathbb{H}^n}kd\gamma<+\infty$, then $\gamma$ is concentrated on a $k$-cyclically monotone set $\Gamma\subseteq\mathbb{H}^n\times\mathbb{H}^n$, i.e.

\begin{equation*}\label{c-CM}
\sum_{i=1}^N k(x_i,y_i) \leq \sum_{i=1}^N k(x_{i},y_{\sigma(i)}),
\end{equation*}
for any $N\geq 2$, any $(x_1,y_1), \dots, (x_N,y_N)\in\Gamma$ and any permutation $\sigma$ of $N$ elements. See  \cite[Chapter 1]{Santambrogiolibro}.

We say that a transport plan $\gamma\in \Pi(\mu,\nu)$ is \textit{induced by a map} if there exists a Borel map $T:\mathbb{H}^n\rightarrow \mathbb{H}^n$ such that $(I \otimes T)_\sharp\mu = \gamma$, where $(I \otimes T)(x) := (x,T(x))$. We will refer to a Borel map $T:\mathbb{H}^n\rightarrow \mathbb{H}^n$ solving \begin{equation}\label{mongepb}
	\inf_{T_{\#}\mu=\nu}\int_{\mathbb{H}^n}k(x,T(x))d\mu(x),
\end{equation}
as an \textit{optimal transport map} for the cost function $k$. If $\gamma\in\Pi_k(\mu,\nu)$ is induced by a map $T$, then $T$ is an optimal transport map for the cost $k$. Moreover, if any $\gamma\in\Pi_k(\mu,\nu)$ is induced by a map, then there exists a unique optimal transport map. Hence also $\gamma\in\Pi_k(\mu,\nu)$ is unique.

When $k=d_{CC}$, from the arguments above it follows that the Monge-Kantorovich problem
\begin{equation}\label{MKH}
	\inf_{\gamma\in \Pi(\mu,\nu)} \int_{\mathbb{H}^n\times\mathbb{H}^n} d_{CC}(x,y)\,d\gamma(x,y),
\end{equation}
admits at least a solution, concentrated on a $d_{CC}$-cyclically monotone set. From now on we will denote by
\begin{equation}\label{23luglio}
    \Pi_1(\mu,\nu):=\left\{\gamma\in\Pi(\mu,\nu):\gamma \textnormal{ solves }\eqref{MKH}\right\}.
\end{equation}

The explicit representation of geodesics, together with the absolute continuity of either $\mu$, or $\nu$, with respect to the Haar measure of the group, imply that any optimal transport plan (for the cost function $d_{CC}$) is concentrated on the set $E\subseteq\mathbb{H}^n\times\mathbb{H}^n$ of pairs of points connected by a unique geodesic, see \eqref{KAPPA} for its definition.
\begin{prop} \label{pi1.1}
    If either $\mu\ll\mathcal{L}^{2n+1}$, or $\nu\ll\mathcal{L}^{2n+1}$, and $\gamma\in \Pi_1(\mu,\nu)$, then for $\gamma$-a.e. $(x,y)\in\mathbb{H}^n\times\mathbb{H}^n$, there exists a unique geodesic between $x$ and $y$, i.e. $\gamma(\mathbb{H}^n\times\mathbb{H}^n\setminus E)=0$.
\end{prop}
See \cite[Lemma 4.1]{DePascale2} for the proof. 


We denote by
\begin{equation}\label{lipset}
    \text{Lip}_1(\mathbb{H}^n, d_{CC}):=\left\{u:\mathbb{H}^n\to\mathbb{R}:|u(x)-u(y)|\leq d_{CC}(x,y),\forall x,y\in\mathbb{H}^n\right\}.
\end{equation}
The following important theorem holds.
\begin{teo}\label{1lip_potential}
There exists a function $u\in \textnormal{Lip}_1(\mathbb{H}^n,d_{CC})$ so that
\begin{equation*}
	 \min_{\gamma\in\Pi(\mu,\nu)}\int_{\mathbb{H}^n\times\mathbb{H}^n} d_{CC}(x,y)\,d\gamma(x,y) 
	= \int_{\mathbb{H}^n} u(x)\,d\mu(x) - \int_{\mathbb{H}^n} u(y)\,d\nu(y),
\end{equation*}
and 
$\gamma\in \Pi(\mu,\nu)$ is optimal if and only if 
\begin{equation*}
    u(x) - u(y) = d_{CC}(x,y),  \quad\text{for }\gamma-\text{a.e. }(x,y)\in \mathbb{H}^n\times\mathbb{H}^n.
\end{equation*}
\end{teo}

We call such a $u\in\text{Lip}_1(\mathbb{H}^n,d_{CC})$ a \textit{Kantorovich potential}.


From now on, we fix a Kantorovic potential $u\in\text{Lip}_1(\mathbb{H}^n,d_{CC})$ and we use it to check the optimality of transport plans. In this way one can select some optimal transport plans that satisfy a monotonicity condition, in the following sense.

If $\gamma\in\Pi_1(\mu,\nu)$, Theorem \ref{1lip_potential} and the Lipschitzianity of $u$ imply that 
\begin{equation*}\label{condition1}
	u(\sigma_{x,y}(t))=u(x)-d_{CC}\left(x,\sigma_{x,y}(t)\right),\quad\forall t\in[0,1],
\end{equation*}
for $\gamma$-a.e $(x,y)\in\mathbb{H}^n\times\mathbb{H}^n$ and any $\sigma_{x,y}$ geodesic between $x$ and $y$.
In this way one can define an order relation on $\sigma_{x,y}$ in the following way: let $t_1,t_2\in[0,1]$, $x'=\sigma_{x,y}(t_1)$ and $x'':=\sigma_{x,y}(t_2)$, then
\begin{equation}\label{orderrelation}
	x'\leq x''\Longleftrightarrow u(x')\geq u(x'').
\end{equation}

We denote by 
$\Pi_2(\mu,\nu)$ the set of transport plans solving the secondary variational problem
\begin{equation}\label{secvarpb}
	\inf_{\gamma \in \Pi_1(\mu,\nu)} \int_{\mathbb{H}^n\times\mathbb{H}^n} d_{CC}(x,y)^2\,d\gamma(x,y).
\end{equation}
This problem admits solutions since the functional $\gamma\mapsto\int_{\mathbb{H}^n\times\mathbb{H}^n}d_{CC}(x,y)^2\,d\gamma(x,y)$  is continuous w.r.t. the weak convergence of measures and $\Pi_1(\mu,\nu)$ is compact w.r.t. the same convergence. Theorem \ref{1lip_potential} allows us to rephrase this problem as a classical Monge-Kantorovich problem \eqref{e:MK} with cost $k(x,y) = \beta(x,y)$, where 
\begin{equation*}
	\beta(x,y) = \begin{cases}
		d_{CC}(x,y)^2 \quad \text{if } u(x)-u(y) = d_{CC}(x,y),\\
		+\infty \qquad \phantom{\text{if}} \text{otherwise}.
	\end{cases}
\end{equation*}
Since $\beta$ is lower semicontinuous and $\int_{\mathbb{H}^n\times\mathbb{H}^n} \beta(x,y) \,d\gamma(x,y)<+\infty$ for all $\gamma \in \Pi_2(\mu,\nu)$, it follows that any $\gamma \in \Pi_2(\mu,\nu)\subset\Pi_1(\mu,\nu)$ is concentrated on a $\beta$-cyclically monotone set $\Gamma$, i.e. 
\begin{equation}\label{var1}
	u(x) - u(y) = d_{CC}(x,y), \quad \forall (x,y)\in \Gamma,
\end{equation}
and
\begin{equation}\label{var2}
	\beta(x,y) +\beta(x',y') \leq \beta(x,y')+\beta(x',y),\quad \forall (x,y),(x',y')\in \Gamma.
\end{equation}

Using the non-branching property of $(\mathbb{H}^n,d_{CC})$ one can prove that geodesics used by an optimal transport plan cannot bifurcate. Moreover, if a transport plan solves also \eqref{secvarpb} then \eqref{var1} and \eqref{var2} imply a one-dimensional monotonicity condition along geodesics. More precisely, the following result \cite[Lemma 4.2 and Lemma 4.3]{DePascale2} holds.
\begin{prop}\label{monotone}
	Let $\gamma\in \Pi_1(\mu,\nu)$. Then, $\gamma$ is concentrated on a set $\Gamma$ such that for all $(x,y)$, $(x',y')\in \Gamma$ such that $x\not=y$ and $x\not=x'$, if $x'$ lies on a geodesic between $x$ and $y$ then all points $x$, $x'$, $y$ and $y'$ lie on the same geodesic.
	
	Moreover if $\gamma\in \Pi_2(\mu,\nu)$, then the condition $x<x'$ implies $y\leq y'$.
\end{prop} 


As far as we know, in the Heisenberg Group has not been proven that any $\gamma\in\Pi_2(\mu,\nu)$ is induced by a map, and hence $\gamma\in\Pi_2(\mu,\nu)$ is unique. See \cite[Theorem 28]{Feldman} or \cite[Theorem 3.18]{Santambrogiolibro} for the analogous result in the Riemannian setting. Anyway in \cite{DePascale2} the authors proved that some particular transport plans in $\Pi_2(\mu,\nu)$, more precisely the ones that can be selected through the variational approximation below, are induced by maps.
\\

Following \cite[Section 5]{DePascale2}, we recall the aforementioned variational approximation procedure. Let $K$ be a compact subset of $\mathbb{H}^n$ such that 
\begin{equation*}
    \text{supp}(\mu)\cup\text{supp}(\nu)\subset K,
\end{equation*}
and let us denote by 
\begin{equation*}
	\Pi:=\{\gamma\in\mathcal{P}(\mathbb{H}^n\times\mathbb{H}^n): (\pi_1)_{\#}\gamma=\mu,\  \text{supp}((\pi_2)_\#\gamma)\subset K\}.
\end{equation*}

For any $\varepsilon\in\mathbb{R}_+$, we can consider the family of minimization problems
\begin{equation}\tag{$P_{\varepsilon}$}\label{varapprox} 
	\min\{C_\varepsilon(\gamma):\gamma\in\Pi\},
\end{equation}
where
\begin{multline*}
	C_{\varepsilon}(\gamma):=\frac{1}{\varepsilon}\, W_1((\pi_2)_\sharp\gamma,\nu) + \int_{\mathbb{H}^n\times\mathbb{H}^n} d_{CC}(x,y)\,d\gamma(x,y) \\ +  \varepsilon \int_{\mathbb{H}^n\times\mathbb{H}^n} d_{CC}(x,y)^2\,d\gamma(x,y) + \varepsilon^{6n+8} \textnormal{card}{(\textnormal{supp}({(\pi_2)_\sharp\gamma}))},
\end{multline*}
where $W_1((\pi_2)_\sharp\gamma,\nu)$ denotes the \textit{1-Wasserstein distance} between the two measures $(\pi_2)_\sharp\gamma$ and $\nu$,
\begin{equation*}
    W_1((\pi_2)_\sharp\gamma,\nu):=\min\left\{\int_{\mathbb{H}^n\times\mathbb{H}^n}d_{CC}(x,y)d\gamma(x,y):\gamma\in\Pi((\pi_2)_\sharp\gamma,\nu)\right\}.
\end{equation*}

For any $\varepsilon>0$, the minimization problem \ref{varapprox} admits at least one solution with finite value. Moreover the following result holds.

\begin{lemma} \label{optpi2}
Let $(\varepsilon_k)_{k\in\mathbb{N}}\subset\mathbb{R}_+$ be such that $\varepsilon_k\underset{k\rightarrow\infty}{\longrightarrow}0$ and $\gamma_{\varepsilon_k}$ be a solution to $(P_{\varepsilon_k})$. If $\gamma_{\varepsilon_k}\rightharpoonup\gamma\in\mathcal{P}(\mathbb{H}^n\times\mathbb{H}^n)$, then $\nu_{\varepsilon_k}:=(\pi_2)_\sharp \gamma_{\varepsilon_k}\rightharpoonup\nu$ and $\gamma \in \Pi_2(\mu,\nu)$.
\end{lemma}

The following theorem (\cite[Theorem 8.1]{DePascale2}) guarantees the existence of optimal transport plans induced by maps, and hence the existence of solutions to \eqref{mongepb}. In particular the optimal transport plans that turn out to be induced by maps are the weak limit of solutions $(\gamma_{\varepsilon_k})_{k\in\mathbb{N}}$ to $(P_{\varepsilon_k})_{k\in\mathbb{N}}$, for some $\varepsilon_k\to0$ as $k\to\infty$. Therefore they are monotone on the transport rays in the sense of \eqref{orderrelation}.

\begin{teo}\label{mainthmbis} If $\mu\ll\mathcal{L}^{2n+1}$, then there exists an optimal transport map $T:\mathbb{H}^n\rightarrow\mathbb{H}^n$ such that $\gamma=(\textnormal{Id}\otimes T)_\#\mu\in\Pi_2(\mu,\nu)$.  
\end{teo}


