\documentclass{amsart}
\usepackage[italian,english]{babel}
\usepackage{amsaddr}

\usepackage[utf8]{inputenc}

\usepackage{enumitem}

\usepackage{biblatex}
\addbibresource{references.bib}
\setlength\bibitemsep{5pt}
\usepackage{csquotes}
\usepackage{lmodern}

\usepackage{amsmath}

\usepackage{amssymb}

\usepackage{bbm}

\usepackage{fancyhdr}

\numberwithin{equation}{section}

\usepackage{xcolor}
%%%%%%%%%%%%%%%%%%%%%%%%%%%%%%%%%%%%%%%%%%%%%%%%%%%%%%%%%%%%%%%%%%% MARGINI
\usepackage{geometry}
\geometry{a4paper,top=3cm,bottom=3cm, left=2.75cm,right=2.75cm,%
heightrounded,bindingoffset=5mm}


\newtheoremstyle{theorem2}
{8pt}
{8pt}
{\itshape}
{}
{\bfseries}
{.}
{.5em}
{}

\newtheoremstyle{definition2}
{8pt}
{8pt}
{\itshape}
{}
{\bfseries}
{.}
{.5em}
{}

\theoremstyle{theorem2}


\newtheorem{lemma}{Lemma}[section]
\newtheorem{teo}[lemma]{Theorem}

\newtheorem{cor}[lemma]{Corollary}
\newtheorem{prop}[lemma]{Proposition}

\theoremstyle{definition2}

\newtheorem{deff}[lemma]{Definition}
\newtheorem{Remark}[lemma]{Remark}
\newtheorem{exa}[lemma]{Example}


\pagestyle{plain}

\cfoot{} 


\rhead[\fancyplain{}{\bfseries\leftmark}]{\fancyplain{}{\bfseries\thepage}} 
\lhead[\fancyplain{}{\bfseries\thepage}]{\fancyplain{}{\bfseries\rightmark}} 



\begin{document}


\title{Transport densities and congested optimal transport problem in the Heisenberg group}

\author{Michele Circelli, Giovanna Citti} 
\address{University of Bologna, Department of Mathematics, 40126 Bologna, Italy}

\keywords{Congested optimal transport, Wardrop equilibrium, Heisenberg group, transport density} 

\begin{abstract}
We adapt the problem of continuous congested optimal transport to the Heisenberg group, equipped with a sub-Riemannian metric. Originally introduced in the Euclidean setting by Carlier, Jimenez, and Santambrogio as a path-dependent variant of the Monge-Kantorovich problem, we significantly restrict the set of admissible curves to horizontal ones. We establish the existence of equilibrium configurations as solutions to a convex minimization problem over a suitable set of measures on horizontal curves. This result is achieved through the notions of horizontal transport density and horizontal traffic intensity.
\end{abstract}

\maketitle

\section{Introduction}
The first formalization of the concept of congested optimal transport in a discrete, Euclidean setting can be traced back to 1952, when Wardrop introduced a notion of equilibrium governing traffic congestion on a network in \cite{Wardrop}. In 1956, Beckmann et al. proposed a variational characterization for such equilibria in \cite{Beckmann1}. For a comprehensive overview, see \cite{Santambrogiolibro} or \cite{Santambrogio2}. In a slightly different context, Bouchitté and Buttazzo introduced the notion of transport density in \cite{Buttazzo1} and \cite{Buttazzo2}, which measures the amount of transport occurring along geodesics within a given region for an optimal transport plan. In \cite{Morel}, Bernot et al. introduced a similar but more abstract framework for describing traffic congestion in networks, expressed in terms of traffic plans, which are probability measures over all possible curves between origin-destination pairs. A few years later, Carlier et al. (in \cite{Santambrogio1}) used this optimal transport approach to propose a continuous version of the traffic congestion problem in networks. They proved the existence of Wardrop equilibria through a convex minimization problem over the set of traffic plans, which is the continuous counterpart of the one proposed in \cite{Beckmann1} for the discrete setting. The traffic intensity they introduced can be considered as the reformulation, in the abstract framework introduced in \cite{Morel}, of notion of transport density. It is worth noting that this minimization problem allows for duality and numerical simulations, as discussed in \cite{Peyre2} and \cite{Peyre}. However, this minimization problem over a set of measures over curves presents many technical difficulties. To address this, in \cite{Brasco} Brasco et al. proved  the equivalence with a technically simpler minimization problem over a set of vector fields with prescribed divergence, introduced in \cite{Beckmann2}. They also proved that Wardrop equilibria are supported on the integral curves, in the DiPerna-Lions sense, of optimal vector fields. For a regularity result about optimal vector fields, see \cite{Colombo}. This new formulation is easier to handle and admits a dual formulation as a classical problem in the calculus of variations. Optimality conditions for this formulation can be written as a PDE, which is the $q$-Laplacian in the simplest case but becomes quite degenerate in more realistic congestion models. See also \cite{brasco2013congested} and \cite{Santambrogioregularity}.
\\

In this paper, we address the continuous congested optimal transport problem in the Heisenberg group $\mathbb{H}^n$, the simplest non-commutative Lie group, which naturally arises in the description of the phase space. This setting is characterized by the choice of $2n$ vector fields and a metric on the sub-bundle of the tangent bundle generated by them. Since the number of given vector fields is strictly smaller than the dimension of the space, the metric is non-Riemannian at every point. All intrinsic objects of the space are defined in terms of these vector fields: in particular, displacement occurs only along their integral curves, called horizontal curves, and the presence of this metric naturally leads to the definition of a sub-Riemannian distance. The Heisenberg group was also proposed by Petitot and Tondut in \cite{Petitot} to describe the functional geometry of the visual cortex (see also \cite{cittisarti}, where the problem was expressed using the tools of sub-Riemannian geometry): they describe the propagation of signals along horizontal curves. Additionally, in \cite{Misic}, the authors state that the propagation of visual signals gives rise to congestion phenomena. With this in mind, we adapt to the Heisenberg group results contained in \cite{Santambrogio1}, which were specific to the Euclidean setting. Addressing this problem in the Heisenberg group is very interesting from a mathematical viewpoint due to its close link with optimal transport theory in this setting. In this direction, the first significant work is \cite{Rigot}, where the authors demonstrated the existence of solutions to the Monge problem associated with the square of the sub-Riemannian distance and established a Brenier-McCann representation theorem. This result was further generalized in \cite{FigalliRifford} to more general (non-homogeneous) sub-Riemannian structures. The first existence result for solutions to the Monge problem associated with the sub-Riemannian distance was provided by De Pascale and Rigot in \cite{DePascale2}. For similar results in more general metric spaces, see \cite{Cavalletti3} and \cite{cavalletti2018overview}.

In this context, we introduce the notion of horizontal transport density, adapting to the geometry of the Heisenberg group the notion previously introduced in the Euclidean setting in \cite{Buttazzo1} and \cite{Buttazzo2}. Following the approach in \cite{Santambrogio}, we establish some summability results of the horizontal transport density for a range of $p$ depending on the geodetic homogeneity of the space. It is worthwhile to note that in $\mathbb{H}^n$, the geodesic dimension does not coincide with either the topological or the homogeneous dimension (see \cite{Juillet}). These results will be instrumental in defining the congested optimal transport problem in this setting. To address this problem, the first obstacle to overcome is lack of non-trivial geodesic subsets in the Heisenberg group, as discussed in \cite{Monti2}. Then, we define a horizontal traffic plan as a probability measure over the space of horizontal curves. The associated horizontal traffic intensity, which measures the amount of traffic along horizontal curves, generalizes the concept of horizontal transport density. Subsequently, inspired by \cite{Santambrogio1}, we introduce a convex minimization problem over the set of horizontal traffic plans, whose solutions are Wardrop equilibria in $\mathbb{H}^n$. To prove the existence of such solutions, we use results from \cite{Santambrogio1} that utilize abstract measure theory instruments. As optimality conditions, we find that such solutions are equilibrium configurations. In fact, traffic plans that solve the minimization problem are supported on geodesics with respect to a suitable congested metric (depending on the traffic plan itself through the traffic intensity). Moreover, they induce transport plans that solve a Monge-Kantorovich problem associated with the congested metric. One of the delicate aspects of the paper is defining this metric, due to the difficulty in defining a length induced by $p$-summable weights. 
\\

The paper is organized as follows: in Section 2, we introduce the Heisenberg Group and recall some well-known results on optimal transport theory in this setting. In Section 3, we introduce the notion of transport density and provide conditions that ensure its $L^p$-summability. In Section 4, we demonstrate that an $L^p$ function induces a weighted distance. In Section 5, we adapt the problem of congested optimal transport to the Heisenberg group and prove the existence of Wardrop equilibria in this setting.

\section{Preliminaries}
\label{sec2}
Deep learning has brought new inspirations to camera calibration, enabling a fully automatic calibration procedure without manual intervention. Here, we first summarize two prevalent paradigms in learning-based camera calibration: regression-based calibration and reconstruction-based calibration. Then, the widely-used learning strategies are reviewed in this research field. The detailed definitions for classical camera models and their corresponding calibration objectives are exhibited in the supplementary material.

\vspace{-0.3cm}

\subsection{Learning Paradigm}
Driven by different architectures of the neural network, the researchers have developed two main paradigms for learning-based camera calibration and its applications.

\noindent \textbf{Regression-based Calibration}
Given an uncalibrated input, the regression-based calibration first extracts the high-level semantic features using stacked convolutional layers. Then, the fully connected layers aggregate the semantic features and form a vector of the estimated calibration objective. The regressed parameters are used to conduct subsequent tasks such as distortion rectification, image warping, camera localization, etc. This paradigm is the earliest and has a dominant role in learning-based camera calibration and its applications. All the first works in various objectives, \textit{e.g.}, intrinsics: Deepfocal \cite{DeepFocal}, extrinsic: PoseNet \cite{PoseNet}, radial distortion: Rong et al. \cite{Rong}, rolling shutter distortion: URS-CNN \cite{URS-CNN}, homography matrix: DHN \cite{DHN}, hybrid parameters: Hold-Geoffroy et al. \cite{Hold-Geoffroy}, camera-LiDAR parameters: RegNet \cite{schneider2017regnet} have been achieved with this paradigm.

\noindent \textbf{Reconstruction-based Calibration}
On the other hand, the reconstruction-based calibration paradigm discards the parameter regression and directly learns the pixel-level mapping function between the uncalibrated input and target, inspired by the conditional image-to-image translation \cite{pix2pix} and dense visual perception\cite{long2015fully, eigen2014depth}. The reconstructed results are then calculated for the pixel-wise loss with the ground truth. In this regard, most reconstruction-based calibration methods \cite{DR-GAN, DDM, DaRecNet, BlindCor} design their network architecture based on the fully convolutional network such as U-Net\cite{ronneberger2015u}. Specifically, an encoder-decoder network, with skip connections between the encoder and decoder features at the same spatial resolution, progressively extracts the features from low-level to high-level and effectively integrates multi-scale features. At the last convolutional layer, the learned features are aggregated into the target channel, reconstructing the calibrated result at the pixel level.

In contrast to the regression-based paradigm, the reconstruction-based paradigm does not require the label of diverse camera parameters. Besides, the imbalance loss problem can be eliminated since it only optimizes the photometric loss of calibrated results. Therefore, the reconstruction-based paradigm enables a blind camera calibration without a strong camera model assumption.

\vspace{-0.3cm}

\subsection{Learning Strategies}
In the following, we review the learning-based camera calibration literature regarding different learning strategies.

\noindent \textbf{Supervised Learning}
Most learning-based camera calibration methods train their networks with the supervised learning strategy, from the classical methods \cite{DeepFocal, PoseNet, DHN, DeepVP, Rong, DeepCalib} to the state-of-the-art methods \cite{DVPD, EvUnroll, FishFormer, DAMG-Homo, SST-Calib}. In terms of the learning paradigm, this strategy supervises the network with the ground truth of the camera parameters (regression-based paradigm) or paired data (reconstruction-based paradigm). In general, they synthesize the training dataset from other large-scale datasets, under the random parameter/transformation sampling and camera model simulation. Some recent works \cite{Zhao, Tan, SPEC, DeepUnrollNet} establish their training dataset using a real-world setup and label the captured images with manual annotations, thereby fostering advancements in this research domain.

\noindent \textbf{Semi-Supervised Learning}
Training the network using an annotated dataset under diverse scenarios is an effective learning strategy. However, human annotation can be prone to errors, leading to inconsistent annotation quality or the inclusion of contaminated data. Consequently, increasing the training dataset to improve performance can be challenging due to the complexity and cost of constructing the dataset. To address this challenge, SS-WPC\cite{SS-WPC} proposes a semi-supervised method for correcting portraits captured by a wide-angle camera. It employs a surrogate task (segmentation) and a semi-supervised method that utilizes direction and range consistency and regression consistency to leverage both labeled and unlabeled data.

\noindent \textbf{Weakly-Supervised Learning}
Although significant progress has been made, data labeling for camera calibration is a notorious costly process, and obtaining perfect ground-truth labels is challenging. As a result, it is often preferable to use weak supervision with machine learning methods. Weakly supervised learning refers to the process of building prediction models through learning with inadequate supervision. Zhu et al. \cite{Zhu} present a weakly supervised camera calibration method for single-view metrology in unconstrained environments, where there is only one accessible image of a scene composed of objects of uncertain sizes. This work leverages 2D object annotations from large-scale datasets, where people and buildings are frequently present and serve as useful ``reference objects'' for determining 3D size.

\begin{figure*}[!t]
  \centering
  \includegraphics[width=1\textwidth]{figures/taxonomy1.pdf}
  %\vspace{-20pt}
  \caption{The structural and hierarchical taxonomy of camera calibration with deep learning. Some classical methods are listed under each category.}
  \label{fig:taxonomy}
  \vspace{-0.2cm}
\end{figure*}

\noindent \textbf{Unsupervised Learning}
Unsupervised learning, commonly referred to as unsupervised machine learning, analyzes and groups unlabeled datasets using machine learning algorithms. UDHN \cite{UDHN} is the first work for a cross-view camera model using unsupervised learning, which estimates the homography matrix of a paired image without the projection labels. By reducing a pixel-wise intensity error that does not require ground truth data, UDHN \cite{UDHN} outperforms previous supervised learning techniques. While preserving superior accuracy and robustness to fluctuation in light, the proposed unsupervised algorithm can also achieve faster inference time. Inspired by this work, increasing more methods leverage the unsupervised learning strategy to estimate the homography such as CA-UDHN \cite{CA-UDHN}, BaseHomo \cite{BasesHomo}, HomoGAN\cite{HomoGAN}, and Liu et al. \cite{Liu}. Besides, UnFishCor \cite{UnFishCor} frees the demands for distortion parameters and designs an unsupervised framework for the wide-angle camera.

\noindent \textbf{Self-supervised Learning}
Robotics is where the phrase ``self-supervised learning'' first appears, as training data is automatically categorized by utilizing relationships between various input sensor signals. Compared to supervised learning, self-supervised learning leverages input data itself as the supervision. Many self-supervised techniques are presented to learn visual characteristics from massive amounts of unlabeled photos or videos without the need for time-consuming and expensive human annotations. SSR-Net \cite{SSR-Net} presents a self-supervised deep homography estimation network, which relaxes the need for ground truth annotations and leverages the invertibility constraints of homography. To be specific, SSR-Net \cite{SSR-Net} utilizes the homography matrix representation in place of other approaches' typically-used 4-point parameterization, to apply the invertibility constraints. SIR \cite{SIR} devises a brand-new self-supervised camera calibration pipeline for wide-angle image rectification, based on the principle that the corrected results of distorted images of the same scene taken with various lenses need to be the same. With self-supervised depth and pose learning as a proxy aim, Fang et al. \cite{Fang} present to self-calibrate a range of generic camera models from raw video, offering for the first time a calibration evaluation of camera model parameters learned solely via self-supervision.

\noindent \textbf{Reinforcement Learning}
Instead of aiming to minimize at each stage, reinforcement learning can maximize the cumulative benefits of a learning process as a whole. To date, DQN-RecNet~\cite{DQN-RecNet} is the first and only work in camera calibration using reinforcement learning. It applies a deep reinforcement learning technique to tackle the fisheye image rectification by a single Markov Decision Process, which is a multi-step gradual calibration scheme. In this situation, the current fisheye image represents the state of the environment. The agent, Deep Q-Network \cite{mnih2015human}, generates an action that should be executed to correct the distorted image.

In the following, we will review the specific methods and literature for learning-based camera calibration. The structural and hierarchical taxonomy is shown in Figure~\ref{fig:taxonomy}. 
	
	\begin{table*}
		\rowcolors{1}{gray!20}{white}
		\centering
		\caption{
			{Details of the learning-based camera calibration and its extended applications from 2015 to 2022, including the method abbreviation, publication, calibration objective, network architecture, loss function, dataset, evaluation metrics, learning strategy, platform, and simulation or not (training data). For the learning strategies, SL, USL, WSL, Semi-SL, SSL, and RL denote supervised learning, unsupervised learning, weakly-supervised learning, semi-supervised learning, self-supervised learning, and reinforcement learning, respectively. }
		}
		\vspace{-6pt}
		\label{table:methods}
		\begin{threeparttable}
			\resizebox{1\textwidth}{!}{
				\setlength\tabcolsep{2pt}
				\renewcommand\arraystretch{0.98}
				% \begin{tabular}{|c|c|r||c|c|c|c|c|c|c|c|c|}  % {lccc}
				\begin{tabular}{c|r||c|c|c|c|c|c|c|c|c}
					\hline
					%\thickhline
					% &\#&
					&\textbf{Method}~~~~~~~~~&\textbf{Publication} &\textbf{Objective} &\textbf{Network}
					&\textbf{Loss Function} & \textbf{Dataset} &\textbf{Evaluation} & \textbf{Learning} &\textbf{Platform} &\textbf{Simulation}\\
					\hline
					\hline
					\multirow{1}{*}{\rotatebox{0}{\textbf{2015}}}
					% &1 &
					&DeepFocal~\cite{DeepFocal} &ICIP &Standard &AlexNet
					&$\mathcal{L}_2$ loss &1DSfM\cite{1DSfM} & Accuracy & SL &Caffe &\\
					&PoseNet~\cite{PoseNet} &ICCV &Standard 
					&GoogLeNet
					&$\mathcal{L}_2$ loss &Cambridge Landmarks\cite{Cambridge_Landmarks} &Accuracy &SL &Caffe& \\
					\hline
					\hline
					\multirow{1}{*}{\rotatebox{0}{\textbf{2016}}}
					% &1&
					&DeepHorizon~\cite{DeepHorizon} &BMVC &Standard &GoogLeNet	&Huber loss &HLW\cite{HLW} & Accuracy & SL &Caffe &\\
					
					&DeepVP~\cite{DeepVP} &CVPR &Standard 
					&AlexNet
					&Logistic loss &YUD\cite{YUD}, ECD\cite{ECD}, HLW\cite{HLW} &Accuracy &SL &Caffe& \\	
					
					&Rong et al.~\cite{Rong} &ACCV &Distortion &AlexNet
					&Softmax loss &ImageNet\cite{ImageNet} &Line length &SL &Caffe&\checkmark\\
					
					&DHN\cite{DHN} &RSSW &Cross-View &VGG
					&$\mathcal{L}_2$ loss &MS-COCO\cite{MS-COCO} &MSE &SL &Caffe&\checkmark\\		\hline
					\hline
					\multirow{1}{*}{\rotatebox{0}{\textbf{2017}}}
					% &1&
					&CLKN~\cite{CLKN} &CVPR &Cross-View  &CNNs	&Hinge loss &MS-COCO\cite{MS-COCO} & MSE & SL &Torch &\checkmark\\
					
		            &HierarchicalNet~\cite{HierarchicalNet} &ICCVW &Cross-View 
					&VGG
					&$\mathcal{L}_2$ loss &MS-COCO\cite{MS-COCO} &MSE &SL &TensorFlow&\checkmark \\
					
					&URS-CNN~\cite{URS-CNN} &CVPR &Distortion 
					&CNNs
					&$\mathcal{L}_2$ loss &Sun\cite{xiao2010sun}, Oxford\cite{philbin2007object}, Zubud\cite{shao2003zubud}, LFW\cite{huang2008labeled} &PSNR, RMSE &SL &Torch&\checkmark\\
					
					&RegNet~\cite{schneider2017regnet} &IV &Cross-Sensor 
					&CNNs
					&$\mathcal{L}_2$ loss &KITTI\cite{KITTI} &MAE &SL &Caffe&\checkmark\\
					
					\hline
					\hline
					\multirow{1}{*}{\rotatebox{0}{\textbf{2018}}}
					% &1&
					&Hold-Geoffroy et al.~\cite{Hold-Geoffroy} &CVPR &Standard &DenseNet	&Entropy loss &SUN360\cite{SUN360} & Human sensitivity & SL &- &\\
					
					&DeepCalib~\cite{DeepCalib} &CVMP &Distortion 
					&Inception-V3
					&Logcosh loss &SUN360\cite{SUN360} &Mean error &SL &TensorFlow&\checkmark \\	
					&FishEyeRecNet~\cite{FishEyeRecNet} &ECCV &Distortion &VGG
					&$\mathcal{L}_2$ loss &ADE20K\cite{ADE20K} &PSNR, SSIM &SL &Caffe&\checkmark\\
					
					&Shi et al.\cite{Shi} &ICPR &Distortion &ResNet
					&$\mathcal{L}_2$ loss &ImageNet\cite{ImageNet} &MSE &SL &PyTorch&\checkmark\\
					
					&DeepFM\cite{DeepFM} &ECCV &Cross-View &ResNet
					&$\mathcal{L}_2$ loss &T\&T\cite{TT}, KITTI\cite{KITTI}, 1DSfM\cite{1DSfM} &F-score, Mean &SL &PyTorch&\checkmark\\
					
					&Poursaeed et al.\cite{Poursaeed} &ECCVW &Cross-View &CNNs
					&$\mathcal{L}_1$, $\mathcal{L}_2$ loss &KITTI\cite{KITTI} &EPI-ABS, EPI-SQR &SL &-& \\
					
					&UDHN\cite{UDHN} &RAL &Cross-View &VGG
					&$\mathcal{L}_1$ loss &MS-COCO\cite{MS-COCO} &RMSE &USL &TensorFlow&\checkmark\\
					
					&PFNet\cite{PFNet} &ACCV &Cross-View &FCN
					&Smooth $\mathcal{L}_1$ loss &MS-COCO\cite{MS-COCO} &MAE &SL &TensorFlow&\checkmark\\
					
					&CalibNet\cite{iyer2018calibnet} &IROS &Cross-Sensor &ResNet
					&Point cloud distance, $\mathcal{L}_2$ loss &KITTI\cite{KITTI} &Geodesic distance, MAE &SL &TensorFlow&\checkmark\\

                        &Chang et al.\cite{chang2018deepvp} &ICRA &Standard &AlexNet
					&Cross-entropy loss &DeepVP-1M~\cite{chang2018deepvp} &MSE, Accuracy &SL &Matconvnet&\\
					
					\hline
					\hline
					\multirow{1}{*}{\rotatebox{0}{\textbf{2019}}}
					% &1&
					&Lopez et al.~\cite{Lopez} &CVPR &Distortion &DenseNet	&Bearing loss &SUN360\cite{SUN360} &MSE & SL &PyTorch &\\
					
					&UprightNet~\cite{UprightNet} &ICCV &Standard &U-Net	&Geometry loss &InteriorNet\cite{InteriorNet}, ScanNet\cite{ScanNet}, SUN360\cite{SUN360} &Mean error & SL &PyTorch &\\
					
					&Zhuang et al.~\cite{Zhuang} &IROS &Distortion &ResNet	&$\mathcal{L}_1$ loss & KITTI\cite{KITTI} &Mean error, RMSE & SL &PyTorch &\checkmark\\
					
					&SSR-Net~\cite{SSR-Net} &PRL &Cross-View &ResNet	&$\mathcal{L}_2$ loss & MS-COCO\cite{MS-COCO} &MAE & SSL &PyTorch &\checkmark\\
					
					&Abbas et al.~\cite{Abbas} &ICCVW &Cross-View &CNNs	&Softmax loss & CARLA\cite{CARLA} &AUC\cite{AUC}, Mean error & SL &TensorFlow &\checkmark\\
					
					&DR-GAN~\cite{DR-GAN} &TCSVT &Distortion &GANs	&Perceptual loss & MS-COCO\cite{MS-COCO} &PSNR, SSIM & SL &TensorFlow &\checkmark\\
					
					&STD~\cite{STD} &TCSVT &Distortion &GANs+CNNs	&Perceptual loss & MS-COCO\cite{MS-COCO} &PSNR, SSIM & SL &TensorFlow &\checkmark\\
					
					&Deep360Up~\cite{Deep360Up} &VR &Standard &DenseNet	&Log-cosh loss\cite{Log-cosh} & SUN360\cite{SUN360} &Mean error & SL &- &\checkmark\\
					
					&UnFishCor~\cite{UnFishCor} &JVCIR &Distortion &VGG	&$\mathcal{L}_1$ loss & Places2\cite{Places2} &PSNR, SSIM & USL &TensorFlow &\checkmark\\
					
					&BlindCor~\cite{BlindCor} &CVPR &Distortion &U-Net	&$\mathcal{L}_2$ loss & Places2\cite{Places2} &MSE & SL &PyTorch &\checkmark\\
					
					&RSC-Net~\cite{RSC-Net} &CVPR &Distortion &ResNet	&$\mathcal{L}_1$ loss & KITTI\cite{KITTI} &Mean error & SL &PyTorch &\checkmark\\
					
					&Xue et al.~\cite{Xue} &CVPR &Distortion &ResNet	&$\mathcal{L}_2$ loss & Wireframes\cite{Wireframes}, SUNCG\cite{SUNCG} &PSNR, SSIM, RPE & SL &PyTorch &\checkmark\\
					
					&Zhao et al.~\cite{Zhao} &ICCV &Distortion &VGG+U-Net	&$\mathcal{L}_1$ loss & Self-constructed+BU-4DFE\cite{BU-4DFE} &Mean error &SL &- &\checkmark\\

					&NeurVPS~\cite{zhou2019neurvps} &NeurIPS &Standard &CNNs	&Binary cross entropy, chamfer-$\mathcal{L}_2$ loss &ScanNet~\cite{ScanNet}, SU3~\cite{SU3} &Angle accuracy &SL &PyTorch &\\

     
					
					\hline
					\hline
					\multirow{1}{*}{\rotatebox{0}{\textbf{2020}}}
					% &1&
					&Sha et al.~\cite{Sha} &CVPR &Cross-View &U-Net	& Cross-entropy loss &World Cup 2014\cite{homayounfar2017sports} &IoU & SL &TensorFlow &\\
				    
				    &Lee et al.~\cite{Lee} &ECCV &Standard &PointNet + CNNs	& Cross-entropy loss &Google Street View\cite{googleStreet}, HLW\cite{HLW} &Mean error, AUC\cite{AUC} & SL &- &\\
				    
				    &MisCaliDet~\cite{MisCaliDet} &ICRA &Distortion &CNNs	& $\mathcal{L}_2$ loss &KITTI\cite{KITTI} &MSE & SL &TensorFlow &\checkmark\\
				    
				    &DeepPTZ~\cite{DeepPTZ} &WACV &Distortion &Inception-V3	& $\mathcal{L}_1$ loss &SUN360\cite{SUN360} &Mean error & SL &PyTorch &\checkmark\\
				    
				    &MHN~\cite{MHN} &CVPR &Cross-View &VGG	&Cross-entropy loss &MS-COCO\cite{MS-COCO}, Self-constructed &MAE & SL &TensorFlow &\checkmark\\
				    
				    &Davidson et al.~\cite{Davidson} &ECCV &Standard &FCN	&Dice loss &SUN360\cite{SUN360} &Accuracy &SL &- &\checkmark\\
				    
				    &CA-UDHN~\cite{CA-UDHN} &ECCV &Cross-View &FCN + ResNet	&Triplet loss &Self-constructed &MSE &USL &PyTorch &\\
				    
				    &DeepFEPE~\cite{DeepFEPE} &IROS &Standard &VGG + PointNet	&$\mathcal{L}_2$ loss &KITTI\cite{KITTI}, ApolloScape\cite{Apolloscape} &Mean error &SL &PyTorch &\\
				    
				    &DDM~\cite{DDM} &TIP &Distortion &GANs	&$\mathcal{L}_1$ loss &MS-COCO\cite{MS-COCO} &PSNR, SSIM &SL &TensorFlow &\checkmark\\
				    
				    &Li et al.~\cite{Li} &TIP &Distortion &CNNs	&Cross-entropy, $\mathcal{L}_1$ loss &CelebA\cite{CelebA} &Cosine distance &SL &- &\checkmark\\
				    
				    &PSE-GAN~\cite{PSE-GAN} &ICPR &Distortion &GANs	&$\mathcal{L}_1$, WGAN loss &Place2\cite{Places2} &MSE &SL &- &\checkmark\\
				    
				    &RDC-Net~\cite{RDC-Net} &ICIP &Distortion &ResNet	&$\mathcal{L}_1$, $\mathcal{L}_2$ loss &ImageNet\cite{ImageNet} &PSNR, SSIM &SL &PyTorch &\checkmark\\
				    
				    &FE-GAN~\cite{FE-GAN} &ICASSP &Distortion &GANs	&$\mathcal{L}_1$, GAN loss &Wireframe\cite{Wireframes}, LSUN\cite{LSUN} &PSNR, SSIM, RMSE &SSL &PyTorch &\checkmark\\
				    
				    &RDCFace~\cite{RDCFace} &CVPR &Distortion &ResNet	&Cross-entropy, $\mathcal{L}_2$ loss &IMDB-Face\cite{IMDB-Face} &Accuracy &SL &- &\checkmark\\
				    
				    &LaRecNet~\cite{LaRecNet} &arXiv &Distortion &ResNet	&$\mathcal{L}_2$ loss &Wireframes\cite{Wireframes}, SUNCG\cite{SUNCG} &PSNR, SSIM, RPE &SL &PyTorch &\checkmark\\
				    
				    &Baradad et al.~\cite{Baradad} &CVPR &Standard &CNNs	&$\mathcal{L}_2$ loss &ScanNet\cite{ScanNet}, NYU\cite{NYU}, SUN360\cite{SUN360} &Mean error, RMS &SL &PyTorch &\\
				    
				    &Zheng et al.~\cite{Zheng} &CVPR &Standard &CNNs	&$\mathcal{L}_1$ loss &FocaLens\cite{FocaLens} &Mean error, PSNR, SSIM &SL &- &\checkmark\\
				    
				    &Zhu et al.~\cite{Zhu} &ECCV &Standard &CNNs + PointNet	&$\mathcal{L}_1$ loss &SUN360\cite{SUN360}, MS-COCO\cite{MS-COCO} &Mean error, Accuracy &WSL &PyTorch &\checkmark\\
				    
				    &DeepUnrollNet~\cite{DeepUnrollNet} &CVPR &Distortion &FCN	&$\mathcal{L}_1$, perceptual, total variation loss &Carla-RS\cite{DeepUnrollNet}, Fastec-RS\cite{DeepUnrollNet}  &PSNR, SSIM &SL &PyTorch &\checkmark\\
				    
				    &RGGNet~\cite{yuan2020rggnet} &RAL &Cross-Sensor &ResNet	&Geodesic distance loss &KITTI\cite{KITTI}  &MSE, MSEE, MRR &SL &TensorFlow &\checkmark\\
				    
				    &CalibRCNN~\cite{shi2020calibrcnn} &IROS &Cross-Sensor &RNNs	&$\mathcal{L}_2$, Epipolar geometry loss &KITTI~\cite{KITTI}  &MAE &SL &TensorFlow &\checkmark\\

				    &SSI-Calib~\cite{zhu2020online} &ICRA &Cross-Sensor &CNNs	&$\mathcal{L}_2$ loss &Pascal VOC 2012~\cite{pascal-voc-2012}  &Mean/standard deviation &SL &TensorFlow &\checkmark\\

				    &SOIC~\cite{wang2020soic} &arXiv &Cross-Sensor &ResNet + PointRCNN	& Cost function &KITTI~\cite{KITTI}  &Mean error &SL &- &\\        

				    &NetCalib~\cite{wu2021netcalib} &ICPR &Cross-Sensor &CNNs	&$\mathcal{L}_1$ loss &KITTI~\cite{KITTI}  &MAE &SL &PyTorch &\checkmark\\

				    &SRHEN~\cite{SRHEN} &ACM-MM &Cross-View &CNNs	&$\mathcal{L}_2$ loss &MS-COCO~\cite{MS-COCO}, SUN397~\cite{SUN360}  &MACE &SL &- &\checkmark\\
                        
				   
					\hline
					\hline
					\multirow{1}{*}{\rotatebox{0}{\textbf{2021}}}
					% &1&
					&StereoCaliNet~\cite{StereoCaliNet} &TCI &Standard &U-Net	&$\mathcal{L}_1$ loss &TAUAgent\cite{TAUAgent}, KITTI\cite{KITTI} &Mean error & SL &PyTorch &\checkmark\\
					
					&CTRL-C~\cite{CTRL-C} &ICCV &Standard &Transformer	&Cross-entropy, $\mathcal{L}_1$ loss &Google Street View\cite{googleStreet}, SUN360\cite{SUN360} &Mean error, AUC\cite{AUC} & SL &PyTorch &\checkmark\\
					
				   &Wakai et al.~\cite{Wakai} &ICCVW &Distortion &DenseNet	&Smooth $\mathcal{L}_1$ loss &StreetLearn\cite{StreetLearn} &Mean error, PSNR, SSIM & SL &- &\checkmark\\
				   &OrdianlDistortion~\cite{OrdianlDistortion} &TIP &Distortion &CNNs	&Smooth $\mathcal{L}_1$ loss & MS-COCO\cite{MS-COCO} &PSNR, SSIM, MDLD & SL &TensorFlow &\checkmark\\
				   
				   &PolarRecNet~\cite{PolarRecNet} &TCSVT &Distortion &VGG + U-Net	&$\mathcal{L}_1$, $\mathcal{L}_2$ loss & MS-COCO\cite{MS-COCO}, LMS\cite{LMS} &PSNR, SSIM, MSE & SL &PyTorch &\checkmark\\
				   
				   &DQN-RecNet~\cite{DQN-RecNet} &PRL &Distortion &VGG	&$\mathcal{L}_2$ loss & Wireframes\cite{Wireframes} &PSNR, SSIM, MSE & RL &PyTorch &\checkmark\\
				   
				   &Tan et al.~\cite{Tan} &CVPR &Distortion &U-Net &$\mathcal{L}_2$ loss & Self-constructed &Accuracy & SL &PyTorch & \\
				   
				   &PCN~\cite{PCN} &CVPR &Distortion &U-Net &$\mathcal{L}_1$, $\mathcal{L}_2$, GAN loss & Place2\cite{Places2} &PSNR, SSIM, FID, CW-SSIM & SL &PyTorch &\checkmark \\
				   
				   &DaRecNet~\cite{DaRecNet} &ICCV &Distortion &U-Net &Smooth $\mathcal{L}_1$, $\mathcal{L}_2$ loss & ADE20K\cite{ADE20K} &PSNR, SSIM & SL &PyTorch &\checkmark \\
				   
				   &DLKFM~\cite{DLKFM} &CVPR &Cross-View &Siamese-Net &$\mathcal{L}_2$ loss & MS-COCO\cite{MS-COCO}, Google Earth, Google Map &MSE & SL &TensorFlow &\checkmark \\
				   
				   &LocalTrans~\cite{LocalTrans} &ICCV &Cross-View &Transformer &$\mathcal{L}_1$ loss & MS-COCO\cite{MS-COCO} &MSE, PSNR, SSIM & SL &PyTorch &\checkmark \\
				   
				   &BasesHomo~\cite{BasesHomo} &ICCV &Cross-View &ResNet &Triplet loss & CA-UDHN\cite{CA-UDHN} &MSE & USL &PyTorch & \\
				   &ShuffleHomoNet~\cite{ShuffleHomoNet} &ICIP &Cross-View &ShuffleNet &$\mathcal{L}_2$ loss & MS-COCO\cite{MS-COCO} &RMSE & SL &TensorFlow &\checkmark \\
				   
				   &DAMG-Homo~\cite{DAMG-Homo} &TCSVT &Cross-View &CNNs &$\mathcal{L}_1$ loss & MS-COCO\cite{MS-COCO}, UDIS\cite{UDIS} &RMSE, PSNR, SSIM & SL &TensorFlow &\checkmark \\
				   
				   &SA-MobileNet~\cite{SA-MobileNet} &BMVC &Standard &MobileNet &Cross-entropy loss& SUN360\cite{SUN360}, ADE20K\cite{ADE20K}, NYU\cite{NYU} &MAE, Accuracy & SL &TensorFlow &\checkmark \\
				   
				   &SPEC~\cite{SPEC} &ICCV &Standard &ResNet &Softargmax-$\mathcal{L}_2$ loss&Self-constructed &W-MPJPE, PA-MPJPE & SL &PyTorch &\checkmark \\
				   
				   &DirectionNet~\cite{DirectionNet} &CVPR &Standard &U-Net &Cosine similarity loss &InteriorNet\cite{InteriorNet}, Matterport3D\cite{Matterport3D}&Mean and median error  & SL &TensorFlow &\checkmark \\
				   
				   &JCD~\cite{JCD} &CVPR &Distortion &FCN &Charbonnier\cite{Charbonnier}, perceptual loss &BS-RSCD \cite{JCD}, Fastec-RS
                   \cite{DeepUnrollNet}&PSNR, SSIM, LPIPS  & SL &PyTorch & \\
                   
                   &LCCNet~\cite{lv2021lccnet} &CVPRW &Cross-Sensor &CNNs &Smooth $\mathcal{L}_1$, $\mathcal{L}_2$ loss &KITTI\cite{KITTI} &MSE  & SL &PyTorch &\checkmark \\
                   
                   &CFNet~\cite{lv2021cfnet} &Sensors &Cross-Sensor &FCN &$\mathcal{L}_1$, Charbonnier\cite{Charbonnier} loss &KITTI\cite{KITTI}, KITTI-360\cite{liao2022kitti} &MAE, MSEE, MRR  & SL &PyTorch &\checkmark \\

                   &Fan\etal~\cite{fan2021inverting} &ICCV &Distortion &U-Net &$\mathcal{L}_1$, perceptual loss &Carla-RS~\cite{DeepUnrollNet}, Fastec-RS~\cite{DeepUnrollNet} &PSNR, SSIM, LPIPS  & SL &PyTorch & \\

                   &SUNet~\cite{SUNet} &ICCV &Distortion &DenseNet + ResNet &$\mathcal{L}_1$, perceptual loss &Carla-RS~\cite{DeepUnrollNet}, Fastec-RS~\cite{DeepUnrollNet} &PSNR, SSIM  & SL &PyTorch & \\

                   &SemAlign~\cite{liu2021semalign} &IROS &Cross-Sensor &CNNs & Semantic alignment loss &KITTI~\cite{KITTI} &Mean/median rotation errors & SL &PyTorch &\checkmark\\
       
				   \hline
				   \hline
				   \multirow{1}{*}{\rotatebox{0}{\textbf{2022}}}
					% &1&
				   &DVPD~\cite{DVPD} &CVPR &Standard &CNNs	&Cross-entropy loss &SU3\cite{SU3}, ScanNet\cite{ScanNet}, YUD\cite{YUD}, NYU\cite{NYU} &Accuracy, AUC\cite{AUC} & SL &PyTorch &\checkmark\\
				   
				   &Fang et al.~\cite{Fang} &ICRA &Standard &CNNs	&$\mathcal{L}_2$ loss &KITTI\cite{KITTI}, EuRoC\cite{EuRoC}, OmniCam\cite{OmniCam} &MRE, RMSE & SSL &PyTorch &\\
				   
				   &CPL~\cite{CPL} &ICASSP &Standard &Inception-V3	&$\mathcal{L}_1$ loss &CARLA\cite{CARLA}, CyclistDetection\cite{CyclistDetection} &MAE & SL &TensorFlow &\checkmark\\
				   
				   &IHN~\cite{IHN} &CVPR &Cross-View  &Siamese-Net	&$\mathcal{L}_1$ loss &MS-COCO\cite{MS-COCO}, Google Earth, Google Map &MACE & SL &PyTorch &\checkmark\\
				   
				   &HomoGAN~\cite{HomoGAN} &CVPR &Cross-View  &GANs	&Cross-entropy, WGAN loss &CA-UDHN\cite{CA-UDHN} &Mean error & USL &PyTorch &\checkmark\\
				   
				   &SS-WPC~\cite{SS-WPC} &CVPR &Distortion  &Transformer	&Cross-entropy, $\mathcal{L}_1$ loss &Tan et al.\cite{Tan} &Accuracy & Semi-SL &PyTorch &\\
				   
				   &AW-RSC~\cite{AW-RSC} &CVPR &Distortion  &CNNs	&Charbonnier\cite{Charbonnier}, perceptual loss &Self-constructed, FastecRS\cite{DeepUnrollNet} &PSNR, SSIM &SL &PyTorch &\\
				   
				   &EvUnroll~\cite{EvUnroll} &CVPR &Distortion  &U-Net	&Charbonnier, perceptual, TV loss &Self-constructed, FastecRS\cite{DeepUnrollNet} &PSNR, SSIM, LPIPS &SL &PyTorch &\\
				   
				   &Do et al.~\cite{Do} &CVPR &Standard  &ResNet&$\mathcal{L}_2$, Robust angular \cite{RobustAngular} loss &Self-constructed, 7-SCENES\cite{7-SCENES} &Median error, Recall &SL &PyTorch &\\
				   
				   &DiffPoseNet~\cite{DiffPoseNet} &CVPR &Standard  &CNNs + LSTM&$\mathcal{L}_2$ loss &TartanAir\cite{TartanAir}, KITTI\cite{KITTI}, TUM-RGBD\cite{TUM-RGBD} &PEE, AEE\cite{AEE} &SSL &PyTorch &\\
				   
				   &SceneSqueezer~\cite{SceneSqueezer} &CVPR &Standard  &Transformer&$\mathcal{L}_1$ loss &RobotCar Seasons\cite{RobotCar}, Cambridge Landmarks\cite{Cambridge_Landmarks}  &Mean error, Recall\cite{AEE} &SL &PyTorch &\\
				   
				   &FocalPose~\cite{FocalPose} &CVPR &Standard  &CNNs&$\mathcal{L}_1$, Huber loss &Pix3D\cite{Pix3D}, CompCars\cite{StanfordCars}, StanfordCars\cite{StanfordCars}  &Median error, Accuracy &SL &PyTorch &\\
				   
				   &DXQ-Net~\cite{jing2022dxq} &arXiv &Cross-Sensor  &CNNs + RNNs&$\mathcal{L}_1$, geodesic loss &KITTI\cite{KITTI}, KITTI-360\cite{liao2022kitti}  &MSE &SL &PyTorch &\checkmark\\
				   
				   &SST-Calib~\cite{SST-Calib} &ITSC &Cross-Sensor  &CNNs &$\mathcal{L}_2$ loss &KITTI\cite{KITTI}  &QAD, AEAD &SL &PyTorch &\checkmark\\
				   &CCS-Net~\cite{zhang2022learning} &IROS &Distortion  &U-Net&$\mathcal{L}_1$ loss &TUM-RGBD\cite{TUM-RGBD} &MAE, RPE &SL &PyTorch &\checkmark\\
				   
				   &FishFormer~\cite{FishFormer} &arXiv &Distortion  &Transformer&$\mathcal{L}_2$ loss &Place2\cite{Places2}, CelebA\cite{CelebA}  &PSNR, SSIM, FID &SL &PyTorch &\checkmark\\
				   
                  &SIR~\cite{SIR} &TIP &Distortion &ResNet &$\mathcal{L}_1$ loss & ADE20K\cite{ADE20K}, WireFrames\cite{Wireframes}, MS-COCO\cite{MS-COCO} &PSNR, SSIM & SSL &PyTorch &\checkmark \\

				   &ATOP~\cite{ATOP} &TIV &Cross-Sensor  &CNNs &Cross entropy loss &Self-constructed + KITTI\cite{KITTI}  &RRE, RTE &SL &- &\\

				   &FusionNet~\cite{wang2022fusionnet} &ICRA &Cross-Sensor  &CNNs+PointNet &$\mathcal{L}_2$ loss &KITTI\cite{KITTI}  &MAE &SL &PyTorch &\checkmark\\

				   &RKGCNet~\cite{RKGCNet} &TIM &Cross-Sensor  &CNNs+PointNet &$\mathcal{L}_1$ loss &KITTI\cite{KITTI}  &MSE &SL &PyTorch &\checkmark\\

                    &GenCaliNet~\cite{GenCaliNet} &ECCV &Distortion &DenseNet	&$\mathcal{L}_2$ loss &StreetLearn\cite{StreetLearn}, SP360\cite{SP360} &MAE, PSNR, SSIM & SL &- &\checkmark\\
       
				   &Liu et al.~\cite{Liu} &TPAMI &Cross-View &ResNet&Triplet loss &Self-constructed  &MSE, Accuracy &USL &PyTorch &\\
				   
				
				\hline
				\end{tabular}
			}
		\end{threeparttable}
	\end{table*}
	
	
	
	
	
	
	
	
	
	
	
	
	


\section{Horizontal transport density on $\mathbb{H}^n$}\label{sectrandens}
In this section we introduce the notion of horizontal transport density, extending to the Heisenberg group the presentation provided in \cite{Santambrogio2}.  A horizontal transport density is a measure representing the amount of transport taking place along geodesics in each region of $\mathbb{H}^n$. In particular, we study conditions under which transport densities are Lebesgue absolutely continuous w.r.t. the Haar measure of the group, with $L^p$ density.


Let $\mu,\nu\in\mathcal{P}_c(\mathbb{H}^n)$ and let us fix a selection of geodesics
\begin{equation}\label{31luglio}
    S:\mathbb{H}^n\times\mathbb{H}^n \rightarrow \textnormal{Geo}(\mathbb{H}^n), 
\end{equation}
$S(x,y)=\sigma_{x,y}\in\textnormal{Geo}(\mathbb{H}^n)$, that is $\gamma$-measurable for any $\gamma\in\Pi_1(\mu,\nu)$, according to \eqref{19marzo1}. One can associate with any optimal transport plan $\gamma\in\Pi_1(\mu,\nu)$ a positive and finite Radon measure $a_\gamma\in\mathcal{M}_+(\mathbb{H}^n)$, defined as 
\begin{equation}\label{transport density}
	\int_{\mathbb{H}^n} \phi(x)da_\gamma(x):=\int_{\mathbb{H}^n\times\mathbb{H}^n}L_\phi(\sigma_{x,y})d\gamma(x,y),\quad\forall\phi\in C_c(\mathbb{H}^n,\mathbb{R}_+).
\end{equation}
Here $L_\phi(\sigma_{x,y})$ denotes the horizontal length of $\sigma_{x,y}$, weighted by $\phi$, see \eqref{3agosto} for its definition. The total mass of $a_\gamma$ satisfies
\begin{equation*}
    a_\gamma(\mathbb{H}^n)\leq\min_{\tilde\gamma\in\Pi(\mu,\nu)}\int_{\mathbb{H}^n\times\mathbb{H}^n}d_{CC}(x,y)d\tilde\gamma(x,y).
\end{equation*}

Moreover, the measure $a_\gamma$ is a compactly supported measure, see \cite{circelli2024continuous}. 
This measure is generally called \textit{horizontal transport density}.  By definition, it also follows that if $A$ is a Borel set, then
\begin{equation}\label{densitylength}
	a_\gamma(A)=\int_{\mathbb{H}^n\times\mathbb{H}^n}\mathcal{H}^1(A\cap S(x,y))d\gamma(x,y).
\end{equation}

Let us remark that if either $\mu\ll\mathcal{L}^{2n+1}$, or $\nu\ll\mathcal{L}^{2n+1}$, then $a_\gamma$ does not depend on the fixed selection $S$. See Proposition \ref{pi1.1}.

\subsection{Absolute continuity of horizontal transport densities}
The first goal is to prove the existence of at least one horizontal transport density that is absolutely continuous w.r.t. the Haar measure of the group.

Given $\gamma\in\Pi_1(\mu,\nu)$, we denote by $\mu_t$ the displacement interpolation between $\mu$ and $\nu$  
\begin{equation*}\label{interpolation}
	\mu_t:=((S_t)_\# \gamma)_{t\in [0,1]}.
\end{equation*}
Hence, the horizontal transport density $a_\gamma$ may be written as 
\begin{equation*}\label{density2}
	a_\gamma=\int_0^1(S_t)_\# (d_{CC}\ \gamma)dt,
\end{equation*}
where $d_{CC}\ \gamma$ is a positive Borel measure on $\mathbb{H}^n\times\mathbb{H}^n$.
Since $\mu$ and $\nu$ have bounded support, then there exists $C>0$ such that $d_{CC}(x,y)\leq C$, for any $(x,y)\in\textnormal{supp}(\gamma)$ and hence 
\begin{equation}
	\label{density3}
	a_\gamma\leq C\int_0^1\mu_t dt.
\end{equation}
In order to prove that $a_\gamma$ is absolutely continuous w.r.t. $\mathcal{L}^{2n+1}$, it is enough to prove that $\mu_t$ is absolutely continuous w.r.t. $\mathcal{L}^{2n+1}$, for almost every $t\in[0,1]$. In this way we would get that, whenever $\mathcal{L}^{2n+1}(A)=0$, then 
\begin{equation}\label{density4}
a_\gamma(A)\leq C\int_0^1\mu_t(A)dt=0.
\end{equation}
\\

We introduce now  the following lemma which guarantees that minimizing geodesics arising in 
optimal plans cannot intersect at intermediate points. This result will be useful in the proof of Theorem \ref{absolutecont}.

\begin{lemma} \label{pi1.3}
	Let $\gamma\in \Pi_1(\mu,\nu)$. Then $\gamma$ is concentrated on a set $\Gamma$ such that $\forall(x,y),(x',y')\in \Gamma$ with $(x,y)\not=(x',y')$, if two transport rays between these two pairs of points intersect at an interior point $z\in\mathbb{H}^n$, then all points $x$, $x'$, $y$, $y'$ and $z$ lie on the same transport ray.
	Moreover if $\gamma\in\Pi_2(\mu,\nu)$, then either $x\leq x'\leq z\leq y\leq y'$ or $x'\leq x\leq z\leq y'\leq y$.
\end{lemma}
\begin{proof}
	We first recall that \eqref{c-CM} reads as
	\begin{equation}\label{aaa}
		d_{CC}(x,y)+d_{CC}(x',y')\leq d_{CC}(x,y')+d_{CC}(x',y),
	\end{equation}
	$\forall (x,y),(x',y')\in\Gamma$.
	Let $\sigma:[0,d_{CC}(x,y)]\rightarrow\mathbb{H}^n$ be a geodesic between $x$ and $y$, $\tilde\sigma:[0,d_{CC}(x',y')]\rightarrow\mathbb{H}^n$ a geodesic between $x'$ and $y'$, $z\in\sigma(0,d_{CC}(x,y))\cap\tilde\sigma(0,d_{CC}(x',y'))$, so $z=\sigma(d_{CC}(x,z))=\tilde\sigma(d_{CC}(x',z))$. We denote by $\alpha$ the curve between $x$ and $y'$ defined in the following way:
    \begin{displaymath}
		\alpha(t):=\begin{cases}
			\sigma\left(\frac{d_{CC}(x,z)}{d_{CC}(x',z)}t\right),\quad &\textnormal{if}\,\ t\in[0,d_{CC}(x',z)],\\
			\tilde\sigma(t),\quad &\textnormal{if}\,\ t\in(d_{CC}(x',z),d_{CC}(x',y')].
		\end{cases}
    \end{displaymath}
    We will prove that $\alpha$ is geodesic between $x$ and $y'$. Indeed, otherwise we would have 
	\begin{multline}\label{aab}
		d_{CC}(x,y')<\ell_H(\alpha)=\ell_H(\alpha_{|[0,d_{CC}(x',z)]})+\ell_H(\alpha_{|[d_{CC}(x',z),d_{CC}(x',y')]})\\=d_{CC}(x,z)+d_{CC}(z,y').
	\end{multline}
	Since $z$ lies on both the geodesic between $x$ and $y$ and the geodesic between $x'$ and $y'$, it follows that 
	\begin{equation}\label{aac}
		\begin{cases}
			d_{CC}(x,y)=d_{CC}(x,z)+d_{CC}(z,y);\\
			d_{CC}(x',y')=d_{CC}(x',z)+d_{CC}(z,y').
		\end{cases}	
	\end{equation}
	By replacing (\ref{aac}) in (\ref{aab}), we obtain:
	\begin{equation}\label{aad}
		d_{CC}(x,y')+d_{CC}(z,y)+d_{CC}(x',z)<d_{CC}(x,y)+d_{CC}(x',y').
	\end{equation}
	By the triangle inequality follows that:
	\begin{displaymath}
		d_{CC}(x',y)\leq d_{CC}(x',z)+d_{CC}(z,y),
	\end{displaymath}
	and then, by replacing this last inequality in (\ref{aad}), we obtain
	\begin{displaymath}
		d_{CC}(x,y')+d_{CC}(x',y)<d_{CC}(x,y)+d_{CC}(x',y'),
	\end{displaymath}
	and this contradicts (\ref{aaa}). It follows that $\tilde\sigma$ and $\alpha$ are geodesics that coincide on the non-trivial interval $[d_{CC}(x',z),d_{CC}(x',y')]$. Since $\mathbb{H}^n$ is non-branching, this implies that $\tilde\sigma$ and $\alpha$ are sub-arcs of the same geodesic, namely $\alpha$ if $d_{CC}(x',z)\leq d_{CC}(x,z)$ and $\tilde\sigma$ otherwise, on which all points $x,x',z,y'$ lie.
	
	The thesis follows from Proposition \ref{monotone}.
\end{proof}

Given a map $T:\mathbb{H}^n\rightarrow\mathbb{H}^n$, from now on we will denote by
\begin{equation*}
	T_t:=S_t\circ(\text{Id}\otimes T):\mathbb{H}^n\rightarrow\mathbb{H}^n,\quad \forall t\in[0,1]
\end{equation*}
where $T_t(x)$ is the point at distance $td_{CC}(x, T(x))$ from $x$ on the selected geodesic $S(x,T(x))$ between $x$ and $T(x)$. In particular if $\gamma\in\Pi_1(\mu,\nu)$ is induced by a transport map, i.e. is of the form $\gamma:=(\text{Id}\otimes T)_\#\mu\in\Pi_1(\mu,\nu)$, then
\begin{equation*}
    \mu_t={(T_{t})}_{\#}\mu. 
\end{equation*}


The previous lemma allows to prove the following result.

\begin{prop}\label{absolutecontinterp}
If $\mu\ll\mathcal{L}^{2n+1}$, then there exists an optimal transport plan $\gamma\in\Pi_2(\mu,\nu)$ such that the measure
\begin{equation}\label{absolutecontinter1}
	\mu_t:=(S_t)_\#\gamma\ll\mathcal{L}^{2n+1},\quad \forall t\in[0,1).
\end{equation}
\end{prop}

\begin{proof}	
First we suppose that $\nu$ is finitely atomic, with atoms $(y^i)_{i=1}^M$. Let $\gamma\in\Pi_2(\mu,\nu)\subset\Pi_1(\mu,\nu)$, as in Theorem \ref{mainthmbis}, which is monotone in the sense of \eqref{orderrelation} and induced by a transport map $T$. Let us denote by $\Gamma\subseteq\mathbb{H}^n\times\mathbb{H}^n$ the set $\gamma$ is concentrated on and \eqref{var1} and \eqref{var2} hold.
	
We denote by $\Omega_i:=T^{-1}(\{y^i\})\cap \pi_1(\Gamma)$: obviously these sets are mutually disjoint and $\mu(\Omega)=1$, where $\Omega:=\bigcup_{i=1}^M\Omega_i$.
	
Now we denote by $\Omega_i(t):=T_t(\Omega_i)$: if we fix $t\in[0,1)$, then $\Omega_i(t)\cap\Omega_j(t)=\emptyset$ for every $i,j=1,\ldots,M$. Indeed, if $\exists\ z\in\Omega_i(t)\cap\Omega_j(t)$ then $\exists\ x^i\in \Omega_i$ and $x^j\in\Omega_j$ such that $(x^i,y^i), (x^j,y^j)\in \Gamma, (x^i,y^i)\not=(x^j,y^j)$ and the geodesics between these two pairs of points intersect at $z$. Since $\gamma\in\Pi_2(\mu,\nu)$, by Theorem \ref{pi1.3} we can suppose that $x^i,y^i,x^j,y^j,z$ belong to the same unit-speed geodesic and $x^i\leq x^j\leq z\leq y^i\leq y^j$. In particular this means, on the one hand, that $td_{CC}(x^i,y^i)=d_{CC}(x_i,z)\geq d_{CC}(x^j,z)=td_{CC}(x^j,y^j)$, hence $d_{CC}(x^i,y^i)\geq d_{CC}(x^j,y^j)$. On the other hand $(1-t)d_{CC}(x^i,y^i)=d_{CC}(z,y_i)\leq d_{CC}(z,y^j)=(1-t)d_{CC}(x^j,y^j)$, hence $d_{CC}(x^i,y^i)\leq d_{CC}(x^j,y^j)$. It follows that $d_{CC}(x^i,y^i)= d_{CC}(x^j,y^j)$ and hence $d_{CC}(x^i,z)=d_{CC}(x^j,z)$ and $d_{CC}(z,y^i)=d_{CC}(z,y^j)$, which in turn implies that $x^j=x^i$ and $y^i=y^j$ and gives a contradiction. However it may happen that $x^i=y^i$ or $x^j=y^j$. Let us suppose that $x^i=y^i=z$: the same computation above implies that $d_{SR}(x^j,y^j)=0$, which in turns implies that $y^i=y^j$ and gives a contradiction.
	
Remember also that $\mu$ is absolutely continuous and hence there exists a correspondence $\varepsilon\mapsto\delta=\delta(\varepsilon)$ such that 
\begin{equation*}
	\mathcal{L}^{2n+1}(A)<\delta(\varepsilon)\Rightarrow\mu(A)<\varepsilon.
\end{equation*}

Let $A\subset\mathbb{H}^n$ be a Borel set, $t\in[0,1)$, then $\mu_t:=(T_t)_{\#}\mu$ is concentrated on $T_t(\text{supp}(\mu))$ and
\begin{equation*}
	\mu_t(A)=\sum_{i=1}^{M}\mu_t(A\cap\Omega_i(t))=\sum_{i=1}^M\mu(T_t^{-1}(A\cap\Omega_i(t)))=\mu\left(\bigcup_{i=1}^M(T_t^{-1}(A\cap\Omega_i(t)))\right),
\end{equation*}
since the sets $T_t^{-1}(A\cap\Omega_i(t))\subseteq\Omega_i$ are disjoint.
We observe that for any $x\in\Omega_i$, $T_t(x)=S_t(x,y^i)$, hence by \eqref{MCP} follows that
\begin{equation*}
    \mathcal{L}^{2n+1}(U)\leq\frac{1}{(1-t)^{2n+3}}\mathcal{L}^{2n+1}(T_t(U)),
\end{equation*}
for any $U\subset\Omega_i$. This in turn implies that
\begin{equation*}
    \mathcal{L}^{2n+1}(T_t^{-1}(A\cap\Omega_i(t)))\leq\frac{1}{(1-t)^{2n+3}}\mathcal{L}^{2n+1}(A\cap \Omega_i(t)),
\end{equation*}
and so 
\begin{equation*}
    \mathcal{L}^{2n+1}\left(\bigcup_{i=1}^M(T_t^{-1}(A\cap\Omega_i(t)))\right)\leq\frac{1}{(1-t)^{2n+3}}\mathcal{L}^{2n+1}(A).
\end{equation*}
Hence, it is sufficient to suppose that $\mathcal{L}^{2n+1}(A)<(1-t)^{2n+3}\delta(\varepsilon)$ to get $\mu_t(A)<\varepsilon$.	This proves that $\mu_t\ll\mathcal{L}^{2n+1}$.
	
Now, if $\nu$ is not finitely atomic, we can take a sequence $(\nu_k)_{k\in\mathbb{N}}$ of atomic measures weakly converging to $\nu$, for instance as in Lemma \ref{optpi2}. For any $k\in\mathbb{N}$, we consider an optimal transport plan $\gamma_k\in\Pi_2(\mu,\nu_k)$ as in the first part of the proof. Hence, the sequence $(\gamma_k)_{k\in\mathbb{N}}$ weakly converges to some optimal transport plan $\gamma\in\Pi_2(\mu,\nu)$; moreover the sequence $(\mu_t^k)_{k\in\mathbb{N}}$ weakly converges to the corresponding $\mu_t:=(S_t)_{\#}\gamma$, thanks to Proposition \ref{pi1.1} and \cite[Lemma 7.3]{DePascale2}. Take a set $A$ such that $\mathcal{L}^{2n+1}(A)<(1-t)^{2n+3}\delta(\varepsilon)$. Since the Lebesgue measure is regular, $A$ is included in an open set $B$ such that $\mathcal{L}^{2n+1}(B)<(1-t)^{2n+3}\delta(\varepsilon)$. Hence $\mu_t^k(B)<\varepsilon,\forall k\in\mathbb{N}$. Passing to the limit and using Portmanteau's Theorem, see \cite[Theorem 2.1]{Billingsley}, we get 
$$\mu_t(A)\leq\mu_t(B)\leq\liminf_k \mu_t^k(B)\leq\varepsilon.$$
This proves that $\mu_t\ll\mathcal{L}^{2n+1}$.\end{proof}

Now we are able to find at least an optimal transport plan $\gamma\in\Pi_1(\mu,\nu)$ such that the interpolation measures $\mu_t$ constructed from $\gamma$ are absolutely continuous for $t<1$.
\\

\begin{teo}\label{absolutecont}
If $\mu\ll\mathcal{L}^{2n+1}$, then there exists an optimal transport plan $\gamma\in\Pi_2(\mu,\nu)$ such that the measure $a_{\gamma}\ll\mathcal{L}^{2n+1}$.
\end{teo}
\begin{proof}
Let $\gamma\in\Pi_2(\mu,\nu)$ satisfying \eqref{absolutecontinter1}. Then, the thesis follows immediately from \eqref{density4} applied to $a_{\gamma}$.
\end{proof}

Obviously the previous argument depends only on one of the two marginals and it is completely symmetric: if $\nu\ll\mathcal{L}^{2n+1}$, again one can get the existence of an optimal transport plan $\gamma\in\Pi_2(\mu,\nu)$ such that the associated horizontal transport density $a_{\gamma}$ is absolute continuous w.r.t. the $(2n+1)$-dimensional Lebesgue measure.

\subsection{$p$-summability of horizontal transport densities}

In this subsection we prove the existence of at least one horizontal transport density belonging to $L^p$, for some values of $p$.

From now on, given $\lambda\in\mathcal{M}_+(\mathbb{H}^n)$ we will write that $\lambda\in L^p$ if $\lambda\ll\mathcal{L}^{2n+1}$, with density $\rho\in L^p$. We will denote by $\|\lambda\|_p:=\|\rho\|_{L^p}$.

Let $\gamma\in\Pi_1(\mu,\nu)$ as in Theorem \ref{absolutecont}. From \eqref{density3} and the Minkowski inequality it follows that
\begin{equation}\label{Minkowski}
	\|a_\gamma\|_p\leq C\int_{0}^1\|\mu_t\|_pdt.
\end{equation}

In order to prove $p$-summability of $a_\gamma$, it is enough to estimate the $L^p$ norm of $\mu_t$ as a function of the variable $t$. This will be established in the following  theorem: 
\begin{prop}\label{summabilityinterp}
If $\mu\in L^p$, for some $p\in[1,\infty]$, then there exists an optimal transport plan $\gamma\in\Pi_2(\mu,\nu)$ such that $\mu_t:=(S_t)_{\#}\gamma\in L^p$ and 
\begin{equation}\label{bb3}
	\|\mu_t\|_p\leq (1-t)^{-(2n+3)/q}\|\mu\|_p,\quad \forall t\in[0,1),
\end{equation}
 where $q:=\frac{p}{p-1}$.
\end{prop}
\begin{proof}
Let us denote by $\rho$ the density of $\mu$ w.r.t. $\mathcal{L}^{2n+1}$. Let us consider first the discrete case: let us assume that the target measure $\nu$ is finitely atomic and let us denote by $(y^i)_{i=1,\ldots,M}$ its atoms. Let us consider an optimal transport plan $\gamma\in\Pi_2(\mu,\nu)$, as in the proof of Proposition \ref{absolutecontinterp}, concentrated on some set $\Gamma$. Since $\gamma$ is induced by a map $T$, we denote by $\Omega_i:=T^{-1}(\{y^i\})\cap\pi_1(\Gamma)$, for $i\in\{1,\ldots,M\}$, so that for $\gamma$-a.e. $(x,y)\in\Omega_i\times\mathbb{H}^n$, we have $y=y^i$. Let us consider the corresponding interpolation measures $\mu_t\ll\mathcal{L}^{2n+1}$ for every $t\in[0,1)$; moreover, for all $\phi\in C_c(\mathbb{H}^n,\mathbb{R}_+)$, by definition of push-forward we get that 
\begin{align*}
    \int\phi(x)d\mu_t(x)=&\sum_{i=1}^M\int_{\Omega_i}\phi(S_t(x,y_i))d\gamma(x,y_i)=\\
	=&\sum_{i=1}^M\int_{\Omega_i}\phi(T_t(x))d\mu(x).
\end{align*}
Let us fix $i\in\{1,\ldots,M\}$ and let us denote by $\rho_t$ the density of $\mu_t$ w.r.t. $\mathcal{L}^{2n+1}$ and by $\rho_t^i:={\rho_t}_{\lfloor \Omega_i}$. Let us take the change of variable $z=S_t(x,y_i)={T_t}_{\lfloor\Omega_i}(x)$. We know from Lemma \ref{pi1.3} and the disjointness of the sets $\Omega_i(t)$ that this map is injective. Then, for all $\phi\in C_c(\mathbb{H}^n,\mathbb{R}_+)$ we get 
\begin{align*}
	\int_{\Omega_i}\phi(x)d\mu^i_t(x)&=\int_{\Omega_i}\phi(T_t(x))\rho(x)dx=\\&=
		\int_{\Omega_i(t)}\phi(z)\rho(T_t^{-1}(z))|\det D_x(S_t(x,y^i))|^{-1}dz.
\end{align*}
Hence, we have that 
\begin{equation*}
    \rho_t^i(z)=\rho(T_t^{-1}(z))|\det D_x(S_t(x,y^i))|^{-1},\quad \text{for a.e. } z\in\Omega_i(t).
\end{equation*}
Consequently, we get 
\begin{align*}
	\|\rho_t^i\|^p_{L^p(\Omega_i(t))}&=\int_{\Omega_i(t)}\rho(T_t^{-1}(z))^p|\det D_x(S_t(x,y^i))|^{-p}dz=\\&=\int_{\Omega_i}\rho(x)^p|\det D_x(S_t(x,y^i))|^{1-p}dx.
\end{align*}
Hence from \eqref{det} it follows that
\begin{align*}
    \|\rho_t^i\|^p_{L^p(\Omega_i(t))}\leq (1-t)^{(1-p)(2n+3)}\|\rho\|^p_{L^p(\Omega_i)},\quad \forall i\in\{1,\ldots,M\}.
\end{align*}
Then, we have
\begin{equation}\label{discretepsumm}
	\|\mu_t\|_p\leq (1-t)^{-(2n+3)/q}\|\mu\|_p,\quad \forall t\in(0,1),
\end{equation}
where $q:=\frac{p}{p-1}$.


If $\nu$ is not finitely atomic, again we take a sequence $(\nu_k)_{k\in\mathbb{N}}$ of atomic measures weakly converging to $\nu$, for instance as in Lemma \ref{optpi2}. We consider a sequence $(\gamma_k)_{k\in\mathbb{N}}\subset\Pi_2(\mu,\nu_k)$ of optimal plans satisfying \eqref{discretepsumm}: this sequence weakly converges to an optimal plan $\gamma\in\Pi_2(\mu,\nu)$ and $\mu_t^k$ weakly converge to the corresponding $\mu_t:=(S_t)_{\#}{\gamma}$, see again Proposition \ref{pi1.1} and \cite[Lemma 7.3]{DePascale2}. Hence, we get that
\begin{equation*}
    \|\mu_t\|_p\leq\liminf_{k \rightarrow 0}\|\mu_t^k\|_p\leq (1-t)^{-(2n+3)/q}\|\mu\|_p.
\end{equation*}
\end{proof}

Now we are able to prove the following theorem.
\begin{prop}\label{summability1}
If $\mu\in L^p$, for some $p\in[1,\infty]$, the following results hold: if $p<\frac{2n+3}{2n+2}$, then there exists $\gamma\in\Pi_2(\mu,\nu)$ such that $a_{\gamma}\in L^p$; otherwise, there exists $\gamma\in\Pi_2(\mu,\nu)$ such that $a_{\gamma}\in L^s$, for $s<\frac{2n+3}{2n+2}$.

\end{prop}

\begin{proof}
Let $\gamma\in\Pi_2(\mu,\nu)$ satisfying \eqref{bb3}. Then, it follows from \eqref{Minkowski} applied to $a_{\gamma}$ that
\begin{equation*}
	\|a_{\gamma}\|_p\leq C\int_0^1 \|\mu_t\|_pdt\leq C\|\mu\|_p\int_0^1(1-t)^{-(2n+3)/q}dt.
\end{equation*}
The last integral is finite whenever $q>2n+3$, i.e. $p<\frac{2n+3}{2n+2}$.
	
If $p\geq\frac{2n+3}{2n+2}$ the thesis follows from the fact that any density in $L^p$ also belongs to any $L^s$ space for $s<p$.
\end{proof}


If also $\nu\in L^p$ then, by symmetry, one can find an optimal transport plan $\tilde\gamma\in\Pi_2(\mu,\nu)$, possibly different from the one in Proposition \ref{summabilityinterp}, such that $\tilde{\mu}_t:=\left(S_t\right)_\#\tilde\gamma\in L^p$ and it satisfies
\begin{equation}\label{bb1}
	\|\tilde{\mu}_t\|_p\leq t^{-(2n+3)/q}\|\nu\|_p,\quad \forall t\in(0,1].
\end{equation}

In the Euclidean setting  
 \cite[Theorem 3.18]{Santambrogiolibro} or \cite{Feldman2}, and in the more general Riemannian setting, see \cite{Feldman}, $\Pi_2(\mu,\nu)$ consists of a unique element, so that it is possible to glue together \eqref{bb3}, for $t\leq\frac{1}{2}$, and \eqref{bb1}, for $t\geq\frac{1}{2}$, and get the existence of a horizontal transport density in $L^p$. Unfortunately, this uniqueness result is still an open problem in the Heisenberg group, hence we cannot glue together \eqref{bb3} and \eqref{bb1}, and deduce anything about the summability of $a_\gamma$.




\section{Weighted distance induced by a $L^p$ density}

In the previous section we introduced the notion of transport density in $\mathbb{H}^n$ and we provided assumptions to ensure that it is in $L^p$. A measure with continuous density naturally induces a weighted length of curves which leads to the definition of control distance. Scope of this section is to extend the definition of this kind of metric to $L^p$ densities. The statement of the results are  apparently similar to the analogous ones in the Euclidean setting (contained in \cite{Santambrogio1}), but the proofs are totally different, due to the geometric properties of the space. 


\subsection{Distance induced by a $p$-summable weight}
Through this section, we suppose that $\Omega\subset\mathbb{H}^{n}$ is an open, bounded set and $\mu,\nu\in\mathcal{P}(\overline{\Omega})$. We denote by 
\begin{equation}\label{acca}
	H:=\big{\{}\sigma\in AC([0,1],\overline{\Omega}):\ \sigma\ \textnormal{is\ horizontal}\big{\}}	 
\end{equation}
the set of horizontal curves on $\overline{\Omega}$ parametrized on $[0,1]$,
viewed as subset of $C([0,1],\overline{\Omega})$ equipped with the topology of uniform convergence.

Let us first recall the definition of weighted length in the regular setting. Let $\phi \in C(\overline{\Omega},\mathbb{R}_+)$ and $\sigma\in C([0,1],\overline{\Omega})$, we call 
\begin{multline*}\label{LPHI}
	L_{\phi}(\sigma):=\sup\bigg{\{}\sum_{i=1}^n \left(\inf_{[t_i,t_{i+1}]}(\phi\circ\sigma)\right)d_{CC}(\sigma(t_i), \sigma(t_{i+1})):\\ 
	([t_i, t_{i+1}])_i \mbox{ is a partition of }[0,1]\bigg{\}}. 
\end{multline*}
As it is well known, see for example \cite[Lemma 2.7]{Santambrogio1}, the function  $\sigma\mapsto L_{\phi}(\sigma)$ is l.s.c., hence Borel, on $C([0,1],\overline{\Omega})$ w.r.t. the uniform convergence. If $\phi\in C(\overline{\Omega},\mathbb{R})$, let us write $\phi=\phi_+-\phi_-$, where $\phi_+$ and $\phi_-$ are the positive and negative part of $\phi$ respectively, hence $L_\phi:=L_{\phi_+}-L_{\phi_-}$ is Borel.
In particular if $\sigma\in H$ and $\phi\in C(\overline{\Omega})$, then $L_{\phi}(\sigma)$ is the  length of the curve $\sigma$ with respect to the weight $\phi$, 
and can be expressed as
\begin{equation}\label{length}
	L_{\phi}(\sigma)=\int_0^1 \phi(\sigma(t))|\dot{\sigma}(t)|_H dt.
\end{equation} 
Moreover it follows that 
\begin{equation}\label{boundlphi}
	0\leq |L_{\phi}(\sigma)|\leq\|\phi\|_{\infty}l_H(\sigma).
\end{equation} 



%%%%%%%%%%%%%%


%%%%%%%%%%%%%


If $\phi\in C(\overline{\Omega},\mathbb{R}_+)$, $\forall x,y\in\overline{\Omega}$ we denote by
\begin{equation}\label{ccontinuous}
	c_\phi(x,y):=\inf\{L_{\phi}(\sigma)\,:\,\sigma\in H^{x,y}\},
\end{equation}
where
    \begin{equation}\label{horcrvxy}
        H^{x, y}:=\{\sigma\in H:\ \sigma(0)=x,\ \sigma(1)=y\}.
    \end{equation}

Let us explicitly recall that the cost function $c_\phi(x,y)$
is a distance, if $\phi$ is continuous and  strictly positive, and it is only a pseudo distance, if $\phi$ is non-negative. 
In order to extend this pseudo distance to summable functions  $\phi$, we start with an estimate of the regularity of $c_\phi(x,y)$ in terms of the $L^{p'}$ norm of $\phi$, 
where
$p':=\frac{p}{p-1}$ is the conjugate exponent of some $p\in(1,+\infty)$. 
 The proof is inspired by \cite[Proposition 3.2]{Santambrogio1} but requires many non trivial changes, due to the geometric structure of $\mathbb{H}^n$. 
\begin{prop}\label{cxicomp}
	If $p'>N$, then there exists $C>0$ such that for every $\phi\in C(\overline{\Omega},\mathbb{R}_+)$ and every $(x,y), (x',y')\in\Omega\times\Omega$, one has:
	\begin{equation*}\label{holderest}
		\vert c_{\phi}(x,y)-c_{\phi}(x',y')\vert \leq C\Vert \phi\Vert_{L^{p'}(\Omega)} \left(d_{CC}(x,x')^{\alpha}+ d_{CC}(y,y')^{\alpha} \right),
	\end{equation*}
	where $\alpha:=1-\frac{N}{p'}$.
	Moreover, if $(\phi_n)_{n\in\mathbb{N}}\subset  C(\overline{\Omega},\mathbb{R}_+)$ is bounded in $L^{p'}$, then $(c_{\phi_n})_{n\in\mathbb{N}}$ admits a sub-sequence that converges in $C(\overline{\Omega}\times\overline{\Omega},\mathbb{R}_+)$. 
\end{prop}
\begin{proof}
   
	Let $\phi\in C(\overline{\Omega},\mathbb{R}_+)$ and $x,y\in\Omega$. For $k>0$ let $\sigma_k\in H^{x,y}$ be such that
	\begin{equation*}
		\int_0^1 \phi(\sigma_k(t))\vert \dot \sigma_k(t)\vert_H dt\leq c_{\phi}(x,y)+\frac{1}{k}.
	\end{equation*}
 In order to study the regularity of $c_\phi$ with respect to the second variable $y$, we choose a point $z_\varepsilon$ 
 which can be connected to $y$ by an horizontal segment. Indeed, we fix  a constant coefficient unitary horizontal vector field i.e a vector field
$ \sum_{j=1}^na_jX_j+b_jX_{n+j}$
such that $(a_1,\ldots,a_n,b_1,\ldots,b_n)\in\mathbb{R}^{2n}$ and $\vert(a_1,\ldots,a_n,b_1,\ldots,b_n)\vert_E=1$, and we choose for all $\varepsilon>0$ the points 
	\begin{equation*}
		z_\varepsilon:=\exp\bigg{(}{\varepsilon\bigg{(}\sum_{j=1}^na_jX_j+b_jX_{n+j}}\bigg{)}\bigg{)}(y),
	\end{equation*}
such that $z_\varepsilon\in \Omega$. Now we modify the curve $\sigma_k$ into a curve  $\sigma_{k,t_0}\in H^{x,z_\varepsilon}$: we choose $t_0\in (0,1)$ and define
	\begin{equation*}
		\sigma_{k,t_0}(t):=
		\begin{cases}
			\sigma_k\big{(}{t\over t_0}\big{)} &\mbox{ if }t\in[0,t_0]\\  
			\tilde{\sigma}_{\varepsilon,y}\big{(}\frac{t-t_0}{1-t_0}\big{)} &\mbox{ if }t\in]t_0,1],
		\end{cases}
	\end{equation*}
	where
	\begin{equation*}
		\tilde{\sigma}_{\varepsilon, y}(t)=\exp\bigg{(}t\varepsilon\bigg{(}\sum_{j=1}^na_jX_j+b_jX_{n+j}\bigg{)}\bigg{)}(y),\quad t\in[0,1].
	\end{equation*}
	We then have, for all $k>0$
	\begin{align*}
		c_\phi(x,z_\varepsilon)&\leq \int_0^1\phi(\sigma_{k,t_0}(t))\vert\dot\sigma_{k,t_0}(t)\vert_H dt=\\
		&=\int_0^1\phi(\sigma_k(t))\vert \dot\sigma_k(t)\vert_H dt+\int_0^1\phi(\tilde{\sigma}_{\varepsilon, y}(t))\vert \dot{\tilde{\sigma}}_{\varepsilon, y}(t)\vert_H dt\leq\\
		&\leq c_\phi(x,y)+\frac{1}{k}+\varepsilon\int_0^1 \phi(\tilde{\sigma}_{\varepsilon, y}(t))dt.
	\end{align*}
	Now, if $k\rightarrow+\infty$, we get
	\begin{equation*}
		\frac{1}{\varepsilon}\bigg{[}c_{\phi}\bigg{(}x,\exp\bigg{(}{\varepsilon\bigg{(}\sum_{j=1}^na_jX_j+b_jX_{n+j}}\bigg{)}\bigg{)}(y)\bigg{)}-c_{\phi}(x,y)\bigg{]}\leq \int_0^1 \phi(\tilde{\sigma}_{\varepsilon, y}(t))dt,
	\end{equation*}
	and, by similar argument:
	\begin{equation*}
		\frac{1}{\varepsilon}\bigg{[}c_{\phi}(x,y)-c_{\phi}\bigg{(}x,\exp\bigg{(}{\varepsilon\bigg{(}\sum_{j=1}^na_jX_j+b_jX_{n+j}}\bigg{)}\bigg{)}(y)\bigg{)}\bigg{]}\leq \int_0^1 \phi(\tilde{\sigma}_{\varepsilon, y}(1-t))dt.
	\end{equation*}

Integrating with respect to $y$, raising to the power $p'$ and using the fact that the function $y\mapsto\tilde{\sigma}_{\varepsilon, y}(t) $ has Jacobian determinant $1$, 
this implies that $c_{\phi}(x,\cdot)\in HW^{1,p'}(\Omega)$, see \eqref{horsob}, and
	\begin{equation}\label{EST}
		\|\nabla_Hc_{\phi}(x,\cdot)\|_{p'}\leq\|\phi\|_{p'},\quad \forall x\in\Omega.
	\end{equation}
	By symmetry we also get that
	\begin{equation}\label{ESTx}
		\|\nabla_Hc_{\phi}(\cdot,y)\|_{p'}\leq\|\phi\|_{p'},\quad \forall y\in\Omega.	
	\end{equation}
	Since $p'>N$ then if follows by \eqref{EST}, \eqref{ESTx} and Morrey's Theorem (see \cite{Capogna}, Chapter 5), that there exists $C>0$ such that
	\begin{align*}
		\vert c_{\phi}(x,y)-c_{\phi}(x,y')\vert \leq C\Vert \phi\Vert_{L^{p'}}  d_{CC}(y,y')^{\alpha},\quad  \forall x, y, y'\in\Omega,\\
		\vert c_{\phi}(x,y)-c_{\phi}(x',y)\vert \leq C\Vert \phi\Vert_{L^{p'}}  d_{CC}(x,x')^{\alpha},\quad   \forall x, x', y\in\Omega.	
	\end{align*}
	This proves \eqref{holderest}. The second claim in the proposition then follows from \eqref{holderest}, the identity $c_{\phi_n}(x,x)=0$ and Ascoli-Arzelà's theorem. 
\end{proof}

From now on, we further assume that $p<\frac{N}{N-1}$. The next goal is to give an equivalent definition for $c_\phi$, that extends the notion for functions just in $L^{p'}$.

\begin{prop}\label{ccoincid}
	If $\phi\in C(\Omega,\mathbb{R}_+)$, then
\begin{equation}\label{barcphi}
	c_\phi(x,y)=\sup\left\{c(x,y)\,:c\in\mathcal{C}(\phi)\right\},
\end{equation}
where 
\begin{equation}\label{Setcphi}
\mathcal{C}(\phi)=\left\{c=\lim_{n\rightarrow+\infty} c_{\phi_n}\,\mbox{ in }C(\overline{\Omega}\times\overline{\Omega})\,:\,(\phi_n)_{n\in\mathbb{N}}\subset C(\overline{\Omega}),\,\phi_n\geq 0,\,
\phi_n\to\phi\,\mbox{ in }L^{p'}\right\}.
\end{equation}
\end{prop}

We first state two technical remarks that will be useful in the proof.

 
\begin{Remark}
Note that if we have a constant coefficient unitary 
horizontal vector 
 $W_1:=a_1X_1+\ldots +a_nX_n+a_{n+1}X_{n+1}+\ldots+a_{2n}X_{2n}\in\mathfrak{h}_1^1$, 
 it is  possible to perform a change of variable which sends the 
 vector $W_1$ to the first element of the canonical orthonormal basis. 
 Indeed, if we  denote by $W_2,\ldots,W_{2n}$ a basis of orthogonal complement $W_1^\perp$ in $\mathfrak{h}_1^1$ with respect to $\left\langle\cdot,\cdot\right\rangle_H$, and by $x$ a point, we can consider the change of variable
\begin{equation}\label{changeofcoordinates}
	\Psi:\mathbb{R}^{2n+1}\to\mathbb{H}^{n},
\quad \Psi(e_1,\ldots,e_{2n+1})=\exp(e_1W_1)\exp\left(\sum_{i=2}^{2n} e_{i}W_{i}+e_{2n+1}X_{2n+1}\right)(x).
\end{equation}
 In this system of coordinates the vector field $W_1$ reads as $d \Psi(W_1)=\partial_{e_1},$ and the point $x$ will be the origin in the new  coordinate system.
\end{Remark}

\begin{Remark} 
First we note that any function $c\in \mathcal{C}(\phi)$ satisfies the triangular inequality. Given a continuous function $\phi$, and three points $x^0,x^1,x^2$, it follows that 
	\begin{align*}
		c_{\phi}(x_0,x_2)&\leq \inf\{L_{\phi}(\sigma_1) + L_{\phi}(\sigma_2):\sigma_1\in H^{x^0,x^1}, \sigma_2\in H^{x^1,x^2}\}\\
		&\leq\sum_{i=1}^2\inf\{L_{\phi}(\sigma):\sigma\in H^{x^{i-1},x^i}\}=\sum_{i=1}^2c_\phi(x^{i-1},x^i).
	\end{align*}
	Hence, passing to the limit in the definition of $c$ we obtain
	\begin{equation}\label{ineq}
		c(x,y)=\lim_{n\to+\infty}c_{\phi_n}(x,y)\leq\lim_{n\to+\infty}\left( \sum_{i=1}^2c_{\phi_n}(x^{i-1},x^i)\right)=\sum_{i=1}^2c(x^{i-1},x^i).
	\end{equation}
\end{Remark}

\begin{proof}[{\it Proof of Proposition \ref{ccoincid}}]
To simplify notations we call
$$\overline{c}_{\phi}= \sup\left\{c(x,y)\,:c\in\mathcal{C}(\phi)\right\},$$
so that we have to prove that $ \overline{c}_{\phi} = c_\phi.$
	First we consider the constant sequence $\phi_n:=\phi,\quad\forall n\in\mathbb{N}$. Then $c_\phi\in\mathcal{C}(\phi)$ and we get that $\overline{c}_{\phi}\geq c_\phi$.
	
	Let us prove the converse inequality. 
	Let $\bar{x},\bar{y}\in\Omega$, $k>0$ and $\sigma\in H^{\bar{x},\bar{y}}$ such that $L_{\phi}(\sigma)<c_\phi(\bar{x},\bar{y})+1/k$. 
 %The fact that $\sigma$ is horizontal means that there exist measurable functions $h_j:[0,1]\to\mathbb{R}$ such that
%	\begin{equation*}
%		\dot\sigma(t)=\sum_{j=1}^{2n}h_j(t)X_j(\sigma(t)).
%	\end{equation*} 
 %a functions $h_j$'s are simple and hence
%	\begin{align*}
%		\dot\sigma(t)=\sum_{j=1}^{2n}\left(\sum_{i=1}^{M}\mathbbm{1}_{[t_{i-1},t_i]}(t)a_{i,j}\right)X_j(\sigma(t)).
%	\end{align*}
	Let us fix a sequence $\phi_n\rightarrow\phi$ in $L^{p'}$ such that $c_{\phi_n}$ converges uniformly to some $c$, we want to prove that $c\leq c_{\phi}$.
  From density of simple functions and continuity of $\phi$ we can assume that there exists 
 a finite decomposition $\{t_0, t_1, \cdots t_M\}$ of the interval $[0,1]$ such that  $\dot\sigma$ is constant and horizontal on the interval $[t_{i-1},t_i]$; in particular
	\begin{equation*}
		L_{\phi_n}(\sigma)=\sum_{i=1}^{M}\int_{t_{i-1}}^{t_i}\phi_n(\sigma(t))|\dot\sigma(t)|_Hdt.
	\end{equation*} 
Let us consider a single interval $[t_{i-1}, t_i]$: up to  a change of coordinates, we can also assume that $|\dot\sigma|_H=1$ on this interval. For this reason, in the change of coordinates  $\Psi_i:\mathbb{R}^{2n+1}\rightarrow\mathbb{H}^{n}$, introduced in \eqref{changeofcoordinates}, we can choose 
$\Phi_i(\sigma(t_{i-1}))=(t_{i-1},0)$ so that  
$\Phi_i(\sigma(t_{i}))=(t_{i},0)$ ,
 %where $W^i_1:=\sum_{j=1}^{2n}a_{i,j}X_j(\sigma(t))$ and we denote by $x^0=\bar{x},x^N=\bar{y}, x^{i}=\sigma(t_i),x^{i-1}=\sigma(t_{i-1})$. Let us fix $i$: we will denote by $W^i_2,\ldots,W_{2n}^i$ a basis of orthogonal complement $(W_1^i)^\perp$ in $\mathfrak{h}_1^1$ with respect to $\left\langle\cdot,\cdot\right\rangle_H$.  Using the change of coordinates. 
and $$\Phi_i \circ \sigma : [t_{i-1}, t_{i}] \to\mathbb{R}^{2n+1}, \quad (\Phi_i \circ \sigma)(t) = (t, 0).$$

We now consider, for every $\delta>0$ and for every $i$, cylindrical neighborhoods $C_{i, \delta} = \{(t, \hat e): t \in [t_{i-1}, t_i], |(0,\hat e)|_H \leq \delta\},$ of the curve $\Phi_i \circ \sigma$, with basis
$S_{i-1} =\{(t_{i-1}, \hat e): |(0,\hat e)|_H \leq \delta\} .$
For every $\hat e\in \mathbb{R}^{2n},$ with  $|(0,\hat e)|_H\leq \delta$, we call  $\sigma_e (t) =\Psi_i(t, \hat e)$. By definition 
$$
c_{\phi_n}\Big(\Psi_i(t_{i-1}, \hat e) , \Psi_i(t_i, \hat e)\Big)\leq 
L_{\phi_n}(\sigma_e \circ \theta_i),
$$
where $\theta_i$ is a  change of coordinate which sends $[0,1]$ to $[t_{i-1}, t_{i}]$. 
Note that 
\begin{equation}\label{questa}
L_{\phi_n}(\sigma_e \circ \theta_i) =  L_{\phi_n\circ\Psi_i }( \Phi_i \circ \sigma_e \circ \theta_i) = \int_{t_{i-1}} ^{t_i}  ( \phi_n\circ\Psi_i)(t, \hat e) dt.
\end{equation}
Hence, integrating on $S_{i-1}$  we get
\begin{equation}\label{cpne}
	\int_{S_{i-1}}c_{\phi_n}\Big(\Psi_i(t_{i-1}, \hat e), \Psi_i(t_i, \hat e)\Big) d\mathcal{H}^{2n}(\hat e)\leq \int_{S_{i-1}} \int_{t_{i-1}}^{t_i} ( \phi_n\circ\Psi_i)(t, \hat e) dt d\mathcal{H}^{2n}(\hat e).
\end{equation}
For $ n \to \infty$ using the uniform convergence of $c_{\phi_n}$ to $c$ and the $L^{p'}$ convergence of $\phi_n$ to $\phi$ we get
that

	\begin{equation*}
		\int_{S_{i-1}} c\Big(\Psi_i(t_{i-1}, \hat e), \Psi_i(t_i, \hat e)\Big))d\mathcal{H}^{2n}(\hat e)\leq\int_{C_i}(\phi\circ\Psi_i) (t, \hat e) d\mathcal{L}^{2n+1} (t, \hat e).
  	\end{equation*}

	Now we divide by the measure of $S_{i-1}$  and pass to the limit as $\delta\rightarrow0^+$.
	Using the fact that $c$ is continuous
	\begin{equation*}
		\lim_{\delta\to0^+}\frac{1}{d\mathcal{H}^{2n}(S_{i-1})}\int_{S_{i-1}} c\Big(\Psi_i(t_{i-1}, \hat e), \Psi_i(t_i, \hat e)\Big))d\mathcal{H}^{2n}(\hat e)=c\Big(\Psi_i(t_{i-1}, 0), \Psi_i(t_i, 0)\Big) = c(x^{i-1},x^i),
	\end{equation*}
where  $x^i = \sigma(t_i)$, 
Analogously  the integral over $C_i =[t_{i-1}, t_i] \times S_{i-1}$ divided by the measure of $S_{i-1} $ converges to the integral on $[t_{i-1}, t_i] $, which is the integral along the curve $\Phi_i\circ\sigma(t)$
	\begin{equation*}
		\lim_{\delta\to 0^+}\frac{1}{d\mathcal{H}^{2n}(S_{i-1})}\int_{C_i}(\phi\circ\Psi_i)(t, \hat e) d\mathcal{L}^{2n+1}(t, \hat e)= \int_{t_{i-1}}^{t_i}
  (\phi\circ\Psi_i)(t, 0) dt = 
\int_{t_{i-1}}^{t_i}  \phi(\sigma(t))|\dot\sigma(t)|_Hdt.
	\end{equation*}
	Then, using \eqref{cpne}, we get that 
	\begin{equation*}
 c(x^{i-1},x^{i}) \leq \int_{t_{i-1}}^{t_i}\phi(\sigma(t))|\dot\sigma(t)|_Hdt,\quad \forall i=1,\ldots,M,
	\end{equation*}
	and then, from \eqref{ineq},
	\begin{equation*}
 c(\bar{x},\bar{y})\leq
 \sum_{i=1}^{M}c(x^{i-1},x^{i}) 
		\leq\sum_{i=1}^{M}\int_{t_{i-1}}^{t_i}\phi(\sigma(t))|\dot\sigma(t)|_Hdt=L_{\phi}(\sigma).
	\end{equation*}
	This gives 
	\begin{equation*}
		c(\bar{x},\bar{y})\leq c_{\phi}(\bar{x},\bar{y})+\frac{1}{k}
	\end{equation*}
	for the choice of $\sigma$ and, since $k$ is arbitrary, it follows that $c(\bar{x},\bar{y})\leq c_{\phi}(\bar{x},\bar{y})$.
\end{proof}


Since  definition \eqref{barcphi},  makes sense also for $L^{p'}$ functions, and extends \eqref{ccontinuous}, we will use it as definition of $c_\phi$
for any non-negative function $\phi\in L^{p'}(\Omega)$: 

\begin{deff}
    If $\phi\in L^{p'}(\Omega)$, then we define
\begin{equation}\label{barcphi}
	c_\phi(x,y)=\sup\left\{c(x,y)\,:c\in\mathcal{C}(\phi)\right\},
\end{equation}
where $\mathcal{C}(\phi)$ has been defined in 
\ref{Setcphi}. 
\end{deff}
 
Let us finally note that the function $c_\phi$ is a pseudo distance. Indeed the properties of the pseudo distance can be obtained passing  to the limit in the inequalities satisfied by $c_{\phi_n}$. In particular 
 the triangular inequality follows from \eqref{ineq}.



\section{Congested Optimal Transport in $\mathbb{H}^n$}
Starting from a transport plan $\gamma$ we defined a transport density
$a_\gamma$, and an associated distance $c_{\phi}$, with $\phi = a_\gamma$. Hence it is possible to study Monge-Kantorovich problem associated to this distance
\begin{equation}\inf_{\gamma\in\Pi(\mu,\nu)}\int_{\overline{\Omega}\times\overline{\Omega}}c_{a_\gamma}(x,y)d\gamma(x,y).
	\end{equation}
This means that a transport plan induces a metric, and a metric induces an optimal transport plan. However, this optimal transport plan will not coincide with the starting one, and will change again the  metric of the space. In order to be able to reach an equilibrium, in which the transport plan remains stable under this minimization, we need to introduce the more abstract notion of \textit{traffic plan}, which depends on  all possible paths, and use  the  metric associated to its \textit{traffic intensity} in the Monge-Kantorovich problem. This will lead to the notion of \textit{Wardrop equilibrium}.

Hence, scope of this section is to adapt to the Heisenberg Group setting the notion of congested optimal transport, and its equilibria, proposed in \cite{Santambrogio1} by Carlier et al. We suppose that $\Omega\subset\mathbb{H}^{n}$ is an open, bounded and geodesically convex set, that models the cortical (or geographical) area on which the dynamic takes place and $\mu,\nu\in\mathcal{P}(\overline{\Omega})$ represent the initial and  final  cortical activity (or distributions of  vehicles and their destinations), respectively.





\subsection{Horizontal traffic plans and traffic intensity}
We now introduce a probability measure $Q$ on the set of continuous curves $C([0,1],\overline{\Omega})$, concentrated on the set $H$ of horizontal paths, defined in \eqref{acca}. According to the notation introduced in \cite{Morel} for the Euclidean setting, we will call \textit{horizontal traffic plan} a probability measure $Q\in\mathcal{P}([0,1],\overline{\Omega})$ such that $Q(H)=1$ and
\begin{equation}\label{avlength}
	\int_{C([0,1],\overline{\Omega})}l_H(\sigma)dQ(\sigma)<+\infty.
\end{equation}
Such measures are introduced to take into account the structure of the space in which we will work. We say that an horizontal traffic plan $Q$ is \textit{admissible} between the measures $\mu$ and $\nu$ if $(e_0)_{\#}Q=\mu$ and $(e_1)_{\#}Q=\nu$, where $e_0$ and $e_1$ are the evaluation maps at times $t=0$ and $t=1$. We denote by 
$$\mathcal{Q}_{\mathbb{H}}(\mu,\nu):=\{\text{admissible horizontal traffic plans between $\mu$ and $\nu$}\};$$ 
this set is not empty, indeed if $\gamma\in\Pi(\mu,\nu)$, we may define \begin{equation}\label{traffpl}
    Q_\gamma:=\int_{\overline{\Omega}\times\overline{\Omega}}\delta_{S(x,y)}d\gamma(x,y)\in\mathcal{Q}_{\mathbb{H}}(\mu,\nu).
\end{equation}


 One can associate to any $Q\in\mathcal{Q}_{\mathbb{H}}(\mu,\nu)$ a positive and finite Borel measure $i_{Q}\in\mathcal{M}_+(\overline{\Omega})$ that we will call \textit{horizontal traffic intensity}, representing the transiting mass associated with such horizontal traffic plan and defined by duality as
\begin{align}\label{defiQ}
    \int_{\overline{\Omega}} \phi(x) di_{Q}(x):=\int_{C([0,1], \overline{\Omega})} L_{\phi}(\sigma) d Q(\sigma),\quad \forall \phi \in C(\overline{\Omega}).
\end{align}
Note that the right hand side is well defined. Indeed, from \eqref{boundlphi} and \eqref{avlength} it follows that $L_{\phi}\in L^1(C([0,1],\overline{\Omega}),Q)$, for every $Q\in\mathcal{Q}_{\mathbb{H}}(\mu,\nu)$.

If we look at the action of $i_{Q}$ on a set $A$, then $i_{Q}(A)$ 
represents the total cumulated traffic in $A$ induced by $Q$.
The notion of horizontal traffic intensity is a path-dependent version of horizontal transport density introduced in Section \ref{sectrandens}. Namely, let $\gamma\in\Pi_1(\mu,\nu)$ and let $Q_\gamma$ be as in \eqref{traffpl}, then
\begin{equation}\label{transporttraffic}
i_{Q_\gamma}=a_\gamma.
\end{equation}


Let us denote by
\begin{equation}
	\mathcal{Q}_{\mathbb{H}}^p(\mu,\nu):=\left\{Q\in\mathcal{Q}_{\mathbb{H}}(\mu,\nu):i_{Q}\in L^p(\Omega)\right\}.
\end{equation}
From now on we further assume that $\mathcal{Q}_{\mathbb{H}}^p(\mu,\nu)\not=\emptyset$. In general this could be not true, but  if $\mu,\nu\in L^p$, Theorem \ref{summability2} and \eqref{transporttraffic} imply the existence of a traffic plan whose associated traffic intensity is a $L^p$ function. 


Using the same argument as in \cite[Section 3.2]{Santambrogio1} one can 
can extend for $\phi\geq 0, \phi\in L^{p'}(\Omega)$ the notion of weighted length  $L_{\phi}$,  and prove the following result.

\begin{teo}\label{prolLxi} Let $Q\in \mathcal{Q}^p_{\mathbb{H}}(\mu,\nu)$, $\phi\in L^{p'},\phi\geq0$, $p<\frac{N}{N-1}$ and $(\phi_n)_{n\in\mathbb{N}}\subset C(\overline{\Omega})$, $\phi_n\geq0$ $\forall n\in\mathbb{N}$, $\phi_n\rightarrow\phi$ in $L^{p'}$, then:
	\begin{enumerate}
		\item[(i)] $(L_{\phi_n})_{n\in\mathbb{N}}$ converges strongly in $L^1(C([0,1],\overline{\Omega}),Q)$ to some limit, independent of the  approximating sequence $(\phi_n)_{n\in\mathbb{N}}$. This limit will be denoted by $L_\phi$. 
		\item[(ii)] The following equality holds:
		\begin{equation}\label{eglc}
			\int_{\Omega} \phi(x) i_{Q}(x)\ dx=\int_{C([0,1],\overline{\Omega})} L_{\phi}(\sigma)\ dQ(\sigma).
		\end{equation}
		\item[(iii)] The following inequality holds for $Q$-a.e. $\sigma\in H$:
		\begin{equation}\label{ineglc}
			L_{\phi}(\sigma)\geq c_{\phi}(\sigma(0),\sigma(1)),
		\end{equation}
	\end{enumerate}
where $c_{\phi}$ is defined in  \eqref{barcphi}.
\end{teo}

\subsection{Wardrop equlibria in $\mathbb{H}^n$}

The congestion effects are captured by a metric associated with $Q$. We consider a \textit{congestion function} $g:\Omega\times\mathbb{R}_+\rightarrow\mathbb{R}_+$, which is continuous and such that
\begin{enumerate}
	\item $g(x,\cdot):\mathbb{R}_+\rightarrow\mathbb{R}_+$ is strictly increasing $\forall x\in\overline{\Omega}$;
	\item $\lim_{i\to\infty}g(x,i)=+\infty, \forall x\in\overline{\Omega}$;
	\item $g(x,0)=c, \forall x\in\overline{\Omega}$, for some $c\in\mathbb{R}, c>0$.
\end{enumerate}

The quantity $g(x,i)$ can be seen as the cost to be paid for passing through $x$ with an amount of traffic $i$. This partially justifies the assumptions on the cost function: the fact that $g(x,\cdot)$ is an increasing function of $i$ means that the more traffic there is, the highest is the cost to reach the destination 
; the assumption $\lim_{i\to\infty}g(x,i)=+\infty$ models the fact that, if there is too much traffic, it is impossible to reach the destination, and $g(x,0)>0$ because the cost is positive even if there is no traffic (in the case of vehicles for instance fuel cost, wear costs ecc.).


Given $Q\in\mathcal{Q}_{\mathbb{H}}(\mu,\nu)$, we denote by
\begin{equation}\label{congfun}
	\phi_{Q}(x):=\begin{cases}
		g(x,i_{Q}(x)),\quad \text{if }i_{Q}\ll\mathcal{L}^{2n+1},\\
		+\infty,\quad \text{otherwise},
	\end{cases}
\end{equation}
where, with abuse of notation, $i_Q(x)$ is the density of the measure $i_Q$ with respect to the Lebesgue measure.
The existence of at least one $Q$ such that $i_{Q}\ll\mathcal{L}^{2n+1}$ depends on $\mu$ and $\nu$. For instance if either $\mu\ll\mathcal{L}^{2n+1}$ or $\nu\ll\mathcal{L}^{2n+1}$, the existence of such a $Q$ follow from  \eqref{transporttraffic} and Theorem \ref{absolutecont}. Let $Q\in\mathcal{Q}(\mu,\nu)$ such that $i_Q\ll\mathcal{L}^{2n+1}$, hence the quantity   
\begin{equation}\label{vehicles}
	\int_{\overline{\Omega} }\phi_{Q}(x)i_{Q}(x)dx
\end{equation}
is well defined and 
represents the total cost paid for commuting between $\mu$ and $\nu$, corresponding to the traffic assignment $Q$. We can also express the transport problem in terms of transport plans. Let $H^{x,y}$ the set defined in \eqref{horcrvxy}
    then any transport from $x$ to $y$ that is performed along a path $\sigma\in H^{x, y}$, pays a cost 
    \begin{equation*}
		L_{\phi_{Q}}(\sigma)=\int_0^1 g\big{(}\sigma(t),i_{Q}(\sigma(t))\big{)}\vert\dot{\sigma}(t)\vert_Hdt.
	\end{equation*}
	An optimal transportation problem with traffic congestion, analogous to the classical Monge-Kantorovich problem, can be stated as 
	\begin{equation}\label{condition2}\inf_{\gamma\in\Pi(\mu,\nu)}\int_{\overline{\Omega}\times\overline{\Omega}}c_{\phi_{Q}}(x,y)d\gamma(x,y),
	\end{equation}
    
    where the distance function has been replaced by the  cost 
    $c_{\phi_{Q}}$ defined in the previous section. Clearly the infimum depends on the choice of $Q$. Moreover, the fact that  $ Q \in \mathcal{Q}_{\mathbb{H}}(\mu,\nu)$, implies that $\gamma_{Q} := (e_0, e_1)_{\#}Q \in\Pi(\mu,\nu)$, but it will be in general different from the minimizer. However a change in all the transport plans can induce to change  the traffic plan. In a urban scenario this change can be imposed by an external authority,  while it takes place naturally in living systems, as in the visual cortex, which we choose as a motivating example for this study. Here the traffic plan is represented by the cortical connectivity and its strength represents its traffic intensity. Within a fixed architecture, the signal tends to choose the best transport plan, minimizing \eqref{condition2}. However, due to cortical plasticity, a learning mechanism is able to change the structure of the connectivity network, in order to optimize the propagation, leading to a changement of the traffic plan. Hence 
    we are interested in horizontal traffic plans $Q$ such that the minimum in this problem is exactly the transport plan $\gamma_{Q}$. 
This is the horizontal version of the notion of Wardrop equilibrium provided in \cite{Santambrogio1}:

\begin{deff}\label{Wardrop}
	A Wardrop equilibrium is an horizontal traffic plan $Q\in\mathcal{Q}_{\mathbb{H}}(\mu,\nu)$ such that
	\begin{enumerate}
		\item $Q\big{(}\left\{\sigma\in H: L_{\phi_Q}(\sigma)=c_{\phi_{Q}}(\sigma(0),\sigma(1))\right\}\big{)}=1$;
		\item $\gamma_{Q}:=(e_0,e_1)_{\#}Q\in\Pi(\mu,\nu)$ solves the Monge-Kantorovich problem
		\begin{equation*}
			\inf_{\gamma\in\Pi(\mu,\nu)}\int_{\overline{\Omega}\times\overline{\Omega}}c_{\phi_{Q}}(x,y)d\gamma(x,y).
		\end{equation*}
	\end{enumerate}
\end{deff}

    
    
%	Obviously the metric above is well-posed only if $i_{Q}\ll\mathcal{L}^{2n+1}$ and $\phi$ is continuous (or at least lower semicontinuous). The goal of the next section is to prove that the metric can be rigorously defined also when $g\left( x,i_{Q}(\cdot)\right)$ is an $L^{p'}$ function for every $x\in\overline{\Omega}$, where $p'>N$ and $N$ is the homogeneous dimension of $\mathbb{H}^n$ defined in \eqref{homogdim}. The reason will be clarified in the last section.

%\begin{Remark}
%	The condition $p<\frac{N}{N-1}$ in Theorem \ref{prolLxi} is not very restrictive. Following \cite[Remark 3.7]{Santambrogio1} and \cite[Proposition 4.4]{Brasco2}, if we consider two discrete measure $\mu$ and $\nu$, with $\mu\not=\nu$, then $\mathcal{Q}_{\mathbb{H}}^p(\mu,\nu)\not=\emptyset$ for $p\in\left(1,\frac{N}{N-1}\right)$ but $\mathcal{Q}_{\mathbb{H}}^{\frac{N}{N-1}}(\mu,\nu)=\emptyset$. Indeed, if we assume that there exists $Q\in\mathcal{Q}_{\mathbb{H}}(\mu,\nu)$ such that $i_{Q}\in L^{\frac{N}{N-1}}(\Omega)$, we can define the vector measure $\textbf{\textsc{w}}_Q$
%	\begin{equation*}\label{vectormeasure}
%		\int_{\overline{\Omega}}X(x)\cdot d\textbf{\textsc{w}}_Q=\int_{C([0,1],\overline{\Omega})}\left(\int_0^1\left\langle X(\sigma(t)),\dot{\sigma}(t)\right\rangle_H dt\right)dQ(\sigma),\quad \forall X\in C(\overline{\Omega},H\mathbb{H}^{n}).
%	\end{equation*}
%	It follows that $|\textbf{\textsc{w}}_Q|\leq i_{Q}$, and hence $\|\textbf{\textsc{w}}_Q\|_{\frac{N}{N-1}}\leq\|i_{Q}\|_{\frac{N}{N-1}}<+\infty$.
%	Moreover it follows that
%	\begin{equation*}\label{flowdivergence}
%		\nabla_H\cdot\textbf{\textsc{w}}_Q=\mu-\nu.
%	\end{equation*} 
%	Since $\mu-\nu\notin HW^{-1,\frac{N}{N-1}}(\Omega)$, we get a contradiction. We have just proved that $\mathcal{Q}_{\mathbb{H}}^{\frac{N}{N-1}}(\mu,\nu)=\emptyset$ as soon as $\mu-\nu\notin HW^{-1,\frac{N}{N-1}}(\Omega)$. We can conclude that if $p\geq\frac{N}{N-1}$ the congestion effects are so strong that the total congested cost in \eqref{lepbme1} is always $+\infty$ as soon as $\mu-\nu\notin HW^{-1,\frac{N}{N-1}}(\Omega)$.
%\end{Remark}

We will see in the next section that the transport plan 
which realizes the equilibrium can be found as a solution of a suitable convex optimization problem. 


\subsection{Existence of Wardrop equilibria as minima of a convex optimization problem}
The following  convex optimization problem has been proposed in \cite{Santambrogio1}, in order to get the existence of equilibria introduced posed in the previous section. We will always assume that
\begin{equation}
\mathcal{Q}_{\mathbb{H}}^p(\mu,\nu):=\left\{Q\in\mathcal{Q}_{\mathbb{H}}(\mu,\nu):i_{Q}\in L^p\right\}\not=\emptyset,
\end{equation}
with $p<\frac{N}{N-1}$, and we call \textit{horizontal congested optimal transport problem}:
\begin{equation}\label{lepbme1}
	\inf_{Q\in \mathcal{Q}_{\mathbb{H}}^p(\mu,\nu)}\int_{\Omega}G(x,i_{Q}(x))dx,
\end{equation}

%\begin{equation}\label{lepbme0}
%	\inf_{Q\in \mathcal{Q}_{\mathbb{H}}(\mu,\nu)} \mathcal{G}(i_{Q}),
%\end{equation}
%where
%\begin{equation}\label{totalcost}
%	\mathcal{G}(i)=\begin{cases}
%		\int_{\Omega}G(x,i(x))dx,\quad  &\mbox{ if $i\ll\mathcal{L}^{2n+1}$},\\
%		+\infty,\quad&\mbox{ otherwise,}
%	\end{cases}	
%\end{equation}
where $G(x,i):=\int_0^ig(x,z)dz$, for some cost function $g$. 
% As we can see in \eqref{congfun}, congestion effects in this model are quite strong: only very diffused horizontal traffic intensity (i.e. absolutely continuous w.r.t $\mathcal{L}^{2n+1}$) are allowed; as soon as there is a low-dimensional concentration of the traffic, there will be an infinite total cost and vehicles get stuck in traffic.
The quantity $\int_{\Omega}G(i_{Q}(x))dx$ can be seen as the total cost of congestion for someone who has the right to impose \textit{who goes where}, in order to minimize the total cost, which in general differs from the total cost paid by a single user, given by \eqref{vehicles}. For this reason we will refer to %\eqref{lepbme1} as the \textit{horizontal congested optimal transport problem} and to
the functional in \eqref{lepbme1} as the \textit{total cost functional}.

We further assume that $\exists a,b\in\mathbb{R}_+$ such that
\begin{equation*}\label{growthg}
	ai^{p-1}\leq g(x,i)\leq b(i^{p-1}+1), \forall i\in\mathbb{R}_+,\forall x\in\Omega,
\end{equation*}
which implies that, for every $x\in\Omega$, the function $g(x,i_{Q}(\cdot))\in L^{p'}(\Omega)$ for every $Q\in\mathcal{Q}^p_{\mathbb{H}}(\mu,\nu)$.


%{\color{blue} - NOTA -  CONCORDO CON BRASCO. NON SI CAPISCE COSA SERVONO QUESTE PROPRIETA', IMMAGINO SERVANO NELLA PROVA, MA NO SI CAPISCE COME}

%Note that Ascoli-Arzelà theorem guarantees that the set 
%\begin{equation*}
%	H_K:=\left\{\sigma\in H:|\dot\sigma|_H\leq K\right\}
%\end{equation*}
%are compact (w.r.t. the uniform convergence) for every $K>0$. Indeed these sets are equicontinuous because 
%\begin{equation*}
%	d_{CC}(\sigma(t_1),\sigma(t_2))\leq\int_{t_1}^{t_2}|\dot\sigma(t)|_Hdt\leq K|t_1-t_1|,\quad \forall t_1,t_2\in[0.1].
%\end{equation*}

%{\color{blue} - NOTA -  END COSA NON COMPRESA}

Under the previous hypothesis, following the same strategy as \cite{Santambrogio1} and using the compactness argument above, one can prove that:

\begin{teo}
	The minimization problem \eqref{lepbme1} admits a solution.
\end{teo}

%\subsection{Optimality Conditions}\label{foc}

We consider the variational inequality characterizing solutions of the convex problem \eqref{lepbme1}. Precisely transport plan  $\overline{Q}\in \mathcal{Q}_{\mathbb{H}}^p(\mu,\nu)$,  solves \eqref{lepbme1} if and only if
\begin{equation}\label{varineq}
	\int_{\Omega} \phi_{\overline{Q}}(x) i_{\overline{Q}}(x)dx=\inf \left\{ \int_{\Omega} \phi_{\overline{Q}}(x) i_{Q}(x)dx \; :\; Q\in\mathcal{Q}_{\mathbb{H}}^p(\mu,\nu)\right\},
\end{equation}
where $\phi_{\overline{Q}}(x)=g(x,i_{\overline{Q}}(x))$.
Following the same strategy as in \cite[Proposition 3.9]{Santambrogio1}, one can also prove that if $\overline{Q}\in \mathcal{Q}_{\mathbb{H}}^p(\mu,\nu)$ solves \eqref{lepbme1}, then
\begin{equation}
	\int_{\Omega} \phi_{\overline{Q}}(x) i_{\overline{Q}}(x)dx=\inf_{\gamma\in\Pi(\mu,\nu)}\int_{\overline{\Omega}\times\overline{\Omega}}c_{\phi_{\overline{Q}}}(x,y)d\gamma(x,y).
\end{equation}

Now we will see that solutions of \eqref{lepbme1} are equilibrium configurations for the Wardrop problem,  introduced in Definition \ref{Wardrop}. 
\begin{Remark}
    If $\overline{Q}$ solves \eqref{lepbme1}, and $\gamma_{\bar Q}:=(e_0,e_1)_{\#}\overline{Q}$,  it follows that
\begin{align*}
	\int_{\overline{\Omega}\times \overline{\Omega}} c_{\phi_{\overline{Q}}}(x,y)d\gamma_{\bar{Q}}(x,y)=\int_{C([0,1],\overline{\Omega})}c_{\phi_{\overline{Q}}}(\sigma(0),\sigma(1))d\overline{Q}(\sigma)\leq\\ \underset{\eqref{ineglc}}{\leq}\int_{C([0,1],\overline{\Omega})} L_{\phi_{\overline{Q}}}(\sigma) d\overline{Q}(\sigma)= \int_{\Omega} \phi_{\overline{Q}}(x) i_{\overline{Q}}(x)dx=\\= \inf_{\gamma\in \Pi(\mu,\nu)} \int_{\overline{\Omega}\times \overline{\Omega}} c_{\phi_{\overline{Q}}}(x,y)d\gamma(x,y).
\end{align*}
Hence, $\gamma_{\bar Q}$ solves the  Monge-Kantorovich problem, associated with the cost function $c_{\phi_{\overline{Q}}}$
\begin{equation}
	\inf_{\gamma\in \Pi(\mu,\nu)} \int_{\overline{\Omega}\times \overline{\Omega}} c_{\phi_{\overline{Q}}}(x,y)d\gamma(x,y).
\end{equation}
\end{Remark}
We also observe that
\begin{align*}
	\int_{C([0,1],\overline{\Omega})} L_{\phi_{\overline{Q}}}(\sigma) d\overline{Q}(\sigma)= \int_{\overline{\Omega}\times\overline{\Omega}} c_{\phi_{\overline{Q}}}(x,y)d\gamma_{\bar Q}(x,y)\\
	=\int_{C([0,1],\overline{\Omega})}c_{\phi_{\overline{Q}}}(\sigma(0),\sigma(1))d\overline{Q}(\sigma)
\end{align*}
and, since $L_{\phi_{\overline{Q}}}(\sigma)\geq c_{\phi_{\overline{Q}}}(\sigma(0),\sigma(1))$, we get
\[L_{\phi_{\overline{Q}}}(\sigma)=c_{\phi_{\overline{Q}}}(\sigma(0),\sigma(1)) \quad\mbox{ for }  \overline{Q} \mbox{-a.e. } \sigma.\]

We remark the fact that the hypothesis $p<\frac{N}{N-1}$ guarantees the conjugate exponent $p'>N$: from \eqref{growthg} it follows that $\phi_{\bar Q}\in L^{p'}(\Omega)$ and then the metric $c_{\phi_{\bar Q}}$ is well defined.

%\subsection{Existence of Wardrop Equilibria }


Following the same argument as in \cite{Santambrogio1}, one can prove the existence of such equilibria:

\begin{teo}\label{cwe}
	Let us assume that $1<p<\frac{N}{N-1}$ then, if $\mathcal{Q}^p_{\mathbb{H}}(\mu,\nu)\not=\emptyset$, there exists an equilibrium. Moreover $Q\in\mathcal{Q}_{\mathbb{H}}(\mu,\nu)$ is an equilibrium if and only if $Q$ solves the minimization problem \eqref{lepbme1}.
\end{teo}

\begin{Remark}
    If we furthermore assume that $G(x,\cdot)$ is strictly convex for every $x\in\Omega$, then given $Q_{1}$ and $Q_{2}$ which solves \eqref{lepbme1} it follows that $i_{Q_{1}}=i_{Q_{2}}$. In other words, equilibria are not necessarily unique but they all induce the same intensity or, equivalently, the same metric.
\end{Remark}
 
As we said before, from a modelistic viewpoint minimizing \eqref{lepbme1}, with the constraint $(e_0)_{\#}Q=\mu$ and $(e_1)_{\#}Q=\nu$, describes a situation we only know which is the the initial and  final  cortical activity (or the  distributions of vehicles and their destination), and we are interested in  minimizing the total cost: in mathematical terms, this corresponds to say that the set of admissible couplings $(e_0, e_1)_{\#}Q$ coincides with the whole set of transport plans $\Pi(\mu,\nu)$. This is often referred as SO (System Optimum) type of movement. 

More constraints could be possibly imposed to the problem. Indeed one could minimize the total cost under the constraint of $(e_0,e_1)_{\#}Q\in\Pi\subset\Pi(\mu,\nu)$. As a particular case, we have $\Pi=\{\bar{\gamma}\}$, that is the coupling is given and, roughly speaking, we a priori know the probability $\bar\gamma(x,y)$ for a vehicle in $x$ (or the signal starting from $x$) to reach a destination $y$. This is often called User Equilibrium (UE) type of movement. 
In this case we look for a traffic plan in the set 
\begin{eqnarray*}
	\mathcal{Q}_{\mathbb{H}}^p(\overline{\gamma}):=\bigg{\{}Q\in \mathcal{P}(C([0,1],\overline{\Omega})) \mbox{ : } Q(H)=1,\ \int_{C([0,1],\overline{\Omega})}l_H(\sigma)dQ(\sigma)<+\infty,\\ (e_0,e_1)_\sharp Q=\overline{\gamma} \text{ and } i_{Q}\in L^p(\Omega)\bigg{\}} 
\end{eqnarray*}
and equilibria are traffic plans belonging to this set and satisfying the first condition of Definition \ref{Wardrop}. If $\mathcal{Q}_{\mathbb{H}}^p(\overline{\gamma}) \not=\emptyset$ the previous arguments can be adapted to this new situation and we can get the existence of equilibria.
Interestingly enough, while in the urban scenario UE is more realistic, since each vehicle has a preferred destination, it has been proved in \cite{Wolfson} (in the  Euclidean setting) that a SO type of movement is more natural for the description of signal propagation in the brain.



\section*{Acknowledgments}
M. C. and G.C. are supported by the project PRIN 2022 F4F2LH - CUP J53D23003760006, G.C. is funded by MNESYS PE12 (PE0000006).

The authors thank L. Brasco and S. Rigot for the useful discussions, suggestions and remarks.

\printbibliography

\end{document}