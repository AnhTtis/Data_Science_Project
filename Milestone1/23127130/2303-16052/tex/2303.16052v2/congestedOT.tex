

\section{Weighted distance induced by a $L^q$ density}
A measure with continuous density naturally induces a weighted length of curves, leading to the definition of weighted distance. The aim of this section is to extend the definition of weighted distance to measures with $L^q$ densities. While the statements of the results appear similar to those in the Euclidean setting (contained in \cite{Santambrogio1}), their proofs differ significantly due to the geometric properties of the space.

We consider a bounded domain $\Omega\subset\mathbb{H}^{n}$ and we assume that its boundary has $C^{1,1}$ regularity, in the Euclidean sense. We consider the space 
\begin{equation}\label{acca}
	H:=\big{\{}\sigma\in \textnormal{Lip}([0,1],\overline{\Omega}):\ \sigma\ \textnormal{is\ horizontal}\big{\}}	 
\end{equation}
of horizontal curves on $\overline{\Omega}$ parameterized on $[0,1]$, viewed as subset of $C([0,1],\overline{\Omega})$ equipped with the topology of uniform convergence. Given $x,y\in\overline\Omega$, we denote by 
\begin{equation}\label{horcrvxy}
    H^{x, y}:=\{\sigma\in H:\ \sigma(0)=x,\ \sigma(1)=y\}.
\end{equation}
The regularity assumption on the boundary of $\Omega$ guarantees the existence of at least one horizontal curve between any pair of points in $\overline\Omega$. This assumption replaces the stronger convexity assumption used in the Euclidean setting in \cite{Santambrogio1},  since non-trivial geodesic convex sets do not exist in the Heisenberg group, as proved in \cite{Monti2}. Therefore, the following important lemma holds.


\begin{lemma}\label{nonemptinesshorcur}
The set $H^{x,y}\not=\emptyset$, for any $x,y\in\overline{\Omega}$. 
\end{lemma}

\begin{proof}
Let $x,y\in\Omega$. Since $\Omega$ is open and connected,  then the statement follows from the Chow-Rashevsky Theorem applied to the manifold $\Omega$, equipped with
the sub-Riemannian structure inherited from $\mathbb{H}^n$, see \cite[Theorem 3.31]{barilariagrachev}.

Now, let us denote by $\mathcal{C}(\partial\Omega)$ the set of \textit{characteristic points} of $\partial\Omega$, that is 
\begin{equation*}    
\mathcal{C}(\partial\Omega):=\left\{x\in\partial\Omega:T_x\partial\Omega=H_x\mathbb{H}^n\right\},
\end{equation*}
and let us remark that at a non-characteristic point $z$ the fiber $H_z\mathbb{H}^n$  can be represented as 
$$H_z\mathbb{H}^n=H_z\partial\Omega \oplus \mathrm{span}\{\textbf{n}_H(z)\}.$$
Here, $H_z\partial\Omega = T_z\partial\Omega \cap H_z\mathbb{H}^n$ denotes the horizontal tangent space to $\partial\Omega$ at the point $z$, and $\textbf{n}_H(z)$ is the horizontal normal at the point $z\in\partial\Omega$, that is the
orthogonal projection of the Euclidean normal to $\Omega$ at $z$ on $H_z\mathbb{H}^n$. Then, for any horizontal vector field $Z$, the vector $Z(z)\in H_z\mathbb{H}^n$ admits an unique projection $\pi(Z(z))$ on the space $H_z\partial\Omega$. One can consider the horizontal vector field $\pi(Z):\Omega\to H\Omega$, $$z\mapsto \pi(Z)(z):=\pi(Z(z)),$$ 
where $\pi(Z)(z)=Z(z)$ if $z$ is a characteristic point.

Let us consider the case $x\in\partial\Omega$ and $y\in\Omega$. If $x\in\partial\Omega$ is a non-characteristic point, then the horizontal normal $\textbf{n}_H$ at the point $x$ does not vanish. As a result, if $\textbf{n}_H(x):=\sum_{i=1}^{2n}\textbf{n}_iX_i(x)$, one can consider the horizontal vector field $Z:=\sum_{i=1}^{2n}\textbf{n}_iX_i\in\mathfrak{h}_1^n$, $\delta >0$ and the horizontal curve
\begin{equation}
    \sigma(t):=\exp(-tZ)(x), 
\end{equation}
such that $\sigma\left([0, \delta]\right)\subset\overline\Omega$ and $z:=\sigma(\delta)\in\Omega$. Now one can consider a horizontal curve between $z$ and $y$, fully contained in $\Omega$.

If $x\in\mathcal{C}(\partial\Omega)$ is a characteristic point, from \cite[Theorem 1.2]{balogh2003size} it follows that the Hausdorff dimension w.r.t. the Euclidean metric is $\dim_E \mathcal{C}(\partial\Omega)<2n$. Then, there exists $v=\sum_{i=1}^{2n}v_iX_i(x)\in T_x\partial\Omega=H_x\mathbb{H}^n$ and $\delta>0$ such that the horizontal curve 
\begin{equation}
    \sigma(t)=\exp\Big(t\sum_{i=1}^{2n}v_i\pi(X_i)\Big)(x)  
\end{equation}
is well-defined, it belongs to $\partial\Omega$ and it is non-characteristic for all $t\in(0, \delta]$. Hence, one can find a horizontal curve between $z=\sigma(\delta)$ and $y\in\Omega$, using the argument above.

If $x,y\in\partial\Omega$, using the previous arguments one can connect them with $x',y'\in \Omega$ and then find a horizontal curve between $x'$ and $y'$, which is contained in $\Omega$.
\end{proof}

Given $\sigma\in H$ and $\phi\in C(\overline{\Omega},\mathbb{R}_+)$, we 
recall that $L_{\phi}(\sigma)$ is the horizontal length of the curve $\sigma$, weighted by $\phi$, introduced in \eqref{3agosto}.

Following \cite[Lemma 2.7]{Santambrogio1} one can prove that, for any $\phi \in C(\overline{\Omega},\mathbb{R}_+)$ and for any $\sigma\in H$, it holds that 
\begin{multline*}\label{LPHI}
	L_{\phi}(\sigma):=\sup\bigg{\{}\sum_{i=1}^n \left(\inf_{[t_i,t_{i+1}]}(\phi\circ\sigma)\right)d_{CC}(\sigma(t_i), \sigma(t_{i+1})):([t_i, t_{i+1}])_i \mbox{ is a partition of }[0,1]\bigg{\}}. 
\end{multline*}
Hence, the functional $H\ni\sigma\mapsto L_{\phi}(\sigma)$ is lower semicontinuous, and hence Borel, w.r.t. the uniform convergence. 

Given a weight $\phi\in C(\overline{\Omega},\mathbb{R}_+)$, we denote its associated weighted distance by
\begin{equation}\label{ccontinuous}
	c_\phi(x,y):=\inf\{L_{\phi}(\sigma)\,:\,\sigma\in H^{x,y}\},\quad \forall (x,y)\in\Omega\times\Omega.
\end{equation}   

Let us explicitly recall that the cost function $c_\phi(x,y)$
is a distance, if $\phi$ is continuous and  strictly positive, and it is only a pseudo distance, if $\phi$ is non-negative. 
In order to extend this pseudo distance to summable functions  $\phi$, we start with an estimate of the regularity of $c_\phi(x,y)$ in terms of the $L^q$ norm of $\phi$, 
where $q:=\frac{p}{p-1}$ is the dual exponent of some $p\in(1,+\infty)$. The proof is inspired by \cite[Proposition 3.2]{Santambrogio1} but requires many non trivial changes, due to the geometric structure of $\mathbb{H}^n$. 

\begin{prop}\label{cxicomp1}
If $q>N$, then there exists $C>0$ such that for any $\phi\in C(\overline\Omega,\mathbb{R}_+)$ and any $(x,y), (x',y')\in\Omega\times\Omega$, it holds
\begin{equation}\label{holderest}
		\vert c_{\phi}(x,y)-c_{\phi}(x',y')\vert \leq C\Vert \phi\Vert_{L^{q}(\Omega)} \left(d_{CC}(x,x')^{\alpha}+ d_{CC}(y,y')^{\alpha} \right),
	\end{equation}
where $\alpha:=1-\frac{N}{q}$.

\end{prop}
\begin{proof}
   
Let $\phi\in C(\overline{\Omega},\mathbb{R}_+)$ and $x,y\in\Omega$. For $k>0$ let $\sigma_k\in H^{x,y}$ be such that
\begin{equation*}
	\int_0^1 \phi(\sigma_k(t))\vert \dot \sigma_k(t)\vert_H dt\leq c_{\phi}(x,y)+\frac{1}{k}.
\end{equation*}
 In order to study the regularity of $c_\phi$ with respect to the second variable $y$, we choose a point $z_\varepsilon$ 
 which can be connected to $y$ by an horizontal segment: i.e. we fix a horizontal vector field $Z\in\mathfrak{h}^{n}_1$, such that $|Z|_H=1$, and we choose for all $\varepsilon>0$ the points and we choose for all $\varepsilon>0$ the points 
\begin{equation*}
	z_\varepsilon:=\exp\left(\varepsilon Z\right)(y),
\end{equation*}
such that $z_\varepsilon\in \Omega$. Now we modify the curve $\sigma_k$ into a curve  $\sigma_{k,t_0}\in H^{x,z_\varepsilon}$: we choose $t_0\in (0,1)$ and define
\begin{equation*}
    \sigma_{k,t_0}(t):=
    \begin{cases}
		\sigma_k\big{(}{\frac{t}{t_0}}\big{)} &\mbox{ if }t\in[0,t_0]\\  
		\tilde{\sigma}_{\varepsilon,y}\big{(}\frac{t-t_0}{1-t_0}\big{)} &\mbox{ if }t\in]t_0,1],
    \end{cases}
\end{equation*}
where
\begin{equation*}
    \tilde{\sigma}_{\varepsilon, y}(t)=\exp\left(t(\varepsilon Z)\right)(y),\quad t\in[0,1].
\end{equation*}
We then have, for all $k>0$
\begin{align*}
    c_\phi(x,z_\varepsilon)&\leq \int_0^1\phi(\sigma_{k,t_0}(t))\vert\dot\sigma_{k,t_0}(t)\vert_H dt=\\
	&=\int_0^1\phi(\sigma_k(t))\vert \dot\sigma_k(t)\vert_H dt+\int_0^1\phi(\tilde{\sigma}_{\varepsilon, y}(t))\vert \dot{\tilde{\sigma}}_{\varepsilon, y}(t)\vert_H dt\leq\\
	&\leq c_\phi(x,y)+\frac{1}{k}+\varepsilon\int_0^1 \phi(\tilde{\sigma}_{\varepsilon, y}(t))dt.
\end{align*}
Now, if $k\rightarrow+\infty$, we get
\begin{equation*}
    \frac{1}{\varepsilon}\left[c_{\phi}\left(x,\exp\left(\varepsilon Z\right)(y)\right)-c_{\phi}(x,y)\right]\leq \int_0^1 \phi(\tilde{\sigma}_{\varepsilon, y}(t))dt,
\end{equation*}
and, by similar argument:
\begin{equation*}
    \frac{1}{\varepsilon}\left[c_{\phi}(x,y)-c_{\phi}\left(x,\exp\left(\varepsilon Z\right)(y)\right)\right]\leq \int_0^1 \phi(\tilde{\sigma}_{\varepsilon, y}(1-t))dt.
\end{equation*}

Integrating with respect to $y$, raising to the power $q$ and using the fact that the function $y\mapsto\tilde{\sigma}_{\varepsilon, y}(t) $ has Jacobian determinant $1$, 
we get that, for any fixed $x$, $Zc_{\phi}(x,\cdot)\in L^{q}(\Omega)$ and $\|Zc_\phi(x,\cdot)\|_{L^q(\Omega)}\leq C\|\phi\|_{L^q(\Omega)}$. Since this holds for every $Z$, we have $c_{\phi}(x,\cdot)\in HW^{1,q}(\Omega)$, see \eqref{horsob}, and
\begin{equation}\label{EST}
    \|\nabla_Hc_{\phi}(x,\cdot)\|_{L^q(\Omega)}\leq\|\phi\|_{L^q(\Omega)},\quad \forall x\in\Omega.
\end{equation}
By symmetry we also get that
\begin{equation}\label{ESTx}
    \|\nabla_Hc_{\phi}(\cdot,y)\|_{L^q(\Omega)}\leq\|\phi\|_{L^q(\Omega)},\quad \forall y\in\Omega.	
\end{equation}

Since $q>N$ then if follows from \eqref{EST}, \eqref{ESTx} and \cite[Theorem 1.11]{garofalo1996isoperimetric}, that there exists $C>0$ such that
\begin{align*}
    \vert c_{\phi}(x,y)-c_{\phi}(x,y')\vert \leq C\Vert \phi\Vert_{L^{q}(\Omega)}  d_{CC}(y,y')^{\alpha},\quad  \forall x, y, y'\in\Omega,\\
    \vert c_{\phi}(x,y)-c_{\phi}(x',y)\vert \leq C\Vert \phi\Vert_{L^{q}(\Omega)}  d_{CC}(x,x')^{\alpha},\quad   \forall x, x', y\in\Omega,	
\end{align*}
with $\alpha=1-\frac{N}{q}.$ This proves \eqref{holderest}. 
\end{proof}


\begin{prop}\label{12aprile}
    If $q>N$, then for any $\phi\in C(\overline\Omega, \mathbb{R}_+)$, the function $c_\phi$ defined in \eqref{ccontinuous} admits a unique continuous extension as a function
\begin{equation*}
    c_\phi:\overline\Omega\times\overline\Omega\to\mathbb{R}_+,
\end{equation*}
with the same modulus of continuity. Moreover the definition at \eqref{ccontinuous} extends to all pairs $(x,y)\in\overline{\Omega}\times\overline{\Omega}$.
\end{prop}

\begin{proof}
The first part of the proof easily follows from \eqref{holderest}.

Let now consider $\phi\in C(\overline\Omega,\mathbb{R}_+)$, $\varepsilon_0>0$ and the continuous function $\phi+\varepsilon_0>0$. Given $(x,y)\in\overline\Omega\times\overline\Omega$ and $\left(x_n,y_n\right)_{n\in\mathbb{N}}\subset\Omega\times\Omega$, $\left(x_n,y_n\right)\to(x,y)$, we have   
\begin{equation*}
    c_{\phi+\varepsilon_0}(x,y):=\lim_{n\to+\infty}c_{\phi+\varepsilon_0}(x_n,y_n).
\end{equation*}
It means that $\forall \varepsilon>0$, there exists $n=n(\varepsilon)$ such that
\begin{equation*}
    \left|c_{\phi+\varepsilon_0}(x,y)-c_{\phi+\varepsilon_0}(x_n,y_n)\right|<\varepsilon,\quad\forall n>n(\varepsilon).
\end{equation*}
By definition of $c_{\phi+\varepsilon_0}(x_n,y_n)$ and by the invariance of $L_{\phi+\varepsilon_0}$ under reparametrization, there exists $\sigma_n\in\tilde{H}^{x_n,y_n}$ such that
\begin{equation*}
    |c_{\phi+\varepsilon_0}(x_n,y_n)-L_{\phi+\varepsilon_0}(\sigma_n)|<\varepsilon.
\end{equation*}
Hence, for any $n>n(\varepsilon)$ it holds that
\begin{equation*}
    \varepsilon_0|\dot{\sigma}_n|_H=\varepsilon_0\ell_{H}(\sigma_n)\leq L_{\phi+\varepsilon_0}(\sigma_n)<c_{\phi+\varepsilon_0}(x_n,y_n)+\varepsilon\leq M+\varepsilon,
\end{equation*}
where $M\geq0$. Then, the Ascoli-Arzelà Theorem implies that $\left(\sigma_n\right)_{n>n(\varepsilon_0)}\subset H$ admits a subsequence $\sigma_{n_k}\to\sigma$ uniformly as $k\to+\infty$. From \cite[Theorem 3.41]{barilariagrachev} it follows that $\sigma\in H^{x,y}$ and the lower semicontinuity implies $L_{\phi+\varepsilon_0}(\sigma)\leq\liminf_{k\to+\infty}L_{\phi+\varepsilon_0}(\sigma_{n_k})=c_{\phi+\varepsilon_0}$. Then, we can conclude that $\forall\varepsilon_0>0$, $\forall (x,y)\in\overline\Omega\times\overline\Omega$, there exists $\sigma\in H^{x,y}$, such that 
\begin{equation*}
    L_{\phi+\varepsilon_0}(\sigma)\leq c_{\phi+\varepsilon_0}(x,y).
\end{equation*}
Moreover,
\begin{equation*}
    L_{\phi}(\sigma)+\varepsilon_0\ell_{H}(\sigma)=L_{\phi+\varepsilon_0}(\sigma)\leq c_{\phi+\varepsilon_0}(x,y)=\lim_{n\to+\infty}c_{\phi+\varepsilon_0}(x_n,y_n)\leq\lim_{n\to+\infty}c_{\phi}(x_n,y_n)+O(\varepsilon_0),
\end{equation*}
hence, letting $\varepsilon_0\to0$, $L_\phi(\sigma)\leq c_\phi(x,y)$.

It remains to prove that $c_{\phi}(x,y)\leq L_{\phi}(\sigma)$, for any $\sigma\in H^{x,y}$.

Let us suppose that $x\in\Omega$ and $y\in\partial\Omega$, all the other cases can be deduced from this one. Let us consider an arbitrary horizontal curve $\sigma\in H^{x,y}$ and, for any $n\in\mathbb{N}$, we take $y_n\in B\left(y,\frac{1}{n}\right)\cap\Omega$. From Remark \ref{nonemptinesshorcur} it follows that there exists a horizontal curve $\sigma_n\in H^{y,y_n}$, such that $\ell_{H}(\sigma_n)\leq 2d_{CC}(y,y_n)$. Hence, $L_\phi(\sigma_n)\leq 2\|\phi\|_\infty d_{CC}(y,y_n)$. Given $t_0\in[0,1]$, we denote by $\tilde\sigma_{n,t_0}\in H^{x,y_n}$ the horizontal curve defined as 
\begin{equation*}
    \tilde\sigma_{n,t_0}(t):=\begin{cases}
        \sigma\left(\frac{t}{t_0}\right),&\text{if }t\in[0,t_0],\\
        \sigma_n\left(\frac{t-t_0}{1-t_0}\right),&\text{if }t\in[t_0,1].
    \end{cases}
\end{equation*}

From the invariance of the weighted sub-Riemannian length under reparametrization, it follows that
\begin{equation*}
    c_\phi(x,y) := \lim_{n\to\infty} c_\phi(x,y_n)\leq \liminf_{n\to\infty} L_\phi (\tilde \sigma_{n,t_0})=\liminf_{n\to\infty}\left(L_\phi (\sigma) + L_\phi(\sigma_n)\right)= L_\phi(\sigma).
\end{equation*}
Since $\sigma\in H^{x,y}$ is arbitrary, the thesis follows.
\end{proof}

\begin{cor}\label{cxicomp}
Let $(\phi_n)_{n\in\mathbb{N}}\subset  C(\overline\Omega,\mathbb{R}_+)$ be a bounded sequence in $L^{q}$, then the sequence $(c_{\phi_n})_{n\in\mathbb{N}}$ admits a subsequence that converges in $C(\overline\Omega\times\overline\Omega)$.
\end{cor}

\begin{proof}
    The existence of a subsequence of $(c_{\phi_n})_{n\in\mathbb{N}}$ that converges in $C(\overline\Omega\times\overline\Omega)$ follows from Ascoli-Arzelà's theorem. Indeed, equicontinuity follows from Proposition \ref{12aprile}, while the pointwise boundness is a consequence of the identity $c_{\phi_n}(x,x)=0$ and \eqref{holderest}.
\end{proof}

From now on, we assume that $q>N$. The next goal is to give an equivalent definition for $c_\phi$, that extends the notion for positive functions in $L^{q}$.

\begin{prop}\label{ccoincid}
	If $\phi\in C(\overline\Omega,\mathbb{R}_+)$, then
\begin{equation*}
	c_\phi(x,y)=\sup\left\{c(x,y)\,:c\in\mathcal{C}(\phi)\right\},\quad \forall(x,y)\in\overline\Omega\times\overline\Omega,
\end{equation*}
where 
\begin{equation}\label{Setcphi}
\mathcal{C}(\phi)=\left\{c=\lim_{n\rightarrow+\infty} c_{\phi_n}\,\textnormal{ in }C(\overline{\Omega}\times\overline{\Omega})\,:\,(\phi_n)_{n\in\mathbb{N}}\subset C(\overline{\Omega},\mathbb{R}_+),\phi_n\to\phi\,\textnormal{ in }L^{q}\right\}.
\end{equation}
\end{prop}

We first state two technical remarks that will be useful in the proof.

 
\begin{Remark}
Note that if we have a constant coefficient unitary 
horizontal vector 
 $W_1:=a_1X_1+\ldots +a_nX_n+a_{n+1}X_{n+1}+\ldots+a_{2n}X_{2n}\in\mathfrak{h}_1^1$, 
 it is  possible to perform a change of variable which sends the 
 vector $W_1$ to the first element of the canonical orthonormal basis. 
 Indeed, if we  denote by $W_2,\ldots,W_{2n}$ a basis of orthogonal complement $W_1^\perp$ in $\mathfrak{h}_1^1$ with respect to $\left\langle\cdot,\cdot\right\rangle_H$, and by $x$ a point, we can consider the change of variables $\Psi:\mathbb{R}^{2n+1}\to\mathbb{H}^{n},$
\begin{equation}\label{changeofcoordinates}
    \Psi(e_1,\ldots,e_{2n+1})=\exp(e_1W_1)\exp\left(\sum_{i=2}^{2n} e_{i}W_{i}+e_{2n+1}X_{2n+1}\right)(x).
\end{equation}
The pullback of the vector field $W_1$ by $\Psi$ is $\Psi_*W_1=\partial_{e_1}$ and the point $x$ will be the origin in the new  coordinate system.
\end{Remark}

\begin{Remark}\label{30luglio}
Any function $c\in \mathcal{C}(\phi)$ is a pseudo distance. Indeed, let $(\phi_n)_{n\in\mathbb{N}}\subset C(\overline\Omega,\mathbb{R}_+)$, such that $\phi_n\to\phi$ in $L^q$ and $c_{\phi_n}\to c$ in $C(\overline\Omega\times\overline\Omega)$. We know that the function $c_{\phi_n}$ is a pseudo distance, for any $n\in\mathbb{N}$. Hence, the thesis follows passing to the limit. In particular, for any $x,y,z\in\overline\Omega$ it holds
\begin{equation}\label{ineq}
		c(x,z)=\lim_{n\to+\infty}c_{\phi_n}(x,z)\leq\lim_{n\to+\infty}\left(c_{\phi_n}(x,y)+c_{\phi_n}(y,z)\right)=c(x,y)+c(y,z).
	\end{equation}
\end{Remark}

\begin{proof}[{\it Proof of Proposition \ref{ccoincid}}]
To simplify notations we call
$$\overline{c}_{\phi}= \sup\left\{c(x,y)\,:c\in\mathcal{C}(\phi)\right\},$$
so that we have to prove that $ \overline{c}_{\phi} = c_\phi.$

First we consider the constant sequence $\phi_n:=\phi,\quad\forall n\in\mathbb{N}$. Then $c_\phi\in\mathcal{C}(\phi)$ and we get that $\overline{c}_{\phi}\geq c_\phi$.

Let us prove the converse inequality. 
Let $x,y\in\overline\Omega$, $k>0$ and $\sigma\in H^{x,y}$ such that $L_{\phi}(\sigma)<c_\phi(x,y)+1/k$. 
Let us fix a sequence $\phi_n\rightarrow\phi$ in $L^{q}$ such that $c_{\phi_n}$ converges uniformly to some $c$, we want to prove that $c\leq c_{\phi}$.
From density of simple functions and continuity of $\phi$ we can assume that there exists a finite decomposition $\{t_0, t_1, \cdots t_M\}$ of the interval $[0,1]$ such that $\dot\sigma$ is constant and horizontal on the interval $[t_{i-1},t_i]$; in particular
\begin{equation*}
    L_{\phi_n}(\sigma)=\sum_{i=1}^{M}\int_{t_{i-1}}^{t_i}\phi_n(\sigma(t))|\dot\sigma(t)|_Hdt.
\end{equation*} 
Let us consider a single interval $[t_{i-1}, t_i]$: up to  a change of coordinates, we can also assume that $|\dot\sigma|_H=1$ on this interval. For this reason, in the change of coordinates  $\Psi_i:\mathbb{R}^{2n+1}\rightarrow\mathbb{H}^{n}$, introduced in \eqref{changeofcoordinates}, we can choose 
$\Phi_i(\sigma(t_{i-1}))=(t_{i-1},0)$ so that  
$\Phi_i(\sigma(t_{i}))=(t_{i},0)$, and $$\Phi_i \circ \sigma : [t_{i-1}, t_{i}] \to\mathbb{R}^{2n+1}, \quad (\Phi_i \circ \sigma)(t) = (t, 0),$$
where $\Phi_i:=\Psi_i^{-1}:\mathbb{H}^n\to\mathbb{R}^{2n+1}$.

We now consider, for every $\delta>0$ and for every $i$, cylindrical neighborhoods $C_{i, \delta} = \{(t, \hat e)\in\mathbb{R}^{2n+1}: t \in [t_{i-1}, t_i], |\hat e|_{\mathbb{R}^{2n}} \leq \delta\},$ of the curve $\Phi_i \circ \sigma$, with basis
$S_{i-1} =\{(t_{i-1}, \hat e): |\hat e|_{\mathbb{R}^{2n}} \leq \delta\} .$
For every $\hat e\in \mathbb{R}^{2n},$ with  $|\hat e|_{\mathbb{R}^{2n}}\leq \delta$, we call  $\sigma_e (t) =\Psi_i(t, \hat e)$. By definition 
$$
c_{\phi_n}\Big(\Psi_i(t_{i-1}, \hat e) , \Psi_i(t_i, \hat e)\Big)\leq 
L_{\phi_n}(\sigma_e \circ \theta_i),
$$
where $\theta_i$ is a  change of coordinate which sends $[0,1]$ to $[t_{i-1}, t_{i}]$. 
Note that 
\begin{equation}\label{questa}
L_{\phi_n}(\sigma_e \circ \theta_i) =  L_{\phi_n\circ\Psi_i }( \Phi_i \circ \sigma_e \circ \theta_i) = \int_{t_{i-1}} ^{t_i}  ( \phi_n\circ\Psi_i)(t, \hat e) dt.
\end{equation}
Hence, integrating on $S_{i-1}$  we get
\begin{equation}\label{cpne}
	\int_{S_{i-1}}c_{\phi_n}\Big(\Psi_i(t_{i-1}, \hat e), \Psi_i(t_i, \hat e)\Big) d\mathcal{L}^{2n}(\hat e)\leq \int_{S_{i-1}} \int_{t_{i-1}}^{t_i} ( \phi_n\circ\Psi_i)(t, \hat e) dt d\mathcal{L}^{2n}(\hat e).
\end{equation}
For $ n \to \infty$ using the uniform convergence of $c_{\phi_n}$ to $c$ and the $L^{q}$ convergence of $\phi_n$ to $\phi$ we get
that
\begin{equation*}
    \int_{S_{i-1}} c\Big(\Psi_i(t_{i-1}, \hat e), \Psi_i(t_i, \hat e)\Big)d\mathcal{L}^{2n}(\hat e)\leq\int_{C_i}(\phi\circ\Psi_i) (t, \hat e) d\mathcal{L}^{2n+1} (t, \hat e).
\end{equation*}

Now we divide by the measure of $S_{i-1}$  and pass to the limit as $\delta\rightarrow0^+$.
Using the fact that $c$ is continuous
\begin{equation*}
	\lim_{\delta\to0^+}\frac{1}{d\mathcal{L}^{2n}(S_{i-1})}\int_{S_{i-1}} c\Big(\Psi_i(t_{i-1}, \hat e), \Psi_i(t_i, \hat e)\Big))d\mathcal{L}^{2n}(\hat e)=c\Big(\Psi_i(t_{i-1}, 0), \Psi_i(t_i, 0)\Big) = c(x^{i-1},x^i),
\end{equation*}
where  $x^i = \sigma(t_i)$, 
Analogously  the integral over $C_i =[t_{i-1}, t_i] \times S_{i-1}$ divided by the measure of $S_{i-1} $ converges to the integral on $[t_{i-1}, t_i] $, which is the integral along the curve $\Phi_i\circ\sigma(t)$
\begin{equation*}
	\lim_{\delta\to 0^+}\frac{1}{d\mathcal{L}^{2n}(S_{i-1})}\int_{C_i}(\phi\circ\Psi_i)(t, \hat e) d\mathcal{L}^{2n+1}(t, \hat e)= \int_{t_{i-1}}^{t_i}
  (\phi\circ\Psi_i)(t, 0) dt = 
\int_{t_{i-1}}^{t_i}  \phi(\sigma(t))|\dot\sigma(t)|_Hdt.
\end{equation*}
Then, using \eqref{cpne}, we get that 
\begin{equation*}
 c(x^{i-1},x^{i}) \leq \int_{t_{i-1}}^{t_i}\phi(\sigma(t))|\dot\sigma(t)|_Hdt,\quad \forall i=1,\ldots,M,
\end{equation*}
and then, from \eqref{ineq},
\begin{equation*}
    c(x,y)\leq
 \sum_{i=1}^{M}c(x^{i-1},x^{i}) 
		\leq\sum_{i=1}^{M}\int_{t_{i-1}}^{t_i}\phi(\sigma(t))|\dot\sigma(t)|_Hdt=L_{\phi}(\sigma).
\end{equation*}
This gives 
\begin{equation*}
    c(x,y)\leq c_{\phi}(x,y)+\frac{1}{k}
\end{equation*}
for the choice of $\sigma$ and, since $k$ is arbitrary, it follows that $c(x,y)\leq c_{\phi}(x,y)$.
\end{proof}

Since definition \eqref{barcphi} makes sense also for $L^{q}$ functions, and extends \eqref{ccontinuous}, we will use it as definition of $c_\phi$, for any non-negative function $\phi\in L^{q}(\Omega)$. 

\begin{deff}
If $\phi\in L^{q}(\Omega)$, $\phi\geq0$ then we define
\begin{equation}\label{barcphi}
    c_\phi(x,y)=\sup\left\{c(x,y)\,:c\in\mathcal{C}(\phi)\right\},
\end{equation}
where $\mathcal{C}(\phi)$ has been defined in 
\ref{Setcphi}. 
\end{deff}

Following the same argument as in \cite[Lemma 3.5]{Santambrogio1} we can prove the following result.
\begin{lemma}\label{existapprox}
	Let $q>N$, $\phi\in L^{q}(\Omega)$, $\phi\geq 0$, then there exists a sequence $(\phi_n)_{n\in\mathbb{N}}\subset C(\overline{\Omega},\mathbb{R}_+),\, \phi_n\to\phi\,\mbox{ in }L^{q}(\Omega)$, such that $c_{\phi_n}$ converges to $c_\phi$ in $C(\overline{\Omega}\times\overline{\Omega})$. 
\end{lemma}
 
Finally, as consequence of Remark \ref{30luglio} and Lemma \ref{existapprox}, the function $c_\phi$ is a pseudo distance.

\section{Congested optimal transport problem in $\mathbb{H}^n$}

The scope of this section is to adapt the congested optimal transport problem, originally proposed by Carlier et al. in \cite{Santambrogio1} for the Euclidean setting, to the Heisenberg Group. This adaptation employs the notion of horizontal traffic intensity, which can be viewed as a more abstract version of the horizontal transport density introduced in Section 3. While the latter is defined by integrating over geodesics with extremes in the the space, the former is obtained by integrating over the whole space of horizontal curves.

\subsection{Horizontal traffic plans and traffic intensity}

Let $\Omega\subset\mathbb{H}^{n}$ be a bounded domain with $C^{1,1}$ boundary, describing the area where  the transport problem takes place. Let $\mu,\nu\in\mathcal{P}(\overline\Omega)$ be two probability measures representing the initial and  final states of the system. We introduce the notion of horizontal traffic plan as a  probability measures over the space $H$ of horizontal curves, defined in \eqref{acca}.

A \textit{horizontal traffic plan admissible between} $\mu$ and $\nu$ is a probability measure $Q\in\mathcal{P}(H)$ such that $(e_0)_{\#}Q=\mu$ and $(e_1)_{\#}Q=\nu$, where $e_0$ and $e_1$ are the evaluation maps at times $t=0$ and $t=1$, respectively.

In the Euclidean setting, the convexity of $\Omega$ guarantees the existence of at least one traffic plan. In view of results contained in \cite{Monti2}, in the next remark we have to replace the convexity of $\Omega$ with a different geometric assumption.

\begin{Remark}
Assume that $\overline{\Omega}$ contains the trajectories of geodesics with extremes in the support of $\mu$ and $\nu$ respectively: 
\begin{equation*}
    \mathcal{T}(\mu,\nu):=\left\{S_t(x,y):x\in\textnormal{supp}(\mu), y\in\textnormal{supp}(\nu), t\in[0,1]\right\}\subseteq\overline\Omega,
\end{equation*}
where $S$ is the selection of geodesics fixed in \eqref{31luglio} and $\gamma\in\Pi(\mu,\nu)$, then  \begin{equation}\label{traffpl}
    Q_\gamma:=S_\#\gamma
\end{equation}
is an horizontal  traffic plan between $\mu$ and $\nu$
\end{Remark}

In the sequel we will   denote 
$$\mathcal{Q}_{\mathbb{H}}(\mu,\nu):=\{\text{horizontal traffic plans admissible between $\mu$ and $\nu$}\}.$$

One can associate to  any horizontal traffic plan  $Q\in\mathcal{Q}_{\mathbb{H}}(\mu,\nu)$ a positive Radon measure $i_{Q}$ over $\overline\Omega$, such that
\begin{align}\label{defiQ}
    \int_{\overline{\Omega}} \phi(x) di_{Q}(x):=\int_{H} L_{\phi}(\sigma) d Q(\sigma),\quad \forall \phi \in C(\overline{\Omega},\mathbb{R}_+).
\end{align}
This measure is called the \textit{horizontal traffic intensity}. 
The value $i_{Q}(A)$  of the intensity on a Borel set $A$,  
provides an estimate of how much traffic there is along the horizontal curves in $A$, selected by the traffic assignment $Q$.

\begin{Remark}
The horizontal traffic intensity is a generalization of the horizontal transport density introduced in Section \ref{sectrandens}: indeed, if $\gamma\in\Pi_1(\mu,\nu)$ and $Q_\gamma$ is as in \eqref{traffpl}, then
\begin{equation}\label{transporttraffic}
i_{Q_\gamma}=a_\gamma.
\end{equation}
\end{Remark}
 


Congestion effects are modeled by a continuous function 
$$g:\mathbb{R}_+\rightarrow\mathbb{R}_+,$$
such that
\begin{enumerate}
    \item $g$ is strictly increasing;
    \item $\lim_{i\to\infty}g(i)=+\infty$.
\end{enumerate}

We refer to such a function as a \textit{congestion function}.

Given $Q\in\mathcal{Q}_{\mathbb{H}}(\mu,\nu)$, we denote by
\begin{equation}\label{congfun}
\phi_{Q}(x):=
\begin{cases}
    g(i_{Q}(x)),\quad &\textnormal{if }i_Q\ll\mathcal{L}^{2n+1},\\
    +\infty,\quad &\textnormal{otherwise},
\end{cases}
\end{equation}
where $i_Q(x)$ is the density of the measure $i_Q$ with respect to the Lebesgue measure $\mathcal{L}^{2n+1}$.  

\begin{Remark}
    The existence of at least one $Q$ such that $i_{Q}\ll\mathcal{L}^{2n+1}$ depends again on $\mu$ and $\nu$. For instance, if either $\mu\ll\mathcal{L}^{2n+1}$ or $\nu\ll\mathcal{L}^{2n+1}$, then the existence of such a $Q$ follow from \eqref{transporttraffic} and Theorem \ref{absolutecont}. 
\end{Remark} 

\subsection{Horizontal Wardrop equilibria}

In this subsection, we introduce the notion of Wardrop equilibrium in the Heisenberg group. This concept was first introduced in the discrete Euclidean setting in \cite{Wardrop} and later formalized in the continuous Euclidean setting in \cite{Santambrogio1}. A Wardrop equilibrium describes the minimization of the cost of transport through two competing minimization problems. First, since the length of curves, weighted by the traffic intensity and the congestion function, represents the cost of transport, an optimal horizontal traffic plan must be concentrated on geodesics with respect to a suitable congested metric. Second, the associated transport plan must solve the Monge-Kantorovich problem associated with this congested metric.

Let $p>1$ and let us suppose that 
\begin{equation}\label{31luglio2}
    \mathcal{Q}_{\mathbb{H}}^p(\mu,\nu):=\left\{Q\in\mathcal{Q}_{\mathbb{H}}(\mu,\nu):i_{Q}\in L^p(\Omega)\right\}\not=\emptyset.
\end{equation}

If for instance $\mu,\nu\in L^p(\Omega)$, then \eqref{31luglio2} holds, see \cite{circelli2024continuous}.

We assume that
that $\exists a,b\in\mathbb{R}_+$ such that the congestion function satisfies
\begin{equation}\label{growthg}
	ai^{p-1}\leq g(i)\leq b(i^{p-1}+1).
\end{equation}

The $(p-1)$ - growth condition \eqref{growthg} implies that 
\begin{equation}\label{3ottbre}
    \phi_Q:=g\circ i_{Q}\in L^{q}(\Omega),\quad \textnormal{with }q:=\frac{p}{p-1}
\end{equation}
for every $Q\in\mathcal{Q}^p_{\mathbb{H}}(\mu,\nu)$.

The first step is to extend the definition of weighted length of horizontal curves to positive $q$-summable weights.

\begin{teo}\label{prolLxi} Let $Q\in \mathcal{Q}^p_{\mathbb{H}}(\mu,\nu)$. If $\phi\in L^{q}(\Omega),\phi\geq0$, with $q>N$, and $(\phi_n)_{n\in\mathbb{N}}\subset C(\overline{\Omega},\mathbb{R}_+)$ is a sequence such that $\phi_n\rightarrow\phi$ in $L^{q}$, then:
\begin{enumerate}
    \item[(i)] $L_{\phi_n}\to L_\phi$ in $L^1(H,Q)$, where $L_\phi$ is independent of the  approximating sequence $(\phi_n)_{n\in\mathbb{N}}$. 
    \item[(ii)] It holds
    \begin{equation}\label{eglc}
        \int_{\Omega} \phi(x) i_{Q}(x)dx=\int_{H} L_{\phi}(\sigma) dQ(\sigma).
    \end{equation}
    \item[(iii)] It holds
    \begin{equation}\label{ineglc}
		L_{\phi}(\sigma)\geq c_{\phi}(\sigma(0),\sigma(1)),\quad \text{ for }Q-\text{a.e. } \sigma\in H,
    \end{equation}
    where $c_{\phi}$ is defined in \eqref{barcphi}.
\end{enumerate}
\end{teo}

The proof follows from Lemma \ref{existapprox} by passing to the limit, see \cite[Lemma 3.6]{Santambrogio1}.

If we suppose that $p < \frac{N}{N-1}$, then the dual exponent $q > N$. From \eqref{3ottbre}, it follows that both the length $L_{\phi_Q}$, weighted by the traffic intensity and the congestion function, and the associated \textit{congested metric} $c_{\phi_Q}$ are well-defined for every $Q \in \mathcal{Q}_H^p(\mu, \nu)$. See \eqref{barcphi} and Theorem \ref{prolLxi}.

The definition of Wardrop equilibria in $\mathbb{H}^n$ can be given as follows.

\begin{deff}\label{Wardrop}
A horizontal Wardrop equilibrium is a horizontal traffic plan $Q\in\mathcal{Q}^p_{\mathbb{H}}(\mu,\nu)$ such that
\begin{equation}\label{6agosto1}
    Q\big{(}\left\{\sigma\in H: L_{\phi_Q}(\sigma)=c_{\phi_{Q}}(\sigma(0),\sigma(1))\right\}\big{)}=1
\end{equation}
and $\gamma_{Q}:=(e_0,e_1)_{\#}Q\in\Pi(\mu,\nu)$ solves the Monge-Kantorovich problem
\begin{equation}\label{6agosto}
    \inf_{\gamma\in\Pi(\mu,\nu)}\int_{\overline{\Omega}\times\overline{\Omega}}c_{\phi_{Q}}(x,y)d\gamma(x,y).
\end{equation}
\end{deff}

This definition describes an equilibrium in the sense that it is not a priori clear that the two problems can be minimized simultaneously.

We now introduce a convex minimization problem, whose solutions are Wardrop equilibria in $\mathbb{H}^n$. This problem is the Heisenberg analogue of the one proposed in \cite{Santambrogio1}, which was strongly inspired by the discrete case \cite{Beckmann1}.

We call \textit{congested optimal transport problem in} $\mathbb{H}^n$ the following minimization problem
\begin{equation}\label{lepbme1}
	\inf_{Q\in \mathcal{Q}_{\mathbb{H}}^p(\mu,\nu)}\int_{\Omega}G(i_{Q}(x))dx,
\end{equation}
where $G(i):=\int_0^ig(z)dz$.

\begin{teo}
    The minimization problem \eqref{lepbme1} admits solutions.
\end{teo}

\begin{proof}
Due to the convexity of $G$, the result immediately follows from direct methods of calculus of variations see \cite[Theorem 2.10]{Santambrogio1}.
\end{proof}

The following theorem guarantees the existence of horizontal Wardrop equilibria, according to Definition \ref{Wardrop}..

\begin{teo}\label{cwe}
If $1<p<\frac{N}{N-1}$ then $Q\in\mathcal{Q}^p_{\mathbb{H}}(\mu,\nu)$ solves \eqref{lepbme1} if, and only if, it is a horizontal Wardrop equilibrium.
\end{teo}

\begin{proof}[Sketch of the proof]
The proof works as in \cite{Santambrogio1}. Here, we just show why solutions to \eqref{lepbme1} are equilibrium solutions.

As in \cite[Proposition 3.9]{Santambrogio1}, one can also prove that, given a solution $\overline{Q}\in \mathcal{Q}_{\mathbb{H}}^p(\mu,\nu)$ to \eqref{lepbme1}, then
\begin{equation}
	\int_{\Omega} \phi_{\overline{Q}}(x) i_{\overline{Q}}(x)dx=\inf_{\gamma\in\Pi(\mu,\nu)}\int_{\overline{\Omega}\times\overline{\Omega}}c_{\phi_{\overline{Q}}}(x,y)d\gamma(x,y).
\end{equation}

Moreover, if we denote by $\gamma_{\overline Q}:=(e_0,e_1)_{\#}\overline{Q}\in\Pi(\mu,\nu)$, it follows that
\begin{align*}
	\int_{\overline{\Omega}\times \overline{\Omega}} c_{\phi_{\overline{Q}}}(x,y)d\gamma_{\overline{Q}}(x,y)=\int_{H}c_{\phi_{\overline{Q}}}(\sigma(0),\sigma(1))d\overline{Q}(\sigma)\leq\\ \underset{\eqref{ineglc}}{\leq}\int_{H} L_{\phi_{\overline{Q}}}(\sigma) d\overline{Q}(\sigma)= \int_{\Omega} \phi_{\overline{Q}}(x) i_{\overline{Q}}(x)dx=\\= \inf_{\gamma\in \Pi(\mu,\nu)} \int_{\overline{\Omega}\times \overline{\Omega}} c_{\phi_{\overline{Q}}}(x,y)d\gamma(x,y).
\end{align*}
Hence, $\gamma_{\overline Q}$ solves the  Monge-Kantorovich problem associated with the congested metric $c_{\phi_{\overline{Q}}}$
\begin{equation}
	\inf_{\gamma\in \Pi(\mu,\nu)} \int_{\overline{\Omega}\times \overline{\Omega}} c_{\phi_{\overline{Q}}}(x,y)d\gamma(x,y).
\end{equation}
We also observe that
\begin{align*}
	\int_{H} L_{\phi_{\overline{Q}}}(\sigma) d\overline{Q}(\sigma)= \int_{\overline{\Omega}\times\overline{\Omega}} c_{\phi_{\overline{Q}}}(x,y)d\gamma_{\overline Q}(x,y)\\
	=\int_{H}c_{\phi_{\overline{Q}}}(\sigma(0),\sigma(1))d\overline{Q}(\sigma)
\end{align*}
and, since $L_{\phi_{\overline{Q}}}(\sigma)\geq c_{\phi_{\overline{Q}}}(\sigma(0),\sigma(1))$, we get
\[L_{\phi_{\overline{Q}}}(\sigma)=c_{\phi_{\overline{Q}}}(\sigma(0),\sigma(1)) \quad\mbox{ for }  \overline{Q} \mbox{-a.e. } \sigma.\]
\end{proof}


\begin{Remark}
Additionally, if $G$ is strictly convex, then given $Q_{1}$ and $Q_{2}$ which solves \eqref{lepbme1} it follows that $i_{Q_{1}}=i_{Q_{2}}$. In other words, equilibria are not necessarily unique but they all induce the same intensity or, equivalently, the same congested metric.
\end{Remark}

Additional constraints can be imposed on the problem. Specifically, one could minimize the total cost under the constraint $(e_0,e_1)_{\#}Q\in\Pi\subset\Pi(\mu,\nu)$ and, as a particular case, consider $\Pi=\{\overline{\gamma}\}$. In this scenario, the set of horizontal traffic plans is defined as:
\begin{equation*}
	\mathcal{Q}_{\mathbb{H}}(\overline{\gamma}):=\left\{Q\in \mathcal{P}(H) \mid (e_0,e_1)_\# Q=\overline{\gamma}\right\};
\end{equation*}
Wardrop equilibria are horizontal traffic plans belonging to:
$$\mathcal{Q}_{\mathbb{H}}^p(\overline{\gamma}):=\left\{Q\in\mathcal{Q}_{\mathbb{H}}(\overline{\gamma}) \mid i_{Q}\in L^p(\Omega)\right\}$$
and satisfy the first condition of Definition \ref{Wardrop}. All previous arguments can be adapted to this case to establish the existence of equilibria.
