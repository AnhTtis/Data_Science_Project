Let $n\geq1$. The $n$-th \textit{Heisenberg group} $\mathbb{H}^n$ is the connected and simply connected Lie group, whose Lie algebra $\mathfrak{h}^n$ is stratified of step $2$, i.e. 
$$ \mathfrak{h}^n= \mathfrak{h}^n_1 \oplus \mathfrak{h}^n_2,$$
where $\mathfrak{h}_1= \mathrm{span} \{X_1, \dots, X_n, X_{n+1}, \dots, X_{2n}\}$ is called \textit{horizontal layer}, $\mathfrak{h}_2=\mathrm{span} \{X_{2n+1} \}$ and the only non-trivial bracket-relation between the vector fields $X_1, \dots, X_{2n+1}$ is 
\begin{equation*}
	[X_j, X_{n+j}]=X_{2n+1},\quad \forall j=1,\ldots n.
\end{equation*}

The horizontal layer $\mathfrak{h}^n_1$ induces a sub-bundle of the tangent bundle, that we denote by $H \mathbb{H}^n$ and whose fibre at any $q \in \mathbb{H}^n$ is 
$$H_q \mathbb{H}^n=\mathrm{span} \{ X_1(q), \dots, X_n(q), X_{n+1}(q), \dots, X_{2n}(q) \}.$$

We call it \textit{horizontal bundle} and any section \textit{horizontal vector field}. 

For any $q \in \mathbb{H}^n$, we consider on $H_q \mathbb{H}^n$ an inner product $\langle \cdot, \cdot \rangle_{H,q}$ that makes $\{X_1(q), \dots, X_{2n}(q)\}$ an orthonormal basis. We denote by $|\cdot|_{H,q}$ the norm associated with such an inner product. 
As it is common in Riemannian geometry, we drop the index $q$ in the inner product, writing $\langle \cdot, \cdot \rangle_H,\forall q\in\mathbb{H}^n$. The same convention shall be adopted for the norm.

Given a smooth vector field $X:\mathbb{H}^n\rightarrow H\mathbb{H}^n$, $t\in(-\varepsilon,\varepsilon)$ and $q_0\in\mathbb{H}^n$, we denote by $\exp(tX)(q_0):=\sigma(t)$, where $\sigma$ is the curve that solves 
\begin{equation*}
	\begin{cases}
		\dot{\sigma}(t)=X(\sigma(t)),\\
		\sigma(0)=q_0.
	\end{cases}
\end{equation*}
If $X\in\mathfrak{h}^n$ and $q_0\in\mathbb{H}^n$, then the previous map is well-defined for any $t\in\mathbb{R}$, see for instance \cite[Proposition 2.1.53]{Bonfiglioli}. Moreover it holds that $$\exp(tX)(q_0)=q_0\cdot\exp(tX)(e),\quad \forall q_0\in\mathbb{H}^n,$$
where $\cdot$ denotes the group law and $e$ is the identity element.

Since the group is also nilpotent, then the exponential map $\mathrm{exp}: \mathfrak{h}^n\to \mathbb{H}^n$ defined as
\begin{equation*}
	\exp(X):=\exp(X)(e)
\end{equation*}
is a global diffeomorphism. Hence, every $q\in\mathbb{H}^n$ can be written in an unique way as
\begin{equation*}
	\label{coordinates}
	q= \mathrm{exp}(x_1 X_1+\dots+x_n X_n+x_{n+1}X_{n+1}+\dots x_{2n}X_{2n}+x_{2n+1}X_{2n+1}),
\end{equation*} 
where $x_i\in\mathbb{R},\forall i=1,\ldots,2n+1$. 
This induces a system of globally defined coordinates on $\mathbb{H}^n$, by identifying any point $q \in \mathbb{H}^n$ with $(x_1,\ldots,x_{2n+1})\in \mathbb{R}^{2n+1}$. From now on we work in this system of coordinates and we write $x=(x_1,\ldots,x_{2n+1})\in\mathbb{H}^n\simeq\mathbb{R}^{2n+1}$. 

In this system of coordinates the group law reads as   
\begin{equation*}\label{grouplaw}
	x \cdot y := \bigg{(}x_1+y_1,\ldots,x_{2n}+y_{2n}, x_{2n+1}+y_{2n+1}+\frac{1}{2}\sum_{j=1}^{2n} (x_jy_{n+j}-x_{n+j}y_j)\bigg{)},
\end{equation*}
and the left-invariant vector fields $X_1,\ldots,X_{2n+1}$ as
\begin{equation*}
	\begin{cases}
		X_j := \partial_{x_j} -\frac{x_{n+j}}{2} \partial_{x_{2n+1}} \,, j=1,\dots,n,\\
		X_{n+j} := \partial_{x_{n+j}}+\frac{x_j}{2} \partial_{x_{2n+1}},\ \ j=1,\dots,n,\\
		X_{2n+1}=\partial_{x_{2n+1}}.	
	\end{cases}
\end{equation*}

Always in coordinates, the unit element $e\in\mathbb{H}^n$ is $0_{\mathbb{R}^{2n+1}}$, the center of the group is 
\begin{equation*}
	L := \{(0,\ldots,0,x_{2n+1}) \in \mathbb{H}^n ;\; x_{2n+1} \in \mathbb{R}\},
\end{equation*}
and, according to the two steps stratification of $\mathfrak{h}^n$, there exists family of non-isotropic dilations that reads as
\begin{equation*}
	\delta_\lambda((x_1,\ldots,x_{2n+1})):= (\lambda x_1,\ldots,\lambda x_{2n},\lambda^2x_{2n+1}),\quad \forall x\in\mathbb{H}^n, \forall \lambda>0.
\end{equation*} 

The homogeneous dimension of $\mathbb{H}^n$ is 
\begin{equation}\label{homogdim}
	N:=\sum_{j=1}^2j\dim(\mathfrak{h}^n_j)=2n+2.
\end{equation}

See \cite{Bonfiglioli}, in particular Section 2, for a general overview on Carnot groups.

Let us just remark that the Lebesgue measure $\mathcal{L}^{2n+1}$, which we shall also denote by $dx$, is the Haar measure of the group $\mathbb{H}^n\simeq\mathbb{R}^{2n+1}$.

Let now consider an open set $\Omega\subseteq\mathbb{H}^n$ and a measurable function $f:\Omega\to\mathbb{R}$. We denote by $$\nabla_Hf=(X_1f,\ldots,X_{2n}f),$$
where $X_jf$ is the derivative of $f$ in the horizontal direction $X_j$, in the sense of distributions. 

Let $1\leq p\leq\infty$, then the space
\begin{equation}\label{horsob}
    HW^{1,p}(\Omega):=\left\{f:\Omega\to\mathbb{R} \text{ measurable : }f\in L^{p}(\Omega),\nabla_Hf\in L^{p}(\Omega)\right\},
\end{equation}
equipped with the norm
\begin{equation*}
    \|f\|_{HW^{1,p}(\Omega)}:=\|f\|_{L^p(\Omega)}+\|\nabla_H f\|_{L^p(\Omega)},
\end{equation*}
is a Banach space. Moreover, for $1\leq p<\infty$, we denote by
\begin{equation*}
    HW_0^{1,p}(\Omega):=\overline{C_0^{\infty}(\Omega)}^{HW^{1,p}(\Omega)}
\end{equation*}
and by
\begin{equation*}
    HW^{-1,p'}(\Omega):=( HW_0^{1,p}(\Omega))'.
\end{equation*}


\subsubsection{\textbf{Carnot-Carath\'eodory distance}}
We can equip $\mathbb{H}^n$ with a sub-Riemannian distance, also known as Carnot-Carath\'eodory distance, that makes it a polish space.

We say that a Lipschitz curve $\sigma\in \text{Lip}([a,b],\mathbb{R}^{2n+1})$, is \textit{horizontal} if its velocity vector $\dot{\sigma}(t)$ belongs to $H_{\sigma(t)}\mathbb{H}^n$ at almost every $t\in[a,b]$ where it exists.
We will denote by
\begin{equation*}
	H([a,b],\mathbb{R}^{2n+1}):=\left\{\sigma\in \text{Lip}\left([a,b],\mathbb{R}^{2n+1}\right):\sigma\text{ is horizontal}\right\}.
\end{equation*}

Given a non negative continuous function $\phi\in C(\mathbb{H}^n,\mathbb{R}_+)$, the \textit{horizontal length of} $\sigma\in H([a,b],\mathbb{R}^{2n+1})$ \textit{weighted by} $\phi$ is
\begin{equation}\label{3agosto}
    L_\phi(\sigma):=\int_a^b\phi(\sigma(t))|\dot\sigma(t)|_Hdt.
\end{equation}
When $\phi\equiv1$, it reads as the \textit{horizontal length} of $\sigma$
\begin{equation}\label{horlength}
\ell_H(\sigma):=\int_a^b|\dot{\sigma}(t)|_H dt.
\end{equation}
Given $x,y\in\mathbb{H}^n$, one can define the \textit{Carnot-Caratheodory distance} (shortly \textit{CC-distance}) between them as 
\begin{equation}\label{CCdistance}
	d_{CC}(x,y):= \inf \  \left\{  \ell_H( \sigma) \ | \  \sigma\in H\left([a,b],\mathbb{R}^{2n+1}\right) , \ \sigma(a)=x, \ \sigma(b)=y \right\}.
\end{equation}

The Chow-Rashevskii theorem guarantees that this distance is well-defined and it induces the Euclidean topology on $\mathbb{H}^n\simeq\mathbb{R}^{2n+1}$, see for instance \cite[Theorem 3.31]{barilariagrachev}. 

This distance is left invariant and 1-homogeneous with respect to the dilations, i.e.
\begin{equation*}
	d_{CC}(x \cdot y, x\cdot z) = d_{CC}(y,z) \quad \text{and} \quad d_{CC}(\delta_\lambda(y), \delta_\lambda(z)) = \lambda\,d_{CC}(y,z)
\end{equation*}
for all $x$, $y$, $z\in\mathbb{H}^n$ and all $\lambda>0$. 

\subsubsection{\textbf{Geodesics}}
We call \textit{minimizing horizontal curve} any $\sigma\in H([a,b],\mathbb{R}^{2n+1})$ such that
\begin{equation*}
	\ell_H(\sigma)=d_{CC}(\sigma(a),\sigma(b)).
\end{equation*}
According to the terminology used in literature, we call \textit{geodesic} any minimizing horizontal curve parametrized proportionally to the arc-length, i.e. $d_{CC} (\sigma(t),\sigma(t'))=|t-t'|v$, where $v=\frac{d_{CC}(\sigma(a),\sigma(b))}{b-a}$ is the constant speed of $\sigma$.  The sub-Riemannian version of the Hopf-Rinow theorem implies that $(\mathbb{H}^n, d_{CC})$ is a geodesic space, see for instance \cite[Theorem 2.4]{FigalliRifford}. Moreover, form \cite[Theorem 7.29]{Villani} follows that $(\mathbb{H}^n, d_{CC})$ is a non-branching metric space, then any two geodesics that coincide on a non-trivial interval coincide on the whole intersection of their intervals of definition.


In the Heisenberg group geodesics have been computed explicitly and so it was possible to detect a set in which these are unique. See \cite{Rigot}, \cite{DePascale2} or \cite{Monti}.

We denote by
\begin{equation} \label{KAPPA}
	E:= \{(x,y)\in \mathbb{H}^n\times \mathbb{H}^n;\,\, x^{-1}\cdot y \not \in L\},
\end{equation}
then it holds the following characterization for geodesics parametrized on $[0,1]$.

\begin{teo}\label{geod} 
A non-trivial geodesic starting from $0$ is the restriction to $[0,1]$ of the curve $$\sigma_{\chi,\theta}(t)=\left(x_1(t),\ldots,x_{2n+1}(t)\right)$$ either of the form
\begin{align}\label{geodform}
    &x_j(t)=\frac{\chi_j\sin(\theta t)-\chi_{n+j}\left(1-\cos(\theta t)\right)}{\theta},\quad j=1,\ldots,n\\
    &x_{n+j}(t)=\frac{\chi_{n+j}\sin(\theta t)+\chi_{j}\left(1-\cos(\theta t)\right)}{\theta},\quad j=1,\ldots,n\\
    &x_{2n+1}(t)=\frac{|\chi|^2}{2\theta^2}\left(\theta t-\sin(\theta t)\right),
\end{align}
for some $\chi \in \mathbb{R}^{2n}\setminus\{0\}$ and $\theta\in [-2\pi,2\pi]\setminus\left\{0\right\}$, or of the form
\begin{equation*}
    \left(x_1(t),\ldots,x_{2n+1}(t)\right)=\left(\chi_1t,\ldots,\chi_{2n}t,0\right), 
\end{equation*}
for some $\chi \in \mathbb{R}^{2n}\setminus\{0\}$ and $\theta=0$. In particular, it holds $$|\chi|_{\mathbb{R}^{2n}}=|\dot\sigma|_H=d_{CC}(0,\sigma(1)).$$
	
In particular it is a horizontal curve and it holds:
\begin{enumerate}
    \item For all $(x,y)\in E$, there is a unique geodesic $x\cdot\sigma_{\chi,\theta}$ parametrized on $[0,1]$ between $x$ and $y$, for some $\chi \in \mathbb{R}^{2n}\setminus\{0\}$ and some $\varphi\in(-2\pi,2\pi)$.
    \item If $(x,y)\not\in E$, then $x^{-1}\cdot y = (0,\ldots,0,z_{2n+1})$ for some $z_{2n+1}\in \mathbb{R}\setminus\{0\}$. Hence, there are infinitely many geodesics parametrized on $[0,1]$ between $x$ and $y$: they are all the curves of the form $x\cdot \sigma_{\chi,2\pi}$, if $z_{2n+1}>0$,  $x\cdot \sigma_{\chi,-2\pi}$, if $z_{2n+1}<0$, for any $\chi \in \mathbb{R}^{2n}$ such that $|\chi|_{\mathbb{R}^{2n}} = \sqrt{4\pi|z_{2n+1}|} $.
\end{enumerate}
\end{teo}

We denote by 
\begin{equation*}
    \textnormal{Geo}(\mathbb{H}^n):=\left\{\sigma\in H\left([0,1],\mathbb{R}^{2n+1}\right):\sigma\textnormal{ is a geodesic}\right\}.
\end{equation*}

From the theory of Souslin sets and general theorems about measurable selections, it follows that there exists a  map 
\begin{equation}\label{19marzo1}
	S:\mathbb{H}^n\times\mathbb{H}^n\rightarrow\textnormal{Geo}(\mathbb{H}^n)
\end{equation}
such that for every $x,y \in \mathbb{H}^n$, the value  $S(x,y)=\sigma_{x,y}$ is a geodesic between $x$ and $y$ and $S$ is $\gamma$-measurable for any positive Borel measure $\gamma$ on $\mathbb{H}^n\times\mathbb{H}^n$ (see for instance \cite[Theorem 6.9.2 and Theorem 7.4.1]{Bogachev}). Moreover $S$ is continuous, hence Borel, on the set $E$, defined in \eqref{KAPPA}. The map $S$ is often called \textit{selection of geodesics} map. 

If $e_t$ is the evaluation map at $t\in[0,1]$, i.e. $e_t(\sigma) := \sigma(t)$ for all $\sigma \in C\left([0,1],\mathbb{R}^{2n+1}\right)$, then the map
\begin{equation}\label{geod01}
	S_t:=e_t\circ S:\mathbb{H}^n\times\mathbb{H}^n\rightarrow\mathbb{H}^n,
\end{equation}
associates with any two points $x,y\in\mathbb{H}^n$, the point $S_t(x,y)$ of $\mathbb{H}^n$ at distance $t \, d_{CC}(x,y)$ from $x$, on the selected geodesic $S(x,y)$ between $x$ and $y$. If we fix $\overline{y}\in\mathbb{H}^n$ and $t\in(0,1)$, then the function $S_t(\cdot,\overline{y})$ is $C^{\infty}$ on $\mathbb{H}^n\setminus(\overline{y}\cdot L)$ and it holds that
\begin{equation}\label{det}
	\det D_x(S_t(x,\overline{y}))\geq(1-t)^{2n+3},
\end{equation}
for all $x\in\mathbb{H}^n\setminus(\overline{y}\cdot L)$. Moreover, for any $y\in\mathbb{H}^n$ and any $A\subset \mathbb{H}^n\setminus \left(y\cdot L\right)$,
\begin{equation}\label{MCP}
	\mathcal{L}^{2n+1}(A) \leq \frac{1}{(1-t)^{2n+3}} \,\mathcal{L}^{2n+1}(S_t(A,y)),
\end{equation}
which means that $(\mathbb{H}^n,d_{CC},\mathcal{L}^{2n+1})$ satisfies a so-called Measure Contraction Property $MCP(0,2n+3)$: see \cite[Section 2]{Juillet} for proofs of these results. The measure contraction property is a generalization to metric measure spaces of the concept of Ricci curvature bounded by below. This notion was introduced by Otha in \cite{ohta2007measure}: it controls the distortion of measures along geodesics. Recently in \cite{badreddine2020measure} the authors proved that every two-step compact sub-Riemannian manifold and every Lipschitz Carnot group satisfy $MCP(0,R)$, for some $R>0$. See also \cite{barilari2018sharp}, \cite{rifford2013ricci}, \cite{rizzi2016measure} and references therein for further results in this direction.

We explicitly recall that formulas \eqref{det} and \eqref{MCP} extend to this setting 
relations well known in the Euclidean setting, where quantity $1-t$ is raised to  dimension of the space. Here the exponent $2n+3$ is neither the topological dimension  $2n+1$, nor the homogeneous dimension $N=2n+2$ of the Heisenberg group. It is the so called geodesic dimension of the space and  Juillet, in the \cite[Remark 2.3]{Juillet}, shows that this exponent is sharp. 


