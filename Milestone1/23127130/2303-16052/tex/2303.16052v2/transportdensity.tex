\section{Horizontal transport density on $\mathbb{H}^n$}\label{sectrandens}
In this section we introduce the notion of horizontal transport density, extending to the Heisenberg group the presentation provided in \cite{Santambrogio2}.  A horizontal transport density is a measure representing the amount of transport taking place along geodesics in each region of $\mathbb{H}^n$. In particular, we study conditions under which transport densities are Lebesgue absolutely continuous w.r.t. the Haar measure of the group, with $L^p$ density.


Let $\mu,\nu\in\mathcal{P}_c(\mathbb{H}^n)$ and let us fix a selection of geodesics
\begin{equation}\label{31luglio}
    S:\mathbb{H}^n\times\mathbb{H}^n \rightarrow \textnormal{Geo}(\mathbb{H}^n), 
\end{equation}
$S(x,y)=\sigma_{x,y}\in\textnormal{Geo}(\mathbb{H}^n)$, that is $\gamma$-measurable for any $\gamma\in\Pi_1(\mu,\nu)$, according to \eqref{19marzo1}. One can associate with any optimal transport plan $\gamma\in\Pi_1(\mu,\nu)$ a positive and finite Radon measure $a_\gamma\in\mathcal{M}_+(\mathbb{H}^n)$, defined as 
\begin{equation}\label{transport density}
	\int_{\mathbb{H}^n} \phi(x)da_\gamma(x):=\int_{\mathbb{H}^n\times\mathbb{H}^n}L_\phi(\sigma_{x,y})d\gamma(x,y),\quad\forall\phi\in C_c(\mathbb{H}^n,\mathbb{R}_+).
\end{equation}
Here $L_\phi(\sigma_{x,y})$ denotes the horizontal length of $\sigma_{x,y}$, weighted by $\phi$, see \eqref{3agosto} for its definition. The total mass of $a_\gamma$ satisfies
\begin{equation*}
    a_\gamma(\mathbb{H}^n)\leq\min_{\tilde\gamma\in\Pi(\mu,\nu)}\int_{\mathbb{H}^n\times\mathbb{H}^n}d_{CC}(x,y)d\tilde\gamma(x,y).
\end{equation*}

Moreover, the measure $a_\gamma$ is a compactly supported measure, see \cite{circelli2024continuous}. 
This measure is generally called \textit{horizontal transport density}.  By definition, it also follows that if $A$ is a Borel set, then
\begin{equation}\label{densitylength}
	a_\gamma(A)=\int_{\mathbb{H}^n\times\mathbb{H}^n}\mathcal{H}^1(A\cap S(x,y))d\gamma(x,y).
\end{equation}

Let us remark that if either $\mu\ll\mathcal{L}^{2n+1}$, or $\nu\ll\mathcal{L}^{2n+1}$, then $a_\gamma$ does not depend on the fixed selection $S$. See Proposition \ref{pi1.1}.

\subsection{Absolute continuity of horizontal transport densities}
The first goal is to prove the existence of at least one horizontal transport density that is absolutely continuous w.r.t. the Haar measure of the group.

Given $\gamma\in\Pi_1(\mu,\nu)$, we denote by $\mu_t$ the displacement interpolation between $\mu$ and $\nu$  
\begin{equation*}\label{interpolation}
	\mu_t:=((S_t)_\# \gamma)_{t\in [0,1]}.
\end{equation*}
Hence, the horizontal transport density $a_\gamma$ may be written as 
\begin{equation*}\label{density2}
	a_\gamma=\int_0^1(S_t)_\# (d_{CC}\ \gamma)dt,
\end{equation*}
where $d_{CC}\ \gamma$ is a positive Borel measure on $\mathbb{H}^n\times\mathbb{H}^n$.
Since $\mu$ and $\nu$ have bounded support, then there exists $C>0$ such that $d_{CC}(x,y)\leq C$, for any $(x,y)\in\textnormal{supp}(\gamma)$ and hence 
\begin{equation}
	\label{density3}
	a_\gamma\leq C\int_0^1\mu_t dt.
\end{equation}
In order to prove that $a_\gamma$ is absolutely continuous w.r.t. $\mathcal{L}^{2n+1}$, it is enough to prove that $\mu_t$ is absolutely continuous w.r.t. $\mathcal{L}^{2n+1}$, for almost every $t\in[0,1]$. In this way we would get that, whenever $\mathcal{L}^{2n+1}(A)=0$, then 
\begin{equation}\label{density4}
a_\gamma(A)\leq C\int_0^1\mu_t(A)dt=0.
\end{equation}
\\

We introduce now  the following lemma which guarantees that minimizing geodesics arising in 
optimal plans cannot intersect at intermediate points. This result will be useful in the proof of Theorem \ref{absolutecont}.

\begin{lemma} \label{pi1.3}
	Let $\gamma\in \Pi_1(\mu,\nu)$. Then $\gamma$ is concentrated on a set $\Gamma$ such that $\forall(x,y),(x',y')\in \Gamma$ with $(x,y)\not=(x',y')$, if two transport rays between these two pairs of points intersect at an interior point $z\in\mathbb{H}^n$, then all points $x$, $x'$, $y$, $y'$ and $z$ lie on the same transport ray.
	Moreover if $\gamma\in\Pi_2(\mu,\nu)$, then either $x\leq x'\leq z\leq y\leq y'$ or $x'\leq x\leq z\leq y'\leq y$.
\end{lemma}
\begin{proof}
	We first recall that \eqref{c-CM} reads as
	\begin{equation}\label{aaa}
		d_{CC}(x,y)+d_{CC}(x',y')\leq d_{CC}(x,y')+d_{CC}(x',y),
	\end{equation}
	$\forall (x,y),(x',y')\in\Gamma$.
	Let $\sigma:[0,d_{CC}(x,y)]\rightarrow\mathbb{H}^n$ be a geodesic between $x$ and $y$, $\tilde\sigma:[0,d_{CC}(x',y')]\rightarrow\mathbb{H}^n$ a geodesic between $x'$ and $y'$, $z\in\sigma(0,d_{CC}(x,y))\cap\tilde\sigma(0,d_{CC}(x',y'))$, so $z=\sigma(d_{CC}(x,z))=\tilde\sigma(d_{CC}(x',z))$. We denote by $\alpha$ the curve between $x$ and $y'$ defined in the following way:
    \begin{displaymath}
		\alpha(t):=\begin{cases}
			\sigma\left(\frac{d_{CC}(x,z)}{d_{CC}(x',z)}t\right),\quad &\textnormal{if}\,\ t\in[0,d_{CC}(x',z)],\\
			\tilde\sigma(t),\quad &\textnormal{if}\,\ t\in(d_{CC}(x',z),d_{CC}(x',y')].
		\end{cases}
    \end{displaymath}
    We will prove that $\alpha$ is geodesic between $x$ and $y'$. Indeed, otherwise we would have 
	\begin{multline}\label{aab}
		d_{CC}(x,y')<\ell_H(\alpha)=\ell_H(\alpha_{|[0,d_{CC}(x',z)]})+\ell_H(\alpha_{|[d_{CC}(x',z),d_{CC}(x',y')]})\\=d_{CC}(x,z)+d_{CC}(z,y').
	\end{multline}
	Since $z$ lies on both the geodesic between $x$ and $y$ and the geodesic between $x'$ and $y'$, it follows that 
	\begin{equation}\label{aac}
		\begin{cases}
			d_{CC}(x,y)=d_{CC}(x,z)+d_{CC}(z,y);\\
			d_{CC}(x',y')=d_{CC}(x',z)+d_{CC}(z,y').
		\end{cases}	
	\end{equation}
	By replacing (\ref{aac}) in (\ref{aab}), we obtain:
	\begin{equation}\label{aad}
		d_{CC}(x,y')+d_{CC}(z,y)+d_{CC}(x',z)<d_{CC}(x,y)+d_{CC}(x',y').
	\end{equation}
	By the triangle inequality follows that:
	\begin{displaymath}
		d_{CC}(x',y)\leq d_{CC}(x',z)+d_{CC}(z,y),
	\end{displaymath}
	and then, by replacing this last inequality in (\ref{aad}), we obtain
	\begin{displaymath}
		d_{CC}(x,y')+d_{CC}(x',y)<d_{CC}(x,y)+d_{CC}(x',y'),
	\end{displaymath}
	and this contradicts (\ref{aaa}). It follows that $\tilde\sigma$ and $\alpha$ are geodesics that coincide on the non-trivial interval $[d_{CC}(x',z),d_{CC}(x',y')]$. Since $\mathbb{H}^n$ is non-branching, this implies that $\tilde\sigma$ and $\alpha$ are sub-arcs of the same geodesic, namely $\alpha$ if $d_{CC}(x',z)\leq d_{CC}(x,z)$ and $\tilde\sigma$ otherwise, on which all points $x,x',z,y'$ lie.
	
	The thesis follows from Proposition \ref{monotone}.
\end{proof}

Given a map $T:\mathbb{H}^n\rightarrow\mathbb{H}^n$, from now on we will denote by
\begin{equation*}
	T_t:=S_t\circ(\text{Id}\otimes T):\mathbb{H}^n\rightarrow\mathbb{H}^n,\quad \forall t\in[0,1]
\end{equation*}
where $T_t(x)$ is the point at distance $td_{CC}(x, T(x))$ from $x$ on the selected geodesic $S(x,T(x))$ between $x$ and $T(x)$. In particular if $\gamma\in\Pi_1(\mu,\nu)$ is induced by a transport map, i.e. is of the form $\gamma:=(\text{Id}\otimes T)_\#\mu\in\Pi_1(\mu,\nu)$, then
\begin{equation*}
    \mu_t={(T_{t})}_{\#}\mu. 
\end{equation*}


The previous lemma allows to prove the following result.

\begin{prop}\label{absolutecontinterp}
If $\mu\ll\mathcal{L}^{2n+1}$, then there exists an optimal transport plan $\gamma\in\Pi_2(\mu,\nu)$ such that the measure
\begin{equation}\label{absolutecontinter1}
	\mu_t:=(S_t)_\#\gamma\ll\mathcal{L}^{2n+1},\quad \forall t\in[0,1).
\end{equation}
\end{prop}

\begin{proof}	
First we suppose that $\nu$ is finitely atomic, with atoms $(y^i)_{i=1}^M$. Let $\gamma\in\Pi_2(\mu,\nu)\subset\Pi_1(\mu,\nu)$, as in Theorem \ref{mainthmbis}, which is monotone in the sense of \eqref{orderrelation} and induced by a transport map $T$. Let us denote by $\Gamma\subseteq\mathbb{H}^n\times\mathbb{H}^n$ the set $\gamma$ is concentrated on and \eqref{var1} and \eqref{var2} hold.
	
We denote by $\Omega_i:=T^{-1}(\{y^i\})\cap \pi_1(\Gamma)$: obviously these sets are mutually disjoint and $\mu(\Omega)=1$, where $\Omega:=\bigcup_{i=1}^M\Omega_i$.
	
Now we denote by $\Omega_i(t):=T_t(\Omega_i)$: if we fix $t\in[0,1)$, then $\Omega_i(t)\cap\Omega_j(t)=\emptyset$ for every $i,j=1,\ldots,M$. Indeed, if $\exists\ z\in\Omega_i(t)\cap\Omega_j(t)$ then $\exists\ x^i\in \Omega_i$ and $x^j\in\Omega_j$ such that $(x^i,y^i), (x^j,y^j)\in \Gamma, (x^i,y^i)\not=(x^j,y^j)$ and the geodesics between these two pairs of points intersect at $z$. Since $\gamma\in\Pi_2(\mu,\nu)$, by Theorem \ref{pi1.3} we can suppose that $x^i,y^i,x^j,y^j,z$ belong to the same unit-speed geodesic and $x^i\leq x^j\leq z\leq y^i\leq y^j$. In particular this means, on the one hand, that $td_{CC}(x^i,y^i)=d_{CC}(x_i,z)\geq d_{CC}(x^j,z)=td_{CC}(x^j,y^j)$, hence $d_{CC}(x^i,y^i)\geq d_{CC}(x^j,y^j)$. On the other hand $(1-t)d_{CC}(x^i,y^i)=d_{CC}(z,y_i)\leq d_{CC}(z,y^j)=(1-t)d_{CC}(x^j,y^j)$, hence $d_{CC}(x^i,y^i)\leq d_{CC}(x^j,y^j)$. It follows that $d_{CC}(x^i,y^i)= d_{CC}(x^j,y^j)$ and hence $d_{CC}(x^i,z)=d_{CC}(x^j,z)$ and $d_{CC}(z,y^i)=d_{CC}(z,y^j)$, which in turn implies that $x^j=x^i$ and $y^i=y^j$ and gives a contradiction. However it may happen that $x^i=y^i$ or $x^j=y^j$. Let us suppose that $x^i=y^i=z$: the same computation above implies that $d_{SR}(x^j,y^j)=0$, which in turns implies that $y^i=y^j$ and gives a contradiction.
	
Remember also that $\mu$ is absolutely continuous and hence there exists a correspondence $\varepsilon\mapsto\delta=\delta(\varepsilon)$ such that 
\begin{equation*}
	\mathcal{L}^{2n+1}(A)<\delta(\varepsilon)\Rightarrow\mu(A)<\varepsilon.
\end{equation*}

Let $A\subset\mathbb{H}^n$ be a Borel set, $t\in[0,1)$, then $\mu_t:=(T_t)_{\#}\mu$ is concentrated on $T_t(\text{supp}(\mu))$ and
\begin{equation*}
	\mu_t(A)=\sum_{i=1}^{M}\mu_t(A\cap\Omega_i(t))=\sum_{i=1}^M\mu(T_t^{-1}(A\cap\Omega_i(t)))=\mu\left(\bigcup_{i=1}^M(T_t^{-1}(A\cap\Omega_i(t)))\right),
\end{equation*}
since the sets $T_t^{-1}(A\cap\Omega_i(t))\subseteq\Omega_i$ are disjoint.
We observe that for any $x\in\Omega_i$, $T_t(x)=S_t(x,y^i)$, hence by \eqref{MCP} follows that
\begin{equation*}
    \mathcal{L}^{2n+1}(U)\leq\frac{1}{(1-t)^{2n+3}}\mathcal{L}^{2n+1}(T_t(U)),
\end{equation*}
for any $U\subset\Omega_i$. This in turn implies that
\begin{equation*}
    \mathcal{L}^{2n+1}(T_t^{-1}(A\cap\Omega_i(t)))\leq\frac{1}{(1-t)^{2n+3}}\mathcal{L}^{2n+1}(A\cap \Omega_i(t)),
\end{equation*}
and so 
\begin{equation*}
    \mathcal{L}^{2n+1}\left(\bigcup_{i=1}^M(T_t^{-1}(A\cap\Omega_i(t)))\right)\leq\frac{1}{(1-t)^{2n+3}}\mathcal{L}^{2n+1}(A).
\end{equation*}
Hence, it is sufficient to suppose that $\mathcal{L}^{2n+1}(A)<(1-t)^{2n+3}\delta(\varepsilon)$ to get $\mu_t(A)<\varepsilon$.	This proves that $\mu_t\ll\mathcal{L}^{2n+1}$.
	
Now, if $\nu$ is not finitely atomic, we can take a sequence $(\nu_k)_{k\in\mathbb{N}}$ of atomic measures weakly converging to $\nu$, for instance as in Lemma \ref{optpi2}. For any $k\in\mathbb{N}$, we consider an optimal transport plan $\gamma_k\in\Pi_2(\mu,\nu_k)$ as in the first part of the proof. Hence, the sequence $(\gamma_k)_{k\in\mathbb{N}}$ weakly converges to some optimal transport plan $\gamma\in\Pi_2(\mu,\nu)$; moreover the sequence $(\mu_t^k)_{k\in\mathbb{N}}$ weakly converges to the corresponding $\mu_t:=(S_t)_{\#}\gamma$, thanks to Proposition \ref{pi1.1} and \cite[Lemma 7.3]{DePascale2}. Take a set $A$ such that $\mathcal{L}^{2n+1}(A)<(1-t)^{2n+3}\delta(\varepsilon)$. Since the Lebesgue measure is regular, $A$ is included in an open set $B$ such that $\mathcal{L}^{2n+1}(B)<(1-t)^{2n+3}\delta(\varepsilon)$. Hence $\mu_t^k(B)<\varepsilon,\forall k\in\mathbb{N}$. Passing to the limit and using Portmanteau's Theorem, see \cite[Theorem 2.1]{Billingsley}, we get 
$$\mu_t(A)\leq\mu_t(B)\leq\liminf_k \mu_t^k(B)\leq\varepsilon.$$
This proves that $\mu_t\ll\mathcal{L}^{2n+1}$.\end{proof}

Now we are able to find at least an optimal transport plan $\gamma\in\Pi_1(\mu,\nu)$ such that the interpolation measures $\mu_t$ constructed from $\gamma$ are absolutely continuous for $t<1$.
\\

\begin{teo}\label{absolutecont}
If $\mu\ll\mathcal{L}^{2n+1}$, then there exists an optimal transport plan $\gamma\in\Pi_2(\mu,\nu)$ such that the measure $a_{\gamma}\ll\mathcal{L}^{2n+1}$.
\end{teo}
\begin{proof}
Let $\gamma\in\Pi_2(\mu,\nu)$ satisfying \eqref{absolutecontinter1}. Then, the thesis follows immediately from \eqref{density4} applied to $a_{\gamma}$.
\end{proof}

Obviously the previous argument depends only on one of the two marginals and it is completely symmetric: if $\nu\ll\mathcal{L}^{2n+1}$, again one can get the existence of an optimal transport plan $\gamma\in\Pi_2(\mu,\nu)$ such that the associated horizontal transport density $a_{\gamma}$ is absolute continuous w.r.t. the $(2n+1)$-dimensional Lebesgue measure.

\subsection{$p$-summability of horizontal transport densities}

In this subsection we prove the existence of at least one horizontal transport density belonging to $L^p$, for some values of $p$.

From now on, given $\lambda\in\mathcal{M}_+(\mathbb{H}^n)$ we will write that $\lambda\in L^p$ if $\lambda\ll\mathcal{L}^{2n+1}$, with density $\rho\in L^p$. We will denote by $\|\lambda\|_p:=\|\rho\|_{L^p}$.

Let $\gamma\in\Pi_1(\mu,\nu)$ as in Theorem \ref{absolutecont}. From \eqref{density3} and the Minkowski inequality it follows that
\begin{equation}\label{Minkowski}
	\|a_\gamma\|_p\leq C\int_{0}^1\|\mu_t\|_pdt.
\end{equation}

In order to prove $p$-summability of $a_\gamma$, it is enough to estimate the $L^p$ norm of $\mu_t$ as a function of the variable $t$. This will be established in the following  theorem: 
\begin{prop}\label{summabilityinterp}
If $\mu\in L^p$, for some $p\in[1,\infty]$, then there exists an optimal transport plan $\gamma\in\Pi_2(\mu,\nu)$ such that $\mu_t:=(S_t)_{\#}\gamma\in L^p$ and 
\begin{equation}\label{bb3}
	\|\mu_t\|_p\leq (1-t)^{-(2n+3)/q}\|\mu\|_p,\quad \forall t\in[0,1),
\end{equation}
 where $q:=\frac{p}{p-1}$.
\end{prop}
\begin{proof}
Let us denote by $\rho$ the density of $\mu$ w.r.t. $\mathcal{L}^{2n+1}$. Let us consider first the discrete case: let us assume that the target measure $\nu$ is finitely atomic and let us denote by $(y^i)_{i=1,\ldots,M}$ its atoms. Let us consider an optimal transport plan $\gamma\in\Pi_2(\mu,\nu)$, as in the proof of Proposition \ref{absolutecontinterp}, concentrated on some set $\Gamma$. Since $\gamma$ is induced by a map $T$, we denote by $\Omega_i:=T^{-1}(\{y^i\})\cap\pi_1(\Gamma)$, for $i\in\{1,\ldots,M\}$, so that for $\gamma$-a.e. $(x,y)\in\Omega_i\times\mathbb{H}^n$, we have $y=y^i$. Let us consider the corresponding interpolation measures $\mu_t\ll\mathcal{L}^{2n+1}$ for every $t\in[0,1)$; moreover, for all $\phi\in C_c(\mathbb{H}^n,\mathbb{R}_+)$, by definition of push-forward we get that 
\begin{align*}
    \int\phi(x)d\mu_t(x)=&\sum_{i=1}^M\int_{\Omega_i}\phi(S_t(x,y_i))d\gamma(x,y_i)=\\
	=&\sum_{i=1}^M\int_{\Omega_i}\phi(T_t(x))d\mu(x).
\end{align*}
Let us fix $i\in\{1,\ldots,M\}$ and let us denote by $\rho_t$ the density of $\mu_t$ w.r.t. $\mathcal{L}^{2n+1}$ and by $\rho_t^i:={\rho_t}_{\lfloor \Omega_i}$. Let us take the change of variable $z=S_t(x,y_i)={T_t}_{\lfloor\Omega_i}(x)$. We know from Lemma \ref{pi1.3} and the disjointness of the sets $\Omega_i(t)$ that this map is injective. Then, for all $\phi\in C_c(\mathbb{H}^n,\mathbb{R}_+)$ we get 
\begin{align*}
	\int_{\Omega_i}\phi(x)d\mu^i_t(x)&=\int_{\Omega_i}\phi(T_t(x))\rho(x)dx=\\&=
		\int_{\Omega_i(t)}\phi(z)\rho(T_t^{-1}(z))|\det D_x(S_t(x,y^i))|^{-1}dz.
\end{align*}
Hence, we have that 
\begin{equation*}
    \rho_t^i(z)=\rho(T_t^{-1}(z))|\det D_x(S_t(x,y^i))|^{-1},\quad \text{for a.e. } z\in\Omega_i(t).
\end{equation*}
Consequently, we get 
\begin{align*}
	\|\rho_t^i\|^p_{L^p(\Omega_i(t))}&=\int_{\Omega_i(t)}\rho(T_t^{-1}(z))^p|\det D_x(S_t(x,y^i))|^{-p}dz=\\&=\int_{\Omega_i}\rho(x)^p|\det D_x(S_t(x,y^i))|^{1-p}dx.
\end{align*}
Hence from \eqref{det} it follows that
\begin{align*}
    \|\rho_t^i\|^p_{L^p(\Omega_i(t))}\leq (1-t)^{(1-p)(2n+3)}\|\rho\|^p_{L^p(\Omega_i)},\quad \forall i\in\{1,\ldots,M\}.
\end{align*}
Then, we have
\begin{equation}\label{discretepsumm}
	\|\mu_t\|_p\leq (1-t)^{-(2n+3)/q}\|\mu\|_p,\quad \forall t\in(0,1),
\end{equation}
where $q:=\frac{p}{p-1}$.


If $\nu$ is not finitely atomic, again we take a sequence $(\nu_k)_{k\in\mathbb{N}}$ of atomic measures weakly converging to $\nu$, for instance as in Lemma \ref{optpi2}. We consider a sequence $(\gamma_k)_{k\in\mathbb{N}}\subset\Pi_2(\mu,\nu_k)$ of optimal plans satisfying \eqref{discretepsumm}: this sequence weakly converges to an optimal plan $\gamma\in\Pi_2(\mu,\nu)$ and $\mu_t^k$ weakly converge to the corresponding $\mu_t:=(S_t)_{\#}{\gamma}$, see again Proposition \ref{pi1.1} and \cite[Lemma 7.3]{DePascale2}. Hence, we get that
\begin{equation*}
    \|\mu_t\|_p\leq\liminf_{k \rightarrow 0}\|\mu_t^k\|_p\leq (1-t)^{-(2n+3)/q}\|\mu\|_p.
\end{equation*}
\end{proof}

Now we are able to prove the following theorem.
\begin{prop}\label{summability1}
If $\mu\in L^p$, for some $p\in[1,\infty]$, the following results hold: if $p<\frac{2n+3}{2n+2}$, then there exists $\gamma\in\Pi_2(\mu,\nu)$ such that $a_{\gamma}\in L^p$; otherwise, there exists $\gamma\in\Pi_2(\mu,\nu)$ such that $a_{\gamma}\in L^s$, for $s<\frac{2n+3}{2n+2}$.

\end{prop}

\begin{proof}
Let $\gamma\in\Pi_2(\mu,\nu)$ satisfying \eqref{bb3}. Then, it follows from \eqref{Minkowski} applied to $a_{\gamma}$ that
\begin{equation*}
	\|a_{\gamma}\|_p\leq C\int_0^1 \|\mu_t\|_pdt\leq C\|\mu\|_p\int_0^1(1-t)^{-(2n+3)/q}dt.
\end{equation*}
The last integral is finite whenever $q>2n+3$, i.e. $p<\frac{2n+3}{2n+2}$.
	
If $p\geq\frac{2n+3}{2n+2}$ the thesis follows from the fact that any density in $L^p$ also belongs to any $L^s$ space for $s<p$.
\end{proof}


If also $\nu\in L^p$ then, by symmetry, one can find an optimal transport plan $\tilde\gamma\in\Pi_2(\mu,\nu)$, possibly different from the one in Proposition \ref{summabilityinterp}, such that $\tilde{\mu}_t:=\left(S_t\right)_\#\tilde\gamma\in L^p$ and it satisfies
\begin{equation}\label{bb1}
	\|\tilde{\mu}_t\|_p\leq t^{-(2n+3)/q}\|\nu\|_p,\quad \forall t\in(0,1].
\end{equation}

In the Euclidean setting  
 \cite[Theorem 3.18]{Santambrogiolibro} or \cite{Feldman2}, and in the more general Riemannian setting, see \cite{Feldman}, $\Pi_2(\mu,\nu)$ consists of a unique element, so that it is possible to glue together \eqref{bb3}, for $t\leq\frac{1}{2}$, and \eqref{bb1}, for $t\geq\frac{1}{2}$, and get the existence of a horizontal transport density in $L^p$. Unfortunately, this uniqueness result is still an open problem in the Heisenberg group, hence we cannot glue together \eqref{bb3} and \eqref{bb1}, and deduce anything about the summability of $a_\gamma$.
