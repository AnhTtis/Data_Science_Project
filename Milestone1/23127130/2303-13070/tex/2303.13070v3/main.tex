\documentclass[a4paper,11pt]{article}
\pdfoutput=1 % if your are submitting a pdflatex (i.e. if you have
             % images in pdf, png or jpg format)

\usepackage{jinstpub} % for details on the use of the package, please
                     % see the JINST-author-manual

\usepackage{color}
\usepackage{lineno}
\usepackage{comment}
\usepackage{siunitx}
\usepackage{orcidlink}
%\usepackage[sort,compress]{natbib}
%\bibliographystyle{utphys}
\bibliographystyle{unsrt}


\usepackage{tikz}
\usetikzlibrary{circuits.logic.US}
\usepackage{siunitx}
\sisetup{separate-uncertainty=true}
%\linenumbers

\title{\boldmath Development of proton beam irradiation system for the NA65/DsTau experiment}

\newcommand{\Aki}[1]{\textcolor{blue}{[AA: {#1}]}}
\newcommand{\Yuri}[1]{\textcolor{red}{[YG: {#1}]}}

%% %simple case: 2 authors, same institution
%% \author{A. Uthor}
%% \author{and A. Nother Author}
%% \affiliation{Institution,\\Address, Country}

% more complex case: 4 authors, 3 institutions, 2 footnotes
\collaboration[]{The DsTau collaboration}
\author[a]{Shigeki Aoki\orcidlink{0000-0002-1092-5037},}
\author[b,c]{Akitaka Ariga\orcidlink{0000-0002-6832-2466},}
\author[d]{Tomoko Ariga\orcidlink{0000-0001-9880-3562},}
\author[e]{Nikolaos Charitonidis\orcidlink{0000-0001-9506-1022},}
\author[f]{Sergey Dmitrievsky\orcidlink{0000-0003-4247-8697},}
\author[g,l]{Radu Dobre\orcidlink{0000-0002-9518-6068},}
\author[g]{Elena Firu\orcidlink{0000-0002-3109-5378},}
\author[f]{Yury Gornushkin\orcidlink{0000-0003-3524-4032},}
\author[h]{Ali Murat Guler\orcidlink{0000-0001-5692-2694},}
\author[b]{Daiki Hayakawa\orcidlink{0000-0003-4253-4484},}
\author[i]{Koichi Kodama\orcidlink{0000-0001-9533-1571},}
\author[j]{Masahiro Komatsu\orcidlink{0000-0002-6423-707X},}
\author[k]{Umut Kose\orcidlink{0000-0001-5380-9354},}
\author[f,l]{M\u{a}d\u{a}lina-Mihaela Miloi\orcidlink{0000-0001-7208-4379},}
\author[b]{Manato Miura\orcidlink{0000-0002-4955-8609},}
\author[j]{Mitsuhiro Nakamura\orcidlink{0009-0002-6032-2741},}
\author[j]{Toshiyuki Nakano\orcidlink{0009-0004-8568-9077},}
\author[g]{Alina-Tania Neagu\orcidlink{0000-0001-6788-4320},}
\author[b,*]{Toranosuke Okumura\note[*]{Corresponding author.}\orcidlink{0000-0002-3266-8713},}
\author[h]{Canay Oz\orcidlink{0000-0002-5113-5779},}
\author[j]{Hiroki Rokujo\orcidlink{0000-0002-3502-493X},}
\author[j]{Osamu Sato\orcidlink{0000-0002-6307-7019},}
\author[f]{Svetlana Vasina\orcidlink{0000-0003-2775-5721},}
\author[m]{Junya Yoshida\orcidlink{0000-0002-9398-746X},}
\author[n]{Masahiro Yoshimoto\orcidlink{0000-0002-4667-0718},}
\author[h]{Emin Yuksel\orcidlink{0009-0008-7861-1879}.}




%@Murat: I'm not sure about his affiliation, please complete...

%\Yuri{The authors list in an order according to the first name! }

% The "\note" macro will give a warning: "Ignoring empty anchor..."
% you can safely ignore it.
\affiliation[a]{Graduate School of Human Development and Environment, Kobe University, Tsurukabuto, Nada, 657-8501 Kobe, Japan}

\affiliation[b]{Department of Physics, Chiba University, 1-33 Yayoi-cho Inage-ku, 263-8522 Chiba, Japan
%Chiba University, Yayoi 1-33, Inage, Chiba, 263-8522 Chiba, Japan
}

\affiliation[c]{Albert Einstein Center for Fundamental Physics, Laboratory for High Energy Physics, University of Bern, Sidlerstrasse 5, CH-3012 Bern, Switzerland}

\affiliation[d]{Faculty of Arts and Science, Kyushu University, 744 Motooka, Nishi-ku, Fukuoka, 819-0395 Japan}

\affiliation[e]{CERN, BE Department, 1 Esplanade des Particules, CH-1211 Meyrin, Switzerland}

\affiliation[f]{Affiliated with an international laboratory covered by a cooperation agreement with CERN}

\affiliation[g]{Laboratory of High Energy Astrophysics and Advanced Technology, Institute of Space Science a subsidiary of INFLPR, 409, Atomistilor Street, Magurele, 077125 Ilfov, Romania}

\affiliation[h]{Physics Department, Middle East Technical University, Dumlup{\i}nar Bulvari, 06800 Ankara, Turkey}

\affiliation[i]{Department of Science Education, Aichi University of Education, 448-8542 Kariya, Japan}

\affiliation[j]{Department of Physics, Nagoya University, Furo-cho, Chikusa-ku, 464-8602 Nagoya, Japan}

\affiliation[k]
{Institute for Particle physics and Astrophysics, ETH Zurich, Otto-Stern-Weg 5, CH-8093 Zurich, Switzerland}

\affiliation[l]
{Faculty of Physics, University of Bucharest, 077125 Bucharest, Romania}

% \affiliation[m]{Advanced Science Research Center, Japan Atomic Energy Agency, 319-1195 Tokai, Japan}
\affiliation[m]{Tohoku University, Sendai city, 980-8577 Sendai, Japan}

% \affiliation[n]{Tohoku University, Sendai city, 980-8577, Sendai, Japan}
\affiliation[n]{RIKEN Nishina Center, RIKEN, 2-1 Hirosawa, Wako, 351-0198 Saitama, Japan}

\emailAdd{toranosuke.okumura@cern.ch}

\abstract{
Tau neutrino is the least studied lepton of the Standard Model (SM).  
The NA65/DsTau experiment targets to investigate $D_s$, the parent particle of the $\nu_\tau$, 
using the nuclear emulsion-based detector and to decrease the systematic uncertainty of $\nu_\tau$ flux prediction from over $50\si{\%}$ to 10\si{\%} for future beam dump experiments. 
In the experiment, the emulsion detectors are exposed to the CERN SPS 400 GeV proton beam. 
To provide optimal conditions for the reconstruction of interactions, the protons are required to be uniformly distributed over the detector's surface with an average density of $10^5 \si{cm^{-2}}$ and the fluctuation of less than 10\%.
To address this issue, we developed a new proton irradiation system called the target mover.
The new target mover provided irradiation with a proton density of \SI{1.01e5}{cm^{-2}} and the density fluctuation of $1.9\pm0.3$\% in the DsTau 2021 run.
}



\keywords{Detector control systems, Beam-line instrumentation

}




\begin{document}
\maketitle
\flushbottom

\section{Introduction}
\label{sec:intro}

\begin{comment}
Recently, flavor anomalies in the heavy meson decay, such as $B \to D^+l^-\nu$ by the LHCb \cite{LHCb:2015gmp}, were indicated, which have attracted the attention of physicists as a hint of the beyond Standard Model physics. 
On the other hand, lepton universality in neutrino scattering has not been studied yet due to insufficient study of the $\nu_\tau$ properties.
The DONuT experiment \cite{DONUT:2000fbd} measured the $\nu_\tau$ cross section using the neutrino beam generated by the interactions between the tungsten target and 800 \si{GeV} protons from the Tevatron accelerator at the Fermilab. So far, this is the only $\nu_\tau$ measurement without effects from neutrino oscillations.
The experimental result has an statistical uncertainty of 33\% due to 9 detected $\nu_\tau$ events, furthermore, it 
%\Yuri{ suffered from an even larger systematic uncertainty of $>$50\%,  mainly due to  a lack of knowledge on the differential production cross-section of $D_s$, a parent particle of $\tau$ and $\nu_\tau$.} 
suffered from an even larger systematic uncertainty of $>$50\%, mainly due to a lack of knowledge on the differential production cross-section of $D_s$, a parent particle of $\tau$, and $\nu_\tau$.
\Yuri{In the future, the tau neutrino measurements are expected from such experiments as ...}
In the furure, the tau neutrino measurements are expected from such experiments as BDF~\cite{Aberle:2022fts}, FASER~\cite{FASER:2019aik}, SND@LHC~\cite{SNDLHC:2022ihg}, however, the understanding of the forward $D_s$, or more in general charm, production would limit understanding of the tau neutrino properties.
\Yuri{I would start the paper from this point and omit the previous stuff. The paper is devoted to the particular technical topic. To associate the physics beyond Standard Model with movement of our stage is somehow too pretentious!}



The NA65/DsTau experiment \cite{DsTau:2019wjb} was proposed to measure the $D_s$ double-differential production cross section in proton-nuclear interactions by detecting $D_s \to \tau \to X$ decays, and to reduce the uncertainty of the $\nu_\tau$ flux from over 50\si{\%} to 10\si{\%} level for future neutrino experiments. 
The project aims to collect approximately 1000 $D_s \to \tau \to X$ decays for 2 $\times 10^8$ proton-nuclear interactions with the emulsion-based detector in the physics data taking in 2021~-~2023. 
The detector module has a piled up structure consisting of 130 emulsion films, 116 plastic sheets, and 13 metallic target plates. 
Two kinds of target materials are used: tungsten plates having a thickness of $\SI{0.5}{mm}$ or molybdenum plates having a thickness of $\SI{1}{mm}$. 
Three target plates at the downstream part of the detector are always of tungsten. 
The size of the module is $\SI{25}{cm} \times \SI{20}{cm} \times \SI{7.5}{cm}$. It weights \SI{12}{kg}. 
The identification of $D_s \to \tau \to X$ decays will be performed with topological information thanks to the high-position and high-angular resolution of the emulsion detector and nanometric precision readout. 
The modules are exposed to the 400 GeV proton beam from the CERN SPS at the North Area experimental hall. 
The primary proton density on the surface of the detector should be uniform at the level of $\SI{e5}{cm^{-2}}$ with less 10\% fluctuation in order to decrease systematic uncertainty caused by the non-uniformity of the track density, such as $K^0$ background to $D^0$ which would be complicated to reproduce in the MC simulation. %Yuri{Not quite clear statement!}
%Yuri{This target track density value is the maximum possible for the analysis at the downstream part of the emulsion detector.}
This target track density value is the maximum possible for the analysis at the downstream part of the emulsion detector.
The data taking rate should be sufficiently high, $\mathcal{O}(10^5)\ \si{\hertz}$, to detect rare charm events in the proton - nuclear interactions.
In order to realize to meet the requirements of the  track density fluctuation and the data taking rate, the proton beam size is enlarged to approximately 2.5 $\si{cm}$ $\times$ 2.5 $\si{cm}$ and each module is exposed to the beam with help of a dedicated stage called the Target Mover (TM).

%%\Aki{Descriptions in introduction and the next section is not logically separated...}

%%\Aki{NA65/DsTau is going to take physics data in 2021-2023.}

%%\Aki{The goal of density fluctuation is to be described.}

\end{comment}
%the Previous introduction


The validation of the Standard Model (SM) and exploration of Beyond Standard Model (BSM) physics are considered to be a paramount mission in particle physics.
Recent results from the LHCb \cite{LHCb:2015gmp}, BaBar \cite{BaBar:2001pki}, and Belle \cite{Belle:2001zzw} (Sec. 7.6 in \cite{HFLAV:2022pwe}) demonstrate hints of possible violation of the Lepton Universality (LU) in $B$ meson decays. 
%As this measurement can be a hint for BSM, it attracts a lot of attention recently.
%While lepton universality in neutrino scattering is advantageous in the validation of the SM, the study of the $\nu_\tau$ properties is insufficient in comparison to the other neutrino flavor, namely $\nu_e$ and $\nu_\mu$.
The study of LU in neutrino interactions can be a new probe for BSM. 
However, the data on $\nu_\tau$ is quite scarce; only a few experiments have reported its detection.
The DONuT experiment \cite{DONUT:2000fbd} directly detected $\nu_\tau$ for the first time and estimated the $\nu_\tau$ interaction cross-section \cite{Furukawa:2008zza}.
However, the cross-section measurement had about 30\% 
statistics error due to the low statistics and about 50\% systematic error due to a poorly constrained $\nu_\tau$ flux.
%Only a handful of $\nu_\tau$ interactions were detected and the $\nu_\tau$ flux was poorly constrained by the other experiments.
The main source of $\nu_\tau$ is the leptonic decay of $D_s$ mesons.
Therefore, a precise measurement of the $D_s$ production cross-section can provide prediction of $\nu_\tau$ fluxes for neutrino experiments like FASER($\nu$) \cite{FASER:2019dxq, FASER:2020gpr}, SND@LHC \cite{SNDLHC:2022ihg} and future experiments proposed at CERN BDF \cite{Aberle:2022fts}.
The NA65/DsTau experiment \cite{DsTau:2019wjb, Aoki:2017spj} at CERN-SPS was proposed to measure $D_s$ production cross-section in proton-nucleus interactions by detecting about $10^3$ $D_s\to\tau\to X$ decays.
This measurement is going to reduce uncertainty in the DONuT's measurement from 50\% to 10\%.

%interaction cross section, which is the only $\nu_\tau$ measurement without effects from neutrino oscillations, so far.
%The experimental result exhibits a statistical uncertainty of 33\% due to the 9 detected $\nu_\tau$ events and is further impacted by a even larger systematic uncertainty >50\%, relative to the measured cross section.

%One of the reasons of the systematic uncertainly is the lack of data on a cross-section of $D_s$ production, which decay is the main source of $\nu_\tau$. 
%Experiments studying tau neutrino, such as BDF \cite{Aberle:2022fts}, FASER \cite{FASER:2019aik}, SND@LHC \cite{SNDLHC:2022ihg}, also require the understanding of the forward $D_s$, or in more general, charm production.
%Therefore, the NA65/DsTau experiment was proposed to measure the $D_s$ double-differential production cross section in proton-nuclear interactions by detecting about 1000 $D_s\to \tau \to X$ decays, in order to reduce the uncertainty from over 50\% to 10\% level. %MMiloi(I think you are saying in the next sentence that you use emulsion, so you can delete if from here). 
The identification of $D_s\to \tau \to X$ decays will be performed by using topological information, thanks to the high spatial and angular resolution of the emulsion-based detectors~\cite{Amsler_2013}.
%to the high-position and high-angular resolution of the emulsion detector 
 

%The emulsion detector module has a piled structure consisting of metallic target plates, emulsion films, and plastic sheets. 
The detector modules are exposed to the CERN SPS 400 GeV proton beam with an intensity order of $10^5$ per spill with a duration of about 4 seconds. 
%during , slowly (during 4.8 seconds) extracted towards the H2 or the H4 line of the North Area. The beam intensity is of the order of $10^{7}$ particles per spill.  
The emulsion accumulates the trajectory of charged particles passing through, however, there is a limit of the track density which can be successfully processed and analyzed.
%Because the emulsion accumulates all tracks of charged particles, there is a limit on the maximum beam density that can be irradiated.
As the proton beam spot is small, the target mover (TM) system (as shown in Figure~\ref{fig:b}) is utilized to uniformly irradiate the whole surface of the emulsion detectors.
The similar movable stages were used in the past experiments~\cite{Kodama:1990zy, Aoki:1988pm, KONOVALOVA2019100401}.
The small scale TM prototype was used during the test runs in 2016 and 2017 then in the pilot run of 2018~\cite{DsTau:2019wjb}.
%With the small TM, we need to exchange the module over 300 times in order to accumulate the number of proton nuclear interaction events that the DsTau requires.
As the detector modules used for 2021 physics run were four times larger than those used in the test and pilot runs, the payload and moving range of the TM should be >\SI{20}{kg} and >$\SI{350}{cm}\times\SI{350}{cm}$, respectively. 
Thus, a new TM with a wide aperture was developed by modifying the TM used in another emulsion experiment, J-PARC E07~\cite{J-PARC:2023}, and adding a new functionality to move the stage of TM with a speed proportional to the beam intensity specifically for the DsTau experiment. 
This paper reports on the development of the new TM and control system, and evaluates their performance in the 2021 physics run.
%As shown in right side of Figure~\ref{fig:2016density}, the proton beam intensity varies over time.
%Right side of Figure~\ref{fig:2016density} shows the track density mapping in the 2016 test run detector.
%In that run, the operation was triggered by the beam extraction signals from SPS and the stage moved at a constant velocity during the beam spill, resulting in a periodical pattern.

%The TM in 2016 was unable to vary the speed according to the beam intensity, resulting in a periodical pattern due to the time structure of the proton beam intensity (as shown in Figure~\ref{fig:2016density}).
\begin{comment}
\begin{figure}[tbp]
\centering
\includegraphics[width=.4\textwidth]{image/one_spill.pdf}
\qquad
\includegraphics[width=.5\textwidth]{image/2016proton_density.png} %%もっと大きく
\caption{\label{fig:2016density}Left: The proton beam intensity measured
by the scintillation counter in the 2016 run, and the ideal speed of the TM stage. Right: The periodical pattern  produced by a combination of the beam's time structure and constant-speed stage movement \cite{DsTau:2019wjb}.}
\end{figure}
Thus, a new TM was developed for the full-scale physics run by modifying the TM with the wide aperture used in other emulsion experiments, J-PARC E07 \cite{J-PARC:2023}, and adding a new function to move the TM in proportion to beam intensity specifically for the DsTau experiment.
This paper reports on the development of the new TM and control system, and evaluates their performance in the 2021 physics data taking.
\end{comment}


\section{The Target Mover and the real-time speed control system}
%The schematic view of the new TM is shown in Figure~\ref{fig:b}.
\begin{figure}[tbp]
\centering
\includegraphics[width=.4\textwidth]{image/Target_Mover_CAD.png}
\qquad
\includegraphics[width=.4\textwidth]{image/TargetMover_Photo.jpg}
\caption{\label{fig:b} Left: The schematic view of the Target Mover used in the J-PARC E07~\cite{J-PARC:2023}. The overall size of the structure is 1370 \si{mm} height, 1800 \si{mm} width, and 400 \si{mm} depth, and the range of the motion is 350 \si{mm} in the vertical direction and 450 \si{mm} in a horizontal one. Right: The picture of the Target Mover used in the DsTau 2021 physics run at the SPS H2 beamline with a detector module and the stage module for mounting it.}
\end{figure}
The TM is a motorized 2-dimensional stage to raster-scan the emulsion module with respect to the beam.
% The stepping motors controlled by the program written in C\# drive the stage.
The stepping motors drive the stage under the control of a computer with a program written in C\# language.
%Compared with the TM in the past experients~\cite{Kodama:1990zy, Aoki:1988pm, KONOVALOVA2019100401} controlled by the dc motors, stepping motors can control stage position and speed more precisely.
Stepping motors offer more precise control over stage position and speed compared to the DC motors that were used to control the TM in previous experiments~\cite{Kodama:1990zy, Aoki:1988pm}.
%\Yuri{J-PARK E07 has been already mentioned few lines before. No need to repeat hear. But the reference, which correct this time, better to move there.}
\begin{comment}
\begin{figure}[tbp]
\centering
\includegraphics[width=.6\textwidth]{image/SequenceOfTheTargetMover.pdf}
\caption{The stage position of the TM in an exposure, starting at the left bottom and the goal was the left top. As discussed in Chapter 4, the y-step size of the raster scan for 2021 run was 15 \si{mm}. This sequence took 2.8 hours.}
\label{fig:d}
\end{figure}
\end{comment}
%the figure of the TM sequence
\begin{comment}
In the test run in 2016, the film size was $\SI{12.5}{cm} \times \SI{10}{cm}$, and another small TM was used.
The operation was triggered by the beam extraction signals from 
SPS and the stage moved at a constant velocity during the beam spill. \Yuri{The variation of the beam intensity during spills is well known, we didn't assume it is constant. }
% based on the assumption that the beam intensity does not change within 4 seconds of beam spill. 
However, the beam intensity during %for
a single spill measured by the scintillation counter was not constant, as shown in Figure~\ref{fig:a} \cite{DsTau:2019wjb}. 
The periodical pattern was observed in the primary protons density in the most upstream film of the module due to the time structure of the proton beam intensity.
%To suppress the influence of proton beam time structure, we developed a new TM capable to deal with new larger modules with the size of $\SI{25}{cm} \times \SI{20}{cm}$, and with the implemented system of the stage speed control according to the beam intensity variation.
\end{comment}
We have implemented additional mechanical support to hold the emulsion modules.
The schematic view of the experimental setup is shown in Figure~\ref{fig:setup_SPS}.
The cross delayed wire chamber (XDWC) measures the proton beam profile.
The hit efficiency of XDWC we used was too low, $<20\%$, to be used as the proton counter.
Therefore, two scintillation counters to obtain proton counts were located behind the TM.
Coincidence were taken in order to minimize the contamination from backgrounds.
The trigger threshold of them was set to well below the MIP level.
%The protons would interact in the detector or the TM.
%These interactions would emit daughter particles and the trigger threshold of the scintillation counters was set to MIP level. 
%Thus, the counts would not be changed even if protons interacted in the detector or TM.
As shown in Figure~\ref{fig:SigConv}, signals from the scintillation counters are sent to a series of NIM modules (discriminator, coincidence module, pre-scaler, and NIM-TTL converter). 
The coincidence signal is then transferred to a Raspberry Pi 4B microcomputer.
The Raspberry Pi counts the pulses and sends the data to the TM control PC every 100 \si{ms}.
A TCP-IP protocol is used for the communication between the Raspberry Pi and the TM control PC.
The TM control PC calculates the optimal stage speed $v_x$ based on the following formulas:
\begin{equation}
\label{eq:y}
\begin{split}
v_x = \frac{I}{\rho\Delta y} \,,
\qquad
I \equiv \frac{\Delta n}{\Delta t} \, , 
\end{split}
\end{equation}
where $\Delta n$ is the count taken by the Raspberry Pi, $\Delta t$ is the time interval of their count measurement ($\sim$ 100 \si{ms}), $\rho$ is the required proton density ($\sim 10^5$ \si{cm^{-2}} for physics run) and $\Delta y$ is the $y$-step size of the raster-scanning, which depends on the beam profile as discussed in Section \ref{sec:2021}.
Figure~\ref{fig:c} shows the flowchart of this system which is called the real-time speed control system (RSCS).
\begin{figure}[tbp]
\centering
\includegraphics[width=.8\textwidth]{image/2021_Setup_SPS_v1.1.png}
\caption{\label{fig:setup_SPS}The Schematic view of the experimental setup of the beam test at CERN SPS.}
\end{figure}
\begin{figure}[tbp]
\centering
\includegraphics[width=.9\textwidth]{image/SignalConverterRSCS.pdf}
\caption{The flowchart of the signal conversion from the scintillation counters to the Raspberry Pi. 
%\Yuri{The flowchart is self explaining. The text is too trivial.}
%The signals from the scintillation counters are digitalized by discriminator.The coincidence module rejects the signals from only one scintillator. 
%Then the prescaler make the number of the signals $\times 1/32$.
%Finally, the signals are converted from the NIM level to TTL, so that the Raspberry Pi catches them.
}
%\Aki{Indicate the pre-scaler's factor.}
\label{fig:SigConv}
\end{figure}
\begin{figure}[tbp]
\centering
\includegraphics[width=.9\textwidth]{image/RSCSFlowChart.drawio.drawio.pdf}
\caption{The flowchart of the RSCS. The left side shows the proton counting process and the right shows the TM control process.
The condition "Reasonable" means that the count data is not too high or low compared with the previous sent count.
In order to prevent the excessive acceleration of the motor, the RSCS rejects counts when that is either more than 50 times or less than 1/50 of the previous count and does not change the stage speed. 
Such "Not reasonable" counts are mainly caused by bit errors in the data transmission. 
This was observed at the development stage, but not in the physics run.}
\label{fig:c}
\end{figure}

%Figure~\ref{fig:raster} shows the stage position and amount of the accumulated protons in an irradiation sequence in 2021 run.

At the beginning, the stage moves to the start position. 
%\Yuri{When the beam appears, the stage moves in X direction for 330 \si{mm} at the speed controlled by the RSCS, then moves in Y direction by ${\Delta y}$ step at a constant speed of $\SI{5}{mm/s}$.}
When the beam is exposed on the module, the stage moves in $x$ direction for 330 \si{mm} at the speed controlled by the RSCS, then moves in $y$ direction by $y$-step at a constant speed of $\SI{5}{mm/s}$.
%\Yuri{Then the stage moves again along X-axis, but in the opposite direction}
Then the stage moves again along $x$ axis, but in the opposite direction.
The TM repeats these steps to expose the entire surface of the module. 
We avoided scanning in y direction because the detector module is so heavy and may affect the RSCS performance. 
%\Yuri{After the module irradiation is completed, the stage goes back to the start position.}
After the module irradiation is completed, the stage goes back to the start position.
In the event of any trouble, the operator could immediately stop the TM and return it to the start position where the detector would not be exposed to the beam. 
Once the issue is resolved, the raster scan could be restarted from where it is interrupted.

The requirement on the proton density is \SI{e5}{cm^{-2}} with less than 10\% fluctuation. 
% in order to reduce the background increase from the MC simulation to 1\%.
This would cause 1\% systematic uncertainty in $D^0$ detection due to fluctuation of misidentified $K^0$.
To be precise, we seeking for $D^0$ by searching for neutral decay around proton interaction vertex. 
Since emulsions does not have time information, neutral decay (e.g. $K^0$) from other proton interaction can be in the region of interest, being the background to $D^0$. 
In order to estimate the background ratio, we will compare the data with the MC simulation with the uniform proton density. 
This is the level of uncertainties that we can tolerate for this analysis.
%avoiding an additional beam irradiation. 



\section{Testing and commissioning of the TM}
The RSCS performance was evaluated 
%\Yuri{using  \si{^{90}Sr} $\beta$ source }
using \si{^{90}Sr} $\beta$ source in Chiba University. 
A lead brick with a weight of 22.7 kg was mounted on the TM instead of the emulsion module, and a 2 \si{kBq} \si{^{90}Sr} source was used to emulate the SPS beam. 
%\Yuri{Since the electrons from the source didn't penetrate through both scintillators and the count rate was too low, we employed only one scintillator counter and omit the pre-scaller.}
Since electrons from the source did not penetrate through both scintillators, the count rate was too low.
Therefore, we employed only one scintillation counter and omit the prescaler. 
%XDWC is beamline equipment and it's difficult to install in our laboratory, thus it is omitted in this test.
%Since the background in the laboratory was negligible (less than 30 $\si{signals/sec}$) and the radioactivity of the source was low enough to detect with the Raspberry Pi without a prescaller, one scintillatior and prescaller was omitted. 
%The RSCS system in the laboratory run counted the signals from one scintillation counter connected to the Raspberry Pi through a discriminator and a NIM-TTL converter. 
%線源を2.5秒間シンチに押し当てた後遠ざけることで発生する信号を1スピルとした
Figure~\ref{fig:e} shows an example of the emulated beam spills using the \si{^{90}Sr} source. 
The pseudo beam spills were emulated by placing the source on the scintillation counter for 2.5 seconds by hand, this was repeated about 950 times.
%\Aki{The figure \ref{fig:e} shows only 2.5 sec. It contradicts with this text.}

%信号強度の図への言及
%シグマが~~の2次元ガウス関数的な強度分布のビームと目標密度~~を仮定してステージスピードを計算
The speed of the TM stage was optimized using data of the 2018 pilot run at the SPS and calculated by Equation~\ref{eq:y}.
%This beam profile parameter was chosen based on the 2018 pilot run beam profile.
In order to emulate the stage velocity similar to one of the real experiments, the required pseudo beam density
%\Yuri{because just target density sounds misleading - here it is desired beam particles density, but we also have the target in the setup (W) and it has a density. So from my point of view better to write "target beam density"} 
and $y$-step size of raster-scanning were set to \SI{1500}{cm^{-2}} and \SI{12}{mm}, respectively. 
%\Yuri{In Fig.9 , right plot mean value is larger than 1500}

%\Aki{The description of the standard deviation of 16mm and $\Delta y$ should be logically reorganized.}
%ステージの軌跡は2章にある図と同様である
\begin{figure}[tbp]
\centering
\includegraphics[width=.4\textwidth]{image/BeamTimeStructure_Lab.pdf}
\caption{\label{fig:e}The emulated beam spill by the \si{^{90}Sr} source. The counts are plotted every \SI{100}{ms} and they are normalized to the counts per second.}
\end{figure}

%左の図はビームシェイプを2次元ガウス関数、TargetMoverがLabRunと同じ動きをしたとしたときのトラックデンシティマッピングであり、右の図は密度分布のヒストグラムである。
% Figure~\ref{fig:e} shows an example of the emulated SPS spill using the \si{^{90}Sr} source. 
% During the laboratory run, 
To emulate the proton density map, the recorded stage position and source intensity were smeared by weighting each signal by a 2D Gaussian, with a sigma of \SI{10}{mm}.
Figure~\ref{fig:g} shows the calculated density map and distribution of the density in each 1 \si{cm^2}.
The density was measured as $\SI{1550 +- 61}{cm^{-2}}$.
The achieved uniformity is approximately $4\si{\%}$, which satisfies the requirement of the experiment ($< 10\si{\%}$). 

\begin{figure}[tbp]
\centering
\includegraphics[width=.4\textwidth]{image/LabRunDensityMap.pdf}
\qquad
\includegraphics[width=.4\textwidth]{image/LabRunDensityDistribution.pdf}
\caption{\label{fig:g}Left: The map of the emulated particle density fluctuation map of the TM test. Right: %\Yuri{The reproduced track density distribution with Gaussian fit} 
The emulated particle density distribution with Gaussian fit.}
\end{figure}
%\Yuri{The units on both pictures better to be ${cm^{-2}}$. Would be better to have a legend on the right plot with the parameters of the fit (mean value and sigma indicated there. Then in the text one can say  }


\section{Physics run at the CERN SPS H2 beamline}
\label{sec:2021}
The NA65/DsTau 2021 physics run was performed at the CERN SPS H2 beamline from $23^\textnormal{rd}$ of September to $7^\textnormal{th}$ of October in 2021.

As shown in Figure~\ref{fig:TM_SPS}, all components were implemented at the CERN SPS H2 beamline. 
\begin{figure}[tbp]
\centering
\includegraphics[width=.6\textwidth]{image/Target_Mover_Setting_Photo.png}
\caption{\label{fig:TM_SPS}%\Yuri{Whole setup placed  (located) at the CERN SPS H@ beamline....}
The experimental setup placed at the CERN SPS H2 beamline for the NA65/DsTau 2021 physics run.}
\end{figure}
%\Aki{Full setup, shown in Figure~\ref{fig:setup_SPS} was implemented.} 
%\Aki{May be it's nice to put a photo of the setup.}
The Raspberry Pi of the RSCS was operated from the control room, outside of the beam area. 
The count data from the Raspberry Pi was transmitted to the TM control PC via Ethernet. %\Yuri{via Ethernet}
The ping value of their communication was less than $\SI{1}{msec}$, the minimum value detectable by the Windows OS.
This network delay is significantly shorter than \SI{100}{ms}, tha data transfer cycle, and negligible for the RSCS performance. 
The beam energy was \SI{400}{GeV}, with an intensity of $5 \times 10^5$ particles per spill of about 4 seconds.
The beam intensity had a time structure as shown in Figure~\ref{fig:beam_intensity_SPS}, and as the peak shows that the data taking rate was up to about \SI{200}{kHz}.
%[data taking rate sentense about 200 kHz]
%The beam profiles on the X and Y axis were measured by the XDWC and the RMS of each profiles was averagely $\SI{12}{mm}$ in x and $\SI{13}{mm}$ in y, respectively. 
%\Yuri{In the text X and Y direction/axis/step are written different way - small x,y, large X,Y. It should be the same through the whole paper! }
The beam profile measured by the XDWC %\Yuri{demonstrated} 
demonstrated RMS values of $\SI{12}{mm}$ in $x$ direction and $\SI{13}{mm}$ in $y$. 
%It was assumed that the positron profile remained unchanged.
%Indeed, there were no profile changes that could have affected the irradiation.
%During the irradiation, the shifters continuously monitored the beam conditions and the TM motion to prevent mis-irradiation by stopping the beam and target motion in case of emergency.
We assumed that the position profile would not change during the irradiation. 
And the operators periodically monitored the beam profiles and the motion of the TM, and no significant deviation of profiles was observed, except for occasional accelerator failures. 
\begin{figure}[tbp]
\centering
\includegraphics[width=.5\textwidth]{image/BeamTimeStructure_SPS.pdf}
\caption{\label{fig:beam_intensity_SPS}
 Example of the time profile of the beam. In certain spills, we observed that  the extracted beam from the accelerator was not very uniform in the time domain, due to quadruple ripples or other effects that from time-to-time affect the slow extraction towards the North Area and the H2 beamline. However, this was compensated and was not problematic for the data taking.}
\end{figure}
%The y-step size of the raster scan was optimized for a uniform proton density distribution. 
Figure~\ref{fig:h} shows a measured beam profile in %\Yuri{in Y (vertical) direction} 
$y$ (vertical) direction. 
In order to find an optimal $y$-step value, the profile was multiple-copied and superimposed with different $y$-steps as shown in Figure~\ref{fig:ySteps}. 
The step size was determined to be \SI{15}{mm} to flatten the %\Yuri{overall protons density distribution}
overall proton density distribution.
\begin{figure}[tbp]
\centering
\includegraphics[width=.4\textwidth]{image/Xprofile.pdf}
\qquad
\includegraphics[width=.4\textwidth]{image/YProfile.pdf}
\caption{\label{fig:h}Left: X (horizontal) profile of the NA65 beam in 2021. Right: Y (vertical) profile of the NA65 beam in 2021.}
\end{figure}
\begin{figure}[tbp]
\centering
\includegraphics[width=1.0\textwidth]{image/SumProfile12_15_18.pdf}
\caption{\label{fig:ySteps}%\Yuri{The overall protons density distribution as a result superposition  of several beam profiles with different y steps of 12 \si{mm}, 15 \si{mm}, and 18 \si{mm}.}
%The total sum histogram of y beam profile with the shifting distance of 12 \si{mm}, 15 \si{mm}, 18 \si{mm}.}
The overall protons density distribution as a superposition of several beam profiles with different $y$-steps of \SI{12}{mm}, \SI{15}{mm}, and \SI{18}{mm}. The deviation is larger in $y$-step \SI{12}{mm} case than \SI{15}{mm} case due to the Y (vertical) profile of the SPS beam having 2nd peaks at around $\pm \SI{12}{mm}$, as Figure~\ref{fig:h} shows.}
\end{figure}

The TM was successfully operated during the 2021 run for 17 detector modules with the size of $\SI{25}{cm} \times \SI{20}{cm} \times \SI{7}{cm}$. 
%and achieved \SI{200}{\kilo\hertz} data taking rate as peak shown in Figure~\ref{fig:beam_intensity_SPS}.
Figure~\ref{fig:raster} describes the TM sequence for one of the modules in the 2021 run.
\begin{figure}[tbp]
\centering  \includegraphics[width=.6\textwidth]{image/RasterScan.pdf}    \caption{The diagram of the TM position and the accumulated number of protons as a function of time for one of the modules in the 2021 physics run. The red line shows the 
%\Yuri{X and Y in cay-stepsetters)}
$x$ position going right and left. 
The blue line shows the $y$ position going up step by step. 
The black line shows the %\Yuri{integrated number of protons that have passed the emulsion module}
integrated number of protons that have passed the emulsion module. 
The flat part corresponds to the period without beam. }
\label{fig:raster}
\end{figure}
%\Yuri{2021 run. Along with the data collected in the 2018 run, about 30\% of the total amount of the proton interactions, planned to be registered in the experiment, were accumulated. }
Together with the data collected in the 2018 pilot run, approximately 30\% of the total amount of the proton interactions, planned to be registered in the experiment, were accumulated.
%run and we accumulated a total of 30\% of the planned proton interactions, including the 2018 physics run. 

After the data taking in H2, the irradiated emulsion modules were dismantled, then the films were developed chemically at the CERN nuclear emulsion facility.
These developed films were transported to Nagoya University 
%\Yuri{University and scanned by HTS automatic microscope}
and scanned by the HTS \cite{10.1093/ptep/ptx131}, the high-speed automatic microscope.
To analyze the density of registered primary protons, the tracks were reconstructed in the 
%\Yuri{10 most upstream emulsion films} 
10 most upstream emulsion films. 
Figure~\ref{fig:i} shows the measured proton track density map with the bin size of $\SI{2}{mm} \times \SI{2}{mm}$, normalized to \SI{1}{cm^2}, and distribution of the proton density in this data sample.
%The tail in the histogram on the right side of Figure~\ref{fig:i} corresponds to a lower density of protons in some bins.  
%This tail may be caused either by the detection or readout inefficiency.
%\Yuri{Isn't the next statement just repeating the same point?}
% Seeing the left side of Figure~\ref{fig:i}, areas with fewer tracks are smaller than the beam size. 
% One of the films of this data has an area with fewer hits.
% And that area is almost similar in size and position to the less track area of the left side of Figure~\ref{fig:i}. 
% Thus, the tail on left side of Figure~\ref{fig:i} must be caused by the hit efficiency of one of the film.
%Because the areas with fewer tracks are smaller than the beam size, the decrease in the proton density is likely caused by local emulsion inefficiency rather than the incorrect speed of the TM stage.
%This inefficiency was likely caused by the variation in the quality of the emulsion layers and scanning errors.

%The mean value of the density $\mu_\rho$ is \SI{0.98 +- 0.03e5}{cm^{-2}}, and the standard deviation $\sigma_\rho$ of the fitted Gaussian function is $0.21 \times 10^4$/\si{cm^2}.
%The variation of the proton density is $\sigma_\rho/\mu_\rho = (2.1\pm0.3)\%$, which satisfies the requirement of $< 10\%$ fluctuation.

The mean value of the density is $\mu_\rho = \SI{1.01e5}{cm^{-2}}$, and the standard deviation of the fitted Gaussian function is $\sigma_\rho = 0.0187 \times 10^5$~\si{cm^{-2}}.
The fluctuation of the proton density is $\sigma_\rho/\mu_\rho = 1.9\pm0.3\%$, which satisfies the requirement of $< 10\%$ fluctuation.

\begin{figure}[tbp]
\centering
\includegraphics[width=.4\textwidth]{image/Density_Map_All_new_cut.pdf}
\qquad
\includegraphics[width=.4\textwidth]{image/Density_Distribution_All_new_cut.pdf}
\caption{\label{fig:i}Left: Track density fluctuation mapping on the emulsion of the DsTau 2021 physics run. Right: The density distribution histogram with Gaussian fitting.}
\end{figure}

\section{Summary}
The NA65/DsTau experiment aims to study tau neutrino production by detecting $D_s \to \tau \to X$ events with the emulsion based detector.
The proton distribution on the detector surface should be uniform with the density of $10^5$ \si{cm^{-2}}.
For physics analysis, the proton density fluctuation should be $<10\%$ and the data taking rate should be $\mathcal{O}(10^5)$ Hz.
%\Yuri{I would skip this sentence about 2016-2017 runs. }The Target Mover in the 2016 and 2017 test runs was too small-scale for the physics run.
%In order to satisfy the requirements for the physics run, the new full-scale TM and the real-time speed control system (RSCS) were implemented.
With the help of the new TM and the RSCS, fluctuation $<10\%$ in the pseudo proton density was achieved in the commissioning run with the radioactive source.
%In the test, the new TM and the RSCS achieved $<4\%$ fluctuation in the pseudo proton density. 
During the DsTau physics run in 2021 at the CERN-SPS H2 beamline, the TM and the RSCS worked successfully and allowed the data taking rate of \SI{200}{\kilo\hertz} and the proton density of $\SI{1.01e05}{cm^{-2}}$ with $ 1.9\pm0.3\%$ fluctuation, which exceeds the requirement of the DsTau experiment. 
%In the proton density mapping, several bins with less tracks are straggling. 
%That appears as the tails on the density distribution histogram.
%These areas are much smaller than the beam size.
%Thus, they should be caused by the scanning process or film efficiency, not by the TM or RSCS performance.


\acknowledgments

We thank for the members of the J-PARC E07 experiment for allowing us to use the TM mechanics.

Funding is gratefully acknowledged from national agencies and Institutions supporting us,
namely:   JSPS KAKENHI for Japan (Grant No. JP 20K23373, JP 18KK0085, JP 17H06926, JP 18H05541), 
CERN-RO(CERN Research Programme)   for Romania (Contract No. 03/03.01.2022)  and  TENMAK for Turkey (Grant No. 2022TENMAK(CERN) A5.H3.F2-1).





%\paragraph{Note added.} This is also a good position for notes added after the paper has been written.
%\Yuri{In the references list the format should be the same and according to the journal rules. Now it is defferent.}



\begin{comment}
% We suggest to always provide author, title and journal data:
% in short all the informations that clearly identify a document.

\begin{thebibliography}{99}

\bibitem{a}
Author, \emph{Title}, \emph{J. Abbrev.} {\bf vol} (year) pg.

\bibitem{b}
Author, \emph{Title},
arxiv:1234.5678.

\bibitem{c}
Author, \emph{Title},
Publisher (year).


% Please avoid comments such as "For a review'', "For some examples",
% "and references therein" or move them in the text. In general,
% please leave only references in the bibliography and move all
% accessory text in footnotes.

% Also, please have only one work for each \bibitem.


\end{thebibliography}

\end{comment}



\bibliography{main}


\end{document}

