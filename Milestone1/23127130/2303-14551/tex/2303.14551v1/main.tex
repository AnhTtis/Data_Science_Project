% ****** Start of file apssamp.tex ******
%
%   This file is part of the APS files in the REVTeX 4.2 distribution.
%   Version 4.2a of REVTeX, December 2014
%
%   Copyright (c) 2014 The American Physical Society.
%
%   See the REVTeX 4 README file for restrictions and more information.
%
% TeX'ing this file requires that you have AMS-LaTeX 2.0 installed
% as well as the rest of the prerequisites for REVTeX 4.2
%
% See the REVTeX 4 README file
% It also requires running BibTeX. The commands are as follows:
%
%  1)  latex apssamp.tex
%  2)  bibtex apssamp
%  3)  latex apssamp.tex
%  4)  latex apssamp.tex
%
\documentclass[%
 reprint,
%superscriptaddress,
%groupedaddress,
%unsortedaddress,
%runinaddress,
%frontmatterverbose, 
%preprint,
%preprintnumbers,
%nofootinbib,
%nobibnotes,
%bibnotes,
 amsmath,amssymb,
 aps,
%pra,
%prb,
%rmp,
%prstab,
%prstper,
%floatfix,
]{revtex4-2}

\usepackage{graphicx}% Include figure files
\usepackage{dcolumn}% Align table columns on decimal point
\usepackage{bm}% bold math
\usepackage{physics}
\usepackage[version=4]{mhchem}
\usepackage{xcolor}
%\usepackage{hyperref}% add hypertext capabilities
%\usepackage[mathlines]{lineno}% Enable numbering of text and display math
%\linenumbers\relax % Commence numbering lines

%\usepackage[showframe,%Uncomment any one of the following lines to test 
%%scale=0.7, marginratio={1:1, 2:3}, ignoreall,% default settings
%%text={7in,10in},centering,
%%margin=1.5in,
%%total={6.5in,8.75in}, top=1.2in, left=0.9in, includefoot,
%%height=10in,a5paper,hmargin={3cm,0.8in},
%]{geometry}
\newcounter{auxFootnote}
\begin{document}
\def\mB{{\mathcal{B}}}
\def\mE{{\mathcal{E}}}
\def\mA{{\mathcal{A}}}
\def\br{{\mathbf{r}}}

%\preprint{APS/123-QED}

\title{Non-equilibrium Theoretical Framework and Universal Design Principles of Oscillation-Driven Catalysis}

\author{Zhongmin Zhang}
\email{zzm@email.unc.edu}
\affiliation{Department of Chemistry,
University of North Carolina,
Chapel Hill, NC 27599-3290, U.S.A.}
\author{Zhiyue Lu}
\email{zhiyuelu@unc.edu}
\affiliation{Department of Chemistry,
University of North Carolina,
Chapel Hill, NC 27599-3290, U.S.A.}




\begin{abstract}
Traditional catalysis theory claims that catalysts speed up reactions by introducing low activation barriers and cannot alter the thermodynamic spontaneous reaction direction. However, if environments change rapidly, catalysts can be driven away from stationary states and exhibit anomalous performance. 
We present a geometric non-equilibrium theory and a control-conjugate landscape to describe and explain anomalous catalytic behaviors in rapidly oscillatory environments. Moreover, we derive a universal design principle for engineering optimal catalytic energy landscapes to achieve desired anomalous catalyst behaviors. Applications include but are not limited to (1) inverting a spontaneous reaction to synthesize high-free-energy molecules, and (2) dissipatively speeding up reactions without lowering activation barriers. In both cases, catalysts autonomously harness energy from non-equilibrium environments to enable such functionalities. 
\end{abstract}

\maketitle
\section{Introduction}
Traditional catalysis theory utilizes catalytic reaction pathways with low activation barriers to speed up both forward and backward reactions. The thermodynamic driving force of a directed catalyzed reaction could come from the free energy change of the reaction itself or from an external steady supply of free energies (e.g., light-harvesting catalysis \cite{Gratzel1983-rx}, or electrolysis \cite{Bard2000-vw}). Recently it has been shown that certain catalysts' performances can be altered or even enhanced when time-varying (oscillating) environments are applied as additional driving forces \cite{Ardagh2019-uj,Gathmann2022-rn,Ardagh2020-iu,Qi2020-jx,Ardagh2019-wb,Shetty2020-sb, berthoumieux2007response, Lemarchand2012-jl, lemarchand2012chemical, berthoumieux2009resonant}. In stochastic thermodynamics, similar observations have been made in energy-pumped enzymes, molecular machines, and flashing ratchets \cite{westerhoff1986enzymes, rozenbaum2004catalytic, qian2005cycle, astumian2001making, yasuda2001resolution,hill1975stochastics, qian1997simple, qian2000simple, astumian2002brownian, reimann2002brownian,astumian1994fluctuation, horowitz2009exact, PhysRevLett.109.203006, astumian2007design, rahav2008directed, sinitsyn2007berry, astumian2007adiabatic, astumian2003adiabatic, astumian2001towards, parrondo1998reversible, astumian2018stochastically, Li1997-xy, reimann1996brownian}. This work theoretically justified that, under environmental oscillation, certain catalysts can demonstrate surprising behavior that can never be achieved in stationary environments. For example, our previous work \cite{Zhang2022-iq} shows that a catalytic molecular machine under oscillatory temperature can invert a spontaneous reaction, converting low-free-energy products into high-free-energy reactants by harnessing energy from the oscillatory environment. These far-from-stationary-state behaviors can not be explained or designed by traditional catalysis theory.

Here presents a novel universal theoretical framework and a geometric design principle to explain and predict anomalous catalytic performances under rapid environmental oscillation. Examples of such performances include enhanced (or inverted) effective thermodynamic driving force, enhanced (or inverted) reaction turnover frequency, and even manipulation of catalytic selectivity. This general theory is applicable to catalysis pumped by arbitrary choices of environmental parameters, including temperature, electric field, light intensity, etc. Most importantly, the theory provides a universal energy-landscape design principle to guide the design of novel catalysts with maximal desired performance under environment oscillation. With simplifying approximations, each \emph{design principle} can be represented by \emph{convenient design rules} directly accessible by experimental test and verification. 

In contrast to our recent work focusing on temperature oscillation-driven molecular machines \cite{Zhang2022-iq}, this work provides a universal design principle of oscillation-driven catalysis with a broader range of applications in chemistry. 
By proposing a novel \emph{control-conjugate landscape} to represent catalysts and their response to arbitrary environmental parameters, we develop a geometric framework to explain the anomalous catalytic performance that can occur under various oscillatory environments (such as electrical voltage, light intensity, etc.).
The performances are not limited to inverting spontaneous chemical reactions and this work may find more general and practical chemistry applications in catalysis, electrochemistry, photochemistry, etc. 

This paper is structured as follows. The Theory section starts with a general kinetic model that captures the cyclic catalytic pathway of an arbitrary catalyst. Then we introduce the concept of control conjugate landscape to describe a catalyst's kinetic properties under different environmental conditions. By introducing the kinetic parameter space, we offer a geometric perspective on the anomalous performances exhibited by catalysts in response to rapidly oscillating environments. From the geometric interpretation, we obtained a universal design principle for oscillation-driven catalysis. In the Application section, we demonstrate two types of anomalous catalysis designed from the general design principle: energy harvesting and enhanced turnover frequency. In the Conclusion and Discussion section, we briefly discuss how the general geometric approach can be extended to more complex environmental parameters and general applications. 


\section{Theory}


\begin{figure}[h]
    \centering
    \includegraphics{fig_landscape.pdf}
    \caption{(a) A simple $(N=3)$-step catalytic cycle, spontaneously converting reactant R to product P. (b) Traditional energy landscapes at two choices of environmental parameters $\lambda_a$ and $\lambda_b$. (c) The control-conjugate landscape for $\lambda$ explicitly illustrates the catalyst's response to a control parameter $\lambda$ in a canonical form (Eq.~\ref{eq:rate}). The control-conjugate activation $\alpha_i$ is the height from the solid line to the dashed line for the $i$-th transition. Here $\alpha_{-1}=\alpha_{-2}=0$ indicates that the transition rates $r_{-1}$ and $r_{-2}$ are independent of $\lambda$. }
    \label{Fig:landscape}
\end{figure}

\subsection{Model}
Consider a general catalytic reaction pathway comprising $N$ intermediate reaction steps. For simplicity, it is illustrated in Fig.~\ref{Fig:landscape} as a 3-state catalytic cycle with $N=3$ intermediate steps, converting a reactant R into a product P per clockwise cycle.
At any stationary environment, each intermediate reaction steps $i$ from $1,2,\cdots N$ defines a forward reaction rate $r_i$ and a backward reaction rate $r_{-i}$.
We assume that at any stationary condition within the range of interest, the reaction spontaneously goes in the forward (clockwise) direction ($\Delta G < 0$). The dynamics of the system can be described by the master equation
\begin{equation}
    \dv{ \vec p }{ t} = \hat R \cdot \vec p
\end{equation}
where the $\vec p$ is the probability of each intermediate state at a given time and $\hat R$ is the transition rate matrix determined by the energy landscape and the environmental condition. In our model, the $\hat R$ is a $N\times N$ matrix whose elements are fully determined by the $2N$ reaction rates of the reaction pathway. Thus the $\hat R$ is equivalent to a $2N$-dim vector $\br=(r_1,r_{-1},r_2,r_{-2},\cdots, r_N,r_{-N})$, where $r_{\pm i}$ represents the forward and backward reaction rates for the $i$-th reaction. 

\subsection{Control-conjugate landscape} 
At any fixed environmental condition, the transition rates of each reaction can be determined by the energy landscape defined over the transition path (see Fig.~\ref{Fig:landscape}b). Each reaction step's reaction rate follows the Arrhenius law
\begin{equation}
\label{eq:ah}
r_{i} = A_i \exp (-E_{a,i}\beta) 
\end{equation}
where $A_i$ is prefactor including concentrations, $\beta=1/kT$ is inverse temperature, and $E_{a,i}$ is the activation energy given by the energy landscape (Fig.~\ref{Fig:landscape}b). We assume that the catalytic reaction occurs in a stationary chemical bath with the concentrations of reactants and products held constant. 

This work focuses on anomalous catalytic behaviors driven by rapidly oscillating environmental conditions. In general, the oscillatory environmental variable (e.g.,reactant or product concentrations, electric field, temperature, light intensity, etc.) can alter the shape of the energy landscape and thus impact the rate $r_{i}$ through $A_i$, $E_{a,i}$, and/or $\beta$. Thus, each catalyst under time-varying environmental conditions evolves on an energy ``seascape'' -- an energy landscape whose shape is controlled by the changing environmental condition. 

Inspired by the form of the Arrhenius law, we introduce a canonical form of reaction rate where we denote each rate's dependence on $\lambda$ in the exponential form with an activation-like quantity $\alpha_{i}$:
\begin{equation}
\label{eq:rate}
    r_{i} = r_{0,i}\exp(-\alpha_{i} \lambda)
\end{equation}
where $r_{0,i}$ is a control-independent factor, $\lambda$ is the \emph{canonical control parameter}, and $\alpha_{i}$ is the \emph{control-conjugate activation}. As illustrated in Fig.~\ref{Fig:landscape}b, given two environmental conditions $\lambda_a$ and $\lambda_b$, the activation energy change caused by the change of $\lambda$ is $\Delta E_{a,i}=RT\alpha_i\Delta \lambda$. 

The Eq.~\ref{eq:rate} leads to a novel representation of energy seascape by a \emph{control-conjugate landscape} pictured in Fig.~\ref{Fig:landscape}c.
In the traditional energy landscape, the inverse temperature $\beta$ universally impacts all reaction rates. In contrast, the conjugate energy landscape description allows selective environmental control that only impacts a few reaction steps. In other words, if the transition rate $r_j$ is independent of the external control $\lambda$, then $\alpha_{j}= 0$. In a special case where we choose to oscillate the environmental temperature $T$, then one can define the canonical control parameter as $\lambda=\beta=1/kT$, and the corresponding control-conjugate activation $\alpha_{i}$ reduces to the traditional activation energy $\alpha_{i}=E_{a,i}$. 


\subsection{Effective kinetics under rapid oscillation}
In general, at a constant environmental condition $\lambda$, traditional kinetics theory allows us to obtain the catalyst's steady-state performance by solving for the steady-state probability $\vec p_{ss}$ from the master equation $\hat R(\lambda) \cdot \vec p_{ss}=0$. At such steady-state, the \emph{catalytic performance} is dictated by the $\vec p_{ss}$ and the $\hat R(\lambda)$. As a result, the stationary catalytic performance can be expressed by a function of the transition rates for all $N$ reactions at the given $\lambda$, 
\begin{equation}\label{eq:fr}
f(\mathbf{r}) = f(r_1,r_{-1},r_2,r_{-2},\cdots, r_N,r_{-N})
\end{equation}
For instance, the performance can be chosen as the turnover frequency (Eq.~\ref{eq:Jr}), thermodynamic affinity (Eq.~\ref{eq:A'}), or catalytic selectivity. 

For time-varying environments, traditional kinetic theory can only be extended to describe slowly-varying environments (quasi-static control), where the system is maintained at slowly varying non-equilibrium steady states (NESS's) at all times. However, when the environment rapidly oscillates, the system can not catch up with the rapidly changing rates. As a result, it can lead to hard-to-solve yet novel performances which can never be observed at NESS. 

To obtain a universal analytic theory for oscillation-driven catalysis, this work considers high-frequency oscillation whose period $\tau$ is much smaller than the fastest transition timescale $\tau \ll \min_{i}(r_i^{-1})$. Under rapid oscillation, the effective reaction rates converge to the time-average reaction rates ${\mathbf r}^\text{eff}$. In this limit, the oscillatory-driven catalyst reaches a pseudo-stationary state\cite{tagliazucchi2014dissipative,Wang2016-fi,Zhang2022-iq}, whose effective performance can be obtained by plugging in the effective rates into Eq.~\ref{eq:fr} as $f({\mathbf r}^\text{eff})$. 
We claim that if the effective rates diverge from any constant-environment rates (see Fig.~\ref{Fig:geo}), the catalyst can demonstrate anomalous performance (e.g., turnover frequency and thermodynamic affinity) when exposed to rapidly oscillating environments. 

For illustrative purposes, in the rest of the paper, we consider a simple binary oscillatory environment with half a period in $\lambda_\text{a}$ and another half in $\lambda_\text{b}$. 
At the fast oscillation limit, each reaction step $i$ has $1/2$ chance to occur with one rate $r_{i,\text{a}}$, and $1/2$ chance to occur with another rate $r_{i,\text{b}}$. Thus, we can compute the effective rate as
\begin{equation}
\label{eq:eff-rate}
    r_i^\text{eff} = \frac{r_{i,\text{a}} + r_{i,\text{b}}}{2}
\end{equation}
as the arithmetic mean of the two instantaneous rates. 
Then the effective performance can be expressed by
\begin{equation}
    f({\mathbf r}^\text{eff}) = f( \frac{{\mathbf r}_{\text{a}}+{\mathbf r}_{\text{b}}}{2})
\end{equation}
which may never be achieved by any stationary environment $f(\mathbf r(\lambda))$.


\subsection{Geometrical design principle} 

\begin{figure*}[ht]
    \centering
    \includegraphics{geo.pdf}
    \caption{The geometric intuition of oscillation-induced catalysis is illustrated in the design space comprising $2N$ reaction rates $\br$. (a) Given reaction rates $\br$, the corresponding catalytic performance $f(\br)$ is captured by a set of isosurfaces $f(\br)=c$ illustrated by light orange surfaces. The gradient of $f$ is the normal vector to the isosurface. A given catalyst's rates under any stationary environment $\lambda$ form a 1-dim locus $\br (\lambda)$. When environments oscillates rapidly between $\lambda_a$ and $\lambda_b$, the catalyst achieves an effective state $\br_{\text{eff}}=(\br_a+\br_b)/2$, which can achieve a new performance $f(\br_{\text{eff}})$. (b) For any catalyst described by the control-conjugate landscape (Eq.~\ref{eq:rate}), the 1-dim locus $\br(\lambda)$ appears to be a straight line in the 2N-dim logarithm scale representation.}
    \label{Fig:geo}
\end{figure*}


Through geometric arguments in a $2N$-dimensional $\br$-space comprising all reaction rates, we provide a universal landscape design principle for novel catalysts to enhance or reduce an arbitrary desired property $f$ under rapid oscillation. In this $2N$-dimensional design space, a catalyst described by a control-conjugate landscape can be represented by a 1-dimensional locus $\br(\lambda)$ parameterized by $\lambda$, shown as a black solid curve in Fig.~\ref{Fig:geo}a. By oscillating $\lambda$, the effective rate $\br^{\text{eff}}$ can be found at the middle point of ${\mathbf r}_{\text{a}}$ and ${\mathbf r}_{\text{b}}$. Geometrically, the more curved the locus $\br(\lambda)$, the larger the deviation of $\br^{\text{eff}}$ from any stationary $\br(\lambda)$. Locally, this deviation is captured by the acceleration vector of the locus as $\ddot{\br}=\dd^2{\mathbf{r}}/\dd{\lambda}^2$. 

To design a catalyst that can achieve enhanced (or suppressed) effective performance $f({\mathbf r}^\text{eff})$, we use a geometrical indicator that combines the deviation vector $\ddot{\br}$ and the gradient vector of performance $\nabla _{\mathbf{r}} f$. Specifically, if the oscillation-induced deviation aligns with (or against) the steepest increment of desired performance $f$, the catalyst can exhibit enhanced (or suppressed) performance under rapid oscillation. 

Mathematically, the design principle to achieve desired performance enhancement (or suppression) can be expressed by the maximization (or minimization) of a universal \emph{design criterion} 
\begin{equation}
\label{eq:CCC}
    \mathcal{C}^{[f]} \equiv \nabla_{\mathbf{r}} f \cdot \ddot{\mathbf{r}} = \sum_{i=\pm 1}^{\pm N} \pdv{f}{r_{i}} \cdot \dv[2]{r_{i}}{\lambda}
\end{equation}
which is the dot product between the gradient vector of performance $f$ and the acceleration vector. This geometry-inspired criterion rigorously quantifies the oscillation-induced performance deviation under a small amplitude of $\lambda$. However, as shown by the following geometrical argument, the criterion applies well even for large-amplitude $\lambda$ oscillations. This is justified because the $\br(\lambda)$ locus bears a well-behaving geometrical shape -- at the logarithm-scale representation of the $\br$-space (see Fig.~\ref{Fig:geo}b), the locus reduces to a straight line, with the angle the line defined by the control-conjugate activations (see Eq.~\ref{eq:rate}). One can verify the result by applying it to design the catalytic inversion of spontaneous reaction under rapid temperature oscillation, where the general principle reduces to the result obtained in our previous work \cite{Zhang2022-iq}.

This geometric analysis not only provides us with a universal design principle but also reveals the geometrical origin of the anomalous oscillation-induced performance change. The performance deviation depends on the nonlinearities of both the rates (Eq.~\ref{eq:rate}) and the performance function (Eq.~\ref{eq:fr}) in the $\br$-space. 



\section{Applications}
The universal design criterion $\mathcal{C}^{[f]}$ can guide the design of oscillation-driven catalysts with desired performance $f$. 
For illustrative purposes, this paper demonstrates two interesting anomalous behavior: (Application 1) oscillation-induced inversion of spontaneous reaction direction, and (Application 2) oscillation-enhanced high turnover frequency without utilizing low activation barriers as required by the traditional theory. Notice that the design principle applies to any desired interest performance $f$, not limited to these two examples. In all cases, the anomalous behavior is sustained by the energy harnessed from the oscillatory environment.  


\begin{figure*}[ht]
    \centering
    \includegraphics{example2.pdf}
    \caption{Demonstrations of application-1 (a,c,e) and application-2 (b,d,f). Dynamics at two stationary environments ($\lambda_a$ and $\lambda_b$) are shown in red and blue; the oscillation-induced effective dynamics are in green.
    (a,b) illustrate stationary reaction rates and effective rates (Eq.~\ref{eq:eff-rate}). (c,d) illustrate two dimensionless stationery free-energy landscapes $G/RT$ and the effective landscape.
    The tilt of the (effective) landscape, i.e., the height difference between two ends, which equals $-\tilde{\mathcal{A}}$, indicates the (effective) spontaneous direction. 
    (e,f) illustrates the detailed steady-state currents of each transition (thickness of arrows) and the steady-state probability of intermediate states (size of disks). $J$ represents the net reaction currents.
    For both applications, we set $r_{1,a}=2$, and $r_{-1,a}=r_{2,a}=r_{-2,a}=r_{3,a}=r_{-3,a}=1$ for $\lambda_a=0$. The rates for $\lambda_b=-1$ can be calculated from $r_{i,a}$'s and $\alpha_i$'s. For both $\lambda_a$ and $\lambda_b$, $\Delta G_\text{rxn}/RT = -\tilde{\mathcal{A}} =-\ln 2$. }
    \label{Fig:app}
\end{figure*}

{\bf Application 1 -- Inverting Thermodynamic Affinity}. Here demonstrated is the optimal catalytic energy landscape for inverting the spontaneous direction of the reaction ($\tilde{\mathcal{A}}$). Such inversion is particularly interesting as the catalyst harnesses energy from the external environment and stores the energy by converting low-free energy products into high-free-energy reactants. We choose the performance $f({\mathbf r})$ to represent the thermodynamic spontaneity (i.e., dimensionless thermodynamic affinity)
\begin{equation}
\label{eq:A'}
    \tilde{\mathcal{A}}({\mathbf r}) = \sum_{i=1}^N \log \frac{r_i}{r_{-i}}
\end{equation}
which is simply related to the reaction Gibbs free energy change via $\tilde{\mathcal{A}}=-\beta \Delta G$.
To maximally invert the spontaneous reaction, one needs to find a catalyst with a very negative $\mathcal{C}^{[\tilde{\mathcal{A}}]}$. In other words, the universal principle Eq.~\ref{eq:CCC} for designing strong catalytic inversion of a spontaneous reaction becomes
\begin{equation}
\label{eq:principle1}
    \mathcal{C}^{[\tilde{\mathcal{A}}]} = \sum_{i=1}^N \alpha_i^2-\sum_{i=-1}^{-N}\alpha_{i}^2
\end{equation}
which is a simple quadratic function that depends only on $\alpha_i$'s, each step's activation level conjugates to the control $\lambda$. 
In conclusion, a designed catalyst with very negative $\mathcal{C}^{[\tilde{\mathcal{A}}]}$ tends to strongly invert the spontaneous reaction direction under rapid oscillatory control. 

The design principle Eq.~\ref{eq:principle1} can be experimentally verified. Here we illustrate an optimal catalyst obtained by the principle under a few constraints: To prevent the control parameter $\lambda$ from directly impacting the spontaneity $\tilde{\mathcal{A}}$, we require that 
\begin{equation}
\label{eq:const}
    \dv{\tilde{\mathcal{A}}}{\lambda} = \sum_{i=1}^N (\alpha_i-\alpha_{-i}) = 0
\end{equation}
and thus, any observed inversion of the reaction's effective direction is purely a result of the catalytic response to rapid environment oscillation. 
By suppressing $\sum_{i=1}^N \alpha_i^2$ and increasing $\sum_{i=-1}^{-N} \alpha_i^2$ under the constraint Eq.~\ref{eq:const}, the optimal reaction-inverting catalyst for $N=3$ can be designed as
\begin{align}
\label{eq:+inv}
\begin{split}
    (\alpha_{1}, \alpha_{2}, \alpha_{3}) &= (\frac{1}{3}\alpha, \frac{1}{3}\alpha, \frac{1}{3}\alpha) \\
    (\alpha_{-1}, \alpha_{-2}, \alpha_{-3}) &= (\alpha, 0, 0)
\end{split}
\end{align}
This design is illustrated in Fig.~\ref{Fig:app}a, where the two sets of reaction rates respectively correspond to environmental conditions $\lambda_a=0$ (red) and $\lambda_b=-1$ (blue), and the effective rates due to rapid oscillation are shown in green colored bars.

Shown in Fig.~\ref{Fig:app}c, the effective reaction energy landscape for this catalyst has an inverted tilted direction compared to the stationary-condition landscapes. In other words, the effective affinity has an opposite sign of the affinity of the stationary environmental condition, leading to the reaction direction inversion. As illustrated in Fig.~\ref{Fig:app}e ($\alpha=12$), under constant environment $\lambda_a$ (or $\lambda_b$), the steady-state current is $J_a=0.833$ (or $J_b=0.176$), whereas a rapid oscillation between these two conditions leads to an average current $J_\text{eff}=-0.158$. Thus the catalyst actively harnesses energy from the environment to drive the reaction against its spontaneous direction. A special case of catalytic thermal engines under temperature oscillation is demonstrated in reference \cite{Zhang2022-iq}.

{\bf Application 2 -- Enhancing reaction TOF (current)}. Here we illustrate the design principle for optimal enhancement of reaction current $J$, which is equivalent to turnover frequency (TOF). The TOF is actively enhanced by energy harnessed from the environment, and cannot be explained by traditional catalysis theory, which focuses on lowering activation barriers. 
In this case, we choose performance $f({\mathbf r})$ as the (pseudo-)stationary current $J({\mathbf r})$, which can be analytically solved for arbitrary catalytic cycles (see reference \cite{Hill1988-tp}). For a $N=3$-step catalytic cycle, the current is
\begin{align}
    J({\mathbf r}) &= \frac{u({\mathbf r})}{v({\mathbf r})}\label{eq:Jr}\\
    u({\mathbf r}) &= r_1 r_2 r_3 - r_{-1} r_{-2} r_{-3} \label{eq:ur}\\
    \begin{split}
    v({\mathbf r}) &= r_1 r_2 + r_2 r_3 + r_3 r_1\\
    &\ + r_{-1} r_{-2} + r_{-2} r_{-3} + r_{-3} r_{-1} \\
    &\ + r_{1} r_{-2} + r_{2} r_{-3} + r_{3} r_{-1}
    \end{split}
\end{align}
Plugging into the universal principle, Eq.~\ref{eq:CCC}, we find the design criteria for enhancing TOF as to increase the value of
\begin{equation}
\label{eq:enhanceJ}
    \mathcal{C}^{[J]} = J (\frac{u_2}{u} - \frac{v_2}{v})
\end{equation}
where
\begin{align}
\begin{split}
    u_2 &= (\alpha_1^2+\alpha_2^2+\alpha_3^2) r_1 r_2 r_3 \\
    &\ - (\alpha_{-1}^2+\alpha_{-2}^2+\alpha_{-3}^2) r_{-1} r_{-2} r_{-3}  \\
\end{split} \\
\begin{split}
    v_2 &= (\alpha_1^2+\alpha_2^2)r_1 r_2 + (\alpha_2^2+\alpha_3^2)r_2 r_3 + (\alpha_3^2+\alpha_1^2)r_3 r_1\\
    &\ + (\alpha_{-1}^2+\alpha_{-2}^2)r_{-1} r_{-2} + (\alpha_{-2}^2+\alpha_{-3}^2)r_{-2} r_{-3} \\
    &\ + (\alpha_{-3}^2+\alpha_{-1}^2)r_{-3} r_{-1} 
    + (\alpha_{1}^2+\alpha_{-2}^2)r_{1} r_{-2} \\
    &\ + (\alpha_{2}^2+\alpha_{-3}^2)r_{2} r_{-3} + (\alpha_{3}^2+\alpha_{-1}^2)r_{3} r_{-1}\\
\end{split} 
\end{align}
Here Eq.~\ref{eq:enhanceJ}, as a function of $r_i$'s as well as $\alpha_i$'s can be directly used as a design principle for the numerical search of the optimal design of oscillation-enhanced catalyst. Moreover, by observing this equation, we can further obtain approximated design rules leading to a simple working example of a prominent current-enhancing catalyst under environment oscillation.
Notice that ${v_2}/{v}$ in Eq.~\ref{eq:enhanceJ} is always positive. Thus to enhance the performance, or achieve a large positive $\mathcal{C}^{[J]}$, we need to find a catalyst of both large ${u_2}/{u}$ and large $J$.
To increase ${u_2}/{u}$, one can enhance $\sum_{i=1}^N \alpha_i^2$ and suppress $\sum_{i=-1}^{-N} \alpha_i^2$.
Combining these considerations, we obtain an intuitive catalytic landscape characterized by 
\begin{align}
\label{eq:enh}
\begin{split}
    (\alpha_{1}, \alpha_{2}, \alpha_{3}) &= (\alpha, \alpha, 0) \\
    (\alpha_{-1}, \alpha_{-2}, \alpha_{-3}) &= (\frac{2}{3}\alpha, \frac{2}{3}\alpha, \frac{2}{3}\alpha)
\end{split}
\end{align}
An example of choosing $\alpha=6$ is summarized in Fig.~\ref{Fig:app}b,d,f.
The reaction rates for $\lambda_a$, $\lambda_b$, and the effective rates are shown in Fig.~\ref{Fig:app}b. The corresponding dimensionless free energy landscapes are shown in Fig.~\ref{Fig:app}d. In Fig.~\ref{Fig:app}f, we verify that the effective reaction TOF (current) reaches $J_{\text{eff}}=5.93$ under rapid oscillation, which is significantly greater than the stationary current of both conditions ($J_a=0.833$, and $J_b=4.05$).




\section{Conclusion and Discussion}
This paper describes a general non-equilibrium theory framework predicting what catalysts could demonstrate anomalous performance under rapid environment oscillation and why. The theory generally applies to catalysts with an arbitrary number of intermediate steps $N$, and proposes a convenient control-conjugate landscape to describe the catalyst's response to different environmental conditions $\lambda$'s. Furthermore, for any oscillatory control parameter, the theory provides a geometric design principle for engineering optimal catalysts with a wide range of oscillation-enhanced performances $f$, such as inverting spontaneous reaction direction or enhancing reaction rate, both empowered by the dissipation of energy. Immediate applications following the two examples listed above can be found in oscillation-enhanced redox reactions and electrolysis, where one can empower a reaction by low-magnitude AC voltages rather than high-magnitude DC voltages. For instance, traditional DC electrolysis is subject to the thermodynamic limit (theoretical cell voltage) plus an extra kinetic limit (overpotential). With novel catalysts, the reaction could occur under oscillatory voltage, even if it is below the theoretical cell voltage. In practice, the weak voltage can avoid undesired side reactions such as electrolysis of the solvent. 

Beyond these two applications, by defining other types of performances $f$, one can achieve more complex and useful performances such as selectivity manipulation. For instance, consider a complex catalytic network with multiple loops, where one loop produces desired products while other loops produce by-products. We can design oscillation-responsiveness to each loop and define the performance $f$ as selectivity, which is the ratio between the net current of the desired-product loop and the total net current among all loops. This allows us to manipulate the catalytic selectivity by adjusting the external oscillatory stimulation.

% r_eff and r_lambda
The fundamental geometric intuition in the design space to explain oscillation-induced anomalous performances is general and not restricted by the simplicity of control parameter $\lambda$ and the control-conjugate landscape. As represented in the $\br$-space, a catalyst under any stationary environment is described by a point $\br(\lambda)$. When the environment rapidly oscillates, the catalyst's kinetic rates are effectively represented by an average point $\br_{\text{eff}}$ in the same $\br$-space, allowing for a new performance that can not be achieved at any stationary condition. Even if the environmental control is complicated (e.g., when reaction rates cannot be simply represented by a control-conjugate landscape, or when the environmental control involves multi-dimensional $\vec \lambda$), one can still utilize the geometric intuition to obtain design principles for desired oscillation-induced performances. 


In the future, this work opens up the opportunity to design catalysts beyond the stationary theory, where catalysts are not simply designed by lowering activation barriers as required by the traditional theory. Beyond what is discussed in this paper, one can also find design principles in terms of the optimal control protocols of $\lambda(t)$. This can potentially allow catalysts to function as environmental sensors, producing signaling molecules only when the environment oscillates according to specific temporal patterns.



\begin{acknowledgements}
We acknowledge the fund from the National Science Foundation Grant DMR-2145256. 
We appreciate helpful discussions and suggestions for the manuscript from Hong Qian, Zhixin Lu, Yosuke Kanai, Jahan Dawlaty, Ruicheng Bao, Shiling Liang, and Jim Cahoon.
\end{acknowledgements}


\bibliography{ref}% Produces the bibliography via BibTeX.


\end{document}
%
% ****** End of file apssamp.tex ******



